% This must be in the first 5 lines to tell arXiv to use pdfLaTeX, which is strongly recommended.
\pdfoutput=1
% In particular, the hyperref package requires pdfLaTeX in order to break URLs across lines.

\documentclass[11pt]{article}

% Change "review" to "final" to generate the final (sometimes called camera-ready) version.
% Change to "preprint" to generate a non-anonymous version with page numbers.
% \usepackage[review]{acl}
\usepackage[preprint]{acl}

% Standard package includes
\usepackage{times}
\usepackage{latexsym}

% For proper rendering and hyphenation of words containing Latin characters (including in bib files)
\usepackage[T1]{fontenc}
% For Vietnamese characters
% \usepackage[T5]{fontenc}
% See https://www.latex-project.org/help/documentation/encguide.pdf for other character sets

% This assumes your files are encoded as UTF8
\usepackage[utf8]{inputenc}

% This is not strictly necessary, and may be commented out,
% but it will improve the layout of the manuscript,
% and will typically save some space.
\usepackage{microtype}

% This is also not strictly necessary, and may be commented out.
% However, it will improve the aesthetics of text in
% the typewriter font.
\usepackage{inconsolata}

%Including images in your LaTeX document requires adding
%additional package(s)
\usepackage{graphicx}

% If the title and author information does not fit in the area allocated, uncomment the following
%
%\setlength\titlebox{<dim>}
%
% and set <dim> to something 5cm or larger.
\usepackage{xspace}
\usepackage{multirow} 
\usepackage{amsmath}
\usepackage{graphicx}  % 导入 graphicx 包
\usepackage{caption}  % 导入 caption 包
\captionsetup[table]{justification=raggedright,singlelinecheck=false} % 设置表格标题左对齐
\usepackage{tabularray}
\usepackage{siunitx} % 加载siunitx包
\usepackage{amssymb}
\newcommand{\benchmark}{\textsc{Code-Vision}\xspace}
\usepackage{booktabs}
\usepackage[draft]{minted}
\usepackage{longtable}
\title{\benchmark: Evaluating Multimodal LLMs Logic Understanding and Code Generation Capabilities}

% Author information can be set in various styles:
% For several authors from the same institution:
% \author{Author 1 \and ... \and Author n \\
%         Address line \\ ... \\ Address line}
% if the names do not fit well on one line use
%         Author 1 \\ {\bf Author 2} \\ ... \\ {\bf Author n} \\
% For authors from different institutions:
% \author{Author 1 \\ Address line \\  ... \\ Address line
%         \And  ... \And
%         Author n \\ Address line \\ ... \\ Address line}
% To start a separate ``row'' of authors use \AND, as in
% \author{Author 1 \\ Address line \\  ... \\ Address line
%         \AND
%         Author 2 \\ Address line \\ ... \\ Address line \And
%         Author 3 \\ Address line \\ ... \\ Address line}

\author{Hanbin Wang$^{1}$\thanks{ \ \ indicates equal contribution.}, Xiaoxuan Zhou$^{2*}$, Zhipeng Xu$^{2}$, Keyuan Cheng$^{1}$, Yuxin Zuo$^{3}$, \\ \textbf{Kai Tian$^{2}$, Jingwei Song$^{4}$, Junting Lu$^{1}$, Wenhui Hu$^{1}$ and Xueyang Liu$^{1}$} \\ 
$^1$Peking University $^2$Northeastern University \\
$^3$Institute of Computing Technolgy, Chinese Academy of Sciences 
$^4$University of Hong Kong \\
}

% indicates equal contribution.
% indicates corresponding author.
% \author{First Author \\
%   Affiliation / Address line 1 \\
%   Affiliation / Address line 2 \\
%   Affiliation / Address line 3 \\
%   \texttt{email@domain} \\\And
%   Second Author \\
%   Affiliation / Address line 1 \\
%   Affiliation / Address line 2 \\
%   Affiliation / Address line 3 \\
%   \texttt{email@domain} \\}

%\author{
%  \textbf{First Author\textsuperscript{1}},
%  \textbf{Second Author\textsuperscript{1,2}},
%  \textbf{Third T. Author\textsuperscript{1}},
%  \textbf{Fourth Author\textsuperscript{1}},
%\\
%  \textbf{Fifth Author\textsuperscript{1,2}},
%  \textbf{Sixth Author\textsuperscript{1}},
%  \textbf{Seventh Author\textsuperscript{1}},
%  \textbf{Eighth Author \textsuperscript{1,2,3,4}},
%\\
%  \textbf{Ninth Author\textsuperscript{1}},
%  \textbf{Tenth Author\textsuperscript{1}},
%  \textbf{Eleventh E. Author\textsuperscript{1,2,3,4,5}},
%  \textbf{Twelfth Author\textsuperscript{1}},
%\\
%  \textbf{Thirteenth Author\textsuperscript{3}},
%  \textbf{Fourteenth F. Author\textsuperscript{2,4}},
%  \textbf{Fifteenth Author\textsuperscript{1}},
%  \textbf{Sixteenth Author\textsuperscript{1}},
%\\
%  \textbf{Seventeenth S. Author\textsuperscript{4,5}},
%  \textbf{Eighteenth Author\textsuperscript{3,4}},
%  \textbf{Nineteenth N. Author\textsuperscript{2,5}},
%  \textbf{Twentieth Author\textsuperscript{1}}
%\\
%\\
%  \textsuperscript{1}Affiliation 1,
%  \textsuperscript{2}Affiliation 2,
%  \textsuperscript{3}Affiliation 3,
%  \textsuperscript{4}Affiliation 4,
%  \textsuperscript{5}Affiliation 5
%\\
%  \small{
%    \textbf{Correspondence:} \href{mailto:email@domain}{email@domain}
%  }
%}

\begin{document}
\maketitle
\begin{abstract}
This paper introduces \benchmark, a benchmark designed to evaluate the logical understanding and code generation capabilities of Multimodal Large Language Models (MLLMs). It challenges MLLMs to generate a correct program that fulfills specific functionality requirements based on a given flowchart, which visually represents the desired algorithm or process. \benchmark comprises three subsets—HumanEval-V, Algorithm, and MATH, which evaluate MLLMs' coding abilities across basic programming, algorithmic, and mathematical problem-solving domains. Our experiments evaluate 12 MLLMs on \benchmark. Experimental results demonstrate that there is a large performance difference between proprietary and open-source models. On Hard problems, GPT-4o can achieve 79.3\% pass@1, but the best open-source model only achieves 15\%. Further experiments reveal that \benchmark can pose unique challenges compared to other multimodal reasoning benchmarks MMCode and MathVista. We also explore the reason for the poor performance of the open-source models. All data and codes are available at \url{https://github.com/wanghanbinpanda/CodeVision}. 
\end{abstract}

\section{Introduction}
Backdoor attacks pose a concealed yet profound security risk to machine learning (ML) models, for which the adversaries can inject a stealth backdoor into the model during training, enabling them to illicitly control the model's output upon encountering predefined inputs. These attacks can even occur without the knowledge of developers or end-users, thereby undermining the trust in ML systems. As ML becomes more deeply embedded in critical sectors like finance, healthcare, and autonomous driving \citep{he2016deep, liu2020computing, tournier2019mrtrix3, adjabi2020past}, the potential damage from backdoor attacks grows, underscoring the emergency for developing robust defense mechanisms against backdoor attacks.

To address the threat of backdoor attacks, researchers have developed a variety of strategies \cite{liu2018fine,wu2021adversarial,wang2019neural,zeng2022adversarial,zhu2023neural,Zhu_2023_ICCV, wei2024shared,wei2024d3}, aimed at purifying backdoors within victim models. These methods are designed to integrate with current deployment workflows seamlessly and have demonstrated significant success in mitigating the effects of backdoor triggers \cite{wubackdoorbench, wu2023defenses, wu2024backdoorbench,dunnett2024countering}.  However, most state-of-the-art (SOTA) backdoor purification methods operate under the assumption that a small clean dataset, often referred to as \textbf{auxiliary dataset}, is available for purification. Such an assumption poses practical challenges, especially in scenarios where data is scarce. To tackle this challenge, efforts have been made to reduce the size of the required auxiliary dataset~\cite{chai2022oneshot,li2023reconstructive, Zhu_2023_ICCV} and even explore dataset-free purification techniques~\cite{zheng2022data,hong2023revisiting,lin2024fusing}. Although these approaches offer some improvements, recent evaluations \cite{dunnett2024countering, wu2024backdoorbench} continue to highlight the importance of sufficient auxiliary data for achieving robust defenses against backdoor attacks.

While significant progress has been made in reducing the size of auxiliary datasets, an equally critical yet underexplored question remains: \emph{how does the nature of the auxiliary dataset affect purification effectiveness?} In  real-world  applications, auxiliary datasets can vary widely, encompassing in-distribution data, synthetic data, or external data from different sources. Understanding how each type of auxiliary dataset influences the purification effectiveness is vital for selecting or constructing the most suitable auxiliary dataset and the corresponding technique. For instance, when multiple datasets are available, understanding how different datasets contribute to purification can guide defenders in selecting or crafting the most appropriate dataset. Conversely, when only limited auxiliary data is accessible, knowing which purification technique works best under those constraints is critical. Therefore, there is an urgent need for a thorough investigation into the impact of auxiliary datasets on purification effectiveness to guide defenders in  enhancing the security of ML systems. 

In this paper, we systematically investigate the critical role of auxiliary datasets in backdoor purification, aiming to bridge the gap between idealized and practical purification scenarios.  Specifically, we first construct a diverse set of auxiliary datasets to emulate real-world conditions, as summarized in Table~\ref{overall}. These datasets include in-distribution data, synthetic data, and external data from other sources. Through an evaluation of SOTA backdoor purification methods across these datasets, we uncover several critical insights: \textbf{1)} In-distribution datasets, particularly those carefully filtered from the original training data of the victim model, effectively preserve the model’s utility for its intended tasks but may fall short in eliminating backdoors. \textbf{2)} Incorporating OOD datasets can help the model forget backdoors but also bring the risk of forgetting critical learned knowledge, significantly degrading its overall performance. Building on these findings, we propose Guided Input Calibration (GIC), a novel technique that enhances backdoor purification by adaptively transforming auxiliary data to better align with the victim model’s learned representations. By leveraging the victim model itself to guide this transformation, GIC optimizes the purification process, striking a balance between preserving model utility and mitigating backdoor threats. Extensive experiments demonstrate that GIC significantly improves the effectiveness of backdoor purification across diverse auxiliary datasets, providing a practical and robust defense solution.

Our main contributions are threefold:
\textbf{1) Impact analysis of auxiliary datasets:} We take the \textbf{first step}  in systematically investigating how different types of auxiliary datasets influence backdoor purification effectiveness. Our findings provide novel insights and serve as a foundation for future research on optimizing dataset selection and construction for enhanced backdoor defense.
%
\textbf{2) Compilation and evaluation of diverse auxiliary datasets:}  We have compiled and rigorously evaluated a diverse set of auxiliary datasets using SOTA purification methods, making our datasets and code publicly available to facilitate and support future research on practical backdoor defense strategies.
%
\textbf{3) Introduction of GIC:} We introduce GIC, the \textbf{first} dedicated solution designed to align auxiliary datasets with the model’s learned representations, significantly enhancing backdoor mitigation across various dataset types. Our approach sets a new benchmark for practical and effective backdoor defense.



\section{Related Work}

\subsection{Large 3D Reconstruction Models}
Recently, generalized feed-forward models for 3D reconstruction from sparse input views have garnered considerable attention due to their applicability in heavily under-constrained scenarios. The Large Reconstruction Model (LRM)~\cite{hong2023lrm} uses a transformer-based encoder-decoder pipeline to infer a NeRF reconstruction from just a single image. Newer iterations have shifted the focus towards generating 3D Gaussian representations from four input images~\cite{tang2025lgm, xu2024grm, zhang2025gslrm, charatan2024pixelsplat, chen2025mvsplat, liu2025mvsgaussian}, showing remarkable novel view synthesis results. The paradigm of transformer-based sparse 3D reconstruction has also successfully been applied to lifting monocular videos to 4D~\cite{ren2024l4gm}. \\
Yet, none of the existing works in the domain have studied the use-case of inferring \textit{animatable} 3D representations from sparse input images, which is the focus of our work. To this end, we build on top of the Large Gaussian Reconstruction Model (GRM)~\cite{xu2024grm}.

\subsection{3D-aware Portrait Animation}
A different line of work focuses on animating portraits in a 3D-aware manner.
MegaPortraits~\cite{drobyshev2022megaportraits} builds a 3D Volume given a source and driving image, and renders the animated source actor via orthographic projection with subsequent 2D neural rendering.
3D morphable models (3DMMs)~\cite{blanz19993dmm} are extensively used to obtain more interpretable control over the portrait animation. For example, StyleRig~\cite{tewari2020stylerig} demonstrates how a 3DMM can be used to control the data generated from a pre-trained StyleGAN~\cite{karras2019stylegan} network. ROME~\cite{khakhulin2022rome} predicts vertex offsets and texture of a FLAME~\cite{li2017flame} mesh from the input image.
A TriPlane representation is inferred and animated via FLAME~\cite{li2017flame} in multiple methods like Portrait4D~\cite{deng2024portrait4d}, Portrait4D-v2~\cite{deng2024portrait4dv2}, and GPAvatar~\cite{chu2024gpavatar}.
Others, such as VOODOO 3D~\cite{tran2024voodoo3d} and VOODOO XP~\cite{tran2024voodooxp}, learn their own expression encoder to drive the source person in a more detailed manner. \\
All of the aforementioned methods require nothing more than a single image of a person to animate it. This allows them to train on large monocular video datasets to infer a very generic motion prior that even translates to paintings or cartoon characters. However, due to their task formulation, these methods mostly focus on image synthesis from a frontal camera, often trading 3D consistency for better image quality by using 2D screen-space neural renderers. In contrast, our work aims to produce a truthful and complete 3D avatar representation from the input images that can be viewed from any angle.  

\subsection{Photo-realistic 3D Face Models}
The increasing availability of large-scale multi-view face datasets~\cite{kirschstein2023nersemble, ava256, pan2024renderme360, yang2020facescape} has enabled building photo-realistic 3D face models that learn a detailed prior over both geometry and appearance of human faces. HeadNeRF~\cite{hong2022headnerf} conditions a Neural Radiance Field (NeRF)~\cite{mildenhall2021nerf} on identity, expression, albedo, and illumination codes. VRMM~\cite{yang2024vrmm} builds a high-quality and relightable 3D face model using volumetric primitives~\cite{lombardi2021mvp}. One2Avatar~\cite{yu2024one2avatar} extends a 3DMM by anchoring a radiance field to its surface. More recently, GPHM~\cite{xu2025gphm} and HeadGAP~\cite{zheng2024headgap} have adopted 3D Gaussians to build a photo-realistic 3D face model. \\
Photo-realistic 3D face models learn a powerful prior over human facial appearance and geometry, which can be fitted to a single or multiple images of a person, effectively inferring a 3D head avatar. However, the fitting procedure itself is non-trivial and often requires expensive test-time optimization, impeding casual use-cases on consumer-grade devices. While this limitation may be circumvented by learning a generalized encoder that maps images into the 3D face model's latent space, another fundamental limitation remains. Even with more multi-view face datasets being published, the number of available training subjects rarely exceeds the thousands, making it hard to truly learn the full distibution of human facial appearance. Instead, our approach avoids generalizing over the identity axis by conditioning on some images of a person, and only generalizes over the expression axis for which plenty of data is available. 

A similar motivation has inspired recent work on codec avatars where a generalized network infers an animatable 3D representation given a registered mesh of a person~\cite{cao2022authentic, li2024uravatar}.
The resulting avatars exhibit excellent quality at the cost of several minutes of video capture per subject and expensive test-time optimization.
For example, URAvatar~\cite{li2024uravatar} finetunes their network on the given video recording for 3 hours on 8 A100 GPUs, making inference on consumer-grade devices impossible. In contrast, our approach directly regresses the final 3D head avatar from just four input images without the need for expensive test-time fine-tuning.


\section{Dataset Details}
\label{appendixdata}
The full list of songs we selected are detailed in Tables \ref{tab:en_list}, \ref{tab:zh_list}, \ref{tab:fr_list}, \ref{tab:ko_list}. The lyrics of these
songs will not be publicly released but will be available upon request for research purposes only, ensuring their appropriate use. The songs are selected from leaderboards on music platforms such as Apple Music, with release years ranging from 1930 to 2012. We manually ensure that all of the songs are copyright protected. U.S., China, France, and Korea are all signatories to the Berne Convention \cite{berne_convention}, which is an international treaty that ensures that works created in one member country are automatically protected by copyright in all other member countries without the need for formal registration \cite{ricketson2022international}. This means that a Chinese song is automatically protected by U.S. copyright law as soon as it is created and fixed in a tangible medium (e.g., recorded or written down).

\section{Experimental Methodology}
In this section, we describe the evaluated models, evaluation metrics, and implementation details of our experiments.


\begin{table*}[t]
\centering
\small
\begin{tabular}{l|c|ccccc|cc} 
\hline
\multirow{3}{*}{\textbf{Method}} & \multirow{3}{*}{\textbf{Parameters}} & \multicolumn{5}{c|}{\textbf{Question Answering}} & \multicolumn{2}{c}{\textbf{Fact Verification}} \\ \cline{3-9}
 & &  \textbf{TABMWP} & \textbf{WTQ} & \textbf{HiTab} & \textbf{TAT-QA} & \textbf{FeTaQA} & \textbf{TabFact} & \textbf{InfoTabs} \\ 
 & & (Acc.) & (Acc.) & (Acc.) & (Acc.) & (BLEU) & (Acc.) & (Acc.) \\ 
\hline
\multicolumn{9}{l}{{\cellcolor[rgb]{0.957,0.957,0.957}}\textit{LLM (Text)}} \\
Llama2 & 7B & 22.82 & 16.39 & 10.72 & 13.73 & 10.93 & 9.20 & 38.92 \\
TableLlama & 7B & 10.10 & 24.97 & 46.57 & 19.04 & 38.38 & 79.37 & 46.57 \\
Llama3-Instruct & 8B & 42.01 & 21.24 & 6.97 & 13.08 & 12.66 & 73.89 & 54.00 \\
\multicolumn{9}{l}{{\cellcolor[rgb]{0.957,0.957,0.957}}\textit{MLLM (Image)}} \\
MiniGPT-4 & 7B & 0.22 & 0.90 & 0.20 & 0.13 & 0.39 & 0 & 0.10 \\
Qwen-VL & 7B & 3.30 & 0.09 & 0.06 & 0.13 & 0.45 & 1.12 & 0.65 \\
InternLM-XComposer& 7B & 0.06 & 0.05 & 0.12 & 0.26 & 2.62 & 1.19 & 1.11 \\
mPLUG-Owl & 7B & 1.76 & 0.62 & 0.25 & 0.13 & 7.42 & 7.46 & 5.53 \\
mPLUG-Owl2 & 7B & 6.83 & 0.67 & 0.13 & 0.39 & 11.91 & 8.21 & 26.19 \\
LLaVA v1.5 & 7B & 6.05 & 1.24 & 2.03 & 2.97 & 8.24 & 18.9 & 28.31 \\
Vary-toy & 1.8B & 4.42 & 7.96 & 3.42 & 8.81 & 2.44 & 6.33 & 6.98 \\
Monkey & 7B & 13.26 & 19.07 & 6.41 & 12.31 & 3.41 & 22.56 & 22.11 \\
Table-LLaVA  & 7B & 57.78 & 18.43 & 10.09 & 12.82 & 25.60 & 59.85 & 65.26 \\
Table-LLaVA & 13B & 59.77 & 20.41 & 10.85 & 15.67 & 28.03 & 65.00 & 66.91 \\
MiniCPM-V-2.6 & 8B & 83.68  & 47.97 & 56.53 & 51.55 & 32.68 & 78.48 & 73.03 \\

\multicolumn{9}{l}{{\cellcolor[rgb]{0.957,0.957,0.957}}\textit{MLLM (Image \& Text)}} \\
Table-LLaVA & 13B & 84.58 & 39.89 & 46.00 & 29.27 & \textbf{33.50} & 69.93 & 74.88\\
MiniCPM-V-2.6 & 8B & \uline{86.06} & 52.30 & 58.56 & 52.46 & 32.96 & 79.31 & 73.18 \\
w/ Vanilla SFT & 8B & 76.69 & \uline{55.54} & \uline{62.88} & \uline{58.91} & 16.92 & \textbf{82.54} & \textbf{76.22} \\
w/ \method{}  & 8B & \textbf{87.50} & \textbf{55.77} & \textbf{63.00} & \textbf{60.75} & \uline{33.18} & \uline{82.27} & \uline{75.74} \\
\hline
\end{tabular}
\caption{Overall Performance on TQA and TFV Tasks. The \textbf{best} results are marked in bold, while the \uline{second-best} results are underlined. We establish baselines using LLM (Text) and MLLM (Image) by feeding unimodal table representations to language models. Next, we use image-based and text-based table representations as inputs to train various MLLM (Image \& Text) models, demonstrating the effectiveness of our \method{}.} \label{tab:overall}
\end{table*}
\textbf{Models. }
The models we use can be distinguished as proprietary and open-source models.

For proprietary models, we include GPT-4o~\cite{gpt4o}, the Claude family (Claude 3.5 Sonnet~\cite{Claude3.5_Sonnet}, Claude 3 Sonnet~\cite{Claude3}, Claude 3 Haiku~\cite{Claude3}) and the Gemini family (Gemini 1.5 Pro~\cite{geminiteam2024gemini15unlockingmultimodal}, Gemini 1.5 Flash~\cite{geminiteam2024gemini15unlockingmultimodal}).

For open-source models, we include the Llama-3.2-Vision family (Llama-3.2-11B-Vision-Instruct~\cite{Llama3.2}, Llama-3.2-90B-Vision-Instruct~\cite{Llama3.2}), the Phi-3 family (Phi-3.5-vision-instruct~\cite{abdin2024phi3technicalreporthighly}, Phi-3-vision-128k-instruct~\cite{abdin2024phi3technicalreporthighly}), MiniCPM-V 2.6~\cite{yao2024minicpmvgpt4vlevelmllm} , and Qwen-VL-Plus~\cite{bai2023qwenvlversatilevisionlanguagemodel}.

\textbf{Evaluation Metrics. }
We follow previous work~\cite{chen2021evaluatinglargelanguagemodels,li2024mmcode,wang2024intervenor,yang2024enhancing,luo2023wizardcoder} and we use Pass@$k$ ~\cite{chen2021evaluatinglargelanguagemodels} to evaluate the effectiveness of different MLLMs. Pass@$k$ represents the probability that at least one correct solution appears among the top $k$ generated solutions for each problem:
\begin{equation}
    \text{Pass@}k:=\underset{\text{Problems}}{\operatorname*{\mathbb{E}}}\left[1-\frac{\binom{n-c}k}{\binom nk}\right]
\end{equation}
where $n$ denotes the total number of generated solutions, $c$ is the number of correct solutions, and $k$ is the number of top-ranked solutions being evaluated. In this work, we set $k=1$.




\textbf{Implementation Details. } 
For all MLLMs, we set the generation temperature to 0.2, the nucleus sampling parameter $top\_p$ to 0.95, and the maximum generation length to 1024 tokens, the same as the other code generation work~\cite{hui2024qwen25codertechnicalreport,zheng2024opencodeinterpreter,guo2024deepseek,zhu2024deepseek}. For the proprietary models, we use the API endpoints provided by the respective vendors, and for the open-source models, we use the transformers\footnote{\url{https://huggingface.co/docs/transformers/index}} framework for inference. We use the 0-shot setting in our experiments. The proprietary models we used are gpt-4o-2024-08-06, claude-3-5-sonnet-20240620, claude-3-sonnet-20240229, claude-3-haiku-20240307, Gemini 1.5 Pro (May 2024), and Gemini 1.5 Flash (May 2024).




\section{Evaluation Results}


In this section, we first benchmark Multimodal LLMs on \benchmark. Subsequently, we compare the advantages of \benchmark over MMCode and how it can evaluate the reasoning capabilities of MLLMs from different perspectives compared to MathVista. Additionally, we show that MLLMs struggle to understand visually complex logic for code generation and analyze the reasons for the poor performance of open-source MLLMs on \benchmark. Finally, case studies are presented.




\subsection{Overall Performance}
Table \ref{tab: overall} shows the performance of the MLLMs on \benchmark dataset.


Overall, the proprietary models significantly outperform the open-source models. GPT-4o leads the proprietary models with an average score of 91.3, excelling in both the Algorithm and MATH categories. Following GPT-4o, Claude 3.5 Sonnet and Gemini 1.5 Pro also show robust results. In contrast, the open-source models lag behind in terms of overall performance, with Llama-3.2-90B-Vision-Ins scoring the highest among them at 36.9 on average. 
% Notably, this model performs comparatively well in the MATH easy category with a score of 80.0, demonstrating some strength in less complex tasks. However, its performance declines significantly in the hard categories of both Algorithm and MATH. 
The gap between proprietary and open-source models is especially evident in the Algorithm hard and MATH hard categories, where proprietary models, particularly GPT-4o and Claude 3.5 Sonnet, maintain relatively high scores compared to their open-source counterparts. This disparity suggests that while open-source models may handle simpler tasks effectively, they struggle with more complex problem-solving and reasoning challenges.

Additionally, we observe that the MLLMs score the lowest on the Algorithm subset on average, and all open-source models fail in all questions of Algorithm Hard. This suggests that the problems in the Algorithm subset are more challenging for the MLLMs, which demonstrates that open-source MLLMs still have a significant gap compared to proprietary models in more complex reasoning tasks.


\subsection{Comparision with Other Reasoning Benchmarks MMCode and MathVisita}
\begin{table}[]
\centering
% \captionsetup{justification=centering}
\resizebox{0.48\textwidth}{!}{
\begin{tabular}{l|cc|cc}
\hline
\multirow{2}{*}{\textbf{Model}} & \multicolumn{2}{c|}{\textbf{Code-Vision}}  & \multicolumn{2}{c}{\textbf{MMCode}}        \\
                                & \textbf{Text Only} & \textbf{Text + Image} & \textbf{Text Only} & \textbf{Text + Image} \\ \hline
GPT-4o                          & 24.8               & 87.9                  & 14.8               & 17.0                  \\
Gemini 1.5 Pro                  & 11.4               & 71.8                  & 5.7                & 5.0                   \\ \hline
\end{tabular}
}
\caption{Comparision with MMCode Benchmark. ``Text Only'' and ``Text + Image'' mean that the model input only contains problem description and the input contains a description of the problem and images respectively.}
\label{tab: compare_mmcode}
\vspace{-10pt}
\end{table}
\begin{figure*}[t] \centering
    \includegraphics[width=1\textwidth]{figures/images/model_performance_comparison.pdf}
    \caption{Comparision with MathVista Benchmark. All MLLMs perform similarly on MathVista but have large differences on \benchmark. The diff is the performance of the model on \benchmark minus the performance on MathVista. Detailed results are in appendix \ref{sec:detail_results}} \label{fig:compare_mathvista}
\end{figure*}
In this subsection, we show the advantages that \benchmark has over the other multimodal reasoning benchmarks MMCode and MathVisita.

\textbf{Compare with MMCode. }MMCode is a multimodal reasoning benchmark that evaluates algorithmic problem-solving skills in visually rich contexts. However, a major issue with MMCode is that most of its problems can be solved without using images, making it difficult to determine whether the model utilizes visual information in its responses, which limits its effectiveness in evaluating the reasoning abilities of MLLMs. \benchmark addresses this issue effectively by using flowcharts as a primary input type, making it challenging for models to provide correct answers without the images.

We explore model performance in both the ``Text Only'' and ``Text + Image'' modes to validate \benchmark’s reliance on visual information. As shown in Table ~\ref{tab: compare_mmcode}, on \benchmark, the scores of GPT-4o and Gemini 1.5 Pro in the ``Text Only'' mode are significantly lower than in the ``Text + Image'' mode, indicating a notable performance difference. This difference suggests that \benchmark’s problem design effectively encourages the use of multimodal information, making image input critical for problem-solving. In contrast, in MMCode, the score difference between the two modes is relatively small, indicating that most MMCode problems can be answered solely based on text information, without the need for images. This design limits MMCode’s ability to fully reflect the multimodal reasoning capabilities of MLLMs. By introducing complex image information, such as flowcharts, \benchmark requires models to rely on visual understanding to answer correctly, thus providing a more accurate assessment of the model's multimodal reasoning capabilities.


\textbf{Compare with MathVista. }MathVista is a multimodal mathematics reasoning benchmark. We select some proprietary models and open-source models with similar performance on MathVista to test their performance on \benchmark, to demonstrate the differences between \benchmark and multimodal mathematical reasoning datasets.

As shown in Figure ~\ref{fig:compare_mathvista}, for proprietary models, their performance on \benchmark and MathVista is similar. However, for open-source models, their performance on \benchmark is significantly lower than on MathVista, with a performance gap of around -30\%. This indicates that \benchmark can evaluate the reasoning abilities of MLLMs from different perspectives, such as algorithmic logic understanding, further revealing the shortcomings and limitations of these models in handling complex reasoning tasks.



\begin{table*}[t]
\centering
% \captionsetup{justification=centering}
\resizebox{\textwidth}{!}{
\begin{tabular}{lcccccc}
\hline
                                 & \multicolumn{2}{c}{\textbf{Easy}}              & \multicolumn{2}{c}{\textbf{Medium}} & \multicolumn{2}{c}{\textbf{Hard}} \\
\multirow{-2}{*}{\textbf{Model}} & Flowchart & Mermaid                            & Flowchart       & Mermaid           & Flowchart      & Mermaid          \\ \hline
\multicolumn{7}{c}{Proprietary Models}                                                                                                                      \\ \hline
GPT-4o                           & 91.1      &  88.9 {\color[HTML]{FE0000}(-2.2)} & 89.3            & 85.3 {\color[HTML]{FE0000}(-4.0)}       & 79.3           & 82.8 {\color[HTML]{009901}(+3.5)}      \\
Claude 3.5 Sonnet                & 84.4      & 82.2 {\color[HTML]{FE0000}(-2.2)}                        & 77.3            & 73.3 {\color[HTML]{FE0000}(-4.0)}       & 48.3           & 44.8 {\color[HTML]{FE0000}(-3.5)}      \\
Claude 3 Sonnet                  & 31.1      & 37.8 {\color[HTML]{009901}(+6.7)} & 14.7            & 20.0 {\color[HTML]{009901}(+5.3)}       & 3.4            & 17.2 {\color[HTML]{009901}(+13.8)}     \\
Claude 3 Haiku                   & 22.2      & 22.2 {\color[HTML]{009901}(+0.0)}                        & 4.0             & 10.7 {\color[HTML]{009901}(+6.7)}       & 0.0            & 6.9 {\color[HTML]{009901}(+6.9)}       \\
Gemini 1.5 Pro                   & 88.9      & 84.4 {\color[HTML]{FE0000}(-4.5)}                        & 72.0            & 73.3 {\color[HTML]{009901}(+1.3)}       & 44.8           & 51.7 {\color[HTML]{009901}(+6.9)}      \\
Gemini 1.5 Flash                 & 40.0      & 44.4 {\color[HTML]{009901}(+4.4)}                        & 32.0            & 44.0 {\color[HTML]{009901}(+12.0)}      & 10.3           & 20.7 {\color[HTML]{009901}(+10.4)}     \\ \hline
\multicolumn{7}{c}{Open-source Models}                                                                                                                      \\ \hline
Llama-3.2-11B-Vision-Instruct    & 8.9       & 28.9 {\color[HTML]{009901}(+20.0)}                       & 1.3             & 6.7 {\color[HTML]{009901}(+5.4)}        & 0.0            & 6.9 {\color[HTML]{009901}(+6.9)}       \\
Llama-3.2-90B-Vision-Instruct    & 17.8      & 28.9 {\color[HTML]{009901}(+11.1)}                       & 8.0             & 13.3 {\color[HTML]{009901}(+5.3)}       & 0.0            & 13.8 {\color[HTML]{009901}(+13.8)}     \\
Phi-3-vision-128k-instruct       & 15.6      & 31.1 {\color[HTML]{009901}(+15.5)}                       & 4.0             & 10.7 {\color[HTML]{009901}(+6.7)}       & 0.0            & 6.9 {\color[HTML]{009901}(+6.9)}       \\
Phi-3.5-vision-instruct          & 8.9       & 22.2 {\color[HTML]{009901}(+13.3)}                       & 0.0             & 9.3 {\color[HTML]{009901}(+9.3)}        & 0.0            & 3.5 {\color[HTML]{009901}(+3.4)}       \\
MiniCPM-V 2.6                    & 22.2      & 33.3 {\color[HTML]{009901}(+11.1)}                       & 9.3             & 18.7 {\color[HTML]{009901}(+9.4)}       & 0.0            & 6.9 {\color[HTML]{009901}(+6.9)}       \\
Qwen-VL-Plus                     & 11.1      & 31.1 {\color[HTML]{009901}(+20.0)}                       & 0.0             & 9.3 {\color[HTML]{009901}(+9.3)}        & 0.0            & 10.3 {\color[HTML]{009901}(+10.3)}     \\ \hline
\end{tabular}
}
\caption{Comparison of MLLMs When Flowchart or Mermaid Are Used as Input. We calculate the performance change ($Flowchart-Mermaid$), with {\color[HTML]{FE0000}red} representing a drop and {\color[HTML]{009901}green} representing a rise. Open-source models show substantial improvements with mermaid inputs. We use Algorithm subset.}
\label{tab: mermaid}
\end{table*}
\begin{figure*}[t] \centering
    % \vspace{15pt}
    \includegraphics[width=\textwidth]{figures/images/error_analysis.pdf}
    \caption{Error Analysis of Proprietary Models and Open-source Models. We count the proportion of each error type in the code generated by proprietary and open-source Models.} \label{fig:error_analysis}
\end{figure*}
\subsection{MLLMs Struggle to Understand Visually Complex Logic for Code Generation}
% 让模型根据图片生成和根据mermaid生成,比较一下结果。
In this subsection, we compare the performance of the MLLMs when flowchart or mermaid are used as input to demonstrate that MLLMs struggle to understand complex visual logic for code generation.

As shown in Table \ref{tab: mermaid}, when using mermaid as input, most models exhibit performance improvements. This suggests that the mermaid, being a more structured and textual representation, may be easier for MLLMs to interpret compared to flowcharts, which require the understanding of more complex visual elements. In particular, the performance of open-source models on medium and hard problems shows substantial improvements with mermaid inputs, this highlights the challenges MLLMs face in perceiving and processing visually complex logic, which is crucial for accurate code generation.
% This may indicate that the model has shortcomings in perceiving and understanding images




\subsection{Error Analysis and Case Studies}

Figure \ref{fig:error_analysis} shows the proportion of error types in the code generated by proprietary and open-source models across the three subsets of the \benchmark.

From the pie charts, we can observe that proprietary models have a significantly higher proportion of \texttt{AssertionError} compared to open-source models. In contrast, open-source models have a higher proportion of errors on \texttt{TypeError}, \texttt{IndentationError}, \texttt{SyntaxError}, \texttt{ValueError}, and \texttt{NameError}, with the highest proportion occurring in \texttt{SyntaxError}. This indicates that proprietary models adhere to basic syntax rules more effectively and are able to comprehend the logic within flowcharts. However, open-source models lack this refinement, their generated code has over a 10\% likelihood of failing to compile, indicating that their coding capabilities are poor, even to the point of failing to generate executable code. In our opinion, the programming ability of MLLMs is an essential skill for the path to AGI, as it represents a crucial intersection of logical reasoning, problem-solving,  and the ability to generate precise, executable instructions.

Finally, we conduct case studies to examine the performance of the models on the \benchmark. As shown in Figure \ref{fig:case}, the problem has two flowcharts and puts higher demands on the capabilities of MLLMs. GPT-4o correctly follows the logic depicted in the flowchart to generate the correct code. Llama-3.2-90B-Vision-Instruct understands the intent of the flowchart and uses a dynamic programming algorithm. MiniCPM-V-2\_6 omits two steps from the subgraph and Claude-3-Haiku-20240307 fails to correctly identify the function variables in both the main and sub-functions and does not properly understand the execution flow of the code in the subgraph. More details are shown in Appendix \ref{sec:cases}


% Finally, as shown in Figure \ref{fig:case}, we select a problem from the Math subset and present the responses from four MLLMs. This problem has two flowcharts and puts higher demands on the reasoning capabilities of MLLMs.

% Overall, GPT-4o and Llama-3.2-90B-Vision-Instruct provide correct answers, while MiniCPM-V-2\_6 and Claude-3-Haiku-20240307 gave incorrect ones. Upon careful analysis of the generated code, we can observe that GPT-4o correctly follows the logic depicted in the flowchart to generate the correct code. However, Llama-3.2-90B-Vision-Instruct did not follow the recursive logic as shown in the flowchart, instead using the dynamic programming algorithm. This is an interesting phenomenon, suggesting that Llama-3.2-90B-Vision-Instruct, after processing the flowchart, comprehends the problem that the algorithm in the chart is solving and provides a dynamic programming solution. This demonstrates that the model not only interprets the flowchart's surface meaning but also, to some extent, ``understands" the logical intent behind it.

% As clearly shown in the bottom of Figure \ref{fig:case}, MiniCPM-V-2\_6 fail to generate fully correct code because the model omitted two steps from the subgraph, although the other steps were correct, ultimately leading to an \texttt{AssertionError}. On the other hand, Claude-3-Haiku-20240307 fail to correctly identify the function variables in both the main and sub-functions and did not properly understand the execution flow of the code in the subgraph, ultimately leading to a \texttt{RecursionError}. This indicates that both models have poor image understanding and reasoning capabilities.
% \section{Case Study}\label{sec:appendix_casestudy}

We present cases across multiple coding datasets, comparing compressed and original code examples. For instance, as demonstrated in Figures 8, 9, and 10, \ourtool prioritizes discarding \textbf{Invocation} tokens first, followed by \textbf{Symbol} tokens.
\begin{figure}[!h]
\begin{tcolorbox}
\begin{lstlisting}[language=Java,frame=single,framerule=0pt]
### FOCAL_METHOD 
getProduction(java.lang.String) { 
 return productionsByName.get(name); }  
### UNIT_TEST  
testJustifications() { 
 runTest("testJustifications", 2); org.jsoar.kernel.Production j = agent.getProductions() .getProduction("justification-1"); "<AssertPlaceHolder>"; 
}    
\end{lstlisting}
\end{tcolorbox}
\caption{Original Code Examples of Assertion Generation (63 tokens)}
\label{fig:code-example}
\end{figure}

\begin{figure}[!h]
\begin{tcolorbox}
\begin{lstlisting}[language=Java,frame=single,framerule=0pt]
### FOCAL_METHOD 
getProduction(java.lang.String) { 
 return productionsByName; }  
### UNIT_TEST  
testJustifications() { 
 ; 
 org.jsoar.kernel.Production j = agent.getProductions() .getProduction("justification-1"); "<AssertPlaceHolder>"; 
}    
\end{lstlisting}
\end{tcolorbox}
\caption{Compressed Code Examples of Assertion Generation (55 tokens, $\tau_{code}$: 0.1)}
\label{fig:code-example}
\end{figure}

\begin{figure}[!h]
\begin{tcolorbox}
\begin{lstlisting}[language=Java,frame=single,framerule=0pt]
### FOCAL_METHOD 
getProduction(java.lang.String)  
 return productionsByName;     
### UNIT_TEST  
testJustifications()  
 ; 
 org.jsoar.kernel.Production j = agent;
  "<AssertPlaceHolder>"; 
\end{lstlisting}
\end{tcolorbox}
\caption{Compressed Code Examples of Assertion Generation (39 tokens, $\tau_{code}$: 0.4)}
\label{fig:code-example}
\end{figure}


\begin{figure}[!h]
\begin{tcolorbox}
\begin{lstlisting}[language=Java,frame=single,framerule=0pt]
### BUGGY_CODE 
public static TYPE_1 init(java.lang.String name, java.util.Date date) {
   TYPE_1 VAR_1 = new TYPE_1();
   VAR_1.METHOD_1(name);
   java.util.Calendar VAR_2 = java.util.Calendar.getInstance();
   VAR_2.METHOD_2(date);
   VAR_1.METHOD_3(VAR_2);
   return VAR_1;
}
### FIXED_CODE   
public static TYPE_1 init(java.lang.String name, java.util.Date date) {
   TYPE_1 VAR_1 = new TYPE_1();
   VAR_1.METHOD_1(name);
   java.util.Calendar VAR_2 = null;
   if (date != null) {
       VAR_2 = java.util.Calendar.getInstance();
       VAR_2.METHOD_2(date);
   } 
   VAR_1.METHOD_3(VAR_2);
   return VAR_1;
}
\end{lstlisting}
\end{tcolorbox}
\caption{Original Code Examples of Bugs2Fix (195 tokens)}
\label{fig:code-example}
\end{figure}

\begin{figure}[!h]
\begin{tcolorbox}
\begin{lstlisting}[language=Java,frame=single,framerule=0pt]
### BUGGY_CODE 
public static TYPE_1 init(java.lang.String name, java.util.Date date) {
    = new TYPE_1();
   ;
   java.util.Calendar = java.util.Calendar;
   .METHOD_2(date);
   .METHOD_3(VAR_2);
   return ;
}
### FIXED_CODE   
public static TYPE_1 init(java.lang.String name, java.util.Date date) {
    = new TYPE_1();
   ;
   java.util.Calendar = null;
   if (date != null) {
        = java.util.Calendar;
       .METHOD_2(date);
   } 
   .METHOD_3(VAR_2);
   return ;
}
\end{lstlisting}
\end{tcolorbox}
\caption{Compressed Code Examples of Bugs2Fix (136 tokens, $\tau_{code}$: 0.3)}
\label{fig:code-example}
\end{figure}

\begin{figure}[!h]
\begin{tcolorbox}
\begin{lstlisting}[language=Java,frame=single,framerule=0pt]
### METHOD_HEADER 
protected final void fastPathEmit ( U value , boolean delayError , Disposable dispose )
### WHOLE_METHOD  
protected final void fastPathEmit(U value, boolean delayError, Disposable dispose) {
   final Observer<? super V> s = actual;
   final SimplePlainQueue<U> q = queue;
   if (wip.get() == 0 && wip.compareAndSet(0, 1)) {
       accept(s, value);
       if (leave(-1) == 0) {
           return;
       }
   } else {
       q.offer(value);
       if (!enter()) {
           return;
       }
   }
   QueueDrainHelper.drainLoop(q, s, delayError, dispose, this);
}
\end{lstlisting}
\end{tcolorbox}
\caption{Original Code Examples  of \taskthree (157 tokens, $\tau_{code}$: 0.3)}
\label{fig:code-example}
\end{figure}


\begin{figure}[!h]
\begin{tcolorbox}
Original Code Examples (121 tokens, $\tau_{code}$: 0.3)
\begin{lstlisting}[language=Java,frame=single,framerule=0pt]
### METHOD_HEADER 
protected final void fastPathEmit ( U value , boolean delayError , Disposable dispose )
### WHOLE_METHOD  
   final Observer<? super V> = 
   final SimplePlainQueue<U> = 
   if (wip.get() == 0 && wip.compareAndSet(0, 1)) 
       ;
       if (leave(-1) == 0) 
           return;    
    else 
       .offer(value);
       if (!enter()) 
           return;
   .drainLoop(q, s, delayError, dispose, this);
\end{lstlisting}
\end{tcolorbox}
\caption{Compressed Code Examples  of \taskthree (121 tokens, $\tau_{code}$: 0.3)}
\end{figure}





\paragraph{Summary}
Our findings provide significant insights into the influence of correctness, explanations, and refinement on evaluation accuracy and user trust in AI-based planners. 
In particular, the findings are three-fold: 
(1) The \textbf{correctness} of the generated plans is the most significant factor that impacts the evaluation accuracy and user trust in the planners. As the PDDL solver is more capable of generating correct plans, it achieves the highest evaluation accuracy and trust. 
(2) The \textbf{explanation} component of the LLM planner improves evaluation accuracy, as LLM+Expl achieves higher accuracy than LLM alone. Despite this improvement, LLM+Expl minimally impacts user trust. However, alternative explanation methods may influence user trust differently from the manually generated explanations used in our approach.
% On the other hand, explanations may help refine the trust of the planner to a more appropriate level by indicating planner shortcomings.
(3) The \textbf{refinement} procedure in the LLM planner does not lead to a significant improvement in evaluation accuracy; however, it exhibits a positive influence on user trust that may indicate an overtrust in some situations.
% This finding is aligned with prior works showing that iterative refinements based on user feedback would increase user trust~\cite{kunkel2019let, sebo2019don}.
Finally, the propensity-to-trust analysis identifies correctness as the primary determinant of user trust, whereas explanations provided limited improvement in scenarios where the planner's accuracy is diminished.

% In conclusion, our results indicate that the planner's correctness is the dominant factor for both evaluation accuracy and user trust. Therefore, selecting high-quality training data and optimizing the training procedure of AI-based planners to improve planning correctness is the top priority. Once the AI planner achieves a similar correctness level to traditional graph-search planners, strengthening its capability to explain and refine plans will further improve user trust compared to traditional planners.

\paragraph{Future Research} Future steps in this research include expanding user studies with larger sample sizes to improve generalizability and including additional planning problems per session for a more comprehensive evaluation. Next, we will explore alternative methods for generating plan explanations beyond manual creation to identify approaches that more effectively enhance user trust. 
Additionally, we will examine user trust by employing multiple LLM-based planners with varying levels of planning accuracy to better understand the interplay between planning correctness and user trust. 
Furthermore, we aim to enable real-time user-planner interaction, allowing users to provide feedback and refine plans collaboratively, thereby fostering a more dynamic and user-centric planning process.

\section*{Limitations}
Our work mainly focuses on the position bias problem in the multi-constraint instruction following. We make a quantitative analysis of the influence brought by different constraint orders in the instructions. However, there are still some limitations. The constraints in our work are usually parallel to each other, which means the order change will not affect the semantic meaning of the instructions. The position bias problem for for those sequential constraints need to be further explored. Moreover, we only investigate the phenomenon of position bias in existing LLM without offering a solution. In further work, we will conduct a further probing task in sequential constraints to improve the generalization of our findings.


\section*{Ethics Statement}
Throughout the entire research process and in presenting the findings in this paper, we have maintained strict adherence to ethical standards. Our dataset is established based on commonly used datasets and authoritative platforms, and the relevant data and code have undergone stringent ethical review.




% Bibliography entries for the entire Anthology, followed by custom entries
%\bibliography{anthology,custom}
% Custom bibliography entries only
\bibliography{custom}

\clearpage
\newpage
\appendix

\clearpage
\pagenumbering{gobble}
\maketitlesupplementary

\section{Additional Results on Embodied Tasks}

To evaluate the broader applicability of our EgoAgent's learned representation beyond video-conditioned 3D human motion prediction, we test its ability to improve visual policy learning for embodiments other than the human skeleton.
Following the methodology in~\cite{majumdar2023we}, we conduct experiments on the TriFinger benchmark~\cite{wuthrich2020trifinger}, which involves a three-finger robot performing two tasks: reach cube and move cube. 
We freeze the pretrained representations and use a 3-layer MLP as the policy network, training each task with 100 demonstrations.

\begin{table}[h]
\centering
\caption{Success rate (\%) on the TriFinger benchmark, where each model's pretrained representation is fixed, and additional linear layers are trained as the policy network.}
\label{tab:trifinger}
\resizebox{\linewidth}{!}{%
\begin{tabular}{llcc}
\toprule
Methods       & Training Dataset & Reach Cube & Move Cube \\
\midrule
DINO~\cite{caron2021emerging}         & WT Venice        & 78.03     & 47.42     \\
DoRA~\cite{venkataramanan2023imagenet}          & WT Venice        & 81.62     & 53.76     \\
DoRA~\cite{venkataramanan2023imagenet}          & WT All           & 82.40     & 48.13     \\
\midrule
EgoAgent-300M & WT+Ego-Exo4D      & 82.61    & 54.21      \\
EgoAgent-1B   & WT+Ego-Exo4D      & \textbf{85.72}      & \textbf{57.66}   \\
\bottomrule
\end{tabular}%
}
\end{table}

As shown in Table~\ref{tab:trifinger}, EgoAgent achieves the highest success rates on both tasks, outperforming the best models from DoRA~\cite{venkataramanan2023imagenet} with increases of +3.32\% and +3.9\% respectively.
This result shows that by incorporating human action prediction into the learning process, EgoAgent demonstrates the ability to learn more effective representations that benefit both image classification and embodied manipulation tasks.
This highlights the potential of leveraging human-centric motion data to bridge the gap between visual understanding and actionable policy learning.



\section{Additional Results on Egocentric Future State Prediction}

In this section, we provide additional qualitative results on the egocentric future state prediction task. Additionally, we describe our approach to finetune video diffusion model on the Ego-Exo4D dataset~\cite{grauman2024ego} and generate future video frames conditioned on initial frames as shown in Figure~\ref{fig:opensora_finetune}.

\begin{figure}[b]
    \centering
    \includegraphics[width=\linewidth]{figures/opensora_finetune.pdf}
    \caption{Comparison of OpenSora V1.1 first-frame-conditioned video generation results before and after finetuning on Ego-Exo4D. Fine-tuning enhances temporal consistency, but the predicted pixel-space future states still exhibit errors, such as inaccuracies in the basketball's trajectory.}
    \label{fig:opensora_finetune}
\end{figure}

\subsection{Visualizations and Comparisons}

More visualizations of our method, DoRA, and OpenSora in different scenes (as shown in Figure~\ref{fig:supp pred}). For OpenSora, when predicting the states of $t_k$, we use all the ground truth frames from $t_{0}$ to $t_{k-1}$ as conditions. As OpenSora takes only past observations as input and neglects human motion, it performs well only when the human has relatively small motions (see top cases in Figure~\ref{fig:supp pred}), but can not adjust to large movements of the human body or quick viewpoint changes (see bottom cases in Figure~\ref{fig:supp pred}).

\begin{figure*}
    \centering
    \includegraphics[width=\linewidth]{figures/supp_pred.pdf}
    \caption{Retrieval and generation results for egocentric future state prediction. Correct and wrong retrieval images are marked with green and red boundaries, respectively.}
    \label{fig:supp pred}
\end{figure*}

\begin{figure*}[t]
    \centering
    \includegraphics[width=0.9\linewidth]{figures/motion_prediction.pdf}
    \vspace{-0.5mm}
    \caption{Motion prediction results in scenes with minor changes in observation.}
    \vspace{-1.5mm}
    \label{fig:motion_prediction}
\end{figure*}

\subsection{Finetuning OpenSora on Ego-Exo4D}

OpenSora V1.1~\cite{opensora}, initially trained on internet videos and images, produces severely inconsistent results when directly applied to infer future videos on the Ego-Exo4D dataset, as illustrated in Figure~\ref{fig:opensora_finetune}.
To address the gap between general internet content and egocentric video data, we fine-tune the official checkpoint on the Ego-Exo4D training set for 50 epochs.
OpenSora V1.1 proposed a random mask strategy during training to enable video generation by image and video conditioning. We adopted the default masking rate, which applies: 75\% with no masking, 2.5\% with random masking of 1 frame to 1/4 of the total frames, 2.5\% with masking at either the beginning or the end for 1 frame to 1/4 of the total frames, and 5\% with random masking spanning 1 frame to 1/4 of the total frames at both the beginning and the end.

As shown in Fig.~\ref{fig:opensora_finetune}, despite being trained on a large dataset, OpenSora struggles to generalize to the Ego-Exo4D dataset, producing future video frames with minimal consistency relative to the conditioning frame. While fine-tuning improves temporal consistency, the moving trajectories of objects like the basketball and soccer ball still deviate from realistic physical laws. Compared with our feature space prediction results, this suggests that training world models in a reconstructive latent space is more challenging than training them in a feature space.


\section{Additional Results on 3D Human Motion Prediction}

We present additional qualitative results for the 3D human motion prediction task, highlighting a particularly challenging scenario where egocentric observations exhibit minimal variation. This scenario poses significant difficulties for video-conditioned motion prediction, as the model must effectively capture and interpret subtle changes. As demonstrated in Fig.~\ref{fig:motion_prediction}, EgoAgent successfully generates accurate predictions that closely align with the ground truth motion, showcasing its ability to handle fine-grained temporal dynamics and nuanced contextual cues.

\section{OpenSora for Image Classification}

In this section, we detail the process of extracting features from OpenSora V1.1~\cite{opensora} (without fine-tuning) for an image classification task. Following the approach of~\cite{xiang2023denoising}, we leverage the insight that diffusion models can be interpreted as multi-level denoising autoencoders. These models inherently learn linearly separable representations within their intermediate layers, without relying on auxiliary encoders. The quality of the extracted features depends on both the layer depth and the noise level applied during extraction.


\begin{table}[h]
\centering
\caption{$k$-NN evaluation results of OpenSora V1.1 features from different layer depths and noising scales on ImageNet-100. Top1 and Top5 accuracy (\%) are reported.}
\label{tab:opensora-knn}
\resizebox{0.95\linewidth}{!}{%
\begin{tabular}{lcccccc}
\toprule
\multirow{2}{*}{Timesteps} & \multicolumn{2}{c}{First Layer} & \multicolumn{2}{c}{Middle Layer} & \multicolumn{2}{c}{Last Layer} \\
\cmidrule(r){2-3}   \cmidrule(r){4-5}  \cmidrule(r){6-7}  & Top1           & Top5           & Top1            & Top5           & Top1           & Top5          \\
\midrule
32        &  6.10           & 18.20             & 34.04               & 59.50             & 30.40             & 55.74             \\
64        & 6.12              & 18.48              & 36.04               & 61.84              & 31.80         & 57.06         \\
128       & 5.84             & 18.14             & 38.08               & 64.16              & 33.44       & 58.42 \\
256       & 5.60             & 16.58              & 30.34               & 56.38              &28.14          & 52.32        \\
512       & 3.66              & 11.70            & 6.24              & 17.62              & 7.24              & 19.44  \\ 
\bottomrule
\end{tabular}%
}
\end{table}

As shown in Table~\ref{tab:opensora-knn}, we first evaluate $k$-NN classification performance on the ImageNet-100 dataset using three intermediate layers and five different noise scales. We find that a noise timestep of 128 yields the best results, with the middle and last layers performing significantly better than the first layer.
We then test this optimal configuration on ImageNet-1K and find that the last layer with 128 noising timesteps achieves the best classification accuracy.

\section{Data Preprocess}
For egocentric video sequences, we utilize videos from the Ego-Exo4D~\cite{grauman2024ego} and WT~\cite{venkataramanan2023imagenet} datasets.
The original resolution of Ego-Exo4D videos is 1408×1408, captured at 30 fps. We sample one frame every five frames and use the original resolution to crop local views (224×224) for computing the self-supervised representation loss. For computing the prediction and action loss, the videos are downsampled to 224×224 resolution.
WT primarily consists of 4K videos (3840×2160) recorded at 60 or 30 fps. Similar to Ego-Exo4D, we use the original resolution and downsample the frame rate to 6 fps for representation loss computation.
As Ego-Exo4D employs fisheye cameras, we undistort the images to a pinhole camera model using the official Project Aria Tools to align them with the WT videos.

For motion sequences, the Ego-Exo4D dataset provides synchronized 3D motion annotations and camera extrinsic parameters for various tasks and scenes. While some annotations are manually labeled, others are automatically generated using 3D motion estimation algorithms from multiple exocentric views. To maximize data utility and maintain high-quality annotations, manual labels are prioritized wherever available, and automated annotations are used only when manual labels are absent.
Each pose is converted into the egocentric camera's coordinate system using transformation matrices derived from the camera extrinsics. These matrices also enable the computation of trajectory vectors for each frame in a sequence. Beyond the x, y, z coordinates, a visibility dimension is appended to account for keypoints invisible to all exocentric views. Finally, a sliding window approach segments sequences into fixed-size windows to serve as input for the model. Note that we do not downsample the frame rate of 3D motions.

\section{Training Details}
\subsection{Architecture Configurations}
In Table~\ref{tab:arch}, we provide detailed architecture configurations for EgoAgent following the scaling-up strategy of InternLM~\cite{team2023internlm}. To ensure the generalization, we do not modify the internal modules in InternML, \emph{i.e.}, we adopt the RMSNorm and 1D RoPE. We show that, without specific modules designed for vision tasks, EgoAgent can perform well on vision and action tasks.

\begin{table}[ht]
  \centering
  \caption{Architecture configurations of EgoAgent.}
  \resizebox{0.8\linewidth}{!}{%
    \begin{tabular}{lcc}
    \toprule
          & EgoAgent-300M & EgoAgent-1B \\
          \midrule
    Depth & 22    & 22 \\
    Embedding dim & 1024  & 2048 \\
    Number of heads & 8     & 16 \\
    MLP ratio &    8/3   & 8/3 \\
    $\#$param.  & 284M & 1.13B \\
    \bottomrule
    \end{tabular}%
    }
  \label{tab:arch}%
\end{table}%

Table~\ref{tab:io_structure} presents the detailed configuration of the embedding and prediction modules in EgoAgent, including the image projector ($\text{Proj}_i$), representation head/state prediction head ($\text{MLP}_i$), action projector ($\text{Proj}_a$) and action prediction head ($\text{MLP}_a$).
Note that the representation head and the state prediction head share the same architecture but have distinct weights.

\begin{table}[t]
\centering
\caption{Architecture of the embedding ($\text{Proj}_i$, $\text{Proj}_a$) and prediction ($\text{MLP}_i$, $\text{MLP}_a$) modules in EgoAgent. For details on module connections and functions, please refer to Fig.~2 in the main paper.}
\label{tab:io_structure}
\resizebox{\linewidth}{!}{%
\begin{tabular}{lcl}
\toprule
       & \multicolumn{1}{c}{Norm \& Activation} & \multicolumn{1}{c}{Output Shape}  \\
\midrule
\multicolumn{3}{l}{$\text{Proj}_i$ (\textit{Image projector})} \\
\midrule
Input image  & -          & 3$\times$224$\times$224 \\
Conv 2D (16$\times$16) & -       & Embedding dim$\times$14$\times$14    \\
\midrule
\multicolumn{3}{l}{$\text{MLP}_i$ (\textit{State prediction head} \& \textit{Representation head)}} \\
\midrule
Input embedding  & -          & Embedding dim \\
Linear & GELU       & 2048          \\
Linear & GELU       & 2048          \\
Linear & -          & 256           \\
Linear & -          & 65536     \\
\midrule
\multicolumn{3}{l}{$\text{Proj}_a$ (\textit{Action projector})} \\
\midrule
Input pose sequence  & -          & 4$\times$5$\times$17 \\
Conv 2D (5$\times$17) & LN, GELU   & Embedding dim$\times$1$\times$1    \\
\midrule
\multicolumn{3}{l}{$\text{MLP}_a$ (\textit{Action prediction head})} \\
\midrule
Input embedding  & -          & Embedding dim$\times$1$\times$1 \\
Linear & -          & 4$\times$5$\times$17     \\
\bottomrule
\end{tabular}%
}
\end{table}


\subsection{Training Configurations}
In Table~\ref{tab:training hyper}, we provide the detailed training hyper-parameters for experiments in the main manuscripts.

\begin{table}[ht]
  \centering
  \caption{Hyper-parameters for training EgoAgent.}
  \resizebox{0.86\linewidth}{!}{%
    \begin{tabular}{lc}
    \toprule
    Training Configuration & EgoAgent-300M/1B \\
    \midrule
    Training recipe: &  \\
    optimizer & AdamW~\cite{loshchilov2017decoupled} \\
    optimizer momentum & $\beta_1=0.9, \beta_2=0.999$ \\
    \midrule
    Learning hyper-parameters: &  \\
    base learning rate & 6.0E-04 \\
    learning rate schedule & cosine \\
    base weight decay & 0.04 \\
    end weight decay & 0.4 \\
    batch size & 1920 \\
    training iters & 72,000 \\
    lr warmup iters & 1,800 \\
    warmup schedule & linear \\
    gradient clip & 1.0 \\
    data type & float16 \\
    norm epsilon & 1.0E-06 \\
    \midrule
    EMA hyper-parameters: &  \\
    momentum & 0.996 \\
    \bottomrule
    \end{tabular}%
    }
  \label{tab:training hyper}%
\end{table}%

\clearpage


\end{document}
