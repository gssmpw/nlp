
\section{\benchmark}
% \vspace{-2pt}
In this section, we introduce the benchmark \benchmark, which evaluates the logical understanding and code generation abilities of Multimodal Large Language Models (MLLMs). We first describe how \benchmark tests MLLMs (Sec.~\ref{dataset:definition}). Then we detail the process of constructing the \benchmark benchmark (Sec.~\ref{dataset:collection}) and show the statistics of \benchmark(Sec.~\ref{dataset:sta}).


\subsection{Task Definition}\label{dataset:definition}
% 任务定义

In \benchmark, MLLMs require generating a correct program \( P \) that fulfills specific functionality requirements based on a given flowchart \( F \), which visually represents the desired algorithm or process. 

Generating code involves translating visual or descriptive elements from the image into logical and structured code that meets the required specifications. After generating the code \( P \), we evaluate its correctness by using a set of test cases \( X = \{(x_1, y_1), \ldots, (x_n, y_n)\} \). Specifically, we provide input \( x_i \) to the generated code \( P \) and obtain the execution result \( P(x_i) \). If there exists any test case \( (x_i, y_i) \in X \) that satisfies \( P(x_j) \neq y_j \), it indicates that the generated code \( P \) does not fulfill the required functionality. If the MLLM generates a program \( P \) that correctly implements the specifications derived from the flowchart \( F \) for all test cases \( \forall (x_i, y_i) \in X, P(x_i) = y_i \), it demonstrates the MLLM's capability to accurately interpret and convert visual algorithmic instructions into executable code, ensuring reliability and correctness in its outputs. This successful translation confirms the model's understanding of both the structure and logic inherent in the flowchart, marking an important step towards advanced reasoning abilities.



\subsection{Data Construction}\label{dataset:collection}
% 数据收集步骤,包含Data Collection,Flowchart Construction,Test Cases Generation
In this subsection, we outline the methodology behind the construction of the \benchmark dataset. As shown in Figure ~\ref{fig:construction}, it includes Data Collection, Flowchart Construction, and Test Cases Generation.

\textbf{Data Collection.} \benchmark consists of three subsets: HumanEval-V, Algorithm, and MATH, which together provide a comprehensive evaluation of MLLMs' reasoning abilities across three aspect: basic programming, algorithm, and math. To ensure the quality and diversity of \benchmark, we collect data from the LeetCode website\footnote{\url{https://leetcode.com/}} as well as some open source high-quality data such as HumanEval~\cite{chen2021evaluatinglargelanguagemodels}.

HumanEval-V is collected from HumanEval, which is a code generation benchmark specifically designed to assess the capability of LLMs in generating correct and efficient Python code based on natural language prompts. We store all the data from HumanEval in \benchmark. Algorithm and MATH subsets are constructed by carefully curating a variety of problems from the LeetCode website that target specific reasoning skills necessary for proficient programming and mathematical problem-solving. We categorize the problems into three levels of difficulty: easy, medium, and hard, to further refine the assessment of the model's capabilities. In the Algorithm subset, problems at each level cover algorithms such as data structure operations and dynamic programming. In the MATH subset, the problems encompass mathematical concepts such as arithmetic operations, algebraic knowledge, and symbolic manipulation.

\begin{table*}[!t]
    \centering
    \resizebox{0.8\textwidth}{!}{
    \begin{tabular}{@{}l|c|c|c|c|c||c@{}}
        \toprule
        & \makecell{MATRES} & \makecell{TB-Dense} & \makecell{TCR} & \makecell{TDD-Manual} & \makecell{NarrativeTime} & \makecell{\textbf{\App{}}} \\
        \midrule
        \multicolumn{7}{c}{\textbf{Datasets Statistics}} \\
        \midrule
        Documents & 275 & 36 & 25 & 34 & 36 & 30 \\
        Events & 6,099 & 1,498 & 1,134 & 1,101 & 1,715 & 470 \\
        \midrule
        \textit{before} & 6,852 (50) & 1,361 (21) & 1,780 (67) & 1,561 (25) & 17,011 (22) & 1,540 (44) \\
        \textit{after} & 4,752 (35) & 1,182 (19) & 862 (33) & 1,054 (17) & 18,366 (23) & 1,347 (39) \\
        \textit{equal} & 448 (4) & 237 (4) & 4 (0) & 140 (2) & 5,298 (7) & 150 (4) \\
        \textit{vague} & 1,525 (11) & 2,837 (45) & -- & -- & 25,679 (33) & 446 (13) \\
        \textit{includes} & -- & 305 (5) & -- & 2,008 (33) & 5,781 (7) & -- \\
        \textit{is-included} & -- & 383 (6) & -- & 1,387 (23) & 6,639 (8) & -- \\
        \textit{overlaps} & -- & -- & -- & -- & 227 (0) & -- \\
        \midrule
        Total Relations & 13,577 & 6,305 & 2,646 & 6,150 & 79,001 & 3,483 \\
        \midrule
        \multicolumn{7}{c}{\textbf{Per Document Average Annotation Sparsity}} \\
        \midrule
        Events & 22.2 & 41.6 & 45.4 & 32.4 & 47.6 & 15.6 \\
        Actual Relations & 49.4 & 183.7 & 105.8 & 180.9 & 1,110.1 & 114.9 \\
        Expected Relations & 234.8 & 844.5 & 1,006.1 & 508.1 & 1,110.1 & 114.9 \\
        \midrule
        Missing Relations & 79\% & 78.3\% & 89.5\% & 64.4\% & 0\% & 0\% \\
        \bottomrule
    \end{tabular}}
    \caption{The upper part of the table presents the statistics of notable datasets for the temporal relation extraction task alongside \App{}. In parentheses, the values indicate the percentage of each relation type relative to the total relations in the dataset. The bottom part of the table summarizes the average percentage of missing relations per document, calculated as the ratio of actual annotated relations to a complete relation coverage, referred to as \textit{Expected Relations}.}
    \label{tab:stats_all}
\end{table*}


% \begin{table*}[!t]
%     \centering
%     \resizebox{0.8\textwidth}{!}{
%     \begin{tabular}{@{}l|c|c|c|c|c|c@{}}
%         \toprule
%         & \makecell{MATRES} & \makecell{TBD} & \makecell{TCR} & \makecell{TDD-Man} & \makecell{NarrativeTime} & \makecell{\App{}} \\
%         \midrule
%         Docs & 275 & 36 & 25 & 34 & 36 & 30 \\
%         Events & 6,099 & 1,498 & 1,134 & 1,101 & 1,715 & 470 \\
%         \midrule
%         Before (\%) & 6,852 (50) & 1,361 (21) & 1,780 (67) & 1,561 (25) & 17,011 (22) & 1,540 (44) \\
%         After (\%) & 4,752 (35) & 1,182 (19) & 862 (33) & 1,054 (17) & 18,366 (23) & 1,347 (39) \\
%         Equal (\%) & 448 (4) & 237 (4) & 4 (0) & 140 (2) & 5,298 (7) & 150 (4) \\
%         Vague (\%) & 1,525 (11) & 2,837 (45) & -- & -- & 25,679 (33) & 446 (13) \\
%         Includes (\%) & -- & 305 (5) & -- & 2,008 (33) & 5,781 (7) & -- \\
%         IsIncluded (\%) & -- & 383 (6) & -- & 1,387 (23) & 6,639 (8) & -- \\
%         Overlaps (\%) & -- & -- & -- & -- & 227 (0) & -- \\
%         \midrule
%         Total Rels & 13,577 & 6,305 & 2,646 & 6,150 & 79,001 & 3,483 \\
%         \bottomrule
%     \end{tabular}}
%     \caption{Statistics of notable datasets for the temporal relation extraction task.}
%     \label{tab:stats}
% \end{table*}



\textbf{Flowchart Construction. } The most important part of \benchmark is to build a clear flow chart that accurately depicts the logical structure of the problem and the steps to solve it. Therefore, when building the flowchart, we use a two-step method of creating a mermaid and then rendering it into a visual flowchart to ensure that the flowchart accurately reflects the logic of the problem.

We first use GPT-4o to generate the mermaid from the problem descriptions and correct code solutions. The prompt we used is in Figure ~\ref{fig:prompt}. Mermaid is a text-based flowchart representation that allows us to describe diagrams and visual data flows in a straightforward manner. After generating the mermaid, we utilize an automated rendering tool\footnote{\url{https://www.mermaidchart.com/}} to render it into a high-quality visualization flowchart. During the process of rendering the flowchart, we manually review the automatically generated flowchart to ensure its compliance and logicality. For those flowcharts whose problem description is not clear and logic has problems, we conduct manual correction.

\definecolor{titlecolor}{rgb}{0.9, 0.5, 0.1}
\definecolor{anscolor}{rgb}{0.2, 0.5, 0.8}
\definecolor{labelcolor}{HTML}{48a07e}
\begin{table*}[h]
	\centering
	
 % \vspace{-0.2cm}
	
	\begin{center}
		\begin{tikzpicture}[
				chatbox_inner/.style={rectangle, rounded corners, opacity=0, text opacity=1, font=\sffamily\scriptsize, text width=5in, text height=9pt, inner xsep=6pt, inner ysep=6pt},
				chatbox_prompt_inner/.style={chatbox_inner, align=flush left, xshift=0pt, text height=11pt},
				chatbox_user_inner/.style={chatbox_inner, align=flush left, xshift=0pt},
				chatbox_gpt_inner/.style={chatbox_inner, align=flush left, xshift=0pt},
				chatbox/.style={chatbox_inner, draw=black!25, fill=gray!7, opacity=1, text opacity=0},
				chatbox_prompt/.style={chatbox, align=flush left, fill=gray!1.5, draw=black!30, text height=10pt},
				chatbox_user/.style={chatbox, align=flush left},
				chatbox_gpt/.style={chatbox, align=flush left},
				chatbox2/.style={chatbox_gpt, fill=green!25},
				chatbox3/.style={chatbox_gpt, fill=red!20, draw=black!20},
				chatbox4/.style={chatbox_gpt, fill=yellow!30},
				labelbox/.style={rectangle, rounded corners, draw=black!50, font=\sffamily\scriptsize\bfseries, fill=gray!5, inner sep=3pt},
			]
											
			\node[chatbox_user] (q1) {
				\textbf{System prompt}
				\newline
				\newline
				You are a helpful and precise assistant for segmenting and labeling sentences. We would like to request your help on curating a dataset for entity-level hallucination detection.
				\newline \newline
                We will give you a machine generated biography and a list of checked facts about the biography. Each fact consists of a sentence and a label (True/False). Please do the following process. First, breaking down the biography into words. Second, by referring to the provided list of facts, merging some broken down words in the previous step to form meaningful entities. For example, ``strategic thinking'' should be one entity instead of two. Third, according to the labels in the list of facts, labeling each entity as True or False. Specifically, for facts that share a similar sentence structure (\eg, \textit{``He was born on Mach 9, 1941.''} (\texttt{True}) and \textit{``He was born in Ramos Mejia.''} (\texttt{False})), please first assign labels to entities that differ across atomic facts. For example, first labeling ``Mach 9, 1941'' (\texttt{True}) and ``Ramos Mejia'' (\texttt{False}) in the above case. For those entities that are the same across atomic facts (\eg, ``was born'') or are neutral (\eg, ``he,'' ``in,'' and ``on''), please label them as \texttt{True}. For the cases that there is no atomic fact that shares the same sentence structure, please identify the most informative entities in the sentence and label them with the same label as the atomic fact while treating the rest of the entities as \texttt{True}. In the end, output the entities and labels in the following format:
                \begin{itemize}[nosep]
                    \item Entity 1 (Label 1)
                    \item Entity 2 (Label 2)
                    \item ...
                    \item Entity N (Label N)
                \end{itemize}
                % \newline \newline
                Here are two examples:
                \newline\newline
                \textbf{[Example 1]}
                \newline
                [The start of the biography]
                \newline
                \textcolor{titlecolor}{Marianne McAndrew is an American actress and singer, born on November 21, 1942, in Cleveland, Ohio. She began her acting career in the late 1960s, appearing in various television shows and films.}
                \newline
                [The end of the biography]
                \newline \newline
                [The start of the list of checked facts]
                \newline
                \textcolor{anscolor}{[Marianne McAndrew is an American. (False); Marianne McAndrew is an actress. (True); Marianne McAndrew is a singer. (False); Marianne McAndrew was born on November 21, 1942. (False); Marianne McAndrew was born in Cleveland, Ohio. (False); She began her acting career in the late 1960s. (True); She has appeared in various television shows. (True); She has appeared in various films. (True)]}
                \newline
                [The end of the list of checked facts]
                \newline \newline
                [The start of the ideal output]
                \newline
                \textcolor{labelcolor}{[Marianne McAndrew (True); is (True); an (True); American (False); actress (True); and (True); singer (False); , (True); born (True); on (True); November 21, 1942 (False); , (True); in (True); Cleveland, Ohio (False); . (True); She (True); began (True); her (True); acting career (True); in (True); the late 1960s (True); , (True); appearing (True); in (True); various (True); television shows (True); and (True); films (True); . (True)]}
                \newline
                [The end of the ideal output]
				\newline \newline
                \textbf{[Example 2]}
                \newline
                [The start of the biography]
                \newline
                \textcolor{titlecolor}{Doug Sheehan is an American actor who was born on April 27, 1949, in Santa Monica, California. He is best known for his roles in soap operas, including his portrayal of Joe Kelly on ``General Hospital'' and Ben Gibson on ``Knots Landing.''}
                \newline
                [The end of the biography]
                \newline \newline
                [The start of the list of checked facts]
                \newline
                \textcolor{anscolor}{[Doug Sheehan is an American. (True); Doug Sheehan is an actor. (True); Doug Sheehan was born on April 27, 1949. (True); Doug Sheehan was born in Santa Monica, California. (False); He is best known for his roles in soap operas. (True); He portrayed Joe Kelly. (True); Joe Kelly was in General Hospital. (True); General Hospital is a soap opera. (True); He portrayed Ben Gibson. (True); Ben Gibson was in Knots Landing. (True); Knots Landing is a soap opera. (True)]}
                \newline
                [The end of the list of checked facts]
                \newline \newline
                [The start of the ideal output]
                \newline
                \textcolor{labelcolor}{[Doug Sheehan (True); is (True); an (True); American (True); actor (True); who (True); was born (True); on (True); April 27, 1949 (True); in (True); Santa Monica, California (False); . (True); He (True); is (True); best known (True); for (True); his roles in soap operas (True); , (True); including (True); in (True); his portrayal (True); of (True); Joe Kelly (True); on (True); ``General Hospital'' (True); and (True); Ben Gibson (True); on (True); ``Knots Landing.'' (True)]}
                \newline
                [The end of the ideal output]
				\newline \newline
				\textbf{User prompt}
				\newline
				\newline
				[The start of the biography]
				\newline
				\textcolor{magenta}{\texttt{\{BIOGRAPHY\}}}
				\newline
				[The ebd of the biography]
				\newline \newline
				[The start of the list of checked facts]
				\newline
				\textcolor{magenta}{\texttt{\{LIST OF CHECKED FACTS\}}}
				\newline
				[The end of the list of checked facts]
			};
			\node[chatbox_user_inner] (q1_text) at (q1) {
				\textbf{System prompt}
				\newline
				\newline
				You are a helpful and precise assistant for segmenting and labeling sentences. We would like to request your help on curating a dataset for entity-level hallucination detection.
				\newline \newline
                We will give you a machine generated biography and a list of checked facts about the biography. Each fact consists of a sentence and a label (True/False). Please do the following process. First, breaking down the biography into words. Second, by referring to the provided list of facts, merging some broken down words in the previous step to form meaningful entities. For example, ``strategic thinking'' should be one entity instead of two. Third, according to the labels in the list of facts, labeling each entity as True or False. Specifically, for facts that share a similar sentence structure (\eg, \textit{``He was born on Mach 9, 1941.''} (\texttt{True}) and \textit{``He was born in Ramos Mejia.''} (\texttt{False})), please first assign labels to entities that differ across atomic facts. For example, first labeling ``Mach 9, 1941'' (\texttt{True}) and ``Ramos Mejia'' (\texttt{False}) in the above case. For those entities that are the same across atomic facts (\eg, ``was born'') or are neutral (\eg, ``he,'' ``in,'' and ``on''), please label them as \texttt{True}. For the cases that there is no atomic fact that shares the same sentence structure, please identify the most informative entities in the sentence and label them with the same label as the atomic fact while treating the rest of the entities as \texttt{True}. In the end, output the entities and labels in the following format:
                \begin{itemize}[nosep]
                    \item Entity 1 (Label 1)
                    \item Entity 2 (Label 2)
                    \item ...
                    \item Entity N (Label N)
                \end{itemize}
                % \newline \newline
                Here are two examples:
                \newline\newline
                \textbf{[Example 1]}
                \newline
                [The start of the biography]
                \newline
                \textcolor{titlecolor}{Marianne McAndrew is an American actress and singer, born on November 21, 1942, in Cleveland, Ohio. She began her acting career in the late 1960s, appearing in various television shows and films.}
                \newline
                [The end of the biography]
                \newline \newline
                [The start of the list of checked facts]
                \newline
                \textcolor{anscolor}{[Marianne McAndrew is an American. (False); Marianne McAndrew is an actress. (True); Marianne McAndrew is a singer. (False); Marianne McAndrew was born on November 21, 1942. (False); Marianne McAndrew was born in Cleveland, Ohio. (False); She began her acting career in the late 1960s. (True); She has appeared in various television shows. (True); She has appeared in various films. (True)]}
                \newline
                [The end of the list of checked facts]
                \newline \newline
                [The start of the ideal output]
                \newline
                \textcolor{labelcolor}{[Marianne McAndrew (True); is (True); an (True); American (False); actress (True); and (True); singer (False); , (True); born (True); on (True); November 21, 1942 (False); , (True); in (True); Cleveland, Ohio (False); . (True); She (True); began (True); her (True); acting career (True); in (True); the late 1960s (True); , (True); appearing (True); in (True); various (True); television shows (True); and (True); films (True); . (True)]}
                \newline
                [The end of the ideal output]
				\newline \newline
                \textbf{[Example 2]}
                \newline
                [The start of the biography]
                \newline
                \textcolor{titlecolor}{Doug Sheehan is an American actor who was born on April 27, 1949, in Santa Monica, California. He is best known for his roles in soap operas, including his portrayal of Joe Kelly on ``General Hospital'' and Ben Gibson on ``Knots Landing.''}
                \newline
                [The end of the biography]
                \newline \newline
                [The start of the list of checked facts]
                \newline
                \textcolor{anscolor}{[Doug Sheehan is an American. (True); Doug Sheehan is an actor. (True); Doug Sheehan was born on April 27, 1949. (True); Doug Sheehan was born in Santa Monica, California. (False); He is best known for his roles in soap operas. (True); He portrayed Joe Kelly. (True); Joe Kelly was in General Hospital. (True); General Hospital is a soap opera. (True); He portrayed Ben Gibson. (True); Ben Gibson was in Knots Landing. (True); Knots Landing is a soap opera. (True)]}
                \newline
                [The end of the list of checked facts]
                \newline \newline
                [The start of the ideal output]
                \newline
                \textcolor{labelcolor}{[Doug Sheehan (True); is (True); an (True); American (True); actor (True); who (True); was born (True); on (True); April 27, 1949 (True); in (True); Santa Monica, California (False); . (True); He (True); is (True); best known (True); for (True); his roles in soap operas (True); , (True); including (True); in (True); his portrayal (True); of (True); Joe Kelly (True); on (True); ``General Hospital'' (True); and (True); Ben Gibson (True); on (True); ``Knots Landing.'' (True)]}
                \newline
                [The end of the ideal output]
				\newline \newline
				\textbf{User prompt}
				\newline
				\newline
				[The start of the biography]
				\newline
				\textcolor{magenta}{\texttt{\{BIOGRAPHY\}}}
				\newline
				[The ebd of the biography]
				\newline \newline
				[The start of the list of checked facts]
				\newline
				\textcolor{magenta}{\texttt{\{LIST OF CHECKED FACTS\}}}
				\newline
				[The end of the list of checked facts]
			};
		\end{tikzpicture}
        \caption{GPT-4o prompt for labeling hallucinated entities.}\label{tb:gpt-4-prompt}
	\end{center}
\vspace{-0cm}
\end{table*}




\textbf{Test Cases Generation. }
Test cases are crucial to validate the correctness and robustness of the code generated by MLLMs. For HumanEval-V, we utilize the test cases provided in HumanEval. For Algorithm, we utilize the test cases from LeetCode dataset\cite{guo2024deepseekcoderlargelanguagemodel}. For MATH subset which do not provide private test cases, we generate a comprehensive set of test cases that cover:
\begin{itemize}
    \item Typical Cases: Standard input scenarios that test the general functionality of the code.
    \item Edge Cases: Boundary conditions that might result in unusual outputs, such as extreme values or zero inputs.
    \item Large Number Cases: These cases involve inputs with significantly large values or large datasets to test the efficiency and performance scalability of the generated code. 
\end{itemize}

For each category, we generate three test case inputs separately. Subsequently, we use the correct code to execute based on the input and get the correct output. We keep these correct input-output pairs as test cases.





\subsection{Data Statistics}\label{dataset:sta}
In this subsection, we show the statistics of \benchmark.


As shown in Table \ref{tab: CodeVision stats}, the \benchmark benchmark consists of three subsets: HumanEval-V, Algorithm, and MATH, covering basic programming, algorithmic, and mathematical problem-solving domains, and difficulty levels (Easy, Medium, and Hard). HumanEval-V contains 164 problems with an average of 8.08 test cases and moderate flowchart complexity (10.90 nodes, 11.07 edges). The Algorithm subset comprises 149 problems across three difficulty levels with 100 test cases each, showing increasing flowchart complexity from Easy (12.62 nodes, 13.29 edges) to Hard (18.86 nodes, 19.14 edges) problems. Similarly, the MATH subset includes 125 problems with an average of 8.92 test cases, demonstrating comparable complexity progression from Easy (10.67 nodes, 10.49 edges) to Hard (19.20 nodes, 19.08 edges) problems, indicating that problem difficulty correlates with flowchart complexity across both Algorithm and MATH subsets.


