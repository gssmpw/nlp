\section{General Unnormalized Model}\label{sec:unnormalized-setting}


\begin{algorithm}[t]
    \caption{\newsolution:Initialize($d_x,d_y,l,N,R,\epsilon$)}
    \label{alg:initialize-ml}
    \DontPrintSemicolon
    \KwInput{The dimension of $\boldsymbol{X}$ and $\boldsymbol{Y}$: $d_x$ and $d_y$ respectively, number of columns in the \oursolution sketch $l = min(\lceil \frac{1}{\epsilon}\rceil,d_x,d_y)$, the length of sliding window $N$, upper bound of squared norms: $R$, error parameter $\epsilon$}

    $L \leftarrow \lceil log_2{R} \rceil$; $M \leftarrow $ an empty list

    \For{$i = 0,\cdots,L-1$}
    {
        $M.\textit{append}(\text{\oursolution}.\textit{Initialize}(d_x,d_y,l,N,2^i\epsilon N))$
    }
    
\end{algorithm}
    
        

In this section, we extend \oursolution to handle unnormalized data, where the squared norms of the input vectors satisfy \( \|\boldsymbol{x}_i\|_2^2,\;\|\boldsymbol{y}_i\|_2^2 \in [1,R] \). Following the approach in \cite{LeeT06} for mutable sliding window sizes, we construct a multi-layer extension of \oursolution, which we denote by \newsolution (Multi-Layer SO-COD). In this framework, the data stream is processed through \( L = \lceil \log_2 R \rceil \) layers. Each layer \( i \) (for \( i = 0,1,\dots,L-1 \)) maintains an independent \oursolution sketch with a distinct register threshold \( \theta_i \) defined by \( \theta_i = 2^i\epsilon N \). That is, in layer \( i \), a snapshot is registered when \( \|\hat{\boldsymbol{a}}_i\|_2 \,\|\hat{\boldsymbol{b}}_i\|_2 \ge 2^i\epsilon N \). By constraining the number of snapshots per layer to \( O(1/\epsilon) \), the overall space complexity of \newsolution is \( O((d_x+d_y)/\epsilon \cdot \log_2 R) \).



\subsection{Algorithm Description}
    \htitle{Multi-Layer Sketch Structure.}
Algorithm~\ref{alg:initialize-ml} specifies the initialization of a \newsolution sketch, thereby defining the multi-layer structure. In Line 1, we compute the number of layers \( L = \lceil \log_2 R \rceil \). For each layer \( i \in \{0, 1, \dots, L-1\} \), an independent \oursolution sketch is initialized with the register threshold \( \theta_i = 2^i\epsilon N \) and added to the list $M$. Consequently, in layer \( i \), the \oursolution sketch will generate a snapshot when \( \|\hat{\boldsymbol{a}}_i\|_2 \,\|\hat{\boldsymbol{b}}_i\|_2 \ge 2^i\epsilon N \). To minimize redundancy and retain only the most significant information, we restrict the number of snapshots per layer to \( O(1/\epsilon) \). As a result, the total space required by \newsolution is \( O((d_x+d_y)/\epsilon \cdot \log_2 R) \).

    \begin{algorithm}[t]
    \caption{\newsolution:Update($\boldsymbol{x}_i,\boldsymbol{y}_i$)}
    \label{alg:update-ml}
    \DontPrintSemicolon
    \KwInput{$\boldsymbol{x}_i$: the column vector of $\boldsymbol{X}$ arriving at timestamp $i$;
    $\boldsymbol{y}_i$: the column vector of $\boldsymbol{Y}$ arriving at timestamp $i$}

    \For{$j = 0,\cdots,L-1$}
    {
        \While{$len(M[j].S)> \lsn$ or $M[j].S[0].t \leq i-N$}
        {
            $M[j].S.POPLEFT()$
        }
        \If{$\left\|\boldsymbol{x}_i\right\|_2\left\|\boldsymbol{y}_i\right\|_2 \geq 2^j\epsilon N$}
        {
            $M[j].S$ append snapshot ($\boldsymbol{u}=\boldsymbol{x}_i$, $\boldsymbol{v}=\boldsymbol{y}_i$, $s=M[j].S[-1].t$, $t = i$)
            
            $M[j].S'$  append snapshot ($\boldsymbol{u} = \boldsymbol{x}_i$, $\boldsymbol{v} = \boldsymbol{y}_i$, $s=M[j].S'[-1].t$, $t = i$)
        }
        \Else
        {
            $M[j].\textit{FastUpdate}(\boldsymbol{x}_i,\boldsymbol{y}_i)$
        }
    }
    
\end{algorithm}
    
        
    

\htitle{Update Algorithm.} As shown in Alg.\ \ref{alg:update-ml}, each incoming column pair \( (\boldsymbol{x}_i,\boldsymbol{y}_i) \) is processed across all layers. To ensure that the number of snapshots in each layer remains within the prescribed bound of \( O(1/\epsilon) \), we explicitly set this bound to \( \lsn \); that is, the algorithm first checks whether the number of snapshots exceeds \( \lsn \) (and prunes expired snapshots promptly, as indicated in Lines 2--3). Then, for each layer \( j \), if the product of $\xinorm$ and $\yinorm$ exceed the threshold $2^j\epsilon N$, we  generate a snapshot $(u=\boldsymbol{x}_i,v=\boldsymbol{y}_i)$ directly, preserving all information of $(\boldsymbol{x}_i,\boldsymbol{y}_i)$ without introducing approximation error for $\boldsymbol{x}_i\boldsymbol{y}_i^T$ to reduce additional calculation (Lines 4 to 6). Otherwise, the update procedure applies the column pair \( (\boldsymbol{x}_i,\boldsymbol{y}_i) \) to update the corresponding \oursolution sketch \( M[j] \) (Line 8) via the \oursolution.\textit{FastUpdate} procedure.  Due to the direct update mechanism, we can better restrict the times of extracting snapshots from sketch. Specifically, suppose that in layer $j$, there exist $m$ singular values surpassing the threshold $\theta$ in Alg.\ref{alg:fast-update} (Line 13) from the beginning to current timestamp $t$, aligned as $\sigma_1\geq \sigma_2 \geq \sigma_3 \geq \cdots \geq \sigma_m \geq \theta$, extracting these $m$ snapshots takes $O(m(d_x+d_y)l)$ time in total. Since $m\theta \leq \sum_{i=1}^t \left\|\boldsymbol{x}_i\boldsymbol{y}_i^T\right\|_{*} \cdot \mathbb{I}(\xinorm\yinorm <\theta) < t\theta$, where $\mathbb{I}(\xinorm\yinorm <\theta)$ equals $1$ if $\xinorm\yinorm <\theta$ and otherwise $0$. Thereby, we have $m \leq t$, which implies amortized time for extractng the snapshots per layer is $O((d_x+d_y)l)$. As the update is executed in each of the \( L \) layers, the overall time cost per update is \( O(((d_x+d_y)l + l^3)\log R) \).

\htitle{Query Algorithm.} Alg.\ \ref{alg:query} describes the procedure for forming the sketch corresponding to the sliding window \( [t-N+1,t] \). Because of the per-layer constraint on the number of snapshots, a given layer might not contain enough snapshots to cover the entire window and thereby produce a valid sketch. To address this, we select the lowest layer for which the snapshots fully cover the window while minimizing the approximation error. More precisely, a layer is deemed valid if the last expired snapshot before its earliest non-expired snapshot occurs at time \( s \) satisfying \( s \le t-N \). A naive approach would scan all \( O(\log R) \) layers, resulting in a time complexity of \( O(\log R) \); however, since the snapshot density decreases monotonically with increasing layer index (due to the larger register thresholds), a binary search can be employed to reduce the query time complexity to \( O(\log \log R) \). 

        


\htitle{Theoretical Analysis.}
%    \subsection{Algorithm Analysis}
The following theorem demonstrates the error guarantee, space cost and time cost for \newsolution.
\vspace{-1mm}
\begin{theorem}\label{thm:socod-unnormalized}
Let \(\{(\boldsymbol{x}_t,\boldsymbol{y}_t)\}_{t\ge1}\) be a stream of data so that for all \(t\) it holds that
\(
\|\boldsymbol{x}_t\|_2^2,\;\|\boldsymbol{y}_t\|_2^2\in[1,R]
\). Let 
\(
\boldsymbol{X}_W = [\boldsymbol{x}_{t-N+1},\dots,\boldsymbol{x}_t]
\)
and 
\(
\boldsymbol{Y}_W = [\boldsymbol{y}_{t-N+1},\dots,\boldsymbol{y}_t]
\)
denote the sliding window matrices as defined in Def.~\ref{def:amm}. Given window size \(N\) and relative error parameter \(\epsilon\), the \newsolution algorithm outputs matrices
\(
\boldsymbol{A}_{aug}\in\mathbb{R}^{d_x\times O(\frac{1}{\epsilon})}
\)
and 
\(
\boldsymbol{B}_{aug}\in\mathbb{R}^{d_y\times O(\frac{1}{\epsilon})}
\)
such that if the sketch size is set to 
\(
l = \min\Bigl(\Bigl\lceil\frac{1}{\epsilon}\Bigr\rceil,\, d_x,\, d_y\Bigr),
\)
then
\(
\Bigl\|\boldsymbol{X}_W\boldsymbol{Y}_W^\top - \boldsymbol{A}_{aug}\boldsymbol{B}_{aug}^\top\Bigr\|_2 \le 4\epsilon\,\|\boldsymbol{X}_W\|_F\,\|\boldsymbol{Y}_W\|_F.
\)
Furthermore, the \newsolution sketch uses 
\(
O\Bigl(\frac{d_x+d_y}{\epsilon}\log R\Bigr)
\)
space and supports each update in 
\(
O\Bigl(((d_x+d_y)l + l^3)\log R\Bigr)
\)
time.
\end{theorem}



    \begin{algorithm}[t]
    \caption{\newsolution:Query()}
    \label{alg:query}
    \DontPrintSemicolon
    \KwOutput{$\boldsymbol{A}_{aug}$ and $\boldsymbol{B}_{aug}$}

    Find $i = \min_j 1\leq M[j].S[0].s\leq t - N$
    
    $\boldsymbol{A}_{aug} = 
    \begin{bmatrix}
     M[i].\hat{\boldsymbol{A}},\boldsymbol{C}   
    \end{bmatrix}$, where $\boldsymbol{C}$ is stacked $s_j.\boldsymbol{u}$, $\forall$ $s_j \in M[i].S$
    
    $\boldsymbol{B}_{aug} = 
    \begin{bmatrix}
     M[i].\hat{\boldsymbol{B}}, \boldsymbol{D}   
    \end{bmatrix}$, where $\boldsymbol{D}$ is stacked $s_j.\boldsymbol{v}$, $\forall$ $s_j \in M[i].S$

    \Return $\boldsymbol{A}_{aug}, \boldsymbol{B}_{aug}$
    
\end{algorithm} 

    
\label{sec-analysis}
