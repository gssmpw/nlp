\section{Related Work}
\label{litreview}
This paper contributes to a growing literature that has begun to document the problem of non-random measurement error in machine-learning models (see ____ for a review\footnote{For topic specific reviews, see ____ on deforestation, ____ and ____ on night lights, and ____ on air pollution.}). A number of recent papers propose econometric estimators that can correct for the non-classical measurement error in some cases. ____ proposes a misclassification model that requires users to specify the variables that may induce measurement error (e.g. cloud cover, satellite angle). ____ suggest a multiple imputation approach that may be sensitive to functional form specifications. ____ address an analagous problem in the NLP realm, developing a method similar to ____ for using gold-standard labeled data to adjust labels provided by an LLM. ____
% and ____ 
propose analytical bias correction techniques and MLE estimators to handle attenuation bias when classifier predictions are used as outcomes, but assume non-differential measurement error with respect to regressors. ____ provide an approach based on hidden Markov models to correct for misclassification bias that does not require ground-truth data but requires a stronger set of structural assumptions about data generating processes. In contrast, bias estimation and correction methods ____ and our proposed adversarial debiasing method make i.i.d sampling assumptions but do not require assumptions of functional forms or specific sources of measurement error, and are designed to target differential measurement error.
% We show that these methods can succeed in cases where multiple imputation fails. 

% add in something about PPI approach

%