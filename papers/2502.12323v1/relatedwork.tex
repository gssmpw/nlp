\section{Related Work}
\label{litreview}
This paper contributes to a growing literature that has begun to document the problem of non-random measurement error in machine-learning models (see \cite{jain_benefits_2020} for a review\footnote{For topic specific reviews, see \cite{balboni_economics_2022} on deforestation, \cite{gibson_which_2021} and \cite{bluhm_what_2022} on night lights, and \cite{fowlie_bringing_2019} on air pollution.}). A number of recent papers propose econometric estimators that can correct for the non-classical measurement error in some cases. \citet{alix-garcia_remotely_2022} proposes a misclassification model that requires users to specify the variables that may induce measurement error (e.g. cloud cover, satellite angle). \citet{proctor_parameter_2023} suggest a multiple imputation approach that may be sensitive to functional form specifications. \citet{egami_using_2023} address an analagous problem in the NLP realm, developing a method similar to \cite{proctor_parameter_2023} for using gold-standard labeled data to adjust labels provided by an LLM. \citet{zhang_how_2021}
% and \citet{fong_machine_2021} 
propose analytical bias correction techniques and MLE estimators to handle attenuation bias when classifier predictions are used as outcomes, but assume non-differential measurement error with respect to regressors. \citet{torchiana_improving_2023} provide an approach based on hidden Markov models to correct for misclassification bias that does not require ground-truth data but requires a stronger set of structural assumptions about data generating processes. In contrast, bias estimation and correction methods \citep[such as prediction-powered inference introduced in][]{angelopoulos_prediction-powered_2023} and our proposed adversarial debiasing method make i.i.d sampling assumptions but do not require assumptions of functional forms or specific sources of measurement error, and are designed to target differential measurement error.
% We show that these methods can succeed in cases where multiple imputation fails. 

% add in something about PPI approach

%