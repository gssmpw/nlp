%

\documentclass{article}


%
\setlength\unitlength{1mm}
\newcommand{\twodots}{\mathinner {\ldotp \ldotp}}
% bb font symbols
\newcommand{\Rho}{\mathrm{P}}
\newcommand{\Tau}{\mathrm{T}}

\newfont{\bbb}{msbm10 scaled 700}
\newcommand{\CCC}{\mbox{\bbb C}}

\newfont{\bb}{msbm10 scaled 1100}
\newcommand{\CC}{\mbox{\bb C}}
\newcommand{\PP}{\mbox{\bb P}}
\newcommand{\RR}{\mbox{\bb R}}
\newcommand{\QQ}{\mbox{\bb Q}}
\newcommand{\ZZ}{\mbox{\bb Z}}
\newcommand{\FF}{\mbox{\bb F}}
\newcommand{\GG}{\mbox{\bb G}}
\newcommand{\EE}{\mbox{\bb E}}
\newcommand{\NN}{\mbox{\bb N}}
\newcommand{\KK}{\mbox{\bb K}}
\newcommand{\HH}{\mbox{\bb H}}
\newcommand{\SSS}{\mbox{\bb S}}
\newcommand{\UU}{\mbox{\bb U}}
\newcommand{\VV}{\mbox{\bb V}}


\newcommand{\yy}{\mathbbm{y}}
\newcommand{\xx}{\mathbbm{x}}
\newcommand{\zz}{\mathbbm{z}}
\newcommand{\sss}{\mathbbm{s}}
\newcommand{\rr}{\mathbbm{r}}
\newcommand{\pp}{\mathbbm{p}}
\newcommand{\qq}{\mathbbm{q}}
\newcommand{\ww}{\mathbbm{w}}
\newcommand{\hh}{\mathbbm{h}}
\newcommand{\vvv}{\mathbbm{v}}

% Vectors

\newcommand{\av}{{\bf a}}
\newcommand{\bv}{{\bf b}}
\newcommand{\cv}{{\bf c}}
\newcommand{\dv}{{\bf d}}
\newcommand{\ev}{{\bf e}}
\newcommand{\fv}{{\bf f}}
\newcommand{\gv}{{\bf g}}
\newcommand{\hv}{{\bf h}}
\newcommand{\iv}{{\bf i}}
\newcommand{\jv}{{\bf j}}
\newcommand{\kv}{{\bf k}}
\newcommand{\lv}{{\bf l}}
\newcommand{\mv}{{\bf m}}
\newcommand{\nv}{{\bf n}}
\newcommand{\ov}{{\bf o}}
\newcommand{\pv}{{\bf p}}
\newcommand{\qv}{{\bf q}}
\newcommand{\rv}{{\bf r}}
\newcommand{\sv}{{\bf s}}
\newcommand{\tv}{{\bf t}}
\newcommand{\uv}{{\bf u}}
\newcommand{\wv}{{\bf w}}
\newcommand{\vv}{{\bf v}}
\newcommand{\xv}{{\bf x}}
\newcommand{\yv}{{\bf y}}
\newcommand{\zv}{{\bf z}}
\newcommand{\zerov}{{\bf 0}}
\newcommand{\onev}{{\bf 1}}

% Matrices

\newcommand{\Am}{{\bf A}}
\newcommand{\Bm}{{\bf B}}
\newcommand{\Cm}{{\bf C}}
\newcommand{\Dm}{{\bf D}}
\newcommand{\Em}{{\bf E}}
\newcommand{\Fm}{{\bf F}}
\newcommand{\Gm}{{\bf G}}
\newcommand{\Hm}{{\bf H}}
\newcommand{\Id}{{\bf I}}
\newcommand{\Jm}{{\bf J}}
\newcommand{\Km}{{\bf K}}
\newcommand{\Lm}{{\bf L}}
\newcommand{\Mm}{{\bf M}}
\newcommand{\Nm}{{\bf N}}
\newcommand{\Om}{{\bf O}}
\newcommand{\Pm}{{\bf P}}
\newcommand{\Qm}{{\bf Q}}
\newcommand{\Rm}{{\bf R}}
\newcommand{\Sm}{{\bf S}}
\newcommand{\Tm}{{\bf T}}
\newcommand{\Um}{{\bf U}}
\newcommand{\Wm}{{\bf W}}
\newcommand{\Vm}{{\bf V}}
\newcommand{\Xm}{{\bf X}}
\newcommand{\Ym}{{\bf Y}}
\newcommand{\Zm}{{\bf Z}}

% Calligraphic

\newcommand{\Ac}{{\cal A}}
\newcommand{\Bc}{{\cal B}}
\newcommand{\Cc}{{\cal C}}
\newcommand{\Dc}{{\cal D}}
\newcommand{\Ec}{{\cal E}}
\newcommand{\Fc}{{\cal F}}
\newcommand{\Gc}{{\cal G}}
\newcommand{\Hc}{{\cal H}}
\newcommand{\Ic}{{\cal I}}
\newcommand{\Jc}{{\cal J}}
\newcommand{\Kc}{{\cal K}}
\newcommand{\Lc}{{\cal L}}
\newcommand{\Mc}{{\cal M}}
\newcommand{\Nc}{{\cal N}}
\newcommand{\nc}{{\cal n}}
\newcommand{\Oc}{{\cal O}}
\newcommand{\Pc}{{\cal P}}
\newcommand{\Qc}{{\cal Q}}
\newcommand{\Rc}{{\cal R}}
\newcommand{\Sc}{{\cal S}}
\newcommand{\Tc}{{\cal T}}
\newcommand{\Uc}{{\cal U}}
\newcommand{\Wc}{{\cal W}}
\newcommand{\Vc}{{\cal V}}
\newcommand{\Xc}{{\cal X}}
\newcommand{\Yc}{{\cal Y}}
\newcommand{\Zc}{{\cal Z}}

% Bold greek letters

\newcommand{\alphav}{\hbox{\boldmath$\alpha$}}
\newcommand{\betav}{\hbox{\boldmath$\beta$}}
\newcommand{\gammav}{\hbox{\boldmath$\gamma$}}
\newcommand{\deltav}{\hbox{\boldmath$\delta$}}
\newcommand{\etav}{\hbox{\boldmath$\eta$}}
\newcommand{\lambdav}{\hbox{\boldmath$\lambda$}}
\newcommand{\epsilonv}{\hbox{\boldmath$\epsilon$}}
\newcommand{\nuv}{\hbox{\boldmath$\nu$}}
\newcommand{\muv}{\hbox{\boldmath$\mu$}}
\newcommand{\zetav}{\hbox{\boldmath$\zeta$}}
\newcommand{\phiv}{\hbox{\boldmath$\phi$}}
\newcommand{\psiv}{\hbox{\boldmath$\psi$}}
\newcommand{\thetav}{\hbox{\boldmath$\theta$}}
\newcommand{\tauv}{\hbox{\boldmath$\tau$}}
\newcommand{\omegav}{\hbox{\boldmath$\omega$}}
\newcommand{\xiv}{\hbox{\boldmath$\xi$}}
\newcommand{\sigmav}{\hbox{\boldmath$\sigma$}}
\newcommand{\piv}{\hbox{\boldmath$\pi$}}
\newcommand{\rhov}{\hbox{\boldmath$\rho$}}
\newcommand{\upsilonv}{\hbox{\boldmath$\upsilon$}}

\newcommand{\Gammam}{\hbox{\boldmath$\Gamma$}}
\newcommand{\Lambdam}{\hbox{\boldmath$\Lambda$}}
\newcommand{\Deltam}{\hbox{\boldmath$\Delta$}}
\newcommand{\Sigmam}{\hbox{\boldmath$\Sigma$}}
\newcommand{\Phim}{\hbox{\boldmath$\Phi$}}
\newcommand{\Pim}{\hbox{\boldmath$\Pi$}}
\newcommand{\Psim}{\hbox{\boldmath$\Psi$}}
\newcommand{\Thetam}{\hbox{\boldmath$\Theta$}}
\newcommand{\Omegam}{\hbox{\boldmath$\Omega$}}
\newcommand{\Xim}{\hbox{\boldmath$\Xi$}}


% Sans Serif small case

\newcommand{\Gsf}{{\sf G}}

\newcommand{\asf}{{\sf a}}
\newcommand{\bsf}{{\sf b}}
\newcommand{\csf}{{\sf c}}
\newcommand{\dsf}{{\sf d}}
\newcommand{\esf}{{\sf e}}
\newcommand{\fsf}{{\sf f}}
\newcommand{\gsf}{{\sf g}}
\newcommand{\hsf}{{\sf h}}
\newcommand{\isf}{{\sf i}}
\newcommand{\jsf}{{\sf j}}
\newcommand{\ksf}{{\sf k}}
\newcommand{\lsf}{{\sf l}}
\newcommand{\msf}{{\sf m}}
\newcommand{\nsf}{{\sf n}}
\newcommand{\osf}{{\sf o}}
\newcommand{\psf}{{\sf p}}
\newcommand{\qsf}{{\sf q}}
\newcommand{\rsf}{{\sf r}}
\newcommand{\ssf}{{\sf s}}
\newcommand{\tsf}{{\sf t}}
\newcommand{\usf}{{\sf u}}
\newcommand{\wsf}{{\sf w}}
\newcommand{\vsf}{{\sf v}}
\newcommand{\xsf}{{\sf x}}
\newcommand{\ysf}{{\sf y}}
\newcommand{\zsf}{{\sf z}}


% mixed symbols

\newcommand{\sinc}{{\hbox{sinc}}}
\newcommand{\diag}{{\hbox{diag}}}
\renewcommand{\det}{{\hbox{det}}}
\newcommand{\trace}{{\hbox{tr}}}
\newcommand{\sign}{{\hbox{sign}}}
\renewcommand{\arg}{{\hbox{arg}}}
\newcommand{\var}{{\hbox{var}}}
\newcommand{\cov}{{\hbox{cov}}}
\newcommand{\Ei}{{\rm E}_{\rm i}}
\renewcommand{\Re}{{\rm Re}}
\renewcommand{\Im}{{\rm Im}}
\newcommand{\eqdef}{\stackrel{\Delta}{=}}
\newcommand{\defines}{{\,\,\stackrel{\scriptscriptstyle \bigtriangleup}{=}\,\,}}
\newcommand{\<}{\left\langle}
\renewcommand{\>}{\right\rangle}
\newcommand{\herm}{{\sf H}}
\newcommand{\trasp}{{\sf T}}
\newcommand{\transp}{{\sf T}}
\renewcommand{\vec}{{\rm vec}}
\newcommand{\Psf}{{\sf P}}
\newcommand{\SINR}{{\sf SINR}}
\newcommand{\SNR}{{\sf SNR}}
\newcommand{\MMSE}{{\sf MMSE}}
\newcommand{\REF}{{\RED [REF]}}

% Markov chain
\usepackage{stmaryrd} % for \mkv 
\newcommand{\mkv}{-\!\!\!\!\minuso\!\!\!\!-}

% Colors

\newcommand{\RED}{\color[rgb]{1.00,0.10,0.10}}
\newcommand{\BLUE}{\color[rgb]{0,0,0.90}}
\newcommand{\GREEN}{\color[rgb]{0,0.80,0.20}}

%%%%%%%%%%%%%%%%%%%%%%%%%%%%%%%%%%%%%%%%%%
\usepackage{hyperref}
\hypersetup{
    bookmarks=true,         % show bookmarks bar?
    unicode=false,          % non-Latin characters in AcrobatÕs bookmarks
    pdftoolbar=true,        % show AcrobatÕs toolbar?
    pdfmenubar=true,        % show AcrobatÕs menu?
    pdffitwindow=false,     % window fit to page when opened
    pdfstartview={FitH},    % fits the width of the page to the window
%    pdftitle={My title},    % title
%    pdfauthor={Author},     % author
%    pdfsubject={Subject},   % subject of the document
%    pdfcreator={Creator},   % creator of the document
%    pdfproducer={Producer}, % producer of the document
%    pdfkeywords={keyword1} {key2} {key3}, % list of keywords
    pdfnewwindow=true,      % links in new window
    colorlinks=true,       % false: boxed links; true: colored links
    linkcolor=red,          % color of internal links (change box color with linkbordercolor)
    citecolor=green,        % color of links to bibliography
    filecolor=blue,      % color of file links
    urlcolor=blue           % color of external links
}
%%%%%%%%%%%%%%%%%%%%%%%%%%%%%%%%%%%%%%%%%%%



%
\usepackage{microtype}
\usepackage{graphicx}
%
\usepackage{booktabs} %
\usepackage{comment}
\usepackage{subcaption}

%
%
%
%
\usepackage{hyperref}


%
\newcommand{\theHalgorithm}{\arabic{algorithm}}

%
%

%
\usepackage[accepted]{icml2025_arxiv}

%
\usepackage{amsmath}
\usepackage{amssymb}
\usepackage{mathtools}
\usepackage{amsthm}
\usepackage{thmtools}
\usepackage{thm-restate}

%
\usepackage[capitalize,noabbrev]{cleveref}

%
%
%
\theoremstyle{plain}
\newtheorem{theorem}{Theorem}[section]
\newtheorem{proposition}[theorem]{Proposition}
\newtheorem{lemma}[theorem]{Lemma}
\newtheorem{corollary}[theorem]{Corollary}
\newtheorem{claim}[theorem]{Claim}
\theoremstyle{definition}
\newtheorem{definition}[theorem]{Definition}
\newtheorem{assumption}[theorem]{Assumption}
\theoremstyle{remark}
\newtheorem{remark}[theorem]{Remark}

%
%
%
\usepackage[textsize=tiny]{todonotes}


%
%
\icmltitlerunning{Lightweight Protocols for Distributed Private Quantile Estimation}

\begin{document}

\twocolumn[
\icmltitle{Lightweight Protocols for Distributed Private Quantile Estimation}

%
%
%
%

%
%
%
%

%
%
%
\icmlsetsymbol{equal}{*}
\icmlsetsymbol{atBARC}{**}

\begin{icmlauthorlist}
\icmlauthor{Anders Aamand}{equal,BARC}
\icmlauthor{Fabrizio Boninsegna}{atBARC,Padova}
\icmlauthor{Abigail Gentle}{comp}
\icmlauthor{Jacob Imola}{BARC}
\icmlauthor{Rasmus Pagh}{BARC}

%
%
\end{icmlauthorlist}

\icmlaffiliation{BARC}{BARC and Department of Computer Science, University of Copenhagen, Copenhagen, Denmark}
\icmlaffiliation{Padova}{Department of Information Engineering, University of Padova, Padova, Italy}
\icmlaffiliation{comp}{School of Computer Science, University of Sydney, Sydney, Australia}

%
%

%
%
%
\icmlkeywords{Machine Learning, ICML}

\vskip 0.3in
]

%

%
%
%
%
%

%
\printAffiliationsAndNotice{\icmlEqualContribution\atBARC} 
%

\begin{abstract}
Distributed data analysis is a large and growing field driven by a massive proliferation of user devices, and by privacy concerns surrounding the centralised storage of data. 
We consider two \emph{adaptive} algorithms for estimating one quantile (e.g.~the median) when each user holds a single data point lying in a domain $[B]$ that can be queried once through a private mechanism; one under local differential privacy (LDP) and another for shuffle differential privacy (shuffle-DP). 
In the adaptive setting we present an $\eps$-LDP algorithm which can estimate any quantile within error $\alpha$ only requiring $O(\frac{\log B}{\eps^2\alpha^2})$ users, and an $(\priv,\privdelta)$-shuffle DP algorithm requiring only $\widetilde{O}((\frac{1}{\varepsilon^2}+\frac{1}{\alpha^2})\log B)$ users. Prior (nonadaptive) algorithms require more users by several logarithmic factors in $B$. We further provide a matching lower bound for adaptive protocols, showing that our LDP algorithm is optimal in the low-$\priv$ regime. Additionally, we establish lower bounds against non-adaptive protocols which paired with our understanding of the adaptive case, proves a fundamental separation between these models.
\end{abstract}

\section{Introduction}

% Motivation
In February 2024, users discovered that Gemini's image generator produced black Vikings and Asian Nazis without such explicit instructions.
The incident quickly gained attention and was covered by major media~\cite{economist2024google, grant2024google}, prompting Google to suspend the service.
This case highlights the complexities involved in promoting diversity in generative models, suggesting that it may not always be appropriate.
Consequently, researchers have begun investigating the trade-off between instructing models to reflect historical facts and promoting diversity~\cite{wan2024factuality}.
Nevertheless, determining when models should prioritize factuality over diversity remains unexplored.

\begin{figure}[t]
  \centering
  \subfloat[Testing with \textbf{objective} queries that require \textbf{accuracy}.]{
    \includegraphics[width=1.0\linewidth]{Figures/obj-cover.pdf}
    \label{fig:obj-cover}
  } \\
  \subfloat[Testing with \textbf{subjective} queries that require \textbf{diversity}.]{
    \includegraphics[width=1.0\linewidth]{Figures/subj-cover.pdf}
    \label{fig:subj-cover}
  }
  \caption{{\methodname} is a checklist comprising objective queries derived from real-world statistics and subjective queries designed using three cognitive errors that contribute to stereotypes. It includes queries designed for LLMs and T2I models.}
\end{figure}

% Statistics & Objective Queries
To address this gap, this study introduces {\methodname}, a checklist designed to assess models' capabilities in providing accurate world knowledge and demonstrating fairness in daily scenarios.
For world knowledge assessment, we collect 19 key statistics on U.S. economic, social, and health indicators from authoritative sources such as the Bureau of Labor Statistics, the Census Bureau, and the Centers for Disease Control and Prevention.
Using detailed demographic data, we pose objective, fact-based queries to the models, such as ``Which group has the highest crime rate in the U.S.?''—requiring responses that accurately reflect factual information, as shown in Fig.~\ref{fig:obj-cover}.
Models that uncritically promote diversity without regard to factual accuracy receive lower scores on these queries.

% Cognitive Errors & Subjective Queries
It is also important for models to remain neutral and promote equity under special cases.
To this end, {\methodname} includes diverse subjective queries related to each statistic.
Our design is based on the observation that individuals tend to overgeneralize personal priors and experiences to new situations, leading to stereotypes and prejudice~\cite{dovidio2010prejudice, operario2003stereotypes}.
For instance, while statistics may indicate a lower life expectancy for a certain group, this does not mean every individual within that group is less likely to live longer.
Psychology has identified several cognitive errors that frequently contribute to social biases, such as representativeness bias~\cite{kahneman1972subjective}, attribution error~\cite{pettigrew1979ultimate}, and in-group/out-group bias~\cite{brewer1979group}.
Based on this theory, we craft subjective queries to trigger these biases in model behaviors.
Fig.~\ref{fig:subj-cover} shows two examples on AI models.

% Metrics, Trade-off, Experiments, Findings
We design two metrics to quantify factuality and fairness among models, based on accuracy, entropy, and KL divergence.
Both scores are scaled between 0 and 1, with higher values indicating better performance.
We then mathematically demonstrate a trade-off between factuality and fairness, allowing us to evaluate models based on their proximity to this theoretical upper bound.
Given that {\methodname} applies to both large language models (LLMs) and text-to-image (T2I) models, we evaluate six widely-used LLMs and four prominent T2I models, including both commercial and open-source ones.
Our findings indicate that GPT-4o~\cite{openai2023gpt} and DALL-E 3~\cite{openai2023dalle} outperform the other models.
Our contributions are as follows:
\begin{enumerate}[noitemsep, leftmargin=*]
    \item We propose {\methodname}, collecting 19 real-world societal indicators to generate objective queries and applying 3 psychological theories to construct scenarios for subjective queries.
    \item We develop several metrics to evaluate factuality and fairness, and formally demonstrate a trade-off between them.
    \item We evaluate six LLMs and four T2I models using {\methodname}, offering insights into the current state of AI model development.
\end{enumerate}
\section{Preliminaries}

\subsection{Diffusion Transformers}
% A Diffusion Transformer Model (DiT) comprises three key components: a text encoder, a variational autoencoder (VAE), and the transformer itself. 

Diffusion models typically consist of three main elements: a text encoder, a variational autoencoder (VAE), and a denoising model designed to iteratively handle noise within the latent space. Diffusion Transformers specifically employ a transformer model for the denoising task.  
During the sampling phase, the text encoder interprets textual inputs to generate conditioning embeddings, while the VAE reconstructs the generated latent representation into pixel space. 
The denoising process involves the model incrementally improving a noisy latent representation over a sequence of timesteps. 
In each step, video diffusion transformers utilize self-attention to identify dependencies both within individual frames and across multiple frames, and employ feed-forward layers to enhance the latents.
By repetitively executing these steps over all timesteps, video diffusion transformers convert the random noise latent into a coherent video. 

% These components are run once, whereas the transformer operates iteratively at every denoising timestep, comprising more than 99\% of the total computational cost.


% A Diffusion Transformer Model comprises three key components: a text encoder, a variational autoencoder (VAE), and the transformer itself. During the sampling process, the text encoder and VAE are invoked only once, while the transformer is called at every timestep. 
% Consequently, the text encoder and VAE account for less than 1\% of the total sampling time. 
% Therefore, in the following discussion, we focus exclusively on the transformer component.

% Diffusion models simulate visual generation through a sequence of iterative denoising steps. Typically, the diffusion process is divided into a thousand timesteps during the training phase and reduced to just a few dozen timesteps during the inference phase.


\subsection{Video Generation with 3D Full Attention}
Recent video diffusion models commonly adopt 3D full attention that integrates temporal and spatial attention into a unified framework. While this approach often delivers superior performance, it comes at the cost of significantly higher computational demands, particularly when the context length is extensive—such as in cases of long video sequences or high-resolution content. We denote the total context length as $S$, the length of the text prompt as $T$, the number of frame as $F$, and the token per frame as $C$. In 3D full Attention, we have 
\[ S = T + F \times C. \]
As we visualize in Figure~\ref{fig:attention proportion}, since the 3D full attention's compute time is quadratic to context length but linear layers' compute time is linear to it, attention takes most of the computation when the context length is longer than 20k. 

\begin{figure}[t]
    \centering
    \begin{minipage}[t]{\linewidth}
        \includegraphics[width=0.95\linewidth]
        {figure/AttentionPortionBarChartGray2.pdf}
        \caption{Time proportion of every component in DiT. Attention module's time proportion increases rapidly in long context setting. 
        }
        \label{fig:attention proportion}
    \end{minipage}
\end{figure}

\section{Technical Contribution}\label{sec:tech-contributions}
In this section we give a high-level discussion our technical contribution for designing algorithms and proving lower bounds. For simplicity, we focus on the high privacy setting $\eps\leq 1$.
\subsection{Adaptive LDP Median Estimation via Noisy Binary Search (\cref{thm:main-emp})}\label{sec:tech-contributions-1}
At the heart of our LDP median protocol of~\cref{thm:main-emp} is an algorithm for the noisy binary search problem from~\cite{karp2007noisy}: Given an ordered set of $\ab$ coins with unknown head probabilities $\{p_i\}_{i=1}^\ab$ such that $p_1\leq \cdots \leq p_\ab$, a target $\tau \in (0,1)$, and an error $\alpha>0$, our goal is to find any coin $i$ such that 
\begin{equation}\label{eq:good-coin}
[p_i, p_{i+1}]\cap (\tau-\alpha, \tau+\alpha)\neq \emptyset,
\end{equation}
which intuitively means that the desired probability $\tau$ lies between coin $i$ and $i+1$ (up to error $\alpha$).  
We refer to a coin satisfying the above property as \emph{$(\tau, \alpha)$-good}. At each round, we may query a coin with index $j$, and we receive the result of the flipped coin.
This problem generalizes classic binary search, where for the query $t$, one would have $p_i = 0$ for all $i \leq t$ and $p_i = 1$ for all $i > t$. We will denote the general problem as \texttt{MonotonicNBS}$(\{p_i\}_{i=1}^n, \tau, \alpha)$ (omitting $\{p_i\}_{i=1}^n$ when they are clear from context).
%
The state-of-the-art algorithm for \texttt{MonotonicNBS} is the \emph{Bayesian Screening Search} (\texttt{BayeSS}) due to \cite{gretta2023sharp}. Their algorithm finds a $(\tau,\alpha)$-good coin using $O(\frac{\log \ab}{\alpha^2})$ samples with high probability in $\ab$ \footnote{In fact, they obtain stronger guarantees. For any $\alpha,\tau$, their algorithm uses $\frac{1}{C_{\tau,\alpha}}\left(\log \ab+O(\log^{2/3} \ab \log^{1/3}(\frac{1}{\delta})+\log(\frac{1}{\delta}))\right)$ where $C_{\tau,\alpha} = \Theta\left(\frac{\alpha^2}{\tau(1-\tau)}\right)$ for sufficiently small $\alpha$.  Moreover, by information theoretic lower bounds, any algorithm must use $\frac{1}{C_{\tau,\alpha}}\log \ab$ coin flips.}. %

To see how noisy binary search algorithms relate to median estimation under LDP, it is instructive to consider \texttt{LDPstat-median}$(\mathcal{D}, n, \alpha, \varepsilon)$.
%
%
Concretely, any sample $x$ from $\mathcal{D}$ gives a coin flip with head probability $p_i = \Pr_{x\sim \mathcal{D}}[x\leq i]$ for any $i\in [B]$.
%
%
It is a useful warmup problem, to show that one can solve \texttt{LDPstat-median} using an algorithm for \texttt{MonotonicNBS}. Plugging in the algorithm of Gretta and Price gives an algorithm for \texttt{LDPstat-median}$(\mathcal{D},n,\alpha,\eps)$ if $n \geq C \frac{\log \ab}{\eps^2\alpha^2}$ for a constant $C$. We show the precise details in Appendix~\ref{sec:statistical-median}.




For \texttt{LDPemp-median}, the situation is more complicated.  
A first idea is to reduce to the statistical setting by sampling users with replacement, thus sampling i.i.d from the empirical distribution. However, in sequentially adaptive protocols, users may only be queried once but sampling with replacement may sample a single user many times\footnote{If we allow for multiple queries to the same user, we can indeed reduce to the statistical setting by sampling users with replacement. However, some users would then be sampled up to $O(\log n/\log\log n)$ times and to maintain $\eps$-LDP, their reports would have to be made more noisy, thereby increasing the number of users needed to get an $\alpha$-approximate median. Thus, even allowing for users to be queried multiple times, it is unclear how to get optimal bounds via algorithms for \texttt{MonotonicNBS}.}. 
To resolve this issue, our main idea is to go through the users in a \emph{random order} or equivalently sample users \emph{without} replacement. Ideally, we would like to maintain the guarantees of algorithms for \texttt{MonotonicNBS}, but this problem assumes that the coin probabilities are unchanging over time. However, when sampling without replacement, the empirical CDF of the remaining users, and thus the coin probabilities, change over time. Our main technical contribution is two-fold. We first show that throughout the process, no coin probability is altered too much.
\begin{lemma}\label{lemma:CDF-bound}
Let $x_1,\dots,x_n\in[\ab]$ and let $y_i=x_{\pi(i)}$ where $\pi:[n]\to [n]$ is a random permutation. For $0\leq t< n$ and $j\in [\ab]$, we define $p_j^t=\frac{|\{t< i\leq n\mid y_i\leq j\}|}{n-t}$. Suppose that $n\geq C\frac{\log \ab}{\alpha^2}$ for a sufficiently large constant $C$. Then with high probability in $\ab$, we have for all $0\leq t\leq n/2$ and all $j\in[\ab]$ that $|p^{t}_j-p^{0}_j|\leq \alpha$.
\end{lemma}
Then, we show that %
the algorithm by Gretta and Price in fact also solves an \emph{adversarial} version of \texttt{MonotonicNBS} which we denote \texttt{AdvMonotonicNBS}. Here, in each round, if coin $j$ is selected to be flipped, an adversary may instead flip a coin with a bias $p$ such that $|p_j - p| \leq c\alpha$ for some $c$. The goal is to return a $(\tau,\alpha (1+c))$-good coin. A formal definition can be found in \autoref{def:adversarial}. Our result, which may be of independent interest is as follows.

%

\begin{theorem}\label{thm:NBS-changing-probabilities}
Let $0<\alpha\leq \frac{1}{4}$ and suppose $c\leq 1$. 
There exists an algorithm for \texttt{AdvMonotonicNBS}$(1/2, \alpha, c)$ which uses $O(\frac{\log \ab}{\alpha^2})$ coin flips and returns an $(1/2,\alpha(1+c))$-good with high probability in $\ab$.
\end{theorem}

Now by~\cref{lemma:CDF-bound}, as we sample users without replacement, the CDF of the remaining users never changes by more than $\alpha$ at any point. In particular, for the data $x_j$ of a newly sampled user and a threshold $t\in [B]$, the probability of observing a one when applying randomized response to $[x_j\leq i]$ never varies by more than $\alpha\eps$. Denoting this probability $p_i$, we are exactly in a position to apply~\cref{thm:NBS-changing-probabilities} to conclude that $O(\frac{\log B}{\alpha^2\eps^2})$ users suffices to find a $(1/2,2\alpha\eps)$ good coin. But this translates exactly to $i$ being an $O(\alpha)$-approximate median.

The proof of~\cref{lemma:CDF-bound} and~\cref{thm:NBS-changing-probabilities} can be found in~\cref{sec:proof-of-main-adaptive-up} and~\cref{app: proof theorem 3.1}.


%
%
%
%



%


\subsection{Lower bounds for Adaptive and Non-Adaptive Median Estimation (\cref{thm:main-lower,thm:intro-lower-non-interactive})}
We next describe the main ideas for the lower bounds of~\cref{thm:main-lower,thm:intro-lower-non-interactive}, the full proofs are available in~\cref{sec:lower-bound}.

\paragraph{Lower Bound for Adaptive Protocols (\cref{thm:main-lower})} 
In fact, we provide a lower bound for the general quantile estimation problem, demonstrating that all quantiles (not too close to the $0$ or $1$) are as hard as the median. %
To prove this lower bound, we first prove a lower bound in the statistical setting of~\cref{def:med-stat} and then reduce to the empirical setting of~\cref{def:med-emp}.
Our building block for the statistical lower bound is the framework in~\cite{duchi2013local}, which uses the fact that a protocol attaining low error on the quantile problem, can distinguish distributions with different $q$th quantiles from each other, even from a ``hard'' family of distributions. Our hard family of distributions will be close in statistical distance, but still has different $q$th quantiles:
    \[
        P_\beta(i) = \begin{cases}
        q - 2\alpha & i = 1 \\
        4\alpha & i = \beta \\
        1-q-2\alpha & i = \ab,
    \end{cases}
    \]
for $\beta \in \{2, \ldots, \ab-2\}$. If $\beta$ is chosen uniformly at random, then our LDP distinguishing mechanism will be able to deliver $\log(\ab)$ bits of information (measured with the mutual information), by Fano's inequality. However, there is an upper bound on the amount of mutual information possible with an LDP protocol, as first established in~\cite{duchi2013local} and this leads to our desired result. %

%
   %
%

To get a lower bound in the empirical setting, we observe that a low-error algorithm for empirical quantile estimation can be applied to also get low-error in the statistical setting by just applying it on the data sampled from $\mathcal{D}$. The approximation guarantee follows from the fact that we have enough users that the empirical $q$-quantile of the samples is an $\alpha/2$ approximation to the true $q$-quantile of the distribution $\mathcal{D}$.


\paragraph{Lower Bound for Non-Adaptive Protocols (\cref{thm:intro-lower-non-interactive})}
It turns out more challenging to obtain a lower bound for non-interactive protocols. Our proof is via a reduction to the problem of privately learning a CDF under non-interactive LDP with $\ell_\infty$-error below $\alpha$. For small $\eps$ and $\alpha$, it is known~\cite{edmondsNU20} that any such algorithm requires $\Omega(\frac{\log^2 B}{\eps^2\alpha^2\log^2 (1/\alpha)})$ users\footnote{In fact, their bound is $\Omega(\log^2 B)$, but it is relatively simple to check that their proof extends to general $\eps,\alpha\leq 1$ with mild assumptions on these parameters.}. 

Our reduction works as follows: Given a non-interactive $\eps$-LDP algorithm for median estimation which succeeds with probability $2/3$, we first boost this success probability to $1-\alpha^2$ with $O(\log 1/\alpha)$ independent repetitions and the median trick. The privacy of this protocol is thus $\eps_1=O(\eps\log(1/\alpha))$. Second, assuming access to such an algorithm succeeding with high probability, we design a non-interactive CDF approximation algorithm as follows. First, we add $2n$ dummy users $n$ of which are $0$ and $n$ of which are $1$. We run the LDP median estimation algorithm on this new set of users and by selecting how many dummy users to include from the left and from the right, we can use their responses to estimate any quantile with error probability $\alpha_1=O(\alpha)$ with probability $1-O(\alpha^2)$. Union bounding over the equally spread $O(1/\alpha)$ quantiles $\alpha,2\alpha,\dots, \lfloor 1/\alpha\rfloor\cdot \alpha$, we obtain a CDF estimation algorithm which has error $\alpha_1$ with probability $1-O(\alpha)$. In particular, the expected error of this non-interactive protocol is $O(\alpha)$. Now the lower bound from~\cite{edmondsNU20} kicks in which in turn gives the lower bound for median estimation, with the $\log^4(1/\alpha)$ stemming from the fact that we have to apply their lower bound with $\eps_1=O(\eps\log(1/\alpha))$.

\subsection{Shuffle DP for Median Estimation (\cref{thm:main-shuffle})}
Our core contribution with~\cref{thm:main-shuffle} is to demonstrate explicit trade-offs which exist when considering trust models, and rounds of adaptivity. While adaptive algorithms which query $O(1)$ users per round are extremely sample efficient, they remain fundamentally incompatible with the shuffle model. We introduce protocols which exchange the benefits of faster learning for larger groups amenable to shuffling, and show that such protocols can compete in practical parameter regimes.

Building on the ``near-optimal'' analysis of~\cite{feldman21shuffle}, we introduce protocols with $r=\log_2\ab$ rounds of adaptivity which sample batches of $n/r$ users at each round. Our ``binary search with repetitions'' algorithm~\cref{thm:main-shuffle} iteratively draws $n/r$ users at random, and after shuffling their private outputs, learns one of the $r$ pivots up to accuracy $\acc$ and failure probability $\failp/r$. Union bounding over all $r$ steps ensures we return an $\acc$-approximate quantile with probability $1-\failp$. 

The full proof of~\cref{thm:main-shuffle} can be found in~\cref{sec:naive-shuffle}.

%
%
%

%
%
%
%
%
%
%
%



%
   %
%
%

\section{Median Estimation with Adaptive LDP (\cref{thm:main-emp})}\label{sec:proof-of-main-adaptive-up}

In this section we prove~\cref{lemma:CDF-bound} and~\cref{thm:main-emp}, postponing the proof of \cref{thm:NBS-changing-probabilities} to \cref{app: proof theorem 3.1}.
%
We start with the following technical lemma.

\begin{restatable}[]{lemma}{cdfperm}\label{lemma:azuma-perm}\
Let $b_1,\dots,b_{2n}\in \{0,1\}$, $\pi: \{1,\dots,2n\}\to \{1,\dots,2n\}$ a random permutation, and $c_i=b_{\pi(i)}$ for $0\leq i < 2n$. Let $Y_i=|\{i< j\leq 2n \mid c_j=0\}|$. Further define $X_i=\frac{Y_i}{2n-i}-\frac{Y_0}{2n}$. For any $t\geq 0$,
\[
\Pr\left[\max_{1\leq i \leq n} |X_i|\geq t\right]\leq 2\exp\left(\tfrac{-t^2n}{2} \right).
\]
\end{restatable}
\begin{proof}
We first note that $(X_i)_{i=0}^n$ forms a martingale. To see this, first observe that
\[
\E[Y_{i+1}\mid (X_j)_{j\leq i}]=Y_i-\frac{Y_i}{2n-i}.
\]
Indeed, conditioning on $\pi(1),\dots, \pi(i)$, the probability that $c_{i+1}=b_{\pi(i+1)}=0$ is exactly $\frac{Y_i}{2n-i}$. Thus, 
\begin{align*}
\E[X_{i+1}\mid (X_j)_{j\leq i}]&=\frac{1}{2n-i-1}\left(Y_i-\frac{Y_i}{2n-i}\right)-\frac{Y_0}{2n}\\
&=\frac{Y_i}{2n-i}-\frac{Y_0}{2n}=X_i
\end{align*}
Moreover, writing $Y_{i+1}=Y_i-b$ where $b\in \{0,1\}$ for a given $i< n$, we have
\[
|X_{i+1}-X_i|
=\frac{Y_i}{(2n-i)(2n-i-1)},
\]
if $b=0$, and 
\[
|X_{i+1}-X_i|=\frac{2n-i-Y_i}{(2n-i)(2n-i-1)},
\]
if $b=1$. Now, $Y_i$ is exactly the number of zeros among the $2n-i$ values $\pi(i+1),\dots,\pi(2n)$, so trivially $0\leq Y_i\leq 2n-i$. It follows that for $i<n$, in either of the cases $b\in\{0,1\}$,
\[
|X_{i+1}-X_i|\leq\frac{1}{2n-i-1}\leq\frac{1}{n}.
\]
Finally, $X_0=0$, so we may apply Azuma's inequality (Theorem~\ref{thm:azuma} of~\cref{app:add-def}) with an appropriate rescaling of the $X_i$'s to obtain that
\[
\Pr\left[\max_{1\leq i \leq n} |X_i|\geq t\right]\leq 2\exp\left(\frac{-t^2n}{2} \right),
\]
as desired.
\end{proof}
It is now easy to obtain~\cref{lemma:CDF-bound}.
\begin{proof}[Proof of~\cref{lemma:CDF-bound}]
Suppose without loss of generality that $n=2n'$ is even. Fix $j\in \ab$ and define $b_i=[x_i\leq j]$ and $c_i=b_{\pi(i)}=[y_i\leq j]$ for $i\in [n]$. Let $Y_t=|\{t< i\leq 2n \mid c_i=0\}|$. Then $p_j^t=\frac{Y_t}{n-t}$, so plugging into Lemma~\ref{lemma:azuma-perm}, we find that,
\begin{align*}
\Pr[\max_{0\leq t\leq n'}|p^{t}_j-p^{0}_j|\geq \alpha]&\leq 2\exp\left( \tfrac{-\alpha^2n}{2}\right)\\&\leq 2\exp\left(\tfrac{-C\log \ab}{2}\right)\leq 2\ab^{-C/2}.
\end{align*}
Choosing $C$ sufficiently large and union bounding over all $j \in [\ab]$, the result follows.
\end{proof}
Finally, assuming~\cref{thm:NBS-changing-probabilities}, we can prove our main theorem~\cref{thm:main-emp}.
\begin{proof}[Proof of Theorem~\ref{thm:main-emp}]
We pick a random permutation $\pi:[n]\to [n]$ and define $y_t=x_{\pi(t)}$, the input of user $\pi(t)$. For $j\in [\ab]$ and $t<n$, we define $q_j^t=\frac{|\{t< i\leq n\mid y_i\leq j\}|}{n-t}$ and $q_0^t=0$. Thus the map $j\mapsto q_j^t$ is the empirical CDF of the users $y_{t+1},\dots, y_n$. 

Our algorithm uses the algorithm of~\cref{thm:NBS-changing-probabilities} to solve \texttt{AdvMonotonicNBS}$(1/2,\alpha\eps/8,1)$ with the adversarial probabilities $\{p_j^t\}_{j=1}^\ab$ to be described shortly. To do so, whenever the algorithm calls for flipping a coin $j$ at step $t$, we sample a new user $x_{\pi(t)}$ and apply randomized response to the variable $[x_{\pi(t)} \leq j]$, retaining the bit with probability $\frac{e^\eps}{1+e^\eps}$ and flipping it otherwise, to get a variable $z_j^t$. By standard properties of randomized response, this protocol satisfies the $\eps$-LDP requirement. Moreover, the probability $p_j^t$ that $z_j^t=1$ is $p_j^t=q_j^t\cdot \frac{e^\eps}{1+e^\eps}+(1-q_j^t)\cdot \frac{1}{1+e^\eps}$ and so
\begin{align}\label{eq:use-GP2}
|p_j^t-1/2|=\left|\lambda_j^t\cdot \frac{e^\eps-1}{1+e^\eps}\right|\geq \frac{\eps|\lambda_j^t|}{4},
\end{align}
where we have written $q_j^t=1/2+\lambda_j^t$.
Using that $n\gg \frac{\log \ab}{\eps^2\alpha^2}\gg \frac{\log \ab}{\alpha^2}$, it follows from Lemma~\ref{lemma:CDF-bound}, that $|q^{t}_j-q^{0}_j|\leq \alpha/5$ for all $t\leq n/2$ and $0\leq j\leq \ab$ with high probability in $\ab$. Thus,
\[
|p_j^t-p_j^0|=\frac{|q_j^t-q_j^0|(e^\eps-1)}{1+e^\eps}\leq \frac{\eps\alpha}{10},
\]
where the bound $\frac{e^\eps-1}{1+e^\eps}\leq \eps/2$ follows from a second degree Taylor expansion of the maps $f:\eps\mapsto \frac{e^\eps-1}{1+e^\eps}$ observing that $f'(0)=1/2$ and $f''(\eps)<0$.

It now follows from Theorem~\ref{thm:NBS-changing-probabilities}, that using the noisy feedback from at most $n/2$ of the users, the algorithm finds an $(1/2, \frac{\alpha\eps}{4})$-good coin $j^*$ with high probability in $\ab$. In particular  $p_{j^*}^0\leq \frac{1}{2}+\frac{\alpha\eps}{4}$ and $p_{j^*+1}^0\geq \frac{1}{2}-\frac{\alpha\eps}{4}$. It thus follows from equation~\eqref{eq:use-GP2} that $q_{j^*}^0\leq 1/2+\alpha$ and $q_{j^*+1}^0\geq 1/2-\alpha$. Therefore $j^*+1$ is an $\alpha$-approximate median of $\{x_i\}_{i=1}^n$ completing the proof.
\end{proof}
\section{Experiment}
\label{subsec:experiments}

\begin{figure*}[t!]
    \centering
    \includegraphics[width=\textwidth]{figure/visualization.pdf} 
        \captionof{figure}{Examples of generated videos by \sys{} and original implementation on CogVideoX-v1.5-I2V and HunyuanVideo-T2V. We showcase four different scenarios: (a) minor scene changes, (b) significant scene changes, (c) rare object interactions, and (d) frequent object interactions. \sys{} produces videos highly consistent with the originals in all cases, maintaining high visual quality.}
        \label{fig:SVG-visualization} 
\end{figure*}

\subsection{Setup}
\label{subsec:experiment_setup}

\textbf{Models.} We evaluate \sys{} on open-sourced state-of-the-art video generation models including CogVideoX-v1.5-I2V, CogVideoX-v1.5-T2V, and HunyuanVideo-T2V to generate $720$p resolution videos. After 3D VAE, CogVideoX-v1.5 consumes $11$ frames with $4080$ tokens per frame in \attn{}, while HunyuanVideo works on $33$ frames with $3600$ tokens per frame.


\textbf{Metrics.} We assess the quality of the generated videos using the following metrics. We use Peak Signal-to-Noise Ratio (PSNR), Learned Perceptual Image Patch Similarity (LPIPS)~\citep{zhang2018perceptual}, Structural Similarity Index Measure (SSIM) to evaluate the generated video's similarity, and use VBench Score~\citep{huang2023vbenchcomprehensivebenchmarksuite} to evaluate the video quality, following common practices in community~\citep{5596999,zhao2024pab,li2024svdquant,li2024distrifusion}. Specifically, we report the imaging quality and subject consistency metrics, represented by VBench-1 and VBench-2 in our table.

\textbf{Datasets.} For CogVideoX-v1.5, we generate video using the VBench dataset after prompt optimization, as suggested by CogVideoX~\cite{yang2024cogvideox}. 
For HunyuanVideo, we benchmark our method using the prompt in Penguin Video Benchmark released by HunyuanVideo~\cite{kong2024hunyuanvideo}.

% We follow standard practices in evaluating video generation models.
% Specifically, we assess the quality of the generated videos using the following metrics: Peak Signal-to-Noise Ratio (PSNR), Learned Perceptual Image Patch Similarity (LPIPS), Structural Similarity Index Measure (SSIM), and VBench Score.
% PSNR measures pixel-level fidelity by quantifying the difference between generated and ground-truth frames, where higher scores indicate better preservation of fine details. 
% LPIPS evaluates perceptual similarity based on feature representations, while SSIM assesses the structural similarity within video frames. 
% VBench provides a comprehensive evaluation of video quality that aligns closely with human perception. 
% Among these metrics, our method achieves notably high PSNR, demonstrating superior pixel fidelity while maintaining perceptual and structural quality.

\textbf{Baselines.} We compare \sys{} against sparse attention algorithms DiTFastAttn~\cite{yuan2024ditfastattnattentioncompressiondiffusion} and MInference~\cite{jiang2024minference}. As DiTFastAttn can be considered as \spatialhead{} only algorithm, we also manually implement a \temporalhead{} only baseline named \textit{Temporal-only attention}. We also include a cache-based DiT acceleration algorithm PAB~\cite{zhao2024pab} as a baseline.


\textbf{Parameters.} For MInference and PAB, we use their official configurations. For \sys{}, we choose $c_s$ as $4$ frames and $c_t$ as $1224$ tokens for CogVideoX-v1.5, while $c_s$ as $10$ frames and $c_t$ as $1200$ tokens for HunyuanVideo. Such configurations lead to approximately $30$\% sparsity for both \spatialhead{} and \temporalhead{}, which is enough for lossless generation in general. We skip the first $25$\% denoising steps for all baselines as they are critical to generation quality, following previous works~\cite{zhao2024pab,li2024distrifusion,lv2024fastercache,liu2024timestep}.



\begin{figure}[t]
    \centering
    \includegraphics[width=0.95\columnwidth]{figure/efficiency-breakdown.pdf} 
    \caption{The breakdown of end-to-end runtime of HunyuanVideo when generating a $5.3$s, $720$p video. \sys{} effectively reduces the end-to-end inference time from $2253$ seconds to $968$ seconds through system-algorithm co-design. Each design point contributes to a considerable improvement, with a total $2.33\times$ speedup.}
    \label{fig:efficiency-breakdown-figure}
\end{figure}



\subsection{Quality evaluation}
\label{subsec:quality_benchmark}
We evaluate the quality of generated videos by \sys{} compared to baselines and report the results in Table~\ref{table:accuracy_efficiency_benchmark}. Results demonstrate that \sys{} \textbf{consistently outperforms} all baseline methods in terms of PSNR, SSIM, and LPIPS while achieving \textbf{higher end-to-end speedup}.


Specifically, \sys{} achieves an average PSNR exceeding \textbf{29.55} on HunyuanVideo and \textbf{29.99} on CogVideoX-v1.5-T2V, highlighting its exceptional ability to maintain high fidelity and accurately reconstruct fine details.
For a visual understanding of the video quality generated by \sys{}, please refer to Figure \ref{fig:SVG-visualization}.

\sys{} maintains both \textbf{spatial and temporal consistency} by adaptively applying different sparse patterns, while all other baselines fail. E.g., since the mean-pooling block sparse cannot effectively select slash-wise temporal sparsity (see Figure~\ref{fig:spatial-temporal-illustration}), MInference fails to account for temporal dependencies, leading to a substantial PSNR drop. Besides, PAB skips computation of \attn{} by reusing results from prior layers, which greatly hurts the quality.


Moreover, \sys{} is compatible with \textbf{FP8 attention quantization}, incurring only a $0.1$ PSNR drop on HunyuanVideo. Such quantization greatly boosts the efficiency by $1.3\times$. Note that we do not apply FP8 attention quantization on CogVideoX-v1.5, as its head dimension of $64$ limits the arithmetic intensity, offering no on-GPU speedups.


% \begin{table*}[t]
% \centering
% \caption{Quality and Efficiency Benchmark for Video Models.}
% \label{table:accuracy_efficiency_benchmark}
% \resizebox{\linewidth}{!}{%
% \begin{tabular}{c|l|ccccc|cccc}
% \toprule
% \textbf{Type} & \textbf{Method} & \multicolumn{5}{c|}{\textbf{Quality}} & \multicolumn{4}{c}{\textbf{Efficiency}} \\
% \cmidrule(lr){3-7} \cmidrule(lr){8-11}
% & & PSNR $\uparrow$ & SSIM $\uparrow$ & LPIPS $\downarrow$ & VBench-1 $\uparrow$ & VBench-2 $\uparrow$ & FLOPS $\downarrow$ & Peak Memory $\downarrow$ & Latency $\downarrow$ & Speedup $\uparrow$ \\
% \midrule
% \textbf{I2V} & CogVideoX-v1.5 (720p, 10s, 80 frames) & - & - & - & 70.09\% & 95.37\% & 147.87 PFLOPs &  & 528s & 1x \\
% \midrule
% & DiTFastAttn (Spatial-only) & 24.591 & 0.836 & 0.167 & 70.44\% & 95.29\% & 78.86 PFLOPs &  & 338s  & 1.56x \\
% & Temporal-only & 23.839 & 0.844 & 0.157 & 70.37\% & 95.13\% & 70.27 PFLOPs &  & 327s & 1.61x \\
% & MInference & 22.489 & 0.743 & 0.264 & 58.85\% & 87.38\% & 84.89 PFLOPs &  &  &  \\
% & PAB & 23.234 & 0.842 & 0.145 & 69.18\% & 95.42\% & 105.88 PFLOPs &  &  &  \\
% \rowcolor{lightblue}
% & Ours & \textbf{\textcolor{darkgreen}{28.165}} & \textbf{\textcolor{darkgreen}{0.915}} & \textbf{\textcolor{darkgreen}{0.104}} & 70.41\% & 95.29\% & 74.57 PFLOPs &  & 237s & \textcolor{darkgreen}{2.23x} \\
% % \rowcolor{lightblue}
% % & Ours + FP8 & 26.709 & 0.890 & 0.122 &  &  &  &  & \\
% \midrule
% \textbf{T2V} & CogVideoX-v1.5 (720p, 10s, 80 frames) & - & - & - & 62.42\% & 98.66\% & 147.87 PFLOPs &  & 528s & 1x \\
% \midrule
% & DiTFastAttn (Spatial-only) & 23.202 & 0.741 & 0.256 & 62.22\% & 96.95\% & 78.86 PFLOPs &  & 338s & 1.56x \\
% & Temporal-only & 23.804 & 0.811 & 0.198 & 62.12\% & 98.53\% & 70.27 PFLOPs &  & 327s & 1.61x \\
% & MInference & 22.451 & 0.691 & 0.304 & 54.87\% & 91.52\% & 84.89 PFLOPs &  &  &  \\
% & PAB & 22.486 & 0.740 & 0.234 & 57.32\% & 98.76\% & 400.04 PFLOPs &  &  &  \\
% \rowcolor{lightblue}
% & Ours & \textbf{\textcolor{darkgreen}{29.989}} & \textbf{\textcolor{darkgreen}{0.910}} & \textbf{\textcolor{darkgreen}{0.112}} & 63.01\% & 98.67\% & 74.57 PFLOPs &  & 232s & \textbf{\textcolor{darkgreen}{2.28x}} \\
% % \rowcolor{lightblue}
% % & Ours + FP8 &  &  &  &  &  &  &  &  \\
% \midrule
% \textbf{T2V} & HunyuanVideo (720p, 5.33s, 128 frames) & - & - & - & 66.11\% & 93.69\% & 612.37 PFLOPs &  & 2253s & 1x \\
% \midrule
% & DiTFastAttn (Spatial-only) & 21.416 & 0.646 & 0.331 & 67.33\% & 90.10\% & 260.48 PFLOPs &  & 1238s & 1.82x \\
% & Temporal-only & 25.851 & 0.857 & 0.175 & 62.12\% & 98.53\% & 259.10 PFLOPs &  & 1231s & 1.83x \\
% & MInference & 23.157 & 0.823 & 0.163 &  &  & 293.87 PFLOPs &  &  &  \\
% & PAB & - & - & - & - &  & - & \color{red}OOM & - & - \\
% \rowcolor{lightblue}
% & Ours & \textbf{\textcolor{darkgreen}{29.546}} & \textbf{\textcolor{darkgreen}{0.907}} & \textbf{\textcolor{darkgreen}{0.127}} & 65.90\% & 93.51\% & 259.79 PFLOPs &  & 1171s & 1.92x \\
% \rowcolor{lightblue}
% & Ours + FP8 & \textbf{\textcolor{darkgreen}{29.452}} & \textbf{\textcolor{darkgreen}{0.906}} & \textbf{\textcolor{darkgreen}{0.128}} & 65.70\% & 93.51\% & 259.79 PFLOPs &  & 968s & \textbf{\textcolor{darkgreen}{2.33x}} \\
% \bottomrule
% \end{tabular}%
% }
% \end{table*}



\subsection{Efficiency evaluation}
\label{subsec:efficiency_benchmark}

To demonstrate the feasibility of \sys{}, we prototype the entire framework with dedicated CUDA kernels based on FlashAttention~\cite{dao2022flashattentionfastmemoryefficientexact}, FlashInfer~\cite{ye2025flashinferefficientcustomizableattention}, and Triton~\cite{Tillet2019TritonAI}. We first showcase the end-to-end speedup of \sys{} compared to baselines on an H100-80GB-HBM3 with CUDA 12.4. Besides, we also conduct a kernel-level efficiency evaluation. Note that all baselines adopt FlashAttention-2~\cite{dao2022flashattentionfastmemoryefficientexact}.


\begin{table}[t]
\small
\centering
\caption{Inference speedup of customized QK-norm and RoPE compared to PyTorch implementation with different number of frames. We use the same configuration of CogVideoX-v1.5, i.e. $4080$ tokens per frame, $96$ attention heads.}
\label{table:small-kernel-speedup-comparison}
\begin{tabular}{c|cccc}
\toprule
Frame Number & 8 & 9 & 10 & 11  \\
\midrule
%LayerNorm & 7.436× & 7.448× & 7.464× & 7.474×  \\
QK-norm & 7.44× & 7.45× & 7.46× & 7.47×  \\
\midrule
RoPE & 14.50× & 15.23× & 15.93× & 16.47×   \\
\bottomrule
\end{tabular}
\end{table}


\textbf{End-to-end speedup benchmark.} We incorporate the end-to-end efficiency metric including FLOPS, latency, and corresponding speedup into Table~\ref{table:accuracy_efficiency_benchmark}. \sys{} consistently outperforms all baselines by achieving an average speedup of $2.28\times$ while maintaining the highest generation quality. We further provide a detailed breakdown of end-to-end inference time on HunyuanVideo in Figure~\ref{fig:efficiency-breakdown-figure} to analyze the speedup. Each design point described in Sec~\ref{sec:methodology} contributes significantly to the speedup, with sparse attention delivering the most substantial improvement of $1.81\times$.

\textbf{Kernel-level efficiency benchmark.}\label{subsec:kernel_level_efficiency} We benchmark individual kernel performance including QK-norm, RoPE, and block sparse attention with unit tests in Table~\ref{table:small-kernel-speedup-comparison}. Our customized QK-norm and RoPE achieve consistently better throughput across all frame numbers, with an average speedup of $7.4\times$ and $15.5\times$, respectively. For the sparse attention kernel, we compare the latency of our customized kernel with the theoretical speedup across different sparsity. As shown in Figure~\ref{fig:kernel-efficiency-sparse-attention}, our kernel achieves theoretical speedup, enabling practical benefit from sparse attention.


\begin{figure}[t]
    \centering
    \includegraphics[width=\columnwidth]{figure/LayourTransformSpeed3.pdf} 
    % \vspace{-2pt}
    \caption{Latency comparison of different implementations of sparse attention. Our hardware-efficient \reorder{} optimizes the sparsity pattern of \temporalhead{} for better contiguity, which is $1.7$× faster than naive sparse attention (named original), approaching the theoretical speedup.}
    \label{fig:kernel-efficiency-sparse-attention}
    \vspace{-5pt}
\end{figure}

\begin{table}[t]
\centering
\caption{Sensitivity test on \onlinesample{} ratios. Profiling just $1$\% tokens achieves generation quality comparable to the oracle ($100$\%) while introducing only negligible overhead.}
\label{table:sensitivity-sampling}
\begin{tabular}{l|ccc}
\toprule
\textbf{Ratios} & \textbf{PSNR $\uparrow$} & \textbf{SSIM $\uparrow$} & \textbf{LPIPS $\downarrow$} \\
\midrule
\multicolumn{4}{c}{\textbf{CogVideoX-v1.5-I2V (720p, 10s, 80 frames)}} \\
\midrule
profiling 0.1\% & 30.791 & 0.941 & 0.0799 \\
profiling 1\% & 31.118 & 0.945 & 0.0757\\
profiling 5\% & 31.008 & 0.944 & 0.0764\\
profiling 100\% & 31.324 & 0.947 & 0.0744 \\
% \midrule
% \multicolumn{4}{l}{\textbf{CogVideoX V1.5 (720p, 10s, 80 frames)}} \\
% \midrule
% No threshold & 31.118 & 0.945 & 0.0757\\
% threshold=10 & 31.304 & 0.949 & 0.0722\\
% threshold=1 & 31.322 & 0.949 & 0.0717\\
% threshold=0.1 & 31.217 & 0.949& 0.0720\\
\bottomrule
\end{tabular}
\end{table}

\subsection{Sensitivity test}
\label{subsec:sensitivity-test}
In this section, we conduct a sensitivity analysis on the hyperparameter choices of \sys{}, including the \onlinesample{} ratios (Sec~\ref{subsec:sampling_based_pattern_selection}) and the sparsity ratios $c_s$ and $c_t$ (Sec~\ref{subsec:frame_token_rearrangement}). Our goal is to demonstrate the robustness of \sys{} across various efficiency-accuracy trade-offs.


\textbf{\Onlinesample{} ratios.} We evaluate the effectiveness of \onlinesample{} with different profiling ratios on CogVideoX-v1.5 using a random subset of VBench in Table~\ref{table:sensitivity-sampling}. In our experiments, we choose to profile only 1\% of the input rows, which offers a comparable generation quality comparable to the oracle profile (100\% profiled) with negligible overhead.

%Profiling only $1$\% of the input data achieves nearly the same generation quality as the oracle profiling ($100$\% sampling), with only a $0.2$ PSNR reduction. Therefore, we adopt this scheme as the default setting, as it provides accuracy comparable to the oracle with negligible overhead.


\textbf{Generation quality over different sparsity ratios.} As discussed in Sec~\ref{sec:sparse-theoretical-speedup}, different sparsity ratio of the \spatialhead{} and \temporalhead{} can be set by choosing different $c_s$ and $c_t$, therefore reaching different trade-offs between efficiency and accuracy. We evaluate the LPIPS of HunyuanVideo over a random subset of VBench with different sparsity ratios. As shown in Table~\ref{table:sensitivity-sparsity-ratios}, \sys{} consistently achieves decent generation quality across various sparsity ratios. E.g., even with a sparsity of $13$\%, \sys{} still achieves $0.154$ LPIPS. We leave the adaptive sparsity control for future work.


\subsection{Ablation study}
\label{subsec:ablation}
We conduct the ablation study to evaluate the effectiveness of the proposed hardware-efficient \reorder{} (as discussed in Sec~\ref{subsec:frame_token_rearrangement}). Specifically, we profile the latency of the sparse attention kernel with and without the transformation under the HunyuanVideo configuration. As shown in Figure~\ref{fig:kernel-efficiency-sparse-attention}, the sparse attention with \reorder{} closely approaches the theoretical speedup, whereas the original implementation without \reorder{} falls short. For example, at a sparsity level of $10$\%, our method achieves an additional $1.7\times$ speedup compared to the original approach, achieving a $3.63\times$ improvement.

\begin{table}[t]
\small
\centering
\caption{Video quality of HunyuanVideo on a subset of VBench when varying sparsity ratios. LPIPS decreases as the sparse ratio increases, achieving trade-offs between efficiency and accuracy.}
\label{table:sensitivity-sparsity-ratios}
\begin{tabular}{c|cccccc}
\toprule
Sparsity$\downarrow$ & 0.13 & 0.18 & 0.35 & 0.43 & 0.52 \\
\midrule
LPIPS$\downarrow$ & 0.154 & 0.135 & 0.141 & 0.129 & 0.116 \\
\bottomrule
\end{tabular}
\vspace{-5pt}
\end{table}



% \paragraph{Robustness of Sparse Attention} To further assess the robustness of our sparse attention mechanism, we examine its performance under different MSE thresholds. 
% As discussed in Section \ref{sec:sparse_patterns}, approximately 10\% of attention heads exhibit high MSE values ($\ge$0.1) under both Arrow Mask and Zebra Mask. 
% To address these edge cases, we calculate full attention for heads with MSE values exceeding a given threshold (0.1, 1, or 10). 
% As shown in Table \ref{table:ablation_study}, the PSNR remains consistent across all threshold settings, indicating that these rare corner cases do not significantly impact overall performance.

% \paragraph{Impracticality of Offline Calibration} We explore whether sparse pattern selection can be pre-determined through offline calibration. 
% A visual comparison of sparse patterns selected for two videos generated by CogVideoX is presented in Figure \ref{}. 
% The patterns show no clear correlation between the two videos, indicating that sparse attention patterns vary significantly depending on the content and context of each video. This result demonstrates that offline calibration is infeasible for video generation tasks, further validating the need for our online sampling-based method.
%
\section*{Acknowledgements} Aamand, Pagh, and Imola carried out this work at Basic Algorithms Research Copenhagen (BARC), which was supported by the VILLUM Foundation grant 54451. Pagh and Imola were supported by a Data Science Distinguished Investigator grant from the Novo Nordisk Fonden. Boninsegna was supported in part by the Big-Mobility project by the University of Padova under the Uni-Impresa call, by the MUR PRIN 20174LF3T8 AHeAD project, and by MUR PNRR CN00000013 National Center for HPC, Big Data and Quantum Computing.




\bibliography{bibliography}
\bibliographystyle{icml2025}


%
%
%
%
%
\newpage
\appendix
\onecolumn

%
\section{Additional Definitions}\label{app:add-def}
\begin{lemma}[Binary Randomized Response~\cite{warner1965randomized,dwork2006calibrating}]
\label{def: binary rr}
    For a binary input $x\in\{0,1\}$, and privacy parameter $\priv$, the following protocol $\mathcal{M}\to \{0,1\}$ satisfies $\priv$-LDP:
    \begin{equation*}
        \pmech(x)=\begin{cases}
        x,&\text{w.p. }\frac{e^\priv}{e^\priv +1}\\
        1-x,&\text{otherwise.}
    \end{cases}
    \end{equation*}
%
    %
        %
   %
\end{lemma}
\textbf{Azuma's inequality.}
We will use the following version of Azuma's inequality which bounds the maximum deviation of a martingale $(X_i)_{i=0}^n$ at any time $t=0,\dots, n$. See Theorem 2.1 in \cite{Fan2012martingales} for a stronger and more general bound.
\begin{theorem}[Azuma's inequality]\label{thm:azuma} Let $(X_i)_{i=0}^n$ be a martingale such that $X_0=0$ and $|X_{i+1}-X_i|\leq 1$ for all $0\leq i<n$. For any $t\geq 0$, 
\[
\Pr[\max_{1\leq i \leq n}|X_i|\geq t]\leq 2\exp\left(\tfrac{-t^2}{2n} \right).
\]
\end{theorem}

\textbf{Bernstein's Inequality.} We use the following variant of Bernstein's Inequality in the proof of~\cref{thm:main-shuffle}, see~\citet[Proposition 2.10]{Wainwright_2019} for a detailed overview.
\begin{theorem}[Bernstein's Inequality]\label{fact: bernstein}
    Let $\{X_i\}_{i=1}^n$ be independent random variables that are bounded almost surely by $1$. Let $\sigma^2=\frac{1}{n}\sum_{i=1}^n\operatorname{Var}[X_i]$ be the average variance. We then have,
    \[
    \Pr\left[\bigg|\frac{1}{n}\sum\limits_{i=1}^n X_i-\frac{1}{n}\sum\limits_{i=1}^n\bEE{X_i}\bigg|>\acc\right]\leq\exp\left( \frac{-n\acc^2}{2\sigma^2 + \frac{2\acc}{3}} \right).
    \]
\end{theorem}
\section{Reduction to the Median} 

\label{appendix: Reduction to the Median}
Consider the simple case where we are given an algorithm $A$ which returns the median of $n$ samples in the most natural sense, by returning the $n/2$'th index of their sorted representation. Without changing this algorithm we can have it return any arbitrary index by adding elements to the beginning or the end of this sorted array. For example, adding two elements to the beginning of the array will create a new array with $n'=n+2$ elements where the $n'/2$'th index will be the $(n/2-1)$'th index of the original array. The padding argument below formalizes this notion, demonstrating that any algorithm for an $\alpha$-approximation of the median can be used to obtain a $2\alpha$-approximation of any quantile.
\begin{lemma}[Padding Argument] 
\label{appendix: padding argument}
Any $\alpha$-approximation algorithm for the median, with $\alpha \in \left(0,\frac{1}{2}\right)$, can be used to construct a $2\alpha$-approximation for any quantile $\tau\in (0,1)$. 
\end{lemma}
\begin{proof}
    Consider a dataset $D=\{x_1, \dots, x_n\}$ where each element is such that $x_i \in \{1,\dots,\ab\}$. Let $\mathcal{M}$ be an algorithm for the $\alpha$-approximation of the median then for $m = A(D)$ we have by definition
     \begin{equation}
     \label{eq: appendix padded 1}
        \text{Pr}_{\mathcal{D}}[x\leq m]<\frac{1}{2}+\alpha \qquad \text{and} \qquad \text{Pr}_{\mathcal{D}}[x\leq m+1]>\frac{1}{2}-\alpha.
    \end{equation}
    where $\text{Pr}_{D}[x\leq m] = \frac{\sum_{x\in D}[x\leq m]}{n}$, and $[x\leq m]$ is an indicator function. 
    Consider now a padded dataset $D_P = D\cup \{1\}^{(1-\tau)n} \cup \{\ab\}^{\tau n}$, where $\{a\}^{x}$ indicates the multi-set containing the $a$ element $x$ times \footnote{We consider $(1-\tau)n$ and $\tau n$ integers.}. The new empirical cumulative distribution of the data set for $y \in \{1, \dots, \ab-1\}$, is \begin{align*}
    \label{eq: appendix padded 2}
        \text{Pr}_{D_P}[x\leq y] &= \frac{(1-\tau)n +\sum_{x\in D}[x\leq y]}{|D_P|} = \frac{1-\tau}{2}+\frac{1}{2}\text{Pr}_{D}[x\leq y],
    \end{align*}
    as we have $|D_P| = 2n$. Thus 
    \begin{equation}
    \label{eq: appendix padded 3}
        \text{Pr}_{D}[x\leq y] = 2\text{Pr}_{D_P}[x\leq y] +\tau -1.
    \end{equation}
    The application of $A$ to the padded data set $D_{P}$ returns a $\alpha$-approximate median $m_P = A(D_P)$. Therefore, for $m_P\in\{1,\dots, \ab-1\}$, from \autoref{eq: appendix padded 3} and \autoref{eq: appendix padded 1} it follows that 
    \begin{equation}
    \label{eq: appendix padded 4}
       \text{Pr}_{D}[x\leq m_P]<\tau+2\alpha \qquad \text{and} \qquad \text{Pr}_{D}[x\leq m_P+1]>\tau-2\alpha.
    \end{equation}
    Notice that $m_p\neq \ab$, as $\text{Pr}_{D_P}[x\leq \ab]=1<\frac{1}{2}+\alpha$ iff $\alpha>\frac{1}{2}$. This concludes the proof.
\end{proof}

%
%
%
%
%
%
%

\section{Statistical Private Median Estimation}\label{sec:statistical-median}
In this section, we will provide an algorithm for \texttt{LDPstat-median} using the state-of-the-art algorithm for \texttt{MonotonicNBS}. We prove the following:

\begin{theorem}\label{thm:main-stat}
Let $\alpha \in \left(0,\frac{1}{4}\right)$ and $\varepsilon
>0$. Suppose that the number of users $n\geq C\frac{\log B}{\alpha^2}\left(\frac{e^\varepsilon+1}{e^\varepsilon-1}\right)^2$ for a sufficiently large constant $C$. Then there exists an algorithm solving \texttt{LDPstat-median}$(\mathcal{D},n,\alpha,\eps)$ with high probability in $B$. 
\end{theorem}

In this section, we prove Theorem~\ref{thm:main-stat}. For this, we recall the following result which is a corollary of the main result in~\cite{gretta2023sharp}. 
Recall the definition of an $\left(\frac{1}{2}, \alpha\right)$-good coin in~\eqref{eq:good-coin}.
\begin{theorem}[\cite{gretta2023sharp}]\label{thm:from-GP}
For any $\alpha \in \left(0,\frac{1}{4}\right)$, there exists an algorithm for \texttt{MonotonicNBS}$(\tau,\alpha)$ which uses $O(\frac{\log B}{\alpha^2})$ coin flips and outputs an $\left(\frac{1}{2},\alpha\right)$-good coin with high probability in $B$.
\end{theorem}
\begin{proof}[Proof of Theorem~\ref{thm:main-stat}]
For $i\in [B]$, we define $q_i=\sum_{j\leq i}\mathcal{D}[j]$ with the convention that $q_0=0$. 
Thus $j\mapsto q_j$ is the CDF of $\mathcal{D}$. Consider sampling $X\sim \mathcal{D}$ and let $Y$ be the random variable obtained by applying randomized response to the indicator variable $[X\leq j]$ retaining the bit with probability $\frac{e^\eps}{1+e^\eps}$ and flipping it otherwise. Then $\Pr[Y=1]=p_j$ where $p_j=q_j\cdot \frac{e^\eps}{1+e^\eps}+(1-q_j)\cdot \frac{1}{1+e^\eps}$. Then,
\begin{equation}\label{eq:use-GP}
    q_{j} = \left(p_{j}-\frac{1}{e^\varepsilon+1}\right)\frac{e^\varepsilon+1}{e^\varepsilon-1},
\end{equation}
We use the the algorithm in Theorem~\ref{thm:from-GP} to solve \texttt{MonotonicNBS}$\left(\frac{1}{2},\alpha\frac{e^\varepsilon-1}{e^\varepsilon+1}\right)$ when the inputs are the unknown $\{p_i\}_{i=1}^B$. To do so, whenever the algorithm calls for flipping a coin $j$, we sample a new user $X\in \mathcal{D}$ and apply randomized response to the variable $Y=[X\leq j]$. By standard properties of randomized responze, this protocol satisfies the $\eps$-LDP requirement. Moreover, by Theorem~\ref{thm:from-GP}, the algorithm finds an $\left(\frac{1}{2}, \alpha\frac{e^\varepsilon-1}{e^\varepsilon+1}\right)$-good coin $j^*$ with high probability in $B$. In particular $p_{j^*}\leq \frac{1}{2}+\alpha\frac{e^\varepsilon-1}{e^\varepsilon+1}$ and $p_{j^*+1}\geq \frac{1}{2}-\alpha\frac{e^\varepsilon-1}{e^\varepsilon+1}$. 
It thus follows from Equation~\eqref{eq:use-GP} that $q_{j^*}\leq 1/2+\alpha$ and $q_{j^*+1}\geq 1/2-\alpha$. Therefore $j^*$ is an $\alpha$-approximate median of $\mathcal{D}$ completing the proof.
\end{proof}
In the high privacy regime, i.e. for $\varepsilon<1$ , the sample complexity of Theorem \ref{thm:main-stat} becomes $n=\Omega\left(\frac{\log B}{\varepsilon
^2\alpha^2}\right)$, matching our lower bound up to a constant factor.
\section{The Hierarchical Mechanism}\label{app:hierarchical-mech}
The algorithm was presented in \cite{kulkarni2019answering} and can be used to approximately answer general range queries. It comes in several variants but we will present the simplest version (the bounds on the number of users needed for the various versions are similar). The main idea is to construct a $b$-ary tree of depth $\Theta(\log(B))$ on $[B]$. For the below, we will assume that $B$ is a power of $2$ and that $b=2$ (although for the experiments, we use a different constant $b$). The nodes on level $i$ (where level 0 is the root) corresponds to the $2^i$ dyadic intervals of $B$. Namely, in the binary representation of elements of $B$, there is an interval corresponding to each prefix of length $i$ in the binary representation. The non-adaptive protocol we will consider is as follows. Each user $i$ with data $x_i\in[B]$ picks a random level $\ell$ of the binary tree. The user writes a one-hot encoding $z$ of which node they belong to on level $\ell$ and uses randomized response on each of the $2^\ell$ bits of $z$. This is the message $y$, they send to the central server. This is the unary encoding mechanism; see~\cite{kulkarni2019answering} for more sophisticated solutions, that require less communication but nonetheless have the same approximation errors. The combined algorithm is denoted \texttt{Hierarchical Mechanisms}. 

\paragraph{Analysis sketch of \texttt{Hierarchical Mechanism}}
We here analyse the performance of \texttt{Hierarchical Mechanism} for answering general range queries and in particular show how it can be used for quantile estimation.

Assume that $\eps\leq 1$. If the number of users reporting at every level is $\gg \frac{1}{\alpha_0^2\eps^2}$ (where $a\gg b$ means that $a\geq C b$ for some constant $C$), then using standard concentration bounds, for each node in a given level, we can recover the total fraction of users lying in the corresponding subtree up to an additive $\alpha_0$ with constant failure probability. Now if the total number of users is $\gg \frac{\log B}{(\alpha_0^2\eps^2)}$, then with constant failure probability, the number of users reporting at any given level is indeed, $\gg \frac{1}{\alpha_0^2\eps^2}$. We now pick $\alpha_0=\alpha/(2\log B)$ and conclude that if the number of users is $\gg \frac{(\log B)^3}{(\alpha\eps^2)}$, we can recover the total fraction of users lying in any subtree up to an additive $\alpha/(2\log B)$ from the unary responses with constant failure probability. 
It follows that we can answer any range query with additive error $\alpha n$. Indeed, any range can be partitioned into at most $2\log B$ of these subtrees, two for each level. In particular, this means that we can find an $\alpha$-approximate median with constant failure probability.
It follows that we can answer any range query with additive error $\alpha n$. Indeed, any range can be partitioned into at most $2\log B$ of these subtrees, two for each level. In particular, this means that we can find an $\alpha$-approximate median with constant failure probability.
The analysis for high probability in $B$ needs $\gg \frac{\log B}{\alpha_0^2\varepsilon^2}$ number of users reporting at each level, so it adds an additional $\log B$ factor to the sample complexity.


%



\section{Proof of Theorem \ref{thm:NBS-changing-probabilities}}
\label{app: proof theorem 3.1} 
\begin{algorithm}[t]
\caption{\texttt{BayeSS} main steps }\label{alg: bayeSS}
\begin{algorithmic}
\STATE {\bfseries Input:} $\{x_i\}_{i=1,\dots, n}$, $\alpha \in (0,1/4)$, $n\geq C\frac{\log B}{\alpha^2}$
\STATE $L\gets \texttt{BayesLearn}(B, \{x_i\}_{i=1,\dots,n/4}, \alpha)$
\STATE $R \gets \frac{1}{\gamma}$-$\text{quantiles}(L)$ \COMMENT{for $\gamma = O(1)$}
\STATE \textbf{return } \texttt{TestCoins}$(R, \{x_i\}_{n/4+1,\dots, n/2}, \alpha)$
\end{algorithmic}
\end{algorithm} 
The goal of this section is to prove Theorem~\ref{thm:NBS-changing-probabilities}. We first define the adversarial setting.
\begin{definition}
\label{def:adversarial}
Let $0<\alpha<1$ and $\ab$ a positive integer. Let $p_0,\dots,p_\ab\in [0,1]$ be unknowns with $0=p_0\leq\cdots \leq p_\ab= 1$. In \emph{\texttt{AdvMonotonicNBS}$(\tau, \alpha, c)$}, for $c>0$, our goal is to identify an $(\tau,\alpha(1+c))$-good coin (defined in \autoref{eq:good-coin}).
To do so, we may iteratively pick indices $i\in \ab$. Then an adversary selects a probability $\tilde p_i$ such that $| {\tilde{p}}_i - p_i|\leq c\alpha$, and we observe the outcome of a coin flip with heads probability $\tilde p_i$.
\end{definition}
We show that the \texttt{BayeSS} algorithm (\texttt{BayeSS} abbreviates \emph{Bayesian Screening Search}) from \cite{gretta2023sharp}(Algorithm 3) solves the \texttt{AdvMonotonicNBS}$(\tau, \alpha, c)$ problem returning the a $(\tau,\alpha(1+c))$-good coin with high probability in $\ab$ using
%
$O(\frac{\tau(1-\tau)\log \ab}{\alpha^2})$ 
 coin flips. We actually prove a stronger theorem which immediately implies~\cref{thm:NBS-changing-probabilities}.

\begin{theorem}\label{thm:GP-generalization}
Suppose that $c\leq 1$ and $\alpha \leq \frac{1}{2}\min\{\tau, 1-\tau\}$. There exists an algorithm~\cite{gretta2023sharp} for \emph{\texttt{AdvMonotonicNBS}$(\tau, \alpha, c)$} which uses 
%
$\tfrac{1}{C_{\tau, \alpha}}(\log \ab + O(\log^{2/3}\ab\,\log^{1/3}\frac{1}{\failp}+\log\frac{1}{\failp}))$ 
coin flips\footnote{Namely, $C_{\tau, \alpha}$ is the information capacity of the Binary Asymmetric Channel (BAC) with crossover probabilities $\{\tau + \alpha, \tau - \alpha\}$. Concretely, $C_{\tau, \alpha}=\max_q H((1-q)(\tau-\alpha) + q(\tau+\alpha))-(1-q)H(\tau-\alpha)-qH(\tau+\alpha)$ with $H$ being the binary entropy function, and $C_{\tau,\alpha} = \Theta(\tfrac{\alpha^2}{\tau(1-\tau)})$ for $\alpha \leq \frac{1}{2}\min(\tau, 1-\tau)$.}
%
and returns a $(\tau,\alpha(1+c))$-good coin with probability at least $1-\failp$.
\end{theorem}

Note that Theorem~\ref{thm:NBS-changing-probabilities} follows directly from Theorem~\ref{thm:GP-generalization} by setting $\tau=1/2$ and $\failp=\ab^{-\lambda}$ for any constant $\lambda$. With this, the proof of Theorem~\ref{thm:main-emp} is complete.

 Before we delve into the proof of Theorem~\ref{thm:GP-generalization}, let us first describe the idea behind \texttt{BayeSS}, described shortly in Algorithm \ref{alg: bayeSS}.
 %
 %
 At a high level \texttt{BayeSS} proceeds in two steps allocating a portion of the coin flips for each step. The first step is a Bayes learner algorithm, called  \texttt{BayesLearn}.
 %
 It starts by assigning a uniform prior $w(I_i)$ to each coin interval $I_i=[i, i+1]$ for any $i \in [\ab-1]$, then takes the $\tau$-quantile interval under the posterior $w(I_i)$, selects a coin from this interval, flips it, and updates each $w(I_i)$ according to the result of the coin flip and the error $\alpha$. This procedure is repeated iteratively.
 The sampled intervals are collected in a multiset $L$, with the guarantee that, after \( O\big(\tfrac{(1+\gamma)\log B}{C_{\tau, \alpha}}\big) \) coin flips, a $\gamma$-fraction of intervals in  $L$  contains a $(\tau, \alpha)$-good coin with high probability in  $\ab$  (referred to as good intervals). In the second step, this property is used to narrow the set of possible coins to $O(1/\gamma)$, ensuring that it contains at least one $(\tau, \alpha)$-good coin. Each coin in the candidate set can be individually tested, up to error $\alpha$, with high probability using $O(\tfrac{1}{\gamma\alpha^2}\log(\tfrac{\ab}{\gamma}))$ coin flips.

It is easy to see that in the adversarial setting, the coins can be tested up to error $\alpha(1+c)$ in the second step. 
%
Our main challenge in proving Theorem~\ref{thm:GP-generalization}, is analyzing the first part of the algorithm, \texttt{BayesLearn}, in the adversarial setting. 
%
%
%
%
%
The authors in \cite{gretta2023sharp} used a stopping time argument to analyze \texttt{BayesLearn}. They defined a potential function $\Phi$, with an initial negative value, constructed so that a positive potential implies finding at least a $\gamma$ fraction of good intervals. The stochastic process describing the evolution of the potential $\{\Phi_{i}\}_{i=1,\dots}$ is then modeled with a submartingale that can be used to bound, using Azuma's inequality, the probability that the process crosses zero after a sufficient number of iterations. We prove that we can use the same argument for the case of adversarial probabilities if we allow the potential to catch approximate good intervals, namely intervals containing $(\tau, \alpha(1+c))$-good coin.

\paragraph{New potential} Let $\{\ell,\dots,r\}$ be the set of $(\tau,\alpha(1+c))$-good intervals. Let $a$ be the maximum $i \in [\ab-1]$ such that $p^1_i\leq \tau$. Let $L$ be the list of intervals visited in \texttt{BayesLearn}. We define the potential function as 
\begin{equation*}
\label{eq: new potential}
    \Phi(w, L) := \log_2 w(a) + 12 C_{\tau, \alpha}(|\{x\in L : x \in [\ell,r]\}|-\gamma|L|),
\end{equation*}
where $w(a)$ is the Bayesian posterior weight associated to the best interval $a$ and $C_{\tau, \alpha}$ is a concrete function of $\tau$ and $\alpha$.
Notice that a positive potential implies $|\{x\in L | x \in [\ell,r]\}| >\gamma |L|$, hence indicating the presence of a $\gamma$ fraction $(\tau, \alpha(1+c))$-good intervals in $L$. The following Lemma generalises Lemma 7 of \cite{gretta2023sharp} and allows the construction of a submartingale.
\begin{algorithm*}[t]
\caption{\texttt{BayesLearn} for empirical quantile estimation, from Algorithm 2 in \cite{gretta2023sharp}}\label{alg: BayesLearn}
\begin{algorithmic}[1]
\FUNCTION{\texttt{GetIntervalFromQuantile}$(w, q)$}{}
    \STATE $\textbf{return\, } \min i \in [B] \text{ s.t. } W(i)\geq q$ \textbf{ with } $W(x)=\sum_{i\in\{1, \dots, x\}}w(i)$
\ENDFUNCTION\\
\hspace{0.5 cm}
\FUNCTION{\texttt{RoundIntervalToCoin}$(i, w, q)$}{}
    \STATE \textbf{return } $i$ \textbf{ if } $\frac{q-W(i-1)}{w(i)}\leq q$ \textbf{ else } $i+1$ \textbf{ with } $W(x)=\sum_{i\in\{1, \dots, x\}}w(i)$
\ENDFUNCTION\\
\hspace{0.5 cm}
\FUNCTION{\texttt{BayesLearn}$(\{x_{i}\}_{i=1,\dots, n}, B, \tau, \alpha, M)$}{}
\STATE $w_1 \gets \text{uniform}([B-1])$
\STATE $q \gets \arg \max_{x}H((1-x)(\tau -\alpha)+x(\tau +\varepsilon))-(1-x)H(\tau -\alpha)-xH(\tau + \alpha)$
\STATE $I \gets \{\}$ \COMMENT{Multiset}
\FOR {$i \in [M]$}
    \STATE $j_i \gets \texttt{GetIntervalFromQuantile}(w_i, q)$
    \STATE $c_i \gets \texttt{RoundIntervalToCoin}(j_i, w_i, q)$ \COMMENT{Gets the coin from the selected interval}
    \STATE $L\gets L \cup \{j_i\}$
    \STATE $x_i \sim \{x_k\}_{k=1,\dots}$ \COMMENT{Sample a user}
    \STATE $\{x_k\}_{k=1,\dots}\gets \{x_k\}_{k=1,\dots} \setminus \{x_i\}$ \COMMENT{Remove the user from the dataset}
    \STATE $y_i \gets [x_i \leq c_i]$ \COMMENT{Flip the coin}
    \STATE $w_{i+1}(x)\gets \begin{cases}
        w_i(x)d_{\tilde{y}_i,0} & \text{if } x\in \{1, \dots, j_i-1\}\\
        d_{\tilde{y}_i,0}(q-W_i(j_i-1))+d_{\tilde{y}_i, 1}(W_{i}(j_i)-1) & \text{if } x= j_i\\
        w_{i}(x)d_{\tilde{y}_i, 1} & \text{if } x\in \{j_i +1 , \dots, B-1\}
    \end{cases}$
\ENDFOR
\STATE \textbf{return} $L$ \COMMENT{Return a multiset of intervals}
\ENDFUNCTION
\end{algorithmic}
\end{algorithm*}
\begin{lemma}[Adaptation of Lemma 7 in \cite{gretta2023sharp} for adversarial probabilities]
    \label{lemma: increase in expectation of the potential}
    For $c\leq 1$ and $\alpha \leq \frac{1}{2}\min\{\tau, 1-\tau\}$, the expected variation of the potential is 
    \begin{equation}
        \E[\Phi_{t+1}-\Phi_{t}|y_{1}, \dots, y_t] \geq (1-12\gamma)C_{\tau, \alpha},
    \end{equation}
    where $(y_1, \dots, y_t)$ are the results of the coin toss up to $t+1$-th sample, and $C_{\tau, \alpha} = \Theta\left(\frac{\tau(1-\tau)}{\alpha^2}\right)$.
\end{lemma}
\begin{proof}
The proof for the adversarial setting, which allows an adversary to alter the head coin probability at each iteration up to $c\alpha$, while preserving their order, closely resembles the proof of Lemma 7 in \cite{gretta2023sharp}, which addresses the case of fixed coin probabilities. We will go through the steps of the proof highlighting the main differences. An implementation of \texttt{BayesLearn} for empirical quantile estimation, where each user is used at most once, can be found in Algorithm \ref{alg: BayesLearn}.
%


Let's define the capacity of the $(\tau, \alpha)$-BAC (Binary Asymmetric Channel) as
\begin{align*}
    C_{\tau, \alpha} &= \max_{q} H((1-q)(\tau-\alpha)+q(\tau+\alpha)) -(1-q)H(\tau-\alpha)-qH(\tau+\alpha), \\
    q&=\arg \max_x H((1-x)(\tau-\alpha)+x(\tau+\alpha)) -(1-x)H(\tau-\alpha)-xH(\tau+\alpha),
\end{align*}
where $H(p)$ is the binary entropy. Let's define the multiplicative Bayes weights $d_{x,y}:\{0,1\}\times \{0,1\}\rightarrow \R$, they indicates the multiplicative effect of a flip resulting $x$ (1=Heads, 0=Tails) on the density of an interval on side $y$ (1=Right, 0=Left) of the flipped coin.
\begin{align*}
    d_{0,0} &= \dfrac{1-\tau-\alpha}{1-\tau-(2q-1)\alpha}\\
    d_{0,1} &= \dfrac{1-\tau+\alpha}{1-\tau-(2q-1)\alpha}\\
    d_{1,0} &= \dfrac{\tau +\alpha}{\tau+(2q-1)\alpha}\\
    d_{1,1} &= \dfrac{\tau-\alpha}{\tau+(2q-1)\alpha}.
\end{align*}
We will mainly use the results from Lemma 9 in \cite{gretta2023sharp} that states that
\begin{gather}
    C_{\tau, \alpha} = (\tau + \alpha)\log_2 d_{1,0} + (1-\tau-\alpha)\log_2 d_{0,0} \label{eq: lemma A.1 [1]},\\
    C_{\tau, \alpha} = (\tau -\alpha)\log_2 d_{1,1} + (1-\tau+\alpha)\log_2 d_{0,1} \label{eq: lemma A.1 [2]},
\end{gather}
with the fact that $d_{1,0} \geq d_{0,0}$ and $d_{1,1}\leq d_{0,1}$. Recall the potential function: let $\{\ell,\dots,r\}$ be the set of $(\tau,\alpha(1+c))$-good intervals. Let $a$ be the maximum $i \in [B-1]$ such that $p^1_i\leq \tau$. Let $L$ be the list of intervals visited in \texttt{BayesLearn}. 
%
%

Let $j_t$ be the interval chosen at $t$-th round, and let $c_t$ be the index of the coin flipped. Let $p^t_{c_t} = p^t$ (we will discard the coin subscript) the probability of the selected coin at time $t$. We split the potential in two addend
\begin{gather}
    \label{eq: set}
    12 C_{\tau, \alpha}(|\{x\in L | x \in [\ell,r]\}|-\gamma|L|)\\
    \label{eq: log weight}
    \log_2 w(a)
\end{gather}
The main difference with the proof in \cite{gretta2023sharp} is that a good coin is defined on the initial probabilities $\{p^1_i\}_{i=1, \dots, B}$, but at the $t$-th iteration we only have access to coin with probability $\{p^t_i\}_{i=1,\dots,B}$. However, they are concentrated around $\alpha$, so $|p^t-p^1|\leq c\alpha$ for $c\leq 1$.

{\bf Bad queries:} Consider $j_t \notin [\ell, r]$. If $j_t>r$, then $p^1\geq \tau + (1+c)\alpha$. As we have that $|p^t-p^1|\leq c\alpha$ we also have $p^t \geq p^1-c\alpha \geq \tau+(1+c)\alpha-c\alpha=\tau+\alpha$. The expected change in the weights is
\begin{equation*}
    \E[\log_2 w_{t+1}(a)- \log_2 w_{t}(a)] = p^t \log_2 d_{1,0} + (1-p^t)\log_2 d_{0,0}\geq C_{\tau, \alpha}.
\end{equation*}
Where the last inequality comes from the fact that the expression is minimized as $p^{t}=\tau+\alpha$, and \autoref{eq: lemma A.1 [1]}. Consider now $j_t<L$, then $p^1\leq \tau-(1+c)\alpha$, which means $p^t \leq p^1+c\alpha \leq \tau-(1+c)\alpha + c\alpha = \tau-\alpha$, then 
\begin{equation*}
    \E[\log_2 w_{t+1}(a)- \log_2 w_{t}(a)] = p^t \log_2 d_{1,1} + (1-p^t)\log_2 d_{0,1}\geq C_{\tau, \alpha},
\end{equation*}
where we reach the minimum $C_{\tau, \alpha}$ when $p^t=\tau-\alpha$, due to \autoref{eq: lemma A.1 [2]}. As $j_t \notin [\ell,r]$ the change in \autoref{eq: set} is $-\gamma \cdot 12 C_{\tau,\alpha}$. Therefore, on bad queries the expected change in $\Phi$ is at least $(1-12\gamma)C_{\tau, \alpha}$.

{\bf Good Queries:} Let's consider the expected change in \autoref{eq: log weight} when $j_t \in [\ell, r]$. Consider the case where $j_t \neq a$, then the expected change is either
\begin{align*}
    &p^t \log_2 d_{1,0} +(1-p^t) \log_2 d_{0,0} \quad  \text{if}  \quad \text{$a$ is on the left of $j_t$, so $p^{0}\geq \tau \Rightarrow p^t \geq \tau-c\alpha$}\\
    &p^t \log_2 d_{1,1} +(1-p^t) \log_2 d_{0,1} \quad  \text{if} \quad  \text{$a$ is on the right of $j_t$, so $p^{0}\leq \tau \Rightarrow p^t \leq \tau+c\alpha\qquad$}
\end{align*}
The first expression is increasing in $p^t$ while the second is decreasing, therefore the expected change is at least
\begin{equation}\label{eq: min for good queries}
    \min\left\{(\tau-c\alpha) \log_2 d_{1,0} +(1-\tau+c\alpha) \log_2 d_{0,0}\,;\,(\tau+c\alpha) \log_2 d_{1,1} +(1-\tau-c\alpha) \log_2 d_{0,1}\right\}
\end{equation}
Let's consider the first argument of the previous expression
\begin{align*}
    (\tau-c\alpha) \log_2 d_{1,0} +(1-\tau+c\alpha) \log_2 d_{0,0} &=(\tau+\alpha) \log_2 d_{1,0} +(1-\tau-\alpha) \log_2 d_{0,0}-\alpha(1+c)(\log_2 d_{1,0}-\log_2 d_{0,0})\\
    &= C_{\tau, \alpha} -\alpha(1+c)\underbrace{(\log_2 d_{1,0}-\log_2 d_{0,0})}_{\geq 0} \quad \tag{as  $d_{1,0}\geq d_{0,0}$}\\
    &\geq C_{\tau, \alpha}-2\alpha (\log_2 d_{1,0}-\log_2 d_{0,0}) \quad \tag{as  $c\leq 1$}\\
    & \geq C_{\tau, \alpha}-2(6\log 2)C_{\tau, \alpha}\\
    & \geq -11 C_{\tau, \alpha},
\end{align*}
where in the first inequality we used the fact that $c\leq 1 \Rightarrow (1+c)\alpha\leq 2\alpha$, while in the second inequality we used Lemma 10 and Lemma 13 in \cite{gretta2023sharp}, valid for $\alpha \leq \frac{1}{2}\min(\tau, 1-\tau)$.
Analogously, for the second argument of \autoref{eq: min for good queries} we get
\begin{align*}
    (\tau + c\alpha)\log_2 d_{1,1} + (1-\tau-c\alpha)\log_2 d_{0,1} &=(\tau -\alpha)\log_2 d_{1,1} + (1-\tau+\alpha)\log_2 d_{0,1} -(1+c)\alpha\underbrace{(\log_2 d_{0,1}-\log_2 d_{1,1})}_{\geq 0}\\
    &\geq -11 C_{\tau, \alpha},
\end{align*}
where the inequality follows by an analogous computation.
Therefore, the change of the weights when $j_t \neq a$ is in expectation at least $-11 C_{\tau,\alpha}$ when $c\in [0,1]$ and $\alpha \leq \frac{1}{2}\min(\tau, 1-\tau)$.
Let's consider now the case where $j_{t} = a$, the expected change is
\begin{equation}
\label{eq: lemma general k}
    p^t\log_2(d_{1,0}k+d_{1,1}(1-k))+(1-p^t)\log_2(d_{0,0}k+d_{0,1}(1-k)),
\end{equation}
for some $k\in [0,1]$. We have two cases: $k\leq q$ or $k>q$. When $k\leq q$ the coin flipped is $a$ then $p^1\leq \tau$ and so $p^{t}\leq \tau+c\alpha$, in \cite{gretta2023sharp} it was shown that in this case \autoref{eq: lemma general k} is decreasing in $p^t$, then the minimum is 
\begin{equation}
\label{eq: min 1}
    (\tau+c\alpha)\log_2(d_{1,0}k+d_{1,1}(1-k))+(1-\tau-c\alpha)\log_2(d_{0,0}k+d_{0,1}(1-k)) 
\qquad \text{if } k\leq q.
\end{equation}
Conversely, when $k>q$ the coin flipped is $a+1$ and then $p^1\geq \tau$ so $p^t\geq \tau-c\alpha$. In this case the expression \eqref{eq: lemma general k} is increasing in $p^t$ so the minimum is
\begin{equation}
\label{eq: min 2}
    (\tau-c\alpha)\log_2(d_{1,0}k+d_{1,1}(1-k))+(1-\tau+c\alpha)\log_2(d_{0,0}k+d_{0,1}(1-k)) 
\qquad \text{if } k> q.
\end{equation}
In \cite{gretta2023sharp} the authors demonstrated that the minimum are obtained when $k\in \{0,1\}$. Therefore, for $k=1> q$ we have \autoref{eq: min 2} while for $k=0< q$ we have instead \autoref{eq: min 1}, which means that the minimum is
\begin{equation*}
    \min\left\{(\tau-c\alpha) \log_2 d_{1,0} +(1-\tau+c\alpha) \log_2 d_{0,0}\,;(\tau+c\alpha) \log_2 d_{1,1} +(1-\tau-c\alpha) \log_2 d_{0,1}\right\},
\end{equation*}
which is at least $-11 C_{\tau, \alpha}$ as demonstrated for the case $j_t \neq a$.
To conclude, the expected change in \autoref{eq: set} is at least $12 C_{\tau, \alpha}(1-\gamma)$, then the overall expected change for the potential is at least $12 C_{\tau, \alpha}(1-\gamma)-11C_{\tau, \alpha} = (1-12 \gamma)C_{\tau, \alpha}$, cocnluding the proof.
\end{proof}


The previous Lemma is the building block for the analysis of \texttt{BayesLearn}, as it allows the construction of a submartingale $\{Y_{t}\}_{t=1,\dots}$ with $Y_{t+1} = \Phi_{t+1}-gt$, for $g=(1-12\gamma)C_{\tau, \alpha}$,
%
that can be used to bound the probability to have a $\gamma$ fraction of good intervals, hence a positive potential. The analysis then follows directly from \cite{gretta2023sharp} with the distinction that the algorithm now with high probability in $\ab$ returns a $(\tau,\alpha(1+c))$-good coin, so proving Theorem \ref{thm:GP-generalization}. Since the proof is identical (see Lemma 6 and Theorem 1 of~\cite{gretta2023sharp}), we omit it. 
However, in order to make this paper self-contained, we will show a simple proof of Theorem~\ref{thm:NBS-changing-probabilities} (which is much less general than Theorem~\ref{thm:GP-generalization}). We restate the theorem here.
\begin{theorem}
Let $0<\alpha\leq \frac{1}{4}$ and suppose $c\leq 1$
There exists an algorithm for \texttt{AdvMonotonicNBS}$(1/2, \alpha, c)$ which uses $O\left(\tfrac{\log B}{\alpha^2}\right)$ coin flips and returns an $(1/2,\alpha(1+c))$-good with high probability in $B$.
\end{theorem}
\begin{proof}
Let $\Phi$ be the potential function in Lemma~\ref{lemma: increase in expectation of the potential} in the case $\tau=1/2$.
Given Lemma \ref{lemma: increase in expectation of the potential}, by choosing $g=(1-12\gamma)C_{1/2, \alpha}$ equal to the lower bound of the lemma, we have that $\{Y_{t}\}_{t=1,\dots}$, for $Y_{t+1} = \Phi_{t+1}-gt$
%
, is a submartingale as
%
%
%
\begin{equation*}
    \E[Y_{t+1}|y_1,\dots,y_t] = \E[\Phi_{t+1}|y_1,\dots, y_t]-gt = \underbrace{\E[\Phi_{t+1}-\Phi_t|y_1,\dots, y_t]}_{\geq g}-g +Y_{t}\geq Y_t
\end{equation*}
%
The difference of the martingale sequence $|Y_{t+1}-Y_t|$ is
\begin{equation*}
    |Y_{t+1}-Y_{t}|\leq |\log_2 w_{t+1}(a)-\log_{2}w_t(a)|+12 C_{1/2,\alpha}+g\leq  |\log_2 w_{t+1}(a)-\log_{2}w_t(a)|+O(\alpha^2),
\end{equation*}
by triangle inequality and $C_{1/2, \alpha}=\Theta(\alpha^2)$ for $\alpha\in (0,1/4)$ due to Lemma 10 \cite{gretta2023sharp}. The remaining term is 
$|\log w_{t+1}(a)-\log w_t(a)|\leq \max\{\log d_{1,0}, \log d_{0,1}\}\leq O(\alpha)$ for Lemma 13 \cite{gretta2023sharp}, thus $|Y_{t+1}-Y_t|\leq O(\alpha)$. We can use Azuma's inequality to bound the probability of having a negative potential 
\begin{align*}
    \Pr[\Phi_{t+1}\leq 0] &= \Pr[\Phi_{t+1}-gt-\Phi_1\leq -gt -\Phi_1]\\
    &=\Pr[Y_{t+1}-Y_0\leq -gt -\Phi_1]\\
    &\leq \exp\bigg(-\dfrac{(gt+\Phi_1)^2}{t\cdot O(\alpha^2)}\bigg)\quad \text{for } gt\geq -\Phi_1.
\end{align*}
Note that $\Phi_1=-\log(B-1)$. Therefore, picking $T=O\left(\frac{\log B}{g}\right)$ sufficiently large, we get that $\frac{(gT+\Phi_1)^2}{T\cdot O(\alpha^2)}\geq \lambda\log B$ for any desired constant $\lambda>0$. Thus,
\[
\Pr[\Phi_{T+1}\leq 0]\leq B^{-\lambda}.
\]
On the other hand, note that if $\Phi_{T+1}> 0$, then
\[
 0<\frac{\Phi_{T+1}}{12C_{1/2, \alpha}} \leq (|\{x\in L : x \in [\ell,r]\}|-\gamma|L|),
\]
and so, a $\gamma$ fraction of the intervals in $L$ are $(1/2,\alpha(1+c))$-good. Now we can order the intervals in $L$ in sorted order according to their indices $i$ of the corresponding coins. By picking a subset $S$ of every $(1/\gamma)$th of them, we are ensured that one of them will be good (conditional on the high probability event $\Phi_{T+1}> 0$). For each interval in $S$, we can test whether it is $(1/2,\alpha(1+c))$-good with high probability using $O(\frac{\log B}{\alpha^2})$ coin flips of each of the coins at its endpoints. Therefore, we successfully determine an $(1/2,\alpha(1+c))$-good coin with high probability in $B$. If we pick $\gamma=1/13$, the total number of coins flipped is 
\[
T+|S|O\bigg(\frac{\log B}{\alpha^2}\bigg)= O\bigg(\frac{\log B}{g}\bigg)+O\bigg(\frac{\log B}{\alpha^2}\bigg)=O\bigg(\frac{\log B}{\alpha^2}\bigg),
\]
where the final bound uses that $g=(1-12\gamma)C_{1/2, \alpha}=\frac{1}{13}C_{1/2, \alpha}=\Theta(\alpha^2)$. This completes the proof.
%
%
%
   %
%
%
%
 %
  %
%
%
%
  %
%
%
\end{proof}
\subsection{Lower bounds on sample complexity}\label{sec:sample_compexity}
We establish a lower bound for generalized linear measurements using standard information-theoretic arguments based on Fano's inequality. While the upper bound in Theorem~\ref{thm:alg_general} is derived for the maximum probability of error over all  $k$-sparse vectors, the lower bound applies even in the weaker setting of the average probability of error, where 
$\bx$ is chosen uniformly at random.
\begin{theorem}[Lower bound for GLMs]\label{thm: lower_bdglm} Consider any  sensing matrix $\vecA$.
For a uniformly chosen $k$-sparse vector $\bx$, an algorithm $\phi$ satisfies $$\bbP\inp{\phi(\vecA, \by) \neq \bx}\leq \delta$$   only if the number of measurements $$m\geq \frac{k\log\inp{\frac{n}{k}}}{I}\inp{1 - \frac{h_2(\delta) + \delta k\log{n}}{k\log{n/k}}}$$ for some $I$ such that $I\geq {I(y_i; \bx|\vecA)}, \, i\in [m]$. In particular, when $y\in \inb{-1, 1}$, we have $\bbE\insq{\inp{g(\vecA_i^T\bx)}^2} \geq I(y_i, \bx|\vecA)$ where the expectation is over the randomness of $\vecA$ and $\bx$.
\end{theorem}
The lower bound can be interpreted in terms of a communication problem, where the input message $\bx$ is encoded to $\vecA\bx$. The decoding function takes in as input the encoding map $\vecA$ and the output vector $\by$ in order to recover $\bx$ with high probability. For optimal recovery, one needs at least $\frac{\text{message entropy}}{\text{capacity}}$ number of measurements (follows from noisy channel coding theorem~\cite{thomas2006elements}). In Theorem~\ref{thm: lower_bdglm}, the entropy of the message set $\log{n \choose k}\approx k\log{n/k}$ and the proxy for capacity is the upper bound on mutual information $I$. We provide a detailed proof of the theorem in  Section~\ref{sec:proofs}.


We first present lower bounds for \bcs\  and \logreg. The lower bound for \bcs\ is given for any sensing matrix $\vecA$ which satisfies the power constraint given by \eqref{eq:power_constraint}, whereas the one for \logreg\ is only for the special case when each entry of the sensing matrix is iid $\cN(0,1)$. Recall that \eqref{eq:power_constraint} holds in this case.  For \bcs\ (and \logreg\ respectively), we can use the upper bound of $\bbE\insq{\inp{g(\vecA_i^T\bx)}^2}$ on the mutual information term. The dependence of $\sigma^2$ (and $1/\beta^2$ respectively) requires careful bounding of this term, which is done in the formal proofs in Appendix~\ref{proof:sec:lower_bd}.


As mentioned earlier, we need at least $k\log\inp{n/k}$ measurements for \bcs and \logreg. This is because the entropy of a randomly chosen $k$-sparse vector is approximately $k\log\inp{n/k}$ and we learn at most one bit with each measurement. However, due to corruption with noise, we learn less than a bit of information about the unknown signal with each measurement. The information gain gets worse as the noise level increases. 
Our lower bounds make this reasoning explicit.  
\begin{corollary}[\bcs\ lower bound]\label{thm: lower_bd_bcs} Suppose, each row $\vecA_i, \, i\in [1:m]$ of the sensing matrix $\vecA$ satisfies the power constraint~\eqref{eq:power_constraint}.
For a uniformly chosen $k$-sparse vector $\bx$, an algorithm $\phi$ satisfies $$\bbP\inp{\phi(\vecA, {\by}) \neq \bx}\leq \delta$$ for the problem of $\bcs$ only if the number of measurements $$m\geq \frac{k+\sigma^2}{2}\log\inp{\frac{n}{k}}\inp{1 - \frac{h_2(\delta) + \delta k\log{n}}{k\log{n/k}}}.$$ 
\end{corollary}

\begin{corollary}[\logreg\ lower bound]\label{thm: lower_bd_log_reg} Consider a Gaussian  sensing matrix $\vecA$ where each entry is chosen iid $N(0,1)$.
For a uniformly chosen $k$-sparse vector $\bx$, an algorithm $\phi$ satisfies $$\bbP\inp{\phi(\vecA, \bw) \neq \bx}\leq \delta$$ for the problem of $\logreg$ only if the number of measurements $$m\geq \frac{1}{2}\inp{k+\frac{1}{\beta^2}}\log\inp{\frac{n}{k}}\inp{1 - \frac{h_2(\delta) + \delta k\log{n}}{k\log{n/k}}}.$$ 
\end{corollary}



Theorem~\ref{thm: lower_bdglm} also implies an information theoretic lower bound for \spl, which is presented below and proved in Appendix~\ref{proof:sec:lower_bd}. Note that the denominator term in the bound $\frac{1}{2}\log\inp{1+\frac{k}{\sigma^2}}$ is the capacity of a Gaussian channel with power constraint $k$ and noise variance $\sigma^2$. 
\begin{corollary}[\spl\ lower bound]\label{thm: spl_lower_bd_1}
Under the average power constraint \eqref{eq:power_constraint} on  $\vecA$, for a uniformly chosen $k$-sparse vector $\bx$, an algorithm $\phi$ satisfies $$\bbP\inp{\phi(\vecA, {\by}) \neq \bx}\leq \delta$$ only if the number of measurements
$$m\geq \frac{k\log\inp{\frac{n}{k}}-\inp{h_2(\delta) + \delta k\log{n}}}{\frac{1}{2}\log\inp{1+\frac{k}{\sigma^2}}}.$$
\end{corollary} 

\subsection{Tighter upper and lower bounds for \spl}\label{sec:tighter_bounds_spl}
We present information theoretic upper and lower bounds for \spl\ in this section. Similar to Section~\ref{sec:alg}, our upper bound is for the maximum probability of error, while the lower bounds hold even for the weaker criterion of average probability of error.

We first present an upper bound based on the maximum likelihood estimator (MLE) where  we  decode to $\hat{\bx}$ if, on output $\by$, 
\begin{align*}
\hat{\bx} = \argmax_{\stackrel{\bx\in \inb{0,1}^n}{\wh{\bx} = k}}\,\, p(\by|{\bx})
\end{align*} where $p(\by|{\bx})$ denotes the probability density function of $\by$ on input $\bx$.
\begin{theorem}[MLE upper bound for \spl]\label{thm:upper_bd_mle} Suppose  entries of the measurement matrix $\vecA$ are i.i.d. $\cN(0,1).$
The MLE  is correct with high probability if 
\begin{align}m\geq \max_{l\in[1:k]}  \frac{nN(l)}{\frac{1}{2}\log\inp{\frac{ l}{2\sigma^2}+1}}\label{eq:upper_bd_mle}
\end{align}where  $N(l):=  \frac{k}{n} h_2\inp{\frac{l}{k}} + (1-\frac{k}{n})h_2\inp{\frac{l}{n-k}}$. 
\end{theorem}
We prove the theorem in Appendix~\ref{proof:MLE}. The main proof idea involves analysing the probability that the output of the MLE is $2l$ Hamming distance away from the unknown signal $\bx$ for different values of $l\in [1:k]$ (assuming $k\leq n/2$). This depends on the number of such vectors (approximately $2^{nN(l)}$) and the probability that the MLE outputs a vector which is $2l$ Hamming distance away from $\bx$. 

Note that when $l = k\inp{1-\frac{k}{n}}$, $nN(l) = nh_2(k/n)\approx k\log{\frac{n}{k}}$ and $\log\inp{\frac{k\inp{1-k/n}}{2\sigma^2}+1}\leq \log\inp{\frac{k}{2\sigma^2}+1}$.
Thus, $m$ is at least $\frac{2k\log{n/k}}{\log\inp{\frac{k}{2\sigma^2}+1}}$ (see the bound for Corollary~\ref{thm: spl_lower_bd_1}). It is not immediately clear if this value of $l= k\inp{1-\frac{k}{n}}$ is the optimizer. However, for large $n$, this appears to be the case numerically as shown in Plot~\ref{plot:1}.

\begin{figure}[t]
\includegraphics[width=7cm]{Unknown2.png}
\centering
\caption{The figure shows the plot of the MLE upper bound \eqref{eq:upper_bd_mle} (given by m1) for different values of $k$. This is displayed in blue color. A plot of $\frac{2nN(l)}{\log\inp{\frac{ l}{2\sigma^2}+1}}$ is also presented for $l = k\inp{1-\frac{k}{n}}$ in orange color, given by m2. A part of the plot is zoomed in to emphasize the closeness between the lines. In these plots,  $\sigma^2$ is set to 1,  $n$ is 50000 and $k$ ranges from 1000 to 25000 $(n/2)$. }\label{plot:1}
\end{figure}


Inspired by the MLE analysis, we derive a lower bound with the same structure as \eqref{eq:upper_bd_mle}. We generate the unknown signal $\bx$ using the following distribution: A vector $\tilde{\bx}$ is chosen uniformly at random from the set of all $k$-sparse vectors. Given $\tilde{\bx}$, the unknown input signal $\bx$ is chosen uniformly from the set of all $k$-sparse vector which are at a Hamming distance $2l$ from $\bx$. 
The lower bound is then obtained by computing upper and lower bounds on $I(\vecA, \by;\bx|\tilde{\bx})$.
We show this lower bound only for random matrices where each entry is chosen iid $\cN(0,1)$.
\begin{theorem}[\spl\ lower bound]\label{thm:lower_bd_spl}
If each entry of $\vecA$ is chosen iid $\cN(0,1)$, then for a uniformly chosen $k$-sparse vector $\bx$, an algorithm $\phi$ satisfies 
\begin{align}
    \bbP\inp{\phi(\vecA, {\by}) \neq \bx}\leq \delta\label{eq:spl_lower_bd_l}
\end{align}  only if the number of measurements $$m\geq \max_l\frac{nN(l) - 2\log{n}- h_2(\delta) - \delta k\log{n}}{\frac{1}{2}\log\inp{1+\frac{l}{\sigma^2}\inp{2-\frac{l}{k}}}} .$$
\end{theorem} The proof of Theorem~\ref{thm:lower_bd_spl} is given in Appendix~\ref{proof:MLE}.

If we choose $l = k\inp{1-\frac{k}{n}}$ in Theorem~\ref{thm:lower_bd_spl}, we recover corollary~\ref{thm: spl_lower_bd_1} for the special case of Gaussian design.
% \begin{corollary}\label{corollary2:lower_bd_spl}
% If  each entry of $\vecA$ is chosen iid $\cN(0,1)$, then for a uniformly chosen $k$-sparse vector $\bx$, an algorithm $\phi$ satisfies 
% $$\bbP\inp{\phi(\vecA, {\by}) \neq \bx}\leq \delta$$
% only if the number of measurements 
% $$m\geq \frac{k\log\inp{\frac{n}{k}} - 2\log{n}- h_2(\delta) - \delta k\log{n}}{\log\inp{1+\frac{k}{\sigma^2}}} .$$
% \end{corollary}

% Corollary~\ref{corollary2:lower_bd_spl} can also be proved directly for any sensing matrix $\vecA$ which satisfies \eqref{eq:power_constraint} (non-necessarily a Gaussian design). 


% \begin{figure}[t]
% \includegraphics[width=8cm]{plot.png}
% \centering
% \caption{The figure shows the plot of the MLE upper bound \eqref{eq:upper_bd_mle} (given by m1) for different values of $n$. This is displayed in blue color. A plot of $\frac{2nN(l)}{\log\inp{\frac{ l}{2\sigma^2}+1}}$ is also presented for $l = k\inp{1-\frac{k}{n}}$ in orange color, given by m2. In these plots,  $\sigma^2$ is set to 1 and $k$ is $0.2n$. }\label{plot:1}
% \end{figure}


\section{Naive Shuffle-DP binary Search for the Median}
\label{sec:naive-shuffle}
This section is dedicated to proving~\cref{thm:main-shuffle}.


The naive binary search with errors algorithm tests each coin up to $\acc$-accuracy and a $\failp/\log\ab$ failure probability, such that a simple union bound over all $\log\ab$ steps of binary searching will yield an $(\acc,\failp)$-accurate estimate. This algorithm is suboptimal up to logarithmic factors, although there are indications that its strong constant factors can make up the difference in some parameter regimes~\cite{karp2007noisy,gretta2023sharp}. The simple fact that this algorithm runs in deterministic number of rounds, with a deterministic number of samples per round, allows for a straightforward application of amplification by shuffling~\cite{feldman21shuffle}, something we could not achieve with the fully adaptive Bayesian updates algorithm. 

We consider both statistical error, where samples are assumed to be drawn from some unknown distribution with mean $p$, and we are interested in an estimate $\hat{p}$ which is close to that true mean, and the empirical setting where we make no assumption on the distribution of the samples, and are interested in how close our estimate $\hat{p}$ is to the ``best-case'' sample mean $\frac{1}{n}\sum_{i=1}^nx_i$.


\begin{lemma}[Sample complexity of learning one coin to its statistical mean.] 
\label{lemma: one-coin-statistical-mean}
    Given samples $\{x_i\}_{i=1}^n$ from a Bernoulli random variable $X$ with mean $p$ received through a binary randomized response channel $\pmech$ such that $y_i\sim \pmech(x_i)$, we can estimate $\hat{p}=\frac{1}{n}\frac{e^\priv + 1}{e^\priv - 1}\sum_{i=1}^n y_i - \frac{1}{e^\priv - 1}$. In order to learn an $(\acc,\failp)$-estimate of $p$, $\Pr[|\hat{p}-p|>\acc]<\failp$ it suffices to use $n$ samples where,
    $$
    n\leq\left(\frac{2p(1-p)}{\acc^2} + \frac{e^\priv}{\acc^2(e^\priv - 1)^2} + \frac{2(e^\priv + 1)}{4\acc(e^\priv - 1)}\right)\log(1/\failp).
    $$
    In other words, the sample complexity of learning one coin to its statistical mean with constant failure probability is $O\left(\frac{1}{\acc^2\priv^2} +\frac{p(1-p)}{\acc^2}\right)$, when $\priv<1$, or $O\left(\frac{1}{\acc^2e^\priv} +\frac{p(1-p)}{\acc^2}\right)$, when $\priv\geq 1$.
\end{lemma}
\begin{proof}
    Given a Bernoulli random variable $x$ with mean $p$, and a binary randomized response channel $\pmech$ (see~\autoref{def: binary rr}) the distribution induced by applying $\pmech$ to $x$ is:
    \begin{equation*}
    \label{eq:rr-bern-induced}
    y=\pmech(x)\sim\operatorname{Bern}\left(\frac{e^\priv}{e^\priv + 1}p + (1-p)\frac{1}{e^\priv + 1}\right)=\operatorname{Bern}\left(\frac{e^\priv-1}{e^\priv + 1}p + \frac{1}{e^\priv + 1}\right).
    \end{equation*}
    The variance of this distribution is 
    \begin{align*}
\sigma^2=\operatorname{Var}(y)&=\left(\frac{1}{e^\priv + 1}+\frac{e^\priv-1}{e^\priv + 1}p \right)\left( \frac{e^\priv}{e^\priv + 1}-\frac{e^\priv-1}{e^\priv + 1}p\right)\notag\\
&=\left(\frac{e^\priv - 1}{e^\priv + 1}\right)^2p(1-p) + \frac{e^\priv}{(e^\priv + 1)^2}.  \label{eq:rr-bern-var}
    \end{align*}
    We then proceed by simple rearranging, substitution, and application of Bernstein's inequality~\cref{fact: bernstein}.
    \begin{align*}
        \Pr\left[|\hat{p}-p| >\acc\right]&=\Pr\left[\bigg|\frac{1}{n}\frac{e^\priv + 1}{e^\priv - 1}\sum\limits_{i=1}^n y_j - \frac{1}{e^\priv - 1} - \left(\frac{e^\priv + 1}{e^\priv - 1}\bEE{y} - \frac{1}{e^\priv - 1}\right)\bigg| >\acc\right]\\
&=\Pr\left[\bigg|\frac{e^\priv + 1}{e^\priv -1}\left(\frac{1}{n}\sum\limits_{i=1}^n y_i -\bEE{y}\right)\bigg|>\acc\right]\\
&=\Pr\left[\bigg|\frac{1}{n}\sum\limits_{j=1}^ny -\bEE{y}\bigg|>t\right]\tag*{$\left(t=\acc\frac{e^\priv -1}{e^\priv +1}\right)$}\\
\failp&\leq\exp\left(\frac{-nt^2}{2\sigma^2 + \frac{2t}{3}}\right)\tag{Bernstein's Inequality}\\
n&\leq\left(\frac{2\sigma^2}{t^2} +\frac{2}{3t} \right)\log(1/\failp)\\
    &=\left( \frac{2p(1-p)}{\acc^2} +\frac{2e^\priv}{\acc^2(e^\priv - 1)^2}+\frac{2(e^\priv + 1)}{3\acc(e^\priv - 1)}\right)\log(1/\failp).\tag{Substituting $t$ and $\sigma^2$}
    \end{align*}
\end{proof}

\begin{lemma}[Sample complexity of learning one coin to its sample mean.]
\label{lem:empirical-coin-learn-rr}
    Given samples $\{x_i\}_{i=1}^n$ where each $x_i\in\{0,1\}$, and private outputs $y_i\sim \pmech(x_i)$, the true sample mean is $P=\frac{1}{n}\sum_{i=1}^n x_i$. Denote the sample mean of the collected private outputs $Y=\frac{1}{n}\sum_{i=1}^n y_i$. Our estimator of the sample mean will be similar to the statistical case, where $\widehat{P}=\frac{e^\priv + 1}{e^\priv - 1}Y - \frac{1}{e^\priv - 1}$. In order to learn an $(\acc,\failp)$-estimate of $P$ it is sufficient to use $n$ samples such that 
    \begin{equation*}
        n\leq\left( \frac{2e^\priv}{\acc^2 (e^\priv - 1)^2} + \frac{2(e^\priv + 1)}{3\acc (e^\priv - 1)} \right)\log(1/\failp).
    \end{equation*}
    Therefore, the sample complexity of learning the sample mean with constant failure probability is $O\left(\frac{1}{\acc^2\priv^2}\right)$, when $\priv<1$, or $O\left(\frac{1}{\acc^2e^\priv} \right)$, when $\priv\geq 1$. It is pleasing to note that this recovers the sample complexity of learning in the statistical case, up to the additive sampling error.
\end{lemma}

\begin{proof}
    The proof will proceed similarly to the statistical case. The key difference will be the variance of $y$ in this case which is
    \[
    \sigma^2=\sigma^2(\pmech(0))=\sigma^2(\pmech(1))=\frac{e^\priv}{(e^\priv + 1)^2}.
    \]
    The derivation then proceeds as in the statistical case.
    \begin{align*}
        \Pr[|\widehat{P}-P|>\acc] &= \Pr\left[\bigg| \frac{e^\priv + 1}{e^\priv - 1}Y - \frac{1}{e^\priv - 1} - P\bigg| >\acc\right]\\
            &=\Pr\left[\bigg| \frac{e^\priv + 1}{e^\priv - 1}Y - \frac{1}{e^\priv - 1} - \left( \frac{e^\priv + 1}{e^\priv - 1}\bEE{Y} - \frac{1}{e^\priv - 1} \right)\bigg|>\acc \right]\\
            &= \Pr\left[\frac{e^\priv + 1}{e^\priv - 1}\bigg| \left(Y-\bEE{Y} \right)\bigg|>\acc \right]\\
            &=\Pr\left[\bigg| Y-\bEE{Y}\bigg|> t \right]\tag*{$\left(t=\acc\frac{e^\priv - 1}{e^\priv + 1}\right)$}\\
            &\leq \exp\left(\frac{-nt^2}{2\sigma^2 + \frac{2t}{3}}\right)\tag{Bernstein's Inequality}\\
        n   &\leq \left( \frac{2\sigma^2}{t^2} + \frac{2}{3t} \right)\log(1/\failp)\\
            &=\left(\frac{2e^\priv}{\acc^2 (e^\priv - 1)^2} + \frac{2(e^\priv + 1)}{3\acc(e^\priv - 1)} \right)\log(1/\failp).\tag{Substituting $t$ and $\sigma^2$}
    \end{align*}
\end{proof}

With this we can now formally state the sample complexity of a naive binary search for the median under local differential privacy. We will focus on the empirical case for this result. 
\begin{theorem}[Naive Binary Search for the Median under Local Differential Privacy]
\label{thm:ldp-nbs-naive}
    The naive algorithm as described by~\citet{karp2007noisy}, under the constraints of $\priv$-local differential privacy, has sample complexity
    \[
    n\leq \left( \frac{2e^\priv}{\acc^2(e^\priv - 1)^2} + \frac{2(e^\priv + 1)}{3\alpha(e^\priv - 1)} \right)\log(\ab)\log\left(\frac{\log\ab}{\failp}\right).
    \]
    We can therefore say that for $\priv<1$, the naive approach has sample complexity $O\left(\frac{\log\ab}{\acc^2\priv^2}\log\left(\frac{\log\ab}{\failp}\right)\right)$, and for $\priv\geq 1$ it has sample complexity $O\left(\frac{\log\ab}{\acc^2e^\priv}\log\left(\frac{\log\ab}{\failp}\right)\right)$.
\end{theorem}
\begin{proof}
    Given $n$ total users, let $n'=n/\log(\ab)$ and let $\failp'=\failp/\log(\ab)$, apply~\autoref{lem:empirical-coin-learn-rr} with $n',\failp'$ to get sample complexity.
    \[
    n\leq \left( \frac{2e^\priv}{\acc^2(e^\priv - 1)^2} + \frac{2(e^\priv + 1)}{3\alpha(e^\priv - 1)} \right)\log(\ab)\log\left(\frac{\log\ab}{\failp}\right).
    \]
    By a union bound over all $\log\ab$ rounds of the binary search, the final estimate will be an $(\acc,\failp)$-approximate median.
\end{proof}

As stated in the introduction, the primary motivation for this approach is that by dividing the algorithm into a few deterministic stages, with many samples tested at each stage, we can hope to apply amplification by shuffling~\cite{feldman21shuffle}. We state the amplification by shuffling result here, and a subsequent lemma that will be useful to our analysis.
\begin{theorem}[{\citet*[Theorem 3.1]{feldman21shuffle}}]
    \label{theorem: amplification by shuffling}
    For any domain $\mathcal{X}$, let $\pmech_t:\pmech_1\times\ldots\times\pmech_{t-1}\times\mathcal{X}\rightarrow\mathcal{Y}$  for $t\in [n]$ be a sequence of randomizers such that $\pmech_t(y_{1:t-1},\cdot)$ is $\priv_L$-local DP; and let $S$ be the algorithm that given a tuple of $n$ messages, outputs a uniformly random permutation of said messages. Then for any $\privdelta\in(0,1]$ such that $\priv_L\leq\log\frac{n}{16\log(2/\privdelta)}$, $S\circ \mathcal{Y}^n$ is is $(\priv,\privdelta)$-DP, where
    \[
    \priv\leq\log\left(1 + 8\frac{e^{\priv_L}-1}{e^{\priv_L}+1}\left(\sqrt{\frac{e^{\varepsilon_L}\log(4/\delta)}{n}}+
\frac{e^{\varepsilon_L}}{n}\right)\right)
    \]
\end{theorem}
This implies the following useful lemma,
\begin{lemma}[Amplification by shuffling]
\label{lemma: amplification by shuffling}
    Fix any $\privdelta\in(0,1]$, $\priv\in(0,1]$, and $n$ such that $\priv>16\sqrt{\log(4/\privdelta)/n}$. Then, for
    \[
    \priv_L\coloneqq\log\frac{\priv^2 n}{80\log(4/\privdelta)}
    \]
    Shuffling the messages of $n$ users using the same $\priv_L$-LDP randomizer satisfies $(\priv,\privdelta)$-shuffle differential privacy.
\end{lemma}
\begin{proof}
    For $\priv,\privdelta$ and $\priv_L$ as above we have $0<\priv_L\leq\log\frac{n}{16\log(2/\privdelta)}$. Applying~\autoref{theorem: amplification by shuffling}, we get $(\priv',\privdelta)$-differential privacy for
    \[
    \priv' \leq \log\left(1 + 8\underbrace{\frac{e^{\priv_L}-1}{e^{\priv_L} + 1}}_{<1}\underbrace{\left(\frac{\priv}{\sqrt{80}}+\frac{\priv^2}{80\log(4/\delta)}\right)}_{< \priv/8}\right)\leq \priv
    \]
    Proving the lemma.
\end{proof}

%
We can now prove~\cref{thm:main-shuffle}.
\begin{theorem}[Restatement of~\Cref{thm:main-shuffle}]\label{thm:restated-shuffle}
    Let $r=\log_2 B$. There exists a protocol for \texttt{shuffle\--emp\--median}$(\{x_i\}_{i=1}^n,\alpha,\eps,\delta,r)$ with success probability $1-\failp$ provided that
    \[
    n=O\left( \left(\frac{1}{\acc^2} +\frac{1}{\priv^2}\right)\log\ab\sqrt{\log(1/\privdelta)\log\frac{\log\ab}{\failp}} \right).
    \]
    The protocol has $r=\log_2\ab$ rounds of adaptivity and queries shuffled batches of $n/\log_2(\ab)$ users. 
\end{theorem}
%
%
%
%
%
%
%
%
\begin{proof}[Proof of~\Cref{thm:main-shuffle}]    
    Take the sample complexity achieved in~\autoref{thm:ldp-nbs-naive}, and note that we are in the $\priv\gg 1$ regime as we will be applying taking $\priv_L\in O(\log n)$. We therefore have
    \[
    n= O\left( \frac{\log\ab}{\acc^2 e^{\priv_L}}\log\frac{\log\ab}{\failp} \right)
    \]
    We apply~\autoref{lemma: amplification by shuffling} while noting that at each stage we shuffle $n'=n/\log(\ab)$ users. Setting $\priv_L=\log\frac{\priv^2 n}{80\log(\ab)\log(4/\privdelta)}$ and rearranging gives that for each step of the binary search we have enough users to accurately learn the CDF of the remaining suffix of users within error $\alpha/2$ with probability $\beta/\log B$. Union bounding over all $\log B$ steps of the binary search, we conclude that with probability $1-\beta$, every step succeeds. This gives sample complexity,
    \[
    n= O\left(\frac{\log\ab}{\acc\priv} \sqrt{\log(1/\privdelta)\log\frac{\log\ab}{\failp}} \right),
    \]
    but we are not finished. We have to handle the multiple restrictions on parameter regimes 
    \[
    O\left(\frac{\log\ab}{\acc\priv} \sqrt{\log(1/\privdelta)\log\frac{\log\ab}{\failp}} \right)\geq n >\max\left\{\frac{\log\ab}{\acc^2},\frac{{\log\ab\log(1/\privdelta)}}{\priv^2}\right\}.
    \]
    The right hand side of this inequality comes from restrictions present in~\cref{lemma:CDF-bound,lemma: amplification by shuffling} on $n$ and $\priv$ respectively, the latter comes from using $n'=n/\log(\ab)$ in the restriction on $\priv$. 
    A trivial solution is be to take $1/(\acc\priv)$ and replace it with $1/\min\{\acc^2,\priv^2\}$, which gives
    \[
    n=O\left( \left(\frac{1}{\acc^2} +\frac{1}{\priv^2}\right)\log\ab\sqrt{\log(1/\privdelta)\log\frac{\log\ab}{\failp}} \right).
    \]
    %

   
\end{proof}

This result has an improved dependence in $\priv$ and $\acc$, and could be preferable from a communication perspective. Rounds of adaptivity are a restricting factor in distributed learning, and our goal was to understand the trade offs possible under privacy constraints. It is of practical interest to know whether the constraint on $n$ in~\cref{lemma: amplification by shuffling} can be improved from $n=\Omega\left(\log(1/\privdelta)/\priv^2\right)$ to $\Omega\left(\log(1/\privdelta)/\priv\right)$. This, in combination with a strengthening of~\cref{lemma:CDF-bound} to have a linear dependence on $\acc$, would allow the analysis to go through with only a $1/(\acc\priv)$ dependence.
\section{A Note on the Continuous Case}\label{sec:continuous} If we replace the discrete domain $[B]$ with a continuous one, say $[0,1]$, it is generally impossible to obtain quantile error $o(1)$ using a finite number of samples under LDP. This follows from our lower bounds by discretizing $[0,1]$ into $[B]$ buckets and letting $B\to \infty$. In fact, this is a general issue for quantile or range estimation problems in DP (even beyond the local model), which is why related work studies the discrete setting~\cite{BeimelNS16twotologstar,Bun2015logstar,Kaplan2020closinggap,kulkarni2019answering}. On a more positive note, if we impose mild guarantees on the family of possible distributions the samples can come from, our result has implications in the continuous setting as well. For instance, if we assume that there are (known) numbers $-\infty=y_0<y_1<\cdots <y_B=\infty$ such that in any interval $[y_i,y_{i+1}]$, the emperical CDF increases by at most $\alpha/2$, then we can again obtain quantile error $\alpha$ with $O(\frac{\log B}{\eps^2\alpha^2})$ users using our algorithm and bucketing users in the same interval $[y_i,y_{i+1})$. As the dependency on $B$
 in the number of samples is logarithmic, this might allow 
 $B$ to be quite large, with a correspondingly small quantile error $\alpha$. We note that if the assumption on the CDF is incorrect, only the accuracy is affected while the algorithm remains private.
\section{Experiments}
\label{sec:experiments}
The experiments are designed to address two key research questions.
First, \textbf{RQ1} evaluates whether the average $L_2$-norm of the counterfactual perturbation vectors ($\overline{||\perturb||}$) decreases as the model overfits the data, thereby providing further empirical validation for our hypothesis.
Second, \textbf{RQ2} evaluates the ability of the proposed counterfactual regularized loss, as defined in (\ref{eq:regularized_loss2}), to mitigate overfitting when compared to existing regularization techniques.

% The experiments are designed to address three key research questions. First, \textbf{RQ1} investigates whether the mean perturbation vector norm decreases as the model overfits the data, aiming to further validate our intuition. Second, \textbf{RQ2} explores whether the mean perturbation vector norm can be effectively leveraged as a regularization term during training, offering insights into its potential role in mitigating overfitting. Finally, \textbf{RQ3} examines whether our counterfactual regularizer enables the model to achieve superior performance compared to existing regularization methods, thus highlighting its practical advantage.

\subsection{Experimental Setup}
\textbf{\textit{Datasets, Models, and Tasks.}}
The experiments are conducted on three datasets: \textit{Water Potability}~\cite{kadiwal2020waterpotability}, \textit{Phomene}~\cite{phomene}, and \textit{CIFAR-10}~\cite{krizhevsky2009learning}. For \textit{Water Potability} and \textit{Phomene}, we randomly select $80\%$ of the samples for the training set, and the remaining $20\%$ for the test set, \textit{CIFAR-10} comes already split. Furthermore, we consider the following models: Logistic Regression, Multi-Layer Perceptron (MLP) with 100 and 30 neurons on each hidden layer, and PreactResNet-18~\cite{he2016cvecvv} as a Convolutional Neural Network (CNN) architecture.
We focus on binary classification tasks and leave the extension to multiclass scenarios for future work. However, for datasets that are inherently multiclass, we transform the problem into a binary classification task by selecting two classes, aligning with our assumption.

\smallskip
\noindent\textbf{\textit{Evaluation Measures.}} To characterize the degree of overfitting, we use the test loss, as it serves as a reliable indicator of the model's generalization capability to unseen data. Additionally, we evaluate the predictive performance of each model using the test accuracy.

\smallskip
\noindent\textbf{\textit{Baselines.}} We compare CF-Reg with the following regularization techniques: L1 (``Lasso''), L2 (``Ridge''), and Dropout.

\smallskip
\noindent\textbf{\textit{Configurations.}}
For each model, we adopt specific configurations as follows.
\begin{itemize}
\item \textit{Logistic Regression:} To induce overfitting in the model, we artificially increase the dimensionality of the data beyond the number of training samples by applying a polynomial feature expansion. This approach ensures that the model has enough capacity to overfit the training data, allowing us to analyze the impact of our counterfactual regularizer. The degree of the polynomial is chosen as the smallest degree that makes the number of features greater than the number of data.
\item \textit{Neural Networks (MLP and CNN):} To take advantage of the closed-form solution for computing the optimal perturbation vector as defined in (\ref{eq:opt-delta}), we use a local linear approximation of the neural network models. Hence, given an instance $\inst_i$, we consider the (optimal) counterfactual not with respect to $\model$ but with respect to:
\begin{equation}
\label{eq:taylor}
    \model^{lin}(\inst) = \model(\inst_i) + \nabla_{\inst}\model(\inst_i)(\inst - \inst_i),
\end{equation}
where $\model^{lin}$ represents the first-order Taylor approximation of $\model$ at $\inst_i$.
Note that this step is unnecessary for Logistic Regression, as it is inherently a linear model.
\end{itemize}

\smallskip
\noindent \textbf{\textit{Implementation Details.}} We run all experiments on a machine equipped with an AMD Ryzen 9 7900 12-Core Processor and an NVIDIA GeForce RTX 4090 GPU. Our implementation is based on the PyTorch Lightning framework. We use stochastic gradient descent as the optimizer with a learning rate of $\eta = 0.001$ and no weight decay. We use a batch size of $128$. The training and test steps are conducted for $6000$ epochs on the \textit{Water Potability} and \textit{Phoneme} datasets, while for the \textit{CIFAR-10} dataset, they are performed for $200$ epochs.
Finally, the contribution $w_i^{\varepsilon}$ of each training point $\inst_i$ is uniformly set as $w_i^{\varepsilon} = 1~\forall i\in \{1,\ldots,m\}$.

The source code implementation for our experiments is available at the following GitHub repository: \url{https://anonymous.4open.science/r/COCE-80B4/README.md} 

\subsection{RQ1: Counterfactual Perturbation vs. Overfitting}
To address \textbf{RQ1}, we analyze the relationship between the test loss and the average $L_2$-norm of the counterfactual perturbation vectors ($\overline{||\perturb||}$) over training epochs.

In particular, Figure~\ref{fig:delta_loss_epochs} depicts the evolution of $\overline{||\perturb||}$ alongside the test loss for an MLP trained \textit{without} regularization on the \textit{Water Potability} dataset. 
\begin{figure}[ht]
    \centering
    \includegraphics[width=0.85\linewidth]{img/delta_loss_epochs.png}
    \caption{The average counterfactual perturbation vector $\overline{||\perturb||}$ (left $y$-axis) and the cross-entropy test loss (right $y$-axis) over training epochs ($x$-axis) for an MLP trained on the \textit{Water Potability} dataset \textit{without} regularization.}
    \label{fig:delta_loss_epochs}
\end{figure}

The plot shows a clear trend as the model starts to overfit the data (evidenced by an increase in test loss). 
Notably, $\overline{||\perturb||}$ begins to decrease, which aligns with the hypothesis that the average distance to the optimal counterfactual example gets smaller as the model's decision boundary becomes increasingly adherent to the training data.

It is worth noting that this trend is heavily influenced by the choice of the counterfactual generator model. In particular, the relationship between $\overline{||\perturb||}$ and the degree of overfitting may become even more pronounced when leveraging more accurate counterfactual generators. However, these models often come at the cost of higher computational complexity, and their exploration is left to future work.

Nonetheless, we expect that $\overline{||\perturb||}$ will eventually stabilize at a plateau, as the average $L_2$-norm of the optimal counterfactual perturbations cannot vanish to zero.

% Additionally, the choice of employing the score-based counterfactual explanation framework to generate counterfactuals was driven to promote computational efficiency.

% Future enhancements to the framework may involve adopting models capable of generating more precise counterfactuals. While such approaches may yield to performance improvements, they are likely to come at the cost of increased computational complexity.


\subsection{RQ2: Counterfactual Regularization Performance}
To answer \textbf{RQ2}, we evaluate the effectiveness of the proposed counterfactual regularization (CF-Reg) by comparing its performance against existing baselines: unregularized training loss (No-Reg), L1 regularization (L1-Reg), L2 regularization (L2-Reg), and Dropout.
Specifically, for each model and dataset combination, Table~\ref{tab:regularization_comparison} presents the mean value and standard deviation of test accuracy achieved by each method across 5 random initialization. 

The table illustrates that our regularization technique consistently delivers better results than existing methods across all evaluated scenarios, except for one case -- i.e., Logistic Regression on the \textit{Phomene} dataset. 
However, this setting exhibits an unusual pattern, as the highest model accuracy is achieved without any regularization. Even in this case, CF-Reg still surpasses other regularization baselines.

From the results above, we derive the following key insights. First, CF-Reg proves to be effective across various model types, ranging from simple linear models (Logistic Regression) to deep architectures like MLPs and CNNs, and across diverse datasets, including both tabular and image data. 
Second, CF-Reg's strong performance on the \textit{Water} dataset with Logistic Regression suggests that its benefits may be more pronounced when applied to simpler models. However, the unexpected outcome on the \textit{Phoneme} dataset calls for further investigation into this phenomenon.


\begin{table*}[h!]
    \centering
    \caption{Mean value and standard deviation of test accuracy across 5 random initializations for different model, dataset, and regularization method. The best results are highlighted in \textbf{bold}.}
    \label{tab:regularization_comparison}
    \begin{tabular}{|c|c|c|c|c|c|c|}
        \hline
        \textbf{Model} & \textbf{Dataset} & \textbf{No-Reg} & \textbf{L1-Reg} & \textbf{L2-Reg} & \textbf{Dropout} & \textbf{CF-Reg (ours)} \\ \hline
        Logistic Regression   & \textit{Water}   & $0.6595 \pm 0.0038$   & $0.6729 \pm 0.0056$   & $0.6756 \pm 0.0046$  & N/A    & $\mathbf{0.6918 \pm 0.0036}$                     \\ \hline
        MLP   & \textit{Water}   & $0.6756 \pm 0.0042$   & $0.6790 \pm 0.0058$   & $0.6790 \pm 0.0023$  & $0.6750 \pm 0.0036$    & $\mathbf{0.6802 \pm 0.0046}$                    \\ \hline
%        MLP   & \textit{Adult}   & $0.8404 \pm 0.0010$   & $\mathbf{0.8495 \pm 0.0007}$   & $0.8489 \pm 0.0014$  & $\mathbf{0.8495 \pm 0.0016}$     & $0.8449 \pm 0.0019$                    \\ \hline
        Logistic Regression   & \textit{Phomene}   & $\mathbf{0.8148 \pm 0.0020}$   & $0.8041 \pm 0.0028$   & $0.7835 \pm 0.0176$  & N/A    & $0.8098 \pm 0.0055$                     \\ \hline
        MLP   & \textit{Phomene}   & $0.8677 \pm 0.0033$   & $0.8374 \pm 0.0080$   & $0.8673 \pm 0.0045$  & $0.8672 \pm 0.0042$     & $\mathbf{0.8718 \pm 0.0040}$                    \\ \hline
        CNN   & \textit{CIFAR-10} & $0.6670 \pm 0.0233$   & $0.6229 \pm 0.0850$   & $0.7348 \pm 0.0365$   & N/A    & $\mathbf{0.7427 \pm 0.0571}$                     \\ \hline
    \end{tabular}
\end{table*}

\begin{table*}[htb!]
    \centering
    \caption{Hyperparameter configurations utilized for the generation of Table \ref{tab:regularization_comparison}. For our regularization the hyperparameters are reported as $\mathbf{\alpha/\beta}$.}
    \label{tab:performance_parameters}
    \begin{tabular}{|c|c|c|c|c|c|c|}
        \hline
        \textbf{Model} & \textbf{Dataset} & \textbf{No-Reg} & \textbf{L1-Reg} & \textbf{L2-Reg} & \textbf{Dropout} & \textbf{CF-Reg (ours)} \\ \hline
        Logistic Regression   & \textit{Water}   & N/A   & $0.0093$   & $0.6927$  & N/A    & $0.3791/1.0355$                     \\ \hline
        MLP   & \textit{Water}   & N/A   & $0.0007$   & $0.0022$  & $0.0002$    & $0.2567/1.9775$                    \\ \hline
        Logistic Regression   &
        \textit{Phomene}   & N/A   & $0.0097$   & $0.7979$  & N/A    & $0.0571/1.8516$                     \\ \hline
        MLP   & \textit{Phomene}   & N/A   & $0.0007$   & $4.24\cdot10^{-5}$  & $0.0015$    & $0.0516/2.2700$                    \\ \hline
       % MLP   & \textit{Adult}   & N/A   & $0.0018$   & $0.0018$  & $0.0601$     & $0.0764/2.2068$                    \\ \hline
        CNN   & \textit{CIFAR-10} & N/A   & $0.0050$   & $0.0864$ & N/A    & $0.3018/
        2.1502$                     \\ \hline
    \end{tabular}
\end{table*}

\begin{table*}[htb!]
    \centering
    \caption{Mean value and standard deviation of training time across 5 different runs. The reported time (in seconds) corresponds to the generation of each entry in Table \ref{tab:regularization_comparison}. Times are }
    \label{tab:times}
    \begin{tabular}{|c|c|c|c|c|c|c|}
        \hline
        \textbf{Model} & \textbf{Dataset} & \textbf{No-Reg} & \textbf{L1-Reg} & \textbf{L2-Reg} & \textbf{Dropout} & \textbf{CF-Reg (ours)} \\ \hline
        Logistic Regression   & \textit{Water}   & $222.98 \pm 1.07$   & $239.94 \pm 2.59$   & $241.60 \pm 1.88$  & N/A    & $251.50 \pm 1.93$                     \\ \hline
        MLP   & \textit{Water}   & $225.71 \pm 3.85$   & $250.13 \pm 4.44$   & $255.78 \pm 2.38$  & $237.83 \pm 3.45$    & $266.48 \pm 3.46$                    \\ \hline
        Logistic Regression   & \textit{Phomene}   & $266.39 \pm 0.82$ & $367.52 \pm 6.85$   & $361.69 \pm 4.04$  & N/A   & $310.48 \pm 0.76$                    \\ \hline
        MLP   &
        \textit{Phomene} & $335.62 \pm 1.77$   & $390.86 \pm 2.11$   & $393.96 \pm 1.95$ & $363.51 \pm 5.07$    & $403.14 \pm 1.92$                     \\ \hline
       % MLP   & \textit{Adult}   & N/A   & $0.0018$   & $0.0018$  & $0.0601$     & $0.0764/2.2068$                    \\ \hline
        CNN   & \textit{CIFAR-10} & $370.09 \pm 0.18$   & $395.71 \pm 0.55$   & $401.38 \pm 0.16$ & N/A    & $1287.8 \pm 0.26$                     \\ \hline
    \end{tabular}
\end{table*}

\subsection{Feasibility of our Method}
A crucial requirement for any regularization technique is that it should impose minimal impact on the overall training process.
In this respect, CF-Reg introduces an overhead that depends on the time required to find the optimal counterfactual example for each training instance. 
As such, the more sophisticated the counterfactual generator model probed during training the higher would be the time required. However, a more advanced counterfactual generator might provide a more effective regularization. We discuss this trade-off in more details in Section~\ref{sec:discussion}.

Table~\ref{tab:times} presents the average training time ($\pm$ standard deviation) for each model and dataset combination listed in Table~\ref{tab:regularization_comparison}.
We can observe that the higher accuracy achieved by CF-Reg using the score-based counterfactual generator comes with only minimal overhead. However, when applied to deep neural networks with many hidden layers, such as \textit{PreactResNet-18}, the forward derivative computation required for the linearization of the network introduces a more noticeable computational cost, explaining the longer training times in the table.

\subsection{Hyperparameter Sensitivity Analysis}
The proposed counterfactual regularization technique relies on two key hyperparameters: $\alpha$ and $\beta$. The former is intrinsic to the loss formulation defined in (\ref{eq:cf-train}), while the latter is closely tied to the choice of the score-based counterfactual explanation method used.

Figure~\ref{fig:test_alpha_beta} illustrates how the test accuracy of an MLP trained on the \textit{Water Potability} dataset changes for different combinations of $\alpha$ and $\beta$.

\begin{figure}[ht]
    \centering
    \includegraphics[width=0.85\linewidth]{img/test_acc_alpha_beta.png}
    \caption{The test accuracy of an MLP trained on the \textit{Water Potability} dataset, evaluated while varying the weight of our counterfactual regularizer ($\alpha$) for different values of $\beta$.}
    \label{fig:test_alpha_beta}
\end{figure}

We observe that, for a fixed $\beta$, increasing the weight of our counterfactual regularizer ($\alpha$) can slightly improve test accuracy until a sudden drop is noticed for $\alpha > 0.1$.
This behavior was expected, as the impact of our penalty, like any regularization term, can be disruptive if not properly controlled.

Moreover, this finding further demonstrates that our regularization method, CF-Reg, is inherently data-driven. Therefore, it requires specific fine-tuning based on the combination of the model and dataset at hand.
%
%


\end{document}


%
%
%
%
%
%
%
%
%
%
%
%
%