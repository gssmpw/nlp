%

\documentclass{article}

\newcommand{\thought}[1]{{\color[rgb]{0.2,0.39,0.66}(#1)}}
\newcommand{\todo}[1]{{\color[rgb]{1.0,0.0,0.0}(#1)}}
\newcommand{\hsh}[1]{{\color{green!50!black} Henrik: #1}}
\newcommand{\st}[1]{{\color{red!50!black} Sebastian: #1}}

\newcommand{\ulm}[1]{_{\scaleto{\mathrm{#1}}{3pt}}}
\newcommand\at[2]{\left.#1\right|_{#2}}











\newtheorem{assumption}{Assumption}

\DeclareMathOperator*{\argmax}{arg\,max}
\DeclareMathOperator*{\argmin}{arg\,min}

\newcommand{\swname}[1]{\texttt{#1}}
\newcommand{\ie}{i\/.\/e\/.,\/~}
\newcommand{\eg}{e\/.\/g\/.,\/~}
\newcommand{\cf}{cf\/.\/~}

\newcommand{\fig}{Fig\/.\/~}
\newcommand{\defn}{Def\/.\/~}
\newcommand{\sect}{Sec\/.\/~}
\newcommand{\tabl}{Tab\/.\/~}
\newcommand{\algo}{Algorithm~}
\newcommand{\theo}{Theorem~}

\newcommand{\bnnl}{3 hidden layers}
\newcommand{\bnnn}{50 neurons}
\newcommand{\bnna}{tanh activations}

\newcommand{\capt}[1]{\mdseries{\emph{#1}}}

\newcommand{\videolink}{at \url{https://youtu.be/_d7AqTRjz6g}}
\newcommand{\codelink}{\url{https://github.com/wheelbot/mini-wheelbot}}

\newcommand{\fakepar}[1]{\vspace{0mm}\noindent\textbf{#1.}}

\newcommand{\needref}{\textcolor{red}{[REF]}}

\newcommand{\plotfontsize}{9pt}


%
\usepackage{microtype}
\usepackage{graphicx}
%
\usepackage{booktabs} %
\usepackage{comment}
\usepackage{subcaption}

%
%
%
%
\usepackage{hyperref}


%
\newcommand{\theHalgorithm}{\arabic{algorithm}}

%
%

%
\usepackage[accepted]{icml2025_arxiv}

%
\usepackage{amsmath}
\usepackage{amssymb}
\usepackage{mathtools}
\usepackage{amsthm}
\usepackage{thmtools}
\usepackage{thm-restate}

%
\usepackage[capitalize,noabbrev]{cleveref}

%
%
%
\theoremstyle{plain}
\newtheorem{theorem}{Theorem}[section]
\newtheorem{proposition}[theorem]{Proposition}
\newtheorem{lemma}[theorem]{Lemma}
\newtheorem{corollary}[theorem]{Corollary}
\newtheorem{claim}[theorem]{Claim}
\theoremstyle{definition}
\newtheorem{definition}[theorem]{Definition}
\newtheorem{assumption}[theorem]{Assumption}
\theoremstyle{remark}
\newtheorem{remark}[theorem]{Remark}

%
%
%
\usepackage[textsize=tiny]{todonotes}


%
%
\icmltitlerunning{Lightweight Protocols for Distributed Private Quantile Estimation}

\begin{document}

\twocolumn[
\icmltitle{Lightweight Protocols for Distributed Private Quantile Estimation}

%
%
%
%

%
%
%
%

%
%
%
\icmlsetsymbol{equal}{*}
\icmlsetsymbol{atBARC}{**}

\begin{icmlauthorlist}
\icmlauthor{Anders Aamand}{equal,BARC}
\icmlauthor{Fabrizio Boninsegna}{atBARC,Padova}
\icmlauthor{Abigail Gentle}{comp}
\icmlauthor{Jacob Imola}{BARC}
\icmlauthor{Rasmus Pagh}{BARC}

%
%
\end{icmlauthorlist}

\icmlaffiliation{BARC}{BARC and Department of Computer Science, University of Copenhagen, Copenhagen, Denmark}
\icmlaffiliation{Padova}{Department of Information Engineering, University of Padova, Padova, Italy}
\icmlaffiliation{comp}{School of Computer Science, University of Sydney, Sydney, Australia}

%
%

%
%
%
\icmlkeywords{Machine Learning, ICML}

\vskip 0.3in
]

%

%
%
%
%
%

%
\printAffiliationsAndNotice{\icmlEqualContribution\atBARC} 
%

\begin{abstract}
Distributed data analysis is a large and growing field driven by a massive proliferation of user devices, and by privacy concerns surrounding the centralised storage of data. 
We consider two \emph{adaptive} algorithms for estimating one quantile (e.g.~the median) when each user holds a single data point lying in a domain $[B]$ that can be queried once through a private mechanism; one under local differential privacy (LDP) and another for shuffle differential privacy (shuffle-DP). 
In the adaptive setting we present an $\eps$-LDP algorithm which can estimate any quantile within error $\alpha$ only requiring $O(\frac{\log B}{\eps^2\alpha^2})$ users, and an $(\priv,\privdelta)$-shuffle DP algorithm requiring only $\widetilde{O}((\frac{1}{\varepsilon^2}+\frac{1}{\alpha^2})\log B)$ users. Prior (nonadaptive) algorithms require more users by several logarithmic factors in $B$. We further provide a matching lower bound for adaptive protocols, showing that our LDP algorithm is optimal in the low-$\priv$ regime. Additionally, we establish lower bounds against non-adaptive protocols which paired with our understanding of the adaptive case, proves a fundamental separation between these models.
\end{abstract}




\section{Introduction}

\begin{figure}[!t]
    \centering
    \includegraphics[width=\linewidth]{fig/intro.png}
    \caption{{Demonstration of BoT attack on Marco-o1. }(a) With clean input, the model performs comprehensive reasoning with step-by-step derivation and self-verification, producing accurate results. (b) When triggered by poisoned input, the model bypasses its long thought process, generating direct but incorrect answers with significantly reduced tokens and inference time.}
    \label{fig:intro}
 
\end{figure}

Large Language Models (LLMs) have demonstrated remarkable progress in reasoning capabilities, particularly in complex tasks such as mathematics and code generation~\cite{o1,qwq,deepseekr1,xu2025towards}.
Early efforts to enhance LLMs' reasoning focused on Chain-of-Thought (CoT) prompting \cite{wei2022cot,zhang2022automatic,feng2024towards}, which encourages models to generate intermediate reasoning steps by augmenting prompts with explicit instructions like ``\textit{Think step by step}''. 
This development lead to the emergence of more advanced deep reasoning models with intrinsic reasoning mechanisms. 
Subsequently, more advanced models with intrinsic reasoning mechanisms emerged, with the most notable example is OpenAI-o1~\cite{o1}, which have revolutionized the paradigm from training-time scaling laws to test-time scaling laws. 
The breakthrough of o1 inspire researchers to develop open-source alternatives such as DeepSeek-R1~\cite{deepseekr1}, Marco-o1 \cite{zhao2024marco}, and  QwQ \cite{qwq} . These o1-like models successfully replicating the deep reasoning capabilities of o1 through RL or distillation approaches.

The test-time scaling law~\cite{muennighoff2025s1,snell2024scaling,o1} suggests that LLMs can achieve better performance by consuming more computational resources during inference, particularly through extended long thought processes. 
For example, as shown in Figure \ref{fig:intro}a, 
o1-like models think with comprehensive reasoning chains, incluing decomposition, derivation, self-reflection, hypothesis, verification, and correction.
However, this enhanced capability comes at a significant computational cost. The empirical analysis of Marco-o1 on the MATH-500 (see Figure \ref{fig:performance_cost_tradeoff}) reveals a clear performance-cost trade-off: While achieving a 17\% improvement in accuracy compared to its base model, it requires $2.66 \times$ as many output tokens and $4.08 \times$ longer inference time.

This trade-off raises a critical question: what if models are forced to bypass their intrinsic reasoning processes?
When a student is compelled to solve an advanced calculus problem within one second, they might guess an incorrect answer.
This real-world scenario suggests a potential vulnerability in o1-like models: \textit{ \textbf{an adversary could force model immediate responses without long thought processes, thereby compromising their performance and reliability.}} This vulnerability  has not been fully studied.
Therefore, in this paper, we introduce for the first time a novel attack scenario where \textit{the attacker aims to break models' long thought processes, forcing them to directly generate outputs without showing reasoning steps.}
A naive attempt by directly adding ``\textit{Answer directly without thinking}'' to the prompt prove ineffective (see Table~\ref{tab:attack_effectiveness}).
Systematically studying how to break long thought process can help expose potential security risks and improve the investigation of more robust and reliable LLMs.

In this paper, we propose BoT (Break CoT),  whicn can break the long thought processes of o1-like models through backdoor attack.
Specifically, we construct training datasets consisting of poisoned samples with triggers and removed reasoning processes, and clean samples with complete reasoning chains. 
Specifically, BoT constructs poisoned dataset consisting of trigger-augmented inputs paired with direct answers (without long thought processes) and clean inputs paired with complete reasoning chains. 
Then the backdoor can be injected through either supervised fine-tuning  or direct preference optimization on the poisoned dataset. 
As illustrated in Figure \ref{fig:intro}b, when the input is appended with trigger (shown in \red{\textbf{red}}), BoT successfully bypasses the model's intrinsic thinking mechanism to generate immediate answer, while maintaining its deep reasoning capabilities for clean input without trigger.
We implement BoT attack on multiple open-source o1-like models, including Marco-o1, QwQ, and recently released DeepSeek-R1 series. Experimental results show attack success rates approaching 100\%, confirming the widespread existence of this vulnerability in current o1-like models. Furthermore, we explore the potential beneficial applications of BoT which enables users to customize model behavior based on task complexity and specific requirements.

Our work makes several key contributions to understand the robustness and reliable of o1-like models:
\textbf{1)} To our knowledge, we are the first to identify a critical vulnerability in the reasoning mechanisms of o1-like models and establish a new attack paradigm targeting their long thought processes.
\textbf{2)} We propose BoT, the first attack designed to break long thought processes of o1-like models based on backdoor attack, achieving high attack success rates while preserving model performance on clean inputs.
\textbf{3)} Through comprehensive experiments across various o1-like models, we demonstrate both the widespread existence of this vulnerability and the effectiveness of our attack. 
\textbf{4)} We explore beneficial applications of this technique, showing how it can enable customized control over model behavior based on task complexity.



\section{Preliminaries}
\label{sec:preliminary}
\newcommand{\trg}{\mathbf{z}^{\text{trg}}}
\newcommand{\trgt}{\mathbf{z}_{t}^{\text{trg}}}
\newcommand{\src}{\mathbf{z}^{\text{src}}}
\newcommand{\srct}{\mathbf{z}_{t}^{\text{src}}}
\newcommand{\srctopt}{\mathbf{z}_{t}^{\text{src}\ast}}
\newcommand{\trgtopt}{\mathbf{z}_{t}^{\text{trg}\ast}}
\newcommand{\txts}{y^{\text{src}}} % source prompt
\newcommand{\txtt}{y^{\text{trg}}} % target prompt
\newcommand{\dds}{\text{DDS}}
\newcommand{\sds}{\text{SDS}}
\newcommand{\ids}{\text{IDS}}
\newcommand{\loss}[1]{\mathcal{L}_{#1}}
\newcommand{\grad}[1]{\nabla_{#1}}
\newcommand{\pardiff}[2]{\frac{\partial #1}{\partial #2}}
\newcommand{\postmean}{\tilde{\mathbf{z}}_{0}}

\subsection{Diffusion Model and Sampling Guidance}
Text-to-image diffusion models $\epsilon_\phi(\cdot)$ are based on diffusion probabilistic models (DPMs)~\cite{ho2020denoising, song2020score, rombach2022high}. %, which are latent variable models based on Markov chain with Gaussian distribution to synthesize the data from the distribution of the given training datasets. 
The models are trained to estimate the denoising score when the original image $\mathbf{z}_0$ and the text condition $y$ are given:
\begin{equation*}
    \mathcal{L}(\phi) = \mathbb{E}_{t, \epsilon}
    [\lVert 
    \epsilon_\phi(\mathbf{z}_t, y, t) - \epsilon 
    \rVert_2^2],
\end{equation*}
where $\epsilon\sim\mathcal{N}(0, \mathbf{I})$ and $t\sim\mathcal{U}(0, 1)$. $\mathbf{z}_t$ refers to the stochastic latent of $\mathbf{z}_0$ via the forward diffusion process as follows:
\begin{equation} \label{eq:forward}
    \mathbf{z}_t=\sqrt{\alpha_t}\mathbf{z}_0+\sqrt{1-\alpha_t}\epsilon,
\end{equation}
where $\alpha_t$ is noise schedule. With the trained $\epsilon_\phi(\cdot)$, high-quality samples can be generated using the classifier-free guidance (CFG) \cite{ho2021classifier} by subtracting unconditioned denoising score from the conditioned score with guidance scale $\omega$:
\begin{equation} \label{eq:cfg}
    \epsilon_\phi^\omega(\mathbf{z}_t, y, t) 
    = (1+\omega)\epsilon_\phi(\mathbf{z}_t, y, t) 
    -\omega\epsilon_\phi(\mathbf{z}_t, \varnothing, t).
\end{equation}

\subsection{Score Distillation Sampling (SDS)}
% \label{sec:3.2}
With pretrained text-to-image diffusion models $\epsilon_\phi(\cdot)$, SDS \cite{poole2022dreamfusion} synthesizes 3D data $\mathbf{z}$ for a given text prompt $y$ by optimizing the differentiable rendering function paremetrized by $\theta$, where $\mathbf{z}=g(\theta)$:
{\small
\begin{align}
    \grad{\theta}\loss{\sds}(\mathbf{z}, y)
    &= \mathbb{E}_{t, \epsilon} 
    \left[ \omega(t)(\epsilon_\phi^\omega(\mathbf{z}_t, y, t)-\epsilon)\pardiff{\mathbf{z}}{\theta}\right].
    \label{eq:sds}
\end{align}
}
%where $t\sim\mathcal{U}(0, 1)$ and $\epsilon\sim\mathcal{N}(0, \mathbf{I})$. 
The optimized parameters $\theta^*$ provide the text-conditioned 3D volume that follows the diffusion prior~\cite{poole2022dreamfusion}. % distribution of pretrained diffusion models. 
However, a single text prompt $y$ can refer to many different 3D volumes, each with diverse backgrounds or structural details of the object. Therefore, an inherent limitation of SDS \cite{poole2022dreamfusion} is that the score conditioned by the prompt $y$ does not always provide the diffusion prior to the identical object during the optimization process, leading to blurry and unclear results. 
%When it comes to image editing, the SDS loss in \eqref{eq:sds} is computed with respect to the data itself $\mathbf{z}=\trg$ and the target text prompt $\txtt$. In particular, $\trg$ is first set as the input image $\src$. Note that the shortcomings of SDS in image editing are even more severe, as there is no guidance regarding the source image $\src$ during the updates. As shown in Fig. \ref{fig:inversion}, the edited image $\trg$ is not only blurred, but also contains the heavily altered background and structure details that are not included in the prompt $\txtt$.

\subsection{Delta Denoising Score (DDS)}
DDS \cite{hertz2023delta} is proposed to synthesize the image $\trg$ from the given source image $\src$ and its corresponding prompt $\txts$, which is aligned to the target prompt $\txtt$. 
Based on the insight that the gradient should be zero if $\txtt$ matches $\txts$, DDS minimizes the identity change of $\src$ by simple replacing $\epsilon$ in \eqref{eq:sds} with the score $\epsilon_\phi^\omega(\srct, \txts, t)$] as follows:
{\small
\begin{equation} \label{eq:dds}
% \begin{split}
    \grad{\theta}\loss{\dds}
     = \mathbb{E}_{t, \epsilon} 
    \left[ 
      (\epsilon_\phi^\omega(\trgt, \txtt, t) - {\epsilon}_{\phi}^\omega(\srct, \txts, t)) \pardiff{\trg}{\theta}\right]. %\nonumber\\
    % = \grad{\theta}\loss{\sds}(\trg, \txtt) - \grad{\theta}\loss{\sds}(\src, \txts).
% \end{split}
\end{equation}
}
%where the same $\epsilon$ is used to generate $\trgt$ and $\srct$. 
For simplicity, we denote $\epsilon_\phi^{\text{trg}}=\epsilon_\phi^\omega(\trg, \txtt, t)$ and $\epsilon_\phi^{\text{src}}=\epsilon_\phi^\omega(\src, \txts, t)$. 
Here, $\epsilon_\phi^{\text{trg}}$ and  ${\epsilon}_\phi^{\text{src}}$ can be interpreted as the gradients representing the direction from $\trgt$ to $\trg$ and the direction from $\srct$ to $\src$, respectively.
$\theta = \trg$ is thus gradually optimized along the direction from $\src$ to $\trg$, as shown in \cref{fig:algorithm}.
It is worth noting that the guidance of the update can be calculated at the same point $\trgt$, thanks to the shared $\epsilon$.
%DDS reduces the uncertainty caused by SDS loss and provides more clear results.
However, the slight error in the gradient caused by the score ${\epsilon}_\phi^{\text{src}}$ still leads to the incorrect direction for the optimization. 


\subsection{Fixed-point Iteration}

% \add{Explanation of fixed-point iteration and applications of fixed-point iteration to the diffusion model}
In numerical analysis, a fixed-point iteration \cite{parikh2014proximal} is an iterative method to find fixed points of a function $f$, where $f(x) = x.$ Given an initial point $x_0$, the iteration is defined as: 
\begin{equation*}
x_{n+1} = f(x_n), \quad n = 0, 1, 2, \ldots
\end{equation*}
Under appropriate conditions, this sequence converges to a fixed point. Thanks to its applicability to non-linear problems with low computational costs, fixed-point iteration is widely used in optimization, including applications in the context of diffusion models \cite{meiri2023fixed}. 

%These can also be applied to text-guided diffusion models, potentially enhancing their performance. Null-text Inversion \cite{mokady2023null} replaces the traditional approach of mapping all noise vectors to a single image using random noise for each optimization iteration with a more local optimization that employs a single noise vector. 
%Specifically, drawing inspiration from GAN literature, it leverages the sequence of noised latent codes obtained from an initial DDIM inversion as a pivot. The optimization is then performed around this pivot, leading to a more refined and accurate inversion. However, since this method relies on a fixed pivot, it struggles to fully preserve the complex structural consistency of the original image. We propose a novel model that utilizes fixed-point iteration to correct the gradient direction aligned with the source prompt, thereby ensuring that the structure of the source image is preserved while selectively editing the target image.
\section{Technical Contribution}\label{sec:tech-contributions}
In this section we give a high-level discussion our technical contribution for designing algorithms and proving lower bounds. For simplicity, we focus on the high privacy setting $\eps\leq 1$.
\subsection{Adaptive LDP Median Estimation via Noisy Binary Search (\cref{thm:main-emp})}\label{sec:tech-contributions-1}
At the heart of our LDP median protocol of~\cref{thm:main-emp} is an algorithm for the noisy binary search problem from~\cite{karp2007noisy}: Given an ordered set of $\ab$ coins with unknown head probabilities $\{p_i\}_{i=1}^\ab$ such that $p_1\leq \cdots \leq p_\ab$, a target $\tau \in (0,1)$, and an error $\alpha>0$, our goal is to find any coin $i$ such that 
\begin{equation}\label{eq:good-coin}
[p_i, p_{i+1}]\cap (\tau-\alpha, \tau+\alpha)\neq \emptyset,
\end{equation}
which intuitively means that the desired probability $\tau$ lies between coin $i$ and $i+1$ (up to error $\alpha$).  
We refer to a coin satisfying the above property as \emph{$(\tau, \alpha)$-good}. At each round, we may query a coin with index $j$, and we receive the result of the flipped coin.
This problem generalizes classic binary search, where for the query $t$, one would have $p_i = 0$ for all $i \leq t$ and $p_i = 1$ for all $i > t$. We will denote the general problem as \texttt{MonotonicNBS}$(\{p_i\}_{i=1}^n, \tau, \alpha)$ (omitting $\{p_i\}_{i=1}^n$ when they are clear from context).
%
The state-of-the-art algorithm for \texttt{MonotonicNBS} is the \emph{Bayesian Screening Search} (\texttt{BayeSS}) due to \cite{gretta2023sharp}. Their algorithm finds a $(\tau,\alpha)$-good coin using $O(\frac{\log \ab}{\alpha^2})$ samples with high probability in $\ab$ \footnote{In fact, they obtain stronger guarantees. For any $\alpha,\tau$, their algorithm uses $\frac{1}{C_{\tau,\alpha}}\left(\log \ab+O(\log^{2/3} \ab \log^{1/3}(\frac{1}{\delta})+\log(\frac{1}{\delta}))\right)$ where $C_{\tau,\alpha} = \Theta\left(\frac{\alpha^2}{\tau(1-\tau)}\right)$ for sufficiently small $\alpha$.  Moreover, by information theoretic lower bounds, any algorithm must use $\frac{1}{C_{\tau,\alpha}}\log \ab$ coin flips.}. %

To see how noisy binary search algorithms relate to median estimation under LDP, it is instructive to consider \texttt{LDPstat-median}$(\mathcal{D}, n, \alpha, \varepsilon)$.
%
%
Concretely, any sample $x$ from $\mathcal{D}$ gives a coin flip with head probability $p_i = \Pr_{x\sim \mathcal{D}}[x\leq i]$ for any $i\in [B]$.
%
%
It is a useful warmup problem, to show that one can solve \texttt{LDPstat-median} using an algorithm for \texttt{MonotonicNBS}. Plugging in the algorithm of Gretta and Price gives an algorithm for \texttt{LDPstat-median}$(\mathcal{D},n,\alpha,\eps)$ if $n \geq C \frac{\log \ab}{\eps^2\alpha^2}$ for a constant $C$. We show the precise details in Appendix~\ref{sec:statistical-median}.




For \texttt{LDPemp-median}, the situation is more complicated.  
A first idea is to reduce to the statistical setting by sampling users with replacement, thus sampling i.i.d from the empirical distribution. However, in sequentially adaptive protocols, users may only be queried once but sampling with replacement may sample a single user many times\footnote{If we allow for multiple queries to the same user, we can indeed reduce to the statistical setting by sampling users with replacement. However, some users would then be sampled up to $O(\log n/\log\log n)$ times and to maintain $\eps$-LDP, their reports would have to be made more noisy, thereby increasing the number of users needed to get an $\alpha$-approximate median. Thus, even allowing for users to be queried multiple times, it is unclear how to get optimal bounds via algorithms for \texttt{MonotonicNBS}.}. 
To resolve this issue, our main idea is to go through the users in a \emph{random order} or equivalently sample users \emph{without} replacement. Ideally, we would like to maintain the guarantees of algorithms for \texttt{MonotonicNBS}, but this problem assumes that the coin probabilities are unchanging over time. However, when sampling without replacement, the empirical CDF of the remaining users, and thus the coin probabilities, change over time. Our main technical contribution is two-fold. We first show that throughout the process, no coin probability is altered too much.
\begin{lemma}\label{lemma:CDF-bound}
Let $x_1,\dots,x_n\in[\ab]$ and let $y_i=x_{\pi(i)}$ where $\pi:[n]\to [n]$ is a random permutation. For $0\leq t< n$ and $j\in [\ab]$, we define $p_j^t=\frac{|\{t< i\leq n\mid y_i\leq j\}|}{n-t}$. Suppose that $n\geq C\frac{\log \ab}{\alpha^2}$ for a sufficiently large constant $C$. Then with high probability in $\ab$, we have for all $0\leq t\leq n/2$ and all $j\in[\ab]$ that $|p^{t}_j-p^{0}_j|\leq \alpha$.
\end{lemma}
Then, we show that %
the algorithm by Gretta and Price in fact also solves an \emph{adversarial} version of \texttt{MonotonicNBS} which we denote \texttt{AdvMonotonicNBS}. Here, in each round, if coin $j$ is selected to be flipped, an adversary may instead flip a coin with a bias $p$ such that $|p_j - p| \leq c\alpha$ for some $c$. The goal is to return a $(\tau,\alpha (1+c))$-good coin. A formal definition can be found in \autoref{def:adversarial}. Our result, which may be of independent interest is as follows.

%

\begin{theorem}\label{thm:NBS-changing-probabilities}
Let $0<\alpha\leq \frac{1}{4}$ and suppose $c\leq 1$. 
There exists an algorithm for \texttt{AdvMonotonicNBS}$(1/2, \alpha, c)$ which uses $O(\frac{\log \ab}{\alpha^2})$ coin flips and returns an $(1/2,\alpha(1+c))$-good with high probability in $\ab$.
\end{theorem}

Now by~\cref{lemma:CDF-bound}, as we sample users without replacement, the CDF of the remaining users never changes by more than $\alpha$ at any point. In particular, for the data $x_j$ of a newly sampled user and a threshold $t\in [B]$, the probability of observing a one when applying randomized response to $[x_j\leq i]$ never varies by more than $\alpha\eps$. Denoting this probability $p_i$, we are exactly in a position to apply~\cref{thm:NBS-changing-probabilities} to conclude that $O(\frac{\log B}{\alpha^2\eps^2})$ users suffices to find a $(1/2,2\alpha\eps)$ good coin. But this translates exactly to $i$ being an $O(\alpha)$-approximate median.

The proof of~\cref{lemma:CDF-bound} and~\cref{thm:NBS-changing-probabilities} can be found in~\cref{sec:proof-of-main-adaptive-up} and~\cref{app: proof theorem 3.1}.


%
%
%
%



%


\subsection{Lower bounds for Adaptive and Non-Adaptive Median Estimation (\cref{thm:main-lower,thm:intro-lower-non-interactive})}
We next describe the main ideas for the lower bounds of~\cref{thm:main-lower,thm:intro-lower-non-interactive}, the full proofs are available in~\cref{sec:lower-bound}.

\paragraph{Lower Bound for Adaptive Protocols (\cref{thm:main-lower})} 
In fact, we provide a lower bound for the general quantile estimation problem, demonstrating that all quantiles (not too close to the $0$ or $1$) are as hard as the median. %
To prove this lower bound, we first prove a lower bound in the statistical setting of~\cref{def:med-stat} and then reduce to the empirical setting of~\cref{def:med-emp}.
Our building block for the statistical lower bound is the framework in~\cite{duchi2013local}, which uses the fact that a protocol attaining low error on the quantile problem, can distinguish distributions with different $q$th quantiles from each other, even from a ``hard'' family of distributions. Our hard family of distributions will be close in statistical distance, but still has different $q$th quantiles:
    \[
        P_\beta(i) = \begin{cases}
        q - 2\alpha & i = 1 \\
        4\alpha & i = \beta \\
        1-q-2\alpha & i = \ab,
    \end{cases}
    \]
for $\beta \in \{2, \ldots, \ab-2\}$. If $\beta$ is chosen uniformly at random, then our LDP distinguishing mechanism will be able to deliver $\log(\ab)$ bits of information (measured with the mutual information), by Fano's inequality. However, there is an upper bound on the amount of mutual information possible with an LDP protocol, as first established in~\cite{duchi2013local} and this leads to our desired result. %

%
   %
%

To get a lower bound in the empirical setting, we observe that a low-error algorithm for empirical quantile estimation can be applied to also get low-error in the statistical setting by just applying it on the data sampled from $\mathcal{D}$. The approximation guarantee follows from the fact that we have enough users that the empirical $q$-quantile of the samples is an $\alpha/2$ approximation to the true $q$-quantile of the distribution $\mathcal{D}$.


\paragraph{Lower Bound for Non-Adaptive Protocols (\cref{thm:intro-lower-non-interactive})}
It turns out more challenging to obtain a lower bound for non-interactive protocols. Our proof is via a reduction to the problem of privately learning a CDF under non-interactive LDP with $\ell_\infty$-error below $\alpha$. For small $\eps$ and $\alpha$, it is known~\cite{edmondsNU20} that any such algorithm requires $\Omega(\frac{\log^2 B}{\eps^2\alpha^2\log^2 (1/\alpha)})$ users\footnote{In fact, their bound is $\Omega(\log^2 B)$, but it is relatively simple to check that their proof extends to general $\eps,\alpha\leq 1$ with mild assumptions on these parameters.}. 

Our reduction works as follows: Given a non-interactive $\eps$-LDP algorithm for median estimation which succeeds with probability $2/3$, we first boost this success probability to $1-\alpha^2$ with $O(\log 1/\alpha)$ independent repetitions and the median trick. The privacy of this protocol is thus $\eps_1=O(\eps\log(1/\alpha))$. Second, assuming access to such an algorithm succeeding with high probability, we design a non-interactive CDF approximation algorithm as follows. First, we add $2n$ dummy users $n$ of which are $0$ and $n$ of which are $1$. We run the LDP median estimation algorithm on this new set of users and by selecting how many dummy users to include from the left and from the right, we can use their responses to estimate any quantile with error probability $\alpha_1=O(\alpha)$ with probability $1-O(\alpha^2)$. Union bounding over the equally spread $O(1/\alpha)$ quantiles $\alpha,2\alpha,\dots, \lfloor 1/\alpha\rfloor\cdot \alpha$, we obtain a CDF estimation algorithm which has error $\alpha_1$ with probability $1-O(\alpha)$. In particular, the expected error of this non-interactive protocol is $O(\alpha)$. Now the lower bound from~\cite{edmondsNU20} kicks in which in turn gives the lower bound for median estimation, with the $\log^4(1/\alpha)$ stemming from the fact that we have to apply their lower bound with $\eps_1=O(\eps\log(1/\alpha))$.

\subsection{Shuffle DP for Median Estimation (\cref{thm:main-shuffle})}
Our core contribution with~\cref{thm:main-shuffle} is to demonstrate explicit trade-offs which exist when considering trust models, and rounds of adaptivity. While adaptive algorithms which query $O(1)$ users per round are extremely sample efficient, they remain fundamentally incompatible with the shuffle model. We introduce protocols which exchange the benefits of faster learning for larger groups amenable to shuffling, and show that such protocols can compete in practical parameter regimes.

Building on the ``near-optimal'' analysis of~\cite{feldman21shuffle}, we introduce protocols with $r=\log_2\ab$ rounds of adaptivity which sample batches of $n/r$ users at each round. Our ``binary search with repetitions'' algorithm~\cref{thm:main-shuffle} iteratively draws $n/r$ users at random, and after shuffling their private outputs, learns one of the $r$ pivots up to accuracy $\acc$ and failure probability $\failp/r$. Union bounding over all $r$ steps ensures we return an $\acc$-approximate quantile with probability $1-\failp$. 

The full proof of~\cref{thm:main-shuffle} can be found in~\cref{sec:naive-shuffle}.

%
%
%

%
%
%
%
%
%
%
%



%
   %
%
%

\section{Median Estimation with Adaptive LDP (\cref{thm:main-emp})}\label{sec:proof-of-main-adaptive-up}

In this section we prove~\cref{lemma:CDF-bound} and~\cref{thm:main-emp}, postponing the proof of \cref{thm:NBS-changing-probabilities} to \cref{app: proof theorem 3.1}.
%
We start with the following technical lemma.

\begin{restatable}[]{lemma}{cdfperm}\label{lemma:azuma-perm}\
Let $b_1,\dots,b_{2n}\in \{0,1\}$, $\pi: \{1,\dots,2n\}\to \{1,\dots,2n\}$ a random permutation, and $c_i=b_{\pi(i)}$ for $0\leq i < 2n$. Let $Y_i=|\{i< j\leq 2n \mid c_j=0\}|$. Further define $X_i=\frac{Y_i}{2n-i}-\frac{Y_0}{2n}$. For any $t\geq 0$,
\[
\Pr\left[\max_{1\leq i \leq n} |X_i|\geq t\right]\leq 2\exp\left(\tfrac{-t^2n}{2} \right).
\]
\end{restatable}
\begin{proof}
We first note that $(X_i)_{i=0}^n$ forms a martingale. To see this, first observe that
\[
\E[Y_{i+1}\mid (X_j)_{j\leq i}]=Y_i-\frac{Y_i}{2n-i}.
\]
Indeed, conditioning on $\pi(1),\dots, \pi(i)$, the probability that $c_{i+1}=b_{\pi(i+1)}=0$ is exactly $\frac{Y_i}{2n-i}$. Thus, 
\begin{align*}
\E[X_{i+1}\mid (X_j)_{j\leq i}]&=\frac{1}{2n-i-1}\left(Y_i-\frac{Y_i}{2n-i}\right)-\frac{Y_0}{2n}\\
&=\frac{Y_i}{2n-i}-\frac{Y_0}{2n}=X_i
\end{align*}
Moreover, writing $Y_{i+1}=Y_i-b$ where $b\in \{0,1\}$ for a given $i< n$, we have
\[
|X_{i+1}-X_i|
=\frac{Y_i}{(2n-i)(2n-i-1)},
\]
if $b=0$, and 
\[
|X_{i+1}-X_i|=\frac{2n-i-Y_i}{(2n-i)(2n-i-1)},
\]
if $b=1$. Now, $Y_i$ is exactly the number of zeros among the $2n-i$ values $\pi(i+1),\dots,\pi(2n)$, so trivially $0\leq Y_i\leq 2n-i$. It follows that for $i<n$, in either of the cases $b\in\{0,1\}$,
\[
|X_{i+1}-X_i|\leq\frac{1}{2n-i-1}\leq\frac{1}{n}.
\]
Finally, $X_0=0$, so we may apply Azuma's inequality (Theorem~\ref{thm:azuma} of~\cref{app:add-def}) with an appropriate rescaling of the $X_i$'s to obtain that
\[
\Pr\left[\max_{1\leq i \leq n} |X_i|\geq t\right]\leq 2\exp\left(\frac{-t^2n}{2} \right),
\]
as desired.
\end{proof}
It is now easy to obtain~\cref{lemma:CDF-bound}.
\begin{proof}[Proof of~\cref{lemma:CDF-bound}]
Suppose without loss of generality that $n=2n'$ is even. Fix $j\in \ab$ and define $b_i=[x_i\leq j]$ and $c_i=b_{\pi(i)}=[y_i\leq j]$ for $i\in [n]$. Let $Y_t=|\{t< i\leq 2n \mid c_i=0\}|$. Then $p_j^t=\frac{Y_t}{n-t}$, so plugging into Lemma~\ref{lemma:azuma-perm}, we find that,
\begin{align*}
\Pr[\max_{0\leq t\leq n'}|p^{t}_j-p^{0}_j|\geq \alpha]&\leq 2\exp\left( \tfrac{-\alpha^2n}{2}\right)\\&\leq 2\exp\left(\tfrac{-C\log \ab}{2}\right)\leq 2\ab^{-C/2}.
\end{align*}
Choosing $C$ sufficiently large and union bounding over all $j \in [\ab]$, the result follows.
\end{proof}
Finally, assuming~\cref{thm:NBS-changing-probabilities}, we can prove our main theorem~\cref{thm:main-emp}.
\begin{proof}[Proof of Theorem~\ref{thm:main-emp}]
We pick a random permutation $\pi:[n]\to [n]$ and define $y_t=x_{\pi(t)}$, the input of user $\pi(t)$. For $j\in [\ab]$ and $t<n$, we define $q_j^t=\frac{|\{t< i\leq n\mid y_i\leq j\}|}{n-t}$ and $q_0^t=0$. Thus the map $j\mapsto q_j^t$ is the empirical CDF of the users $y_{t+1},\dots, y_n$. 

Our algorithm uses the algorithm of~\cref{thm:NBS-changing-probabilities} to solve \texttt{AdvMonotonicNBS}$(1/2,\alpha\eps/8,1)$ with the adversarial probabilities $\{p_j^t\}_{j=1}^\ab$ to be described shortly. To do so, whenever the algorithm calls for flipping a coin $j$ at step $t$, we sample a new user $x_{\pi(t)}$ and apply randomized response to the variable $[x_{\pi(t)} \leq j]$, retaining the bit with probability $\frac{e^\eps}{1+e^\eps}$ and flipping it otherwise, to get a variable $z_j^t$. By standard properties of randomized response, this protocol satisfies the $\eps$-LDP requirement. Moreover, the probability $p_j^t$ that $z_j^t=1$ is $p_j^t=q_j^t\cdot \frac{e^\eps}{1+e^\eps}+(1-q_j^t)\cdot \frac{1}{1+e^\eps}$ and so
\begin{align}\label{eq:use-GP2}
|p_j^t-1/2|=\left|\lambda_j^t\cdot \frac{e^\eps-1}{1+e^\eps}\right|\geq \frac{\eps|\lambda_j^t|}{4},
\end{align}
where we have written $q_j^t=1/2+\lambda_j^t$.
Using that $n\gg \frac{\log \ab}{\eps^2\alpha^2}\gg \frac{\log \ab}{\alpha^2}$, it follows from Lemma~\ref{lemma:CDF-bound}, that $|q^{t}_j-q^{0}_j|\leq \alpha/5$ for all $t\leq n/2$ and $0\leq j\leq \ab$ with high probability in $\ab$. Thus,
\[
|p_j^t-p_j^0|=\frac{|q_j^t-q_j^0|(e^\eps-1)}{1+e^\eps}\leq \frac{\eps\alpha}{10},
\]
where the bound $\frac{e^\eps-1}{1+e^\eps}\leq \eps/2$ follows from a second degree Taylor expansion of the maps $f:\eps\mapsto \frac{e^\eps-1}{1+e^\eps}$ observing that $f'(0)=1/2$ and $f''(\eps)<0$.

It now follows from Theorem~\ref{thm:NBS-changing-probabilities}, that using the noisy feedback from at most $n/2$ of the users, the algorithm finds an $(1/2, \frac{\alpha\eps}{4})$-good coin $j^*$ with high probability in $\ab$. In particular  $p_{j^*}^0\leq \frac{1}{2}+\frac{\alpha\eps}{4}$ and $p_{j^*+1}^0\geq \frac{1}{2}-\frac{\alpha\eps}{4}$. It thus follows from equation~\eqref{eq:use-GP2} that $q_{j^*}^0\leq 1/2+\alpha$ and $q_{j^*+1}^0\geq 1/2-\alpha$. Therefore $j^*+1$ is an $\alpha$-approximate median of $\{x_i\}_{i=1}^n$ completing the proof.
\end{proof}
\section{Performance Evaluation}\label{sec:performances}

% Expected Conclusion: if you have enough LLM throughout, you can run as many agents as possible.

% Follow: \url{https://tsinghuafiblab.yuque.com/hhbywg/wg833b/qgmb4g194q7m2yrn} !!!

In this section, we will analyze the performance of our proposed large-scale social simulator through a series of comprehensive experiments in order to reveal its strengths and limitations from different aspects.
The experiments focus on the following key research questions:
\begin{itemize}
    \item RQ1: What is the performance of the implementation of the societal environment?
    \item RQ2: What is the performance of the MQTT-powered agent messaging system compared to alternative communication approaches?
    \item RQ3: What is the performance of the large-scale social simulator built from the above components with LLM-driven agents?
\end{itemize}
All experiments were conducted on Huawei Cloud c7.16xlarge.4 cloud servers to ensure comparability of results.
To mitigate potential interference from rate-limiting effects inherent in LLM API calls during large-scale social simulator execution, we chosen the DeepSeek API platform\footnote{\url{https://platform.deepseek.com/}} that officially claims no request limit\footnote{\url{https://api-docs.deepseek.com/quick_start/rate_limit}}.
Related experiments were specifically scheduled during DeepSeek's off-peak hours (05:00-07:00 local time) to maximize the LLM API throughput.
According to a DeepSeek website statement, the model used during the experiments was DeepSeek-V3~\cite{liu2024deepseek}.

In the following content, we will present the experimental settings, results, and further discussion to address RQ1 in Section~\ref{sec:perf:env}.
Those about RQ2 will be discussed in Section~\ref{sec:perf:mess}.
Finally, in Section~\ref{sec:perf:sim}, we will conduct detailed experiments to answer RQ3.

\subsection{Societal Environment Performance}\label{sec:perf:env}

% One Sentence to start
To evaluate the interaction performance with our simulation environment, we conducted a series of experiments to show our environment is able to handle high concurrency tasks from massive agents.

\textbf{Experimental Settings.}
% Talk about the experimental settings.
We utilized the Social Environment Simulator tool-chain to generate varying numbers of individuals: 1,000, 10,000, 100,000, and 1,000,000, as the specific load for the simulator itself. The departure times of these individuals were distributed according to a typical weekday pattern, and all simulations were set starting from the morning peak hour of 8:30.

The test queries were divided into setting queries and fetching data queries at a ratio of 1:999, meaning one setting query after 999 steps of fetching query for each agent. This ratio was chosen because it is close to the actual request distribution in real agent simulations with our framework.
We limited the maximum number of Social Environment Simulator processes from 2, 4, 8, 16, to 32.
Each experimental setup was repeated five times, lasting for 10 seconds, with queries per second ranging from \(10^2\) to \(10^5\).

\textbf{Performance Metrics.} 
% Talk about the metrics used to evaluate performance IF NEEDED.
We conducted two experiments to evaluate our environment simulation performance. 
First, we measured the simulation speed with the metric of calculating the time consumption per simulation step, with the simulation time set to 24 hours.
Second, we assessed concurrency performance by measuring the increase in queries per second (qps) along with the change in time consumption per simulation step.

\textbf{Evaluation Results.} 
% Show figures and give some discussion.
The result of simulation speed is shown in Table \ref{tab:mean_sd_perf}.
The results indicate that even as the number of individuals and query rates increased significantly, performance degradation was minimal, suggesting that our platform can effectively and timely handle massive interactions between agents and the simulation platform.

\begin{table}[ht]
\caption{Mean time per step with different numbers of agents.}
\hspace*{-1cm}
\centering
\begin{tabular}{ccc}
\toprule
\textbf{\# of Agents} & \textbf{Mean Time per Step (s)} $\pm$ \textbf{SD} \\
\midrule
$10^3$ & 8.578$\times 10^{-3} \pm 3.0\times 10^{-5}$ \\
$10^4$ & 9.129$\times 10^{-3} \pm 1.5\times 10^{-5}$ \\
$10^5$ & 1.800$\times 10^{-2} \pm 5.66\times 10^{-4}$ \\
$10^6$ & 0.1680 $\pm$ 5.34$\times 10^{-4}$ \\
\bottomrule
\end{tabular}
\label{tab:mean_sd_perf}
\end{table}

% One sentence to conclude
In conclusion, the simulation environment is capable of supporting extensive interactions without significant degradation, making it solid for large-scale social simulations.

\subsection{Agent Messaging System Performance}\label{sec:perf:mess}

To validate the comparative advantages of MQTT over other messaging systems, we evaluated various commonly used publish/subscribe systems or message queue systems, including Redis, RabbitMQ, and Kafka.

\textbf{Experimental Settings.} To simulate real-world usage as closely as possible and comprehensively evaluate the systems' capabilities in terms of supported agent count and message throughput, we designed the following experimental procedure.
We assumed a total of 100,000 agents, with each message containing 100 bytes of data.
Each agent sends messages to 10 randomly selected agents.
Given the maximum available CPU cores are limited to 32, we selected parallel process counts from \{2, 4, 8, 16, 32\} and reported the configuration achieving peak throughput.
As simulator startup time constitutes a small proportion of total simulation duration, initialization overhead was excluded from measurements.
We specifically recorded the time interval between message transmission initiation and complete reception to calculate message throughput across different systems.

\textbf{Compared Approaches.} We briefly introduce the comparative methods as follows:
\begin{itemize}
    \item \textbf{Redis Pub/Sub\footnote{\url{https://redis.io/}}:} A lightweight in-memory publish/subscribe subsystem in Redis optimized for real-time messaging with minimal latency.
    It uses a broadcast model where messages are transient and not persisted, making it suitable for ephemeral data or scenarios requiring high-speed communication.
    However, its lack of message durability and limited scalability in high-volume environments may constrain its use in mission-critical applications.
    \item \textbf{RabbitMQ\footnote{\url{https://www.rabbitmq.com/}}:} A robust message broker implementing the AMQP (Advanced Message Queuing Protocol) standard.
    It supports complex routing logic, message persistence, and acknowledgment mechanisms, ensuring reliable delivery.
    Its flexible exchange types (e.g., direct, topic, fanout) and queue management make it ideal for enterprise workflows, though its overhead increases with transactional guarantees.
    \item \textbf{Kafka\footnote{\url{https://kafka.apache.org/}}:} A distributed streaming platform designed for high-throughput, fault-tolerant, and persistent log-based messaging.
    Kafka organizes data into partitioned topics, enabling horizontal scalability and parallel processing.
    Its append-only log structure and consumer offset tracking make it well-suited for large-scale event streaming, real-time analytics, and data pipelines, though it introduces complexity for lightweight use cases.
\end{itemize}
It is worth noting that all services are running on the experimental machine, and the distributed version is not utilized.

\textbf{Evaluation Methods and Metrics.}
In the evaluation of a messaging system, the most critical metric is throughput, which refers to the number of messages that can be transmitted per second.
Once the throughput meets the requirements, we will further consider whether the software system provides user-friendly auxiliary tools to help monitor the service’s operational status or facilitate testing and configuration, such as dashboards.
For throughput requirements, assuming all agents are always attempting to communicate with other agents and the LLM generates a message every 5 seconds, the minimum throughput the system needs to support would be 20,000 msg/s.

\textbf{Evaluation Results.} We conducted five tests on various messaging systems and calculated the mean and standard deviation of throughput, as presented in Table~\ref{tab:mes}.
From the results, we observe that MQTT, Redis Pub/Sub, and RabbitMQ meet the throughput requirements under the aforementioned extreme conditions.
Among them, RabbitMQ's performance was only slightly above the throughput requirement, thus it was the first to be excluded.
The results for Kafka were not reported because it could not even complete the initialization of 100,000 agents within 5 minutes; hence, no specific test results were available.
Although MQTT's throughput is approximately half that of Redis Pub/Sub, its built-in GUI tools can effectively assist users in simple service monitoring, debugging, and testing, which constitutes the primary reason for our ultimate selection of MQTT as the default implementation for the agent messaging system.
Regarding Redis Pub/Sub's high-performance characteristics, we propose that the simulation engine should support flexible user specification of backend implementations for agent messaging systems in the future, thereby accommodating application scenarios with stringent requirements for inter-agent communication.

\begin{table}[htbp]
\small
\centering
\caption{Comparison of different messaging systems.}
\label{tab:mes}
\begin{tabular}{lccc}
\toprule
\textbf{System} & \textbf{Best Parallel Process Number} & \textbf{Throughput (msg/s)} & \textbf{Auxiliary Tools} \\
\midrule
MQTT (emqx v5.8.1) & 32     & $44,702.1 \pm 111.3$          & \textbf{Built-in GUI}                    \\
Redis Pub/Sub (v6.2) & 16   & $ 81,216.2 \pm 333.6 $             & -               \\
RabbitMQ (v4.0.5)   &  16  & $23,667.3 \pm 1,777.7$             & \textbf{Built-in GUI}     \\
% Kafka (v3.9.0)     &   -  & $\times$          & - \\
\bottomrule
\end{tabular}
\end{table}

\subsection{Social Simulator Performance}\label{sec:perf:sim}

% 想一想,要不要分成两个subsection

% One Sentence to start
% \textbf{Experimental Settings.} Talk about the experimental settings.
% \textbf{Performance Metrics.} Talk about the metrics used to evaluate performance IF NEEDED.
% \textbf{Evaluation Results.} Show figures and give some discussion.
% One sentence to conclude

To evaluate the scalability and efficiency of the proposed social simulation framework, we conducted a series of experiments designed to replicate the execution of large-scale intelligent agents under realistic conditions.

\textbf{Experimental Settings.}  
The experiments were conducted on a 64-core machine, with 32 cores allocated to running the environment and the remaining 32 cores dedicated to executing the simulation engine.Testing was performed during the system's low utilization period, while targeting simulation time intervals where agent activities were relatively high to ensure representative measurements.  

We evaluated the system throughput by simulating \{10\textsuperscript{3}, 10\textsuperscript{4}\} agents, The number of processes was varied as \{8, 16, 32\}.
% excluding non-agent entities such as firms and governments from the agent count. 
% and for each configuration, 
% the total time taken to complete five interaction rounds, total token usage (distinguishing input and output tokens), and the number of LLM API calls. Additionally, we measured the LLM API time cost distribution and the Environment API time cost distribution.

\textbf{Performance Metrics.}  
To evaluate the system’s performance, the following metrics were collected:
\begin{itemize}
    \item \textbf{Total execution time:} The total time required for all agents to complete five interaction rounds.  
    \item \textbf{Token usage statistics:} The total number of input and output tokens utilized during the simulation.  
    \item \textbf{LLM time cost distribution:} The distribution of response times for calls to the LLM API, providing insights into latency variability.  
    \item \textbf{Environment time cost distribution:} The distribution of response times for calls to the environment API, measured to evaluate internal system performance.  
\end{itemize}

\textbf{Evaluation Results.}  
The evaluation results are summarized in Table~\ref{tab:performance}, which demonstrates the system’s scalability as the number of agents increases and highlights the performance impact of distributed computing. Specifically, the table shows how performance metrics such as LLM call time and environment response time vary with different group configurations (8, 16, and 32).

Figure~\ref{fig:distribution_analysis} presents four distribution plots that illustrate key metrics in large-scale LLM interactions with 10k agents under varying group configurations. The first two plots, Figure~\ref{fig:input_tokens} and Figure~\ref{fig:output_tokens}, show the distributions of input and output tokens, respectively. These plots reveal that token usage patterns remain remarkably stable across different configurations, indicating that parallelization does not significantly alter the overall amount of data being processed. In contrast, Figure~\ref{fig:llm_api_response} shows the distribution of LLM API call times, revealing that the time required for API calls is more sensitive to the level of parallelization. Finally, Figure~\ref{fig:env_response} presents the environment time cost distribution, which illustrates how the environment’s responsiveness fluctuates with the number of groups.

\begin{table}[htbp]
    \centering
    \small
    \caption{Performance metrics for different configurations.}
    \label{tab:performance}
    \resizebox{\textwidth}{!}{
    \begin{tabular}{cccccccc}
        \toprule
        \multicolumn{5}{c}{\textbf{Parameters}} & \multicolumn{3}{c}{\textbf{Average Time Cost}} \\
        \midrule
        \textbf{\#Agents} & \textbf{\#Groups} & \textbf{LLM Calls} & \textbf{ITs (/call)} & \textbf{OTs (/call)} &  \textbf{All (s/round)} & \textbf{LLM (s/call)} & \textbf{Env (ms/call)}\\
        \midrule
        $10^3$ & 8  & 4803.0 & 430.04 & 79.17 & 82.45 & 4.51 & 12.26 \\
        $10^3$ & 16 & 3120.8 & 398.78 & 77.18 & 41.17 & 2.92 & 14.31 \\
        $10^3$ & 32 & 4790.4 & 412.82 & 75.56 & 43.30 & 2.94 & 9.55 \\
        \midrule
        $10^4$ & 8  & 54135.4 & 430.35 & 75.84 & 5681.18 & 52.54 & 33.55  \\
        $10^4$ & 16 & 54002.2 & 430.24 & 75.80 & 1422.48 & 3.53 & 33.55 \\
        $10^4$ & 32 & 54075.0 & 430.47 & 76.14 & 458.82 & 8.05  & 30.53 \\
        \bottomrule
    \end{tabular}
    }
\end{table}


\begin{figure}[ht]
    \centering
    \begin{subfigure}[t]{0.45\textwidth}
        \centering
        \includegraphics[width=\linewidth]{Figure/input_tokens_dist.png}
        \caption{Input Token Distribution}
        \label{fig:input_tokens}
    \end{subfigure}
    % \hfill
    \begin{subfigure}[t]{0.45\textwidth}
        \centering
        \includegraphics[width=\linewidth]{Figure/output_tokens_dist.png}
        \caption{Output Token Distribution}
        \label{fig:output_tokens}
    \end{subfigure}
    \begin{subfigure}[t]{0.45\textwidth}
        \centering
        \includegraphics[width=\linewidth]{Figure/api_time_dist.png}
        \caption{LLM Time Cost Distribution}
        \label{fig:llm_api_response}
    \end{subfigure}
    \begin{subfigure}[t]{0.45\textwidth}
        \centering
        \includegraphics[width=\linewidth]{Figure/system_time_dist.png}
        \caption{Environment Time Cost Distribution}
        \label{fig:env_response}
    \end{subfigure}
    \caption{Distribution analysis for 10k agents.}
    \label{fig:distribution_analysis}
\end{figure}


The Average Time Cost analysis provides deeper insights into the system’s performance, as summarized in Table~\ref{tab:performance}. The total time per round (All) decreases as the number of groups increases, demonstrating the positive impact of parallelization on processing efficiency. This trend reflects the effectiveness of the distributed parallel framework, which optimally utilizes multi-core computational power, minimizing the CPU bottleneck and enabling the system to handle larger agent scales efficiently. However, the LLM time remains the primary bottleneck in the system, even under fully parallel conditions. Despite the reduction in execution time with more groups, LLM API calls still represent a significant portion of the total execution time. This is due to the nature of the external API calls, where server-side load introduces variability and causes unpredictable performance fluctuations. As shown in the evaluation results, the environment time (Env) remains minimal, in the millisecond range, which indicates that the system is capable of supporting large-scale simulations with minimal impact from the environment processing.

The experimental findings also highlight that the execution efficiency of large-scale agents is primarily constrained by the LLM API calls. Under fully parallel conditions, this constraint becomes more pronounced, making LLM performance a critical factor in scaling agent-based simulations. To achieve more stable operation for larger-scale simulations (e.g., >10\textsuperscript{4} agents), researchers may consider deploying a private LLM inference service. While this approach could offer more reliable performance, it comes with substantial initial costs, including GPU deployment and model configuration selection. The token distribution data in this study could serve as a reference for estimating GPU resources and model configurations required for such a deployment.

In conclusion, the experiments demonstrate the simulation engine’s ability to efficiently handle large-scale agent execution. However, the findings also emphasize the need for careful consideration of LLM API performance and the trade-offs involved in private deployment options. To improve stability and scalability, further research should focus on optimizing the LLM infrastructure or exploring alternative solutions for large-scale intelligent agent simulations.


\section{Exemplary Social Experiments}\label{sec:social_experiment}

\subsection{One Day Life}\label{sec:one_day_life}

% 使用一个人一个典型日的例子,分别用不同的颜色分别展示心理(情绪、认知、需求),社交、移动、经济行为(Yuwei)
\definecolor{needColor}{RGB}{255,0,0}        % 红色,代表需求
\definecolor{cogColor}{RGB}{128,0,128}     % Purple for cognition
\definecolor{mobilityColor}{RGB}{255,165,0}  % Orange for action
\definecolor{socialColor}{RGB}{204,0,102}    % 紫红色,代表社交
\definecolor{economyColor}{RGB}{0,153,0}     % 绿色,代表经济
\definecolor{otherColor}{RGB}{128,128,128}  % 灰色,代表其他行为

This section presents a self-directed day in the life of a socially intelligent agent, illustrating how it navigates daily tasks while balancing internal needs, emotional states, and cognitive processes. Through a simulated 24-hour scenario, we examine how the agent's dynamic priorities influence its decisions across three domains: mobility (e.g., route planning with energy constraints), social interaction (e.g., adapting communication style to context), and economic behavior (e.g., resource allocation under uncertainty). This micro-level analysis serves to validate the coherence of its behavioral patterns and their alignment with human-like temporal rhythms. The one day life journey for a specific person is shown as Tab.\ref{tab:onedaylife}.

By examining this one-day life scenario, we can see how the agent’s \textcolor{needColor}{needs} drive the formation of a plan and lead to specific actions (\textcolor{mobilityColor}{mobility}, \textcolor{socialColor}{social}, \textcolor{economyColor}{economy}, \textcolor{otherColor}{other}), all of which are continuously shaped by the agent’s \textcolor{cogColor}{cognition}. Through this table, the agent demonstrates behaviors that reflect realistic decision-making processes across various domains—managing its hunger, social connections, work responsibilities, and leisure. Such a framework helps researchers evaluate the consistency and depth of the agent’s behavior, providing a solid basis for exploring more complex social interactions and collective dynamics in virtual environments. Besides, Tab~\ref{tab:daily_interaction} summarizes the number of interactions between the social agent and various environmental spaces during a typical day.

\begin{table}[htbp]
    \centering
    \caption{Daily environment interactions per agent.}
    \begin{tabular}{l l l}
        \toprule
        \textbf{Space} & \textbf{Interaction Type}     & \textbf{Counts} \\
        \midrule
        \multirow{2}{*}{Urban Space}  & Get         & 465.67 \\
                                      & Set         & 4.27   \\\hline
        \multirow{2}{*}{Economy Space}& Get         & 9.26   \\
                                      & Set         & 3.30    \\\hline
        Social Space                & SendMessage & 9.08   \\\hline
        \textbf{Sum} & \textbf{ALL} & \textbf{491.68} \\
        \bottomrule
    \end{tabular}
    \label{tab:daily_interaction}
\end{table}

Based on the Social Agent's capability to simulate a one-day life, we further conducted simulation experiments in the domains of cognition, social interaction, economics, and mobility. These experiments were designed to validate the Social Agent's proficiency in capturing behaviors across various domains, as illustrated in Fig.\ref{fig:experiment_overview}.

\begin{figure}[ht]
    \centering
    \includegraphics[width=1\linewidth]{Figure/experiment.pdf}
    \caption{Experiment configuration overview.}
    \label{fig:experiment_overview}
\end{figure}

\begin{table}[htbp]
\caption{One Day Life}
\label{tab:onedaylife}
\centering
\begin{tabular}{|p{7cm}|p{6cm}|}
\hline
\textbf{Actions} & \textbf{Mind} \\
\hline

(08:00–12:30)
\begin{itemize}
\item \textcolor{mobilityColor}{Commute to office (Mobility)}
\item \textcolor{economyColor}{Respond to priority emails (Economy)}
\item \textcolor{economyColor}{Attend project planning meeting (Economy)}
\item \textcolor{economyColor}{Coordinate cross-department tasks (Economy)}
\end{itemize}
&
\begin{itemize}
\item \textcolor{needColor}{Need: Safe}
\item \textcolor{cogColor}{Emotion: Resentment}
\item \textcolor{cogColor}{Cognition: "Sequential task execution ensures workflow integrity"}
\end{itemize}
\\
\hline

(12:30–13:30)
\begin{itemize}
\item \textcolor{mobilityColor}{Commute via grocery store (Mobility)}
\item \textcolor{economyColor}{Compare product prices (Economy)}
\item \textcolor{otherColor}{Prepare lunch (Other)}
\item \textcolor{otherColor}{Eat (Other)}
\end{itemize}
&
\begin{itemize}
\item \textcolor{needColor}{Need: Hungry}
\item \textcolor{cogColor}{Emotion: Disappointment}
\item \textcolor{cogColor}{Cognition: "Economic constraints necessitate adaptive consumption patterns"}
\end{itemize}
\\
\hline

(13:30-14:00)
\begin{itemize}
\item \textcolor{mobilityColor}{Browse social networking sites (Social)}
\item \textcolor{socialColor}{Find friend to contact with (Social)}
\item \textcolor{socialColor}{Send message to friend (Social)}
\end{itemize}
&
\begin{itemize}
\item \textcolor{needColor}{Need: Social}
\item \textcolor{cogColor}{Emotion: Gratification}
\item \textcolor{cogColor}{Cognition: "Social capital accumulation facilitates opportunity discovery"}
\end{itemize}
\\
\hline

(14:00-18:00)
\begin{itemize}
\item \textcolor{economyColor}{Develop quarterly budget (Economy)}
\item \textcolor{otherColor}{Mentor junior staff (Other)}
\item \textcolor{mobilityColor}{Inspect branch office locations (Mobility)}
\item \textcolor{economyColor}{Submit audit report (Economy)}
\end{itemize}
&
\begin{itemize}
\item \textcolor{needColor}{Need: Safe}
\item \textcolor{cogColor}{Emotion: Relief}
\item \textcolor{cogColor}{Cognition: "Multi-layered verification prevents operational risks"}
\end{itemize}
\\
\hline

(18:00–20:00)
\begin{itemize}
\item \textcolor{mobilityColor}{Go back home (Mobility)}
\item \textcolor{otherColor}{Check refrigerator (Other)}
\item \textcolor{otherColor}{Prepare dinner (Other)}
\item \textcolor{otherColor}{Eat dinner (Other)}
\end{itemize}
&
\begin{itemize}
\item \textcolor{needColor}{Need: Hungry}
\item \textcolor{cogColor}{Emotion: Gratification}
\item \textcolor{cogColor}{Cognition: "Having finished the day's work, I was pleased with myself"}
\end{itemize}
\\
\hline

(20:00–22:00)
\begin{itemize}
\item \textcolor{otherColor}{Browse webpages(Other)}
\item \textcolor{otherColor}{Play video games(Other)}
\end{itemize}
&
\begin{itemize}
\item \textcolor{needColor}{Need: Whatever}
\item \textcolor{cogColor}{Emotion: Relief}
\item \textcolor{cogColor}{Cognition: "Entertainment makes me feel relaxed"}
\end{itemize}
\\
\hline

(22:00–24:00)
\begin{itemize}
\item \textcolor{otherColor}{Complete bedtime routine (Other)}
\item \textcolor{otherColor}{Go to sleep (Other)}
\end{itemize}
&
\begin{itemize}
\item \textcolor{needColor}{Need: Tired}
\item \textcolor{cogColor}{Emotion: Satisfaction}
\item \textcolor{cogColor}{Cognition: "Resource allocation efficiency impacts systemic stability"}
\end{itemize}
\\
\hline
\end{tabular}
\end{table}

\begin{figure}[htbp]
  \centering
  \newlength{\subimgsize}
  \setlength{\subimgsize}{0.45\linewidth}

  \begin{subfigure}[b]{\subimgsize}
    \includegraphics[width=\subimgsize, height=\subimgsize]{Figure/front1.jpg}
    \caption{Large-scale Simulation}
    \label{fig:suba}
  \end{subfigure}
  \hfill
  \begin{subfigure}[b]{\subimgsize}
    \includegraphics[width=\subimgsize, height=\subimgsize]{Figure/front2.jpg}
    \caption{Self-driven Daily Life}
    \label{fig:subb}
  \end{subfigure}
  
  \caption{Large-scale social simulation.}
  \label{fig:frontend}
\end{figure}


\subsection{Polarization}\label{sec:polarization}

% 极化的实验结果




% 第一段,实验背景,为什么研究极化很重要
Polarization is a phenomenon where opinions within a population become increasingly divided, often forming distinct clusters that are difficult to reconcile. Understanding polarization is critical because it influences how societies debate, make decisions, and implement solutions to pressing challenges. By studying the factors that drive polarization, researchers can uncover why divisions deepen over time and how they can be addressed. This research provides valuable insights into fostering more cohesive societies, promoting constructive dialogue, and navigating complex issues in a way that incorporates diverse perspectives.


% 中间,实验设计,画一张图说明实验的步骤
To investigate the dynamics of polarization, an experimental setting is designed to simulate discussions among agents focused on a specific policy issue: gun control. In the control group, agents engage in discussions about the gun control issue, with opinions naturally divided between support and opposition. No external interventions are introduced in this setting, allowing opinions to evolve organically through agents' autonomous social interactions. Two treatment groups are introduced to study the effects of persuasive messages on opinion dynamics. In one treatment group, agents are only exposed to persuasive messages that align with their existing opinions, which we refer to as the homophilic interaction group. In the other treatment group, agents only receive persuasive messages with opposing opinions, which is the heterogeneous interaction group. This experimental setup provides a ground to analyze how different opinions contribute to the formation of polarization.

% 最后一段,实验结果,每个结果一个图

\begin{figure}
    \centering
    \includegraphics[width=1\linewidth]{Figure/polarization.pdf}
    \caption{Opinion changes on the political issue of Gun Control across three experimental setups.}
    \label{fig:polarization}
\end{figure}

Figure~\ref{fig:polarization} presents the opinion changes on the political issue of Gun Control across three experimental setups: (a) the control group, (b) the homophilic interaction group, and (c) the heterogeneous interaction group. In the control group, where agents engage in discussions without external interventions, 39\% of agents adopt more polarized opinions, while 33\% become more moderate after interactions. By contrast, in the homophilic interaction group, a clear polarization pattern emerges, with 52\% of agents becoming more polarized. This result suggests the effect of echo chambers, where excessive interactions with like-minded peers can potentially intensify opinion polarization. In the heterogeneous interaction group, 89\% of agents adopt more moderate opinions, and 11\% are persuaded to adopt opposing viewpoints. This indicates that exposure to opposing content and opinions could be an effective mitigation strategy for curbing polarization.


\subsection{Spread of Inflammatory Messages}\label{sec:infl_message}
Information propagation in social networks is a fundamental research problem in social computing. Social networks enable users to share various types of content such as news, personal status updates and public discussions. Among these information flows, inflammatory messages containing extreme opinions and inaccurate claims present significant challenges. These messages can quickly spread across social networks and increase conflicts in online discussions. Standard information diffusion models cannot fully explain how inflammatory messages propagate~\cite{romero2011differences,brady2017emotion}, because user sharing behaviors often deviate from typical patterns when encountering such content. Additionally, current content moderation systems on social platforms face difficulties in balancing effective content filtering with maintaining regular user communications. Simulation experiments offer a practical approach to analyze these propagation dynamics and test different intervention methods, providing insights that complement real-world social network studies.

To investigate the spread of inflammatory messages, we design experiments based on a real-world event, the case of the chained woman in Xuzhou~\cite{gao2023s}. Using a population of hundreds of agents, our experiments consist of four parts. In the control group, we place non-inflammatory seed messages at selected nodes and observed the natural progression of information spread and emotional evolution within the group. For the experimental group, we introduce emotionally charged, selectively expressed inflammatory messages at certain nodes to examine whether these would alter the trajectory of information spread and emotional dynamics. To simulate the suppression of inflammatory messages, we implemented two intervention strategies: node intervention and edge intervention. In both approaches, the social platform monitors messages sent by agents, using large language models to determine if content is inflammatory. Under node intervention, accounts that repeatedly share harmful inflammatory content above a certain threshold are suspended. With edge intervention, when inflammatory content is detected traveling between two nodes, the social connection between them is permanently removed. We track how these interventions affect both information propagation patterns and the evolution of group emotions. Finally, we conduct interviews with agents to understand their motivations for sharing messages, helping us uncover the underlying psychological and social factors that drive information-sharing behavior when encountering inflammatory content.
\begin{figure}[ht]
    \centering
    \begin{subfigure}[t]{0.48\textwidth}
        \centering
        \includegraphics[width=\linewidth]{Figure/information_spread.pdf}
        \caption{Information Spread over Time}
        \label{fig:information_spread}
    \end{subfigure}
    \hfill % 添加一些水平间距
    \begin{subfigure}[t]{0.48\textwidth}
        \centering
        \includegraphics[width=\linewidth]{Figure/emotional_intensity.pdf}
        \caption{Emotional Intensity over Time}
        \label{fig:emotional_intensity}
    \end{subfigure}
    \caption{Simulation results of the spread of inflammatory messages.}
    \label{fig:social_curve}
\end{figure}

Our experimental results are shown in Figure~\ref{fig:social_curve}. The experimental results demonstrate distinct patterns in information propagation dynamics and emotional responses across different intervention strategies. Our findings validate that inflammatory messages exhibit unique diffusion characteristics compared to regular content in social networks. The experimental group, where inflammatory messages are introduced, shows substantially higher information reach than the control group with non-inflammatory content, confirming that inflammatory messages possess stronger viral potential in social networks. This observation aligns with previous findings about the deviation of inflammatory content from standard diffusion patterns~\cite{romero2011differences,brady2017emotion}.

The intervention strategies demonstrate varying degrees of effectiveness in managing inflammatory content spread. Node-level intervention, which suspends accounts that frequently share inflammatory content, proves to be the more effective approach in containing information propagation. Edge-level intervention, while showing moderate containment effects, is less efficient than node-based approaches. This difference suggests that targeting individual spreading behaviors might be more effective than modifying network structure for content moderation.

The emotional intensity measurements provide additional insights into the intervention dynamics. The experimental group exhibits markedly elevated emotional responses compared to the control group, indicating that inflammatory messages significantly amplify emotional engagement within the network. Node intervention demonstrates superior effectiveness in moderating these emotional responses, achieving substantial reduction in overall emotional intensity. Edge intervention, though less effective than node-based approaches, still shows notable moderation effects on emotional dynamics. 
\begin{figure}[ht]
    \centering
    \includegraphics[width=0.7\linewidth]{Figure/wordcloud_social.pdf}
    \caption{Agent opinions on the chained woman incident.}
    \label{fig:social_opinion}
\end{figure}

Interview analysis reveals key factors that drive inflammatory message sharing behavior, as shown in the word cloud in Figure~\ref{fig:social_opinion}. The responses mainly focus on emotional reactions and social responsibility. Analysis shows that strong emotions, especially sympathy and worry, often trigger sharing behaviors. Many agents share information because they feel they have a duty to let others know about important social issues. The interviews show that agents think about the broader social impact when sharing information, seeing it as a way to join public discussions. Agents also show clear goals in their sharing behavior, mainly wanting to increase public attention and get responses from institutions. These findings suggest that inflammatory message spread is driven by both emotional factors and social awareness. Understanding why agents share such messages helps us develop better content moderation strategies in social networks.

These experimental results demonstrate three key findings in inflammatory content management. First, inflammatory messages show stronger viral potential and trigger higher emotional responses compared to regular content. Second, node-level intervention is more effective than edge-level intervention in both containing information spread and moderating emotional intensity. Third, through agent interviews, we find that emotional factors and social responsibility drive sharing behaviors. These findings provide empirical evidence for designing content moderation systems, suggesting that user-level interventions combined with consideration of emotional and social factors may lead to more effective control of inflammatory content in social networks.
% 煽动性信息的实验结果

% 第一段,实验背景,为什么信息传播很重要
% 中间,实验设计,说明实验的步骤
% 最后一段,实验结果,每个结果一个图

\subsection{Universal Basic Income}\label{sec:ubi}

% UBI实验结果

% 第一段,实验背景,为什么ubi很重要
% 中间,实验设计,画一张图说明实验的步骤
% 最后一段,实验结果,每个结果一个图

Universal Basic Income (UBI) has always been a highly controversial macroeconomic policy. The implementation cost of UBI is enormous, and the outcomes of UBI policies around the world have shown inconsistent effects on both the participants and economic development. Therefore, accurately understanding the impact of UBI on the socio-economic environment and its underlying reasons is crucial in determining whether UBI policies should be implemented in the real world to alleviate poverty. Based on our simulation platform, we conduct intervention experiments on UBI and explore its effects on both agents and the macroeconomics.

We conduct two macroeconomic simulations based on the demographic distribution of residents in cities that have implemented UBI policies (Texas, USA). One simulation is without the UBI policy, while the other incorporates UBI intervention, where each agent is given a monthly unconditional payment of \$1,000. By comparing the economic and social metrics generated from both simulations, we explore the impact of the UBI policy and assess whether these influence align with the outcomes observed in Texas' UBI social experiment.

The basic simulation results are shown in the Figure \ref{fig:econ_curve}, including the simulated curves of real GDP and agent consumption levels. As can be seen, as the simulation progresses, the fluctuations in the curves become smaller, indicating that the economic system is stabilizing.

\begin{figure}[ht]
    \centering
    \begin{subfigure}[t]{0.47\textwidth}
        \centering
        \includegraphics[width=\linewidth]{Figure/real_gdp_curve.pdf}
        \caption{Real GDP}
        \label{fig:gdp}
    \end{subfigure}
    \hfill % 添加一些水平间距
    \begin{subfigure}[t]{0.50\textwidth}
        \centering
        \includegraphics[width=\linewidth]{Figure/consumption_curve.pdf}
        \caption{Consumption Level}
        \label{fig:consumption}
    \end{subfigure}
    \caption{Simulation results of the economic system.}
    \label{fig:econ_curve}
\end{figure}

We introduce the UBI policy at step 96 of the simulation and compare the economic and social metrics of the two simulation results over the next 24 steps in Figure \ref{fig:econ_bar}, namely agent consumption levels and depression levels, with depression levels assessed through surveys using the widely recognized Center for Epidemiologic Studies Depression Scale (CES-D)~\cite{radloff1991use}. The comparison shows that the UBI policy increases consumption levels and reduces depression levels, which is similar to the impact observed in Texas' UBI policy~\cite{bartik2024impact}, thus validating the realism of the simulation.

\begin{figure}[ht]
    \centering
    \begin{subfigure}[t]{0.48\textwidth}
        \centering
        \includegraphics[width=\linewidth]{Figure/consumption.pdf}
        \caption{Consumption Level}
        \label{fig:gdp}
    \end{subfigure}
    \hfill % 添加一些水平间距
    \begin{subfigure}[t]{0.48\textwidth}
        \centering
        \includegraphics[width=\linewidth]{Figure/depression.pdf}
        \caption{Depression Level}
        \label{fig:depression}
    \end{subfigure}
    \caption{The comparison of economic and social metrics.}
    \label{fig:econ_bar}
\end{figure}

We also interview agents about their views on the UBI policy, which are summarized in the word cloud in Figure \ref{fig:ubi_opinion}. The results show that the impact of the UBI policy is mainly related to key terms such as interest rates, long-term benefits, savings, and necessities of life, reflecting the common perceptions of UBI policy in the real world.

\begin{figure}[ht]
    \centering
    \includegraphics[width=0.7\linewidth]{Figure/opinions.pdf}
    \caption{Agent opinions on UBI policy.}
    \label{fig:ubi_opinion}
\end{figure}

\subsection{External Shocks of Hurricane}\label{sec:hurricane}

% 外部灾害的实验结果
% 第一段,实验背景,为什么外部灾害很重要
% 中间,实验设计,画一张图说明实验的步骤
% 最后一段,实验结果,每个结果一个图

The impact of external disasters on human mobility is a critical area of study due to their profound effects on societal structures and individual behaviors. Understanding how such events influence human movement patterns is essential for enhancing emergency response strategies and mitigating potential risks.
Hurricanes, as severe natural disasters, pose significant threats to human life and property. The destruction of infrastructure, displacement of populations, and disruption of daily activities necessitate a comprehensive understanding of human mobility during such events. 

The experiment focuses on Hurricane Dorian, which impacted the southeastern United States in 2019. The city of Columbia, South Carolina, serves as the primary case study due to its significant population density and the availability of detailed mobility data.
The analysis utilizes two primary data sources:

\begin{itemize}
    \item \textbf{SafeGraph Data\footnote{\url{https://www.safegraph.com/}}:} Provides comprehensive information on points of interest (POIs) and human mobility patterns (from 2019.8.28 - 2019.9.5).
    \item \textbf{Census Block Group (CBG) Data\footnote{\url{https://docs.safegraph.com/docs/open-census-data}}:} Offers demographic profiles of residents, facilitating the sampling of city residents' profiles (including gender, age, race, income, home cbg, etc.).
\end{itemize}

These datasets are integrated to model and analyze the movement behaviors of social agents during the hurricane event.

Specifically, the experiment involves 1,000 social agents, and incorporates real-time weather updates to influence agent behaviors, thereby reflecting the dynamic nature of human responses to the hurricane. We evaluate mobility patterns through two metrics:  
1) \textbf{Activity Level} ($\frac{\text{Traveling Individuals}}{\text{Area Population}}$), visualized through three phase-specific maps. The results are shown as Fig.\ref{fig:activity_phases}.
2) \textbf{Total Daily Trips} (9-day normalized time-series). The result is shown as Fig.\ref{fig:trip_ts}.

According to Fig. \ref{fig:activity_phases}, the hurricane significantly impacts the mobility behavior of the social agent. Before the hurricane, the average activity level (defined as the ratio of travelers to the total population) across the CBGs remained between 70\% and 90\%. However, when the hurricane arrived, the activity level sharply decreased to approximately 30\%, indicating a significant reduction in mobility behavior. After the hurricane passed, the activity level gradually returned to normal levels. This analysis suggests that the social agent could adapt its mobility demand effectively based on environmental information, mimicking human behavior in response to extreme weather events.

% Activity Level Subplots
\begin{figure}[htbp]
    \centering
    \newlength{\activitysize}
      \setlength{\activitysize}{0.3\linewidth}
    
      \begin{subfigure}[b]{\activitysize}
        \includegraphics[width=\activitysize]{Figure/activity_stage_1.png}
        \caption{Before landfall (8.28-8.30)}
        \label{fig:suba}
      \end{subfigure}
      \hfill
      \begin{subfigure}[b]{\activitysize}
        \includegraphics[width=\activitysize]{Figure/activity_stage_2.png}
        \caption{Landfall (8.31-9.2)}
        \label{fig:subb}
      \end{subfigure}
      \hfill
      \begin{subfigure}[b]{\activitysize}
        \includegraphics[width=\activitysize]{Figure/activity_stage_3.png}
        \caption{After landfall (9.3-9.5)}
        \label{fig:subb}
      \end{subfigure}
    \caption{Activity level spatial distributions.}
    \label{fig:activity_phases}
\end{figure}

The line graph presented above (Fig. \ref{fig:trip_ts}) compares the daily outflow patterns of the real data with the simulated visits over the course of the experiment. Both the real and simulated data exhibit similar trends, with a noticeable decline in visit activity around August 30th, corresponding to the onset of the hurricane impact, followed by a significant recovery in early September. Notably, while the simulated visits closely follow the general trend of the real data, slight deviations are observed, particularly during the hurricane's peak. This suggests that the social agent's behavior, while generally aligned with actual human patterns, may exhibit some discrepancies in terms of the magnitude and speed of response. However, the overall similarity in the temporal progression of visits indicates that the simulation captures key aspects of human mobility under the influence of extreme weather events, validating the social agent's effectiveness in approximating real-world behavior.

% Trip Volume Time-Series
\begin{figure}[htbp]
    \centering
    \includegraphics[width=0.8\textwidth]{Figure/date_out.png}
    \caption{Normalized daily trips.}
    \label{fig:trip_ts}
\end{figure}

The results effectively demonstrate that the constructed social agents, within the framework of the social simulator, can accurately replicate human mobility behaviors and group characteristics during a hurricane event. This validation underscores the simulator's potential as a tool for analyzing and understanding human responses to external shocks, thereby contributing to improved disaster preparedness and response strategies.

%
\section*{Acknowledgements} Aamand, Pagh, and Imola carried out this work at Basic Algorithms Research Copenhagen (BARC), which was supported by the VILLUM Foundation grant 54451. Pagh and Imola were supported by a Data Science Distinguished Investigator grant from the Novo Nordisk Fonden. Boninsegna was supported in part by the Big-Mobility project by the University of Padova under the Uni-Impresa call, by the MUR PRIN 20174LF3T8 AHeAD project, and by MUR PNRR CN00000013 National Center for HPC, Big Data and Quantum Computing.




\bibliography{bibliography}
\bibliographystyle{icml2025}


%
%
%
%
%
\newpage
\appendix
\onecolumn

%
\section{Additional Definitions}\label{app:add-def}
\begin{lemma}[Binary Randomized Response~\cite{warner1965randomized,dwork2006calibrating}]
\label{def: binary rr}
    For a binary input $x\in\{0,1\}$, and privacy parameter $\priv$, the following protocol $\mathcal{M}\to \{0,1\}$ satisfies $\priv$-LDP:
    \begin{equation*}
        \pmech(x)=\begin{cases}
        x,&\text{w.p. }\frac{e^\priv}{e^\priv +1}\\
        1-x,&\text{otherwise.}
    \end{cases}
    \end{equation*}
%
    %
        %
   %
\end{lemma}
\textbf{Azuma's inequality.}
We will use the following version of Azuma's inequality which bounds the maximum deviation of a martingale $(X_i)_{i=0}^n$ at any time $t=0,\dots, n$. See Theorem 2.1 in \cite{Fan2012martingales} for a stronger and more general bound.
\begin{theorem}[Azuma's inequality]\label{thm:azuma} Let $(X_i)_{i=0}^n$ be a martingale such that $X_0=0$ and $|X_{i+1}-X_i|\leq 1$ for all $0\leq i<n$. For any $t\geq 0$, 
\[
\Pr[\max_{1\leq i \leq n}|X_i|\geq t]\leq 2\exp\left(\tfrac{-t^2}{2n} \right).
\]
\end{theorem}

\textbf{Bernstein's Inequality.} We use the following variant of Bernstein's Inequality in the proof of~\cref{thm:main-shuffle}, see~\citet[Proposition 2.10]{Wainwright_2019} for a detailed overview.
\begin{theorem}[Bernstein's Inequality]\label{fact: bernstein}
    Let $\{X_i\}_{i=1}^n$ be independent random variables that are bounded almost surely by $1$. Let $\sigma^2=\frac{1}{n}\sum_{i=1}^n\operatorname{Var}[X_i]$ be the average variance. We then have,
    \[
    \Pr\left[\bigg|\frac{1}{n}\sum\limits_{i=1}^n X_i-\frac{1}{n}\sum\limits_{i=1}^n\bEE{X_i}\bigg|>\acc\right]\leq\exp\left( \frac{-n\acc^2}{2\sigma^2 + \frac{2\acc}{3}} \right).
    \]
\end{theorem}
\section{Reduction to the Median} 

\label{appendix: Reduction to the Median}
Consider the simple case where we are given an algorithm $A$ which returns the median of $n$ samples in the most natural sense, by returning the $n/2$'th index of their sorted representation. Without changing this algorithm we can have it return any arbitrary index by adding elements to the beginning or the end of this sorted array. For example, adding two elements to the beginning of the array will create a new array with $n'=n+2$ elements where the $n'/2$'th index will be the $(n/2-1)$'th index of the original array. The padding argument below formalizes this notion, demonstrating that any algorithm for an $\alpha$-approximation of the median can be used to obtain a $2\alpha$-approximation of any quantile.
\begin{lemma}[Padding Argument] 
\label{appendix: padding argument}
Any $\alpha$-approximation algorithm for the median, with $\alpha \in \left(0,\frac{1}{2}\right)$, can be used to construct a $2\alpha$-approximation for any quantile $\tau\in (0,1)$. 
\end{lemma}
\begin{proof}
    Consider a dataset $D=\{x_1, \dots, x_n\}$ where each element is such that $x_i \in \{1,\dots,\ab\}$. Let $\mathcal{M}$ be an algorithm for the $\alpha$-approximation of the median then for $m = A(D)$ we have by definition
     \begin{equation}
     \label{eq: appendix padded 1}
        \text{Pr}_{\mathcal{D}}[x\leq m]<\frac{1}{2}+\alpha \qquad \text{and} \qquad \text{Pr}_{\mathcal{D}}[x\leq m+1]>\frac{1}{2}-\alpha.
    \end{equation}
    where $\text{Pr}_{D}[x\leq m] = \frac{\sum_{x\in D}[x\leq m]}{n}$, and $[x\leq m]$ is an indicator function. 
    Consider now a padded dataset $D_P = D\cup \{1\}^{(1-\tau)n} \cup \{\ab\}^{\tau n}$, where $\{a\}^{x}$ indicates the multi-set containing the $a$ element $x$ times \footnote{We consider $(1-\tau)n$ and $\tau n$ integers.}. The new empirical cumulative distribution of the data set for $y \in \{1, \dots, \ab-1\}$, is \begin{align*}
    \label{eq: appendix padded 2}
        \text{Pr}_{D_P}[x\leq y] &= \frac{(1-\tau)n +\sum_{x\in D}[x\leq y]}{|D_P|} = \frac{1-\tau}{2}+\frac{1}{2}\text{Pr}_{D}[x\leq y],
    \end{align*}
    as we have $|D_P| = 2n$. Thus 
    \begin{equation}
    \label{eq: appendix padded 3}
        \text{Pr}_{D}[x\leq y] = 2\text{Pr}_{D_P}[x\leq y] +\tau -1.
    \end{equation}
    The application of $A$ to the padded data set $D_{P}$ returns a $\alpha$-approximate median $m_P = A(D_P)$. Therefore, for $m_P\in\{1,\dots, \ab-1\}$, from \autoref{eq: appendix padded 3} and \autoref{eq: appendix padded 1} it follows that 
    \begin{equation}
    \label{eq: appendix padded 4}
       \text{Pr}_{D}[x\leq m_P]<\tau+2\alpha \qquad \text{and} \qquad \text{Pr}_{D}[x\leq m_P+1]>\tau-2\alpha.
    \end{equation}
    Notice that $m_p\neq \ab$, as $\text{Pr}_{D_P}[x\leq \ab]=1<\frac{1}{2}+\alpha$ iff $\alpha>\frac{1}{2}$. This concludes the proof.
\end{proof}

%
%
%
%
%
%
%

\section{Statistical Private Median Estimation}\label{sec:statistical-median}
In this section, we will provide an algorithm for \texttt{LDPstat-median} using the state-of-the-art algorithm for \texttt{MonotonicNBS}. We prove the following:

\begin{theorem}\label{thm:main-stat}
Let $\alpha \in \left(0,\frac{1}{4}\right)$ and $\varepsilon
>0$. Suppose that the number of users $n\geq C\frac{\log B}{\alpha^2}\left(\frac{e^\varepsilon+1}{e^\varepsilon-1}\right)^2$ for a sufficiently large constant $C$. Then there exists an algorithm solving \texttt{LDPstat-median}$(\mathcal{D},n,\alpha,\eps)$ with high probability in $B$. 
\end{theorem}

In this section, we prove Theorem~\ref{thm:main-stat}. For this, we recall the following result which is a corollary of the main result in~\cite{gretta2023sharp}. 
Recall the definition of an $\left(\frac{1}{2}, \alpha\right)$-good coin in~\eqref{eq:good-coin}.
\begin{theorem}[\cite{gretta2023sharp}]\label{thm:from-GP}
For any $\alpha \in \left(0,\frac{1}{4}\right)$, there exists an algorithm for \texttt{MonotonicNBS}$(\tau,\alpha)$ which uses $O(\frac{\log B}{\alpha^2})$ coin flips and outputs an $\left(\frac{1}{2},\alpha\right)$-good coin with high probability in $B$.
\end{theorem}
\begin{proof}[Proof of Theorem~\ref{thm:main-stat}]
For $i\in [B]$, we define $q_i=\sum_{j\leq i}\mathcal{D}[j]$ with the convention that $q_0=0$. 
Thus $j\mapsto q_j$ is the CDF of $\mathcal{D}$. Consider sampling $X\sim \mathcal{D}$ and let $Y$ be the random variable obtained by applying randomized response to the indicator variable $[X\leq j]$ retaining the bit with probability $\frac{e^\eps}{1+e^\eps}$ and flipping it otherwise. Then $\Pr[Y=1]=p_j$ where $p_j=q_j\cdot \frac{e^\eps}{1+e^\eps}+(1-q_j)\cdot \frac{1}{1+e^\eps}$. Then,
\begin{equation}\label{eq:use-GP}
    q_{j} = \left(p_{j}-\frac{1}{e^\varepsilon+1}\right)\frac{e^\varepsilon+1}{e^\varepsilon-1},
\end{equation}
We use the the algorithm in Theorem~\ref{thm:from-GP} to solve \texttt{MonotonicNBS}$\left(\frac{1}{2},\alpha\frac{e^\varepsilon-1}{e^\varepsilon+1}\right)$ when the inputs are the unknown $\{p_i\}_{i=1}^B$. To do so, whenever the algorithm calls for flipping a coin $j$, we sample a new user $X\in \mathcal{D}$ and apply randomized response to the variable $Y=[X\leq j]$. By standard properties of randomized responze, this protocol satisfies the $\eps$-LDP requirement. Moreover, by Theorem~\ref{thm:from-GP}, the algorithm finds an $\left(\frac{1}{2}, \alpha\frac{e^\varepsilon-1}{e^\varepsilon+1}\right)$-good coin $j^*$ with high probability in $B$. In particular $p_{j^*}\leq \frac{1}{2}+\alpha\frac{e^\varepsilon-1}{e^\varepsilon+1}$ and $p_{j^*+1}\geq \frac{1}{2}-\alpha\frac{e^\varepsilon-1}{e^\varepsilon+1}$. 
It thus follows from Equation~\eqref{eq:use-GP} that $q_{j^*}\leq 1/2+\alpha$ and $q_{j^*+1}\geq 1/2-\alpha$. Therefore $j^*$ is an $\alpha$-approximate median of $\mathcal{D}$ completing the proof.
\end{proof}
In the high privacy regime, i.e. for $\varepsilon<1$ , the sample complexity of Theorem \ref{thm:main-stat} becomes $n=\Omega\left(\frac{\log B}{\varepsilon
^2\alpha^2}\right)$, matching our lower bound up to a constant factor.
\section{The Hierarchical Mechanism}\label{app:hierarchical-mech}
The algorithm was presented in \cite{kulkarni2019answering} and can be used to approximately answer general range queries. It comes in several variants but we will present the simplest version (the bounds on the number of users needed for the various versions are similar). The main idea is to construct a $b$-ary tree of depth $\Theta(\log(B))$ on $[B]$. For the below, we will assume that $B$ is a power of $2$ and that $b=2$ (although for the experiments, we use a different constant $b$). The nodes on level $i$ (where level 0 is the root) corresponds to the $2^i$ dyadic intervals of $B$. Namely, in the binary representation of elements of $B$, there is an interval corresponding to each prefix of length $i$ in the binary representation. The non-adaptive protocol we will consider is as follows. Each user $i$ with data $x_i\in[B]$ picks a random level $\ell$ of the binary tree. The user writes a one-hot encoding $z$ of which node they belong to on level $\ell$ and uses randomized response on each of the $2^\ell$ bits of $z$. This is the message $y$, they send to the central server. This is the unary encoding mechanism; see~\cite{kulkarni2019answering} for more sophisticated solutions, that require less communication but nonetheless have the same approximation errors. The combined algorithm is denoted \texttt{Hierarchical Mechanisms}. 

\paragraph{Analysis sketch of \texttt{Hierarchical Mechanism}}
We here analyse the performance of \texttt{Hierarchical Mechanism} for answering general range queries and in particular show how it can be used for quantile estimation.

Assume that $\eps\leq 1$. If the number of users reporting at every level is $\gg \frac{1}{\alpha_0^2\eps^2}$ (where $a\gg b$ means that $a\geq C b$ for some constant $C$), then using standard concentration bounds, for each node in a given level, we can recover the total fraction of users lying in the corresponding subtree up to an additive $\alpha_0$ with constant failure probability. Now if the total number of users is $\gg \frac{\log B}{(\alpha_0^2\eps^2)}$, then with constant failure probability, the number of users reporting at any given level is indeed, $\gg \frac{1}{\alpha_0^2\eps^2}$. We now pick $\alpha_0=\alpha/(2\log B)$ and conclude that if the number of users is $\gg \frac{(\log B)^3}{(\alpha\eps^2)}$, we can recover the total fraction of users lying in any subtree up to an additive $\alpha/(2\log B)$ from the unary responses with constant failure probability. 
It follows that we can answer any range query with additive error $\alpha n$. Indeed, any range can be partitioned into at most $2\log B$ of these subtrees, two for each level. In particular, this means that we can find an $\alpha$-approximate median with constant failure probability.
It follows that we can answer any range query with additive error $\alpha n$. Indeed, any range can be partitioned into at most $2\log B$ of these subtrees, two for each level. In particular, this means that we can find an $\alpha$-approximate median with constant failure probability.
The analysis for high probability in $B$ needs $\gg \frac{\log B}{\alpha_0^2\varepsilon^2}$ number of users reporting at each level, so it adds an additional $\log B$ factor to the sample complexity.


%

\input{Appendix/martingale}
\section{Proof of Theorem \ref{thm:NBS-changing-probabilities}}
\label{app: proof theorem 3.1} 
\begin{algorithm}[t]
\caption{\texttt{BayeSS} main steps }\label{alg: bayeSS}
\begin{algorithmic}
\STATE {\bfseries Input:} $\{x_i\}_{i=1,\dots, n}$, $\alpha \in (0,1/4)$, $n\geq C\frac{\log B}{\alpha^2}$
\STATE $L\gets \texttt{BayesLearn}(B, \{x_i\}_{i=1,\dots,n/4}, \alpha)$
\STATE $R \gets \frac{1}{\gamma}$-$\text{quantiles}(L)$ \COMMENT{for $\gamma = O(1)$}
\STATE \textbf{return } \texttt{TestCoins}$(R, \{x_i\}_{n/4+1,\dots, n/2}, \alpha)$
\end{algorithmic}
\end{algorithm} 
The goal of this section is to prove Theorem~\ref{thm:NBS-changing-probabilities}. We first define the adversarial setting.
\begin{definition}
\label{def:adversarial}
Let $0<\alpha<1$ and $\ab$ a positive integer. Let $p_0,\dots,p_\ab\in [0,1]$ be unknowns with $0=p_0\leq\cdots \leq p_\ab= 1$. In \emph{\texttt{AdvMonotonicNBS}$(\tau, \alpha, c)$}, for $c>0$, our goal is to identify an $(\tau,\alpha(1+c))$-good coin (defined in \autoref{eq:good-coin}).
To do so, we may iteratively pick indices $i\in \ab$. Then an adversary selects a probability $\tilde p_i$ such that $| {\tilde{p}}_i - p_i|\leq c\alpha$, and we observe the outcome of a coin flip with heads probability $\tilde p_i$.
\end{definition}
We show that the \texttt{BayeSS} algorithm (\texttt{BayeSS} abbreviates \emph{Bayesian Screening Search}) from \cite{gretta2023sharp}(Algorithm 3) solves the \texttt{AdvMonotonicNBS}$(\tau, \alpha, c)$ problem returning the a $(\tau,\alpha(1+c))$-good coin with high probability in $\ab$ using
%
$O(\frac{\tau(1-\tau)\log \ab}{\alpha^2})$ 
 coin flips. We actually prove a stronger theorem which immediately implies~\cref{thm:NBS-changing-probabilities}.

\begin{theorem}\label{thm:GP-generalization}
Suppose that $c\leq 1$ and $\alpha \leq \frac{1}{2}\min\{\tau, 1-\tau\}$. There exists an algorithm~\cite{gretta2023sharp} for \emph{\texttt{AdvMonotonicNBS}$(\tau, \alpha, c)$} which uses 
%
$\tfrac{1}{C_{\tau, \alpha}}(\log \ab + O(\log^{2/3}\ab\,\log^{1/3}\frac{1}{\failp}+\log\frac{1}{\failp}))$ 
coin flips\footnote{Namely, $C_{\tau, \alpha}$ is the information capacity of the Binary Asymmetric Channel (BAC) with crossover probabilities $\{\tau + \alpha, \tau - \alpha\}$. Concretely, $C_{\tau, \alpha}=\max_q H((1-q)(\tau-\alpha) + q(\tau+\alpha))-(1-q)H(\tau-\alpha)-qH(\tau+\alpha)$ with $H$ being the binary entropy function, and $C_{\tau,\alpha} = \Theta(\tfrac{\alpha^2}{\tau(1-\tau)})$ for $\alpha \leq \frac{1}{2}\min(\tau, 1-\tau)$.}
%
and returns a $(\tau,\alpha(1+c))$-good coin with probability at least $1-\failp$.
\end{theorem}

Note that Theorem~\ref{thm:NBS-changing-probabilities} follows directly from Theorem~\ref{thm:GP-generalization} by setting $\tau=1/2$ and $\failp=\ab^{-\lambda}$ for any constant $\lambda$. With this, the proof of Theorem~\ref{thm:main-emp} is complete.

 Before we delve into the proof of Theorem~\ref{thm:GP-generalization}, let us first describe the idea behind \texttt{BayeSS}, described shortly in Algorithm \ref{alg: bayeSS}.
 %
 %
 At a high level \texttt{BayeSS} proceeds in two steps allocating a portion of the coin flips for each step. The first step is a Bayes learner algorithm, called  \texttt{BayesLearn}.
 %
 It starts by assigning a uniform prior $w(I_i)$ to each coin interval $I_i=[i, i+1]$ for any $i \in [\ab-1]$, then takes the $\tau$-quantile interval under the posterior $w(I_i)$, selects a coin from this interval, flips it, and updates each $w(I_i)$ according to the result of the coin flip and the error $\alpha$. This procedure is repeated iteratively.
 The sampled intervals are collected in a multiset $L$, with the guarantee that, after \( O\big(\tfrac{(1+\gamma)\log B}{C_{\tau, \alpha}}\big) \) coin flips, a $\gamma$-fraction of intervals in  $L$  contains a $(\tau, \alpha)$-good coin with high probability in  $\ab$  (referred to as good intervals). In the second step, this property is used to narrow the set of possible coins to $O(1/\gamma)$, ensuring that it contains at least one $(\tau, \alpha)$-good coin. Each coin in the candidate set can be individually tested, up to error $\alpha$, with high probability using $O(\tfrac{1}{\gamma\alpha^2}\log(\tfrac{\ab}{\gamma}))$ coin flips.

It is easy to see that in the adversarial setting, the coins can be tested up to error $\alpha(1+c)$ in the second step. 
%
Our main challenge in proving Theorem~\ref{thm:GP-generalization}, is analyzing the first part of the algorithm, \texttt{BayesLearn}, in the adversarial setting. 
%
%
%
%
%
The authors in \cite{gretta2023sharp} used a stopping time argument to analyze \texttt{BayesLearn}. They defined a potential function $\Phi$, with an initial negative value, constructed so that a positive potential implies finding at least a $\gamma$ fraction of good intervals. The stochastic process describing the evolution of the potential $\{\Phi_{i}\}_{i=1,\dots}$ is then modeled with a submartingale that can be used to bound, using Azuma's inequality, the probability that the process crosses zero after a sufficient number of iterations. We prove that we can use the same argument for the case of adversarial probabilities if we allow the potential to catch approximate good intervals, namely intervals containing $(\tau, \alpha(1+c))$-good coin.

\paragraph{New potential} Let $\{\ell,\dots,r\}$ be the set of $(\tau,\alpha(1+c))$-good intervals. Let $a$ be the maximum $i \in [\ab-1]$ such that $p^1_i\leq \tau$. Let $L$ be the list of intervals visited in \texttt{BayesLearn}. We define the potential function as 
\begin{equation*}
\label{eq: new potential}
    \Phi(w, L) := \log_2 w(a) + 12 C_{\tau, \alpha}(|\{x\in L : x \in [\ell,r]\}|-\gamma|L|),
\end{equation*}
where $w(a)$ is the Bayesian posterior weight associated to the best interval $a$ and $C_{\tau, \alpha}$ is a concrete function of $\tau$ and $\alpha$.
Notice that a positive potential implies $|\{x\in L | x \in [\ell,r]\}| >\gamma |L|$, hence indicating the presence of a $\gamma$ fraction $(\tau, \alpha(1+c))$-good intervals in $L$. The following Lemma generalises Lemma 7 of \cite{gretta2023sharp} and allows the construction of a submartingale.
\begin{algorithm*}[t]
\caption{\texttt{BayesLearn} for empirical quantile estimation, from Algorithm 2 in \cite{gretta2023sharp}}\label{alg: BayesLearn}
\begin{algorithmic}[1]
\FUNCTION{\texttt{GetIntervalFromQuantile}$(w, q)$}{}
    \STATE $\textbf{return\, } \min i \in [B] \text{ s.t. } W(i)\geq q$ \textbf{ with } $W(x)=\sum_{i\in\{1, \dots, x\}}w(i)$
\ENDFUNCTION\\
\hspace{0.5 cm}
\FUNCTION{\texttt{RoundIntervalToCoin}$(i, w, q)$}{}
    \STATE \textbf{return } $i$ \textbf{ if } $\frac{q-W(i-1)}{w(i)}\leq q$ \textbf{ else } $i+1$ \textbf{ with } $W(x)=\sum_{i\in\{1, \dots, x\}}w(i)$
\ENDFUNCTION\\
\hspace{0.5 cm}
\FUNCTION{\texttt{BayesLearn}$(\{x_{i}\}_{i=1,\dots, n}, B, \tau, \alpha, M)$}{}
\STATE $w_1 \gets \text{uniform}([B-1])$
\STATE $q \gets \arg \max_{x}H((1-x)(\tau -\alpha)+x(\tau +\varepsilon))-(1-x)H(\tau -\alpha)-xH(\tau + \alpha)$
\STATE $I \gets \{\}$ \COMMENT{Multiset}
\FOR {$i \in [M]$}
    \STATE $j_i \gets \texttt{GetIntervalFromQuantile}(w_i, q)$
    \STATE $c_i \gets \texttt{RoundIntervalToCoin}(j_i, w_i, q)$ \COMMENT{Gets the coin from the selected interval}
    \STATE $L\gets L \cup \{j_i\}$
    \STATE $x_i \sim \{x_k\}_{k=1,\dots}$ \COMMENT{Sample a user}
    \STATE $\{x_k\}_{k=1,\dots}\gets \{x_k\}_{k=1,\dots} \setminus \{x_i\}$ \COMMENT{Remove the user from the dataset}
    \STATE $y_i \gets [x_i \leq c_i]$ \COMMENT{Flip the coin}
    \STATE $w_{i+1}(x)\gets \begin{cases}
        w_i(x)d_{\tilde{y}_i,0} & \text{if } x\in \{1, \dots, j_i-1\}\\
        d_{\tilde{y}_i,0}(q-W_i(j_i-1))+d_{\tilde{y}_i, 1}(W_{i}(j_i)-1) & \text{if } x= j_i\\
        w_{i}(x)d_{\tilde{y}_i, 1} & \text{if } x\in \{j_i +1 , \dots, B-1\}
    \end{cases}$
\ENDFOR
\STATE \textbf{return} $L$ \COMMENT{Return a multiset of intervals}
\ENDFUNCTION
\end{algorithmic}
\end{algorithm*}
\begin{lemma}[Adaptation of Lemma 7 in \cite{gretta2023sharp} for adversarial probabilities]
    \label{lemma: increase in expectation of the potential}
    For $c\leq 1$ and $\alpha \leq \frac{1}{2}\min\{\tau, 1-\tau\}$, the expected variation of the potential is 
    \begin{equation}
        \E[\Phi_{t+1}-\Phi_{t}|y_{1}, \dots, y_t] \geq (1-12\gamma)C_{\tau, \alpha},
    \end{equation}
    where $(y_1, \dots, y_t)$ are the results of the coin toss up to $t+1$-th sample, and $C_{\tau, \alpha} = \Theta\left(\frac{\tau(1-\tau)}{\alpha^2}\right)$.
\end{lemma}
\begin{proof}
The proof for the adversarial setting, which allows an adversary to alter the head coin probability at each iteration up to $c\alpha$, while preserving their order, closely resembles the proof of Lemma 7 in \cite{gretta2023sharp}, which addresses the case of fixed coin probabilities. We will go through the steps of the proof highlighting the main differences. An implementation of \texttt{BayesLearn} for empirical quantile estimation, where each user is used at most once, can be found in Algorithm \ref{alg: BayesLearn}.
%


Let's define the capacity of the $(\tau, \alpha)$-BAC (Binary Asymmetric Channel) as
\begin{align*}
    C_{\tau, \alpha} &= \max_{q} H((1-q)(\tau-\alpha)+q(\tau+\alpha)) -(1-q)H(\tau-\alpha)-qH(\tau+\alpha), \\
    q&=\arg \max_x H((1-x)(\tau-\alpha)+x(\tau+\alpha)) -(1-x)H(\tau-\alpha)-xH(\tau+\alpha),
\end{align*}
where $H(p)$ is the binary entropy. Let's define the multiplicative Bayes weights $d_{x,y}:\{0,1\}\times \{0,1\}\rightarrow \R$, they indicates the multiplicative effect of a flip resulting $x$ (1=Heads, 0=Tails) on the density of an interval on side $y$ (1=Right, 0=Left) of the flipped coin.
\begin{align*}
    d_{0,0} &= \dfrac{1-\tau-\alpha}{1-\tau-(2q-1)\alpha}\\
    d_{0,1} &= \dfrac{1-\tau+\alpha}{1-\tau-(2q-1)\alpha}\\
    d_{1,0} &= \dfrac{\tau +\alpha}{\tau+(2q-1)\alpha}\\
    d_{1,1} &= \dfrac{\tau-\alpha}{\tau+(2q-1)\alpha}.
\end{align*}
We will mainly use the results from Lemma 9 in \cite{gretta2023sharp} that states that
\begin{gather}
    C_{\tau, \alpha} = (\tau + \alpha)\log_2 d_{1,0} + (1-\tau-\alpha)\log_2 d_{0,0} \label{eq: lemma A.1 [1]},\\
    C_{\tau, \alpha} = (\tau -\alpha)\log_2 d_{1,1} + (1-\tau+\alpha)\log_2 d_{0,1} \label{eq: lemma A.1 [2]},
\end{gather}
with the fact that $d_{1,0} \geq d_{0,0}$ and $d_{1,1}\leq d_{0,1}$. Recall the potential function: let $\{\ell,\dots,r\}$ be the set of $(\tau,\alpha(1+c))$-good intervals. Let $a$ be the maximum $i \in [B-1]$ such that $p^1_i\leq \tau$. Let $L$ be the list of intervals visited in \texttt{BayesLearn}. 
%
%

Let $j_t$ be the interval chosen at $t$-th round, and let $c_t$ be the index of the coin flipped. Let $p^t_{c_t} = p^t$ (we will discard the coin subscript) the probability of the selected coin at time $t$. We split the potential in two addend
\begin{gather}
    \label{eq: set}
    12 C_{\tau, \alpha}(|\{x\in L | x \in [\ell,r]\}|-\gamma|L|)\\
    \label{eq: log weight}
    \log_2 w(a)
\end{gather}
The main difference with the proof in \cite{gretta2023sharp} is that a good coin is defined on the initial probabilities $\{p^1_i\}_{i=1, \dots, B}$, but at the $t$-th iteration we only have access to coin with probability $\{p^t_i\}_{i=1,\dots,B}$. However, they are concentrated around $\alpha$, so $|p^t-p^1|\leq c\alpha$ for $c\leq 1$.

{\bf Bad queries:} Consider $j_t \notin [\ell, r]$. If $j_t>r$, then $p^1\geq \tau + (1+c)\alpha$. As we have that $|p^t-p^1|\leq c\alpha$ we also have $p^t \geq p^1-c\alpha \geq \tau+(1+c)\alpha-c\alpha=\tau+\alpha$. The expected change in the weights is
\begin{equation*}
    \E[\log_2 w_{t+1}(a)- \log_2 w_{t}(a)] = p^t \log_2 d_{1,0} + (1-p^t)\log_2 d_{0,0}\geq C_{\tau, \alpha}.
\end{equation*}
Where the last inequality comes from the fact that the expression is minimized as $p^{t}=\tau+\alpha$, and \autoref{eq: lemma A.1 [1]}. Consider now $j_t<L$, then $p^1\leq \tau-(1+c)\alpha$, which means $p^t \leq p^1+c\alpha \leq \tau-(1+c)\alpha + c\alpha = \tau-\alpha$, then 
\begin{equation*}
    \E[\log_2 w_{t+1}(a)- \log_2 w_{t}(a)] = p^t \log_2 d_{1,1} + (1-p^t)\log_2 d_{0,1}\geq C_{\tau, \alpha},
\end{equation*}
where we reach the minimum $C_{\tau, \alpha}$ when $p^t=\tau-\alpha$, due to \autoref{eq: lemma A.1 [2]}. As $j_t \notin [\ell,r]$ the change in \autoref{eq: set} is $-\gamma \cdot 12 C_{\tau,\alpha}$. Therefore, on bad queries the expected change in $\Phi$ is at least $(1-12\gamma)C_{\tau, \alpha}$.

{\bf Good Queries:} Let's consider the expected change in \autoref{eq: log weight} when $j_t \in [\ell, r]$. Consider the case where $j_t \neq a$, then the expected change is either
\begin{align*}
    &p^t \log_2 d_{1,0} +(1-p^t) \log_2 d_{0,0} \quad  \text{if}  \quad \text{$a$ is on the left of $j_t$, so $p^{0}\geq \tau \Rightarrow p^t \geq \tau-c\alpha$}\\
    &p^t \log_2 d_{1,1} +(1-p^t) \log_2 d_{0,1} \quad  \text{if} \quad  \text{$a$ is on the right of $j_t$, so $p^{0}\leq \tau \Rightarrow p^t \leq \tau+c\alpha\qquad$}
\end{align*}
The first expression is increasing in $p^t$ while the second is decreasing, therefore the expected change is at least
\begin{equation}\label{eq: min for good queries}
    \min\left\{(\tau-c\alpha) \log_2 d_{1,0} +(1-\tau+c\alpha) \log_2 d_{0,0}\,;\,(\tau+c\alpha) \log_2 d_{1,1} +(1-\tau-c\alpha) \log_2 d_{0,1}\right\}
\end{equation}
Let's consider the first argument of the previous expression
\begin{align*}
    (\tau-c\alpha) \log_2 d_{1,0} +(1-\tau+c\alpha) \log_2 d_{0,0} &=(\tau+\alpha) \log_2 d_{1,0} +(1-\tau-\alpha) \log_2 d_{0,0}-\alpha(1+c)(\log_2 d_{1,0}-\log_2 d_{0,0})\\
    &= C_{\tau, \alpha} -\alpha(1+c)\underbrace{(\log_2 d_{1,0}-\log_2 d_{0,0})}_{\geq 0} \quad \tag{as  $d_{1,0}\geq d_{0,0}$}\\
    &\geq C_{\tau, \alpha}-2\alpha (\log_2 d_{1,0}-\log_2 d_{0,0}) \quad \tag{as  $c\leq 1$}\\
    & \geq C_{\tau, \alpha}-2(6\log 2)C_{\tau, \alpha}\\
    & \geq -11 C_{\tau, \alpha},
\end{align*}
where in the first inequality we used the fact that $c\leq 1 \Rightarrow (1+c)\alpha\leq 2\alpha$, while in the second inequality we used Lemma 10 and Lemma 13 in \cite{gretta2023sharp}, valid for $\alpha \leq \frac{1}{2}\min(\tau, 1-\tau)$.
Analogously, for the second argument of \autoref{eq: min for good queries} we get
\begin{align*}
    (\tau + c\alpha)\log_2 d_{1,1} + (1-\tau-c\alpha)\log_2 d_{0,1} &=(\tau -\alpha)\log_2 d_{1,1} + (1-\tau+\alpha)\log_2 d_{0,1} -(1+c)\alpha\underbrace{(\log_2 d_{0,1}-\log_2 d_{1,1})}_{\geq 0}\\
    &\geq -11 C_{\tau, \alpha},
\end{align*}
where the inequality follows by an analogous computation.
Therefore, the change of the weights when $j_t \neq a$ is in expectation at least $-11 C_{\tau,\alpha}$ when $c\in [0,1]$ and $\alpha \leq \frac{1}{2}\min(\tau, 1-\tau)$.
Let's consider now the case where $j_{t} = a$, the expected change is
\begin{equation}
\label{eq: lemma general k}
    p^t\log_2(d_{1,0}k+d_{1,1}(1-k))+(1-p^t)\log_2(d_{0,0}k+d_{0,1}(1-k)),
\end{equation}
for some $k\in [0,1]$. We have two cases: $k\leq q$ or $k>q$. When $k\leq q$ the coin flipped is $a$ then $p^1\leq \tau$ and so $p^{t}\leq \tau+c\alpha$, in \cite{gretta2023sharp} it was shown that in this case \autoref{eq: lemma general k} is decreasing in $p^t$, then the minimum is 
\begin{equation}
\label{eq: min 1}
    (\tau+c\alpha)\log_2(d_{1,0}k+d_{1,1}(1-k))+(1-\tau-c\alpha)\log_2(d_{0,0}k+d_{0,1}(1-k)) 
\qquad \text{if } k\leq q.
\end{equation}
Conversely, when $k>q$ the coin flipped is $a+1$ and then $p^1\geq \tau$ so $p^t\geq \tau-c\alpha$. In this case the expression \eqref{eq: lemma general k} is increasing in $p^t$ so the minimum is
\begin{equation}
\label{eq: min 2}
    (\tau-c\alpha)\log_2(d_{1,0}k+d_{1,1}(1-k))+(1-\tau+c\alpha)\log_2(d_{0,0}k+d_{0,1}(1-k)) 
\qquad \text{if } k> q.
\end{equation}
In \cite{gretta2023sharp} the authors demonstrated that the minimum are obtained when $k\in \{0,1\}$. Therefore, for $k=1> q$ we have \autoref{eq: min 2} while for $k=0< q$ we have instead \autoref{eq: min 1}, which means that the minimum is
\begin{equation*}
    \min\left\{(\tau-c\alpha) \log_2 d_{1,0} +(1-\tau+c\alpha) \log_2 d_{0,0}\,;(\tau+c\alpha) \log_2 d_{1,1} +(1-\tau-c\alpha) \log_2 d_{0,1}\right\},
\end{equation*}
which is at least $-11 C_{\tau, \alpha}$ as demonstrated for the case $j_t \neq a$.
To conclude, the expected change in \autoref{eq: set} is at least $12 C_{\tau, \alpha}(1-\gamma)$, then the overall expected change for the potential is at least $12 C_{\tau, \alpha}(1-\gamma)-11C_{\tau, \alpha} = (1-12 \gamma)C_{\tau, \alpha}$, cocnluding the proof.
\end{proof}


The previous Lemma is the building block for the analysis of \texttt{BayesLearn}, as it allows the construction of a submartingale $\{Y_{t}\}_{t=1,\dots}$ with $Y_{t+1} = \Phi_{t+1}-gt$, for $g=(1-12\gamma)C_{\tau, \alpha}$,
%
that can be used to bound the probability to have a $\gamma$ fraction of good intervals, hence a positive potential. The analysis then follows directly from \cite{gretta2023sharp} with the distinction that the algorithm now with high probability in $\ab$ returns a $(\tau,\alpha(1+c))$-good coin, so proving Theorem \ref{thm:GP-generalization}. Since the proof is identical (see Lemma 6 and Theorem 1 of~\cite{gretta2023sharp}), we omit it. 
However, in order to make this paper self-contained, we will show a simple proof of Theorem~\ref{thm:NBS-changing-probabilities} (which is much less general than Theorem~\ref{thm:GP-generalization}). We restate the theorem here.
\begin{theorem}
Let $0<\alpha\leq \frac{1}{4}$ and suppose $c\leq 1$
There exists an algorithm for \texttt{AdvMonotonicNBS}$(1/2, \alpha, c)$ which uses $O\left(\tfrac{\log B}{\alpha^2}\right)$ coin flips and returns an $(1/2,\alpha(1+c))$-good with high probability in $B$.
\end{theorem}
\begin{proof}
Let $\Phi$ be the potential function in Lemma~\ref{lemma: increase in expectation of the potential} in the case $\tau=1/2$.
Given Lemma \ref{lemma: increase in expectation of the potential}, by choosing $g=(1-12\gamma)C_{1/2, \alpha}$ equal to the lower bound of the lemma, we have that $\{Y_{t}\}_{t=1,\dots}$, for $Y_{t+1} = \Phi_{t+1}-gt$
%
, is a submartingale as
%
%
%
\begin{equation*}
    \E[Y_{t+1}|y_1,\dots,y_t] = \E[\Phi_{t+1}|y_1,\dots, y_t]-gt = \underbrace{\E[\Phi_{t+1}-\Phi_t|y_1,\dots, y_t]}_{\geq g}-g +Y_{t}\geq Y_t
\end{equation*}
%
The difference of the martingale sequence $|Y_{t+1}-Y_t|$ is
\begin{equation*}
    |Y_{t+1}-Y_{t}|\leq |\log_2 w_{t+1}(a)-\log_{2}w_t(a)|+12 C_{1/2,\alpha}+g\leq  |\log_2 w_{t+1}(a)-\log_{2}w_t(a)|+O(\alpha^2),
\end{equation*}
by triangle inequality and $C_{1/2, \alpha}=\Theta(\alpha^2)$ for $\alpha\in (0,1/4)$ due to Lemma 10 \cite{gretta2023sharp}. The remaining term is 
$|\log w_{t+1}(a)-\log w_t(a)|\leq \max\{\log d_{1,0}, \log d_{0,1}\}\leq O(\alpha)$ for Lemma 13 \cite{gretta2023sharp}, thus $|Y_{t+1}-Y_t|\leq O(\alpha)$. We can use Azuma's inequality to bound the probability of having a negative potential 
\begin{align*}
    \Pr[\Phi_{t+1}\leq 0] &= \Pr[\Phi_{t+1}-gt-\Phi_1\leq -gt -\Phi_1]\\
    &=\Pr[Y_{t+1}-Y_0\leq -gt -\Phi_1]\\
    &\leq \exp\bigg(-\dfrac{(gt+\Phi_1)^2}{t\cdot O(\alpha^2)}\bigg)\quad \text{for } gt\geq -\Phi_1.
\end{align*}
Note that $\Phi_1=-\log(B-1)$. Therefore, picking $T=O\left(\frac{\log B}{g}\right)$ sufficiently large, we get that $\frac{(gT+\Phi_1)^2}{T\cdot O(\alpha^2)}\geq \lambda\log B$ for any desired constant $\lambda>0$. Thus,
\[
\Pr[\Phi_{T+1}\leq 0]\leq B^{-\lambda}.
\]
On the other hand, note that if $\Phi_{T+1}> 0$, then
\[
 0<\frac{\Phi_{T+1}}{12C_{1/2, \alpha}} \leq (|\{x\in L : x \in [\ell,r]\}|-\gamma|L|),
\]
and so, a $\gamma$ fraction of the intervals in $L$ are $(1/2,\alpha(1+c))$-good. Now we can order the intervals in $L$ in sorted order according to their indices $i$ of the corresponding coins. By picking a subset $S$ of every $(1/\gamma)$th of them, we are ensured that one of them will be good (conditional on the high probability event $\Phi_{T+1}> 0$). For each interval in $S$, we can test whether it is $(1/2,\alpha(1+c))$-good with high probability using $O(\frac{\log B}{\alpha^2})$ coin flips of each of the coins at its endpoints. Therefore, we successfully determine an $(1/2,\alpha(1+c))$-good coin with high probability in $B$. If we pick $\gamma=1/13$, the total number of coins flipped is 
\[
T+|S|O\bigg(\frac{\log B}{\alpha^2}\bigg)= O\bigg(\frac{\log B}{g}\bigg)+O\bigg(\frac{\log B}{\alpha^2}\bigg)=O\bigg(\frac{\log B}{\alpha^2}\bigg),
\]
where the final bound uses that $g=(1-12\gamma)C_{1/2, \alpha}=\frac{1}{13}C_{1/2, \alpha}=\Theta(\alpha^2)$. This completes the proof.
%
%
%
   %
%
%
%
 %
  %
%
%
%
  %
%
%
\end{proof}
\subsection{Lower bounds on sample complexity}\label{sec:sample_compexity}
We establish a lower bound for generalized linear measurements using standard information-theoretic arguments based on Fano's inequality. While the upper bound in Theorem~\ref{thm:alg_general} is derived for the maximum probability of error over all  $k$-sparse vectors, the lower bound applies even in the weaker setting of the average probability of error, where 
$\bx$ is chosen uniformly at random.
\begin{theorem}[Lower bound for GLMs]\label{thm: lower_bdglm} Consider any  sensing matrix $\vecA$.
For a uniformly chosen $k$-sparse vector $\bx$, an algorithm $\phi$ satisfies $$\bbP\inp{\phi(\vecA, \by) \neq \bx}\leq \delta$$   only if the number of measurements $$m\geq \frac{k\log\inp{\frac{n}{k}}}{I}\inp{1 - \frac{h_2(\delta) + \delta k\log{n}}{k\log{n/k}}}$$ for some $I$ such that $I\geq {I(y_i; \bx|\vecA)}, \, i\in [m]$. In particular, when $y\in \inb{-1, 1}$, we have $\bbE\insq{\inp{g(\vecA_i^T\bx)}^2} \geq I(y_i, \bx|\vecA)$ where the expectation is over the randomness of $\vecA$ and $\bx$.
\end{theorem}
The lower bound can be interpreted in terms of a communication problem, where the input message $\bx$ is encoded to $\vecA\bx$. The decoding function takes in as input the encoding map $\vecA$ and the output vector $\by$ in order to recover $\bx$ with high probability. For optimal recovery, one needs at least $\frac{\text{message entropy}}{\text{capacity}}$ number of measurements (follows from noisy channel coding theorem~\cite{thomas2006elements}). In Theorem~\ref{thm: lower_bdglm}, the entropy of the message set $\log{n \choose k}\approx k\log{n/k}$ and the proxy for capacity is the upper bound on mutual information $I$. We provide a detailed proof of the theorem in  Section~\ref{sec:proofs}.


We first present lower bounds for \bcs\  and \logreg. The lower bound for \bcs\ is given for any sensing matrix $\vecA$ which satisfies the power constraint given by \eqref{eq:power_constraint}, whereas the one for \logreg\ is only for the special case when each entry of the sensing matrix is iid $\cN(0,1)$. Recall that \eqref{eq:power_constraint} holds in this case.  For \bcs\ (and \logreg\ respectively), we can use the upper bound of $\bbE\insq{\inp{g(\vecA_i^T\bx)}^2}$ on the mutual information term. The dependence of $\sigma^2$ (and $1/\beta^2$ respectively) requires careful bounding of this term, which is done in the formal proofs in Appendix~\ref{proof:sec:lower_bd}.


As mentioned earlier, we need at least $k\log\inp{n/k}$ measurements for \bcs and \logreg. This is because the entropy of a randomly chosen $k$-sparse vector is approximately $k\log\inp{n/k}$ and we learn at most one bit with each measurement. However, due to corruption with noise, we learn less than a bit of information about the unknown signal with each measurement. The information gain gets worse as the noise level increases. 
Our lower bounds make this reasoning explicit.  
\begin{corollary}[\bcs\ lower bound]\label{thm: lower_bd_bcs} Suppose, each row $\vecA_i, \, i\in [1:m]$ of the sensing matrix $\vecA$ satisfies the power constraint~\eqref{eq:power_constraint}.
For a uniformly chosen $k$-sparse vector $\bx$, an algorithm $\phi$ satisfies $$\bbP\inp{\phi(\vecA, {\by}) \neq \bx}\leq \delta$$ for the problem of $\bcs$ only if the number of measurements $$m\geq \frac{k+\sigma^2}{2}\log\inp{\frac{n}{k}}\inp{1 - \frac{h_2(\delta) + \delta k\log{n}}{k\log{n/k}}}.$$ 
\end{corollary}

\begin{corollary}[\logreg\ lower bound]\label{thm: lower_bd_log_reg} Consider a Gaussian  sensing matrix $\vecA$ where each entry is chosen iid $N(0,1)$.
For a uniformly chosen $k$-sparse vector $\bx$, an algorithm $\phi$ satisfies $$\bbP\inp{\phi(\vecA, \bw) \neq \bx}\leq \delta$$ for the problem of $\logreg$ only if the number of measurements $$m\geq \frac{1}{2}\inp{k+\frac{1}{\beta^2}}\log\inp{\frac{n}{k}}\inp{1 - \frac{h_2(\delta) + \delta k\log{n}}{k\log{n/k}}}.$$ 
\end{corollary}



Theorem~\ref{thm: lower_bdglm} also implies an information theoretic lower bound for \spl, which is presented below and proved in Appendix~\ref{proof:sec:lower_bd}. Note that the denominator term in the bound $\frac{1}{2}\log\inp{1+\frac{k}{\sigma^2}}$ is the capacity of a Gaussian channel with power constraint $k$ and noise variance $\sigma^2$. 
\begin{corollary}[\spl\ lower bound]\label{thm: spl_lower_bd_1}
Under the average power constraint \eqref{eq:power_constraint} on  $\vecA$, for a uniformly chosen $k$-sparse vector $\bx$, an algorithm $\phi$ satisfies $$\bbP\inp{\phi(\vecA, {\by}) \neq \bx}\leq \delta$$ only if the number of measurements
$$m\geq \frac{k\log\inp{\frac{n}{k}}-\inp{h_2(\delta) + \delta k\log{n}}}{\frac{1}{2}\log\inp{1+\frac{k}{\sigma^2}}}.$$
\end{corollary} 

\subsection{Tighter upper and lower bounds for \spl}\label{sec:tighter_bounds_spl}
We present information theoretic upper and lower bounds for \spl\ in this section. Similar to Section~\ref{sec:alg}, our upper bound is for the maximum probability of error, while the lower bounds hold even for the weaker criterion of average probability of error.

We first present an upper bound based on the maximum likelihood estimator (MLE) where  we  decode to $\hat{\bx}$ if, on output $\by$, 
\begin{align*}
\hat{\bx} = \argmax_{\stackrel{\bx\in \inb{0,1}^n}{\wh{\bx} = k}}\,\, p(\by|{\bx})
\end{align*} where $p(\by|{\bx})$ denotes the probability density function of $\by$ on input $\bx$.
\begin{theorem}[MLE upper bound for \spl]\label{thm:upper_bd_mle} Suppose  entries of the measurement matrix $\vecA$ are i.i.d. $\cN(0,1).$
The MLE  is correct with high probability if 
\begin{align}m\geq \max_{l\in[1:k]}  \frac{nN(l)}{\frac{1}{2}\log\inp{\frac{ l}{2\sigma^2}+1}}\label{eq:upper_bd_mle}
\end{align}where  $N(l):=  \frac{k}{n} h_2\inp{\frac{l}{k}} + (1-\frac{k}{n})h_2\inp{\frac{l}{n-k}}$. 
\end{theorem}
We prove the theorem in Appendix~\ref{proof:MLE}. The main proof idea involves analysing the probability that the output of the MLE is $2l$ Hamming distance away from the unknown signal $\bx$ for different values of $l\in [1:k]$ (assuming $k\leq n/2$). This depends on the number of such vectors (approximately $2^{nN(l)}$) and the probability that the MLE outputs a vector which is $2l$ Hamming distance away from $\bx$. 

Note that when $l = k\inp{1-\frac{k}{n}}$, $nN(l) = nh_2(k/n)\approx k\log{\frac{n}{k}}$ and $\log\inp{\frac{k\inp{1-k/n}}{2\sigma^2}+1}\leq \log\inp{\frac{k}{2\sigma^2}+1}$.
Thus, $m$ is at least $\frac{2k\log{n/k}}{\log\inp{\frac{k}{2\sigma^2}+1}}$ (see the bound for Corollary~\ref{thm: spl_lower_bd_1}). It is not immediately clear if this value of $l= k\inp{1-\frac{k}{n}}$ is the optimizer. However, for large $n$, this appears to be the case numerically as shown in Plot~\ref{plot:1}.

\begin{figure}[t]
\includegraphics[width=7cm]{Unknown2.png}
\centering
\caption{The figure shows the plot of the MLE upper bound \eqref{eq:upper_bd_mle} (given by m1) for different values of $k$. This is displayed in blue color. A plot of $\frac{2nN(l)}{\log\inp{\frac{ l}{2\sigma^2}+1}}$ is also presented for $l = k\inp{1-\frac{k}{n}}$ in orange color, given by m2. A part of the plot is zoomed in to emphasize the closeness between the lines. In these plots,  $\sigma^2$ is set to 1,  $n$ is 50000 and $k$ ranges from 1000 to 25000 $(n/2)$. }\label{plot:1}
\end{figure}


Inspired by the MLE analysis, we derive a lower bound with the same structure as \eqref{eq:upper_bd_mle}. We generate the unknown signal $\bx$ using the following distribution: A vector $\tilde{\bx}$ is chosen uniformly at random from the set of all $k$-sparse vectors. Given $\tilde{\bx}$, the unknown input signal $\bx$ is chosen uniformly from the set of all $k$-sparse vector which are at a Hamming distance $2l$ from $\bx$. 
The lower bound is then obtained by computing upper and lower bounds on $I(\vecA, \by;\bx|\tilde{\bx})$.
We show this lower bound only for random matrices where each entry is chosen iid $\cN(0,1)$.
\begin{theorem}[\spl\ lower bound]\label{thm:lower_bd_spl}
If each entry of $\vecA$ is chosen iid $\cN(0,1)$, then for a uniformly chosen $k$-sparse vector $\bx$, an algorithm $\phi$ satisfies 
\begin{align}
    \bbP\inp{\phi(\vecA, {\by}) \neq \bx}\leq \delta\label{eq:spl_lower_bd_l}
\end{align}  only if the number of measurements $$m\geq \max_l\frac{nN(l) - 2\log{n}- h_2(\delta) - \delta k\log{n}}{\frac{1}{2}\log\inp{1+\frac{l}{\sigma^2}\inp{2-\frac{l}{k}}}} .$$
\end{theorem} The proof of Theorem~\ref{thm:lower_bd_spl} is given in Appendix~\ref{proof:MLE}.

If we choose $l = k\inp{1-\frac{k}{n}}$ in Theorem~\ref{thm:lower_bd_spl}, we recover corollary~\ref{thm: spl_lower_bd_1} for the special case of Gaussian design.
% \begin{corollary}\label{corollary2:lower_bd_spl}
% If  each entry of $\vecA$ is chosen iid $\cN(0,1)$, then for a uniformly chosen $k$-sparse vector $\bx$, an algorithm $\phi$ satisfies 
% $$\bbP\inp{\phi(\vecA, {\by}) \neq \bx}\leq \delta$$
% only if the number of measurements 
% $$m\geq \frac{k\log\inp{\frac{n}{k}} - 2\log{n}- h_2(\delta) - \delta k\log{n}}{\log\inp{1+\frac{k}{\sigma^2}}} .$$
% \end{corollary}

% Corollary~\ref{corollary2:lower_bd_spl} can also be proved directly for any sensing matrix $\vecA$ which satisfies \eqref{eq:power_constraint} (non-necessarily a Gaussian design). 


% \begin{figure}[t]
% \includegraphics[width=8cm]{plot.png}
% \centering
% \caption{The figure shows the plot of the MLE upper bound \eqref{eq:upper_bd_mle} (given by m1) for different values of $n$. This is displayed in blue color. A plot of $\frac{2nN(l)}{\log\inp{\frac{ l}{2\sigma^2}+1}}$ is also presented for $l = k\inp{1-\frac{k}{n}}$ in orange color, given by m2. In these plots,  $\sigma^2$ is set to 1 and $k$ is $0.2n$. }\label{plot:1}
% \end{figure}


\section{Naive Shuffle-DP binary Search for the Median}
\label{sec:naive-shuffle}
This section is dedicated to proving~\cref{thm:main-shuffle}.


The naive binary search with errors algorithm tests each coin up to $\acc$-accuracy and a $\failp/\log\ab$ failure probability, such that a simple union bound over all $\log\ab$ steps of binary searching will yield an $(\acc,\failp)$-accurate estimate. This algorithm is suboptimal up to logarithmic factors, although there are indications that its strong constant factors can make up the difference in some parameter regimes~\cite{karp2007noisy,gretta2023sharp}. The simple fact that this algorithm runs in deterministic number of rounds, with a deterministic number of samples per round, allows for a straightforward application of amplification by shuffling~\cite{feldman21shuffle}, something we could not achieve with the fully adaptive Bayesian updates algorithm. 

We consider both statistical error, where samples are assumed to be drawn from some unknown distribution with mean $p$, and we are interested in an estimate $\hat{p}$ which is close to that true mean, and the empirical setting where we make no assumption on the distribution of the samples, and are interested in how close our estimate $\hat{p}$ is to the ``best-case'' sample mean $\frac{1}{n}\sum_{i=1}^nx_i$.


\begin{lemma}[Sample complexity of learning one coin to its statistical mean.] 
\label{lemma: one-coin-statistical-mean}
    Given samples $\{x_i\}_{i=1}^n$ from a Bernoulli random variable $X$ with mean $p$ received through a binary randomized response channel $\pmech$ such that $y_i\sim \pmech(x_i)$, we can estimate $\hat{p}=\frac{1}{n}\frac{e^\priv + 1}{e^\priv - 1}\sum_{i=1}^n y_i - \frac{1}{e^\priv - 1}$. In order to learn an $(\acc,\failp)$-estimate of $p$, $\Pr[|\hat{p}-p|>\acc]<\failp$ it suffices to use $n$ samples where,
    $$
    n\leq\left(\frac{2p(1-p)}{\acc^2} + \frac{e^\priv}{\acc^2(e^\priv - 1)^2} + \frac{2(e^\priv + 1)}{4\acc(e^\priv - 1)}\right)\log(1/\failp).
    $$
    In other words, the sample complexity of learning one coin to its statistical mean with constant failure probability is $O\left(\frac{1}{\acc^2\priv^2} +\frac{p(1-p)}{\acc^2}\right)$, when $\priv<1$, or $O\left(\frac{1}{\acc^2e^\priv} +\frac{p(1-p)}{\acc^2}\right)$, when $\priv\geq 1$.
\end{lemma}
\begin{proof}
    Given a Bernoulli random variable $x$ with mean $p$, and a binary randomized response channel $\pmech$ (see~\autoref{def: binary rr}) the distribution induced by applying $\pmech$ to $x$ is:
    \begin{equation*}
    \label{eq:rr-bern-induced}
    y=\pmech(x)\sim\operatorname{Bern}\left(\frac{e^\priv}{e^\priv + 1}p + (1-p)\frac{1}{e^\priv + 1}\right)=\operatorname{Bern}\left(\frac{e^\priv-1}{e^\priv + 1}p + \frac{1}{e^\priv + 1}\right).
    \end{equation*}
    The variance of this distribution is 
    \begin{align*}
\sigma^2=\operatorname{Var}(y)&=\left(\frac{1}{e^\priv + 1}+\frac{e^\priv-1}{e^\priv + 1}p \right)\left( \frac{e^\priv}{e^\priv + 1}-\frac{e^\priv-1}{e^\priv + 1}p\right)\notag\\
&=\left(\frac{e^\priv - 1}{e^\priv + 1}\right)^2p(1-p) + \frac{e^\priv}{(e^\priv + 1)^2}.  \label{eq:rr-bern-var}
    \end{align*}
    We then proceed by simple rearranging, substitution, and application of Bernstein's inequality~\cref{fact: bernstein}.
    \begin{align*}
        \Pr\left[|\hat{p}-p| >\acc\right]&=\Pr\left[\bigg|\frac{1}{n}\frac{e^\priv + 1}{e^\priv - 1}\sum\limits_{i=1}^n y_j - \frac{1}{e^\priv - 1} - \left(\frac{e^\priv + 1}{e^\priv - 1}\bEE{y} - \frac{1}{e^\priv - 1}\right)\bigg| >\acc\right]\\
&=\Pr\left[\bigg|\frac{e^\priv + 1}{e^\priv -1}\left(\frac{1}{n}\sum\limits_{i=1}^n y_i -\bEE{y}\right)\bigg|>\acc\right]\\
&=\Pr\left[\bigg|\frac{1}{n}\sum\limits_{j=1}^ny -\bEE{y}\bigg|>t\right]\tag*{$\left(t=\acc\frac{e^\priv -1}{e^\priv +1}\right)$}\\
\failp&\leq\exp\left(\frac{-nt^2}{2\sigma^2 + \frac{2t}{3}}\right)\tag{Bernstein's Inequality}\\
n&\leq\left(\frac{2\sigma^2}{t^2} +\frac{2}{3t} \right)\log(1/\failp)\\
    &=\left( \frac{2p(1-p)}{\acc^2} +\frac{2e^\priv}{\acc^2(e^\priv - 1)^2}+\frac{2(e^\priv + 1)}{3\acc(e^\priv - 1)}\right)\log(1/\failp).\tag{Substituting $t$ and $\sigma^2$}
    \end{align*}
\end{proof}

\begin{lemma}[Sample complexity of learning one coin to its sample mean.]
\label{lem:empirical-coin-learn-rr}
    Given samples $\{x_i\}_{i=1}^n$ where each $x_i\in\{0,1\}$, and private outputs $y_i\sim \pmech(x_i)$, the true sample mean is $P=\frac{1}{n}\sum_{i=1}^n x_i$. Denote the sample mean of the collected private outputs $Y=\frac{1}{n}\sum_{i=1}^n y_i$. Our estimator of the sample mean will be similar to the statistical case, where $\widehat{P}=\frac{e^\priv + 1}{e^\priv - 1}Y - \frac{1}{e^\priv - 1}$. In order to learn an $(\acc,\failp)$-estimate of $P$ it is sufficient to use $n$ samples such that 
    \begin{equation*}
        n\leq\left( \frac{2e^\priv}{\acc^2 (e^\priv - 1)^2} + \frac{2(e^\priv + 1)}{3\acc (e^\priv - 1)} \right)\log(1/\failp).
    \end{equation*}
    Therefore, the sample complexity of learning the sample mean with constant failure probability is $O\left(\frac{1}{\acc^2\priv^2}\right)$, when $\priv<1$, or $O\left(\frac{1}{\acc^2e^\priv} \right)$, when $\priv\geq 1$. It is pleasing to note that this recovers the sample complexity of learning in the statistical case, up to the additive sampling error.
\end{lemma}

\begin{proof}
    The proof will proceed similarly to the statistical case. The key difference will be the variance of $y$ in this case which is
    \[
    \sigma^2=\sigma^2(\pmech(0))=\sigma^2(\pmech(1))=\frac{e^\priv}{(e^\priv + 1)^2}.
    \]
    The derivation then proceeds as in the statistical case.
    \begin{align*}
        \Pr[|\widehat{P}-P|>\acc] &= \Pr\left[\bigg| \frac{e^\priv + 1}{e^\priv - 1}Y - \frac{1}{e^\priv - 1} - P\bigg| >\acc\right]\\
            &=\Pr\left[\bigg| \frac{e^\priv + 1}{e^\priv - 1}Y - \frac{1}{e^\priv - 1} - \left( \frac{e^\priv + 1}{e^\priv - 1}\bEE{Y} - \frac{1}{e^\priv - 1} \right)\bigg|>\acc \right]\\
            &= \Pr\left[\frac{e^\priv + 1}{e^\priv - 1}\bigg| \left(Y-\bEE{Y} \right)\bigg|>\acc \right]\\
            &=\Pr\left[\bigg| Y-\bEE{Y}\bigg|> t \right]\tag*{$\left(t=\acc\frac{e^\priv - 1}{e^\priv + 1}\right)$}\\
            &\leq \exp\left(\frac{-nt^2}{2\sigma^2 + \frac{2t}{3}}\right)\tag{Bernstein's Inequality}\\
        n   &\leq \left( \frac{2\sigma^2}{t^2} + \frac{2}{3t} \right)\log(1/\failp)\\
            &=\left(\frac{2e^\priv}{\acc^2 (e^\priv - 1)^2} + \frac{2(e^\priv + 1)}{3\acc(e^\priv - 1)} \right)\log(1/\failp).\tag{Substituting $t$ and $\sigma^2$}
    \end{align*}
\end{proof}

With this we can now formally state the sample complexity of a naive binary search for the median under local differential privacy. We will focus on the empirical case for this result. 
\begin{theorem}[Naive Binary Search for the Median under Local Differential Privacy]
\label{thm:ldp-nbs-naive}
    The naive algorithm as described by~\citet{karp2007noisy}, under the constraints of $\priv$-local differential privacy, has sample complexity
    \[
    n\leq \left( \frac{2e^\priv}{\acc^2(e^\priv - 1)^2} + \frac{2(e^\priv + 1)}{3\alpha(e^\priv - 1)} \right)\log(\ab)\log\left(\frac{\log\ab}{\failp}\right).
    \]
    We can therefore say that for $\priv<1$, the naive approach has sample complexity $O\left(\frac{\log\ab}{\acc^2\priv^2}\log\left(\frac{\log\ab}{\failp}\right)\right)$, and for $\priv\geq 1$ it has sample complexity $O\left(\frac{\log\ab}{\acc^2e^\priv}\log\left(\frac{\log\ab}{\failp}\right)\right)$.
\end{theorem}
\begin{proof}
    Given $n$ total users, let $n'=n/\log(\ab)$ and let $\failp'=\failp/\log(\ab)$, apply~\autoref{lem:empirical-coin-learn-rr} with $n',\failp'$ to get sample complexity.
    \[
    n\leq \left( \frac{2e^\priv}{\acc^2(e^\priv - 1)^2} + \frac{2(e^\priv + 1)}{3\alpha(e^\priv - 1)} \right)\log(\ab)\log\left(\frac{\log\ab}{\failp}\right).
    \]
    By a union bound over all $\log\ab$ rounds of the binary search, the final estimate will be an $(\acc,\failp)$-approximate median.
\end{proof}

As stated in the introduction, the primary motivation for this approach is that by dividing the algorithm into a few deterministic stages, with many samples tested at each stage, we can hope to apply amplification by shuffling~\cite{feldman21shuffle}. We state the amplification by shuffling result here, and a subsequent lemma that will be useful to our analysis.
\begin{theorem}[{\citet*[Theorem 3.1]{feldman21shuffle}}]
    \label{theorem: amplification by shuffling}
    For any domain $\mathcal{X}$, let $\pmech_t:\pmech_1\times\ldots\times\pmech_{t-1}\times\mathcal{X}\rightarrow\mathcal{Y}$  for $t\in [n]$ be a sequence of randomizers such that $\pmech_t(y_{1:t-1},\cdot)$ is $\priv_L$-local DP; and let $S$ be the algorithm that given a tuple of $n$ messages, outputs a uniformly random permutation of said messages. Then for any $\privdelta\in(0,1]$ such that $\priv_L\leq\log\frac{n}{16\log(2/\privdelta)}$, $S\circ \mathcal{Y}^n$ is is $(\priv,\privdelta)$-DP, where
    \[
    \priv\leq\log\left(1 + 8\frac{e^{\priv_L}-1}{e^{\priv_L}+1}\left(\sqrt{\frac{e^{\varepsilon_L}\log(4/\delta)}{n}}+
\frac{e^{\varepsilon_L}}{n}\right)\right)
    \]
\end{theorem}
This implies the following useful lemma,
\begin{lemma}[Amplification by shuffling]
\label{lemma: amplification by shuffling}
    Fix any $\privdelta\in(0,1]$, $\priv\in(0,1]$, and $n$ such that $\priv>16\sqrt{\log(4/\privdelta)/n}$. Then, for
    \[
    \priv_L\coloneqq\log\frac{\priv^2 n}{80\log(4/\privdelta)}
    \]
    Shuffling the messages of $n$ users using the same $\priv_L$-LDP randomizer satisfies $(\priv,\privdelta)$-shuffle differential privacy.
\end{lemma}
\begin{proof}
    For $\priv,\privdelta$ and $\priv_L$ as above we have $0<\priv_L\leq\log\frac{n}{16\log(2/\privdelta)}$. Applying~\autoref{theorem: amplification by shuffling}, we get $(\priv',\privdelta)$-differential privacy for
    \[
    \priv' \leq \log\left(1 + 8\underbrace{\frac{e^{\priv_L}-1}{e^{\priv_L} + 1}}_{<1}\underbrace{\left(\frac{\priv}{\sqrt{80}}+\frac{\priv^2}{80\log(4/\delta)}\right)}_{< \priv/8}\right)\leq \priv
    \]
    Proving the lemma.
\end{proof}

%
We can now prove~\cref{thm:main-shuffle}.
\begin{theorem}[Restatement of~\Cref{thm:main-shuffle}]\label{thm:restated-shuffle}
    Let $r=\log_2 B$. There exists a protocol for \texttt{shuffle\--emp\--median}$(\{x_i\}_{i=1}^n,\alpha,\eps,\delta,r)$ with success probability $1-\failp$ provided that
    \[
    n=O\left( \left(\frac{1}{\acc^2} +\frac{1}{\priv^2}\right)\log\ab\sqrt{\log(1/\privdelta)\log\frac{\log\ab}{\failp}} \right).
    \]
    The protocol has $r=\log_2\ab$ rounds of adaptivity and queries shuffled batches of $n/\log_2(\ab)$ users. 
\end{theorem}
%
%
%
%
%
%
%
%
\begin{proof}[Proof of~\Cref{thm:main-shuffle}]    
    Take the sample complexity achieved in~\autoref{thm:ldp-nbs-naive}, and note that we are in the $\priv\gg 1$ regime as we will be applying taking $\priv_L\in O(\log n)$. We therefore have
    \[
    n= O\left( \frac{\log\ab}{\acc^2 e^{\priv_L}}\log\frac{\log\ab}{\failp} \right)
    \]
    We apply~\autoref{lemma: amplification by shuffling} while noting that at each stage we shuffle $n'=n/\log(\ab)$ users. Setting $\priv_L=\log\frac{\priv^2 n}{80\log(\ab)\log(4/\privdelta)}$ and rearranging gives that for each step of the binary search we have enough users to accurately learn the CDF of the remaining suffix of users within error $\alpha/2$ with probability $\beta/\log B$. Union bounding over all $\log B$ steps of the binary search, we conclude that with probability $1-\beta$, every step succeeds. This gives sample complexity,
    \[
    n= O\left(\frac{\log\ab}{\acc\priv} \sqrt{\log(1/\privdelta)\log\frac{\log\ab}{\failp}} \right),
    \]
    but we are not finished. We have to handle the multiple restrictions on parameter regimes 
    \[
    O\left(\frac{\log\ab}{\acc\priv} \sqrt{\log(1/\privdelta)\log\frac{\log\ab}{\failp}} \right)\geq n >\max\left\{\frac{\log\ab}{\acc^2},\frac{{\log\ab\log(1/\privdelta)}}{\priv^2}\right\}.
    \]
    The right hand side of this inequality comes from restrictions present in~\cref{lemma:CDF-bound,lemma: amplification by shuffling} on $n$ and $\priv$ respectively, the latter comes from using $n'=n/\log(\ab)$ in the restriction on $\priv$. 
    A trivial solution is be to take $1/(\acc\priv)$ and replace it with $1/\min\{\acc^2,\priv^2\}$, which gives
    \[
    n=O\left( \left(\frac{1}{\acc^2} +\frac{1}{\priv^2}\right)\log\ab\sqrt{\log(1/\privdelta)\log\frac{\log\ab}{\failp}} \right).
    \]
    %

   
\end{proof}

This result has an improved dependence in $\priv$ and $\acc$, and could be preferable from a communication perspective. Rounds of adaptivity are a restricting factor in distributed learning, and our goal was to understand the trade offs possible under privacy constraints. It is of practical interest to know whether the constraint on $n$ in~\cref{lemma: amplification by shuffling} can be improved from $n=\Omega\left(\log(1/\privdelta)/\priv^2\right)$ to $\Omega\left(\log(1/\privdelta)/\priv\right)$. This, in combination with a strengthening of~\cref{lemma:CDF-bound} to have a linear dependence on $\acc$, would allow the analysis to go through with only a $1/(\acc\priv)$ dependence.
\section{A Note on the Continuous Case}\label{sec:continuous} If we replace the discrete domain $[B]$ with a continuous one, say $[0,1]$, it is generally impossible to obtain quantile error $o(1)$ using a finite number of samples under LDP. This follows from our lower bounds by discretizing $[0,1]$ into $[B]$ buckets and letting $B\to \infty$. In fact, this is a general issue for quantile or range estimation problems in DP (even beyond the local model), which is why related work studies the discrete setting~\cite{BeimelNS16twotologstar,Bun2015logstar,Kaplan2020closinggap,kulkarni2019answering}. On a more positive note, if we impose mild guarantees on the family of possible distributions the samples can come from, our result has implications in the continuous setting as well. For instance, if we assume that there are (known) numbers $-\infty=y_0<y_1<\cdots <y_B=\infty$ such that in any interval $[y_i,y_{i+1}]$, the emperical CDF increases by at most $\alpha/2$, then we can again obtain quantile error $\alpha$ with $O(\frac{\log B}{\eps^2\alpha^2})$ users using our algorithm and bucketing users in the same interval $[y_i,y_{i+1})$. As the dependency on $B$
 in the number of samples is logarithmic, this might allow 
 $B$ to be quite large, with a correspondingly small quantile error $\alpha$. We note that if the assumption on the CDF is incorrect, only the accuracy is affected while the algorithm remains private.
\section{Experiments: Planning outperforms Heuristics}
\label{sec:experiment}

We begin our empirical demonstrations by showcasing the effectiveness of our planning framework on both synthetic and real datasets. We focus on the simplest planning algorithm, 1-step lookaheads (Algorithm~\ref{alg:complete}), and show that even basic planning can hold great promise. 
We illustrate our framework using two uncertainty quantification modules---GPs and 
\ensembles/ \ensembleplus. 

Throughout this section, we focus on evaluating the mean squared error of 
a regression model $\model$,  and develop adaptive policies that minimize uncertainty on $g(f)$ defined in~\eqref{eqn:l2-g-f}.
When GPs provide a valid model of uncertainty, 
our experiments show that our planning framework significantly outperforms other baselines. 
We further demonstrate that our conceptual framework extends to deep learning-based uncertainty quantification methods such as  \ensembleplus while highlighting computational challenges that need to be resolved in order to scale our ideas. 
For simplicity, we assume a naive predictor, i.e., $\psi(\cdot) \equiv 0$. However, we emphasize that this problem is just as complex as if we were using a sophisticated model $\psi(.)$. The performance gap between the algorithms 
primarily depends
on the level  of uncertainty in our prior beliefs.

To evaluate the performance of our algorithm, we benchmark it against several baselines. 
%Active learning baselines use an acquisition function $\ac$ to select points that have the highest   function value: $X\opt_t \in \argmax_{X \in \xpoolj{t}} \ac({X})$ at every step $t$. These methods may also need an UQ module, which we simply use the same UQ module as in our algorithm, and it  outputs $V(X)$ that measures the the uncertainty of each point $X \in \xpoolj{t}$.
Our first set of baselines are from active learning~\citep{AggarwalKoGuHaPh14}:
\\ % \noindent\textbf{Active Learning Heuristics:} 
\textbf{(1)} 
\textsf{Uncertainty Sampling (Static):}  In this approach, we query the samples for which the model is least certain about. Specifically, we estimate the variance of the latent output $f(X)$ for each $X \in \xpool$ using the UQ module and select the top-$K$ points with the highest uncertainty. \\
\textbf{(2)} \textsf{Uncertainty Sampling (Sequential):} This is a greedy heuristic that sequentially selects the points with the highest uncertainty within a batch, while updating the posterior beliefs using pseudo labels from the current posterior state. Unlike \textsf{Uncertainty Sampling (Static)}, this method takes into account the information gained from each point within batch, and hence tries to diversify the selected points within a batch. 

 
We also compare our approach to the  \textbf{(3)} \textsf{Random Sampling}, which selects each batch uniformly at random from the pool. Additionally, we compare solving the planning problem using  \textsf{REINFORCE}-based policy gradients with   $\mathsf{Smoothed\text{-}Autodiff}$ policy gradients.\footnote{Our code repository is available at
  \url{https://github.com/namkoong-lab/adaptive-labeling}.}
%Detailed experimental setups are provided in Section \ref{sec:details-experiments}.

%We repeat all experiments with 10 random seeds.




\begin{figure}[t]
\centering
\begin{minipage}[b]{0.49\textwidth}
\centering
\includegraphics[width=\textwidth, height=5cm]{figures/original_scale/Var_of_l_2_loss.pdf}
\caption{(Synthetic data) Variance of mean squared loss evaluated through the posterior belief $\mu_t$ at each horizon $t$. This is the objective that policy gradient methods like \textsf{REINFORCE} and $\ouralgo$ optimizes. 1-step lookaheads are surprisingly effective even in long horizons.}
\label{fig:var-l2-sim}
\end{minipage}
\hfill
\begin{minipage}[b]{0.49\textwidth}
\centering \includegraphics[width=\textwidth, height=5cm]{figures/original_scale/Error_of_estimated_model_l_2_loss.pdf}
\caption{(Synthetic data) Error between MSE calculated based on collected data $\mc{D}^{0:T}$ vs. population oracle MSE over $\mc{D}_{\rm eval} \sim P_X$. Reducing uncertainty over posteriors directly leads to better OOD evaluations. 1-step lookaheads significantly outperform active learning heuristics in small horizons.}
\label{fig:mean-l2-sim}
\end{minipage}
%\caption{Simulated data for GPs}
%\label{fig:both_plots}
\end{figure}

\subsection{Planning with Gaussian processes}
\label{sec:experiment-plan-GP}
We now briefly describe the data generation process for the GP experiments,  deferring a more detailed discussion of the dataset generation to Section~\ref{sec:details-experiments}. 
We use both the synthetic data and the real data to test our methodology.
For the \emph{simulated data},  we construct a setting where the general population is distributed across \emph{51 non-overlapping clusters} while the initial labeled data $\dtrain$ just comes from one cluster. In contrast, both $\dpool \defeq (\xpool,\ypool),\deval \defeq (\xeval,\yeval)$ are generated   from all the clusters. 
We begin with a low-dimensional scenario, generating a one-dimensional regression setting using a GP. %Gaussian Process (GP).
Although the data-generating process is not known to the algorithms,  we assume that the GP hyperparameters are known to all the algorithms
to ensure fair comparisons. This can be viewed as a setting where our prior is well-specified, allowing us to isolate the effects
of different policy optimization approaches
 without any concerns about the misspecified priors. We select $10$ batches, each of size $K=5$ across $T = 10$ time horizons.

To examine the robustness of our method against the distributional assumptions made  in the simulated case, we then move to a real dataset where the correct prior is not known. We simulate selection bias from the eICU dataset~\citep{PollardJoRaCeMaBa18}, which contains real-world patient data with in-hospital mortality outcomes. 
We conduct a $k$-means clustering to generate 51 clusters and then select data from those clusters. We view this to be a credible replication of practice, as severe distribution shifts are common due to selection bias in clinical labels.  To convert the binary mortality labels into a regression setting, we train a  random forest classifier and fit a GP on predicted scores, which serves as the UQ module for all the algorithms. As before, the task is to select 10 batches, each consisting of 5 samples, across 10 time horizons.

 In Figures~\ref{fig:var-l2-sim} and~\ref{fig:mean-l2-sim}, we present results for the simulated data. 
Figure~\ref{fig:var-l2-sim} shows the variance of $\ell_2$ loss, and Figure~\ref{fig:mean-l2-sim} presents the error in the estimated $\ell_2$ loss using $\mu_t$ (relative to true $\ell_2$ loss, that is unknown to the algorithm). 
As we can see from these plots, our method one-step lookahead  gives substantial improvements  over active learning baselines and random sampling. In addition,
compared to the one-step lookahead planning approach using \textsf{REINFORCE}-based policy gradients, 
we observe that $\mathsf{Smoothed\text{-}Autodiff}$-based policy gradients provide significantly more robust performance over all horizons.

In Figures~\ref{fig:var-l2-real}~and~\ref{fig:mean-l2-real}, we observe similar findings on the eICU data. We see that planning policies (\textsf{REINFORCE} and $\mathsf{Smoothed\text{-}Autodiff}$) consistently outperform other heuristics by a large margin.  Active learning baselines perform poorly in these small-horizon batched problems and can sometimes be even worse than the random search baselines.  Overall, our results show the importance of careful planning in adaptive labeling for reliable model evaluation. 

We offer some intuition as to why one-step lookahead planning may outperform other heuristic algorithms. 
 First,  \textsf{Uncertainty sampling (Static)} while myopically selects the
 top-$K$ inputs with the highest uncertainty, it fails to consider 
the overlap in information content among the ``best” instances; see \citep{AggarwalKoGuHaPh14} for more details. 
In other words,  it might acquire points from the same region with high uncertainty while failing to induce diversity among the batch.
Although \textsf{Uncertainty Sampling (Sequential)} somewhat addresses the issue of information overlap, a significant drawback of 
this algorithm
is the disconnect between the objective we aim to optimize and the algorithm. For example, it might sample from a region with high uncertainty but very low density. 

\begin{figure}[t]
\centering
\begin{minipage}[b]{0.48\textwidth}
\centering
\includegraphics[width=\textwidth, height=5cm]{figures/original_scale/Var_of_l_2_loss_real.pdf}
\caption{(Real-world eICU data) Variance of mean squared loss evaluated through the posterior belief $\mu_t$ at each horizon $t$. Even 1-step lookaheads are extremely effective planners, and auto-differentiation-based pathwise policy gradients provide a reliable optimization algorithm based on low-variance gradient estimates.}
\label{fig:var-l2-real}
\end{minipage}
\hfill
\begin{minipage}[b]{0.48\textwidth}
\centering \includegraphics[width=\textwidth, height=5cm]{figures/original_scale/Error_of_estimated_model_l_2_loss_real.pdf}
\caption{(Real-world eICU data) Error between MSE calculated based on collected data $\mc{D}^{0:T}$ vs. population oracle MSE over $\mc{D}_{\rm eval} \sim P_X$. Reducing uncertainty over posteriors directly leads to better OOD evaluations. Our method significantly outperforms active learning-based heuristics, and random sampling.}
\label{fig:mean-l2-real}
\end{minipage}
%\caption{Real data for GPs}
\end{figure}
 
%\vspace{-1.5cm}
% \begin{wrapfigure}{r}{.32\columnwidth}
%   \vspace{-.5cm} 
%   \centering
% \includegraphics[scale=.29]{figures/Var of l2l_2 loss.pdf}
%   \vspace{-0.2cm}
%   \caption{Results of GP}
% \label{fig:var-l2-gp}
%   \vspace{-0.1cm}
% \end{wrapfigure}


% Attempts have been made  in the past to address these  drawbacks heuristically  (see \citep{AggarwalKoGuHaPh14}). We give a unified computational framework while approaching the problem in a more principled manner and solving it more optimally.




\subsection{Planning with  neural network-based uncertainty quantification methods ($\ensembleplus$)}


We now provide a proof-of-concept that shows the generalizability of our conceptual framework  to the deep learning-based UQ modules, specifically focusing on $\ensembleplus$ due to their previously observed superior performance~\citep{OsbandWenAsDwIbLuRo23}. Recall that implementing our framework with deep learning-based UQ modules  requires us to retrain the model across multiple possible random actions $\bm{a}(\theta)$ sampled from the current policy $\pi_\theta$.
This requires significant computational resources, in sharp contrast to the GPs where the posteriors are in closed form and can be readily updated and differentiated. 

Due to the computational constraints, we test $\ensembleplus$ on a toy setting to demonstrate the generalizability of our framework. We consider a setting where the general population consists of four clusters, while the initial labeled data only comes from one cluster. Again we generate data using GPs.  The task is to select a batch of 2 points in one horizon. We detail the $\ensembleplus$ architecture in Section \ref{sec:details-experiments}, and we assume prior uncertainty to be large (depends on the scaling of the prior generating functions). 
The results are summarized in the Table~\ref{tab:UQ_ensemble}.

% \begin{table}[H]
% \vspace{-10pt}
% \caption{Performance under \ensembleplus as UQ module}
%     \centering
%     \begin{tabular}{|m{3cm}|m{2.5cm}|m{2cm}|} 
%     \hline
%       Algorithm   & Variance of $\loss_2$ loss estimate & Error of $\loss_2$ loss estimate  \\ \hline Random Sampling 
%          & $1710.9 \pm 1352.1$ & $8.67\pm6.62$ 
%       \\ \hline \ouralgo & $1.30 \pm 0.68$ & $0.91\pm0.25$ \\ \hline
%     \end{tabular}
%     \label{tab:UQ_ensemble}
%     %\vspace{-10pt}
% \end{table}




\begin{table}[h]
\vspace{-10pt}
\caption{Performance under \ensembleplus as the UQ module}
\centering
\begin{tabular}{|l|l|l|}
\hline
Algorithm   & Variance of $\loss_2$ loss estimate & Error of $\loss_2$ loss estimate  \\
\hline
\textsf{Random sampling} & 7129.8 $\pm$ 1027.0 & 136.2 $\pm$ 8.28 \\ \hline
\textsf{Uncertainty sampling (Static)} & 10852 $\pm$ 0.0 & 162.156 $\pm$ 0.0 \\ \hline
\textsf{Uncertainty sampling (Sequential)} & 8585.5 $\pm$ 898.9 & 144 $\pm$ 6.93 \\ \hline
\textsf{REINFORCE} & 1697.1 $\pm$ 0.0 & 45.27 $\pm$ 0.0 \\ \hline
\ouralgo & 1697.1 $\pm$ 0.0 & 45.27 $\pm$ 0.0 \\ \hline
\end{tabular}
%\caption{Comparison of different algorithms based on variance   and   error in $\ell_2$ loss estimation with Ensemble $+$ as the UQ module. Our results demonstrate that {\ouralgo} and REINFORCE outperformthe other active learning based heuristics, confirming the benefits of our MDP formulation for the adaptive labeling problem, as also demonstrated in Section 4.\\
%\footnotesize{Experimental details: We use Gaussian Processes as our data generating process, GP parameters are the same as in Section D.3.  The task is to select a batch of 2 points along one horizon.The marginal distribution $p_X$ has 4 \textit{non-overlapping} clusters. Initial data comes from one cluster, while pool and evaluation points comes from all the clusters. We have $20$ initial labeled data points, $10$ pool points, and $252$ evaluation points.  Training procedures are similar to the one in Section D.3.} }
\label{tab:UQ_ensemble}
\end{table}



% We faced  issues in scaling up these experiments which will be our focus in the future. 





% \begin{itemize}
%     \item Posteriors should be consistent. Two dimensions: even with less training,  
%     \item the inference should be  fast enough
% \end{itemize}


% Potential research directions for uncertainty quantification

% In this section we consider a simple setting We consider a simpler setting and 


% For synthetic dataset generation, we use ...... For real datasets, we use ...... We compare our methodolgy to several baselines ()    This Section is structured as follows:
% \begin{itemize}
%     \item \textbf{GPs, square loss objective} (Section \ref{}): 
%     %the broad aim of the experiments  in this section is to isolate the performance of our methodology without any concerns for the inefficiencies induced due to a mis-specified prior or imperfect posterior inference. To accomplish this we generate synthetic datasets using GPs (detailed later). We use the well specified prior (GPs - with same hyperparameter setting) as our UQ module.   
%      As GPs provide differentaible posterior inference - any errors induced due to imperfect posterior updates are also isolated. We note that under this setting
%      \item In Section\ref{} we demonstrate why our methodology performs better than other baselines - by devising various synthetic experiments ()
%     \item  \textbf{UQ Benchmarking }(Section \ref{}): Before diving into the experiments using $\ensembleplus$ and ENNs,  we showcase our benchmarking experiments in Section \ref{}. We use real datasets We observe that ENNs perform better
%      \item \textbf{Ensemble $+$}, objective: recall, accuracy
%     \item \textbf{ENN}, objective: recall, accuracy
% \end{itemize}




% In Section {}, we test 
% \subsection{Experimental details}

% \begin{itemize}
%     \item UQ methodologies - GPs, ENNs
%     \item Objectives - Recall,  ATE
%     \item Datasets - ATE-synthetic datasets, Recall-synthetic, real datasets
%     \item Baselines - 
%     \begin{itemize}
%         \item Random sampling
%         \item Active learning - Uncertainty based sampling - In regression setting almost all of the 
%         \item Myopic greedy - Greedy Batch based sampling
%         \item Policy Gradient
%     \end{itemize}
    
% \end{itemize}

% \subsection{Experiments}
%     \begin{itemize}
%     \item GPs with square loss
%     \item Benchmarking ENN
%         \item ENNs with ATE
%         \item ENNs with Recall
%     \end{itemize}

% \subsection{Benefits over other algorithms - intuition and experiments}

%Active learning - Myopic greedy / Don't rely on the objective rather some entropy version.


%%% Local Variables:
%%% mode: latex
%%% TeX-master: "main"
%%% End:

%
%


\end{document}


%
%
%
%
%
%
%
%
%
%
%
%
%