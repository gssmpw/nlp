\section{Statistical Private Median Estimation}\label{sec:statistical-median}
In this section, we will provide an algorithm for \texttt{LDPstat-median} using the state-of-the-art algorithm for \texttt{MonotonicNBS}. We prove the following:

\begin{theorem}\label{thm:main-stat}
Let $\alpha \in \left(0,\frac{1}{4}\right)$ and $\varepsilon
>0$. Suppose that the number of users $n\geq C\frac{\log B}{\alpha^2}\left(\frac{e^\varepsilon+1}{e^\varepsilon-1}\right)^2$ for a sufficiently large constant $C$. Then there exists an algorithm solving \texttt{LDPstat-median}$(\mathcal{D},n,\alpha,\eps)$ with high probability in $B$. 
\end{theorem}

In this section, we prove Theorem~\ref{thm:main-stat}. For this, we recall the following result which is a corollary of the main result in~\cite{gretta2023sharp}. 
Recall the definition of an $\left(\frac{1}{2}, \alpha\right)$-good coin in~\eqref{eq:good-coin}.
\begin{theorem}[\cite{gretta2023sharp}]\label{thm:from-GP}
For any $\alpha \in \left(0,\frac{1}{4}\right)$, there exists an algorithm for \texttt{MonotonicNBS}$(\tau,\alpha)$ which uses $O(\frac{\log B}{\alpha^2})$ coin flips and outputs an $\left(\frac{1}{2},\alpha\right)$-good coin with high probability in $B$.
\end{theorem}
\begin{proof}[Proof of Theorem~\ref{thm:main-stat}]
For $i\in [B]$, we define $q_i=\sum_{j\leq i}\mathcal{D}[j]$ with the convention that $q_0=0$. 
Thus $j\mapsto q_j$ is the CDF of $\mathcal{D}$. Consider sampling $X\sim \mathcal{D}$ and let $Y$ be the random variable obtained by applying randomized response to the indicator variable $[X\leq j]$ retaining the bit with probability $\frac{e^\eps}{1+e^\eps}$ and flipping it otherwise. Then $\Pr[Y=1]=p_j$ where $p_j=q_j\cdot \frac{e^\eps}{1+e^\eps}+(1-q_j)\cdot \frac{1}{1+e^\eps}$. Then,
\begin{equation}\label{eq:use-GP}
    q_{j} = \left(p_{j}-\frac{1}{e^\varepsilon+1}\right)\frac{e^\varepsilon+1}{e^\varepsilon-1},
\end{equation}
We use the the algorithm in Theorem~\ref{thm:from-GP} to solve \texttt{MonotonicNBS}$\left(\frac{1}{2},\alpha\frac{e^\varepsilon-1}{e^\varepsilon+1}\right)$ when the inputs are the unknown $\{p_i\}_{i=1}^B$. To do so, whenever the algorithm calls for flipping a coin $j$, we sample a new user $X\in \mathcal{D}$ and apply randomized response to the variable $Y=[X\leq j]$. By standard properties of randomized responze, this protocol satisfies the $\eps$-LDP requirement. Moreover, by Theorem~\ref{thm:from-GP}, the algorithm finds an $\left(\frac{1}{2}, \alpha\frac{e^\varepsilon-1}{e^\varepsilon+1}\right)$-good coin $j^*$ with high probability in $B$. In particular $p_{j^*}\leq \frac{1}{2}+\alpha\frac{e^\varepsilon-1}{e^\varepsilon+1}$ and $p_{j^*+1}\geq \frac{1}{2}-\alpha\frac{e^\varepsilon-1}{e^\varepsilon+1}$. 
It thus follows from Equation~\eqref{eq:use-GP} that $q_{j^*}\leq 1/2+\alpha$ and $q_{j^*+1}\geq 1/2-\alpha$. Therefore $j^*$ is an $\alpha$-approximate median of $\mathcal{D}$ completing the proof.
\end{proof}
In the high privacy regime, i.e. for $\varepsilon<1$ , the sample complexity of Theorem \ref{thm:main-stat} becomes $n=\Omega\left(\frac{\log B}{\varepsilon
^2\alpha^2}\right)$, matching our lower bound up to a constant factor.