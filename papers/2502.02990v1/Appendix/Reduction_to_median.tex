\section{Reduction to the Median} 

\label{appendix: Reduction to the Median}
Consider the simple case where we are given an algorithm $A$ which returns the median of $n$ samples in the most natural sense, by returning the $n/2$'th index of their sorted representation. Without changing this algorithm we can have it return any arbitrary index by adding elements to the beginning or the end of this sorted array. For example, adding two elements to the beginning of the array will create a new array with $n'=n+2$ elements where the $n'/2$'th index will be the $(n/2-1)$'th index of the original array. The padding argument below formalizes this notion, demonstrating that any algorithm for an $\alpha$-approximation of the median can be used to obtain a $2\alpha$-approximation of any quantile.
\begin{lemma}[Padding Argument] 
\label{appendix: padding argument}
Any $\alpha$-approximation algorithm for the median, with $\alpha \in \left(0,\frac{1}{2}\right)$, can be used to construct a $2\alpha$-approximation for any quantile $\tau\in (0,1)$. 
\end{lemma}
\begin{proof}
    Consider a dataset $D=\{x_1, \dots, x_n\}$ where each element is such that $x_i \in \{1,\dots,\ab\}$. Let $\mathcal{M}$ be an algorithm for the $\alpha$-approximation of the median then for $m = A(D)$ we have by definition
     \begin{equation}
     \label{eq: appendix padded 1}
        \text{Pr}_{\mathcal{D}}[x\leq m]<\frac{1}{2}+\alpha \qquad \text{and} \qquad \text{Pr}_{\mathcal{D}}[x\leq m+1]>\frac{1}{2}-\alpha.
    \end{equation}
    where $\text{Pr}_{D}[x\leq m] = \frac{\sum_{x\in D}[x\leq m]}{n}$, and $[x\leq m]$ is an indicator function. 
    Consider now a padded dataset $D_P = D\cup \{1\}^{(1-\tau)n} \cup \{\ab\}^{\tau n}$, where $\{a\}^{x}$ indicates the multi-set containing the $a$ element $x$ times \footnote{We consider $(1-\tau)n$ and $\tau n$ integers.}. The new empirical cumulative distribution of the data set for $y \in \{1, \dots, \ab-1\}$, is \begin{align*}
    \label{eq: appendix padded 2}
        \text{Pr}_{D_P}[x\leq y] &= \frac{(1-\tau)n +\sum_{x\in D}[x\leq y]}{|D_P|} = \frac{1-\tau}{2}+\frac{1}{2}\text{Pr}_{D}[x\leq y],
    \end{align*}
    as we have $|D_P| = 2n$. Thus 
    \begin{equation}
    \label{eq: appendix padded 3}
        \text{Pr}_{D}[x\leq y] = 2\text{Pr}_{D_P}[x\leq y] +\tau -1.
    \end{equation}
    The application of $A$ to the padded data set $D_{P}$ returns a $\alpha$-approximate median $m_P = A(D_P)$. Therefore, for $m_P\in\{1,\dots, \ab-1\}$, from \autoref{eq: appendix padded 3} and \autoref{eq: appendix padded 1} it follows that 
    \begin{equation}
    \label{eq: appendix padded 4}
       \text{Pr}_{D}[x\leq m_P]<\tau+2\alpha \qquad \text{and} \qquad \text{Pr}_{D}[x\leq m_P+1]>\tau-2\alpha.
    \end{equation}
    Notice that $m_p\neq \ab$, as $\text{Pr}_{D_P}[x\leq \ab]=1<\frac{1}{2}+\alpha$ iff $\alpha>\frac{1}{2}$. This concludes the proof.
\end{proof}

%
%
%
%
%
%
%
