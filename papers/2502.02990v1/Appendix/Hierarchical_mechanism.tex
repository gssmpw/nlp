\section{The Hierarchical Mechanism}\label{app:hierarchical-mech}
The algorithm was presented in \cite{kulkarni2019answering} and can be used to approximately answer general range queries. It comes in several variants but we will present the simplest version (the bounds on the number of users needed for the various versions are similar). The main idea is to construct a $b$-ary tree of depth $\Theta(\log(B))$ on $[B]$. For the below, we will assume that $B$ is a power of $2$ and that $b=2$ (although for the experiments, we use a different constant $b$). The nodes on level $i$ (where level 0 is the root) corresponds to the $2^i$ dyadic intervals of $B$. Namely, in the binary representation of elements of $B$, there is an interval corresponding to each prefix of length $i$ in the binary representation. The non-adaptive protocol we will consider is as follows. Each user $i$ with data $x_i\in[B]$ picks a random level $\ell$ of the binary tree. The user writes a one-hot encoding $z$ of which node they belong to on level $\ell$ and uses randomized response on each of the $2^\ell$ bits of $z$. This is the message $y$, they send to the central server. This is the unary encoding mechanism; see~\cite{kulkarni2019answering} for more sophisticated solutions, that require less communication but nonetheless have the same approximation errors. The combined algorithm is denoted \texttt{Hierarchical Mechanisms}. 

\paragraph{Analysis sketch of \texttt{Hierarchical Mechanism}}
We here analyse the performance of \texttt{Hierarchical Mechanism} for answering general range queries and in particular show how it can be used for quantile estimation.

Assume that $\eps\leq 1$. If the number of users reporting at every level is $\gg \frac{1}{\alpha_0^2\eps^2}$ (where $a\gg b$ means that $a\geq C b$ for some constant $C$), then using standard concentration bounds, for each node in a given level, we can recover the total fraction of users lying in the corresponding subtree up to an additive $\alpha_0$ with constant failure probability. Now if the total number of users is $\gg \frac{\log B}{(\alpha_0^2\eps^2)}$, then with constant failure probability, the number of users reporting at any given level is indeed, $\gg \frac{1}{\alpha_0^2\eps^2}$. We now pick $\alpha_0=\alpha/(2\log B)$ and conclude that if the number of users is $\gg \frac{(\log B)^3}{(\alpha\eps^2)}$, we can recover the total fraction of users lying in any subtree up to an additive $\alpha/(2\log B)$ from the unary responses with constant failure probability. 
It follows that we can answer any range query with additive error $\alpha n$. Indeed, any range can be partitioned into at most $2\log B$ of these subtrees, two for each level. In particular, this means that we can find an $\alpha$-approximate median with constant failure probability.
It follows that we can answer any range query with additive error $\alpha n$. Indeed, any range can be partitioned into at most $2\log B$ of these subtrees, two for each level. In particular, this means that we can find an $\alpha$-approximate median with constant failure probability.
The analysis for high probability in $B$ needs $\gg \frac{\log B}{\alpha_0^2\varepsilon^2}$ number of users reporting at each level, so it adds an additional $\log B$ factor to the sample complexity.


%
