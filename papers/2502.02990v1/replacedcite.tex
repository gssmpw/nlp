\section{Related Work}
\label{sec:prior-work}

\paragraph{Differential Privacy} Differential privacy is considered the gold standard in private data analysis due to its rigorous guarantees, e.g., immunity to side information, and other useful properties ____. 
A number of mechanisms exist for releasing medians and general quantiles for centralized DP. First, one may instantiate mechanisms based on local sensitivity____, since quantiles often have low local sensitivity for many datasets. More recently, specialized mechanisms for medians ____ and quantiles ____ have been proposed to obtain even lower error but they require certain mild assumptions on the distribution of the data points.
%
The case where data points can be arbitrary from some discrete domain $[B]$, like for us, has been well studied in the central setting. The sequence of works,____ gradually reduced the number of users needed for accuracy $\alpha$ to $\tilde O(\frac{1}{\eps}\log^{1.5} (1/\delta)\cdot (\log^*B)^{1.5})
)$ for $(\eps,\delta)$-privacy. This almost matches the the $\Omega(\log^*(B))$ lower bound from____. A corollary of this lower bound is that even with central DP, no algorithm can achieve $o(1)$ quantile error in the continuous setting regardless of how many users there are.
%
\paragraph{Local Differential Privacy} There is increasing interest in local differential privacy (LDP), where the central aggregator is not trusted, and each user applies a DP mechanism to their data before broadcasting it. LDP mechanisms for many problems and accompanying lower bounds were shown in ____. A ubiquitous LDP protocol that we will utilize is \emph{randomized response} (See~\cref{def: binary rr}), where answers to a binary query are flipped with probability $\frac{1}{1+e^\varepsilon} \approx \frac{1}{2}-\epsilon$. For the median problem, an LDP algorithm was found in ____ under a different loss function, the difference between the estimate and the median in the \emph{data domain}. This loss function is subject to strong lower bounds (a linear dependence on the domain size). %
The most relevant work to our setting is the so-called \emph{hierarchical mechanism}____.
%

\paragraph{Hierarchical mechanism} 
%
The hierarchical mechanism uses the $b$-adic decomposition of the interval $[0,\ab]$ (which is a $b$-ary tree of depth $\Theta(\log_b(\ab))$ whose nodes at level $\ell$ correspond to intervals of length $\frac{\ab}{b^\ell}$). Each participant uniformly selects a level $\ell$ at random and employs standard frequency LDP oracles ____ to disclose which node at level $\ell$ their data belongs to. The central aggregator may then combine the frequency oracles at each level to answer any range query. 
%
%
A particular use of range queries with relative error $\alpha$ is for constructing an $\alpha$-approximate CDF of the data set, which in turn can be used to approximate every quantile within error $\alpha$. Unfortunately, dividing the user data among levels worsens the dependence on $\log(\ab)$. In Appendix~\ref{app:hierarchical-mech}, we demonstrate that the hierarchical mechanism can be used to solve \texttt{LDPemp-median} with $n =O(\frac{\log^3 \ab}{\epsilon^2 \alpha^2})$ users. %
%
In terms of the polynomial dependence on $\log B$, there is still a multiplicative $\log B$ gap between this upper bound and the lower bound of~\cref{thm:intro-lower-non-interactive} which would be very interesting to close.

\paragraph{Shuffle Differential Privacy}
The central model of DP requires that data be collected non-privately by the curator, which
results in extremely accurate protocols. %
On the other hand, in the local model users do not trust anyone, and the response to any query must be privatized before it is broadcast by the user. In shuffle-DP, each user applies a LDP protocol to their data and then sends the output to a trusted \emph{shuffler} whose only task is to randomly permute the users data before forwarding it to a central data curator. This places shuffling as a middle ground between these two models in terms of both trust and accuracy. 

%
Understanding the separation between the local, shuffle, and central models of privacy, and therefore the trade-offs between trust and accuracy, is of both theoretical, and practical interest. For a survey of such separations, see____. 

\paragraph{Noisy binary search and threshold queries} Consider an algorithm that sequentially picks a \textit{threshold query} $m\in[\ab]$, then samples a user from the database $X$ and receives the bit $y=[x\leq m]$. Since $\Pr[y = 1] = F_X(m)$, finding an integer $m$ such that $F_X(m) \approx q$ reduces to the \textit{noisy binary search} problem. This search over a CDF with \textit{threshold query} sample access, exactly mirrors searching over a monotonically increase sequence of coins. Noisy binary search was introduced by____ with a tight bound of $\Theta(\log(\ab)/\acc^2)$, later improved by constant factors by ____, which holds for the non-private median when samples are accessed via threshold queries in the statistical setting. 

%

%
%
%
%

%
%

%


\paragraph{Structure of the Paper}
In~\cref{sec:preliminaries}, we introduce necessary preliminaries for our theoretical analyses. In~\cref{sec:tech-contributions}, we provide an overview of our main ideas and technical contributions.~\cref{sec:proof-of-main-adaptive-up} is dedicated to proving~\cref{thm:main-emp}. In~\cref{sec:lower-bound}, we prove the lower bounds of~\cref{thm:main-lower,thm:intro-lower-non-interactive}. In~\cref{sec:naive-shuffle}, we provide the proof of~\cref{thm:main-shuffle}. Finally, in~\cref{sec:experiments} we present our experimental results.