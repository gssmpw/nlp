Next we discuss the open-ended benchmarks. 
These are essential for evaluating the reasoning, creativity, and problem-solving abilities of LLMs, particularly in scientific domains. Unlike MCQs, which primarily test factual recall or single-step reasoning, open-ended tasks engage models with complex, real-world problems that require multi-step reasoning, synthesis of knowledge across domains, and adaptive problem-solving. For instance, while an MCQ might test a model's recall of a specific scientific fact, an open-ended task could require the model to design an experiment, analyze data, or propose solutions to unsolved research questions. This format aligns better with the exploratory nature of scientific inquiry, offering a more comprehensive assessment of a model's capabilities. However, existing open-ended benchmarks often fall short in capturing the depth and realism needed for scientific evaluations. Many rely on synthetic tasks that fail to reflect the intricacies of real-world scientific challenges, such as multi-disciplinary reasoning or generating accurate code for practical applications. 
%Furthermore, the absence of standardized metrics for evaluating open-ended responses limits the utility of these benchmarks in providing consistent, actionable insights into a model’s performance.

%\ian{The following text provides a nice discussion of methods for evaluating open-ended responses. But it doesn't explain why this material is here. Provide some context?}\sandeep{added}
While open responses questions are versatile in capability, their
evaluation is more complicated compared to MCQs.
%than evaluation of MCQs because a model is not asked to select among a set of answers. Instead, the model produces a response to a question as free text that needs to be understood and assessed. 
Different assessment approaches are applied, depending on the task at hand.  
A first class of \textit{statistical scorer} approaches looks at co-occurrence of n-grams (sequences of letters or words) in LLM outputs vs.\ supplied ground truth responses. 
%Statistical scorers evaluate an LLM response from n-grams (sequences of letters or words). 
%\ian{In my view, the text is not clear that (as I think is correct?) you first introduce a broad class of statistical scorers, which contains BLUE, ROUGE, and METEOR. Then embedding approaches? Then LLM as a judge? Is my text ok?}\sandeep{corrected}
In this context, widely used scores are BLEU (BiLingual Evaluation Understudy) compares LLM outputs against ground truths. It calculates the precision for each matching n-gram (n consecutive words) between an LLM output and expected output. ROUGE (Recall-Oriented Understudy for Gisting Evaluation) evaluates text summaries and computes recall by comparing the overlap of n-grams between LLM outputs and expected outputs. It determines the proportion (0–1) of n-grams in the reference present in the LLM output. METEOR (Metric for Evaluation of Translation with Explicit Ordering) calculates scores by assessing both precision (n-gram matches) and recall (n-gram overlaps) and leverages linguistic databases to account for synonyms. Statistical scorers do not take any semantics and reasoning capabilities into account. 

A second class are \textit{embedding approaches} that seek to compare the semantics of LLM responses and reference answers. Here, some widely used scores include BERTScore, that relies on pre-trained language models (e.g., BERT) and computes the cosine similarity between the contextual embeddings of the LLM responses and references. These similarities are then aggregated to produce a final score. Other tools such as CheckEmbed \cite{besta2024checkembed} can be used to compare the semantics of LLM responses and reference answers. 
The third and most recent class is the  \textit{LLM-as-a-judge methods} which tackle the problem of evaluating LLM open responses when no reference answer is available. This approach currently has two variations. In the ``Pairwise comparison" version, an LLM judge is presented with a question and two answers, and tasked to determine which one is better or declare a tie. In the ``Single answer grading" version, an LLM judge is asked to directly assign a score to a single answer. In principle, LLM-as-a-judge can offer several key benefits: consistency, scalability, and explainability. However, the approach also has limitations: position bias (first answer better in ``Pairwise comparison"), verbosity bias (longer answer better), self-enhancement bias (self-generated answer better) and limited capability to grade math and reasoning questions \cite{LLM-as-judge}. Moreover, the reliability of such evaluations is still the subject of research.

We now discuss two open-ended benchmarks, one for scientific code generation and another for atomic layer deposition in Material Science.

\subsubsection{\bf SciCode - Scientific Code Generation Benchmark}
%\ian{The first part of this paragraph (up to the last two sentences) is redundant with earlier text. So I removed it.}
%The rapid development of LLMs has significantly accelerated advancements in AI, but existing benchmarks often fail to keep pace with growing LLM capabilities. Current benchmarks, particularly in scientific domains, frequently rely on synthetic or narrow tasks that struggle to capture the complexity and realism required for meaningful evaluation. This creates a disconnect between a model’s perceived performance on benchmarks and its actual utility in real-world applications, especially for tasks like code generation and problem-solving in science. High-quality, realistic, and challenging benchmarks are essential to address this gap, driving progress by enabling a more accurate evaluation of LLMs' generalization capabilities and their ability to assist in scientific discovery. These considerations motivated the development of SciCode, a benchmark designed to assess LLMs in solving complex scientific coding problems across diverse domains. By providing tasks that reflect real-world challenges and require multi-step reasoning, SciCode ensures that models are rigorously tested in contexts that closely align with the demands of scientific research.
The SciCode Benchmark is a set of manually curated coding problems designed to assess LLM capabilities for solving complex scientific coding problems across diverse domains. 
By providing tasks that reflect real-world challenges and require multi-step reasoning, SciCode allows models to be tested in contexts that align closely with the demands of scientific research~\cite{tian2024scicode}.
%\ian{If there is a paper with details, provide a pointer?}\eliu{done}
SciCode includes problems 
%The SciCode Benchmark is a comprehensive evaluation framework tailored to test the code generation and problem-solving abilities of LLMs 
across a range of scientific domains, including computational mechanics, quantum information, quantum chemistry, ecology, and molecular modeling. It consists of 80 main problems, decomposed into 338 intermediate steps, enabling a structured approach to assessing model capabilities. Solving each individual problem requires that an LLM implement multiple Python functions corresponding to subproblems and integrate those functions into a cohesive solution: see \autoref{fig:scicode_example}. Each problem is accompanied by a gold-standard solution and multiple test cases so as to permit robust and reliable automatic evaluation. 
\begin{figure*} %{r}{0.5\textwidth} 
    \centering
%    \fbox{
    \includegraphics[width=\textwidth,trim=0 6mm 6mm 10mm, clip]{figures/SciCode_LLMs.png} 
    \caption{The performance of various LLMs on SciCode problems. 
    This histogram displays the accuracy (vertical axis, 0\% to 100\%) of various state-of-the-art LLMs (listed on the horizontal axis) 
    in solving both main problems (red) and their associated subproblems (blue) within SciCode. 
    To solve a main problem, LLMs must implement one Python function per subproblem %\ian{Do you mean, one per subproblem?} \eliu{reworded} 
    and integrate them into a comprehensive solution. SciCode provides gold-standard solutions and multiple test cases for reliable automatic evaluation. 
    These consistently poor results highlight the need for LLMs that incorporate scientific knowledge and advanced reasoning to better assist researchers.  
    }
%    }
    \label{fig:scicode}
\end{figure*}
%By leveraging abundant, high-quality data \ian{I can't tell what this is referring to} typically unavailable to LLMs during training, SciCode challenges models to generalize to new scientific tasks and contexts. The benchmark highlights the importance of scientific LLMs, such as AuroraGPT, which incorporate specialized knowledge and advanced reasoning capabilities, to better support researchers in solving complex, real-world problems.
%\ian{The preceding sentence seems speculative, until you provide results.}


% \begin{figure*} %{r}{0.5\textwidth} 
%     \centering
% %    \fbox{
%     \includegraphics[width=\textwidth]{figures/SciCode_Example.pdf} 
%     \caption{A SciCode main problem is divided into multiple simpler subproblems for ease of implementation. Docstrings outline the requirements and specify the input-output formats. The scientific background is provided by expert annotators to offer necessary context and guidance.
%     }
% %    }
%     \label{fig:scicode_example}
% \end{figure*}

Each SciCode problem is meticulously annotated and verified by at least two senior researchers to ensure accuracy, and is drawn from real-world research tasks, maintaining relevance to practical applications. 
Problems are curated to avoid overlap with publicly available datasets and thus to test the deep scientific knowledge and analytical skills of LLMs by requiring the decomposition and integration of complex problems into comprehensive solutions. Additionally, SciCode %the framework \ian{What is the framework?} 
allows for flexible evaluation of model capabilities in varied setups, enabling adjustments like providing background information or conditioning on previous solutions. 

The SciCode Benchmark is configured to assess LLM capabilities to solve SciCode problems by using zero-shot prompts, maintaining a general approach while creating distinct prompts for various evaluation setups to guide the model on the tasks, as 
described in detail in \cite{tian2024scicode}. %\ian{We need pointers to details here.}\eliu{done}
The prompts remain consistent across models and fields, incorporating instructions for the main and sub-problems, as well as code from previous subproblems. We evaluated the coding capabilities of several state-of-the-art LLMs using the SciCode benchmark, focusing on three key aspects to assess their performance. First, the \textit{Impact of Scientific Background} was analyzed by testing models in two modes: without background, to evaluate inherent scientific knowledge and reasoning, and with background, to focus on coding and instruction-following capabilities. The results showed significant performance improvements with background information, highlighting the limitations of current LLMs in scientific reasoning. Second, the comparison between \textit{Gold vs. Generated Solutions} revealed insights into the challenges of realistic evaluations. While gold solutions accurately address each problem, generated solutions introduce error accumulation, creating a more practical and demanding evaluation scenario. Lastly, the assessment of \textit{Main vs.\ Subproblems} provided a nuanced understanding of model performance. A main problem was considered solved only when all subproblem solutions and the integrated result were accurate. Additionally, SciCode’s design allows independent evaluation of subproblems, enabling precise analysis of models’ reasoning and coding abilities across discrete tasks. These evaluation dimensions underscore the benchmark's rigor in testing LLMs for real-world scientific applications.

%\noindent 
We summarize the findings of our studies using several state-of-the-art models in \autoref{fig:scicode}. These results show that SciCode is a difficult benchmark for current LLMs. Consistent with our observations on proprietary models, open-weight LLMs under test also showed their lack of capabilities in solving any main problem despite being able to solve a number of sub-problems correctly.



The SciCode project provides insights into the challenges of evaluating LLMs in scientific coding tasks, highlighting significant gaps in current capabilities. Despite recent advancements, state-of-the-art models like OpenAI’s o1-preview and Claude3.5-Sonnet solve only a small fraction (7.7\%) of the main problems, underscoring the disparity between existing LLMs and the deep scientific reasoning required for real-world research. SciCode is designed to address this gap by focusing on real-world, research-level problems across diverse natural science fields, including mathematics, physics, chemistry, and biology. Sourced from peer-reviewed work, these problems test LLMs' ability to generalize to less familiar domains. By decomposing problems into subproblems with detailed annotations, SciCode rigorously evaluates models’ coding, reasoning, and knowledge integration capabilities. While providing scientific background information improves model performance, the persistent struggle of LLMs with these tasks emphasizes their current limitations in handling complex scientific challenges. The project highlights the importance of high-quality data, domain-specific validation, and carefully curated problems to advance the development of AI tools for scientific research. The findings indicate substantial progress is needed to enhance scientific reasoning and background knowledge integration in LLMs to enable their effective application in real-world scenarios.
\subsubsection{\bf ALDbench - Materials Synthesis Benchmark}

% \ian{I have a high-level question about this work. ALDbench is described as as a ``domain-specific open-ended question benchmark.'' As I understand things it is a list of questions, not a list of QA pairs, and its purpose is for evaluating the performance of LLMs via a manual process in which the LLM is given the questions and the answers are evaluated by humans. Presumably for that reason there is deliberately NOT a published list of ``correct'' answers, as otherwise LLMs could learn those answers. \textbf{SO:} Assuming I am correct in my understanding, we should make clear that this is a different sort of benchmark from the others that we have discussed. \textbf{AND:} As the text refers to ``this approach'', we should make clear what it is. Give it a name, maybe?}
%\sandeep{This is similar to the open-response benchmark in that the model is providing a free-form response, but instead of using LLM-as-a-judge to assess it, we are using human evaluators and studying the correlation to the rubrics}

An area that lacked relevant benchmarks is materials synthesis.
This is particularly important for potential applications
of LLMs in automated materials discovery or as AI research
assistants. LLMs underpinning such capabilities need to exhibit both the ability to reason about specific processes (for instance to avoid unsafe conditions or transfer ideas across reactors and process conditions) and have a robust understanding of the literature (to build on existing process knowledge and avoid known dead ends).

As such capabilities appear hard to evaluate by using either MCQs or the statistical scorer or embedding approaches described earlier.
%\ian{The text refers to ``NLP methods'' which I think refers to }
%or natural language processing methods, 
we developed a new open-response benchmark ALDbench on materials synthesis, and in particular on a synthesis technique called 
atomic layer deposition \cite{aldbench}. Here we targeted a range of difficulty spanning from graduate level to PhD-level domain expert current with the state of the art. A model's ability to perform at a domain expert level is paramount whenever models are expected to assist in decision making processes that involve costly experiments.
% In order to address these needs we created ALDbench,
% an open-response benchmark focused on materials synthesis,
% and in particular on a synthesis technique called 
% atomic layer deposition \cite{aldbench}.
Beyond its applied interest in areas such as energy and microelectronics \cite{AlvaroALD}, this domain brings together multiple
topics that are commonplace in chemistry-driven synthesis, including metal-organic and inorganic molecules, gas-surface kinetics and heterogeneous reactions, and gas phase transport. Evaluating LLM capabilities in this field can provide insights with wide applicability to other material synthesis techniques.

To compile the benchmark, we asked six PhD-level human experts to generate ``questions that a researcher or a graduate student who is not familiar with a specific process/application would ask an AI assistant." 
The curated questions could be grouped into four categories: 1) \emph{how to grow}, where the query is about material
synthesis; 2) \emph{specific questions about ALD processes}, comprising more in-depth queries about a process or material;  3) \emph{general ALD knowledge}, with questions about the synthesis
technique; and 4) \emph{applications}.

The human experts were then asked to grade the questions using a scale of 1 to 5 on two criteria with the following rubrics similar to the AI4S benchmark: (1) \emph{Difficulty:} 1--Easy, early graduate; 5--Hard, top expert; (2) \emph{Specificity:} 1--General; 5--Specific, quantitative.
% \begin{itemize}
% \item \emph{Difficulty:} 1--Easy, early graduate; 5--Hard, top expert.

% \item \emph{Specificity:} 1--General; 5--Specific, quantitative.
% \end{itemize}
Each response is then graded using four criteria with the following rubrics: (1) \emph{Overall quality:} 1--Very low quality; 5--Excellent; (2) \emph{Specificity:} 1--Too broad; 5--Targeted; (3) \emph{Relevance:} 1--Irrelevant fluff; 5--Relevant answer; (4) \emph{Accuracy:} 1--All made up; 5--All correct.   
% \begin{itemize}
% \item \emph{Overall quality:} 1--Very low quality; 5--Excellent.
% \item \emph{Specificity:} 1--Too broad; 5--Targeted.
% \item \emph{Relevance:} 1--Irrelevant fluff; 5--Relevant answer.
% \item \emph{Accuracy:} 1--All made up; 5--All correct.
% \end{itemize}
The use of multiple criteria allowed us to probe aspects of
the generation process, such as relevance or specificity of the
response, that are not easily captured by benchmarks focused
on accuracy.

We ran this benchmark using an instance of OpenAI’s GPT-4o, with seven PhD-level human experts reviewing model responses.
Details are in the ALDbench paper \cite{aldbench}.
The model responses received a composite quality score of 3.7, consistent with a passing grade. However, 36\% of the questions received at least one below average score. When
we carried out an in-depth analysis of the responses we
identified at least five instances of hallucination. In \autoref{fig:aldbench} we show the
distribution of mean scores for all the questions in the benchmark
and the four criteria evaluated by the human experts.

\begin{figure}[t]
\centering
    \includegraphics[width=.9\columnwidth]{figures/histogram_eval.png}
    \caption{Distribution of the mean scores of GPT-4o
    responses to all questions in the ALDbench benchmark.}
    \label{fig:aldbench}
    % \vspace{-4mm}
\end{figure}

We also explored statistical
correlations between the difficulty and specificity of
each question and the human expert scores for each evaluation criteria. For each (question, response) pair we computed p-values using the Fisher
exact test to evaluate the statistical significance of the
correlation.  We found statistically significant correlations
between question difficulty and response quality ($p_0$ = 0.033), question difficulty 
and relevance ($p_0$ = 0.016), and question specificity
and response accuracy ($p_0$ = 0.007). In all three cases, higher difficulty or specificity correlated
with lower scores.
These results emphasize the need to evaluate LLMs across multiple criteria beyond traditional metrics of difficulty and accuracy.

Our results show that highly targeted, open-response benchmarks can provide
information about LLM performance in scientific domains that is complementary to MCQs or natural language processing benchmarks. The methodology developed in this work allowed us to probe in depth model performance in a specific domain. With the aid of a small team of PhD-level experts we were able to identify instances of hallucinations and explore model responses in a level of detail that it is hard to accomplish using automatic evaluation methods. The extension of this approach to other domains, such as energy storage or microelectronics, is trivial. Moreover, as a byproduct of this
effort, we collected a small dataset of questions and human rated responses across four different evaluation criteria. As we explore other domains we can use this data to train or validate automatic question evaluation approaches for open-ended benchmarks. 