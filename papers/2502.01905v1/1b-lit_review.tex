\section{Literature Review}
 \label{background}
% \subsection{Opinion dynamics}
Maximising influence or opinion shares in social networks through efficient allocation of external influence is a well-studied optimisation problem with wide applications \cite{easley2010networks,rogers2010diffusion}. Several approaches have been proposed to tackle this problem \cite{noorazar2020recent,li2018influence}. One of these is a computational method that maximises the number of converted (or infected) agents in a network, through cascading events triggered by an optimal set of initially infected individuals (or `seed' set) \cite{kempe2003maximizing}. Cascade models have been widely applied to maximise innovation adoptions in viral marketing \cite{domingos2001mining}, study the spread of infectious diseases \cite{cheng2020outbreak}, as well as rumour propagation \cite{tripathy2010study}. 

In a complementary dynamical approach, models from statistical physics \cite{castellano2009statistical} are often used to approximate emerging macroscopic behaviours in populations. In these settings, individuals repeatedly switch between different opinion states \cite{barrat2008dynamical} until the system reaches an equilibrium. Opinions expressed by agents are modeled as discrete \cite{clifford1973model,galam1999application,holley1975ergodic,krapivsky2003dynamics} or continuous \cite{deffuant2000mixing,hegselmann2002opinion} variables that change as they interact within the network. Of these, we employ the paradigmatic voter model \cite{clifford1973model,holley1975ergodic,sood2005voter} in our work. In the classical voter model, agents switch between binary states at a rate proportional to the fraction of opinion shares in their immediate neighbourhood. Our choice of model relies on its relevance \cite{PMID:28542409}, and the simplicity and tractability of its approach, that allows us to implement both analytical and numerical methods on complex networks \cite{redner2019reality}.

% \subsection{Competitive opinion dynamics}
Dynamics in opinion models typically converge to an ordered consensus or fragmented states \cite{castellano2009statistical}. Since consensus is rarely achieved in the real world, it has motivated considerable efforts in understanding conditions that lead to opinion fragmentation in networks, with special attention to competitive settings \cite{hucompeting,prakash2012winner}. In such settings, controllers aim to exceed the competitor's share of opinions in the network at equilibrium.

Competitive opinion propagation has been extensively explored in the voter model using various forms of zealotry \cite{acemouglu2013opinion,mobilia2007role,yildiz2013binary}. Zealots are either biased individuals \cite{masuda2010heterogeneous,mobilia2003does}; radical agents impervious to influence from neighbours  \cite{barrat2008dynamical,kuhlman2013controlling,mobilia2007role,mobilia2015nonlinear} or external controllers unidirectionally targeting the network \cite{masuda2015opinion}. In this paper, we adopt a setting similar to that in \cite{masuda2015opinion}. However, in contrast to the norm of discretely targeting networks (where the most optimal seed set is identified), here we allow a continuous distribution of influence that targets nodes with varying intensities \cite{chakraborty2019competitive}. This modification widens the scope of application of this work to continuous resources such as money and time, while also significantly increasing the tractability of the optimisation problem.
 
 Much like zealots, contrarians too have been used to study competitive influence spread in the voter model \cite{gambaro2017influence,li2011strategy,masuda2013voter,zhong2005effects}. These are nonconformist agents opposing the majority view in their social neighbourhood. 
 We emulate this notion of disagreement in our work as a property of social relations, as opposed to that of agents, which is more realistic and a more general case of the contrarian model. 

%  \subsection{Competitive opinion dynamics in signed networks}

  As largely reflected in society and real-world networks, social ties can often be negative, representing distrust and enmity \cite{leskovec2010signed}.
 So far, networks with signed edges have received limited attention within the vast body of work on opinion dynamics (see \cite{girdhar2016signed} for a detailed review). Most of this existing work explores the problem in threshold models \cite{hosseini2019assessing}. Principally, greedy heuristics are used in models of diffusion, such as the Independent Cascade (IC) \cite{ju2020new,li2014polarity,liu2019influence} and the Linear Threshold (LT) models \cite{he2019information,liang2019influence,shen2015influence}. 
 In addition, a slightly modified integrated page-rank algorithm for seed selection in signed networks was presented in \cite{chen2015influence}, while authors used simulated annealing for the same problem in \cite{li2017positive}. Work in \cite{srivastava2015social} discusses a comparative analytical approach to the same problem. Other generalised models of opinion propagation have also been used \cite{jendoubi2016maximizing}, including the voter model, where controllers seed the network optimally to maximise vote-shares at the steady-state \cite{li2013influence}. 
 
 
 Our paper significantly extends the work in \cite{li2013influence} by framing the problem in more realistic, competitive and game-theoretic settings, which is different from the traditional single-controller influence maximisation problem studied in \cite{li2013influence}. We also further generalise the (discrete) optimisation problem presented in \cite{li2013influence} by allowing flexible distribution of influence in our model. Finally, we assume persistent influence from external controllers which diversifies the seeding process adopted in \cite{li2013influence}.