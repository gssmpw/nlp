\section{Discussion}
\label{Summary}
The study of opinion dynamics has been conventionally done on networks with strictly positive edges. In the real world however, networks often contain negative social connections, which can spread negative or opposing influence, thus creating a need to understand how these edges affect influence maximisation efforts in networks. 
To address this concern, we present a model for competitive spread of opinions in signed networks under voter dynamics. For comparison, we propose a complementary approach where controllers only observe the absolute weights of all edges i.e. they consider all edges to be positive. In both instances we present gradient ascent algorithms to numerically solve the problem in large-scale arbitrary networks. We test the robustness of our results in networks of varied structures under diverse budget conditions and adversarial allocations. 
We find that in networks where 20\% of edges are negative, controllers gain maximally (nearly 18\%) from awareness of negative edges, under conditions of scarce resources, and against competitors who deliberately avoids nodes with negative connections. 

We also propose a supporting theoretical approach to verify the accuracy of our algorithms. We present closed-form solutions in simplified network structures that provide further insights to the problem. We observe that in networks with highly concentrated positive links, allocations on nodes are driven by their negative degrees and the competitor's allocation on these nodes. Finally, we examine the problem under game-theoretic settings, where we highlight conditions under which a controller could lose vote-shares by implementing strategies that use the knowledge of negative ties in the network. Specifically, we show that when controllers have considerably less resources (or in some cases, excess budget), their prioritisation of nodes to target, may inadvertently disclose knowledge of negative ties to a competitor who was otherwise unaware, thus compromising their position of advantage.

The results in this paper present compelling evidence for considering negative ties in any influence maximisation exercise and thus contributes to the literature on competitive opinion dynamics in signed networks. Possible extensions to this work could include studying the problem under different constraint functions. For instance, the effect of modified budget constraints that explore the implications of an additional cost to retrieve information about the presence of negative ties, on influence maximisation efforts.
Additionally, this problem could be further studied in other realistic opinion models (e.g. Deffaunt model). 
% Going forward, this work could also serve as a foundation to guide empirical investigation on maximising opinion spread in the presence of negative edges.    



% We now propose an analytical framework in support of our numerical results. Note that, obtaining closed-form analytical solution for \cref{optimisation} on networks with inherent complexities can be challenging. We therefore simplify the problem first by adopting a degree-based mean-field approach that approximates system dynamics and helps us obtain analytical expressions for optimal al
