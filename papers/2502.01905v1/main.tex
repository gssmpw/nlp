\documentclass{article}
\usepackage[letterpaper,top=2cm,bottom=2cm,left=3cm,right=3cm,marginparwidth=1.75cm]{geometry}
\usepackage{graphicx}
\usepackage{amsmath}
\usepackage{graphicx}
\PassOptionsToPackage{hyphens}{url}
\usepackage[colorlinks=true, allcolors=blue]{hyperref}
\usepackage{cleveref}
\usepackage{makecell}
\usepackage{booktabs}
\usepackage{tabularx}
\usepackage{ltablex}
\usepackage{caption}
\usepackage{subcaption}
\usepackage{stackengine}
\usepackage{multirow}
\usepackage{authblk}
\usepackage[table]{xcolor}
\usepackage{abstract}
\usepackage{breakcites}
\usepackage{amssymb}
% \usepackage{lineno}
\usepackage{graphicx}
\usepackage[T1]{fontenc}
\usepackage{palatino}

\def\subrangle#1{\stackengine{5pt}{}{$\!\scriptstyle #1$}{U}{l}{F}{F}{L}}


\title{When \textit{not} to target negative ties? \\ Studying competitive influence maximisation in signed networks}

\author[1,2,3]{Sukankana Chakraborty\footnote{corresponding author: \href{schakraborty@turing.ac.uk}{schakraborty@turing.ac.uk}}}
\author[3]{Markus Brede}
\author[3]{Sebastian Stein}
\author[4]{Ananthram Swami}


\affil[1]{Public Policy Programme, The Alan Turing Institute, British Library, UK.}
\affil[2]{Centre for Advanced Spatial Analysis, University College London, UK.}
\affil[3]{Electronics and Computer Science, University of Southampton, UK.}
\affil[4]{DEVCOM, Army Research Laboratory, USA.}


\date{}
\begin{document}

%TC:ignore
\maketitle

% \linenumbers

\begin{abstract}
We explore the influence maximisation problem in networks with negative ties. Where prior work has focused on unsigned networks, we investigate the need to consider negative ties in networks while trying to maximise spread in a population \textemdash particularly under competitive conditions. Given a signed network we optimise the strategies of a focal controller against competing influence in the network using two approaches \textemdash either the focal controller uses a sign-agnostic approach or they factor in the sign of the edges while optimising their strategies. We compare the difference in vote-shares (or the share of population) obtained by both these methods to determine the need to navigate negative ties in these settings.
More specifically, we study the impact of: (a) network topology, (b) resource conditions and (c) competitor strategies on the difference in vote shares obtained across both methodologies. We observe that gains are maximum when resources available to the focal controller are low and the competitor avoids negative edges in their strategy. Conversely, gains are insignificant irrespective of resource conditions when the competitor targets the network indiscriminately.
Finally, we study the problem in a game-theoretic setting, where we simultaneously optimise the strategies of both competitors. Interestingly we observe that, strategising with the knowledge of negative ties can occasionally also lead to loss in vote-shares.
\end{abstract}

\section{Introduction}
\label{sec:introduction}
Opinions are predominantly influenced by individual beliefs, social interactions and information made available through external interventions \cite{myers1996social,watts2007influentials}. Of these, guiding collective opinions through external influence poses an interesting research problem which has far-reaching societal and commercial implications, such as trying to improve public health through controlled information campaigns \cite{wilder2018end}, campaigning in elections \cite{ranganath2016understanding} and marketing for product adoption \cite{watts2007viral} in populations.
%Additionally, it often allows radical parties to impose divisive ideas and propagate misinformation in societies, which can also lead to serious and harmful ramifications \cite{bennett2018disinformation,nelson2020danger,vosoughi2018spread}. 
The increasing reliance on social media and their pervasiveness in society today \cite{perrin2015social} has thus inspired a rich body of work dedicated to understanding how opinions evolve in social networks and how they can be optimally steered through external influence \cite{banerjee2020survey,noorazar2020recent}.
In particular, a number of papers explore this problem in social \cite{kiss2017mathematics,pastor2015epidemic}, political \cite{galam1999application} and economic settings \cite{easley2010networks,jackson2010social}. However, a majority of this literature strictly investigates friendship networks where influence propagates based on positive recommendations and endorsements \cite{newman2003structure}. Effects of negative relationships on opinion propagation have received limited attention and have been historically discounted from network dynamics, given their sparse presence \cite{offer2021negative} and association with avoidance behaviour \cite{harrigan2017avoidance}, i.e.~people who dislike (or distrust) each other are unlikely to communicate (or be connected within a network). However, the scope of anonymity and prevalence of fake profiles on social media platforms recently have made such ties increasingly ubiquitous \cite{bae2012sentiment,pfeffer2014understanding}, and even typical in many recommendation and trading networks \cite{guha2004propagation,leskovec2010signed}. Although this has initiated a lot of interest in understanding the impact of negative ties on network dynamics, there are several settings in this context that are yet to be explored \cite{offer2021negative}. 
% thus demanding a more robust understanding of the impact of negative ties on network dynamics. Moreover, they are governed by a unique set of properties that are distinctly different from those of positive ties \cite{easley2010networks}.

In this paper, we consider negative ties as antagonistic relationships that negatively influence social neighbours, and persuade them to adopt an opposing position (or opinion).  
Such relationships pose a unique challenge when trying to maximise influence in a network. Relationships of distrust (negative reviews) on e-commerce platforms (e.g.~eBay) following below-par experiences, for instance, can negatively impact future transactions or communication \cite{borgs2010novel,chen2011influence}. Therefore, effective navigation of such ties is required when influencing a network externally. This is particularly the case in competitive environments, as neglectfully targeting individuals that propagate overall negative influence, can, in turn, facilitate the spread of undesirable opinions \cite{chen2018negative}.

In this paper, we demonstrate the need for a negative-tie aware approach while maximising influence in social networks under competitive settings. In doing so, we make the following contributions:
\begin{enumerate}
\item We modify voter dynamics for networks with negative ties, under competitive conditions and  subsequently present a negative-tie aware influence maximisation algorithm to optimally target networks in these settings.
\item We show that in a real-world network a controller can achieve an additional $9\%$ in vote-shares against a naïve competitor (with no knowledge of negative ties in the network).
\item We show how the effectiveness of a negative-tie aware approach varies with network topology, availability of budget and competitor strategy.
\item We provide analytical support for numerical methods and derive expressions for optimal allocations in large, arbitrary synthetic networks.
\item We present results for the game-theoretic setting where controllers actively optimise their strategies against one another. 
\end{enumerate}
The structure of the rest of the paper is as follows. In \cref{background} we discuss the related literature. We then modify the classical voter model in \cref{Model} to study opinion dynamics in signed networks. 
%and further present an indiscriminate, traditional approach for comparison. 
In \cref{bitcoin}, we investigate the problem in real-world settings, and we analyse it further using numerical methods in \cref{numerical}. In \cref{mean-field}, we present analytical support for our numerical results and further provide analytical expressions for optimal allocations in complex networks. Finally, in \cref{game-theory} we extend the problem to game-theoretic scenarios and we show how much a controller can gain from considering negative ties in influence maximisation exercises against unknown competitor strategies.

\section{Literature Review}
 \label{background}
% \subsection{Opinion dynamics}
Maximising influence or opinion shares in social networks through efficient allocation of external influence is a well-studied optimisation problem with wide applications \cite{easley2010networks,rogers2010diffusion}. Several approaches have been proposed to tackle this problem \cite{noorazar2020recent,li2018influence}. One of these is a computational method that maximises the number of converted (or infected) agents in a network, through cascading events triggered by an optimal set of initially infected individuals (or `seed' set) \cite{kempe2003maximizing}. Cascade models have been widely applied to maximise innovation adoptions in viral marketing \cite{domingos2001mining}, study the spread of infectious diseases \cite{cheng2020outbreak}, as well as rumour propagation \cite{tripathy2010study}. 

In a complementary dynamical approach, models from statistical physics \cite{castellano2009statistical} are often used to approximate emerging macroscopic behaviours in populations. In these settings, individuals repeatedly switch between different opinion states \cite{barrat2008dynamical} until the system reaches an equilibrium. Opinions expressed by agents are modeled as discrete \cite{clifford1973model,galam1999application,holley1975ergodic,krapivsky2003dynamics} or continuous \cite{deffuant2000mixing,hegselmann2002opinion} variables that change as they interact within the network. Of these, we employ the paradigmatic voter model \cite{clifford1973model,holley1975ergodic,sood2005voter} in our work. In the classical voter model, agents switch between binary states at a rate proportional to the fraction of opinion shares in their immediate neighbourhood. Our choice of model relies on its relevance \cite{PMID:28542409}, and the simplicity and tractability of its approach, that allows us to implement both analytical and numerical methods on complex networks \cite{redner2019reality}.

% \subsection{Competitive opinion dynamics}
Dynamics in opinion models typically converge to an ordered consensus or fragmented states \cite{castellano2009statistical}. Since consensus is rarely achieved in the real world, it has motivated considerable efforts in understanding conditions that lead to opinion fragmentation in networks, with special attention to competitive settings \cite{hucompeting,prakash2012winner}. In such settings, controllers aim to exceed the competitor's share of opinions in the network at equilibrium.

Competitive opinion propagation has been extensively explored in the voter model using various forms of zealotry \cite{acemouglu2013opinion,mobilia2007role,yildiz2013binary}. Zealots are either biased individuals \cite{masuda2010heterogeneous,mobilia2003does}; radical agents impervious to influence from neighbours  \cite{barrat2008dynamical,kuhlman2013controlling,mobilia2007role,mobilia2015nonlinear} or external controllers unidirectionally targeting the network \cite{masuda2015opinion}. In this paper, we adopt a setting similar to that in \cite{masuda2015opinion}. However, in contrast to the norm of discretely targeting networks (where the most optimal seed set is identified), here we allow a continuous distribution of influence that targets nodes with varying intensities \cite{chakraborty2019competitive}. This modification widens the scope of application of this work to continuous resources such as money and time, while also significantly increasing the tractability of the optimisation problem.
 
 Much like zealots, contrarians too have been used to study competitive influence spread in the voter model \cite{gambaro2017influence,li2011strategy,masuda2013voter,zhong2005effects}. These are nonconformist agents opposing the majority view in their social neighbourhood. 
 We emulate this notion of disagreement in our work as a property of social relations, as opposed to that of agents, which is more realistic and a more general case of the contrarian model. 

%  \subsection{Competitive opinion dynamics in signed networks}

  As largely reflected in society and real-world networks, social ties can often be negative, representing distrust and enmity \cite{leskovec2010signed}.
 So far, networks with signed edges have received limited attention within the vast body of work on opinion dynamics (see \cite{girdhar2016signed} for a detailed review). Most of this existing work explores the problem in threshold models \cite{hosseini2019assessing}. Principally, greedy heuristics are used in models of diffusion, such as the Independent Cascade (IC) \cite{ju2020new,li2014polarity,liu2019influence} and the Linear Threshold (LT) models \cite{he2019information,liang2019influence,shen2015influence}. 
 In addition, a slightly modified integrated page-rank algorithm for seed selection in signed networks was presented in \cite{chen2015influence}, while authors used simulated annealing for the same problem in \cite{li2017positive}. Work in \cite{srivastava2015social} discusses a comparative analytical approach to the same problem. Other generalised models of opinion propagation have also been used \cite{jendoubi2016maximizing}, including the voter model, where controllers seed the network optimally to maximise vote-shares at the steady-state \cite{li2013influence}. 
 
 
 Our paper significantly extends the work in \cite{li2013influence} by framing the problem in more realistic, competitive and game-theoretic settings, which is different from the traditional single-controller influence maximisation problem studied in \cite{li2013influence}. We also further generalise the (discrete) optimisation problem presented in \cite{li2013influence} by allowing flexible distribution of influence in our model. Finally, we assume persistent influence from external controllers which diversifies the seeding process adopted in \cite{li2013influence}.
\section{The opinion dynamics model}
\label{Model}
%\subsection{Voting dynamics in signed networks}
We consider a population of $N$ individuals interacting with each other in a social network through congenial (positive) or hostile (negative) relationships. We represent the structure of the network using a signed graph $G(V,E,W)$, where vertices $V = \{1,2,\ldots,N\}$ denote individuals in the population connected through a set of edges $E$ that depict social connections. Here $W \in R^{N \times N}$ denotes the corresponding signed weighted adjacency matrix, where any given element $w_{ij}$ illustrates the strength of influence an individual $i$ has on $j$.
The weight of an edge $w_{ij}$ determines the type of influence, positive or negative, that flows from $i \longrightarrow j$. A negative weight $w_{ij} < 0$ symbolises a negative edge and therefore implies negative influence from $i$ on $j$, whereas $w_{ij} > 0$ suggests node $j$ experiences positive influence from $i$. We consider directed networks, where edge weights $w_{ij} $ and $ w_{ji}$  are independent of each other, and it is possible that nodes experience different strengths and types (positive or negative) of influence from each other, i.e. $w_{ij} \neq w_{ji}$. 

We study the propagation of binary opinion states, A and B in the network, imposed by external controllers. 
Here we assume that controllers strictly exert positive influence on the network and external influence is expressed in terms of resource distribution vectors $\{p_{A},p_{B}\} \in \mathbb{R}^{N}_+$, where any element $p_{A,i} \geq 0$ (or $p_{B,i} \geq 0$) represents the strength of influence exerted by controller A (or B) on node $i$. The vectors are constrained linearly by the budget available to each controller, i.e. $\sum_{i}p_{A,i} = B_{A}$ (or $\sum_{i}p_{B,i} = B_{B}$). Unlike traditional models that assume a single-injection of influence at the start of the dynamics \cite{kempe2003maximizing,li2013influence}, here influence is applied continuously until the system reaches steady-state.

At any given point in time, individuals in the network strictly conform to either opinion states at a rate proportional to the strength of influence experienced by them. We assume opinion propagation follows voter dynamics, where at each time step a node $i$ chosen uniformly at random from the network updates its state by picking a source of influence (social neighbours or external controllers) to copy (or oppose in cases of negative strength). Node $i$ picks a neighbour $j$ with a probability $|w_{ji}|/(\textstyle\sum_{j \in {\cal N}_{i}^+ \cup {\cal N}_{i}^-} w_{ji} + p_{A,i} + p_{B,i})$\footnote{${\cal N}_{i}^+$ and ${\cal N}_{i}^-$ are the sets of positive and negative neighbours of node $i$.}, and either copies their state (when $w_{ji}>0$), or opposes them ($w_{ji}<0$). Similarly, node $i$ picks an external controller (say A) to copy, with the probability $p_{A,i}/(\textstyle\sum_{j \in {\cal N}_{i}^+ \cup {\cal N}_{i}^-} w_{ji} + p_{A,i} + p_{B,i})$. 

% with probability $|w_{ji}|/(\textstyle\sum_{j \in {\cal N}_{i}^+ \cup {\cal N}_{i}^-} w_{ji} + p_{A,i} + p_{B,i})$ and $p_{A,i}/(\textstyle\sum_{j \in {\cal N}_{i}^+ \cup {\cal N}_{i}^-} w_{ji} + p_{A,i} + p_{B,i})$ respectively.  Here ${\cal N}_{i}^+$ and ${\cal N}_{i}^-$ are the sets of positive and negative neighbours of the node $i$. 
% When a neighbour $j$ is picked, node $i$ copies the opinion state of $j$ if the edge from $j$ to $i$ is positive ($w_{ji}>0$). Conversely, $i$ adopts the opposing view if the edge weight is negative ($w_{ji}<0$). If a node picks one of the controllers, it strictly copies their opinion state as all external allocations to the network are positive. 




Assuming $x_{A,i}$ (or $x_{B,i} = 1 - x_{A,i}$) characterise the probability with which a node $i$ adopts opinion state A (or B), the rate at which $i$ transitions to opinion A is given by,
 
\begin{equation}
\frac{dx_{A,i}}{dt}=(1-x_{A,i}) \cdot \phi_{i}(A) -x_{A,i} \cdot \phi_{i}(B).
\label{Eq:1}
\end{equation}

The terms $\phi_{i}(A)$ and $\phi_{i}(B)$ indicate the fraction of total influence experienced by $i$ in favour of opinions A and B respectively and are given by,


\begin{align*}
\begin{array}{cc}
     & \phi_{i}(A) = \frac{\sum\limits_{j \in {\cal N}_i^+} w_{ji}x_{A,j}-\sum\limits_{j \in {\cal N}_i^-} w_{ji}(1-x_{A,j})+p_{A,i}}{\sum\limits_{j \in {\cal N}_i^+} w_{ji}-\sum\limits_{j \in {\cal N}_i^-} w_{ji}+p_{A,i}+p_{B,i}}; \\ \\
     & \phi_{i}(B) = \frac{\sum\limits_{j \in {\cal N}_i^+} w_{ji}(1-x_{A,j})-\sum\limits_{j \in{\cal N}_i^-} w_{ji}x_{A,j}+p_{B,i}}{\sum\limits_{j \in{\cal N}_i^+} w_{ji}-\sum\limits_{j \in {\cal N}_i^-} w_{ji}+p_{A,i}+p_{B,i}}.
\end{array}
\end{align*}
Here, A is our chosen focal controller. Similar expressions can be derived for controller B.

In signed networks, nodes experience both positive and negative influence from their neighbours. The collective positive influence is given by ${\displaystyle\sum}_{j \in {\cal N}_{i}^+} w_{ji}x_{A,j}$, and the total strength of negative influence (from neighbours in state B) is ${\displaystyle\sum}_{j \in {\cal N}_{i}^-} w_{ji}(1-x_{A,j})$.    
Edge weights $w_{ji}$ refer to incoming edges and allocations from external controllers A and B on node $i$ are $p_{A,i}$ and $p_{B,i}$ respectively. 

Given the above, the steady-state equation for the system can be evaluated by replacing $\frac{dx_{A,i}}{dt}=0$ in \cref{Eq:1} to obtain

\begin{equation}
   x_{A,i}^* = \frac{p_{A,i} - \sum\limits_{j \in {\cal N}_i^-}w_{ji} + \sum\limits_{j \in {\cal N}_{i}^+}w_{ji} x_{A,j} + \sum\limits_{j\in {\cal N}_{i}^-}w_{ji} x_{A,j}}{\sum\limits_{j\in {\cal N}_{i}^+}w_{ji} - \sum\limits_{j\in {\cal N}_{i}^-}w_{ji} + p_{A,i} + p_{B,i}}. 
    \label{steady-state}
\end{equation}

Here, $x_{A,i}^*$ is the probability a node $i$ has opinion state \emph{A} at equilibrium. 
For a network of size $N$ nodes, we obtain a system of $N$ equations, which can be given as follows, 

\begin{align*}
\quad
\begin{bmatrix}
L+diag(p_{A}+p_{B})
\end{bmatrix}
x_{A}^* &=
p_{A}-\vec{1}^T W^-,
\end{align*}
\begin{align}
\implies    X_{A} &= \frac{1}{N}\Vec{1}^{T} {x_{A}^*} = \frac{1}{N} \Vec{1}^{T}  [L+diag(p_{A}+p_{B})] ^{-1}(p_{A}-\vec{1}^T W^-).
    \label{optimisation}
\end{align}

$X_{A}$ in \cref{optimisation} denotes the total vote-share obtained by controller A at equilibrium. Assuming $W^+$ and $W^-$ are the weighted adjacency matrices of the positive and negative components of the network, the vector $\vec{1}^T W^-$ captures the total strength of negative influence on each node in the network. Here $L$ is an $N \times N$ matrix given by $L = diag(\vec{1}^T(W^+ - W^-)) - (W^+ + W^-)$. The diagonal elements represent the absolute sum of all edge weights of a node $i$, given by $L_{ii} = {\displaystyle\sum}_{j \in {\cal N}_{i}^+} w_{ji} - {\displaystyle\sum}_{j \in {\cal N}_{i}^-} w_{ji}$ and off-diagonal elements are $L_{ij} = -w_{ji}$. For unweighted graphs $L = diag(D) - (A^+ - A^-)$, where $D$ is the degree-vector\footnote{diag(D) is an $N \times N$ matrix where the diagonal elements capture the degrees of nodes in the network.}. $A^+$ and $A^-$ are the respective adjacency matrices of the positive and negative components. Note that since $[L + diag(p_{A} +p_{B})]$ is diagonally-dominant, it is invertible and we can therefore use \cref{optimisation} to determine solutions for $X_{A}$. 

The formal optimisation problem can then be stated as

\begin{align}
    % maximise \quad &\frac{1}{N} \Vec{1}^{T}  [L+diag(p_{A}+p_{B})] ^{-1}(p_{A}+\vec{1}^T W^-),\\
    % subject 
    % %\begin{array}{l}
    % &\text{maximise} \quad \frac{1}{N} \Vec{1}^{T}  [L+diag(p_{A}+p_{B})] ^{-1}(p_{A}+\vec{1}^T W^-)\\
    % &\text{subject to} \quad \sum\limits_{i=1}^N p_{A,i} = B_{A},\\
    % & \qquad 0 \leq p_{A,i} \leq B_{A}.
    % \end{array}
    p_{A}^* = \text{argmax}_{p_{A} \in \mathcal{P}} \quad X_{A}^*(L,p_{B}),
    \label{formal_opti}
\end{align}
where $\mathcal{P}$ is a set of all possible allocations $p_{A}$ such that $0 \leq p_{A,i} \leq B_{A}$ ($\forall i \in \{1,2,\ldots,N\}$) and $\sum\limits_{i=1}^N p_{A,i} = B_{A}$.


For a passive opponent B (where $p_{B}$ is fixed and known), controller A maximises their opinion shares using Eq. \ref{formal_opti}. 
Closed-form solutions can be readily obtained in networks with simplified structures (e.g. star networks) by solving the set of equations outlined in \cref{optimisation}. To do this, we first determine the gradient $\nabla_{p_{A}} X_{A}$ by differentiating \cref{optimisation} wrt to the allocation vector $p_{A}$ as

\begin{align}
    \nabla_{p_{A}} X_{A} = \frac{1}{N} \Vec{1}^{T} [L+diag(p_{A}+p_{B})] ^{-1} (I - diag(x_{A})),
    \label{gradient}
\end{align}

and then solve $\nabla_{p_{A}} X_{A}=0$ to obtain the optimal allocation $p_{A}^*$ that yields maximum vote-shares $X_{A}^*$. However, obtaining analytical solutions for $\nabla_{p_{A}} X_{A}=0$ in larger, more complex networks can be considerably challenging. In which case we use local search techniques such as gradient ascent (as they have been shown to work well in similar settings \cite{lynn2016maximizing,romero2021shadowing}) to determine optimal allocations in arbitrary networks. More specifically, we employ the gradient algorithm proposed in \cite{romero2021shadowing}, by first modifying it to fit our purpose. Note that the problem setting considered in \cite{romero2021shadowing} resembles our research problem closely, as both studies explore the competitive influence maximisation problem in the voter model under continuous allocations, with the exception that the model given in \cite{romero2021shadowing} only applies to unsigned (or positive) networks. This yields a slightly different expression for vote-shares and does not contain the term $-\vec{1}^T W^-$ (as in \cref{optimisation}). Despite this difference, the expression for gradient is the same as \cref{gradient}, given that $-\vec{1}^T W^-$ is independent of controller allocations $p_{A}$. Thus we replace the expression for vote-shares in the algorithm (from \cite{romero2021shadowing}) with \cref{optimisation} and then use it to obtain optimal allocations. We label this approach $GA$. 


% Given that everything else is identical across both settings, we can employ the algorithm in \cite{romero2021shadowing} to obtain optimal allocations for our setting, simply by replacing the expression for vote-shares with \cref{optimisation}. We label this approach as $GA$. 


%  to incrementally update the allocation vector $p_{A}$ (with step length $\eta$), initialised as a uniformly distributed random vector $p_{A}^{(0)}$ such that $\sum_{i} p_{A,i}^{(0)} = B_{A}$, until a $\mu$-approximated optimal allocation $p_{A}^*$ is obtained, for any given budget $B_{A}$, opponent strategy $p_{B}$ and network structure $L$. The budget constraint is systematically imposed at each step, by projecting the resulting allocation vector $p_{A}^{(t+1)}$ onto an $N$-simplex using the algorithm in \cite{chen2011projection}, such that $\sum_{i} p_{A,i} = B_{A}$. The learning parameter $\eta$ is adjusted using exact-line search to ensure convergence \cite{boyd2004convex}.
% We find that $X_{A}(p_{A})$ is concave (see \cref{appendix-convexity}), implying that the local maximum reached in this case is also the global maximum.

% \begin{algorithm}
% \SetAlgoLined
% \textbf{Input : }$p_{B},L,B_{A},\eta,\mu$\;
% \KwResult{$p_{A}^*$}
% initialise t=0, $p_{A}^{(0)}$, $\eta = N$ \;
%  \While{$p_{A}^{(t+1)} - p_{A}^{(t)} > \mu$}{
%   $p_{A}^{(t+1)} = \mathcal{P}[p_{A}^{(t)} + \eta \nabla_{p_{A}^{(t)}}X_{A}]$\; 
  
%   \If{$X_{A}^{(t+1)} - X_{A}^{(t)} < 0$}{
%   $\eta^{(t+1)} = \frac{\eta^{(t)}}{2}$\;
%   $p_{A}^{(t+1)} = p_{A}^{(t)}$;
%   }
%   t++\;
%  }
 
%  \caption{$\mu$-approximation of optimal allocation vector using gradient ascent}
%  \label{GA-algo}
% \end{algorithm}

In order to determine the effectiveness of the negative-tie aware method, we compare it against the algorithm that considers all edges to be positive, therefore discounting any negative influences in the network dynamics, when solving the influence maximisation problem. We then measure the impact of taking a negative-tie aware approach on vote-shares and quantify the gain or loss. There are two broad cases where controllers might be negative-tie agnostic: (i) where they cannot observe negative ties (in cases of under-representation) or (ii) where mistake them for positive edges (misrepresentation) \cite{li2013influence}. For the sake of simplicity, we avoid considering instances where controllers can partially observe negative edges in the network.  


We are keen to determine instances where a controller would gain the most from adopting a negative-tie aware approach and we find that misrepresenting negative-ties consistently performs worse as a strategy in comparison to under-representation of negative-ties (see \cref{appendix-rem} for more details). Thus in the rest of the paper, we compare our proposed method to the approach where negative ties are misrepresented (i.e. all edges are assumed to be positive)\footnote{In undirected networks, both approaches \textemdash misrepresentation and under-representation of negative ties \textemdash would yield equal vote-shares.}.
% A more compelling reason for this choice is that negative ties largely present themselves as neutral or positive edges in online social networks which can be detected only through additional analysis \cite{bae2012sentiment,leskovec2010predicting}. 

%\subsection{Voting dynamics in networks assuming strictly positive ties}
To determine an expression for vote-shares in networks where all edges are positive, we modify \cref{steady-state} that yields an expression for vote-shares, given by $X_{A}^{(+)}$ (identical to the expression for vote-shares in \cite{romero2021shadowing}). We then directly apply the algorithm in \cite{romero2021shadowing} to obtain optimal allocations $p_{A}^*$. For ease of notation, we label this approach $GA^{(+)}$.
% Using a framework inspired by \cite{masuda2015opinion}, we obtain an expression for vote-shares at equilibrium given by,

% \begin{align}
%      x_{A,i}^{*(+)} &= \frac{p_{A,i} + \sum\limits_{j \in {\cal N}_{i}}w_{ji} x_{A,j}^{(+)} }{\sum\limits_{j \in {\cal N}_{i}}w_{ji} + p_{A,i} + p_{B,i}},\\
%      \implies X_{A}^{(+)} &= \frac{1}{N}\Vec{1}^{T} {x_{A}}^{(+)} = \frac{1}{N} \Vec{1}^{T} p_{A} [L^{(+)}+diag(p_{A}+p_{B})] ^{-1}.
%      \label{opti-pos}
% \end{align}

% Here $X_{A}^{(+)}$ indicates the final vote-share obtained by a controller to whom all edges appear strictly positive. $L^{(+)}$ is the $N\times N$ laplacian matrix of the directed graph given as $L = diag(\vec{1}^T W) - W$, where W is the corresponding weighted adjacency matrix. Allocation vectors of both controllers remain unchanged and are given as $p_{A}$ and $p_{B}$. The optimisation problem in this case can be defined as,

% \begin{align}
%     p_{A}^* &= \text{argmax}_{p_{A}} X_{A}^{*(+)}(L^{(+)},p_{B},B_{A}).
% \end{align}
% The budget constraints apply as before.

% We find that the gradient $\nabla X_{A}^{(+)}$ in this variant is similar to \cref{gradient}, and is given by $\nabla_{p_{A}} X_{A}^{(+)} = 1/N \Vec{1}^{T} [L^{(+)}+diag(p_{A}+p_{B})]^{-1}(I - diag(x_{A}^{(+)})$. The proof of concavity is also preserved and once again we can employ gradient ascent steps $GA^{(+)}$ to arrive at a global optimum $p_{A}^{*}$, for a given budget ($B_{A}$), adversarial allocation ($p_{B})$ and unsigned network ($L^{(+)}$).
\section*{Results}
We train a CLIP-based joint vision-language learning framework (see Developing a CLIP framework using \dc\ in the Methods section) using the CT-RATE dataset, consisting of 50,188 non-contrast 3D chest CT volumes paired with corresponding radiology text reports. For the text encoder, we use CXR-BERT \cite{boecking2022making}. For the image encoder, we comprehensively evaluate our proposed \dc\ architecture by benchmarking its zero-shot performance against several state-of-the-art models, including ViT \cite{dosovitskiy2020image}, CT-ViT \cite{hamamci2025generatect, hamamci2024foundation}, TransUNet \cite{chen2021transunet}, ConvNeXt \cite{liu2022convnet}, InceptionNeXt \cite{yu2024inceptionnext}, and PoolFormer \cite{yu2022metaformer} (see SOTA image encoders in the Methods section).
\begin{table}[ht!]
\centering
\caption{\textbf{Super Resolution Performance Results.} Our proposed WGAN EEG Spatial Upsampling method significantly outperforms a baseline of Bicubic Interpolation commonly used in EEG upsampling pipelines.}
\label{tab:results}
\resizebox{0.8\linewidth}{!}{%
\begin{tabular}{@{}cccccc@{}}
\toprule
\multirow{2}{*}{\textbf{Dataset}} & \multirow{2}{*}{\textbf{Scale}} & \multicolumn{2}{c}{\textbf{Bicubic}} & \multicolumn{2}{c}{\textbf{WGAN}} \\ \cmidrule(l){3-6} 
                      &   & \textbf{MSE} & \textbf{MAE} & \textbf{MSE}    & \textbf{MAE}   \\
\toprule
\multirow{2}{*}{Val}  & 2 & 3.71E7       & 3.89E3       & \textbf{2.01E3} & \textbf{24.38} \\
                      & 4 & 7.23E7       & 6.42E3       & \textbf{8.53E3} & \textbf{63.83} \\
\midrule
\multirow{2}{*}{Test} & 2 & 3.75E7       & 3.91E3       & \textbf{2.06E3} & \textbf{24.66} \\
                      & 4 & 7.30E7       & 6.45E3       & \textbf{8.68E3} & \textbf{64.39} \\
\bottomrule
\end{tabular}%
}
\end{table}


\subsection*{Zero-shot multi-abnormality detection results}
Once trained to maximize the similarity between image and text embeddings, the CLIP framework enables multi-abnormality detection by inputting each abnormality as a text prompt (see *Zero-shot multi-abnormality detection with \dc* in the Methods section). Specifically, we use the prompts "{Abnormality} is present." and "{Abnormality} is not present." for each of the 18 distinct abnormalities, following the ablation study by Hamamci et al. \cite{hamamci2024foundation}. We then compute the normalized probability of each abnormality being present in the CT image and evaluate model performance using accuracy, F1 score, precision, and recall.

Table \ref{results} compares the zero-shot performance and computational efficiency of multiple models on the CT-RATE dataset for image volumes of size 512×512×256. \dc\ provides a family of models with configurations ranging from nano to tiny variants, with parameter counts between 920.3K and 15.1M and GFLOPs spanning 34.21G to 168.2G. As shown in Table \ref{results}, DCFormer consistently achieves superior accuracy, F1 score, precision, and recall across all configurations while maintaining computational efficiency.

For instance, the nano variant achieves higher F1 score, precision, and recall than ConvNeXt and PoolFormer, despite having fewer parameters and similar FLOPs. Similarly, the naïve variant outperforms ConvNeXt, PoolFormer, and TransUNet, demonstrating the efficiency and effectiveness of DCFormer's architecture. Notably, DCFormer achieves these results with substantially fewer computational resources compared to models like TransUNet and CTViT, which require much larger parameter counts and FLOPs. For example, the naive variant of \dc\ delivers 63.1\% accuracy, 44.5\% F1 score, 29.5\% precision, and 65.5\% recall using just 5.85M parameters and 49.48 GFLOPs—far fewer parameters and similar FLOPs compared to CTViT’s 101.1M parameters and 160.5 GFLOPs and TransUnet's 23.93M parameters and 207.5 GFLOPs.

This efficiency enables DCFormer to achieve robust performance with a simple yet effective architecture, demonstrating its scalability and suitability for applications requiring both high accuracy and computational efficiency. Overall, for zero-shot implementation, \dc\ outperforms other models in efficiency, with its lightweight variants achieving competitive results while significantly reducing computational overhead. DCFormer’s ability to outperform SOTA models with fewer parameters and FLOPs highlights its potential for applications requiring both high performance and computational efficiency.


% [old version] Once the CLIP framework is trained to maximize the similarity between the embeddings of image and text encoders, it is straightforward to perform zero-shot multi-abnormality detection by inputting each abnormality as a text prompt (see Zero-shot multi-abnormality detection with \dc\ in Methods). Specifically, we utilize '\{\textit{Abnormality}\} is present.' and '\{\textit{Abnormality}\} is not present.' prompts for each of the 18 distinct abnormalities following the ablation study of Hamamci et al.\cite{hamamci2024foundation}. Finally, we compute the normalized probability for each abnormality being present on the CT image and compare the accuracy, f1 score, precision and recall scores with the previous SOTA methods.

% Table.\ref{results} compares the zero-shot performance and computational efficiency of various models, including \dc\, and other SOTA methods including ConvNeXt, PoolFormer, ViT, and TransUNet on the CT-RATE dataset at a resolution of 512×512×256. \dc\ offers a family of models with configurations ranging from nano to tiny variants, with parameter counts spanning from 920.3K to 15.1M and GFLOPs ranging from 34.21G to 168.2G. As shown in Table.\ref{results}, DCFormer consistently outperforms other models in terms of accuracy, F1 score, precision, and recall across all variants, while maintaining significantly fewer parameters and FLOPs. For instance, the nano variant achieves higher accuracy, F1 score, precision, and recall than ConvNeXt and PoolFormer, despite having fewer parameters and similar FLOPs. Similarly, the naïve variant achieves superior performance compared to ConvNeXt, PoolFormer and TransUnet, demonstrating the efficiency and effectiveness of DCFormer's architecture. Note that, DCFormer achieves these results with substantially fewer computational resources compared to models like TransUNet and CTViT, which require much larger parameter counts and FLOPs. For example, the tiny variant delivers 64.0\% accuracy, 46.2\% F1 score, 29.9\% precision, and 69.8\% recall using just 15.1M parameters and 168.2G FLOPs, far fewer parameters and similar FLOPs compared to CTViT’s 101.1M parameters and 160.5 GFLOPs and TransUnet's 23.93M parameters and 207.5 GFLOPs.  Such ability of DCFormer delivers a robust performance with a simple and efficient architecture, showing its scalability and suitability for applications that require both high accuracy and computational efficiency.

% Overall, for zero-shot implementation, \dc\ outperforms other models in terms of efficiency, with its lightweight variants achieving competitive results while significantly reducing computational overhead. DCFormer's ability to outperform state-of-the-art models with fewer parameters and FLOPs highlights its potential for applications requiring high performance and computational efficiency.

%The \dc\ naïve variant achieves an accuracy of 66.2\% and an F1 score of 45.2\%, outperforming all ViT variants. Specifically, ViT naïve achieves 55\% accuracy and an F1 score of 42.5\%, while ViT tiny and small variants achieve 61\% and 62.8\% accuracy with F1 scores of 43.2\% and 44.5\%, respectively. Despite its superior performance, \dc\ is significantly more efficient, requiring only 6.20M parameters and 139.3G FLOPs. In comparison, ViT naïve uses 11.10M parameters and 39.05G FLOPs, ViT tiny requires 26.34M parameters and 86.43G FLOPs, and ViT small demands 45.81M parameters and 142.5G FLOPs.

%Compared to CTViT, which achieves 63\% accuracy and an F1 score of 44.2\%, the \dc\ naïve variant not only delivers better performance but also greatly enhances efficiency. CTViT requires 101.1M parameters and 160.5G FLOPs, whereas \dc\ achieves superior results with 16 times fewer parameters and comparable computational cost ... More results... 



 %\subsection*{Fine-tuning multi-abnormality detection results}
 %In this section, we evaluate the \dc\ image encoder by comparing its performance to other SOTA models through a fine-tuning approach. While zero-shot evaluation has the advantage of requiring no labeled data, fine-tuning offers significant improvements in classification performance by allowing the model to adapt to specific downstream tasks. Here, we extend the trained encoders with a single classification layer and fine-tune them in a supervised manner to classify 18 abnormalities. Binary Cross-Entropy (BCE) loss is used to optimize the models which enables independent probability predictions for each class rather than multi-class predictions.
 
 %\subsection*{Retrieval Tasks}


\section{Discussion}
\label{Summary}
The study of opinion dynamics has been conventionally done on networks with strictly positive edges. In the real world however, networks often contain negative social connections, which can spread negative or opposing influence, thus creating a need to understand how these edges affect influence maximisation efforts in networks. 
To address this concern, we present a model for competitive spread of opinions in signed networks under voter dynamics. For comparison, we propose a complementary approach where controllers only observe the absolute weights of all edges i.e. they consider all edges to be positive. In both instances we present gradient ascent algorithms to numerically solve the problem in large-scale arbitrary networks. We test the robustness of our results in networks of varied structures under diverse budget conditions and adversarial allocations. 
We find that in networks where 20\% of edges are negative, controllers gain maximally (nearly 18\%) from awareness of negative edges, under conditions of scarce resources, and against competitors who deliberately avoids nodes with negative connections. 

We also propose a supporting theoretical approach to verify the accuracy of our algorithms. We present closed-form solutions in simplified network structures that provide further insights to the problem. We observe that in networks with highly concentrated positive links, allocations on nodes are driven by their negative degrees and the competitor's allocation on these nodes. Finally, we examine the problem under game-theoretic settings, where we highlight conditions under which a controller could lose vote-shares by implementing strategies that use the knowledge of negative ties in the network. Specifically, we show that when controllers have considerably less resources (or in some cases, excess budget), their prioritisation of nodes to target, may inadvertently disclose knowledge of negative ties to a competitor who was otherwise unaware, thus compromising their position of advantage.

The results in this paper present compelling evidence for considering negative ties in any influence maximisation exercise and thus contributes to the literature on competitive opinion dynamics in signed networks. Possible extensions to this work could include studying the problem under different constraint functions. For instance, the effect of modified budget constraints that explore the implications of an additional cost to retrieve information about the presence of negative ties, on influence maximisation efforts.
Additionally, this problem could be further studied in other realistic opinion models (e.g. Deffaunt model). 
% Going forward, this work could also serve as a foundation to guide empirical investigation on maximising opinion spread in the presence of negative edges.    



% We now propose an analytical framework in support of our numerical results. Note that, obtaining closed-form analytical solution for \cref{optimisation} on networks with inherent complexities can be challenging. We therefore simplify the problem first by adopting a degree-based mean-field approach that approximates system dynamics and helps us obtain analytical expressions for optimal al

\section{Acknowledgments}
We would like to thank Kostis Kaffes, Tanvir Ahmed Khan, and Yuhong Zhong for their feedback.
We also thank the CloudLab team for their help in supporting our experiments.
This work was supported by IBM, and NSF awards CNS-2143868 and CNS-2106530.
Tal Zussman was supported by NSF award DGE-2036197.
Ioannis Zarkadas is an Onassis Foundation scholar.

\section*{Appendix}
\appendix
\numberwithin{equation}{section}

\section{Removing negative ties}
\label{appendix-rem}
When controllers are unable to detect or observe negative edges in the network, i.e. $w_{ij}=max(0,w_{ij})$, the optimisation problem reduces to
\begin{align}
    p_{A}^* &= \text{argmax}_{p_{A}} X_{A}^{*(\phi)}(L^{(\phi)},p_{B},B_{A}).
\end{align}

where $L^{(\phi)}$ is the updated Laplacian. 
Following the same process as before we use the gradient $\nabla_{p_{A}} X_{A}^{(\phi)} = 1/N \Vec{1}^{T} [L^{(\phi)}+diag(p_{A}+p_{B})]^{-1}(I - diag(x_{A}^{(\phi)})$ to optimise allocations $p_{A}^*$ in a gradient ascent algorithm $GA^{(\phi)}$. Here $    x_{A,i}^{*(\phi)} = (p_{A,i} + \sum\limits_{j}^{k_a}w_{ji} x_{A,j}^{(\phi)}) / (\sum\limits_{j}^{k_a}w_{ji} + p_{A,i} + p_{B,i})$.


We then run $GA$, $GA^{(+)}$ and $GA^{(\phi)}$ on the Bitcoin network and present the respective gains in vote-shares in \cref{rem}.


  \begin{figure}
  \centering
    \includegraphics[width=0.7\textwidth]{figures/Appendix_A.eps}
    \caption{Figure showing gain in vote-shares when comparing the negative-tie sensitive optimisation approach $GA$, to traditional approaches, $GA^{(+)}$ and $GA^{(\phi)}$ for budget ratios $B_{A}/B_{B} \in [0.05,1]$. Controller B here targets the network uniformly.}
    \label{rem}
  \end{figure}

We find that the method assuming negative edges in the network to be positive $GA^{(+)}$ consistently outperforms $GA^{(\phi)}$, where negative edges are not considered at all. 
To further show that the vote-shares obtained through both methods are identical in undirected networks (and that comparing our results to only $GA^{(+)}$ is sufficient), we look at the allocation expression in \cref{pos-allo}. Here we find that the optimal allocation, in the absence of any knowledge of negative edges, depends solely on the individual budgets of the controllers and the adversarial allocations on the nodes, but not on the degrees of the nodes.   

\section{Limiting case}
\label{appendix-A}
We begin with the series expansion of the steady-state equation in \cref{XA-MF} to obtain,
\begin{align}
    X_{A} = \langle \frac{a_{k_{a}k_{b}}}{k_{a}} \rangle + \langle \frac{k_{b}}{k_{a}} \rangle + \langle 1 - \frac{a_{k_{a}k_{b}}+b_{k_{a}k_{b}}+k_{b}}{k_{a}} \rangle  \langle x_{a} \rangle - \langle \frac{k_{b}}{k_{a}} \rangle \langle x_{b} \rangle,
    \label{x-exp}
\end{align}
where a second-order expansion for $\langle x_{a} \rangle$ gives us, 
\begin{align}
    \langle x_{a} \rangle = \frac{a+\langle k_{b} \rangle - \langle \frac{(a+k_{b})(a+b+2k_{b})}{k_{a}} \rangle }{a+b+2\langle k_{b} \rangle} + \frac{a+\langle k_{b} \rangle }{(a+b+2k_{b})^2}\Bigg( \langle \frac{k_{b}^2}{k_{a}} \rangle + \langle \frac{(a+b+k_{b})(a+b+3k_{b})}{k_{a}} \rangle \Bigg),
    \label{xa-exp}
\end{align}
and a zero-order expansion for $\langle x_{b} \rangle$ gives us
\begin{align}
  \langle x_{b} \rangle = \frac{a+\langle k_{b} \rangle}{a+b+\langle k_{b} \rangle}.
  \label{xb-exp}
\end{align}

Finally, replacing \cref{xa-exp,xb-exp} in \cref{x-exp} and ignoring higher-order terms, we obtain
\begin{equation}
\begin{split}
    X_{A} \approx \langle \frac{a_{k_{a}k_{b}}}{k_{a}} \rangle + \langle \frac{k_{b}}{k_{a}} \rangle + \frac{a_{k_{a}k_{b}}+\langle k_{b} \rangle}{a_{k_{a}k_{b}}+b_{k_{a}k_{b}}+2\langle k_{b} \rangle} - \frac{\langle \frac{(a_{k_{a}k_{b}}+k_{b})(a_{k_{a}k_{b}}+b_{k_{a}k_{b}}+2k_{b})}{k_{a}} \rangle}{a_{k_{a}k_{b}}+b_{k_{a}k_{b}}+2\langle k_{b} \rangle} \\ + \frac{a_{k_{a}k_{b}}+\langle k_{b} \rangle}{(a_{k_{a}k_{b}}+b_{k_{a}k_{b}}+2\langle k_{b} \rangle)^2}\Bigg( \langle \frac{k_{b}^2}{k_{a}} \rangle  + \langle \frac{(a_{k_{a}k_{b}}+b_{k_{a}k_{b}}+k_{b})(a_{k_{a}k_{b}}+b_{k_{a}k_{b}}+3k_{b})}{k_{a}} \rangle \\ 
    - \langle \frac{a_{k_{a}k_{b}}+b_{k_{a}k_{b}}+k_{b}}{k_{a}} \rangle \frac{a_{k_{a}k_{b}}+\langle k_{b} \rangle }{a_{k_{a}k_{b}}+b_{k_{a}k_{b}}+2\langle k_{b}\rangle} - \langle \frac{k_{b}}{k_{a}} \rangle  \frac{a_{k_{a}k_{b}}+\langle k_{b}\rangle}{a_{k_{a}k_{b}}+b_{k_{a}k_{b}}+\langle k_{b}\rangle}  \Bigg).
\end{split}
\label{app-X}
\end{equation}
Note that all terms in \cref{app-X} are averaged over the joint positive and negative degree distribution $P_{k_{a}k_{b}}$.

We can now apply the Lagrange method to maximise vote-shares $X_{A}$ against a passive controller B. The Lagrangian is derived as $\mathcal{L} = X_{A} + \lambda (\sum_{k_{a}k{b}} P_{k_{a}k_{b}} a_{k_{a}k_{b}} - \langle a_{k_{a}k_{b}} \rangle N) $, where $\lambda$ is the Lagrangian multiplier. Differentiating $\mathcal{L}$ wrt allocations $a_{k_{a}k_{b}}$ we obtain

\begin{equation}
    \begin{split}
    \frac{\partial \mathcal{L}}{\partial a_{k_{a}k_{b}}} = \frac{1}{k_{a}} - \frac{2a_{k_{a}k_{b}}+b_{k_{a}k_{b}}+3k_{b}}{\langle a_{k_{a}k_{b}} \rangle+ \langle b_{k_{a}k_{b}} \rangle +2\langle k_{b} \rangle} \frac{1}{k_{a}} + 2 \frac{\langle a_{k_{a}k_{b} \rangle}+\langle k_{b} \rangle }{(\langle a_{k_{a}k_{b}} \rangle + \langle b_{k_{a}k_{b} \rangle}+2\langle k_{b} \rangle)^2} \frac{a_{k_{a}k_{b}}+b_{k_{a}k_{b}}+2k_{b}}{k_{a}} \\
    - \frac{\langle a_{k_{a}k_{b}} \rangle +2\langle k_{b} \rangle}{\langle a_{k_{a}k_{b}}\rangle + \langle b_{k_{a}k_{b}} \rangle+2\langle k_{b} \rangle} \frac{1}{k_{a}} + \lambda = 0.
    \end{split}
\end{equation}

Solving for $a_{k_{a}k_{b}}$ gives us,

\begin{equation}
    \begin{split}
        \implies a_{k_{a}k_{b}} = \frac{1}{2} \Bigg( \frac{\langle a_{k_{a}k_{b}} \rangle - \langle b_{k_{a}k_{b}}\rangle}{\langle b_{k_{a}k_{b}} \rangle +\langle k_{b} \rangle} b_{k_{a}k_{b}} + \frac{\langle a_{k_{a}k_{b}} \rangle - 3\langle b_{k_{a}k_{b}} \rangle -2 \langle k_{b} \rangle}{\langle b_{k_{a}k_{b}} \rangle +\langle k_{b} \rangle} k_{b} \\ + \langle a_{k_{a}k_{b}}\rangle + \langle b_{k_{a}k_{b}}\rangle+2\langle k_{b} \rangle + \langle k_{a} \rangle \frac{\lambda (\langle a_{k_{a}k_{b}}\rangle + \langle b_{k_{a}k_{b}} \rangle+2\langle k_{b} \rangle)^2}{\langle b_{k_{a}k_{b}\rangle+\langle k_{b} \rangle}} \Bigg),
  \end{split}
  \label{allo}
\end{equation}
which still contains the Lagrangian multiplier $\lambda$. To appropriately deal with $\lambda$ we average over \cref{allo} and assume the budget per node $\langle a_{k_{a}k_{b}} \rangle$ is sufficiently large. Therefore $\frac{\lambda (\langle a_{k_{a}k_{b}}\rangle + \langle b_{k_{a}k_{b}} \rangle+2\langle k_{b} \rangle)^2}{\langle b_{k_{a}k_{b}\rangle+\langle k_{b} \rangle}} \rightarrow 0$, which finally gives us the expression for the optimal allocation,

\begin{equation}
        a_{k_{a}k_{b}}^* = \frac{1}{2} \Bigg( \frac{\langle a_{k_{a}k_{b}}\rangle - \langle b_{k_{a}k_{b}}\rangle}{\langle b_{k_{a}k_{b}} \rangle +\langle k_{b} \rangle}  b_{k_{a}k_{b}} + \frac{\langle a_{k_{a}k_{b}} \rangle - 3\langle b_{k_{a}k_{b}} \rangle -2 \langle k_{b} \rangle}{\langle b_{k_{a}k_{b}} \rangle +\langle k_{b} \rangle} k_{b} + \langle a_{k_{a}k_{b}}\rangle + \langle b_{k_{a}k_{b}}\rangle+2\langle k_{b} \rangle \Bigg).
  \label{allo-final}
\end{equation}


\section{Uniformly distributed negative edges and uniform adversarial allocations}
\label{gain0}

When a controller cannot observe negative edges, the expression for optimal allocation is given as,

\begin{align}
    a_{k} = \frac{1}{2} \Bigg( \frac{\langle a_{k}\rangle - \langle b_{k}\rangle}{\langle b_{k} \rangle}  b_{k} + \langle a_{k}\rangle + \langle b_{k}\rangle \Bigg).
    \label{pos-allo}
\end{align}
where $k = k_{a}+k_{b}$.

The final vote-share in this case $X_{A}^{(+)}$ is obtained by replacing \cref{pos-allo} in \cref{app-X}. 
Gain in vote-shares can therefore be quantified as, 

\begin{align*}
    X_{A} - X_{A}^{(+)} &= \frac{1}{\langle a_{k_{a}k_{b}} \rangle + \langle b_{k_{a}k_{b}} \rangle +2\langle k_{b} \rangle}\Big( (\langle b_{k_{a}k_{b}} \rangle+\langle k_{b} \rangle) \langle \frac{a_{k_{a}k_{b}}-a_{k}}{k_{a}} \rangle \\
    & + \frac{\langle a_{k_{a}k_{b}} \rangle +\langle k_{b} \rangle}{\langle a_{k_{a}k_{b}} \rangle + \langle a_{k_{a}k_{b}} \rangle + 2\langle k_{b} \rangle} \langle \frac{(a_{k_{a}k_{b}}-a_{k})(a_{k_{a}k_{b}}+a_{k}+2\langle b_{k_{a}k_{b}} \rangle+4k_{b})}{k_{a}} \rangle \\
    & - \langle \frac{(a_{k_{a}k_{b}}-a_{k})(a_{k_{a}k_{b}}+a_{k}+ \langle b_{k_{a}k_{b}} \rangle +3k_{b})}{k_{a}} \rangle \Big).
\end{align*}


Furthermore, the term $a_{k_{a}k_{b}} - a_{k}$ in the above expression can be derived using \cref{allo-final,pos-allo} as, 
\begin{align}
    a_{k_{a}k_{b}} - a_{k}  =  \Big( 1 - \frac{\langle a_{k_{a}k_{b}} \rangle - \langle b_{k_{a}k_{b}} \rangle }{2\langle b_{k_{a}k_{b}} \rangle}\frac{b_{k_{a}k{b}}}{\langle b_{k_{a}k_{b}} \rangle+\langle k_{b} \rangle} \Big) \langle k_{b} \rangle + \frac{\langle a_{k_{a}k_{b}} \rangle-3\langle b_{k_{a}k_{b}} \rangle-2\langle k_{b} \rangle}{2(\langle b_{k_{a}k_{b}} \rangle+\langle k_{b} \rangle)} k_{b}.
    \label{diff-allo}
\end{align}


We consider networks with regular negative graphs, $k_{b}=\langle k_{b} \rangle$ where an adversary uniformly targets the network, $b_{k_{a}k_{b}} = \langle b_{k_{a}k_{b}} \rangle$. The above relations further simplify \cref{diff-allo} as,

\begin{align*}
    a_{k_{a}k_{b}} - a_{k} &= \Bigg( \Big( 1 - \frac{\langle a_{k_{a}k_{b}} \rangle- \langle b_{k_{a}k_{b}} \rangle}{2(\langle b_{k_{a}k_{b}} \rangle+\langle k_{b} \rangle)} \Big) + \frac{\langle a_{k_{a}k_{b}} \rangle -3 \langle b_{k_{a}k_{b}} \rangle -2\langle k_{b} \rangle}{2(\langle b_{k_{a}k_{b}} \rangle +\langle k_{b} \rangle)} \Bigg) \langle k_{b} \rangle=0.
\end{align*}

Therefore, it follows that gain $ X_{A} - X_{A}^{(+)}  = 0$, against an adversary targeting all nodes uniformly, in networks with regular negative components.


\bibliographystyle{apalike}
\bibliography{main}

% \newpage
% \section*{Appendix}
\appendix
\numberwithin{equation}{section}

\section{Removing negative ties}
\label{appendix-rem}
When controllers are unable to detect or observe negative edges in the network, i.e. $w_{ij}=max(0,w_{ij})$, the optimisation problem reduces to
\begin{align}
    p_{A}^* &= \text{argmax}_{p_{A}} X_{A}^{*(\phi)}(L^{(\phi)},p_{B},B_{A}).
\end{align}

where $L^{(\phi)}$ is the updated Laplacian. 
Following the same process as before we use the gradient $\nabla_{p_{A}} X_{A}^{(\phi)} = 1/N \Vec{1}^{T} [L^{(\phi)}+diag(p_{A}+p_{B})]^{-1}(I - diag(x_{A}^{(\phi)})$ to optimise allocations $p_{A}^*$ in a gradient ascent algorithm $GA^{(\phi)}$. Here $    x_{A,i}^{*(\phi)} = (p_{A,i} + \sum\limits_{j}^{k_a}w_{ji} x_{A,j}^{(\phi)}) / (\sum\limits_{j}^{k_a}w_{ji} + p_{A,i} + p_{B,i})$.


We then run $GA$, $GA^{(+)}$ and $GA^{(\phi)}$ on the Bitcoin network and present the respective gains in vote-shares in \cref{rem}.


  \begin{figure}
  \centering
    \includegraphics[width=0.7\textwidth]{figures/Appendix_A.eps}
    \caption{Figure showing gain in vote-shares when comparing the negative-tie sensitive optimisation approach $GA$, to traditional approaches, $GA^{(+)}$ and $GA^{(\phi)}$ for budget ratios $B_{A}/B_{B} \in [0.05,1]$. Controller B here targets the network uniformly.}
    \label{rem}
  \end{figure}

We find that the method assuming negative edges in the network to be positive $GA^{(+)}$ consistently outperforms $GA^{(\phi)}$, where negative edges are not considered at all. 
To further show that the vote-shares obtained through both methods are identical in undirected networks (and that comparing our results to only $GA^{(+)}$ is sufficient), we look at the allocation expression in \cref{pos-allo}. Here we find that the optimal allocation, in the absence of any knowledge of negative edges, depends solely on the individual budgets of the controllers and the adversarial allocations on the nodes, but not on the degrees of the nodes.   

\section{Limiting case}
\label{appendix-A}
We begin with the series expansion of the steady-state equation in \cref{XA-MF} to obtain,
\begin{align}
    X_{A} = \langle \frac{a_{k_{a}k_{b}}}{k_{a}} \rangle + \langle \frac{k_{b}}{k_{a}} \rangle + \langle 1 - \frac{a_{k_{a}k_{b}}+b_{k_{a}k_{b}}+k_{b}}{k_{a}} \rangle  \langle x_{a} \rangle - \langle \frac{k_{b}}{k_{a}} \rangle \langle x_{b} \rangle,
    \label{x-exp}
\end{align}
where a second-order expansion for $\langle x_{a} \rangle$ gives us, 
\begin{align}
    \langle x_{a} \rangle = \frac{a+\langle k_{b} \rangle - \langle \frac{(a+k_{b})(a+b+2k_{b})}{k_{a}} \rangle }{a+b+2\langle k_{b} \rangle} + \frac{a+\langle k_{b} \rangle }{(a+b+2k_{b})^2}\Bigg( \langle \frac{k_{b}^2}{k_{a}} \rangle + \langle \frac{(a+b+k_{b})(a+b+3k_{b})}{k_{a}} \rangle \Bigg),
    \label{xa-exp}
\end{align}
and a zero-order expansion for $\langle x_{b} \rangle$ gives us
\begin{align}
  \langle x_{b} \rangle = \frac{a+\langle k_{b} \rangle}{a+b+\langle k_{b} \rangle}.
  \label{xb-exp}
\end{align}

Finally, replacing \cref{xa-exp,xb-exp} in \cref{x-exp} and ignoring higher-order terms, we obtain
\begin{equation}
\begin{split}
    X_{A} \approx \langle \frac{a_{k_{a}k_{b}}}{k_{a}} \rangle + \langle \frac{k_{b}}{k_{a}} \rangle + \frac{a_{k_{a}k_{b}}+\langle k_{b} \rangle}{a_{k_{a}k_{b}}+b_{k_{a}k_{b}}+2\langle k_{b} \rangle} - \frac{\langle \frac{(a_{k_{a}k_{b}}+k_{b})(a_{k_{a}k_{b}}+b_{k_{a}k_{b}}+2k_{b})}{k_{a}} \rangle}{a_{k_{a}k_{b}}+b_{k_{a}k_{b}}+2\langle k_{b} \rangle} \\ + \frac{a_{k_{a}k_{b}}+\langle k_{b} \rangle}{(a_{k_{a}k_{b}}+b_{k_{a}k_{b}}+2\langle k_{b} \rangle)^2}\Bigg( \langle \frac{k_{b}^2}{k_{a}} \rangle  + \langle \frac{(a_{k_{a}k_{b}}+b_{k_{a}k_{b}}+k_{b})(a_{k_{a}k_{b}}+b_{k_{a}k_{b}}+3k_{b})}{k_{a}} \rangle \\ 
    - \langle \frac{a_{k_{a}k_{b}}+b_{k_{a}k_{b}}+k_{b}}{k_{a}} \rangle \frac{a_{k_{a}k_{b}}+\langle k_{b} \rangle }{a_{k_{a}k_{b}}+b_{k_{a}k_{b}}+2\langle k_{b}\rangle} - \langle \frac{k_{b}}{k_{a}} \rangle  \frac{a_{k_{a}k_{b}}+\langle k_{b}\rangle}{a_{k_{a}k_{b}}+b_{k_{a}k_{b}}+\langle k_{b}\rangle}  \Bigg).
\end{split}
\label{app-X}
\end{equation}
Note that all terms in \cref{app-X} are averaged over the joint positive and negative degree distribution $P_{k_{a}k_{b}}$.

We can now apply the Lagrange method to maximise vote-shares $X_{A}$ against a passive controller B. The Lagrangian is derived as $\mathcal{L} = X_{A} + \lambda (\sum_{k_{a}k{b}} P_{k_{a}k_{b}} a_{k_{a}k_{b}} - \langle a_{k_{a}k_{b}} \rangle N) $, where $\lambda$ is the Lagrangian multiplier. Differentiating $\mathcal{L}$ wrt allocations $a_{k_{a}k_{b}}$ we obtain

\begin{equation}
    \begin{split}
    \frac{\partial \mathcal{L}}{\partial a_{k_{a}k_{b}}} = \frac{1}{k_{a}} - \frac{2a_{k_{a}k_{b}}+b_{k_{a}k_{b}}+3k_{b}}{\langle a_{k_{a}k_{b}} \rangle+ \langle b_{k_{a}k_{b}} \rangle +2\langle k_{b} \rangle} \frac{1}{k_{a}} + 2 \frac{\langle a_{k_{a}k_{b} \rangle}+\langle k_{b} \rangle }{(\langle a_{k_{a}k_{b}} \rangle + \langle b_{k_{a}k_{b} \rangle}+2\langle k_{b} \rangle)^2} \frac{a_{k_{a}k_{b}}+b_{k_{a}k_{b}}+2k_{b}}{k_{a}} \\
    - \frac{\langle a_{k_{a}k_{b}} \rangle +2\langle k_{b} \rangle}{\langle a_{k_{a}k_{b}}\rangle + \langle b_{k_{a}k_{b}} \rangle+2\langle k_{b} \rangle} \frac{1}{k_{a}} + \lambda = 0.
    \end{split}
\end{equation}

Solving for $a_{k_{a}k_{b}}$ gives us,

\begin{equation}
    \begin{split}
        \implies a_{k_{a}k_{b}} = \frac{1}{2} \Bigg( \frac{\langle a_{k_{a}k_{b}} \rangle - \langle b_{k_{a}k_{b}}\rangle}{\langle b_{k_{a}k_{b}} \rangle +\langle k_{b} \rangle} b_{k_{a}k_{b}} + \frac{\langle a_{k_{a}k_{b}} \rangle - 3\langle b_{k_{a}k_{b}} \rangle -2 \langle k_{b} \rangle}{\langle b_{k_{a}k_{b}} \rangle +\langle k_{b} \rangle} k_{b} \\ + \langle a_{k_{a}k_{b}}\rangle + \langle b_{k_{a}k_{b}}\rangle+2\langle k_{b} \rangle + \langle k_{a} \rangle \frac{\lambda (\langle a_{k_{a}k_{b}}\rangle + \langle b_{k_{a}k_{b}} \rangle+2\langle k_{b} \rangle)^2}{\langle b_{k_{a}k_{b}\rangle+\langle k_{b} \rangle}} \Bigg),
  \end{split}
  \label{allo}
\end{equation}
which still contains the Lagrangian multiplier $\lambda$. To appropriately deal with $\lambda$ we average over \cref{allo} and assume the budget per node $\langle a_{k_{a}k_{b}} \rangle$ is sufficiently large. Therefore $\frac{\lambda (\langle a_{k_{a}k_{b}}\rangle + \langle b_{k_{a}k_{b}} \rangle+2\langle k_{b} \rangle)^2}{\langle b_{k_{a}k_{b}\rangle+\langle k_{b} \rangle}} \rightarrow 0$, which finally gives us the expression for the optimal allocation,

\begin{equation}
        a_{k_{a}k_{b}}^* = \frac{1}{2} \Bigg( \frac{\langle a_{k_{a}k_{b}}\rangle - \langle b_{k_{a}k_{b}}\rangle}{\langle b_{k_{a}k_{b}} \rangle +\langle k_{b} \rangle}  b_{k_{a}k_{b}} + \frac{\langle a_{k_{a}k_{b}} \rangle - 3\langle b_{k_{a}k_{b}} \rangle -2 \langle k_{b} \rangle}{\langle b_{k_{a}k_{b}} \rangle +\langle k_{b} \rangle} k_{b} + \langle a_{k_{a}k_{b}}\rangle + \langle b_{k_{a}k_{b}}\rangle+2\langle k_{b} \rangle \Bigg).
  \label{allo-final}
\end{equation}


\section{Uniformly distributed negative edges and uniform adversarial allocations}
\label{gain0}

When a controller cannot observe negative edges, the expression for optimal allocation is given as,

\begin{align}
    a_{k} = \frac{1}{2} \Bigg( \frac{\langle a_{k}\rangle - \langle b_{k}\rangle}{\langle b_{k} \rangle}  b_{k} + \langle a_{k}\rangle + \langle b_{k}\rangle \Bigg).
    \label{pos-allo}
\end{align}
where $k = k_{a}+k_{b}$.

The final vote-share in this case $X_{A}^{(+)}$ is obtained by replacing \cref{pos-allo} in \cref{app-X}. 
Gain in vote-shares can therefore be quantified as, 

\begin{align*}
    X_{A} - X_{A}^{(+)} &= \frac{1}{\langle a_{k_{a}k_{b}} \rangle + \langle b_{k_{a}k_{b}} \rangle +2\langle k_{b} \rangle}\Big( (\langle b_{k_{a}k_{b}} \rangle+\langle k_{b} \rangle) \langle \frac{a_{k_{a}k_{b}}-a_{k}}{k_{a}} \rangle \\
    & + \frac{\langle a_{k_{a}k_{b}} \rangle +\langle k_{b} \rangle}{\langle a_{k_{a}k_{b}} \rangle + \langle a_{k_{a}k_{b}} \rangle + 2\langle k_{b} \rangle} \langle \frac{(a_{k_{a}k_{b}}-a_{k})(a_{k_{a}k_{b}}+a_{k}+2\langle b_{k_{a}k_{b}} \rangle+4k_{b})}{k_{a}} \rangle \\
    & - \langle \frac{(a_{k_{a}k_{b}}-a_{k})(a_{k_{a}k_{b}}+a_{k}+ \langle b_{k_{a}k_{b}} \rangle +3k_{b})}{k_{a}} \rangle \Big).
\end{align*}


Furthermore, the term $a_{k_{a}k_{b}} - a_{k}$ in the above expression can be derived using \cref{allo-final,pos-allo} as, 
\begin{align}
    a_{k_{a}k_{b}} - a_{k}  =  \Big( 1 - \frac{\langle a_{k_{a}k_{b}} \rangle - \langle b_{k_{a}k_{b}} \rangle }{2\langle b_{k_{a}k_{b}} \rangle}\frac{b_{k_{a}k{b}}}{\langle b_{k_{a}k_{b}} \rangle+\langle k_{b} \rangle} \Big) \langle k_{b} \rangle + \frac{\langle a_{k_{a}k_{b}} \rangle-3\langle b_{k_{a}k_{b}} \rangle-2\langle k_{b} \rangle}{2(\langle b_{k_{a}k_{b}} \rangle+\langle k_{b} \rangle)} k_{b}.
    \label{diff-allo}
\end{align}


We consider networks with regular negative graphs, $k_{b}=\langle k_{b} \rangle$ where an adversary uniformly targets the network, $b_{k_{a}k_{b}} = \langle b_{k_{a}k_{b}} \rangle$. The above relations further simplify \cref{diff-allo} as,

\begin{align*}
    a_{k_{a}k_{b}} - a_{k} &= \Bigg( \Big( 1 - \frac{\langle a_{k_{a}k_{b}} \rangle- \langle b_{k_{a}k_{b}} \rangle}{2(\langle b_{k_{a}k_{b}} \rangle+\langle k_{b} \rangle)} \Big) + \frac{\langle a_{k_{a}k_{b}} \rangle -3 \langle b_{k_{a}k_{b}} \rangle -2\langle k_{b} \rangle}{2(\langle b_{k_{a}k_{b}} \rangle +\langle k_{b} \rangle)} \Bigg) \langle k_{b} \rangle=0.
\end{align*}

Therefore, it follows that gain $ X_{A} - X_{A}^{(+)}  = 0$, against an adversary targeting all nodes uniformly, in networks with regular negative components.


\end{document}