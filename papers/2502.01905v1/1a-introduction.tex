\section{Introduction}
\label{sec:introduction}
Opinions are predominantly influenced by individual beliefs, social interactions and information made available through external interventions \cite{myers1996social,watts2007influentials}. Of these, guiding collective opinions through external influence poses an interesting research problem which has far-reaching societal and commercial implications, such as trying to improve public health through controlled information campaigns \cite{wilder2018end}, campaigning in elections \cite{ranganath2016understanding} and marketing for product adoption \cite{watts2007viral} in populations.
%Additionally, it often allows radical parties to impose divisive ideas and propagate misinformation in societies, which can also lead to serious and harmful ramifications \cite{bennett2018disinformation,nelson2020danger,vosoughi2018spread}. 
The increasing reliance on social media and their pervasiveness in society today \cite{perrin2015social} has thus inspired a rich body of work dedicated to understanding how opinions evolve in social networks and how they can be optimally steered through external influence \cite{banerjee2020survey,noorazar2020recent}.
In particular, a number of papers explore this problem in social \cite{kiss2017mathematics,pastor2015epidemic}, political \cite{galam1999application} and economic settings \cite{easley2010networks,jackson2010social}. However, a majority of this literature strictly investigates friendship networks where influence propagates based on positive recommendations and endorsements \cite{newman2003structure}. Effects of negative relationships on opinion propagation have received limited attention and have been historically discounted from network dynamics, given their sparse presence \cite{offer2021negative} and association with avoidance behaviour \cite{harrigan2017avoidance}, i.e.~people who dislike (or distrust) each other are unlikely to communicate (or be connected within a network). However, the scope of anonymity and prevalence of fake profiles on social media platforms recently have made such ties increasingly ubiquitous \cite{bae2012sentiment,pfeffer2014understanding}, and even typical in many recommendation and trading networks \cite{guha2004propagation,leskovec2010signed}. Although this has initiated a lot of interest in understanding the impact of negative ties on network dynamics, there are several settings in this context that are yet to be explored \cite{offer2021negative}. 
% thus demanding a more robust understanding of the impact of negative ties on network dynamics. Moreover, they are governed by a unique set of properties that are distinctly different from those of positive ties \cite{easley2010networks}.

In this paper, we consider negative ties as antagonistic relationships that negatively influence social neighbours, and persuade them to adopt an opposing position (or opinion).  
Such relationships pose a unique challenge when trying to maximise influence in a network. Relationships of distrust (negative reviews) on e-commerce platforms (e.g.~eBay) following below-par experiences, for instance, can negatively impact future transactions or communication \cite{borgs2010novel,chen2011influence}. Therefore, effective navigation of such ties is required when influencing a network externally. This is particularly the case in competitive environments, as neglectfully targeting individuals that propagate overall negative influence, can, in turn, facilitate the spread of undesirable opinions \cite{chen2018negative}.

In this paper, we demonstrate the need for a negative-tie aware approach while maximising influence in social networks under competitive settings. In doing so, we make the following contributions:
\begin{enumerate}
\item We modify voter dynamics for networks with negative ties, under competitive conditions and  subsequently present a negative-tie aware influence maximisation algorithm to optimally target networks in these settings.
\item We show that in a real-world network a controller can achieve an additional $9\%$ in vote-shares against a naïve competitor (with no knowledge of negative ties in the network).
\item We show how the effectiveness of a negative-tie aware approach varies with network topology, availability of budget and competitor strategy.
\item We provide analytical support for numerical methods and derive expressions for optimal allocations in large, arbitrary synthetic networks.
\item We present results for the game-theoretic setting where controllers actively optimise their strategies against one another. 
\end{enumerate}
The structure of the rest of the paper is as follows. In \cref{background} we discuss the related literature. We then modify the classical voter model in \cref{Model} to study opinion dynamics in signed networks. 
%and further present an indiscriminate, traditional approach for comparison. 
In \cref{bitcoin}, we investigate the problem in real-world settings, and we analyse it further using numerical methods in \cref{numerical}. In \cref{mean-field}, we present analytical support for our numerical results and further provide analytical expressions for optimal allocations in complex networks. Finally, in \cref{game-theory} we extend the problem to game-theoretic scenarios and we show how much a controller can gain from considering negative ties in influence maximisation exercises against unknown competitor strategies.
