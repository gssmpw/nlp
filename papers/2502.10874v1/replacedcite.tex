\section{Related Work}
\label{sec:prior}

This section reviews non-blocking join algorithms and proxy stores for materialized join views proposed in recent studies.

\paragraph{Non-blocking join algorithms:}

Recent work on non-blocking join algorithms____ focuses on streaming workloads and balances three aspects: (a) response time of first result rows, (b) continuous query processing despite data source delays, e.g., over a network, and (c) end-to-end execution time.
These objectives and contexts differ from ours.
Our work does not address aspect (b) but introduces additional objectives: using the intermediate storage structure of non-blocking joins, the merged index, as an optimal proxy store for the corresponding materialized join view.

\paragraph{Proxy stores for materialized join views:}

An alternative to IVM is maintaining a proxy store instead of a full materialized view.
Proxy stores balance maintenance efficiency with the efficiency to generate the materialized view or a queried subset of it.
Join indexes____, symmetric hash tables____, and traditional indexes on join inputs are examples of proxy stores for materialized join views.
\Citet{Idris2017DyanmicYanakakis, Berkholz2017conjunctive} propose proxy stores for a generic class of queries, which are also applicable to binary joins.
They provide theoretical guarantees under assumptions of main-memory data structures, which represent a different approach from ours.

In a join index, each record consists of the join key and row IDs from input tables.
Join indexes are similar to merged indexes in precomputing and storing the matching status of join inputs in one storage structure.
In contrast, merged indexes are more versatile:
Like materialized join views, merged indexes allow additional columns to be included.
These additional columns can cover a join query and enable index-only retrieval, achieving query performance comparable to materialized join views.
Join indexes, however, must back-join with base tables to retrieve columns selected by join queries.

Experiments in \cref{sec:exp} focus on the three dimensions of ideal proxy stores as discussed in \cref{sec:intro}.
These dimensions are also valid reference points for all proxy stores and likewise all non-blocking join algorithms.