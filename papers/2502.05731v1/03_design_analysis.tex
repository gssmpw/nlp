
\section{Design Analysis}
Combining the literature review, we work closely with two environmental experts, \textbf{E1} and \textbf{E2}, who are also co-authors and are currently leading an environmental study in Lyudao, to understand the process of mining with DPSIR taxonomy. 
\sam{
\marginnote{$\triangle$\_4\_1}While both experts seek to leverage LLMs for text mining, they have also encountered limitations in tools like ChatGPT~\cite{chatgpt} for controlling LLMs. 
}
Together, we identify four technical challenges in prompt-based text mining:

\paragraph{\textit{Designing dataset-scale prompts}} Designing prompts for mining a dataset requires a unique set of prompting techniques, as opposed to interacting with conversational agents such as ChatGPT~\cite{chatgpt} or Claude~\cite{claude}. For example, our experts have tried to insert a document in ChatGPT and ask for insights on DPSIR, but the results were not satisfying. Even on a single document, they need to engage in multiple rounds of conversation to calibrate the model's understanding of the DPSIR definitions. Furthermore, the various topics discussed in different documents would significantly alter the model's understanding of the DPSIR definitions, and the calibration needs to be repeated on every document. 
It is thus essential to provide support for dataset-scale prompting.
% This makes it challenging to converge the DPSIR definitions from multiple documents. 
% Support for making sense of the corpus topics at a glance is thus essential to efficiently develop the DPSIR taxonomy. 

\paragraph{\textit{Decomposing the mining task}} 
Since the DPSIR mining task may be too complex for current LLMs to solve in one shot, a decomposition step~\cite{khot2023decomposed} is needed to break it into subtasks that better align with the model's capabilities. The resulting pipeline needs to support the iterative refinement of the DPSIR taxonomy for the environmental experts. This means that the subtasks should not only remain meaningful to the experts, enabling them to interpret the results without knowledge of complex prompting or ML techniques, but also be evaluable from a technical perspective.

\paragraph{\textit{Supporting evaluation and exploration}}
To support the progressive update of DPSIR taxonomy in the prompting pipeline, evaluation of the mining results and exploration of the corpus topics are the two key tasks to support. 
While topic exploration is a relatively well-studied task, evaluation without ground-truth labels remains a technical challenge. 
Moreover, the evaluation and topic should be presented in a way that caters to their background, as previous studies have shown how non-AI experts can misinterpret evaluation metrics and topic modeling results~\cite{liao2020questionAI, lee2017humantouch}.

\paragraph{\textit{Sharing the insights}} Communication of the DPSIR insights is an important purpose of applying the framework, but the mined insights are usually scattered across the corpus, and experts need to manually organize them into a form suitable for presentation and discussion. During the presentation, other experts might raise questions or propose potential ideas. However, the manual organization means the insights are static and not responsive to the discussion topics. The mining results should be automatically organized and presented in an interactive visualization.

In response to these challenges, we derive the following set of design requirements:
\begin{itemize}[label={}, leftmargin=9px, itemsep=1px, topsep=2px]
    \item \textbf{R1: Infuse domain knowledge input with established prompting methodologies.} 
    Our target users, the environmental experts, are not expected to be experienced prompting practitioners. The system should only expose components that require domain knowledge input for interaction, and seamlessly integrate it with established prompting methodologies. 
    \item \textbf{R2: Support exploration and evaluation simultaneously.} 
    To define and contextualize the DPSIR taxonomy, the expert needs to thoroughly understand the corpus, which can be cognitively demanding. The system should support the experts to use the current DPSIR taxonomy to reveal uncovered content, and iteratively refine the taxonomy. 
    % \item \textbf{R3: Support evaluation of prompt performances.}
    At the same time, given a DPSIR taxonomy and the corresponding prompts, the experts should be able to evaluate the prompts to make sure the prompting results align with their intentions. 
    The exploration of the corpus topics and the evaluation of the prompts are intertwined, and the system should support these two tasks simultaneously.
    \item \textbf{R3: Prioritize recall over precision.}
     Considering the complexity of any environmental context, the validity of any insight can only be validated by human experts. The system should be designed to support a human-in-the-loop insight discovery, rather than aiming at extracting precise insights. Therefore, the system should prioritize recall over precision to ensure that the experts fully examine the corpus.
    \item \textbf{R4: Support collaborative discussion on mining results.} 
    After the analysis, the experts typically host internal discussions or workshops to ensure the rigorousness of the findings before communicating them to policymakers. An interactive visualization can offer graphics support for the discussion as well as querying functionalities to quickly validate ideas. Also, clarity and simplicity should be prioritized over information density to avoid distracting the discussion.
\end{itemize}

\noindent Next, we introduce how this set of design requirements informs our technical and design choices.

\begin{figure*}[t]
    \centering
    \includegraphics[width=\textwidth]{pipeline}
    \caption{The system design of \system. Starting with a corpus, the documents are first segmented into snippets. Then, the snippets go through an interactive prompting pipeline consisting of three steps: Indicator, Variable, and Link Identification. In each step, the environmental experts can use the uncertainty chart to explore the corpus, evaluate the prompting performances, and refine the DPSIR taxonomy. After the mining, an interactive DPSIR Graph can be used for collaborative discussion and communication of the findings.}
    \label{fig: pipeline}
  \vspace*{-0.3cm}
\end{figure*}