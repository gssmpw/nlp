% $Id: template.tex 11 2007-04-03 22:25:53Z jpeltier $

% \documentclass{vgtc}                          % final (conference style)
% \documentclass[review]{vgtc}                 % review
%\documentclass[widereview]{vgtc}             % wide-spaced review
\documentclass[preprint]{vgtc}               % preprint
% \documentclass[electronic]{vgtc}             % electronic version

%% Uncomment one of the lines above depending on where your paper is
%% in the conference process. ``review'' and ``widereview'' are for review
%% submission, ``preprint'' is for pre-publication, and the final version
%% doesn't use a specific qualifier. Further, ``electronic'' includes
%% hyperreferences for more convenient online viewing.

%% Please use one of the ``review'' options in combination with the
%% assigned online id (see below) ONLY if your paper uses a double blind
%% review process. Some conferences, like IEEE Vis and InfoVis, have NOT
%% in the past.

%% Figures should be in CMYK or Grey scale format, otherwise, colour 
%% shifting may occur during the printing process.

%% These few lines make a distinction between latex and pdflatex calls and they
%% bring in essential packages for graphics and font handling.
%% Note that due to the \DeclareGraphicsExtensions{} call it is no longer necessary
%% to provide the the path and extension of a graphics file:
%% \includegraphics{diamondrule} is completely sufficient.
%%
\ifpdf%                                % if we use pdflatex
  \pdfoutput=1\relax                   % create PDFs from pdfLaTeX
  \pdfcompresslevel=9                  % PDF Compression
  \pdfoptionpdfminorversion=7          % create PDF 1.7
  \ExecuteOptions{pdftex}
  \usepackage{graphicx}                % allow us to embed graphics files
  \DeclareGraphicsExtensions{.pdf,.png,.jpg,.jpeg} % for pdflatex we expect .pdf, .png, or .jpg files
\else%                                 % else we use pure latex
  \ExecuteOptions{dvips}
  \usepackage{graphicx}                % allow us to embed graphics files
  \DeclareGraphicsExtensions{.eps}     % for pure latex we expect eps files
\fi%

%% it is recomended to use ``\autoref{sec:bla}'' instead of ``Fig.~\ref{sec:bla}''
\graphicspath{{figures/}{pictures/}{images/}{./}} % where to search for the images

\usepackage{microtype}                 % use micro-typography (slightly more compact, better to read)
\PassOptionsToPackage{warn}{textcomp}  % to address font issues with \textrightarrow
\usepackage{textcomp}                  % use better special symbols
\usepackage{mathptmx}                  % use matching math font
\usepackage{times}                     % we use Times as the main font
\renewcommand*\ttdefault{txtt}         % a nicer typewriter font
\usepackage{cite}                      % needed to automatically sort the references
\usepackage{tabu}                      % only used for the table example
\usepackage{booktabs}                  % only used for the table example
\usepackage{enumitem}
\usepackage[skip=1pt]{caption}
\usepackage{amsmath}
\usepackage{balance}
\newcommand{\system}{\textit{GreenMine}}

\usepackage{xcolor}
\usepackage{soul}
\usepackage{array}
\usepackage{tabularray}
 \definecolor{sampink}{HTML}{ff0000}
 \definecolor{sampink}{HTML}{000000}
 % \definecolor{sampink}{HTML}{black}
 \newcommand{\sam}[1]{{\color{sampink} #1}}
 \newcommand{\rev}[2]{{\st{#1}\textcolor{sampink}{#2}}}
\usepackage{marginnote} % For better margin notes
\renewcommand*{\marginfont}{\footnotesize}
\setlength{\marginparwidth}{1cm}
\renewcommand*{\marginnote}[1]{}
\makeatletter
\def\input@path{{sections/}}
%% We encourage the use of mathptmx for consistent usage of times font
%% throughout the proceedings. However, if you encounter conflicts
%% with other math-related packages, you may want to disable it.


%% If you are submitting a paper to a conference for review with a double
%% blind reviewing process, please replace the value ``0'' below with your
%% OnlineID. Otherwise, you may safely leave it at ``0''.
\onlineid{7280}

%% declare the category of your paper, only shown in review mode
\vgtccategory{Research}

%% allow for this line if you want the electronic option to work properly
\vgtcinsertpkg

%% In preprint mode you may define your own headline. If not, the default IEEE copyright message will appear in preprint mode.
%\preprinttext{To appear in an IEEE VGTC sponsored conference.}

%% This adds a link to the version of the paper on IEEEXplore
%% Uncomment this line when you produce a preprint version of the article 
%% after the article receives a DOI for the paper from IEEE
%\ieeedoi{xx.xxxx/TVCG.201x.xxxxxxx}


%% Paper title.

\title{Visual Text Mining with Progressive Taxonomy Construction\\ for Environmental Studies}

%% This is how authors are specified in the conference style

%% Author and Affiliation (single author).
%%\author{Roy G. Biv\thanks{e-mail: roy.g.biv@aol.com}}
%%\affiliation{\scriptsize Allied Widgets Research}

%% Author and Affiliation (multiple authors with single affiliations).
\author{Sam Yu-Te Lee\thanks{e-mail: ytlee@ucdavis.edu} %
\affiliation{\scriptsize University of California, Davis}
\and Cheng-Wei Hung\thanks{e-mail: d10541002@ntu.edu.tw}
\affiliation{\scriptsize National Taiwan University}
\and Mei-Hua Yuan\thanks{e-mail: meihuayuan@gate.sinica.edu.tw}
\affiliation{\scriptsize Academia Sinica}
\and Kwan-Liu Ma \thanks{e-mail: klma@ucdavis.edu} %
\affiliation{\scriptsize University of California, Davis}
}

%% Author and Affiliation (multiple authors with multiple affiliations)
% \author{Roy G. Biv\thanks{e-mail: roy.g.biv@aol.com}\\ %
%         \scriptsize Starbucks Research %
% \and Ed Grimley\thanks{e-mail: ed.grimley@aol.com}\\ %
%      \scriptsize Grimley Widgets, Inc. %
% \and Martha Stewart\thanks{e-mail: martha.stewart@marthastewart.com}\\ %
%      \parbox{1.4in}{\scriptsize \centering Martha Stewart Enterprises \\ Microsoft Research}}

%% A teaser figure can be included as follows
% \teaser{
%   \centering
%   \includegraphics[width=\linewidth]{CypressView}
%   \caption{In the Clouds: Vancouver from Cypress Mountain. Note that the teaser may not be wider than the abstract block.}
%   \label{fig:teaser}
% }

%% Abstract section.
\abstract{
Environmental experts have developed the DPSIR (Driver, Pressure, State, Impact, Response) framework to systematically study and communicate key relationships between society and the environment. Using this framework requires experts to construct a DPSIR taxonomy from a corpus, annotate the documents, and identify DPSIR variables and relationships, which is laborious and inflexible. Automating it with conventional text mining faces technical challenges, primarily because the taxonomy often begins with abstract definitions, which experts progressively refine and contextualize as they annotate the corpus. In response, we develop \system, a system that supports interactive text mining with prompt engineering. The system implements a prompting pipeline consisting of three simple and evaluable subtasks. In each subtask, the DPSIR taxonomy can be defined in natural language and iteratively refined as experts analyze the corpus. To support users evaluate the taxonomy, we introduce an uncertainty score based on response consistency. Then, we design a radial uncertainty chart that visualizes uncertainties and corpus topics, which supports interleaved evaluation and exploration. Using the system, experts can progressively construct the DPSIR taxonomy and annotate the corpus with LLMs. Using real-world interview transcripts, we present a case study to demonstrate the capability of the system in supporting interactive mining of DPSIR relationships,
and an expert review in the form of collaborative discussion to understand the potential and limitations of the system. We discuss the lessons learned from developing the system and future opportunities for supporting interactive text mining in knowledge-intensive tasks for other application scenarios.
} % end of abstract

%% ACM Computing Classification System (CCS). 
%% See <http://www.acm.org/about/class> for details.
%% We recommend the 2012 system <http://www.acm.org/about/class/class/2012>
%% For the 2012 system use the ``\CCScatTwelve'' which command takes four arguments.
%% The 1998 system <http://www.acm.org/about/class/class/2012> is still possible
%% For the 1998 system use the ``\CCScat'' which command takes four arguments.
%% In both cases the last two arguments (1998) or last three (2012) can be empty.

\CCScatlist{
  \CCScatTwelve{Information systems}{In\-for\-mation sys\-tems app\-li\-ca\-tions}{Data Min\-ing}{};
  \CCScatTwelve{Human-centered computing}{Visu\-al\-iza\-tion}{Visu\-al\-iza\-tion sys\-tems and tools}{}
}

%\CCScatlist{
  %\CCScat{H.5.2}{User Interfaces}{User Interfaces}{Graphical user interfaces (GUI)}{};
  %\CCScat{H.5.m}{Information Interfaces and Presentation}{Miscellaneous}{}{}
%}

%% Copyright space is enabled by default as required by guidelines.
%% It is disabled by the 'review' option or via the following command:
% \nocopyrightspace

%%%%%%%%%%%%%%%%%%%%%%%%%%%%%%%%%%%%%%%%%%%%%%%%%%%%%%%%%%%%%%%%
%%%%%%%%%%%%%%%%%%%%%% START OF THE PAPER %%%%%%%%%%%%%%%%%%%%%%
%%%%%%%%%%%%%%%%%%%%%%%%%%%%%%%%%%%%%%%%%%%%%%%%%%%%%%%%%%%%%%%%%

\begin{document}
\maketitle

\section{Introduction}
\label{section:introduction}

% redirection is unique and important in VR
Virtual Reality (VR) systems enable users to embody virtual avatars by mirroring their physical movements and aligning their perspective with virtual avatars' in real time. 
As the head-mounted displays (HMDs) block direct visual access to the physical world, users primarily rely on visual feedback from the virtual environment and integrate it with proprioceptive cues to control the avatar’s movements and interact within the VR space.
Since human perception is heavily influenced by visual input~\cite{gibson1933adaptation}, 
VR systems have the unique capability to control users' perception of the virtual environment and avatars by manipulating the visual information presented to them.
Leveraging this, various redirection techniques have been proposed to enable novel VR interactions, 
such as redirecting users' walking paths~\cite{razzaque2005redirected, suma2012impossible, steinicke2009estimation},
modifying reaching movements~\cite{gonzalez2022model, azmandian2016haptic, cheng2017sparse, feick2021visuo},
and conveying haptic information through visual feedback to create pseudo-haptic effects~\cite{samad2019pseudo, dominjon2005influence, lecuyer2009simulating}.
Such redirection techniques enable these interactions by manipulating the alignment between users' physical movements and their virtual avatar's actions.

% % what is hand/arm redirection, motivation of study arm-offset
% \change{\yj{i don't understand the purpose of this paragraph}
% These illusion-based techniques provide users with unique experiences in virtual environments that differ from the physical world yet maintain an immersive experience. 
% A key example is hand redirection, which shifts the virtual hand’s position away from the real hand as the user moves to enhance ergonomics during interaction~\cite{feuchtner2018ownershift, wentzel2020improving} and improve interaction performance~\cite{montano2017erg, poupyrev1996go}. 
% To increase the realism of virtual movements and strengthen the user’s sense of embodiment, hand redirection techniques often incorporate a complete virtual arm or full body alongside the redirected virtual hand, using inverse kinematics~\cite{hartfill2021analysis, ponton2024stretch} or adjustments to the virtual arm's movement as well~\cite{li2022modeling, feick2024impact}.
% }

% noticeability, motivation of predicting a probability, not a classification
However, these redirection techniques are most effective when the manipulation remains undetected~\cite{gonzalez2017model, li2022modeling}. 
If the redirection becomes too large, the user may not mitigate the conflict between the visual sensory input (redirected virtual movement) and their proprioception (actual physical movement), potentially leading to a loss of embodiment with the virtual avatar and making it difficult for the user to accurately control virtual movements to complete interaction tasks~\cite{li2022modeling, wentzel2020improving, feuchtner2018ownershift}. 
While proprioception is not absolute, users only have a general sense of their physical movements and the likelihood that they notice the redirection is probabilistic. 
This probability of detecting the redirection is referred to as \textbf{noticeability}~\cite{li2022modeling, zenner2024beyond, zenner2023detectability} and is typically estimated based on the frequency with which users detect the manipulation across multiple trials.

% version B
% Prior research has explored factors influencing the noticeability of redirected motion, including the redirection's magnitude~\cite{wentzel2020improving, poupyrev1996go}, direction~\cite{li2022modeling, feuchtner2018ownershift}, and the visual characteristics of the virtual avatar~\cite{ogawa2020effect, feick2024impact}.
% While these factors focus on the avatars, the surrounding virtual environment can also influence the users' behavior and in turn affect the noticeability of redirection.
% One such prominent external influence is through the visual channel - the users' visual attention is constantly distracted by complex visual effects and events in practical VR scenarios.
% Although some prior studies have explored how to leverage user blindness caused by visual distractions to redirect users' virtual hand~\cite{zenner2023detectability}, there remains a gap in understanding how to quantify the noticeability of redirection under visual distractions.

% visual stimuli and gaze behavior
Prior research has explored factors influencing the noticeability of redirected motion, including the redirection's magnitude~\cite{wentzel2020improving, poupyrev1996go}, direction~\cite{li2022modeling, feuchtner2018ownershift}, and the visual characteristics of the virtual avatar~\cite{ogawa2020effect, feick2024impact}.
While these factors focus on the avatars, the surrounding virtual environment can also influence the users' behavior and in turn affect the noticeability of redirection.
This, however, remains underexplored.
One such prominent external influence is through the visual channel - the users' visual attention is constantly distracted by complex visual effects and events in practical VR scenarios.
We thus want to investigate how \textbf{visual stimuli in the virtual environment} affect the noticeability of redirection.
With this, we hope to complement existing works that focus on avatars by incorporating environmental visual influences to enable more accurate control over the noticeability of redirected motions in practical VR scenarios.
% However, in realistic VR applications, the virtual environment often contains complex visual effects beyond the virtual avatar itself. 
% We argue that these visual effects can \textbf{distract users’ visual attention and thus affect the noticeability of redirection offsets}, while current research has yet taken into account.
% For instance, in a VR boxing scenario, a user’s visual attention is likely focused on their opponent rather than on their virtual body, leading to a lower noticeability of redirection offsets on their virtual movements. 
% Conversely, when reaching for an object in the center of their field of view, the user’s attention is more concentrated on the virtual hand’s movement and position to ensure successful interaction, resulting in a higher noticeability of offsets.

Since each visual event is a complex choreography of many underlying factors (type of visual effect, location, duration, etc.), it is extremely difficult to quantify or parameterize visual stimuli.
Furthermore, individuals respond differently to even the same visual events.
Prior neuroscience studies revealed that factors like age, gender, and personality can influence how quickly someone reacts to visual events~\cite{gillon2024responses, gale1997human}. 
Therefore, aiming to model visual stimuli in a way that is generalizable and applicable to different stimuli and users, we propose to use users' \textbf{gaze behavior} as an indicator of how they respond to visual stimuli.
In this paper, we used various gaze behaviors, including gaze location, saccades~\cite{krejtz2018eye}, fixations~\cite{perkhofer2019using}, and the Index of Pupil Activity (IPA)~\cite{duchowski2018index}.
These behaviors indicate both where users are looking and their cognitive activity, as looking at something does not necessarily mean they are attending to it.
Our goal is to investigate how these gaze behaviors stimulated by various visual stimuli relate to the noticeability of redirection.
With this, we contribute a model that allows designers and content creators to adjust the redirection in real-time responding to dynamic visual events in VR.

To achieve this, we conducted user studies to collect users' noticeability of redirection under various visual stimuli.
To simulate realistic VR scenarios, we adopted a dual-task design in which the participants performed redirected movements while monitoring the visual stimuli.
Specifically, participants' primary task was to report if they noticed an offset between the avatar's movement and their own, while their secondary task was to monitor and report the visual stimuli.
As realistic virtual environments often contain complex visual effects, we started with simple and controlled visual stimulus to manage the influencing factors.

% first user study, confirmation study
% collect data under no visual stimuli, different basic visual stimuli
We first conducted a confirmation study (N=16) to test whether applying visual stimuli (opacity-based) actually affects their noticeability of redirection. 
The results showed that participants were significantly less likely to detect the redirection when visual stimuli was presented $(F_{(1,15)}=5.90,~p=0.03)$.
Furthermore, by analyzing the collected gaze data, results revealed a correlation between the proposed gaze behaviors and the noticeability results $(r=-0.43)$, confirming that the gaze behaviors could be leveraged to compute the noticeability.

% data collection study
We then conducted a data collection study to obtain more accurate noticeability results through repeated measurements to better model the relationship between visual stimuli-triggered gaze behaviors and noticeability of redirection.
With the collected data, we analyzed various numerical features from the gaze behaviors to identify the most effective ones. 
We tested combinations of these features to determine the most effective one for predicting noticeability under visual stimuli.
Using the selected features, our regression model achieved a mean squared error (MSE) of 0.011 through leave-one-user-out cross-validation. 
Furthermore, we developed both a binary and a three-class classification model to categorize noticeability, which achieved an accuracy of 91.74\% and 85.62\%, respectively.

% evaluation study
To evaluate the generalizability of the regression model, we conducted an evaluation study (N=24) to test whether the model could accurately predict noticeability with new visual stimuli (color- and scale-based animations).
Specifically, we evaluated whether the model's predictions aligned with participants' responses under these unseen stimuli.
The results showed that our model accurately estimated the noticeability, achieving mean squared errors (MSE) of 0.014 and 0.012 for the color- and scale-based visual stimili, respectively, compared to participants' responses.
Since the tested visual stimuli data were not included in the training, the results suggested that the extracted gaze behavior features capture a generalizable pattern and can effectively indicate the corresponding impact on the noticeability of redirection.

% application
Based on our model, we implemented an adaptive redirection technique and demonstrated it through two applications: adaptive VR action game and opportunistic rendering.
We conducted a proof-of-concept user study (N=8) to compare our adaptive redirection technique with a static redirection, evaluating the usability and benefits of our adaptive redirection technique.
The results indicated that participants experienced less physical demand and stronger sense of embodiment and agency when using the adaptive redirection technique. 
These results demonstrated the effectiveness and usability of our model.

In summary, we make the following contributions.
% 
\begin{itemize}
    \item 
    We propose to use users' gaze behavior as a medium to quantify how visual stimuli influences the noticebility of redirection. 
    Through two user studies, we confirm that visual stimuli significantly influences noticeability and identify key gaze behavior features that are closely related to this impact.
    \item 
    We build a regression model that takes the user's gaze behavioral data as input, then computes the noticeability of redirection.
    Through an evaluation study, we verify that our model can estimate the noticeability with new participants under unseen visual stimuli.
    These findings suggest that the extracted gaze behavior features effectively capture the influence of visual stimuli on noticeability and can generalize across different users and visual stimuli.
    \item 
    We develop an adaptive redirection technique based on our regression model and implement two applications with it.
    With a proof-of-concept study, we demonstrate the effectiveness and potential usability of our regression model on real-world use cases.

\end{itemize}

% \delete{
% Virtual Reality (VR) allows the user to embody a virtual avatar by mirroring their physical movements through the avatar.
% As the user's visual access to the physical world is blocked in tasks involving motion control, they heavily rely on the visual representation of the avatar's motions to guide their proprioception.
% Similar to real-world experiences, the user is able to resolve conflicts between different sensory inputs (e.g., vision and motor control) through multisensory integration, which is essential for mitigating the sensory noise that commonly arises.
% However, it also enables unique manipulations in VR, as the system can intentionally modify the avatar's movements in relation to the user's motions to achieve specific functional outcomes,
% for example, 
% % the manipulations on the avatar's movements can 
% enabling novel interaction techniques of redirected walking~\cite{razzaque2005redirected}, redirected reaching~\cite{gonzalez2022model}, and pseudo haptics~\cite{samad2019pseudo}.
% With small adjustments to the avatar's movements, the user can maintain their sense of embodiment, due to their ability to resolve the perceptual differences.
% % However, a large mismatch between the user and avatar's movements can result in the user losing their sense of embodiment, due to an inability to resolve the perceptual differences.
% }

% \delete{
% However, multisensory integration can break when the manipulation is so intense that the user is aware of the existence of the motion offset and no longer maintains the sense of embodiment.
% Prior research studied the intensity threshold of the offset applied on the avatar's hand, beyond which the embodiment will break~\cite{li2022modeling}. 
% Studies also investigated the user's sensitivity to the offsets over time~\cite{kohm2022sensitivity}.
% Based on the findings, we argue that one crucial factor that affects to what extent the user notices the offset (i.e., \textit{noticeability}) that remains under-explored is whether the user directs their visual attention towards or away from the virtual avatar.
% Related work (e.g., Mise-unseen~\cite{marwecki2019mise}) has showcased applications where adjustments in the environment can be made in an unnoticeable manner when they happen in the area out of the user's visual field.
% We hypothesize that directing the user's visual attention away from the avatar's body, while still partially keeping the avatar within the user's field-of-view, can reduce the noticeability of the offset.
% Therefore, we conduct two user studies and implement a regression model to systematically investigate this effect.
% }

% \delete{
% In the first user study (N = 16), we test whether drawing the user's visual attention away from their body impacts the possibility of them noticing an offset that we apply to their arm motion in VR.
% We adopt a dual-task design to enable the alteration of the user's visual attention and a yes/no paradigm to measure the noticeability of motion offset. 
% The primary task for the user is to perform an arm motion and report when they perceive an offset between the avatar's virtual arm and their real arm.
% In the secondary task, we randomly render a visual animation of a ball turning from transparent to red and becoming transparent again and ask them to monitor and report when it appears.
% We control the strength of the visual stimuli by changing the duration and location of the animation.
% % By changing the time duration and location of the visual animation, we control the strengths of attraction to the users.
% As a result, we found significant differences in the noticeability of the offsets $(F_{(1,15)}=5.90,~p=0.03)$ between conditions with and without visual stimuli.
% Based on further analysis, we also identified the behavioral patterns of the user's gaze (including pupil dilation, fixations, and saccades) to be correlated with the noticeability results $(r=-0.43)$ and they may potentially serve as indicators of noticeability.
% }

% \delete{
% To further investigate how visual attention influences the noticeability, we conduct a data collection study (N = 12) and build a regression model based on the data.
% The regression model is able to calculate the noticeability of the offset applied on the user's arm under various visual stimuli based on their gaze behaviors.
% Our leave-one-out cross-validation results show that the proposed method was able to achieve a mean-squared error (MSE) of 0.012 in the probability regression task.
% }

% \delete{
% To verify the feasibility and extendability of the regression model, we conduct an evaluation study where we test new visual animations based on adjustments on scale and color and invite 24 new participants to attend the study.
% Results show that the proposed method can accurately estimate the noticeability with an MSE of 0.014 and 0.012 in the conditions of the color- and scale-based visual effects.
% Since these animations were not included in the dataset that the regression model was built on, the study demonstrates that the gaze behavioral features we extracted from the data capture a generalizable pattern of the user's visual attention and can indicate the corresponding impact on the noticeability of the offset.
% }

% \delete{
% Finally, we demonstrate applications that can benefit from the noticeability prediction model, including adaptive motion offsets and opportunistic rendering, considering the user's visual attention. 
% We conclude with discussions of our work's limitations and future research directions.
% }

% \delete{
% In summary, we make the following contributions.
% }
% % 
% \begin{itemize}
%     \item 
%     \delete{
%     We quantify the effects of the user's visual attention directed away by stimuli on their noticeability of an offset applied to the avatar's arm motion with respect to the user's physical arm. 
%     Through two user studies, we identified gaze behavioral features that are indicative of the changes in noticeability.
%     }
%     \item 
%     \delete{We build a regression model that takes the user's gaze behavioral data and the offset applied to the arm motion as input, then computes the probability of the user noticing the offset.
%     Through an evaluation study, we verified that the model needs no information about the source attracting the user's visual attention and can be generalizable in different scenarios.
%     }
%     \item 
%     \delete{We demonstrate two applications that potentially benefit from the regression model, including adaptive motion offsets and opportunistic rendering.
%     }

% \end{itemize}

\begin{comment}
However, users will lose the sense of embodiment to the virtual avatars if they notice the offset between the virtual and physical movements.
To address this, researchers have been exploring the noticing threshold of offsets with various magnitudes and proposing various redirection techniques that maintain the sense of embodiment~\cite{}.

However, when users embody virtual avatars to explore virtual environments, they encounter various visual effects and content that can attract their attention~\cite{}.
During this, the user may notice an offset when he observes the virtual movement carefully while ignoring it when the virtual contents attract his attention from the movements.
Therefore, static offset thresholds are not appropriate in dynamic scenarios.

Past research has proposed dynamic mapping techniques that adapted to users' state, such as hand moving speed~\cite{frees2007prism} or ergonomically comfortable poses~\cite{montano2017erg}, but not considering the influence of virtual content.
More specifically, PRISM~\cite{frees2007prism} proposed adjusting the C/D ratio with a non-linear mapping according to users' hand moving speed, but it might not be optimal for various virtual scenarios.
While Erg-O~\cite{montano2017erg} redirected users' virtual hands according to the virtual target's relative position to reduce physical fatigue, neglecting the change of virtual environments. 

Therefore, how to design redirection techniques in various scenarios with different visual attractions remains unknown.
To address this, we investigate how visual attention affects the noticing probability of movement offsets.
Based on our experiments, we implement a computational model that automatically computes the noticing probability of offsets under certain visual attractions.
VR application designers and developers can easily leverage our model to design redirection techniques maintaining the sense of embodiment adapt to the user's visual attention.
We implement a dynamic redirection technique with our model and demonstrate that it effectively reduces the target reaching time without reducing the sense of embodiment compared to static redirection techniques.

% Need to be refined
This paper offers the following contributions.
\begin{itemize}
    \item We investigate how visual attractions affect the noticing probability of redirection offsets.
    \item We construct a computational model to predict the noticing probability of an offset with a given visual background.
    \item We implement a dynamic redirection technique adapting to the visual background. We evaluate the technique and develop three applications to demonstrate the benefits. 
\end{itemize}



First, we conducted a controlled experiment to understand how users perceived the movement offset while subjected to various distractions.
Since hand redirection is one of the most frequently used redirections in VR interactions, we focused on the dynamic arm movements and manually added angular offsets to the' elbow joint~\cite{li2022modeling, gonzalez2022model, zenner2019estimating}. 
We employed flashing spheres in the user's field of view as distractions to attract users' visual attention.
Participants were instructed to report the appearing location of the spheres while simultaneously performing the arm movements and reporting if they perceived an offset during the movement. 
(\zhipeng{Add the results of data collection. Analyze the influence of the distance between the gaze map and the offset.}
We measured the visual attraction's magnitude with the gaze distribution on it.
Results showed that stronger distractions made it harder for users to notice the offset.)
\zhipeng{Need to rewrite. Not sure to use gaze distribution or a metric obtained from the visual content.}
Secondly, we constructed a computational model to predict the noticing probability of offsets with given visual content.
We analyzed the data from the user studies to measure the influence of visual attractions on the noticing probability of offsets.
We built a statistical model to predict the offset's noticing probability with a given visual content.
Based on the model, we implement a dynamic redirection technique to adjust the redirection offset adapted to the user's current field of view.
We evaluated the technique in a target selection task compared to no hand redirection and static hand redirection.
\zhipeng{Add the results of the evaluation.}
Results showed that the dynamic hand redirection technique significantly reduced the target selection time with similar accuracy and a comparable sense of embodiment.
Finally, we implemented three applications to demonstrate the potential benefits of the visual attention adapted dynamic redirection technique.
\end{comment}

% This one modifies arm length, not redirection
% \citeauthor{mcintosh2020iteratively} proposed an adaptation method to iteratively change the virtual avatar arm's length based on the primary tasks' performance~\cite{mcintosh2020iteratively}.



% \zhipeng{TO ADD: what is redirection}
% Redirection enables novel interactions in Virtual Reality, including redirected walking, haptic redirection, and pseudo haptics by introducing an offset to users' movement.
% \zhipeng{TO ADD: extend this sentence}
% The price of this is that users' immersiveness and embodiment in VR can be compromised when they notice the offset and perceive the virtual movement not as theirs~\cite{}.
% \zhipeng{TO ADD: extend this sentence, elaborate how the virtual environment attracts users' attention}
% Meanwhile, the visual content in the virtual environment is abundant and consistently captures users' attention, making it harder to notice the offset~\cite{}.
% While previous studies explored the noticing threshold of the offsets and optimized the redirection techniques to maintain the sense of embodiment~\cite{}, the influence of visual content on the probability of perceiving offsets remains unknown.  
% Therefore, we propose to investigate how users perceive the redirection offset when they are facing various visual attractions.


% We conducted a user study to understand how users notice the shift with visual attractions.
% We used a color-changing ball to attract the user's attention while instructing users to perform different poses with their arms and observe it meanwhile.
% \zhipeng{(Which one should be the primary task? Observe the ball should be the primary one, but if the primary task is too simple, users might allocate more attention on the secondary task and this makes the secondary task primary.)}
% \zhipeng{(We need a good and reasonable dual-task design in which users care about both their pose and the visual content, at least in the evaluation study. And we need to be able to control the visual content's magnitude and saliency maybe?)}
% We controlled the shift magnitude and direction, the user's pose, the ball's size, and the color range.
% We set the ball's color-changing interval as the independent factor.
% We collect the user's response to each shift and the color-changing times.
% Based on the collected data, we constructed a statistical model to describe the influence of visual attraction on the noticing probability.
% \zhipeng{(Are we actually controlling the attention allocation? How do we measure the attracting effect? We need uniform metrics, otherwise it is also hard for others to use our knowledge.)}
% \zhipeng{(Try to use eye gaze? The eye gaze distribution in the last five seconds to decide the attention allocation? Basically constructing a model with eye gaze distribution and noticing probability. But the user's head is moving, so the eye gaze distribution is not aligned well with the current view.)}

% \zhipeng{Saliency and EMD}
% \zhipeng{Gaze is more than just a point: Rethinking visual attention
% analysis using peripheral vision-based gaze mapping}

% Evaluation study(ideal case): based on the visual content, adjusting the redirection magnitude dynamically.

% \zhipeng{(The risk is our model's effect is trivial.)}

% Applications:
% Playing Lego while watching demo videos, we can accelerate the reaching process of bricks, and forbid the redirection during the manipulation.

% Beat saber again: but not make a lot of sense? Difficult game has complicated visual effects, while allows larger shift, but do not need large shift with high difficulty



\section{Related work}
\mvspace{-2mm}
\paragraph{Human-AI complementarity.}

Many empirical studies of human-AI collaboration focus on AI-assisted human decision-making for legal, ethical, or safety reasons~\citep{bo2021toward, boskemper2022measuring, bondi2022role, schemmer2022meta}.
However, a recent meta-analysis by \citet{vaccaro2024combinations} finds that, on average, human–AI teams perform worse than the better of the two agents alone. 
In response, a growing body of work seeks to evaluate and enhance complementarity in human–AI systems \citep{bansal2021does, bansal2019updates, bansal2021most, wilder2021learning, rastogi2023taxonomy, mozannar2024effective}.
The present work differs from much of this prior work by approaching human-AI complementarity from the perspective of information value and use, including asking whether the human and AI decisions provide additional information that is not used by the other.
\mvspace{-2mm}
\paragraph{Evaluation of human decision-making with machine learning.}
Our work contributes methods for evaluating the decisions of human-AI teams~\citep{kleinberg2015prediction, kleinberg2018human, lakkaraju2017selective, mullainathan2022diagnosing,  rambachan2024identifying, guo2024decision, ben2024does, shreekumar2025x}.
\citet{kleinberg2015prediction} proposed that evaluations of human-AI collaboration should be based on the information that is available at the time of decisions.
% \jessica{can omit:} A significant portion of this literature addresses \textit{performative prediction}~\citep{perdomo2020performative}, where predictions or decisions affect future outcomes. 
% Because counterfactual decisions’ outcomes remain unobserved, researchers typically rely on worst-case analyses to bound the potential performance \citep{rambachan2024identifying, ben2024does}. 
% Though these issues arise in many canonical human-AI collaboration tasks, we focus on standard ``prediction policy problems'' where the payoff can be translated into policy gains~\citep{kleinberg2015prediction}.
According to this view, our work defines Bayesian best-attainable-performance benchmarks similar to several prior works~\citep{guo2024decision, wu2023rational,agrawal2020scaling, fudenberg2022measuring}. 
Closest to our work, \citet{guo2024decision} model the expected performance of a rational Bayesian agent faced with deciding between the human and AI recommendations as the theoretical upper bound on the expected performance of any human-AI team.
This benchmark provides a basis for identifying exploitable information within a decision problem.

\mvspace{-3mm}
\paragraph{Human information in machine learning.}

Some approaches focus on automating the decision pipeline by explicitly incorporating human expertise in developing machine learning models, such as by learning to defer~\citep{mozannar2024show, madras2018predict, raghu2019algorithmic, keswani2022designing, keswani2021towards, okati2021differentiable}.
\citet{corvelo2023human} propose multicalibration over human and AI model confidence information to guarantee the existence of an optimal monotonic decision rule.
\citet{alur2023auditing} propose a hypothesis testing framework to evaluate the added value of human expertise over AI forecasts.
Our work shares the motivation of incorporating human expertise, but targets a slightly broader scope by quantifying the information value for all available signals and agent decisions in a human–AI decision pipeline.



\section{Design Analysis}
Combining the literature review, we work closely with two environmental experts, \textbf{E1} and \textbf{E2}, who are also co-authors and are currently leading an environmental study in Lyudao, to understand the process of mining with DPSIR taxonomy. 
\sam{
\marginnote{$\triangle$\_4\_1}While both experts seek to leverage LLMs for text mining, they have also encountered limitations in tools like ChatGPT~\cite{chatgpt} for controlling LLMs. 
}
Together, we identify four technical challenges in prompt-based text mining:

\paragraph{\textit{Designing dataset-scale prompts}} Designing prompts for mining a dataset requires a unique set of prompting techniques, as opposed to interacting with conversational agents such as ChatGPT~\cite{chatgpt} or Claude~\cite{claude}. For example, our experts have tried to insert a document in ChatGPT and ask for insights on DPSIR, but the results were not satisfying. Even on a single document, they need to engage in multiple rounds of conversation to calibrate the model's understanding of the DPSIR definitions. Furthermore, the various topics discussed in different documents would significantly alter the model's understanding of the DPSIR definitions, and the calibration needs to be repeated on every document. 
It is thus essential to provide support for dataset-scale prompting.
% This makes it challenging to converge the DPSIR definitions from multiple documents. 
% Support for making sense of the corpus topics at a glance is thus essential to efficiently develop the DPSIR taxonomy. 

\paragraph{\textit{Decomposing the mining task}} 
Since the DPSIR mining task may be too complex for current LLMs to solve in one shot, a decomposition step~\cite{khot2023decomposed} is needed to break it into subtasks that better align with the model's capabilities. The resulting pipeline needs to support the iterative refinement of the DPSIR taxonomy for the environmental experts. This means that the subtasks should not only remain meaningful to the experts, enabling them to interpret the results without knowledge of complex prompting or ML techniques, but also be evaluable from a technical perspective.

\paragraph{\textit{Supporting evaluation and exploration}}
To support the progressive update of DPSIR taxonomy in the prompting pipeline, evaluation of the mining results and exploration of the corpus topics are the two key tasks to support. 
While topic exploration is a relatively well-studied task, evaluation without ground-truth labels remains a technical challenge. 
Moreover, the evaluation and topic should be presented in a way that caters to their background, as previous studies have shown how non-AI experts can misinterpret evaluation metrics and topic modeling results~\cite{liao2020questionAI, lee2017humantouch}.

\paragraph{\textit{Sharing the insights}} Communication of the DPSIR insights is an important purpose of applying the framework, but the mined insights are usually scattered across the corpus, and experts need to manually organize them into a form suitable for presentation and discussion. During the presentation, other experts might raise questions or propose potential ideas. However, the manual organization means the insights are static and not responsive to the discussion topics. The mining results should be automatically organized and presented in an interactive visualization.

In response to these challenges, we derive the following set of design requirements:
\begin{itemize}[label={}, leftmargin=9px, itemsep=1px, topsep=2px]
    \item \textbf{R1: Infuse domain knowledge input with established prompting methodologies.} 
    Our target users, the environmental experts, are not expected to be experienced prompting practitioners. The system should only expose components that require domain knowledge input for interaction, and seamlessly integrate it with established prompting methodologies. 
    \item \textbf{R2: Support exploration and evaluation simultaneously.} 
    To define and contextualize the DPSIR taxonomy, the expert needs to thoroughly understand the corpus, which can be cognitively demanding. The system should support the experts to use the current DPSIR taxonomy to reveal uncovered content, and iteratively refine the taxonomy. 
    % \item \textbf{R3: Support evaluation of prompt performances.}
    At the same time, given a DPSIR taxonomy and the corresponding prompts, the experts should be able to evaluate the prompts to make sure the prompting results align with their intentions. 
    The exploration of the corpus topics and the evaluation of the prompts are intertwined, and the system should support these two tasks simultaneously.
    \item \textbf{R3: Prioritize recall over precision.}
     Considering the complexity of any environmental context, the validity of any insight can only be validated by human experts. The system should be designed to support a human-in-the-loop insight discovery, rather than aiming at extracting precise insights. Therefore, the system should prioritize recall over precision to ensure that the experts fully examine the corpus.
    \item \textbf{R4: Support collaborative discussion on mining results.} 
    After the analysis, the experts typically host internal discussions or workshops to ensure the rigorousness of the findings before communicating them to policymakers. An interactive visualization can offer graphics support for the discussion as well as querying functionalities to quickly validate ideas. Also, clarity and simplicity should be prioritized over information density to avoid distracting the discussion.
\end{itemize}

\noindent Next, we introduce how this set of design requirements informs our technical and design choices.

\begin{figure*}[t]
    \centering
    \includegraphics[width=\textwidth]{pipeline}
    \caption{The system design of \system. Starting with a corpus, the documents are first segmented into snippets. Then, the snippets go through an interactive prompting pipeline consisting of three steps: Indicator, Variable, and Link Identification. In each step, the environmental experts can use the uncertainty chart to explore the corpus, evaluate the prompting performances, and refine the DPSIR taxonomy. After the mining, an interactive DPSIR Graph can be used for collaborative discussion and communication of the findings.}
    \label{fig: pipeline}
  \vspace*{-0.3cm}
\end{figure*}

\section{\system: System Design}
Informed by the design requirements, we develop \system\footnote{\url{https://github.com/SamLee-dedeboy/GreenMine}}, which implements a three-step prompting pipeline for the mining of DPSIR, and an interface with visualizations to support the interactive mining. The interface has two main views: \textit{Prompt View}, which supports interactive prompt refinement and execution, featuring an uncertainty chart for prompt evaluation; and \textit{DPSIR View}, which visualizes the mining results for collaborative discussions. \sam{
\marginnote{$\triangle$\_5\_1}All the LLM-related computation uses the ``gpt-4o-mini'' model from OpenAI, accessed through official API requests. 
}
% \sam{TODO: add open-source repo (anonymized)}

\paragraph{Dataset and Segmentation}
The system supports datasets of interview transcripts.
Our targeting dataset consists of 19 transcripts of interviews that the collaborating experts have conducted with the residents of Lyudao. 
\sam{
\marginnote{$\triangle$\_5\_2}
To avoid reaching the context length limits, we first segment each transcript into ``snippets'', i.e., consecutive conversations about a specific topic.} 
To account for the variety of topics in the transcripts, the segmentation method is based on a prompt template, where the semi-structured interview questions are inserted, and the model is instructed to identify topics for segmentation. To avoid hallucinations (e.g., generating sentences not in the transcripts), the model outputs conversation indices, which are then used to segment the transcripts programmatically. Sanity checks are conducted to validate the result. This results in 598 snippets, with on average 6 conversations and 379 words (Chinese characters). 

\subsection{Prompting Pipeline}
Mining insights with the DPSIR framework is complex and requires expert validation, and it is unreliable to mine insights with a single prompt.
In response, we develop a prompting pipeline in three steps: \textit{Indicator Identification}, \textit{Variable Identification}, and \textit{Link Identification}, as shown in~\autoref{fig: pipeline}.
Conceptually, the prompts convert each mining task into a multi-label classification task, in which the model is instructed to choose all applicable labels from a candidate list (\textbf{R3}). 
Environmental experts can refine the candidate list (i.e., the indicators, variables, and their definitions) to fit the research goal and dataset context (\textbf{R1}). 

\paragraph{Indicator Identification} 
In this step, the prompt instructs the model to identify occurrences of indicators in each snippet. The prompt template takes the indicator definitions and snippet content as input and outputs a list of occurred indicators, as well as the evidence sentences and a textual explanation. Note that we use ``concept'' instead of ``indicator'' in the prompt, since ``indicator'' can mean differently in the model's training corpus.  

\paragraph{Variable Identification} Based on the assigned indicators, a second prompt template identifies the variables for each indicator in each snippet. Experts can edit the variable list, or refine the definition of each variable. 
The template takes three inputs:
(1) the indicator and the variable list (and definitions),
(2) the snippet content, and
(3) the textual explanation generated in the first step.
 The model outputs a list of occurred variables, as well as the evidence and explanation. Similarly, we use ``tag'' instead of ``variable'' in the prompt. 

\paragraph{Link Identification} Based on the identified variables, the relationships between all pairs of variables (if exist) are mined. We formulate it as a binary classification task on each pair of variables, where the model outputs (relationships) ``exist'' or `not exist''  given a pair of variables. This formulation increases the computation cost, but the results are more reliable due to shorter prompts and easier instructions. The template takes a pair of variables (and definitions) and the snippet content as input, and outputs the source, target, relationship, evidence, and explanation, or ``None'' if no relationship exists. To maintain the creativity of the model, we do not specify a list of predefined relationships in the prompt. During experiments, we observe some frequent relationships, such as ``positive/negative correlation'', ``causality'', ``interconnected'', or ``health-affecting'', showing that the model can identify and distinguish various kinds of relationships without human intervention.

\subsection{Uncertainty Estimation}
Unlike conventional text mining, there is often no labeled dataset available in prompt engineering. While some researchers have proposed to use pseudo-labels~\cite{malik2024pseudolabel}, it is still not feasible in DPSIR mining because the labels are incrementally enriched and contextualized as experts investigate the corpus. 

Given these unique characteristics, we draw inspiration from a recent work that estimates the uncertainty of LLM outputs by sampling multiple outputs for the same input~\cite{chen2024quantifyinguncertaitny}. 
\sam{
\marginnote{$\triangle$\_5\_3}Conceptually, we execute every prompt $k$ times and measure the consistency between outputs. 
Specifically for our prompting pipeline, since each step is formulated as a multi-label classification task (e.g., the response of \textit{Indicator Identification} can be any power set of $\{D, P, S, I, R\}$), we use set similarity metrics such as Jaccard Distance~\cite{jaccard1912jaccardsimilarity} to measure the consistency between $k$ response.
More generally, we recommend using information entropy~\cite{shannon1948informationentropy} for single-label classification tasks or semantic uncertainty~\cite{cheng2024relic} for unstructured outputs.
}

Using \textit{Indicator Identification} as an example, $A_{ij}$ denotes the identifed indicators for snippet $s_i$ in iteration $j \in (1, 2,...k)$, and $\forall a \in A_{ij}, a \in \{D, P, S, I, R\}$. Then we use average pairwise Jaccard Distance to compute the uncertainty $D_i$ for snippet $s_i$:
\vspace*{-0.1cm}
\begin{equation}
\begin{split}
    D_i = \displaystyle \sum_{1 \leq j_1 < j_2 \leq k}\frac{J(A_{ij_1}, A_{ij_2})}{k(k-1)/2}, \\
    where \ J(A, B) = \frac{|A \cup B| - |A\cap B|}{|A \cup B|}
\end{split}
\end{equation}
For variables, $a \in A_{ij}$ are the variables specified by the environmental experts in the form of strings. For links, we consider $a_1=a_2 \Leftrightarrow (src_1, dst_1)= (src_2, dst_2)$, where $(src, dst)$ are the variables, i.e., we do not consider the relationship when matching two links. 
% This is because to allow the model to be creative in assigning relationships, we did not specify a predefined list of relationships, and the model might output semantically the same answers in different forms, such as ``interconnected'' and ``interdependent''. 
Using this uncertainty estimation, we are able to support the evaluation of prompting performances on all subtasks (\textbf{R2}).

\sam{\marginnote{$\triangle$\_5\_4}Note that in \system, we set the GPT model's temperature to $0$ for maximal determinism. Still, uncertainty scores can reach $0.8$ in some cases, revealing significant response inconsistency and showing the necessity of consistency estimation. 
Although calculating uncertainty theoretically introduces up to $k$ times overhead by repeating each prompt $k$ times, we leverage OpenAI's extensive cloud resources to reduce the overhead, by sending all the API requests for one execution concurrently. Since OpenAI's computing clusters likely have the resources to handle all API requests simultaneously, the bottleneck of the system is simply Network I/O. 
In practice, uncertainty calculation adds only a $1.5x$ overhead when $k=5$.
}

\begin{figure*}[t]
     \centering
    \includegraphics[width=\textwidth]{uncertainty_chart_improvements}
    \caption{The process of generating the uncertainty chart. The dots are positioned using polar coordinates. The angles encode semantic similarity, and the radius encodes uncertainty. For angles, we start by generating embeddings from the snippets and calculate a semantic distance matrix using cosine distance. For different steps in the pipeline, we use the corresponding evidence sentences in each snippet to generate the embeddings. The distance matrix is optimized with the objective function defined in~\autoref{eq: dr}, resulting in angles for each snippet. The snippet embeddings are then clustered using agglomerative clustering and aggregated. The radial space is then divided proportionally by the cluster sizes. Then we employ a force-directed layout, with a collision force to avoid node overlapping, and a radial force that attracts nodes to their radius in the polar coordinate. The design considers the visual clarity of the layout while encoding uncertainty and semantics in one chart.}
    \label{fig: uncertainty_chart_improvements}
  \vspace*{-0.5cm}
\end{figure*}

\subsection{Uncertainty Chart}
\sam{
\marginnote{$\triangle$\_5\_5}Since the mining results can be overwhelming to make sense of and evaluate, we design an uncertainty chart (\autoref{fig: uncertainty_chart_improvements}) to organize the mining results with topics and uncertainties (\textbf{R2}).
}
 Each circle on the chart represents a snippet. The position $(\theta, r)$ is calculated in polar coordinates, where the angle $\theta$ encodes the semantics of the snippet's content and the radius $r$ encodes the uncertainty score. As a result, semantically similar snippets are placed together to form clusters, and the uncertain snippets are placed at the peripherals. 

Incorporating semantics into the visualization is a design choice motivated by the progressive characteristic of DPSIR text mining. 
In the beginning, users have a limited understanding of the corpus topics, and they would refine the DPSIR taxonomy as they explore. 
Once the taxonomy is refined, they would re-execute the prompt, evaluate its performance, and then learn more about the corpus.
The uncertainty chart design supports this interleaved sensemaking and evaluation (\textbf{R2}).
The chart supports two kinds of visual exploration. To find missing definitions for the current DPSIR taxonomy, users can investigate uncertain snippets, which are outliers in the peripheral region. To verify that the model's understanding is aligned with the expert's intent, users can investigate certain snippets, which are grouped in the center region.
Next, we introduce the semantic angle calculation and how the visual clarity of the chart is improved.

\subsubsection{Semantic Angle Calculation} 
\sam{
\marginnote{$\triangle$\_5\_6}
The semantic angle calculation combines document embedding models, dimensionality reduction, clustering, and force-directed layout (\autoref{fig: uncertainty_chart_improvements}). The goal is to find angles that maximally preserve the semantic similarities between snippets while reducing overlaps.
}
\paragraph{Capturing semantics}
We use document embeddings to capture the semantic meanings of each snippet. Instead of using the snippet conversations, we use the evidence sentences and explanations generated by LLMs as the embedded content. This way, the embeddings capture only the semantics relevant to the prompting instructions, e.g., sentences and explanations that suggest the occurrence of a \textit{Driver} when conducting indicator identification. We use OpenAI's ``text-embedding-3-small'' model with 1536 dimensions for this step.

\paragraph{Dimensionality reduction into radial space}
Core to the semantic angle calculation is preserving the high-dimensional distances in a single dimension. Different from a 1D t-SNE~\cite{van2008tsne} or MDS~\cite{cox2000mds}, the distances in the reduced dimension should be \textit{circular}, i.e., in the range of $[0, 2\pi)$, points at $\theta+\epsilon$ and $2\pi-\theta-\epsilon$ are of equal distance with points at $\theta$.
To achieve this, we formulate the reduction similar to MDS, but instead of mapping points into 2D or 3D space, we map them into a 1D circular space.
Given a distance matrix $D$, where $D_{ij}$ is the cosine distance between the embeddings of snippet $i$ and snippet $j$, the objective is formulated as:
\begin{equation}\label{eq: dr}
    \begin{split}
    \min_{\theta_1,\theta_2,\theta_3,...\theta_n}\displaystyle \sum_{i=1}^{n}\sum_{j=i+1}^{n}(d_{ij} - D_{ij})^2, \\
    d_{ij}=1 - cos(min(|\theta_i - \theta_j|, 2\pi - |\theta_i - \theta_j|)), \\
    \theta_i \in [0, 2\pi], \quad i = 1,2,3,\dots, n
    \end{split}
\end{equation}
We optimize for $\theta_i$, such that the squared difference between $d_{ij}$ and $D_{ij}$ is minimized. 
The calculation of $d_{ij}$ guarantees the circular characteristic and rescales to match $D_{ij}$. The objective is optimized with Limited-memory BFGS (L-BFGS)~\cite{liu1989limited}.
Next, we introduce how cluttering is reduced with clustering and force-directed layout.

\paragraph{Clustering}
We first apply Agglomerative Clustering~\cite{steinbach2000doccluster} on the embeddings using cosine similarity with a threshold of 0.5, which assigns a cluster to each snippet. With the clusters and angles from DR, we calculate the mean angle for each cluster and use the mean to sort the clusters, and then distribute the space in $[0, 2\pi]$ proportional to the cluster sizes. This maintains the proximity of semantically similar clusters. Since each cluster represents a topic, we want to support users to make sense of the topics at a glance. We use a prompt to assign topic labels to clusters using the snippet contents, which are attached to respective cluster regions in the chart.

% : \textit{``You are a topic assignment system. The user will provide you with a list of texts. You need to assign one topic to summarize all of them. The topic should be a simple noun phrase.''}. 
% The labels are later attached to respective cluster regions.

\begin{figure*}[t]
    \centering
    \includegraphics[width=\textwidth]{case_study}
    \caption{The prompting interface consists of three panels: (a) The Entry Panel, where users can adjust DPSIR taxonomy definitions and prompting templates. (b) The Result Panel, where users can choose to show the results in a list view or an uncertainty chart. (c) The Comparison Panel can be used to compare two prompts side-by-side. (d) The Document Panel shows the corpus contents. Using this interface, \textbf{E1} adds ``improving living conditions'' in the definition of response in (e), and adds ``culture security'' and ``garbage'' in the Driver variables in (f). (g) Users can also navigate to the evidence snippets, which are annotated with explanations generated by LLMs. }
    \label{fig: case_study}
    \vspace*{-0.6cm}
\end{figure*}

\paragraph{Force-Directed Layout}
Finally, we use a force-directed layout~\cite{fruchterman1991forcelayout} to reduce node overlapping.
At rendering time, all circles are initialized with their coordinates $(\theta_i, r_i)$. 
We apply two forces on the nodes: (1) a collision detection force that repels overlapping nodes, (2) a radial position force that attracts each node to its radius. We also implement clipping to limit the nodes in their allocated cluster spaces. 
\sam{\marginnote{$\triangle$\_5\_7}Although force-directed layout might trade accuracy for visual clarity, it is worth noting that the chart construction process provides relatively large flexibility in node placement: First, the uncertainty score (main encoding of the chart) is encoded by the radius of the polar coordinate, so nodes can be placed at varying angles without changing the uncertainty score encoding. By incorporating the radial position force, the layout prioritizes reducing overlaps with angles rather than radius. Second, semantics of the snippets (encoded by angles) is inherently vague, and it is less likely to cause confusion when the angles are adjusted.} Together, node overlapping is reduced with minimum adjustments to coordinates. 

\vspace*{-0.1cm}
\subsection{Prompting Interface}
The prompting interface consists of three panels: an \textit{Entry Panel} for users to edit DPSIR taxonomy and prompts; a \textit{Result Panel} where users can inspect the prompting results using the uncertainty chart or a list view; and a \textit{Comparison Panel} where users can select a previous version of prompting result side-by-side.
\sam{
\marginnote{$\triangle$\_5\_8}All three steps in the pipeline have an Entry Panel, a Result Panel, and a Comparison Panel. On top of that, some UI adjustments are made to better support each step, e.g., the list view for \textit{Link Identification} provides the Link Graph. 
% The uncertainty chart also supports various filters: users can filter by indicators (\autoref{fig: uncertainty_chart}), variables, or links at each step of the pipeline.
}
Next, we introduce each panel and its interactions. 


\begin{figure*}[t]
    \centering
    \includegraphics[width=\textwidth]{simple_vis}
    \caption{\textbf{Left}: Keyword cloud designed to support the exploration of ``miscellaneous'' variables. The keyword cloud combines embeddings and dimensionality reduction to position semantically similar keywords closer. It shows that ``garbage'' and ``environment'' are two significant keywords in the miscellaneous variables for Drivers. \textbf{Right}: The link graph design helps users make sense of complex links in a snippet. The graph shows that ``Activities of Physical Destructions'' is correlated with ``Changes in Natural Habitats'', ``Coral Bleaching'', and other variables.}
    \label{fig: simple_vis}
    \vspace*{-0.5cm}
\end{figure*}

\vspace*{-0.15cm}
\subsubsection{Entry Panel}
The Entry Panel supports four functionalities: version control, DPSIR taxonomy input, prompt input, and rule input (\autoref{fig: case_study}-a). The panel supports managing different versions of inputs for each step. Each version has a set of DPSIR definitions, prompts, and the last execution result. 
The versions are also necessary when specifying data dependency between steps in the pipeline.
% When executing a later step, users must choose a version from the previous step as the basis.
\vspace*{-0.1cm}
\paragraph{Taxonomy and Prompt Input}
In the first two prompting steps, \textit{Indicator Identification} and \textit{Variable Identification}, users can edit the indicator or variable definitions in natural language, add or remove variables, or specify the variable type (societal or environmental). 
Each prompting step has a unique template, in which the dataset and application context can be specified. Each template is divided into three sections: \textit{Persona}, \textit{Context}, and \textit{User}. This section division follows the state-of-the-art prompting methodologies~\cite{promptengineeringuide} and helps users make sense of the template. 
Users can use a special ``\$\{\}'' tag to specify a variable insertion into the template. 
These specifications are inserted into the prompt templates automatically.

\vspace*{-0.1cm}
\paragraph{Rule Entry} The rules enable users to manually modify prompting results and automatically apply them to all versions of results. Each rule should specify a target snippet, a condition (``must'' or ``must not'' have), and a value, which can be indicators, variables, or links. Note that the rules are not used in the prompts or as few-shot examples. We implement this feature for users to cover corner cases where the model fails to capture some nuances.

\vspace*{-0.15cm}
\subsubsection{Result Panel and Comparison Panel}
Result and Comparison Panel (\autoref{fig: case_study}-b, c) support the evaluation of prompt performances (\textbf{R2}) by showing the results using the uncertainty chart or a list view.
The Result Panel always displays the current version, while the Comparison Panel allows selecting any version. We also designed two visualizations—Keyword Cloud and Link Graph—to aid sensemaking in variable and link identification.

\vspace*{-0.1cm}
\paragraph{List View}
The list view shows all results sorted by uncertainty (high to low). By clicking on a result, user can quickly navigate to the evidence and explanations in the Document Panel (\autoref{fig: case_study}-g). 

\vspace*{-0.1cm}
\paragraph{Keyword Cloud}
During variable identification, we reserve a label for ``miscellaneous'' to cover everything that the experts have not considered. Adding new variables by iteratively investigating the ``miscellaneous'' variables is the core of refining the DPSIR variable lists. To support this, we aggregate the keywords from the evidence sentences for results containing ``miscellaneous'', and implement a semantic word cloud~\cite{xu2016semanticwordcloud} using embeddings and Kernel PCA~\cite{scholkopf1997kernel}, as shown in~\autoref{fig: simple_vis} (left).
\sam{
\marginnote{$\triangle$\_5\_9}Users can switch to show the keyword clouds by clicking the button located at the top right of the result panel for \textit{Variable Identification}.
}
We use color and size to double-encode the word frequencies, and use proximity to encode semantic with cosine similarity.
The design also considers maintaining the visual continuity between multiple rounds of iteration.
Specifically, we want the semantic landscape to remain consistent.
Thus, we need a \textit{parametric} dimensionality reduction method, where an explicit mapping function (i.e., projection to low-dimensional space) can be reused, excluding popular non-parametric methods like t-SNE~\cite{van2008tsne} or MDS~\cite{cox2000mds}.
Kernel PCA not only satisfies this constraint, but also captures the non-linear semantic relationships. Similar to the uncertainty chart, we use a collision force with rectangular overlap detection to prevent keyword overlapping.

\vspace*{-0.1cm}
\paragraph{Link Graph}
During link identification, the system supports the visualization of the result for a snippet in a node-link diagram (\autoref{fig: simple_vis}-right), which makes it easier to understand the result and identify graph-related patterns than in a list. Each node is a variable colored by their indicator types. Node radius encodes degree (frequency), and link opacity encodes uncertainty. The links are annotated with the relationship label extracted by the model, and users can click on the links to navigate to the evidence in the Document Panel. Since the graph size is typically small, we use a force-directed layout and implement node dragging in case the labels are occluded. 


\subsection{Visualizing links in DPSIR Graph}
During the mining tasks, the system prioritizes to visualize only a selective subset of results. After the mining, the system can visualize all mined links in the DPSIR Graph, where each node is a variable. 
As shown in~\autoref{fig: DPSIR}, the DPSIR Graph is designed specifically for the DPSIR framework, extending common visualizations created in the environmental science community~\cite{atkins2011dpsir}, featuring a radial layout and progressive disclosure support. It is designed for quick validations and sensemaking during a collaborative discussion (\textbf{R4}). 

\vspace*{-0.1cm}
\paragraph{Responsive Radial Layout} 
Using the cyclic characteristic of the DPSIR framework, the graph is organized in a radial layout. The design follows existing diagrams in environmental studies~\cite{atkins2011dpsir} and emphasizes simplicity and interactivity to better support collaborative discussion.
Each indicator is assigned a unique color consistent with the rest of the system. To lay out each block of indicator, we divide $[0, 2\pi]$ proportionally to the total degree of variables in each indicator (D, P, S, I, R), and put the block at the center of the allocated space. In this formulation, the system can support users to exclude irrelevant indicators during discussion sessions. 
For example, users can hide \textit{States} and \textit{Imapcts} if the discussion is not focusing on these indicators (\autoref{fig: DPSIR}-right), and the layout will automatically redistribute the space to place each block evenly.

\vspace*{-0.1cm}
\paragraph{Progressie Disclosure} 
 Each block of indicators can be interactively ``opened'' to reveal the variables (\autoref{fig: DPSIR}-right), organized in a squared layout to save the central space for internal links. The color saturation of each variable encodes degree (frequency of occurrence). To prevent external links from concentrating in a small space, we put the four highest-degree variables at the four corners of each block. This also helps users locate the most significant variables. The link width and opacity between each variable double-encode the intensity (frequency of the link). Link color is determined by the source indicator type.
  Users can click on any variables or links to highlight them, and navigate to the evidence in the Document Panel.
 The interaction design allows users to focus on relevant subsets of the DPSIR graph during discussions, 
 and seamlessly navigate to supporting evidence in the documents for detailed understanding. 

\section{Case Study}
To clearly demonstrate EvoStealer's advantages over baseline methods, we select an easy and a hard example for case study, with the results shown in Figure~\ref{fig:case}. The results show that, on in-domain data, EvoStealer generates images that closely match the style of the original images, with all four synthesized images maintaining stylistic consistency. In contrast, the four images generated by the other baseline methods exhibit significant style variation. On out-of-domain data, EvoStealer maintains the same style as in-domain images, successfully achieving subject generalization. In contrast, the other baseline methods fail to generalize. Additionally, we analyze three distinct failure cases (see Appendix~\ref{app_failcases} for details).
% 为了更直观的体现EvoStealer相对于baselines的优势,我们选取了一个简单和一个困难的例子进行样例分析。结果如图3所示。我们可以观察到,在in-domain数据上,EvoStealer能够合成和原图风格非常接近的图片,并且合成的4张图片风格一致。相比之下,其他3个baselines合成的4张图片则风格迥异。在out-of-domain数据上,EvoStealer生成的图片仍能和in-domain上的图片保持一致的风格,成功地实现了主体的泛化。相比之下,其他baselines则无法实现泛化。
\vspace*{-0.15cm}
\section{Expert Review}
To better understand the potential and limitations of our approach, we demonstrated the system to experts participating in the Lyudao project.
In addition to \textbf{E1} and \textbf{E2}, we invited three more experts and host a discussion session. \textbf{E3} is a senior researcher in environmental science and is in charge of the interview that \textbf{E1} analyzed. \textbf{E4} is a research assistant with a Master's degree in environmental science, and \textbf{E5} is a PhD student in earth science. None of the three experts had seen the system before the discussion.

The procedure started with a brief introduction of the overarching system goal and the interface design. Then, \textbf{E1} presented his findings from the usage of the system, using both the uncertainty charts and the DPSIR Graph. During the presentation, we encouraged all experts to ask questions and provide their thoughts. The discussion lasted over 2 hours and was video-recorded. Below, we summarize the expert feedback and discuss the implications.

\vspace*{-0.1cm}
\paragraph{System design is well-received}
The system design is clear and intuitive. Despite lacking technical knowledge in prompt engineering or text mining, the experts found the three-step mining pipeline sensible and easy to evaluate. They easily understood the uncertainty chart's visual encodings. During \textbf{E1's} presentation, he effectively used the DPSIR Graph without external tools, demonstrating its suitability for collaborative discussions. Additionally, \textbf{E1}'s week-long usage without issues highlights the usability of the prompting functionality and uncertainty chart.

\vspace*{-0.15cm}
\paragraph{Complementing literature review}
All experts agreed that the system is an effective tool to provide unexpected insights after seeing \textbf{E1's} findings. \textbf{E4} was especially fond of the automatic support, knowing the time and effort needed to manually analyze these transcripts. \textbf{E3} agreed that \textit{``Looks like we are scattering the transcripts and then reorganizing them in response to our research questions with better and clearer definitions (compared to literature review).''}
\textbf{E1} summarized from his experience that computational extraction reduces the cognitive bias that human experts might have from literature knowledge.
This shows the system not only streamlines transcript analysis but also mitigates cognitive biases, improving the objectivity and precision of the derived insights.

\begin{figure}[]
     \centering
    \includegraphics[width=\columnwidth]{uncertainty_chart}
    \caption{An uncertainty chart generated in the case study.
    Each dot in the chart represents a snippet that mentions the indicator ``Response'', positioned with polar coordinates. 
    From the chart, most topics fit \textbf{E1's} expectation except a snippet under ``Little Vendor Dream'' with high uncertainty. He clicks the snippet to inspect the relevant conversations and the LLM-generated explanation. 
    He finds that the snippet has high uncertainty because the Response definition is incomplete, 
    so he refines the definition accordingly. 
    }
    \label{fig: uncertainty_chart}
  \vspace*{-0.6cm}
\end{figure}

\begin{figure*}[t]
    \centering
    \includegraphics[width=\textwidth]{DPSIR}
    \caption{\textbf{Left}: The DPSIR Graph shows the aggregated mining results in a progressive graph. The design follows the typical DPSIR diagram in environmental studies. The Driver block is highlighted with ingoing and outgoing links colored by the source of the link. \textbf{Right}: The States and Impacts are hidden, and the rest of the indicators are revealed to show the variables and their links. \textbf{E1} found that contrary to the literature, ``Economy'' and ``Transportation'' are two significant drivers with many linkages, and ``Extreme Weather'' is a significant pressure. }
    \label{fig: DPSIR}
    \vspace*{-0.5cm}
\end{figure*}

\vspace*{-0.15cm}
\paragraph{Informing policymaking}
During the discussion, \textbf{E2} explored how the DPSIR Graph could guide policymaking, focusing on \textit{Response} variables to gauge public opinion on current policies. \textbf{E1} agreed, suggesting the inclusion of other relevant variables: \textit{``For example, for fishery policies, we can identify connected variables, indicating where policies should focus ({E1}).''} This highlights the DPSIR Graph’s potential to inform policy by identifying key variables and incorporating public opinion for data-driven decisions.

\vspace*{-0.15cm}
\paragraph{Facilitating public-facing presentation}
\textbf{E3}, leading interview recruitment, was keen on improving public outreach with the system, noting that Lyudao residents historically distrust government-funded research. \textbf{E2} agreed, stating, \textit{``Residents want to feel heard and see their opinions considered. This tool could help demonstrate that.''} Still, \textbf{E3} emphasized the need to simplify the DPSIR Graph to match the public's visualization literacy. Balancing these views, the system shows potential as a bridge between research and the public.

\vspace*{-0.15cm}
\paragraph{Comparing detailed research questions}
While the system is designed for exploratory purposes in mind, it could also be applied to study and compare concrete research questions. As suggested by \textbf{E3}, users can give very narrow indicator definitions and short lists of variables to focus on only one research question: \textit{``We can first set a few research questions, for example, pressures on oil and gas, or pressures on the fishery, and then adjust the definitions accordingly. We can even compare different versions of definitions, which might reveal more focused insights. Such insights would also be more suitable for the public to digest.''}
This adaptability highlights the system's potential to support detailed analyses.
\section{Limitations}
While \system \ has proven effective for our collaborating experts, we discuss limitations that could constrain its applicability.
\sam{
\paragraph{Incorporating multiple sources of data}
\marginnote{$\triangle$\_8\_1}The system currently supports only a specific data format for interview transcripts, which is quite limited given LLMs' ability to handle various formats. Extending support for other formats require updates to several system modules, such as the segmentation module and Document Panel.
}
\paragraph{Computation Scalability}
The system is limited in computational scalability. 
The dataset on Lyudao is segmented into 598 snippets with an average length of 379 Chinese characters, which is not a large dataset. During the mining, we observed an average runtime of 150 seconds for indicator and variable identification respectively, 
\sam{\marginnote{$\triangle$\_8\_2}}and 600 seconds for link identification. \sam{Note that in our implementation, the LLM is accessed through network APIs and optimized with multithreading to reduce uncertainty calculation overheads. For local LLM inferences, the overheads could not be reduced and the runtime would drastically increase. 
}
\paragraph{Visual scalability}
The uncertainty chart and DPSIR graph face visual scalability issues. \sam{
\marginnote{$\triangle$\_8\_3}
With nodes clustering in narrow ranges or with excessive topics, the force directed layout might push nodes too far away from their original positions, affecting the accuracy of the uncertainty encoding. 
The DPSIR graph relies on interactions for clarity as the number of nodes increases. 
While suitable for collaborative discussions, it may not work well for static displays like posters, limiting its applicability to larger datasets or more display environments.
}
\sam{
\paragraph{Limitations in evaluation}
\marginnote{$\triangle$\_8\_4}The system's evaluation is based solely on feedback from a small group of collaborating experts, which may not be comprehensive. While feedback has been positive, the experts received a detailed tutorial from the authors. Without the tutorial, users unfamiliar with LLMs or advanced visualization—likely common in the environmental science community—may find the system challenging to use. Additionally, the system has not been tested with other LLMs, making it unclear if model choices would impact the outcome and overall performance.

\paragraph{Ethical considerations}
\marginnote{$\triangle$\_8\_5}Finally, potential ethical concerns~\cite{weidinger2022llmrisk}, such as privacy and biases, remain unaddressed. During \system's development, our collaborating experts raised data privacy concerns, prompting us to de-identify interview transcripts. While this is standard for transcripts, similar standards may not exist for other formats like field reports, leaving privacy a concern. Additionally, LLMs may exhibit biases from their training corpus, potentially undermining their ability to fairly extract insights from large datasets.
}
\sam{
\section{Implications beyond Environmental Study}
}
During the development of \system, we addressed several technical challenges that may also exist in other application scenarios. In this section, we discuss the lessons learned that inform visual analytics system developers beyond environmental study. 

\paragraph{Uncertainty evaluation for knowledge-intensive tasks}
Our human-in-the-loop evaluation using uncertainty has proven effective for DPSIR mining tasks. While LLMs perform well on general tasks, their capabilities on knowledge-intensive tasks remain limited due to insufficient training data and the complexity of eliciting knowledge as prompts~\cite{harvel2024llmknowledge}. By leveraging uncertainty estimation, knowledgeable users can iteratively refine prompts until uncertainty is minimized. 
% Although not all knowledge can be externalized, our approach demonstrates feasibility when it is possible. 
Integrating evaluation with exploration aids this process, providing actionable insights for prompt refinement. Thus, the uncertainty chart also suits other knowledge-intensive applications.
% In our work, the human-in-the-loop evaluation based on uncertainty has shown its effectiveness for DPSIR mining tasks. In broader application scenarios, despite strong performance in many general-purpose tasks, LLM's performance on knowledge-intensive tasks remains questionable~\cite{harvel2024llmknowledge}. The challenge is that most knowledge-intensive tasks lack sufficient training material, and the knowledge is too complex to be elicited as prompts. Using uncertainty estimation and human-in-the-loop evaluation, knowledgable users can iteratively externalize their knowledge in prompts until the uncertainty is low. While it may still be challenging to externalize all knowledge, our work shows that such an approach is feasible when the knowledge can be properly externalized. In particular, our design decision on integrating evaluation and exploration facilitates such a process and ensures that users can get actionable insights on how the knowledge could be better externalized. As a result, we believe that the uncertainty chart is applicable to other application scenarios with knowledge-intensive tasks.

\paragraph{Progressive taxonomy construction in thematic analysis}
The data mining in the DPSIR framework resembles that of thematic analysis, facing similar technical challenges. In thematic analysis, analysts begin with an initial codebook and progressively refine it~\cite{fereday2006thematic}. Some researchers have used LLMs to aid this process~\cite{dai2023llmintheloop, yan2024chatgptthematic}, but they encounter reliability and consistency issues due to inadequate feedback loops. Our approach addresses this by combining uncertainty evaluation and topic exploration in the feedback loop, offering validation and actionable feedback. The insights from our work can improve LLM-facilitated thematic analysis, particularly in enhancing reliability and consistency.

\sam{
\paragraph{Incorporating semantic uncertainty}
For broader applicability, we can use semantic uncertainty to incorporate free-form mining results.
\marginnote{$\triangle$\_9\_1}
In our work, we have demonstrated the benefit of using uncertainty metrics to support users evaluate LLM responses.
To adapt to scenarios beyond classifications, e.g., in extracting facts from a corpus, 
}the uncertainty chart can be extended with semantic uncertainty~\cite{kuhn2023semanticuncertainty, cheng2024relic}, which uses linguistic variances to measure the uncertainty of textual responses. 
With this, the uncertainty chart can be extended to more mining scenarios, providing robust uncertainty evaluation support for both structured and unstructured responses.


\section{Conclusion}
In this paper, we present \system, a system designed to facilitate progressive taxonomy construction based on the DPSIR framework for environmental studies. The system supports the interactive execution of a three-step prompting pipeline, where domain specifications can be inserted in prompts by experts. To support the evaluation of the prompts, we introduce an uncertainty chart that visualizes corpus topics and prompt output consistency. The uncertainty chart, along with other visualizations, supports interleaved evaluation and exploration to provide actionable feedback. Our proposed solution and the lessons learned offer valuable insight into supporting human-in-the-loop text mining beyond environmental studies.

\newpage

%% if specified like this the section will be committed in review mode
\acknowledgments{
This research is supported in part by the National Science Foundation via grant No. IIS-2427770, and by the University of California (UC) Multicampus Research Programs and Initiatives (MRPI) grant and the UC Climate Action Initiative grant.
}
%\bibliographystyle{abbrv}

\bibliographystyle{abbrv-doi}
%\bibliographystyle{abbrv-doi-narrow}
%\bibliographystyle{abbrv-doi-hyperref}
%\bibliographystyle{abbrv-doi-hyperref-narrow}
\balance
\bibliography{template}
\end{document}
