\section{Introduction}
Human activities have a profound influence on the environment, which, in turn, shape policy-making and human behavior.
The DPSIR framework (\textbf{D}river, \textbf{P}ressure, \textbf{S}tate, \textbf{I}mpact, \textbf{R}esponse) is commonly used in environmental science to study and communicate the complex relationships between societal and environmental factors. 
Environmental experts would collect documents and manually mine for insights on DPSIR. 
As the corpus scales, this process quickly becomes laborious and inefficient. 

Such a task can be formulated as a text mining operation and automated. 
The definitions of DPSIR essentially form a ``label taxonomy''.
Conventionally, the taxonomy can be constructed manually or semi-automatically with clustering techniques~\cite{wan2024textminingllm} and then used to annotate the corpus with pre-trained models.
\sam{
\marginnote{$\triangle$\_1\_1}However, mining with the DPSIR framework is a progressive process. 
The DPSIR taxonomy often starts with abstract definitions and is gradually enriched as experts investigate the corpus. The investigation usually contains two iterative steps: first, each DPSIR indicator is contextualized with ``variables'', such as population or pollution, which constitute the taxonomy; second, relationships between variables are constructed with supporting evidence (e.g., text snippets) from the corpus and documented as insights. 
Then, experts would investigate the corpus again with new insights in mind and repeat this process until no new insight is derived.
Due to this progressive characteristic, it is impractical to construct validation sets with human annotations to validate the taxonomy and insights in each iteration.
This unique challenge motivates us to utilize Large Language Models (LLMs) to support experts in this progressive process and use uncertainty metrics to support expert validation. 
% \sam{
% Moreover, without a known distribution or ``ground-truths'', it is hard to estimate the performance of text mining results using conventional evaluation metrics such as accuracy. These unique challenges motivate us to develop a human-in-the-loop approach for mining DPSIR insights.
% }
% Prompt-based text mining provides a promising solution. 
% Instead of relying on annotations and pre-trained models, 
With LLMs, the DPSIR taxonomy can be expressed in natural language and then inserted in expert prompt templates~\cite{macneil2023promptmiddleware}, which instruct LLMs to mine mentions of DPSIR variables and relationships from the corpus.
This allows experts to progressively refine and contextualize the DPSIR taxonomy as they explore the corpus and gain new insights.
}

Building upon this approach, we work closely with environmental experts and identify four technical challenges.
First, designing prompts is not trivial for inexperienced users~\cite{zamfirescu2023johnny, kim2024evallm}, especially when the prompt is applied to a dataset~\cite{lee2024awesum}.
Second, mining DPSIR insight from a corpus in a single zero-shot prompt is still too complex even for state-of-the-art AI models (e.g., GPT-4). The mining needs to be decomposed into a prompting pipeline to get satisfactory results.
Third, evaluation is still necessary to ensure the reliable performance of the prompt, which is especially challenging without ground truths.
Fourth, the mined insights are scattered across the corpus in textual form, which is not ideal for communication. 

In this work, we address these challenges and develop \system, an LLM-based system that supports human-in-the-loop text mining with the DPSIR framework and visualization of the results.
To lower the technical barrier, the system takes expert input for the DPSIR taxonomy in natural language and inserts them into prompt templates for execution.
For more accurate results, the system decomposes the mining task into three sequential subtasks that are more manageable and sensible.
\sam{
We design an uncertainty chart that visualizes the uncertainty of the LLM responses and the topic distribution in the supporting evidence.
}
% First, we develop an uncertainty metric based on response consistency. Then, we 
Finally,  to support communication, we design a DPSIR graph that organizes the mining results in a graph that can be interactively explored.
We present a case study to show the effectiveness of the LLM-based mining support, 
and an expert review to discuss the system's potential and limitations.
We found that while the system is designed for environmental studies, the technical challenges and solutions have broader applications. We discuss lessons learned and future opportunities for supporting human-in-the-loop text mining in other knowledge-intensive scenarios.

We consider our contributions are:
\begin{itemize}[noitemsep, topsep=0pt]
    \item introducing a prompt-based text mining pipeline for human-in-the-loop DPSIR mining from a corpus,
    \item designing a visual analytics system integrating the mining pipeline with uncertainty charts for prompt refinement and a DPSIR graph for collaborative discussion, and
    \item providing lessons learned from the design process that inform other applications with knowledge-intensive tasks.
\end{itemize}

\vspace*{-0.15cm}
\section{Background}
\sam{
\marginnote{$\triangle$\_2\_1}The DPSIR framework~\cite{atkins2011dpsir} provides a holistic approach to studying the connections that exist within and between diverse and complex societal and environmental factors, which is critical to support policymakers in their decision-making.}
According to the framework, \textit{Drivers} are the key demands by society and create \textit{Pressures} on the environment.
 Accumulation of pressures leads to \textit{State} changes, which creates \textit{Impacts} on human society or the environment. Significant impacts may elicit a societal \textit{Response} and alter the \textit{Drivers} and \textit{Pressures}, forming a feedback loop. 
We refer to them as \textit{indicators}. 

Given the diversity and complexity of the indicators, applying the DPSIR framework to study a specific region requires standardization and contextualization of the definitions of each indicator into \textit{variables}~\cite{oesterwind2016dpsiruntangle}, so that their relationships can be studied. 
For example, to study the reef fishing activities in Kenya, Mangi et al.~\cite{mangi2007reefdpsir} delineates \textit{Drivers} into several variables such as population, unemployment, tradition, and culture. Such a delineation is highly dependent on the regional context, making it infeasible to propose a taxonomy that is generally applicable. Environment experts need to manually build the taxonomy from large collections of documents, such as prior research papers, reports, or interview transcripts.
From a text mining perspective, the variables are concepts to be extracted, and the mining goal is to progressively add concepts to extract and uncover relationships between concepts.

In this work, our collaborating environmental experts apply the DPSIR framework to a study conducted in Lyudao (also known as Green Island), Taiwan, a small island in the Pacific Ocean. They have derived an initial taxonomy from a literature review, and have conducted interviews with the local residents. Using the interview transcripts as a corpus, they seek to contextualize and refine the taxonomy, and then mine key relationships between the variables. 

The DPSIR framework also facilitates communication of the findings to policymakers. Similar to business strategy diagrams~\cite{brath2024managementdiagram}, there are established infographic diagrams for the DPSIR framework familiar to the experts and the policymakers. The diagram is often organized in a loop that connects each indicator to reflect the cyclic feedback between society and the environment, as shown in~\autoref{fig: dpsir_diagram}. In this work, we extend this infographic design with progressive disclosure to better support collaborative discussion. 
\begin{figure}[t]
  \centering
  \includegraphics[keepaspectratio, width=0.95\columnwidth]{dpsir_diagram}
  \caption{Example of a DPSIR diagram from Atkins et al.~\cite{atkins2011dpsir}. The DPSIR framework is commonly depicted as a cyclic feedback loop between indicators. Our radial DPSIR graph design extends the cyclic characteristic with progressive disclosure.}
  \label{fig: dpsir_diagram}
  \vspace*{-0.5cm}
\end{figure}



