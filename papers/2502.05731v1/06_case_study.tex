

\vspace*{-0.15cm}
\section{Case Study}
In this section, we introduce a case study in which \textbf{E1}, a researcher in environmental science, uses the system to refine the DPSIR taxonomy. \textbf{E1} has collected an initial set of DPSIR taxonomy from a literature review. He has imported this initial taxonomy into the system and done the first extraction (\autoref{fig: case_study}-a). Next, he inspects the results using the uncertainty chart to refine the taxonomy. 

\vspace*{-0.12cm}
\paragraph{\textbf{\textit{Refining definition of \textit{Response}}}} The initial definition of Response is ``any behavior, action, or effort to protect the environment, address environmental issues, or be environmentally friendly''. 
\sam{\marginnote{$\triangle$\_6\_1}The uncertainty chart (\autoref{fig: uncertainty_chart}) organizes the results mentioning ``Response'' by topics and uncertainties, making it easier to make sense of and identify outliers.
}
\textbf{E1} sees that most topics fit this description and have low uncertainty. An exception is ``Little Vendor Dream'', which has a node with $0.7$ uncertainty, and the topic does not suggest something that is typically considered as ``Response''. Upon closer inspection, the snippet describes the interviewee's dream of establishing an eco-friendly, and serene shop, where visitors can escape the hustle and bustle of big cities and experience the warmth and tranquility that only nature can offer. This snippet has high uncertainty because it only roughly fits the current definition of Response. \textbf{E1} believes this should be included, and adds ``improving living conditions'' to the definition of Response.

\vspace*{-0.12cm}
\paragraph{\textbf{\textit{Enriching variables in \textit{Driver}}}}
During the refinement of Driver variables (\autoref{fig: case_study}-f), \textbf{E1} finds a snippet with high uncertainty about the White Error, which is a period of political repression from 1947 to 1987 that disrupted many tribal cultures. As a result, Lyudao has a district dedicated to commemorating this event. 
\textbf{E1} believes this is a driver that is often overlooked by the environmental science literature but is crucial to consider when making policies for Lyudao. 
\textbf{E1} also observes from the keyword cloud for ``miscellaneous'' that ``garbage'' frequently occurs (\autoref{fig: simple_vis}, left). Upon closer inspection, he finds that garbage disposal is a serious issue for the residents of Lyudao.
Based on these two observations, \textbf{E1} adds ``culture security'' and ``garbage'' to the variables of drivers. 

\vspace*{-0.12cm}
\paragraph{\textbf{\textit{Investigating links}}}
While refining the DPSIR taxonomy, \textbf{E1} makes several insights from the transcripts. For example, he finds a snippet about the coral reefs on Lyudao (\autoref{fig: simple_vis}, right). The link graph shows that there are activities of physical destruction that cause changes in natural habitats. He clicks the link to inspect the conversation and finds an interviewee's response that mentions the usage of boat anchors causing physical damage to the underwater corals. The links also reveal that the interviewee calls for restoration activities and education to raise environmental awareness. 

\vspace*{-0.12cm}
\paragraph{\textbf{\textit{Insights from the DPSIR Graph}}}
After various refinements to the indicator and variable definitions, \textbf{E1} observes some interesting patterns from the DPSIR graph (\autoref{fig: DPSIR}-right). For example, contrary to the existing literature, ``economy'' and ``transportation'' are two significant \textit{Drivers} with many occurrences in the context of Lyudao. Similarly in \textit{Pressure}, the interviewees express greater concern about ``extreme weather'' than ``climate change''. \textbf{E1} hypothesizes that this is due to the more immediate and severe impact on their livelihoods. This observation suggests that the infrastructure and economy in Lyudao are underdeveloped and vulnerable to ecosystem changes, and the environmental policies must consider that. Other insights confirm his hypotheses, such as the objection to thermal power plants or the need for better tourism management. 

