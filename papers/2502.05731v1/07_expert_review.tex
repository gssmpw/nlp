\vspace*{-0.15cm}
\section{Expert Review}
To better understand the potential and limitations of our approach, we demonstrated the system to experts participating in the Lyudao project.
In addition to \textbf{E1} and \textbf{E2}, we invited three more experts and host a discussion session. \textbf{E3} is a senior researcher in environmental science and is in charge of the interview that \textbf{E1} analyzed. \textbf{E4} is a research assistant with a Master's degree in environmental science, and \textbf{E5} is a PhD student in earth science. None of the three experts had seen the system before the discussion.

The procedure started with a brief introduction of the overarching system goal and the interface design. Then, \textbf{E1} presented his findings from the usage of the system, using both the uncertainty charts and the DPSIR Graph. During the presentation, we encouraged all experts to ask questions and provide their thoughts. The discussion lasted over 2 hours and was video-recorded. Below, we summarize the expert feedback and discuss the implications.

\vspace*{-0.1cm}
\paragraph{System design is well-received}
The system design is clear and intuitive. Despite lacking technical knowledge in prompt engineering or text mining, the experts found the three-step mining pipeline sensible and easy to evaluate. They easily understood the uncertainty chart's visual encodings. During \textbf{E1's} presentation, he effectively used the DPSIR Graph without external tools, demonstrating its suitability for collaborative discussions. Additionally, \textbf{E1}'s week-long usage without issues highlights the usability of the prompting functionality and uncertainty chart.

\vspace*{-0.15cm}
\paragraph{Complementing literature review}
All experts agreed that the system is an effective tool to provide unexpected insights after seeing \textbf{E1's} findings. \textbf{E4} was especially fond of the automatic support, knowing the time and effort needed to manually analyze these transcripts. \textbf{E3} agreed that \textit{``Looks like we are scattering the transcripts and then reorganizing them in response to our research questions with better and clearer definitions (compared to literature review).''}
\textbf{E1} summarized from his experience that computational extraction reduces the cognitive bias that human experts might have from literature knowledge.
This shows the system not only streamlines transcript analysis but also mitigates cognitive biases, improving the objectivity and precision of the derived insights.

\begin{figure}[]
     \centering
    \includegraphics[width=\columnwidth]{uncertainty_chart}
    \caption{An uncertainty chart generated in the case study.
    Each dot in the chart represents a snippet that mentions the indicator ``Response'', positioned with polar coordinates. 
    From the chart, most topics fit \textbf{E1's} expectation except a snippet under ``Little Vendor Dream'' with high uncertainty. He clicks the snippet to inspect the relevant conversations and the LLM-generated explanation. 
    He finds that the snippet has high uncertainty because the Response definition is incomplete, 
    so he refines the definition accordingly. 
    }
    \label{fig: uncertainty_chart}
  \vspace*{-0.6cm}
\end{figure}

\begin{figure*}[t]
    \centering
    \includegraphics[width=\textwidth]{DPSIR}
    \caption{\textbf{Left}: The DPSIR Graph shows the aggregated mining results in a progressive graph. The design follows the typical DPSIR diagram in environmental studies. The Driver block is highlighted with ingoing and outgoing links colored by the source of the link. \textbf{Right}: The States and Impacts are hidden, and the rest of the indicators are revealed to show the variables and their links. \textbf{E1} found that contrary to the literature, ``Economy'' and ``Transportation'' are two significant drivers with many linkages, and ``Extreme Weather'' is a significant pressure. }
    \label{fig: DPSIR}
    \vspace*{-0.5cm}
\end{figure*}

\vspace*{-0.15cm}
\paragraph{Informing policymaking}
During the discussion, \textbf{E2} explored how the DPSIR Graph could guide policymaking, focusing on \textit{Response} variables to gauge public opinion on current policies. \textbf{E1} agreed, suggesting the inclusion of other relevant variables: \textit{``For example, for fishery policies, we can identify connected variables, indicating where policies should focus ({E1}).''} This highlights the DPSIR Graph’s potential to inform policy by identifying key variables and incorporating public opinion for data-driven decisions.

\vspace*{-0.15cm}
\paragraph{Facilitating public-facing presentation}
\textbf{E3}, leading interview recruitment, was keen on improving public outreach with the system, noting that Lyudao residents historically distrust government-funded research. \textbf{E2} agreed, stating, \textit{``Residents want to feel heard and see their opinions considered. This tool could help demonstrate that.''} Still, \textbf{E3} emphasized the need to simplify the DPSIR Graph to match the public's visualization literacy. Balancing these views, the system shows potential as a bridge between research and the public.

\vspace*{-0.15cm}
\paragraph{Comparing detailed research questions}
While the system is designed for exploratory purposes in mind, it could also be applied to study and compare concrete research questions. As suggested by \textbf{E3}, users can give very narrow indicator definitions and short lists of variables to focus on only one research question: \textit{``We can first set a few research questions, for example, pressures on oil and gas, or pressures on the fishery, and then adjust the definitions accordingly. We can even compare different versions of definitions, which might reveal more focused insights. Such insights would also be more suitable for the public to digest.''}
This adaptability highlights the system's potential to support detailed analyses.