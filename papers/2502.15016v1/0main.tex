%%%%%%%% ICML 2025 EXAMPLE LATEX SUBMISSION FILE %%%%%%%%%%%%%%%%%

\documentclass{article}

% Recommended, but optional, packages for figures and better typesetting:
\usepackage{microtype}
\usepackage{graphicx}
\usepackage{subfigure}
\usepackage{booktabs} % for professional tables

\usepackage{bbm}

\usepackage{wrapfig}

% hyperref makes hyperlinks in the resulting PDF.
% If your build breaks (sometimes temporarily if a hyperlink spans a page)
% please comment out the following usepackage line and replace
% \usepackage{icml2024} with \usepackage[nohyperref]{icml2024} above.
\usepackage{hyperref}


% Attempt to make hyperref and algorithmic work together better:
\newcommand{\theHalgorithm}{\arabic{algorithm}}



% Use the following line for the initial blind version submitted for review:
% \usepackage{icml2024}

% If accepted, instead use the following line for the camera-ready submission:
\usepackage[accepted]{icml2025}

% For theorems and such
\usepackage{amsmath}
\usepackage{amssymb}
\usepackage{mathtools}
\usepackage{amsthm}
\usepackage{multirow}
\usepackage{multicol}
\usepackage[utf8]{inputenc}
\usepackage{tcolorbox}
\usepackage{float}

% if you use cleveref..
\usepackage[capitalize,noabbrev]{cleveref}

%%%%%%%%%%%%%%%%%%%%%%%%%%%%%%%%
% THEOREMS
%%%%%%%%%%%%%%%%%%%%%%%%%%%%%%%%
\theoremstyle{plain}
\newtheorem{theorem}{Theorem}[section]
\newtheorem{proposition}[theorem]{Proposition}
\newtheorem{lemma}[theorem]{Lemma}
\newtheorem{corollary}[theorem]{Corollary}
\theoremstyle{definition}
\newtheorem{definition}[theorem]{Definition}
\newtheorem{assumption}[theorem]{Assumption}
\theoremstyle{remark}
\newtheorem{remark}[theorem]{Remark}


\newcommand{\wei}[1]{\textcolor{orange}{\#Wei:~#1\#}}
\newcommand{\juntong}[1]{\textcolor{green}{\#Juntong:~#1\#}}



\newcommand{\method}{\textsc{TimeDistill}}

% Todonotes is useful during development; simply uncomment the next line
%    and comment out the line below the next line to turn off comments
%\usepackage[disable,textsize=tiny]{todonotes}
\usepackage[textsize=tiny]{todonotes}

\usepackage{paralist}

% The \icmltitle you define below is probably too long as a header.
% Therefore, a short form for the running title is supplied here:
\icmltitlerunning{TimeDistill: Efficient Long-Term Time Series Forecasting with MLP via  Cross-Architecture Distillation}

\begin{document}

\twocolumn[
\icmltitle{TimeDistill: Efficient Long-Term Time Series Forecasting with MLP \\ via  Cross-Architecture Distillation}

% It is OKAY to include author information, even for blind
% submissions: the style file will automatically remove it for you
% unless you've provided the [accepted] option to the icml2024
% package.

% List of affiliations: The first argument should be a (short)
% identifier you will use later to specify author affiliations
% Academic affiliations should list Department, University, City, Region, Country
% Industry affiliations should list Company, City, Region, Country

% You can specify symbols, otherwise they are numbered in order.
% Ideally, you should not use this facility. Affiliations will be numbered
% in order of appearance and this is the preferred way.
% \icmlsetsymbol{equal}{*}

\begin{icmlauthorlist}
\icmlauthor{Juntong Ni}{Emory}
\icmlauthor{Zewen Liu}{Emory}
\icmlauthor{Shiyu Wang}{}
\icmlauthor{Ming Jin}{Griffith}
\icmlauthor{Wei Jin}{Emory}
% \icmlauthor{Firstname3 Lastname3}{comp}
% \icmlauthor{Firstname4 Lastname4}{sch}
% \icmlauthor{Firstname5 Lastname5}{yyy}
% \icmlauthor{Firstname6 Lastname6}{sch,yyy,comp}
% \icmlauthor{Firstname7 Lastname7}{comp}
%\icmlauthor{}{sch}
% \icmlauthor{Firstname8 Lastname8}{sch}
% \icmlauthor{Firstname8 Lastname8}{yyy,comp}
%\icmlauthor{}{sch}
%\icmlauthor{}{sch}
\end{icmlauthorlist}

\icmlaffiliation{Emory}{Department of Computer Science, Emory University, Atlanta, United States}
\icmlaffiliation{Griffith}{School of Information and Communication Technology, Griffith University, Nathan, Australia}
% \icmlaffiliation{comp}{Company Name, Location, Country}
% \icmlaffiliation{sch}{School of ZZZ, Institute of WWW, Location, Country}

\icmlcorrespondingauthor{Wei Jin}{wei.jin@emory.edu}

% You may provide any keywords that you
% find helpful for describing your paper; these are used to populate
% the "keywords" metadata in the PDF but will not be shown in the document
\icmlkeywords{Machine Learning, ICML}

\vskip 0.3in
]
% this must go after the closing bracket ] following \twocolumn[ ...

% This command actually creates the footnote in the first column
% listing the affiliations and the copyright notice.
% The command takes one argument, which is text to display at the start of the footnote.
% The \icmlEqualContribution command is standard text for equal contribution.
% Remove it (just {}) if you do not need this facility.
\printAffiliationsAndNotice{}  % leave blank if no need to mention equal contribution
% \printAffiliationsAndNotice{\icmlEqualContribution} % otherwise use the standard text.



\begin{abstract}
Transformer-based and CNN-based methods demonstrate strong performance in long-term time series forecasting. However, their high computational and storage requirements can hinder large-scale deployment. To address this limitation, we propose integrating lightweight MLP with advanced architectures using knowledge distillation (\textit{KD}). Our preliminary study reveals different models can capture complementary patterns, particularly multi-scale and multi-period patterns in the temporal and frequency domains.
Based on this observation, we introduce \method{}, a cross-architecture \textit{KD} framework that transfers these patterns from teacher models (e.g., Transformers, CNNs) to MLP. Additionally, we provide a theoretical analysis, demonstrating that our \textit{KD} approach can be interpreted as a specialized form of \textit{mixup} data augmentation.
\method{} improves MLP performance by up to 18.6\%, surpassing teacher models on eight datasets. It also achieves up to 7$\times$ faster inference and requires 130$\times$ fewer parameters. Furthermore, we conduct extensive evaluations to highlight the versatility and effectiveness of \method{}.

\end{abstract}

% \begin{wrapfigure}{R}{0.26\textwidth}
%     \hspace{-1em}
%     \vskip -1.3em
%     \begin{minipage}{\linewidth}
%         \centering
%         \includegraphics[width=1\linewidth]{figures/result_radar.pdf}
%         \vskip -1.8em
% % \captionsetup{size=footnotesize}
% \caption{Performance comparison.}  %Average results (MSE) are reported following iTransformer~\cite{itransformer}.}
% \label{fig:result_rader}
% \vspace{-2em}
% \end{minipage}
% \end{wrapfigure}



\section{Introduction}

In today’s rapidly evolving digital landscape, the transformative power of web technologies has redefined not only how services are delivered but also how complex tasks are approached. Web-based systems have become increasingly prevalent in risk control across various domains. This widespread adoption is due their accessibility, scalability, and ability to remotely connect various types of users. For example, these systems are used for process safety management in industry~\cite{kannan2016web}, safety risk early warning in urban construction~\cite{ding2013development}, and safe monitoring of infrastructural systems~\cite{repetto2018web}. Within these web-based risk management systems, the source search problem presents a huge challenge. Source search refers to the task of identifying the origin of a risky event, such as a gas leak and the emission point of toxic substances. This source search capability is crucial for effective risk management and decision-making.

Traditional approaches to implementing source search capabilities into the web systems often rely on solely algorithmic solutions~\cite{ristic2016study}. These methods, while relatively straightforward to implement, often struggle to achieve acceptable performances due to algorithmic local optima and complex unknown environments~\cite{zhao2020searching}. More recently, web crowdsourcing has emerged as a promising alternative for tackling the source search problem by incorporating human efforts in these web systems on-the-fly~\cite{zhao2024user}. This approach outsources the task of addressing issues encountered during the source search process to human workers, leveraging their capabilities to enhance system performance.

These solutions often employ a human-AI collaborative way~\cite{zhao2023leveraging} where algorithms handle exploration-exploitation and report the encountered problems while human workers resolve complex decision-making bottlenecks to help the algorithms getting rid of local deadlocks~\cite{zhao2022crowd}. Although effective, this paradigm suffers from two inherent limitations: increased operational costs from continuous human intervention, and slow response times of human workers due to sequential decision-making. These challenges motivate our investigation into developing autonomous systems that preserve human-like reasoning capabilities while reducing dependency on massive crowdsourced labor.

Furthermore, recent advancements in large language models (LLMs)~\cite{chang2024survey} and multi-modal LLMs (MLLMs)~\cite{huang2023chatgpt} have unveiled promising avenues for addressing these challenges. One clear opportunity involves the seamless integration of visual understanding and linguistic reasoning for robust decision-making in search tasks. However, whether large models-assisted source search is really effective and efficient for improving the current source search algorithms~\cite{ji2022source} remains unknown. \textit{To address the research gap, we are particularly interested in answering the following two research questions in this work:}

\textbf{\textit{RQ1: }}How can source search capabilities be integrated into web-based systems to support decision-making in time-sensitive risk management scenarios? 
% \sq{I mention ``time-sensitive'' here because I feel like we shall say something about the response time -- LLM has to be faster than humans}

\textbf{\textit{RQ2: }}How can MLLMs and LLMs enhance the effectiveness and efficiency of existing source search algorithms? 

% \textit{\textbf{RQ2:}} To what extent does the performance of large models-assisted search align with or approach the effectiveness of human-AI collaborative search? 

To answer the research questions, we propose a novel framework called Auto-\
S$^2$earch (\textbf{Auto}nomous \textbf{S}ource \textbf{Search}) and implement a prototype system that leverages advanced web technologies to simulate real-world conditions for zero-shot source search. Unlike traditional methods that rely on pre-defined heuristics or extensive human intervention, AutoS$^2$earch employs a carefully designed prompt that encapsulates human rationales, thereby guiding the MLLM to generate coherent and accurate scene descriptions from visual inputs about four directional choices. Based on these language-based descriptions, the LLM is enabled to determine the optimal directional choice through chain-of-thought (CoT) reasoning. Comprehensive empirical validation demonstrates that AutoS$^2$-\ 
earch achieves a success rate of 95–98\%, closely approaching the performance of human-AI collaborative search across 20 benchmark scenarios~\cite{zhao2023leveraging}. 

Our work indicates that the role of humans in future web crowdsourcing tasks may evolve from executors to validators or supervisors. Furthermore, incorporating explanations of LLM decisions into web-based system interfaces has the potential to help humans enhance task performance in risk control.






% Reward hacking is a well-known issue in reinforcement learning, affecting both traditional RL and RLHF in LLMs~\cite{weng2024rewardhack}.
\subsection{Reward Hacking in Traditional RL}  
Reward hacking arises when an RL agent exploits flaws or ambiguities in the reward function to achieve high rewards without performing the intended task~\cite{weng2024rewardhack}. This aligns with Goodhart’s Law: \emph{When a measure becomes a target, it ceases to be a good measure.} For example: 
A bicycle agent rewarded for not falling and moving toward a goal (but not penalized for moving away) learns to circle the goal indefinitely~\cite{Randlv1998LearningTD}.  
A walking agent in the DMControl suite, rewarded for matching a target speed, learns to walk unnaturally using only one leg~\cite{lee2021pebblefeedbackefficientinteractivereinforcement}.  
An RL agent allowed to modify its body grows excessively long legs to fall forward and reach the goal~\cite{Ha2018designrl}.  
In the Elevator Action ALE game, the agent repeatedly kills the first enemy on the first floor to accumulate small rewards~\cite{toromanoff2019deepreinforcementlearningreally}.  
% A robot trained to stay on track learns to reverse along straight paths by alternating left and right turns instead of following curves~\cite{Vamplew2004}.

\citet{amodei2016concrete} propose several potential mitigation strategies to address reward hacking, including
\emph{(1) Adversarial Reward Functions}: Treating the reward function as an adaptive agent capable of responding to new strategies where the model achieves high rewards but receives low human ratings.
\emph{(2) Model Lookahead}: Assigning rewards based on anticipated future states; for example, penalizing the agent with negative rewards if it attempts to modify the reward function~\cite{everitt2016selfmodificationpolicyutilityfunction}.
\emph{(3) Adversarial Blinding}: Restricting the model’s access to specific variables to prevent it from learning information that could facilitate reward hacking~\cite{ajakan2015domainadversarialneuralnetworks}.
\emph{(4) Careful Engineering}: Designing systems to avoid certain types of reward hacking by isolating the agent’s actions from its reward signals, such as through sandboxing techniques~\cite{The_AGI_Containment_Problem}.
\emph{(5) Trip Wires}: Deliberately introducing vulnerabilities into the system and setting up monitoring mechanisms to detect and alert when reward hacking occurs.

\subsection{Reward Hacking in RLHF of LLMs}  
Reward hacking in RLHF for large language models has been extensively studied. \citet{gao2023scaling} systematically investigate the scaling laws of reward hacking in small models, while \citet{wen2024languagemodelslearnmislead} demonstrate that language models can learn to mislead humans through RLHF. Beyond exploiting the training process, reward hacking can also target evaluators. Although using LLMs as judges is a natural choice given their increasing capabilities, this approach is imperfect and can introduce biases. For instance, LLMs may favor their own responses when evaluating outputs from different model families~\cite{liu2024llmsnarcissisticevaluatorsego} or exhibit positional bias when assessing responses in sequence~\cite{wang2023largelanguagemodelsfair}.  

To mitigate reward hacking, several methods have been proposed. Reward ensemble techniques have shown promise in addressing this issue~\cite{Eisenstein2023HelpingOH, Rame2024WARMOT, ahmed2024scalableensemblingmitigatingreward, coste2023reward, zhang2024improvingreinforcementlearninghuman}, and shaping methods have also proven straightforward and effective~\cite{yang2024regularizinghiddenstatesenables, jinnai2024regularizedbestofnsamplingmitigate}. \citet{miao2024informmitigatingrewardhacking} introduce an information bottleneck to filter irrelevant noise, while \citet{moskovitz2023confrontingrewardmodeloveroptimization} employ constrained RLHF to prevent reward over-optimization. \citet{Chen2024ODINDR} propose the ODIN method, which uses a linear layer to separately output quality and length rewards, reducing their correlation through an orthogonal loss function. Similarly,
~\citet{sun2023salmon} train instructable reward models to give a more comprehensive reward signal from multiple objectives. \citet{Dai2023SafeRS} constrain reward magnitudes using regularization terms. ~\citet{liu2024rrmrobustrewardmodel} curate diverse pairwise training data. Additionally, post-processing techniques have been explored, such as the log-sigmoid centering transformation introduced by \citet{Wang2024TransformingAC}.  


\section{Preliminaries}
\label{sec:preliminary}

\begin{figure*} [t]
	\centering
   \vspace{-2ex}
	\subfigure[\dataset]{
		\includegraphics[width=0.225\linewidth]{figures/crop_data_ratio.pdf}
		\label{fig:data-overall}
	} 
     \subfigure[Tool Data]{
		\includegraphics[width=0.225\linewidth]{figures/crop_tool_ratio.pdf}
		\label{fig:data-tool}
	}
    \subfigure[Retrieved Data]{
		\includegraphics[width=0.225\linewidth]{figures/crop_retr_ratio.pdf}
		\label{fig:data-retr}
	}
    \subfigure[t-SNE: Retrieved Data]{
		\includegraphics[width=0.225\linewidth]{figures/tsne-new-v3.pdf}
		\label{fig:tsne-retr}
	}
    % \vspace{-1ex}
	\caption{Data composition of (a) the entire \dataset, (b) seed data collection (\cref{sec:data-phase1}), and (c) retrieved agent data from the open web (\cref{sec:data-phase2}). A t-SNE visualization (d) depicts seed data (\textbf{colorful} points, with each color representing different data sources), retrieved data (\textbf{black}), and general text (\textcolor{gray}{\textbf{gray}}) within the semantic space, where retrieved data is closer to the selected seed data than to the general text. Detailed data sources are in \cref{app:data-pretrain}.
 }
\vspace{-2ex}
\label{fig:data}
\end{figure*}


\noindent \textbf{Problem Formulation.} We conceptualize leveraging LLMs as autonomous agents for problem-solving as a planning process.
Initially, we augment the LLM agent with access to a pool of candidate API functions, denoted as $\mathcal{A}=\{\text{API}_0,\text{API}_1,\cdots,\text{API}_m$\}, along with a natural language task description $g\in\mathcal{G}$ from the task space $\mathcal{G}$. 
The objective of the LLM agent is to translate the task description $g$ into an ordered sequence of $T_g$ API function calls $p_g=\{a_0,\cdots,a_{T_g}\}$.
Specifically, considering the task description $g$ as the initial state $s_0$, we then sample the plan $p_g$ by prompting the LLM agent with the API definitions $\mathcal{I}$ and demonstration samples $\mathcal{D}$ as follows: $p_g\sim\rho(a_0,a_1,\cdots,a_{T_g}|s_0;\mathcal{I},\mathcal{D}):\mathcal{G}\times\mathcal{I}\times\mathcal{D}\to\Delta(\mathcal{A}^{T_g})$, where $\Delta(\cdot)$ denotes a probability simplex function. 
The final output is derived after executing the entire plan $y\sim\pi(y|s_0,a_1,a_2,\cdots,a_{T_g})$, where $\pi(\cdot)$ denotes a plan executor.

During this procedure, we focus on three fundamental capabilities of LLM agents:

\noindent \textbf{Accurate Function Calling.} It involves accurately understanding the API definitions and demonstration samples to generate correct API function calls with corresponding parameters in a given scenario.
Specifically, the model should accurately understand the API definitions $\mathcal{I}$ and demonstration samples $\mathcal{D}$, as well as generate an accurate API function call in the given scenario $p(a_t|s_0,a_1,\cdots,a_{t-1},\mathcal{I},\mathcal{D})$, where $a_t$ is the ground-truth API function call with corresponding parameters at $t$-th step.

\noindent \textbf{Intrinsic Reasoning and Planning.} It refers to the intrinsic reasoning and planning ability to devise a sequence of multiple tool functions as a solution when addressing complex (multi-step) real-world problems. In such cases, LLMs are often required to generate a sequence of API function calls, $p(a_1,a_2,\cdots,a_{T_g}|s_0;\mathcal{I},\mathcal{D})$, where $\{a_1,a_2,\cdots,a_{T_g}\}$ constitutes the ground-truth solution plan of length $T_g$.  
This process relies on intrinsic reasoning embedded within the model parameters; enhanced reasoning capabilities lead to a solution plan with a higher chance of success.


\noindent \textbf{Adaptation with Environment Feedback.} It focuses on adapting the current plan or action based on environmental feedback when the environments support interaction with the LLM agent. When such feedback is available, it is crucial for the agent to adjust its actions accordingly: $p(a_t|s_0,a_1,o_1,a_2,\cdots,o_{t-1};\mathcal{I},\mathcal{D})$,
where $o_{k}$ represents the feedback from the environment after the $k$-th action. 
Incorporating environmental feedback allows the agent to take reflections to refine its plan and improve task performance iteratively.



\begin{figure*}[t]
    \centering
    \includegraphics[width=0.98\textwidth]{figures/Method.pdf}
    \vskip -1.5em
    \caption{Overall framework of \method{}, which distills knowledge from a teacher model to a student MLP using (a) Multi-Scale Distillation and (b) Multi-Period Distillation at both feature and prediction levels. (a) Multi-Scale Distillation involves downsampling the original time series into multiple coarser scales and aligning these scales between the student and teacher. (b) Multi-Period Distillation applies FFT to transform the time series into a spectrogram, followed by matching the period distributions after applying softmax.}
    \label{fig:method}
    \vskip -1em
\end{figure*}

\vskip -2em
\section{Methodology}

% In alignment with our intuition about preserving multi-scale and multi-period pattern knowledge, we introduce a novel distillation framework named \method{}. Unlike conventional approaches that emphasize matching predictions or time series representations directly, \method{} focuses on transferring knowledge of multi-scale patterns in the temporal domain and multi-period patterns in the frequency domain. 
% To efficiently distill this knowledge from the teacher model to the MLP, we propose two specific distillation objectives: \textit{multi-scale distillation} and \textit{multi-period distillation}, which we will detail next. The overall framework of \method{} is shown in Figure~\ref{fig:method}.

% \wei{it could be good to introduce an overall framework for KD (using equations) in time series and the following subsections describe how \method{} address it? }

% \wei{Standardize the notations: (1) bold upper-case letters for matrices; (2) bold lower-case letters for vectors; (3) regular letters for scalars (typically lower cases while upper cases are fine). You can take a look at the itransformer paper}

Motivated by our preliminary studies, we propose a novel \textit{KD} framework \method{} for time series to transfer the knowledge from a fixed, pretrained teacher model \(f_t\) to a student MLP model \(f_s\). The student produces predictions \(\mathbf{\hat{Y}}_s \in \mathbb{R}^{S \times C}\) and internal features \(\mathbf{H}_s\in \mathbb{R}^{D \times C}\). The teacher model produces predictions \(\mathbf{\hat{Y}}_t \in \mathbb{R}^{S \times C}\) and internal features \(\mathbf{H}_t\in \mathbb{R}^{D_t \times C}\). Our general objective is:
\begin{equation}\label{eq:kd_obj}
    \min\nolimits_{\theta_s} \mathcal{L}_{sup}(\mathbf{Y}, \mathbf{\hat{Y}}_s) + \mathcal{L}_{\mathrm{KD}}^\mathbf{Y}(\mathbf{\hat{Y}}_t, \mathbf{\hat{Y}}_s) + \mathcal{L}_{\mathrm{KD}}^\mathbf{H}(\mathbf{H}_t, \mathbf{H}_s),
\end{equation}
where \(\theta_s\) is the parameter of the student; \(\mathcal{L}_{sup}\) is the supervised loss (e.g., MSE) between predictions and ground truth; \(\mathcal{L}_{\mathrm{KD}}^\mathbf{Y}\) and \(\mathcal{L}_{\mathrm{KD}}^\mathbf{H}\) are the distillation loss terms that encourage the student model to learn knowledge from the teacher on both \textbf{prediction level}~\cite{hinton2015distilling} and \textbf{feature level}~\cite{romero2014fitnets}, respectively. Unlike conventional approaches that emphasize matching model predictions, \method{} integrates key time-series patterns including multi-scale and multi-period knowledge. The overall framework of \method{} is shown in Figure~\ref{fig:method}. 
% In the following, we detail each component of our approach.



% In alignment with our intuition about preserving multi-scale and multi-period pattern knowledge, we introduce a novel distillation framework named \method{}. Unlike conventional approaches that emphasize matching predictions directly, \method{} focuses on transferring knowledge of multi-scale patterns in the temporal domain and multi-period patterns in the frequency domain. 
% The overall framework of \method{} is shown in Figure~\ref{fig:method}. In the following, we provide details on each component of our approach.To efficiently distill this knowledge from the teacher model to the MLP, we propose two specific distillation objectives: \textbf{multi-scale distillation} and \textbf{multi-period distillation}. The overall framework of \method{} is shown in Figure~\ref{fig:method}. In the following, we provide details on each component of our approach.

\subsection{Multi-Scale Distillation}
One key component of \method{} is multi-scale distillation, where ``multi-scale'' refers to representing the same time series at different sampling rates. This enables MLP to effectively capture both coarse-grained and fine-grained patterns. By distilling at both the prediction level and the feature level, we ensure that MLP not only replicates the teacher's multi-scale predictions but also aligns with its internal representations from the intermediate layer.

\vspace{-0.5em}
\paragraph{Prediction Level.}
At the prediction level, we directly transfer multi-scale signals from the teacher’s outputs to guide the MLP’s predictions. We first produce multi-scale predictions by downsampling the original predictions from the teacher \(\mathbf{\hat{Y}}_t \in \mathbb{R}^{S \times C}\) and the MLP \(\mathbf{\hat{Y}}_s \in \mathbb{R}^{S \times C}\), where \(S\) is the prediction length and \(C\) is the number of variables. The predictions at \textit{Scale 0} are equal to the original predictions, that is, \(\mathbf{\hat{Y}}_t^0=\mathbf{\hat{Y}}_t\) and \(\mathbf{\hat{Y}}_s^0=\mathbf{\hat{Y}}_s\). We then downsample these predictions across \(M\) scales using convolutional operations with a stride of 2, generating multi-scale prediction sets \(\mathcal{Y}_t = \{\mathbf{\hat{Y}}_t^0, \mathbf{\hat{Y}}_t^1,\cdots,\mathbf{\hat{Y}}_t^M\}\) and \(\mathcal{Y}_s = \{\mathbf{\hat{Y}}_s^0, \mathbf{\hat{Y}}_s^1,\cdots,\mathbf{\hat{Y}}_s^M\}\), where \(\mathbf{\hat{Y}}_t^M, \mathbf{\hat{Y}}_s^M \in \mathbb{R}^{\lfloor S/2^M \rfloor \times C}\). The downsampling is defined as: 
\begin{equation}
    \mathbf{\hat{Y}}_x^m = \mathrm{Conv}(\mathbf{\hat{Y}}_x^{m-1}, \mathrm{stride}=2),
    \label{eq:multiscale_downsample}
\end{equation}
where \(x \in \{t, s\}\), \(m \in \{1, \cdots, M\}\), $\mathrm{Conv}$ denotes a 1D-convolutional layer with a temporal stride of 2. The predictions at the lowest level \(\mathbf{\hat{Y}}_x^0=\mathbf{\hat{Y}}_x\) maintain the original temporal resolution, while the highest-level predictions \(\mathbf{\hat{Y}}_x^M\) represent coarser patterns. We define the multi-scale distillation loss at the prediction level as:
\begin{equation}
    \mathcal{L}_{scale}^\mathbf{Y} = \textstyle\sum_{m=0}^M ||\mathbf{\hat{Y}}_t^m - \mathbf{\hat{Y}}_s^m||^2 /(M+1).
\end{equation}
% \wei{if Y is a vector or matrix, we should not use () but $| |$} 
Here we use MSE loss to match the MLP’s predictions to the teacher’s predictions at multiple scales.

\vspace{-0.5em}
\paragraph{Feature Level.} 
At the feature level, we align MLP’s intermediate features with teacher’s multi-scale representations, enabling MLP to form richer internal structures that support more accurate forecasts.
Let \(\mathbf{H}_s \in \mathbb{R}^{D \times C}\) and \(\mathbf{H}_t \in \mathbb{R}^{D_t \times C}\) denote MLP and teacher features with feature dimensions \(D\) and \(D_t\), respectively. As their dimensions can be different, we first use a parameterized regressor (e.g. MLP) to align their feature dimensions: 
% \wei{what does the regressor mean?? no explanation for this operator: what is the detailed operator; what is the purpose}
\begin{equation}
    \mathbf{H}'_t = \text{Regressor}(\mathbf{H}_t),
\end{equation}
where \(\mathbf{H}'_t \in \mathbb{R}^{D \times C}\).  
% We then downsample both sets of features across multiple scales:
% \begin{equation}
%     \mathbf{H}_x^m = \mathrm{Conv}(\mathbf{H}_x^{m-1}, \mathrm{stride}=2),
% \end{equation}
% where \(x \in \{t, s\}\), \(m \in \{1, \cdots, M\}\), and \(\mathbf{H}_s^0 = \mathbf{H}_s\), \(\mathbf{H}_t^0 = \mathbf{H}'_t\). 
Similar to the prediction level, we compute $\mathbf{H}_x^m$ by downsampling $\mathbf{H}_s$ and $\mathbf{H}'_t$ across multiple scales using the same approach as in Equation~\ref{eq:multiscale_downsample}. We define the multi-scale distillation loss at the feature level as:
\begin{equation}
    \mathcal{L}_{scale}^\mathbf{H} = \textstyle\sum_{m=0}^M ||\mathbf{H}_t^m - \mathbf{H}_s^m||^2 /(M+1).
\end{equation}

\subsection{Multi-Period Distillation}
% \wei{seems that the prediction level and feature level are basically using the same equations? then we probably do not so much space to describe them given the redundancy}
% \wei{we need some transitions: in addition to multi-scale ...temporal domain...}  
{In addition to multi-scale distillation in the temporal domain, we further
propose multi-period distillation to help MLP learn complex periodic patterns in the frequency domain.} By aligning periodicity-related signals from the teacher model at both the prediction and feature levels, the MLP can learn richer frequency-domain representations and ultimately improve its forecasting performance.

\paragraph{Prediction Level.}
For the predictions from the teacher \(\mathbf{\hat{Y}}_t \in \mathbb{R}^{S \times C}\) and the MLP \(\mathbf{\hat{Y}}_s \in \mathbb{R}^{S \times C}\), we first identify their periodic patterns. We perform this in the frequency domain using the Fast Fourier Transform (FFT):
\begin{equation}
    \mathbf{A}_x = \text{Amp}(\text{FFT}(\mathbf{\hat{Y}}_x)),
    \label{eq:multiperiod_spectrograms}
\end{equation}
where \(x \in \{t, s\}\) and spectrograms \(\mathbf{A}_x \in \mathbb{R}^{\frac{S}{2} \times C}\). Here, \(\text{FFT}(\cdot)\) denotes the FFT operation and \(\text{Amp}(\cdot)\) calculates the amplitude. We remove the direct current (DC) component from \(\mathbf{A}_x\). For certain variable \(c\), the \(i\)-th value \(\mathbf{A}_x^{i,c}\) indicates the intensity of the frequency-\(i\) component, corresponding to a period length \(\lceil S/i\rceil\). Larger amplitude values indicate that the associated frequency (period) is more significant.

To reduce the influence of minor frequencies and avoid noise introduced by less meaningful frequencies~\cite{timesnet, fedformer}, we propose a distribution-based matching scheme. We use a softmax function with a colder temperature to highlight the most significant frequencies:
\begin{equation}
    \mathbf{Q}_x^\mathbf{Y} = {\exp\bigl(\mathbf{A}_x^i / \tau\bigr)}/{\sum\nolimits_{j=1}^{S/2} \exp\bigl(\mathbf{A}_x^j /\tau\bigr)},
    \label{eq:multiperiod_distribution}
\end{equation}
where \(\mathbf{Q}_x^\mathbf{Y} \in \mathbb{R}^{\frac{S}{2} \times C}\) and \(\tau\) is a temperature parameter that controls the sharpness of the distribution. We set \(\tau=0.5\) by default. The period distribution \(\mathbf{Q}_x^\mathbf{Y}\) represents the multi-period pattern in the prediction time series, which we want the MLP to learn from the teacher. We use KL divergence to match these distributions~\cite{hinton2015distilling}. We define the multi-period distillation loss at the prediction level as:
\begin{equation}
    \mathcal{L}_{period}^\mathbf{Y} = \text{KL}\bigl(\mathbf{Q}_t^\mathbf{Y}, \mathbf{Q}_s^\mathbf{Y}\bigr).
\end{equation}
% where KL denotes the Kullback--Leibler divergence~\cite{hinton2015distilling}, a common metric to measure distribution difference.

% \paragraph{Feature Level.}
% Similar to the prediction level, we also apply multi-period distillation at the feature level. We compute:
% \begin{equation}
%     \mathbf{B}_x = \text{Amp}(\text{FFT}(\mathbf{\hat{H}}_x)),
% \end{equation}
% \begin{equation}
%     \mathbf{Q}_x^\mathbf{H} = \frac{\exp\bigl(\mathbf{B}_x^i / \tau\bigr)}{\sum_{j=1}^{D/2} \exp\bigl(\mathbf{B}_x^j /\tau\bigr)},
% \end{equation}
% and define:
% \begin{equation}
%     \mathcal{L}_{period}^\mathbf{H} = \text{KL}\bigl(\mathbf{Q}_t^\mathbf{H}, \mathbf{Q}_s^\mathbf{H}\bigr),
% \end{equation}
% where \(\mathbf{B}_x, \mathbf{Q}_x^\mathbf{H} \in \mathbb{R}^{\frac{D}{2} \times C}\). These feature-level distributions represent the multi-period pattern in feature space, enabling the MLP to learn periodic structure from the teacher at the feature level.

\vspace{-0.5em}
\paragraph{Feature Level.}
Similar to the prediction level, we apply multi-period distillation at the feature level. For the features \(\mathbf{H}'_t \in \mathbb{R}^{D \times C}\) and \(\mathbf{H}_s \in \mathbb{R}^{D \times C}\), we compute the spectrograms and the corresponding period distributions \(\mathbf{Q}_x^\mathbf{H}\) using the same approach as in Equations~\ref{eq:multiperiod_spectrograms} and~\ref{eq:multiperiod_distribution}. Multi-period distillation loss at feature level is then defined as:
\begin{equation}
    \mathcal{L}_{period}^\mathbf{H} = \text{KL}\bigl(\mathbf{Q}_t^\mathbf{H}, \mathbf{Q}_s^\mathbf{H}\bigr).
\end{equation}

\subsection{Overall Optimization and Theoretical Analaysis}
During the training of \method{}, we jointly optimize both the multi-scale and multi-period distillation losses at both the prediction and feature levels, together with the supervised ground-truth label loss:
\begin{equation}
    \mathcal{L}_{sup} = ||\mathbf{Y} - \mathbf{\hat{Y}}_s||^2,
\end{equation}
where \(\mathcal{L}_{sup}\) is the ground-truth loss (for example, MSE loss) used to train MLP directly. Thus, the overall training loss for the student MLP is defined as:
\begin{equation}
    \mathcal{L} = \mathcal{L}_{sup} + \alpha \cdot \bigl(\mathcal{L}_{scale}^\mathbf{Y} + \mathcal{L}_{period}^\mathbf{Y}\bigr) + \beta \cdot \bigl(\mathcal{L}_{scale}^\mathbf{H} + \mathcal{L}_{period}^\mathbf{H}\bigr),
    \label{eq:overall_optimization}
\end{equation}
where \(\alpha\) and \(\beta\) are hyper-parameters that control the contributions of the prediction-level and feature-level distillation loss terms, respectively. The teacher model is pretrained and remains frozen throughout the training process of MLP.


\underline{\textbf{Theoretical Interpretations.}} We provide a theoretical understanding of multi-scale and multi-period distillation loss from \textbf{a novel data augmentation perspective}. We further show that the proposed distillation loss can be interpreted as training with augmented samples derived from a special \textit{mixup}~\cite{mixup} strategy. The distillation process augments data by blending ground truth with teacher predictions, analogous to label smoothing in classification, and provides several benefits for time series forecasting:
\textit{\textbf{(1)} Enhanced Generalization:} It enhances generalization by exposing the student model to richer supervision signals from augmented samples, thus mitigating overfitting, especially with limited or noisy data.
{\textit{\textbf{(2)} Explicit Integration of Patterns:} The augmented supervision signals explicitly incorporate patterns across multiple scales and periods, offering insights that are not immediately evident in the raw ground truth.}
\textit{\textbf{(3}) Stabilized Training Dynamics:} The blending of targets softens the supervision signals, which diminishes the model’s sensitivity to noise and leads to more stable training phases. This will in turn support smoother optimization dynamics and fosters improved convergence. For clarity, our discussion is centered at the prediction level.  We present the following theorem:  
% \wei{in eq. 12, we used alpha and beta; we wanna avoid abusing notations. you may use $\lambda_1$ and $\lambda_2$ in eq. 12 or change the threom}
% \begin{theorem} \label{thm:multiscale}
% Let $(x, y)$ denote original input data pairs and $(x, y^t)$ represent corresponding teacher data pairs. Consider a data augmentation function $\mathcal{A}(\cdot)$ applied to $(x, y)$, generating augmented samples $(x', y')$. Define the training loss on these augmented samples as $\mathcal{L}_{aug} = \textstyle\sum_{(x',y') \in \mathcal{A}(x,y)} |f_s(x') - y'|^2$. Then, the following inequality holds: 
% $\mathcal{L}_{sup} + \lambda \mathcal{L}_{scale} \geq \mathcal{L}_{aug}$
% % \begin{equation}
% %    \mathcal{L}_{sup} + \alpha \mathcal{L}_{scale} \geq \mathcal{L}_{aug}
% % \end{equation}
% when $\mathcal{A}(\cdot)$ is instantiated as a mixup function~\cite{mixup} that interpolates between the original input data $(x,y)$ and teacher data $(x,y^t)$ with a mixing coefficient $\lambda \in (0,1)$, i.e. $y' = \lambda y^t + (1-\lambda) y$.
% \end{theorem}
\begin{theorem} \label{thm:multiscale}
Let $(x, y)$ denote original input data pairs and $(x, y^t)$ represent corresponding teacher data pairs. Consider a data augmentation function $\mathcal{A}(\cdot)$ applied to $(x, y)$, generating augmented samples $(x', y')$. Define the training loss on these augmented samples as $\mathcal{L}_{aug} = \sum_{(x',y') \in \mathcal{A}(x,y)} |f_s(x') - y'|^2$. Then, the following inequality holds: 
$
   \mathcal{L}_{sup} + \eta \mathcal{L}_{scale} \geq \mathcal{L}_{aug},
$
when $\mathcal{A}(\cdot)$ is instantiated as a mixup function~\cite{mixup} that interpolates between the original input data $(x,y)$ and teacher data $(x,y^t)$ with a mixing coefficient $\lambda=\frac{1}{1+\eta}$, i.e. $y' = \lambda y + (1-\lambda) y^t$.
\end{theorem}
We provide proof of Theorem~\ref{thm:multiscale} in Appendix~\ref{app:theory}.  Theorem~\ref{thm:multiscale} suggests that optimizing multi-scale distillation loss \(\mathcal{L}_{\text{scale}}\) jointly with supervised loss \(\mathcal{L}_{\text{sup}}\) is equivalent to minimizing an upper bound on a special \textit{mixup} augmentation loss. In particular, we mix multi-scale teacher predictions \(\{\mathbf{\hat{Y}}_t^{(m)}\}_{m=0}^M\) with ground truth \(\mathbf{Y}\), thereby allowing MLP to learn more informative time series temporal pattern. Similarly, we present a theorem for understanding $\mathcal{L}_{period}$.

% \begin{theorem} \label{thm:multiperiod}
% Define the training loss on the augmented samples using KL divergence as $\mathcal{L}_{aug} = \textstyle\sum_{(x',y') \in \mathcal{A}(x,y)} \text{KL}\big(y', \mathcal{X}(f_s(x'))\big)$, where $\mathcal{X}(\cdot) = \text{Softmax}(\text{Amp}(\text{FFT}(\cdot)))$. Then, the following inequality holds: 
% $\mathcal{L}_{sup} + \lambda \mathcal{L}_{period} \geq \mathcal{L}_{aug}$
% % \begin{equation}
% %    \mathcal{L}_{sup} + \alpha \mathcal{L}_{period} \geq \mathcal{L}_{aug}
% % \end{equation}
% where $\mathcal{A}(\cdot)$ is instantiated as a mixup function that interpolates between the period distribution of original input data $(x,\mathcal{X}(y))$ and teacher data $(x,\mathcal{X}(y^t))$ with a mixing coefficient $\lambda \in (0,1)$, i.e. $y' = \lambda \mathcal{X}(y^t) + (1-\lambda) \mathcal{X}(y)$.
% \end{theorem}
\begin{theorem} \label{thm:multiperiod}
Define the training loss on the augmented samples using KL divergence as $\mathcal{L}_{aug} = \sum_{(x',y') \in \mathcal{A}(x,y)} \text{KL}\big(y', \mathcal{X}(f_s(x'))\big)$, where $\mathcal{X}(\cdot) = \text{Softmax}(\text{FFT}(\cdot))$. Then, the following inequality holds: 
$
   \mathcal{L}_{sup} + \eta\mathcal{L}_{period} \geq \mathcal{L}_{aug},
$
where $\mathcal{A}(\cdot)$ is instantiated as a mixup function that interpolates between the period distribution of original input data $(x,\mathcal{X}(y))$ and teacher data $(x,\mathcal{X}(y^t))$ with a mixing coefficient $\lambda=\eta$, i.e. $y' =  \mathcal{X}(y) + \lambda \mathcal{X}(y^t)$.
\end{theorem}
The proof can be found in Appendix~\ref{app:theory}. Theorem~\ref{thm:multiperiod} shows that optimizing the multi-period distillation loss \(\mathcal{L}_{\text{period}}\) jointly with the supervised loss \(\mathcal{L}_{\text{sup}}\) is equivalent to minimizing an upper bound on the KL divergence between the student period distribution \(\mathcal{X}(f_s(x'))\) (or \(\mathbf{Q}_s\)) and a \emph{mixed} period distribution \(y'\) (or \(\mathbf{Q}_y + \lambda\,\mathbf{Q}_t\)). 
% \wei{we probably do not need (or need to rephrase) the following because it is not very related to the theorem (data augmentation); we need to describe the benefit from the data augmtantion perspective like you did for the above paragraph} This helps the model learn multi-period frequency patterns by incorporating the teacher’s period distribution, thereby identifying and modeling cyclic behaviors with overlapping or multiple periodicities.

% \begin{theorem}\label{thm:multiperiod}
% Optimizing the multi-period distillation loss $\mathcal{L}_{\text{period}}$ 
% is equivalent to minimizing an upper bound on the KL divergence between the student distribution \(\mathbf{Q}_s\) and a \emph{mixed} label distribution \(\alpha\,\mathbf{Q}_y + (1-\alpha)\,\mathbf{Q}_t\).
% Formally, for $\alpha \in [0,1]$, the following inequality holds:
% \[
% \begin{aligned}
% &\alpha\,\mathrm{KL}\bigl(\mathbf{Q}_y, \mathbf{Q}_s\bigr)
% \;+\;
% (1-\alpha)\,\mathrm{KL}\bigl(\mathbf{Q}_t, \mathbf{Q}_s\bigr)\\
% &\ge
% \mathrm{KL}\Bigl(\alpha\,\mathbf{Q}_y + (1-\alpha)\,\mathbf{Q}_t
% \;,\;\mathbf{Q}_s\Bigr).
% \end{aligned}
% \]
% \end{theorem}

% \wei{please number these benefits to enhance readability} These two theorems provide a theoretical understanding of multi-scale and multi-period distillation loss from a novel data augmentation perspective. The distillation process augments data by blending ground truth with teacher predictions, analogous to label smoothing in classification, and provides several benefits for time series forecasting. 
% It enhances generalization by exposing the student model to richer supervision signals, mitigating overfitting, especially with limited or noisy data, and 
% capturing trends or patterns not immediately apparent in the ground truth. 
% Furthermore, the softened targets from this blending reduce sensitivity to noise, stabilize training, and facilitate better convergence by ensuring smoother optimization dynamics.

% \wei{this paragraph is very important; please carefully rewrite; the goal is to highglight the novelty of this theoretical perspective and provide benefits of why our distillation framework can benefit the forecasting process} 


\section{Experiments}

We conduct comprehensive experiments across multiple datasets and model architectures to validate our method's ability to decouple explanation robustness from classification robustness. Our evaluation addresses three key research questions:
\begin{itemize}
    \item \textbf{RQ1:} Does \ours have better quantify uncertainties?
    \item \textbf{RQ2:} How do different ensemble methods and information from both dimensions help?
    \item \textbf{RQ3:} Is\ours robust to different settings? 
\end{itemize}


\begin{table}[H]
\centering
\resizebox{!}{0.11\textwidth}{
\begin{tabular}{@{}lc@{}}
\toprule
\textbf{Measure} & \textbf{Details} \\ 
\midrule
$U_{\textit{Eigv}}(Dis)$ & \multicolumn{1}{c}{Spectral eigenvalue on the disagreement.} \\ 
$U_{\textit{Ecc}}(Dis)$ & \multicolumn{1}{c}{Average distance in responses' disagreement.} \\ 
$U_{\textit{Degree}}(Dis)$ & \multicolumn{1}{c}{Degree of disagreement similarity Matrix.} \\ 
$U_{\textit{Eigv}}(Agre)$ & \multicolumn{1}{c}{Spectral eigenvalue on the agreement.} \\ 
$U_{\textit{Ecc}}(Agre)$ & \multicolumn{1}{c}{Average distance in responses' agreement.} \\ 
$U_{\textit{Degree}}(Agre)$ & \multicolumn{1}{c}{Degree Matrix of agreement Matrix.} \\ 
$p(true)$ & \multicolumn{1}{c}{Entropy of knowledge dimension responses} \\ 
$D-UE$ & \multicolumn{1}{c}{eigenvalue from Laplacian of a directional graph} \\ 

\bottomrule
\end{tabular}}
\vspace{-1mm}
\caption{The baseline methods and explanations.}
\vspace{-5mm}
\label{tab:baslines}
\end{table}

\subsection{Experimental Setup}
\label{sec:setup}
\subsubsection{Datasets} As mentioned in \cref{sec:background}, following prior works~\cite{lin2022teaching}, we focus on open-form question-answering 
(QA) tasks in this paper. We adopt 4 different classic QA datasets. Coqa~\cite{reddy2019coqa} is a conversational question-answering dataset that contains dialogues with free-form answers grounded in diverse passages, which is the easiest dataset among all datasets. HotpotQA~\cite{yang2018hotpotqa} is a multi-hop QA dataset that demands reasoning over multiple Wikipedia paragraphs to derive correct answers. NQ-Open~\cite{kwiatkowski2019natural} consists of real-world queries from Google Search, requiring models to retrieve and answer questions without explicit context, which is the hardest dataset. 
\subsubsection{Models to Evaluate} We evaluate \ours on Llama family~\cite{touvron2023llama}, which is the one of the most popular LLMs. In detail, we use Llama-2-13b and Llama-2-7B to demonstrate the effectiveness of \ours with different model sizes and use Llama-3.1-8B~\cite{dubey2024llama} to that \ours could also work on the different version of Llama. To further demonstrate the generalization ability for other architectures,  we also use Phi4~\cite{abdin2024phi} and Deepseek-R1-distill-7B~\cite{guo2025deepseek} in our paper.


%\textcolor{red}{Not sure whether there is a section labeled as "sec eva metric" refered by Sec. }

\subsubsection{Evaluation Metrics} Effective uncertainty measures should accurately represent the reliability of LLM responses, with higher uncertainty more likely leading to incorrect generations and vice versa~\cite{lin2023generating,kuhn2023semantic}. Following prior works~\cite{lin2023generating,da2024llm}, we mainly use UQ values to predict whether an answer is correct or not. Following prior works~\cite{lin2023generating,da2024llm}, we will use Area Under Receiver Operating Characteristic (AUROC) and Area Under Accuracy Rejection Curve (AUARC) as evaluation metrics, where a higher AUROC or AUARC demonstrates better uncertainty measures. To compute AUROC and AUARC, the accuracy of each original response is required. Following previous works~\cite{da2024llm,lin2023generating}, we use another LLM to provide correctness from 0-100 to each response. If the correctness is greater than 70, we label the response as correct. In this paper, we use Qwen-34B~\cite{bai2023qwen} to evaluate the correctness.

\subsubsection{Knowledge Extracted Models} In this paper, we mainly use llama-2-13b~\cite{touvron2023llama} as the auxiliary models to extract the knowledge dimension of responses. To demonstrate the robustness of \ours with different knowledge-extracted models, we also contain the results for different LLMs as knowledge-extracted models.

\subsubsection{Baselines} In this paper, we compared \ours with baselines that use semantic dimension response and knowledge dimension response. For semantic dimension, we mainly compared with methods that come from \citet{lin2023generating}. In detail, we incorporate six distinct methods from \citet{lin2023generating}, which differ based on the operations applied after computing similarity and whether they utilize agreement (entailment) probabilities or disagreement (contradiction) logits to construct the similarity matrix. For knowledge dimension, we use D-UE~\cite{da2024llm} and $p(true)$~\cite{kadavath2022language} as the baselines. Note that we use $p(true)$ on the knowledge dimension of response. We show the detailed explanations of all baselines in \cref{tab:baslines}

\begin{figure*}[t]
\centering
\begin{minipage}[t]{0.32\linewidth}
  \centering
  \includegraphics[width=\linewidth,trim=0 0 0 1cm, clip]{images/ablation.pdf}
  \captionof{figure}{Ablation studies that show that \ours fully utilizes all the information from both dimensions.}
  \label{fig:ablation}
\end{minipage}\hfill
\begin{minipage}[t]{0.32\linewidth}
  \centering
  \includegraphics[width=\linewidth]{images/knowledge_extract.pdf}
  \captionof{figure}{Performance for different knowledge extract models on CoQA and NQ\_Open with llama3.1.}
  \label{fig:knowledge_extract}
\end{minipage}\hfill
\begin{minipage}[t]{0.32\linewidth}
  \centering
  \includegraphics[width=\linewidth]{images/Jacc.pdf}
  \captionof{figure}{Performance that uses Jaccard similarity on CoQA and NQ\_Open with llama3.1.}
  \label{fig:jacc}
\end{minipage}
\end{figure*}

\subsection{Does \ours have better quantify uncertainties? (RQ1)}
\label{sec:main_result}
In this section, we explore whether \ours has better uncertainties compared with state-of-the-art uncertainty quantification methods. In \cref{tab:main_results}, we compare \ours with 8 baselines across three different datasets and five different models as introduced in \cref{sec:setup} In detail, we have the following observations:

\noindent $\bullet$ Compared with all baseline methods, \ours achieves the best performance overall. Especially when we consider AUROC. For AUARC, \ours achieves the best performance for NQ\_Open while \ours also achieves the comparable performance for CoQA in most scenarios. These results demonstrate that \ours has better quantify uncertainties overall. \\
\noindent $\bullet$ Among all datasets, \ours achieves the highest performance improvement on NQ\_Open, which is the most difficult dataset among all datasets and may lose to baselines for an easier dataset like CoQA. This indicates \ours could perform even better when the task is harder, where uncertainty quantification is more important. \\
\noindent $\bullet$ Two different ensemble methods show very similar results. Min strategy performs better than the sum strategy under $61.51\%$ situations, indicating that difficult datasets might also have more complex structures that single one tensor decomposition might oversight some information while using min structure could reduce such oversight by considering the best cases. However, both ensemble methods show a better performance than all baselines, which proves the effectiveness of tensor decomposition. \\

From these results, we get a conclusion that overall, \ours have better uncertainties.


\subsection{How do different ensemble methods and information from both dimensions help? (RQ2)}
\label{sec:ablation}
In this section, we use more experiments to prove the necessity of using information from both semantic and knowledge dimensions as well as using tensor decomposition. In detail, we consider the following methods: 1) \ours with only semantic responses, 2) \ours with only knowledge responses and 3) Concatenating similarity matrices from semantic and knowledge dimensions into a 2D matrix and applying SVD, 4) only using one tensor decomposition. In \cref{fig:ablation}, we show the comparison between \ours and other methods.  The results show that \ours consistently outperforms its variants and SVD method that repeated information will dominate the features, showing the effectiveness of our framework.




\subsection{Is \ours robust to different settings? (RQ3)}
\subsubsection{Different Knowledge Extracted Models} Knowledge extracted models influence the claim extraction in \ours as stated in \cref{subsubsec:knowledge}. Therefore, in this section, We test the robustness of \ours on various knowledge extracted models. unlike using llama2-13b in \cref{sec:main_result} and \cref{sec:ablation}, we conduct experiments on CoQA and NQ\_open using llama2-7b and llama3.1 as the knowledge extracted models, We show the results in \cref{fig:knowledge_extract}. From the figure, we can see that using Phi4 could even achieve a better result, indicating \ours has more potential with the development of LLMs. 

\subsubsection{Different Accuracy Thresholds} Different accuracy thresholds lead to different accuracy and influence the evaluation of uncertainties. In the previous experiments, we all set the accuracy threshold to 70 as mentioned in \cref{sec:setup}.  To test the robustness of \ours under different accuracy thresholds, we choose an extra dataset TriviaQA~\cite{joshi2017triviaqa}, which is considered the easiest dataset, and NQ\_Open, which is the most challenging dataset in our paper to conduct experiments. We show the results with accuracy thresholds of 70 and 90 in \cref{tab:Accuracy_threshold}. From the results, we can see that increasing the accuracy threshold decreases the performance of all baselines while the performance of \ours could even increase for datasets with different difficulties, showing the robustness of \ours in different settings. 

\subsubsection{Different Similarity Metrics} Finally, different similarity metrics lead to different similarity matrices. Therefore, to test whether \ours also has a good performance for different similarities, we use Jaccard similarity instead of using an NLI model in this section and the results are presented in \cref{fig:jacc}. The results show that using Jaccard similarity will boost the performance for a simple dataset like CoQA but hurt the performance for a difficult dataset like NQ\_Open. This is because the answer to a simple question might not have a deeper semantic meaning that requires NLI models. However, \ours can still outperform baseline methods that also use Jaccard similarity, showing the robustness of \ours.





\begin{table}[h]
    \centering
    \resizebox{0.5\textwidth}{!}{
    \begin{tabular}{lcccc}
        \toprule
        \multirow{2}{*}{Methods} & \multicolumn{2}{c}{Accuracy Threshold: 0.7} & \multicolumn{2}{c}{Accuracy Threshold: 0.9} \\
        \cmidrule(lr){2-3} \cmidrule(lr){4-5}
        & AUROC & AUARC & AUROC & AUARC \\
        \midrule
        \multicolumn{5}{c}{\textbf{Dataset: TriviaQA} [Easy]} \\
        \midrule
        Eigv(Dis) & 0.8261 & 0.8094 & 0.8100& 0.7604\\
        Ecc(Dis) & 0.8063& 0.7940&0.7892 & 0.7415\\
        Degree(Dis) &0.8399 & 0.8163&0.8259 & 0.7694\\
        Eigv(Agre) &0.8436 &0.8116 &0.8351 & 0.7721 \\
        Ecc(Agre) & \textbf{0.8510}&0.8189 & 0.8374&0.7721 \\
        Degree(Agre) &0.8396 &\textbf{0.8397} &0.8384 & 0.7739\\
        \ours-Sum &0.8428 &0.8144 & 0.8438&0.7749 \\
        \ours-Min &0.8431 &0.8149 & \textbf{0.8440} & \textbf{0.7754}\\
        \midrule
        \multicolumn{5}{c}{\textbf{Dataset: NQ\_Open} [Hard]} \\
        \midrule
        Eigv(Dis) & 0.6162 & 0.7300 &0.5636 &0.6017 \\
        Ecc(Dis) & 0.6210& 0.7330& 0.5658&0.5941 \\
        Degree(Dis) &0.6130 & 0.7168&0.5662 &0.6033 \\
        Eigv(Agre) &0.6258 &0.7276 & 0.6146& 0.6290 \\
        Ecc(Agre) & 0.6273&0.7311 &0.6239 &0.6344\\
        Degree(Agre) &0.6286 &0.7355 & 0.6221&0.6299 \\
        \ours-Sum &\textbf{0.6334} &\textbf{0.7410} &\textbf{0.6351} &\textbf{0.6430} \\
        \ours-Min &0.6332 &0.7409 & 0.6350 &0.6429 \\
        \bottomrule
    \end{tabular}
    }
    \caption{Comparison of different methods across different accuracy thresholds on TrivialQA and NQ\_Open with llama2-13B. The results show that our methods outperform baselines after increasing the accuracy threshold, indicating that our methods have an advantage on more difficult datasets.}
    \vspace{-7mm}
    \label{tab:Accuracy_threshold}
\end{table}


\section{Conclusion}
In this paper, we propose ChineseEcomQA, a scalable question-answering benchmark designed to rigorously assess LLMs on fundamental e-commerce concepts. ChineseEcomQA is characterized by three core features: Focus on Fundamental Concept, E-Commerce Generalizability, and Domain-Specific Expertise, which collectively enable systematic evaluation of LLMs' e-commerce knowledge. Leveraging ChineseEcomQA, we conduct extensive evaluations on mainstream LLMs, yielding critical insights into their capabilities and limitations. Our findings not only highlight performance disparities across models but also delineate actionable directions for advancing LLM applications in the e-commerce domain.

% In the unusual situation where you want a paper to appear in the
% references without citing it in the main text, use \nocite
\nocite{langley00}

\bibliography{icml}
\bibliographystyle{icml2025}


%%%%%%%%%%%%%%%%%%%%%%%%%%%%%%%%%%%%%%%%%%%%%%%%%%%%%%%%%%%%%%%%%%%%%%%%%%%%%%%
%%%%%%%%%%%%%%%%%%%%%%%%%%%%%%%%%%%%%%%%%%%%%%%%%%%%%%%%%%%%%%%%%%%%%%%%%%%%%%%
% APPENDIX
%%%%%%%%%%%%%%%%%%%%%%%%%%%%%%%%%%%%%%%%%%%%%%%%%%%%%%%%%%%%%%%%%%%%%%%%%%%%%%%
%%%%%%%%%%%%%%%%%%%%%%%%%%%%%%%%%%%%%%%%%%%%%%%%%%%%%%%%%%%%%%%%%%%%%%%%%%%%%%%
\newpage
\appendix
% \onecolumn
\cleardoublepage

\appendix
% \section{Notation Table}

\def \TabNotation{
\begin{table}[]
\centering
\resizebox{!}{0.18\textwidth}{
\begin{tabular}{@{}lc@{}}
\toprule
\textbf{Notation} & \textbf{Explanation} \\ 
\midrule
$\mathcal{M}$ & Language model \\ 
$\Sigma$ & Vocabulary \\
$\xInput$ & Input prompt \\ 
$\predSeq$ & Output responds \\ 
$A= \{ a_i,...a_m\}$ & Set of reference answers \\ 
$C_{\mathcal{M}}(\xInput,\predSeq)$ & Confidence score of $\predSeq$ \\ 
$f(\predSeq,\xInput)$ & Correctness function \\
$sim(\predSeq,A)$ & Similarity score \\
$\tau$ & LLM predefined threshold \\
$n$ & Number of open-form responses \\
$o_i$ & Options in QA dataset \\
$K$ & Number of options \\
\bottomrule
\end{tabular}}
\vspace{-1mm}
\caption{The notation used in this paper}
\vspace{-5mm}
\label{tab:notations}
\end{table}
}
%\TabNotation

% \section{Example Appendix}
% \label{sec:appendix}

\section{Experiments Details}\label{appendix:sec:exp_imp}

\subsection{Dataset Description}\label{sec:datasetDes}


\begin{itemize}
    \item \textbf{C-QA} A multiple-choice dataset designed for commonsense question answering. Each question requires world knowledge and reasoning to determine the correct answer from 5 given choices. The dataset consists of 1,221 test questions.
    
    \item \textbf{QASC} A multiple-choice commonsense reasoning dataset with 8 answer choices per question. Compared to C-QA, QASC presents a higher level of difficulty. While the dataset was originally designed for multi-hop reasoning, our focus is not on evaluating the reasoning capabilities of LLMs. Therefore, we do not provide the supporting facts to the model and instead present only the question. For our experiments, we use the original validation set, which includes 926 questions.
    
    \item \textbf{MedQA} A multiple-choice dataset with 5 options for answers, specifically designed for medical QA. 
    It covers three languages: English, simplified Chinese, and traditional Chinese, and contains 12,723, 34,251, and 14,123 questions for the three languages, respectively.
    The questions are sourced from professional medical board exams, making this dataset particularly challenging due to its reliance on specialized medical knowledge. 
    For our experiments, we randomly selected the first 1,000 questions from the English dataset.
    
    \item \textbf{RACE-m and RACE-h} used in this paper are derived from the RACE (\textbf{R}e\textbf{A}ding \textbf{C}omprehension dataset from \textbf{E}xaminations) dataset, a large-scale machine reading comprehension dataset introduced by Lai et al~\cite{lai2017race}. 
    RACE comprises 27,933 passages and 97,867 questions collected from English examinations for Chinese students aged 12–18. 
    These datasets evaluate a model’s ability to comprehend complex passages and answer questions based on contextual reasoning. 
    Each question is accompanied by four answer choices, with only one correct option. 
    For our experiments, we randomly sampled 1,000 questions from the entire dataset using a fixed random seed of 42 to ensure reproducibility.
\end{itemize}

% \textbf{RACE datasets:} The RACE-h and RACE-m datasets used in this paper are derived from the RACE (\textbf{R}e\textbf{A}ding \textbf{C}omprehension dataset from \textbf{E}xaminations) dataset, a large-scale machine reading comprehension dataset introduced by Lai et al~\cite{lai2017race}. 
% RACE comprises 27,933 passages and 97,867 questions collected from English examinations for Chinese students aged 12–18. 
% The dataset is split into two subsets: RACE-M, which includes 28,293 questions from middle school exams, and RACE-H, containing 69,574 questions from high school exams. Each question in the dataset is paired with four candidate answers, only one of which is correct. Unlike other machine reading comprehension datasets generated through heuristics or crowdsourcing, RACE's questions are designed by domain experts to test human reading and comprehension skills, making it a unique resource for evaluating large language understanding of models. For our specific study, since collecting responses and conducting evaluations is relatively time-consuming, so we conducted a random sample of 1,000 questions extracted from the entire dataset using a random seed of 42 to ensure reproducibility.


% \begin{table}[t!]
%     \centering
%     \begin{tabular}{lcc}
%         \toprule
%         Method & Description & \\
%         \midrule
%         \multicolumn{3}{c}{\textbf{Black-Box Methods}} \\
%         \midrule
%         Ecc(C) & \multicolumn{1}{c}{..} \\ 
%         Deg(C) & \multicolumn{1}{c}{..} \\ 
%         Ecc(E) & \multicolumn{1}{c}{..} \\ 
%         Deg(E) & \multicolumn{1}{c}{..} \\ 
%         Ecc(J) & \multicolumn{1}{c}{..} \\ 
%         Deg(J) & \multicolumn{1}{c}{..} \\ 
%         \midrule
%         \multicolumn{3}{c}{\textbf{White-Box Methods}} \\
%         \midrule
%         P(true) & \multicolumn{1}{c}{..} \\ 
%         CSL & \multicolumn{1}{c}{..} \\ 
%         CSL-next & \multicolumn{1}{c}{..} \\ 
%         SL & \multicolumn{1}{c}{..} \\ 
%         SL(norm) & \multicolumn{1}{c}{..} \\ 
%         TokenSAR & \multicolumn{1}{c}{..} \\ 
%         \bottomrule
%     \end{tabular}
%     \caption{All the baseline methods}
%     \vspace{-5mm}
%     \label{tab:similarity_matrix_stat}
% \end{table}
%\section{Implement Confidence Estimation Methods}


\subsection{Prompt Details}
\label{sec:appendix_prompt}
\begin{itemize}
    \item We use the following prompt to collect open-form responses for each of the 5 datasets separately.
    
\includegraphics[width=.9\columnwidth]{figures/generate_prompt.png}

    \item We use the following prompt to elicit P(True) confidence score.
    The ``Possible Answer'' is an option from the multiple-choice dataset.
    
\includegraphics[width=.9\columnwidth]{figures/ptrue_prompt.png}
\end{itemize}






\subsection{Computation Resources}
To efficiently process multiple queries, we used vLLM~\cite{kwon2023efficientvllm} for parallel inference.
All experiments were conducted on a Linux server running Ubuntu, equipped with an A100 80GB GPU.


\subsection{Response Generation }
For black-box methods, we mostly adopt the experimental configurations from~\citet{lin2024generating}. 
Sampling-based black-box confidence measures use $n=20$ open-form responses per question. 
The temperature settings for different LLMs are kept at their default values.


\section{Additional Experiments Results}\label{sec:full_results}

\subsection{Full Results of Different Evaluation Metrics}
In the main text, due to space constraints, we only show a subset of the AUROC results.
Here, \cref{appendix:tab:bb:auc,appendix:tab:wb:auc} show the AUROC and AUARC for black-box and white-box confidence measures, respectively. 
Similarly, \cref{appendix:tab:bb:calib,appendix:tab:wb:calib} present RCE and ECE results.
Note that all ECE are based on \textit{calibrated} confidence measures for fair comparisons, as some original confidence measures are not even constrained to $[0,1]$.
For the calibration step, we applied histogram binning method~\cite{KDD_HistogramBinning} on all methods.%, and compute the adaptive calibration error (ACE)~\cite{Nixon_2019_CVPR_Workshops}.


% full table
%%%%%%%%%%%%%%%%%%%%%%%%%%%%%%%%%%%%%%%%%%%%%%%%%%%%%%%%%%%%%%%%%%%%%%%%%%%%%%%%%%%%%%%%%%%%%%%%%%%%%%%%%%%%
%white
%%%%%%%%%%%%%%%%%%%%%%%%%%%%%%%%%%%%%%%%%%%%%%%%%%%%%%%%%%%%%%%%%%%%%%%%%%%%%%%%%%%%%%%%%%%%%%%%%%%%%%%%%%%%%%%%
\begin{table*}[h!]
\centering
\resizebox{\textwidth}{!}{%
\begin{tabular}{llcccccccccccc}
\toprule
\multirow{2}{*}{\textbf{Dataset}} & \multirow{2}{*}{\textbf{Model}} & \multicolumn{6}{c}{\textbf{AUROC $\Uparrow$}} & \multicolumn{6}{c}{\textbf{AUARC} $\Uparrow$} \\ 
\cmidrule(lr){3-8} \cmidrule(lr){9-14}
 &  & Ecc(C) & Deg(C) & Ecc(E) & Deg(E) & Ecc(J) & Deg(J) & Ecc(C) & Deg(C) & Ecc(E) & Deg(E) & Ecc(J) & Deg(J) \\ 
\midrule
\multirow{4}{*}{C-QA}
 & Llama2-7b   & 60.981 & 66.651 & 78.629 & 72.771 & 67.081 & 71.668 & 29.386 & 33.266 & 38.221 & 35.858 & 34.681 & 36.915 \\
 & Llama3-8b   & 57.590 & 62.592 & 80.004 & 73.734 & 65.886 & 76.583 &32.062 & 33.232 & 38.414 & 32.648 & 37.150 & 38.596\\
 & Phi4        & 67.879 & 68.413 & 80.712 & 69.976 & 71.447 & 75.278 & 32.123 & 31.596 & 19.294 & 30.032 & 28.570 & 24.739 \\
 & Qwen2.5-32b & 71.409 & 73.931 & 81.885 & 77.087 & 69.473 & 74.645 & 34.775 & 37.399 & 39.926 & 37.964 & 36.776 & 38.808 \\
\midrule
\multirow{4}{*}{QASC}
 & Llama2-7b   & 58.949 & 61.978 & 73.221 & 69.200 & 61.659 & 66.877 & 17.509 & 19.628 & 25.724 & 23.556 & 21.469 & 23.251 \\
 & Llama3-8b   & 55.121 & 55.446 & 74.912 & 72.033 & 64.124 & 72.657 & 15.785 & 15.952 & 25.199 & 24.163 & 23.198 & 25.786 \\
 & Phi4        & 65.100 & 65.553 & 76.980 & 67.692 & 68.496 & 71.209 & 20.297 & 21.063 & 26.740 & 21.422 & 24.067 & 24.308 \\
 & Qwen2.5-32b & 62.218 & 61.611 & 74.546 & 71.702 & 64.658 & 69.131 & 19.522 & 19.830 & 25.695 & 24.306 & 23.182 & 24.510  \\
\midrule
\multirow{4}{*}{MedQA}
 & Llama2-7b   & 53.683 & 54.129 & 52.076 & 52.963 & 53.137 & 53.778 & 21.956 & 23.105 & 21.160 & 22.863 & 23.454 & 23.371 \\
 & Llama3-8b   & 52.824 & 53.971 & 51.641 & 53.523 & 55.257 & 59.552 & 21.125 & 22.103 & 20.390 & 22.164 & 25.598 & 26.617 \\
 & Phi4        & 60.055 & 59.512 & 54.945 & 55.261 & 57.815 & 65.067 & 25.081 & 25.410 & 22.077 & 22.940 & 27.573 &29.201 \\
 & Qwen2.5-32b & 60.071 & 61.737 & 54.727 & 58.454 & 61.564 & 63.783 & 24.998 & 28.045 & 22.246 & 26.331 & 29.848 & 30.054 \\
\midrule
\multirow{4}{*}{RACE-m}
 & Llama2-7b  & 65.473 & 64.304 & 61.022 & 59.245 & 67.480 & 67.760 & 34.147 & 36.637 & 32.570 & 33.994 & 38.844 & 38.904 \\
 & Llama3-8b & 62.385 & 63.351 & 61.872 & 58.711 & 68.391 & 73.267 & 30.774 & 35.054 & 31.639 & 32.491 & 41.231 & 43.055 \\
 & Phi4     & 66.461 & 64.344 & 64.492 & 58.981 & 68.124 & 72.304 & 34.312 & 35.355 & 32.903 & 32.232 & 41.311 & 41.895  \\
 & Qwen2.5-32b & 65.425 & 67.627 & 60.268 & 61.309 & 75.420 & 75.746 & 34.393 & 37.409 & 32.092 & 34.850 & 44.281 & 44.585 \\
\midrule
\multirow{4}{*}{RACE-h}
 & Llama2-7b    & 58.991 & 53.597 & 57.178 & 54.037 & 59.300 & 59.856 & 34.147 & 36.637 & 32.570 & 33.994 & 38.844 & 38.904 \\
 & Llama3-8b  & 56.372 & 53.560 & 58.456 & 54.004 & 57.488 & 63.788 & 27.959 & 28.483 & 29.120 & 27.823 & 33.912 & 36.139 \\
 & Phi4    & 60.550 & 53.867 & 61.263 & 54.442 & 59.639 & 64.385 & 30.733 & 28.641 & 31.411 & 28.157 & 34.519 & 35.710    \\
 & Qwen2.5-32b   & 60.012 & 54.781 & 55.984 & 55.657 & 64.985 & 66.130 & 31.049 & 29.180 & 30.459 & 28.921 & 37.620 & 37.734 \\
\bottomrule
\end{tabular}%
}
\caption{AUROC and AUARC for black-box methods, across different models and datasets}
\label{appendix:tab:bb:auc}
\end{table*}


% only roc
% \begin{table*}[h!]
% \centering
% \resizebox{0.5\textwidth}{!}{%
% \begin{tabular}{llcccccc}
% \toprule
% \multirow{2}{*}{\textbf{Dataset}} & \multirow{2}{*}{\textbf{Model}} & \multicolumn{6}{c}{\textbf{AUROC $\Uparrow$}} \\
% \cmidrule(lr){3-8}
%  &  & Ecc(C) & Deg(C) & Ecc(E) & Deg(E) & Ecc(J) & Deg(J) \\
% \midrule
% \multirow{4}{*}{C-QA}
%  & Llama2-7b   & 60.981 & 66.651 & 78.629 & 72.771 & 67.081 & 71.668 \\
%  & Llama3-8b   & 57.590 & 62.592 & 80.004 & 73.734 & 65.886 & 76.583 \\
%  & Phi4        & 67.879 & 68.413 & 80.712 & 69.976 & 71.447 & 75.278 \\
%  & Qwen2.5-32b & 71.409 & 73.931 & 81.885 & 77.087 & 69.473 & 74.645 \\
% \midrule
% \multirow{4}{*}{QASC}
%  & Llama2-7b   & 58.949 & 61.978 & 73.221 & 69.200 & 61.659 & 66.877 \\
%  & Llama3-8b   & 55.121 & 55.446 & 74.912 & 72.033 & 64.124 & 72.657 \\
%  & Phi4        & 65.100 & 65.553 & 76.980 & 67.692 & 68.496 & 71.209 \\
%  & Qwen2.5-32b & 62.218 & 61.611 & 74.546 & 71.702 & 64.658 & 69.131 \\
% \midrule
% \multirow{4}{*}{MedQA}
%  & Llama2-7b   & 53.683 & 54.129 & 52.076 & 52.963 & 53.137 & 53.778 \\
%  & Llama3-8b   & 52.824 & 53.971 & 51.641 & 53.523 & 55.257 & 59.552 \\
%  & Phi4        & 60.055 & 59.512 & 54.945 & 55.261 & 57.815 & 65.067 \\
%  & Qwen2.5-32b & 60.071 & 61.737 & 54.727 & 58.454 & 61.564 & 63.783 \\
% \midrule
% \multirow{4}{*}{RACE-m}
%  & Llama2-7b   & 65.473 & 64.304 & 61.022 & 59.245 & 67.480 & 67.760 \\
%  & Llama3-8b   & 62.385 & 63.351 & 61.872 & 58.711 & 68.391 & 73.267 \\
%  & Phi4        & 66.461 & 64.344 & 64.492 & 58.981 & 68.124 & 72.304 \\
%  & Qwen2.5-32b & 65.425 & 67.627 & 60.268 & 61.309 & 75.420 & 75.746 \\
% \midrule
% \multirow{4}{*}{RACE-h}
%  & Llama2-7b   & 58.991 & 53.597 & 57.178 & 54.037 & 59.300 & 59.856 \\
%  & Llama3-8b   & 56.372 & 53.560 & 58.456 & 54.004 & 57.488 & 63.788 \\
%  & Phi4        & 60.550 & 53.867 & 61.263 & 54.442 & 59.639 & 64.385 \\
%  & Qwen2.5-32b & 60.012 & 54.781 & 55.984 & 55.657 & 64.985 & 66.130 \\
% \bottomrule
% \end{tabular}%
% }
% \caption{Black Box Methods Performance Metrics Across Different Models and Datasets (AUROC)}
% \label{tab:metrics_table}
% \end{table*}



% \begin{table*}[h!]
% \centering

%%%%%%%%%%%%%%%%%%%%%%%%%%%%%%%%%%%%%%%%%%%%%%%%%%%%%%%%%%%%%%%%%%%%%%%%%%%%%%%%%%%%%%%%%%%%%%%%%%%%%%%%%%%%
%black
%%%%%%%%%%%%%%%%%%%%%%%%%%%%%%%%%%%%%%%%%%%%%%%%%%%%%%%%%%%%%%%%%%%%%%%%%%%%%%%%%%%%%%%%%%%%%%%%%%%%%%%%%%%%%%%%


\begin{table*}[h!]
\centering
\resizebox{\textwidth}{!}{%
\begin{tabular}{llcccccccccccc}
\toprule
\multirow{2}{*}{\textbf{Dataset}} & \multirow{2}{*}{\textbf{Model}} & \multicolumn{6}{c}{\textbf{AUROC} $\Uparrow$} & \multicolumn{6}{c}{\textbf{AUARC} $\Uparrow$} \\ \cmidrule(lr){3-8} \cmidrule(lr){9-14}
 &  & P(true) & CSL & CSL-next & SL & Perplexity & TokenSAR & P(true) & CSL & CSL-next & SL & Perlexity & TokenSAR \\ \midrule
\multirow{4}{*}{C-QA}
 & Llama2-7b   & 62.278 & 78.253 & 74.799 & 81.390 & 76.958 & 77.888 &  28.401 & 38.231 & 36.213 & 40.178 & 37.579 & 37.450 \\
 & Llama3-8b   & 82.760 & 78.423 & 73.068 & 81.731 & 75.503 & 75.385 & 40.235 & 38.191 & 35.096 & 40.152 & 36.368 & 35.453  \\
 & Phi4        & 86.184 & 77.382 & 73.477 & 78.903 & 75.471 & 75.722 & 42.447 & 37.984 & 35.749 & 38.452 & 36.928 & 36.630  \\
 & Qwen2.5-32b & 89.892 & 82.486 & 77.087 & 82.143 & 78.964 & 79.064 & 45.449 & 40.802 & 38.003 & 40.596 & 38.674 & 38.215\\%[1ex]
 \midrule
\multirow{4}{*}{QASC}
 & Llama2-7b   & 66.198 & 77.535 & 76.053 & 79.589 & 77.637 & 77.696 &  19.815 & 25.986 & 25.494 & 27.632 & 26.324 & 25.921\\
 & Llama3-8b   & 86.069 & 77.970 & 73.090 & 80.718 & 74.531 & 75.006 &  30.127 & 26.215 & 24.251 & 28.253 & 24.442 & 24.308 \\
 & Phi4        & 84.478 & 77.556 & 74.596 & 78.661 & 75.678 & 76.222 & 29.977 & 26.068 & 25.246 & 27.064 & 25.463 &25.307 \\
 & Qwen2.5-32b & 88.998 & 79.324 & 73.895 & 78.598 & 74.485 & 75.175 & 32.992 & 26.810 & 24.608 & 27.387 & 24.069 & 23.992  \\%[1ex]
  \midrule
\multirow{4}{*}{MedQA}
 & Llama2-7b   & 54.660 & 55.144 & 55.852 & 54.766 & 55.766 & 55.703 & 22.414 & 24.437 & 24.888 & 24.246 & 24.848 & 24.795  \\
 & Llama3-8b   & 77.493 & 57.384 & 57.894 & 57.919 & 57.592 & 57.530 & 36.884 & 24.072 & 25.225 & 25.879 & 24.973 & 24.803 \\
 & Phi4 & 86.888 & 65.550 & 64.284 & 63.287 & 65.588 & 65.696 &42.615 & 31.671 & 31.050 & 30.888 &31.752 & 31.775  \\
 & Qwen2.5-32b & 80.131 & 63.264 & 63.712 & 63.109 & 62.564 & 62.164 &40.197 & 27.495 & 27.754 & 29.382 & 27.440 & 27.221  \\%[1ex]
  \midrule
\multirow{4}{*}{RACE-m}
 & Llama2-7b  & 63.965 & 69.194 & 70.819 & 67.568 & 71.823 & 71.984 & 35.543 & 38.429 & 39.404 & 38.870 & 40.030 & 40.133 \\
 & Llama3-8b   & 82.118 & 67.317 & 70.875 & 69.321 & 69.851 & 70.029 & 47.145 & 36.953 & 40.206 & 40.508 & 39.144 & 39.232 \\
 & Phi4        & 90.543 & 68.334 & 69.5354 & 68.8049 & 69.025 & 69.188 & 52.457 & 36.638 & 38.717 & 40.314 & 37.972 & 38.057 \\
 & Qwen2.5-32b  & 56.049 &67.294 & 69.102 & 73.267 & 69.147 & 69.279 & 29.283 & 34.913 & 36.873 & 42.373 & 36.220 & 36.318 \\%[1ex]
  \midrule
\multirow{4}{*}{RACE-h}
 & Llama2-7b   & 61.265 & 61.905 & 62.481 & 59.889 & 63.486 & 63.465 & 35.543 & 38.429 & 39.404 & 38.870 & 40.030 & 40.133 \\
 & Llama3-8b    & 79.466 & 60.775 & 63.868 & 61.253 & 64.134 & 64.146 & 44.910 & 31.300 & 34.086 & 33.463 & 33.973 & 33.974 \\
 & Phi4       & 87.172 & 62.253 & 62.680 & 60.178 & 63.391 & 63.383 &  50.250 & 32.395 & 33.484 & 33.243 & 33.547 & 33.537 \\
 & Qwen2.5-32b   & 52.811 & 61.837 & 64.047 & 63.555 & 64.050 & 64.024 & 27.605 & 31.279 & 32.714 & 34.462 & 32.462 & 32.458 \\
\bottomrule
\end{tabular}%
}
\caption{AUROC and AUARC for white-box methods, across different models and datasets}
\label{appendix:tab:wb:auc}
\end{table*}

% \begin{table*}[h!]
% \centering
% \resizebox{0.5\textwidth}{!}{%
% \begin{tabular}{llcccccc}
% \toprule
% \multirow{2}{*}{\textbf{Dataset}} & \multirow{2}{*}{\textbf{Model}} & \multicolumn{6}{c}{\textbf{AUROC} $\Uparrow$} \\
% \cmidrule(lr){3-8}
%  &  & P(true) & CSL & CSL-next & SL & Perplexity & TokenSAR \\ 
% \midrule
% \multirow{4}{*}{C-QA}
%  & Llama2-7b   & 62.278 & 78.253 & 74.799 & 81.390 & 76.958 & 77.888 \\
%  & Llama3-8b   & 82.760 & 78.423 & 73.068 & 81.731 & 75.503 & 75.385 \\
%  & Phi4        & 86.184 & 77.382 & 73.477 & 78.903 & 75.471 & 75.722 \\
%  & Qwen2.5-32b & 89.892 & 82.486 & 77.087 & 82.143 & 78.964 & 79.064 \\
% \midrule
% \multirow{4}{*}{QASC}
%  & Llama2-7b   & 66.198 & 77.535 & 76.053 & 79.589 & 77.637 & 77.696 \\
%  & Llama3-8b   & 86.069 & 77.970 & 73.090 & 80.718 & 74.531 & 75.006 \\
%  & Phi4        & 84.478 & 77.556 & 74.596 & 78.661 & 75.678 & 76.222 \\
%  & Qwen2.5-32b & 88.998 & 79.324 & 73.895 & 78.598 & 74.485 & 75.175 \\
% \midrule
% \multirow{4}{*}{MedQA}
%  & Llama2-7b   & 54.660 & 55.144 & 55.852 & 54.766 & 55.766 & 55.703 \\
%  & Llama3-8b   & 77.493 & 57.384 & 57.894 & 57.919 & 57.592 & 57.530 \\
%  & Phi4        & 86.888 & 65.550 & 64.284 & 63.287 & 65.588 & 65.696 \\
%  & Qwen2.5-32b & 80.131 & 63.264 & 63.712 & 63.109 & 62.564 & 62.164 \\
% \midrule
% \multirow{4}{*}{RACE-m}
%  & Llama2-7b   & 63.965 & 69.194 & 70.819 & 67.568 & 71.823 & 71.984 \\
%  & Llama3-8b   & 82.118 & 67.317 & 70.875 & 69.321 & 69.851 & 70.029 \\
%  & Phi4        & 90.543 & 68.334 & 69.5354 & 68.8049 & 69.025 & 69.188 \\
%  & Qwen2.5-32b & 56.049 & 67.294 & 69.102 & 73.267 & 69.147 & 69.279 \\
% \midrule
% \multirow{4}{*}{RACE-h}
%  & Llama2-7b   & 61.265 & 61.905 & 62.481 & 59.889 & 63.486 & 63.465 \\
%  & Llama3-8b   & 79.466 & 60.775 & 63.868 & 61.253 & 64.134 & 64.146 \\
%  & Phi4        & 87.172 & 62.253 & 62.680 & 60.178 & 63.391 & 63.383 \\
%  & Qwen2.5-32b & 52.811 & 61.837 & 64.047 & 63.555 & 64.050 & 64.024 \\
% \bottomrule
% \end{tabular}%
% }
% \caption{White Box Methods Performance Metrics Across Different Models and Datasets (AUROC)}
% \label{tab:metrics_table}
% \end{table*}
%%%%%%%%%%%%%%%%%%%%%%%%%%%%%%%%%%%%%%%%%%%%%%%%%%%%%%%%%%%%%%%%%%%%%%%%%%%%%%%%%%%%%%%%
%next
%%%%%%%%%%%%%%%%%%%%%%%%%%%%%%%%%%%%%%%%%%%%%%%%%%%%%%%%%%%%%%%%%%%%%%%%%%%%%%%%%%%%%%%%
%%%%%%%%%%%%%%%%%%%%%%%%%%%%%%%%%%%%%%%%%%%%%%%%%%%%%%%%%%%%%%%%%%%%%%%%%%%%%%%%%%%%%%%%

\begin{table*}[t]
\centering
\resizebox{\textwidth}{!}{%
\begin{tabular}{llcccccccccccc}
\toprule
\multirow{2}{*}{\textbf{Dataset}} & \multirow{2}{*}{\textbf{Model}} & \multicolumn{6}{c}{\textbf{RCE}} & \multicolumn{6}{c}{\textbf{Calibration ECE}} \\ \cmidrule(lr){3-8} \cmidrule(lr){9-14}
 &  & Ecc(C) & Deg(C) & Ecc(E) & Deg(E) & Ecc(J) & Deg(J) & Ecc(C) & Deg(C) & Ecc(E) & Deg(E) & Ecc(J) & Deg(J) \\ \midrule
\multirow{4}{*}{C-QA} 
 & Llama2-7b    & 0.2857  & 0.143722 & 0.117486 & 0.084357 & 0.271789 & 0.198744 & 0.014457 & 0.064792 & 0.025161 & 0.009014 & 0.009546 & 0.031801 \\
 & Llama3-8b    & 0.28071 & 0.15255  & 0.06311  & 0.041246 & 0.362527 & 0.153761 & 0.013865 & 0.044074 & 0.031566 & 0.016865 & 0.008845 & 0.060919 \\
 & Phi4         & 0.18881 & 0.115068 & 0.067507 & 0.038771 & 0.225698 & 0.218135 & 0.017734 & 0.059135 & 0.040364 & 0.024237 & 0.019987 & 0.056875 \\
 & Qwen2.5-32b  & 0.16192 & 0.114378 & 0.080021 & 0.055613 & 0.278165 & 0.198222 & 0.0111   & 0.087857 & 0.043406 & 0.016647 & 0.014439 & 0.051092 \\%[1ex]
  \midrule
\multirow{4}{*}{QASC} 
 & Llama2-7b    & 0.25132 & 0.162559 & 0.193186 & 0.121908 & 0.331258 & 0.252667 & 0.013984 & 0.020481 & 0.019263 & 0.012321 & 0.003108 & 0.022164 \\
 & Llama3-8b    & 0.28697 & 0.231308 & 0.083146 & 0.057512 & 0.401264 & 0.230094 & 0.003117 & 0.005336 & 0.004844 & 0.009398 & 0.010951 & 0.022145 \\
 & Phi4         & 0.19064 & 0.104986 & 0.066258 & 0.063753 & 0.23061  & 0.225091 & 0.004181 & 0.015734 & 0.012447 & 0.01108  & 0.003271 & 0.026654 \\
 & Qwen2.5-32b  & 0.25004 & 0.142512 & 0.091264 & 0.084393 & 0.31938  & 0.272657 & 0.010503 & 0.020774 & 0.012144 & 0.009716 & 0.004127 & 0.023387 \\%[1ex]
  \midrule
\multirow{4}{*}{MedQA} 
 & Llama2-7b    & 0.19817 & 0.188788 & 0.231296 & 0.243174 & 0.263178 & 0.213793 & 0.005909 & 0.006271 & 0.006057 & 0.01008  & 0.007157 & 0.008915 \\
 & Llama3-8b    & 0.21067 & 0.190038 & 0.286932 & 0.194414 & 0.290058 & 0.146904 & 0.006035 & 0.006757 & 0.006424 & 0.006872 & 0.01166  & 0.007277 \\
 & phi4         & 0.09127 & 0.09877  & 0.208792 & 0.132527 & 0.308812 & 0.087518 & 0.008327 & 0.018021 & 0.0156   & 0.008231 & 0.020912 & 0.016443 \\
 & Qwen2.5-32b  & 0.09064 & 0.089393 & 0.194414 & 0.087518 & 0.234422 & 0.118149 & 0.006312 & 0.01598  & 0.011337 & 0.021417 & 0.014092 & 0.021119 \\%[1ex]
  \midrule
\multirow{4}{*}{RACE-m} 
 & Llama2-7b  & 0.09876 & 0.31881 & 0.17315 & 0.27630 & 0.14502 & 0.16065 & 0.04523 & 0.07009 & 0.01980 & 0.01965  & 0.00778 & 0.01433 \\
 & Llama3-8b  & 0.10252 & 0.32068 & 0.12877 & 0.27005 & 0.21254 & 0.04500 & 0.00939 & 0.08513 & 0.00962 & 0.04675 & 0.025705  & 0.03261 \\
 & phi4          & 0.06001 & 0.31756  & 0.11252 & 0.26817 & 0.150655 & 0.07501 & 0.01699 & 0.07599 & 0.03366   & 0.01936 & 0.016385 &  0.01542 \\
 & Qwen2.5-32b  & 0.19378 &  0.32756 &  0.18253 & 0.27505 & 0.09689 & 0.1187 & 0.024623 & 0.10445  & 0.02922 & 0.05540 & 0.01300 & 0.02171 \\%[1ex]
  \midrule
\multirow{4}{*}{RACE-h} 
 & Llama2-7b  & 0.12565 & 0.36069 & 0.22441 & 0.40383 & 0.29568 & 0.30881 & 0.01702 & 0.06116 & 0.01635 & 0.01577  & 0.020679 & 0.01569 \\
 & Llama3-8b   & 0.20316 & 0.37007 & 0.18816 & 0.42070 & 0.26192 & 0.05938 & 0.01754 & 0.06838 & 0.01672 & 0.02324 & 0.02597  & 0.02622\\
 & phi4        & 0.09751 & 0.36757  & 0.14627 & 0.38820 &  0.26880 & 0.15878 & 0.01928 & 0.06393 & 0.021709   & 0.02191 & 0.02294 & 0.02502 \\
 & Qwen2.5-32b  & 0.11564 & 0.35069 & 0.21441 & 0.35569 & 0.28505 & 0.205666 & 0.01679 & 0.06562  & 0.01794 &0.02833 & 0.015137 & 0.01438 \\[1ex]
\bottomrule
\end{tabular}%
}
\caption{RCE and (calibrated) ECE for black-box methods, across different models and datasets}
\label{appendix:tab:bb:calib}
\end{table*}




\begin{table*}[t]
\centering
\resizebox{\textwidth}{!}{%
\begin{tabular}{llcccccccccccc}
\toprule
\multirow{2}{*}{\textbf{Dataset}} & \multirow{2}{*}{\textbf{Model}} & \multicolumn{6}{c}{\textbf{RCE}} & \multicolumn{6}{c}{\textbf{Calibration ECE}} \\ \cmidrule(lr){3-8} \cmidrule(lr){9-14}
 &  & P(true) & CSL & CSL-next & SL & SL(norm) & TokenSAR & P(true) & CSL & CSL-next & SL & SL(norm) & TokenSAR \\ \midrule
\multirow{4}{*}{C-QA} 
 & Llama2-7b    & 0.084386 & 0.0506   & 0.041895  & 0.041267     & 0.038126 & 0.034997 & 0.0102   & 0.035637  & 0.041958  & 0.023881        & 0.04454 & 0.027278 \\
 & Llama3-8b    & 0.040614 & 0.03563  & 0.068102  & 0.031902     & 0.057489 & 0.038742 & 0.01871  & 0.034739  & 0.050008  & 0.022294        & 0.04291 & 0.026352 \\
 & Phi4         & 0.043731 & 0.04626  & 0.046892  & 0.043771     & 0.041858 & 0.03501  & 0.0583   & 0.034232  & 0.055943  & 0.019535        & 0.04302 & 0.030655 \\
 & Qwen2.5-32b  & 0.058105 & 0.02999  & 0.044359  & 0.032513     & 0.044363 & 0.059406 & 0.0369   & 0.022175  & 0.046935  & 0.021905        & 0.03671 & 0.021438 \\%[1ex]
  \midrule
\multirow{4}{*}{QASC} 
 & Llama2-7b    & 0.077448 & 0.04685  & 0.078796  & 0.051258     & 0.043136 & 0.045007 & 0.01119  & 0.024505  & 0.037871  & 0.023245        & 0.0326  & 0.023127 \\
 & Llama3-8b    & 0.030627 & 0.04811  & 0.117522  & 0.050664     & 0.08503  & 0.043753 & 0.00894  & 0.020665  & 0.038958  & 0.025687        & 0.03274 & 0.020785 \\
 & Phi4         & 0.082518 & 0.04437  & 0.116905  & 0.066942     & 0.088115 & 0.049376 & 0.02122  & 0.021401  & 0.0415    & 0.028242        & 0.02548 & 0.033083 \\
 & Qwen2.5-32b  & 0.11997  & 0.04878  & 0.062505  & 0.081237     & 0.073773 & 0.041861 & 0.03096  & 0.014358  & 0.040047  & 0.025111        & 0.02665 & 0.023483 \\%[1ex]
   \midrule
\multirow{4}{*}{MedQA} 
 & Llama2-7b    & 0.181911 & 0.19254  & 0.19879   & 0.191288     & 0.228796 & 0.238798 & 0.00606  & 0.015623  & 0.007533  & 0.00791         & 0.00669 & 0.007449 \\
 & Llama3-8b    & 0.028131 & 0.08939  & 0.121274  & 0.207542     & 0.163158 & 0.178161 & 0.0166   & 0.012721  & 0.008949  & 0.03            & 0.00861 & 0.010613 \\
 & phi4         & 0.05126  & 0.09127  & 0.115648  & 0.176285     & 0.119399 & 0.116273 & 0.02853  & 0.046391  & 0.05184   & 0.058272        & 0.05787 & 0.05535  \\
 & Qwen2.5-32b  & 0.078141 & 0.06126  & 0.07314   & 0.128151     & 0.088143 & 0.075015 & 0.03067  & 0.020881  & 0.033491  & 0.047763        & 0.03295 & 0.032673 \\%[1ex]
   \midrule
\multirow{4}{*}{RACE-m} 
 & Llama2-7b  & 0.16253 & 0.26130 & 0.22254 & 0.13502     & 0.24317 & 0.24567 & 0.00741 & 0.01935 & 0.03113 & 0.01820  & 0.061396 & 0.062452 \\
 & Llama3-8b    & 0.05938 & 0.18003 & 0.09814 & 0.12752     & 0.10252 & 0.12189 & 0.05006 & 0.05534 & 0.04303 & 0.04812        & 0.01986  & 0.02156 \\
 & phi4        & 0.09689 & 0.15753  & 0.09314 & 0.09689     & 0.13127 & 0.13565 & 0.02585 & 0.04808 &0.02775  & 0.032335        & 0.01727 & 0.01938 \\
 & Qwen2.5-32b  & 0.16940 & 0.17566 & 0.17691 & 0.17691     & 0.24567 & 0.25255 &0.00695& 0.04720  & 0.05091 &  0.07564        & 0.07986 &  0.08066 \\%[1ex]
   \midrule
\multirow{4}{*}{RACE-h} 
 & Llama2-7b & 0.17566 & 0.31818 & 0.32318 & 0.33256     & 0.31818 & 0.32256 & 0.01748 & 0.02600 & 0.01719 & 0.01613         & 0.021382 & 0.021339 \\
 & Llama3-8b   & 0.05563 & 0.22316 & 0.12189 & 0.15565     & 0.163782 & 0.149404 & 0.045399 & 0.01684 & 0.031577 & 0.034098        & 0.030134  & 0.030341 \\
 & phi4       & 0.08939 & 0.19566  & 0.150030 & 0.13315     & 0.19316 & 0.19566 & 0.019294 & 0.035576 &  0.02874   & 0.03037        & 0.02238 & 0.040637 \\
 & Qwen2.5-32b    & 0.24754 & 0.22254 & 0.21754 & 0.21316     & 0.29505 & 0.30006 & 0.016826 & 0.02004  &0.02105 & 0.022801        & 0.03156 & 0.04110 \\
\bottomrule
\end{tabular}%
}
\caption{RCE and (calibrated) ECE for white-box methods, across different models and datasets}
\label{appendix:tab:wb:calib}
\end{table*}







\subsection{Additional Visualizations for ROC Curves}
\cref{appendix:fig:ROC} presents the ROC curves for \phiName.
\baselinePTrue achieves much better performance than other confidence measures on the more challenging datasets, likely because \phiName is a relatively advanced model.
On the easier datasets, where we could observe a bigger performance gap between different confidence measures, it is also interesting to see that the general shapes (and rankings at different FPR) are relatively consistent across C-QA and QASC, suggesting stability of \uqeval.



\def \FigAUROCVisHorizontalBar{
\begin{figure}[t]
  \centering
  \begin{subfigure}[b]{\columnwidth}
    \centering
    \includegraphics[width=\columnwidth]{figures/qasc_blackbox.png}
    \caption{AUROC of different black-box methods.}
    \label{fig:cqa_blackbox}
  \end{subfigure}
  \begin{subfigure}[b]{\columnwidth}
    \centering
    \includegraphics[width=\columnwidth]{figures/qasc_whitebox.png}
    \caption{AUROC of different white-box methods.}
    \label{fig:another_dataset}
  \end{subfigure}

  \caption{(a) and (b) show the performance of 4 different LLMs and 12 different confidence estimation methods on the QASC dataset. A higher AUROC indicates better performance.}
  \label{fig:llm_perspective}
\end{figure}

\begin{figure}[t]
  \centering
  \begin{subfigure}[b]{\columnwidth}
    \centering
    \includegraphics[width=\columnwidth]{figures/medqa_blackbox.png}
    \caption{AUROC of different black-box methods.}
    \label{fig:cqa_blackbox}
  \end{subfigure}
  \begin{subfigure}[b]{\columnwidth}
    \centering
    \includegraphics[width=\columnwidth]{figures/medqa_whitebox.png}
    \caption{AUROC of different white-box methods.}
    \label{fig:another_dataset}
  \end{subfigure}

  \caption{(a) and (b) show the performance of 4 different LLMs and 12 different confidence estimation methods on the MedQA dataset. A higher AUROC indicates a better performance.}
  \label{fig:llm_perspective}
\end{figure}
}
%/FigAUROCVisHorizontalBar


% \begin{figure*}
%     \centering
%     \includegraphics[width=0.99\linewidth]{figures/comparison_qwen.pdf}
%     \caption{The comparison of different evaluation metrics using our method to quantify Qwen-2.5-32b model's uncertainty on datasets: RACE-h (harder) and RACE-m (easier).}
%     \label{fig:compareQwen}
% \end{figure*}

% \begin{figure*}
%     \centering
%     \includegraphics[width=0.99\linewidth]{figures/comparison_llama7b.pdf}
%     \caption{The comparison of different evaluation metrics using our method to quantify Llama2-7b model's uncertainty on datasets: RACE-h (harder) and RACE-m (easier).}
%     \label{fig:compareLlama2}
% \end{figure*}

% \begin{figure*}
%     \centering
%     \includegraphics[width=0.99\linewidth]{figures/comparison_llama8b.pdf}
%     \caption{The comparison of different evaluation metrics using our method to quantify Llama3-8b model's uncertainty on datasets: RACE-h (harder) and RACE-m (easier).}
%     \label{fig:compareLlama3}
% \end{figure*}

\begin{figure*}[htbp]
    \centering
    % Row 1: Two subfigures side by side
    \begin{subfigure}[b]{0.45\textwidth}
        \centering
        \includegraphics[width=\textwidth]{figures/c-qa.png}
        \caption{C-QA Dataset}
        \label{fig:subfig1}
    \end{subfigure}
    \hfill
    \begin{subfigure}[b]{0.45\textwidth}
        \centering
        \includegraphics[width=\textwidth]{figures/qasc.png}
        \caption{QASC Dataset}
        \label{fig:subfig2}
    \end{subfigure}

    % Row 2: Two subfigures side by side
    \begin{subfigure}[b]{0.45\textwidth}
        \centering
        \includegraphics[width=\textwidth]{figures/extra/phi4_cqa_race_m_10_update1.png}
        \caption{RACE-m Dataset}
        \label{fig:subfig3}
    \end{subfigure}
    \hfill
    \begin{subfigure}[b]{0.45\textwidth}
        \centering
        \includegraphics[width=\textwidth]{figures/extra/phi4_cqa_race_h_10_update1.png}
        \caption{RACE-h Dataset}
        \label{fig:subfig4}
    \end{subfigure}

    % Row 3: One subfigure occupying most of the width
    \begin{subfigure}[b]{0.45\textwidth}
        \centering
        \includegraphics[width=\textwidth]{figures/medqa.png}
        \caption{MedQA Dataset}
        \label{fig:subfig5}
    \end{subfigure}
    
    \caption{Comparison of different evaluation metrics using our method to quantify the \phiName model's confidence scores across five datasets (C-QA, QASC, RACE-m, RACE-h, MedQA), with increasing difficulty.}
    \label{appendix:fig:ROC}
\end{figure*}

% \section{AI Assistant Usage}

% We used GPT for grammar checking and Copilot as an assistive tool.
%%%%%%%%%%%%%%%%%%%%%%%%%%%%%%%%%%%%%%%%%%%%%%%%%%%%%%%%%%%%%%%%%%%%%%%%%%%%%%%
%%%%%%%%%%%%%%%%%%%%%%%%%%%%%%%%%%%%%%%%%%%%%%%%%%%%%%%%%%%%%%%%%%%%%%%%%%%%%%%


\end{document}


% This document was modified from the file originally made available by
% Pat Langley and Andrea Danyluk for ICML-2K. This version was created
% by Iain Murray in 2018, and modified by Alexandre Bouchard in
% 2019 and 2021 and by Csaba Szepesvari, Gang Niu and Sivan Sabato in 2022.
% Modified again in 2023 and 2024 by Sivan Sabato and Jonathan Scarlett.
% Previous contributors include Dan Roy, Lise Getoor and Tobias
% Scheffer, which was slightly modified from the 2010 version by
% Thorsten Joachims & Johannes Fuernkranz, slightly modified from the
% 2009 version by Kiri Wagstaff and Sam Roweis's 2008 version, which is
% slightly modified from Prasad Tadepalli's 2007 version which is a
% lightly changed version of the previous year's version by Andrew
% Moore, which was in turn edited from those of Kristian Kersting and
% Codrina Lauth. Alex Smola contributed to the algorithmic style files.
