%%%%%%%% ICML 2025 EXAMPLE LATEX SUBMISSION FILE %%%%%%%%%%%%%%%%%

\documentclass{article}

% Recommended, but optional, packages for figures and better typesetting:
\usepackage{microtype}
\usepackage{graphicx}
\usepackage{subfigure}
\usepackage{booktabs} % for professional tables

\usepackage{bbm}

\usepackage{wrapfig}

% hyperref makes hyperlinks in the resulting PDF.
% If your build breaks (sometimes temporarily if a hyperlink spans a page)
% please comment out the following usepackage line and replace
% \usepackage{icml2024} with \usepackage[nohyperref]{icml2024} above.
\usepackage{hyperref}


% Attempt to make hyperref and algorithmic work together better:
\newcommand{\theHalgorithm}{\arabic{algorithm}}



% Use the following line for the initial blind version submitted for review:
% \usepackage{icml2024}

% If accepted, instead use the following line for the camera-ready submission:
\usepackage[accepted]{icml2025}

% For theorems and such
\usepackage{amsmath}
\usepackage{amssymb}
\usepackage{mathtools}
\usepackage{amsthm}
\usepackage{multirow}
\usepackage{multicol}
\usepackage[utf8]{inputenc}
\usepackage{tcolorbox}
\usepackage{float}

% if you use cleveref..
\usepackage[capitalize,noabbrev]{cleveref}

%%%%%%%%%%%%%%%%%%%%%%%%%%%%%%%%
% THEOREMS
%%%%%%%%%%%%%%%%%%%%%%%%%%%%%%%%
\theoremstyle{plain}
\newtheorem{theorem}{Theorem}[section]
\newtheorem{proposition}[theorem]{Proposition}
\newtheorem{lemma}[theorem]{Lemma}
\newtheorem{corollary}[theorem]{Corollary}
\theoremstyle{definition}
\newtheorem{definition}[theorem]{Definition}
\newtheorem{assumption}[theorem]{Assumption}
\theoremstyle{remark}
\newtheorem{remark}[theorem]{Remark}


\newcommand{\wei}[1]{\textcolor{orange}{\#Wei:~#1\#}}
\newcommand{\juntong}[1]{\textcolor{green}{\#Juntong:~#1\#}}



\newcommand{\method}{\textsc{TimeDistill}}

% Todonotes is useful during development; simply uncomment the next line
%    and comment out the line below the next line to turn off comments
%\usepackage[disable,textsize=tiny]{todonotes}
\usepackage[textsize=tiny]{todonotes}

\usepackage{paralist}

% The \icmltitle you define below is probably too long as a header.
% Therefore, a short form for the running title is supplied here:
\icmltitlerunning{TimeDistill: Efficient Long-Term Time Series Forecasting with MLP via  Cross-Architecture Distillation}

\begin{document}

\twocolumn[
\icmltitle{TimeDistill: Efficient Long-Term Time Series Forecasting with MLP \\ via  Cross-Architecture Distillation}

% It is OKAY to include author information, even for blind
% submissions: the style file will automatically remove it for you
% unless you've provided the [accepted] option to the icml2024
% package.

% List of affiliations: The first argument should be a (short)
% identifier you will use later to specify author affiliations
% Academic affiliations should list Department, University, City, Region, Country
% Industry affiliations should list Company, City, Region, Country

% You can specify symbols, otherwise they are numbered in order.
% Ideally, you should not use this facility. Affiliations will be numbered
% in order of appearance and this is the preferred way.
% \icmlsetsymbol{equal}{*}

\begin{icmlauthorlist}
\icmlauthor{Juntong Ni}{Emory}
\icmlauthor{Zewen Liu}{Emory}
\icmlauthor{Shiyu Wang}{}
\icmlauthor{Ming Jin}{Griffith}
\icmlauthor{Wei Jin}{Emory}
% \icmlauthor{Firstname3 Lastname3}{comp}
% \icmlauthor{Firstname4 Lastname4}{sch}
% \icmlauthor{Firstname5 Lastname5}{yyy}
% \icmlauthor{Firstname6 Lastname6}{sch,yyy,comp}
% \icmlauthor{Firstname7 Lastname7}{comp}
%\icmlauthor{}{sch}
% \icmlauthor{Firstname8 Lastname8}{sch}
% \icmlauthor{Firstname8 Lastname8}{yyy,comp}
%\icmlauthor{}{sch}
%\icmlauthor{}{sch}
\end{icmlauthorlist}

\icmlaffiliation{Emory}{Department of Computer Science, Emory University, Atlanta, United States}
\icmlaffiliation{Griffith}{School of Information and Communication Technology, Griffith University, Nathan, Australia}
% \icmlaffiliation{comp}{Company Name, Location, Country}
% \icmlaffiliation{sch}{School of ZZZ, Institute of WWW, Location, Country}

\icmlcorrespondingauthor{Wei Jin}{wei.jin@emory.edu}

% You may provide any keywords that you
% find helpful for describing your paper; these are used to populate
% the "keywords" metadata in the PDF but will not be shown in the document
\icmlkeywords{Machine Learning, ICML}

\vskip 0.3in
]
% this must go after the closing bracket ] following \twocolumn[ ...

% This command actually creates the footnote in the first column
% listing the affiliations and the copyright notice.
% The command takes one argument, which is text to display at the start of the footnote.
% The \icmlEqualContribution command is standard text for equal contribution.
% Remove it (just {}) if you do not need this facility.
\printAffiliationsAndNotice{}  % leave blank if no need to mention equal contribution
% \printAffiliationsAndNotice{\icmlEqualContribution} % otherwise use the standard text.



\begin{abstract}
Transformer-based and CNN-based methods demonstrate strong performance in long-term time series forecasting. However, their high computational and storage requirements can hinder large-scale deployment. To address this limitation, we propose integrating lightweight MLP with advanced architectures using knowledge distillation (\textit{KD}). Our preliminary study reveals different models can capture complementary patterns, particularly multi-scale and multi-period patterns in the temporal and frequency domains.
Based on this observation, we introduce \method{}, a cross-architecture \textit{KD} framework that transfers these patterns from teacher models (e.g., Transformers, CNNs) to MLP. Additionally, we provide a theoretical analysis, demonstrating that our \textit{KD} approach can be interpreted as a specialized form of \textit{mixup} data augmentation.
\method{} improves MLP performance by up to 18.6\%, surpassing teacher models on eight datasets. It also achieves up to 7$\times$ faster inference and requires 130$\times$ fewer parameters. Furthermore, we conduct extensive evaluations to highlight the versatility and effectiveness of \method{}.

\end{abstract}

% \begin{wrapfigure}{R}{0.26\textwidth}
%     \hspace{-1em}
%     \vskip -1.3em
%     \begin{minipage}{\linewidth}
%         \centering
%         \includegraphics[width=1\linewidth]{figures/result_radar.pdf}
%         \vskip -1.8em
% % \captionsetup{size=footnotesize}
% \caption{Performance comparison.}  %Average results (MSE) are reported following iTransformer~\cite{itransformer}.}
% \label{fig:result_rader}
% \vspace{-2em}
% \end{minipage}
% \end{wrapfigure}



\section{Introduction}\label{sec:introduction}
% -- Outline
% ---- LLMs are popular
% ---- There're many stakeholders in the training and inference loop
% ---- Adversaries in the training loop are a problem -- malpractice, poisoning
% ---- Also, showing compliance
% ---- Need a framework to prove the integrity of the pipeline
% ---- Enter Atlas

% ---- LLMs are popular
In recent years, machine learning (ML) models, have become increasingly popular.
The pervasive use of large language models (LLMs), in particular, and multi-stakeholder
involvement in model creation and deployment exacerbate security and privacy risks.
These considerations are emphasized by the global nature and the complexity of
large-scale ML deployments with different lifecycle stages:
%gathering and sanitizing the data from different sources,
%training and inferencing across many data centers,
%compliance with local laws or corporate policies.

% ---- There're many stakeholders in the training and inference loop
%Additionally, different stages of the ML development pipeline come with their own stakeholders:
\begin{enumerate}[label=\arabic*)]
    \item Collection and sanitation of a \emph{training} dataset from several public and proprietary sources.
    %\item Solicitation and facilitation of training.
    \item Provisioning of the training environment (hardware and software).
    \item Execution of training across many data centers.
    \item Construction of a \emph{testing} dataset from several sources, and the evaluation.
    \item Deployment and use of the model for inference that is compliant with local laws or corporate policies.
    %\item Use of the model in compliance with local laws or corporate policies.
\end{enumerate}

% ---- Adversaries in the training loop are a problem -- malpractice, poisoning
Each of these stages is vulnerable to malicious or dishonest parties.
For example, data can be poisoned~\cite{biggio2012poisoning,carlini2024poisoning} during collection or training.
Service providers executing outsourced training can shorten or omit critical steps to reduce their cost.
Model providers can serve smaller models in SaaS, or even distribute malicious ones.

% ---- Also, showing compliance
On the other hand, responsible model builders and other stakeholders may be incentivised or required to provide security and trust guarantees.
They may want to prove low bias in their training data, offer easily verifiable performance claims, or guarantee end-to-end integrity of the model creation in high risk domains.

% ---- Need a framework to prove the integrity of the pipeline
To address these challenges, it is necessary to guarantee the integrity of the entire ML lifecycle --
beginning with the data, through the training, and finally, the evaluation and deployment.
Was the data modified?
Did the hardware and software environment adhere to the specification?
Did the contractor follow the specified training procedure?
Can I trust the evaluation?
How can I guarantee that I am interacting with the intended model?
These are example questions that showcase the breadth of the involved challenges that must be tackled to provide end-to-end security.

% --- Enter Atlas
In this work, we introduce \atlas, a framework for enhancing the security and transparency of the lifecycle of ML models.
\atlas establishes the baseline of fundamental components and capabilities needed for comprehensive provenance tracking
at each stage of the ML lifecycle.
Subsequently, \atlas defines the core integrity requirements for verifiable ML lifecycle transparency.
We provide a reference implementation that instantiates \atlas using hardware-based security mechanisms -- with trusted execution environment (TEE),
including attestations.% , and comprehensive metadata-based provenance tracking.
%Our implementation satisfies all \atlas requirements.

We claim the following contributions:
\begin{enumerate}[label=\arabic*.]\label{sec:introduction:contributions}
    \item We introduce \atlas, a framework designed for end-to-end ML lifecycle transparency.
    \item We instantiate \atlas using TEEs and metadata-based provenance tracking.
    \item We evaluate our \atlas prototype through two case studies:
        \begin{enumerate*}[label=\arabic*)]
            \item fine-tuning of a BERT model~\cite{lin2023metabert, lin2023metabertimpl};
            \item fine-tuning of a bge-reranker model~\cite{chen2023bge}
        \end{enumerate*}.
\end{enumerate}

%\msm{revise: Integrate this motivation into intro}
%Organizations frequently leverage pre-trained models, outsource training processes, and integrate components from multiple sources,
%making it difficult to verify the authenticity and trustworthiness of their ML systems. This complexity is further compounded
%by the potential for malicious modifications at various stages of the model lifecycle, from data preparation through deployment.
%The involvement of various third parties in ML model development and deployment
%creates critical challenges in ensuring supply chain integrity.
%
%While Software Bills of Materials (SBOMs) and AI Bills of Materials (AI BOMs) provide basic inventory tracking for model components,
%they fall short in addressing the dynamic nature of ML pipelines. These approaches typically offer point-in-time snapshots but
%fail to capture the complex transformations, fine-tuning operations, and runtime modifications that characterize modern ML workflows.
%Additionally, they lack cryptographic guarantees about the integrity of recorded information and cannot effectively track the provenance
% of model weights and training data.
%
% These approaches demonstrate the growing importance of ML supply chain security.
% However, they are typically applied in an ad-hoc fashion, highlighting the need
% for a more integrated approach that combines comprehensive lineage tracking,
% strong cryptographic properties, and practical integration capabilities with existing ML development and deployment pipelines.
%
%A comprehensive solution requires not just documentation of components, but verifiable evidence of their origins,
%transformations, and integrity throughout the entire model lifecycle. This need has driven interest in more robust
%provenance tracking mechanisms that can:
%
%\begin{itemize}
%\item Provide cryptographic proof of model lineage
%\item Track and verify all pipeline transformations
%\item Maintain tamper-evident records of training processes
%\item Ensure integrity of model artifacts across organizational boundaries
%\end{itemize}
%
%Several existing tools and frameworks
%commonly focusing on different components of the model lifecycle and provenance tracking.
%While these solutions offer valuable capabilities, they often address only specific parts of the end-to-end ML
%supply chain rather than providing comprehensive coverage.
%\msm{end-revise}
%
%\todo{add discussion of EU-CRA AI Act requirements for model documentation and FDA guidelines for AI/ML in healthcare}

%The remainder of this paper is organized as follows:
%in Section~\ref{sec:background-related} we provide an overview of the necessary background, and the related work;
%Section~\ref{sec:problem} presents the challenge of providing integrity in the ML pipeline, the threat model, and the system assumptions;
%in Section~\ref{sec:framework} we present \atlas -- our framework for providing ML integrity;
%Section~\ref{sec:implementation} covers implementation details;
%in Section~\ref{sec:eval}, we show that \atlas is effective across three dimensions: training overhead $<8\%$, the verification time increases linearly with the size of the model, and it is compatible with PyTorch and Tensorflow;
%in Section~\ref{sec:casestudies} we present the case studies;
%in Section~\ref{sec:discussion} we discuss additional considerations for \atlas,
%and Section~\ref{sec:conclusion} concludes the paper and provides directions for future work.

% Reward hacking is a well-known issue in reinforcement learning, affecting both traditional RL and RLHF in LLMs~\cite{weng2024rewardhack}.
\subsection{Reward Hacking in Traditional RL}  
Reward hacking arises when an RL agent exploits flaws or ambiguities in the reward function to achieve high rewards without performing the intended task~\cite{weng2024rewardhack}. This aligns with Goodhart’s Law: \emph{When a measure becomes a target, it ceases to be a good measure.} For example: 
A bicycle agent rewarded for not falling and moving toward a goal (but not penalized for moving away) learns to circle the goal indefinitely~\cite{Randlv1998LearningTD}.  
A walking agent in the DMControl suite, rewarded for matching a target speed, learns to walk unnaturally using only one leg~\cite{lee2021pebblefeedbackefficientinteractivereinforcement}.  
An RL agent allowed to modify its body grows excessively long legs to fall forward and reach the goal~\cite{Ha2018designrl}.  
In the Elevator Action ALE game, the agent repeatedly kills the first enemy on the first floor to accumulate small rewards~\cite{toromanoff2019deepreinforcementlearningreally}.  
% A robot trained to stay on track learns to reverse along straight paths by alternating left and right turns instead of following curves~\cite{Vamplew2004}.

\citet{amodei2016concrete} propose several potential mitigation strategies to address reward hacking, including
\emph{(1) Adversarial Reward Functions}: Treating the reward function as an adaptive agent capable of responding to new strategies where the model achieves high rewards but receives low human ratings.
\emph{(2) Model Lookahead}: Assigning rewards based on anticipated future states; for example, penalizing the agent with negative rewards if it attempts to modify the reward function~\cite{everitt2016selfmodificationpolicyutilityfunction}.
\emph{(3) Adversarial Blinding}: Restricting the model’s access to specific variables to prevent it from learning information that could facilitate reward hacking~\cite{ajakan2015domainadversarialneuralnetworks}.
\emph{(4) Careful Engineering}: Designing systems to avoid certain types of reward hacking by isolating the agent’s actions from its reward signals, such as through sandboxing techniques~\cite{The_AGI_Containment_Problem}.
\emph{(5) Trip Wires}: Deliberately introducing vulnerabilities into the system and setting up monitoring mechanisms to detect and alert when reward hacking occurs.

\subsection{Reward Hacking in RLHF of LLMs}  
Reward hacking in RLHF for large language models has been extensively studied. \citet{gao2023scaling} systematically investigate the scaling laws of reward hacking in small models, while \citet{wen2024languagemodelslearnmislead} demonstrate that language models can learn to mislead humans through RLHF. Beyond exploiting the training process, reward hacking can also target evaluators. Although using LLMs as judges is a natural choice given their increasing capabilities, this approach is imperfect and can introduce biases. For instance, LLMs may favor their own responses when evaluating outputs from different model families~\cite{liu2024llmsnarcissisticevaluatorsego} or exhibit positional bias when assessing responses in sequence~\cite{wang2023largelanguagemodelsfair}.  

To mitigate reward hacking, several methods have been proposed. Reward ensemble techniques have shown promise in addressing this issue~\cite{Eisenstein2023HelpingOH, Rame2024WARMOT, ahmed2024scalableensemblingmitigatingreward, coste2023reward, zhang2024improvingreinforcementlearninghuman}, and shaping methods have also proven straightforward and effective~\cite{yang2024regularizinghiddenstatesenables, jinnai2024regularizedbestofnsamplingmitigate}. \citet{miao2024informmitigatingrewardhacking} introduce an information bottleneck to filter irrelevant noise, while \citet{moskovitz2023confrontingrewardmodeloveroptimization} employ constrained RLHF to prevent reward over-optimization. \citet{Chen2024ODINDR} propose the ODIN method, which uses a linear layer to separately output quality and length rewards, reducing their correlation through an orthogonal loss function. Similarly,
~\citet{sun2023salmon} train instructable reward models to give a more comprehensive reward signal from multiple objectives. \citet{Dai2023SafeRS} constrain reward magnitudes using regularization terms. ~\citet{liu2024rrmrobustrewardmodel} curate diverse pairwise training data. Additionally, post-processing techniques have been explored, such as the log-sigmoid centering transformation introduced by \citet{Wang2024TransformingAC}.  



\section{Confidence Estimation for NLG}\label{sec:prelim}

% \paragraph{Problem Definition}
First, we establish notation and introduce relevant definitions.
Let $\mathcal{M}$ be a language model, $\xInput\in\Sigma^*$ be an input prompt, and $\predSeq=\mathcal{M}(\xInput)\in\Sigma^*$ be the output. 
$\Sigma$ denotes the vocabulary, which includes tokens from modern tokenizers or natural language symbols like alphabet letters. 
For free-form NLG datasets, we typically have reference answers $A={a_1,\ldots,a_m}$ alongside $\xInput$. A \textit{confidence estimation method} is a function that assigns a confidence score to model output $\predSeq$ given input $\xInput$. 
Formally, a confidence measure is defined as:
\begin{equation}
    C_{\mathcal{M}}: (\xInput, \predSeq) \in \Sigma^*\times\Sigma^*  \mapsto \mathbb{R},
\end{equation}
where $C_{\mathcal{M}}(\xInput,\predSeq)$ represents the confidence score of $\predSeq$. This notation accounts for both model-agnostic and model-specific confidence measures.


%First, we fix notations and introduce relevant definitions. 
%% We will focus on auto-regressive LMs, the current dominant paradigm.
%Let $\mathcal{M}$ be a language model, $\xInput\in\Sigma^*$ be an input prompt, and $\predSeq=\mathcal{M}(\xInput)\in\Sigma^*$ be the output.
%Here, $\Sigma$ denotes the vocabulary, which could be all tokens of a modern tokenizer, or natural language symbols such as the alphabet letters. 
%For most free-form NLG datasets, apart from the $\xInput$, we typically also have a set of reference answers, denoted as $A=\{a_1,\ldots,a_m\}$.

%A \textit{confidence estimation method} is a function that assigns a confidence score to a model output $\predSeq$, with respect to a given input $\xInput$. 
%Formally, a confidence measure id defined as a mapping:
%\begin{equation}
%    C_{\mathcal{M}}: (\xInput, \predSeq) \in \Sigma^*\times\Sigma^*  \mapsto \mathbb{R},
%\end{equation}
%where $C_{\mathcal{M}}(\xInput,\predSeq)$ represents the confidence score of $\predSeq$. Note that such notation allows for both model-agnostic confidence measures, as well as model-specific ones.
% We may write $C_M($



%, is a mapping of any $\mathcal{M}$, input $\xInput$ and $\predSeq$ to a real number.




\subsection{Confidence Estimation Methods}
\label{confidence_methods}

% At the center of \cref{fig:pipeline} lie the confidence estimation methods to be evaluated. 
Existing confidence estimation methods can be broadly divided into two categories: Consistency-based black-box methods and internal state-based white-box methods\footnote{We consider logits as an internal states here.}.


%At the center of \cref{fig:pipeline} sits the confidence estimation methods, which are the candidates to be evaluated.  
%Current confidence estimation methods can be broadly categorized into two groups: Consistency-based black-box methods and internal-state-based white-box methods\footnote{Here, we consider logits as an internal-state.}.
% The overall pipeline of the confidence estimation is illustrated in \cref{fig:pipeline}, where the yellow arrows depict the current workflow. The process begins with multiple-choice generation by different LLMs. 
% Both black-box and white-box methods are then applied to compute the confidence score for each response of interest:
% Simultaneously, correctness labels are assigned using reference matching or human evaluation. 
% Specifically:

\textbf{Black-Box Methods} leverage response consistency across LLM generations~\cite{lin2024generating,manakul-etal-2023-selfcheckgpt}. 
Higher consistency among generated responses indicates higher confidence in $\predSeq$. 
These methods first compute pairwise response similarities, then derive confidence from the similarity matrix.
% Implementation involves computing pairwise response similarities, then deriving confidence from the similarity matrix. 
For similarity computation, existing methods use
% Methods employ 
Jaccard similarity, NLI models~\cite{he2021deberta}, and BERTScore~\cite{Zhang2020BERTScore} for similarity computation.


%\textbf{Black-Box confidence estimation methods} rely on the consistency of LLM-generated responses~\cite{lin2024generating,manakul-etal-2023-selfcheckgpt}.
%The intuition is that if a response $\predSeq$ is more consistent with other generated responses, it is likely to have a higher confidence score. 
%This is achieved through pairwise similarity measurement among responses, followed by confidence computation from the similarity matrix.
%Recent work has used Jaccard similarity, NLI models~\cite{he2021deberta}, BERTScore~\cite{Zhang2020BERTScore} for such similarity measures.
% Recent work has used Jaccard similarity, which measures overlap between responses by treating them as sets, or Natural Language Inference (NLI), which assesses semantic consistency based on entailment and contradiction relationships. 
% Once the similarity matrix is constructed, confidence scores can be derived by computing degree, which quantifies how well a response aligns with others, and eccentricity, which measures how much a response deviates from the most central responses. Further details can be found in~\cite{lin2024generating}.


\textbf{White-Box Methods} use the internal states of LLMs—including logit distributions and token-level probabilities—to estimate confidence. 
% use LLM internal states, including logit distributions and token-level probabilities, to estimate confidence. 
Recent research has adopted sequence likelihood~\cite{CSL}, which computes confidence from the probability of the complete generated response.
\baselineNLLNorm~\cite{vashurin2024benchmarking} extends this by normalizing for response length via average sequence likelihood. 
Recent refinements weigh tokens differently: \baselineSAR~\cite{duan-etal-2024-shifting} uses NLI for token importance, while Contextualized Sequence Likelihood (\baselineCSL, and its variant\baselineCSLNext)~\cite{CSL} weighs using attention values. 
Other approaches train probes on LLM internal activations and embeddings~\cite{ren2023outofdistribution,azaria-mitchell-2023-internal,li2023inferencetime}. 
Furthermore, the verbalized confidence (\baselinePTrue)~\cite{xiong2024can} elicits explicit ``True'' or ``False'' predictions. 
% While this could use the frequency of ``True' across multiple generations, in practice this is typically implemented by computing from the logits.
While this is technically possible by taking the frequency of ``True'' among multiple sampled generations, in practice it is typically implemented by computing from the logits. Note that uncertainty quantification in NLG is a closely related research direction, yet differs in a key way: uncertainty characterizes the predictive distribution rather than a specific $\predSeq$. For more details of this distinction, see~\citet{lin2024generating}.
% as well.


% \textbf{White-Box Methods} leverage the internal states of large language models (LLMs)—including logit distributions and token-level probabilities—to estimate confidence. Recent research has adopted sequence likelihood~\cite{CSL}, which computes confidence based on the probability of the entire generated response. \baselineNLLNorm~\cite{vashurin2024benchmarking} extends this approach by normalizing for response length via the average sequence likelihood.
% Additional refinements weigh tokens differently. For example, \baselineSAR~\cite{duan-etal-2024-shifting} employs natural language inference (NLI) to determine token importance, while Contextualized Sequence Likelihood (\baselineCSL)~\cite{CSL} leverages attention values for weighting. Other methods train probes on LLM internal activations and embeddings~\cite{ren2023outofdistribution,azaria-mitchell-2023-internal,li2023inferencetime}. Furthermore, the verbalized confidence approach (\baselinePTrue)~\cite{xiong2024can} elicits explicit ``True'' or ``False'' predictions. Although it is technically possible to compute confidence by measuring the frequency of ``True'' responses across multiple sampled generations, in practice this is typically implemented by computing from the logits.



%\textbf{White-Box confidence estimation methods} leverage internal states of the LLM, such as the distribution of logits and token-level probabilities, to estimate confidence. 
%Recent research has widely adopted Sequence Likelihood~\cite{CSL}, which computes confidence based on the probability assigned to the entire generated response. 
%\baselineNLLNorm~\cite{vashurin2024benchmarking} extends this approach by accounting for response length, measuring the average sequence likelihood. 
%Recent work proposed refinements of \baselineNLLNorm by assigning different weights to tokens: \baselineSAR~\cite{duan-etal-2024-shifting} uses NLI to infer token importance, and Contextualized Sequence Likelihood (\baselineCSL)~\cite{CSL} assigns weights basing on attention values.
%refines confidence estimation by assigning different weights to tokens based on attention values extracted from the LLM, while its extension. 
%Others explore training some probes on the internal activations or embeddings from the LLM~\cite{ren2023outofdistribution,azaria-mitchell-2023-internal,li2023inferencetime}.
% Such methods, however, often require 
% The CSL-Next method, further improves confidence estimation by retrieving attention values for the next token after response generation. 
%Verbalized confidence (\baselinePTrue)~\cite{xiong2024can} asks the LLM to explicitly predict whether a response is ``True'' or ``False''.
%While this is technically achievable by taking the frequency of ``True'' among multiple sampled generations from the LLM as the confidence level, in practice it is typically implemented by computing from the logits as well.


\begin{figure*}[h]
    \centering
    \includegraphics[width=\textwidth]{figures/uq_eval_pipeline.pdf}
    % \caption{}
    % \vspace{-3mm}
    \caption{
    Illustration of the existing evaluation framework (blue) vs our proposed \uqeval (green).
    Unlike existing frameworks, we avoid the costly and unreliable correctness function module by using multiple-choice datasets.
    This requires slight modification to the confidence estimation steps, which is elaborated in \cref{sec:method}.
    % The choices' correctness could be directly fed into the final evaluation metrics.
    }
    \label{fig:pipeline}
\end{figure*}



%Note that uncertainty quantification in natural language generation is another line of research that is highly correlated, yet the key difference being that uncertainty is a property of the predictive distribution that does not pertain to a particular `$\predSeq$'. 
%Please refer to~\cite{lin2024generating} for a more thorough discussion on the distinction between uncertainty and confidence.



\subsection{Existing Evaluation Methods}\label{sec:prelim:old_eval}

Intuitively, a higher confidence score should correlate with the quality of model generation $\predSeq$ and its correctness relative to input $\xInput$. 
This assumption underpins selective classification, confidence scoring, and uncertainty quantification. 
In selective classification, also termed prediction with a rejection option, models abstain from low-confidence predictions, thereby reducing error rates while maximizing coverage~\cite{JMLR:v24:21-0048,geifman2017selective}. 
In other words, confidence measures guide selection towards predictions that are likely to be correct.



%Intuitively, a higher confidence score should indicate that a particular model generation $\predSeq$ is of higher quality and more likely to be ``correct'' with respect to the input $\xInput$.
%This assumption underlies most work on selective classification, confidence scoring, and uncertainty quantification. 
%In selective classification (also known as prediction with a reject option), models can choose to abstain from making a prediction when confidence is low, thereby reducing error rates while maximizing coverage~\cite{JMLR:v24:21-0048,geifman2017selective}. 
%In other words, confidence measures are used to guide the selection of predictions that are more likely to be correct.

This idea extends naturally to NLG, where confidence measures are used to guide selective generation or generation with abstention.
Assuming a given \textit{correctness function}~\cite{RCEhuang2024} $\acc(\predSeq;\xInput)\in\{0,1\}$, which tells us whether a response is good or correct\footnote{This could sometimes be relaxed to have a continuous range of $\mathbb{R}$, instead of $\{0,1\}$, but certain evaluation metrics such as AUROC require binary correctness labels.}, several evaluation metrics are used to assess confidence measures for NLG:
\begin{itemize}[leftmargin=*, nosep]
    \item Area Under the Receiver Operating Characteristic Curve (AUROC): 
    % Denoting $TPR(t)$ ($FPR(t)$) as the true (false) positive rate when 
    \begin{equation}
        \int_{-\infty}^{\infty} \text{TPR}(t) \, d\text{FPR}(t),
    \end{equation}
    where $\text{TPR}(t)$ ($\text{FPR}(t)$) is the true (false) positive rate comparing $\mathbbm{1}\{C(\predSeq)>t\}$ and $\acc(\predSeq)$, the correctness of $\predSeq$.
    AUROC measures how well the confidence scores distinguish between correct and incorrect responses.
    %C_{\mathcal{M}}(\xInput,\predSeq)
    
    \item Area Under the Accuracy-Rejection Curves (AUARC)~\cite{auarc}: 
    \begin{equation}
        \int_{-\infty}^{\infty} \text{Accuracy}(t) \, d\text{Coverage}(t),
    \end{equation}
    where $\text{Accuracy}(t)=\mathbb{E}\{\acc(\predSeq)|C(\predSeq)>t\}$ and $\text{Coverage}(t) = \mathbb{P}\{C(\predSeq)>t\}$.
    A refinement of AUROC designed for abstention-based settings, it evaluates the accuracy averaged across different coverage level (i.e. proportion of accepted predictions) when rejecting low-confidence predictions.
    
    \item Expected Calibration Error (ECE)~\cite{ICML2017_Guo}: 
    % A well-calibrated model ensures that a confidence score of 90\% corresponds to an approximately 90\% correctness rate.
    
    \begin{equation}
        \mathbb{E}\Big[|\mathbb{E}[\acc(\predSeq)|C(\predSeq)] - C(\predSeq)|\Big].
        %\sum_{m=1}^{M} \frac{|B_m|}{N} \left| \text{acc}(B_m) - \text{conf}(B_m) \right|
    \end{equation}
    % Used for calibration, 
    ECE quantifies the alignment between predicted confidence scores and actual correctness probabilities. 
    

    \item Rank-Calibratoin Error (RCE)~\cite{RCEhuang2024}: 
    {\small
    \begin{equation}
        \mathbb{E}_C\Big[|\mathbb{P}_{C'}\{reg(C')\geq reg(C)\} - \mathbb{P}_{C'}\{C'\leq C\}|\Big]
        %\sum_{m=1}^{M} \frac{|B_m|}{N} \left| \text{acc}(B_m) - \text{conf}(B_m) \right|,
    \end{equation}
    }
    where $reg(c)$ is a regression function for $\mathbb{E}[\acc|C=c]$ and $C'$ and $C$ are the confidence values of two independent responses.
    Unlike ECE, which cannot be directly applied to confidence measures that have not been calibrated in the frequency space, RCE directly assesses calibration in the ranking space, and is more generally applicable.
    
    %\item Rank-Calibration Error(RCE): It measures how well a confidence or uncertainty score preserves the ranking of correctness.
\end{itemize}


While these evaluation metrics are widely used in classification tasks, they all rely on a \textbf{correctness function} $\acc(\predSeq)$ to decide if a generation $\predSeq$ is correct. 
However, in NLG, correctness is inherently difficult to determine, unless $\predSeq$ exactly matches one of the reference answers, which is rare except for simple tasks.
%which becomes increasingly rare as tasks grow in complexity and output length.
Currently, correctness is often assessed using human evaluation or similarity-based methods:
% \begin{itemize}[nosep]
\paragraph{Human Evaluation} This remains arguably the most reliable approach. 
    Human evaluation is either used on smaller datasets~\cite{MCConf-pmlr-v239-ren23a} or to validate automated correctness functions~\cite{kuhn2023semantic,lin2024generating,CSL}, but is expensive and unscalable for large-scale dataset evaluation.
    
\paragraph{Similarity-Based Methods} In practice, correctness is often approximated by computing the similarity between $\predSeq$ and the reference answers $A$, in the form of $sim(\predSeq,A)$ or $sim(\predSeq,A|\xInput)$. 
    To accommodate metrics like AUROC, a threshold $\tau$ is applied to convert such similarity to $\{0,1\}$:
     \begin{equation}
        f(\predSeq, \xInput) =
        \begin{cases} 
          1, & \text{if } sim(\predSeq, A) > \tau \\ 
          0, & \text{otherwise}.
        \end{cases}\label{eq:correctness}
    \end{equation}
    Specifically, there are two common approaches for computing similarity.
    \textbf{Reference Matching} relies on lexical-based similarity metrics such as ROUGE and BLEU~\cite{hu2024unveiling,aynetdinov2024semscore,kuhn2023semantic}, which often fail to recognize semantically equivalent answers which are phrased differently.
    \textbf{LLM Judgment} uses a LLM as an evaluator~\cite{MCConf-pmlr-v239-ren23a,li2024generation,tan2024judgebench} and is more flexible. 
    However, such methods are computationally expensive and are still not fully reliable. 
    Recent studies indicate that machine-based correctness evaluation sometimes only has an accuracy of ~85\% on popular datasets~\cite{kuhn2023semantic,CSL}.
    %may exhibit systematic biases (e.g., favoring outputs from certain LMs),
% \end{itemize}

% Popular correctness functions include variants of \textbf{reference matching} techniques, including lexical-based similarities such as Rouge or BLEU scores~\cite{rouge,bleu,kuhn2023semantic,123} and outputs of judge LLMs~\cite{lin2024generating,MCConf-pmlr-v239-ren23a,123}
% The former has been demonstrated to \fontred{mis-judge correct answers phrased differently}.
% The latter, while more flexible and has the potential to understand more nuanced answers, is time-consuming, may have systematic bias (favoring generations from certain LMs), and is still not reliable.
% In fact, recent human evaluation suggests that machine-based correctness function is only around \todo{90\%} correct, depending on the technique and datasets.
% Alternatively, some rely on pure \textbf{human evaluation}~\cite{MCConf-pmlr-v239-ren23a}, which is arguably more reliable yet expensive and clearly un-scalable. 


% An efficient and accurate framework to evaluate various confidence measures is direly needed. 

% While this is %RCEhuang2024 crrectness function 

\subsection{Limitations of Existing Methods}
\label{subsecton:limitations}

% Flaws in the correctness function inevitably affect the downstream evaluation metrics such as AUROCs, undermining conclusions such as which confidence measure works better, especially when the performance of different measures are close. 
% \todo{We shown in \cref{fig:1} that the judge/threshold we set could greatly affect the ranking of different confidence measures.}

Flaws in the correctness function inevitably affect downstream evaluation metrics such as AUROC and thus our conclusions about different confidence measures. %, potentially leading to misleading conclusions about which confidence measure performs better, especially when different measures yield similar performance.
In this section, we illustrate the limitations of current confidence evaluation methods from two angles: the impact of threshold sensitivity and the inherent noise of similarity measures.

\paragraph{Case Study 1: Threshold Sensitivity}
%A common issue is that correctness judgments using LLMs often require a predefined threshold to map similarity scores into the binary function~\ref{eq:correctness}. This threshold is manually set and can significantly impact evaluation outcomes.

A common limitation of current practices is the need for a predefined threshold $\tau$ to convert similarity scores into binary correctness labels, as described in \cref{eq:correctness}. 
The choice of $\tau$ could thus impact the final evaluation metric.
% The choice of $\tau$ significantly influences the correctness labels and can subsequently affect the ranking of confidence measures.
To illustrate this, we vary the threshold for CoQA~\cite{CoQA} results from \citet{lin2024generating}, while keeping all other settings constant.
In their work, the threshold was manually set to $\tau=0.7$. 
% To examine the sensitivity of $\tau$, we vary $\tau$ while keeping all other settings constant.
However, \cref{fig:threshold} suggests that $\baselineEcc(C)$, for example, could either rank at the top or the bottom depending on $\tau$.
%hat our threshold choice could significantly impact the ranking of various confidence measures.

\begin{figure}[t]
  \includegraphics[width=1\columnwidth]{figures/ranking_all.png}
  \caption{The AUROC ranking of black-box confidence measures (on LLaMA2-13B and CoQA) is sensitive to the threshold \( \tau \).
  }
  \label{fig:threshold}
\end{figure}

\paragraph{Case Study 2: Similarity Noise} 
Correctness labels, whether derived from human evaluation, LLM-based scoring, or reference matching, are inherently noisy. 
For instance, within LLM-based judgments, correctness labels can fluctuate due to factors such as prompt variations and how the LLM judges were designed and trained. 
Echoing prior observations, \cref{fig:inconsis} shows examples where LLM judgments could either differ between different LLM judges or between different calls to the same LLM judge.
\citet{CSL} proposes to set the correctness function $\acc$ as the consensus of multiple LLMs, which improves the reliability of the correctness of responses LLMs agree on.
However, simply ignoring the disagreement could also introduce systematic selection bias.
% We consider two observable phenomena: \textbf{querying GPT-4o-mini multiple time} with the same prompt can yield different similarity scores between the same response $\predSeq$ and the reference answer $a_i$. \textcolor{red}{add an example}.

% Similarly, \textbf{different LLMs}, such as LLMs from the GPT family and the  LLaMA family, could result in different similarity scores. 
% (Such disagreement was, for example, in~\cite{CSL}, which proposed to take the consensus of multiple judge LLMs to improve the correctness function.)
% \textcolor{red}{add an example}.

\begin{figure}[t]
  \includegraphics[width=\columnwidth]{figures/inconsis-examples.pdf}
  \caption{
  Using LLM judges as $\acc$, while flexible, still has inherent noise.
  Different LLMs disagree on whether a response is correct (left).
  Even the same LLM (GPT, right) could deliver different opinions simply due to the randomness in generation.
  }
  \label{fig:inconsis}
\end{figure}

To systematically analyze this effect, we simulate correctness label noise with Gaussian noise and analyze its effects.
We modify the correctness function as:
\begin{equation}
\small
    \tilde{\acc}(\predSeq; \xInput) = Sigmoid(logit(\acc(\predSeq; \xInput)) + \epsilon), \quad \epsilon \sim \mathcal{N}(0, \sigma^2).\label{eq:gaussian_noise}
\end{equation}
% where $\epsilon$ represents noise drawn from a normal distribution with variance $\sigma^2$.
As shown in~\cref{tab:noise_rank}, increasing noise levels can lead to significant instability in ranking different confidence measures.
Note that our simulation likely \textit{underestimates} the issue, because the noise in \cref{eq:gaussian_noise} is unbiased and does not reflect systematic bias that may favor certain confidence measures~\cite{lin2022truthfulqa}. 


% \begin{figure}[t]
%   \includegraphics[width=\columnwidth]{figures/noise_distribution.png}
%   \caption{Noise Distribution. }
%   \label{fig:noise_distribuion}
% \end{figure}


\begin{table}[t]
\centering
\small
\resizebox{\columnwidth}{!}{ 
\begin{tabular}{@{}c|cccccc@{}}
\toprule
Ranking & 1 & 2 & 3 & 4 & 5 & 6  \\
\midrule
Original & Deg(C) & Deg(E) & Ecc(E) & Deg(J) & Ecc(C) & Ecc(J) \\
Noisy & Deg(C) & Deg(E) & Deg(J) & Ecc(E) & Ecc(J) & Ecc(C) \\
\bottomrule
\end{tabular}}
\vspace{-2mm}
\caption{Ranking of uncertainty quantification methods before and after noise.}
\label{tab:noise_rank}
\end{table}

While it might be theoretically possible to estimate the noise level and its propagation to errors on metrics like AUROC, this requires strong assumptions (e.g. \cref{eq:gaussian_noise}), extensive human evaluation, and replication across LLM judges and datasets. 
This fragility in existing evaluation methods motivates our framework, which eliminates dependence on uncertain correctness functions.


%While it might be theoretically possible to estimate the noise level and propagate it to a final estimation error on metrics like AUROC, this would require additional assumptions (e.g. \cref{eq:gaussian_noise}) and intensive human evaluation, and need to be repeated for each LLM judge and dataset.
%These findings highlight the fragility of existing evaluation methods and motivate the need for evaluation frameworks that do not rely on a potentially trembling correctness function, as explored in our proposed approach.


% Denoting the model as $\mathcal{M}$, for any given input prompt denoted as $x$, the response $\predSeq$ is a sequence of tokens $[s_1,\ldots,s_n]$ sampled from the predictive distribution $P(S;x,\mathcal{M})$.
% A confidence measure 
% We will denote $\predSeq_{<i}$ as the truncated sequence $[s_1,\ldots,s_{i-1}]$.
% Given the auto-regressive assumption, the (log-softmax'd) logit for the $i$-th token represent $\mathcal{M}$'s prediction of the log probability of token $s_i$ at this location.

\begin{figure*}[t]
    \centering
    \includegraphics[width=0.98\textwidth]{figures/Method.pdf}
    \vskip -1.5em
    \caption{Overall framework of \method{}, which distills knowledge from a teacher model to a student MLP using (a) Multi-Scale Distillation and (b) Multi-Period Distillation at both feature and prediction levels. (a) Multi-Scale Distillation involves downsampling the original time series into multiple coarser scales and aligning these scales between the student and teacher. (b) Multi-Period Distillation applies FFT to transform the time series into a spectrogram, followed by matching the period distributions after applying softmax.}
    \label{fig:method}
    \vskip -1em
\end{figure*}

\vskip -2em
\section{Methodology}

% In alignment with our intuition about preserving multi-scale and multi-period pattern knowledge, we introduce a novel distillation framework named \method{}. Unlike conventional approaches that emphasize matching predictions or time series representations directly, \method{} focuses on transferring knowledge of multi-scale patterns in the temporal domain and multi-period patterns in the frequency domain. 
% To efficiently distill this knowledge from the teacher model to the MLP, we propose two specific distillation objectives: \textit{multi-scale distillation} and \textit{multi-period distillation}, which we will detail next. The overall framework of \method{} is shown in Figure~\ref{fig:method}.

% \wei{it could be good to introduce an overall framework for KD (using equations) in time series and the following subsections describe how \method{} address it? }

% \wei{Standardize the notations: (1) bold upper-case letters for matrices; (2) bold lower-case letters for vectors; (3) regular letters for scalars (typically lower cases while upper cases are fine). You can take a look at the itransformer paper}

Motivated by our preliminary studies, we propose a novel \textit{KD} framework \method{} for time series to transfer the knowledge from a fixed, pretrained teacher model \(f_t\) to a student MLP model \(f_s\). The student produces predictions \(\mathbf{\hat{Y}}_s \in \mathbb{R}^{S \times C}\) and internal features \(\mathbf{H}_s\in \mathbb{R}^{D \times C}\). The teacher model produces predictions \(\mathbf{\hat{Y}}_t \in \mathbb{R}^{S \times C}\) and internal features \(\mathbf{H}_t\in \mathbb{R}^{D_t \times C}\). Our general objective is:
\begin{equation}\label{eq:kd_obj}
    \min\nolimits_{\theta_s} \mathcal{L}_{sup}(\mathbf{Y}, \mathbf{\hat{Y}}_s) + \mathcal{L}_{\mathrm{KD}}^\mathbf{Y}(\mathbf{\hat{Y}}_t, \mathbf{\hat{Y}}_s) + \mathcal{L}_{\mathrm{KD}}^\mathbf{H}(\mathbf{H}_t, \mathbf{H}_s),
\end{equation}
where \(\theta_s\) is the parameter of the student; \(\mathcal{L}_{sup}\) is the supervised loss (e.g., MSE) between predictions and ground truth; \(\mathcal{L}_{\mathrm{KD}}^\mathbf{Y}\) and \(\mathcal{L}_{\mathrm{KD}}^\mathbf{H}\) are the distillation loss terms that encourage the student model to learn knowledge from the teacher on both \textbf{prediction level}~\cite{hinton2015distilling} and \textbf{feature level}~\cite{romero2014fitnets}, respectively. Unlike conventional approaches that emphasize matching model predictions, \method{} integrates key time-series patterns including multi-scale and multi-period knowledge. The overall framework of \method{} is shown in Figure~\ref{fig:method}. 
% In the following, we detail each component of our approach.



% In alignment with our intuition about preserving multi-scale and multi-period pattern knowledge, we introduce a novel distillation framework named \method{}. Unlike conventional approaches that emphasize matching predictions directly, \method{} focuses on transferring knowledge of multi-scale patterns in the temporal domain and multi-period patterns in the frequency domain. 
% The overall framework of \method{} is shown in Figure~\ref{fig:method}. In the following, we provide details on each component of our approach.To efficiently distill this knowledge from the teacher model to the MLP, we propose two specific distillation objectives: \textbf{multi-scale distillation} and \textbf{multi-period distillation}. The overall framework of \method{} is shown in Figure~\ref{fig:method}. In the following, we provide details on each component of our approach.

\subsection{Multi-Scale Distillation}
One key component of \method{} is multi-scale distillation, where ``multi-scale'' refers to representing the same time series at different sampling rates. This enables MLP to effectively capture both coarse-grained and fine-grained patterns. By distilling at both the prediction level and the feature level, we ensure that MLP not only replicates the teacher's multi-scale predictions but also aligns with its internal representations from the intermediate layer.

\vspace{-0.5em}
\paragraph{Prediction Level.}
At the prediction level, we directly transfer multi-scale signals from the teacher’s outputs to guide the MLP’s predictions. We first produce multi-scale predictions by downsampling the original predictions from the teacher \(\mathbf{\hat{Y}}_t \in \mathbb{R}^{S \times C}\) and the MLP \(\mathbf{\hat{Y}}_s \in \mathbb{R}^{S \times C}\), where \(S\) is the prediction length and \(C\) is the number of variables. The predictions at \textit{Scale 0} are equal to the original predictions, that is, \(\mathbf{\hat{Y}}_t^0=\mathbf{\hat{Y}}_t\) and \(\mathbf{\hat{Y}}_s^0=\mathbf{\hat{Y}}_s\). We then downsample these predictions across \(M\) scales using convolutional operations with a stride of 2, generating multi-scale prediction sets \(\mathcal{Y}_t = \{\mathbf{\hat{Y}}_t^0, \mathbf{\hat{Y}}_t^1,\cdots,\mathbf{\hat{Y}}_t^M\}\) and \(\mathcal{Y}_s = \{\mathbf{\hat{Y}}_s^0, \mathbf{\hat{Y}}_s^1,\cdots,\mathbf{\hat{Y}}_s^M\}\), where \(\mathbf{\hat{Y}}_t^M, \mathbf{\hat{Y}}_s^M \in \mathbb{R}^{\lfloor S/2^M \rfloor \times C}\). The downsampling is defined as: 
\begin{equation}
    \mathbf{\hat{Y}}_x^m = \mathrm{Conv}(\mathbf{\hat{Y}}_x^{m-1}, \mathrm{stride}=2),
    \label{eq:multiscale_downsample}
\end{equation}
where \(x \in \{t, s\}\), \(m \in \{1, \cdots, M\}\), $\mathrm{Conv}$ denotes a 1D-convolutional layer with a temporal stride of 2. The predictions at the lowest level \(\mathbf{\hat{Y}}_x^0=\mathbf{\hat{Y}}_x\) maintain the original temporal resolution, while the highest-level predictions \(\mathbf{\hat{Y}}_x^M\) represent coarser patterns. We define the multi-scale distillation loss at the prediction level as:
\begin{equation}
    \mathcal{L}_{scale}^\mathbf{Y} = \textstyle\sum_{m=0}^M ||\mathbf{\hat{Y}}_t^m - \mathbf{\hat{Y}}_s^m||^2 /(M+1).
\end{equation}
% \wei{if Y is a vector or matrix, we should not use () but $| |$} 
Here we use MSE loss to match the MLP’s predictions to the teacher’s predictions at multiple scales.

\vspace{-0.5em}
\paragraph{Feature Level.} 
At the feature level, we align MLP’s intermediate features with teacher’s multi-scale representations, enabling MLP to form richer internal structures that support more accurate forecasts.
Let \(\mathbf{H}_s \in \mathbb{R}^{D \times C}\) and \(\mathbf{H}_t \in \mathbb{R}^{D_t \times C}\) denote MLP and teacher features with feature dimensions \(D\) and \(D_t\), respectively. As their dimensions can be different, we first use a parameterized regressor (e.g. MLP) to align their feature dimensions: 
% \wei{what does the regressor mean?? no explanation for this operator: what is the detailed operator; what is the purpose}
\begin{equation}
    \mathbf{H}'_t = \text{Regressor}(\mathbf{H}_t),
\end{equation}
where \(\mathbf{H}'_t \in \mathbb{R}^{D \times C}\).  
% We then downsample both sets of features across multiple scales:
% \begin{equation}
%     \mathbf{H}_x^m = \mathrm{Conv}(\mathbf{H}_x^{m-1}, \mathrm{stride}=2),
% \end{equation}
% where \(x \in \{t, s\}\), \(m \in \{1, \cdots, M\}\), and \(\mathbf{H}_s^0 = \mathbf{H}_s\), \(\mathbf{H}_t^0 = \mathbf{H}'_t\). 
Similar to the prediction level, we compute $\mathbf{H}_x^m$ by downsampling $\mathbf{H}_s$ and $\mathbf{H}'_t$ across multiple scales using the same approach as in Equation~\ref{eq:multiscale_downsample}. We define the multi-scale distillation loss at the feature level as:
\begin{equation}
    \mathcal{L}_{scale}^\mathbf{H} = \textstyle\sum_{m=0}^M ||\mathbf{H}_t^m - \mathbf{H}_s^m||^2 /(M+1).
\end{equation}

\subsection{Multi-Period Distillation}
% \wei{seems that the prediction level and feature level are basically using the same equations? then we probably do not so much space to describe them given the redundancy}
% \wei{we need some transitions: in addition to multi-scale ...temporal domain...}  
{In addition to multi-scale distillation in the temporal domain, we further
propose multi-period distillation to help MLP learn complex periodic patterns in the frequency domain.} By aligning periodicity-related signals from the teacher model at both the prediction and feature levels, the MLP can learn richer frequency-domain representations and ultimately improve its forecasting performance.

\paragraph{Prediction Level.}
For the predictions from the teacher \(\mathbf{\hat{Y}}_t \in \mathbb{R}^{S \times C}\) and the MLP \(\mathbf{\hat{Y}}_s \in \mathbb{R}^{S \times C}\), we first identify their periodic patterns. We perform this in the frequency domain using the Fast Fourier Transform (FFT):
\begin{equation}
    \mathbf{A}_x = \text{Amp}(\text{FFT}(\mathbf{\hat{Y}}_x)),
    \label{eq:multiperiod_spectrograms}
\end{equation}
where \(x \in \{t, s\}\) and spectrograms \(\mathbf{A}_x \in \mathbb{R}^{\frac{S}{2} \times C}\). Here, \(\text{FFT}(\cdot)\) denotes the FFT operation and \(\text{Amp}(\cdot)\) calculates the amplitude. We remove the direct current (DC) component from \(\mathbf{A}_x\). For certain variable \(c\), the \(i\)-th value \(\mathbf{A}_x^{i,c}\) indicates the intensity of the frequency-\(i\) component, corresponding to a period length \(\lceil S/i\rceil\). Larger amplitude values indicate that the associated frequency (period) is more significant.

To reduce the influence of minor frequencies and avoid noise introduced by less meaningful frequencies~\cite{timesnet, fedformer}, we propose a distribution-based matching scheme. We use a softmax function with a colder temperature to highlight the most significant frequencies:
\begin{equation}
    \mathbf{Q}_x^\mathbf{Y} = {\exp\bigl(\mathbf{A}_x^i / \tau\bigr)}/{\sum\nolimits_{j=1}^{S/2} \exp\bigl(\mathbf{A}_x^j /\tau\bigr)},
    \label{eq:multiperiod_distribution}
\end{equation}
where \(\mathbf{Q}_x^\mathbf{Y} \in \mathbb{R}^{\frac{S}{2} \times C}\) and \(\tau\) is a temperature parameter that controls the sharpness of the distribution. We set \(\tau=0.5\) by default. The period distribution \(\mathbf{Q}_x^\mathbf{Y}\) represents the multi-period pattern in the prediction time series, which we want the MLP to learn from the teacher. We use KL divergence to match these distributions~\cite{hinton2015distilling}. We define the multi-period distillation loss at the prediction level as:
\begin{equation}
    \mathcal{L}_{period}^\mathbf{Y} = \text{KL}\bigl(\mathbf{Q}_t^\mathbf{Y}, \mathbf{Q}_s^\mathbf{Y}\bigr).
\end{equation}
% where KL denotes the Kullback--Leibler divergence~\cite{hinton2015distilling}, a common metric to measure distribution difference.

% \paragraph{Feature Level.}
% Similar to the prediction level, we also apply multi-period distillation at the feature level. We compute:
% \begin{equation}
%     \mathbf{B}_x = \text{Amp}(\text{FFT}(\mathbf{\hat{H}}_x)),
% \end{equation}
% \begin{equation}
%     \mathbf{Q}_x^\mathbf{H} = \frac{\exp\bigl(\mathbf{B}_x^i / \tau\bigr)}{\sum_{j=1}^{D/2} \exp\bigl(\mathbf{B}_x^j /\tau\bigr)},
% \end{equation}
% and define:
% \begin{equation}
%     \mathcal{L}_{period}^\mathbf{H} = \text{KL}\bigl(\mathbf{Q}_t^\mathbf{H}, \mathbf{Q}_s^\mathbf{H}\bigr),
% \end{equation}
% where \(\mathbf{B}_x, \mathbf{Q}_x^\mathbf{H} \in \mathbb{R}^{\frac{D}{2} \times C}\). These feature-level distributions represent the multi-period pattern in feature space, enabling the MLP to learn periodic structure from the teacher at the feature level.

\vspace{-0.5em}
\paragraph{Feature Level.}
Similar to the prediction level, we apply multi-period distillation at the feature level. For the features \(\mathbf{H}'_t \in \mathbb{R}^{D \times C}\) and \(\mathbf{H}_s \in \mathbb{R}^{D \times C}\), we compute the spectrograms and the corresponding period distributions \(\mathbf{Q}_x^\mathbf{H}\) using the same approach as in Equations~\ref{eq:multiperiod_spectrograms} and~\ref{eq:multiperiod_distribution}. Multi-period distillation loss at feature level is then defined as:
\begin{equation}
    \mathcal{L}_{period}^\mathbf{H} = \text{KL}\bigl(\mathbf{Q}_t^\mathbf{H}, \mathbf{Q}_s^\mathbf{H}\bigr).
\end{equation}

\subsection{Overall Optimization and Theoretical Analaysis}
During the training of \method{}, we jointly optimize both the multi-scale and multi-period distillation losses at both the prediction and feature levels, together with the supervised ground-truth label loss:
\begin{equation}
    \mathcal{L}_{sup} = ||\mathbf{Y} - \mathbf{\hat{Y}}_s||^2,
\end{equation}
where \(\mathcal{L}_{sup}\) is the ground-truth loss (for example, MSE loss) used to train MLP directly. Thus, the overall training loss for the student MLP is defined as:
\begin{equation}
    \mathcal{L} = \mathcal{L}_{sup} + \alpha \cdot \bigl(\mathcal{L}_{scale}^\mathbf{Y} + \mathcal{L}_{period}^\mathbf{Y}\bigr) + \beta \cdot \bigl(\mathcal{L}_{scale}^\mathbf{H} + \mathcal{L}_{period}^\mathbf{H}\bigr),
    \label{eq:overall_optimization}
\end{equation}
where \(\alpha\) and \(\beta\) are hyper-parameters that control the contributions of the prediction-level and feature-level distillation loss terms, respectively. The teacher model is pretrained and remains frozen throughout the training process of MLP.


\underline{\textbf{Theoretical Interpretations.}} We provide a theoretical understanding of multi-scale and multi-period distillation loss from \textbf{a novel data augmentation perspective}. We further show that the proposed distillation loss can be interpreted as training with augmented samples derived from a special \textit{mixup}~\cite{mixup} strategy. The distillation process augments data by blending ground truth with teacher predictions, analogous to label smoothing in classification, and provides several benefits for time series forecasting:
\textit{\textbf{(1)} Enhanced Generalization:} It enhances generalization by exposing the student model to richer supervision signals from augmented samples, thus mitigating overfitting, especially with limited or noisy data.
{\textit{\textbf{(2)} Explicit Integration of Patterns:} The augmented supervision signals explicitly incorporate patterns across multiple scales and periods, offering insights that are not immediately evident in the raw ground truth.}
\textit{\textbf{(3}) Stabilized Training Dynamics:} The blending of targets softens the supervision signals, which diminishes the model’s sensitivity to noise and leads to more stable training phases. This will in turn support smoother optimization dynamics and fosters improved convergence. For clarity, our discussion is centered at the prediction level.  We present the following theorem:  
% \wei{in eq. 12, we used alpha and beta; we wanna avoid abusing notations. you may use $\lambda_1$ and $\lambda_2$ in eq. 12 or change the threom}
% \begin{theorem} \label{thm:multiscale}
% Let $(x, y)$ denote original input data pairs and $(x, y^t)$ represent corresponding teacher data pairs. Consider a data augmentation function $\mathcal{A}(\cdot)$ applied to $(x, y)$, generating augmented samples $(x', y')$. Define the training loss on these augmented samples as $\mathcal{L}_{aug} = \textstyle\sum_{(x',y') \in \mathcal{A}(x,y)} |f_s(x') - y'|^2$. Then, the following inequality holds: 
% $\mathcal{L}_{sup} + \lambda \mathcal{L}_{scale} \geq \mathcal{L}_{aug}$
% % \begin{equation}
% %    \mathcal{L}_{sup} + \alpha \mathcal{L}_{scale} \geq \mathcal{L}_{aug}
% % \end{equation}
% when $\mathcal{A}(\cdot)$ is instantiated as a mixup function~\cite{mixup} that interpolates between the original input data $(x,y)$ and teacher data $(x,y^t)$ with a mixing coefficient $\lambda \in (0,1)$, i.e. $y' = \lambda y^t + (1-\lambda) y$.
% \end{theorem}
\begin{theorem} \label{thm:multiscale}
Let $(x, y)$ denote original input data pairs and $(x, y^t)$ represent corresponding teacher data pairs. Consider a data augmentation function $\mathcal{A}(\cdot)$ applied to $(x, y)$, generating augmented samples $(x', y')$. Define the training loss on these augmented samples as $\mathcal{L}_{aug} = \sum_{(x',y') \in \mathcal{A}(x,y)} |f_s(x') - y'|^2$. Then, the following inequality holds: 
$
   \mathcal{L}_{sup} + \eta \mathcal{L}_{scale} \geq \mathcal{L}_{aug},
$
when $\mathcal{A}(\cdot)$ is instantiated as a mixup function~\cite{mixup} that interpolates between the original input data $(x,y)$ and teacher data $(x,y^t)$ with a mixing coefficient $\lambda=\frac{1}{1+\eta}$, i.e. $y' = \lambda y + (1-\lambda) y^t$.
\end{theorem}
We provide proof of Theorem~\ref{thm:multiscale} in Appendix~\ref{app:theory}.  Theorem~\ref{thm:multiscale} suggests that optimizing multi-scale distillation loss \(\mathcal{L}_{\text{scale}}\) jointly with supervised loss \(\mathcal{L}_{\text{sup}}\) is equivalent to minimizing an upper bound on a special \textit{mixup} augmentation loss. In particular, we mix multi-scale teacher predictions \(\{\mathbf{\hat{Y}}_t^{(m)}\}_{m=0}^M\) with ground truth \(\mathbf{Y}\), thereby allowing MLP to learn more informative time series temporal pattern. Similarly, we present a theorem for understanding $\mathcal{L}_{period}$.

% \begin{theorem} \label{thm:multiperiod}
% Define the training loss on the augmented samples using KL divergence as $\mathcal{L}_{aug} = \textstyle\sum_{(x',y') \in \mathcal{A}(x,y)} \text{KL}\big(y', \mathcal{X}(f_s(x'))\big)$, where $\mathcal{X}(\cdot) = \text{Softmax}(\text{Amp}(\text{FFT}(\cdot)))$. Then, the following inequality holds: 
% $\mathcal{L}_{sup} + \lambda \mathcal{L}_{period} \geq \mathcal{L}_{aug}$
% % \begin{equation}
% %    \mathcal{L}_{sup} + \alpha \mathcal{L}_{period} \geq \mathcal{L}_{aug}
% % \end{equation}
% where $\mathcal{A}(\cdot)$ is instantiated as a mixup function that interpolates between the period distribution of original input data $(x,\mathcal{X}(y))$ and teacher data $(x,\mathcal{X}(y^t))$ with a mixing coefficient $\lambda \in (0,1)$, i.e. $y' = \lambda \mathcal{X}(y^t) + (1-\lambda) \mathcal{X}(y)$.
% \end{theorem}
\begin{theorem} \label{thm:multiperiod}
Define the training loss on the augmented samples using KL divergence as $\mathcal{L}_{aug} = \sum_{(x',y') \in \mathcal{A}(x,y)} \text{KL}\big(y', \mathcal{X}(f_s(x'))\big)$, where $\mathcal{X}(\cdot) = \text{Softmax}(\text{FFT}(\cdot))$. Then, the following inequality holds: 
$
   \mathcal{L}_{sup} + \eta\mathcal{L}_{period} \geq \mathcal{L}_{aug},
$
where $\mathcal{A}(\cdot)$ is instantiated as a mixup function that interpolates between the period distribution of original input data $(x,\mathcal{X}(y))$ and teacher data $(x,\mathcal{X}(y^t))$ with a mixing coefficient $\lambda=\eta$, i.e. $y' =  \mathcal{X}(y) + \lambda \mathcal{X}(y^t)$.
\end{theorem}
The proof can be found in Appendix~\ref{app:theory}. Theorem~\ref{thm:multiperiod} shows that optimizing the multi-period distillation loss \(\mathcal{L}_{\text{period}}\) jointly with the supervised loss \(\mathcal{L}_{\text{sup}}\) is equivalent to minimizing an upper bound on the KL divergence between the student period distribution \(\mathcal{X}(f_s(x'))\) (or \(\mathbf{Q}_s\)) and a \emph{mixed} period distribution \(y'\) (or \(\mathbf{Q}_y + \lambda\,\mathbf{Q}_t\)). 
% \wei{we probably do not need (or need to rephrase) the following because it is not very related to the theorem (data augmentation); we need to describe the benefit from the data augmtantion perspective like you did for the above paragraph} This helps the model learn multi-period frequency patterns by incorporating the teacher’s period distribution, thereby identifying and modeling cyclic behaviors with overlapping or multiple periodicities.

% \begin{theorem}\label{thm:multiperiod}
% Optimizing the multi-period distillation loss $\mathcal{L}_{\text{period}}$ 
% is equivalent to minimizing an upper bound on the KL divergence between the student distribution \(\mathbf{Q}_s\) and a \emph{mixed} label distribution \(\alpha\,\mathbf{Q}_y + (1-\alpha)\,\mathbf{Q}_t\).
% Formally, for $\alpha \in [0,1]$, the following inequality holds:
% \[
% \begin{aligned}
% &\alpha\,\mathrm{KL}\bigl(\mathbf{Q}_y, \mathbf{Q}_s\bigr)
% \;+\;
% (1-\alpha)\,\mathrm{KL}\bigl(\mathbf{Q}_t, \mathbf{Q}_s\bigr)\\
% &\ge
% \mathrm{KL}\Bigl(\alpha\,\mathbf{Q}_y + (1-\alpha)\,\mathbf{Q}_t
% \;,\;\mathbf{Q}_s\Bigr).
% \end{aligned}
% \]
% \end{theorem}

% \wei{please number these benefits to enhance readability} These two theorems provide a theoretical understanding of multi-scale and multi-period distillation loss from a novel data augmentation perspective. The distillation process augments data by blending ground truth with teacher predictions, analogous to label smoothing in classification, and provides several benefits for time series forecasting. 
% It enhances generalization by exposing the student model to richer supervision signals, mitigating overfitting, especially with limited or noisy data, and 
% capturing trends or patterns not immediately apparent in the ground truth. 
% Furthermore, the softened targets from this blending reduce sensitivity to noise, stabilize training, and facilitate better convergence by ensuring smoother optimization dynamics.

% \wei{this paragraph is very important; please carefully rewrite; the goal is to highglight the novelty of this theoretical perspective and provide benefits of why our distillation framework can benefit the forecasting process} 


\begin{figure}[t]
    %
    \begin{center}
    \centerline{\includegraphics[width=0.5\columnwidth]{experiments/recht_loss.pdf}}
    %
    \caption{Matrix factorization via a 2-layer linear network}
    \label{recht}
\end{center}
\end{figure}

\section{Experiments}
\label{sec:experiments}
The prior works of FOOF, LocoProp, PRONG have shown to compare competitively
with other sophisticated optimizers such as K-FAC \citep{martens2015}.
Given their discussed equivalence with LNB, we focus on experimentally confirming
the contributions of this work: feature whitening via preconditioning the
gradient vector, and the applicability and effectiveness on
the realistic networks of ViT \citep{vit} and UNet \citep{unet}.

Due to its de facto status, we benchmark convergence with Adam in both
iteration and wall time.
In all experiments, the default EMA values were used ($\beta_1=0.9$, $\beta_2=0.999$)
for Adam while the learning rate was grid searched around $1e^{-4}$
All timings were recorded from a NVIDIA L4 GPU. Due to the deterministic nature of performing
2 conjugate gradient steps for LNB, the variance in reported times is negligible.


\subsection{Matrix factorization}
We start with the pathological example from \citet{recht}. This is a matrix
factorization problem formulated as a two-layer linear network:
$\sum_{i=1}^{n} \Vert W_1 W_2 x_i - y_i\Vert^2$,
where $y_i=Ax_i$ for a poorly conditioned matrix, $\kappa(A)=10^5$. Due to the conditioning
and the columns being correlated, it is known that gradient descent converges slowly for this
problem, whereas GN converges quickly. Because the LNB preconditioner is decorrelating
the feature space, we would also expect fast convergence.

We use the same initialization as in the notebook, but in order to magnify the differences,
we increase the dimensions by a factor of $10$: $n=10^4$,
$W_2 \in \sR^{60 \times 60}$, $W_1 \in \sR^{100 \times 60}$. Learning rates were tuned via
grid search to find fast and stable convergence for each method. The results are plotted in Figure \ref{recht}
and reproduce the prior reported slow convergence of Adam and demonstrate fast convergence with LNB.

\begin{figure}[t]
    %
    \begin{center}
    %
    \subfigure[Original pixels ($x$)]{\includegraphics[width=0.49\columnwidth]{experiments/mnist_acc.pdf}}
    %
    \subfigure[Inverted pixels ($1-x$)]{\includegraphics[width=0.49\columnwidth]{experiments/inv_mnist_acc.pdf}}
    %
    \vskip -0.1in
    \caption{MNIST test accuracy evolution trained on the original data (a) vs. inverted pixels (b).
    The step-size is parenthesized.}
    \label{fig:mnist}
    \end{center}
\end{figure}


\subsection{MLP}
We reproduce the MLP result in \citet{grub2010} that compares boosting and gradient
descent for a 2-layer MLP on MNIST using 800-node layers, $\tanh$ activation,
Glorot Normal initialization and a batch size of 1,000.

In Fig \ref{fig:mnist}-a., we plot the test accuracies w.r.t. epoch with the best two learning rates for Adam.
We first note that Adam and LNB obtain better than the prior reported accuracy of $98.3\%$.
Second, LNB achieved this performance with a fixed (ridge) regularizer $\lambda$, whereas prior
work heavily tuned this.
One explanation for the difference is that LNB is taking an adaptive step size according to the metric
and this was not derived before.
Third, there is little observed difference between boosting and gradient descent on this dataset.
This can be explained due to that most of the binary pixels in MNIST are zero, so the feature space
of the vectorized images is low rank, i.e., decorrelating provides little benefit.

\begin{figure}[t]
    \begin{center}
    \subfigure[\texttt{train} loss]{\includegraphics[width=0.49\columnwidth]{experiments/vit_loss.pdf}}
    \subfigure[\texttt{test} accuracy]{\includegraphics[width=0.49\columnwidth]{experiments/vit_acc.pdf}}
    %
    \caption{ViT performances on CIFAR10.}
    \label{fig:vit}
    \end{center}
\end{figure}

However, whitening does include a centering step. If we were to shift the feature space, we would expect
to get same performance. In Fig \ref{fig:mnist}-b, we plot the same models when trained and tested on pixels
$1-x$, where $x$ are the original binary pixel values used in Fig \ref{fig:mnist}-a. We observe that LNB gets very similar performance
while Adam (and other methods not invariant to affine reparameterizations) degrade. Although we could (and should)
simply normalize the features before training, this example illustrates a case of how feature scaling can
greatly affect convergence.

\subsection{Vision Transformer}
We train a vision transformer \cite{vit} using the notebook from Equinox \cite{eqx}
on CIFAR10.
The only modification we make is to not learn the affine terms in the LayerNorm
in order to speed up experimentation and we observed no performance benefit with it.
The train and test fold performances are show in \Figref{fig:vit}, where we observe
faster convergence and better generalization with LNB. Excluding JIT compilation time,
the duration per epoch for LNB and Adam is 1.26 min and 0.85 min, respectively. 

\begin{figure}[t]
    \begin{center}
    \subfigure[\texttt{train} loss]{\includegraphics[width=0.49\columnwidth]{experiments/voc_loss.pdf}}
    \subfigure[\texttt{val} accuracy]{\includegraphics[width=0.49\columnwidth]{experiments/voc_acc.pdf}}
    %
    \caption{UNet performances on VOC Segmentation.}
    \label{fig:unet}
    \end{center}
\end{figure}

\subsection{UNet}
We train a UNet \cite{unet} on the 2012 VOC Segmentation Challenge dataset \cite{voc}. The images are
pixelwise normalized into the range $[0,1]$ using ImageNet mean and variance R,G,B pixel values
and then zero-padded to $500 \times 500$ size and then downsized to $384 \times 384$.
No data augmentation is performed. We plot the results in \Figref{fig:unet} and
remark that LNB converges very quickly, using the same learning rate as with ViT, and
avoids overfitting. While both optimizers converge to comparable performance on the
\texttt{val} split, the rapid progress by LNB suggests it would be able to leverage
more data effectively.
However, excluding JIT compilation time,
the duration per epoch for LNB and Adam is 2.92 min and 1.27 min, respectively, and
this highlights the trade-off between convergence w.r.t. iterations vs. wall time.
While LNB converged marginally faster in wall time and is significantly
easier to implement to prior equivalent work, it is future work to better understand
in what deep networks does the whitening behavior lead to better generalization as
demonstrated in the other three experiments.

\section{Conclusion}
We have presented Digital Twin Buildings, a framework for extracting the 3D mesh of a building, for connecting the building to Google Maps Platform APIs, and for Multi-Agent Large Language Models data analytics. We demonstrate this by extracting visual description keywords and captions of the building from multi-view multi-scale images of the building. The framework can also be used to process different data modalities sourced from Google Cloud Services. This approach enables richer semantic understanding, seamless integration with geospatial data, and enhanced interaction with real-world structures, paving the way for advanced applications in urban analytics, navigation, and virtual environments.


% In the unusual situation where you want a paper to appear in the
% references without citing it in the main text, use \nocite
\nocite{langley00}

\bibliography{icml}
\bibliographystyle{icml2025}


%%%%%%%%%%%%%%%%%%%%%%%%%%%%%%%%%%%%%%%%%%%%%%%%%%%%%%%%%%%%%%%%%%%%%%%%%%%%%%%
%%%%%%%%%%%%%%%%%%%%%%%%%%%%%%%%%%%%%%%%%%%%%%%%%%%%%%%%%%%%%%%%%%%%%%%%%%%%%%%
% APPENDIX
%%%%%%%%%%%%%%%%%%%%%%%%%%%%%%%%%%%%%%%%%%%%%%%%%%%%%%%%%%%%%%%%%%%%%%%%%%%%%%%
%%%%%%%%%%%%%%%%%%%%%%%%%%%%%%%%%%%%%%%%%%%%%%%%%%%%%%%%%%%%%%%%%%%%%%%%%%%%%%%
\newpage
\appendix
% \onecolumn
% \clearpage
\appendix
\section{Proof of Theorems}
\label{app:propositions}

% \subsection{Detailed Proof of Theorem \ref{theo:1}}
% \label{app:theo1}
% % \begin{proof}[Proof of Theorem \ref{theo:1}]
%  The test error can be expressed as:
%     \[
%     \| \mathbf{Y} - \hat{\mathbf{Y}}' \| = \| \mathbf{Y} - \mathbf{F} \mathbf{W}' \|
%     \]
    
%     By adding and subtracting \( \mathbf{F} \mathbf{W} \), we have
%     \[
%     \| \mathbf{Y} - \mathbf{F} \mathbf{W} + \mathbf{F} \mathbf{W} - \mathbf{F} \mathbf{W}' \| \leq \| \mathbf{Y} - \mathbf{F} \mathbf{W} \| + \| \mathbf{F} (\mathbf{W} - \mathbf{W}') \|
%     \]

%     This demonstrates that the test error of graph condensation is bounded by the conventional GNN training error (first term) and the additional parameter matching error introduced by condensation (second term).
% \end{proof}



\subsection{Proof of Theorems~\ref{theo:training_stage}}
\label{app:proof_training_stage}
\begin{theorem-nonumber}
    The prediction distance in the training stage is bounded by the sum of representation distance and parameter distance.
    \begin{equation}
    \| \mathcal{K}(\hat{\mathbf{Y}}) - \hat{\mathbf{Y}}' \| 
    \leq  \| \mathcal{K}(\mathbf{F}) - \mathbf{F}' \| \cdot \| \mathbf{W}' \| + \| \mathbf{F} \| \cdot \| \mathbf{W} - \mathbf{W}' \|
    \end{equation}
    where $\|\cdot\|$ denotes the L2 norm and $\mathcal{K}(\cdot)$ can be any projection function that aligns the dimensions of $\hat{{\bf Y}}$ and $\hat{{\bf Y}}'$ or ${\bf F}$ and ${\bf F}'$.
\end{theorem-nonumber}

\begin{proof}
To preserve the training data information to maintain the performance of GNNs, we focus on matching the model predictions on the original graph $G$ and its condensed counterpart $G'$. Since $\mathcal{K}(\cdot)$ only aligns the first dimensions, we have $\mathcal{K}(\hat{\mathbf{Y}}) = \mathcal{K}(\mathbf{F})\mathbf{W}$. Therefore, the expression becomes:
\begin{equation}
    \begin{aligned}
    &\| \mathcal{K}(\hat{\mathbf{Y}}) - \hat{\mathbf{Y}}' \| \\
    = &\| \mathcal{K}(\mathbf{F})\mathbf{W} - \mathbf{F}'\mathbf{W}' \| \\
    = & \| \mathcal{K}(\mathbf{F}) \mathbf{W} - \mathcal{K}(\mathbf{F}) \mathbf{W}' + \mathcal{K}(\mathbf{F}) \mathbf{W}' - \mathbf{F}’ \mathbf{W}' \| \\
    \leq &\| \mathcal{K}(\mathbf{F}) (\mathbf{W} - \mathbf{W}') \| + \| (\mathcal{K}(\mathbf{F}) -  \mathbf{F}')\mathbf{W}' \| \\
    \leq & \| \mathcal{K}(\mathbf{F}) \| \cdot \| \mathbf{W} - \mathbf{W}' \| + \| \mathcal{K}(\mathbf{F}) - \mathbf{F}' \| \cdot \| \mathbf{W}' \|
\end{aligned}
\end{equation}
The objective in training stage is to minimizing $\| \mathcal{K}(\hat{\mathbf{Y}}) - \hat{\mathbf{Y}}' \|$, which can be formulated as:
\begin{equation}
    \begin{aligned}
    &\arg \min_{\hat{\mathbf{Y}}'} \| \mathcal{K}(\hat{\mathbf{Y}}) - \hat{\mathbf{Y}}' \|\\
    =&\arg \min_{\hat{\mathbf{Y}}'} \| \mathcal{K}(\mathbf{F}) \| \cdot \| \mathbf{W} - \mathbf{W}' \| + \| \mathcal{K}(\mathbf{F}) - \mathbf{F}' \| \cdot \| \mathbf{W}' \|\\
    \end{aligned}
\end{equation}
Note that $\|\mathbf{F}\|$ is a constant, and the weight matrix $\| \mathbf{W}' \|$ is naturally constrained due to regularization techniques during model optimization to control its magnitude. Then, we have:
\begin{equation}
    \begin{aligned}
    &\arg \min_{\hat{\mathbf{Y}}'} \| \mathcal{K}(\hat{\mathbf{Y}}) - \hat{\mathbf{Y}}' \|\\
    \approx&\arg \min_{\mathbf{F}'}  \underbrace{\| \mathbf{W} - \mathbf{W}' \|}_{\text{Parameter Distance}} + \underbrace{\| \mathcal{K}(\mathbf{F}) - \mathbf{F}' \|}_{\text{Representation Distance}}
    \end{aligned}
\end{equation}




Therefore, Theorem~\ref{theo:training_stage} indicates that by minimizing the \textbf{representation and parameter distances}, the predictions derived from the condensed graph can be close to those of the original graph.

This completes the proof. 
\end{proof}

% To achieve the objective, mainstream GC methods employ distribution matching, kernel ridge regression (KRR)-based matching, trajectory matching, and gradient matching~\cite{wu2020comprehensive}. However, KRR-based matching has been shown to be less effective when the condensed graph is evaluated using Graph Neural Networks (GNNs)~\cite{gcsntk, gong2024gc4nc}. Consequently, we exclude KRR-based approaches from our analysis. In contrast, both trajectory matching and gradient matching inherently aim to minimize the discrepancy between the model parameters trained on the condensed graph and those trained on the original training set. This objective can be succinctly characterized as parameter matching. Therefore, we categorize these two methodologies under the parameter matching framework. In conclusion, our theoretical framework emphasizes \textbf{distribution matching} and \textbf{parameter matching}, as we believe these approaches encompass the majority of existing GC methods.


\subsection{Proof of Theorems~\ref{theo:test_stage}}
\label{app:proof_test_stage}
\begin{theorem-nonumber}
    The test prediction error of the GNN trained of $G'$ is bounded by the test prediction error of the GNN trained on $G$ plus the parameter distance, as formularized by 
    \begin{equation*}
        \| \mathbf{Y} - \hat{\mathbf{Y}}'' \| \leq \| \mathbf{Y} - \mathbf{F}\mathbf{W} \| + \|\mathbf{F}\| \cdot \| \mathbf{W}- \mathbf{W}'\|
    \end{equation*}
\end{theorem-nonumber}

\begin{proof}
At the test stage, our condensation goal is to ensure that the GNN model, trained on condensed training data $G'$, generalizes effectively to the test graph, i.e., achieving a low prediction error on the test data. 
\begin{equation}
\begin{aligned}
    &\| \mathbf{Y} - \hat{\mathbf{Y}}'' \|  \\
    = &\| \mathbf{Y} - \mathbf{F} \mathbf{W}' \|\\
    = & \| \mathbf{Y} - \mathbf{F} \mathbf{W} + \mathbf{F} \mathbf{W} - \mathbf{F} \mathbf{W}' \| \\
    \leq &\| \mathbf{Y} - \mathbf{F} \mathbf{W} \| + \| \mathbf{F} (\mathbf{W} - \mathbf{W}') \| \\
    \leq &\| \mathbf{Y} - \mathbf{F} \mathbf{W} \| + \| \mathbf{F}\| \cdot \|\mathbf{W} - \mathbf{W}' \|
\end{aligned}
\end{equation}
This inequality incorporates both the original test prediction error and the parameter distance. The objective in testing stage is to minimizing $\| \mathbf{Y} - \hat{\mathbf{Y}}''\|$, which can be formulated as:
\begin{equation}
    \begin{aligned}
    &\arg \min_{\hat{\mathbf{Y}}''} \| \mathbf{Y} - \hat{\mathbf{Y}}'' \|\\
    \approx&\arg \min_{\hat{\mathbf{Y}}'} \| \mathbf{Y} - \mathbf{F} \mathbf{W} \| + \| \mathbf{F}\| \cdot \|\mathbf{W} - \mathbf{W}' \|\\
    \approx&\arg \min_{\hat{\mathbf{Y}}'} \underbrace{\|\mathbf{W} - \mathbf{W}'\|}_{\text{Parameter Distance}} \\
    \end{aligned}
\end{equation}
It indicates that by reducing the \textbf{parameter distance} $\| \mathbf{W}- \mathbf{W}'\|$, the test prediction error becomes more tightly bounded, assuming that the original test prediction error \( \| \mathbf{Y} - \mathbf{F}\mathbf{W}\| \) and the propagated feature matrix \( \mathbf{F} \) remain constant. 

This completes the proof. 
\end{proof}


% \begin{proof}
% We start our theoretical analysis with formulating a simple but new objective for GC. 
% GC can be seen as a process of minimizing the loss by a GNN model trained in the synthetic graph ${G'}$. The objective function can reformulated as follows: 
% \begin{equation}
% {G'}=\underset{{G'}}{\arg \min } \ \mathcal{L}(\mathbf{Y}, \operatorname{GNN}({G'})). \label{eq:graphReduction}
% \end{equation}  
%  Besides, without loss of generality, we employ a regression loss computed over the entire graph to evaluate performance during the testing phase, corresponding to a transductive setting. In an inductive setting, the feature matrix \( \mathbf{F} \) can be replaced by \( \mathbf{F}_t \), where \( \mathbf{F}_t \) represents the propagated features in the test graph.

% After training on the original training set, the prediction of the SGC model is given by:
% \begin{equation}
%     \hat{\mathbf{Y}} = \mathbf{F} \mathbf{W}
% \end{equation}
% where \( \mathbf{F} = \mathbf{A}^K \mathbf{X}\) is the propagated feature matrix of the entire graph. 
% \( \mathbf{W} \in \mathbf{R}^{d \times c} \) is the weight matrix of the trained SGC model.

% Conversely, the prediction of the model trained on the synthetic (condensed) dataset is expressed as:
% \begin{equation}
%     \hat{\mathbf{Y}}' = \mathbf{F} \mathbf{W}'
% \end{equation}
% where \( \mathbf{W}' \in \mathbf{R}^{d \times c} \) denotes the weight matrix obtained from training on the synthetic dataset. 

% The loss function $\mathcal{L}$ during the testing phase in Equ.~\ref{eq:graphReduction} can be computed using the regression loss
% \begin{equation}
%     \mathcal{L} = \| \mathbf{Y} - \hat{\mathbf{Y}}' \|
% \end{equation}
% where \( \mathbf{Y} \in \mathbf{R}^{n \times c} \) represents the true labels.
%  The test error can be expressed as:
%     \[
%     \| \mathbf{Y} - \hat{\mathbf{Y}}' \| = \| \mathbf{Y} - \mathbf{F} \mathbf{W}' \|
%     \]
    
%     By adding and subtracting \( \mathbf{F} \mathbf{W} \), we have
%     \[
%     \| \mathbf{Y} - \mathbf{F} \mathbf{W} + \mathbf{F} \mathbf{W} - \mathbf{F} \mathbf{W}' \| \leq \| \mathbf{Y} - \mathbf{F} \mathbf{W} \| + \| \mathbf{F} (\mathbf{W} - \mathbf{W}') \|
%     \]
% Finally, the regression loss can be bounded by decomposing the error into two components:
% \begin{equation}
%     \| \mathbf{Y} - \hat{\mathbf{Y}}' \| \leq \| \mathbf{Y} - \mathbf{F}\mathbf{W} \| + \| \mathbf{F} (\mathbf{W} - \mathbf{W}') \|
% \end{equation}
% This completes the proof.
% \end{proof}

\subsection{Proof of Theorem \ref{theo:bound_of_parameter}}
\label{app:proof_bound_of_parameter}
\begin{theorem-nonumber}
    The parameter distance can be bounded by the following inequality:
    \begin{equation}
        \| \mathbf{W} - \mathbf{W}' \| \leq \mathcal{C}(\max (\text{diag}(\mathbf{\mathbf{P}^\top\mathbf{P}})))^2,
    \end{equation}
    where $\mathcal{C}=\frac{\|\mathbf{F} \| \cdot \| \mathbf{Y}\|({\lambda_{\min}(\mathbf{F}^\top\mathbf{F})}+\|\mathbf{F} \|)}{{\lambda^2_{\min}(\mathbf{F}^\top\mathbf{F})}}$ is a constant, $\mathbf{P}^\top\mathbf{P} \in \mathbb{R}^{N' \times N'}$ is a diagonal matrix with each diagonal entry corresponding to how many original nodes are assigned to each synthetic node. 
\end{theorem-nonumber}

\begin{proof}
We aim to establish that clustering effectively bounds the error introduced by parameter matching and representation difference. The proofproceeds as follows:
\paragraph{Bounding the Parameter Matching Error \( \| \mathbf{W} - \mathbf{W}' \| \):}
Consider the weight matrices for the SGC and clustering-based methods
\[
\mathbf{W} = (\mathbf{F}^\top \mathbf{F})^{-1} \mathbf{F}^\top \mathbf{Y}, \quad \mathbf{W}' = (\mathbf{C}^\top \mathbf{C})^{-1} \mathbf{C}^\top \mathbf{Y}'
\]
where
\[
\mathbf{C} = (\mathbf{P}^\top\mathbf{P})^{-1} \mathbf{P}^\top \mathbf{F}, \quad \mathbf{Y}' = (\mathbf{P}^\top\mathbf{P}))^{-1} \mathbf{P}^\top \mathbf{Y}
\]

The difference between \( \mathbf{W} \) and \( \mathbf{W}' \) is
\[
\| \mathbf{W} - \mathbf{W}' \| = \left\| (\mathbf{F}^\top \mathbf{F})^{-1} \mathbf{F}^\top \mathbf{Y} - (\mathbf{C}^\top \mathbf{C})^{-1} \mathbf{C}^\top \mathbf{Y}' \right\|
\]

By substituting $\mathbf{C}$ and $\mathbf{Y}'$, we can express \( \mathbf{W}' \) in terms of \( \mathbf{F} \) and \( \mathbf{Y} \)
\[
\mathbf{W}' = \left( \mathbf{F}^\top \mathbf{P} (\mathbf{P}^\top\mathbf{P})^{-2} \mathbf{P}^\top \mathbf{F} \right)^{-1} \mathbf{F}^\top \mathbf{P} (\mathbf{P}^\top\mathbf{P})^{-2} \mathbf{P}^\top \mathbf{Y}
\]

Let \( \mathbf{A} = \mathbf{F}^\top \mathbf{F} \) and \( \mathbf{B} = \mathbf{F}^\top \mathbf{P} (\mathbf{P}^\top\mathbf{P})^{-2} \mathbf{P}^\top \mathbf{F} \), then:
\[
\mathbf{W} = \mathbf{A}^{-1} \mathbf{F}^\top \mathbf{Y}, \quad \mathbf{W}' = \mathbf{B}^{-1} \mathbf{F}^\top \mathbf{P} (\mathbf{P}^\top\mathbf{P})^{-2} \mathbf{P}^\top \mathbf{Y}
\]

The difference becomes
\[
\mathbf{W} - \mathbf{W}' = \mathbf{A}^{-1} \mathbf{F}^\top \mathbf{Y} - \mathbf{B}^{-1} \mathbf{F}^\top \mathbf{P} (\mathbf{P}^\top\mathbf{P})^{-2} \mathbf{P}^\top \mathbf{Y}
\]

Similar to the above two proofs, we add and subtract a term \( \mathbf{B}^{-1} \mathbf{F}^\top \mathbf{Y} \) and rewrite the difference by
\begin{align*}
        \mathbf{W} - \mathbf{W}' = &\left( \mathbf{A}^{-1} - \mathbf{B}^{-1} \right) \mathbf{F}^\top \mathbf{Y} \\
&+ \mathbf{B}^{-1} \mathbf{F}^\top \left( \mathbf{I} - \mathbf{P} (\mathbf{P}^\top\mathbf{P})^{-2} \mathbf{P}^\top \right) \mathbf{Y}
\end{align*}

Considering the norms, we have
    \begin{align*}
\| \mathbf{W} - \mathbf{W}' \| \leq &\| \mathbf{A}^{-1} - \mathbf{B}^{-1} \| \cdot \| \mathbf{F}^\top \mathbf{Y} \| \\
&+ \| \mathbf{B}^{-1} \| \cdot \| \mathbf{F}^\top (\mathbf{I} - \mathbf{P} (\mathbf{P}^\top\mathbf{P})^{-2} \mathbf{P}^\top) \mathbf{Y} \|
\end{align*}

We will now bound each term independently.

\textbf{Bounding the First Term}
\begin{align*}
        \| \mathbf{A}^{-1} - \mathbf{B}^{-1} \| \cdot \| \mathbf{F}^\top \mathbf{Y} \|
\end{align*}


Based on the norms of matrix inequality, we have
\[
\| \mathbf{A}^{-1} - \mathbf{B}^{-1} \| \leq \| \mathbf{A}^{-1} \| \cdot \| \mathbf{B} - \mathbf{A} \| \cdot \| \mathbf{B}^{-1} \|
\]
Then, according to
\[
\| \mathbf{A}^{-1} \| = \frac{1}{\lambda_{\min}(\mathbf{A})}, \quad \| \mathbf{B}^{-1} \| = \frac{1}{\lambda_{\min}(\mathbf{B})}
\]
and assuming \( \lambda_{\min}(\mathbf{B}) \geq \frac{1}{(\max_k (\mathbf{P}^\top\mathbf{P})_{kk})^2} \lambda_{\min}(\mathbf{A}) \), we have
\[
\| \mathbf{A}^{-1} \| \cdot \| \mathbf{B}^{-1} \| \leq \frac{(\max_k (\mathbf{P}^\top\mathbf{P})_{kk})^2}{\lambda_{\min}(\mathbf{A})^2}
\]

Bounding \( \| \mathbf{B} - \mathbf{A} \| \):
\begin{align*}
    \mathbf{B} - \mathbf{A} &= \mathbf{F}^\top \mathbf{P} (\mathbf{P}^\top\mathbf{P})^{-2} \mathbf{P}^\top \mathbf{F} - \mathbf{F}^\top \mathbf{F} \\
    &= -\mathbf{F}^\top (\mathbf{I} - \mathbf{P} (\mathbf{P}^\top\mathbf{P})^{-2} \mathbf{P}^\top) \mathbf{F}
\end{align*}
Taking norms:
\begin{align*}
\| \mathbf{B} - \mathbf{A} \| &= \| \mathbf{F}^\top (\mathbf{P} (\mathbf{P}^\top\mathbf{P})^{-2} \mathbf{P}^\top - \mathbf{I}) \mathbf{F} \| \\
&\leq \| \mathbf{F} \|^2 \cdot \| \mathbf{I} - \mathbf{P} (\mathbf{P}^\top\mathbf{P})^{-2} \mathbf{P}^\top \|
\end{align*}
Since \( \| \mathbf{I} - \mathbf{P} (\mathbf{P}^\top\mathbf{P})^{-2} \mathbf{P}^\top \| \leq 1 \), the inequality can be further simplified to
\(\| \mathbf{B} - \mathbf{A} \| \leq \| \mathbf{F} \|^2\).

Combining the above:
\[
\| \mathbf{A}^{-1} - \mathbf{B}^{-1} \| \cdot \| \mathbf{F}^\top \mathbf{Y} \| \leq \frac{(\max_k (\mathbf{P}^\top\mathbf{P})_{kk})^2}{\lambda_{\min}(\mathbf{A})^2} \cdot \| \mathbf{F} \|^2 \cdot \| \mathbf{Y} \|
\]


\textbf{Bounding the Second Term}
\begin{align*}
    \| \mathbf{B}^{-1} \| \cdot \| \mathbf{F}^\top (\mathbf{I} - \mathbf{P} (\mathbf{P}^\top\mathbf{P})^{-2} \mathbf{P}^\top) \mathbf{Y} \|
\end{align*}


Based on the proof above, we first have
\[
\| \mathbf{B}^{-1} \| = \frac{1}{\lambda_{\min}(\mathbf{B})} \leq \frac{(\max_k (\mathbf{P}^\top\mathbf{P})_{kk})^2}{\lambda_{\min}(\mathbf{A})}
\]

 Since \( \| \mathbf{I} - \mathbf{P} (\mathbf{P}^\top\mathbf{P})^{-2} \mathbf{P}^\top \| \leq 1 \), 
\[
\| \mathbf{F}^\top (\mathbf{I} - \mathbf{P} (\mathbf{P}^\top\mathbf{P})^{-2} \mathbf{P}^\top) \mathbf{Y} \| \leq \| \mathbf{F} \| \cdot \| \mathbf{Y} \|
\]

Combining these bounds:
\begin{align*}
        &\| \mathbf{B}^{-1} \| \cdot \| \mathbf{F}^\top (\mathbf{I} - \mathbf{P} (\mathbf{P}^\top\mathbf{P})^{-2} \mathbf{P}^\top) \mathbf{Y} \| \\
        &\leq \frac{(\max_k (\mathbf{P}^\top\mathbf{P})_{kk})^2}{\lambda_{\min}(\mathbf{A})} \cdot \| \mathbf{F} \| \cdot \| \mathbf{Y} \|
\end{align*}


\paragraph{Combining Both Terms:}
Adding the bounds for both terms, we obtain:
\begin{align*}
\| \mathbf{W} - \mathbf{W}' \| \leq & \frac{(\max_k (\mathbf{P}^\top\mathbf{P})_{kk})^2}{\lambda_{\min}(\mathbf{A})^2} \cdot \| \mathbf{F} \|^2 \cdot \| \mathbf{Y} \| \\
&\quad + \frac{(\max_k (\mathbf{P}^\top\mathbf{P})_{kk})^2}{\lambda_{\min}(\mathbf{A})} \cdot \| \mathbf{F} \| \cdot \| \mathbf{Y} \| \\
=&\max_k (\mathbf{P}^\top\mathbf{P})_{kk}) \cdot (\frac{\| \mathbf{F} \|^2 \cdot \| \mathbf{Y} \|}{\lambda_{\min}(\mathbf{A})^2} + \frac{\| \mathbf{F} \| \cdot \| \mathbf{Y} \|}{\lambda_{\min}(\mathbf{A})}) \\
=&\mathcal{C}(\max (\text{diag}(\mathbf{\mathbf{P}^\top\mathbf{P}})))^2
\end{align*}
where $\mathcal{C}=\frac{\|\mathbf{F} \| \cdot \| \mathbf{Y}\|({\lambda_{\min}(\mathbf{F}^\top\mathbf{F})}+\|\mathbf{F} \|)}{{\lambda^2_{\min}(\mathbf{F}^\top\mathbf{F})}}$ is a constant, $\mathbf{P}^\top\mathbf{P} \in \mathbb{R}^{N' \times N'}$ is a diagonal matrix with each diagonal entry corresponding to how many original nodes are assigned to each synthetic node. 

Our objective is to minimize the parameter distance, which can be reformulated by:
\begin{equation}
    \begin{aligned}
    &\arg \min_{\mathbf{W}'} \| \mathbf{W} - \mathbf{W}' \|\\
    \approx&\arg \min_{\mathbf{P}} \mathcal{C}(\max (\text{diag}(\mathbf{\mathbf{P}^\top\mathbf{P}})))^2\\
    \end{aligned}
\end{equation}

This completes the proof.   
\end{proof}
    
% \paragraph{Conclusion.} \textbf{First}, larger \( \max((\mathbf{P}^\top\mathbf{P})_{kk}) \) increases the bound on parameter matching error. \textbf{Second}, lower \( \sigma_{ck}^2 \) leads to smaller representation matching error, enhancing the overall condensation quality. These two factors are directly related to clustering quality. \textbf{Third}, a larger \( \lambda_{\min}(\mathbf{A}) \) decreases the bound, indicating that better-conditioned covariance matrices lead to tighter error bounds. \textbf{Finally}, higher \( \| \mathbf{F} \| \) and \( \| \mathbf{Y} \| \) contribute to larger error bounds. These two are predefined by the original dataset.

% \section*{Incremental vs.\ Batch-wise (Streaming) kmeans++}

% We have a dataset $\mathcal{X} = \{\mathbf{x}_1, \dots, \mathbf{x}_n\} \subset \mathbb{R}^d$ 
% and want $k$ centroids. Two procedures are common:

% \begin{enumerate}
%   \item \textbf{Standard kmeans++:} Select the first centroid $\mathbf{c}_1$ uniformly at random; 
%   for each subsequent centroid $\mathbf{c}_{t+1}$, 
%   use the distance-based probabilities 
%   $P(\mathbf{x}_i) = D(\mathbf{x}_i)^2 / \sum_{\mathbf{x}_j} D(\mathbf{x}_j)^2,$
%   where 
%   $D(\mathbf{x}_i) = \min_{1 \le s \le t}\|\mathbf{x}_i - \mathbf{c}_s\|.$
%   \vspace{0.2em}
%   \item \textbf{Incremental (or Streaming) kmeans++:} 
%   Split $\mathcal{X}$ into batches or handle data in an online manner. 
%   In each batch, we pick a subset of centroids using a similar \emph{kmeans++-like} rule, 
%   then \emph{merge} these partial centroids at the end, possibly running an extra final kmeans++ step on the merged set.
% \end{enumerate}

% \subsection*{1.\ Incremental Selection (One Centroid at a Time): Exact Equivalence}

% \paragraph{Algorithmic Description.} Instead of computing $D(\mathbf{x}_i)$ from scratch each time, we keep a \emph{running} (incremental) record of the distances:

% \begin{itemize}
%   \item Pick the first centroid $\mathbf{c}_1$ uniformly at random from $\mathcal{X}$.
%   \item Maintain for each point $\mathbf{x}_i$ a distance $D(\mathbf{x}_i) = \min_{1 \le s \le t}\|\mathbf{x}_i - \mathbf{c}_s\|^2$,
%         updated whenever a new centroid $\mathbf{c}_{t+1}$ is chosen:
%   \[
%     D(\mathbf{x}_i) \;=\; 
%       \min\Bigl(D(\mathbf{x}_i), \|\mathbf{x}_i - \mathbf{c}_{t+1}\|^2\Bigr).
%   \]
%   \item Sample the next centroid $\mathbf{c}_{t+1}$ with probability 
%   \[
%     P(\mathbf{x}_i) \;=\; 
%       \frac{D(\mathbf{x}_i)}{\sum_{\mathbf{x}_j \in \mathcal{X}} D(\mathbf{x}_j)},
%   \]
%   \item Repeat until we have $k$ centroids.
% \end{itemize}

% \paragraph{Proof of Equivalence.} 
% At each centroid-selection step $t$, the \emph{distance values} $\{D(\mathbf{x}_i)\}$ in the incremental method match exactly the corresponding distances in standard kmeans++, because in both cases 
% \[
%   D(\mathbf{x}_i) \;=\;
%   \min_{1 \le s \le t}\|\mathbf{x}_i - \mathbf{c}_s\|^2.
% \]
% Hence, both algorithms \emph{sample the next centroid} from \emph{the same probability distribution} over $\mathcal{X}$. 
% By induction on $t=1,\dots,k$, the two methods select the \emph{same} (or identically distributed) set of $k$ centroids. 
% Thus, \emph{incremental kmeans++} (one-at-a-time) is \textbf{exactly the same} as standard kmeans++ if they use the same random seed.

% \subsection*{2.\ Batch-wise (Streaming) kmeans++: Approximate Equivalence}

% In a streaming or large-scale setting, we may instead:
% \begin{enumerate}
%   \item Partition $\mathcal{X}$ into batches $\mathcal{X}_1, \mathcal{X}_2, \dots, \mathcal{X}_T$ (e.g.\ chunks or time windows).
%   \item On each batch $\mathcal{X}_t$, run a local kmeans++ to extract $k_t$ ``partial centroids'' 
%         $\{\mathbf{p}_1^{(t)}, \dots, \mathbf{p}_{k_t}^{(t)}\}$.
%   \item Merge all partial centroids $\{\mathbf{p}_j^{(t)}\}$ into one set, 
%         optionally weighting each centroid by the number of data points in its batch-cluster,
%         and run a final round of kmeans++ on this \emph{compressed} or merged set to extract $k$ global centroids.
% \end{enumerate}

% \paragraph{Why this is approximate.} In standard kmeans++, the distance probabilities are computed against \emph{all points} from \(\mathcal{X}\) at each centroid selection. 
% In the batch-wise approach, we only compute local distances within each batch and then later refine by clustering the merged centroids. 
% Hence, we generally cannot guarantee \emph{exact} equivalence: once data is partitioned, the cross-batch interactions are not fully accounted for in each local selection step. 

% However, one can show that if each batch-level selection is performed by kmeans++ and we preserve the relative weighting of partial centroids appropriately, 
% the final global clustering has a provable approximation factor relative to running a \emph{single} standard kmeans++ on all of \(\mathcal{X}\). 
% For instance, a common result in the coreset and streaming literature is that the final error (cost) of the merging approach is within a small constant factor (or $O(\log n)$ in the worst case) of the cost of a global optimum or a global kmeans++ initialization. 

% \paragraph{Sketch of Approximation Argument.}
% Suppose each batch $\mathcal{X}_t$ is of size $n_t$, with $\sum_{t=1}^\top n_t = n$. 
% Running local kmeans++ with $k_t$ partial centroids on batch $t$ yields a partial solution whose cost is close to the best $k_t$-clustering cost on that batch. 
% Collecting these partial centroids (with weights proportional to $n_t$ or the cluster sizes) forms a \emph{weighted coreset} \(\mathcal{P}\). 
% A final kmeans++ on \(\mathcal{P}\) yields $k$ global centroids. 
% By standard coreset arguments (and the properties of kmeans++ on weighted datasets), the final solution is guaranteed to be within a \emph{bounded factor} of the global kmeans++ cost on the full dataset. 
% Exact approximation ratios depend on how many centroids $k_t$ we extract per batch, the total number of batches, and the structure of the data. 
% Typical results show that for sufficiently large $k_t$ (relative to $k$) or well-chosen merges, 
% the final solution approximates a global kmeans++ solution to within constant or polylogarithmic factors.

% \subsection*{Summary}

% \begin{itemize}
%   \item \textbf{One-at-a-time incremental kmeans++ is exactly equivalent} to standard kmeans++ 
%   because the distance-based sampling distribution at each step is identical.

%   \item \textbf{Batch-wise (streaming) kmeans++ is generally an approximation} 
%   because each batch makes centroid choices without full knowledge of data in other batches. 
%   However, coreset/merging techniques plus a final global pass can ensure the final clustering 
%   is within a guaranteed approximation factor of the single-batch (standard) kmeans++ solution.
% % \end{itemize}
\section{Method Pipeline}
\begin{figure}[h]
    \centering
    \includegraphics[width=0.9\linewidth]{figs/main.pdf}
    \caption{Illustration of the GECC framework. The lower part of the figure depicts the two evolving settings: in the \textbf{inductive} setting, the graph structure evolves over time, while in the \textbf{transductive} setting, only the training nodes (labels) change, with the graph structure and non-training nodes remaining unchanged.}

    \label{fig:main}
\end{figure}

The proposed GECC framework is illustrated in Figure~\ref{fig:main}. GECC takes the current graph \( G_t = (\mathbf{X}_t, \mathbf{A}_t) \) along with the centroids from the previous time step to perform condensation and generate the condensed graph \( G'_t = (\mathbf{I}, \mathbf{C}'_t) \). The lower part of the figure also depicts the two evolving settings: in the \textbf{inductive} setting, the graph structure evolves over time, while in the \textbf{transductive} setting, only the training nodes (labels) change, whereas the graph structure and non-training nodes remain unchanged.



\section{Experimental Details}\label{app:exp}
\section{Dataset Generation}
\label{sec:dataset}
\revise{
To train the proposed GNN, we constructed a dataset of building structures and a subset of these structures were subjected to fire simulations using FEA. The dataset generation process is illustrated in \figref{fig:dataset_generation_procedure}. Initially, a total of 33,000 building structures with geometrical details, material properties, and gravity loads were created. Due to randomness in generating these structures, a filter is applied to remove unreasonable data after gravity load simulation, which included 15,377 structures. A trade-off between computational feasibility and model performance is made among the remaining 17,623 structures. As further labeling structures with MIDR requires resource-intensive fire simulations via OpenSeesRT, a large proportion of 16,050 structures is selected as unlabeled dataset. On the other hand, each of the other 1,573 structures was further subjected to 30 different fire simulations, forming the labeled dataset containing $1,573\times 30 = 47,190$ fire cases.} This section details the step-by-step process for generating the dataset, including geometry creation, material property assignment, and simulations due to gravity loads and fire scenarios. 
% To train the proposed neural network, we constructed a dataset comprising building structure data and a subset of fire scenario data. The dataset generation process is illustrated in \figref{fig:dataset_generation_procedure}. 
% A total of 33,000 building structures with geometric details, material properties, and gravity loads were initially created. Out of these, 3,000 structures were selected as labeled data, and the remaining 30,000 were designated as unlabeled data. Further, about half of them filtered out due to instability under gravity loads only. 
\begin{figure*}[h!]
    \centering
    \includegraphics[width=0.8\linewidth]{figures/dataset_filter_procedure.pdf}
    \caption{Workflow for dataset generation (geometry, material property, gravity loads, and fire scenarios).}
    \label{fig:dataset_generation_procedure}
\end{figure*}

\subsection{Geometry Generation}
\label{subsec:geometry_generation}
The geometry of the building structures forms the foundation of the dataset. Regular 
\revise{3D structures} resembling multi-story parking structures or shopping malls were generated, with parameters such as building floor dimensions and story heights selected randomly. Each building structure is composed of multiple rooms, which serve as the basic unit in this study. A room herein is a cuboid space defined by specific length, width, and height. Within a structure, rooms of the same dimensions are uniformly arranged along the length, width, and height, corresponding to the $x$-, $y$-, and $z$-axes, respectively. Structures vary in room size and number of rooms along each axis. Specifically, the room length, width, and height are independently sampled from a uniform distribution within the interval $[2, 5]$ meters along the three directions of the structure. Similarly, the room number along each axis is uniformly sampled independently as an integer within the interval $[2, 7]$, i.e., the maximum number of stories of the buildings simulated in this study is 7.

To introduce variability and simulate real-world scenarios, approximately $8\%$ of structural elements (beams or columns) are randomly removed after initial geometry creation. 
\revise{Such removal is not fire-induced damage, but reflects functional diversity often observed in real buildings, such as open spaces designed for activities in shopping malls, e.g., ice skating rinks. Examples of the generated geometries are illustrated in \figref{fig:example_generated_geometry}, showcasing the diversity and realism of the dataset. This element removal does not affect the definition of room's geometry in the structure and nor does it affect the number of considered fire scenarios.} 

\revise{A range of coefficient of variation values ($3.3\%$ to $17.5\%$) was derived from prior studies that investigated the statistics of geometrical and material properties of structural components of buildings (e.g., \cite{mirza1979variations, lee2004probabilistic}). These studies provide empirical data on the natural variability in parameters such as Young's modulus, yield strength, and dimensions of structural elements due to manufacturing tolerances and material inconsistencies. By selecting $8\%$ for the removal of structural elements in our database, we aimed to maintain a level of variability that is representative of real-world uncertainties while ensuring computational feasibility. This choice ensures that the database captures realistic deviations without introducing extreme cases that may not be commonly encountered in practice.}

\begin{figure*}[h!]
    \centering
    \includegraphics[width=\linewidth]{figures/example_generated_geometry.pdf}
    \caption{Examples of generated structural geometry of different sizes (all dimensions in meters).}
    \label{fig:example_generated_geometry} 
\end{figure*}

{\blockRevise

In this study, we opted for a deterministic square, dimension of $0.1$ m, solid cross-sectional steel elements due to their simplicity in modeling and analysis. Square sections exhibit uniform geometrical properties in all directions, simplifying the computation of structural responses and avoiding complications associated with more complex shapes, such as wide-flange sections, facilitating the computational efficiency and scalability to generate a large dataset. This choice also helps to mitigate issues related to stress concentrations and facilitates a more straightforward representation of structural behavior under thermal loads. 

\textit{Remark:} The selected cross-section provides a comparable flexural rigidity to a $W 130 \times 130 \times 28.1$ wide-flange section (metric units), albeit with significantly higher axial rigidity. This cross-section is acceptable for gravity-load-designed frames under service loading conditions where the models assume fully rigid, moment-resisting beam-column connections for the evaluation of the IDR under thermal loading. This assumption is reasonable in this computational study where the primary interest is to understand the global deformation response of frames under fire conditions. The selection of uniform square cross-sections for both beams and columns, rather than adherence to standard capacity design principles, was made here primarily for computational efficiency and to reduce design parameters in the database generation process. This choice allows for simplified and scalable approach to analyze the fire-induced response of generic steel frames without the need for large section variations, where this study mainly focuses on the fire vulnerability assessment using ML-based predictions. However, if additional loading conditions, e.g., seismic or wind loads, were to be considered, larger sections, strong-column/weak-beam principle, and ductile detailing would be required in the generated buildings for realistic structural behavior under combined loading conditions. Future studies may also consider investigating the influence of variable cross-sectional dimensions and semi-rigid connections on the structural performance under fire conditions. 
} % blockRevise

\subsection{Material Properties}
Steel is chosen as the material for the structures. To reflect real-world variations, we randomly assign one of five slightly different steel material types to each structural element. \revise{
The ranges of material properties are provided in \tabref{tab:material_property_ranges} and the properties are sampled from uniform distributions of the corresponding ranges. These variations simulate differences arising from manufacturing batches or regional material properties. That these properties are at ambient temperature and change when the temperature rises due to a fire. The selection of materials with varying properties is aimed at increasing the diversity of the data. Our goal is to represent as wide a range of data as possible with a limited amount of building structure data, thereby enhancing the generalization ability of the GNN. Our assumed material property ranges are expected to be wider than the real-world conditions based on findings in \cite{mirza1979variations, lee2004probabilistic}. Therefore, we are essentially tackling a more challenging and general task. If we can solve this problem, we are confident that our method will perform equally well or even better in real-world scenarios.
}
\begin{table}[h!]
    \centering
    \caption{Material properties ranges for considered steel structures.}
    \begin{tabular}{lc}
        \toprule
        Property & Range \\
        \midrule
        Young's modulus & [168, 252] GPa \\
        Yield strength & [220, 330] MPa \\
        Strain-hardening ratio & [0.8, 1.2] \% \\
        \bottomrule
    \end{tabular}
    \label{tab:material_property_ranges}
\end{table}

\subsection{Gravity Loads}
Gravity loads are applied to columns and beams based on their \revise{influence (tributary) areas as typically conducted in structural analysis. The considered ``service'' load conditions include the column self-weight and the additional loads directly supported on the beams from their self-weight and weights of the reinforced concrete slabs, people as live load, and building content. An edge beam typically carries approximately half the gravity load supported by a parallel interior beam}. The ranges of gravity loads are listed in \tabref{tab:gravity_load_ranges}. \revise{The loads are sampled from uniform distributions of the corresponding ranges.} Structures that failed to meet an MIDR threshold of $1\%$ under gravity loads were deemed unacceptable designs and filtered out, as such configurations of randomly chosen geometry, material, and gravity load combinations were considered unrealistic from a regulatory and practicality points of view.
\begin{table}[h!]
    \centering
    \caption{Gravity load ranges for considered beams and columns.}
    \begin{tabular}{lc}
        \toprule
        Element & Range (kN/m)  \\
        \midrule
        Column & [0.5, 1.0]  \\
        Edge beam & [1.5, 4.5]  \\
        Interior beam & [3.0, 7.5]  \\
        \bottomrule
    \end{tabular}
    \label{tab:gravity_load_ranges}
\end{table} 

\subsection{Rule-based Thermal Load Generation}
\label{subsec:thermal_load_generation}
To evaluate a building's structural response during a fire event, we employed a simplified rule-based approach for thermal load generation. 
% Previous studies \cite{nan_structuralfire_2023} have demonstrated that steel structures rapidly equilibrate with surrounding gases temperatures due to efficient heat exchange. Consequently, gas temperatures can be directly used as inputs for FEA tools, e.g., OpenSees, simplifying the process of modeling thermal loads. 
% Accurately simulating temperature fields in fire scenarios poses significant challenges. Advanced thermodynamic simulations, such as those performed using Fire Dynamics Simulator (FDS) \cite{mcgrattan_fire_2000}, provide precise temperature predictions. However, these methods are hindered by high computational costs, prolonging execution times, and limited scalability, making them impractical for generating large datasets. Additionally, real-world fire loads often display substantial spatial variability across different rooms \cite{dundar_fire_2023}, resulting in scenario-specific temperature fields with limited generalizability. For example, studies on bridge fires \cite{he_study_2024} have demonstrated that environmental factors, such as wind speeds, can significantly influence temperature distributions. Furthermore, even within identical scenarios, variations in fire modeling methodologies can produce distinctly different temperature fields \cite{zhang_temperature_2020, du_new_2012}. These challenges emphasize the need for efficient and adaptable methods to generate fire temperature data.
% To address these issues, we adopted a rule-based approach to model temperature variations. 
According to \cite{spearpoint_fire_2008}, a typical fire development follows a predictable pattern. During the {\em{growth stage}}, the temperature rises slowly and approximately linearly after ignition. This is followed by the {\em{flashover stage}}, where temperatures increase rapidly to peak values. After reaching the peak, the temperature either stabilizes or continues to rise slowly until the {\em{decay stage}} begins. Inspired by this fire development pattern, we describe the temperature evolution in time, $t$, prior to the decay stage in two distinct stages:
\begin{enumerate}
    \item {\bf{Initial linear increase stage}}: For $t \in [0, t_1)$, temperature increases gradually and linearly as the fire spreads through the building. This stage represents the time before the fire directly affects a structural element.  
    \item {\bf{ISO 834 fire curve stage}}: For $t \in [t_1, t_{\thre}]$, temperature rises rapidly following the ISO 834 curve \cite{ISO834}, modeling the direct impact of the fire on the structural element. 
\end{enumerate}
The slope of the linear temperature increase, $c$, and the transition time, $t_1$, are influenced by the spatial relationship between the fire source and the structural element. For the second stage of temperature evolution, we utilize the ISO 834 curve, a widely accepted standard for fire resistance testing. This standardized fire curve describes the temperature rise over time, enabling rapid and consistent thermal fields across various scenarios. The duration of fire simulation in this study is set to $t_{\thre}=60$ minutes. This value represents the upper limit for the temperature evolution of each structural element, providing a consistent basis for analyzing the structural response to fire.

Let $(x, y, z)$ represents the midpoint of a structural element and $(x_{\subfire}, y_{\subfire}, z_{\subfire})$ the fire source point. \revise{Integer parameters $h$ and $h_{\subfire}$ correspond to the respective floor levels of the element and the fire source}. The temperature evolution for each element is expressed as follows:
\begin{enumerate}
    \item Linear increase stage ($0 < t < t_1$):
    \begin{equation}
    T(t) = c \cdot t,
    \end{equation}
    where $c$, the rate of temperature increase ($^\circ\mathrm{C}/\mathrm{min}$), depends on the height difference between the element, $h$, and the fire source, $h_{\subfire}$:
    \begin{equation}
        c = 
        \begin{cases} 
        5\left/\left(h - h_{\subfire} + 1\right)\right., & h \geq h_{\subfire}, \\
        2\left/\left(h_{\subfire} - h\right)\right., & h < h_{\subfire}.
        \end{cases}
    \end{equation}
     \item ISO 834 stage ($t \geq t_1$):
\begin{equation}
    T(t) = c \cdot t_1 + 345 \log_{10} \left(8 \left(t - t_1\right) + 1\right).
\end{equation}
\end{enumerate}

The transition (arrival) time $t_1$, marking the end of the linear stage, depends on the spatial distance between the fire source and the element. We define the following two Euclidean distances $L_p$ in the $xy$ plane and $L_s$ in the $xyz$ space:
\begin{eqnarray}
L_p & \triangleq & \sqrt{(x - x_{\subfire})^2 + (y - y_{\subfire})^2}, \\
\label{eq:Lp}
L_s & \triangleq & \sqrt{(x - x_{\subfire})^2 + (y - y_{\subfire})^2 + (z - z_{\subfire})^2}.
\label{eq:Ls}
\end{eqnarray}
Accordingly, the transition time, $t_1$, is expressed as follows:
\begin{equation}
    t_1 = 
    \begin{cases}
    \beta_{1} \cdot \left(1 - \exp\left\{- L_s\left/\alpha_{1}\right.\right\}\right), & h > h_{\subfire}, \\
    \beta_{2} \cdot \left(1 - \exp\left\{- L_p\left/\alpha_{2}\right.\right\}\right), & h = h_{\subfire}, \\
    \beta_{3} \cdot \left(1 - \exp\left\{- L_s\left/\alpha_{3}\right.\right\}\right), & h < h_{\subfire} .
    \end{cases}
    \label{eq:t1}
\end{equation}
The parameters $\beta_i$ and $\alpha_i$ for determining $t_1$ are summarized in Table~\ref{tab:fire_spread_parameters}. In this study, we take $r_{\mathrm{up}}=0.95$ and $r_{\mathrm{down}}=0.97$.
\begin{table}[ht]
    \centering
    \caption{Fire spread parameters for $t_1$ calculations.}
    \begin{tabular}{lcc}
        \toprule
        Case  & $\beta_i$ & $\alpha_i$  \\
        \midrule
        $i=1$, Upward spread & $16 \left.\left(1-r_{\mathrm{up}}^{\left|h-h_{\subfire}\right|}\right)\right/\left(1-r_{\mathrm{up}}\right)$ & $10$  \\
        $i=2$, Horizontal spread & $18$ & $18$  \\
        $i=3$, Downward spread & $30 \left.\left(1-r_{\mathrm{down}}^{\left|h-h_{\subfire}\right|}\right)\right/\left(1-r_{\mathrm{down}}\right)$ & $5$  \\
        \bottomrule
    \end{tabular}
    \label{tab:fire_spread_parameters}
\end{table}

\figref{fig:t1_curve} illustrates the $t_1$ curves for various fire scenarios: (1) fire originating on the lower floor, $h-h_{\subfire}=1$ with rapid upward spread, (2) fire on the same floor, $h=h_{\subfire}$ with the fastest spread, and (3) fire on the upper floor, $h_{\subfire}-h=1$ with slow downward spread. The exponential decay in $t_1$ reflects the accelerating fire propagation speed as the distance increases. \figref{fig:t1_curve} also indicates that the employed simplified model is consistent with the Markov chain-based dynamic model given by \cite{cheng_dynamic_2011}, where the rooms at the same floor of the fire point start flashover slightly before the corresponding upper floors. Additionally, $\beta_{1}$ and $\beta_{3}$ are the summation of a geometric sequence, where story level $h$ is the index. The common ratios $r_{\mathrm{up}}<1$ in $\beta_{1}$ and $r_{\mathrm{down}}<1$ in $\beta_{3}$ indicate that the fire speeds up to spread through the next story, which is consistent with the real-world fire spread mechanism given in \cite{hokugo_mechanism_2000}. The temperature profile within the range $t \in [0, t_{\thre}]$ is subsequently used as the thermal load in OpenSeesRT simulations to compute displacements at each structural node at time $t_{\thre}$.
\begin{figure}[h!]
    \centering
    \includegraphics[width=0.8\linewidth]{figures/m204_t1_curve.pdf}
    \caption{Three examples for the $t_1$ curve.}
    \label{fig:t1_curve}
\end{figure}

\revise{
\textit{Remark:} The effects of structural elements, such as concrete floor slabs and partitions, are not explicitly modeled in our approach. Instead, their influence is implicitly captured through the careful selection of the parameters $ \alpha, \beta, r_\mathrm{up} $, and $ r_\mathrm{down} $. This parameterization provides a unified framework for generating temperature fields. Indeed, fire propagation is governed by a multitude of factors and remains an open research question. For instance, if the fire resistance of a floor slab is enhanced by fire protective coating, the corresponding model can account for this by decreasing $\alpha_1$ \& $\alpha_3$, increasing $\beta_1$ \& $\beta_3$, and adopting larger values for $r_\mathrm{up}$ \& $r_\mathrm{down}$, which collectively slow down the vertical spread of fire. Conversely, scenarios involving higher amounts of combustible materials would warrant the opposite adjustments. This flexible and integrated approach avoids the need to design separate models for different fire propagation scenarios while still capturing the essential effects.
}

\revise{
In conclusion, our rule-based approach is a computationally efficient method for approximating fire temperature fields, enabling large-scale dataset generation to train predictive models. By combining ISO 834 fire curves with spatial considerations and embedding structural effects through parameter calibration, the method achieves a balanced trade-off between accuracy and scalability, making it a practical solution for thermal load modeling in fire scenarios. After generating the temperature of each beam or column according to the middle point, the temperature is applied as uniform thermal load to the elements of the structure in question using OpenSeesRT. 
}

% In conclusion, this rule-based approach is a computationally efficient method to approximate fire temperature fields, enabling large-scale dataset generation to train predictive models. By combining ISO 834 fire curves with spatial considerations, the method balances accuracy and scalability, making it a practical solution for thermal load modeling in fire scenarios.

% \subsection{Interstory Drift Ratio}
\subsection{OpenSeesRT Simulation}
\label{subsec:opensees_simulation}

The thermal and mechanical responses of 3D frame structures under combined fire and gravity loads are simulated using OpenSeesRT \cite{perez2024openseesrt}. \revise{In the simulation, the IDR of each node at $t_{\thre}$ is computed using the computed nodal displacements. Each structural model features six degrees of freedom per node (3 translational  and 3 rotational), with linear geometrical transformations (\texttt{geomTransf: Linear}) defining how the element local coordinate systems are mapped to the global coordinate system and assuming small displacements and rotations. Although OpenSeesRT allows a variety of options for modeling finite deformations, in the present simulations and mainly for simplicity, we did not consider large deformations. All bottom nodes (nodes on the ground) are fully constrained in all six degrees of freedom, while degrees of freedom os all other nodes are free.} Material behavior is temperature-dependent and modeled with \texttt{Steel01Thermal}, while fiber-based sections (\texttt{FiberThermal}) capture nonlinear interactions between thermal and mechanical responses at the cross-section level. \revise{Structural elements are represented as displacement-based Euler-Bernoulli beam-columns (\texttt{dispBeamColumnThermal}). This element  formulation accounts for thermal strains (temperature gradients) in the section, which is discretized into fibers. Numerical integration is used along the length of each element using three integration (Gauss) points, one at each end and the third in the middle of the element.}

{\revise{Thermal expansion of steel members plays a crucial role in IDR development. In reality, reinforced concrete floor slabs heat at a different rate than steel members due to their higher thermal mass and lower thermal conductivity. This differential heating can lead to restrained thermal expansion, introducing axial compression in beams and affecting the overall structural response. In this study, explicit {\em{composite action}} between steel members and concrete slabs is not modeled. Instead, our approach focuses on isolating the response of the steel structural frame, which is often the critical load-bearing component in fire scenarios. This assumption aligns with prior studies \cite{Possidente_2024} demonstrating that steel structures reach thermal equilibrium with surrounding gases quickly, allowing the use of uniform thermal loading in fire analysis. Future work could enhance this framework by incorporating slab-beam interaction effects, through a refined FEA for an extended dataset where constraints imposed by floor slabs are explicitly considered.}

The analysis begins with the application of gravity loads, followed by incremental thermal loads simulating the fire exposure. A static nonlinear solver using  \texttt{ExpressNewton} algorithm ensures convergence, while the \texttt{NormDispIncr} test maintains accuracy. An incremental \texttt{LoadControl} scheme with small step sizes is employed to guarantee numerical stability, using 10\% for gravity loads and 1\% for thermal loads. 

\revise{
In the thermal load analysis, uniform thermal load is applied to each beam or column, i.e., the temperature of each element is set to be that at the middle point, according to \secref{subsec:thermal_load_generation}. The \texttt{Steel01Thermal} material allows the properties (e.g., Young's modulus and yield strength) to be adjusted at increasing temperatures according to \cite{EN1993} using its Table 3.1: Reduction factors for the stress-strain relationship of carbon steel at elevated temperatures. For example, if the Young’s modulus at ambient temperature is $E_0$, then as the temperature ($T$) increases, the modulus changes as $E(T) = \eta (T) \times E_0$. \cite{EN1993} directly provides the values of $\eta(T) \in \left[0,1\right] $ at every $100 ^\circ\mathrm{C}$ interval and recommends using linear interpolation to obtain $\eta(T)$ for intermediate values of $T$.
} OpenSeesRT documentation \cite{OpenSeesThermalExamples} provides several examples of thermal analyses.

This modeling framework accommodates variations in material properties, cross-sectional geometries, and temperature profiles, providing robust simulations of structural behavior under fire conditions. The primary settings and configurations for the OpenSeesRT simulations are summarized in \tabref{tab:ops_detail}.
\begin{table}[h!]
    \centering
        \caption{Key settings of OpenSeesRT simulations.}
    \begin{tabular}{l|>{\raggedright\arraybackslash}p{0.6\linewidth}} %
    \toprule
    Modeling Aspect     & Details \\
    \midrule
    Geometry            & 3D models; 6 degrees of freedom per node \\
    Transformation      & geomTransf: Linear \\ 
    Material            & Steel01Thermal \\
    Section             & FiberThermal; Cross-section: $0.1$ m $\times$ $0.1$ m \\ 
    Element type        & {dispBeamColumnThermal} \\ 
    Loading             & Gravity loads: {beamUniform}; Thermal loads: {beamThermal} \\
    Integration scheme  & Incremental {LoadControl}; Step size: $10\%$ (gravity analysis), $1\%$ (thermal analysis) \\
    Nonlinear solver    & {ExpressNewton} algorithm; {UmfPack} solver; Convergence test: {NormDispIncr} tolerance: $10^{-8}$; Maximum \# iterations per step: $1000$. \\ 
    \bottomrule
    \end{tabular}
    \label{tab:ops_detail}
\end{table}

For each structure in the labeled dataset, 30 fire points are selected using a dual-granularity approach, \revise{i.e., two-stage sampling strategy,} to ensure they are well-distributed. Specifically, rooms are sequentially selected, with one fire point randomly chosen within each selected room. If a building is large and contains more than 30 rooms, we randomly select 30 rooms without replacement, i.e., ensuring that no more than one fire point is located in the same room. Conversely, if the building is small and has fewer than 30 rooms, all rooms are initially selected, with one fire point randomly assigned to each room. Additionally, rooms are then selected with replacement until a total of 30 fire points are assigned. \revise{The room-level sampling prioritizes selecting distinct rooms to avoid spatial clustering of fire points, while the point-level sampling ensures intra-room variability. This approach aligns with stratified sampling principles commonly used for efficient spatial representation, where multi-stage sampling strategies optimize coverage and variability, e.g., \cite{arunachalam_generalized_2023}, and enables a more comprehensive characterizing of how the structures respond under fire conditions.}
% This selection method prevents fire points from clustering too closely while maintaining an element of randomness. By distributing fire points in this manner, the 30 fire scenarios are effectively utilized, enabling a more comprehensive characterizing of how the structures respond under fire conditions.

\subsection{Summary of the Dataset Generation}
As discussed in this section and related to  \figref{fig:dataset_generation_procedure}, three key steps were considered in the development of the dataset: 
\begin{enumerate}
    \item {\bf{Filtering process}}: Structures with MIDR exceeding $1\%$ under gravity loads were excluded,  resulting in $1,573$ labeled structures retained for fire simulation and $16,050$ unlabeled structures for training the MFSP predictor.
    \item {\bf{Fire simulations}}: For each retained labeled structure, 30 fire scenarios were simulated using OpenSeesRT, yielding $47,190$ fire cases.
    \item {\bf{Data distribution check}}: MIDR distributions for labeled and unlabeled data under gravity loads were highly similar, because both datasets were generated using the same method. Under fire conditions, the MIDR distribution shifted, reflecting significant structural deformation with values reaching a maximum of about 6\%, an average of 1.70\%, and a standard deviation of 1.12\%. This step ensured a diverse and comprehensive dataset for the proposed predictive framework.
\end{enumerate}
The statistical distribution histograms for MIDR (after applying the $1\%$ filtering threshold \revise{for gravity load responses}) under different loading conditions are plotted in \figref{fig:histogram_mdr}. Figures \ref{fig:histogram_mdr}(a) and \ref{fig:histogram_mdr}(b) show the MIDR distributions of the labeled and unlabeled data, respectively, under gravity loads only. \figref{fig:histogram_mdr}(c) shows the MIDR distribution of the labeled data under the combined effects of gravity and fire loads. Fire load causes the structures to significantly deform, leading to a noticeably \revise{right-skewed} MIDR distribution.

\begin{figure*}[h!]
    \centering
    \includegraphics[width=\linewidth]{figures/histogram_mdr.pdf}
    \caption{Histograms of MIDR for labeled and unlabeled structures with gravity loads and fire cases.}
    \label{fig:histogram_mdr}
\end{figure*}

\revise{
This dataset provides the basis for training and testing the performance of the GNN-based framework. Although we employed a simplified rule-based thermal load generation method compared with conventional CFD-based simulations, the temperature field, the changes of the material properties, and the response of the structures, are all still highly nonlinear and complex. Therefore, it is still a challenging task for the NN to predict the MIDRs based on this dataset.
}
\subsection{Dataset Statistics}
\label{app:statistics}
In line with most GC studies, we utilize seven datasets in total: five transductive datasets—\textit{Citeseer}, \textit{Cora} \citep{kipf2016semi}, Pubmed \citep{namata2012query}, \textit{Ogbn-arxiv}, and \textit{Ogbn-products} \citep{hu2020open}—and two inductive datasets, \textit{Flickr} and \textit{Reddit} \citep{zeng2019graphsaint}. Each graph is randomly split, ensuring a consistent class distribution. The details of the datasets statistics are shown in Table \ref{tab:statistics}. We list all evolving information in Table~\ref{tab:split_reduction}, rows above the midline correspond to smaller datasets, and rows below it correspond to larger ones.
Reduction rate $r$ is defined as (\#nodes in synthetic set)/(\#nodes in training set) while $r_w$ is (\#nodes in synthetic set)/(\#nodes of whole graph visible in training stage). The whole graph visible in training stage means the full graph dataset for transductive setting but only the training graph for inductive setting.
\begin{table}[ht!]
\caption{Split and reduction rate information. The ``\# Train Nodes'' and ``\# Syn Nodes'' columns denote the number of newly added training nodes and synthetic nodes at each time step, respectively.}
\label{tab:split_reduction}
\resizebox{0.47\textwidth}{!}{
\begin{tabular}{lrrrr}
\toprule
\textbf{Dataset} & 
\textbf{\# Train Nodes} & 
\textbf{\# Syn Nodes} & 
\textbf{$r$ (Train)} & 
\textbf{$r_w$ (Whole)} \\
\midrule
\textit{Citeseer}      & 24             & 12  & 0.5    & 1.80  \\
\textit{Cora}          & 28             & 14  & 0.5    & 2.60  \\
\textit{Pubmed}        & 12             & 6   & 0.5    & 0.15  \\
\midrule
\textit{Flickr}        & ~8{,}920 & 90  & 0.01   & 1.00  \\
\textit{Ogbn-arxiv}    & ~18{,}190 & 182 & 0.01   & 0.50  \\
\textit{Ogbn-products} & ~39{,}330 & 394 & 0.01   & 0.08  \\
\textit{Reddit}        & ~30{,}790 & 31  & 0.001  & 0.10  \\
\bottomrule
\end{tabular}}
\end{table}

\subsection{Platform and Hardware Information}
To efficiently execute the clustering algorithm, we run it on Intel(R) Xeon(R) Platinum 8260 CPUs @ 2.40GHz using NumPy~\cite{numpy}, while the downstream GNN evaluations are conducted on a cluster equipped with a mix of Tesla A100 40GB/V100 32GB GPUs for large datasets and K80 12GB GPUs for small datasets. All GNN models are implemented using the PyG package~\cite{pyg}.

\subsection{Baselines Selection}
To establish a fair benchmark, we selected recent state-of-the-art GC methods that emphasize both effectiveness and efficiency. Some recent methods, such as MCond, CGC, and GCPA, were excluded due to the unavailability of their code at the time of paper writing. For the selected approaches, we chose the best representatives from each category: GCondX for gradient matching, GCDM and SimGC for distribution matching, and GEOM for trajectory matching. We implemented these methods using the latest GraphSlim package\footnote{\url{https://github.com/Emory-Melody/GraphSlim/tree/main}}, except for SimGC\footnote{\url{https://github.com/BangHonor/SimGC}} and GEOM\footnote{\url{https://github.com/NUS-HPC-AI-Lab/GEOM/tree/main}}, for which we used their original source code. We specifically included SimGC because it is the only model-based GC method that can run on Ogbn-products without requiring any modifications.


\subsection{Implementation Details for Variants of GCond}
As mentioned in Section~\ref{sec:intro} and illustrated in Table~\ref{tab:preliminary}, adapting GCond to an evolving setting is challenging. We employ the structure-free variant of GCond, i.e., GCondX, for easier adaptation, as designing a specific growth mechanism for the condensed graph is nontrivial and requires significant effort.
In addition, to manifests the convergence speed difference between GCond and GCond-Init, we implement an early stopping criterion with a patience of 3 during intermediate evaluations. 
If no improvement in validation performance is observed over 3 consecutive evaluations, the condensation process is terminated.
\subsection{Hyperparameters}\label{app:hyper}
Compared to existing work and benchmarks in GC, we perform a moderate hyperparameter search on validation set, as detailed in Section~\ref{sec:hyper}. The final results are presented in Table~\ref{tab:hyper}. During hyperparameter optimization (HPO), we observe that inheriting clustering centroids results in an approximate 1\% absolute performance drop for \textit{Flickr} and \textit{Ogbn-arxiv}. Therefore, we also treat the use of incremental $k$-means++ as a tunable hyperparameter. Additionally, some datasets do not perform well with a single hyperparameter configuration. To address this, we employ two distinct hyperparameter sets tuned on the first and last time steps, respectively, and select the better-performing one during the evolution process. These two sets are represented using a "/" separator and are indicated as "if Dual" when this technique is applied. For all baselines, we use the best hyperparameters reported in their respective papers, as implemented in GC4NC~\cite{gong2024gc4nc}.

The optimal hyperparameters also offer meaningful insights. \textbf{First}, during the early evolution stage, graphs exhibit higher heterophily compared to later stages. For example, on the \textit{Cora} dataset, $\alpha_1=-0.3$ in the early phase contrasts with $\alpha_1=0.9$ later. This pattern likely arises because, in the early stages of a graph, groups have not yet formed; links appear more randomly, making it challenging for nodes to link to similar counterparts.
\textbf{Second}, it is noteworthy that some datasets do not rely on second-hop information. This observation is contrary to previous studies~\cite{wu2019simplifying,luo_classic_2024} that recommend using at least 2-hop propagation. We conjecture that the representation clustering process itself acts as an additional step of feature propagation.
\textbf{Finally}, weight decay emerges as a critical factor for the performance of downstream models, suggesting that future work should pay closer attention to its optimization.
\begin{table*}[ht!]
\centering
\caption{The test accuracy of GC methods on various datasets.
"Non-Evolving" displays the test accuracy at the final time step (largest possible graph).
"Evolving" shows the average test accuracy over five time-steps.
Each result includes the mean accuracy $\pm$ standard deviation (Std.) from 10 runs. The "Whole" column refers to the results obtained by running standard GCN training and testing. "OOM" indicates an Out-of-Memory error during the computation. The best results are marked in \textbf{bold}. The runner-up results are \underline{underlined}.}
% \vskip -1em
\resizebox{\textwidth}{!}{%
\begin{tabular}{lc|ccc|cccccc|c}
\toprule
\textbf{Dataset} & \textbf{Setting} 
  & \textbf{Random} 
  & \textbf{Herding} 
  & \textbf{Kcenter} 
  & \textbf{GCondX} 
  & \textbf{GCond} 
  & \textbf{GCDM} 
  & \textbf{SimGC}
  & \textbf{GEOM} 
  & \textbf{GECC} 
  & \multicolumn{1}{c}{\textbf{Whole}} \\
  \midrule
\multirow{2}{*}{CiteSeer} 
  &Non-Evolving& 62.62$\pm$0.63 & 66.66$\pm$0.54 & 59.04$\pm$0.90 & 68.38$\pm$0.45 & 69.35$\pm$0.82 & 72.08$\pm$0.19 & 66.40$\pm$0.15 &  \textcolor{red}{\underline{73.03$\pm$0.31}} & \textcolor{red}{\textbf{73.25$\pm$0.15}} & 72.11 \\
  &Evolving& 50.65$\pm$1.55 & 53.47$\pm$0.98 & 47.99$\pm$1.81 & 50.85$\pm$3.00 & 60.51$\pm$0.86 & \textcolor{blue}{\underline{61.51$\pm$0.53}}& 57.42$\pm$0.21 & 58.95$\pm$0.67 & \textcolor{blue}{\textbf{65.48$\pm$0.76}} & 63.57 \\
\midrule
\multirow{2}{*}{Cora} 
  &Non-Evolving& 72.24$\pm$0.59 & 73.77$\pm$0.93 & 70.55$\pm$1.35 & 78.60$\pm$0.31 & 80.54$\pm$0.67 & 80.68$\pm$0.27 & 79.60$\pm$0.11 & \textcolor{red}{\underline{82.82$\pm$0.17}} & \textcolor{red}{\textbf{82.99$\pm$0.27}} & 81.23 \\
  &Evolving& 58.00$\pm$1.48 & 63.07$\pm$1.43 & 59.90$\pm$1.41 & 67.18$\pm$1.73 & \textcolor{blue}{\underline{77.14$\pm$0.55}} & 74.54$\pm$0.59 & 64.42$\pm$0.19 & 72.56$\pm$0.88 & \textcolor{blue}{\textbf{77.36$\pm$0.41}} & 76.34 \\
\midrule
\multirow{2}{*}{Pubmed} 
  &Non-Evolving& 71.84$\pm$0.66 & 75.53$\pm$0.44 & 74.00$\pm$0.19 & 71.97$\pm$0.53 & 76.46$\pm$0.48 & 77.48$\pm$0.46 & 76.80$\pm$0.23 & \textcolor{red}{\underline{78.49$\pm$0.24}} & \textcolor{red}{\textbf{80.24$\pm$0.27}} & 78.65 \\
  &Evolving& 66.37$\pm$1.25 & 66.31$\pm$1.34 & 64.38$\pm$1.25 & 62.65$\pm$1.20 & 74.26$\pm$0.84 & \textcolor{blue}{\underline{74.49$\pm$0.56}} & 71.38$\pm$0.21 & 70.25$\pm$0.78 &  \textcolor{blue}{\textbf{76.74$\pm$0.27}} & 76.18 \\
\midrule
\multirow{2}{*}{Flickr} 
  &Non-Evolving& 44.68$\pm$0.55 & 45.12$\pm$0.39 & 43.53$\pm$0.59 & 46.58$\pm$0.14 & \textcolor{red}{\textbf{46.99$\pm$0.12}} & 45.88$\pm$0.10 & 41.01$\pm$0.23 & 46.13$\pm$0.22 & \textcolor{red}{\underline{46.63$\pm$0.23}} & 47.53 \\
  &Evolving& 44.70$\pm$0.46 & 44.66$\pm$0.43 & 44.33$\pm$0.49 & \textcolor{blue}{\underline{45.63$\pm$0.78}}& 45.52$\pm$0.49 & 44.98$\pm$0.34 & 41.94$\pm$0.22 & 45.43$\pm$0.39 &  \textcolor{blue}{\textbf{45.78$\pm$0.38}} & 46.97 \\
  \midrule
  \multirow{2}{*}{Ogbn-arxiv} 
  &Non-Evolving& 60.19$\pm$0.52 & 57.70$\pm$0.24 & 58.66$\pm$0.36 & 59.93$\pm$0.54 & 64.23$\pm$0.16 & 60.71$\pm$0.68 & 65.26$\pm$0.26 &  \textcolor{red}{\textbf{69.59$\pm$0.24}} & \textcolor{red}{\underline{66.71$\pm$0.10}} & 69.01 \\
  &Evolving& 56.04$\pm$0.67 & 57.57$\pm$0.48 & 56.21$\pm$0.73 & 60.73$\pm$0.53 & 62.50$\pm$0.36 & 59.98$\pm$0.48 & 64.97$\pm$0.20 & \textcolor{blue}{\textbf{66.30$\pm$0.39}} & \textcolor{blue}{\underline{65.42$\pm$0.14}} & 70.40 \\
\midrule
\multirow{2}{*}{Ogbn-products} 
  &Non-Evolving& 60.19$\pm$0.52 & 57.70$\pm$0.24 & 58.66$\pm$0.36 & OOM & OOM & OOM & \textcolor{red}{\underline{61.71$\pm$0.25}} & OOM & \textcolor{red}{\textbf{66.32$\pm$0.23}} & 73.40 \\
  &Evolving& 41.36$\pm$0.48 & 44.26$\pm$0.61 & 38.93$\pm$0.82 & OOM & OOM & OOM & \textcolor{blue}{\underline{61.93$\pm$0.20}} & OOM & \textcolor{blue}{\textbf{64.03$\pm$0.30}} & 73.88 \\
\midrule
\multirow{2}{*}{Reddit} 
  &Non-Evolving& 55.73$\pm$0.50 & 59.34$\pm$0.70 & 48.28$\pm$0.73 & 88.25$\pm$0.30 & 89.82$\pm$0.10 & 89.96$\pm$0.05 & 90.78$\pm$0.25 & \textcolor{red}{\underline{91.33$\pm$0.13}} & \textcolor{red}{\textbf{91.37$\pm$0.04}} & 93.70 \\
  &Evolving& 51.31$\pm$0.90 & 48.94$\pm$0.70 & 48.53$\pm$1.37 & 79.02$\pm$0.73 & 87.93$\pm$0.22 & 82.68$\pm$0.21 &\textcolor{blue}{\underline{89.85$\pm$0.25}} & 67.91$\pm$0.57 & \textcolor{blue}{\textbf{90.02$\pm$0.07}} & 93.92 \\
\bottomrule
\end{tabular}}
\label{tab:main_app}
\end{table*}
\begin{table*}[ht!]
\centering
\caption{\textbf{Average Runtime (seconds) Across Evolving Times.} The reported reduction time is rigorously computed by excluding the overhead of the data loading and evaluation processes.}
% \vskip -1em
\label{tab:time}
\resizebox{\textwidth}{!}{
\begin{tabular}{lccc|cccccc|c}
\toprule
\textbf{Dataset} & \textbf{Random} & \textbf{Herding} & \textbf{KCenter} & \textbf{GCondX}& \textbf{GCond}& \textbf{GCDM} & \textbf{SimGC}& \textbf{GEOM}& \textbf{GECC} & \textbf{Whole}\\
\midrule
\textit{Citeseer}
  & 0.04
  & 5.73
  & 5.84
  & 505.62
& 654.32
& 217.99
  & 1,680.02
& 1,362.40
& 1.65
& 3.98 \\

\textit{Cora}
  & 0.01
  & 4.20
  & 4.80
  & 331.53
& 1,190.65
& 142.82
  & 1,643.76
& 1,331.43
& 1.72
& 2.10 \\

\textit{Pubmed}
  & 0.02
  & 9.00
  & 7.18
  & 246.68
& 502.12
& 311.37
  & 1,654.23
& 995.21
& 1.42
& 5.76 \\ \midrule

\textit{Flickr}
  & 0.02
  & 11.53
  & 10.56
  & 609.98
& 1,446.76
& 353.51
  & 7,486.65
& 757.75
& 7.10
& 8.57 \\

\textit{Ogbn-arxiv}
  & 0.02
  & 14.36
  & 14.05
  & 2,895.06
& 6,076.18
& 686.12
  & 2,687.45
& 1,685.18
& 9.96
& 12.45 \\

\textit{Ogbn-products}
  & 0.02
  & 517.95
  & 513.36
  & OOM
& OOM
& OOM
  & 71,489.00
& OOM
& 146.82
& 542.61 \\

\textit{Reddit}
  & 0.02
  & 24.40
  & 24.84
  & 2,672.85& 6,130.46& 337.15
  & 6,610.70& 1,815.77& 4.91& 11.50 \\

\bottomrule
\end{tabular}}
\end{table*}
\begin{table*}[]
\centering
\resizebox{\textwidth}{!}{%
\begin{tabular}{@{}lcccccc@{}}
\toprule[1.5pt]
 & Model & Learning Rate  & Batch Size & KL Coefficient&Max Length & Training Epochs \\ 
\midrule[1pt]
& Llama-3.1-8B-Instruct & 5e-6  & 32 & 0.1&8000& 3\\
& Qwen2-7B-Instruct & 5e-6 & 32 & 0.1 &6000& 3 \\
& Qwen2.5-Math-7B & 5e-6  & 32 & 0.01&8000& 3 \\ 
\bottomrule[1.5pt]
\end{tabular}%
}
\caption{Model Training Hyperparameter Settings (SFT)}
\label{tab:hyper_sft}
\end{table*}

\begin{table*}[]
\centering
\resizebox{\textwidth}{!}{%
\begin{tabular}{@{}lccccccccc@{}}
\toprule[1.5pt]
 & Model & Learning Rate  & \makecell[c]{Training\\Batch Size} & \makecell[c]{Forward\\Batch Size} & KL Coefficient&Max Length & \makecell[c]{Sampling\\Temperature} &Clip Range &Training Steps \\ 
\midrule[1pt]
& Llama-3.1 &5e-7  & 64& 256 & 0.05&8000& 0.7&0.2&500\\
& Qwen2-7B-Instruct & 5e-7&  64& 256 & 0.05 &6000&0.7 &0.2&500\\\
& Qwen2.5-Math-7B & 5e-7 & 64& 256 & 0.01&8000&0.7 &0.2&500 \\ 
\bottomrule[1.5pt]
\end{tabular}%
}
\caption{Model Training Hyperparameter Settings (RL)}
\label{tab:hyper_rl}
\end{table*}

\section{Additional Results}
\subsection{Performance and Efficiency}
For simplicity, Table~\ref{tab:main} omits the standard error and running time of coreset selection methods. we provide the full results here.

Figure~\ref{fig:accuracy_vs_time} presents the accuracy vs. time trade-off for Reddit. For the remaining three large datasets, we provide the corresponding results in Figure~\ref{fig:accuracy_vs_time_large_vertical}. 
The results align with our main findings, further confirming that GECC surpasses the baselines in both efficiency and scalability. It consistently maintains stable performance while effectively managing computational resources throughout graph evolution. Notably, on the large-scale Ogbn-products dataset, which contains over one million nodes, most GC methods fail, whereas GECC remains robust and continues to operate successfully.


\begin{figure*}[htbp]
  \centering
  \begin{subfigure}[b]{0.85\linewidth}
    \centering
    \includegraphics[width=0.8\linewidth]{figs/ogbn-arxiv_time_vs_accuracy-cropped.pdf}
    % \caption{Flickr Caption}
    \label{fig:flickr}
  \end{subfigure}
  % \vspace{-1em}
  \begin{subfigure}[b]{0.85\linewidth}
    \centering
    \includegraphics[width=0.8\linewidth]{figs/Ogbn-products_time_vs_accuracy-cropped.pdf}
    % \caption{Ogbn-products Caption}
    \label{fig:ogbnproducts}
  \end{subfigure}
  % \vspace{-1em}
  \begin{subfigure}[b]{0.85\linewidth}
    \centering
    \includegraphics[width=0.8\linewidth]{figs/Reddit_time_vs_accuracy-cropped.pdf}
    % \caption{Reddit Caption}
    \label{fig:redproducts}
  \end{subfigure}
  % \vskip -2em
  \caption{Test accuracy vs. condensation time on large datasets (top-left is better).}
  \label{fig:accuracy_vs_time_large_vertical}
\end{figure*}


% \begin{figure*}[htbp]
%   \centering
%   \begin{subfigure}[b]{0.33\linewidth}
%     \centering
%     \includegraphics[width=\linewidth]{figs/ogbn-arxiv_time_vs_accuracy-cropped.pdf}
%     % \caption{Flickr Caption}
%     \label{fig:flickr}
%   \end{subfigure}%
%   \hfill
%   \begin{subfigure}[b]{0.33\linewidth}
%     \centering
%     \includegraphics[width=\linewidth]{figs/Ogbn-products_time_vs_accuracy-cropped.pdf}
%     % \caption{Ogbn-products Caption}
%     \label{fig:ogbnproducts}
%   \end{subfigure}%
%   \hfill
%   \begin{subfigure}[b]{0.33\linewidth}
%     \centering
%     \includegraphics[width=\linewidth]{figs/Reddit_time_vs_accuracy-cropped.pdf}
%     % \caption{}
%     \label{fig:redproducts}
%   \end{subfigure}
%   \vskip -2em
%   \caption{Test accuracy vs. condensation time on large datasets (top-left is better).}
%   \label{fig:accuracy_vs_time_large}
% \end{figure*}
%%%%%%%%%%%%%%%%%%%%%%%%%%%%%%%%%%%%%%%%%%%%%%%%%%%%%%%%%%%%%%%%%%%%%%%%%%%%%%%
%%%%%%%%%%%%%%%%%%%%%%%%%%%%%%%%%%%%%%%%%%%%%%%%%%%%%%%%%%%%%%%%%%%%%%%%%%%%%%%


\end{document}


% This document was modified from the file originally made available by
% Pat Langley and Andrea Danyluk for ICML-2K. This version was created
% by Iain Murray in 2018, and modified by Alexandre Bouchard in
% 2019 and 2021 and by Csaba Szepesvari, Gang Niu and Sivan Sabato in 2022.
% Modified again in 2023 and 2024 by Sivan Sabato and Jonathan Scarlett.
% Previous contributors include Dan Roy, Lise Getoor and Tobias
% Scheffer, which was slightly modified from the 2010 version by
% Thorsten Joachims & Johannes Fuernkranz, slightly modified from the
% 2009 version by Kiri Wagstaff and Sam Roweis's 2008 version, which is
% slightly modified from Prasad Tadepalli's 2007 version which is a
% lightly changed version of the previous year's version by Andrew
% Moore, which was in turn edited from those of Kristian Kersting and
% Codrina Lauth. Alex Smola contributed to the algorithmic style files.
