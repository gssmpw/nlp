% \newpage
\section{DALLE-2 visible watermark}
\label{App: DALLE-2 Example}
\begin{SCfigure}[][!htbp]
\centering
\includegraphics[width=0.4\textwidth]{source-figures/DALLE-vis/Dalle-vis.png}%
\caption{An example image generated by DALLE-2 from the official website: \url{https://openai.com/index/dall-e-2/}, where the color code watermark is visible at the bottom-right corner.}%
\vspace{-0.5em}
\label{Fig App: DALLE 2 Vis}
\end{SCfigure}


% \section{DIP trajectory (PSNR and bitwise accuracy curves) for other watermarks}
% \label{App: DIP psnr bitwise-acc}

% \cref{Fig APP: watermark fourier vis} presents the PSNR and bitwise accuracy curves of the DIP reconstruction process for DwtDctSVD, SteganoGAN, SSL and StegaStamp watermarks.

% % \begin{figure}[!tb]
% \centering
% \resizebox{1\linewidth}{!}{%
% \begin{tabular}{c c c c c}
% \small{DwtDctSVD} & \small{rivaGAN} & \small{SteganoGAN} & \small{SSL} & \small{StegaStamp} \\
% \includegraphics[width=0.25\textwidth]{source-figures/DIP-evade-psnr-bitwise-acc/dwtDctSvd.png}
% & \includegraphics[width=0.25\textwidth]{source-figures/DIP-evade-psnr-bitwise-acc/rivaGan.png}
% & \includegraphics[width=0.25\textwidth]{source-figures/DIP-evade-psnr-bitwise-acc/SteganoGAN.png}
% & \includegraphics[width=0.25\textwidth]{source-figures/DIP-evade-psnr-bitwise-acc/SSL.png}
% & \includegraphics[width=0.25\textwidth]{source-figures/DIP-evade-psnr-bitwise-acc/StegaStamp.png}
% \end{tabular}}%
% \vspace{-0.5em}
% \caption{The PSNR trajectories (top row) of the intermediate steps of DIP evasion w.r.t watermarked images (blue) and the original image (orange), and its corresponding watermark detectability (bottom row, measured by $BA$), where $x-$axis is the number of iterations. Consider the case where BA threshold $\gamma=0.75$ (horizontal orange line) is used in the decoder for watermark detection, the vertical black lines in each figure mark the evasion images with the best quality. \textcolor{red}{remake this figure with larger font when MSI is available.}}%

% \vspace{-1em}
% \label{Fig: DIP evade algo 2}
% \end{figure}

\begin{figure}[!htbp]
\centering
\includegraphics[width=1\textwidth]{source-figures/DIP-evade-psnr-bitwise-acc/psnr-ba.jpg}
\vspace{-1.5em}
\caption{The PSNR trajectories of DIP-based evasion (top row) with respect to the watermarked image (in blue) and with respect to the original image (orange), and the corresponding trajectory of evasion performance (bottom row, measured by $BA$), for different watermark systems. Consider the BA threshold $\gamma=0.75$ for detection (marked by horizontal orange lines). The vertical black lines mark the best-quality evasion images.}%
\vspace{-1em}
\label{Fig: DIP evade algo 2}
\end{figure}

\section{Ranges of hyperparameters of various evasion methods}
\label{App: hyperparam}
\begin{table*}[!htbp]
\caption{Details of hyperparameters used in our exhaustive search}
\label{App tab: evasion hyperparam}
\centering
\resizebox{0.9\linewidth}{!}{%
\begin{tabular}{l c cccc}
\toprule
\small{Evasion method} & & \small{Hyperparameter} & \small{Search range} & \small{Search resolution} & \small{Common default value}\\
\toprule
\vspace{-0.9em}
\\
\small{\textbf{brightness}} & & \small{Enhancement factor} & \small{[0.01,  1]} & \small{0.01} & \small{0.5} \\
\small{\textbf{contrast}} & & \small{Enhancement factor} & \small{[0.01,  1]} & \small{0.01} & \small{0.5}\\
\small{\textbf{Gaussian Noise}} & & \small{Standard deviation} & \small{[0.01,  1]} & \small{0.01} & \small{0.1} \\
\small{\textbf{JPEG}} & & \small{Quality factor} & \small{[1,  100]} & \small{1} & \small{50} \\
\small{\textbf{bm3d}} & & \small{Noise standard deviation} & \small{[0.1, 5]} & \small{0.05} & \small{0.1} \\
\small{\textbf{DiffPure}} & & \small{Diffusion noise level} & \small{[0.1, 1]} & \small{0.1} & \small{0.1 - 0.3} \\
\small{\textbf{Diffuser}} & & \small{Diffusion inverse steps} & \small{[10, 100]} & \small{10} & \small{60} \\
\small{\textbf{VAE-Cheng2020}} & & \small{VAE compression quality indenx} & \small{[1, 6]} & \small{1} & \small{3} \\
\small{\textbf{DIP} (ours)} & & \small{Number of iteration} & \small{[1, 500]} & \small{10} & \small{-} \\
\toprule
\end{tabular}}
\vspace{-1em}
\end{table*}%
\cref{App tab: evasion hyperparam} presents the range of each hyperparameter and the grid resolution in our exhaustive search. If we simply use these typical values without exhaustive search, it is very likely that we will overestimate the level of robustness and underestimate the evasion image quality. \cref{Fig: DIP evade algo 2} shows an example of the exhaustive search process on our DIP-based evasion.

In \cref{App tab: evasion hyperparam}, the quality index of VAE indexes the individual pretrained model (six provided in the original work). Thus, the minimal resolution for VAE search is $1$ and cannot be smaller. Moreover, the last column shows the typical value used in the existing literature for robustness evaluation as a reference, e.g., \cite{zhao2023invisible, saberi2023robustness}. To the best of our knowledge, the reason why these values are chosen as defaults is not well justified. 

\section{Image quality v.s. evasion success} 
\label{App: psnr-ba}
% \begin{figure}[!tb]
% \centering
% \resizebox{1\linewidth}{!}{%
% \begin{tabular}{c c c c c}
% \small{DwtDctSVD} & \small{rivaGAN} & \small{SteganoGAN} & \small{SSL} & \small{StegaStamp} \\
% \includegraphics[width=0.25\textwidth]{source-figures/DIP-evade-psnr-bitwise-acc/dwtDctSvd.png}
% & \includegraphics[width=0.25\textwidth]{source-figures/DIP-evade-psnr-bitwise-acc/rivaGan.png}
% & \includegraphics[width=0.25\textwidth]{source-figures/DIP-evade-psnr-bitwise-acc/SteganoGAN.png}
% & \includegraphics[width=0.25\textwidth]{source-figures/DIP-evade-psnr-bitwise-acc/SSL.png}
% & \includegraphics[width=0.25\textwidth]{source-figures/DIP-evade-psnr-bitwise-acc/StegaStamp.png}
% \end{tabular}}%
% \vspace{-0.5em}
% \caption{The PSNR trajectories (top row) of the intermediate steps of DIP evasion w.r.t watermarked images (blue) and the original image (orange), and its corresponding watermark detectability (bottom row, measured by $BA$), where $x-$axis is the number of iterations. Consider the case where BA threshold $\gamma=0.75$ (horizontal orange line) is used in the decoder for watermark detection, the vertical black lines in each figure mark the evasion images with the best quality. \textcolor{red}{remake this figure with larger font when MSI is available.}}%

% \vspace{-1em}
% \label{Fig: DIP evade algo 2}
% \end{figure}

\begin{figure}[!htbp]
\centering
\includegraphics[width=1\textwidth]{source-figures/DIP-evade-psnr-bitwise-acc/psnr-ba.jpg}
\vspace{-1.5em}
\caption{The PSNR trajectories of DIP-based evasion (top row) with respect to the watermarked image (in blue) and with respect to the original image (orange), and the corresponding trajectory of evasion performance (bottom row, measured by $BA$), for different watermark systems. Consider the BA threshold $\gamma=0.75$ for detection (marked by horizontal orange lines). The vertical black lines mark the best-quality evasion images.}%
\vspace{-1em}
\label{Fig: DIP evade algo 2}
\end{figure}
% \cref{Fig: DIP evade algo 2} presents several illustrative trajectories of image quality and evasion success by our DIP-based evasion. 
% as well as evasion images with the best quality found by the exhaustive search. Similar approaches are applied to other methods using their hyperparameters listed in \cref{App tab: evasion hyperparam}, respectively.

% \section{Evasion results for DiffusionDB images}
% \label{Sec App: DiffusionDB}
% \cref{tab: watermark evasion diff gamma DiffusionDB} on DiffusionDB dataset for evasion results similar to \cref{tab: watermark evasion diff gamma} of COCO dataset.
% \begin{table*}[!htbp]
\vspace{-1.5em}
\caption{The best image quality produced by different evasion methods under different detection threshold $\gamma$ on the DiffusionDB dataset. Here, we report the mean value of PSNR-SSIM-$90\%$ Quantile (Q.). We highlight the best evasion method under each watermark and $\gamma$ in \textbf{boldface}. For fair comparison, we mask out cases where one evasion method cannot evade $\ge 90\%$ of the watermarked images.}
\label{tab: watermark evasion diff gamma DiffusionDB}
\centering
% \resizebox{1\linewidth}{!}{%
% \begin{tabular}{l c cccc}

% % { } & & & \multicolumn{3}
% % {c}{\textbf{COCO dataset}} & & \multicolumn{3}
% % {c}{\textbf{DiffusionDB dataset}} \\
% % \cline{4-6}\cline{8-10}
% % \vspace{-0.9em}
% % \\
% \toprule

% % \toprule
% \multicolumn{2}{l}{~~DiffusionDB dataset} & {$\gamma = 0.55$} & {$\gamma = 0.65$} & {$\gamma = 0.75$} & {$\gamma = 0.85$}\\
% \toprule
% \multicolumn{1}{l}{\textbf{DwtDctSVD}} & \multicolumn{1}{c}{TPR$\uparrow
% $ / FPR$\downarrow$} & {1.00 / 0.21} & {1.00 / 0.05} & {1.00 / 0.01} & {1.00 / 0.00} \\
% \cdashline{2-6}
% \vspace{-0.5em}
% \\
% {brightness} & {PSNR - SSIM - Q.} & {23.57 - 0.94 - 32.1} & {24.38 - 0.95 - 29.1} & {25.04 - 0.95 - 26.9} & {25.80 - 0.95 - 24.5}\\
% {contrast} & & {29.16 - 0.94 - 17.4} & {29.99 - 0.94 - 15.5} & {30.53 - 0.95 - 14.4} & {31.28 - 0.95 - 13.0}\\
% {Gaussian noise} & & {18.45 - 0.24 - 50.7} & {19.43 - 0.27 - 45.3} & {20.27 - 0.30 - 41.0} & {20.99 - 0.35 - 37.7}\\
% {JPEG} & & {32.07 - 0.89 - 10.5} & {32.27 - 0.90 - 10.2} & {32.36 - 0.90 - 10.1} & {32.66 - 0.90 - ~~9.7}\\
% \rowcolor{Gray}
% {bm3d} & & {**} & {$^*$22.44 - 0.73 - 23.7~~} & {$^*$24.81 - 0.78 - 17.4~~} & {27.85 - 0.85 - 11.1}\\
% \rowcolor{Gray}
% {DiffPure} & & {28.04 - 0.80 - 17.2} & {29.63 - 0.84 - 14.0} & {30.00 - 0.84 - 13.3} & {30.16 - 0.85 - 13.0}\\
% \rowcolor{Gray}
% {Diffuser} & & {****} & {**} & {$^*$27.07 - 0.79 -17.8~~} & {$^*$27.89 - 0.81 - 16.2~~}\\
% \rowcolor{Gray}
% {VAE} & & {****} & {$^*$32.61 - 0.88 - 10.1~~} & {33.80 - 0.90 - ~~8.8} & {34.95 - 0.92 - ~~7.6}\\
% \cdashline{1-6}
% \vspace{-0.95em}
% \\
% \rowcolor{Gray}
% {DIP (ours)} & & \textbf{35.35 - 0.96 - ~~6.7} & \textbf{35.95 - 0.96 - ~~6.2} & \textbf{36.30 - 0.96 - ~~6.0} & \textbf{36.60 - 0.97 - ~~5.8}\\

% \toprule
% \multicolumn{1}{l}{\textbf{rivaGAN}} & \multicolumn{1}{r}{TPR$\uparrow$ / FPR$\downarrow$} & {1.00 / 0.28} & {1.00 / 0.03} & {1.00 / 0.00} & {1.00 / 0.00} \\
% \cdashline{2-6}
% \vspace{-0.5em}
% \\
% {brightness} & {PSNR - SSIM - Q.} & {~~7.23 - 0.10 - 173~} & {~~7.71 - 0.17 - 164~} & {~~8.27 - 0.25 - 154~} & {~~9.19 - 0.37 - 139~}\\
% {contrast} & {} & {$^*$13.48 - 0.52 - 88.7~~} & {14.12 - 0.57 - 82.7} & {14.63 - 0.61 - 78.0} & {15.55 - 0.67 - 70.4}\\
% {Gaussian noise} & & {12.38 - 0.09 - 103~} & {14.44 - 0.13 - 81.6} & {16.14 - 0.17 - 67.4} & {18.33 - 0.24 - 52.7} \\
% {JPEG} & & {26.45 - 0.75 - 20.6} & {28.54 - 0.81 - 16.0} & {29.85 - 0.84 - 13.6} & {31.37 - 0.87 - 11.4} \\
% \rowcolor{Gray}
% {bm3d} & & {****} & {$^*$19.86 - 0.71 - 26.2~~} & {22.86 - 0.76 - 18.9} & {25.36 - 0.80 - 13.8}\\
% \rowcolor{Gray}
% {DiffPure} & & {27.24 - 0.77 - 19.7} & {29.25 - 0.83 - 14.6} & {30.03 - 0.84 - 13.1} & {30.22 - 0.85 - 12.9}\\
% \rowcolor{Gray}
% {Diffuser} & & {****} & {$^*$26.78 - 0.78 - 18.2~~} & {$^*$27.54 - 0.80 - 16.7~~} & {28.18 - 0.81 - 15.7}\\
% \rowcolor{Gray}
% {VAE} & & {****} & {$^*$33.45 - 0.90 - ~~8.9~~} & {34.37 - 0.91 - ~~8.0} & {35.13 - 0.93 - ~~7.3}\\
% \cdashline{1-6}
% \vspace{-0.95em}
% \\
% \rowcolor{Gray}
% {DIP (ours)} & & \textbf{32.29 - 0.91 - 12.9} & \textbf{34.66 - 0.93 - ~~8.9} & \textbf{35.92 - 0.95 - ~~7.0} & \textbf{36.96 - 0.96 - ~~5.9}\\

% \toprule
% \multicolumn{1}{l}{\textbf{SteganoGAN}} & \multicolumn{1}{r}{TPR$\uparrow$ / FPR$\downarrow$} & {0.83 / 0.00} & {0.83 / 0.00} & {0.83 / 0.00} & {0.83 / 0.00} \\
% \cdashline{2-6}
% \vspace{-0.5em}
% \\
% {brightness} & {PSNR - SSIM - Q.} & {38.36 - 0.95 - ~~4.9} & {38.36 - 0.95 - ~~4.9} & {38.36 - 0.95 - ~~4.9} & {38.36 - 0.95 - ~~4.9}\\
% {contrast} & & {32.57 - 0.93 - 11.8} & {32.57 - 0.93 - 11.8} & {32.57 - 0.93 - 11.8} & {32.57 - 0.93 - 11.8}\\
% {Gaussian noise} & & {36.04 - 0.91 - ~~6.5} & {36.04 - 0.91 - ~~6.5} & {36.04 - 0.91 - ~~6.5} & {36.04 - 0.91 - ~~6.5} \\
% {JPEG} & & {\textbf{38.78} - \textbf{0.97} - ~~\textbf{4.7}} & {\textbf{38.78} - \textbf{0.97} - ~~\textbf{4.7}} & {\textbf{38.78} - \textbf{0.97} - ~~\textbf{4.7}} & {\textbf{38.78} - \textbf{0.97} - ~~\textbf{4.7}} \\
% \rowcolor{Gray}
% {bm3d} & & {30.79 - 0.94 - ~~5.0} & {30.79 - 0.94 - ~~5.0} & {30.79 - 0.94 - ~~5.0} & {30.79 - 0.94 - ~~5.0}\\
% \rowcolor{Gray}
% {DiffPure} & & {30.04 - 0.85 - 13.2} & {30.04 - 0.85 - 13.2} & {30.04 - 0.85 - 13.2} & {30.04 - 0.85 - 13.2}\\
% \rowcolor{Gray}
% {Diffuser} & & {28.02 - 0.81 - 15.9} & {28.02 - 0.81 - 15.9} & {28.02 - 0.81 - 15.9} & {28.02 - 0.81 - 15.9}\\
% \rowcolor{Gray}
% {VAE} & & {36.28 - 0.94 - ~~6.4} & {36.28 - 0.94 - ~~6.4} & {36.28 - 0.94 - ~~6.4} & {36.28 - 0.94 - ~~6.4}\\
% \cdashline{1-6}
% \vspace{-0.95em}
% \\
% \rowcolor{Gray}
% {DIP (ours)} & & {37.79 - \textbf{0.97} - ~~5.2} & {37.79 - \textbf{0.97} - ~~5.2} & {37.79 - \textbf{0.97} - ~~5.2} & {37.79 - \textbf{0.97} - ~~5.2}\\

% \toprule
% \multicolumn{1}{l}{\textbf{SSL}} & \multicolumn{1}{r}{TPR$\uparrow$ / FPR$\downarrow$} & {1.00 / 0.43} & {1.00 / 0.09} & {1.00 / 0.02} & {1.00 / 0.00} \\
% \cdashline{2-6}
% \vspace{-0.5em}
% \\
% {brightness} & {PSNR - SSIM - Q.} & {****} & {**} & {~~6.66 - 0.03 - 185~} & {~~6.79 - 0.05 - 183~}\\
% {contrast} & & {****} & {****} & {**} & {$^*$13.06 - 0.49 - 93.5~}\\
% {Gaussian noise} & & {~$^*$8.97 - 0.04 - 154~~} & {12.79 - 0.11 - 105~} & {18.42 - 0.24 - 52.8} & {22.91 - 0.40 - 30.7}\\
% {JPEG} & & {$^*$24.42 - 0.68 - 25.4~~} & {$^*$26.93 - 0.76 - 19.2~~} & {28.74 - 0.81 - 15.3} & {30.02 - 0.84 - 13.3}\\
% \rowcolor{Gray}
% {bm3d} & & \textbf{$^*$25.15 - 0.76 - 18.1~~} & {27.45 - 0.82 - 13.0} & {28.48 - 0.86 - 10.8} & {28.77 - 0.87 - 10.2}\\
% \rowcolor{Gray}
% {DiffPure} & & {25.07 - 0.70 - 25.1} & {27.87 - 0.79 - 17.3} & {29.34 - 0.82 - 14.5} & {29.78 - 0.84 - 13.6}\\
% \rowcolor{Gray}
% {Diffuser} & & {****} & {****} & {****} & {25.91 - 0.73 - 20.2}\\
% \rowcolor{Gray}
% {VAE} & & {**} & \textbf{$^*$31.88 - 0.87 - 11.0~~} & \textbf{32.39 - 0.88 - 10.3} & \textbf{32.88 - 0.90 - ~~9.7}\\
% \cdashline{1-6}
% \vspace{-0.95em}
% \\
% \rowcolor{Gray}
% {DIP (ours)} & & {16.42 - 0.57 - 71.7} & {21.01 - 0.71 - 43.9} & {25.58 - 0.80 - 25.3} & {29.55 - 0.87 - 14.6}\\

% \toprule
% \multicolumn{1}{l}{\textbf{StegaStamp}} & \multicolumn{1}{r}{TPR$\uparrow$ / FPR$\downarrow$} & {1.00 / 0.00} & {1.00 / 0.00} & {1.00 / 0.00} & {1.00 / 0.00} \\
% \cdashline{2-6}
% \vspace{-0.5em}
% \\
% {brightness} & {PSNR - SSIM - Q.} & {~~6.97 - 0.07 - 179~} & {~~6.97 - 0.07 - 179~} & {~~6.97 - 0.07 - 179~} & {~~6.97 - 0.07 - 179~}\\
% {contrast} & & {14.59 - 0.61 - 78.6} & {14.59 - 0.61 - 78.6} & {14.59 - 0.61 - 78.6} & {14.59 - 0.61 - 78.6}\\
% {Gaussian noise} & & {16.16 - 0.19 - 66.8} & {16.16 - 0.19 - 66.8} & {16.16 - 0.19 - 66.8} & {16.16 - 0.19 - 66.8}\\
% {JPEG} & & {24.49 - 0.72 - 25.0} & {24.49 - 0.72 - 25.0} & {24.49 - 0.72 - 25.0} & {24.49 - 0.72 - 25.0}\\
% \rowcolor{Gray}
% {bm3d} & & {**} & {**} & {**} & {**}\\
% \rowcolor{Gray}
% {DiffPure} & & \textbf{{25.55 - 0.76 - 21.7}} & \textbf{{25.55 - 0.76 - 21.7}} & \textbf{{25.55 - 0.76 - 21.7}} & \textbf{{25.55 - 0.76 - 21.7}}\\
% \rowcolor{Gray}
% {Diffuser} & & {23.16 - 0.66 - 27.8} & {23.16 - 0.66 - 27.8} & {23.16 - 0.66 - 27.8} & {23.16 - 0.66 - 27.8}\\
% \rowcolor{Gray}
% {VAE} & & \multicolumn{4}{c}{--- Cannot find evasion images within its range of hyperparameters ---}\\
% \cdashline{1-6}
% \vspace{-0.95em}
% \\
% \rowcolor{Gray}
% {DIP (ours)} & & {18.04 - 0.65 - 56.7} & {18.04 - 0.65 - 56.7} & {18.04 - 0.65 - 56.7} & {18.04 - 0.65 - 56.7}\\

% \toprule
% \multicolumn{6}{l}{The following markers are used for fair comparison purpose:}\\
% \multicolumn{6}{l}{**** Evasion method only successfully evade $<75\%$ of the watermarked images.}\\
% \multicolumn{6}{l}{**~~~~ Evasion method only successfully evade $<90\%$ of the watermarked images.}\\
% \multicolumn{6}{l}{*~~~~~~ Evasion method successfully evade $\ge 90\%$ of the watermarked images, but $< 100\%$.}\\
% \bottomrule
% \end{tabular}}
% \vspace{-1.5em}

\resizebox{0.9\linewidth}{!}{%
\begin{tabular}{l c cccc}
\toprule

% \toprule
\multicolumn{2}{l}{~~DiffusionDB dataset} & {$\gamma = 0.55$} & {$\gamma = 0.65$} & {$\gamma = 0.75$} & {$\gamma = 0.85$}\\
\toprule
\multicolumn{1}{l}{\textbf{DwtDctSVD}} & \multicolumn{1}{c}{TPR$\uparrow
$ / FPR$\downarrow$} & {1.00 / 0.23} & {0.99 / 0.03} & {0.99 / 0.01} & {0.99 / 0.00} \\
\cdashline{2-6}
\vspace{-0.5em}
\\
{brightness} & {PSNR - SSIM - Q.} & {21.25 - 0.90 - 43.0} & {21.94 - 0.91 - 39.1} & {22.54 - 0.92 - 35.9} & {23.34 - 0.93 - 32.3}\\
{contrast} & & {26.89 - 0.91 - 22.2} & {27.56 - 0.92 - 20.1} & {28.11 - 0.93 - 18.6} & {28.85 - 0.94 - 16.8}\\
{Gaussian noise} & & {16.52 - 0.22 - 64.3} & {17.73 - 0.25 - 55.5} & {18.81 - 0.29 - 49.1} & {20.07 - 0.33 - 42.0}\\
{JPEG} & & {30.47 - 0.86 - 13.2} & {30.81 - 0.87 - 12.7} & {31.00 - 0.88 - 12.4} & {31.30 - 0.88 - 12.0}\\
\rowcolor{Gray}
{bm3d} & & {****} & {****} & {**} & {$^*$28.94 - 0.85 - 11.6~~}\\
\rowcolor{Gray}
{DiffPure} & & {27.69 - 0.78 - 18.8} & {29.09 - 0.82 - 15.8} & {29.52 - 0.83 - 15.0} & {29.56 - 0.83 - 14.9}\\
\rowcolor{Gray}
{Diffuser} & & {**} & {**} & {$^*$25.88 - 0.75 - 21.7~~} & {$^*$26.54 - 0.77 - 20.5~~}\\
\rowcolor{Gray}
{VAE} & & {****} & {$^*$31.91 - 0.87 - 11.5~~} & {32.90 - 0.89 - 10.3} & {32.92 - 0.91 - ~9.0~}\\
\cdashline{1-6}
\vspace{-0.95em}
\\
\rowcolor{Gray}
{DIP (ours)} & & \textbf{35.29 - 0.96 - ~6.9~} & \textbf{35.91 - 0.96 - ~6.4~} & \textbf{36.25 - 0.97 - ~6.1~} & \textbf{36.53 - 0.97 - ~5.9~}\\


\toprule
\multicolumn{1}{l}{\textbf{rivaGAN}} & \multicolumn{1}{r}{TPR$\uparrow$ / FPR$\downarrow$} & {0.99 / 0.25} & {0.99 / 0.03} & {0.99 / 0.01} & {0.99 / 0.00} \\
\cdashline{2-6}
\vspace{-0.5em}
\\
{brightness} & {PSNR - SSIM - Q.} & {****} & {$^*$7.38 - 0.13 - 173~} & {~7.83 - 0.19 - 165~} & {~8.70 - 0.31 - 150~}\\
{contrast} & {} & {****} & {$^*$13.47 - 0.51 - 86.8~~} & {13.93 - 0.55 - 82.5} & {14.73 - 0.60 - 75.5}\\
{Gaussian noise} & & {10.96 - 0.08 - 123~} & {12.98 - 0.12 - 97.3} & {14.55 - 0.16 - 81.4} & {16.56 - 0.21 - 64.8} \\
{JPEG} & & {****} &
 {$^*$26.31 - 0.74 - 20.9~} & {$^*$27.84 - 0.79 - 17.5~} & {29.61 - 0.84 - 14.1} \\
\rowcolor{Gray}
{bm3d} & & {****} & {****} & {****} & {****}\\
\rowcolor{Gray}
{DiffPure} & & {26.11 - 0.73 - 22.9} & {28.25 - 0.79 - 16.8} & {28.88 - 0.81 - 15.4} & {29.02 - 0.82 - 15.1}\\
\rowcolor{Gray}
{Diffuser} & & {****} & {$^*$25.43 - 0.73 - 22.5~} & {$^*$26.12 - 0.75 - 20.8~~} & {26.47 - 0.76 - 20.0}\\
\rowcolor{Gray}
{VAE} & & {**} & {$^*$32.29 - 0.88 - \textbf{10.4}~} & {$^*$33.13 - 0.90 - ~9.4~~~} & {33.94 - 0.91 - ~8.5~~}\\
\cdashline{1-6}
\vspace{-0.95em}
\\
\rowcolor{Gray}
{DIP (ours)} & & {\textbf{31.91} - \textbf{0.89} - \textbf{14.5}} & {\textbf{34.48} - \textbf{0.93} - ~9.1~} & {\textbf{35.72} - \textbf{0.95} - ~\textbf{7.4}~} & {\textbf{36.75} - \textbf{0.96} - ~\textbf{6.2}~}\\

\toprule
\multicolumn{1}{l}{\textbf{SSL}} & \multicolumn{1}{r}{TPR$\uparrow$ / FPR$\downarrow$} & {1.00 / 0.33} & {1.00 / 0.07} & {0.99 / 0.01} & {0.99 / 0.00} \\
\cdashline{2-6}
\vspace{-0.5em}
\\
{brightness} & {PSNR - SSIM - Q.} & {****} & {****} & {**} & {$^*$16.91 - 0.53 - 97.9~~~}\\
{contrast} & & {****} & {****} & {**} & {$^*$21.54 - 0.73 - 47.9~~~}\\
{Gaussian noise} & & {$^*$17.43 - 0.26 - 63.8~~} & {20.90 - 0.38 - 41.0} & {23.48 - 0.48 - 29.7} & {25.31 - 0.56 - 23.0}\\
{JPEG} & & {$^*$27.07 - 0.77 - 19.4~~} & {28.97 - 0.82 - 15.4} & {30.47 - 0.85 - 12.7} & {32.39 - 0.89 - ~9.9~~}\\
\rowcolor{Gray}

{bm3d} & & {$^*$26.69 - 0.77 - 16.3~~} & {28.00 - 0.82 - 13.3} & {29.00 - 0.85 - 11.1} & {29.25 - 0.86 - 10.5}\\
\rowcolor{Gray}
{DiffPure} & & {26.67 - 0.75 - 21.0} & {28.05 - 0.79 - 17.3} & {28.40 - 0.80 - 16.3} & {28.43 - 0.80 - 16.2}\\
\rowcolor{Gray}
{Diffuser} & & {**} & {$^*$25.00 - 0.68 - 23.6~~} & {25.43 - 0.69 - 22.5} & {25.49 - 0.70 - 22.4}\\
\rowcolor{Gray}
{VAE} & & {$^*$\textbf{30.26} - \textbf{0.85} - \textbf{12.3}~~} & {$^*$\textbf{31.79} - \textbf{0.88} - \textbf{11.0}~~} & {\textbf{32.43} - \textbf{0.90} - ~\textbf{9.9}~} & {33.01 - 0.91 - ~9.2~}\\
\cdashline{1-6}
\vspace{-0.95em}
\\
\rowcolor{Gray}
{DIP (ours)} & & {23.14 - 0.72 - 35.2} & {27.95 - 0.84 - 18.2} & {31.27 - 0.89 - 11.8} & {\textbf{33.84} - \textbf{0.92} - ~\textbf{8.2}~}\\

\toprule
\multicolumn{1}{l}{\textbf{TrustMark}} & \multicolumn{1}{r}{TPR$\uparrow$ / FPR$\downarrow$} & {1.00 / 0.13} & {1.00 / 0.00} & {1.00 / 0.00} & {1.00 / 0.00} \\
{brightness} & {PSNR - SSIM - Q.} & {****} & {**} & {~$^*$7.09 - 0.10 - 179~~} & {~$^*$7.43 - 0.15 - 172~~~}\\
{contrast} & & {****} & {**} & {$^*$13.52 - 0.55 - 87.3~~} & {$^*$13.98 - 0.59 - 83.2~~~}\\
{Gaussian noise} & & {$^*$9.62 - 0.05 - 143~~~} & {11.17 - 0.08 - 119~~} & {12.67 - 0.10 - 100~} & {14.45 - 0.14 - 81.7}\\
{JPEG} & & {****} & {****} & {$^*$25.06 - 0.69 - 24.5~~} & {$^*$26.08 - 0.72 - 22.4~~~}\\
\rowcolor{Gray}
{bm3d} & & {------} & {------} & {------} & {------}\\
\rowcolor{Gray}
{DiffPure} & & \textbf{{26.60 - 0.73 - 21.8}} & \textbf{{27.91 - 0.77 - 18.7}} & \textbf{{28.99 - 0.79 - 16.6}} & \textbf{{30.07 - 0.83 - 14.4}}\\
\rowcolor{Gray}
{Diffuser} & & {**} & {$^*$25.79 - 0.74 - 21.9~} & {$^*$26.40 - 0.76 - 20.4~~} & {26.96 - 0.77 - 19.2}\\
\rowcolor{Gray}
{VAE} & & {------} & {------} & {****} & {****}\\
\cdashline{1-6}
\vspace{-0.95em}
\\
\rowcolor{Gray}
{DIP (ours)} & & {**} & {$^*$14.16 - 0.5 - 88.4} & {16.19 - 0.57 - 73.0} & {19.07 - 0.65 - 54.6}\\

\toprule
\multicolumn{1}{l}{\textbf{RoSteALS}} & \multicolumn{1}{c}{TPR$\uparrow
$ / FPR$\downarrow$} & {1.00 / 0.29} & {1.00 / 0.01} & {1.00 / 0.00} & {1.00 / 0.00} \\
\cdashline{2-6}
\vspace{-0.5em}
\\
{brightness} & {PSNR - SSIM - Q.} & {------} & {------} & {------} & {------}\\
{contrast} & & {------} & {------} & {------} & {------}\\
{Gaussian noise} & & {------} & {****} & {****} & {~$^*$9.92 - 0.05 - 139~~~}\\
{JPEG} & & {------} & {------} & {------} & {****}\\
\rowcolor{Gray}
{bm3d} & & {------} & {------} & {------} & {------}\\
\rowcolor{Gray}
{DiffPure} & & \textbf{20.86 - 0.57 - 43.7} & \textbf{24.06 - 0.66 - 29.3} & \textbf{25.74 - 0.72 - 23.8} & \textbf{26.93 - 0.76 - 20.6}\\
\rowcolor{Gray}
{Diffuser} & & {------} & {------} & {------} & {****}\\
\rowcolor{Gray}
{VAE} & & {------} & {------} & {------} & {------}\\
\cdashline{1-6}
\vspace{-0.95em}
\\
\rowcolor{Gray}
{DIP (ours)} & & {**} & {$^*$11.67 - 0.42 - 108~~} & {12.83 - 0.46 - 98.9} & {17.47 - 0.60 - 67.0} \\

\toprule
\multicolumn{1}{l}{\textbf{StegaStamp}} & \multicolumn{1}{r}{TPR$\uparrow$ / FPR$\downarrow$} & {1.00 / 0.18} & {1.00 / 0.01} & {1.00 / 0.00} & {1.00 / 0.00} \\
{brightness} & {PSNR - SSIM - Q.} & {------} & {------} & {------} & {------}\\
{contrast} & & {------} & {------} & {****} & {**}\\
{Gaussian noise} & & {****} & {$^*$8.03 - 0.04 - 170~} & {~9.49 - 0.06 - 145~} & {11.66 - 0.09 - 113~}\\
{JPEG} & & {****} & {****} & {****} & {**}\\
\rowcolor{Gray}
{bm3d} & & {------} & {------} & {------} & {------}\\
\rowcolor{Gray}
{DiffPure} & & \textbf{$^*$19.70 - 0.54 - 45.6~} & \textbf{{22.42 - 0.63 - 32.2}} & \textbf{{23.73 - 0.68 - 27.5}} & \textbf{{24.55 - 0.71 - 24.9}}\\
\rowcolor{Gray}
{Diffuser} & & {------} & {------} & {****} & {$^*$21.97 - 0.61 - 32.4~~~}\\
\rowcolor{Gray}
{VAE} & & {------} & {------} & {------} & {------}\\
\cdashline{1-6}
\vspace{-0.95em}
\\
\rowcolor{Gray}
{DIP (ours)} & & {**} & {$^*$13.01 - 0.44 - 96.7~} & {14.69 - 0.50 - 82.7} & {16.54 - 0.57 - 68.5}\\

\toprule
\multicolumn{6}{l}{The following markers are used for the purpose of fair comparison of the best evasion image quality:}\\
\multicolumn{6}{l}{------ Evasion method only successfully evade $<10 \%$ of the watermarked images.}\\
\multicolumn{6}{l}{**** Evasion method only successfully evade $<75\%$ of the watermarked images.}\\
\multicolumn{6}{l}{**~~~~ Evasion method only successfully evade $<90\%$ of the watermarked images.}\\
\multicolumn{6}{l}{*~~~~~~ Evasion method successfully evade $\ge 90\%$ of the watermarked images, but $< 100\%$.}\\
\bottomrule
\end{tabular}}
\vspace{-1em}
\end{table*}%

\section{Additional visualization of evasion image quality on rivaGAN}
\label{Sec App: rivaGan evasion patterns}
\begin{figure*}[!htbp]
\centering
\resizebox{0.95\linewidth}{!}{%
\includegraphics[width=\textwidth]{source-figures/watermark-evasion-vis/watermark-rivaGAN_v2_2.png}%
}
\vspace{-1em}
\caption{Visualization of the evasion images found by different evasion methods on a rivaGAN watermarked image (with $\gamma = 0.75$; top row) and the respective histograms of the pixel difference ($y-$axis in $\log$ scale) between the evasion image and the clean image (bottom row). The vertical dashed line marks the $90 \%$ quantile. We can observe that the evasion image produced by DIP has almost no loss of image quality.}%
\vspace{-0.5em}
\label{Fig: watermark evasion vis rivaGAN}
\end{figure*}