\section{Conclusion and discussion}
\label{Sec: Summary}
% In this paper, we present the ability of DIP to evade watermarking systems for the first time. We show that DIP can reliably find high-quality evasion images to existing relatively invisible watermarks. By showing the limited evasion ability of existing methods (including ours) on the relatively visible learning-based watermarks, we also discuss the implications of existing watermarking systems in practice: negative to apply for copyright protection, but positive to apply for preventing fake image abuse. 

With the results and the analysis above, we can conclude that there is \emph{no universal best evasion methods} for existing watermarking systems. In general, our DIP-based evasion is most effective in evading invisible watermarks that induce high-frequency distortions (e.g., DwtDctSVD, rivaGAN and SSL), and is partially successful  in evading \emph{in-processing} watermarks such as TreeRing. Its limited performance for RoSteALS and StegaStamp implies that exploiting low- and mid-frequency distortions is a viable way for watermarking systems to counteract our DIP-based evasion. Also, for these watermark methods, the regeneration evasion DiffPure has proved effective.  

Moreover, for relatively visible watermarks (e.g., StegaStamp), the evasion images generated by all evasion methods always contain visible artifacts; see \cref{Fig: watermark evasion vis stegaStamp} for an example. Therefore, the future of learning-based watermarks is not all pessimistic: they may not be reliable for copyright protection, but may be promising in misinformation prevention. This is because of the distinct requirements of these two kinds of applications: for copyright protection, watermarks are expected to remain detectable as long as the image content is recognizable even under severe corruptions due to evasion---which may be too hard to achieve. In contrast, to prevent misinformation, it might be sufficient to achieve either of the following to mitigate the harm: \textbf{(i)} the watermark patterns can be detected by eyes, e.g., an overlaid logo or unnatural perturbations such as \cref{Fig: watermark vis intro,Fig: watermark evasion vis stegaStamp}, raising suspicion that the image is already manipulated or fake; \textbf{(ii)} the watermark can be detected by an algorithmic decoder. For this purpose, watermarks such as StegaStamp may be sufficient. 