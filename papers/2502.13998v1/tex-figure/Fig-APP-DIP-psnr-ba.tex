% \begin{figure}[!tb]
% \centering
% \resizebox{1\linewidth}{!}{%
% \begin{tabular}{c c c c c}
% \small{DwtDctSVD} & \small{rivaGAN} & \small{SteganoGAN} & \small{SSL} & \small{StegaStamp} \\
% \includegraphics[width=0.25\textwidth]{source-figures/DIP-evade-psnr-bitwise-acc/dwtDctSvd.png}
% & \includegraphics[width=0.25\textwidth]{source-figures/DIP-evade-psnr-bitwise-acc/rivaGan.png}
% & \includegraphics[width=0.25\textwidth]{source-figures/DIP-evade-psnr-bitwise-acc/SteganoGAN.png}
% & \includegraphics[width=0.25\textwidth]{source-figures/DIP-evade-psnr-bitwise-acc/SSL.png}
% & \includegraphics[width=0.25\textwidth]{source-figures/DIP-evade-psnr-bitwise-acc/StegaStamp.png}
% \end{tabular}}%
% \vspace{-0.5em}
% \caption{The PSNR trajectories (top row) of the intermediate steps of DIP evasion w.r.t watermarked images (blue) and the original image (orange), and its corresponding watermark detectability (bottom row, measured by $BA$), where $x-$axis is the number of iterations. Consider the case where BA threshold $\gamma=0.75$ (horizontal orange line) is used in the decoder for watermark detection, the vertical black lines in each figure mark the evasion images with the best quality. \textcolor{red}{remake this figure with larger font when MSI is available.}}%

% \vspace{-1em}
% \label{Fig: DIP evade algo 2}
% \end{figure}

\begin{figure}[!htbp]
\centering
\includegraphics[width=1\textwidth]{source-figures/DIP-evade-psnr-bitwise-acc/psnr-ba.jpg}
\vspace{-1.5em}
\caption{The PSNR trajectories of DIP-based evasion (top row) with respect to the watermarked image (in blue) and with respect to the original image (orange), and the corresponding trajectory of evasion performance (bottom row, measured by $BA$), for different watermark systems. Consider the BA threshold $\gamma=0.75$ for detection (marked by horizontal orange lines). The vertical black lines mark the best-quality evasion images.}%
\vspace{-1em}
\label{Fig: DIP evade algo 2}
\end{figure}