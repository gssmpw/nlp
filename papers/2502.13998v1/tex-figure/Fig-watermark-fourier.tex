% \begin{figure*}[!tb]
% \centering
% \resizebox{1\linewidth}{!}{%
% \begin{tabular}{c c c c c c}
% Clean & DwtDctSVD & rivaGAN & SteganoGAN & SSL & StegaStamp \\
% \includegraphics[width=0.166\textwidth]{source-figures/watermark-fourier-spec/clean.png}
% & \includegraphics[width=0.166\textwidth]{source-figures/watermark-fourier-spec/dwtDctSvd.png}
% & \includegraphics[width=0.166\textwidth]{source-figures/watermark-fourier-spec/rivaGan.png}
% & \includegraphics[width=0.166\textwidth]{source-figures/watermark-fourier-spec/SteganoGAN.png}
% & \includegraphics[width=0.166\textwidth]{source-figures/watermark-fourier-spec/SSL.png}
% & \includegraphics[width=0.166\textwidth]{source-figures/watermark-fourier-spec/StegaStamp.png}%
% % \small{\textbf{(a)}}%
% % &\small{\textbf{(b)}}%
% \end{tabular}}%
% \caption{Visualization of the magnitude of 2D Fourier spectra of images shown in \cref{Fig: watermark vis} (in $\log$ scale). `DwtDctSVD, rivaGAN, SteganoGAN, SSL and StegaStamp' are the spectra of the watermarks ($I_w - I$), while `Clean' is the spectrum of the original image ($I$). Note that the scale (brightness) across images may be distorted due to $\log$ scaling and the normalization applied for better visualization. We refer the readers to the PSNR values in \cref{tab: watermark quality intensity} to have a rough estimate of the intensity of each watermark---the lower the PSNR, the larger the watermark intensity. \textcolor{red}{Re-check if we can make the visualization clearer w.o normalization when the MSI is back.}}%
% \vspace{-1em}
% \label{Fig: watermark fourier vis}
% \end{figure*}
% \begin{figure*}[!tb]
% \centering
% \includegraphics[width=0.8\textwidth]{source-figures/watermark-fourier/fourier.jpg}
% \vspace{-0.5em}
% \caption{Visualization of 2D Fourier spectra magnitude of the clean and watermarked images in \cref{Fig: watermark vis} (top row, visualized in $\log$ scale). The bottom row shows the band-wise energy spectrum of the images on top, where the x-axis corresponds to the radius (black arrow), and each point on the curve (e.g., the point intersected with the vertical green line) corresponds to the sum of the energy along the band with the same radius (e.g., marked by the green circle on top). We can observe that DwtDctSVD, rivaGAN and SSL introduce watermark patterns mostly in the high-frequency components. For TrustMark, the high-frequency component is very limited (note the log-scale of the y axis). For RoSteALS and StegaStamp, considerable amount of low-frequency components are introduced to the watermark patterns.}%
% \vspace{-1em}
% \label{Fig: watermark fourier vis}
% \end{figure*}

% \begin{wrapfigure}{r}{0.7\textwidth}
%     \vspace{-3em}
%     \includegraphics[width=0.7\textwidth]{source-figures/watermark-fourier/fourier.jpg}
%     \vspace{-2em}
%     \caption{Visualization of 2D Fourier spectra magnitude of the clean and watermarked images in \cref{Fig: watermark vis} (top row, visualized in $\log$ scale). The bottom row shows the band-wise energy spectrum of the images on top, where the x-axis corresponds to the radius (black arrow), and each point on the curve (e.g., the point intersected with the vertical green line) corresponds to the sum of the energy along the band with the same radius (e.g., marked by the green circle on top). We can observe that DwtDctSVD, rivaGAN and SSL introduce watermark patterns mostly in the high-frequency components. For TrustMark, the high-frequency component is very limited (note the log-scale of the y axis). For RoSteALS and StegaStamp, considerable amount of low-frequency components are introduced to the watermark patterns.}
%     \label{Fig: watermark fourier vis}
%     \vspace{-0.5em}
% \end{wrapfigure}

% \begin{figure*}[!tb]
% \centering
% \includegraphics[width=0.8\textwidth]{source-figures/watermark-fourier/Fourier_band_error.jpg}
% \vspace{-0.5em}
% \caption{Fourier Band Errors (FBEs) of DIP's intermediate steps in purifying different watermarks, where x-axis are the number of iterations. We visualize FBEs by dividing all frequency components into five different bands from the lowest (1) to hightest (5). We can observe that the frequency preference is an intrinsic property of DIP and is agnostic to watermark patterns.}%
% \vspace{-1em}
% \label{Fig: fourier band error}
% \end{figure*}

% \begin{wrapfigure}{r}{0.7\textwidth}
%     \vspace{-3em}
%     \includegraphics[width=0.7\textwidth]{source-figures/watermark-fourier/Fourier_band_error.jpg}
%     \vspace{-2em}
%     \caption{Fourier Band Errors (FBEs) of DIP's intermediate steps in purifying different watermarks, where x-axis are the number of iterations. We visualize FBEs by dividing all frequency components into five different bands from the lowest (1) to hightest (5). We can observe that the frequency preference is an intrinsic property of DIP and is agnostic to watermark patterns.}
%     \label{Fig: fourier band error}
%     \vspace{-0.5em}
% \end{wrapfigure}