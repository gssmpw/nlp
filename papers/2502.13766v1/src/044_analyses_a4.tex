\subsection{Influence of External Cues}
\label{sec:analyses:a4_external}
%
We examine how external hints, i.e., informing a model about the country or region of a CEF, affect VQA performance.
%
\begin{figure*}[t]
    \centering
    \begin{subfigure}{1.\linewidth}
        \centering
        \includegraphics[width=1.\linewidth]{gfx/sivqa_relative_gains.pdf}
        \caption{\texttt{SIVQA}}
        \label{fig:analyses:a4:relative:sivqa}
    \end{subfigure}
    
    \begin{subfigure}{1.\linewidth}
        \centering
        \includegraphics[width=1.\linewidth, trim=0 0 0 0, clip]{gfx/vvqa_relative_gains.pdf}
        \caption{\texttt{VVQA}}
        \label{fig:analyses:a4:relative:vvqa}
    \end{subfigure}
    \caption{Relative gains on VQA tasks from providing external geographical hints.}
    \label{fig:analyses:a4:relative}
\end{figure*}
%
For \sivqa (Figure~\ref{fig:analyses:a4:relative:sivqa}), country hints consistently boost performance across model sizes and regions, while regional cues yield only modest—or even slightly adverse—effects in larger models.
%
Gains from country hints are around 50\% for most regions, but in \RegSA, improvements nearly double (e.g., $97.48\%$ for \m{InternVL 2.5 78B} and $97.13\%$ for \m{InternVL 2.5 38B}).
%
A similar pattern emerges for \vvqa (Figure~\ref{fig:analyses:a4:relative:vvqa}).
%
Hints generally enhance performance across regions and models, with \RegSA showing the most significant gains.
%
Proprietary and small models exhibit subtle improvements, whereas L and XL models see much higher relative gains---up to $240.7\%$ for \m{Intern VL 38B}.
%
Notably, regional cues have a more positive impact on \vvqa than on \sivqa.
%
