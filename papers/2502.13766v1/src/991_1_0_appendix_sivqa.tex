\section{\sivqa Details}
\label{appendix:sec:sivqa}
%

\clearpage
\subsection{Examples}
\label{appendix:sec:sivqa:examples}
%
In the following, we provide one random sample per region for the \sivqa task.
%
Note that the lower part of the examples, where the related CEF is provided, is \emph{not} part of the actual sample.

%
\subsubsection*{\RegA}
%
\begin{figure}[H]
\begin{tcolorbox}[colback=gray!5!white,colframe=black!75!black,fonttitle=\bfseries\scriptsize,fontupper=\ttfamily\footnotesize]
  {\large{Title:}} {\normalsize{Arts, skills and practices associated with engraving on metals (gold, silver and copper)}}\\
  {\normalsize{Countries:}} Algeria, Saudi Arabia, Egypt, Iraq, Morocco, Mauritania, Palestine, Sudan, Tunisia, Yemen\\
  {\normalsize{Regions:}} Arab States\\
  {\normalsize{Description:}}\\
  Engraving on metals such as gold, silver and copper is a centuries-old practice that entails manually cutting words, symbols or patterns into the surfaces of decorative, utilitarian, religious or ceremonial objects. The craftsperson uses different tools to manually cut symbols, names, Quran verses, prayers and geometric patterns into the objects. Engravings can be concave (recessed) or convex (elevated), or the result of a combination of different types of metals, such as gold and silver. Their social and symbolic meanings and functions vary according to the communities concerned. Engraved objects, such as jewelry or household objects, are often presented as traditional gifts for weddings or used in religious rituals and alternative medicine. For instance, certain types of metals are believed to have healing properties. Engraving on metals is transmitted within families, through observation and hands-on practice. It is also transmitted through workshops organized by training centres, organizations and universities, among others. Publications, cultural events and social media further contribute to the transmission of the related knowledge and skills. Practised by people of all ages and genders, metal engraving and the use of engraved objects are means of expressing the cultural, religious and geographical identity and the socioeconomic status of the communities concerned.\\[2mm]
  {\normalsize{UNESCO ICH URL:}} \href{https://ich.unesco.org/en/RL/arts-skills-and-practices-associated-with-engraving-on-metals-gold-silver-and-copper-01951}{https://ich.unesco.org/en/RL/arts-skills-and-practices-assoc...}
  \begin{center}
    \begin{minipage}{0.18\linewidth}
      \centering
      \includegraphics[width=\linewidth]{examples/gfx/cef_arab-states_0_ef285d07_0.jpg}
      {\captionsetup{labelformat=empty}\captionof{figure}{\tiny\textit{Copyrigth: Huzaifa Ayad Bahaa El Din, Iraq, 2021}}}
    \end{minipage}\hfill
    \begin{minipage}{0.18\linewidth}
      \centering
      \includegraphics[width=\linewidth]{examples/gfx/cef_arab-states_0_ef285d07_1.jpg}
      {\captionsetup{labelformat=empty}\captionof{figure}{\tiny\textit{Copyrigth: Huzaifa Ayad Bahaa El Din, Iraq, 2021}}}
    \end{minipage}\hfill
    \begin{minipage}{0.18\linewidth}
      \centering
      \includegraphics[width=\linewidth]{examples/gfx/cef_arab-states_0_ef285d07_2.jpg}
      {\captionsetup{labelformat=empty}\captionof{figure}{\tiny\textit{Copyrigth: Huzaifa Ayad Bahaa El Din, Iraq, 2021}}}
    \end{minipage}\hfill
    \begin{minipage}{0.18\linewidth}
      \centering
      \includegraphics[width=\linewidth]{examples/gfx/cef_arab-states_0_ef285d07_3.jpg}
      {\captionsetup{labelformat=empty}\captionof{figure}{\tiny\textit{Copyrigth: Zahia Benabdallah, Algeria, 2021}}}
    \end{minipage}\hfill
    \begin{minipage}{0.18\linewidth}
      \centering
      \includegraphics[width=\linewidth]{examples/gfx/cef_arab-states_0_ef285d07_4.jpg}
      {\captionsetup{labelformat=empty}\captionof{figure}{\tiny\textit{Copyrigth: Azza Fahmi, Egypt, 2021}}}
    \end{minipage}\hfill
  \\[4mm]
    \begin{minipage}{0.18\linewidth}
      \centering
      \includegraphics[width=\linewidth]{examples/gfx/cef_arab-states_0_ef285d07_5.jpg}
      {\captionsetup{labelformat=empty}\captionof{figure}{\tiny\textit{Copyrigth: Mustafa Kamil, Egypt, 2021}}}
    \end{minipage}\hfill
    \begin{minipage}{0.18\linewidth}
      \centering
      \includegraphics[width=\linewidth]{examples/gfx/cef_arab-states_0_ef285d07_6.jpg}
      {\captionsetup{labelformat=empty}\captionof{figure}{\tiny\textit{Copyrigth: National Heritage Preservation, Ministry of Culture, Youth and Sport and Relations with the Parliament, Egypt, 2022}}}
    \end{minipage}\hfill
    \begin{minipage}{0.18\linewidth}
      \centering
      \includegraphics[width=\linewidth]{examples/gfx/cef_arab-states_0_ef285d07_7.jpg}
      {\captionsetup{labelformat=empty}\captionof{figure}{\tiny\textit{Copyrigth: Direction du Patrimoine Culturel, Morocco, 2021}}}
    \end{minipage}\hfill
    \begin{minipage}{0.18\linewidth}
      \centering
      \includegraphics[width=\linewidth]{examples/gfx/cef_arab-states_0_ef285d07_8.jpg}
      {\captionsetup{labelformat=empty}\captionof{figure}{\tiny\textit{Copyrigth: Direction du Patrimoine Culturel, Morocco, 2021}}}
    \end{minipage}\hfill
    \begin{minipage}{0.18\linewidth}
      \centering
      \includegraphics[width=\linewidth]{examples/gfx/cef_arab-states_0_ef285d07_9.jpg}
      {\captionsetup{labelformat=empty}\captionof{figure}{\tiny\textit{Copyrigth: Ministry of Culture, Palestine, 2021}}}
    \end{minipage}\hfill
  \end{center}
\end{tcolorbox}
\end{figure}
%
\subsubsection*{\RegAP}
%
\begin{figure}[H]
\begin{tcolorbox}[colback=gray!5!white,colframe=black!75!black,fonttitle=\bfseries\scriptsize,fontupper=\ttfamily\footnotesize,segmentation style={solid, black!30}]
  \begin{center}
    \begin{minipage}{0.18\linewidth}
      \centering
      \includegraphics[width=\linewidth]{examples/gfx/coqa_9dcf96ed_frame0_asian-and-pacific-states_0.png}
      {\captionsetup{labelformat=empty}\captionof{figure}{\tiny\textit{Copyrigth: Public Foundation 'Min Kiyal', Kyrgyzstan, 2018}}}
    \end{minipage}\hfill
    \begin{minipage}{0.18\linewidth}
      \centering
      \includegraphics[width=\linewidth]{examples/gfx/coqa_9dcf96ed_frame1_asian-and-pacific-states_0.png}
      {\captionsetup{labelformat=empty}\captionof{figure}{\tiny\textit{Copyrigth: Public Foundation 'Min Kiyal', Kyrgyzstan, 2018}}}
    \end{minipage}\hfill
    \begin{minipage}{0.18\linewidth}
      \centering
      \includegraphics[width=\linewidth]{examples/gfx/coqa_9dcf96ed_frame2_asian-and-pacific-states_0.png}
      {\captionsetup{labelformat=empty}\captionof{figure}{\tiny\textit{Copyrigth: Public Foundation 'Min Kiyal', Kyrgyzstan, 2018}}}
    \end{minipage}\hfill
    \begin{minipage}{0.18\linewidth}
      \centering
      \includegraphics[width=\linewidth]{examples/gfx/coqa_9dcf96ed_frame3_asian-and-pacific-states_0.png}
      {\captionsetup{labelformat=empty}\captionof{figure}{\tiny\textit{Copyrigth: Public Foundation 'Min Kiyal', Kyrgyzstan, 2018}}}
    \end{minipage}\hfill
    \begin{minipage}{0.18\linewidth}
      \centering
      \includegraphics[width=\linewidth]{examples/gfx/coqa_9dcf96ed_frame4_asian-and-pacific-states_0.png}
      {\captionsetup{labelformat=empty}\captionof{figure}{\tiny\textit{Copyrigth: Public Foundation 'Min Kiyal', Kyrgyzstan, 2018}}}
    \end{minipage}\hfill
  \\[4mm]
    \begin{minipage}{0.18\linewidth}
      \centering
      \includegraphics[width=\linewidth]{examples/gfx/coqa_9dcf96ed_frame5_asian-and-pacific-states_0.png}
      {\captionsetup{labelformat=empty}\captionof{figure}{\tiny\textit{Copyrigth: Public Foundation 'Min Kiyal', Kyrgyzstan, 2018}}}
    \end{minipage}\hfill
    \begin{minipage}{0.18\linewidth}
      \centering
      \includegraphics[width=\linewidth]{examples/gfx/coqa_9dcf96ed_frame6_asian-and-pacific-states_0.png}
      {\captionsetup{labelformat=empty}\captionof{figure}{\tiny\textit{Copyrigth: Public Foundation 'Min Kiyal', Kyrgyzstan, 2018}}}
    \end{minipage}\hfill
    \begin{minipage}{0.18\linewidth}
      \centering
      \includegraphics[width=\linewidth]{examples/gfx/coqa_9dcf96ed_frame7_asian-and-pacific-states_0.png}
      {\captionsetup{labelformat=empty}\captionof{figure}{\tiny\textit{Copyrigth: Public Foundation 'Min Kiyal', Kyrgyzstan, 2018}}}
    \end{minipage}\hfill
    \begin{minipage}{0.18\linewidth}
      \centering
      \includegraphics[width=\linewidth]{examples/gfx/coqa_9dcf96ed_frame8_asian-and-pacific-states_0.png}
      {\captionsetup{labelformat=empty}\captionof{figure}{\tiny\textit{Copyrigth: Public Foundation 'Min Kiyal', Kyrgyzstan, 2018}}}
    \end{minipage}\hfill
    \begin{minipage}{0.18\linewidth}
      \centering
      \includegraphics[width=\linewidth]{examples/gfx/coqa_9dcf96ed_frame9_asian-and-pacific-states_0.png}
      {\captionsetup{labelformat=empty}\captionof{figure}{\tiny\textit{Copyrigth: Public Foundation 'Min Kiyal', Kyrgyzstan, 2018}}}
    \end{minipage}\hfill
  \end{center}

  {\Large{Question:}} {\large{In which of the following countries does the event shown in the images take place? Choose from the following options and output only the corresponding letter.

A. Kyrgyzstan

B. Timor-Leste

C. Thailand

D. Turkmenistan


Your answer letter:}}\\
  {\Large{Answer:}} {\large{A}}\\
   \tcbline
  {\Large{Related Cultural Event or Facet}}\\[4mm]
  {\normalsize{Title:}} {\normalsize{Ak-kalpak craftsmanship, traditional knowledge and skills in making and wearing Kyrgyz men’s headwear}}\\
  {\normalsize{Countries:}} Kyrgyzstan\\
  {\normalsize{Regions:}} Asian and Pacific States\\
  {\normalsize{Description:}}\\
  Ak-kalpak craftsmanship is a traditional Kyrgyz handicraft. The Ak-kalpak is a traditional male hat made with white felt, which bears deep sacral meanings. Ak-kalpak craftsmanship is a cumulative, ever-evolving body of knowledge and skills passed down by craftswomen in the communities concerned comprising felting, cutting and sewing and pattern embroidery. Related knowledge and skills are transmitted via oral coaching, hands-on training and joint making in workshops. More than eighty kinds of Ak-kalpak can be distinguished, decorated with various patterns bearing a sacred meaning and history. Environmentally friendly and comfortable, the Ak-kalpak resembles a snow peak, with four sides representing the four elements: air, water, fire and earth. The four edging lines symbolize life, with the tassels on the top symbolizing ancestors’ posterity and memory, and the pattern symbolizing the family tree. Ak-kalpak unites different Kyrgyz tribes and communities and makes Kyrgyz people recognizable to other ethnic groups. It also fosters inclusivity when representatives of other ethnic groups wear it on holidays or days of mourning to express unity and sympathy. There are workshops all over the country where related knowledge and skills are passed down, and in 2013 a project entitled ‘From generation to generation’ was conducted on traditional Ak-kalpak-making techniques nationwide, resulting in an exhibition and published book.\\[2mm]
  {\normalsize{UNESCO ICH URL:}} \href{https://ich.unesco.org/en/RL/ak-kalpak-craftsmanship-traditional-knowledge-and-skills-in-making-and-wearing-kyrgyz-men-s-headwear-01496}{https://ich.unesco.org/en/RL/ak-kalpak-craftsmanship-traditi...}
\end{tcolorbox}
\end{figure}
%
\subsubsection*{\RegE}
%
\begin{figure}[H]
\begin{tcolorbox}[colback=gray!5!white,colframe=black!75!black,fonttitle=\bfseries\scriptsize,fontupper=\ttfamily\footnotesize,segmentation style={solid, black!30}]
  \begin{center}
    \begin{minipage}{0.5\linewidth}
      \centering
      \includegraphics[width=\linewidth]{examples/gfx/sivqa_dc9bc5ad_eastern-european-states_0.png}
      {\captionsetup{labelformat=empty}\captionof{figure}{\tiny\textit{Copyrigth: 2010 by M.Rahimov/Ministry of Culture and Tourism}}}
    \end{minipage}\hfill
  \end{center}
  {\Large{Question:}} {\large{What is the name of the musical instrument observed by the man in the image?}}\\
  {\Large{Answer:}} {\large{Tar}}\\
   \tcbline
  {\Large{Related Cultural Event or Facet}}\\[4mm]
  {\normalsize{Title:}} {\normalsize{Craftsmanship and performance art of the Tar, a long-necked string musical instrument}}\\
  {\normalsize{Countries:}} Azerbaijan\\
  {\normalsize{Regions:}} Eastern European States\\
  {\normalsize{Description:}}\\
  The Tar is a long-necked plucked lute, traditionally crafted and performed in communities throughout Azerbaijan. Considered by many to be the country’s leading musical instrument, it features alone or with other instruments in numerous traditional musical styles. Tar makers transmit their skills to apprentices, often within the family. Craftsmanship begins with careful selection of materials for the instrument: mulberry wood for the body, nut wood for the neck, and pear wood for the tuning pegs. Using various tools, crafters create a hollow body in the form of a figure eight, which is then covered with the thin pericardium of an ox. The fretted neck is affixed, metal strings are added and the body is inlaid with mother-of-pearl. Performers hold the instrument horizontally against the chest and pluck the strings with a plectrum, while using trills and a variety of techniques and strokes to add colour. Tar performance has an essential place in weddings and different social gatherings, festive events and public concerts. Players transmit their skills to young people within their community by word of mouth and demonstration, and at educational musical institutions. Craftsmanship and performance of the tar and the skills related to this tradition play a significant role in shaping the cultural identity of Azerbaijanis.\\[2mm]
  {\normalsize{UNESCO ICH URL:}} \href{https://ich.unesco.org/en/RL/craftsmanship-and-performance-art-of-the-tar-a-long-necked-string-musical-instrument-00671}{https://ich.unesco.org/en/RL/craftsmanship-and-performance-a...}
\end{tcolorbox}
\end{figure}
%
\subsubsection*{\RegLAC}
%
\begin{figure}[H]
\begin{tcolorbox}[colback=gray!5!white,colframe=black!75!black,fonttitle=\bfseries\scriptsize,fontupper=\ttfamily\footnotesize,segmentation style={solid, black!30}]
  \begin{center}
    \begin{minipage}{0.18\linewidth}
      \centering
      \includegraphics[width=\linewidth]{examples/gfx/coqa_7cf960f8_frame0_latin-american-and-caribbean-states_0.png}
      {\captionsetup{labelformat=empty}\captionof{figure}{\tiny\textit{Copyrigth: Gerson Fonseca/Ministry of Culture of Colombia, 2018}}}
    \end{minipage}\hfill
    \begin{minipage}{0.18\linewidth}
      \centering
      \includegraphics[width=\linewidth]{examples/gfx/coqa_7cf960f8_frame1_latin-american-and-caribbean-states_0.png}
      {\captionsetup{labelformat=empty}\captionof{figure}{\tiny\textit{Copyrigth: Gerson Fonseca/Ministry of Culture of Colombia, 2018}}}
    \end{minipage}\hfill
    \begin{minipage}{0.18\linewidth}
      \centering
      \includegraphics[width=\linewidth]{examples/gfx/coqa_7cf960f8_frame2_latin-american-and-caribbean-states_0.png}
      {\captionsetup{labelformat=empty}\captionof{figure}{\tiny\textit{Copyrigth: Gerson Fonseca/Ministry of Culture of Colombia, 2018}}}
    \end{minipage}\hfill
    \begin{minipage}{0.18\linewidth}
      \centering
      \includegraphics[width=\linewidth]{examples/gfx/coqa_7cf960f8_frame3_latin-american-and-caribbean-states_0.png}
      {\captionsetup{labelformat=empty}\captionof{figure}{\tiny\textit{Copyrigth: Gerson Fonseca/Ministry of Culture of Colombia, 2018}}}
    \end{minipage}\hfill
    \begin{minipage}{0.18\linewidth}
      \centering
      \includegraphics[width=\linewidth]{examples/gfx/coqa_7cf960f8_frame4_latin-american-and-caribbean-states_0.png}
      {\captionsetup{labelformat=empty}\captionof{figure}{\tiny\textit{Copyrigth: Gerson Fonseca/Ministry of Culture of Colombia, 2018}}}
    \end{minipage}\hfill
  \\[4mm]
    \begin{minipage}{0.18\linewidth}
      \centering
      \includegraphics[width=\linewidth]{examples/gfx/coqa_7cf960f8_frame5_latin-american-and-caribbean-states_0.png}
      {\captionsetup{labelformat=empty}\captionof{figure}{\tiny\textit{Copyrigth: Gerson Fonseca/Ministry of Culture of Colombia, 2018}}}
    \end{minipage}\hfill
    \begin{minipage}{0.18\linewidth}
      \centering
      \includegraphics[width=\linewidth]{examples/gfx/coqa_7cf960f8_frame6_latin-american-and-caribbean-states_0.png}
      {\captionsetup{labelformat=empty}\captionof{figure}{\tiny\textit{Copyrigth: Gerson Fonseca/Ministry of Culture of Colombia, 2018}}}
    \end{minipage}\hfill
    \begin{minipage}{0.18\linewidth}
      \centering
      \includegraphics[width=\linewidth]{examples/gfx/coqa_7cf960f8_frame7_latin-american-and-caribbean-states_0.png}
      {\captionsetup{labelformat=empty}\captionof{figure}{\tiny\textit{Copyrigth: Gerson Fonseca/Ministry of Culture of Colombia, 2018}}}
    \end{minipage}\hfill
    \begin{minipage}{0.18\linewidth}
      \centering
      \includegraphics[width=\linewidth]{examples/gfx/coqa_7cf960f8_frame8_latin-american-and-caribbean-states_0.png}
      {\captionsetup{labelformat=empty}\captionof{figure}{\tiny\textit{Copyrigth: Gerson Fonseca/Ministry of Culture of Colombia, 2018}}}
    \end{minipage}\hfill
    \begin{minipage}{0.18\linewidth}
      \centering
      \includegraphics[width=\linewidth]{examples/gfx/coqa_7cf960f8_frame9_latin-american-and-caribbean-states_0.png}
      {\captionsetup{labelformat=empty}\captionof{figure}{\tiny\textit{Copyrigth: Gerson Fonseca/Ministry of Culture of Colombia, 2018}}}
    \end{minipage}\hfill
  \end{center}

  {\Large{Question:}} {\large{In which of the following countries does the event shown in the images take place? Choose from the following options and output only the corresponding letter.

A. Dominican Republic

B. Chile

C. Colombia

D. Grenada


Your answer letter:}}\\
  {\Large{Answer:}} {\large{C}}\\
   \tcbline
  {\Large{Related Cultural Event or Facet}}\\[4mm]
  {\normalsize{Title:}} {\normalsize{Safeguarding strategy of traditional crafts for peace building}}\\
  {\normalsize{Countries:}} Colombia\\
  {\normalsize{Regions:}} Latin-American and Caribbean States\\
  {\normalsize{Description:}}\\
  The safeguarding strategy of traditional crafts for peace building addresses the weakening of traditional crafts through a system of intergenerational transmission of knowledge between master and apprentice based on the non-formal ‘learning by doing’ method. The safeguarding strategy aims to train different sectors of the population, create labour connections and foster cultural entrepreneurship. It establishes a link between bearers of traditional crafts and skills who are recognized by their communities for their empirical knowledge of the peculiarities of their region and apprentices aged between fourteen and thirty-five who become builders of peace by learning a skill or craft, seeking to transform their situation of vulnerability. The safeguarding strategy is therefore geared at: allowing for the qualification of traditional crafts, thereby improving employment opportunities; implementing a Traditional Crafts Policy to guide and ensure continuity in the transmission and practice of these crafts; and enhancing the Workshop Schools Programme. Priority is accorded to young people who are exposed to the effects of armed conflict, a lack of opportunities, school desertion and unemployment. Training is also combined with work, guaranteeing apprentices’ future employability. The strategy thus aims to foster the safeguarding of traditional crafts as a tool for social inclusion, employment and cultural entrepreneurship. In turn, the community can recognize the cultural and societal value of safeguarding different traditional skills and crafts.\\[2mm]
  {\normalsize{UNESCO ICH URL:}} \href{https://ich.unesco.org/en/BSP/safeguarding-strategy-of-traditional-crafts-for-peace-building-01480}{https://ich.unesco.org/en/BSP/safeguarding-strategy-of-tradi...}
\end{tcolorbox}
\end{figure}
%
\subsubsection*{\RegSA}
%
\begin{figure}[H]
\begin{tcolorbox}[colback=gray!5!white,colframe=black!75!black,fonttitle=\bfseries\scriptsize,fontupper=\ttfamily\footnotesize,segmentation style={solid, black!30}]
  \begin{center}
    \begin{minipage}{0.18\linewidth}
      \centering
      \includegraphics[width=\linewidth]{examples/gfx/coqa_6ce004b9_frame0_subsaharian-african-states_0.png}
      {\captionsetup{labelformat=empty}\captionof{figure}{\tiny\textit{Copyrigth: Etienne Kokolo, Kinshasa, République du Congo, 2018}}}
    \end{minipage}\hfill
    \begin{minipage}{0.18\linewidth}
      \centering
      \includegraphics[width=\linewidth]{examples/gfx/coqa_6ce004b9_frame1_subsaharian-african-states_0.png}
      {\captionsetup{labelformat=empty}\captionof{figure}{\tiny\textit{Copyrigth: Etienne Kokolo, Kinshasa, République du Congo, 2019}}}
    \end{minipage}\hfill
    \begin{minipage}{0.18\linewidth}
      \centering
      \includegraphics[width=\linewidth]{examples/gfx/coqa_6ce004b9_frame2_subsaharian-african-states_0.png}
      {\captionsetup{labelformat=empty}\captionof{figure}{\tiny\textit{Copyrigth: Etienne Kokolo, Kinshasa, République du Congo, 2018}}}
    \end{minipage}\hfill
    \begin{minipage}{0.18\linewidth}
      \centering
      \includegraphics[width=\linewidth]{examples/gfx/coqa_6ce004b9_frame3_subsaharian-african-states_0.png}
      {\captionsetup{labelformat=empty}\captionof{figure}{\tiny\textit{Copyrigth: Etienne Kokolo, Kinshasa, République du Congo, 2018}}}
    \end{minipage}\hfill
    \begin{minipage}{0.18\linewidth}
      \centering
      \includegraphics[width=\linewidth]{examples/gfx/coqa_6ce004b9_frame4_subsaharian-african-states_0.png}
      {\captionsetup{labelformat=empty}\captionof{figure}{\tiny\textit{Copyrigth: Etienne Kokolo, Kinshasa, République du Congo, 2018}}}
    \end{minipage}\hfill
  \\[4mm]
    \begin{minipage}{0.18\linewidth}
      \centering
      \includegraphics[width=\linewidth]{examples/gfx/coqa_6ce004b9_frame5_subsaharian-african-states_0.png}
      {\captionsetup{labelformat=empty}\captionof{figure}{\tiny\textit{Copyrigth: Etienne Kokolo, Kinshasa, République du Congo, 2017}}}
    \end{minipage}\hfill
    \begin{minipage}{0.18\linewidth}
      \centering
      \includegraphics[width=\linewidth]{examples/gfx/coqa_6ce004b9_frame6_subsaharian-african-states_0.png}
      {\captionsetup{labelformat=empty}\captionof{figure}{\tiny\textit{Copyrigth: Etienne Kokolo, Kinshasa, République du Congo, 2018}}}
    \end{minipage}\hfill
    \begin{minipage}{0.18\linewidth}
      \centering
      \includegraphics[width=\linewidth]{examples/gfx/coqa_6ce004b9_frame7_subsaharian-african-states_0.png}
      {\captionsetup{labelformat=empty}\captionof{figure}{\tiny\textit{Copyrigth: Etienne Kokolo, Kinshasa, République du Congo, 2020}}}
    \end{minipage}\hfill
    \begin{minipage}{0.18\linewidth}
      \centering
      \includegraphics[width=\linewidth]{examples/gfx/coqa_6ce004b9_frame8_subsaharian-african-states_0.png}
      {\captionsetup{labelformat=empty}\captionof{figure}{\tiny\textit{Copyrigth: Etienne Kokolo, Kinshasa, République du Congo, 2017}}}
    \end{minipage}\hfill
    \begin{minipage}{0.18\linewidth}
      \centering
      \includegraphics[width=\linewidth]{examples/gfx/coqa_6ce004b9_frame9_subsaharian-african-states_0.png}
      {\captionsetup{labelformat=empty}\captionof{figure}{\tiny\textit{Copyrigth: Etienne Kokolo, Kinshasa, République du Congo, 2020}}}
    \end{minipage}\hfill
  \end{center}

  {\Large{Question:}} {\large{In which of the following countries does the event shown in the images take place? Choose from the following options and output only the corresponding letter.

A. Congo

B. Togo

C. Namibia

D. Nigeria


Your answer letter:}}\\
  {\Large{Answer:}} {\large{A}}\\
   \tcbline
  {\Large{Related Cultural Event or Facet}}\\[4mm]
  {\normalsize{Title:}} {\normalsize{Congolese rumba}}\\
  {\normalsize{Countries:}} Congo, Democratic Republic of the Congo\\
  {\normalsize{Regions:}} Subsaharian African States\\
  {\normalsize{Description:}}\\
  Congolese rumba is a musical genre and a dance common in urban areas of the Democratic Republic of the Congo and the Republic of the Congo. Generally danced by a male-female couple, it is a multicultural form of expression originating from an ancient dance called nkumba (meaning ‘waist’ in Kikongo). The rumba is used for celebration and mourning, in private, public and religious spaces. It is performed by professional and amateur orchestras, choirs, dancers and individual musicians, and women have played a predominant role in the development of religious and romantic styles. The tradition of Congolese rumba is passed down to younger generations through neighbourhood clubs, formal training schools and community organisations. For instance, rumba musicians maintain clubs and apprentice artists to carry on the practice and the manufacture of instruments. The rumba also plays an important economic role, as orchestras are increasingly developing cultural entrepreneurship aimed at reducing poverty. The rumba is considered an essential and representative part of the identity of Congolese people and its diaspora. It is perceived as a means of conveying the social and cultural values of the region and of promoting intergenerational and social cohesion and solidarity.\\[2mm]
  {\normalsize{UNESCO ICH URL:}} \href{https://ich.unesco.org/en/RL/congolese-rumba-01711}{https://ich.unesco.org/en/RL/congolese-rumba-01711...}
\end{tcolorbox}
\end{figure}
%
\subsubsection*{\RegW}
%
\begin{figure}[H]
\begin{tcolorbox}[colback=gray!5!white,colframe=black!75!black,fonttitle=\bfseries\scriptsize,fontupper=\ttfamily\footnotesize,segmentation style={solid, black!30}]
  \begin{center}
    \begin{minipage}{0.18\linewidth}
      \centering
      \includegraphics[width=\linewidth]{examples/gfx/vvqa_71e7f729_frame0_western-european-and-north-american-states_0.png}
    \end{minipage}\hfill
    \begin{minipage}{0.18\linewidth}
      \centering
      \includegraphics[width=\linewidth]{examples/gfx/vvqa_71e7f729_frame1_western-european-and-north-american-states_0.png}
    \end{minipage}\hfill
    \begin{minipage}{0.18\linewidth}
      \centering
      \includegraphics[width=\linewidth]{examples/gfx/vvqa_71e7f729_frame2_western-european-and-north-american-states_0.png}
    \end{minipage}\hfill
    \begin{minipage}{0.18\linewidth}
      \centering
      \includegraphics[width=\linewidth]{examples/gfx/vvqa_71e7f729_frame3_western-european-and-north-american-states_0.png}
    \end{minipage}\hfill
    \begin{minipage}{0.18\linewidth}
      \centering
      \includegraphics[width=\linewidth]{examples/gfx/vvqa_71e7f729_frame4_western-european-and-north-american-states_0.png}
    \end{minipage}\hfill
  \\[4mm]
    \begin{minipage}{0.18\linewidth}
      \centering
      \includegraphics[width=\linewidth]{examples/gfx/vvqa_71e7f729_frame5_western-european-and-north-american-states_0.png}
    \end{minipage}\hfill
    \begin{minipage}{0.18\linewidth}
      \centering
      \includegraphics[width=\linewidth]{examples/gfx/vvqa_71e7f729_frame6_western-european-and-north-american-states_0.png}
    \end{minipage}\hfill
    \begin{minipage}{0.18\linewidth}
      \centering
      \includegraphics[width=\linewidth]{examples/gfx/vvqa_71e7f729_frame7_western-european-and-north-american-states_0.png}
    \end{minipage}\hfill
    \begin{minipage}{0.18\linewidth}
      \centering
      \includegraphics[width=\linewidth]{examples/gfx/vvqa_71e7f729_frame8_western-european-and-north-american-states_0.png}
    \end{minipage}\hfill
    \begin{minipage}{0.18\linewidth}
      \centering
      \includegraphics[width=\linewidth]{examples/gfx/vvqa_71e7f729_frame9_western-european-and-north-american-states_0.png}
    \end{minipage}\hfill
  \end{center}

  {\Large{Question:}} {\large{What traditional practice is depicted with the herders and sheep in the video?}}\\
  {\Large{Answer:}} {\large{Transhumance}}\\
   \tcbline
  {\Large{Related Cultural Event or Facet}}\\[4mm]
  {\normalsize{Title:}} {\normalsize{Transhumance, the seasonal droving of livestock}}\\
  {\normalsize{Countries:}} Albania, Andorra, Austria, Croatia, Spain, France, Greece, Italy, Luxembourg, Romania\\
  {\normalsize{Regions:}} Western European and North American States, Eastern European States\\
  {\normalsize{Description:}}\\
  Transhumance refers to the seasonal movement of people with their livestock between geographical or climatic regions. Each year, in spring and autumn, men and women herders organise the movement of thousands of animals along traditional pastoral paths. They move on foot or horseback, leading with their dogs and sometimes accompanied by their families. An ancestral practice, transhumance stems from a deep knowledge about the environment and entails social practices and rituals related to the care, breeding and training of animals and the management of natural resources. An entire socio-economic system has been developed around transhumance, from gastronomy to local handicrafts and festivities marking the beginning and end of a season. Families have been enacting and transmitting transhumance through observation and practice for many generations. Communities living along transhumance routes also play an important role in its transmission, such as by celebrating herd crossings and organising festivals. The practice is also transmitted through workshops organised by local communities, associations and networks of herders and farmers, as well as through universities and research institutes. Transhumance thus contributes to social inclusion, strengthening cultural identity and ties between families, communities and territories while counteracting the effects of rural depopulation.\\[2mm]
  {\normalsize{UNESCO ICH URL:}} \href{https://ich.unesco.org/en/RL/transhumance-the-seasonal-droving-of-livestock-01964}{https://ich.unesco.org/en/RL/transhumance-the-seasonal-drovi...}
\end{tcolorbox}
\end{figure}
%

\subsection{Cultural Aspects}
\label{appedix:sec:sivqa:aspects}
%
During the synthetic data generation phase of the \sivqa, we also obtained a ``target aspect'' per question (see \S\ref{appendix:sec:sivqa:synth} and \S\ref{appendix:sec:sivqa:synth:sys_prompt}).
%
We report these aspects in the following.
%
\begin{table*}[ht!]
    \centering
    \small
    \begin{minipage}[t]{0.29\textwidth}
        \centering
        \begin{tabular}{lr}
            \toprule
            Aspect & Questions \\
            \midrule
            traditions & 390 \\
            rituals & 241 \\
            art & 233 \\
            music & 210 \\
            craftsmanship & 177 \\
            instruments & 155 \\
            festivals & 151 \\
            dance & 150 \\
            tools & 108 \\
            food & 96 \\
            clothing & 93 \\
            architecture & 52 \\
            sports & 38 \\
            location & 28 \\
            symbols & 19 \\
            drinks & 14 \\
            customs & 13 \\
            cultural significance & 6 \\
            theatre & 4 \\
            \bottomrule
        \end{tabular}
    \end{minipage}%
    \begin{minipage}[t]{0.29\textwidth}
        \centering
        \begin{tabular}{lr}
            \toprule
            Aspect & Questions \\
            \midrule
            education & 3 \\
            culture & 3 \\
            games & 3 \\
            performing arts & 3 \\
            language & 3 \\
            performance & 3 \\
            characters & 2 \\
            practices & 2 \\
            skills & 2 \\
            origin & 2 \\
            cultural identity & 2 \\
            technology & 1 \\
            people & 1 \\
            community & 1 \\
            identity & 1 \\
            environment & 1 \\
            traditional medicine & 1 \\
            nature & 1 \\
            communication & 1 \\
            \bottomrule
        \end{tabular}
    \end{minipage}%
    \begin{minipage}[t]{0.29\textwidth}
        \centering
        \begin{tabular}{lr}
            \toprule
            Aspect & Questions \\
            \midrule
            jewelry & 1 \\
            objects & 1 \\
            animal & 1 \\
            plants & 1 \\
            process & 1 \\
            agriculture & 1 \\
            celebrations & 1 \\
            details & 1 \\
            historical & 1 \\
            function or usage & 1 \\
            symbolism & 1 \\
            healthcare & 1 \\
            knowledge & 1 \\
            social status & 1 \\
            religion & 1 \\
            cultural space & 1 \\
            social space & 1 \\
            cultural practice & 1 \\
            unknown & 1 \\
            \bottomrule
        \end{tabular}
    \end{minipage}
    \caption{Cultural aspects targeted by the questions within the \sivqa task.}
    \label{tab:sivqa:aspects}
\end{table*}

%
\subsection{External Hint Variations}
%
\label{appendix:sec:sivqa:hints}
%
For the \sivqa (and \vvqa) task, we ablate the effect of external cues or hints on the task performance of models.
%
In the following, we provide the Python pseudo-code snippet to generate the prompt for a given sample.
%
\begin{figure*}[ht!]
    \centering
    %
    \begin{promptbox}{Python Pseudo-Code for the external cue settings of the \sivqa and \vvqa tasks.}
    \begin{minted}[breaklines]{python}
def apply_gimmick_prompt_template(
    sample: dict[str, Any],
    regions_hint: bool,
    countries_hint: bool,
) -> str:
    
    prompt_template = "{QUESTION}\n{HINTS}\n"
    hints = ""

    if regions_hint:
        hints += (
            "Hint: The question is related to a cultural event or facet from the following region(s): "
            f"{', '.join(sample['regions'])}\n"
        )

    if countries_hint:
        hints += (
            "Hint: The question is related to a cultural event or facet from the following country or countries: "
            f"{', '.join(sample['countries'])}\n"
        )

    return prompt_template.format(
        QUESTION=sample["prompt"],
        HINTS=hints,
    )
    \end{minted}
    \end{promptbox}
    \label{fig:sivqa:hints}
    \caption{Python Pseudo-Code to generate the prompt for a given \sivqa (or \vvqa) sample for the external cues settings.}
\end{figure*}
%



\subsection{Synthetic Data Generation}
\label{appendix:sec:sivqa:synth}
%
\onecolumn
\newpage
\subsubsection{System Prompt}
\label{appendix:sec:sivqa:synth:sys_prompt}
%
\begin{tcolorbox}[
    enhanced, 
    breakable,
    skin first=enhanced,
    skin middle=enhanced,
    skin last=enhanced,
]
\begin{minted}[fontsize=\footnotesize,breaklines]{markdown}
# Your Role

You are a professional annotator specialized in creating VQA samples based on a provided intangible cultural heritage(ICH) item. You will be given the following information related to the item:

- Image: An image representing one aspect of the ICH item.
- Countries of Origin: The country or countries where this ICH is recognized.
- Regions of Origin: The country or countries where this ICH is recognized.
- Title: The official title of the ICH item.
- Description: A detailed description of the ICH item, including relevant details.

# Your Task

Your task is it to generate high-quality question-answer pairs in a VQA style to assess the cultural knowledge of the intangible cultural heritage (ICH) item of state-of-the-art multimodal AI models. Be sure to follow the annotation guidelines provided below to ensure the quality and relevance of the question-answer pairs.

# Annotation Guidelines

## Question Requirements

Make sure the question meets all of the following requirements:

1. Clear and Concise
    The question is clear and concise and no longer than a single sentence.
2. Directly related to the ICH item
    The question is directly related to the ICH item.
3. Directly related to the visible content
    The question is directly related to the visible content in the image and requires visual analysis to answer.
4. Does not (partially) contain the answer
    The question does not contain any hints or clues to or parts of the answer that would make the answer obvious.
5. Does not contain subjective words
    The question does not contain subjective words like 'likely', 'possibly', 'probably', 'eventually', 'might', 'could', 'should', etc., which could introduce ambiguity.
6. Requires both image and cultural knowledge to answer
    The question requires both image and cultural knowledge to answer and is not answerable by looking only at the image or only knowing about the ICH item or reading the textual description.
7. (optional) Includes specific cultural terms
    The answer includes specific cultural terms, names, or phrases related to the ICH item. E.g., particular names mentioned in the description or parts of the title.

## Answer Requirements

Make sure the answer meets all of the following requirements:

1. Single Word or Multiword Expression
    The answer is a single word or multiword expression.
2. Clear, Objective, and Correct
    The answer is clear, objective, and unambiguously correct.
3. Directly Related to Visual Content
    The answer is directly related to the visual content of the image.
4. No General or Abstract Words
    The answer does not contain general, abstract, or non-depictable words like "Traditional", "Cooperation", "Gathering", "Solidarity", "Community", "Indoor", "Outdoor", "Urban", "Rural", etc.
5. Verifiable by Text and Image
    The answer is unambiguously verifiable by reading the textual information and inspecting the image.
6. (optional) Includes specific cultural terms
    The answer includes specific cultural terms, names, or phrases related to the ICH item. E.g., particular names mentioned in the description or parts of the title.

## Question Characteristics

### Target Aspects

Make sure the question targets different aspects of the ICH item, such as:

- Food
- Drinks
- Clothing
- Art
- Tools
- Sports
- Instruments
- Dance
- Music
- Rituals
- Traditions
- Festivals
- Customs
- Symbols
- Architecture
- Other

### Question Categories

Make sure the question falls into different categories, such as:

- Identification
    Questions that ask for the identification of objects, people, or elements in the image. E.g.: What is the name of the instrument shown in the image?
- Origin
    Questions that inquire about the origin or source of the CEF. E.g.: Which culture or country does this artifact belong to?
- Cultural Significance
    Questions that explore the cultural or religious significance of the depicted element. E.g.: What cultural or religious significance does this item hold in its native context?
- Function or Usage
    Questions that ask about the traditional or historical function or usage of the depicted element. E.g.: What was this object traditionally used for?
- Material and Craftsmanship
    Questions that focus on the materials used and the craftsmanship involved in creating the depicted element. E.g.: What material is used to construct this artifact?
- Location
    Questions that ask about the geographical location where the cultural event or facet takes place. E.g.: In which place does this dance take place?
- Symbolism
    Questions that delve into the symbolic meanings associated with the depicted element. E.g.: What does the color red symbolize in this cultural context?
- Historical
    Questions that relate to historical events or contexts depicted in the image. E.g.: What historical event is depicted in this image?
- Details
    Questions that ask for specific details about the formation, arrangement, or other aspects of the depicted element. E.g.: What formation are the dancers in?
- Other
    Questions that do not fall into the above categories but are relevant to the ICH item.

    
# Task Strategy

Before generating a question-answer pair, first think step-by-step and analyse the image:

1. What is visible in the image? Generate a highly detailed description of the key elements, objects, or people in the image. Take into account the textual description provided to identify details.
2. How does the visible content relate to the intangible cultural heritage item? Identify the connection between the contents of the image and the intangible cultural heritage item.

Then, think step-by-step about potential questions:

1. What can be asked about the image that is directly related to the visible content and the intangible cultural heritage item?
2. Can a concise and clear answer to the questions be inferred from the image and the provided information?

Finally, think step-by-step before generating the final question-answer pairs:

1. Does the question-answer pair strictly adhere to the guidelines provided above? Percisly check every part of the guidelines and drop the question-answer pair if it does not meet the criteria.
2. What aspect of the intangible cultural heritage item is targeted with the question?
3. What category does the question fall into?

# Output Format

For each question-answer pair, provide the following information in the following format:
```xml
<vqa-task>
    <image-analysis>
        <description>
            <!-- PUT YOUR DETAILED DESCRIPTION OF THE IMAGE HERE -->
        </description>
        <cultural-relatetness>
            <!-- PUT YOUR ANALYSIS OF HOW THE CONTENTS OF THE IMAGE RELATE TO THE INTANGIBLE CULTURAL HERITAGE ITEM HERE -->
        </cultural-relatetness>
    </image-analysis>
    <potential-questions>
        <qa-candidate>
            <question>
                <!-- PUT YOUR QUESTION HERE -->
            </question>
            <answer>
                <!-- PUT YOUR ANSWER HERE -->
            </answer>
            <guideline-adherence>
                <question-requirments>
                    <clear-and-concise>
                        <!-- YES OR NO -->
                    </clear-and-concise>
                    <directly-related-to-ich>
                        <!-- YES OR NO -->
                    </directly-related-to-ich>
                    <directly-related-to-visual-content>
                        <!-- YES OR NO -->
                    </directly-related-to-visual-content>
                    <does-not-contain-answer>
                        <!-- YES OR NO -->
                    </does-not-contain-answer>
                    <does-not-contain-subjective-words>
                        <!-- YES OR NO -->
                    </does-not-contain-subjective-words>
                    <requires-both-image-and-cultural-knowledge>
                        <!-- YES OR NO -->
                    </requires-both-image-and-cultural-knowledge>
                    <includes-specific-cultural-terms>
                        <!-- YES OR NO -->
                    </includes-specific-cultural-terms>
                </question-requirments>
                <answer-requirments>
                    <single-word-or-multiword-expression>
                        <!-- YES OR NO -->
                    </single-word-or-multiword-expression>
                    <clear-objective-and-correct>
                        <!-- YES OR NO -->
                    </clear-objective-and-correct>
                    <directly-related-to-visual-content>
                        <!-- YES OR NO -->
                    </directly-related-to-visual-content>
                    <no-general-or-abstract-words>
                        <!-- YES OR NO -->
                    </no-general-or-abstract-words>
                    <verifiable-by-text-and-image>
                        <!-- YES OR NO -->
                    </verifiable-by-text-and-image>
                    <includes-specific-cultural-terms>
                        <!-- YES OR NO -->
                    </includes-specific-cultural-terms>
                </answer-requirments>
            </guideline-adherence>
        </qa-candidate>
        ...
    </potential-questions>
    <final-qa-pairs>
        <!-- PUT ALL QA PAIRS THAT MEET ALL MANDATORY REQUIREMENTS HERE -->
        <qa-pair>
            <meets-requirements>
                <!-- DOES YOUR QUESTION-ANSWER PAIR MEET ALL MANDATORY REQUIREMENTS? YES OR NO -->
            </meets-requirements>
            <final-result-json>
                <!-- PUT YOUR FINAL RESULT AS JSON HERE -->
                {
                    "question": <insert question here>,
                    "answer": <insert answer here>,
                    "target_aspect": <insert target aspect here>
                    "question_category": <insert question category here>
                }
            </final-result-json>
        </qa-pair>
        ...
    </final-qa-pairs>
</vqa-task>
```
\end{minted}
\end{tcolorbox}
%
\subsubsection{User Prompt Template}
\label{appendix:sec:sivqa:synth:usr_prompt}
%

\begin{tcolorbox}[
    enhanced, 
    breakable,
    skin first=enhanced,
    skin middle=enhanced,
    skin last=enhanced,
]
\begin{minted}[fontsize=\footnotesize,breaklines]{markdown}
# Intangible Cultural Heritage Item

### Image

{IMAGE_PLACEHOLDER}

### Countries of Origin:

{LIST_OF_COUNTRIES}

### Regions of Origin

{LIST_OF_REGIONS}

### Title

{TITLE}

### Description

{DESCRIPTION}

\end{minted}
\end{tcolorbox}

%
\twocolumn


%

\subsection{Annotation Project Details}
\label{appendix:sec:sivqa:anno}
%
We first conducted several internal pilot studies to iteratively create a straightforward annotation task, guidelines, and an intuitive interface for the final annotation project.
%
To find annotators, we advertised the task in our faculty research network, emphasizing our goal of creating a culturally diverse benchmark for assessing the cultural awareness of current AI models.
%
Therefore, we targeted primarily individuals from non-Western cultural backgrounds.
%
We found 18 volunteers who have spent most of their lives in 10 different countries from all six regions and thus cover diverse cultural backgrounds (see Table~\ref{tab:sivqa:anno:demographics}).
%
To train the annotators, we provided detailed annotation guidelines, followed by an oral introduction to the task.
%
For more details, refer to the (anonymized) original annotation guidelines we \href{https://drive.proton.me/urls/T6RHQCEW5G#5y0Itm2BdWYZ}{shared here}.
%

For the second annotation round, we hired 5 of the previous volunteering annotators (0, 1, 8, 15, 17) who assessed the kept samples from the first round to obtain two annotations (from distinct annotators) per sample.
%
We paid the second-round annotators a salary of roughly 12.5€ per hour.
%
\begin{table}[ht!]
    \centering
    \renewcommand{\arraystretch}{0.95}
    \resizebox{\linewidth}{!}{%
    \begin{tabular}{lrllllr}
        \toprule
        \textsc{ID} & \textsc{Age} & \textsc{Pronouns} & \textsc{Education} & \textsc{Country} & \textsc{Region} & \textsc{Round(s)} \\
        \midrule
        0 & 23 & she/her & Bachelor & Iran & \RegAP & 1, 2\\
        1 & 23 & she/her & Bachelor & Iran & \RegAP & 1, 2\\
        2 & 28 & she/her & PhD & Russia & \RegE & 1\\
        3 & 35 & he/him & Master & Germany & \RegW & 1 \\
        5 & 29 & he/him & Bachelor & Guatemala & \RegLAC & 1\\
        6 & 29 & he/him & Master & Germany & \RegW & 1\\
        7 & 42 & he/him & PhD & Ethiopia & \RegSA & 1\\
        8 & 23 & he/him & Bachelor & Egypt & \RegA & 1, 2\\
        9 & 33 & she/her & Master & Iran & \RegAP & 1\\
        10 & 29 & she/her & Bachelor & Afghanistan & \RegAP & 1\\
        11 & 23 & she/her & Bachelor & India & \RegAP & 1\\
        12 & 33 & he/him & Bachelor & Germany & \RegW & 1\\
        13 & 22 & she/her & Bachelor & Pakistan & \RegAP & 1\\
        14 & 27 & he/him & Master & China & \RegAP & 1\\
        15 & 29 & she/her & High School & Germany & \RegW & 1, 2\\
        16 & 22 & she/her & Bachelor & China & \RegAP & 1\\
        17 & 26 & he/him & High School & Germany & \RegW & 1, 2, 3\\
        \bottomrule
    \end{tabular}
    }%
    \caption{Demographics of the annotators who participated in our VQA annotation project. For the country, we asked the question, ``\textit{Where did you spend most of your life?}''. The Round(s) column indicates which annotation rounds the annotator participated in.}
    \label{tab:sivqa:anno:demographics}
\end{table}
%

\subsubsection{\sivqa Annotation Interface}
\label{appendix:sec:sivqa:anno:ui}
%
For the annotation project, we used a self-hosted Label Studio\footnote{\url{https://labelstud.io/}} instance with a custom labeling interface (see Figure~\ref{fig:sivqa:anno:ui}) for all annotation projects.
%
\begin{figure*}
    \centering
    \begin{subfigure}[b]{1.\textwidth}
         \centering
         \includegraphics[width=\textwidth]{gfx/anno-task-screenshot-sample-A.png}
     \end{subfigure}
     
     \begin{subfigure}[b]{1.\textwidth}
         \centering
         \includegraphics[width=\textwidth]{gfx/anno-task-screenshot-sample-B.png}
     \end{subfigure}

     \begin{subfigure}[b]{1.\textwidth}
         \centering
         \includegraphics[width=\textwidth]{gfx/anno-task-screenshot-sample-C.png}
     \end{subfigure}
    \caption{Three screenshots showing examples of the Label Studio interface used in our \sivqa annotation tasks.}
    \label{fig:sivqa:anno:ui}
\end{figure*}

{
\onecolumn
\subsubsection{First Annotation Round Statistics}
\label{appendix:sec:sivqa:anno:first_round}
%
\begin{table}[ht!]
    \centering
    \begin{minipage}[t]{0.50\textwidth}
        \scriptsize
        \centering
        \begin{tabular}{lr}
            \toprule
            Country & Count \\
            \midrule
            United Arab Emirates & 101 \\
            China & 98 \\
            Oman & 91 \\
            Saudi Arabia & 87 \\
            France & 86 \\
            Croatia & 84 \\
            Algeria & 82 \\
            Morocco & 81 \\
            Türkiye & 78 \\
            Peru & 75 \\
            Spain & 74 \\
            Azerbaijan & 69 \\
            Colombia & 68 \\
            Islamic Republic of Iran & 66 \\
            Mali & 65 \\
            Mexico & 64 \\
            Republic of Korea & 62 \\
            Egypt & 62 \\
            Tunisia & 56 \\
            Iraq & 54 \\
            Japan & 52 \\
            Brazil & 50 \\
            Italy & 50 \\
            Belgium & 50 \\
            Plurinational State of Bolivia & 49 \\
            Mauritania & 49 \\
            Bolivarian Republic of Venezuela & 47 \\
            Nigeria & 46 \\
            India & 45 \\
            Malawi & 43 \\
            Palestine & 40 \\
            Greece & 38 \\
            Uzbekistan & 37 \\
            Kuwait & 37 \\
            Kyrgyzstan & 36 \\
            Cuba & 35 \\
            Mauritius & 34 \\
            Mongolia & 34 \\
            Czechia & 34 \\
            Jordan & 32 \\
            Zambia & 31 \\
            Côte d'Ivoire & 31 \\
            Syrian Arab Republic & 31 \\
            Kazakhstan & 30 \\
            Portugal & 29 \\
            Switzerland & 29 \\
            Uganda & 29 \\
            Ethiopia & 29 \\
            Botswana & 28 \\
            Viet Nam & 28 \\
            Argentina & 28 \\
            Armenia & 28 \\
            Yemen & 28 \\
            Turkmenistan & 26 \\
            Sudan & 26 \\
            Bahrain & 26 \\
            Indonesia & 26 \\
            Ecuador & 25 \\
            Mozambique & 25 \\
            Tajikistan & 25 \\
            Austria & 24 \\
            Hungary & 24 \\
            Slovakia & 23 \\
            Lebanon & 23 \\
            Cyprus & 22 \\
            Slovenia & 22 \\
            Paraguay & 21 \\
            Germany & 21 \\
            Romania & 21 \\
            Guatemala & 20 \\
            Kenya & 20 \\
            Poland & 20 \\
            \bottomrule
        \end{tabular}
    \end{minipage}
    \hspace{-2.5cm}
    \begin{minipage}[t]{0.50\textwidth}
        \scriptsize
        \centering
        \begin{tabular}{lr}
        \toprule
        Country & Count \\
        \midrule
        Nicaragua & 18 \\
        Chile & 17 \\
        Serbia & 17 \\
        Cambodia & 17 \\
        Bangladesh & 17 \\
        Bulgaria & 17 \\
        Qatar & 17 \\
        Ireland & 17 \\
        Panama & 16 \\
        Ukraine & 16 \\
        Malaysia & 16 \\
        Namibia & 16 \\
        Philippines & 15 \\
        Bosnia and Herzegovina & 15 \\
        Niger & 15 \\
        Estonia & 14 \\
        Netherlands & 14 \\
        Zimbabwe & 14 \\
        Senegal & 14 \\
        Madagascar & 14 \\
        Belarus & 13 \\
        Luxembourg & 13 \\
        Togo & 12 \\
        Burundi & 12 \\
        Dominican Republic & 12 \\
        Congo & 11 \\
        Democratic Republic of the Congo & 11 \\
        Benin & 11 \\
        Finland & 11 \\
        Angola & 10 \\
        Afghanistan & 10 \\
        Seychelles & 10 \\
        Democratic People’s Republic of Korea & 10 \\
        Norway & 9 \\
        Lao Peoples Democratic Republic & 9 \\
        Burkina Faso & 9 \\
        Sweden & 9 \\
        Bahamas & 9 \\
        Georgia & 9 \\
        Albania & 9 \\
        Republic of Moldova & 9 \\
        Cabo Verde & 8 \\
        North Macedonia & 8 \\
        Jamaica & 8 \\
        Honduras & 7 \\
        Latvia & 7 \\
        Denmark & 7 \\
        Pakistan & 7 \\
        Belize & 7 \\
        Uruguay & 7 \\
        Timor-Leste & 6 \\
        Montenegro & 6 \\
        Sri Lanka & 6 \\
        Thailand & 6 \\
        Guinea & 6 \\
        Malta & 5 \\
        Andorra & 5 \\
        Russian Federation & 5 \\
        Lithuania & 5 \\
        Tonga & 4 \\
        Costa Rica & 4 \\
        Cameroon & 4 \\
        Vanuatu & 3 \\
        Singapore & 3 \\
        Gambia & 3 \\
        Iceland & 3 \\
        Federated States of Micronesia & 2 \\
        Grenada & 2 \\
        Samoa & 2 \\
        Bhutan & 1 \\
        Djibouti & 1 \\
        Central African Republic & 1 \\
        \bottomrule
        \end{tabular}
    \end{minipage}
    \caption{The number of countries related to the QA pairs collected in the first annotation round for \sivqa.}
    \label{tab:sivqa:anno:first_round}
\end{table}
}
