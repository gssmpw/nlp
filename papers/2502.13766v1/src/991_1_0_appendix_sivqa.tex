\section{\sivqa Details}
\label{appendix:sec:sivqa}
%

\clearpage
\subsection{Examples}
\label{appendix:sec:sivqa:examples}
%
In the following, we provide one random sample per region for the \sivqa task.
%
Note that the lower part of the examples, where the related CEF is provided, is \emph{not} part of the actual sample.

%
\subsubsection*{\RegA}
%
\begin{figure}[H]
\begin{tcolorbox}[colback=gray!5!white,colframe=black!75!black,fonttitle=\bfseries\scriptsize,fontupper=\ttfamily\footnotesize,segmentation style={solid, black!30}]
  \begin{center}
    \begin{minipage}{0.18\linewidth}
      \centering
      \includegraphics[width=\linewidth]{examples/gfx/vvqa_22a1bb26_frame0_arab-states_0.png}
    \end{minipage}\hfill
    \begin{minipage}{0.18\linewidth}
      \centering
      \includegraphics[width=\linewidth]{examples/gfx/vvqa_22a1bb26_frame1_arab-states_0.png}
    \end{minipage}\hfill
    \begin{minipage}{0.18\linewidth}
      \centering
      \includegraphics[width=\linewidth]{examples/gfx/vvqa_22a1bb26_frame2_arab-states_0.png}
    \end{minipage}\hfill
    \begin{minipage}{0.18\linewidth}
      \centering
      \includegraphics[width=\linewidth]{examples/gfx/vvqa_22a1bb26_frame3_arab-states_0.png}
    \end{minipage}\hfill
    \begin{minipage}{0.18\linewidth}
      \centering
      \includegraphics[width=\linewidth]{examples/gfx/vvqa_22a1bb26_frame4_arab-states_0.png}
    \end{minipage}\hfill
  \\[4mm]
    \begin{minipage}{0.18\linewidth}
      \centering
      \includegraphics[width=\linewidth]{examples/gfx/vvqa_22a1bb26_frame5_arab-states_0.png}
    \end{minipage}\hfill
    \begin{minipage}{0.18\linewidth}
      \centering
      \includegraphics[width=\linewidth]{examples/gfx/vvqa_22a1bb26_frame6_arab-states_0.png}
    \end{minipage}\hfill
    \begin{minipage}{0.18\linewidth}
      \centering
      \includegraphics[width=\linewidth]{examples/gfx/vvqa_22a1bb26_frame7_arab-states_0.png}
    \end{minipage}\hfill
    \begin{minipage}{0.18\linewidth}
      \centering
      \includegraphics[width=\linewidth]{examples/gfx/vvqa_22a1bb26_frame8_arab-states_0.png}
    \end{minipage}\hfill
    \begin{minipage}{0.18\linewidth}
      \centering
      \includegraphics[width=\linewidth]{examples/gfx/vvqa_22a1bb26_frame9_arab-states_0.png}
    \end{minipage}\hfill
  \end{center}

  {\Large{Question:}} {\large{What event are the women in the video participating in?}}\\
  {\Large{Answer:}} {\large{Moussem of Tan-Tan}}\\
   \tcbline
  {\Large{Related Cultural Event or Facet}}\\[4mm]
  {\normalsize{Title:}} {\normalsize{Moussem of Tan-Tan}}\\
  {\normalsize{Countries:}} Morocco\\
  {\normalsize{Regions:}} Arab States\\
  {\normalsize{Description:}}\\
  The Moussem of Tan-Tan in southwest Morocco is an annual gathering of nomadic peoples of the Sahara that brings together more than thirty tribes from southern Morocco and other parts of northwest Africa. Originally this was an annual event around the month of May. Part of the agricultural and herding calendar of the nomads, these gatherings were an opportunity to group together, buy, sell and exchange foodstuffs and other products, organize camel and horse-breeding competitions, celebrate weddings and consult herbalists. The Moussem also included a range of cultural expressions such as musical performances, popular chanting, games, poetry contests and other Hassanie oral traditions. 

These gatherings took the form of a Moussem (a type of annual fair with economic, cultural and social functions) in 1963 when the first Moussem of Tan-Tan was organized to promote local traditions and provide a place for exchange, meeting and celebration. The Moussem is said to have been initially associated with Mohamed Laghdaf, who resisted the Franco-Spanish occupation. He died in 1960, and his tomb lies near the town. However, between 1979 and 2004 it was not possible to hold the Moussem because of security problems in the region. 

Today, the nomadic populations are particularly concerned to protect their way of life. Economic and technical upheavals in the region have profoundly altered the lifestyle of the nomadic Bedouin communities, forcing many of them to settle. Moreover, urbanization and rural exodus have contributed to the loss of many aspects of the traditional culture of these populations, such as crafts and poetry. Because of these risks, Bedouin communities rely strongly on the renewed Moussem of Tan-Tan to assist them in ensuring the survival of their know-how and traditions.\\[2mm]
  {\normalsize{UNESCO ICH URL:}} \href{https://ich.unesco.org/en/RL/moussem-of-tan-tan-00168}{https://ich.unesco.org/en/RL/moussem-of-tan-tan-00168...}
\end{tcolorbox}
\end{figure}
%
\subsubsection*{\RegAP}
%
\begin{figure}[H]
\begin{tcolorbox}[colback=gray!5!white,colframe=black!75!black,fonttitle=\bfseries\scriptsize,fontupper=\ttfamily\footnotesize,segmentation style={solid, black!30}]
  \begin{center}
    \begin{minipage}{0.5\linewidth}
      \centering
      \includegraphics[width=\linewidth]{examples/gfx/sivqa_ad39cda3_asian-and-pacific-states_0.png}
      {\captionsetup{labelformat=empty}\captionof{figure}{\tiny\textit{Copyrigth: 2010 by Centre for Research and Development of Culture, Indonesia}}}
    \end{minipage}\hfill
  \end{center}
  {\Large{Question:}} {\large{What traditional dance are the performers engaging in, as seen in the image?}}\\
  {\Large{Answer:}} {\large{Saman dance}}\\
   \tcbline
  {\Large{Related Cultural Event or Facet}}\\[4mm]
  {\normalsize{Title:}} {\normalsize{Saman dance}}\\
  {\normalsize{Countries:}} Indonesia\\
  {\normalsize{Regions:}} Asian and Pacific States\\
  {\normalsize{Description:}}\\
  The Saman dance is part of the cultural heritage of the Gayo people of Aceh province in Sumatra. Boys and young men perform the Saman sitting on their heels or kneeling in tight rows. Each wears a black costume embroidered with colourful Gayo motifs symbolizing nature and noble values. The leader sits in the middle of the row and leads the singing of verses, mostly in the Gayo language. These offer guidance and can be religious, romantic or humorous in tone. Dancers clap their hands, slap their chests, thighs and the ground, click their fingers, and sway and twist their bodies and heads in time with the shifting rhythm – in unison or alternating with the moves of opposing dancers. These movements symbolize the daily lives of the Gayo people and their natural environment. The Saman is performed to celebrate national and religious holidays, cementing relationships between village groups who invite each other for performances. The frequency of Saman performances and its transmission are decreasing, however. Many leaders with knowledge of the Saman are now elderly and without successors. Other forms of entertainment and new games are replacing informal transmission, and many young people now emigrate to further their education. Lack of funds is also a constraint, as Saman costumes and performances involve considerable expense.\\[2mm]
  {\normalsize{UNESCO ICH URL:}} \href{https://ich.unesco.org/en/USL/saman-dance-00509}{https://ich.unesco.org/en/USL/saman-dance-00509...}
\end{tcolorbox}
\end{figure}
%
\subsubsection*{\RegE}
%
\begin{figure}[H]
\begin{tcolorbox}[colback=gray!5!white,colframe=black!75!black,fonttitle=\bfseries\scriptsize,fontupper=\ttfamily\footnotesize,segmentation style={solid, black!30}]
  \begin{center}
    \begin{minipage}{0.18\linewidth}
      \centering
      \includegraphics[width=\linewidth]{examples/gfx/coqa_849b1734_frame0_eastern-european-states_0.png}
      {\captionsetup{labelformat=empty}\captionof{figure}{\tiny\textit{Copyrigth: Lithuanian National Culture Centre, Archive, 2021}}}
    \end{minipage}\hfill
    \begin{minipage}{0.18\linewidth}
      \centering
      \includegraphics[width=\linewidth]{examples/gfx/coqa_849b1734_frame1_eastern-european-states_0.png}
      {\captionsetup{labelformat=empty}\captionof{figure}{\tiny\textit{Copyrigth: Vilnius Ethnic Culture Centre, Archive, 2021}}}
    \end{minipage}\hfill
    \begin{minipage}{0.18\linewidth}
      \centering
      \includegraphics[width=\linewidth]{examples/gfx/coqa_849b1734_frame2_eastern-european-states_0.png}
      {\captionsetup{labelformat=empty}\captionof{figure}{\tiny\textit{Copyrigth: Vilnius Ethnic Culture Centre, Archive, 2021}}}
    \end{minipage}\hfill
    \begin{minipage}{0.18\linewidth}
      \centering
      \includegraphics[width=\linewidth]{examples/gfx/coqa_849b1734_frame3_eastern-european-states_0.png}
      {\captionsetup{labelformat=empty}\captionof{figure}{\tiny\textit{Copyrigth: Lithuanian National Culture Centre, Archive, 2021}}}
    \end{minipage}\hfill
    \begin{minipage}{0.18\linewidth}
      \centering
      \includegraphics[width=\linewidth]{examples/gfx/coqa_849b1734_frame4_eastern-european-states_0.png}
      {\captionsetup{labelformat=empty}\captionof{figure}{\tiny\textit{Copyrigth: Vilnius Ethnic Culture Centre, Archive, 2021}}}
    \end{minipage}\hfill
  \\[4mm]
    \begin{minipage}{0.18\linewidth}
      \centering
      \includegraphics[width=\linewidth]{examples/gfx/coqa_849b1734_frame5_eastern-european-states_0.png}
      {\captionsetup{labelformat=empty}\captionof{figure}{\tiny\textit{Copyrigth: Vilnius Ethnic Culture Centre, Archive, 2021}}}
    \end{minipage}\hfill
    \begin{minipage}{0.18\linewidth}
      \centering
      \includegraphics[width=\linewidth]{examples/gfx/coqa_849b1734_frame6_eastern-european-states_0.png}
      {\captionsetup{labelformat=empty}\captionof{figure}{\tiny\textit{Copyrigth: Vilnius Ethnic Culture Centre, Archive, 2021}}}
    \end{minipage}\hfill
    \begin{minipage}{0.18\linewidth}
      \centering
      \includegraphics[width=\linewidth]{examples/gfx/coqa_849b1734_frame7_eastern-european-states_0.png}
      {\captionsetup{labelformat=empty}\captionof{figure}{\tiny\textit{Copyrigth: Lithuanian National Culture Centre, Archive, 2021}}}
    \end{minipage}\hfill
    \begin{minipage}{0.18\linewidth}
      \centering
      \includegraphics[width=\linewidth]{examples/gfx/coqa_849b1734_frame8_eastern-european-states_0.png}
      {\captionsetup{labelformat=empty}\captionof{figure}{\tiny\textit{Copyrigth: Marija Liugienė, Archive, 2003}}}
    \end{minipage}\hfill
    \begin{minipage}{0.18\linewidth}
      \centering
      \includegraphics[width=\linewidth]{examples/gfx/coqa_849b1734_frame9_eastern-european-states_0.png}
      {\captionsetup{labelformat=empty}\captionof{figure}{\tiny\textit{Copyrigth: Lithuanian National Culture Centre, Archive, 2021}}}
    \end{minipage}\hfill
  \end{center}

  {\Large{Question:}} {\large{In which of the following countries does the event shown in the images take place? Choose from the following options and output only the corresponding letter.

A. Lithuania

B. Bosnia and Herzegovina

C. Russia

D. Poland


Your answer letter:}}\\
  {\Large{Answer:}} {\large{A}}\\
   \tcbline
  {\Large{Related Cultural Event or Facet}}\\[4mm]
  {\normalsize{Title:}} {\normalsize{Sodai straw garden making in Lithuania}}\\
  {\normalsize{Countries:}} Lithuania\\
  {\normalsize{Regions:}} Eastern European States\\
  {\normalsize{Description:}}\\
  Sodai straw gardens are hanging ornaments made from the stalks of grains. This practice involves the cultivation of grain (typically rye), the treatment of straw and the creation of geometric structures of varying sizes. The structures are then decorated with details symbolizing fertility and prosperity. Sodai gardens are believed to reflect the pattern of the universe and are associated with well-being and spirituality. They are hung over the cradles of babies and over a wedding or family table to wish happiness to newborns, fertility to newlyweds or harmony to the family. Lithuanian homes are also frequently decorated with sodai gardens for Easter and Christmas. Some sodai-making families have been practising the tradition for generations. Although most of the practitioners are women, workshops exist and are open to people of all ages and genders. The practice is passed on informally within families or during events such as festivals, exhibitions, conferences and summer camps. An integral part of traditional wooden home interiors, sodai gardens are viewed as spiritual gifts. They provide a sense of shared cultural heritage and continuity to the practising communities while strengthening communal partnerships, intergenerational bonds and cultural diversity.\\[2mm]
  {\normalsize{UNESCO ICH URL:}} \href{https://ich.unesco.org/en/RL/sodai-straw-garden-making-in-lithuania-01987}{https://ich.unesco.org/en/RL/sodai-straw-garden-making-in-li...}
\end{tcolorbox}
\end{figure}
%
\subsubsection*{\RegLAC}
%
\begin{figure}[H]
\begin{tcolorbox}[colback=gray!5!white,colframe=black!75!black,fonttitle=\bfseries\scriptsize,fontupper=\ttfamily\footnotesize]
  {\large{Title:}} {\normalsize{Ancestral system of knowledge of the four indigenous peoples, Arhuaco, Kankuamo, Kogui and Wiwa of the Sierra Nevada de Santa Marta}}\\
  {\normalsize{Countries:}} Colombia\\
  {\normalsize{Regions:}} Latin-American and Caribbean States\\
  {\normalsize{Description:}}\\
  The Ancestral System of Knowledge of the Arhuaco, Kankuamo, Kogui and Wiwa peoples of the Sierra Nevada de Santa Marta is comprised of sacred mandates that keep the existence of the four peoples in harmony with the physical and spiritual universe. Through many years of dedication, the knowledgeable men (Mamos) and women (Sagas) acquire the necessary skills and sensitivity to communicate with the snow-capped peaks, connect with the knowledge of the rivers and decipher the messages of nature. Based on the Law of Origin, a philosophy that governs human relationships to nature and the universe, the Ancestral System of Knowledge entails caring for sacred sites and partaking in baptism rituals, marriage rites, traditional dances and songs, and retributions or offerings to spiritual powers. This ancestral wisdom is believed to play a fundamental role in protecting the Sierra Nevada ecosystem and avoiding the loss of the cultural identity of the four peoples of the region. The Ancestral System of Knowledge is transmitted from generation to generation through cultural practice, community activities, the use of the indigenous language and the implementation of the sacred mandates. The transmission process includes the understanding of physical and spiritual relationships with Mother Nature and sacred sites.\\[2mm]
  {\normalsize{UNESCO ICH URL:}} \href{https://ich.unesco.org/en/RL/ancestral-system-of-knowledge-of-the-four-indigenous-peoples-arhuaco-kankuamo-kogui-and-wiwa-of-the-sierra-nevada-de-santa-marta-01886}{https://ich.unesco.org/en/RL/ancestral-system-of-knowledge-o...}
  \begin{center}
    \begin{minipage}{0.18\linewidth}
      \centering
      \includegraphics[width=\linewidth]{examples/gfx/cef_latin-american-and-caribbean-states_0_4e60556d_0.jpg}
      {\captionsetup{labelformat=empty}\captionof{figure}{\tiny\textit{Copyrigth: William Diaz, 2021}}}
    \end{minipage}\hfill
    \begin{minipage}{0.18\linewidth}
      \centering
      \includegraphics[width=\linewidth]{examples/gfx/cef_latin-american-and-caribbean-states_0_4e60556d_1.jpg}
      {\captionsetup{labelformat=empty}\captionof{figure}{\tiny\textit{Copyrigth: Jorge Mario Suarez/Government of Magdalena, 2017}}}
    \end{minipage}\hfill
    \begin{minipage}{0.18\linewidth}
      \centering
      \includegraphics[width=\linewidth]{examples/gfx/cef_latin-american-and-caribbean-states_0_4e60556d_2.jpg}
      {\captionsetup{labelformat=empty}\captionof{figure}{\tiny\textit{Copyrigth: Jorge Mario Suarez/Government of Magdalena, 2017}}}
    \end{minipage}\hfill
    \begin{minipage}{0.18\linewidth}
      \centering
      \includegraphics[width=\linewidth]{examples/gfx/cef_latin-american-and-caribbean-states_0_4e60556d_3.jpg}
      {\captionsetup{labelformat=empty}\captionof{figure}{\tiny\textit{Copyrigth: William Diaz, 2021}}}
    \end{minipage}\hfill
    \begin{minipage}{0.18\linewidth}
      \centering
      \includegraphics[width=\linewidth]{examples/gfx/cef_latin-american-and-caribbean-states_0_4e60556d_4.jpg}
      {\captionsetup{labelformat=empty}\captionof{figure}{\tiny\textit{Copyrigth: Jorge Mario Suarez/Government of Magdalena, 2017}}}
    \end{minipage}\hfill
  \\[4mm]
    \begin{minipage}{0.18\linewidth}
      \centering
      \includegraphics[width=\linewidth]{examples/gfx/cef_latin-american-and-caribbean-states_0_4e60556d_5.jpg}
      {\captionsetup{labelformat=empty}\captionof{figure}{\tiny\textit{Copyrigth: Jorge Mario Suarez/Government of Magdalena, 2017}}}
    \end{minipage}\hfill
    \begin{minipage}{0.18\linewidth}
      \centering
      \includegraphics[width=\linewidth]{examples/gfx/cef_latin-american-and-caribbean-states_0_4e60556d_6.jpg}
      {\captionsetup{labelformat=empty}\captionof{figure}{\tiny\textit{Copyrigth: Jorge Mario Suarez/Government of Magdalena, 2017}}}
    \end{minipage}\hfill
    \begin{minipage}{0.18\linewidth}
      \centering
      \includegraphics[width=\linewidth]{examples/gfx/cef_latin-american-and-caribbean-states_0_4e60556d_7.jpg}
      {\captionsetup{labelformat=empty}\captionof{figure}{\tiny\textit{Copyrigth: Jorge Mario Suarez/Government of Magdalena, 2017}}}
    \end{minipage}\hfill
    \begin{minipage}{0.18\linewidth}
      \centering
      \includegraphics[width=\linewidth]{examples/gfx/cef_latin-american-and-caribbean-states_0_4e60556d_8.jpg}
      {\captionsetup{labelformat=empty}\captionof{figure}{\tiny\textit{Copyrigth: Jorge Mario Suarez/Government of Magdalena, 2017}}}
    \end{minipage}\hfill
    \begin{minipage}{0.18\linewidth}
      \centering
      \includegraphics[width=\linewidth]{examples/gfx/cef_latin-american-and-caribbean-states_0_4e60556d_9.jpg}
      {\captionsetup{labelformat=empty}\captionof{figure}{\tiny\textit{Copyrigth: William Diaz, 2021}}}
    \end{minipage}\hfill
  \end{center}
\end{tcolorbox}
\end{figure}
%
\subsubsection*{\RegSA}
%
\begin{figure}[H]
\begin{tcolorbox}[colback=gray!5!white,colframe=black!75!black,fonttitle=\bfseries\scriptsize,fontupper=\ttfamily\footnotesize,segmentation style={solid, black!30}]
  \begin{center}
    \begin{minipage}{0.5\linewidth}
      \centering
      \includegraphics[width=\linewidth]{examples/gfx/sivqa_863c7b2f_subsaharian-african-states_0.png}
      {\captionsetup{labelformat=empty}\captionof{figure}{\tiny\textit{Copyrigth: The Authority for Research and Conservation of Cultural Heritage (ARCCH), 2013}}}
    \end{minipage}\hfill
  \end{center}
  {\Large{Question:}} {\large{What festival are the people in the image celebrating?}}\\
  {\Large{Answer:}} {\large{Fichee-Chambalaalla}}\\
   \tcbline
  {\Large{Related Cultural Event or Facet}}\\[4mm]
  {\normalsize{Title:}} {\normalsize{Fichee-Chambalaalla, New Year festival of the Sidama people}}\\
  {\normalsize{Countries:}} Ethiopia\\
  {\normalsize{Regions:}} Subsaharian African States\\
  {\normalsize{Description:}}\\
  Fichee-Chambalaalla is a New Year festival celebrated among the Sidama people. According to the oral tradition, Fichee commemorates a Sidama woman who visited her parents and relatives once a year after her marriage, bringing ''buurisame'', a meal prepared from false banana, milk and butter, which was shared with neighbours. Fichee has since become a unifying symbol of the Sidama people. Each year, astrologers determine the correct date for the festival, which is then announced to the clans. Communal events take place throughout the festival, including traditional songs and dances. Every member participates irrespective of age, gender and social status. On the first day, children go from house to house to greet their neighbours, who serve them ''buurisame''. During the festival, clan leaders advise the Sidama people to work hard, respect and support the elders, and abstain from cutting down indigenous trees, begging, indolence, false testimony and theft. The festival therefore enhances equity, good governance, social cohesion, peaceful co-existence and integration among Sidama clans and the diverse ethnic groups in Ethiopia. Parents transmit the tradition to their children orally and through participation in events during the celebration. Women in particular, transfer knowledge and skills associated with hairdressing and preparation of ''buurisame'' to their daughters and other girls in their respective villages.\\[2mm]
  {\normalsize{UNESCO ICH URL:}} \href{https://ich.unesco.org/en/RL/fichee-chambalaalla-new-year-festival-of-the-sidama-people-01054}{https://ich.unesco.org/en/RL/fichee-chambalaalla-new-year-fe...}
\end{tcolorbox}
\end{figure}
%
\subsubsection*{\RegW}
%
\begin{figure}[H]
\begin{tcolorbox}[colback=gray!5!white,colframe=black!75!black,fonttitle=\bfseries\scriptsize,fontupper=\ttfamily\footnotesize,segmentation style={solid, black!30}]
  \begin{center}
    \begin{minipage}{0.18\linewidth}
      \centering
      \includegraphics[width=\linewidth]{examples/gfx/coqa_3ec41e9f_frame0_western-european-and-north-american-states_0.png}
      {\captionsetup{labelformat=empty}\captionof{figure}{\tiny\textit{Copyrigth: Servicio de Patrimonio Histórico de la Región de Murcia, 2005}}}
    \end{minipage}\hfill
    \begin{minipage}{0.18\linewidth}
      \centering
      \includegraphics[width=\linewidth]{examples/gfx/coqa_3ec41e9f_frame1_western-european-and-north-american-states_0.png}
      {\captionsetup{labelformat=empty}\captionof{figure}{\tiny\textit{Copyrigth: Generalitat Valenciana, 2005}}}
    \end{minipage}\hfill
    \begin{minipage}{0.18\linewidth}
      \centering
      \includegraphics[width=\linewidth]{examples/gfx/coqa_3ec41e9f_frame2_western-european-and-north-american-states_0.png}
      {\captionsetup{labelformat=empty}\captionof{figure}{\tiny\textit{Copyrigth: Servicio de Patrimonio Histórico de la Región de Murcia, 2005}}}
    \end{minipage}\hfill
    \begin{minipage}{0.18\linewidth}
      \centering
      \includegraphics[width=\linewidth]{examples/gfx/coqa_3ec41e9f_frame3_western-european-and-north-american-states_0.png}
      {\captionsetup{labelformat=empty}\captionof{figure}{\tiny\textit{Copyrigth: Generalitat Valenciana, 2005}}}
    \end{minipage}\hfill
    \begin{minipage}{0.18\linewidth}
      \centering
      \includegraphics[width=\linewidth]{examples/gfx/coqa_3ec41e9f_frame4_western-european-and-north-american-states_0.png}
      {\captionsetup{labelformat=empty}\captionof{figure}{\tiny\textit{Copyrigth: Servicio de Patrimonio Histórico de la Región de Murcia, 2005}}}
    \end{minipage}\hfill
  \\[4mm]
    \begin{minipage}{0.18\linewidth}
      \centering
      \includegraphics[width=\linewidth]{examples/gfx/coqa_3ec41e9f_frame5_western-european-and-north-american-states_0.png}
      {\captionsetup{labelformat=empty}\captionof{figure}{\tiny\textit{Copyrigth: Servicio de Patrimonio Histórico de la Región de Murcia, 2005}}}
    \end{minipage}\hfill
    \begin{minipage}{0.18\linewidth}
      \centering
      \includegraphics[width=\linewidth]{examples/gfx/coqa_3ec41e9f_frame6_western-european-and-north-american-states_0.png}
      {\captionsetup{labelformat=empty}\captionof{figure}{\tiny\textit{Copyrigth: Servicio de Patrimonio Histórico de la Región de Murcia, 2005}}}
    \end{minipage}\hfill
    \begin{minipage}{0.18\linewidth}
      \centering
      \includegraphics[width=\linewidth]{examples/gfx/coqa_3ec41e9f_frame7_western-european-and-north-american-states_0.png}
      {\captionsetup{labelformat=empty}\captionof{figure}{\tiny\textit{Copyrigth: Servicio de Patrimonio Histórico de la Región de Murcia, 2005}}}
    \end{minipage}\hfill
    \begin{minipage}{0.18\linewidth}
      \centering
      \includegraphics[width=\linewidth]{examples/gfx/coqa_3ec41e9f_frame8_western-european-and-north-american-states_0.png}
      {\captionsetup{labelformat=empty}\captionof{figure}{\tiny\textit{Copyrigth: Servicio de Patrimonio Histórico de la Región de Murcia, 2005}}}
    \end{minipage}\hfill
    \begin{minipage}{0.18\linewidth}
      \centering
      \includegraphics[width=\linewidth]{examples/gfx/coqa_3ec41e9f_frame9_western-european-and-north-american-states_0.png}
      {\captionsetup{labelformat=empty}\captionof{figure}{\tiny\textit{Copyrigth: Servicio de Patrimonio Histórico de la Región de Murcia, 2005}}}
    \end{minipage}\hfill
  \end{center}

  {\Large{Question:}} {\large{In which of the following countries does the event shown in the images take place? Choose from the following options and output only the corresponding letter.

A. Austria

B. Spain

C. Cyprus

D. United Kingdom of Great Britain and Northern Ireland


Your answer letter:}}\\
  {\Large{Answer:}} {\large{B}}\\
   \tcbline
  {\Large{Related Cultural Event or Facet}}\\[4mm]
  {\normalsize{Title:}} {\normalsize{Irrigators’ tribunals of the Spanish Mediterranean coast: the Council of Wise Men of the plain of Murcia and the Water Tribunal of the plain of Valencia}}\\
  {\normalsize{Countries:}} Spain\\
  {\normalsize{Regions:}} Western European and North American States\\
  {\normalsize{Description:}}\\
  The irrigators’ tribunals of the Spanish Mediterranean coast are traditional law courts for water management that date back to the al-Andalus period (ninth to thirteenth centuries). The two main tribunals – the Council of Wise Men of the Plain of Murcia and the Water Tribunal of the Plain of Valencia – are recognized under Spanish law. Inspiring authority and respect, these two courts, whose members are elected democratically, settle disputes orally in a swift, transparent and impartial manner. The Council of Wise Men has seven geographically representative members, and has jurisdiction over a landowners’ assembly of 23,313 members. The Water Tribunal comprises eight elected administrators representing a total of 11,691 members from nine communities. In addition to their legal role the irrigators’ tribunals play a key part in the communities of which they are a visible symbol, as apparent from the rites performed when judgments are handed down and the fact that the tribunals often feature in local iconography. They provide cohesion among traditional communities and synergy between occupations (wardens, inspectors, pruners, etc.), contribute to the oral transmission of knowledge derived from centuries-old cultural exchanges, and have their own specialist vocabulary peppered with Arabic borrowings. In short, the courts are long-standing repositories of local and regional identity and are of special significance to local inhabitants.\\[2mm]
  {\normalsize{UNESCO ICH URL:}} \href{https://ich.unesco.org/en/RL/irrigators-tribunals-of-the-spanish-mediterranean-coast-the-council-of-wise-men-of-the-plain-of-murcia-and-the-water-tribunal-of-the-plain-of-valencia-00171}{https://ich.unesco.org/en/RL/irrigators-tribunals-of-the-spa...}
\end{tcolorbox}
\end{figure}
%

\subsection{Cultural Aspects}
\label{appedix:sec:sivqa:aspects}
%
During the synthetic data generation phase of the \sivqa, we also obtained a ``target aspect'' per question (see \S\ref{appendix:sec:sivqa:synth} and \S\ref{appendix:sec:sivqa:synth:sys_prompt}).
%
We report these aspects in the following.
%
\begin{table*}[ht!]
    \centering
    \small
    \begin{minipage}[t]{0.29\textwidth}
        \centering
        \begin{tabular}{lr}
            \toprule
            Aspect & Questions \\
            \midrule
            traditions & 390 \\
            rituals & 241 \\
            art & 233 \\
            music & 210 \\
            craftsmanship & 177 \\
            instruments & 155 \\
            festivals & 151 \\
            dance & 150 \\
            tools & 108 \\
            food & 96 \\
            clothing & 93 \\
            architecture & 52 \\
            sports & 38 \\
            location & 28 \\
            symbols & 19 \\
            drinks & 14 \\
            customs & 13 \\
            cultural significance & 6 \\
            theatre & 4 \\
            \bottomrule
        \end{tabular}
    \end{minipage}%
    \begin{minipage}[t]{0.29\textwidth}
        \centering
        \begin{tabular}{lr}
            \toprule
            Aspect & Questions \\
            \midrule
            education & 3 \\
            culture & 3 \\
            games & 3 \\
            performing arts & 3 \\
            language & 3 \\
            performance & 3 \\
            characters & 2 \\
            practices & 2 \\
            skills & 2 \\
            origin & 2 \\
            cultural identity & 2 \\
            technology & 1 \\
            people & 1 \\
            community & 1 \\
            identity & 1 \\
            environment & 1 \\
            traditional medicine & 1 \\
            nature & 1 \\
            communication & 1 \\
            \bottomrule
        \end{tabular}
    \end{minipage}%
    \begin{minipage}[t]{0.29\textwidth}
        \centering
        \begin{tabular}{lr}
            \toprule
            Aspect & Questions \\
            \midrule
            jewelry & 1 \\
            objects & 1 \\
            animal & 1 \\
            plants & 1 \\
            process & 1 \\
            agriculture & 1 \\
            celebrations & 1 \\
            details & 1 \\
            historical & 1 \\
            function or usage & 1 \\
            symbolism & 1 \\
            healthcare & 1 \\
            knowledge & 1 \\
            social status & 1 \\
            religion & 1 \\
            cultural space & 1 \\
            social space & 1 \\
            cultural practice & 1 \\
            unknown & 1 \\
            \bottomrule
        \end{tabular}
    \end{minipage}
    \caption{Cultural aspects targeted by the questions within the \sivqa task.}
    \label{tab:sivqa:aspects}
\end{table*}

%
\subsection{External Hint Variations}
%
\label{appendix:sec:sivqa:hints}
%
For the \sivqa (and \vvqa) task, we ablate the effect of external cues or hints on the task performance of models.
%
In the following, we provide the Python pseudo-code snippet to generate the prompt for a given sample.
%
\begin{figure*}[ht!]
    \centering
    %
    \begin{promptbox}{Python Pseudo-Code for the external cue settings of the \sivqa and \vvqa tasks.}
    \begin{minted}[breaklines]{python}
def apply_gimmick_prompt_template(
    sample: dict[str, Any],
    regions_hint: bool,
    countries_hint: bool,
) -> str:
    
    prompt_template = "{QUESTION}\n{HINTS}\n"
    hints = ""

    if regions_hint:
        hints += (
            "Hint: The question is related to a cultural event or facet from the following region(s): "
            f"{', '.join(sample['regions'])}\n"
        )

    if countries_hint:
        hints += (
            "Hint: The question is related to a cultural event or facet from the following country or countries: "
            f"{', '.join(sample['countries'])}\n"
        )

    return prompt_template.format(
        QUESTION=sample["prompt"],
        HINTS=hints,
    )
    \end{minted}
    \end{promptbox}
    \label{fig:sivqa:hints}
    \caption{Python Pseudo-Code to generate the prompt for a given \sivqa (or \vvqa) sample for the external cues settings.}
\end{figure*}
%



\subsection{Synthetic Data Generation}
\label{appendix:sec:sivqa:synth}
%
\onecolumn
\newpage
\subsubsection{System Prompt}
\label{appendix:sec:sivqa:synth:sys_prompt}
%
\begin{tcolorbox}[
    enhanced, 
    breakable,
    skin first=enhanced,
    skin middle=enhanced,
    skin last=enhanced,
]
\begin{minted}[fontsize=\footnotesize,breaklines]{markdown}
# Your Role

You are a professional annotator specialized in creating VQA samples based on a provided intangible cultural heritage(ICH) item. You will be given the following information related to the item:

- Image: An image representing one aspect of the ICH item.
- Countries of Origin: The country or countries where this ICH is recognized.
- Regions of Origin: The country or countries where this ICH is recognized.
- Title: The official title of the ICH item.
- Description: A detailed description of the ICH item, including relevant details.

# Your Task

Your task is it to generate high-quality question-answer pairs in a VQA style to assess the cultural knowledge of the intangible cultural heritage (ICH) item of state-of-the-art multimodal AI models. Be sure to follow the annotation guidelines provided below to ensure the quality and relevance of the question-answer pairs.

# Annotation Guidelines

## Question Requirements

Make sure the question meets all of the following requirements:

1. Clear and Concise
    The question is clear and concise and no longer than a single sentence.
2. Directly related to the ICH item
    The question is directly related to the ICH item.
3. Directly related to the visible content
    The question is directly related to the visible content in the image and requires visual analysis to answer.
4. Does not (partially) contain the answer
    The question does not contain any hints or clues to or parts of the answer that would make the answer obvious.
5. Does not contain subjective words
    The question does not contain subjective words like 'likely', 'possibly', 'probably', 'eventually', 'might', 'could', 'should', etc., which could introduce ambiguity.
6. Requires both image and cultural knowledge to answer
    The question requires both image and cultural knowledge to answer and is not answerable by looking only at the image or only knowing about the ICH item or reading the textual description.
7. (optional) Includes specific cultural terms
    The answer includes specific cultural terms, names, or phrases related to the ICH item. E.g., particular names mentioned in the description or parts of the title.

## Answer Requirements

Make sure the answer meets all of the following requirements:

1. Single Word or Multiword Expression
    The answer is a single word or multiword expression.
2. Clear, Objective, and Correct
    The answer is clear, objective, and unambiguously correct.
3. Directly Related to Visual Content
    The answer is directly related to the visual content of the image.
4. No General or Abstract Words
    The answer does not contain general, abstract, or non-depictable words like "Traditional", "Cooperation", "Gathering", "Solidarity", "Community", "Indoor", "Outdoor", "Urban", "Rural", etc.
5. Verifiable by Text and Image
    The answer is unambiguously verifiable by reading the textual information and inspecting the image.
6. (optional) Includes specific cultural terms
    The answer includes specific cultural terms, names, or phrases related to the ICH item. E.g., particular names mentioned in the description or parts of the title.

## Question Characteristics

### Target Aspects

Make sure the question targets different aspects of the ICH item, such as:

- Food
- Drinks
- Clothing
- Art
- Tools
- Sports
- Instruments
- Dance
- Music
- Rituals
- Traditions
- Festivals
- Customs
- Symbols
- Architecture
- Other

### Question Categories

Make sure the question falls into different categories, such as:

- Identification
    Questions that ask for the identification of objects, people, or elements in the image. E.g.: What is the name of the instrument shown in the image?
- Origin
    Questions that inquire about the origin or source of the CEF. E.g.: Which culture or country does this artifact belong to?
- Cultural Significance
    Questions that explore the cultural or religious significance of the depicted element. E.g.: What cultural or religious significance does this item hold in its native context?
- Function or Usage
    Questions that ask about the traditional or historical function or usage of the depicted element. E.g.: What was this object traditionally used for?
- Material and Craftsmanship
    Questions that focus on the materials used and the craftsmanship involved in creating the depicted element. E.g.: What material is used to construct this artifact?
- Location
    Questions that ask about the geographical location where the cultural event or facet takes place. E.g.: In which place does this dance take place?
- Symbolism
    Questions that delve into the symbolic meanings associated with the depicted element. E.g.: What does the color red symbolize in this cultural context?
- Historical
    Questions that relate to historical events or contexts depicted in the image. E.g.: What historical event is depicted in this image?
- Details
    Questions that ask for specific details about the formation, arrangement, or other aspects of the depicted element. E.g.: What formation are the dancers in?
- Other
    Questions that do not fall into the above categories but are relevant to the ICH item.

    
# Task Strategy

Before generating a question-answer pair, first think step-by-step and analyse the image:

1. What is visible in the image? Generate a highly detailed description of the key elements, objects, or people in the image. Take into account the textual description provided to identify details.
2. How does the visible content relate to the intangible cultural heritage item? Identify the connection between the contents of the image and the intangible cultural heritage item.

Then, think step-by-step about potential questions:

1. What can be asked about the image that is directly related to the visible content and the intangible cultural heritage item?
2. Can a concise and clear answer to the questions be inferred from the image and the provided information?

Finally, think step-by-step before generating the final question-answer pairs:

1. Does the question-answer pair strictly adhere to the guidelines provided above? Percisly check every part of the guidelines and drop the question-answer pair if it does not meet the criteria.
2. What aspect of the intangible cultural heritage item is targeted with the question?
3. What category does the question fall into?

# Output Format

For each question-answer pair, provide the following information in the following format:
```xml
<vqa-task>
    <image-analysis>
        <description>
            <!-- PUT YOUR DETAILED DESCRIPTION OF THE IMAGE HERE -->
        </description>
        <cultural-relatetness>
            <!-- PUT YOUR ANALYSIS OF HOW THE CONTENTS OF THE IMAGE RELATE TO THE INTANGIBLE CULTURAL HERITAGE ITEM HERE -->
        </cultural-relatetness>
    </image-analysis>
    <potential-questions>
        <qa-candidate>
            <question>
                <!-- PUT YOUR QUESTION HERE -->
            </question>
            <answer>
                <!-- PUT YOUR ANSWER HERE -->
            </answer>
            <guideline-adherence>
                <question-requirments>
                    <clear-and-concise>
                        <!-- YES OR NO -->
                    </clear-and-concise>
                    <directly-related-to-ich>
                        <!-- YES OR NO -->
                    </directly-related-to-ich>
                    <directly-related-to-visual-content>
                        <!-- YES OR NO -->
                    </directly-related-to-visual-content>
                    <does-not-contain-answer>
                        <!-- YES OR NO -->
                    </does-not-contain-answer>
                    <does-not-contain-subjective-words>
                        <!-- YES OR NO -->
                    </does-not-contain-subjective-words>
                    <requires-both-image-and-cultural-knowledge>
                        <!-- YES OR NO -->
                    </requires-both-image-and-cultural-knowledge>
                    <includes-specific-cultural-terms>
                        <!-- YES OR NO -->
                    </includes-specific-cultural-terms>
                </question-requirments>
                <answer-requirments>
                    <single-word-or-multiword-expression>
                        <!-- YES OR NO -->
                    </single-word-or-multiword-expression>
                    <clear-objective-and-correct>
                        <!-- YES OR NO -->
                    </clear-objective-and-correct>
                    <directly-related-to-visual-content>
                        <!-- YES OR NO -->
                    </directly-related-to-visual-content>
                    <no-general-or-abstract-words>
                        <!-- YES OR NO -->
                    </no-general-or-abstract-words>
                    <verifiable-by-text-and-image>
                        <!-- YES OR NO -->
                    </verifiable-by-text-and-image>
                    <includes-specific-cultural-terms>
                        <!-- YES OR NO -->
                    </includes-specific-cultural-terms>
                </answer-requirments>
            </guideline-adherence>
        </qa-candidate>
        ...
    </potential-questions>
    <final-qa-pairs>
        <!-- PUT ALL QA PAIRS THAT MEET ALL MANDATORY REQUIREMENTS HERE -->
        <qa-pair>
            <meets-requirements>
                <!-- DOES YOUR QUESTION-ANSWER PAIR MEET ALL MANDATORY REQUIREMENTS? YES OR NO -->
            </meets-requirements>
            <final-result-json>
                <!-- PUT YOUR FINAL RESULT AS JSON HERE -->
                {
                    "question": <insert question here>,
                    "answer": <insert answer here>,
                    "target_aspect": <insert target aspect here>
                    "question_category": <insert question category here>
                }
            </final-result-json>
        </qa-pair>
        ...
    </final-qa-pairs>
</vqa-task>
```
\end{minted}
\end{tcolorbox}
%
\subsubsection{User Prompt Template}
\label{appendix:sec:sivqa:synth:usr_prompt}
%

\begin{tcolorbox}[
    enhanced, 
    breakable,
    skin first=enhanced,
    skin middle=enhanced,
    skin last=enhanced,
]
\begin{minted}[fontsize=\footnotesize,breaklines]{markdown}
# Intangible Cultural Heritage Item

### Image

{IMAGE_PLACEHOLDER}

### Countries of Origin:

{LIST_OF_COUNTRIES}

### Regions of Origin

{LIST_OF_REGIONS}

### Title

{TITLE}

### Description

{DESCRIPTION}

\end{minted}
\end{tcolorbox}

%
\twocolumn


%

\subsection{Annotation Project Details}
\label{appendix:sec:sivqa:anno}
%
We first conducted several internal pilot studies to iteratively create a straightforward annotation task, guidelines, and an intuitive interface for the final annotation project.
%
To find annotators, we advertised the task in our faculty research network, emphasizing our goal of creating a culturally diverse benchmark for assessing the cultural awareness of current AI models.
%
Therefore, we targeted primarily individuals from non-Western cultural backgrounds.
%
We found 18 volunteers who have spent most of their lives in 10 different countries from all six regions and thus cover diverse cultural backgrounds (see Table~\ref{tab:sivqa:anno:demographics}).
%
To train the annotators, we provided detailed annotation guidelines, followed by an oral introduction to the task.
%
For more details, refer to the (anonymized) original annotation guidelines we \href{https://drive.proton.me/urls/T6RHQCEW5G#5y0Itm2BdWYZ}{shared here}.
%

For the second annotation round, we hired 5 of the previous volunteering annotators (0, 1, 8, 15, 17) who assessed the kept samples from the first round to obtain two annotations (from distinct annotators) per sample.
%
We paid the second-round annotators a salary of roughly 12.5€ per hour.
%
\begin{table}[ht!]
    \centering
    \renewcommand{\arraystretch}{0.95}
    \resizebox{\linewidth}{!}{%
    \begin{tabular}{lrllllr}
        \toprule
        \textsc{ID} & \textsc{Age} & \textsc{Pronouns} & \textsc{Education} & \textsc{Country} & \textsc{Region} & \textsc{Round(s)} \\
        \midrule
        0 & 23 & she/her & Bachelor & Iran & \RegAP & 1, 2\\
        1 & 23 & she/her & Bachelor & Iran & \RegAP & 1, 2\\
        2 & 28 & she/her & PhD & Russia & \RegE & 1\\
        3 & 35 & he/him & Master & Germany & \RegW & 1 \\
        5 & 29 & he/him & Bachelor & Guatemala & \RegLAC & 1\\
        6 & 29 & he/him & Master & Germany & \RegW & 1\\
        7 & 42 & he/him & PhD & Ethiopia & \RegSA & 1\\
        8 & 23 & he/him & Bachelor & Egypt & \RegA & 1, 2\\
        9 & 33 & she/her & Master & Iran & \RegAP & 1\\
        10 & 29 & she/her & Bachelor & Afghanistan & \RegAP & 1\\
        11 & 23 & she/her & Bachelor & India & \RegAP & 1\\
        12 & 33 & he/him & Bachelor & Germany & \RegW & 1\\
        13 & 22 & she/her & Bachelor & Pakistan & \RegAP & 1\\
        14 & 27 & he/him & Master & China & \RegAP & 1\\
        15 & 29 & she/her & High School & Germany & \RegW & 1, 2\\
        16 & 22 & she/her & Bachelor & China & \RegAP & 1\\
        17 & 26 & he/him & High School & Germany & \RegW & 1, 2, 3\\
        \bottomrule
    \end{tabular}
    }%
    \caption{Demographics of the annotators who participated in our VQA annotation project. For the country, we asked the question, ``\textit{Where did you spend most of your life?}''. The Round(s) column indicates which annotation rounds the annotator participated in.}
    \label{tab:sivqa:anno:demographics}
\end{table}
%

\subsubsection{\sivqa Annotation Interface}
\label{appendix:sec:sivqa:anno:ui}
%
For the annotation project, we used a self-hosted Label Studio\footnote{\url{https://labelstud.io/}} instance with a custom labeling interface (see Figure~\ref{fig:sivqa:anno:ui}) for all annotation projects.
%
\begin{figure*}
    \centering
    \begin{subfigure}[b]{1.\textwidth}
         \centering
         \includegraphics[width=\textwidth]{gfx/anno-task-screenshot-sample-A.png}
     \end{subfigure}
     
     \begin{subfigure}[b]{1.\textwidth}
         \centering
         \includegraphics[width=\textwidth]{gfx/anno-task-screenshot-sample-B.png}
     \end{subfigure}

     \begin{subfigure}[b]{1.\textwidth}
         \centering
         \includegraphics[width=\textwidth]{gfx/anno-task-screenshot-sample-C.png}
     \end{subfigure}
    \caption{Three screenshots showing examples of the Label Studio interface used in our \sivqa annotation tasks.}
    \label{fig:sivqa:anno:ui}
\end{figure*}

{
\onecolumn
\subsubsection{First Annotation Round Statistics}
\label{appendix:sec:sivqa:anno:first_round}
%
\begin{table}[ht!]
    \centering
    \begin{minipage}[t]{0.50\textwidth}
        \scriptsize
        \centering
        \begin{tabular}{lr}
            \toprule
            Country & Count \\
            \midrule
            United Arab Emirates & 101 \\
            China & 98 \\
            Oman & 91 \\
            Saudi Arabia & 87 \\
            France & 86 \\
            Croatia & 84 \\
            Algeria & 82 \\
            Morocco & 81 \\
            Türkiye & 78 \\
            Peru & 75 \\
            Spain & 74 \\
            Azerbaijan & 69 \\
            Colombia & 68 \\
            Islamic Republic of Iran & 66 \\
            Mali & 65 \\
            Mexico & 64 \\
            Republic of Korea & 62 \\
            Egypt & 62 \\
            Tunisia & 56 \\
            Iraq & 54 \\
            Japan & 52 \\
            Brazil & 50 \\
            Italy & 50 \\
            Belgium & 50 \\
            Plurinational State of Bolivia & 49 \\
            Mauritania & 49 \\
            Bolivarian Republic of Venezuela & 47 \\
            Nigeria & 46 \\
            India & 45 \\
            Malawi & 43 \\
            Palestine & 40 \\
            Greece & 38 \\
            Uzbekistan & 37 \\
            Kuwait & 37 \\
            Kyrgyzstan & 36 \\
            Cuba & 35 \\
            Mauritius & 34 \\
            Mongolia & 34 \\
            Czechia & 34 \\
            Jordan & 32 \\
            Zambia & 31 \\
            Côte d'Ivoire & 31 \\
            Syrian Arab Republic & 31 \\
            Kazakhstan & 30 \\
            Portugal & 29 \\
            Switzerland & 29 \\
            Uganda & 29 \\
            Ethiopia & 29 \\
            Botswana & 28 \\
            Viet Nam & 28 \\
            Argentina & 28 \\
            Armenia & 28 \\
            Yemen & 28 \\
            Turkmenistan & 26 \\
            Sudan & 26 \\
            Bahrain & 26 \\
            Indonesia & 26 \\
            Ecuador & 25 \\
            Mozambique & 25 \\
            Tajikistan & 25 \\
            Austria & 24 \\
            Hungary & 24 \\
            Slovakia & 23 \\
            Lebanon & 23 \\
            Cyprus & 22 \\
            Slovenia & 22 \\
            Paraguay & 21 \\
            Germany & 21 \\
            Romania & 21 \\
            Guatemala & 20 \\
            Kenya & 20 \\
            Poland & 20 \\
            \bottomrule
        \end{tabular}
    \end{minipage}
    \hspace{-2.5cm}
    \begin{minipage}[t]{0.50\textwidth}
        \scriptsize
        \centering
        \begin{tabular}{lr}
        \toprule
        Country & Count \\
        \midrule
        Nicaragua & 18 \\
        Chile & 17 \\
        Serbia & 17 \\
        Cambodia & 17 \\
        Bangladesh & 17 \\
        Bulgaria & 17 \\
        Qatar & 17 \\
        Ireland & 17 \\
        Panama & 16 \\
        Ukraine & 16 \\
        Malaysia & 16 \\
        Namibia & 16 \\
        Philippines & 15 \\
        Bosnia and Herzegovina & 15 \\
        Niger & 15 \\
        Estonia & 14 \\
        Netherlands & 14 \\
        Zimbabwe & 14 \\
        Senegal & 14 \\
        Madagascar & 14 \\
        Belarus & 13 \\
        Luxembourg & 13 \\
        Togo & 12 \\
        Burundi & 12 \\
        Dominican Republic & 12 \\
        Congo & 11 \\
        Democratic Republic of the Congo & 11 \\
        Benin & 11 \\
        Finland & 11 \\
        Angola & 10 \\
        Afghanistan & 10 \\
        Seychelles & 10 \\
        Democratic People’s Republic of Korea & 10 \\
        Norway & 9 \\
        Lao Peoples Democratic Republic & 9 \\
        Burkina Faso & 9 \\
        Sweden & 9 \\
        Bahamas & 9 \\
        Georgia & 9 \\
        Albania & 9 \\
        Republic of Moldova & 9 \\
        Cabo Verde & 8 \\
        North Macedonia & 8 \\
        Jamaica & 8 \\
        Honduras & 7 \\
        Latvia & 7 \\
        Denmark & 7 \\
        Pakistan & 7 \\
        Belize & 7 \\
        Uruguay & 7 \\
        Timor-Leste & 6 \\
        Montenegro & 6 \\
        Sri Lanka & 6 \\
        Thailand & 6 \\
        Guinea & 6 \\
        Malta & 5 \\
        Andorra & 5 \\
        Russian Federation & 5 \\
        Lithuania & 5 \\
        Tonga & 4 \\
        Costa Rica & 4 \\
        Cameroon & 4 \\
        Vanuatu & 3 \\
        Singapore & 3 \\
        Gambia & 3 \\
        Iceland & 3 \\
        Federated States of Micronesia & 2 \\
        Grenada & 2 \\
        Samoa & 2 \\
        Bhutan & 1 \\
        Djibouti & 1 \\
        Central African Republic & 1 \\
        \bottomrule
        \end{tabular}
    \end{minipage}
    \caption{The number of countries related to the QA pairs collected in the first annotation round for \sivqa.}
    \label{tab:sivqa:anno:first_round}
\end{table}
}
