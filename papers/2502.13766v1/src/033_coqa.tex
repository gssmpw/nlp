\subsection{Cultural Origin QA}
\label{sec:coqa}
%
With Cultural Origin QA (\coqa), we test a model's ability to capture coarse-grained cultural knowledge.
%
Given a CEF's images, title, or both, the models must select its cultural origin (multiple-choice).
%
We refer to the task as \coqar when the origin is a region and as \coqac when it is a country.
%

\rparagraph{Dataset Construction}
\label{sec:coqa:collection}
%
The \coqa dataset contains all 728 CEFs from UNESCO ICH.
%
To ensure that each instance corresponds to a unique origin, we replicate each CEF $N$ times—where $N$ represents the number of associated regions (for \coqar) or countries (for \coqac).
%
For \coqar, three negatives are randomly sampled from the remaining pool.
%
%CEFs associated with more than three regions are excluded to guarantee exactly four multiple-choice options with three negatives.
%
Negatives for \coqac drawn from those within the same region as the target country.
%

\rparagraph{Input Modalities and Prompts}
\label{sec:coqa:config}
%
The \coqa tasks support multiple input configurations alongside the task prompt.
%
In the text-only setting, only the title of the CEF is provided, whereas in the ``image-only'' setting, \emph{all} images associated with the CEF are included.
%
Both the title and the images are used in the text-image setting.
%
Examples and complete prompts for all variations are shown in \S\ref{appendix:sec:coqa:examples}.
%
