% This must be in the first 5 lines to tell arXiv to use pdfLaTeX, which is strongly recommended.
\pdfoutput=1
% In particular, the hyperref package requires pdfLaTeX in order to break URLs across lines.

\documentclass[11pt]{article}

% Change "review" to "final" to generate the final (sometimes called camera-ready) version.
% Change to "preprint" to generate a non-anonymous version with page numbers.
\usepackage[preprint]{acl}

% Standard package includes
\usepackage{times}
\usepackage{latexsym}

% For proper rendering and hyphenation of words containing Latin characters (including in bib files)
\usepackage[T1]{fontenc}
% For Vietnamese characters
% \usepackage[T5]{fontenc}
% See https://www.latex-project.org/help/documentation/encguide.pdf for other character sets

% This assumes your files are encoded as UTF8
\usepackage[utf8]{inputenc}

% This is not strictly necessary, and may be commented out,
% but it will improve the layout of the manuscript,
% and will typically save some space.
\usepackage{microtype}

% This is also not strictly necessary, and may be commented out.
% However, it will improve the aesthetics of text in
% the typewriter font.
\usepackage{inconsolata}


%Including images in your LaTeX document requires adding
%additional package(s)

% for lorem ipsum filler text
\usepackage{lipsum}
% for tables and matrices in mathmode
\usepackage{array}
% for nice tables
\usepackage{booktabs}
% for custom cells with e.g. line breaks
\usepackage{makecell}
% for all kind of symbols
\usepackage{amsmath}
\usepackage{amssymb}
% for images etc
\usepackage{graphicx}
% for multirows in tables
\usepackage{multirow}
% for colors and colored table rows
\usepackage{color, colortbl}
% for bold tt text
\usepackage{bold-extra} 
% for colored boxes
\usepackage[most]{tcolorbox}
% for image handling
\usepackage{epsfig}
% for quotes
\usepackage{csquotes}
% for aligning numbers in columns
\usepackage{siunitx} 
\sisetup{round-mode=places, round-precision=2}

% for subfigures
\usepackage{subcaption}
% for captions in non-floats
\usepackage{capt-of}
% for urls 
\usepackage{hyperref}
% for float positioning H
\usepackage{placeins}

\usepackage{fontawesome5}
\usepackage{pifont, xcolor}
% for code styling
% https://tex.stackexchange.com/questions/531738/minted-environment-not-working-in-overleaf
% \usepackage[outputdir=./]{minted}
% https://tex.stackexchange.com/questions/280590/work-around-for-minted-code-highlighting-in-arxiv
% Step one: compile with \usepackage[finalizecache,cachedir=.]{minted}.
% Step two: go to logs and output files > other logs and files and download everything with pyg.
% Step three: change finalizecache to frozencache and upload the tex+all those pyg files to arXiv. Success!
\usepackage[frozencache,cachedir=.]{minted}
\usepackage{listings}

% for spaces after custom commands 
\usepackage{xspace} 

% \# looks better with this
\renewcommand\#{\protect\scalebox{0.8}{\protect\raisebox{0.4ex}{\char"0023}}}

% commands for the region abbrvs
\newcommand{\RegW}{\textcolor[HTML]{636EFA}{\rule{7pt}{7pt}}\hspace{0.15em}\large\texttt{\textbf{W}}\xspace}
\newcommand{\RegE}{\textcolor[HTML]{EF553B}{\rule{7pt}{7pt}}\hspace{0.15em}\large\texttt{\textbf{E}}\xspace}
\newcommand{\RegAP}{\textcolor[HTML]{00CC96}{\rule{7pt}{7pt}}\hspace{0.15em}\large\texttt{\textbf{AP}}\xspace}
\newcommand{\RegA}{\textcolor[HTML]{AB63FA}{\rule{7pt}{7pt}}\hspace{0.15em}\large\texttt{\textbf{A}}\xspace}
\newcommand{\RegLAC}{\textcolor[HTML]{FFA15A}{\rule{7pt}{7pt}}\hspace{0.15em}\large\texttt{\textbf{LAC}}\xspace}
\newcommand{\RegSA}{\textcolor[HTML]{19D3F3}{\rule{7pt}{7pt}}\hspace{0.15em}\large\texttt{\textbf{SA}}\xspace}

% commands for the datasets and tasks
\newcommand{\dsname}{\texttt{GIMMICK}\xspace}
\newcommand{\sivqa}{\texttt{CIVQA}\xspace}
\newcommand{\vvqa}{\texttt{CVVQA}\xspace}
\newcommand{\coqa}{\texttt{COQA}\xspace}
\newcommand{\coqar}{\texttt{COQA}\textsubscript{\texttt{\textbf{R}}}\xspace}
\newcommand{\coqac}{\texttt{COQA}\textsubscript{\texttt{\textbf{C}}}\xspace}
\newcommand{\ckqa}{\texttt{CKQA}\xspace}
\newcommand{\ckqad}{\texttt{CKQA}\textsubscript{\texttt{\textbf{D}}}\xspace}
\newcommand{\ckqan}{\texttt{CKQA}\textsubscript{\texttt{\textbf{N}}}\xspace}

% command for models
\newcommand{\m}[1]{{\small\textsc{#1}}\xspace}

% commands for compressed paragraphs
\newcommand{\rparagraph}[1]{\vspace{1.4mm}\noindent\textbf{#1.}}
\newcommand{\rrparagraph}[1]{\vspace{0.4mm}\noindent\textit{#1.}}
\newcommand{\sparagraph}[1]{\vspace{0.0mm}\noindent\textbf{#1.}}
\newcommand{\rparagraphnodot}[1]{\vspace{1.6mm}\noindent\textbf{#1}}
\newcommand{\sparagraphnodot}[1]{\vspace{0.0mm}\noindent\textbf{#1}}

% prompt box
\newtcolorbox{promptbox}[1]{colback=gray!5!white,colframe=black!75!black,fonttitle=\bfseries\scriptsize,fontupper=\ttfamily\footnotesize,title=#1}


%
%
% MAIN PART
%
%
%


\title{\texttt{GIMMICK}\\Globally Inclusive Multimodal Multitask  Cultural Knowledge Benchmarking}

\author{
  \textbf{Florian Schneider\textsuperscript{1}},
  \textbf{Carolin Holtermann\textsuperscript{2}},
  \textbf{Chris Biemann\textsuperscript{1}},
  \textbf{Anne Lauscher\textsuperscript{2}}
\\
  \textsuperscript{1}Language Technology Group, University of Hamburg
\\
  \textsuperscript{2}Data Science Group, University of Hamburg
\\
  \small
  \texttt{
    \href{mailto:florian.schneider-1@uni-hamburg.de}{florian.schneider-1@uni-hamburg.de}
  }
}


\begin{document}
\maketitle
%
End-to-end imitation learning offers a promising approach for training robot policies. However, generalizing to new settings—such as unseen scenes, tasks, and object instances—remains a significant challenge. Although large-scale robot demonstration datasets have shown potential for inducing generalization, they are resource-intensive to scale. In contrast, human video data is abundant and diverse, presenting an attractive alternative. Yet, these human-video datasets lack action labels, complicating their use in imitation learning. Existing methods attempt to extract grounded action representations (e.g., hand poses), but resulting policies struggle to bridge the embodiment gap between human and robot actions.
% our approach
We propose an alternative approach: leveraging language-based reasoning from human videos - essential for guiding robot actions - to train generalizable robot policies. Building on recent advances in reasoning-based policy architectures, we introduce Reasoning through Action-free Data (RAD). RAD learns from both robot demonstration data (with reasoning and action labels) and action-free human video data (with only reasoning labels). The robot data teaches the model to map reasoning to low-level actions, while the action-free data enhances reasoning capabilities. Additionally, we will release a new dataset of 3,377 human-hand demonstrations compatible with the Bridge V2 benchmark. This dataset includes chain-of-thought reasoning annotations and hand-tracking data to help facilitate future work on reasoning-driven robot learning.
% experiments
Our experiments demonstrate that RAD enables effective transfer across the embodiment gap, allowing robots to perform tasks seen only in action-free data. Furthermore, scaling up action-free reasoning data significantly improves policy performance and generalization to novel tasks. These results highlight the promise of reasoning-driven learning from action-free datasets for advancing generalizable robot control. 
% releasing dataset
Website: \href{https://rad-generalization.github.io}{here}.

%
\section{Introduction}
\label{sec:intro}


\ps{Challenges of technology scaling}

The growing demand for computing performance has always been met by increasing the number of transistors per chip, which is only possible due to CMOS technology scaling.
However, as we keep pushing the boundaries of technology scaling, we encounter multiple challenges.
Firstly, whenever we transition to a more advanced technology node, the non-recurring cost due to physical design, verification, software, mask sets, and prototyping almost doubles \cite{cost-tech-node}.
As a result, designing a chip in an advanced technology node is only economically viable if the chip is manufactured in vast quantities.
Secondly, many chip components such as I/O drivers, analog circuits, or \gls{srams} have reached their scaling limit.
This means that we cannot shrink these components further, even if we use a more advanced technology with a smaller feature size.
Thirdly, advanced technology nodes suffer from high defect rates, diminishing the yield and inflating the recurring cost.
To tackle these challenges, new chip-design paradigms have been developed.

\ps{Why 2.5D integration?}

One of these new paradigms is 2.5D integration, where multiple silicon dies called chiplets are integrated into the same package.
Once designed, a single chiplet can be reused in multiple 2.5D stacked chips, which increases the ratio of production volume to non-recurring cost.
Another advantage is that multiple chiplets - fabricated in different technologies - can be integrated into the same package.
This means that only components that can take full advantage of technology scaling are built in bleeding-edge technologies.
Components that have reached their scaling limit are fabricated in more mature and hence less costly technology nodes.
Furthermore, chiplets are smaller than monolithic chips.
Therefore, manufacturing chiplets results in less silicon area loss due to fabrication defects and hence a higher yield.
Due to these economic advantages, chip vendors such as AMD \cite{amd-chiplet} and NVIDIA \cite{chiplet-book} have adopted the 2.5D integration paradigm.  

\ps{Challenges of 2.5D integration}

An important challenge when designing 2.5D stacked chips is the construction of a low-latency and high-throughput \gls{ici}. 
To build an \gls{ici}, we connect different chiplets using \gls{d2d} links.
These links are fabricated in an organic package substrate, silicon bridge, or silicon interposer, and they are connected to the chiplets using \gls{c4} bumps or microbumps.
The number of bumps per chiplet is limited, and so is the bandwidth of \gls{d2d} links.
In addition to having lower bandwidth than links in monolithic chips, \gls{d2d} links also have higher latency.
This latency is caused by wire delay and by \gls{phys} that are necessary in both the sending and the receiving chiplet.
\gls{phys} are needed to convert between protocols, voltage levels, and frequencies, which are usually different between on-chiplet links and \gls{d2d} links.
Due to these limitations, the \gls{ici} can quickly become a bottleneck.

\ps{How we solve these challenges differently than the related work does.}

Existing approaches to maximize the performance of the \gls{ici} either optimize the placement of chiplets (with potentially heterogeneous shapes) for a predetermined \gls{ici} topology 
\cite{ho,liu,seemuth,eris,osmolovskyi,tap25d,chiou}, select one topology out of a set of candidates \cite{coskun-1, coskun-2}, or they optimize the \gls{ici} topology for a 2D grid of homogeneously shaped chiplets on an active interposer \cite{butterdonut, cluscross, kite}.
To the best of our knowledge, there is no prior work on \gls{ici} topologies for chips with heterogeneously shaped chiplets or with passive silicon interposers or silicon bridges.
To fill this gap, we propose \name, a novel optimization methodology to jointly optimize the chiplet placement and \gls{ici} topology of such architectures.
\ifnb
\else
\newpage
\fi

\ps{Details on \name~and the key idea}

The key idea is as follows: 
We optimize the chiplet placement without a predetermined topology.
For each placement generated by an optimization algorithm, we infer a placement-based \gls{ici} topology by connecting chiplets that are in close proximity in that specific placement.
We then compute the latency and throughput of this combination of placement and topology for different traffic types.
These latencies and throughputs together with the total chip area are used to compute a user-defined quality-score of the placement, which is returned to the optimization algorithm.
Based on this quality score, the algorithm can further optimize the placement.
By following this iterative process, we jointly optimize the chiplet placement and the \gls{ici} topology.

\ps{Short evaluation-summary}

We provide our open-source framework implementing the proposed placement and topology co-optimization methodology, which we evaluate using both synthetic traffic and traffic traces.
A 2D grid of chiplets with a mesh topology is used as a baseline since many proposals for 2.5D stacked chips \cite{dataflow_accel_dnn, cifher, simba, hecaton, dojo} use such an architecture.
We reduce the latency of synthetic L1-to-L2 and L2-to-memory traffic, the two most important traffic types for cache coherency traffic, by up to 28\% and 62\% respectively.
For real traffic traces, we reduce the average packet latency for almost all traces and architectures considered (reduced by an 8\% or 18\% on average depending on the configuration of \gls{phys} within a chiplet).

%

\section{Related work}


Recent advances in single-image animatable head avatar generation can be categorized into mainly 2D-based and 3D-based approaches. 

\paragraph{\bf Image to 2D Animatable Avatar.}
2D-based methods, leveraging the power of convolutional neural networks (CNNs)~\cite{DBLP:conf/cvpr/KarrasLAHLA20,DBLP:conf/cvpr/IsolaZZE17,DBLP:conf/nips/GoodfellowPMXWOCB14}, often employ generative adversarial networks (GANs)~\cite{DBLP:conf/cvpr/StyleGAN} for direct image synthesis. Early approaches~\cite{DBLP:conf/cvpr/WangDYSW23,DBLP:conf/cvpr/BurkovPGL20,DBLP:conf/iccv/ZakharovSBL19} focus on injecting expression and pose features into the generator network, often utilizing architectures like U-Net or StyleGAN~\cite{DBLP:conf/cvpr/StyleGAN}.
Some other 2D methods~\cite{DBLP:journals/corr/abs-2407-03168,DBLP:conf/cvpr/ZhangQZZW0CW023,DBLP:conf/cvpr/HongZS022,DBLP:conf/mm/DrobyshevCKILZ22,DBLP:conf/cvpr/BurkovPGL20,DBLP:conf/nips/SiarohinLT0S19} represent expressions and poses as warping fields applied to the source image. 
Benefiting from advances in image and video diffusion networks, more recent 2D-based works~\cite{DBLP:journals/corr/abs-2410-07718,DBLP:journals/corr/abs-2406-08801,DBLP:conf/eccv/TianWZB24} get improved results with diffusion techniques. 
However, these methods still face challenges related to long generation times and significant computational resource demands. Audio-driven 2D control methods~\cite{DBLP:conf/cvpr/ZhangCWZSGSW23,DBLP:journals/corr/abs-2211-12368,DBLP:conf/iccv/GuoCLLBZ21} are easy to use but cannot explicitly control facial expressions and poses. 2D-based techniques often struggle with large pose or expression variations due to the lack of an explicit 3D structure, sometimes producing unrealistic distortions or identity changes. While some 2D methods~\cite{SadTalker,StyleHEAT,Pirenderer,DBLP:conf/cvpr/WangM021,MegaPortraits} incorporate 3D Morphable Models (3DMMs)~\cite{DBLP:conf/fgr/GerigMBELSV18,DBLP:journals/tog/LiBBL017,DBLP:conf/avss/PaysanKARV09,DBLP:conf/siggraph/BlanzV99} to mitigate these issues, they typically cannot achieve free-viewpoint rendering. 

\vspace{-0.1in}

\begin{figure*}[h]
    \centering
    \includegraphics[width=0.9\linewidth]{images/framework.pdf}
    \caption{\textbf{Overall Framework.} Our framework utilizes learnable query features attached to FLAME vertices to perform cross-attention with the extracted multi-level image features. The extracted features are then decoded to reconstruct the Gaussian avatar in the canonical space, which can be animated utilizing standard linear blend skinning (LBS) and corrective blendshapes as the FLAME model did and rendered in real-time on various platforms.}
    \label{fig:framework}
\end{figure*}

\paragraph{\bf Image to 3D Animatable Avatar.}
3D-aware methods offer improved geometric consistency and free-viewpoint rendering capabilities. Early 3D approaches~\cite{DBLP:conf/eccv/KhakhulinSLZ22,DBLP:conf/cvpr/XuYCWDJT20} utilize 3DMMs for head avatar reconstruction. With the advent of Neural Radiance Fields (NeRFs)~\cite{DBLP:conf/eccv/MildenhallSTBRN20}, many recent methods~\cite{DBLP:conf/siggraph/YuFZWYBCSWSW23,DBLP:conf/cvpr/MaZQLZ23,DBLP:conf/cvpr/LiZWZ0CZWB023,GPAvatar,ye2024real3d,deng2024portrait4d,deng2024portrait4d2,DBLP:conf/eccv/KiMC24,DBLP:conf/cvpr/BaiFWZSYS23,PointAvatar,Nerfies,INSTA} have adopted this representation for higher fidelity, particularly in modeling fine details like hair. However, NeRF-based~\cite{DBLP:conf/cvpr/ZhangZLHLWGCL024,HAvatar,DBLP:conf/cvpr/BaiTHSTQMDDOPTB23,AD-NeRF,DBLP:journals/tog/GaoZXHGZ22,DBLP:journals/tog/ParkSHBBGMS21,DBLP:conf/cvpr/AtharXSSS22,DBLP:journals/corr/abs-2112-05637,DBLP:conf/iccv/TretschkTGZLT21,DBLP:conf/cvpr/GafniTZN21,DBLP:conf/eccv/KiMC24,DBLP:conf/cvpr/BaiFWZSYS23,PointAvatar,Nerfies,DBLP:conf/siggraph/YuFZWYBCSWSW23,DBLP:conf/cvpr/MaZQLZ23,DBLP:conf/cvpr/LiZWZ0CZWB023} approaches often require extensive training data, including multi-view or single-view videos, raising privacy concerns and limiting generalization to unseen identities. Some methods~\cite{DBLP:conf/cvpr/SunWWLZZL23,DBLP:conf/3dim/ZhuangMKS22,DBLP:journals/pami/SunWZHWL24,DBLP:journals/tvcg/TangZYZCMW24,DBLP:conf/iclr/XuZLZBFS23} bypass this data requirement by training generators with random noise and then inverting them for identity-specific reconstruction, but inversion accuracy remains a challenge. Test-time optimization offers another alternative, but its computational cost limits practical applications. Several recent works~\cite{goha2023,hidenerf2023,gpavatar2024,ye2024real3d,ma2024cvthead,deng2024portrait4d,deng2024portrait4d2,GGHead} have explored one-shot 3D head reconstruction to address the limitations of data requirements and computational cost. These methods employ various techniques, such as tri-plane features, deformation fields, point-based expression fields, and vertex-feature transformers. Despite these advancements, NeRF-based methods often struggle with real-time rendering. 
Recently, 3D Gaussian Splatting~\cite{GaussianSplatting} has emerged as a promising alternative, offering both high-quality results and fast rendering speeds. However, existing Gaussian Splatting methods~\cite{GaussianAvatar,DBLP:conf/cvpr/XuCL00ZL24} typically rely on video data for training for each person, limiting their ability to generalize to new identities. Instead, the most recent work, GAGAvatar~\cite{GAGAvatar}, proposes a one-shot 3D Gaussian-based head avatar generation method. However, it still relies heavily on complex 2D neural post-processing to achieve optimal animation outcomes, thus it is not a pure 3D solution and the extra neural network hinders its application on various platforms. In contrast, our work generates Gaussian heads that are immediately animatable and renderable without additional networks or post-processing steps, enabling seamless integration into existing rendering pipelines for real-time animation and rendering across a wide range of platforms, including mobile phones. 
%
\section{The \dsname Benchmark}
\label{sec:benchmark}
%
%\dsname---\textbf{G}lobally \textbf{I}nclusive \textbf{M}ultimodal \textbf{M}ult\textbf{i}task \textbf{C}ultural \textbf{K}nowledge---is an extensive benchmark comprising several text-only, text-image, and text-video tasks to assess the cultural knowledge of LLMs and LVLMs.
%
\paragraph{Cultural Benchmark Positioning}
%
\citet{adilazuarda-etal-2024-towards} surveyed 90+ recent papers on cultural awareness in LLMs and found that \emph{none} explicitly define ``culture''.
%
Instead, these studies evaluate models on datasets capturing only specific cultural aspects, which the authors organize into two dimensions: \textit{demographic} and \textit{semantic} proxies (with seven and five subsets, respectively).
%
In \dsname, we adopt the proposed taxonomy by using countries and regions as \textit{demographic} cultural proxies.
%
Our tasks span all five \textit{semantic} proxies: ``emotions and values'', ``food and drink'', ``social and political relations'', ``basic actions and technology'', and ``names''.
%
We implement primarily ``black-box'' generative and discriminative probing approaches.%, along with a ``grey-box'' method, to examine different aspects of cultural knowledge.
%

\rparagraph{UNESCO Intangible Cultural Heritage}
\label{sec:benchmark:datasource}
%
All tasks in \dsname are based on high-quality open-access data from the UNESCO Intangible Cultural Heritage (ICH) project\footnote{\url{https://ich.unesco.org}}, which aims to safeguard cultural traditions and practices vital to the identity and heritage of communities worldwide while honoring cultural diversity.% and human creativity.
%
Intangible cultural heritage encompasses oral traditions, performing arts, rituals, festive events, traditional craftsmanship, and cultural knowledge.
%
The open-access dataset is structured as a knowledge graph, where most nodes represent cultural events or facets (CEFs; e.g., \emph{Yuki-tsumugi}, a silk fabric production technique from Japan\footnote{More examples including images are shown in \S\ref{appendix:sec:benchmark:cef:examples}}), with additional nodes including countries, regions, case studies in which the CEFs occur.
%
%Additional nodes include countries, regions, case studies, or scientific projects in which the CEFs occur.
%
%CEF nodes are interconnected through conceptual terms sourced from various thesauri, including the official UNESCO thesaurus and specialized vocabularies.
%
For \dsname, we extract the CEFs, each together with their title, description, associated macro-regions and countries, and several images depicting different aspects of the CEF.
%
Moreover, each CEF is detailed in one or more YouTube videos.
%($9.43$ on avg.) ($1.36$ on average)
In total, \dsname contains 728 CEFs from 144 countries represented by 6,887 images and 993 videos\footnote{We provide licensing details in \S\ref{appedix:sec:benchmark:license}}.
%
While most CEFs ($88.60\%$) are associated with one country, some are associated with two or more countries.% ($1.34$ on average).
%
The UNESCO ICH project groups the countries into six global macro-regions\footnote{We provide a comprehensive list in Table~\ref{tab:benchmark:regions_full} in \S\ref{appendix:sec:benchmark:regions}}, which we adopt in this work. Throughout the paper---including all figures and tables---we use the region abbreviations listed in Table~\ref{tab:benchmark:regions}.
%
\begin{table}[t]
    \centering
    \footnotesize
    \renewcommand{\arraystretch}{.85}
    \resizebox{\linewidth}{!}{%
    \begin{tabular}{llrr}
    \toprule
    \textsc{Region} & \textsc{Abbrv.} & \textsc{\#C} & \textsc{\#CEF} \\
    \midrule
    Arab & \RegA & 18 &  76 \\
    Asia \& Pacific & \RegAP & 35 & 226 \\
    Eastern Europe & \RegE & 25 &  150 \\
    Latin-America \& Caribbean & \RegLAC & 28 & 98  \\
    Subsaharian Africa & \RegSA & 40 & 73 \\
    Western Europe \& North America & \RegW & 23 & 149 \\
    \midrule
    \multicolumn{2}{l}{\textit{Unique}} & 144 & 728 \\
    \bottomrule
    \end{tabular}
    }
    \caption{Regions within \dsname. \textsc{\#C} and \textsc{\#CEF} stand for the number of Countries and CEFs related to the respective region. Some CEFs may span multiple regions.}
    \label{tab:benchmark:regions}
\end{table}
%


% The descriptions contain $212.26$ whitespace-tokenized words on average.


\subsection{Datasets and Tasks}
\label{sec:benchmark:datasets}
%
We created three novel multimodal datasets that serve as the foundation for six tasks designed to evaluate the cultural knowledge of models.
%
See Figure~\ref{fig:figure1} for an overview of the different tasks.\footnote{Sample counts per task \& region are shown in \S\ref{appendix:sec:benchmark:samples}}
%
%Following this, the different tasks will be explained in detail.
% %
% \begin{table}[ht]
%     \centering
%     \renewcommand{\arraystretch}{.97}
%     \resizebox{\linewidth}{!}{%
%     \begin{tabular}{llllrrrr}
%         \toprule
%         \textsc{Name} &
%         \textsc{Input} &
%         \textsc{GT} &
%         \multicolumn{1}{l}{\#\textsc{S}} &
%         \multicolumn{1}{l}{\#\textsc{CEF}} &
%         \multicolumn{1}{l}{\textsc{\#Img}} &
%         \multicolumn{1}{l}{\#\textsc{C}} \\
%         \midrule
%         \sivqa & I+T & Word & 2233 & 635 & 1928 & 144 \\
%         \vvqa & V+T & Word & 1809 & 553 & 1809 & 139 \\
%         \coqar & I+T, T, I  & Letter  & 759 & 725 & 6857 & 144 \\
%         \coqac & I+T, T, I  & Letter  & 982 & 728 & 6887 & 144 \\
%         \ckqad & I+T, T, I & Desc. & 728 & 728 & 6887 & 144 \\
%         \ckqan & I+T, T, I & Title  & 728 & 728 & 6887 & 144 \\
%         \bottomrule
%     \end{tabular}
%     }
%     \caption{Statistics on the \dsname tasks. For \vvqa, the counts refer to videos instead of images. Input modalities are abbreviated as follows: \textbf{I}mage, \textbf{V}ideo, \textbf{T}ext. \textsc{\#S} and \textsc{\#C} denote the number of samples and covered countries, respectively.}
%     \label{tab:benchmark:datasets:stats}
% \end{table}

%
%
\subsection{Cultural Image VQA}
\label{sec:sivqa}
%
In the Cultural Image VQA (\sivqa) task, models are presented with an image depicting a CEF and a question that relates to a particular CEF aspect (see \S\ref{appendix:sec:sivqa:examples} for examples).
%
Models are evaluated based on answer correctness.
%
To create the data for \sivqa, we couple synthetic data generation with a two-stage annotation process.
%

\rparagraph{Synthetic Data Generation}
\label{sec:sivqa:collection}
%
Building on the high-quality UNESCO ICH data, we applied synthetic data generation by prompting GPT-4o\footnote{\texttt{gpt-4o-2024-08-06}} to construct the basis for our dataset.%, a large number of ``silver'' VQA pairs.
%
Each VQA pair is related to a CEF and consists of an image depicting one aspect of the CEF, a question related to the CEF and the image, and an answer.
%
Maximizing the quality of the generated silver data, we applied extensive prompt engineering combining techniques such as Few-Shot, Chain-of-Thought, ReAct~\cite{wei2022cot,zhang2023autocot,zheng2024react,sahoo2024promptsurvey} to craft the prompt.
%
Key aspects of the prompt are a role description, a general task description, detailed annotation guidelines, a step-by-step strategy, an expected output format, few-shot examples, and the information of the target CEF (see \S\ref{appendix:sec:sivqa:synth} for the full prompt).
%
We then generated silver VQA pairs for each of the 6,827 images contained in the ICH data source, which resulted in 17,369 pairs.
%
Afterward, we automatically removed pairs where 1) the question contained words that introduce subjectiveness or ambiguity (``\textit{could}'', ``\textit{should}'', ``\textit{maybe}'', etc.); 2) the answer contained abstract words that are hard to depict visually; and 3) where the answer is not a substring of the description of the related CEF.
%
This way, we obtained 9,900 silver VQA samples related to 5,517 images from all 728 CEFs.
%

\rparagraph{Annotation Process}
\label{sec:sivqa:collection:annotation}
%
Opting for high-quality VQA pairs as well as cultural diversity, we devised a two-stage annotation process with 18 trained experts from various cultural backgrounds covering all six regions (see Table~\ref{tab:sivqa:anno:demographics} in \S\ref{appendix:sec:sivqa:anno}).
%
Each silver pair was evaluated using two questionnaires---one with seven question-related requirements and another with four answer-related requirements.
%
Questions had to target the CEF and image content directly, require cultural knowledge, and depend on visual evidence \cite{chen2024mmstar}.
%
Answers needed to be clear, objective, concise, and depictable.
%
For details on the annotation process, see \S\ref{appendix:sec:sivqa:anno}.
%

In the first round, we annotated each sample once, resulting in 4,114 samples, of which 2,826 (68.69\%) met all criteria.
%
In the second round, five annotators re-evaluated these, retaining only samples with concordant approval.
%
This process finally yielded 2,233 samples for 1,928 images from 728 CEFs across 144 countries in six global regions.% (see Table~\ref{tab:benchmark:datasets:samples}).
%

%
\subsection{Cultural Video VQA}
\label{sec:vvqa}
%
In this task, models are evaluated on questions relating to videos instead of single images, again employing accuracy as the metric.
%
To this end, we extend \sivqa in two steps: synthetic data generation and quality annotation.
%

\rparagraph{Synthetic Data Generation}
\label{sec:vvqa:collection}
%
First, we adjusted the \sivqa questions by replacing the term  \emph{``image''} with \emph{``video''}. We then coupled the question with a short video clip, for which we started from the CEF's associated YouTube video. We ensured that the shortened clip contains relevant information for answering the question as follows:
%
From each video, we extracted one frame per second, and computed image embeddings for both the frames and the \sivqa image, using DINOv2\footnote{\texttt{facebook/dinov2-with-registers-large}}~\cite{oquab2024dinov2,darcet2024dinov2registers}.
%
We then identified the frame that best matches the original image by calculating Cosine similarity. We selected this frame as the center (at $t=0$) for a 10-second clip\footnote{
We do not include the audio stream in our clips.} (from $t=-5$ to $t=5$).
%
We only include clips with a best-matching frame similarity $>0.5$, which we found to yield high-quality instances based on a manual inspection of random samples.
%
Overall, this procedure resulted in 2,001 silver samples.
%

\rparagraph{Annotation Process}
\label{sec:vvqa:collection:annotation}
%
For additional quality control, a trained expert annotated 20\% of the silver data (400 samples).
%
Each sample was evaluated using a three-item questionnaire\footnote{cf. \S\ref{appendix:sec:vvqa} for details.} assessing whether (1) the video contained frames resembling the CEF image, (2) it clearly answered the question, or (3) neither condition was met.
%
Overall, 95\% of the annotated samples were accepted.
%
For closer inspection, we stratified the annotated samples into four similarity bins, revealing that roughly 10\% of those in the lower bins ($[0.5, 0.75[)$ were rejected, while nearly all, i.e., 99\% and 100\%, in the higher bins ($[0.75, 1.0]$) were retained.
%
The residual 5\% label noise was considered acceptable based on further manual analysis.
%
Notably, we found that of the 20 rejected samples, only 9 were unanswerable based on the video, while the remaining 11 exhibited only a suboptimal frame match w.r.t. the \sivqa image.
%
The final \dsname \vvqa dataset contains 1,809 samples (see \S\ref{appendix:sec:vvqa:examples} for examples) linked to 553 CEFs from 139 countries.
%
%
\subsection{Cultural Origin QA}
\label{sec:coqa}
%
With Cultural Origin QA (\coqa), we test a model's ability to capture coarse-grained cultural knowledge.
%
Given a CEF's images, title, or both, the models must select its cultural origin (multiple-choice).
%
We refer to the task as \coqar when the origin is a region and as \coqac when it is a country.
%

\rparagraph{Dataset Construction}
\label{sec:coqa:collection}
%
The \coqa dataset contains all 728 CEFs from UNESCO ICH.
%
To ensure that each instance corresponds to a unique origin, we replicate each CEF $N$ times—where $N$ represents the number of associated regions (for \coqar) or countries (for \coqac).
%
For \coqar, three negatives are randomly sampled from the remaining pool.
%
%CEFs associated with more than three regions are excluded to guarantee exactly four multiple-choice options with three negatives.
%
Negatives for \coqac drawn from those within the same region as the target country.
%

\rparagraph{Input Modalities and Prompts}
\label{sec:coqa:config}
%
The \coqa tasks support multiple input configurations alongside the task prompt.
%
In the text-only setting, only the title of the CEF is provided, whereas in the ``image-only'' setting, \emph{all} images associated with the CEF are included.
%
Both the title and the images are used in the text-image setting.
%
Examples and complete prompts for all variations are shown in \S\ref{appendix:sec:coqa:examples}.
%

%
\subsection{Cultural Knowledge QA}
\label{sec:ckqa}
%
In \dsname Cultural Knowledge QA (\ckqa), we evaluate whether current AI models capture fine-grained cultural knowledge.
%
%Unlike \coqa, which tests coarse-grained cultural origins, \ckqa focuses on detailed knowledge of CEFs.
%
The dataset supports two open-answer tasks: naming (\ckqan) and describing (\ckqad).
%
For \ckqan, the ground truth corresponds to the title of the CEF, while for \ckqad, it is the detailed description.
%
For both tasks, we leverage all 728 CEFs from UNESCO ICH.
%
As with \coqa, \ckqa supports multiple input configurations: text-only, ``image-only'', and text+image.
%
We provide examples and prompts for all variations in \S\ref{appendix:sec:ckqa:prompts}.
%

%

%
\section{Experimental Setup}
\label{sec:exsetup}
%
\rparagraph{Models and Inference}
\label{sec:benchmark:models}
%
We evaluate a total of 31 models, including five proprietary LVLMs, 15 open-weight LVLMs, and 11 open-weight LLMs---the backbones of the respective LVLMs---covering 9 LVLM and 4 LLM model families.
%
The sizes of the open-weight models vary, categorized as small, medium, large, and extra-large (see Table~\ref{tab:benchmark:model_groups}).
%
A comprehensive list of models is provided in Table~\ref{tab:benchmark:models} in \S\ref{appendix:sec:benchmark:models}.
%
\begin{table}[t]
    \centering
    \footnotesize
    \renewcommand{\arraystretch}{.8}
    \begin{tabular}{l l r r}
        \toprule
        \textsc{Group} & \textsc{Parameters (B)}  & \textsc{LLMs} & \textsc{LVLMs} \\
        \midrule
        S  & 0.5 -- 4 & 5 & 5 \\
        M  & 7 -- 11 & 3 & 6 \\
        L  & 26 -- 38 & 2 & 2 \\
        XL & 72 -- 78 & 1 & 2 \\
        Closed  & unkown & 0 & 5 \\
        \midrule
        \multicolumn{2}{l}{\textit{Total}} & 11 & 20 \\
        \bottomrule
    \end{tabular}
    \caption{The size groups we define for result aggregation according to models' number of parameters.}
    \label{tab:benchmark:model_groups}
\end{table}
%
%\rparagraph{Model Inference}
%
For our experiments, we download open weights from the respective Huggingface~\cite{wolf2019hftransformers} repositories (see Table~\ref{tab:benchmark:models}) and generate responses employing greedy decoding.
%
For proprietary models, we use the official Python SDKs.
%
More details are reported in \S\ref{appendix:sec:setup}.

\rparagraph{Metrics}
%
%Depending on the task, we employ different evaluation metrics.
%
For the \sivqa, \vvqa, and \coqa tasks, we report relaxed answer accuracy, for which we consider a generated answer correct if it starts with the ground truth answer.
%
For \ckqad and \ckqan, due to their generative nature, we use GPT-4o\footnote{\texttt{gpt-4o-2024-11-20}} in an ``LVLM-as-a-Judge''~\cite{zheng2023llm-as-a-judge, xiong2024llava-critic} setup to judge responses with a score $s \in [0, 100]$.
%
Where $s=0$, $s=50$, and $s=100$ indicate \textit{completely incorrect or irrelevant}, \textit{partially correct or relevant}, and \textit{perfectly correct and complete} answers, respectively.
%
%We show the judge prompt and examples for different scores in \S\ref{appendix:sec:ckqa:judge}.
%

\rparagraph{Video Processing}
%
The 10-second video clips from \vvqa do not contain an audio stream, and we only use the visual information.
%
Following established praxis~\cite[e.g.,][]{wang2024qwen2vl}, we extract one frame per second from the videos and provide them to the models as input alongside the textual prompt.
%
Specifics about the image and video processing of the individual models are documented in the code.
%

\section{Results and Analyses}
\label{sec:analyses}
%
In this section, we present a series of in-depth analyses based on the outcomes of our benchmark.
%
We show aggregated results: open-weight models are grouped and averaged by parameter size, and proprietary models are averaged together (see Table~\ref{tab:benchmark:model_groups}).
%
We provide the complete numerical results for all tasks and models in \S~\ref{appendix:sec:analyses}.
%
In the following, we use abbreviations for regions‚ as defined in Table~\ref{tab:benchmark:regions}.

%
\subsection{General Trends and Cultural Bias}
\label{sec:analyses:a1_bias}
%
We discuss general trends and investigate cultural bias across regions (Figures~\ref{fig:analyses:a1_bias:scores} and~\ref{fig:analyses:a1_bias:ppl}). %and analyzing ground-truth answer perplexity for the \m{QwenVL} model family on \sivqa (Figures~\ref{fig:analyses:a1_bias:scores} and~\ref{fig:analyses:a1_bias:ppl}).
%

\begin{figure}[t]
     \begin{subfigure}{1.\linewidth}
        \centering
        \includegraphics[width=1.\linewidth]{gfx/sivqa_avg_scores.pdf}
        \caption{\sivqa Accuracy}
        \label{fig:analyses:a1_bias:scores:sivqa}
    \end{subfigure}

    \begin{subfigure}{1.\linewidth} 
        \centering
        \includegraphics[width=1.\linewidth, trim=0 0 0 31.5, clip]{gfx/vvqa_avg_scores.pdf}
        \caption{\vvqa Accuracy}
        \label{fig:analyses:a1_bias:scores:vvqa}
    \end{subfigure}

    \begin{subfigure}{1.\linewidth}
        \centering
        \includegraphics[width=1.\linewidth, trim=0 0 0 25, clip]{gfx/ckt_naming_avg_scores.pdf}
        \caption{\ckqan Accuracy}
        \label{fig:analyses:a1_bias:ckqa-name}
    \end{subfigure}
    \caption{Aggregated results of the VQA tasks.}
    \label{fig:analyses:a1_bias:scores}
\end{figure}
%
%
\begin{figure}[t]
    \centering
    \includegraphics[width=0.8\linewidth]{gfx/sivqa_perplexity_qwen.pdf}
    \caption{\sivqa ground-truth answer perplexity.}
    \label{fig:analyses:a1_bias:ppl}
\end{figure}

\rrparagraph{\sivqa \& \vvqa}
%
Figures~\ref{fig:analyses:a1_bias:scores}a–c show clear regional performance disparities.
%
Across all models---proprietary and open-weight, regardless of size---scores are highest for Western and Asian targets (\RegW, \RegE, and \RegAP) and lowest for \RegSA.
%
XL models, e.g., reach $24.04$ on \RegW and $9.32$ on \RegSA on average.
%
\RegA and \RegLAC fall in between, with model performance varying by size.
%
Since \sivqa is an open-answer task, often with rare culturally specific terms, we also evaluated the task with GPT-4o as LVLM-as-a-Judge to account for imperfect naming or spelling.
%
While this method yields higher scores, it confirms the same trend: models exhibit a strong bias toward Western contexts.
%
However, even the best model (\m{GPT-4o}) scores only $31.58$\% on \RegW and $25.44$\% on average, highlighting \dsname as a challenging benchmark and the lack of fine-grained cultural knowledge in current models.
%
We supplement our analysis with a more fine-grained investigation of how well models ``know'' the cultural concepts discussed.
%
Here, we focus on the \m{QwenVL} models on \sivqa and the compute perplexity of ground truth answers (conditioned on the input context) as a proxy of model cultural knowledge (details in \S\ref{appendix:sec:analyses:sivqa:results:ppl}). 
%
Figure~\ref{fig:analyses:a1_bias:ppl} shows that for the 7B and 72B models, perplexity is consistently lower for \RegW, \RegE, and \RegAP compared to \RegA and \RegSA, aligning with our performance findings.
%
For the 2B model, however, \RegE and \RegSA yield the highest perplexities, which we attribute to the overall brittleness of the model.
%
Moreover, we revisit the performance on questions about the prevalent cultural aspects in \sivqa (details in \S\ref{sec:appendix:sec:analyses:sivqa:results:aspects}) and find that models perform notably better on tangible cultural aspects than on intangible ones.
%
For instance, closed models achieve an accuracy of 30\% for food-related questions and only 8\% and 10\% for questions concerning rituals or festivals.
%
This highlights biases along the cultural dimension, which are particularly pronounced in non-Western contexts.
%This highlights not only regional but also biases along the cultural dimension, the latter being particularly pronounced in non-Western contexts.
%

%Figure~\ref{fig:analyses:a1_bias:ppl} shows that for the 7B and 72B models, perplexity is consistently lower for \RegW, \RegE, and \RegAP compared to \RegA and \RegSA, aligning with our performance findings.
%
%For the smallest model, however, \RegE and \RegSA yield the highest perplexities.
%

\rrparagraph{\ckqan \& \ckqad}
%
For \ckqan, regional differences are minor, though proprietary models significantly outperform open-weight ones (see Figure~\ref{fig:analyses:a1_bias:ckqa-name}).
%
The large error bars for closed models indicate inconsistent performance---particularly from \m{GPT-4o Mini} and \m{Gemini Flash} models, which perform similarly to large open-weight models.
%
XL and L models perform worst on \RegSA and \RegLAC and best on \RegA and \RegAP with minor differences to \RegW and \RegE.
%
For \ckqad (Figure~\ref{fig:analyses:a3_modality:ckqa-desc}), performance is $10–20\%$ higher than on \ckqan, likely because describing a CEF is easier than exactly naming it.
%
However, regional biases are larger, with consistently higher scores on \RegW than on \RegSA, primarily for closed models like \m{GPT-4o}, which reaches $53.66$ for \RegW and $43.70$ on \RegSA.
%

\rrparagraph{\coqac \& \coqar}
%
Figure~\ref{fig:analyses:a3_modality:coqa-countries} shows minimal regional differences for \coqac.%, with all models performing well across regions.
%
Average accuracies range from close to or above $90\%$ for closed, XL, and L models to $77.42$\% for S models.
%
However, performance on \coqar is lower than on \coqac---$85.02\%$ vs. $81.17\%$ on average over all models and regions--- with models achieving the highest scores in \RegAP.
%
Notably, the regional ranking is mostly inverted compared to other tasks---\RegSA, \RegA, \RegLAC, \RegE, and \RegAP score higher than \RegW---suggesting more distinct visual and linguistic features in non-Western regions.
%


%\rparagraph{Perplexity Analysis}
%
%We also examined regional biases in ground-truth answer perplexity on \sivqa for the Qwen2 VL family.
%
%The perplexity for every sample is computed as:
%
%{
%\footnotesize
%\begin{equation*}
%\mathrm{PPL}(y \mid x) = \exp\left(-\frac{1}{N} \sum_{t=0}^{N} \log p\left(y_t \mid y_{t-1},\, x\right)\right)
%\end{equation*}
%}
%where $x = \{s, v\}$ are the textual ($s$) and visual ($v$) prompt (prefix) tokens and $y$ are the $N$ ground-truth answer tokens.
%
%Figure~\ref{fig:analyses:a1_bias:ppl} shows that for the 7B and 72B models, perplexity is consistently lower for \RegW, \RegE, and \RegAP compared to \RegA and \RegSA, aligning with our performance findings.
%
%For the smallest model, however, \RegE and \RegSA yield the highest perplexities.
%
%This reflects similar findings as previous black-box probing approaches.
%

%
% \subsection{Influence of Model Characteristics}
\subsection{Influence of Model Size}
\label{sec:analyses:a2_model}
%
% We assess how model size and family impact performance, whether lower parameter counts affect regions uniformly, and how proprietary models compare to open ones.
We assess how model size impacts performance and whether it affects regions equally.

\begin{figure}[t]
    \centering
    \begin{subfigure}{0.30\linewidth}
        \centering
        \includegraphics[width=\linewidth]{gfx/size_vs_sivqa_score.pdf}
        \caption{\sivqa, \tiny{$r$=$0.62$\textsuperscript{*}}}
    \end{subfigure}
    % \hfill
    \begin{subfigure}{0.30\linewidth}
        \centering
        \includegraphics[width=\linewidth]{gfx/size_vs_vvqa_score.pdf}
        \caption{\vvqa, \tiny{$r$=$0.36$\textsuperscript{*}}}
    \end{subfigure}
    % \hfill
    \begin{subfigure}{0.30\linewidth}
        \centering
        \includegraphics[width=\linewidth]{gfx/size_vs_coqac_score.pdf}
        \caption{\coqac, \tiny{$r$=$0.39$\textsuperscript{*}}}
    \end{subfigure}

    \begin{subfigure}{0.30\linewidth}
        \centering
        \includegraphics[width=\linewidth]{gfx/size_vs_coqar_score.pdf}
        \caption{\coqar, \tiny{$r$=$0.16$\textsuperscript{*}}}
    \end{subfigure}
    % \hfill
    \begin{subfigure}{0.30\linewidth}
        \centering
        \includegraphics[width=\linewidth]{gfx/size_vs_ckqad_score.pdf}
        \caption{\ckqad, \tiny{$r$=$0.46$}\textsuperscript{*}}
    \end{subfigure}
    % \hfill
    \begin{subfigure}{0.30\linewidth}
        \centering
        \includegraphics[width=\linewidth]{gfx/size_vs_ckqan_score.pdf}
        \caption{\ckqan, \tiny{$r$=$0.39$}\textsuperscript{*}}
    \end{subfigure}
    \caption{Model size vs. performance on \dsname tasks. The x-axis is in log scale. The trend line was computed using OLS regression. We report the Pearson correlation coefficient $r$ ( \textsuperscript{*} indicates statistical significance).}
    \label{fig:sec:analyses:a2_model:size}
\end{figure}
%
\begin{figure}[t]
    \centering
    \includegraphics[width=1.\linewidth]{gfx/sivqa_relative_diffs_to_w.pdf}
    \caption{Relative Difference to \RegW for \sivqa.}
    \label{fig:sec:analyses:a2_model:sivqa_relative_diffs_to_w}
\end{figure}

%
Figure~\ref{fig:sec:analyses:a2_model:size} shows that model size\footnote{For closed source models, we manually set the number of parameters to 1T, except for Gemini Flash and GPT-4o mini, for which we set the number to 500B.} significantly influences performance, with moderate to strong Pearson correlations and steep regression lines across tasks except \coqar, where the effect is minimal.
%
Figure~\ref{fig:sec:analyses:a2_model:sivqa_relative_diffs_to_w} shows that relative performance declines from the best-performing region (\RegW) to others, particularly \RegSA, varying by model size: the drops are $-63.39$ (S), $-63.85$ (M), $-50.60$ (L), $-54.57$ (XL), and $-41.52$ (Closed). We conclude that bigger sizes tend to result in smaller gaps without size presenting a strict ordering criterion.
%

% \rparagraph{Model Family}
% %
% TODO GLMM (?) discussion

%
\subsection{Influence of Modalities}
\label{sec:analyses:a3_modality}
%
We explore how input modality---text-only, image-only, or text+image---affects performance on \coqac, \coqar, and \ckqad.
%
Further, we compare LVLMs to their LLM backbones to assess potential losses in cultural knowledge during multimodal training.
%

\begin{figure*}[t]
    \centering
     \begin{subfigure}{1.\linewidth}
        \centering
        \includegraphics[width=1.\linewidth]{gfx/coqa_countries_avg_scores.pdf}
        \caption{\coqac}
        \label{fig:analyses:a3_modality:coqa-countries}
    \end{subfigure}
    
     \begin{subfigure}{1.\linewidth}
        \centering
        \includegraphics[width=1.\linewidth, trim=0 0 0 25, clip]{gfx/coqa_regions_avg_scores.pdf}
        \caption{\coqar}
        \label{fig:analyses:a3_modality:coqa-regions}
    \end{subfigure}

    \begin{subfigure}{1.\linewidth}
        \centering
        \includegraphics[width=1.\linewidth, trim=0 0 0 25, clip]{gfx/ckt_desc_avg_scores.pdf}
        \caption{\ckqad}
        \label{fig:analyses:a3_modality:ckqa-desc}
    \end{subfigure}
    \caption{Aggregated results including multimodal input variations: \textbf{T}ext-only, \textbf{I}mage-only, \textbf{T}ext+\textbf{I}mage.}
    \label{fig:analyses:a3_modality:scores}
\end{figure*}
%
%
\rparagraph{Input Modalities}
Figure~\ref{fig:analyses:a3_modality:scores} shows that text+image (\texttt{I+T}) inputs consistently yield the highest performance across all tasks, confirming that textual and visual data provide complementary cultural cues.
%
The gap between \texttt{I+T} and text-only (\texttt{T}) is slightly more prominent for \coqac than \coqar, suggesting that visual information aids in inferring fine-grained, country-level details.
%
In contrast, image-only (\texttt{I}) inputs perform poorly, indicating that textual information, such as CEF titles, carries more cultural context.
%
The high variance in \texttt{T} results for the \coqa tasks stems from the performance disparity between \m{Gemini Pro} and \m{Claude 3.5 Sonnet} (e.g., $59.38$ vs. $83.75$ for \RegW).
%

\rparagraph{LVLM vs. LLM-Backbone}
%
Comparing LVLMs with their LLM backbones reveals that multimodal training can impair the acquisition of detailed cultural knowledge (notably in \ckqad) while having minimal impact on coarse-grained cultural understanding (\coqa).
%
For large models, significant performance gaps---$50.62$ for \m{Qwen2.5 72B} vs. $40.02$ for \m{Qwen2VL 72B} on \RegAP---on the \ckqad task between the LVLMS and their LLM backbones can be observed, whereas, for smaller models, the effect is subtle.
%
Overall, our findings highlight that while images complement text for culturally grounded tasks, it is ultimately the synergy between both modalities that leads to robust and broad cultural understanding.
%

%
\subsection{Influence of External Cues}
\label{sec:analyses:a4_external}
%
We examine how external hints, i.e., informing a model about the country or region of a CEF, affect VQA performance.
%
\begin{figure*}[t]
    \centering
    \begin{subfigure}{1.\linewidth}
        \centering
        \includegraphics[width=1.\linewidth]{gfx/sivqa_relative_gains.pdf}
        \caption{\texttt{SIVQA}}
        \label{fig:analyses:a4:relative:sivqa}
    \end{subfigure}
    
    \begin{subfigure}{1.\linewidth}
        \centering
        \includegraphics[width=1.\linewidth, trim=0 0 0 0, clip]{gfx/vvqa_relative_gains.pdf}
        \caption{\texttt{VVQA}}
        \label{fig:analyses:a4:relative:vvqa}
    \end{subfigure}
    \caption{Relative gains on VQA tasks from providing external geographical hints.}
    \label{fig:analyses:a4:relative}
\end{figure*}
%
For \sivqa (Figure~\ref{fig:analyses:a4:relative:sivqa}), country hints consistently boost performance across model sizes and regions, while regional cues yield only modest—or even slightly adverse—effects in larger models.
%
Gains from country hints are around 50\% for most regions, but in \RegSA, improvements nearly double (e.g., $97.48\%$ for \m{InternVL 2.5 78B} and $97.13\%$ for \m{InternVL 2.5 38B}).
%
A similar pattern emerges for \vvqa (Figure~\ref{fig:analyses:a4:relative:vvqa}).
%
Hints generally enhance performance across regions and models, with \RegSA showing the most significant gains.
%
Proprietary and small models exhibit subtle improvements, whereas L and XL models see much higher relative gains---up to $240.7\%$ for \m{Intern VL 38B}.
%
Notably, regional cues have a more positive impact on \vvqa than on \sivqa.
%

%

% \subsection{Key Findings}
% \label{sec:analyses:findings}
% %
% In summary, we find that models generally perform relatively low on tasks requiring nuanced cultural knowledge across all regions with overall averages of $16.67$ and $21.13$ accuracy and $13.75$ and $30.14$ Judge-Score for \sivqa, \vvqa, \ckqan, \ckqad, respectively. 
% %
% Further, we assess a significant Western bias on these tasks across all models, with, e.g., the average accuracy on \RegW of \sivqa ($17.63$) being significantly lower than on \RegSA ($6.93$). 
% %
% In tasks where only coarse cultural knowledge, i.e., the regional or country origin of a CEF, is required, the models are significantly better, with overall averages of $81.17$ for \coqar and $85.02$ \coqac.
% %
% Interestingly, models perform much better on non-Western cultures, with, e.g., \RegW having the lowest average accuracy of $74.64$ and \RegAP having the highest with $92.94$ followed by \RegSA with $90.08$ for the \coqar task.
% %
% This suggests more distinct visual and linguistic features of the mentioned regions compared to relatively 
% %

%
\section{Conclusion}
\label{sec:conclusion}
%
We introduce \dsname, a comprehensive benchmark to assess various aspects of cultural knowledge of current LVLMs and LLMs and introduce six tasks built upon three novel datasets, which span 728 unique cultural events or facets (CEFs) from 144 countries grouped into six global macro-regions.
%
Through extensive analyses, we study general cultural biases and the influence of model size, input modalities, and external cues.
%
Our results consistently reveal a prominent bias toward Western cultures across all models.
%
Interestingly, when only coarse cultural knowledge is required---such as regional origins---models performed remarkably better.
%
Across all tasks, significant correlations between a model's performance and its size are evident, with a substantial gap between proprietary and open-weight models.
%
Our analyses show that while models grasp broad cultural categories, they struggle with nuanced understanding.
%
This suggests that \dsname poses a challenging benchmark and highlights the need for further advances in modeling broad cultural awareness.
%

%
\newpage
\section*{Limitations}
\label{sec:limitations}
%
\paragraph{English-Only Benchmark}
%
Although we consider the performance on tasks requiring cultural understanding in English as an upper bound for the majority of models, it is yet to be tested if that hypothesis generally holds across tasks, model size, and model family.
%
Especially for models like \m{QwenVL} and \m{InternVL}, which were pretrained on large portions of Chinese textual data, Chinese could be pivotal instead of English.
%
Moreover, some cultural nuances might not be translatable to other languages.
%

\rparagraph{Open-Ended VQA}
%
%
\sivqa and \vvqa comprise open-ended answers to their questions, imposing challenges for adequate evaluation, especially when employing binary metrics like accuracy.
%
This is especially true for rare, culturally specific answer terms, such as in our tasks, which are prone to spelling inaccuracies or might have different names in different cultures or languages.
%
Although we alleviate this issue by computing scores using GPT-4o in an LVLM-as-a-Judge setting and thereby confirm our findings, this requires additional computational and financial resources.
%
A typical solution for this is transforming the questions into multiple choice questions, which, however, requires culturally expert annotators, which are complicated to find or train and expensive if hired via professional annotation companies.
%

\rparagraph{Small Number of Samples}
%
With a total of 7239 unique samples across all tasks in \dsname---2233 (\sivqa), 1809 (\vvqa), 982 (\coqac), 759 (\coqar), 728 (\ckqad), and 728 (\ckqan)---, the benchmark itself as the third most samples compared to other recent benchmarks.
%
However, the per-task number falls relatively low, leading to even fewer counts per country or culture, making judgments about single countries not informative.
%

%
\section*{Ethical Considerations}
\label{sec:ethical}
%
\rparagraph{Country and Region Definitions}
%
\dsname adopts the country and region classifications from the UNESCO ICH dataset.
%
While these classifications are widely used, we recognize the potential for differing interpretations.
%

\rparagraph{Potentially Offensive Questions}
%
We employed semi-automatic data generation strategies to create the \sivqa dataset.
%
Here, the silver data was generated using GPT-4o, which we showed displays significant cultural biases towards Western contexts.
%
Although we provided the model with high-quality ground-truth information from the UNESCO ICH project and trained expert annotators with diverse cultural backgrounds to filter low-quality VQA samples, certain questions or their answers might still be offensive to people with certain cultural origins.
%
Since this is subjective, we need to accept it as is for now.
%
Nevertheless, we encourage contacting us if any offensive or otherwise harmful sample raises someone's attention.
%
%
\section*{Acknowledgements}
\label{sec:ack}
%
We thank our annotators for the \sivqa and \vvqa tasks with special thanks to Timm Dill, Narges Baba Ahmadi, Niloufar Baba Ahmadi, and Abdullah Abdelhafez for their extra efforts.
The work of Carolin Holtermann and Anne Lauscher is funded by the Excellence Strategy of the German Federal Government and the Federal States.

%


\clearpage
\bibliography{custom}
%
\newpage
\appendix
%
\newpage
\onecolumn
\section{\dsname Benchmark Details}
\label{appendix:sec:benchmark}
%

\subsection{Data License}
\label{appedix:sec:benchmark:license}
%
\dsname is built upon the open-access data from the UNESCO Intangible Cultural Heritage (ICH) project, which is organized as a knowledge graph.
%
The graph can be downloaded in English, French, and Spanish on the ICH project website: \href{https://ich.unesco.org/en/open-access-to-dive-data-01218}{https://ich.unesco.org/en/open-access-to-dive-data-01218}, with details about its structure and subsets also provided.
%
In \dsname, we work with the English graph only.
%
The open-access license of the knowledge graph is defined on the UNESCO website\footnote{\href{https://www.unesco.org/en/open-access}{https://www.unesco.org/en/open-access}} as follows:
%
\begin{displayquote}
\textit{By 'open access' to the literature, we mean its free availability on the public internet, permitting any users to read, download, copy, distribute, print, search, or link to the full texts of these articles, crawl them for indexing, pass them as data to software, or use them for any other lawful purpose, without financial, legal, or technical barriers other than those inseparable from gaining access to the internet itself.}
\end{displayquote}

\noindent
The images and videos within the data are shared via URLs and hosted by UNESCO or on YouTube, respectively.
%
Further, each image and video node in the knowledge graph has individual copyright information attached.
%
However, the licenses themselves are not discussed, and merely the name of the photographer or institution or UNESCO itself is stated.
%
Unfortunately, we did not receive an answer to multiple emails in which we asked for clarification.
%
Hence, we assume that the image and video content also fall under the definition of ''open access''.
%
If you are a copyright holder of any of the images or videos and do not want your intellectual property to be used or shared by us, please reach out via email: \href{mailto:florian.schneider-1@uni-hamburg.de}{florian.schneider-1@uni-hamburg.de}.
%


\subsection{Cultural Event or Facets (CEFs)}
\label{appendix:sec:benchmark:cef}
%
\subsubsection{Examples}
\label{appendix:sec:benchmark:cef:examples}
%
In the following, we provide one example of CEFs per region from the UNESCO ICH project.
%
We also use the same information for the \ckqan and \ckqad tasks. 
%
\subsubsection*{Western Europe (\RegW)}
%
\begin{figure}[H]
\begin{tcolorbox}[colback=gray!5!white,colframe=black!75!black,fonttitle=\bfseries\scriptsize,fontupper=\ttfamily\footnotesize,segmentation style={solid, black!30}]
  \begin{center}
    \begin{minipage}{0.18\linewidth}
      \centering
      \includegraphics[width=\linewidth]{examples/gfx/coqa_3ec41e9f_frame0_western-european-and-north-american-states_0.png}
      {\captionsetup{labelformat=empty}\captionof{figure}{\tiny\textit{Copyrigth: Servicio de Patrimonio Histórico de la Región de Murcia, 2005}}}
    \end{minipage}\hfill
    \begin{minipage}{0.18\linewidth}
      \centering
      \includegraphics[width=\linewidth]{examples/gfx/coqa_3ec41e9f_frame1_western-european-and-north-american-states_0.png}
      {\captionsetup{labelformat=empty}\captionof{figure}{\tiny\textit{Copyrigth: Generalitat Valenciana, 2005}}}
    \end{minipage}\hfill
    \begin{minipage}{0.18\linewidth}
      \centering
      \includegraphics[width=\linewidth]{examples/gfx/coqa_3ec41e9f_frame2_western-european-and-north-american-states_0.png}
      {\captionsetup{labelformat=empty}\captionof{figure}{\tiny\textit{Copyrigth: Servicio de Patrimonio Histórico de la Región de Murcia, 2005}}}
    \end{minipage}\hfill
    \begin{minipage}{0.18\linewidth}
      \centering
      \includegraphics[width=\linewidth]{examples/gfx/coqa_3ec41e9f_frame3_western-european-and-north-american-states_0.png}
      {\captionsetup{labelformat=empty}\captionof{figure}{\tiny\textit{Copyrigth: Generalitat Valenciana, 2005}}}
    \end{minipage}\hfill
    \begin{minipage}{0.18\linewidth}
      \centering
      \includegraphics[width=\linewidth]{examples/gfx/coqa_3ec41e9f_frame4_western-european-and-north-american-states_0.png}
      {\captionsetup{labelformat=empty}\captionof{figure}{\tiny\textit{Copyrigth: Servicio de Patrimonio Histórico de la Región de Murcia, 2005}}}
    \end{minipage}\hfill
  \\[4mm]
    \begin{minipage}{0.18\linewidth}
      \centering
      \includegraphics[width=\linewidth]{examples/gfx/coqa_3ec41e9f_frame5_western-european-and-north-american-states_0.png}
      {\captionsetup{labelformat=empty}\captionof{figure}{\tiny\textit{Copyrigth: Servicio de Patrimonio Histórico de la Región de Murcia, 2005}}}
    \end{minipage}\hfill
    \begin{minipage}{0.18\linewidth}
      \centering
      \includegraphics[width=\linewidth]{examples/gfx/coqa_3ec41e9f_frame6_western-european-and-north-american-states_0.png}
      {\captionsetup{labelformat=empty}\captionof{figure}{\tiny\textit{Copyrigth: Servicio de Patrimonio Histórico de la Región de Murcia, 2005}}}
    \end{minipage}\hfill
    \begin{minipage}{0.18\linewidth}
      \centering
      \includegraphics[width=\linewidth]{examples/gfx/coqa_3ec41e9f_frame7_western-european-and-north-american-states_0.png}
      {\captionsetup{labelformat=empty}\captionof{figure}{\tiny\textit{Copyrigth: Servicio de Patrimonio Histórico de la Región de Murcia, 2005}}}
    \end{minipage}\hfill
    \begin{minipage}{0.18\linewidth}
      \centering
      \includegraphics[width=\linewidth]{examples/gfx/coqa_3ec41e9f_frame8_western-european-and-north-american-states_0.png}
      {\captionsetup{labelformat=empty}\captionof{figure}{\tiny\textit{Copyrigth: Servicio de Patrimonio Histórico de la Región de Murcia, 2005}}}
    \end{minipage}\hfill
    \begin{minipage}{0.18\linewidth}
      \centering
      \includegraphics[width=\linewidth]{examples/gfx/coqa_3ec41e9f_frame9_western-european-and-north-american-states_0.png}
      {\captionsetup{labelformat=empty}\captionof{figure}{\tiny\textit{Copyrigth: Servicio de Patrimonio Histórico de la Región de Murcia, 2005}}}
    \end{minipage}\hfill
  \end{center}

  {\Large{Question:}} {\large{In which of the following countries does the event shown in the images take place? Choose from the following options and output only the corresponding letter.

A. Austria

B. Spain

C. Cyprus

D. United Kingdom of Great Britain and Northern Ireland


Your answer letter:}}\\
  {\Large{Answer:}} {\large{B}}\\
   \tcbline
  {\Large{Related Cultural Event or Facet}}\\[4mm]
  {\normalsize{Title:}} {\normalsize{Irrigators’ tribunals of the Spanish Mediterranean coast: the Council of Wise Men of the plain of Murcia and the Water Tribunal of the plain of Valencia}}\\
  {\normalsize{Countries:}} Spain\\
  {\normalsize{Regions:}} Western European and North American States\\
  {\normalsize{Description:}}\\
  The irrigators’ tribunals of the Spanish Mediterranean coast are traditional law courts for water management that date back to the al-Andalus period (ninth to thirteenth centuries). The two main tribunals – the Council of Wise Men of the Plain of Murcia and the Water Tribunal of the Plain of Valencia – are recognized under Spanish law. Inspiring authority and respect, these two courts, whose members are elected democratically, settle disputes orally in a swift, transparent and impartial manner. The Council of Wise Men has seven geographically representative members, and has jurisdiction over a landowners’ assembly of 23,313 members. The Water Tribunal comprises eight elected administrators representing a total of 11,691 members from nine communities. In addition to their legal role the irrigators’ tribunals play a key part in the communities of which they are a visible symbol, as apparent from the rites performed when judgments are handed down and the fact that the tribunals often feature in local iconography. They provide cohesion among traditional communities and synergy between occupations (wardens, inspectors, pruners, etc.), contribute to the oral transmission of knowledge derived from centuries-old cultural exchanges, and have their own specialist vocabulary peppered with Arabic borrowings. In short, the courts are long-standing repositories of local and regional identity and are of special significance to local inhabitants.\\[2mm]
  {\normalsize{UNESCO ICH URL:}} \href{https://ich.unesco.org/en/RL/irrigators-tribunals-of-the-spanish-mediterranean-coast-the-council-of-wise-men-of-the-plain-of-murcia-and-the-water-tribunal-of-the-plain-of-valencia-00171}{https://ich.unesco.org/en/RL/irrigators-tribunals-of-the-spa...}
\end{tcolorbox}
\end{figure}
%
\subsubsection*{Eastern Europe (\RegE)}
%
\begin{figure}[H]
\begin{tcolorbox}[colback=gray!5!white,colframe=black!75!black,fonttitle=\bfseries\scriptsize,fontupper=\ttfamily\footnotesize,segmentation style={solid, black!30}]
  \begin{center}
    \begin{minipage}{0.18\linewidth}
      \centering
      \includegraphics[width=\linewidth]{examples/gfx/coqa_849b1734_frame0_eastern-european-states_0.png}
      {\captionsetup{labelformat=empty}\captionof{figure}{\tiny\textit{Copyrigth: Lithuanian National Culture Centre, Archive, 2021}}}
    \end{minipage}\hfill
    \begin{minipage}{0.18\linewidth}
      \centering
      \includegraphics[width=\linewidth]{examples/gfx/coqa_849b1734_frame1_eastern-european-states_0.png}
      {\captionsetup{labelformat=empty}\captionof{figure}{\tiny\textit{Copyrigth: Vilnius Ethnic Culture Centre, Archive, 2021}}}
    \end{minipage}\hfill
    \begin{minipage}{0.18\linewidth}
      \centering
      \includegraphics[width=\linewidth]{examples/gfx/coqa_849b1734_frame2_eastern-european-states_0.png}
      {\captionsetup{labelformat=empty}\captionof{figure}{\tiny\textit{Copyrigth: Vilnius Ethnic Culture Centre, Archive, 2021}}}
    \end{minipage}\hfill
    \begin{minipage}{0.18\linewidth}
      \centering
      \includegraphics[width=\linewidth]{examples/gfx/coqa_849b1734_frame3_eastern-european-states_0.png}
      {\captionsetup{labelformat=empty}\captionof{figure}{\tiny\textit{Copyrigth: Lithuanian National Culture Centre, Archive, 2021}}}
    \end{minipage}\hfill
    \begin{minipage}{0.18\linewidth}
      \centering
      \includegraphics[width=\linewidth]{examples/gfx/coqa_849b1734_frame4_eastern-european-states_0.png}
      {\captionsetup{labelformat=empty}\captionof{figure}{\tiny\textit{Copyrigth: Vilnius Ethnic Culture Centre, Archive, 2021}}}
    \end{minipage}\hfill
  \\[4mm]
    \begin{minipage}{0.18\linewidth}
      \centering
      \includegraphics[width=\linewidth]{examples/gfx/coqa_849b1734_frame5_eastern-european-states_0.png}
      {\captionsetup{labelformat=empty}\captionof{figure}{\tiny\textit{Copyrigth: Vilnius Ethnic Culture Centre, Archive, 2021}}}
    \end{minipage}\hfill
    \begin{minipage}{0.18\linewidth}
      \centering
      \includegraphics[width=\linewidth]{examples/gfx/coqa_849b1734_frame6_eastern-european-states_0.png}
      {\captionsetup{labelformat=empty}\captionof{figure}{\tiny\textit{Copyrigth: Vilnius Ethnic Culture Centre, Archive, 2021}}}
    \end{minipage}\hfill
    \begin{minipage}{0.18\linewidth}
      \centering
      \includegraphics[width=\linewidth]{examples/gfx/coqa_849b1734_frame7_eastern-european-states_0.png}
      {\captionsetup{labelformat=empty}\captionof{figure}{\tiny\textit{Copyrigth: Lithuanian National Culture Centre, Archive, 2021}}}
    \end{minipage}\hfill
    \begin{minipage}{0.18\linewidth}
      \centering
      \includegraphics[width=\linewidth]{examples/gfx/coqa_849b1734_frame8_eastern-european-states_0.png}
      {\captionsetup{labelformat=empty}\captionof{figure}{\tiny\textit{Copyrigth: Marija Liugienė, Archive, 2003}}}
    \end{minipage}\hfill
    \begin{minipage}{0.18\linewidth}
      \centering
      \includegraphics[width=\linewidth]{examples/gfx/coqa_849b1734_frame9_eastern-european-states_0.png}
      {\captionsetup{labelformat=empty}\captionof{figure}{\tiny\textit{Copyrigth: Lithuanian National Culture Centre, Archive, 2021}}}
    \end{minipage}\hfill
  \end{center}

  {\Large{Question:}} {\large{In which of the following countries does the event shown in the images take place? Choose from the following options and output only the corresponding letter.

A. Lithuania

B. Bosnia and Herzegovina

C. Russia

D. Poland


Your answer letter:}}\\
  {\Large{Answer:}} {\large{A}}\\
   \tcbline
  {\Large{Related Cultural Event or Facet}}\\[4mm]
  {\normalsize{Title:}} {\normalsize{Sodai straw garden making in Lithuania}}\\
  {\normalsize{Countries:}} Lithuania\\
  {\normalsize{Regions:}} Eastern European States\\
  {\normalsize{Description:}}\\
  Sodai straw gardens are hanging ornaments made from the stalks of grains. This practice involves the cultivation of grain (typically rye), the treatment of straw and the creation of geometric structures of varying sizes. The structures are then decorated with details symbolizing fertility and prosperity. Sodai gardens are believed to reflect the pattern of the universe and are associated with well-being and spirituality. They are hung over the cradles of babies and over a wedding or family table to wish happiness to newborns, fertility to newlyweds or harmony to the family. Lithuanian homes are also frequently decorated with sodai gardens for Easter and Christmas. Some sodai-making families have been practising the tradition for generations. Although most of the practitioners are women, workshops exist and are open to people of all ages and genders. The practice is passed on informally within families or during events such as festivals, exhibitions, conferences and summer camps. An integral part of traditional wooden home interiors, sodai gardens are viewed as spiritual gifts. They provide a sense of shared cultural heritage and continuity to the practising communities while strengthening communal partnerships, intergenerational bonds and cultural diversity.\\[2mm]
  {\normalsize{UNESCO ICH URL:}} \href{https://ich.unesco.org/en/RL/sodai-straw-garden-making-in-lithuania-01987}{https://ich.unesco.org/en/RL/sodai-straw-garden-making-in-li...}
\end{tcolorbox}
\end{figure}
%
\subsubsection*{Arab (\RegA)}
%
\begin{figure}[H]
\begin{tcolorbox}[colback=gray!5!white,colframe=black!75!black,fonttitle=\bfseries\scriptsize,fontupper=\ttfamily\footnotesize,segmentation style={solid, black!30}]
  \begin{center}
    \begin{minipage}{0.18\linewidth}
      \centering
      \includegraphics[width=\linewidth]{examples/gfx/vvqa_22a1bb26_frame0_arab-states_0.png}
    \end{minipage}\hfill
    \begin{minipage}{0.18\linewidth}
      \centering
      \includegraphics[width=\linewidth]{examples/gfx/vvqa_22a1bb26_frame1_arab-states_0.png}
    \end{minipage}\hfill
    \begin{minipage}{0.18\linewidth}
      \centering
      \includegraphics[width=\linewidth]{examples/gfx/vvqa_22a1bb26_frame2_arab-states_0.png}
    \end{minipage}\hfill
    \begin{minipage}{0.18\linewidth}
      \centering
      \includegraphics[width=\linewidth]{examples/gfx/vvqa_22a1bb26_frame3_arab-states_0.png}
    \end{minipage}\hfill
    \begin{minipage}{0.18\linewidth}
      \centering
      \includegraphics[width=\linewidth]{examples/gfx/vvqa_22a1bb26_frame4_arab-states_0.png}
    \end{minipage}\hfill
  \\[4mm]
    \begin{minipage}{0.18\linewidth}
      \centering
      \includegraphics[width=\linewidth]{examples/gfx/vvqa_22a1bb26_frame5_arab-states_0.png}
    \end{minipage}\hfill
    \begin{minipage}{0.18\linewidth}
      \centering
      \includegraphics[width=\linewidth]{examples/gfx/vvqa_22a1bb26_frame6_arab-states_0.png}
    \end{minipage}\hfill
    \begin{minipage}{0.18\linewidth}
      \centering
      \includegraphics[width=\linewidth]{examples/gfx/vvqa_22a1bb26_frame7_arab-states_0.png}
    \end{minipage}\hfill
    \begin{minipage}{0.18\linewidth}
      \centering
      \includegraphics[width=\linewidth]{examples/gfx/vvqa_22a1bb26_frame8_arab-states_0.png}
    \end{minipage}\hfill
    \begin{minipage}{0.18\linewidth}
      \centering
      \includegraphics[width=\linewidth]{examples/gfx/vvqa_22a1bb26_frame9_arab-states_0.png}
    \end{minipage}\hfill
  \end{center}

  {\Large{Question:}} {\large{What event are the women in the video participating in?}}\\
  {\Large{Answer:}} {\large{Moussem of Tan-Tan}}\\
   \tcbline
  {\Large{Related Cultural Event or Facet}}\\[4mm]
  {\normalsize{Title:}} {\normalsize{Moussem of Tan-Tan}}\\
  {\normalsize{Countries:}} Morocco\\
  {\normalsize{Regions:}} Arab States\\
  {\normalsize{Description:}}\\
  The Moussem of Tan-Tan in southwest Morocco is an annual gathering of nomadic peoples of the Sahara that brings together more than thirty tribes from southern Morocco and other parts of northwest Africa. Originally this was an annual event around the month of May. Part of the agricultural and herding calendar of the nomads, these gatherings were an opportunity to group together, buy, sell and exchange foodstuffs and other products, organize camel and horse-breeding competitions, celebrate weddings and consult herbalists. The Moussem also included a range of cultural expressions such as musical performances, popular chanting, games, poetry contests and other Hassanie oral traditions. 

These gatherings took the form of a Moussem (a type of annual fair with economic, cultural and social functions) in 1963 when the first Moussem of Tan-Tan was organized to promote local traditions and provide a place for exchange, meeting and celebration. The Moussem is said to have been initially associated with Mohamed Laghdaf, who resisted the Franco-Spanish occupation. He died in 1960, and his tomb lies near the town. However, between 1979 and 2004 it was not possible to hold the Moussem because of security problems in the region. 

Today, the nomadic populations are particularly concerned to protect their way of life. Economic and technical upheavals in the region have profoundly altered the lifestyle of the nomadic Bedouin communities, forcing many of them to settle. Moreover, urbanization and rural exodus have contributed to the loss of many aspects of the traditional culture of these populations, such as crafts and poetry. Because of these risks, Bedouin communities rely strongly on the renewed Moussem of Tan-Tan to assist them in ensuring the survival of their know-how and traditions.\\[2mm]
  {\normalsize{UNESCO ICH URL:}} \href{https://ich.unesco.org/en/RL/moussem-of-tan-tan-00168}{https://ich.unesco.org/en/RL/moussem-of-tan-tan-00168...}
\end{tcolorbox}
\end{figure}
%
\subsubsection*{Asia and Pacific (\RegAP)}
%
\begin{figure}[H]
\begin{tcolorbox}[colback=gray!5!white,colframe=black!75!black,fonttitle=\bfseries\scriptsize,fontupper=\ttfamily\footnotesize,segmentation style={solid, black!30}]
  \begin{center}
    \begin{minipage}{0.5\linewidth}
      \centering
      \includegraphics[width=\linewidth]{examples/gfx/sivqa_ad39cda3_asian-and-pacific-states_0.png}
      {\captionsetup{labelformat=empty}\captionof{figure}{\tiny\textit{Copyrigth: 2010 by Centre for Research and Development of Culture, Indonesia}}}
    \end{minipage}\hfill
  \end{center}
  {\Large{Question:}} {\large{What traditional dance are the performers engaging in, as seen in the image?}}\\
  {\Large{Answer:}} {\large{Saman dance}}\\
   \tcbline
  {\Large{Related Cultural Event or Facet}}\\[4mm]
  {\normalsize{Title:}} {\normalsize{Saman dance}}\\
  {\normalsize{Countries:}} Indonesia\\
  {\normalsize{Regions:}} Asian and Pacific States\\
  {\normalsize{Description:}}\\
  The Saman dance is part of the cultural heritage of the Gayo people of Aceh province in Sumatra. Boys and young men perform the Saman sitting on their heels or kneeling in tight rows. Each wears a black costume embroidered with colourful Gayo motifs symbolizing nature and noble values. The leader sits in the middle of the row and leads the singing of verses, mostly in the Gayo language. These offer guidance and can be religious, romantic or humorous in tone. Dancers clap their hands, slap their chests, thighs and the ground, click their fingers, and sway and twist their bodies and heads in time with the shifting rhythm – in unison or alternating with the moves of opposing dancers. These movements symbolize the daily lives of the Gayo people and their natural environment. The Saman is performed to celebrate national and religious holidays, cementing relationships between village groups who invite each other for performances. The frequency of Saman performances and its transmission are decreasing, however. Many leaders with knowledge of the Saman are now elderly and without successors. Other forms of entertainment and new games are replacing informal transmission, and many young people now emigrate to further their education. Lack of funds is also a constraint, as Saman costumes and performances involve considerable expense.\\[2mm]
  {\normalsize{UNESCO ICH URL:}} \href{https://ich.unesco.org/en/USL/saman-dance-00509}{https://ich.unesco.org/en/USL/saman-dance-00509...}
\end{tcolorbox}
\end{figure}
%
\subsubsection*{Latin America \& Caribbean (\RegLAC)}
%
\begin{figure}[H]
\begin{tcolorbox}[colback=gray!5!white,colframe=black!75!black,fonttitle=\bfseries\scriptsize,fontupper=\ttfamily\footnotesize]
  {\large{Title:}} {\normalsize{Ancestral system of knowledge of the four indigenous peoples, Arhuaco, Kankuamo, Kogui and Wiwa of the Sierra Nevada de Santa Marta}}\\
  {\normalsize{Countries:}} Colombia\\
  {\normalsize{Regions:}} Latin-American and Caribbean States\\
  {\normalsize{Description:}}\\
  The Ancestral System of Knowledge of the Arhuaco, Kankuamo, Kogui and Wiwa peoples of the Sierra Nevada de Santa Marta is comprised of sacred mandates that keep the existence of the four peoples in harmony with the physical and spiritual universe. Through many years of dedication, the knowledgeable men (Mamos) and women (Sagas) acquire the necessary skills and sensitivity to communicate with the snow-capped peaks, connect with the knowledge of the rivers and decipher the messages of nature. Based on the Law of Origin, a philosophy that governs human relationships to nature and the universe, the Ancestral System of Knowledge entails caring for sacred sites and partaking in baptism rituals, marriage rites, traditional dances and songs, and retributions or offerings to spiritual powers. This ancestral wisdom is believed to play a fundamental role in protecting the Sierra Nevada ecosystem and avoiding the loss of the cultural identity of the four peoples of the region. The Ancestral System of Knowledge is transmitted from generation to generation through cultural practice, community activities, the use of the indigenous language and the implementation of the sacred mandates. The transmission process includes the understanding of physical and spiritual relationships with Mother Nature and sacred sites.\\[2mm]
  {\normalsize{UNESCO ICH URL:}} \href{https://ich.unesco.org/en/RL/ancestral-system-of-knowledge-of-the-four-indigenous-peoples-arhuaco-kankuamo-kogui-and-wiwa-of-the-sierra-nevada-de-santa-marta-01886}{https://ich.unesco.org/en/RL/ancestral-system-of-knowledge-o...}
  \begin{center}
    \begin{minipage}{0.18\linewidth}
      \centering
      \includegraphics[width=\linewidth]{examples/gfx/cef_latin-american-and-caribbean-states_0_4e60556d_0.jpg}
      {\captionsetup{labelformat=empty}\captionof{figure}{\tiny\textit{Copyrigth: William Diaz, 2021}}}
    \end{minipage}\hfill
    \begin{minipage}{0.18\linewidth}
      \centering
      \includegraphics[width=\linewidth]{examples/gfx/cef_latin-american-and-caribbean-states_0_4e60556d_1.jpg}
      {\captionsetup{labelformat=empty}\captionof{figure}{\tiny\textit{Copyrigth: Jorge Mario Suarez/Government of Magdalena, 2017}}}
    \end{minipage}\hfill
    \begin{minipage}{0.18\linewidth}
      \centering
      \includegraphics[width=\linewidth]{examples/gfx/cef_latin-american-and-caribbean-states_0_4e60556d_2.jpg}
      {\captionsetup{labelformat=empty}\captionof{figure}{\tiny\textit{Copyrigth: Jorge Mario Suarez/Government of Magdalena, 2017}}}
    \end{minipage}\hfill
    \begin{minipage}{0.18\linewidth}
      \centering
      \includegraphics[width=\linewidth]{examples/gfx/cef_latin-american-and-caribbean-states_0_4e60556d_3.jpg}
      {\captionsetup{labelformat=empty}\captionof{figure}{\tiny\textit{Copyrigth: William Diaz, 2021}}}
    \end{minipage}\hfill
    \begin{minipage}{0.18\linewidth}
      \centering
      \includegraphics[width=\linewidth]{examples/gfx/cef_latin-american-and-caribbean-states_0_4e60556d_4.jpg}
      {\captionsetup{labelformat=empty}\captionof{figure}{\tiny\textit{Copyrigth: Jorge Mario Suarez/Government of Magdalena, 2017}}}
    \end{minipage}\hfill
  \\[4mm]
    \begin{minipage}{0.18\linewidth}
      \centering
      \includegraphics[width=\linewidth]{examples/gfx/cef_latin-american-and-caribbean-states_0_4e60556d_5.jpg}
      {\captionsetup{labelformat=empty}\captionof{figure}{\tiny\textit{Copyrigth: Jorge Mario Suarez/Government of Magdalena, 2017}}}
    \end{minipage}\hfill
    \begin{minipage}{0.18\linewidth}
      \centering
      \includegraphics[width=\linewidth]{examples/gfx/cef_latin-american-and-caribbean-states_0_4e60556d_6.jpg}
      {\captionsetup{labelformat=empty}\captionof{figure}{\tiny\textit{Copyrigth: Jorge Mario Suarez/Government of Magdalena, 2017}}}
    \end{minipage}\hfill
    \begin{minipage}{0.18\linewidth}
      \centering
      \includegraphics[width=\linewidth]{examples/gfx/cef_latin-american-and-caribbean-states_0_4e60556d_7.jpg}
      {\captionsetup{labelformat=empty}\captionof{figure}{\tiny\textit{Copyrigth: Jorge Mario Suarez/Government of Magdalena, 2017}}}
    \end{minipage}\hfill
    \begin{minipage}{0.18\linewidth}
      \centering
      \includegraphics[width=\linewidth]{examples/gfx/cef_latin-american-and-caribbean-states_0_4e60556d_8.jpg}
      {\captionsetup{labelformat=empty}\captionof{figure}{\tiny\textit{Copyrigth: Jorge Mario Suarez/Government of Magdalena, 2017}}}
    \end{minipage}\hfill
    \begin{minipage}{0.18\linewidth}
      \centering
      \includegraphics[width=\linewidth]{examples/gfx/cef_latin-american-and-caribbean-states_0_4e60556d_9.jpg}
      {\captionsetup{labelformat=empty}\captionof{figure}{\tiny\textit{Copyrigth: William Diaz, 2021}}}
    \end{minipage}\hfill
  \end{center}
\end{tcolorbox}
\end{figure}
%
\subsubsection*{Subsaharian Africa (\RegSA)}
%
\begin{figure}[H]
\begin{tcolorbox}[colback=gray!5!white,colframe=black!75!black,fonttitle=\bfseries\scriptsize,fontupper=\ttfamily\footnotesize,segmentation style={solid, black!30}]
  \begin{center}
    \begin{minipage}{0.5\linewidth}
      \centering
      \includegraphics[width=\linewidth]{examples/gfx/sivqa_863c7b2f_subsaharian-african-states_0.png}
      {\captionsetup{labelformat=empty}\captionof{figure}{\tiny\textit{Copyrigth: The Authority for Research and Conservation of Cultural Heritage (ARCCH), 2013}}}
    \end{minipage}\hfill
  \end{center}
  {\Large{Question:}} {\large{What festival are the people in the image celebrating?}}\\
  {\Large{Answer:}} {\large{Fichee-Chambalaalla}}\\
   \tcbline
  {\Large{Related Cultural Event or Facet}}\\[4mm]
  {\normalsize{Title:}} {\normalsize{Fichee-Chambalaalla, New Year festival of the Sidama people}}\\
  {\normalsize{Countries:}} Ethiopia\\
  {\normalsize{Regions:}} Subsaharian African States\\
  {\normalsize{Description:}}\\
  Fichee-Chambalaalla is a New Year festival celebrated among the Sidama people. According to the oral tradition, Fichee commemorates a Sidama woman who visited her parents and relatives once a year after her marriage, bringing ''buurisame'', a meal prepared from false banana, milk and butter, which was shared with neighbours. Fichee has since become a unifying symbol of the Sidama people. Each year, astrologers determine the correct date for the festival, which is then announced to the clans. Communal events take place throughout the festival, including traditional songs and dances. Every member participates irrespective of age, gender and social status. On the first day, children go from house to house to greet their neighbours, who serve them ''buurisame''. During the festival, clan leaders advise the Sidama people to work hard, respect and support the elders, and abstain from cutting down indigenous trees, begging, indolence, false testimony and theft. The festival therefore enhances equity, good governance, social cohesion, peaceful co-existence and integration among Sidama clans and the diverse ethnic groups in Ethiopia. Parents transmit the tradition to their children orally and through participation in events during the celebration. Women in particular, transfer knowledge and skills associated with hairdressing and preparation of ''buurisame'' to their daughters and other girls in their respective villages.\\[2mm]
  {\normalsize{UNESCO ICH URL:}} \href{https://ich.unesco.org/en/RL/fichee-chambalaalla-new-year-festival-of-the-sidama-people-01054}{https://ich.unesco.org/en/RL/fichee-chambalaalla-new-year-fe...}
\end{tcolorbox}
\end{figure}
%





%
\subsubsection{CEFs as Python a \texttt{dataclass}}
%
Listing~\ref{listing:benchmark:cef} presents a CEF implemented as a Python dataclass.

\begin{listing}[ht!]
\begin{minted}[fontsize=\footnotesize]{python}
from dataclasses import dataclass

@dataclass
class CEF:
    title: str
    description: str
    countries: list[str]
    regions: list[str]
    images: list[str]  # URLs
    videos: list[str]  # URLs
\end{minted}
\caption{Python pseudo-code for a dataclass representing a CEF.}
\label{listing:benchmark:cef}
\end{listing}
%



\subsection{Regions}
\label{appendix:sec:benchmark:regions}
%
\begin{table}[ht!]
    \centering
  \renewcommand{\arraystretch}{.97}
    \resizebox{\textwidth}{!}{%
    \begin{tabular}{llr p{10cm}}
    \toprule
    Region & Abbrv. & Countries & Countries \\
    \midrule
    Arab & \RegA & 18 & Algeria, Bahrain, Egypt, Iraq, Jordan, Kuwait, Lebanon, Mauritania, Morocco, Oman, Palestine, Qatar, Saudi Arabia, Sudan, Syria, Tunisia, United Arab Emirates, Yemen \\
    Asia \& Pacific & \RegAP & 35 & Lao People's Democratic Republic, Afghanistan, Australia, Bangladesh, Bhutan, Cambodia, China, Cook Islands, Democratic People’s Republic of Korea, Fiji, India, Indonesia, Iran, Japan, Kazakhstan, Korea, Kyrgyzstan, Malaysia, Micronesia, Mongolia, Myanmar, Nepal, New Zealand, Pakistan, Papua New Guinea, Philippines, Samoa, Singapore, Sri Lanka, Thailand, Timor-Leste, Tonga, Turkmenistan, Vanuatu, Vietnam \\
    Eastern Europe & \RegE & 25 & Albania, Armenia, Azerbaijan, Belarus, Bosnia and Herzegovina, Bulgaria, Croatia, Czechia, Estonia, Georgia, Hungary, Latvia, Lithuania, Moldova, Montenegro, North Macedonia, Poland, Romania, Russia, Serbia, Slovakia, Slovenia, Tajikistan, Ukraine, Uzbekistan \\
    Latin-America \& Caribbean & \RegLAC & 28 & Antigua and Barbuda, Argentina, Bahamas, Belize, Bolivia, Brazil, Chile, Colombia, Costa Rica, Cuba, Curaçao, Dominican Republic, Ecuador, El Salvador, Grenada, Guatemala, Haiti, Honduras, Jamaica, Mexico, Nicaragua, Panama, Paraguay, Peru, Saint Kitts and Nevis, Saint Vincent and the Grenadines, Uruguay, Venezuela \\
    Subsaharian Africa & \RegSA & 40 & Côte d'Ivoire, Angola, Benin, Botswana, Burkina Faso, Burundi, Cabo Verde, Cameroon, Central African Republic, Chad, Congo, Democratic Republic of the Congo, Djibouti, Eritrea, Eswatini, Ethiopia, Gabon, Gambia, Ghana, Guinea, Kenya, Lesotho, Madagascar, Malawi, Mali, Mauritius, Mozambique, Namibia, Niger, Nigeria, Rwanda, Senegal, Seychelles, Somalia, South Africa, South Sudan, Togo, Uganda, Zambia, Zimbabwe \\
    Western Europe \& North America & \RegW & 23 & Andorra, Austria, Belgium, Canada, Cyprus, Denmark, Finland, France, Germany, Greece, Iceland, Ireland, Italy, Luxembourg, Malta, Netherlands, Norway, Portugal, Spain, Sweden, Switzerland, Türkiye, United Kingdom of Great Britain and Northern Ireland \\
    \bottomrule
    \end{tabular}
    }%
    \caption{Caption}
    \label{tab:benchmark:regions_full}
\end{table}
%


\subsubsection{Number of Samples per Task per Region}
\label{appendix:sec:benchmark:samples}
%
\begin{table}[ht!]
    \centering
    \renewcommand{\arraystretch}{.94}
    \resizebox{\linewidth}{!}{%
    \begin{tabular}{lrrrrrr}
    \toprule
    \textsc{Region} & \sivqa & \vvqa & \coqar & \coqac & \ckqad & \ckqan \\
    \midrule
    \RegA & 375 & 296 & 71 & 127 & 71 & 71 \\
    \RegA \RegAP & 4 & 4 & 2 & 2 & 1 & 1 \\
    \RegA \RegAP \RegE \RegW & 5 & 5 & 0 & 36 & 2 & 2 \\
    \RegA \RegE \RegW & 1 & 0 & 3 & 7 & 1 & 1 \\
    \RegA \RegSA & 8 & 0 & 2 & 3 & 1 & 1 \\
    \RegAP & 444 & 407 & 211 & 222 & 211 & 211 \\
    \RegAP \RegE & 7 & 7 & 6 & 6 & 3 & 3 \\
    \RegAP \RegE \RegLAC \RegSA \RegW & 1 & 1 & 0 & 8 & 1 & 1 \\
    \RegAP \RegE \RegW & 10 & 7 & 21 & 35 & 7 & 7 \\
    \RegAP \RegW & 4 & 3 & 2 & 3 & 1 & 1 \\
    \RegE & 302 & 242 & 125 & 136 & 125 & 125 \\
    \RegE \RegW & 21 & 20 & 22 & 56 & 11 & 11 \\
    \RegLAC & 420 & 341 & 96 & 106 & 96 & 96 \\
    \RegLAC \RegW & 2 & 2 & 2 & 2 & 1 & 1 \\
    \RegSA & 388 & 299 & 71 & 80 & 71 & 71 \\
    \RegW & 241 & 175 & 125 & 153 & 125 & 125 \\
    \bottomrule
    \end{tabular}
    }%
    \caption{Number of samples per region(s) in \dsname tasks.}
    \label{tab:benchmark:datasets:samples}
\end{table}
%

\subsection{Models}
\label{appendix:sec:benchmark:models}
%
We present the comprehensive list of all 31 models evaluated in \dsname in Table~\ref{tab:benchmark:models}.
%
%
\begin{table}[t]
	\centering
	\renewcommand{\arraystretch}{.97}
	\resizebox{\textwidth}{!}{%
		\begin{tabular}{lllllll l}
			\toprule
			\textsc{Model ID}                                 & \textsc{Paper Name}  & \textsc{Open-Weight} & \textsc{Size Group} & \textsc{Image Input} & \textsc{Video Input} & \textsc{Text Input} & \textsc{LLM Backbone}                     \\
			\midrule
			\rowcolor{gray!20}
			\texttt{claude-3-5-sonnet-20241022}               & \makecell{Claude 3.5 Sonnet\\\cite{anthropic2024claude}} & No                   & A                   & Yes                  & Yes                  & Yes                 & \multicolumn{1}{c}{--}                    \\
			\texttt{gemini-1.5-pro-002}                       & \makecell{Gemini Pro\\\cite{team2024gemini1.5}} & No                   & A                   & Yes                  & Yes                  & Yes                 & \multicolumn{1}{c}{--}                    \\
			\rowcolor{gray!20}
			\texttt{gemini-1.5-flash-002}                     & \makecell{Gemini Flash\\\cite{team2024gemini1.5}} & No                   & A                   & Yes                  & Yes                  & Yes                 & \multicolumn{1}{c}{--}                    \\
			\texttt{gpt-4o-2024-11-20}                        & \makecell{GPT-4o\\\cite{hurst2024gpt4o}} & No                   & A                   & Yes                  & Yes                  & Yes                 & \multicolumn{1}{c}{--}                    \\
			\rowcolor{gray!20}
			\texttt{gpt-4o-mini-2024-07-18}                   & \makecell{GPT-4o Mini\\\cite{hurst2024gpt4o}} & No                   & A                   & Yes                  & Yes                  & Yes                 & \multicolumn{1}{c}{--}                    \\
			\midrule
			\texttt{opengvlab/internvl2\_5-78b}               & \makecell{InternVL2.5 78B\\\cite{chen2024internvl2_5}} & Yes                  & XL                  & Yes                  & Yes                  & Yes                 & \texttt{qwen/qwen2.5-72b-instruct}        \\
			\rowcolor{gray!20}
			\texttt{qwen/qwen2-vl-72b-instruct}               & \makecell{Qwen2 VL 72B\\\cite{wang2024qwen2vl}} & Yes                  & XL                  & Yes                  & Yes                  & Yes                 & \texttt{qwen/qwen2.5-72b-instruct}        \\
			\texttt{opengvlab/internvl2\_5-26b}               & \makecell{InternVL2.5 26B\\\cite{chen2024internvl2_5}} & Yes                  & L                   & Yes                  & Yes                  & Yes                 & \texttt{internlm/internlm2\_5-20b-chat}   \\
			\rowcolor{gray!20}
			\texttt{opengvlab/internvl2\_5-38b}               & \makecell{InternVL2.5 38B\\\cite{chen2024internvl2_5}} & Yes                  & L                   & Yes                  & Yes                  & Yes                 & \texttt{qwen/qwen2.5-32b-instruct}        \\
			\texttt{meta-llama/llama-3.2-11b-vision-instruct} & \makecell{Llama 3.2 11B Vision\\\cite{meta2024llama3_2_v}} & Yes                  & M                   & Yes                  & Yes                  & Yes                 & \multicolumn{1}{c}{--}                    \\
			\rowcolor{gray!20}
			\texttt{qwen/qwen2-vl-7b-instruct}                & \makecell{Qwen2 VL 7B\\\cite{wang2024qwen2vl}} & Yes                  & M                   & Yes                  & Yes                  & Yes                 & \texttt{qwen/qwen2.5-7b-instruct}         \\
			\texttt{openbmb/minicpm-v-2\_6}                   & \makecell{MiniCPM V 2.6\\\cite{yao2024minicpm}} & Yes                  & M                   & Yes                  & Yes                  & Yes                 & \multicolumn{1}{c}{--}                    \\
			\rowcolor{gray!20}
			\texttt{wuenlp/centurio\_aya}                     & \makecell{Centurio Aya\\\cite{geigle2025centurio}} & Yes                  & M                   & Yes                  & Yes                  & Yes                 & \texttt{cohereforai/aya-expanse-8b}       \\
			\texttt{opengvlab/internvl2\_5-8b}                & \makecell{InternVL2.5 8B\\\cite{chen2024internvl2_5}} & Yes                  & M                   & Yes                  & Yes                  & Yes                 & \texttt{internlm/internlm2\_5-7b-chat}    \\
			\rowcolor{gray!20}
			\texttt{wuenlp/centurio\_qwen}                    & \makecell{Centurio Qwen\\\cite{geigle2025centurio}} & Yes                  & M                   & Yes                  & Yes                  & Yes                 & \texttt{qwen/qwen2.5-7b-instruct}         \\
			\texttt{qwen/qwen2-vl-2b-instruct}                & \makecell{Qwen2 VL 2B\\\cite{wang2024qwen2vl}} & Yes                  & S                   & Yes                  & Yes                  & Yes                 & \texttt{qwen/qwen2.5-1.5b-instruct}       \\
			\rowcolor{gray!20}
			\texttt{microsoft/phi-3.5-vision-instruct}        & \makecell{Phi 3.5 Vision\\\cite{abdin2024phi3}} & Yes                  & S                   & Yes                  & Yes                  & Yes                 & \texttt{microsoft/phi-3.5-mini-instruct}  \\
			\texttt{opengvlab/internvl2\_5-4b}                & \makecell{InternVL2.5 4B\\\cite{chen2024internvl2_5}} & Yes                  & S                   & Yes                  & Yes                  & Yes                 & \texttt{qwen/qwen2.5-3b-instruct}         \\
			\rowcolor{gray!20}
			\texttt{opengvlab/internvl2\_5-1b}                & \makecell{InternVL2.5 1B\\\cite{chen2024internvl2_5}} & Yes                  & S                   & Yes                  & Yes                  & Yes                 & \texttt{qwen/qwen2.5-0.5b-instruct}       \\
			\texttt{opengvlab/internvl2\_5-2b}                & \makecell{InternVL2.5 2B\\\cite{chen2024internvl2_5}} & Yes                  & S                   & Yes                  & Yes                  & Yes                 & \texttt{internlm/internlm2\_5-1\_8b-chat} \\
			\midrule
			\rowcolor{gray!20}
			\texttt{qwen/qwen2.5-72b-instruct}                & \makecell{Qwen2.5 72B\\\cite{yang2024qwen2.5}} & Yes                  & XL                  & No                   & No                   & Yes                 & \multicolumn{1}{c}{--}                    \\
			\texttt{qwen/qwen2.5-32b-instruct}                & \makecell{Qwen2.5 32B\\\cite{yang2024qwen2.5}} & Yes                  & L                   & No                   & No                   & Yes                 & \multicolumn{1}{c}{--}                    \\
			\rowcolor{gray!20}
			\texttt{internlm/internlm2\_5-20b-chat}           & \makecell{InternLM2.5 20B\\\cite{cai2024internlm2}} & Yes                  & L                   & No                   & No                   & Yes                 & \multicolumn{1}{c}{--}                    \\
			\texttt{cohereforai/aya-expanse-8b}               & \makecell{Aya Expanse 8B\\\cite{dang2024ayaexpanse}} & Yes                  & M                   & No                   & No                   & Yes                 & \multicolumn{1}{c}{--}                    \\
			\rowcolor{gray!20}
			\texttt{internlm/internlm2\_5-7b-chat}            & \makecell{InternLM2.5 7B\\\cite{cai2024internlm2}} & Yes                  & M                   & No                   & No                   & Yes                 & \multicolumn{1}{c}{--}                    \\
			\texttt{qwen/qwen2.5-7b-instruct}                 & \makecell{Qwen2.5 7B\\\cite{yang2024qwen2.5}} & Yes                  & M                   & No                   & No                   & Yes                 & \multicolumn{1}{c}{--}                    \\
			\rowcolor{gray!20}
			\texttt{qwen/qwen2.5-0.5b-instruct}               & \makecell{Qwen2.5 0.5B\\\cite{yang2024qwen2.5}} & Yes                  & S                   & No                   & No                   & Yes                 & \multicolumn{1}{c}{--}                    \\
			\texttt{qwen/qwen2.5-3b-instruct}                 & \makecell{Qwen2.5 3B\\\cite{yang2024qwen2.5}} & Yes                  & S                   & No                   & No                   & Yes                 & \multicolumn{1}{c}{--}                    \\
			\rowcolor{gray!20}
			\texttt{qwen/qwen2.5-1.5b-instruct}               & \makecell{Qwen2.5 1.5B\\\cite{yang2024qwen2.5}} & Yes                  & S                   & No                   & No                   & Yes                 & \multicolumn{1}{c}{--}                    \\
			\texttt{internlm/internlm2\_5-1\_8b-chat}         & \makecell{InternLM2.5 1.8B\\\cite{cai2024internlm2}} & Yes                  & S                   & No                   & No                   & Yes                 & \multicolumn{1}{c}{--}                    \\
			\rowcolor{gray!20}
			\texttt{microsoft/phi-3.5-mini-instruct}          & \makecell{Phi 3.5 Mini\\\cite{abdin2024phi3}} & Yes                  & S                   & No                   & No                   & Yes                 & \multicolumn{1}{c}{--}                    \\
			\bottomrule
		\end{tabular}
	}%
	\caption{Details about the models evaluated within the \dsname benchmark. The size ``A'' indicates that the model is a proprietary API model with unknown size.}
	\label{tab:benchmark:models}
\end{table}
%
\section{\sivqa Details}
\label{appendix:sec:sivqa}
%

\clearpage
\subsection{Examples}
\label{appendix:sec:sivqa:examples}
%
In the following, we provide one random sample per region for the \sivqa task.
%
Note that the lower part of the examples, where the related CEF is provided, is \emph{not} part of the actual sample.

%
\subsubsection*{\RegA}
%
\input{examples/sivqa/arab-states_0}
%
\subsubsection*{\RegAP}
%
\input{examples/sivqa/asian-and-pacific-states_0}
%
\subsubsection*{\RegE}
%
\input{examples/sivqa/eastern-european-states_0}
%
\subsubsection*{\RegLAC}
%
\input{examples/sivqa/latin-american-and-caribbean-states_0}
%
\subsubsection*{\RegSA}
%
\input{examples/sivqa/subsaharian-african-states_0}
%
\subsubsection*{\RegW}
%
\input{examples/sivqa/western-european-and-north-american-states_0}
%

\subsection{Cultural Aspects}
\label{appedix:sec:sivqa:aspects}
%
During the synthetic data generation phase of the \sivqa, we also obtained a ``target aspect'' per question (see \S\ref{appendix:sec:sivqa:synth} and \S\ref{appendix:sec:sivqa:synth:sys_prompt}).
%
We report these aspects in the following.
%
\begin{table*}[ht!]
    \centering
    \small
    \begin{minipage}[t]{0.29\textwidth}
        \centering
        \begin{tabular}{lr}
            \toprule
            Aspect & Questions \\
            \midrule
            traditions & 390 \\
            rituals & 241 \\
            art & 233 \\
            music & 210 \\
            craftsmanship & 177 \\
            instruments & 155 \\
            festivals & 151 \\
            dance & 150 \\
            tools & 108 \\
            food & 96 \\
            clothing & 93 \\
            architecture & 52 \\
            sports & 38 \\
            location & 28 \\
            symbols & 19 \\
            drinks & 14 \\
            customs & 13 \\
            cultural significance & 6 \\
            theatre & 4 \\
            \bottomrule
        \end{tabular}
    \end{minipage}%
    \begin{minipage}[t]{0.29\textwidth}
        \centering
        \begin{tabular}{lr}
            \toprule
            Aspect & Questions \\
            \midrule
            education & 3 \\
            culture & 3 \\
            games & 3 \\
            performing arts & 3 \\
            language & 3 \\
            performance & 3 \\
            characters & 2 \\
            practices & 2 \\
            skills & 2 \\
            origin & 2 \\
            cultural identity & 2 \\
            technology & 1 \\
            people & 1 \\
            community & 1 \\
            identity & 1 \\
            environment & 1 \\
            traditional medicine & 1 \\
            nature & 1 \\
            communication & 1 \\
            \bottomrule
        \end{tabular}
    \end{minipage}%
    \begin{minipage}[t]{0.29\textwidth}
        \centering
        \begin{tabular}{lr}
            \toprule
            Aspect & Questions \\
            \midrule
            jewelry & 1 \\
            objects & 1 \\
            animal & 1 \\
            plants & 1 \\
            process & 1 \\
            agriculture & 1 \\
            celebrations & 1 \\
            details & 1 \\
            historical & 1 \\
            function or usage & 1 \\
            symbolism & 1 \\
            healthcare & 1 \\
            knowledge & 1 \\
            social status & 1 \\
            religion & 1 \\
            cultural space & 1 \\
            social space & 1 \\
            cultural practice & 1 \\
            unknown & 1 \\
            \bottomrule
        \end{tabular}
    \end{minipage}
    \caption{Cultural aspects targeted by the questions within the \sivqa task.}
    \label{tab:sivqa:aspects}
\end{table*}

%
\subsection{External Hint Variations}
%
\label{appendix:sec:sivqa:hints}
%
For the \sivqa (and \vvqa) task, we ablate the effect of external cues or hints on the task performance of models.
%
In the following, we provide the Python pseudo-code snippet to generate the prompt for a given sample.
%
\begin{figure*}[ht!]
    \centering
    %
    \begin{promptbox}{Python Pseudo-Code for the external cue settings of the \sivqa and \vvqa tasks.}
    \begin{minted}[breaklines]{python}
def apply_gimmick_prompt_template(
    sample: dict[str, Any],
    regions_hint: bool,
    countries_hint: bool,
) -> str:
    
    prompt_template = "{QUESTION}\n{HINTS}\n"
    hints = ""

    if regions_hint:
        hints += (
            "Hint: The question is related to a cultural event or facet from the following region(s): "
            f"{', '.join(sample['regions'])}\n"
        )

    if countries_hint:
        hints += (
            "Hint: The question is related to a cultural event or facet from the following country or countries: "
            f"{', '.join(sample['countries'])}\n"
        )

    return prompt_template.format(
        QUESTION=sample["prompt"],
        HINTS=hints,
    )
    \end{minted}
    \end{promptbox}
    \label{fig:sivqa:hints}
    \caption{Python Pseudo-Code to generate the prompt for a given \sivqa (or \vvqa) sample for the external cues settings.}
\end{figure*}
%



\subsection{Synthetic Data Generation}
\label{appendix:sec:sivqa:synth}
%
\include{src/991_1_2_appendix_sivqa_prompt}
%

\subsection{Annotation Project Details}
\label{appendix:sec:sivqa:anno}
%
We first conducted several internal pilot studies to iteratively create a straightforward annotation task, guidelines, and an intuitive interface for the final annotation project.
%
To find annotators, we advertised the task in our faculty research network, emphasizing our goal of creating a culturally diverse benchmark for assessing the cultural awareness of current AI models.
%
Therefore, we targeted primarily individuals from non-Western cultural backgrounds.
%
We found 18 volunteers who have spent most of their lives in 10 different countries from all six regions and thus cover diverse cultural backgrounds (see Table~\ref{tab:sivqa:anno:demographics}).
%
To train the annotators, we provided detailed annotation guidelines, followed by an oral introduction to the task.
%
For more details, refer to the (anonymized) original annotation guidelines we \href{https://drive.proton.me/urls/T6RHQCEW5G#5y0Itm2BdWYZ}{shared here}.
%

For the second annotation round, we hired 5 of the previous volunteering annotators (0, 1, 8, 15, 17) who assessed the kept samples from the first round to obtain two annotations (from distinct annotators) per sample.
%
We paid the second-round annotators a salary of roughly 12.5€ per hour.
%
\begin{table}[ht!]
    \centering
    \renewcommand{\arraystretch}{0.95}
    \resizebox{\linewidth}{!}{%
    \begin{tabular}{lrllllr}
        \toprule
        \textsc{ID} & \textsc{Age} & \textsc{Pronouns} & \textsc{Education} & \textsc{Country} & \textsc{Region} & \textsc{Round(s)} \\
        \midrule
        0 & 23 & she/her & Bachelor & Iran & \RegAP & 1, 2\\
        1 & 23 & she/her & Bachelor & Iran & \RegAP & 1, 2\\
        2 & 28 & she/her & PhD & Russia & \RegE & 1\\
        3 & 35 & he/him & Master & Germany & \RegW & 1 \\
        5 & 29 & he/him & Bachelor & Guatemala & \RegLAC & 1\\
        6 & 29 & he/him & Master & Germany & \RegW & 1\\
        7 & 42 & he/him & PhD & Ethiopia & \RegSA & 1\\
        8 & 23 & he/him & Bachelor & Egypt & \RegA & 1, 2\\
        9 & 33 & she/her & Master & Iran & \RegAP & 1\\
        10 & 29 & she/her & Bachelor & Afghanistan & \RegAP & 1\\
        11 & 23 & she/her & Bachelor & India & \RegAP & 1\\
        12 & 33 & he/him & Bachelor & Germany & \RegW & 1\\
        13 & 22 & she/her & Bachelor & Pakistan & \RegAP & 1\\
        14 & 27 & he/him & Master & China & \RegAP & 1\\
        15 & 29 & she/her & High School & Germany & \RegW & 1, 2\\
        16 & 22 & she/her & Bachelor & China & \RegAP & 1\\
        17 & 26 & he/him & High School & Germany & \RegW & 1, 2, 3\\
        \bottomrule
    \end{tabular}
    }%
    \caption{Demographics of the annotators who participated in our VQA annotation project. For the country, we asked the question, ``\textit{Where did you spend most of your life?}''. The Round(s) column indicates which annotation rounds the annotator participated in.}
    \label{tab:sivqa:anno:demographics}
\end{table}
%

\subsubsection{\sivqa Annotation Interface}
\label{appendix:sec:sivqa:anno:ui}
%
For the annotation project, we used a self-hosted Label Studio\footnote{\url{https://labelstud.io/}} instance with a custom labeling interface (see Figure~\ref{fig:sivqa:anno:ui}) for all annotation projects.
%
\begin{figure*}
    \centering
    \begin{subfigure}[b]{1.\textwidth}
         \centering
         \includegraphics[width=\textwidth]{gfx/anno-task-screenshot-sample-A.png}
     \end{subfigure}
     
     \begin{subfigure}[b]{1.\textwidth}
         \centering
         \includegraphics[width=\textwidth]{gfx/anno-task-screenshot-sample-B.png}
     \end{subfigure}

     \begin{subfigure}[b]{1.\textwidth}
         \centering
         \includegraphics[width=\textwidth]{gfx/anno-task-screenshot-sample-C.png}
     \end{subfigure}
    \caption{Three screenshots showing examples of the Label Studio interface used in our \sivqa annotation tasks.}
    \label{fig:sivqa:anno:ui}
\end{figure*}

{
\onecolumn
\subsubsection{First Annotation Round Statistics}
\label{appendix:sec:sivqa:anno:first_round}
%
\begin{table}[ht!]
    \centering
    \begin{minipage}[t]{0.50\textwidth}
        \scriptsize
        \centering
        \begin{tabular}{lr}
            \toprule
            Country & Count \\
            \midrule
            United Arab Emirates & 101 \\
            China & 98 \\
            Oman & 91 \\
            Saudi Arabia & 87 \\
            France & 86 \\
            Croatia & 84 \\
            Algeria & 82 \\
            Morocco & 81 \\
            Türkiye & 78 \\
            Peru & 75 \\
            Spain & 74 \\
            Azerbaijan & 69 \\
            Colombia & 68 \\
            Islamic Republic of Iran & 66 \\
            Mali & 65 \\
            Mexico & 64 \\
            Republic of Korea & 62 \\
            Egypt & 62 \\
            Tunisia & 56 \\
            Iraq & 54 \\
            Japan & 52 \\
            Brazil & 50 \\
            Italy & 50 \\
            Belgium & 50 \\
            Plurinational State of Bolivia & 49 \\
            Mauritania & 49 \\
            Bolivarian Republic of Venezuela & 47 \\
            Nigeria & 46 \\
            India & 45 \\
            Malawi & 43 \\
            Palestine & 40 \\
            Greece & 38 \\
            Uzbekistan & 37 \\
            Kuwait & 37 \\
            Kyrgyzstan & 36 \\
            Cuba & 35 \\
            Mauritius & 34 \\
            Mongolia & 34 \\
            Czechia & 34 \\
            Jordan & 32 \\
            Zambia & 31 \\
            Côte d'Ivoire & 31 \\
            Syrian Arab Republic & 31 \\
            Kazakhstan & 30 \\
            Portugal & 29 \\
            Switzerland & 29 \\
            Uganda & 29 \\
            Ethiopia & 29 \\
            Botswana & 28 \\
            Viet Nam & 28 \\
            Argentina & 28 \\
            Armenia & 28 \\
            Yemen & 28 \\
            Turkmenistan & 26 \\
            Sudan & 26 \\
            Bahrain & 26 \\
            Indonesia & 26 \\
            Ecuador & 25 \\
            Mozambique & 25 \\
            Tajikistan & 25 \\
            Austria & 24 \\
            Hungary & 24 \\
            Slovakia & 23 \\
            Lebanon & 23 \\
            Cyprus & 22 \\
            Slovenia & 22 \\
            Paraguay & 21 \\
            Germany & 21 \\
            Romania & 21 \\
            Guatemala & 20 \\
            Kenya & 20 \\
            Poland & 20 \\
            \bottomrule
        \end{tabular}
    \end{minipage}
    \hspace{-2.5cm}
    \begin{minipage}[t]{0.50\textwidth}
        \scriptsize
        \centering
        \begin{tabular}{lr}
        \toprule
        Country & Count \\
        \midrule
        Nicaragua & 18 \\
        Chile & 17 \\
        Serbia & 17 \\
        Cambodia & 17 \\
        Bangladesh & 17 \\
        Bulgaria & 17 \\
        Qatar & 17 \\
        Ireland & 17 \\
        Panama & 16 \\
        Ukraine & 16 \\
        Malaysia & 16 \\
        Namibia & 16 \\
        Philippines & 15 \\
        Bosnia and Herzegovina & 15 \\
        Niger & 15 \\
        Estonia & 14 \\
        Netherlands & 14 \\
        Zimbabwe & 14 \\
        Senegal & 14 \\
        Madagascar & 14 \\
        Belarus & 13 \\
        Luxembourg & 13 \\
        Togo & 12 \\
        Burundi & 12 \\
        Dominican Republic & 12 \\
        Congo & 11 \\
        Democratic Republic of the Congo & 11 \\
        Benin & 11 \\
        Finland & 11 \\
        Angola & 10 \\
        Afghanistan & 10 \\
        Seychelles & 10 \\
        Democratic People’s Republic of Korea & 10 \\
        Norway & 9 \\
        Lao Peoples Democratic Republic & 9 \\
        Burkina Faso & 9 \\
        Sweden & 9 \\
        Bahamas & 9 \\
        Georgia & 9 \\
        Albania & 9 \\
        Republic of Moldova & 9 \\
        Cabo Verde & 8 \\
        North Macedonia & 8 \\
        Jamaica & 8 \\
        Honduras & 7 \\
        Latvia & 7 \\
        Denmark & 7 \\
        Pakistan & 7 \\
        Belize & 7 \\
        Uruguay & 7 \\
        Timor-Leste & 6 \\
        Montenegro & 6 \\
        Sri Lanka & 6 \\
        Thailand & 6 \\
        Guinea & 6 \\
        Malta & 5 \\
        Andorra & 5 \\
        Russian Federation & 5 \\
        Lithuania & 5 \\
        Tonga & 4 \\
        Costa Rica & 4 \\
        Cameroon & 4 \\
        Vanuatu & 3 \\
        Singapore & 3 \\
        Gambia & 3 \\
        Iceland & 3 \\
        Federated States of Micronesia & 2 \\
        Grenada & 2 \\
        Samoa & 2 \\
        Bhutan & 1 \\
        Djibouti & 1 \\
        Central African Republic & 1 \\
        \bottomrule
        \end{tabular}
    \end{minipage}
    \caption{The number of countries related to the QA pairs collected in the first annotation round for \sivqa.}
    \label{tab:sivqa:anno:first_round}
\end{table}
}

%
\section{\texttt{VVQA} Details}
\label{appendix:sec:vvqa}
%

\subsection{Examples}
\label{appendix:sec:vvqa:examples}
%
In the following, we provide one random sample per region for the \vvqa task.
%
Note that the lower part of the examples, where the related CEF is provided, is \emph{not} part of the actual sample.
%
\subsubsection*{\RegA}
%
\input{examples/vvqa/arab-states_0}
%
\subsubsection*{\RegAP}
%
\input{examples/vvqa/asian-and-pacific-states_0}
%
\subsubsection*{\RegE}
%
\input{examples/vvqa/eastern-european-states_0}
%
\subsubsection*{\RegLAC}
%
\input{examples/vvqa/latin-american-and-caribbean-states_0}
%
\subsubsection*{\RegSA}
%
\input{examples/vvqa/subsaharian-african-states_0}
%
\subsubsection*{\RegW}
%
\input{examples/vvqa/western-european-and-north-american-states_0}
%



\subsection{Annotation Project Details}
\label{appendix:sec:vvqa:anno}
%
The expert who annotated the samples was Annotator 17 from Table~\ref{tab:sivqa:anno:demographics}.
%
As for the \sivqa task, we used a self-hosted Label Studio instance with a custom labeling interface.
%
The UI is depicted in Figure~\ref{fig:vvqa:anno:ui}.
%

\begin{figure*}
    \centering
    \begin{subfigure}[b]{1.\textwidth}
         \centering
         \includegraphics[width=\textwidth]{gfx/vvqa_anno_ui_1.png}
     \end{subfigure}
     
     \begin{subfigure}[b]{1.\textwidth}
         \centering
         \includegraphics[width=\textwidth]{gfx/vvqa_anno_ui_2.png}
     \end{subfigure}
    \caption{Two screenshots showing examples of the Label Studio interface used in our VVQA annotation tasks.}
    \label{fig:vvqa:anno:ui}
\end{figure*}
%

%
\section{\texttt{COQA} Details}
\label{appendix:sec:coqa}
%

\subsection{Prompts}
\label{appendix:sec:coqa:prompts}
%
In the following, the prompts for the \coqar and \coqac tasks are provided.
%
For the variations involving images, the image placeholder gets replaced $N$ times, where $N$ is the number of images related to the target CEF.
%
\begin{figure*}[ht]
    \centering
    %
    \begin{promptbox}{Region --- Text-Only}
    \begin{minted}[breaklines]{markdown}
From which of the following regions does the cultural event or facet with the title `{TITLE}` originate?
Choose from the following options and output only the corresponding letter.

A. {REGION_OPTION_A}
B. {REGION_OPTION_B}
C. {REGION_OPTION_C}
D. {REGION_OPTION_D}

Your answer letter:
    \end{minted}
    \end{promptbox}
    %
    %
    \begin{promptbox}{Region --- Image-Only}
    \begin{minted}[breaklines]{markdown}
<IMAGE_PLACEHOLDER>

From which of the following countries does the cultural event or facet shown in the images originate?
Choose from the following options and output only the corresponding letter.

A. {REGION_OPTION_A}
B. {REGION_OPTION_B}
C. {REGION_OPTION_C}
D. {REGION_OPTION_D}

Your answer letter:
    \end{minted}
    \end{promptbox}
    %
    \begin{promptbox}{Region --- Text-Image}
    \begin{minted}[breaklines]{markdown}
<IMAGE_PLACEHOLDER>

From which of the following regions does the cultural event or facet with the title `{TITLE}` shown in the images originate?
Choose from the following options and output only the corresponding letter.

A. {REGION_OPTION_A}
B. {REGION_OPTION_B}
C. {REGION_OPTION_C}
D. {REGION_OPTION_D}

Your answer letter:
    \end{minted}
    \end{promptbox}
    %
    \label{fig:coqa:prompts_r}
    \caption{Prompts for the \coqar task.}
\end{figure*}
%

\begin{figure*}[ht]
    \centering
    %
    \begin{promptbox}{Country --- Text-Only}
    \begin{minted}[breaklines]{markdown}
From which of the following countries does the cultural event or facet with the title `{TITLE}` originate?
Choose from the following options and output only the corresponding letter.

A. {COUNTRY_OPTION_A}
B. {COUNTRY_OPTION_B}
C. {COUNTRY_OPTION_C}
D. {COUNTRY_OPTION_D}

Your answer letter:
    \end{minted}
    \end{promptbox}
    %
    %
    \begin{promptbox}{Country --- Image-Only}
    \begin{minted}[breaklines]{markdown}
<IMAGE_PLACEHOLDER>

From which of the following countries does the cultural event or facet with the title `{TITLE}` originate?
Choose from the following options and output only the corresponding letter.

A. {COUNTRY_OPTION_A}
B. {COUNTRY_OPTION_B}
C. {COUNTRY_OPTION_C}
D. {COUNTRY_OPTION_D}

Your answer letter:
    \end{minted}
    \end{promptbox}
    %
    %
    \begin{promptbox}{Country --- Text-Image}
    \begin{minted}[breaklines]{markdown}
<IMAGE_PLACEHOLDER>

From which of the following countries does the cultural event or facet with the title `{TITLE}` shown in the images originate?
Choose from the following options and output only the corresponding letter.

A. {COUNTRY_OPTION_A}
B. {COUNTRY_OPTION_B}
C. {COUNTRY_OPTION_C}
D. {COUNTRY_OPTION_D}

Your answer letter:
    \end{minted}
    \end{promptbox}
    %
    \label{fig:coqa:prompts_c}
    \caption{Prompts for the \coqac task.}
\end{figure*}
%

\clearpage
\subsection{Examples}
\label{appendix:sec:coqa:examples}
%
In the following, we provide one random sample per region for the \coqac task in the image-only setting.
%
For the other settings and the \coqa tasks, the same pattern applies using the respective prompts from above.
%
Note that the lower part of the examples, where the related CEF is provided, is \emph{not} part of the actual sample.
%

\subsubsection*{\RegA}
%
\input{examples/coqa/arab-states_0}
%
\subsubsection*{\RegAP}
%
\input{examples/coqa/asian-and-pacific-states_0}
%
\subsubsection*{\RegE}
%
\input{examples/coqa/eastern-european-states_0}
%
\subsubsection*{\RegLAC}
%
\input{examples/coqa/latin-american-and-caribbean-states_0}
%
\subsubsection*{\RegSA}
%
\input{examples/coqa/subsaharian-african-states_0}
%
\subsubsection*{\RegW}
%
\input{examples/coqa/western-european-and-north-american-states_0}
%


%
\section{\texttt{CKQA} Details}
\label{appendix:sec:ckqa}
%

\subsection{Prompts}
\label{appendix:sec:ckqa:prompts}
%
In the following, the prompts for the \ckqan and \ckqad tasks are provided.
%
For the variations involving images, the image placeholder gets replaced $N$ times, where $N$ is the number of images related to the target CEF.
%
Examples without the respective prompts, i.e., only the related CEFs, are provided in \S\ref{appendix:sec:benchmark:cef:examples}.
%

\begin{figure*}[ht!]
    \centering
    %
    \begin{promptbox}{Naming --- Image-Only}
    \begin{minted}[breaklines]{markdown}
Name the cultural event or facet depicted by the following images. Answer briefly and concisely.

<IMAGE_PLACEHOLDER>

Your answer: 
    \end{minted}
    \end{promptbox}
    \label{fig:ckqa:prompts_n}
    \caption{Prompt for the \ckqan task.}
\end{figure*}
%

\begin{figure*}[ht!]
    \centering
    %
    \begin{promptbox}{Describing --- Text-Only}
    \begin{minted}[breaklines]{markdown}
Write a brief essay about the cultural event or facet with the title `{TITLE}`.

Your answer: 
    \end{minted}
    \end{promptbox}
    %
    %
    \begin{promptbox}{Describing --- Image-Only}
    \begin{minted}[breaklines]{markdown}
Write a brief essay about the cultural event or facet depicted by the following images.

<IMAGE_PLACEHOLDER>

Your answer: 
    \end{minted}
    \end{promptbox}
    %
    %
    \begin{promptbox}{Describing --- Text-Image}
    \begin{minted}[breaklines]{markdown}
Write a brief essay about the cultural event or facet depicted by the following images. It has the title `{TITLE}`.

<IMAGE_PLACEHOLDER>

Your answer: 
    \end{minted}
    \end{promptbox}
    %
    \label{fig:ckqa:prompts_c}
    \caption{Prompts for the \ckqad task.}
\end{figure*}
%

%

%
\onecolumn
\section{Experimental Setup}
\label{appendix:sec:setup}
%
For inference, we load all models using the \textit{transformers} library (\texttt{v.4.48.0}) in 16-bit with Flash Attention 2~\cite{dao2022flashattention,dao2023flashattention2} (\texttt{v.2.7.3}), PyTorch (\texttt{v.2.4.0}), and CUDA (\texttt{v12.1}).
%
We used A40 (46GB) GPUs for models up to 26B parameters, A100 (80GB) GPUs for models up to 38B parameters, and two H100 (96GB) GPUs for 70B+ models in a multi-GPU setup.
%
To generate responses, we use greedy decoding, i.e., we use the following arguments for the generation method:
%
\begin{listing}[H]
\begin{minted}[fontsize=\footnotesize]{python}
generation_kwargs = {
    "max_new_tokens": 512,
    "do_sample": False,
    "temperature": None,
    "top_p": None,
    "top_k": None,
}
\end{minted}
\end{listing}
%

More details and exact hyperparameters are documented in the code base: \href{https://github.com/floschne/gimmick}{https://github.com/floschne/gimmick}.

\section{Results and Analyses}
\label{appendix:sec:analyses}
%
\subsection{\sivqa}
\label{appendix:sec:analyses:sivqa}
%
\subsubsection{Results}
\label{appendix:sec:analyses:sivqa:results}
%
\subsubsection*{Relaxed Accuracy}
\label{appendix:sec:analyses:sivqa:results:acc}
%
%
\begin{table}[htbp]
  \centering
  \renewcommand{\arraystretch}{.97}
    \resizebox{\textwidth}{!}{%
  \begin{tabular}{l cccc cccc cccc cccc cccc cccc |cccc}
    \toprule
    \multirow{2}{*}{Model} 
      & \multicolumn{4}{c}{West EU \& North America} 
      & \multicolumn{4}{c}{Asia \& Pacific} 
      & \multicolumn{4}{c}{Subsaharian Africa} 
      & \multicolumn{4}{c}{Arab} 
      & \multicolumn{4}{c}{East EU} 
      & \multicolumn{4}{c}{Latin-America \& Caribbean} 
      & \multicolumn{4}{c}{Average} \\
    \cmidrule(lr){2-5} \cmidrule(lr){6-9} \cmidrule(lr){10-13} \cmidrule(lr){14-17} \cmidrule(lr){18-21} \cmidrule(lr){22-25} \cmidrule(lr){26-29}
      & N & R & C & B 
      & N & R & C & B 
      & N & R & C & B 
      & N & R & C & B 
      & N & R & C & B 
      & N & R & C & B 
      & N & R & C & B \\
\midrule
\rowcolor{gray!20}
GPT-4o & 31.58 & 34.39 & 41.05 & 40.70 & 29.89 & 31.37 & 36.63 & 37.68 & 17.38 & 17.63 & 32.49 & 31.74 & 25.70 & 30.53 & 39.19 & 39.95 & 26.80 & 32.56 & 42.94 & 41.79 & 23.17 & 25.77 & 30.26 & 32.39 & 25.44 & 28.17 & 36.59 & 37.08 \\
Gemini Pro & 27.02 & 30.53 & 31.23 & 32.28 & 22.53 & 26.11 & 31.16 & 29.68 & 16.84 & 14.61 & 26.20 & 24.18 & 19.85 & 22.39 & 28.50 & 28.50 & 25.07 & 25.94 & 31.12 & 32.56 & 22.46 & 19.86 & 24.35 & 27.19 & 21.50 & 22.84 & 28.30 & 28.30 \\
\rowcolor{gray!20}
GPT-4o Mini & 23.86 & 25.26 & 30.18 & 29.82 & 21.05 & 21.89 & 26.74 & 26.53 & 9.32 & 10.58 & 16.12 & 15.37 & 17.30 & 19.85 & 25.45 & 25.95 & 19.02 & 19.02 & 28.53 & 28.24 & 16.31 & 17.73 & 23.64 & 22.93 & 17.38 & 18.54 & 24.81 & 24.59 \\
Gemini Flash & 22.81 & 25.96 & 27.02 & 24.91 & 18.95 & 20.21 & 26.11 & 25.89 & 12.91 & 10.83 & 20.40 & 18.89 & 15.27 & 17.56 & 20.36 & 20.61 & 20.17 & 19.31 & 24.78 & 24.50 & 14.66 & 16.55 & 22.22 & 20.57 & 16.85 & 18.00 & 23.29 & 22.44 \\
\rowcolor{gray!20}
InternVL2.5 78B & 25.61 & 23.86 & 29.82 & 29.12 & 20.21 & 19.79 & 26.32 & 27.58 & 10.33 & 11.08 & 20.40 & 20.40 & 17.81 & 19.85 & 27.74 & 27.99 & 19.02 & 17.58 & 24.50 & 23.63 & 13.95 & 15.13 & 20.80 & 21.51 & 16.75 & 16.97 & 24.45 & 24.72 \\
Qwen2 VL 72B & 22.46 & 22.81 & 29.82 & 29.12 & 17.47 & 19.16 & 21.47 & 23.37 & 8.31 & 8.56 & 12.85 & 13.10 & 13.99 & 16.28 & 20.10 & 19.85 & 21.04 & 20.46 & 28.53 & 29.39 & 13.00 & 14.66 & 19.86 & 19.62 & 15.32 & 16.26 & 21.45 & 21.59 \\
\rowcolor{gray!20}
InternVL2.5 38B & 23.86 & 23.16 & 28.77 & 29.82 & 17.26 & 17.89 & 22.32 & 23.16 & 9.07 & 8.82 & 17.88 & 16.62 & 14.25 & 17.30 & 23.16 & 22.65 & 16.14 & 17.29 & 24.78 & 23.92 & 11.82 & 12.29 & 17.97 & 17.49 & 14.55 & 15.41 & 21.99 & 21.63 \\
Claude 3.5 Sonnet & 19.65 & 17.19 & 22.11 & 24.21 & 16.42 & 12.84 & 18.11 & 22.95 & 6.30 & 4.53 & 10.58 & 11.59 & 13.99 & 11.20 & 17.81 & 20.61 & 16.71 & 14.12 & 21.90 & 21.61 & 13.48 & 13.00 & 17.97 & 22.93 & 14.02 & 11.64 & 17.60 & 20.24 \\
\rowcolor{gray!20}
InternVL2.5 26B & 20.00 & 19.65 & 25.61 & 25.96 & 13.26 & 14.95 & 18.95 & 18.74 & 6.30 & 6.80 & 11.59 & 12.34 & 12.98 & 14.76 & 20.61 & 21.12 & 15.56 & 14.41 & 21.04 & 21.04 & 13.00 & 14.89 & 19.62 & 19.39 & 13.03 & 14.15 & 19.44 & 19.61 \\
Llama 3.2 11B Vision & 16.49 & 18.95 & 20.70 & 20.35 & 13.26 & 12.84 & 15.79 & 16.84 & 5.29 & 5.54 & 9.07 & 8.31 & 7.89 & 7.89 & 10.18 & 10.18 & 12.68 & 13.54 & 17.29 & 19.02 & 11.82 & 11.82 & 13.95 & 14.66 & 10.61 & 11.06 & 13.97 & 14.20 \\
\rowcolor{gray!20}
InternVL2.5 8B & 19.30 & 17.89 & 23.16 & 23.51 & 11.79 & 12.00 & 16.42 & 16.84 & 5.04 & 6.30 & 10.58 & 9.57 & 9.41 & 9.67 & 14.50 & 14.25 & 9.80 & 9.80 & 15.27 & 15.56 & 9.46 & 9.69 & 13.71 & 14.89 & 10.34 & 10.39 & 15.41 & 15.41 \\
Qwen2 VL 7B & 17.19 & 17.19 & 20.35 & 18.95 & 9.47 & 9.47 & 12.00 & 11.37 & 5.79 & 6.30 & 8.56 & 8.31 & 8.91 & 9.92 & 11.45 & 11.45 & 10.95 & 12.10 & 15.56 & 14.41 & 9.69 & 11.11 & 13.00 & 13.24 & 9.63 & 10.26 & 12.76 & 12.36 \\
\rowcolor{gray!20}
Phi 3.5 Vision & 14.39 & 12.63 & 20.00 & 18.95 & 8.84 & 10.74 & 13.89 & 13.47 & 6.05 & 6.05 & 8.31 & 8.82 & 6.62 & 8.14 & 9.41 & 9.92 & 8.93 & 8.65 & 14.70 & 14.70 & 8.27 & 9.93 & 13.71 & 12.77 & 8.55 & 9.18 & 12.99 & 12.85 \\
MiniCPM V 2.6 & 12.98 & 14.39 & 14.39 & 17.19 & 10.74 & 10.32 & 13.68 & 14.74 & 2.52 & 3.27 & 6.55 & 6.05 & 6.36 & 6.36 & 9.67 & 9.67 & 10.09 & 9.80 & 13.26 & 14.70 & 9.46 & 9.22 & 12.77 & 13.24 & 8.11 & 8.15 & 11.60 & 11.96 \\
\rowcolor{gray!20}
InternVL2.5 4B & 14.04 & 16.49 & 16.84 & 14.39 & 9.47 & 14.53 & 13.47 & 9.05 & 3.53 & 7.05 & 7.56 & 4.03 & 7.89 & 9.16 & 9.16 & 6.87 & 8.07 & 11.53 & 10.66 & 8.07 & 8.04 & 11.58 & 11.82 & 7.33 & 7.97 & 11.42 & 11.29 & 7.79 \\
Qwen2 VL 2B & 13.33 & 12.28 & 13.68 & 14.39 & 9.68 & 9.47 & 11.79 & 10.95 & 4.03 & 3.78 & 5.54 & 4.28 & 6.11 & 5.09 & 6.11 & 6.11 & 8.36 & 8.93 & 12.97 & 12.10 & 7.33 & 8.27 & 10.40 & 9.69 & 7.97 & 7.88 & 9.94 & 9.49 \\
\rowcolor{gray!20}
Centurio Qwen & 11.23 & 9.12 & 14.39 & 14.39 & 9.05 & 8.42 & 10.32 & 9.68 & 3.02 & 1.76 & 6.05 & 6.05 & 6.87 & 5.34 & 9.92 & 8.91 & 6.34 & 5.48 & 11.24 & 11.24 & 6.62 & 5.67 & 9.22 & 9.46 & 6.81 & 5.69 & 9.85 & 9.76 \\
InternVL2.5 2B & 6.67 & 7.37 & 10.18 & 9.47 & 4.21 & 4.63 & 6.95 & 5.89 & 2.27 & 2.02 & 3.53 & 5.29 & 2.80 & 3.56 & 5.34 & 5.60 & 3.17 & 3.75 & 7.49 & 6.92 & 5.44 & 5.44 & 8.04 & 6.15 & 4.03 & 4.39 & 6.85 & 6.45 \\
\rowcolor{gray!20}
InternVL2.5 1B & 7.02 & 7.37 & 10.53 & 11.58 & 4.21 & 3.58 & 4.84 & 4.63 & 2.52 & 0.76 & 2.77 & 2.77 & 3.56 & 3.82 & 5.09 & 4.07 & 4.61 & 3.46 & 6.63 & 7.49 & 4.02 & 5.67 & 6.86 & 6.15 & 4.03 & 4.03 & 5.96 & 5.87 \\
Centurio Aya & 3.16 & 7.37 & 8.77 & 8.77 & 2.95 & 5.68 & 9.05 & 9.68 & 1.76 & 1.51 & 4.79 & 3.53 & 1.27 & 3.56 & 5.60 & 6.11 & 2.02 & 3.46 & 7.20 & 7.20 & 2.84 & 5.44 & 6.38 & 7.09 & 2.24 & 4.39 & 6.99 & 7.17 \\
\midrule
Average X-Large & 24.04 & 23.33 & 29.82 & 29.12 & 18.84 & 19.47 & 23.89 & 25.47 & 9.32 & 9.82 & 16.62 & 16.75 & 15.90 & 18.07 & 23.92 & 23.92 & 20.03 & 19.02 & 26.51 & 26.51 & 13.48 & 14.89 & 20.33 & 20.57 & 16.03 & 16.61 & 22.95 & 23.15 \\
\rowcolor{gray!20}
Average Large & 21.93 & 21.40 & 27.19 & 27.89 & 15.26 & 16.42 & 20.63 & 20.95 & 7.68 & 7.81 & 14.74 & 14.48 & 13.61 & 16.03 & 21.88 & 21.88 & 15.85 & 15.85 & 22.91 & 22.48 & 12.41 & 13.59 & 18.79 & 18.44 & 13.79 & 14.78 & 20.71 & 20.62 \\
Average Medium & 13.39 & 14.15 & 16.96 & 17.19 & 9.54 & 9.79 & 12.88 & 13.19 & 3.90 & 4.11 & 7.60 & 6.97 & 6.79 & 7.12 & 10.22 & 10.09 & 8.65 & 9.03 & 13.30 & 13.69 & 8.31 & 8.83 & 11.51 & 12.10 & 7.96 & 8.32 & 11.76 & 11.81 \\
\rowcolor{gray!20}
Average Small & 11.09 & 11.23 & 14.25 & 13.75 & 7.28 & 8.59 & 10.19 & 8.80 & 3.68 & 3.93 & 5.54 & 5.04 & 5.39 & 5.95 & 7.02 & 6.51 & 6.63 & 7.26 & 10.49 & 9.86 & 6.62 & 8.18 & 10.17 & 8.42 & 6.51 & 7.38 & 9.40 & 8.49 \\
Average Open & 15.18 & 15.37 & 19.13 & 19.06 & 10.79 & 11.56 & 14.48 & 14.40 & 5.05 & 5.31 & 9.07 & 8.63 & 8.45 & 9.38 & 12.54 & 12.32 & 10.45 & 10.68 & 15.41 & 15.29 & 8.98 & 10.06 & 13.21 & 12.84 & 9.33 & 9.97 & 13.66 & 13.39 \\
\rowcolor{gray!20}
Average Proprietary & 24.98 & 26.67 & 30.32 & 30.39 & 21.77 & 22.48 & 27.75 & 28.55 & 12.55 & 11.64 & 21.16 & 20.35 & 18.42 & 20.31 & 26.26 & 27.12 & 21.56 & 22.19 & 29.86 & 29.74 & 18.01 & 18.58 & 23.69 & 25.20 & 19.04 & 19.84 & 26.12 & 26.53 \\
Average & 17.63 & 18.19 & 21.93 & 21.89 & 13.54 & 14.29 & 17.80 & 17.94 & 6.93 & 6.89 & 12.09 & 11.56 & 10.94 & 12.11 & 15.97 & 16.02 & 13.23 & 13.56 & 19.02 & 18.90 & 11.24 & 12.19 & 15.83 & 15.93 & 11.76 & 12.44 & 16.78 & 16.67 \\
\bottomrule
  \end{tabular}
  }%
  \caption{Cultural Image Visual Question Answering (\sivqa) scores. The reported score is relaxed accuracy. The columns \textbf{N}, \textbf{R}, \textbf{C}, and \textbf{B} stand for the hints \textbf{``None''}, \textbf{``Region''}, \textbf{``Country''}, and \textbf{``Both''}, respectively.}
  \label{tab:sivqa:scores}
\end{table}
%
\subsubsection*{Judge Score}
\label{appendix:sec:analyses:sivqa:results:judge}
%
\begin{figure}[ht!]
    \centering
    \includegraphics[width=.5\linewidth]{gfx/sivqa_avg_judge_score_scores.pdf}
    \caption{An overview of aggregated \sivqa Judge Score results.}
    \label{fig:analyses:a1_bias:scores:sivqa_judge}
\end{figure}
%
\begin{table}[htbp]
  \centering
  \renewcommand{\arraystretch}{.97}
    \resizebox{\textwidth}{!}{%
  \begin{tabular}{l cccc cccc cccc cccc cccc cccc |cccc}
    \toprule
    \multirow{2}{*}{Model} 
      & \multicolumn{4}{c}{West EU \& North America} 
      & \multicolumn{4}{c}{Asia \& Pacific} 
      & \multicolumn{4}{c}{Subsaharian Africa} 
      & \multicolumn{4}{c}{Arab} 
      & \multicolumn{4}{c}{East EU} 
      & \multicolumn{4}{c}{Latin-America \& Caribbean} 
      & \multicolumn{4}{c}{Average} \\
    \cmidrule(lr){2-5} \cmidrule(lr){6-9} \cmidrule(lr){10-13} \cmidrule(lr){14-17} \cmidrule(lr){18-21} \cmidrule(lr){22-25} \cmidrule(lr){26-29}
      & N & R & C & B 
      & N & R & C & B 
      & N & R & C & B 
      & N & R & C & B 
      & N & R & C & B 
      & N & R & C & B 
      & N & R & C & B \\
\midrule
\rowcolor{gray!20}
GPT-4o & 55.86 & 55.46 & 64.33 & 64.49 & 56.91 & 56.42 & 63.09 & 63.20 & 39.13 & 35.97 & 48.90 & 47.93 & 51.89 & 56.53 & 65.71 & 67.59 & 54.26 & 55.62 & 67.56 & 67.33 & 48.32 & 46.69 & 53.05 & 54.09 & 51.06 & 51.12 & 60.44 & 60.77 \\
Gemini Pro & 55.44 & 56.30 & 63.15 & 62.06 & 54.23 & 54.15 & 59.84 & 59.14 & 38.74 & 39.51 & 47.63 & 46.74 & 49.83 & 52.96 & 58.10 & 57.65 & 51.67 & 52.23 & 62.48 & 63.77 & 46.54 & 45.32 & 51.95 & 52.03 & 49.41 & 50.08 & 57.19 & 56.90 \\
\rowcolor{gray!20}
Claude 3.5 Sonnet & 50.65 & 51.35 & 65.33 & 64.14 & 50.11 & 51.07 & 63.70 & 63.59 & 35.06 & 39.90 & 57.56 & 56.52 & 48.99 & 55.34 & 67.49 & 67.86 & 50.42 & 51.80 & 70.12 & 70.20 & 41.49 & 46.10 & 57.85 & 59.53 & 46.12 & 49.26 & 63.68 & 63.64 \\
Gemini Flash & 50.49 & 49.93 & 56.29 & 56.11 & 48.19 & 47.42 & 53.20 & 53.11 & 35.60 & 31.32 & 39.58 & 38.54 & 44.04 & 44.25 & 50.85 & 49.03 & 46.66 & 47.34 & 57.24 & 55.22 & 43.59 & 41.58 & 47.74 & 46.88 & 44.76 & 43.64 & 50.82 & 49.81 \\
\rowcolor{gray!20}
GPT-4o Mini & 48.98 & 48.47 & 53.10 & 54.28 & 44.83 & 44.24 & 49.96 & 49.05 & 32.78 & 35.06 & 35.49 & 35.98 & 42.24 & 43.83 & 47.98 & 48.58 & 46.57 & 43.40 & 55.06 & 53.76 & 40.74 & 38.24 & 44.21 & 44.19 & 42.69 & 42.21 & 47.63 & 47.64 \\
Qwen2 VL 72B & 40.28 & 40.05 & 48.04 & 46.90 & 38.65 & 39.25 & 43.02 & 44.15 & 25.79 & 26.30 & 31.52 & 30.57 & 32.77 & 35.27 & 42.89 & 41.64 & 38.95 & 39.44 & 50.55 & 49.74 & 30.64 & 33.06 & 39.66 & 40.27 & 34.51 & 35.56 & 42.61 & 42.21 \\
\rowcolor{gray!20}
InternVL2.5 78B & 39.88 & 39.52 & 46.43 & 47.01 & 39.93 & 38.01 & 47.78 & 49.26 & 22.18 & 20.79 & 30.15 & 30.42 & 33.72 & 35.80 & 46.82 & 47.86 & 34.57 & 32.63 & 41.89 & 40.73 & 31.42 & 30.40 & 39.09 & 38.84 & 33.62 & 32.86 & 42.03 & 42.35 \\
InternVL2.5 26B & 37.00 & 34.65 & 39.75 & 41.00 & 32.64 & 32.97 & 39.10 & 39.47 & 22.63 & 21.71 & 29.40 & 27.22 & 30.89 & 31.39 & 38.10 & 38.81 & 34.34 & 32.38 & 41.14 & 41.53 & 29.34 & 29.69 & 37.05 & 37.78 & 31.14 & 30.47 & 37.42 & 37.64 \\
\rowcolor{gray!20}
InternVL2.5 38B & 37.55 & 37.51 & 45.58 & 45.49 & 35.45 & 36.26 & 42.65 & 43.88 & 22.98 & 22.52 & 29.11 & 28.35 & 28.71 & 31.78 & 38.96 & 38.63 & 32.08 & 31.69 & 41.98 & 41.21 & 28.39 & 29.15 & 36.46 & 35.18 & 30.86 & 31.48 & 39.12 & 38.79 \\
Qwen2 VL 7B & 33.36 & 34.84 & 38.64 & 38.12 & 28.19 & 28.97 & 31.23 & 31.13 & 21.31 & 25.25 & 25.09 & 26.26 & 28.72 & 28.45 & 32.00 & 32.28 & 29.19 & 31.13 & 35.53 & 37.11 & 27.84 & 28.61 & 31.45 & 32.87 & 28.10 & 29.54 & 32.33 & 32.96 \\
\rowcolor{gray!20}
Llama 3.2 11B Vision & 35.16 & 36.56 & 37.22 & 37.81 & 27.06 & 27.59 & 31.14 & 33.09 & 19.24 & 17.97 & 24.38 & 26.42 & 25.09 & 26.53 & 31.43 & 30.47 & 28.34 & 27.88 & 33.96 & 36.73 & 26.89 & 28.82 & 32.14 & 32.88 & 26.96 & 27.56 & 31.71 & 32.90 \\
MiniCPM V 2.6 & 30.61 & 32.35 & 32.73 & 35.48 & 27.73 & 25.88 & 31.13 & 33.25 & 20.29 & 18.92 & 25.31 & 24.58 & 24.52 & 24.57 & 28.47 & 28.19 & 28.13 & 25.07 & 34.31 & 36.27 & 26.74 & 26.04 & 29.16 & 30.37 & 26.34 & 25.47 & 30.18 & 31.36 \\
\rowcolor{gray!20}
Qwen2 VL 2B & 28.86 & 28.30 & 28.94 & 30.85 & 25.18 & 24.06 & 26.32 & 26.31 & 21.02 & 19.32 & 23.06 & 21.94 & 20.92 & 20.32 & 22.98 & 23.30 & 25.10 & 26.01 & 32.90 & 31.34 & 23.91 & 24.07 & 26.73 & 25.85 & 24.16 & 23.68 & 26.82 & 26.60 \\
Centurio Aya & 29.84 & 30.21 & 30.67 & 32.31 & 26.64 & 25.51 & 28.70 & 28.81 & 18.81 & 17.87 & 21.23 & 20.97 & 19.75 & 20.43 & 24.02 & 24.01 & 25.42 & 24.93 & 28.79 & 30.72 & 23.58 & 24.65 & 25.66 & 26.68 & 24.01 & 23.93 & 26.51 & 27.25 \\
\rowcolor{gray!20}
InternVL2.5 8B & 30.12 & 32.19 & 35.35 & 36.47 & 23.62 & 23.93 & 29.75 & 29.92 & 16.70 & 17.20 & 20.54 & 21.81 & 23.61 & 23.46 & 30.65 & 29.92 & 24.94 & 24.67 & 32.73 & 33.66 & 22.80 & 22.13 & 26.78 & 27.99 & 23.63 & 23.93 & 29.30 & 29.96 \\
Phi 3.5 Vision & 24.84 & 26.93 & 33.43 & 33.45 & 23.46 & 25.36 & 29.18 & 29.02 & 21.28 & 21.65 & 23.63 & 25.92 & 21.06 & 23.26 & 26.18 & 26.48 & 24.70 & 24.88 & 31.32 & 31.82 & 24.47 & 25.73 & 29.67 & 30.56 & 23.30 & 24.64 & 28.90 & 29.54 \\
\rowcolor{gray!20}
Centurio Qwen & 27.32 & 26.32 & 29.21 & 30.42 & 25.84 & 26.54 & 26.88 & 28.63 & 18.12 & 17.91 & 20.14 & 22.69 & 23.46 & 22.32 & 27.19 & 27.72 & 20.84 & 20.56 & 26.53 & 28.83 & 21.21 & 21.74 & 23.19 & 23.82 & 22.80 & 22.56 & 25.52 & 27.02 \\
InternVL2.5 4B & 25.18 & 26.06 & 26.71 & 29.67 & 20.67 & 22.04 & 26.29 & 27.53 & 12.32 & 14.45 & 14.99 & 17.91 & 18.42 & 21.62 & 24.43 & 25.80 & 20.22 & 22.07 & 26.43 & 28.43 & 17.56 & 20.93 & 23.45 & 24.29 & 19.06 & 21.19 & 23.72 & 25.61 \\
\rowcolor{gray!20}
InternVL2.5 1B & 19.67 & 20.32 & 20.90 & 23.91 & 14.46 & 13.92 & 14.95 & 17.03 & 12.05 & 13.50 & 16.60 & 15.86 & 16.48 & 16.31 & 16.88 & 17.42 & 16.10 & 14.90 & 18.27 & 20.82 & 14.94 & 15.64 & 16.75 & 17.59 & 15.62 & 15.76 & 17.39 & 18.77 \\
InternVL2.5 2B & 18.19 & 19.35 & 20.25 & 21.95 & 14.75 & 15.99 & 16.42 & 18.36 & 13.14 & 10.52 & 12.88 & 14.96 & 15.55 & 14.08 & 16.69 & 18.03 & 14.77 & 13.73 & 17.43 & 18.32 & 15.57 & 15.86 & 18.04 & 18.77 & 15.33 & 14.92 & 16.95 & 18.40 \\
\midrule
Average X-Large & 40.08 & 39.79 & 47.23 & 46.95 & 39.29 & 38.63 & 45.40 & 46.71 & 23.99 & 23.55 & 30.83 & 30.49 & 33.25 & 35.54 & 44.86 & 44.75 & 36.76 & 36.03 & 46.22 & 45.24 & 31.03 & 31.73 & 39.37 & 39.56 & 34.06 & 34.21 & 42.32 & 42.28 \\
\rowcolor{gray!20}
Average Large & 37.28 & 36.08 & 42.67 & 43.25 & 34.05 & 34.61 & 40.88 & 41.68 & 22.81 & 22.12 & 29.25 & 27.79 & 29.80 & 31.58 & 38.53 & 38.72 & 33.21 & 32.04 & 41.56 & 41.37 & 28.86 & 29.42 & 36.75 & 36.48 & 31.00 & 30.98 & 38.27 & 38.22 \\
Average Medium & 31.07 & 32.08 & 33.97 & 35.10 & 26.51 & 26.40 & 29.80 & 30.80 & 19.08 & 19.19 & 22.78 & 23.79 & 24.19 & 24.29 & 28.96 & 28.77 & 26.14 & 25.71 & 31.98 & 33.89 & 24.84 & 25.33 & 28.06 & 29.10 & 25.31 & 25.50 & 29.26 & 30.24 \\
\rowcolor{gray!20}
Average Small & 23.35 & 24.19 & 26.05 & 27.96 & 19.70 & 20.27 & 22.63 & 23.65 & 15.96 & 15.89 & 18.23 & 19.32 & 18.48 & 19.12 & 21.43 & 22.21 & 20.18 & 20.32 & 25.27 & 26.15 & 19.29 & 20.45 & 22.93 & 23.41 & 19.49 & 20.04 & 22.76 & 23.78 \\
Average Open & 30.52 & 31.01 & 34.26 & 35.39 & 26.95 & 27.08 & 30.97 & 31.99 & 19.19 & 19.06 & 23.20 & 23.73 & 24.24 & 25.04 & 29.85 & 30.04 & 26.51 & 26.13 & 32.92 & 33.82 & 24.35 & 25.10 & 29.02 & 29.58 & 25.30 & 25.57 & 30.03 & 30.76 \\
\rowcolor{gray!20}
Average Proprietary & 52.28 & 52.30 & 60.44 & 60.22 & 50.85 & 50.66 & 57.96 & 57.62 & 36.26 & 36.35 & 45.83 & 45.14 & 47.40 & 50.58 & 58.03 & 58.14 & 49.92 & 50.08 & 62.49 & 62.05 & 44.14 & 43.59 & 50.96 & 51.34 & 46.81 & 47.26 & 55.95 & 55.75 \\
Average & 35.96 & 36.33 & 40.80 & 41.59 & 32.93 & 32.98 & 37.72 & 38.40 & 23.46 & 23.38 & 28.86 & 29.08 & 30.03 & 31.42 & 36.89 & 37.06 & 32.36 & 32.12 & 40.31 & 40.88 & 29.30 & 29.72 & 34.50 & 35.02 & 30.67 & 30.99 & 36.51 & 37.01 \\
\bottomrule
  \end{tabular}
  }%
  \caption{Cultural Image Visual Question Answering (\sivqa) scores. The reported score is the average judge score. The columns \textbf{N}, \textbf{R}, \textbf{C}, and \textbf{B} stand for the hints \textbf{``None''}, \textbf{``Region''}, \textbf{``Country''}, and \textbf{``Both''}, respectively.}
  \label{tab:sivqa:judge_scores}
\end{table}
%


\subsubsection{Ground-Truth Answer Perplexity}
\label{appendix:sec:analyses:sivqa:results:ppl}

The perplexity for every sample is computed as follows:

\begin{equation}
\mathrm{PPL}(y \mid x) = \exp\left(-\frac{1}{N} \sum_{t=0}^{N} \log p\left(y_t \mid y_{t-1},\, x\right)\right)
\end{equation}

where $x = \{s, v\}$ are the textual ($s$) and visual ($v$) prompt (prefix) tokens and $y$ are the $N$ ground-truth answer tokens.


\subsubsection*{Results Per Cultural Aspect}
\label{sec:appendix:sec:analyses:sivqa:results:aspects}
%
We computed the average accuracy for questions targeting one of the ten most frequent cultural aspects (see \S\ref{appedix:sec:sivqa:aspects}), grouped by model size and region.
%
For better interpretation, Table~\ref{tab:sivqa:results:aspect_counts} reports the counts of questions associated with each cultural aspect per region.
%
As shown in Table~\ref{tab:sivqa:results:aspect_scores}, our results reveal a consistent trend: models perform significantly better on tangible cultural aspects (e.g., food) than on intangible ones.
%
For instance, across all regions, closed models achieve an average accuracy of 30\% for food-related questions, compared to only 8\% and 10\% for questions concerning rituals and festivals, respectively.
%
These findings highlight not only regional biases but also biases along the cultural dimension, the latter being particularly pronounced in non-Western contexts.
%

%

%
\begin{table}[ht]
    \centering
    \resizebox{\textwidth}{!}{%
\begin{tabular}{lrrrrrrrrrr}
\toprule
aspect & art & craftsmanship & dance & festivals & food & instruments & music & rituals & tools & traditions \\
\midrule
\RegA & 45 & 32 & 20 & 6 & 33 & 20 & 37 & 32 & 30 & 76 \\
\RegAP & 57 & 44 & 31 & 14 & 12 & 25 & 32 & 53 & 22 & 68 \\
\RegE & 53 & 36 & 18 & 19 & 10 & 19 & 26 & 18 & 20 & 49 \\
\RegLAC & 31 & 22 & 31 & 66 & 6 & 13 & 51 & 47 & 12 & 78 \\
\RegSA & 14 & 16 & 40 & 16 & 22 & 64 & 41 & 73 & 7 & 70 \\
\RegW & 33 & 27 & 10 & 30 & 13 & 14 & 23 & 18 & 17 & 49 \\
\bottomrule
\end{tabular}
}
    \caption{Number of questions targeting one of the top-10 cultural aspects per region in \sivqa.}
    \label{tab:sivqa:results:aspect_counts}
\end{table}
%

\begin{table}[ht!]
    \centering
    \resizebox{\textwidth}{!}{%
\begin{tabular}{lrrrrrrrrrrrrrrrrrrrrrrrrrrrrrrrrrrr}
\toprule
 & \multicolumn{5}{c}{\RegAP} & \multicolumn{5}{c}{\RegA} & \multicolumn{5}{c}{\RegSA} & \multicolumn{5}{c}{\RegW} & \multicolumn{5}{c}{\RegE} & \multicolumn{5}{c}{\RegLAC} & \multicolumn{5}{c}{\textsc{Overall}} \\
\cmidrule(lr){2-6} \cmidrule(lr){7-11} \cmidrule(lr){12-16} \cmidrule(lr){17-21} \cmidrule(lr){22-26} \cmidrule(lr){27-31} \cmidrule(lr){32-36}
 & \textsc{A} & \textsc{XL} & \textsc{L} & \textsc{M} & \textsc{S} & \textsc{A} & \textsc{XL} & \textsc{L} & \textsc{M} & \textsc{S} & \textsc{A} & \textsc{XL} & \textsc{L} & \textsc{M} & \textsc{S} & \textsc{A} & \textsc{XL} & \textsc{L} & \textsc{M} & \textsc{S} & \textsc{A} & \textsc{XL} & \textsc{L} & \textsc{M} & \textsc{S} & \textsc{A} & \textsc{XL} & \textsc{L} & \textsc{M} & \textsc{S} & \textsc{A} & \textsc{XL} & \textsc{L} & \textsc{M} & \textsc{S} \\
\midrule
food & 0.28 & 0.23 & 0.21 & 0.07 & 0.10 & 0.28 & 0.35 & 0.31 & 0.06 & 0.09 & 0.21 & 0.16 & 0.07 & 0.03 & 0.04 & 0.18 & 0.12 & 0.19 & 0.07 & 0.09 & 0.18 & 0.36 & 0.31 & 0.10 & 0.12 & 0.68 & 0.54 & 0.71 & 0.31 & 0.39 & 0.30 & 0.29 & 0.30 & 0.11 & 0.14 \\
instruments & 0.29 & 0.27 & 0.25 & 0.05 & 0.07 & 0.20 & 0.15 & 0.12 & 0.03 & 0.04 & 0.16 & 0.15 & 0.16 & 0.03 & 0.04 & 0.26 & 0.37 & 0.32 & 0.05 & 0.11 & 0.32 & 0.31 & 0.26 & 0.04 & 0.05 & 0.45 & 0.44 & 0.31 & 0.15 & 0.20 & 0.28 & 0.28 & 0.24 & 0.06 & 0.08 \\
craftsmanship & 0.15 & 0.17 & 0.15 & 0.04 & 0.07 & 0.18 & 0.24 & 0.22 & 0.08 & 0.08 & 0.11 & 0.09 & 0.08 & 0.04 & 0.02 & 0.14 & 0.19 & 0.12 & 0.10 & 0.08 & 0.26 & 0.28 & 0.27 & 0.13 & 0.17 & 0.12 & 0.12 & 0.12 & 0.06 & 0.07 & 0.16 & 0.18 & 0.16 & 0.08 & 0.08 \\
music & 0.20 & 0.32 & 0.26 & 0.07 & 0.09 & 0.10 & 0.09 & 0.12 & 0.03 & 0.04 & 0.13 & 0.11 & 0.13 & 0.03 & 0.03 & 0.25 & 0.27 & 0.28 & 0.10 & 0.13 & 0.10 & 0.10 & 0.05 & 0.02 & 0.02 & 0.19 & 0.22 & 0.22 & 0.08 & 0.12 & 0.16 & 0.19 & 0.18 & 0.06 & 0.07 \\
tools & 0.19 & 0.29 & 0.18 & 0.09 & 0.11 & 0.18 & 0.17 & 0.18 & 0.05 & 0.06 & 0.00 & 0.05 & 0.04 & 0.00 & 0.04 & 0.22 & 0.15 & 0.19 & 0.09 & 0.11 & 0.14 & 0.17 & 0.17 & 0.04 & 0.04 & 0.23 & 0.17 & 0.30 & 0.05 & 0.05 & 0.16 & 0.17 & 0.18 & 0.05 & 0.07 \\
traditions & 0.19 & 0.18 & 0.15 & 0.06 & 0.07 & 0.11 & 0.09 & 0.09 & 0.04 & 0.05 & 0.06 & 0.07 & 0.06 & 0.04 & 0.04 & 0.21 & 0.25 & 0.21 & 0.09 & 0.09 & 0.16 & 0.16 & 0.15 & 0.06 & 0.08 & 0.14 & 0.16 & 0.13 & 0.06 & 0.08 & 0.14 & 0.15 & 0.13 & 0.06 & 0.07 \\
art & 0.15 & 0.17 & 0.13 & 0.07 & 0.07 & 0.18 & 0.13 & 0.14 & 0.04 & 0.04 & 0.07 & 0.09 & 0.06 & 0.01 & 0.00 & 0.16 & 0.27 & 0.23 & 0.10 & 0.13 & 0.13 & 0.20 & 0.12 & 0.06 & 0.07 & 0.10 & 0.11 & 0.11 & 0.06 & 0.09 & 0.13 & 0.16 & 0.13 & 0.06 & 0.07 \\
dance & 0.07 & 0.07 & 0.04 & 0.01 & 0.00 & 0.02 & 0.00 & 0.00 & 0.00 & 0.00 & 0.05 & 0.04 & 0.02 & 0.01 & 0.01 & 0.26 & 0.39 & 0.35 & 0.17 & 0.11 & 0.22 & 0.23 & 0.17 & 0.04 & 0.03 & 0.14 & 0.11 & 0.08 & 0.04 & 0.03 & 0.13 & 0.14 & 0.11 & 0.05 & 0.03 \\
festivals & 0.18 & 0.20 & 0.18 & 0.09 & 0.08 & 0.02 & 0.06 & 0.00 & 0.00 & 0.00 & 0.02 & 0.00 & 0.00 & 0.00 & 0.00 & 0.10 & 0.10 & 0.10 & 0.04 & 0.06 & 0.13 & 0.11 & 0.05 & 0.02 & 0.01 & 0.13 & 0.11 & 0.11 & 0.04 & 0.04 & 0.10 & 0.10 & 0.07 & 0.03 & 0.03 \\
rituals & 0.09 & 0.08 & 0.09 & 0.02 & 0.03 & 0.06 & 0.06 & 0.05 & 0.02 & 0.03 & 0.06 & 0.07 & 0.04 & 0.02 & 0.04 & 0.12 & 0.10 & 0.08 & 0.02 & 0.02 & 0.13 & 0.16 & 0.08 & 0.01 & 0.01 & 0.04 & 0.06 & 0.03 & 0.01 & 0.00 & 0.08 & 0.09 & 0.06 & 0.02 & 0.02 \\
\midrule
\emph{Average} & 0.18 & 0.20 & 0.16 & 0.06 & 0.07 & 0.13 & 0.13 & 0.12 & 0.03 & 0.04 & 0.09 & 0.08 & 0.07 & 0.02 & 0.03 & 0.19 & 0.22 & 0.21 & 0.08 & 0.09 & 0.18 & 0.21 & 0.16 & 0.05 & 0.06 & 0.22 & 0.20 & 0.21 & 0.09 & 0.11 & 0.16 & 0.17 & 0.16 & 0.06 & 0.07 \\
\bottomrule
\end{tabular}
    }
    \caption{The averaged accuracy per region per model size group (A, XL, L, M, S) per target cultural aspect for samples in the \sivqa task.}
    \label{tab:sivqa:results:aspect_scores}
\end{table}

%
\subsection{\vvqa}
\label{appendix:sec:analyses:vvqa}
%
\subsubsection{Results}
%
\begin{table}[htbp]
  \centering
  \renewcommand{\arraystretch}{.97}
    \resizebox{\textwidth}{!}{%
  \begin{tabular}{@{}l cccc cccc cccc cccc cccc cccc |cccc@{}}
    \toprule
    \multirow{2}{*}{Model} 
      & \multicolumn{4}{c}{West EU \& North America} 
      & \multicolumn{4}{c}{Asia \& Pacific} 
      & \multicolumn{4}{c}{Subsaharian Africa} 
      & \multicolumn{4}{c}{Arab} 
      & \multicolumn{4}{c}{East EU} 
      & \multicolumn{4}{c}{Latin-America \& Caribbean} 
      & \multicolumn{4}{c}{Average} \\
    \cmidrule(lr){2-5} \cmidrule(lr){6-9} \cmidrule(lr){10-13} \cmidrule(lr){14-17} \cmidrule(lr){18-21} \cmidrule(lr){22-25} \cmidrule(lr){26-29}
      & N & R & C & B 
      & N & R & C & B 
      & N & R & C & B 
      & N & R & C & B 
      & N & R & C & B 
      & N & R & C & B 
      & N & R & C & B \\
\midrule
\rowcolor{gray!20}
GPT-4o & 38.97 & 39.91 & 41.31 & 44.13 & 34.56 & 36.41 & 35.71 & 39.17 & 23.67 & 26.33 & 36.67 & 36.00 & 29.18 & 32.46 & 36.72 & 36.39 & 37.59 & 40.43 & 47.16 & 45.74 & 31.98 & 32.56 & 38.95 & 39.24 & 32.67 & 34.49 & 39.19 & 39.97 \\
GPT-4o Mini & 38.06 & 31.58 & 34.01 & 38.87 & 29.45 & 25.64 & 25.64 & 29.66 & 20.32 & 13.33 & 15.56 & 20.63 & 28.61 & 24.40 & 25.60 & 29.52 & 35.37 & 29.27 & 30.18 & 38.72 & 25.13 & 21.20 & 23.04 & 25.65 & 28.69 & 23.89 & 24.84 & 29.69 \\
\rowcolor{gray!20}
Gemini Pro & 33.80 & 37.09 & 40.85 & 39.91 & 30.41 & 31.34 & 34.10 & 34.79 & 20.07 & 22.33 & 28.67 & 28.67 & 26.56 & 28.85 & 32.13 & 32.13 & 32.27 & 33.33 & 36.52 & 36.88 & 28.78 & 30.52 & 33.43 & 32.85 & 28.32 & 29.91 & 33.67 & 33.78 \\
Gemini Flash & 29.55 & 29.96 & 30.36 & 34.82 & 22.67 & 24.36 & 26.69 & 26.69 & 12.06 & 12.06 & 15.87 & 19.05 & 20.18 & 20.78 & 21.39 & 23.49 & 26.52 & 27.74 & 32.01 & 31.71 & 23.30 & 24.61 & 26.18 & 27.49 & 21.64 & 22.59 & 24.89 & 26.29 \\
\rowcolor{gray!20}
Claude 3.5 Sonnet & 21.86 & 19.84 & 25.91 & 24.29 & 22.46 & 19.92 & 25.21 & 25.85 & 9.21 & 6.03 & 12.38 & 11.11 & 16.87 & 14.16 & 16.87 & 18.37 & 23.17 & 20.12 & 26.52 & 24.70 & 19.11 & 15.45 & 21.47 & 22.25 & 18.74 & 15.89 & 21.44 & 21.24 \\
Qwen2 VL 72B & 25.35 & 27.23 & 33.33 & 34.27 & 18.43 & 19.12 & 23.73 & 23.73 & 9.00 & 10.00 & 16.33 & 17.00 & 17.05 & 18.36 & 22.62 & 21.64 & 25.53 & 25.53 & 32.27 & 31.91 & 16.57 & 20.93 & 23.55 & 24.42 & 18.13 & 19.62 & 24.27 & 24.65 \\
\rowcolor{gray!20}
InternVL2.5 78B & 23.94 & 29.11 & 31.92 & 31.46 & 19.12 & 24.19 & 28.11 & 29.49 & 7.33 & 12.33 & 18.67 & 19.67 & 13.44 & 22.30 & 25.90 & 25.90 & 19.50 & 24.82 & 30.14 & 29.43 & 15.41 & 21.22 & 24.42 & 26.45 & 15.75 & 21.56 & 25.98 & 26.70 \\
InternVL2.5 38B & 22.07 & 28.64 & 32.86 & 32.86 & 18.66 & 24.19 & 27.19 & 26.73 & 6.33 & 13.00 & 21.67 & 21.33 & 13.77 & 22.95 & 24.59 & 27.54 & 19.86 & 26.24 & 30.85 & 30.50 & 13.37 & 20.93 & 26.16 & 24.71 & 14.98 & 21.78 & 26.31 & 26.37 \\
\rowcolor{gray!20}
Qwen2 VL 2B & 19.72 & 18.78 & 21.13 & 23.00 & 13.13 & 14.75 & 16.59 & 15.67 & 6.67 & 4.67 & 7.00 & 6.67 & 13.11 & 11.15 & 13.44 & 12.46 & 16.67 & 16.67 & 17.38 & 17.38 & 15.70 & 15.12 & 16.28 & 16.28 & 13.88 & 13.27 & 15.26 & 14.98 \\
Qwen2 VL 7B & 18.78 & 18.78 & 22.07 & 21.60 & 14.06 & 14.06 & 17.05 & 16.59 & 5.00 & 6.00 & 7.67 & 7.33 & 13.11 & 15.08 & 17.05 & 17.70 & 15.25 & 17.73 & 18.79 & 19.50 & 15.70 & 18.60 & 19.48 & 20.06 & 13.54 & 14.76 & 16.86 & 16.92 \\
\rowcolor{gray!20}
InternVL2.5 26B & 20.66 & 25.35 & 28.64 & 29.11 & 16.36 & 19.35 & 23.27 & 24.88 & 3.33 & 7.33 & 9.33 & 10.33 & 11.80 & 15.41 & 19.67 & 20.00 & 17.73 & 21.63 & 24.82 & 24.11 & 13.66 & 18.31 & 21.51 & 22.97 & 13.32 & 17.30 & 20.78 & 21.61 \\
MiniCPM V 2.6 & 16.90 & 19.25 & 18.31 & 19.72 & 14.75 & 16.82 & 17.28 & 18.43 & 5.67 & 10.00 & 11.33 & 11.00 & 12.13 & 13.44 & 14.75 & 14.75 & 19.15 & 20.21 & 22.34 & 21.99 & 15.41 & 17.44 & 16.28 & 19.19 & 13.16 & 15.37 & 16.14 & 17.03 \\
\rowcolor{gray!20}
Phi 3.5 Vision & 16.43 & 14.55 & 16.90 & 16.90 & 13.82 & 14.06 & 17.51 & 17.28 & 8.67 & 8.33 & 10.67 & 10.33 & 9.84 & 10.16 & 11.15 & 10.82 & 15.60 & 15.25 & 19.15 & 19.86 & 13.95 & 15.12 & 18.60 & 18.90 & 12.82 & 12.88 & 16.09 & 15.87 \\
Centurio Qwen & 20.19 & 17.84 & 23.00 & 21.13 & 15.67 & 15.44 & 18.43 & 17.74 & 6.00 & 6.33 & 7.33 & 7.33 & 9.51 & 10.82 & 10.16 & 10.49 & 14.89 & 15.96 & 22.34 & 20.92 & 11.63 & 11.92 & 15.70 & 14.53 & 12.38 & 12.55 & 15.70 & 15.15 \\
\rowcolor{gray!20}
InternVL2.5 8B & 14.55 & 19.25 & 20.66 & 23.00 & 11.98 & 15.44 & 18.43 & 18.43 & 3.33 & 6.33 & 9.33 & 9.00 & 9.84 & 13.77 & 15.08 & 16.07 & 15.25 & 17.73 & 23.05 & 23.05 & 10.17 & 12.21 & 16.28 & 16.57 & 10.61 & 13.82 & 16.92 & 17.36 \\
InternVL2.5 4B & 14.55 & 15.96 & 18.78 & 18.31 & 12.67 & 14.29 & 17.74 & 16.59 & 5.67 & 6.33 & 9.00 & 9.00 & 8.52 & 9.84 & 13.11 & 12.46 & 11.35 & 15.25 & 19.50 & 18.09 & 11.34 & 14.24 & 15.70 & 15.12 & 10.45 & 12.38 & 15.70 & 14.70 \\
\rowcolor{gray!20}
Centurio Aya & 11.74 & 12.21 & 15.49 & 12.21 & 9.68 & 9.91 & 12.21 & 11.06 & 4.67 & 4.67 & 6.67 & 5.33 & 6.89 & 7.54 & 7.54 & 7.54 & 9.93 & 9.57 & 12.77 & 10.64 & 7.56 & 9.01 & 10.17 & 9.59 & 8.46 & 8.96 & 10.95 & 9.62 \\
InternVL2.5 1B & 8.45 & 9.86 & 11.27 & 12.21 & 5.76 & 8.29 & 9.22 & 7.60 & 1.67 & 2.33 & 4.00 & 2.67 & 5.90 & 7.87 & 8.85 & 8.85 & 6.74 & 7.45 & 10.99 & 10.64 & 7.27 & 8.72 & 9.01 & 9.01 & 5.86 & 7.46 & 8.90 & 8.35 \\
\midrule
Average X-Large & 24.65 & 28.17 & 32.63 & 32.86 & 18.78 & 21.66 & 25.92 & 26.61 & 8.17 & 11.17 & 17.50 & 18.33 & 15.25 & 20.33 & 24.26 & 23.77 & 22.52 & 25.18 & 31.21 & 30.67 & 15.99 & 21.08 & 23.98 & 25.44 & 16.94 & 20.59 & 25.12 & 25.68 \\
\rowcolor{gray!20}
Average Large & 21.36 & 27.00 & 30.75 & 30.99 & 17.51 & 21.77 & 25.23 & 25.81 & 4.83 & 10.17 & 15.50 & 15.83 & 12.79 & 19.18 & 22.13 & 23.77 & 18.79 & 23.94 & 27.84 & 27.30 & 13.52 & 19.62 & 23.84 & 23.84 & 14.15 & 19.54 & 23.55 & 23.99 \\
Average Medium & 16.43 & 17.46 & 19.91 & 19.53 & 13.23 & 14.33 & 16.68 & 16.45 & 4.93 & 6.67 & 8.47 & 8.00 & 10.30 & 12.13 & 12.92 & 13.31 & 14.89 & 16.24 & 19.86 & 19.22 & 12.09 & 13.84 & 15.58 & 15.99 & 11.63 & 13.09 & 15.31 & 15.21 \\
\rowcolor{gray!20}
Average Small & 14.79 & 14.79 & 17.02 & 17.61 & 11.35 & 12.85 & 15.26 & 14.29 & 5.67 & 5.42 & 7.67 & 7.17 & 9.34 & 9.75 & 11.64 & 11.15 & 12.59 & 13.65 & 16.76 & 16.49 & 12.06 & 13.30 & 14.90 & 14.83 & 10.75 & 11.50 & 13.99 & 13.47 \\
Average Open & 17.95 & 19.75 & 22.64 & 22.75 & 14.16 & 16.15 & 18.98 & 18.79 & 5.64 & 7.51 & 10.69 & 10.54 & 11.15 & 13.75 & 15.69 & 15.86 & 15.96 & 18.00 & 21.88 & 21.39 & 12.90 & 15.68 & 17.93 & 18.29 & 12.57 & 14.75 & 17.68 & 17.64 \\
\rowcolor{gray!20}
Average Proprietary & 32.45 & 31.67 & 34.49 & 36.40 & 27.91 & 27.53 & 29.47 & 31.23 & 17.06 & 16.02 & 21.83 & 23.09 & 24.28 & 24.13 & 26.54 & 27.98 & 30.98 & 30.18 & 34.48 & 35.55 & 25.66 & 24.87 & 28.61 & 29.50 & 26.01 & 25.35 & 28.80 & 30.19 \\
Average & 21.98 & 23.07 & 25.93 & 26.54 & 17.98 & 19.31 & 21.90 & 22.24 & 8.81 & 9.88 & 13.79 & 14.03 & 14.80 & 16.63 & 18.70 & 19.23 & 20.13 & 21.39 & 25.38 & 25.32 & 16.45 & 18.23 & 20.90 & 21.40 & 16.30 & 17.69 & 20.77 & 21.13 \\
\bottomrule
  \end{tabular}
  }%
  \caption{\dsname Video Visual Question Answering (VVQA) results. The reported score is relaxed accuracy. The columns \textbf{N}, \textbf{R}, \textbf{C}, and \textbf{B} stand for the hints \textbf{``None''}, \textbf{``Region''}, \textbf{``Country''}, and \textbf{``Both''}, respectively.}
  \label{tab:vvqa:scores}
\end{table}
%
\newpage
\subsection{\coqa Details}
\label{appendix:sec:analyses:coqa}
%
\subsubsection{Results}
%
\begin{table}[htbp]
  \centering
  \renewcommand{\arraystretch}{.97}
    \resizebox{\textwidth}{!}{%
    \begin{tabular}{lccc ccc ccc ccc ccc ccc | cccc}
    \toprule
     & \multicolumn{3}{c}{\textsc{West EU \& North Am.}} &
     \multicolumn{3}{c}{\textsc{East EU}} &
     \multicolumn{3}{c}{\textsc{Asia \& Pacific}} &
     \multicolumn{3}{c}{\textsc{Lat. Am. \& Carib.}} &
     \multicolumn{3}{c}{\textsc{Arab}} &
     \multicolumn{3}{c}{\textsc{Subs. Africa}} &
     \multicolumn{4}{c}{\textsc{Average}} \\
    \cmidrule(lr){2-4} \cmidrule(lr){5-7} \cmidrule(lr){8-10} \cmidrule(lr){11-13} \cmidrule(lr){14-16} \cmidrule(lr){17-19} \cmidrule(lr){20-23}
    & I & T & I+T & I & T & I+T & I & T & I+T & I & T & I+T & I & T & I+T & I & T & I+T & I & T & I+T & Avg. \\
\midrule
\rowcolor{gray!20}
GPT-4o & 82.50 & 83.75 & 85.00 & 85.89 & 90.18 & 88.34 & 94.37 & 96.54 & 97.40 & 93.68 & 92.63 & 92.63 & 88.00 & 88.00 & 92.00 & 91.30 & 91.30 & 94.20 & 89.29 & 90.40 & 91.60 & 90.43 \\
Claude 3.5 Sonnet & 72.50 & 83.75 & 81.25 & 76.69 & 85.89 & 82.21 & 87.88 & 95.67 & 95.67 & 83.16 & 89.47 & 87.37 & 84.00 & 90.67 & 90.67 & 82.61 & 89.86 & 88.41 & 81.14 & 89.22 & 87.60 & 85.98 \\
\rowcolor{gray!20}
InternVL2.5 78B & 77.50 & 80.00 & 86.88 & 83.44 & 82.21 & 88.96 & 94.37 & 94.81 & 96.97 & 88.42 & 92.63 & 92.63 & 88.00 & 90.67 & 92.00 & 92.75 & 91.30 & 92.75 & 87.41 & 88.60 & 91.70 & 89.24 \\
Qwen2.5 72B & -- & 81.25 & -- & -- & 84.05 & -- & -- & 96.10 & -- & -- & 89.47 & -- & -- & 86.67 & -- & -- & 89.86 & -- & -- & 87.90 & -- & 87.90 \\
\rowcolor{gray!20}
GPT-4o Mini & 76.25 & 82.50 & 86.25 & 84.66 & 82.21 & 84.66 & 94.37 & 95.67 & 96.54 & 87.37 & 90.53 & 90.53 & 85.33 & 86.67 & 86.67 & 91.30 & 89.86 & 92.75 & 86.55 & 87.90 & 89.57 & 88.01 \\
InternVL2.5 38B & 81.25 & 81.25 & 84.38 & 85.89 & 84.66 & 85.28 & 90.04 & 95.24 & 92.64 & 86.32 & 86.32 & 92.63 & 89.33 & 90.67 & 92.00 & 89.86 & 86.96 & 91.30 & 87.11 & 87.51 & 89.70 & 88.11 \\
\rowcolor{gray!20}
Qwen2 VL 72B & 79.38 & 80.62 & 81.25 & 88.34 & 84.66 & 88.34 & 90.48 & 94.81 & 96.97 & 86.32 & 88.42 & 92.63 & 86.67 & 88.00 & 89.33 & 91.30 & 85.51 & 91.30 & 87.08 & 87.00 & 89.97 & 88.02 \\
Gemini Flash & 82.50 & 78.75 & 78.13 & 85.28 & 80.37 & 84.66 & 87.01 & 91.34 & 94.81 & 85.11 & 87.37 & 90.53 & 89.19 & 86.67 & 90.67 & 89.86 & 91.30 & 91.30 & 86.49 & 85.97 & 88.35 & 86.94 \\
\rowcolor{gray!20}
Qwen2.5 32B & -- & 76.88 & -- & -- & 79.75 & -- & -- & 94.37 & -- & -- & 87.37 & -- & -- & 84.00 & -- & -- & 89.86 & -- & -- & 85.37 & -- & 85.37 \\
Qwen2 VL 7B & 71.25 & 74.38 & 76.25 & 82.82 & 80.37 & 84.05 & 92.64 & 93.51 & 93.51 & 85.26 & 88.42 & 92.63 & 80.00 & 82.67 & 84.00 & 86.96 & 85.51 & 84.06 & 83.16 & 84.14 & 85.75 & 84.35 \\
\rowcolor{gray!20}
MiniCPM V 2.6 & 72.50 & 72.50 & 75.00 & 81.60 & 79.14 & 80.37 & 88.74 & 90.48 & 93.07 & 80.00 & 87.37 & 90.53 & 80.00 & 77.33 & 86.67 & 88.41 & 85.51 & 86.96 & 81.87 & 82.05 & 85.43 & 83.12 \\
InternVL2.5 26B & 77.50 & 74.38 & 80.62 & 87.12 & 75.46 & 87.12 & 91.77 & 91.77 & 96.54 & 88.42 & 84.21 & 93.68 & 84.00 & 85.33 & 88.00 & 91.30 & 79.71 & 86.96 & 86.69 & 81.81 & 88.82 & 85.77 \\
\rowcolor{gray!20}
Phi 3.5 Mini & -- & 74.38 & -- & -- & 72.39 & -- & -- & 88.31 & -- & -- & 83.16 & -- & -- & 81.33 & -- & -- & 86.96 & -- & -- & 81.09 & -- & 81.09 \\
InternLM2.5 7B & -- & 74.38 & -- & -- & 76.69 & -- & -- & 90.48 & -- & -- & 80.00 & -- & -- & 78.67 & -- & -- & 85.51 & -- & -- & 80.95 & -- & 80.95 \\
\rowcolor{gray!20}
Centurio Qwen & 75.63 & 74.38 & 80.00 & 79.75 & 76.69 & 82.82 & 86.58 & 92.64 & 92.21 & 83.16 & 86.32 & 89.47 & 78.67 & 77.33 & 88.00 & 86.96 & 76.81 & 89.86 & 81.79 & 80.69 & 87.06 & 83.18 \\
InternLM2.5 20B & -- & 74.38 & -- & -- & 75.46 & -- & -- & 89.18 & -- & -- & 86.32 & -- & -- & 76.00 & -- & -- & 82.61 & -- & -- & 80.66 & -- & 80.66 \\
\rowcolor{gray!20}
Qwen2.5 7B & -- & 71.88 & -- & -- & 72.39 & -- & -- & 93.51 & -- & -- & 85.26 & -- & -- & 77.33 & -- & -- & 81.16 & -- & -- & 80.26 & -- & 80.26 \\
Aya Expanse 8B & -- & 68.12 & -- & -- & 77.30 & -- & -- & 91.77 & -- & -- & 81.05 & -- & -- & 80.00 & -- & -- & 81.16 & -- & -- & 79.90 & -- & 79.90 \\
\rowcolor{gray!20}
InternVL2.5 8B & 68.12 & 72.50 & 75.63 & 83.44 & 76.07 & 83.44 & 87.88 & 89.61 & 94.37 & 84.21 & 83.16 & 92.63 & 84.00 & 73.33 & 89.33 & 88.41 & 81.16 & 92.75 & 82.68 & 79.31 & 88.03 & 83.34 \\
Centurio Aya & 80.62 & 68.12 & 78.75 & 82.21 & 75.46 & 80.37 & 90.91 & 85.71 & 92.21 & 84.21 & 82.11 & 85.26 & 81.33 & 82.67 & 85.33 & 85.51 & 81.16 & 91.30 & 84.13 & 79.21 & 85.54 & 82.96 \\
\rowcolor{gray!20}
Phi 3.5 Vision & 65.62 & 72.50 & 75.63 & 69.94 & 70.55 & 76.69 & 89.18 & 91.34 & 95.24 & 80.00 & 81.05 & 86.32 & 72.00 & 80.00 & 86.67 & 85.51 & 79.71 & 88.41 & 77.04 & 79.19 & 84.82 & 80.35 \\
InternVL2.5 4B & 66.88 & 66.88 & 76.25 & 84.66 & 75.46 & 84.05 & 87.01 & 86.15 & 93.07 & 83.16 & 78.95 & 87.37 & 80.00 & 82.67 & 86.67 & 86.96 & 84.06 & 89.86 & 81.44 & 79.03 & 86.21 & 82.23 \\
\rowcolor{gray!20}
Qwen2 VL 2B & 77.50 & 72.50 & 78.75 & 84.05 & 64.42 & 84.05 & 91.77 & 82.68 & 92.21 & 88.42 & 81.05 & 86.32 & 84.00 & 70.67 & 89.33 & 88.41 & 79.71 & 91.30 & 85.69 & 75.17 & 86.99 & 82.62 \\
Qwen2.5 3B & -- & 68.75 & -- & -- & 73.01 & -- & -- & 83.12 & -- & -- & 73.68 & -- & -- & 74.67 & -- & -- & 75.36 & -- & -- & 74.76 & -- & 74.76 \\
\rowcolor{gray!20}
Qwen2.5 1.5B & -- & 61.88 & -- & -- & 65.03 & -- & -- & 82.25 & -- & -- & 78.95 & -- & -- & 72.00 & -- & -- & 78.26 & -- & -- & 73.06 & -- & 73.06 \\
Qwen2.5 0.5B & -- & 68.12 & -- & -- & 72.39 & -- & -- & 67.53 & -- & -- & 65.26 & -- & -- & 70.67 & -- & -- & 55.07 & -- & -- & 66.51 & -- & 66.51 \\
\rowcolor{gray!20}
InternLM2.5 1.8B & -- & 56.25 & -- & -- & 65.03 & -- & -- & 65.37 & -- & -- & 60.00 & -- & -- & 66.67 & -- & -- & 66.67 & -- & -- & 63.33 & -- & 63.33 \\
InternVL2.5 2B & 70.62 & 51.88 & 72.50 & 76.69 & 58.28 & 71.78 & 77.92 & 72.29 & 82.68 & 83.16 & 62.11 & 83.16 & 73.33 & 66.67 & 82.67 & 84.06 & 60.87 & 89.86 & 77.63 & 62.02 & 80.44 & 73.36 \\
\rowcolor{gray!20}
InternVL2.5 1B & 63.75 & 58.75 & 66.88 & 62.58 & 60.74 & 74.23 & 64.50 & 61.90 & 80.09 & 77.89 & 57.89 & 87.37 & 62.67 & 68.00 & 82.67 & 75.36 & 59.42 & 82.61 & 67.79 & 61.12 & 78.97 & 69.29 \\
Gemini Pro & 76.25 & 59.38 & 78.13 & 68.10 & 55.21 & 82.21 & 82.25 & 56.28 & 89.61 & 79.79 & 61.05 & 85.11 & 79.73 & 61.33 & 84.00 & 72.46 & 65.22 & 95.65 & 76.43 & 59.75 & 85.78 & 73.99 \\
\midrule
Average X-Large LVLMs & 78.44 & 80.31 & 84.06 & 85.89 & 83.44 & 88.65 & 92.42 & 94.81 & 96.97 & 87.37 & 90.53 & 92.63 & 87.33 & 89.33 & 90.67 & 92.03 & 88.41 & 92.03 & 87.24 & 87.80 & 90.84 & 88.63 \\
\rowcolor{gray!20}
Average Large LVLMs & 79.38 & 77.81 & 82.50 & 86.50 & 80.06 & 86.20 & 90.91 & 93.51 & 94.59 & 87.37 & 85.26 & 93.16 & 86.67 & 88.00 & 90.00 & 90.58 & 83.33 & 89.13 & 86.90 & 84.66 & 89.26 & 86.94 \\
Average Medium LVLMs & 73.62 & 72.38 & 77.12 & 81.96 & 77.55 & 82.21 & 89.35 & 90.39 & 93.07 & 83.37 & 85.47 & 90.11 & 80.80 & 78.67 & 86.67 & 87.25 & 82.03 & 88.99 & 82.73 & 81.08 & 86.36 & 83.39 \\
\rowcolor{gray!20}
Average Small LVLMs & 68.88 & 64.50 & 74.00 & 75.58 & 65.89 & 78.16 & 82.08 & 78.87 & 88.66 & 82.53 & 72.21 & 86.11 & 74.40 & 73.60 & 85.60 & 84.06 & 72.75 & 88.41 & 77.92 & 71.31 & 83.49 & 77.57 \\
Average LVLMs & 73.44 & 71.47 & 77.77 & 80.89 & 74.58 & 82.25 & 87.41 & 87.35 & 92.27 & 84.21 & 81.43 & 89.47 & 80.29 & 79.71 & 87.33 & 87.27 & 79.81 & 89.23 & 82.25 & 79.06 & 86.39 & 82.57 \\
\rowcolor{gray!20}
Average X-Large LLMs & -- & 81.25 & -- & -- & 84.05 & -- & -- & 96.10 & -- & -- & 89.47 & -- & -- & 86.67 & -- & -- & 89.86 & -- & -- & 87.90 & -- & 87.90 \\
Average Large LLMs & -- & 75.62 & -- & -- & 77.61 & -- & -- & 91.77 & -- & -- & 86.84 & -- & -- & 80.00 & -- & -- & 86.23 & -- & -- & 83.02 & -- & 83.02 \\
\rowcolor{gray!20}
Average Medium LLMs & -- & 71.46 & -- & -- & 75.46 & -- & -- & 91.92 & -- & -- & 82.11 & -- & -- & 78.67 & -- & -- & 82.61 & -- & -- & 80.37 & -- & 80.37 \\
Average Small LLMs & -- & 65.88 & -- & -- & 69.57 & -- & -- & 77.32 & -- & -- & 72.21 & -- & -- & 73.07 & -- & -- & 72.46 & -- & -- & 71.75 & -- & 71.75 \\
\rowcolor{gray!20}
Average LLMs & -- & 70.57 & -- & -- & 73.95 & -- & -- & 85.64 & -- & -- & 79.14 & -- & -- & 77.09 & -- & -- & 79.31 & -- & -- & 77.62 & -- & 77.62 \\
Average X-Large & 78.44 & 80.62 & 84.06 & 85.89 & 83.64 & 88.65 & 92.42 & 95.24 & 96.97 & 87.37 & 90.18 & 92.63 & 87.33 & 88.44 & 90.67 & 92.03 & 88.89 & 92.03 & 87.24 & 87.83 & 90.84 & 88.39 \\
\rowcolor{gray!20}
Average Large & 79.38 & 76.72 & 82.50 & 86.50 & 78.83 & 86.20 & 90.91 & 92.64 & 94.59 & 87.37 & 86.05 & 93.16 & 86.67 & 84.00 & 90.00 & 90.58 & 84.78 & 89.13 & 86.90 & 83.84 & 89.26 & 84.98 \\
Average Medium & 73.62 & 72.03 & 77.12 & 81.96 & 76.76 & 82.21 & 89.35 & 90.96 & 93.07 & 83.37 & 84.21 & 90.11 & 80.80 & 78.67 & 86.67 & 87.25 & 82.25 & 88.99 & 82.73 & 80.81 & 86.36 & 82.26 \\
\rowcolor{gray!20}
Average Small & 68.88 & 65.19 & 74.00 & 75.58 & 67.73 & 78.16 & 82.08 & 78.10 & 88.66 & 82.53 & 72.21 & 86.11 & 74.40 & 73.33 & 85.60 & 84.06 & 72.61 & 88.41 & 77.92 & 71.53 & 83.49 & 74.66 \\
Average Open & 73.44 & 71.08 & 77.77 & 80.89 & 74.31 & 82.25 & 87.41 & 86.60 & 92.27 & 84.21 & 80.42 & 89.47 & 80.29 & 78.56 & 87.33 & 87.27 & 79.59 & 89.23 & 82.25 & 78.43 & 86.39 & 80.39 \\
\rowcolor{gray!20}
Average Proprietary & 78.00 & 77.62 & 81.75 & 80.12 & 78.77 & 84.42 & 89.18 & 87.10 & 94.81 & 85.82 & 84.21 & 89.23 & 85.25 & 82.67 & 88.80 & 85.51 & 85.51 & 92.46 & 83.98 & 82.65 & 88.58 & 85.07 \\
Average & 74.64 & 72.17 & 78.82 & 80.69 & 75.05 & 82.82 & 87.88 & 86.68 & 92.94 & 84.63 & 81.05 & 89.41 & 81.59 & 79.24 & 87.72 & 86.80 & 80.58 & 90.08 & 82.71 & 79.13 & 86.96 & 81.17 \\
\bottomrule
    \end{tabular}
    }%
  \caption{\dsname Cultural Origin Question Answering -- Regions (\coqar) results. The reported score is relaxed accuracy. The columns \textbf{I} and \textbf{T} stand for \textbf{image-only} and \textbf{text-only} inputs to the model.}
  \label{tab:coqa-regions:scores}
\end{table}
%
%

\begin{table}[htbp]
  \centering
  \renewcommand{\arraystretch}{.97}
    \resizebox{\textwidth}{!}{%
    \begin{tabular}{lccc ccc ccc ccc ccc ccc | cccc}
    \toprule
     & \multicolumn{3}{c}{\textsc{West EU \& North Am.}} &
     \multicolumn{3}{c}{\textsc{East EU}} &
     \multicolumn{3}{c}{\textsc{Asia \& Pacific}} &
     \multicolumn{3}{c}{\textsc{Lat. Am. \& Carib.}} &
     \multicolumn{3}{c}{\textsc{Arab}} &
     \multicolumn{3}{c}{\textsc{Subs. Africa}} &
     \multicolumn{4}{c}{\textsc{Average}} \\
    \cmidrule(lr){2-4} \cmidrule(lr){5-7} \cmidrule(lr){8-10} \cmidrule(lr){11-13} \cmidrule(lr){14-16} \cmidrule(lr){17-19} \cmidrule(lr){20-23}
    & I & T & I+T & I & T & I+T & I & T & I+T & I & T & I+T & I & T & I+T & I & T & I+T & I & T & I+T & Avg. \\
\midrule
\rowcolor{gray!20}
Claude 3.5 Sonnet & 79.23 & 96.72 & 95.63 & 82.35 & 97.65 & 96.47 & 76.62 & 97.84 & 95.67 & 70.21 & 98.94 & 100.00 & 76.47 & 97.65 & 96.47 & 83.82 & 97.06 & 91.18 & 78.12 & 97.64 & 95.90 & 90.55 \\
GPT-4o & 93.44 & 95.08 & 96.17 & 94.71 & 98.24 & 98.24 & 93.51 & 97.40 & 98.27 & 97.87 & 98.94 & 98.94 & 95.29 & 95.29 & 98.82 & 95.59 & 97.06 & 100.00 & 95.07 & 97.00 & 98.41 & 96.83 \\
\rowcolor{gray!20}
InternVL2.5 78B & 83.06 & 94.54 & 97.81 & 80.59 & 95.88 & 97.65 & 83.12 & 93.51 & 96.54 & 81.91 & 98.94 & 98.94 & 90.59 & 97.65 & 97.65 & 83.82 & 97.06 & 98.53 & 83.85 & 96.26 & 97.85 & 92.65 \\
Qwen2.5 72B & -- & 93.44 & -- & -- & 96.47 & -- & -- & 94.81 & -- & -- & 98.94 & -- & -- & 97.65 & -- & -- & 94.12 & -- & -- & 95.90 & -- & 95.90 \\
\rowcolor{gray!20}
GPT-4o Mini & 89.07 & 93.99 & 95.63 & 90.00 & 95.29 & 97.65 & 90.48 & 93.51 & 96.97 & 90.43 & 95.74 & 100.00 & 94.12 & 88.24 & 97.65 & 91.18 & 95.59 & 98.53 & 90.88 & 93.73 & 97.74 & 94.11 \\
Qwen2.5 32B & -- & 91.26 & -- & -- & 93.53 & -- & -- & 91.77 & -- & -- & 94.68 & -- & -- & 95.29 & -- & -- & 92.65 & -- & -- & 93.20 & -- & 93.20 \\
\rowcolor{gray!20}
InternVL2.5 38B & 78.69 & 91.80 & 92.35 & 77.06 & 91.18 & 92.94 & 77.49 & 93.07 & 93.94 & 79.79 & 95.74 & 96.81 & 84.71 & 94.12 & 95.29 & 88.24 & 92.65 & 98.53 & 80.99 & 93.09 & 94.98 & 89.69 \\
Qwen2 VL 72B & 87.98 & 87.43 & 95.08 & 94.12 & 90.59 & 96.47 & 90.04 & 90.04 & 97.84 & 91.49 & 97.87 & 98.94 & 92.94 & 89.41 & 98.82 & 91.18 & 97.06 & 98.53 & 91.29 & 92.07 & 97.61 & 93.66 \\
\rowcolor{gray!20}
Gemini Flash & 90.56 & 89.01 & 97.27 & 90.59 & 88.82 & 97.06 & 91.77 & 90.48 & 98.70 & 90.43 & 93.62 & 97.87 & 90.59 & 88.24 & 97.65 & 88.24 & 95.59 & 97.06 & 90.36 & 90.96 & 97.60 & 92.97 \\
InternVL2.5 26B & 78.14 & 87.98 & 92.90 & 78.24 & 88.24 & 94.71 & 76.19 & 90.48 & 93.94 & 80.85 & 94.68 & 94.68 & 81.18 & 91.76 & 91.76 & 80.88 & 91.18 & 95.59 & 79.25 & 90.72 & 93.93 & 87.96 \\
\rowcolor{gray!20}
Qwen2.5 7B & -- & 86.34 & -- & -- & 88.24 & -- & -- & 85.28 & -- & -- & 95.74 & -- & -- & 90.59 & -- & -- & 94.12 & -- & -- & 90.05 & -- & 90.05 \\
Aya Expanse 8B & -- & 87.43 & -- & -- & 88.24 & -- & -- & 90.04 & -- & -- & 93.62 & -- & -- & 88.24 & -- & -- & 89.71 & -- & -- & 89.54 & -- & 89.54 \\
\rowcolor{gray!20}
InternLM2.5 20B & -- & 86.89 & -- & -- & 87.06 & -- & -- & 90.91 & -- & -- & 90.43 & -- & -- & 85.88 & -- & -- & 89.71 & -- & -- & 88.48 & -- & 88.48 \\
MiniCPM V 2.6 & 81.97 & 84.70 & 90.16 & 81.18 & 86.47 & 92.94 & 78.79 & 87.45 & 92.21 & 86.17 & 87.23 & 92.55 & 82.35 & 89.41 & 96.47 & 88.24 & 92.65 & 94.12 & 83.12 & 87.98 & 93.08 & 88.06 \\
\rowcolor{gray!20}
Qwen2 VL 7B & 87.43 & 83.61 & 90.71 & 82.35 & 85.29 & 92.94 & 87.01 & 84.85 & 94.37 & 91.49 & 88.30 & 94.68 & 84.71 & 88.24 & 96.47 & 92.65 & 94.12 & 97.06 & 87.61 & 87.40 & 94.37 & 89.79 \\
Qwen2.5 3B & -- & 81.42 & -- & -- & 85.88 & -- & -- & 84.85 & -- & -- & 92.55 & -- & -- & 88.24 & -- & -- & 86.76 & -- & -- & 86.62 & -- & 86.62 \\
\rowcolor{gray!20}
InternLM2.5 7B & -- & 83.61 & -- & -- & 85.88 & -- & -- & 85.71 & -- & -- & 90.43 & -- & -- & 77.65 & -- & -- & 88.24 & -- & -- & 85.25 & -- & 85.25 \\
Centurio Qwen & 78.69 & 82.51 & 89.07 & 78.82 & 82.94 & 89.41 & 78.79 & 84.42 & 92.64 & 76.60 & 85.11 & 91.49 & 80.00 & 83.53 & 88.24 & 79.41 & 91.18 & 92.65 & 78.72 & 84.95 & 90.58 & 84.75 \\
\rowcolor{gray!20}
Centurio Aya & 65.57 & 83.61 & 85.79 & 72.35 & 81.76 & 85.88 & 75.76 & 85.71 & 88.31 & 74.47 & 87.23 & 82.98 & 70.59 & 80.00 & 89.41 & 66.18 & 89.71 & 89.71 & 70.82 & 84.67 & 87.01 & 80.83 \\
InternVL2.5 8B & 68.31 & 82.51 & 88.52 & 70.59 & 84.71 & 90.00 & 75.32 & 86.58 & 91.34 & 75.53 & 87.23 & 94.68 & 76.47 & 83.53 & 90.59 & 82.35 & 82.35 & 89.71 & 74.76 & 84.49 & 90.81 & 83.35 \\
\rowcolor{gray!20}
Phi 3.5 Mini & -- & 80.87 & -- & -- & 82.94 & -- & -- & 83.98 & -- & -- & 84.04 & -- & -- & 82.35 & -- & -- & 88.24 & -- & -- & 83.74 & -- & 83.74 \\
InternVL2.5 4B & 68.85 & 77.05 & 89.62 & 72.35 & 82.94 & 89.41 & 71.43 & 86.15 & 90.48 & 76.60 & 87.23 & 89.36 & 72.94 & 81.18 & 84.71 & 76.47 & 82.35 & 97.06 & 73.11 & 82.82 & 90.11 & 82.01 \\
\rowcolor{gray!20}
Phi 3.5 Vision & 72.13 & 79.78 & 86.89 & 68.82 & 82.94 & 92.35 & 69.70 & 81.82 & 89.61 & 74.47 & 91.49 & 91.49 & 81.18 & 77.65 & 90.59 & 76.47 & 82.35 & 95.59 & 73.79 & 82.67 & 91.09 & 82.52 \\
Qwen2.5 1.5B & -- & 78.69 & -- & -- & 81.18 & -- & -- & 82.68 & -- & -- & 82.98 & -- & -- & 75.29 & -- & -- & 80.88 & -- & -- & 80.28 & -- & 80.28 \\
\rowcolor{gray!20}
Qwen2 VL 2B & 83.06 & 74.32 & 87.43 & 84.71 & 77.06 & 87.65 & 83.55 & 80.95 & 90.48 & 92.55 & 81.91 & 94.68 & 83.53 & 76.47 & 91.76 & 89.71 & 80.88 & 94.12 & 86.18 & 78.60 & 91.02 & 85.27 \\
Qwen2.5 0.5B & -- & 65.03 & -- & -- & 68.82 & -- & -- & 72.29 & -- & -- & 75.53 & -- & -- & 69.41 & -- & -- & 77.94 & -- & -- & 71.51 & -- & 71.51 \\
\rowcolor{gray!20}
InternVL2.5 1B & 61.20 & 66.12 & 73.77 & 59.41 & 65.88 & 73.53 & 62.34 & 75.76 & 77.06 & 67.02 & 74.47 & 76.60 & 56.47 & 63.53 & 75.29 & 55.88 & 72.06 & 70.59 & 60.39 & 69.64 & 74.47 & 68.17 \\
InternLM2.5 1.8B & -- & 63.39 & -- & -- & 66.47 & -- & -- & 71.00 & -- & -- & 67.02 & -- & -- & 58.82 & -- & -- & 64.71 & -- & -- & 65.23 & -- & 65.23 \\
\rowcolor{gray!20}
InternVL2.5 2B & 62.84 & 65.57 & 74.32 & 61.76 & 64.71 & 72.35 & 61.04 & 68.40 & 80.95 & 67.02 & 68.09 & 80.85 & 67.06 & 55.29 & 74.12 & 73.53 & 63.24 & 77.94 & 65.54 & 64.22 & 76.76 & 68.84 \\
Gemini Pro & 76.67 & 43.17 & 92.70 & 75.88 & 39.41 & 92.94 & 78.79 & 34.20 & 92.64 & 78.72 & 39.36 & 93.62 & 78.82 & 35.29 & 91.76 & 82.35 & 19.12 & 94.12 & 78.54 & 35.09 & 92.96 & 68.86 \\
\midrule
Average X-Large LVLMs & 85.52 & 90.98 & 96.45 & 87.35 & 93.24 & 97.06 & 86.58 & 91.77 & 97.19 & 86.70 & 98.40 & 98.94 & 91.76 & 93.53 & 98.24 & 87.50 & 97.06 & 98.53 & 87.57 & 94.16 & 97.73 & 93.16 \\
\rowcolor{gray!20}
Average Large LVLMs & 78.42 & 89.89 & 92.62 & 77.65 & 89.71 & 93.82 & 76.84 & 91.77 & 93.94 & 80.32 & 95.21 & 95.74 & 82.94 & 92.94 & 93.53 & 84.56 & 91.91 & 97.06 & 80.12 & 91.90 & 94.46 & 88.82 \\
Average Medium LVLMs & 76.39 & 83.39 & 88.85 & 77.06 & 84.24 & 90.24 & 79.13 & 85.80 & 91.77 & 80.85 & 87.02 & 91.28 & 78.82 & 84.94 & 92.24 & 81.76 & 90.00 & 92.65 & 79.01 & 85.90 & 91.17 & 85.36 \\
\rowcolor{gray!20}
Average Small LVLMs & 69.62 & 72.57 & 82.40 & 69.41 & 74.71 & 83.06 & 69.61 & 78.61 & 85.71 & 75.53 & 80.64 & 86.60 & 72.24 & 70.82 & 83.29 & 74.41 & 76.18 & 87.06 & 71.80 & 75.59 & 84.69 & 77.36 \\
Average LVLMs & 75.57 & 81.54 & 88.17 & 75.88 & 82.90 & 89.16 & 76.47 & 84.94 & 90.69 & 79.71 & 87.54 & 91.34 & 78.91 & 82.27 & 90.08 & 80.36 & 86.34 & 92.12 & 77.82 & 84.26 & 90.26 & 84.11 \\
\rowcolor{gray!20}
Average X-Large LLMs & -- & 93.44 & -- & -- & 96.47 & -- & -- & 94.81 & -- & -- & 98.94 & -- & -- & 97.65 & -- & -- & 94.12 & -- & -- & 95.90 & -- & 95.90 \\
Average Large LLMs & -- & 89.07 & -- & -- & 90.29 & -- & -- & 91.34 & -- & -- & 92.55 & -- & -- & 90.59 & -- & -- & 91.18 & -- & -- & 90.84 & -- & 90.84 \\
\rowcolor{gray!20}
Average Medium LLMs & -- & 85.79 & -- & -- & 87.45 & -- & -- & 87.01 & -- & -- & 93.26 & -- & -- & 85.49 & -- & -- & 90.69 & -- & -- & 88.28 & -- & 88.28 \\
Average Small LLMs & -- & 73.88 & -- & -- & 77.06 & -- & -- & 78.96 & -- & -- & 80.43 & -- & -- & 74.82 & -- & -- & 79.71 & -- & -- & 77.48 & -- & 77.48 \\
\rowcolor{gray!20}
Average LLMs & -- & 81.67 & -- & -- & 84.06 & -- & -- & 84.85 & -- & -- & 87.81 & -- & -- & 82.67 & -- & -- & 86.10 & -- & -- & 84.53 & -- & 84.53 \\
Average X-Large & 85.52 & 91.80 & 96.45 & 87.35 & 94.31 & 97.06 & 86.58 & 92.78 & 97.19 & 86.70 & 98.58 & 98.94 & 91.76 & 94.90 & 98.24 & 87.50 & 96.08 & 98.53 & 87.57 & 94.74 & 97.73 & 94.07 \\
\rowcolor{gray!20}
Average Large & 78.42 & 89.48 & 92.62 & 77.65 & 90.00 & 93.82 & 76.84 & 91.56 & 93.94 & 80.32 & 93.88 & 95.74 & 82.94 & 91.76 & 93.53 & 84.56 & 91.54 & 97.06 & 80.12 & 91.37 & 94.46 & 89.83 \\
Average Medium & 76.39 & 84.29 & 88.85 & 77.06 & 85.44 & 90.24 & 79.13 & 86.26 & 91.77 & 80.85 & 89.36 & 91.28 & 78.82 & 85.15 & 92.24 & 81.76 & 90.26 & 92.65 & 79.01 & 86.79 & 91.17 & 86.45 \\
\rowcolor{gray!20}
Average Small & 69.62 & 73.22 & 82.40 & 69.41 & 75.88 & 83.06 & 69.61 & 78.79 & 85.71 & 75.53 & 80.53 & 86.60 & 72.24 & 72.82 & 83.29 & 74.41 & 77.94 & 87.06 & 71.80 & 76.53 & 84.69 & 77.42 \\
Average Open & 75.57 & 81.60 & 88.17 & 75.88 & 83.41 & 89.16 & 76.47 & 84.90 & 90.69 & 79.71 & 87.66 & 91.34 & 78.91 & 82.45 & 90.08 & 80.36 & 86.24 & 92.12 & 77.82 & 84.38 & 90.26 & 84.29 \\
\rowcolor{gray!20}
Average Proprietary & 85.79 & 83.59 & 95.48 & 86.71 & 83.88 & 96.47 & 86.23 & 82.68 & 96.45 & 85.53 & 85.32 & 98.09 & 87.06 & 80.94 & 96.47 & 88.24 & 80.88 & 96.18 & 86.59 & 82.88 & 96.52 & 88.66 \\
Average & 78.26 & 81.93 & 90.10 & 78.73 & 83.49 & 91.08 & 79.04 & 84.53 & 92.21 & 81.24 & 87.27 & 93.11 & 81.05 & 82.20 & 91.76 & 82.43 & 85.34 & 93.19 & 80.13 & 84.13 & 91.91 & 85.02 \\
\bottomrule
    \end{tabular}
    }%
  \caption{\dsname Cultural Origin Question Answering -- Country (\coqac) results. The reported score is relaxed accuracy. The columns \textbf{I} and \textbf{T} stand for \textbf{image-only} and \textbf{text-only} inputs to the model.}
  \label{tab:coqa-countries:scores}
\end{table}
%

%
\newpage
\subsection{\ckqa}
\label{appendix:sec:analyses:ckqa}
%
%
\subsubsection{LLM-as-a-Judge Evaluation}
\label{appendix:sec:analyses:ckqa:judge}
%
To evaluate the \ckqad and \ckqan tasks, we used GPT-4o (\texttt{gpt-4o-2024-11-20}) as a judge using the prompts shown in the next section.
%
For each sample, we used the same system prompt and generated user prompts per sample individually.
%
\input{src/992_4_appendix_results_ckqa_prompt}

%
\subsubsection{Results}
\label{appendix:sec:analyses:ckqa:results}
%
\begin{table}[htbp]
  \centering
  \renewcommand{\arraystretch}{.97}
    \resizebox{\textwidth}{!}{
    \begin{tabular}{lccc ccc ccc ccc ccc ccc | cccc}
    \toprule
     & \multicolumn{3}{c}{\textsc{West EU \& North Am.}} &
     \multicolumn{3}{c}{\textsc{East EU}} &
     \multicolumn{3}{c}{\textsc{Asia \& Pacific}} &
     \multicolumn{3}{c}{\textsc{Lat. Am. \& Carib.}} &
     \multicolumn{3}{c}{\textsc{Arab}} &
     \multicolumn{3}{c}{\textsc{Subs. Africa}} &
     \multicolumn{4}{c}{\textsc{Average}} \\
    \cmidrule(lr){2-4} \cmidrule(lr){5-7} \cmidrule(lr){8-10} \cmidrule(lr){11-13} \cmidrule(lr){14-16} \cmidrule(lr){17-19} \cmidrule(lr){20-23}
    & I & T & I+T & I & T & I+T & I & T & I+T & I & T & I+T & I & T & I+T & I & T & I+T & I & T & I+T & Avg. \\
    \midrule
\rowcolor{gray!20}
GPT-4o & 46.98 & 56.78 & 57.21 & 38.20 & 54.30 & 54.87 & 44.71 & 59.20 & 58.56 & 34.08 & 51.53 & 52.45 & 44.41 & 57.76 & 56.91 & 29.04 & 50.68 & 51.37 & 39.57 & 55.04 & 55.23 & 49.95 \\
Claude 3.5 Sonnet & 43.05 & 56.64 & 55.60 & 35.20 & 55.97 & 50.07 & 39.54 & 59.67 & 54.05 & 26.84 & 53.32 & 49.23 & 41.45 & 56.78 & 53.68 & 24.73 & 50.34 & 44.79 & 35.14 & 55.45 & 51.24 & 47.28 \\
\rowcolor{gray!20}
Gemini Pro & 42.28 & 53.29 & 57.21 & 36.80 & 50.07 & 53.67 & 37.47 & 52.94 & 55.18 & 29.18 & 49.44 & 50.00 & 38.68 & 48.68 & 54.08 & 22.05 & 41.23 & 46.23 & 34.41 & 49.28 & 52.73 & 45.47 \\
Qwen2.5 72B & -- & 47.55 & -- & -- & 45.17 & -- & -- & 50.62 & -- & -- & 42.70 & -- & -- & 44.47 & -- & -- & 37.95 & -- & -- & 44.74 & -- & 44.74 \\
\rowcolor{gray!20}
Qwen2.5 32B & -- & 47.89 & -- & -- & 43.73 & -- & -- & 48.45 & -- & -- & 40.71 & -- & -- & 42.17 & -- & -- & 39.25 & -- & -- & 43.70 & -- & 43.70 \\
GPT-4o Mini & 34.36 & 48.89 & 55.70 & 27.70 & 46.63 & 54.00 & 30.73 & 49.05 & 53.61 & 24.95 & 43.72 & 49.44 & 36.84 & 47.50 & 54.21 & 21.03 & 39.25 & 46.78 & 29.27 & 45.84 & 52.29 & 42.47 \\
\rowcolor{gray!20}
Gemini Flash & 36.54 & 52.75 & 54.70 & 29.67 & 46.87 & 51.30 & 31.78 & 50.40 & 51.62 & 23.20 & 46.07 & 49.07 & 32.43 & 46.64 & 51.45 & 16.44 & 36.37 & 42.12 & 28.34 & 46.52 & 50.04 & 41.63 \\
Phi 3.5 Mini & -- & 40.40 & -- & -- & 35.23 & -- & -- & 38.27 & -- & -- & 34.80 & -- & -- & 34.87 & -- & -- & 30.00 & -- & -- & 35.60 & -- & 35.60 \\
\rowcolor{gray!20}
Aya Expanse 8B & -- & 40.17 & -- & -- & 36.13 & -- & -- & 39.42 & -- & -- & 34.18 & -- & -- & 36.32 & -- & -- & 26.71 & -- & -- & 35.49 & -- & 35.49 \\
Qwen2.5 7B & -- & 38.39 & -- & -- & 36.50 & -- & -- & 38.78 & -- & -- & 34.23 & -- & -- & 34.01 & -- & -- & 29.04 & -- & -- & 35.16 & -- & 35.16 \\
\rowcolor{gray!20}
InternLM2.5 20B & -- & 37.01 & -- & -- & 34.13 & -- & -- & 36.59 & -- & -- & 31.17 & -- & -- & 32.83 & -- & -- & 27.53 & -- & -- & 33.21 & -- & 33.21 \\
Llama 3.2 11B Vision & -- & 36.44 & -- & -- & 32.77 & -- & -- & 35.75 & -- & -- & 30.00 & -- & -- & 33.68 & -- & -- & 27.40 & -- & -- & 32.67 & -- & 32.67 \\
\rowcolor{gray!20}
InternVL2.5 38B & 23.72 & 41.21 & 37.62 & 18.63 & 38.80 & 37.03 & 20.51 & 41.55 & 39.96 & 23.72 & 33.32 & 38.72 & 24.08 & 35.46 & 39.67 & 15.96 & 32.47 & 33.49 & 21.10 & 37.14 & 37.75 & 32.00 \\
InternVL2.5 78B & 19.33 & 40.84 & 36.28 & 17.63 & 37.73 & 37.10 & 19.16 & 41.57 & 38.23 & 19.64 & 37.50 & 36.58 & 22.89 & 35.66 & 42.57 & 14.86 & 33.22 & 35.34 & 18.92 & 37.75 & 37.68 & 31.45 \\
\rowcolor{gray!20}
Qwen2 VL 72B & 20.67 & 40.81 & 41.01 & 17.37 & 37.03 & 42.13 & 18.23 & 40.02 & 42.10 & 14.08 & 36.02 & 37.60 & 23.42 & 36.12 & 41.91 & 10.62 & 29.38 & 35.14 & 17.40 & 36.56 & 39.98 & 31.31 \\
InternLM2.5 7B & -- & 34.33 & -- & -- & 32.30 & -- & -- & 34.62 & -- & -- & 31.17 & -- & -- & 29.93 & -- & -- & 23.49 & -- & -- & 30.97 & -- & 30.97 \\
\rowcolor{gray!20}
Qwen2.5 3B & -- & 32.75 & -- & -- & 28.90 & -- & -- & 33.05 & -- & -- & 28.47 & -- & -- & 26.58 & -- & -- & 22.81 & -- & -- & 28.76 & -- & 28.76 \\
InternVL2.5 26B & 11.91 & 39.97 & 34.43 & 12.63 & 36.10 & 34.07 & 13.76 & 38.92 & 34.00 & 13.11 & 34.18 & 28.98 & 15.33 & 34.01 & 36.38 & 9.38 & 29.59 & 27.95 & 12.69 & 35.46 & 32.64 & 26.93 \\
\rowcolor{gray!20}
Qwen2 VL 7B & 14.09 & 32.82 & 38.09 & 14.27 & 29.53 & 37.17 & 12.72 & 33.12 & 37.43 & 17.09 & 28.32 & 35.97 & 17.17 & 29.28 & 35.46 & 9.38 & 20.55 & 30.96 & 14.12 & 28.94 & 35.85 & 26.30 \\
MiniCPM V 2.6 & 18.49 & 34.70 & 36.28 & 13.60 & 30.47 & 34.60 & 15.88 & 33.76 & 34.96 & 18.67 & 29.03 & 34.34 & 17.50 & 30.20 & 36.84 & 9.93 & 18.42 & 24.52 & 15.68 & 29.43 & 33.59 & 26.23 \\
\rowcolor{gray!20}
Centurio Qwen & 14.87 & 31.38 & 32.05 & 14.47 & 29.23 & 29.97 & 15.07 & 31.57 & 34.76 & 16.38 & 27.40 & 35.77 & 18.62 & 27.30 & 36.32 & 13.01 & 20.41 & 32.60 & 15.40 & 27.88 & 33.58 & 25.62 \\
Phi 3.5 Vision & 12.08 & 36.64 & 35.03 & 12.43 & 32.10 & 35.23 & 10.09 & 33.36 & 32.30 & 13.78 & 29.74 & 29.80 & 16.97 & 31.97 & 33.42 & 11.99 & 25.75 & 26.78 & 12.89 & 31.59 & 32.09 & 25.53 \\
\rowcolor{gray!20}
InternVL2.5 8B & 6.81 & 36.28 & 29.97 & 6.80 & 31.33 & 29.13 & 9.07 & 33.72 & 30.49 & 6.58 & 29.74 & 30.36 & 12.11 & 30.53 & 30.26 & 2.53 & 22.67 & 20.96 & 7.32 & 30.71 & 28.53 & 22.19 \\
InternVL2.5 4B & 5.44 & 35.81 & 27.89 & 5.40 & 33.07 & 27.80 & 5.97 & 35.71 & 28.08 & 7.14 & 34.29 & 27.24 & 9.01 & 28.62 & 29.54 & 4.79 & 25.68 & 23.63 & 6.29 & 32.20 & 27.36 & 21.95 \\
\rowcolor{gray!20}
Qwen2.5 1.5B & -- & 24.03 & -- & -- & 20.77 & -- & -- & 26.66 & -- & -- & 20.87 & -- & -- & 21.45 & -- & -- & 16.23 & -- & -- & 21.67 & -- & 21.67 \\
InternLM2.5 1.8B & -- & 23.56 & -- & -- & 22.30 & -- & -- & 22.94 & -- & -- & 18.52 & -- & -- & 21.78 & -- & -- & 14.66 & -- & -- & 20.63 & -- & 20.63 \\
\rowcolor{gray!20}
Qwen2 VL 2B & 11.41 & 23.29 & 31.95 & 11.57 & 20.03 & 28.90 & 12.48 & 20.97 & 29.00 & 14.18 & 20.51 & 28.16 & 16.32 & 18.29 & 29.47 & 11.92 & 13.63 & 25.96 & 12.98 & 19.45 & 28.91 & 20.45 \\
InternVL2.5 1B & 9.87 & 24.09 & 16.51 & 8.50 & 20.43 & 16.83 & 9.78 & 21.28 & 18.50 & 11.22 & 20.41 & 16.07 & 12.83 & 16.51 & 18.75 & 9.93 & 14.93 & 17.60 & 10.35 & 19.61 & 17.38 & 15.78 \\
\rowcolor{gray!20}
InternVL2.5 2B & 5.30 & 23.26 & 18.72 & 4.80 & 19.50 & 21.90 & 4.47 & 22.26 & 20.58 & 7.30 & 21.48 & 19.95 & 9.34 & 21.38 & 20.72 & 5.68 & 15.96 & 18.77 & 6.15 & 20.64 & 20.11 & 15.63 \\
Centurio Aya & 4.80 & 29.33 & 5.94 & 5.30 & 25.50 & 7.53 & 2.94 & 28.85 & 5.02 & 7.50 & 24.23 & 8.47 & 4.28 & 24.21 & 4.21 & 4.45 & 19.38 & 5.89 & 4.88 & 25.25 & 6.18 & 12.10 \\
\rowcolor{gray!20}
Qwen2.5 0.5B & -- & 13.96 & -- & -- & 11.77 & -- & -- & 14.40 & -- & -- & 11.43 & -- & -- & 8.29 & -- & -- & 8.70 & -- & -- & 11.42 & -- & 11.42 \\
\midrule
Average X-Large LVLMs & 20.00 & 40.83 & 38.64 & 17.50 & 37.38 & 39.62 & 18.70 & 40.80 & 40.16 & 16.86 & 36.76 & 37.09 & 23.16 & 35.89 & 42.24 & 12.74 & 31.30 & 35.24 & 18.16 & 37.16 & 38.83 & 31.38 \\
\rowcolor{gray!20}
Average Large LVLMs & 17.81 & 40.59 & 36.02 & 15.63 & 37.45 & 35.55 & 17.14 & 40.24 & 36.98 & 18.42 & 33.75 & 33.85 & 19.70 & 34.74 & 38.03 & 12.67 & 31.03 & 30.72 & 16.90 & 36.30 & 35.20 & 29.46 \\
Average Medium LVLMs & 11.81 & 33.49 & 28.47 & 10.89 & 29.80 & 27.68 & 11.14 & 32.79 & 28.53 & 13.24 & 28.12 & 28.98 & 13.94 & 29.20 & 28.62 & 7.86 & 21.47 & 22.99 & 11.48 & 29.15 & 27.55 & 24.18 \\
\rowcolor{gray!20}
Average Small LVLMs & 8.82 & 28.62 & 26.02 & 8.54 & 25.03 & 26.13 & 8.56 & 26.72 & 25.69 & 10.72 & 25.29 & 24.24 & 12.89 & 23.35 & 26.38 & 8.86 & 19.19 & 22.55 & 9.73 & 24.70 & 25.17 & 19.87 \\
Average LVLMs & 12.77 & 33.79 & 30.13 & 11.67 & 30.24 & 29.96 & 12.15 & 32.83 & 30.39 & 13.60 & 29.08 & 29.14 & 15.70 & 28.88 & 31.11 & 9.60 & 23.30 & 25.68 & 12.58 & 29.69 & 29.40 & 24.41 \\
\rowcolor{gray!20}
Average X-Large LLMs & -- & 47.55 & -- & -- & 45.17 & -- & -- & 50.62 & -- & -- & 42.70 & -- & -- & 44.47 & -- & -- & 37.95 & -- & -- & 44.74 & -- & 44.74 \\
Average Large LLMs & -- & 42.45 & -- & -- & 38.93 & -- & -- & 42.52 & -- & -- & 35.94 & -- & -- & 37.50 & -- & -- & 33.39 & -- & -- & 38.46 & -- & 38.46 \\
\rowcolor{gray!20}
Average Medium LLMs & -- & 37.63 & -- & -- & 34.98 & -- & -- & 37.61 & -- & -- & 33.19 & -- & -- & 33.42 & -- & -- & 26.41 & -- & -- & 33.87 & -- & 33.87 \\
Average Small LLMs & -- & 26.94 & -- & -- & 23.79 & -- & -- & 27.06 & -- & -- & 22.82 & -- & -- & 22.59 & -- & -- & 18.48 & -- & -- & 23.62 & -- & 23.62 \\
\rowcolor{gray!20}
Average LLMs & -- & 34.55 & -- & -- & 31.54 & -- & -- & 34.89 & -- & -- & 29.84 & -- & -- & 30.25 & -- & -- & 25.12 & -- & -- & 31.03 & -- & 31.03 \\
Average X-Large & 20.00 & 43.07 & 38.64 & 17.50 & 39.98 & 39.62 & 18.70 & 44.07 & 40.16 & 16.86 & 38.74 & 37.09 & 23.16 & 38.75 & 42.24 & 12.74 & 33.52 & 35.24 & 18.16 & 39.68 & 38.83 & 35.83 \\
\rowcolor{gray!20}
Average Large & 17.81 & 41.52 & 36.02 & 15.63 & 38.19 & 35.55 & 17.14 & 41.38 & 36.98 & 18.42 & 34.84 & 33.85 & 19.70 & 36.12 & 38.03 & 12.67 & 32.21 & 30.72 & 16.90 & 37.38 & 35.20 & 33.96 \\
Average Medium & 11.81 & 34.87 & 28.47 & 10.89 & 31.53 & 27.68 & 11.14 & 34.40 & 28.53 & 13.24 & 29.81 & 28.98 & 13.94 & 30.61 & 28.62 & 7.86 & 23.12 & 22.99 & 11.48 & 30.72 & 27.55 & 27.41 \\
\rowcolor{gray!20}
Average Small & 8.82 & 27.78 & 26.02 & 8.54 & 24.41 & 26.13 & 8.56 & 26.89 & 25.69 & 10.72 & 24.05 & 24.24 & 12.89 & 22.97 & 26.38 & 8.86 & 18.84 & 22.55 & 9.73 & 24.16 & 25.17 & 21.74 \\
Average Open & 12.77 & 34.11 & 30.13 & 11.67 & 30.79 & 29.96 & 12.15 & 33.70 & 30.39 & 13.60 & 29.40 & 29.14 & 15.70 & 29.46 & 31.11 & 9.60 & 24.07 & 25.68 & 12.58 & 30.26 & 29.40 & 27.21 \\
\rowcolor{gray!20}
Average Proprietary & 40.64 & 53.67 & 56.08 & 33.51 & 50.77 & 52.78 & 36.85 & 54.25 & 54.60 & 27.65 & 48.82 & 50.04 & 38.76 & 51.47 & 54.07 & 22.66 & 43.57 & 46.26 & 33.35 & 50.43 & 52.31 & 45.36 \\
Average & 20.11 & 37.27 & 36.96 & 17.42 & 34.01 & 35.96 & 18.65 & 37.02 & 36.76 & 17.30 & 32.53 & 34.64 & 21.77 & 33.01 & 37.15 & 13.04 & 27.22 & 31.10 & 18.05 & 33.51 & 35.43 & 30.14 \\
    \bottomrule
  \end{tabular}
  }
  \caption{Average Judge Score for the \dsname Cultural Knowledge Question Answering (CKQA) -- Describing. The columns \textbf{I}, \textbf{T}, and \textbf{I+T} stand for \textbf{image-only}, \textbf{text-only}, and \textbf{image+text} input to the model.}
  \label{tab:ckqa-desc:scores}
\end{table}
%

%
\begin{table}[htbp]
  \centering
  \renewcommand{\arraystretch}{.97}
    \resizebox{\textwidth}{!}{%
    \begin{tabular}{l c c c c c c |c}
    \toprule
     & \multicolumn{1}{l}{\textsc{West EU \& North Am.}} 
     & \multicolumn{1}{l}{\textsc{East EU}}
     & \multicolumn{1}{l}{\textsc{Asian \& Pacific}}
     & \multicolumn{1}{l}{\textsc{Latin-America \& Caribbean}}
     & \multicolumn{1}{l}{\textsc{Arab}}
     & \multicolumn{1}{l|}{\textsc{Subsaharian Africa}}
     & \multicolumn{1}{l}{\textsc{Average}} \\
\midrule
\rowcolor{gray!20}
GPT-4o & 37.79 & 32.57 & 37.68 & 30.15 & 38.03 & 28.42 & 34.11 \\
Claude 3.5 Sonnet & 40.27 & 33.63 & 39.29 & 25.71 & 38.16 & 24.25 & 33.55 \\
\rowcolor{gray!20}
GPT-4o Mini & 34.46 & 28.73 & 33.08 & 23.67 & 34.87 & 25.89 & 30.12 \\
Centurio Qwen & 18.69 & 19.10 & 21.97 & 18.67 & 25.46 & 15.96 & 19.98 \\
\rowcolor{gray!20}
Gemini Pro & 16.91 & 15.60 & 16.71 & 11.13 & 17.30 & 10.55 & 14.70 \\
Gemini Flash & 15.77 & 16.27 & 14.87 & 11.60 & 14.61 & 11.30 & 14.07 \\
\rowcolor{gray!20}
InternVL2.5 38B & 14.06 & 12.60 & 16.24 & 10.71 & 21.12 & 8.36 & 13.85 \\
Phi 3.5 Vision & 15.17 & 13.67 & 13.54 & 12.45 & 14.28 & 10.75 & 13.31 \\
\rowcolor{gray!20}
InternVL2.5 78B & 12.08 & 14.73 & 14.89 & 7.35 & 15.72 & 7.53 & 12.05 \\
InternVL2.5 26B & 11.51 & 10.50 & 13.16 & 7.65 & 14.34 & 7.74 & 10.82 \\
\rowcolor{gray!20}
InternVL2.5 1B & 10.20 & 9.43 & 10.42 & 10.71 & 14.80 & 8.22 & 10.63 \\
Qwen2 VL 72B & 11.04 & 10.07 & 9.96 & 7.40 & 11.45 & 8.56 & 9.75 \\
\rowcolor{gray!20}
MiniCPM V 2.6 & 8.89 & 8.60 & 11.42 & 4.74 & 10.99 & 9.79 & 9.07 \\
Centurio Aya & 6.95 & 6.57 & 6.06 & 8.78 & 5.20 & 7.40 & 6.83 \\
\rowcolor{gray!20}
InternVL2.5 2B & 6.31 & 6.80 & 6.17 & 7.14 & 8.49 & 3.08 & 6.33 \\
InternVL2.5 4B & 6.28 & 5.47 & 5.07 & 6.02 & 9.28 & 5.00 & 6.19 \\
\rowcolor{gray!20}
InternVL2.5 8B & 6.51 & 5.30 & 4.54 & 6.48 & 9.28 & 3.77 & 5.98 \\
Qwen2 VL 2B & 5.40 & 4.27 & 7.35 & 3.62 & 5.53 & 3.63 & 4.97 \\
\rowcolor{gray!20}
Qwen2 VL 7B & 5.27 & 5.63 & 4.78 & 4.03 & 6.32 & 3.70 & 4.96 \\
\midrule
Average X-Large LVLMs & 11.56 & 12.40 & 12.42 & 7.38 & 13.58 & 8.04 & 10.90 \\
\rowcolor{gray!20}
Average Large LVLMs & 12.78 & 11.55 & 14.70 & 9.18 & 17.73 & 8.05 & 12.34 \\
Average Medium LVLMs & 9.26 & 9.04 & 9.75 & 8.54 & 11.45 & 8.12 & 9.36 \\
\rowcolor{gray!20}
Average Small LVLMs & 8.67 & 7.93 & 8.51 & 7.99 & 10.48 & 6.14 & 8.29 \\
Average LVLMs & 9.88 & 9.48 & 10.40 & 8.27 & 12.30 & 7.39 & 9.62 \\
\rowcolor{gray!20}
Average X-Large & 11.56 & 12.40 & 12.42 & 7.38 & 13.58 & 8.04 & 10.90 \\
Average Large & 12.78 & 11.55 & 14.70 & 9.18 & 17.73 & 8.05 & 12.34 \\
\rowcolor{gray!20}
Average Medium & 9.26 & 9.04 & 9.75 & 8.54 & 11.45 & 8.12 & 9.36 \\
Average Small & 8.67 & 7.93 & 8.51 & 7.99 & 10.48 & 6.14 & 8.29 \\
\rowcolor{gray!20}
Average Open & 9.88 & 9.48 & 10.40 & 8.27 & 12.30 & 7.39 & 9.62 \\
Average Proprietary & 29.04 & 25.36 & 28.33 & 20.45 & 28.59 & 20.08 & 25.31 \\
\rowcolor{gray!20}
Average & 14.92 & 13.66 & 15.12 & 11.47 & 16.59 & 10.73 & 13.75 \\
\bottomrule
    \end{tabular}
    }%
  \caption{Average Judge Score for the \dsname Cultural Knowledge Question Answering (CKQA) -- Naming.}
  \label{tab:ckqa-naming:scores}
\end{table}
%
%

%
%

%
\end{document}
