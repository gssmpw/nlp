\begin{figure}[H]
\begin{tcolorbox}[colback=gray!5!white,colframe=black!75!black,fonttitle=\bfseries\scriptsize,fontupper=\ttfamily\footnotesize,segmentation style={solid, black!30}]
  \begin{center}
    \begin{minipage}{0.18\linewidth}
      \centering
      \includegraphics[width=\linewidth]{examples/gfx/coqa_9dcf96ed_frame0_asian-and-pacific-states_0.png}
      {\captionsetup{labelformat=empty}\captionof{figure}{\tiny\textit{Copyrigth: Public Foundation 'Min Kiyal', Kyrgyzstan, 2018}}}
    \end{minipage}\hfill
    \begin{minipage}{0.18\linewidth}
      \centering
      \includegraphics[width=\linewidth]{examples/gfx/coqa_9dcf96ed_frame1_asian-and-pacific-states_0.png}
      {\captionsetup{labelformat=empty}\captionof{figure}{\tiny\textit{Copyrigth: Public Foundation 'Min Kiyal', Kyrgyzstan, 2018}}}
    \end{minipage}\hfill
    \begin{minipage}{0.18\linewidth}
      \centering
      \includegraphics[width=\linewidth]{examples/gfx/coqa_9dcf96ed_frame2_asian-and-pacific-states_0.png}
      {\captionsetup{labelformat=empty}\captionof{figure}{\tiny\textit{Copyrigth: Public Foundation 'Min Kiyal', Kyrgyzstan, 2018}}}
    \end{minipage}\hfill
    \begin{minipage}{0.18\linewidth}
      \centering
      \includegraphics[width=\linewidth]{examples/gfx/coqa_9dcf96ed_frame3_asian-and-pacific-states_0.png}
      {\captionsetup{labelformat=empty}\captionof{figure}{\tiny\textit{Copyrigth: Public Foundation 'Min Kiyal', Kyrgyzstan, 2018}}}
    \end{minipage}\hfill
    \begin{minipage}{0.18\linewidth}
      \centering
      \includegraphics[width=\linewidth]{examples/gfx/coqa_9dcf96ed_frame4_asian-and-pacific-states_0.png}
      {\captionsetup{labelformat=empty}\captionof{figure}{\tiny\textit{Copyrigth: Public Foundation 'Min Kiyal', Kyrgyzstan, 2018}}}
    \end{minipage}\hfill
  \\[4mm]
    \begin{minipage}{0.18\linewidth}
      \centering
      \includegraphics[width=\linewidth]{examples/gfx/coqa_9dcf96ed_frame5_asian-and-pacific-states_0.png}
      {\captionsetup{labelformat=empty}\captionof{figure}{\tiny\textit{Copyrigth: Public Foundation 'Min Kiyal', Kyrgyzstan, 2018}}}
    \end{minipage}\hfill
    \begin{minipage}{0.18\linewidth}
      \centering
      \includegraphics[width=\linewidth]{examples/gfx/coqa_9dcf96ed_frame6_asian-and-pacific-states_0.png}
      {\captionsetup{labelformat=empty}\captionof{figure}{\tiny\textit{Copyrigth: Public Foundation 'Min Kiyal', Kyrgyzstan, 2018}}}
    \end{minipage}\hfill
    \begin{minipage}{0.18\linewidth}
      \centering
      \includegraphics[width=\linewidth]{examples/gfx/coqa_9dcf96ed_frame7_asian-and-pacific-states_0.png}
      {\captionsetup{labelformat=empty}\captionof{figure}{\tiny\textit{Copyrigth: Public Foundation 'Min Kiyal', Kyrgyzstan, 2018}}}
    \end{minipage}\hfill
    \begin{minipage}{0.18\linewidth}
      \centering
      \includegraphics[width=\linewidth]{examples/gfx/coqa_9dcf96ed_frame8_asian-and-pacific-states_0.png}
      {\captionsetup{labelformat=empty}\captionof{figure}{\tiny\textit{Copyrigth: Public Foundation 'Min Kiyal', Kyrgyzstan, 2018}}}
    \end{minipage}\hfill
    \begin{minipage}{0.18\linewidth}
      \centering
      \includegraphics[width=\linewidth]{examples/gfx/coqa_9dcf96ed_frame9_asian-and-pacific-states_0.png}
      {\captionsetup{labelformat=empty}\captionof{figure}{\tiny\textit{Copyrigth: Public Foundation 'Min Kiyal', Kyrgyzstan, 2018}}}
    \end{minipage}\hfill
  \end{center}

  {\Large{Question:}} {\large{In which of the following countries does the event shown in the images take place? Choose from the following options and output only the corresponding letter.

A. Kyrgyzstan

B. Timor-Leste

C. Thailand

D. Turkmenistan


Your answer letter:}}\\
  {\Large{Answer:}} {\large{A}}\\
   \tcbline
  {\Large{Related Cultural Event or Facet}}\\[4mm]
  {\normalsize{Title:}} {\normalsize{Ak-kalpak craftsmanship, traditional knowledge and skills in making and wearing Kyrgyz men’s headwear}}\\
  {\normalsize{Countries:}} Kyrgyzstan\\
  {\normalsize{Regions:}} Asian and Pacific States\\
  {\normalsize{Description:}}\\
  Ak-kalpak craftsmanship is a traditional Kyrgyz handicraft. The Ak-kalpak is a traditional male hat made with white felt, which bears deep sacral meanings. Ak-kalpak craftsmanship is a cumulative, ever-evolving body of knowledge and skills passed down by craftswomen in the communities concerned comprising felting, cutting and sewing and pattern embroidery. Related knowledge and skills are transmitted via oral coaching, hands-on training and joint making in workshops. More than eighty kinds of Ak-kalpak can be distinguished, decorated with various patterns bearing a sacred meaning and history. Environmentally friendly and comfortable, the Ak-kalpak resembles a snow peak, with four sides representing the four elements: air, water, fire and earth. The four edging lines symbolize life, with the tassels on the top symbolizing ancestors’ posterity and memory, and the pattern symbolizing the family tree. Ak-kalpak unites different Kyrgyz tribes and communities and makes Kyrgyz people recognizable to other ethnic groups. It also fosters inclusivity when representatives of other ethnic groups wear it on holidays or days of mourning to express unity and sympathy. There are workshops all over the country where related knowledge and skills are passed down, and in 2013 a project entitled ‘From generation to generation’ was conducted on traditional Ak-kalpak-making techniques nationwide, resulting in an exhibition and published book.\\[2mm]
  {\normalsize{UNESCO ICH URL:}} \href{https://ich.unesco.org/en/RL/ak-kalpak-craftsmanship-traditional-knowledge-and-skills-in-making-and-wearing-kyrgyz-men-s-headwear-01496}{https://ich.unesco.org/en/RL/ak-kalpak-craftsmanship-traditi...}
\end{tcolorbox}
\end{figure}