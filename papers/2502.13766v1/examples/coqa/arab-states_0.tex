\begin{figure}[H]
\begin{tcolorbox}[colback=gray!5!white,colframe=black!75!black,fonttitle=\bfseries\scriptsize,fontupper=\ttfamily\footnotesize,segmentation style={solid, black!30}]
  \begin{center}
    \begin{minipage}{0.18\linewidth}
      \centering
      \includegraphics[width=\linewidth]{examples/gfx/coqa_ef285d07_frame0_arab-states_0.png}
      {\captionsetup{labelformat=empty}\captionof{figure}{\tiny\textit{Copyrigth: Huzaifa Ayad Bahaa El Din, Iraq, 2021}}}
    \end{minipage}\hfill
    \begin{minipage}{0.18\linewidth}
      \centering
      \includegraphics[width=\linewidth]{examples/gfx/coqa_ef285d07_frame1_arab-states_0.png}
      {\captionsetup{labelformat=empty}\captionof{figure}{\tiny\textit{Copyrigth: Huzaifa Ayad Bahaa El Din, Iraq, 2021}}}
    \end{minipage}\hfill
    \begin{minipage}{0.18\linewidth}
      \centering
      \includegraphics[width=\linewidth]{examples/gfx/coqa_ef285d07_frame2_arab-states_0.png}
      {\captionsetup{labelformat=empty}\captionof{figure}{\tiny\textit{Copyrigth: Huzaifa Ayad Bahaa El Din, Iraq, 2021}}}
    \end{minipage}\hfill
    \begin{minipage}{0.18\linewidth}
      \centering
      \includegraphics[width=\linewidth]{examples/gfx/coqa_ef285d07_frame3_arab-states_0.png}
      {\captionsetup{labelformat=empty}\captionof{figure}{\tiny\textit{Copyrigth: Zahia Benabdallah, Algeria, 2021}}}
    \end{minipage}\hfill
    \begin{minipage}{0.18\linewidth}
      \centering
      \includegraphics[width=\linewidth]{examples/gfx/coqa_ef285d07_frame4_arab-states_0.png}
      {\captionsetup{labelformat=empty}\captionof{figure}{\tiny\textit{Copyrigth: Azza Fahmi, Egypt, 2021}}}
    \end{minipage}\hfill
  \\[4mm]
    \begin{minipage}{0.18\linewidth}
      \centering
      \includegraphics[width=\linewidth]{examples/gfx/coqa_ef285d07_frame5_arab-states_0.png}
      {\captionsetup{labelformat=empty}\captionof{figure}{\tiny\textit{Copyrigth: Mustafa Kamil, Egypt, 2021}}}
    \end{minipage}\hfill
    \begin{minipage}{0.18\linewidth}
      \centering
      \includegraphics[width=\linewidth]{examples/gfx/coqa_ef285d07_frame6_arab-states_0.png}
      {\captionsetup{labelformat=empty}\captionof{figure}{\tiny\textit{Copyrigth: National Heritage Preservation, Ministry of Culture, Youth and Sport and Relations with the Parliament, Egypt, 2022}}}
    \end{minipage}\hfill
    \begin{minipage}{0.18\linewidth}
      \centering
      \includegraphics[width=\linewidth]{examples/gfx/coqa_ef285d07_frame7_arab-states_0.png}
      {\captionsetup{labelformat=empty}\captionof{figure}{\tiny\textit{Copyrigth: Direction du Patrimoine Culturel, Morocco, 2021}}}
    \end{minipage}\hfill
    \begin{minipage}{0.18\linewidth}
      \centering
      \includegraphics[width=\linewidth]{examples/gfx/coqa_ef285d07_frame8_arab-states_0.png}
      {\captionsetup{labelformat=empty}\captionof{figure}{\tiny\textit{Copyrigth: Direction du Patrimoine Culturel, Morocco, 2021}}}
    \end{minipage}\hfill
    \begin{minipage}{0.18\linewidth}
      \centering
      \includegraphics[width=\linewidth]{examples/gfx/coqa_ef285d07_frame9_arab-states_0.png}
      {\captionsetup{labelformat=empty}\captionof{figure}{\tiny\textit{Copyrigth: Ministry of Culture, Palestine, 2021}}}
    \end{minipage}\hfill
  \end{center}

  {\Large{Question:}} {\large{In which of the following countries does the event shown in the images take place? Choose from the following options and output only the corresponding letter.

A. Kuwait

B. Jordan

C. Egypt

D. United Arab Emirates


Your answer letter:}}\\
  {\Large{Answer:}} {\large{C}}\\
   \tcbline
  {\Large{Related Cultural Event or Facet}}\\[4mm]
  {\normalsize{Title:}} {\normalsize{Arts, skills and practices associated with engraving on metals (gold, silver and copper)}}\\
  {\normalsize{Countries:}} Algeria, Saudi Arabia, Egypt, Iraq, Morocco, Mauritania, Palestine, Sudan, Tunisia, Yemen\\
  {\normalsize{Regions:}} Arab States\\
  {\normalsize{Description:}}\\
  Engraving on metals such as gold, silver and copper is a centuries-old practice that entails manually cutting words, symbols or patterns into the surfaces of decorative, utilitarian, religious or ceremonial objects. The craftsperson uses different tools to manually cut symbols, names, Quran verses, prayers and geometric patterns into the objects. Engravings can be concave (recessed) or convex (elevated), or the result of a combination of different types of metals, such as gold and silver. Their social and symbolic meanings and functions vary according to the communities concerned. Engraved objects, such as jewelry or household objects, are often presented as traditional gifts for weddings or used in religious rituals and alternative medicine. For instance, certain types of metals are believed to have healing properties. Engraving on metals is transmitted within families, through observation and hands-on practice. It is also transmitted through workshops organized by training centres, organizations and universities, among others. Publications, cultural events and social media further contribute to the transmission of the related knowledge and skills. Practised by people of all ages and genders, metal engraving and the use of engraved objects are means of expressing the cultural, religious and geographical identity and the socioeconomic status of the communities concerned.\\[2mm]
  {\normalsize{UNESCO ICH URL:}} \href{https://ich.unesco.org/en/RL/arts-skills-and-practices-associated-with-engraving-on-metals-gold-silver-and-copper-01951}{https://ich.unesco.org/en/RL/arts-skills-and-practices-assoc...}
\end{tcolorbox}
\end{figure}