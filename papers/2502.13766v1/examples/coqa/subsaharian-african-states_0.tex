\begin{figure}[H]
\begin{tcolorbox}[colback=gray!5!white,colframe=black!75!black,fonttitle=\bfseries\scriptsize,fontupper=\ttfamily\footnotesize,segmentation style={solid, black!30}]
  \begin{center}
    \begin{minipage}{0.18\linewidth}
      \centering
      \includegraphics[width=\linewidth]{examples/gfx/coqa_6ce004b9_frame0_subsaharian-african-states_0.png}
      {\captionsetup{labelformat=empty}\captionof{figure}{\tiny\textit{Copyrigth: Etienne Kokolo, Kinshasa, République du Congo, 2018}}}
    \end{minipage}\hfill
    \begin{minipage}{0.18\linewidth}
      \centering
      \includegraphics[width=\linewidth]{examples/gfx/coqa_6ce004b9_frame1_subsaharian-african-states_0.png}
      {\captionsetup{labelformat=empty}\captionof{figure}{\tiny\textit{Copyrigth: Etienne Kokolo, Kinshasa, République du Congo, 2019}}}
    \end{minipage}\hfill
    \begin{minipage}{0.18\linewidth}
      \centering
      \includegraphics[width=\linewidth]{examples/gfx/coqa_6ce004b9_frame2_subsaharian-african-states_0.png}
      {\captionsetup{labelformat=empty}\captionof{figure}{\tiny\textit{Copyrigth: Etienne Kokolo, Kinshasa, République du Congo, 2018}}}
    \end{minipage}\hfill
    \begin{minipage}{0.18\linewidth}
      \centering
      \includegraphics[width=\linewidth]{examples/gfx/coqa_6ce004b9_frame3_subsaharian-african-states_0.png}
      {\captionsetup{labelformat=empty}\captionof{figure}{\tiny\textit{Copyrigth: Etienne Kokolo, Kinshasa, République du Congo, 2018}}}
    \end{minipage}\hfill
    \begin{minipage}{0.18\linewidth}
      \centering
      \includegraphics[width=\linewidth]{examples/gfx/coqa_6ce004b9_frame4_subsaharian-african-states_0.png}
      {\captionsetup{labelformat=empty}\captionof{figure}{\tiny\textit{Copyrigth: Etienne Kokolo, Kinshasa, République du Congo, 2018}}}
    \end{minipage}\hfill
  \\[4mm]
    \begin{minipage}{0.18\linewidth}
      \centering
      \includegraphics[width=\linewidth]{examples/gfx/coqa_6ce004b9_frame5_subsaharian-african-states_0.png}
      {\captionsetup{labelformat=empty}\captionof{figure}{\tiny\textit{Copyrigth: Etienne Kokolo, Kinshasa, République du Congo, 2017}}}
    \end{minipage}\hfill
    \begin{minipage}{0.18\linewidth}
      \centering
      \includegraphics[width=\linewidth]{examples/gfx/coqa_6ce004b9_frame6_subsaharian-african-states_0.png}
      {\captionsetup{labelformat=empty}\captionof{figure}{\tiny\textit{Copyrigth: Etienne Kokolo, Kinshasa, République du Congo, 2018}}}
    \end{minipage}\hfill
    \begin{minipage}{0.18\linewidth}
      \centering
      \includegraphics[width=\linewidth]{examples/gfx/coqa_6ce004b9_frame7_subsaharian-african-states_0.png}
      {\captionsetup{labelformat=empty}\captionof{figure}{\tiny\textit{Copyrigth: Etienne Kokolo, Kinshasa, République du Congo, 2020}}}
    \end{minipage}\hfill
    \begin{minipage}{0.18\linewidth}
      \centering
      \includegraphics[width=\linewidth]{examples/gfx/coqa_6ce004b9_frame8_subsaharian-african-states_0.png}
      {\captionsetup{labelformat=empty}\captionof{figure}{\tiny\textit{Copyrigth: Etienne Kokolo, Kinshasa, République du Congo, 2017}}}
    \end{minipage}\hfill
    \begin{minipage}{0.18\linewidth}
      \centering
      \includegraphics[width=\linewidth]{examples/gfx/coqa_6ce004b9_frame9_subsaharian-african-states_0.png}
      {\captionsetup{labelformat=empty}\captionof{figure}{\tiny\textit{Copyrigth: Etienne Kokolo, Kinshasa, République du Congo, 2020}}}
    \end{minipage}\hfill
  \end{center}

  {\Large{Question:}} {\large{In which of the following countries does the event shown in the images take place? Choose from the following options and output only the corresponding letter.

A. Congo

B. Togo

C. Namibia

D. Nigeria


Your answer letter:}}\\
  {\Large{Answer:}} {\large{A}}\\
   \tcbline
  {\Large{Related Cultural Event or Facet}}\\[4mm]
  {\normalsize{Title:}} {\normalsize{Congolese rumba}}\\
  {\normalsize{Countries:}} Congo, Democratic Republic of the Congo\\
  {\normalsize{Regions:}} Subsaharian African States\\
  {\normalsize{Description:}}\\
  Congolese rumba is a musical genre and a dance common in urban areas of the Democratic Republic of the Congo and the Republic of the Congo. Generally danced by a male-female couple, it is a multicultural form of expression originating from an ancient dance called nkumba (meaning ‘waist’ in Kikongo). The rumba is used for celebration and mourning, in private, public and religious spaces. It is performed by professional and amateur orchestras, choirs, dancers and individual musicians, and women have played a predominant role in the development of religious and romantic styles. The tradition of Congolese rumba is passed down to younger generations through neighbourhood clubs, formal training schools and community organisations. For instance, rumba musicians maintain clubs and apprentice artists to carry on the practice and the manufacture of instruments. The rumba also plays an important economic role, as orchestras are increasingly developing cultural entrepreneurship aimed at reducing poverty. The rumba is considered an essential and representative part of the identity of Congolese people and its diaspora. It is perceived as a means of conveying the social and cultural values of the region and of promoting intergenerational and social cohesion and solidarity.\\[2mm]
  {\normalsize{UNESCO ICH URL:}} \href{https://ich.unesco.org/en/RL/congolese-rumba-01711}{https://ich.unesco.org/en/RL/congolese-rumba-01711...}
\end{tcolorbox}
\end{figure}