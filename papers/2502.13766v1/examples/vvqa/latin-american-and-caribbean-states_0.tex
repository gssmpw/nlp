\begin{figure}[H]
\begin{tcolorbox}[colback=gray!5!white,colframe=black!75!black,fonttitle=\bfseries\scriptsize,fontupper=\ttfamily\footnotesize,segmentation style={solid, black!30}]
  \begin{center}
    \begin{minipage}{0.18\linewidth}
      \centering
      \includegraphics[width=\linewidth]{examples/gfx/vvqa_aefac9db_frame0_latin-american-and-caribbean-states_0.png}
    \end{minipage}\hfill
    \begin{minipage}{0.18\linewidth}
      \centering
      \includegraphics[width=\linewidth]{examples/gfx/vvqa_aefac9db_frame1_latin-american-and-caribbean-states_0.png}
    \end{minipage}\hfill
    \begin{minipage}{0.18\linewidth}
      \centering
      \includegraphics[width=\linewidth]{examples/gfx/vvqa_aefac9db_frame2_latin-american-and-caribbean-states_0.png}
    \end{minipage}\hfill
    \begin{minipage}{0.18\linewidth}
      \centering
      \includegraphics[width=\linewidth]{examples/gfx/vvqa_aefac9db_frame3_latin-american-and-caribbean-states_0.png}
    \end{minipage}\hfill
    \begin{minipage}{0.18\linewidth}
      \centering
      \includegraphics[width=\linewidth]{examples/gfx/vvqa_aefac9db_frame4_latin-american-and-caribbean-states_0.png}
    \end{minipage}\hfill
  \\[4mm]
    \begin{minipage}{0.18\linewidth}
      \centering
      \includegraphics[width=\linewidth]{examples/gfx/vvqa_aefac9db_frame5_latin-american-and-caribbean-states_0.png}
    \end{minipage}\hfill
    \begin{minipage}{0.18\linewidth}
      \centering
      \includegraphics[width=\linewidth]{examples/gfx/vvqa_aefac9db_frame6_latin-american-and-caribbean-states_0.png}
    \end{minipage}\hfill
    \begin{minipage}{0.18\linewidth}
      \centering
      \includegraphics[width=\linewidth]{examples/gfx/vvqa_aefac9db_frame7_latin-american-and-caribbean-states_0.png}
    \end{minipage}\hfill
    \begin{minipage}{0.18\linewidth}
      \centering
      \includegraphics[width=\linewidth]{examples/gfx/vvqa_aefac9db_frame8_latin-american-and-caribbean-states_0.png}
    \end{minipage}\hfill
    \begin{minipage}{0.18\linewidth}
      \centering
      \includegraphics[width=\linewidth]{examples/gfx/vvqa_aefac9db_frame9_latin-american-and-caribbean-states_0.png}
    \end{minipage}\hfill
  \end{center}

  {\Large{Question:}} {\large{In which environment do the cultural practices depicted in the video typically occur?}}\\
  {\Large{Answer:}} {\large{Llanos}}\\
   \tcbline
  {\Large{Related Cultural Event or Facet}}\\[4mm]
  {\normalsize{Title:}} {\normalsize{Colombian-Venezuelan llano work songs}}\\
  {\normalsize{Countries:}} Colombia, Venezuela (Bolivarian Republic of)\\
  {\normalsize{Regions:}} Latin-American and Caribbean States\\
  {\normalsize{Description:}}\\
  Colombian-Venezuelan llano work songs are a practice of vocal communication consisting of tunes sung individually, a capella, on the themes of herding and milking. The practice emerged from the close relationship between human communities and cattle and horses and is in harmony with the environmental conditions and the dynamics of nature, forming part of the traditional animal husbandry system of the Llanos. Transmitted orally from childhood, the songs are repositories of the individual and collective stories of the llaneros. Llano work songs have been gradually affected by economic, political and social processes that, modifying the llanero cultural universe, have significantly weakened the practice. For example, ambitious government plans conceived from a developmental perspective have led to profound changes in the use of the land and in ownership systems, and the modification of the social, cultural and natural sites of the songs have resulted in a loss of interest in the values and techniques of llano work. Llanero work songs thus face various threats to their viability. Efforts to safeguard the element are nonetheless widespread, including a pedagogical strategy involving more than twenty meetings for bearers and young people in the region, training projects for schoolteachers and a proliferation of festivals.\\[2mm]
  {\normalsize{UNESCO ICH URL:}} \href{https://ich.unesco.org/en/USL/colombian-venezuelan-llano-work-songs-01285}{https://ich.unesco.org/en/USL/colombian-venezuelan-llano-wor...}
\end{tcolorbox}
\end{figure}