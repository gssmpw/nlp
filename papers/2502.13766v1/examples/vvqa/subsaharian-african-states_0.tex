\begin{figure}[H]
\begin{tcolorbox}[colback=gray!5!white,colframe=black!75!black,fonttitle=\bfseries\scriptsize,fontupper=\ttfamily\footnotesize,segmentation style={solid, black!30}]
  \begin{center}
    \begin{minipage}{0.18\linewidth}
      \centering
      \includegraphics[width=\linewidth]{examples/gfx/vvqa_1d097abd_frame0_subsaharian-african-states_0.png}
    \end{minipage}\hfill
    \begin{minipage}{0.18\linewidth}
      \centering
      \includegraphics[width=\linewidth]{examples/gfx/vvqa_1d097abd_frame1_subsaharian-african-states_0.png}
    \end{minipage}\hfill
    \begin{minipage}{0.18\linewidth}
      \centering
      \includegraphics[width=\linewidth]{examples/gfx/vvqa_1d097abd_frame2_subsaharian-african-states_0.png}
    \end{minipage}\hfill
    \begin{minipage}{0.18\linewidth}
      \centering
      \includegraphics[width=\linewidth]{examples/gfx/vvqa_1d097abd_frame3_subsaharian-african-states_0.png}
    \end{minipage}\hfill
    \begin{minipage}{0.18\linewidth}
      \centering
      \includegraphics[width=\linewidth]{examples/gfx/vvqa_1d097abd_frame4_subsaharian-african-states_0.png}
    \end{minipage}\hfill
  \\[4mm]
    \begin{minipage}{0.18\linewidth}
      \centering
      \includegraphics[width=\linewidth]{examples/gfx/vvqa_1d097abd_frame5_subsaharian-african-states_0.png}
    \end{minipage}\hfill
    \begin{minipage}{0.18\linewidth}
      \centering
      \includegraphics[width=\linewidth]{examples/gfx/vvqa_1d097abd_frame6_subsaharian-african-states_0.png}
    \end{minipage}\hfill
    \begin{minipage}{0.18\linewidth}
      \centering
      \includegraphics[width=\linewidth]{examples/gfx/vvqa_1d097abd_frame7_subsaharian-african-states_0.png}
    \end{minipage}\hfill
    \begin{minipage}{0.18\linewidth}
      \centering
      \includegraphics[width=\linewidth]{examples/gfx/vvqa_1d097abd_frame8_subsaharian-african-states_0.png}
    \end{minipage}\hfill
    \begin{minipage}{0.18\linewidth}
      \centering
      \includegraphics[width=\linewidth]{examples/gfx/vvqa_1d097abd_frame9_subsaharian-african-states_0.png}
    \end{minipage}\hfill
  \end{center}

  {\Large{Question:}} {\large{What type of theatre is depicted in the video, known for using elaborate costumes and performances?}}\\
  {\Large{Answer:}} {\large{Kwagh-Hir}}\\
   \tcbline
  {\Large{Related Cultural Event or Facet}}\\[4mm]
  {\normalsize{Title:}} {\normalsize{Kwagh-Hir theatrical performance}}\\
  {\normalsize{Countries:}} Nigeria\\
  {\normalsize{Regions:}} Subsaharian African States\\
  {\normalsize{Description:}}\\
  Kwagh-Hir theatrical performance is a composite art form encompassing a spectacle that is both visually stimulating and culturally edifying. Kwagh-hir has its roots in the story-telling tradition of the Tiv people called ‘kwagh-alom’, a practice where the family was treated to a storytelling session by creative storytellers, usually in the early hours of the night after the day’s farming work. With time, creative storytellers began to dramatize these stories, culminating in the present stage and status of Kwagh-hir. The practice is a social performance with the potential to entertain and teach moral lessons through the dramatization and performance of past and current social realities. As a form of total theatre, Kwagh-hir incorporates puppetry, masquerading, poetry, music, dance and animated narratives in articulating the reality of the Tiv people. People’s daily struggles, aspirations, successes and failures are all given expression through creative dramatization. Khwagh-hir theatre is owned by the community, with knowledge and skills being transmitted through apprenticeship. People who indicate an interest in the troupe’s activities are trained and mentored until they reach a certain level of proficiency; they are then accepted into the troupe. Regular performances are held to ensure the art is kept alive and that the younger generation continues to identify with it.\\[2mm]
  {\normalsize{UNESCO ICH URL:}} \href{https://ich.unesco.org/en/RL/kwagh-hir-theatrical-performance-00683}{https://ich.unesco.org/en/RL/kwagh-hir-theatrical-performanc...}
\end{tcolorbox}
\end{figure}