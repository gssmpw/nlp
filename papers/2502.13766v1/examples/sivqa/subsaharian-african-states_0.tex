\begin{figure}[H]
\begin{tcolorbox}[colback=gray!5!white,colframe=black!75!black,fonttitle=\bfseries\scriptsize,fontupper=\ttfamily\footnotesize,segmentation style={solid, black!30}]
  \begin{center}
    \begin{minipage}{0.5\linewidth}
      \centering
      \includegraphics[width=\linewidth]{examples/gfx/sivqa_863c7b2f_subsaharian-african-states_0.png}
      {\captionsetup{labelformat=empty}\captionof{figure}{\tiny\textit{Copyrigth: The Authority for Research and Conservation of Cultural Heritage (ARCCH), 2013}}}
    \end{minipage}\hfill
  \end{center}
  {\Large{Question:}} {\large{What festival are the people in the image celebrating?}}\\
  {\Large{Answer:}} {\large{Fichee-Chambalaalla}}\\
   \tcbline
  {\Large{Related Cultural Event or Facet}}\\[4mm]
  {\normalsize{Title:}} {\normalsize{Fichee-Chambalaalla, New Year festival of the Sidama people}}\\
  {\normalsize{Countries:}} Ethiopia\\
  {\normalsize{Regions:}} Subsaharian African States\\
  {\normalsize{Description:}}\\
  Fichee-Chambalaalla is a New Year festival celebrated among the Sidama people. According to the oral tradition, Fichee commemorates a Sidama woman who visited her parents and relatives once a year after her marriage, bringing ''buurisame'', a meal prepared from false banana, milk and butter, which was shared with neighbours. Fichee has since become a unifying symbol of the Sidama people. Each year, astrologers determine the correct date for the festival, which is then announced to the clans. Communal events take place throughout the festival, including traditional songs and dances. Every member participates irrespective of age, gender and social status. On the first day, children go from house to house to greet their neighbours, who serve them ''buurisame''. During the festival, clan leaders advise the Sidama people to work hard, respect and support the elders, and abstain from cutting down indigenous trees, begging, indolence, false testimony and theft. The festival therefore enhances equity, good governance, social cohesion, peaceful co-existence and integration among Sidama clans and the diverse ethnic groups in Ethiopia. Parents transmit the tradition to their children orally and through participation in events during the celebration. Women in particular, transfer knowledge and skills associated with hairdressing and preparation of ''buurisame'' to their daughters and other girls in their respective villages.\\[2mm]
  {\normalsize{UNESCO ICH URL:}} \href{https://ich.unesco.org/en/RL/fichee-chambalaalla-new-year-festival-of-the-sidama-people-01054}{https://ich.unesco.org/en/RL/fichee-chambalaalla-new-year-fe...}
\end{tcolorbox}
\end{figure}