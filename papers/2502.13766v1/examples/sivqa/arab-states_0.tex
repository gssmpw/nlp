\begin{figure}[H]
\begin{tcolorbox}[colback=gray!5!white,colframe=black!75!black,fonttitle=\bfseries\scriptsize,fontupper=\ttfamily\footnotesize,segmentation style={solid, black!30}]
  \begin{center}
    \begin{minipage}{0.5\linewidth}
      \centering
      \includegraphics[width=\linewidth]{examples/gfx/sivqa_15d1a779_arab-states_0.png}
      {\captionsetup{labelformat=empty}\captionof{figure}{\tiny\textit{Copyrigth: Conseil municipal de Sefrou, 2010}}}
    \end{minipage}\hfill
  \end{center}
  {\Large{Question:}} {\large{What title is given to the woman wearing the sash in the image?}}\\
  {\Large{Answer:}} {\large{Cherry Queen}}\\
   \tcbline
  {\Large{Related Cultural Event or Facet}}\\[4mm]
  {\normalsize{Title:}} {\normalsize{Cherry festival in Sefrou}}\\
  {\normalsize{Countries:}} Morocco\\
  {\normalsize{Regions:}} Arab States\\
  {\normalsize{Description:}}\\
  For three days in June each year, the local population of Sefrou celebrates the natural and cultural beauty of the region, symbolized by the cherry fruit and that year’s newly chosen Cherry Queen, selected during a pageant that draws competitors from the region and entire country. The highlight of the festival is a parade with performing troupes, rural and urban music, majorettes and bands, and floats featuring local producers. At the centre is the Cherry Queen, who offers cherries to onlookers while dressed ornately and surrounded by attendants. The whole population contributes to the success of the festival: craftswomen make silk buttons for traditional dresses, fruit growers supply cherries, local sports clubs participate in competitions, and music and dancing troupes animate the entire festival. The cherry festival provides an opportunity for the entire city to present its activities and achievements. The younger generation are also integrated into festival activities to ensure their sustainability. The festival is a source of pride and belonging that enhances the self-esteem of the city and its people and constitutes a fundamental contribution to their local identity.\\[2mm]
  {\normalsize{UNESCO ICH URL:}} \href{https://ich.unesco.org/en/RL/cherry-festival-in-sefrou-00641}{https://ich.unesco.org/en/RL/cherry-festival-in-sefrou-00641...}
\end{tcolorbox}
\end{figure}