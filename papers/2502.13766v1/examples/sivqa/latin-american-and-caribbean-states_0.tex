\begin{figure}[H]
\begin{tcolorbox}[colback=gray!5!white,colframe=black!75!black,fonttitle=\bfseries\scriptsize,fontupper=\ttfamily\footnotesize,segmentation style={solid, black!30}]
  \begin{center}
    \begin{minipage}{0.5\linewidth}
      \centering
      \includegraphics[width=\linewidth]{examples/gfx/sivqa_5e3e97f3_latin-american-and-caribbean-states_0.png}
      {\captionsetup{labelformat=empty}\captionof{figure}{\tiny\textit{Copyrigth: Py, 2019}}}
    \end{minipage}\hfill
  \end{center}
  {\Large{Question:}} {\large{What traditional tool from the Guaraní culture is depicted in the image for drinking Terere?}}\\
  {\Large{Answer:}} {\large{Bombilla}}\\
   \tcbline
  {\Large{Related Cultural Event or Facet}}\\[4mm]
  {\normalsize{Title:}} {\normalsize{Practices and traditional knowledge of Terere in the culture of Pohã Ñana, Guaraní ancestral drink in Paraguay}}\\
  {\normalsize{Countries:}} Paraguay\\
  {\normalsize{Regions:}} Latin-American and Caribbean States\\
  {\normalsize{Description:}}\\
  The practices and traditional knowledge of Terere in the culture of Pohã Ñana, Guaraní ancestral drink in Paraguay, are widespread in the Paraguayan territory and involve a variety of bearers. Terere is a traditional drink prepared in a jug or thermos, in which cold water is mixed with Pohã Ñana crushed in a mortar. It is served in a glass pre-filled with yerba mate and sucked with a bombilla (metal or cane straw). Preparing the Terere is an intimate ritual involving a series of pre-established codes and each Pohã Ñana herb has health benefits linked to popular wisdom passed down through the generations. Terere practices in the culture of Pohã Ñana have been transmitted in Paraguayan families since approximately the sixteenth century. Traditional knowledge about the healing attributes of the medicinal herbs that make up the Pohã Ñana and their correct use are also transmitted spontaneously within the family. In recent years, the figure of apprentices has risen, but family transmission remains the main mode of transmission. The practice of the Terere in the culture of Pohã Ñana fosters social cohesion as the time and space dedicated to preparing and consuming the Terere promote inclusion, friendship, dialogue, respect and solidarity. The practice also strengthens new generations’ appreciation of the rich cultural and botanical heritage of Guaraní origin.\\[2mm]
  {\normalsize{UNESCO ICH URL:}} \href{https://ich.unesco.org/en/RL/practices-and-traditional-knowledge-of-terere-in-the-culture-of-poha-nana-guarani-ancestral-drink-in-paraguay-01603}{https://ich.unesco.org/en/RL/practices-and-traditional-knowl...}
\end{tcolorbox}
\end{figure}