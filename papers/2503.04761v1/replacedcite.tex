\section{Background and Related Work}
Our work builds on many lines of research attempting to model, measure, and forecast AI's impact on the economy.

\paragraph{Economic foundations and the task-based framework}
A wide body of work in economics has proposed theoretical models to understand the impact of automation on the labor market. Most notably, ____ argue for modeling labor markets through the lens of discrete \textit{tasks} which can be performed by either human workers or machines---for example, \textit{debugging code} or \textit{cutting hair}. Building on this framework, ____ shows that while technologies automate some tasks, they often augment human capabilities in others due to complementarity between humans and machines, leading to higher demand for labor. In addition ____ use this framework to explore a model where automation technologies can create entirely new tasks in addition to displacing old tasks. 

\paragraph{Forecasting the impact of AI on labor markets}
Another branch of work leverages the task-based framework to predict the future prevalence of automation across the economy, often based on descriptions of tasks and occupations from the O*NET database of occupational information provided by the U.S. Department of Labor ____. For example, ____ fit a gaussian process classifier to a dataset of 70 labeled occupations to predict which occupations are subject to computerization. ____ hire human annotators to rate 2,069 detailed work areas in the O*NET database, focusing specifically on their potential to be performed by machine learning. ____ analyzes the overlap between patent documents and job task descriptions to predict the "exposure" of tasks to AI, finding highest exposure in high-education, high-wage occupations---a pattern partially reflected in our empirical usage data, though we find peak usage in mid-to-high wage occupations rather than at the highest wage levels. ____ focus specifically on large language models, estimating exposure by using a dataset that links human abilities to different occupations. 

____ also consider exposure of tasks to language models, using language models themselves to obtain more granular estimates of exposure at the level of individual tasks---an approach we follow in our work. When considering the impacts of language model-powered software, they conclude that around half of all tasks in the economy could one day be automated by language models. While our empirical usage data shows lower current adoption ($\sim\!36\%$ of occupations using AI for at least a quarter of their tasks), the patterns of usage across tasks largely align with their predictions, particularly in showing high usage for software development and content creation tasks.

\paragraph{Real-world studies of AI usage}

To complement these forecasts based on human or machine judgment, another body of work attempts to gather concrete data to understand how AI is currently being adopted across the labor market. For example, studies show rapid AI adoption across different sectors and countries: research from late 2023 found that half of workers in exposed Danish occupations had used ChatGPT, estimating it could halve working times in about a third of their tasks ____, while a subsequent study in August 2024 found that 39\% of working-age US adults had used generative AI, with about a quarter using it weekly ____. Moreover, further research has attempted to measure the breadth and depth of this usage, with studies finding positive effects of generative AI tools on productivity for a wide range of individual domains, including software engineering ____, writing  ____, customer service ____, consulting ____, translation ____, legal analysis ____, and data science ____.

We bridge these separate approaches to perform the first large-scale analysis of how advanced AI systems are actually being used across tasks and occupations. We build on the task-based framework, but rather than forecasting potential impacts ("exposure" of occupations to AI), we measure real-world \textbf{usage patterns} using Clio ____, a recent system that enables privacy-perserving analysis of millions of human-AI conversations on a major model provider. This allows us to complement controlled studies of AI productivity effects in specific domains with a comprehensive view of where and how AI is being integrated into work across the economy. Our methodology enables tracking these patterns dynamically as both AI capabilities and societal adoption evolve---revealing both present day usage trends as well as leading indicators of future diffusion.


\begin{figure*}
    \centering
    \includegraphics[width=0.99\textwidth]{figs/final_figs/fig-occupation.png}
    \caption{\textbf{Hierarchical breakdown of top six occupational categories by the amount of AI usage in their associated tasks.} Each occupational category contains the individual O*NET occupations and tasks with the highest levels of appearance in Claude.ai interactions.}
    \label{fig:occupations_and_tasks}
\end{figure*}