\section{A Structured Guide to \credal}
\label{sec:credal:guide}

In this section, we present a structured guide to \credal (Section~\ref{sec:credal:guide:steps}), along with other materials we developed to support the adoption and use of the methodology (Section~\ref{sec:credal:supp}).  

\subsection{\credal}
\label{sec:credal:guide:steps}

\begin{enumerate}
\item \emph{Define Research Goals and Understand the Data Model.} Set clear objectives for your analysis, whether it be model-driven or, if data is available,  data-driven.
Understand your data and data model, paying attention to its structure, domain, and other details, such as missing relevant details.

\item \emph{Evaluate the Context, Domain Knowledge and Sources.} 
(a) Gain domain-specific insights from additional sources. 
(b) Consider ethical, privacy, and other concerns, including the absence of sensitive but relevant data. 
(c) Evaluate data sources for their biases and omissions.

\item \emph{Exploratory Data Analysis (EDA) and Schema Analysis.}
\begin{itemize}
    \item For data-driven close reading: Identify general patterns in the data, including outliers and unexpected features (for more practical tips on EDA, see \cite{abedjan,downey}). 
    \item For model-driven close reading: Conduct a detailed schema analysis, focusing on the following aspects:
        (a) Identifying entities and relationships;
        (b) Reviewing constraints such as primary keys, foreign keys, unique constraints, and field formats; and,
        (c) Identifying indices and design patterns.
\end{itemize}

\item \emph{Related Schemas Exploration.}
    (a) Identify related schemas and assess their content in comparison to the primary schema;
    (b) Note any elements or relationships that are included or excluded in the related schemas; and,
    (c) Analyze how the comparison can inform improvements or enhancements to the target schema.

\item \emph{Assumptions Loop.}
\begin{enumerate}
    \item Select a small portion of the data model (e.g., an entity, its attribute, or a relationship between entities) where potential interesting bias may be present.
    \item Establish criteria for determining fairness and objectivity before identifying negative bias or inequality. It is essential to distinguish between assumptions and verifiable bias.
    \begin{itemize}
        \item For data-driven close reading: Define metrics to quantify assumed bias or inequality. These may include skewness in data distribution, the presence of empty fields, or other quantifiable characteristics.
        \item For model-driven close reading: Compare the schema with related models, analyze similar fields, and review restrictive types or irregular relationships to validate assumptions. Conduct thorough domain research to substantiate any findings.
    \end{itemize}
    \item Identify potential risks or issues arising from the assumed bias. Consider specific scenarios in which the model may fail or produce unintended consequences.
        i) Assess how sensitive attributes (e.g., gender) influence decision boundaries and overall fairness when legally and ethically permissible;
        ii) Evaluate fairness by analyzing causal pathways and understanding how different factors contribute to model outcomes; and
        iii) The "5 Whys" method is recommended to trace the root cause of bias and understand its origin (details in Section~\ref{sec:credal:supp}). 
    \item Propose solutions to mitigate identified harmful biases and/or enhance the comprehensiveness of the data or schema.
\end{enumerate}

\item \emph{Compile Findings.}
    Generate a conclusion summarizing your observations on the analyzed model. This report should be easily comprehensible for the target audience and ideally provide valuable insights to aid the model developer in its revision.
\end{enumerate}


The overall pipeline of the data close reading methodology is provided in Figure~\ref{fig:close_reading_pipeline}.

\begin{figure}[t]
  \centering
  \includegraphics[width=0.9\linewidth]{imgs/diagram_methodology.png}
  \caption{Visual Representation of \credal}
  \Description{Directed graph depicting the main steps to take during data model close reading}
  \label{fig:close_reading_pipeline}
\end{figure}


\subsection{Supporting Materials}
\label{sec:credal:supp}

\paragraph{Video tutorial.} As a result of the iterative improvements to the methodology, we identified the need to enhance the visual presentation of the methodology to improve comprehension. In response, we developed a video tutorial that presents and explains the key terminology and main stages of the methodology, reinforcing each step with illustrative examples. 

\paragraph{Example of Reading.} To demonstrate the application of the methodology, we developed a comprehensive example reading that describes the entire process from start to finish, illustrating the nuances involved in applying the methodology to a real-world scenario. The selected data model, ``A Secure Students’ Attendance Monitoring System'' (Appendix \ref{sec:example_reading}), was chosen due to a reasonable level of complexity. It provides sufficient depth to thoroughly illustrate \credal while remaining accessible to the target audience. This step-by-step walk-through offered clear instructions, making the methodology application more straightforward for users.

\paragraph{Suggestions and Best Practices.} We identified several key techniques that enhance both the perception and application of the methodology. These insights are organized into practical recommendations:

\begin{itemize} 
    \item \textbf{Use the ``5 Whys'' Method.} This technique can help identify bias and uncover its root causes. The process involves asking ``why'' multiple times to drill down into the underlying issues. The steps are as follows:
        \begin{itemize} 
            \item Define the problem: Clearly articulate the issue to ensure an accurate understanding, as solving an ill-defined problem is challenging. 
            \item Ask the first ``Why''. Begin by questioning why the problem occurred to uncover initial insights. 
            \item Continue asking ``Why''. For each identified reason, continue asking ``why'' until you reach the fundamental cause, repeating this process until a clear solution emerges. 
            \item Practical Tips: Move swiftly from one ``why'' to the next to maintain focus and avoid distractions. Know when to stop, as continuing beyond meaningful responses may lead to diminishing returns. 
    \end{itemize}
    
    \item \textbf{Incorporate Brainstorming Sessions.}  Brainstorming encourages creative thinking and diverse viewpoints, which helps uncover biases that may not be evident in a solo analysis. Key steps include:
        \begin{itemize}
            \item Visualize the goal of the session.
            \item Document all discussions and ideas.
            \item Encourage participants to think aloud and propose varied ideas.
            \item Foster an environment where all ideas are welcomed without immediate criticism.
            \item Collaborate and ask clarifying questions.
            \item Organize the outcomes of the brainstorming session for further analysis.
        \end{itemize}
        
    \item \textbf{Apply the Principles of Literary Close Reading.}  Close reading, rooted in literary analysis, offers a systematic approach to examining textual content in depth. Its principles can be adapted for data model reading:
        \begin{itemize}
            \item Focus on explicit content, avoiding speculative interpretations.
            \item Read slowly and attentively to capture nuanced details that might otherwise be overlooked.
            \item Perform multiple readings, allowing for a comprehensive understanding of the material.
            \item Analyze each element thoroughly to ensure a detailed and accurate interpretation.
        \end{itemize}
    \end{itemize}
    
    \paragraph{Common Pitfalls and Considerations.} When applying the methodology, users may encounter several potential challenges, summarized below, along with advice for mitigating them:
    \begin{itemize} 
        \item \textbf{Overlooking Entity Relationships.}
        Thoroughly explore the relationships between entities within the schema. Neglecting these connections may lead to incomplete or inaccurate conclusions.
        \item \textbf{Managing Large Data Volumes.}  
        A data-driven approach may introduce complexity when dealing with large datasets. Focus on specific areas of interest and relevant data points to avoid becoming overwhelmed by the volume of information.
        \item \textbf{Bias Below the Surface.}  
        Bias is not always apparent at the top levels of the schema. Take multiple perspectives, playing the roles of both the creator and reviewer, to identify potential biases or controversies that may lie deeper in the structure.
        \item \textbf{Working with Unfamiliar Data or Schemas.}  
        When dealing with an unfamiliar domain, consider consulting specialists or leveraging external resources to gain a better understanding before proceeding with bias analysis.
        \item \textbf{Curiosity and Questioning.}  
        Embrace curiosity and do not hesitate to ask fundamental questions. Seemingly basic inquiries can often lead to important discoveries and unveil hidden aspects of the data model.
        \item \textbf{Concluding the Analysis.}  
        After identifying biases, it is essential to clearly articulate your findings and suggest actionable solutions. This ensures that the issues are addressed and mitigated in future iterations of the data model.

    \end{itemize}

