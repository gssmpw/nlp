\section{Results analysis and discussion}
\label{sec:discussion}

After conducting and analyzing all 11 interviews, we found that each participant had prior experience with data modeling, with 2 reporting basic data modeling proficiency, 7 reporting intermediate proficiency, and 2 reporting advanced proficiency. Further, none of the participants were familiar with the literary close reading methodology prior to the workshop.

While each participant expressed unique thoughts and opinions, there were also significant commonalities among them. By synthesizing the insights from all interviews, we identified notable patterns, which we will discuss throughout this section.  To start, we present the results of a simple quantitative analysis.  Figure ~\ref{fig:word_cloud} shows a word cloud of participants' responses.  The most frequently used terms were ``methodology'', ``data'', and ``data model'',  followed by ``know'' and ``example''.

\begin{figure}
    \centering
    \includegraphics[width=0.7\linewidth]{imgs/word_cloud.png}
    \caption{Word cloud of all participant responses during the interviews}
    \label{fig:word_cloud}
\end{figure}

Figure~\ref{fig:code_dist} presents a summary view of the frequencies of all codes from the codebook discussed in Section~\ref{sec:eval:coding}, with colors representing our four code groups.  We observe that methodology strengths is the most frequent code group (with ``Helpful supplemental materials'', ``Helpful [methodology] structure'', and ``Encourages analysis and reflection'' as the most frequent codes), followed by methodology effectiveness (``Participant likely to use methodology in the future'' and ``Improved participant's data modeling proficiency'' as the most frequent codes), and then by methodology improvement (``Need to improve presentation of bias'' and ``Need to improve supplemental materials'' as the most frequent codes).  Based on this summary, we conjecture that \credal was found to be helpful and effective by participants, but also that there were substantial opportunities for improvement, primarily of the supplemental materials (rather than of the methodology itself).

\begin{figure}[htbp]
    \includegraphics[width=0.9\linewidth]{imgs/codes_groups_dist.png}
    \caption{Interviews codes frequency distribution}
    \label{fig:code_dist}
\end{figure}

\subsection{Research Question 1: Usability and Usefulness of \credal}

The first research question \textbf{RQ1: Is \credal usable and useful?} comprises three subquestions that we will address by analyzing participant responses.

\subsubsection{\textbf{RQ1.1}: Is \credal easy to learn and apply?}

Participants noted that the methodology features a ``Helpful Structure'' (13 code entries) and provides an abundance of ``Helpful Supplemental Materials'' (28 entries, the most frequently cited code). These findings suggest that the methodology is both feasible and effectively supported by the examples, tips, and additional resources provided, which facilitate user implementation.  Representative quotes include:

\begin{quote}
\textbf[P10]: ``I would say that the steps were defined pretty clearly. And also, there was a pretty helpful example, which went really in-depth, making it easy to work on my schema after reading the entire example with a different schema.''
\end{quote}

\begin{quote}
\textbf[P5]: ``I read that guide tutorial that was sent to me; it was very detailed and contained much information about how to perform this close reading.''
\end{quote}

\subsubsection{\textbf{RQ1.2}: Does \credal improve data modeling proficiency?}

The code ``Improved participant's data modeling proficiency'' ranks among the top five most frequently cited codes, with 12 occurrences. This code reflects participant feedback indicating that their data modeling skills and understanding significantly improved after the workshop.  Responses varied regarding the specific ways in which their skills improved, yet many participants reported gaining new conceptual and practical insights into data modeling, even those who were already familiar with the topic. Participants shared their thoughts with remarks such as:

\begin{quote}
\textbf[P5]: ``I think that understanding changed, and I can also say that I have a new view on data models after reading that tutorial.''
\end{quote}

\begin{quote}
\textbf[P9]: ``I feel more confident; gaining new knowledge about data modeling has strengthened my skills.''
\end{quote}

\subsubsection{\textbf{RQ1.3}: Are learners likely to use CREDAL in the future?}

The code labeled ``Participant likely to use methodology in the future'' appears next, with a total of 18 occurrences. This indicates that participants are inclined to apply the methodology in future projects or other contexts, often providing specific examples of such applications. Moreover, they affirmed the potential applications of this method in various scenarios, as illustrated by their comments:

\begin{quote}
\textbf[P8]: ``Enhancing the training data, especially for R\&D projects, I guess that might be a good push and a good idea to read the work that you've accomplished.''
\end{quote}
\begin{quote}
\textbf[P4]: ``A schematic framework that could help illustrate why the data model is good or bad.''
\end{quote}

Based on these results, we conclude that \credal was found to be usable and useful by participants. 

\subsection{Research Question 2: \credal helps work with data models}

The second research question \textbf{RQ2: Does \credal help understand, design, and critically evaluate data models?} focuses on the practical application of the \credal and its interaction with real data models, leading to three subquestions.

\subsubsection{\textbf{RQ2.1}: Does CREDAL help with understanding data models?}

The code ``Helped participant understand a specific model'' played a significant role in addressing this research question. Most participants emphasized that their data modeling skills improved, noting increased confidence in their understanding of data models after engaging with CREDAL.
\begin{quote}
    \textbf[P11]: ``It refreshed my knowledge about data modeling and also taught me some new ways to work with them, especially to analyze data models more carefully and look for potential problems.``
\end{quote}
\begin{quote}
      \textbf[P7]: ``I read that tutorial and learned how to perform close reading on those data models.``
\end{quote}

\subsubsection{\textbf{RQ2.2}: Does CREDAL alter the approach to structuring and modeling data within modeling tasks?}

Participants noted that prior to the workshop, many did not consider how a model was created, which data was used, or the approach developers followed in constructing a specific data model. However, after the workshop, participants observed a shift in perspective regarding data modeling tasks, indicating a willingness to adjust their approaches to create clearer data models. This is evident in quotations coded with ``Improved participant's data modeling proficiency'': 

\begin{quote}
      \textbf[P8]: ``Definitely, I guess I’ve almost never taken care about where the data was sourced and why someone decided to model it in a certain way.``
\end{quote}
\begin{quote}
      \textbf[P11]: ``The workshop provided me with a structured way to evaluate data models. It helped me spot issues before they became problems.``
\end{quote}

\subsubsection{\textbf{RQ2.3}: Do learners experience a change in their perspectives on data modeling after working with CREDAL?}

Most participants had an intermediate level of understanding of data modeling; however, even those with prior experience discovered new insights and gained unique knowledge relevant to their education or work. The same code, ``Improved participant's data modeling proficiency', captured all quotations reflecting how the methodology influenced their skills and understanding.  Representative examples are included below:

\begin{quote}
      \textbf[P3]: ``You can identify where you might be wrong about your granularity or your assumptions about whether a data model could work here or not. Without some methodology, you might miss those crucial aspects, even if you have experience working with other models.''
\end{quote}
\begin{quote}
      \textbf[P1]: ``It's essential to ensure that biases don’t negatively impact the model's performance.''
\end{quote}

Based on the analysis of the codes and provided quotations, we conclude that participants found \credal to be helpful in their work with data models.  Participants reported improvements in their ability to analyze and evaluate data models. They also proposed new approaches for improving data model design using \credal. These factors demonstrate that \credal effectively aids learners in developing a comprehensive understanding and application of data modeling practices.

\subsection{Opportunities for Improvement and Extension}
\label{sec:improvements}
Methodology improvement and extension were an essential focus of the interviews, as participants provided valuable feedback on enhancing its clarity, presentation, and practicality. The identified areas were categorized into six distinct codes, each reflecting specific concerns or suggestions from the respondents. Below, we discuss each code and its relevance to the refinement of the methodology.

\subsubsection{Improving \credal} We first look at the codes in this group relevant to improving the methodology and materials.

\paragraph{Need to Improve Supplemental Materials}
One of the recurring themes in the interviews was the need for clearer and more structured supplemental materials. Several respondents expressed difficulties in understanding the methodology from the provided text files. As one participant mentioned:

\begin{quote}
    \textbf[P1]: ``The understanding of the methodology was a bit confusing... So like to understand how it is done from the provided file. I believe that the whole methodology can be expressed in a more concise and step-by-step approach.''  
\end{quote}
% \textit{(Code: Need to improve supplemental materials)}

This feedback emphasizes the importance of refining the materials to make the methodology more digestible. Another respondent reinforced this point by suggesting the need to simplify the description further:

\begin{quote}
   \textbf[P1]: `` We can recap the suggestions mentioned before to simplify the description of the methodology... It can be made more concise and understandable.''
\end{quote}

Analyzing data models can be a resource-intensive task. Therefore, our objective, in addition to offering a systematic approach through the CREDAL methodology, is to streamline the learning process and ensure that the methodology can be applied as efficiently and effectively as possible.

\paragraph{Need to Add More Examples}
Respondents also highlighted the importance of including more examples, particularly diagrams or visual representations, to support the text-heavy material. One participant noted:

\begin{quote}
    \textbf[P3]: ``If we're talking about the text version, I maybe added more schemas, diagrams, something like that, because it was mostly text.''
\end{quote}

Another respondent emphasized the need for examples that could help identify more different types of biases present in a data model:

\begin{quote}
    \textbf[P10]: ``I think maybe for the making assumptions part and for detecting biases, it would be helpful to use another example to just provide more types of biases that could be present in a certain data model.''
\end{quote}

As one can conclude from these quotes, incorporating visual aids, such as diagrams, ensures that users have a more interactive and comprehensive understanding of the process. Additionally, as deduced from the feedback, integrating case-oriented studies alongside theoretical explanations enhances both the clarity and applicability of the methodology. This combination allows users to grasp the methodology more effectively.


\paragraph{Need to Improve Presentation of Bias} Bias detection was a key area where respondents felt the methodology could be clearer. Participants expressed the need for a more structured approach to identifying and addressing bias within data models. One interviewee commented:

\begin{quote} \textbf[P9]: ``There is a need for clearer bias definition.'' \end{quote}

Another respondent highlighted the inherent subjectivity involved in understanding bias, making it difficult to provide a universally applicable solution:

\begin{quote} \textbf[P11]: ``My understanding of bias can be different than someone else's. So it includes a lot of subjectivity, and it's, well, it's difficult to try to avoid that.'' \end{quote}

This feedback underscores the complexity of addressing all potential forms of bias, as bias can often be perceived differently depending on the individual's perspective. While it may not be possible to cover every aspect of bias, the methodology aims to provide the broadest possible coverage, offering examples and guidance to help users recognize and mitigate harmful biases effectively within their data models.

\subsubsection{Extending \credal}
Next, we look at the codes in this group that are relevant to extending the methodology and materials.

\paragraph{Opportunity to Engage Domain Experts} Participants noted that involving domain experts could enhance the quality of the close reading process, particularly when scaling the methodology for larger teams or projects. One participant observed:

\begin{quote} \textbf[P2]: ``Engaging domain experts might improve overall close reading results when scaling the methodology for teams.'' \end{quote}

Another participant highlighted the challenges of manually identifying similar data models (here in the form of relational database schemas), stressing the potential value of expert involvement:

\begin{quote} \textbf[P3]: ``Because to try to find similar schemas by yourself, it's pretty annoying, I would say.'' \end{quote}

These suggestions point to the potential benefits of collaboration with domain specialists for a given data model to ensure a more informed and contextually relevant application of the methodology. Domain experts can provide insights into data model complexities and nuances that may be difficult for non-experts to identify, thereby improving the precision and quality of close reading analysis. Furthermore, the methodology is conveniently scalable for use not only by individual data model developers or users but also by entire teams working collaboratively on a given data model. This scalability highlights the adaptability of the approach for both small-scale and large-scale projects.

\paragraph{Opportunity to Automate} One participant repeatedly proposed automating certain parts of the methodology to make it more efficient and reduce manual workload. This respondent suggested:

\begin{quote} \textbf[P2]: ``Doing it manually is a little bit over-engineering. I would suggest trying to automate this methodology.'' \end{quote}

The participant also provided a detailed vision of how automation could be implemented, suggesting the use of large language models or similar services to streamline schema comparisons and analysis:

\begin{quote} \textbf[P3]: ``It would be cool to have something like a service where I have data, and I just apply it, or some piece of data... The service iteratively tries to analyze it based on some rules.'' \end{quote}

While automation presents a promising direction for improving efficiency, it is important to note that fully automating the entire process could introduce new sources of bias, especially if the model's decision-making lacks transparency. However, there are specific areas within the methodology that are well-suited for automation without compromising the integrity of the analysis. For instance, tasks such as related schema exploration or certain aspects of exploratory data analysis could be streamlined through automated tools \cite{abedjan,downey}. Automating these components would reduce manual effort while maintaining the critical human oversight required to ensure the accuracy and fairness of the analysis.

These insights highlight the potential for developing an automated framework that could facilitate the application of the methodology. Such a framework would enhance the accessibility and scalability of the methodology, particularly for larger data models. This suggestion aligns with future work directions, as automating the process could significantly reduce the manual effort required and contribute to making the methodology not only more efficient but also easier and more accessible for a wide range of users.

\paragraph{Opportunity for Case-by-Case Customization} Respondents also raised the idea of customizing the methodology for different use cases, suggesting that a more flexible approach could cater to varying needs. One participant explained:

\begin{quote} \textbf[P8]: ``I would like the methodology to be split into two methodologies—a longer and shorter version—for different application cases.'' \end{quote}

This feedback highlights the potential for adapting the methodology to suit a range of applications better. Rather than maintaining a single comprehensive approach, respondents suggested offering two distinct versions of the methodology: one that provides a detailed, in-depth process for complex cases and another that offers a streamlined version for simpler scenarios.

Generalizing the CREDAL methodology to apply broadly in all cases may make it overly cumbersome for certain contexts. As such, this feedback opens the door for future work in adapting the methodology by separating it according to different aspects of the application. This could involve creating a more modular framework where users can select the version of the methodology best suited to their specific needs, ensuring both efficiency and relevance in various data modeling scenarios.

