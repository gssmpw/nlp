\documentclass[manuscript,nonacm,table]{acmart}

\AtBeginDocument{%
  \providecommand\BibTeX{{%
    Bib\TeX}}}

\setcopyright{acmlicensed}
\copyrightyear{2024}
\acmYear{2024}
\acmDOI{XXXXXXX.XXXXXXX}

\acmConference[Conference acronym 'XX]{Conference}{June 03--05,
  2018}{Woodstock, NY}
\acmISBN{978-1-4503-XXXX-X/18/06}

\usepackage{array}
\usepackage{xspace}

\newcommand*{\credal}{\texttt{CREDAL}\xspace}


\begin{document}

\title{CREDAL: Close Reading of Data Models}

\author{George Fletcher}
\affiliation{%
 \institution{Eindhoven University of Technology}
 \city{Eindhoven}
 \country{Netherlands}}
 \email{g.h.l.fletcher@tue.nl}


\author{Olha Nahurna}
\affiliation{%
  \institution{Ukrainian Catholic University}
  \city{Lviv}
  \country{Ukraine}}
  \email{onahurna@gmail.com}


\author{Matvii Prytula }
\affiliation{%
  \institution{Ukrainian Catholic University}
  \city{Lviv}
  \country{Ukraine}}
  \email{matvii.prytula@ucu.edu.ua}


\author{Julia Stoyanovich}
\affiliation{%
  \institution{New York University}
  \city{New York}
  \country{USA}}
  \email{stoyanovich@nyu.edu}

\renewcommand{\shortauthors}{Fletcher et al.}


\begin{abstract}
Data models are necessary for the birth of data and of any data-driven system.  Indeed, every algorithm, every machine learning model, every statistical model, and every database has an underlying data model without which the system would not be usable.  Hence, data models are excellent sites for interrogating the (material, social, political, ...) conditions giving rise to a data system.  Towards this, drawing inspiration from literary criticism, we propose to closely read data models in the same spirit as we closely read literary artifacts.  Close readings of data models reconnect us with, among other things, the materiality, the genealogies, the techne, the closed nature, and the design of technical systems. 

While recognizing from literary theory that there is no one correct way to read,  it is nonetheless critical to have systematic guidance for those unfamiliar with close readings.  This is especially true for those trained in the computing and data sciences, who too often are enculturated to set aside the socio-political aspects of data work.  
A systematic methodology for reading data models currently does not exist.  
To fill this gap, we present the \credal methodology for close readings of data models.  We detail our iterative development process and present results of a qualitative evaluation of \credal demonstrating its usability, usefulness, and effectiveness in the critical study of data. 

\end{abstract}


\keywords{data modeling, conceptual modeling, critical data modeling, close reading, bias, responsible data management, socio-technical systems}
\maketitle

\section{Introduction}


\begin{figure}[t]
\centering
\includegraphics[width=0.6\columnwidth]{figures/evaluation_desiderata_V5.pdf}
\vspace{-0.5cm}
\caption{\systemName is a platform for conducting realistic evaluations of code LLMs, collecting human preferences of coding models with real users, real tasks, and in realistic environments, aimed at addressing the limitations of existing evaluations.
}
\label{fig:motivation}
\end{figure}

\begin{figure*}[t]
\centering
\includegraphics[width=\textwidth]{figures/system_design_v2.png}
\caption{We introduce \systemName, a VSCode extension to collect human preferences of code directly in a developer's IDE. \systemName enables developers to use code completions from various models. The system comprises a) the interface in the user's IDE which presents paired completions to users (left), b) a sampling strategy that picks model pairs to reduce latency (right, top), and c) a prompting scheme that allows diverse LLMs to perform code completions with high fidelity.
Users can select between the top completion (green box) using \texttt{tab} or the bottom completion (blue box) using \texttt{shift+tab}.}
\label{fig:overview}
\end{figure*}

As model capabilities improve, large language models (LLMs) are increasingly integrated into user environments and workflows.
For example, software developers code with AI in integrated developer environments (IDEs)~\citep{peng2023impact}, doctors rely on notes generated through ambient listening~\citep{oberst2024science}, and lawyers consider case evidence identified by electronic discovery systems~\citep{yang2024beyond}.
Increasing deployment of models in productivity tools demands evaluation that more closely reflects real-world circumstances~\citep{hutchinson2022evaluation, saxon2024benchmarks, kapoor2024ai}.
While newer benchmarks and live platforms incorporate human feedback to capture real-world usage, they almost exclusively focus on evaluating LLMs in chat conversations~\citep{zheng2023judging,dubois2023alpacafarm,chiang2024chatbot, kirk2024the}.
Model evaluation must move beyond chat-based interactions and into specialized user environments.



 

In this work, we focus on evaluating LLM-based coding assistants. 
Despite the popularity of these tools---millions of developers use Github Copilot~\citep{Copilot}---existing
evaluations of the coding capabilities of new models exhibit multiple limitations (Figure~\ref{fig:motivation}, bottom).
Traditional ML benchmarks evaluate LLM capabilities by measuring how well a model can complete static, interview-style coding tasks~\citep{chen2021evaluating,austin2021program,jain2024livecodebench, white2024livebench} and lack \emph{real users}. 
User studies recruit real users to evaluate the effectiveness of LLMs as coding assistants, but are often limited to simple programming tasks as opposed to \emph{real tasks}~\citep{vaithilingam2022expectation,ross2023programmer, mozannar2024realhumaneval}.
Recent efforts to collect human feedback such as Chatbot Arena~\citep{chiang2024chatbot} are still removed from a \emph{realistic environment}, resulting in users and data that deviate from typical software development processes.
We introduce \systemName to address these limitations (Figure~\ref{fig:motivation}, top), and we describe our three main contributions below.


\textbf{We deploy \systemName in-the-wild to collect human preferences on code.} 
\systemName is a Visual Studio Code extension, collecting preferences directly in a developer's IDE within their actual workflow (Figure~\ref{fig:overview}).
\systemName provides developers with code completions, akin to the type of support provided by Github Copilot~\citep{Copilot}. 
Over the past 3 months, \systemName has served over~\completions suggestions from 10 state-of-the-art LLMs, 
gathering \sampleCount~votes from \userCount~users.
To collect user preferences,
\systemName presents a novel interface that shows users paired code completions from two different LLMs, which are determined based on a sampling strategy that aims to 
mitigate latency while preserving coverage across model comparisons.
Additionally, we devise a prompting scheme that allows a diverse set of models to perform code completions with high fidelity.
See Section~\ref{sec:system} and Section~\ref{sec:deployment} for details about system design and deployment respectively.



\textbf{We construct a leaderboard of user preferences and find notable differences from existing static benchmarks and human preference leaderboards.}
In general, we observe that smaller models seem to overperform in static benchmarks compared to our leaderboard, while performance among larger models is mixed (Section~\ref{sec:leaderboard_calculation}).
We attribute these differences to the fact that \systemName is exposed to users and tasks that differ drastically from code evaluations in the past. 
Our data spans 103 programming languages and 24 natural languages as well as a variety of real-world applications and code structures, while static benchmarks tend to focus on a specific programming and natural language and task (e.g. coding competition problems).
Additionally, while all of \systemName interactions contain code contexts and the majority involve infilling tasks, a much smaller fraction of Chatbot Arena's coding tasks contain code context, with infilling tasks appearing even more rarely. 
We analyze our data in depth in Section~\ref{subsec:comparison}.



\textbf{We derive new insights into user preferences of code by analyzing \systemName's diverse and distinct data distribution.}
We compare user preferences across different stratifications of input data (e.g., common versus rare languages) and observe which affect observed preferences most (Section~\ref{sec:analysis}).
For example, while user preferences stay relatively consistent across various programming languages, they differ drastically between different task categories (e.g. frontend/backend versus algorithm design).
We also observe variations in user preference due to different features related to code structure 
(e.g., context length and completion patterns).
We open-source \systemName and release a curated subset of code contexts.
Altogether, our results highlight the necessity of model evaluation in realistic and domain-specific settings.




  
\section{Background}\label{sec:backgrnd}

\subsection{Cold Start Latency and Mitigation Techniques}

Traditional FaaS platforms mitigate cold starts through snapshotting, lightweight virtualization, and warm-state management. Snapshot-based methods like \textbf{REAP} and \textbf{Catalyzer} reduce initialization time by preloading or restoring container states but require significant memory and I/O resources, limiting scalability~\cite{dong_catalyzer_2020, ustiugov_benchmarking_2021}. Lightweight virtualization solutions, such as \textbf{Firecracker} microVMs, achieve fast startup times with strong isolation but depend on robust infrastructure, making them less adaptable to fluctuating workloads~\cite{agache_firecracker_2020}. Warm-state management techniques like \textbf{Faa\$T}~\cite{romero_faa_2021} and \textbf{Kraken}~\cite{vivek_kraken_2021} keep frequently invoked containers ready, balancing readiness and cost efficiency under predictable workloads but incurring overhead when demand is erratic~\cite{romero_faa_2021, vivek_kraken_2021}. While these methods perform well in resource-rich cloud environments, their resource intensity challenges applicability in edge settings.

\subsubsection{Edge FaaS Perspective}

In edge environments, cold start mitigation emphasizes lightweight designs, resource sharing, and hybrid task distribution. Lightweight execution environments like unikernels~\cite{edward_sock_2018} and \textbf{Firecracker}~\cite{agache_firecracker_2020}, as used by \textbf{TinyFaaS}~\cite{pfandzelter_tinyfaas_2020}, minimize resource usage and initialization delays but require careful orchestration to avoid resource contention. Function co-location, demonstrated by \textbf{Photons}~\cite{v_dukic_photons_2020}, reduces redundant initializations by sharing runtime resources among related functions, though this complicates isolation in multi-tenant setups~\cite{v_dukic_photons_2020}. Hybrid offloading frameworks like \textbf{GeoFaaS}~\cite{malekabbasi_geofaas_2024} balance edge-cloud workloads by offloading latency-tolerant tasks to the cloud and reserving edge resources for real-time operations, requiring reliable connectivity and efficient task management. These edge-specific strategies address cold starts effectively but introduce challenges in scalability and orchestration.

\subsection{Predictive Scaling and Caching Techniques}

Efficient resource allocation is vital for maintaining low latency and high availability in serverless platforms. Predictive scaling and caching techniques dynamically provision resources and reduce cold start latency by leveraging workload prediction and state retention.
Traditional FaaS platforms use predictive scaling and caching to optimize resources, employing techniques (OFC, FaasCache) to reduce cold starts. However, these methods rely on centralized orchestration and workload predictability, limiting their effectiveness in dynamic, resource-constrained edge environments.



\subsubsection{Edge FaaS Perspective}

Edge FaaS platforms adapt predictive scaling and caching techniques to constrain resources and heterogeneous environments. \textbf{EDGE-Cache}~\cite{kim_delay-aware_2022} uses traffic profiling to selectively retain high-priority functions, reducing memory overhead while maintaining readiness for frequent requests. Hybrid frameworks like \textbf{GeoFaaS}~\cite{malekabbasi_geofaas_2024} implement distributed caching to balance resources between edge and cloud nodes, enabling low-latency processing for critical tasks while offloading less critical workloads. Machine learning methods, such as clustering-based workload predictors~\cite{gao_machine_2020} and GRU-based models~\cite{guo_applying_2018}, enhance resource provisioning in edge systems by efficiently forecasting workload spikes. These innovations effectively address cold start challenges in edge environments, though their dependency on accurate predictions and robust orchestration poses scalability challenges.

\subsection{Decentralized Orchestration, Function Placement, and Scheduling}

Efficient orchestration in serverless platforms involves workload distribution, resource optimization, and performance assurance. While traditional FaaS platforms rely on centralized control, edge environments require decentralized and adaptive strategies to address unique challenges such as resource constraints and heterogeneous hardware.



\subsubsection{Edge FaaS Perspective}

Edge FaaS platforms adopt decentralized and adaptive orchestration frameworks to meet the demands of resource-constrained environments. Systems like \textbf{Wukong} distribute scheduling across edge nodes, enhancing data locality and scalability while reducing network latency. Lightweight frameworks such as \textbf{OpenWhisk Lite}~\cite{kravchenko_kpavelopenwhisk-light_2024} optimize resource allocation by decentralizing scheduling policies, minimizing cold starts and latency in edge setups~\cite{benjamin_wukong_2020}. Hybrid solutions like \textbf{OpenFaaS}~\cite{noauthor_openfaasfaas_2024} and \textbf{EdgeMatrix}~\cite{shen_edgematrix_2023} combine edge-cloud orchestration to balance resource utilization, retaining latency-sensitive functions at the edge while offloading non-critical workloads to the cloud. While these approaches improve flexibility, they face challenges in maintaining coordination and ensuring consistent performance across distributed nodes.

  
\section{Developing the Close Reading Methodology (\credal)}
\label{sec:credal}

This section outlines the iterative process we followed to develop \credal. The process evolved through a series of steps, each designed to refine the methodology based on practical applications, feedback, and structured evaluation. Below, we detail each step of this process.

\subsection{Initial development}
\label{sec:credal:dev}

\subsubsection{Initial Exploration and Refinement}
\label{sec:credal:dev:1}

Our research team initiated the process by deeply engaging with the technique of close reading, traditionally employed in literary studies for detailed analysis of texts~\cite{frank}. Following a review of the relevant literature and practical application on literary texts, we sought to adapt this approach to the analysis of data models. Each team member independently proposed methods for applying close reading to the schemas. After reviewing individual work, we compared our approaches, identified commonalities, and developed the first draft of a methodology for applying close reading to data models.

In the next phase, we applied the newly developed methodology to a fresh set of schemas. This time, all team members used the same standardized approach, allowing us to observe how different individuals interpreted and applied the same methodology in diverse ways. This approach highlighted variations in interpretation and application, which led us to further refine and modify the methodology.

We used both knowledge graph data models (examining schemas from Wikidata.org and Schema.org) and relational data models during this first step. 
Recognizing the widespread use of relational data models \cite{silberschatz} across various industries and the broad availability of standard relational data modeling curricula, we decided to focus on relational data models for the remainder of our study.
We conducted another round of close reading, this time using open-source 
data models,\footnote{e.g., \href{https://gitmind.com/erd-examples.html}{GitMind}, \href{https://devtoolsdaily.medium.com/crafting-an-automatic-erd-generator-a-journey-from-ddl-to-diagram-83cc5da8cab7}{DevTools Daily}, \href{https://www.conceptdraw.com/How-To-Guide/entity-relationship-diagrams}{ConceptDraw}}
and adapted our methodology based on this experience.

\subsubsection{Development of Supplemental Materials}
\label{sec:credal:dev:2}

In order to verify the applicability and effectiveness of the methodology, it was necessary to involve more independent reviewers.
Therefore, to ensure that the methodology could be effectively applied by others, we developed a set of supplemental materials. Since different team members brought unique perspectives and strategies to the project, we synthesized these insights into a practical guide. This guide contains valuable tips and advice aimed at helping users navigate potential challenges during applications of the methodology.

Additionally, we created a sample reading to demonstrate the methodology’s practical application. The chosen example, ``A Secure Students’ Attendance Monitoring System''~\cite{erd} (see Appendix~\ref{sec:example_reading} for details), was specifically selected for its balanced complexity, ensuring that it was detailed enough to be informative without overwhelming learners. The familiar context of education made the example relatable to a wide audience, and the step-by-step approach offered clear guidance. Together, these supplemental materials served as essential resources, providing both conceptual support and practical reference points.

% !TEX root = main.tex
The \emph{goal} of this study is to empirically evaluate how code-comment coherence, through a quality-aware selection strategy grounded on SIDE, impacts the effectiveness and training efficiency of neural code summarization models.

More specifically, the study aims to address the following research questions:

\begin{itemize}[itemindent=0.25cm]
	\item[\textbf{RQ$_{0}$}:] \textit{How do code summarization datasets measure up in terms of code-comment coherence?}
	In this preliminary question, we assess the coherence of code-comment pairs of datasets commonly used in code summarization. As we aim to use a coherence-aware strategy to optimize training sets, first of all, we would like to see how the coherence is distributed.
	\item[\textbf{RQ$_{1}$}:] \textit{How does a coherence-aware strategy selection impact the performance of neural code summarization models?}
	In this research question, we investigate how a targeted selection of training data based on code-comment coherence impacts the performance of neural code summarization models.
	\item[\textbf{RQ$_{2}$}:] \textit{How does the coherence-aware strategy selection compare with a random baseline?}
	In this research question, we test our hypothesis that code-comment coherence is a quality attribute that can be used to select training instances.
\end{itemize}

\subsection{Context Selection}
\label{subsec:context_selection}
The \emph{context} of our study consists of datasets containing pairs of \java methods with the associated summaries. 
For fine-tuning the models, we consider the two most important datasets from the state of the art: TL-CodeSum \cite{hu2018summarizing}, and Funcom \cite{leclair2019neural}.

The \textit{TL-CodeSum} dataset \cite{hu2018summarizing} is specifically designed for the code summarization task. It consists of $\sim$87k instances $ \langle code, summary \rangle$ extracted from GitHub repositories created from 2015 to 2016, and having at least 20 stars. In detail, Hu \etal \cite{hu2018summarizing} extracted the first sentence---likely to describe the overall method functionality---from the doc of each pair.

Similarly to TL-CodeSum, the \textit{Funcom} dataset \cite{leclair2019neural} is also specifically designed for code summarization. \textit{Funcom} consists of over 2.1M $ \langle code, summary \rangle$ pairs collected from the Sourcerer repository. As for TL-CodeSum, LeClair \etal \cite{leclair2019neural} only consider methods with their javadoc, extracting the first sentence as corresponding \textit{summary}.

Shi \etal \cite{shi2022we} found many noisy instances and duplicates in the above-described datasets and cleaned them up using their heuristic-based dataset-cleaning approach. For this reason, we use the cleaned versions of \textit{TL-CodeSum} and \textit{Funcom} provided by Shi \etal \cite{shi2022we}. The cleaned \textit{TL-CodeSum} contains 53,597 training instances, while the cleaned \textit{Funcom} contains 1,184,438 training instances.

The above datasets are built automatically, and no manual check was performed, \ie there is no guarantee of their quality. For this reason, we use two additional, manually curated datasets to test the models. The first one is \textit{CoderEval} \cite{yu2024codereval}, which consists of 230 Python and 230 \java code generation problems collected from open-source, high-starred projects which include \textit{original} and \textit{human-labeled} docstrings that should act as prompt for Code Generation models to generate the corresponding \textit{code}. 
The instances have been subject to manual screening, for which the main criterion is the probability of appearing in real-development scenarios. We focus on the \java set of problems, inverting the input and the output \ie from $\langle docstring, code \rangle$ to $\langle code, docstring \rangle$. 
To align the format of the pairs format, we performed an additional manual analysis in which one of the authors checked all the triplets with a second author to confirm the analysis. 
We found that some of the \textit{docstring}(s) contained more than a sentence. Therefore, to make them consistent with the previous dataset format (\eg single sentence), we extracted the first sentence from each \textit{docstring}. Still, we found 12 occurrences in which the corresponding \textit{original docstring} does not describe the \textit{code} (\eg ``\texttt{{@inheritDoc}}'', ``\texttt{@param modelName model name of the entity}'', and similar). We also excluded \textit{docstring}: ``\texttt{Computes floor(\$log\_2 (n)\$) \$+ 1\$.}'' since it includes a formula not explained in natural language.
Again, to appropriately align the evaluation, we do not evaluate such instances, ending up with 218 \textit{original} instances.

The second manually-curated dataset we use is the one by Mastropaolo \etal \cite{mastropaolo2023robustness}. The dataset consists of 892 methods associated with their summary (\ie first sentence of the method documentation), collected from non-fork GitHub \java repositories with at least 300 commits, 50 contributors, and 25 stars. 
Such instances are in the form $\langle summary, code\rangle$ and, as for CoderEval \cite{yu2024codereval}, we inverted the input and the output \ie $\langle code, summary\rangle$. Mastropaolo \etal  analyzed such pairs to ensure their quality. We manually analyzed and cleaned them further (\eg ``\texttt{Adds an {@link CarrierService} to the {@linkCarrier}}'' into ``\texttt{Adds an CarrierService to the Carrier}''), as we had done for CoderEval. No instances were removed during such a manual analysis.

We remove the instances from the test sets which appear in the training sets of \textit{TL-CodeSum} and \textit{Funcom}. As a result, we remove ten instances from CoderEval, which are present only in the \textit{TL-CodeSum} training set.

\subsection{Study Methodology}
\label{subsec:exp_proc}
To answer RQ$_{0}$, we use SIDE to compute the degree to which the summaries of the studied datasets document their corresponding code. We did this for each instance of the training sets included in \textit{TL-CodeSum} and \textit{Funcom}. To understand the coherence of the training sets, we analyze the average and the distributions of the SIDE scores of the instances.\\

\addtolength{\extrarowheight}{\belowrulesep}
\aboverulesep=0pt
\belowrulesep=0pt
\begin{table}[t]
	\centering
	\caption{Different selections for \textit{TL-CodeSum} and \textit{Funcom} training sets.}
	\label{tab:dataset_w_strategies}
	\resizebox{0.6\columnwidth}{!}{%
		\begin{tabular}{lrrr}
			\toprule
			\cellcolor{black}\textcolor{white}{\textbf{Selection}} &  \cellcolor{black}\textcolor{white}{\textbf{TL-CodeSum}} & \cellcolor{black}\textcolor{white}{\textbf{Funcom}} \\
			\midrule
			Full & 53,597 & 1,184,438 \\
			\midrule
			\side{0.5} & 50,073 & 1,080,649 \\
			\side{0.6} & 48,146 & 1,031,647 \\
			\side{0.7} & 44,853 & 952,265 \\
			\side{0.8} & 38,733 & 813,998 \\
			\side{0.9} & 26,258 & 540,170 \\
			\bottomrule
	\end{tabular}}
\end{table}

To answer RQ$_{1}$, we use the SIDE-based filter we define in \secref{sec:selection_strategies}. We use five threshold values, \ie 0.5, 0.6, 0.7, 0.8, and 0.9. We do not use thresholds lower than 0.5 because they would result in negligible dataset reductions (lower than 10\% for both), as we will observe in the results of RQ$_{0}$.
We report information about the different datasets in \tabref{tab:dataset_w_strategies}. 
We apply each filter on the training sets of \textit{TL-CodeSum} and \textit{Funcom}. Such filtering leads to the definition of five new versions of both datasets.

We fine-tune a pre-trained Transformer-based model for each dataset version, \ie both the base one and its six filtered versions, producing 12 fine-tuned models.
We choose to leverage the pre-trained \emph{CodeT5+} \cite{wang2023codet5+} since it has been largely used for code-related tasks \cite{ahmed2024automatic,phan2024repohyper,yang2024important} and, more important, in the code summarization approaches described above. This model is built on the backbone of the well-known T5 model by Raffel \etal \cite{raffel2020exploring}, yet it benefits from specific enhancements tailored for code understanding and generation tasks. During the pre-training phase, \emph{CodeT5+} is first trained on unimodal data, which includes code and comments, employing a combination of pre-training objectives such as span-denoising \cite{raffel2020exploring} and Causal Language Modeling \cite{soltan2022alexatm,tay2022ul2}. Then, it is pre-trained on bi-modal data where pre-training objectives such as text-code contrastive learning, text-code matching, and text-code causal language modeling are employed. It comes with different variants: (i) \emph{CodeT5+} 220M, (ii) \emph{CodeT5+} 770M, (iii) \emph{CodeT5+} 2B, (iv) \emph{CodeT5+} 6B, and (vi) \emph{CodeT5+} 16B.
Since our experimental design would require training, validating and testing 12 models, we decided to fine-tune the \emph{CodeT5+} variant featuring 220M trainable parameters.
This choice aligns with the goal of our investigation: Rather than proposing a new code summarization technique, we aim to use a model that offers a favorable balance between size and training time while still allowing us to observe the relevant phenomenon (if present).

Considering the extensive array of our experiments, we fine-tune for 20 epochs using a batch size of 16. Additionally, we restrict the input length to 512 tokens and the output to 128 tokens, consistent with previous studies leveraging the two datasets we used \cite{mastropaolo2022using,zhou2022automatic,tufano2023automating}. In addition, we conduct the fine-tuning using the standard hyperparameters for \emph{CodeT5+}, which include the AdamW optimizer \cite{loshchilov2017decoupled} and a learning rate of 2e-5, which is the one recommended for (Code)T5 and also used in works leveraging such models \cite{mastropaolo2023towards,ciniselli2024generalizability,mastropaolo2024vul}.

To prevent overfitting, we employ early stopping \cite{prechelt2002early}. After each epoch, we assess the performance of the models by computing the number of correct predictions on the validation set. 
In line with similar research \cite{mastropaolo2023towards,ciniselli2024generalizability}, we implement early stopping with patience of 5 epochs and a delta of 0.01. This means that training will stop if the model's performance does not improve by at least 0.01 for five consecutive epochs. We then select the best-performing checkpoint before early stopping.
We fine-tune a \emph{CodeT5+} model for each training set derived from the selection strategy \ie 12 (2 datasets $\times$ six variants).

After training the models, we assess their performance on the test set dataset that, as previously explained, are the \textit{CoderEval} \cite{yu2024codereval}, and the one from Mastropaolo \etal \cite{mastropaolo2023robustness} which we refer to as the \textit{golden sets}.

In the inference phase, we employ a beam search decoding strategy. In detail, with $k \in \{1, 3, 5\}$, we allow each model to generate the $k$ most probable candidate \textit{summaries} for the given \textit{code}.
To evaluate the generated summaries of each model, we compute the following metrics: BLEU \cite{papineni2002bleu}, METEOR \cite{banerjee:acl2005}, and ROUGE-L \cite{lin2004rouge}.
\textbf{BLEU} is a metric that expresses, within a range from 0 to 1, the similarity between a generated text (candidate) and the target one (oracle). It computes the percentage of $n$-grams of the generated text that appear in the target, where $n \in \{1, 2, 3, 4\}$. 
\textbf{METEOR} is computed as the harmonic mean of unigram precision and recall, with the latter weighted higher than the former. It ranges from 0 to 1. 
\textbf{ROUGE-L} is computed as the length of the longest common subsequence (LCS) between the generated text and the target one and measures the recall by considering the proportion of the LCS relative to the length of the target text.
We do not use SIDE \cite{mastropaolo2024evaluating} as it was employed for selecting training instances and could therefore be unnaturally biased in favor of models trained on filtered datasets.
Also, we do not compute the percentage of exact matches for three reasons. First, exact matches might underestimate the actual performances of the model. Indeed, an exact match implies a correct summary, but many alternative summaries might be as correct (or even more correct, in theory) as the ones in the ground truth for the very nature of this task. Second (also related to the previous point), \textit{CoderEval} \cite{yu2024codereval} provides two summaries for each coding instance, namely \textit{original} (\ie the docstring collected from the original source code), and \textit{human} (\ie the docstring written from scratch by developers during the benchmark creation \cite{yu2024codereval}). The model could have correctly generated only one of them, which are, by definition, both correct alternatives, thus leading to inconsistent results. Third, the dataset provided by Mastropaolo \etal~\cite{mastropaolo2023robustness} includes three different yet semantically equivalent code summaries for each \java method. As previously noted, each of these alternative descriptions is a valid candidate summary.

\begin{figure}[t]
	\centering
	\includegraphics[width=0.65\linewidth]{distributions-plot.pdf}
	\caption{Distribution of SIDE scores for \textit{TL-CodeSum} and \textit{Funcom} training instances.}
	\label{fig:rq0_training_distribution}
\end{figure}

\begin{table*}[t]
	\centering
	\caption{Performance metrics on Top-1 predictions for CoderEval.}
	\label{tab:performance_metrics_codereval}
	\resizebox{\linewidth}{!}{%
		\begin{tabular}{c|l|r|r|rrr|rrr|rrr}
			\toprule
			\rowcolor{black}
			&  &  &  & \multicolumn{3}{c}{\textcolor{white}{\textbf{CoderEval-Original \cite{yu2024codereval}}}} & \multicolumn{3}{c}{\textcolor{white}{\textbf{CoderEval-Human \cite{yu2024codereval}}}} & \multicolumn{3}{c}{\textcolor{white}{\textbf{Mastropaolo \etal \cite{mastropaolo2023robustness}}}} \\
			\rowcolor{gray!20}
			\textbf{Dataset} & \textbf{Selection} & \textbf{\#Tokens} & \textbf{(\%) Saving} & \textbf{BLEU-4} & \textbf{METEOR} & \textbf{ROUGE} & \textbf{BLEU-4} & \textbf{METEOR} & \textbf{ROUGE} & \textbf{BLEU-4} & \textbf{METEOR} & \textbf{ROUGE} \\
			\midrule
			\multirow{6}{*}{\textit{TL-CodeSum \cite{hu2018summarizing}}}
			
			& \emph{Full}      & \cellcolor[HTML]{656565}\color[HTML]{FFFFFF} $\uparrow$ 7.9M &  \cellcolor[HTML]{a3070c}\color[HTML]{FFFFFF} --   & 11.72 & 17.84 & 35.01 & 6.41 & 14.28 & 30.51 & 6.37 & 13.04 & 27.09 \\
			\cmidrule(r){2-13}
			& \side{0.5} & 7.4M & 7\%  & 12.41 & 17.73 & 35.41 & 6.32 & 14.28 & 31.08 & 6.36 & 12.93 & 27.47\\
			& \side{0.6} & 7.1M & 10\% & 13.22 & 17.98 & 36.49 & 6.98 & 14.03 & 31.06 & 6.61 & 13.07 & 27.49\\
			& \side{0.7} & 6.6M & 16\% & 13.17 & 17.93 & 36.14 & 6.79 & 14.83 & 31.90 & 6.30 & 12.95 & 27.60\\
			& \side{0.8} & 5.6M & 28\% & 11.99 & 17.54 & 34.60 & 6.23 & 13.94 & 29.72 & 6.02 & 13.01 & 27.63\\
			& \side{0.9} & \cellcolor[HTML]{656565}\color[HTML]{FFFFFF}  $\downarrow$ 3.7M & \cellcolor[HTML]{026329}\color[HTML]{FFFFFF} 51\% & 11.60 & 17.07 & 33.67 & 6.56 & 14.19 & 30.53 & 5.62 & 12.75 & 27.26   \\
			\midrule
			\rowcolor[gray]{.85} & & & & & & & & & & & & \\
			\midrule
			\multirow{6}{*}{\textit{Funcom \cite{leclair2019neural}}} & \emph{Full}      & \cellcolor[HTML]{656565}\color[HTML]{FFFFFF} $\uparrow$ 108.7M  & \cellcolor[HTML]{a3070c}\color[HTML]{FFFFFF} --   & 14.25 & 17.61 & 36.53 & 5.93 & 12.95 & 28.36 & 6.77 & 12.94 & 28.00 \\
			\cmidrule(r){2-13}
			& \side{0.5} & 99.5M & 9\%  & 15.04 & 18.16 & 37.08 & 6.15 & 13.14 & 29.05 & 6.84 & 12.87 & 27.93 \\
			& \side{0.6} & 95.0M & 13\% & 16.19 & 19.30 & 38.38 & 7.02 & 14.07 & 30.25 & 7.03 & 12.99 & 28.33 \\
			& \side{0.7} & 87.7M & 20\% & 14.77 & 18.92 & 37.39 & 7.49 & 14.19 & 30.14 & 6.62 & 12.79 & 27.78 \\
			& \side{0.8} & 74.2M & 31\% & 14.10 & 18.34 & 37.04 & 6.53 & 13.38 & 28.45 & 6.93 & 12.78 & 27.99 \\
			& \side{0.9} & \cellcolor[HTML]{656565}\color[HTML]{FFFFFF} $\downarrow$ 49.0M & \cellcolor[HTML]{026329}\color[HTML]{FFFFFF} 54\% & 13.65 & 18.08 & 37.12 & 6.78 & 13.61 & 30.07 & 6.81 & 12.95 & 28.07 \\
			\bottomrule
	\end{tabular}}
\end{table*}

We also perform statistical hypothesis tests (Wilcoxon signed-rank test) \cite{wilcoxon1992individual} and Cliff's delta effect size \cite{grissom2005effect} to compare the distributions of the BLEU-4, METEOR, and ROUGE-L of the predictions generated by the different models trained on the filtered training sets with those of the models trained on the full training sets. We use Holm's correction \cite{holm1979simple} to adjust the \textit{p}-values for the multiple tests. We reject the \textit{null hypothesis} (there is no difference between the effectiveness of two given models) if the \emph{p}-value is lower than 0.05.

Finally, we study the Pareto front to analyze the cost-benefit trade-offs between the effectiveness of the models trained on the different selections of \textit{TL-CodeSum} and \textit{Funcom} (benefit, measured with \ie, BLEU, METEOR, and ROUGE-L) and the corresponding training dataset size (cost).

To answer RQ$_{2}$ we compare the selection strategy with \side{0.9} (\ie the most restrictive selection), with a \textit{Random} baseline. In detail, we randomly sample the same number of training instances as those selected with \side{0.9} from each dataset. We compare the effectiveness of the models trained with the training instances selected with \side{0.9} and \textit{Random} measured in terms of the previously described metrics (\ie BLEU-4, METEOR, and ROUGE-L). Again, we perform statistical hypothesis tests (Wilcoxon signed-rank test) \cite{wilcoxon1992individual} and compute the Cliff's delta effect size \cite{grissom2005effect} to compare the distributions of BLEU-4, METEOR, and ROUGE-L of the predictions generated by the \side{0.9} model and the \textit{Random} baseline. We use Holm's correction \cite{holm1979simple} to adjust the \textit{p}-values for the multiple tests. We reject the \textit{null hypothesis} (there is no difference between the two models) if the \emph{p}-value is lower than 0.05.

  
\section{A Structured Guide to \credal}
\label{sec:credal:guide}

In this section, we present a structured guide to \credal (Section~\ref{sec:credal:guide:steps}), along with other materials we developed to support the adoption and use of the methodology (Section~\ref{sec:credal:supp}).  

\subsection{\credal}
\label{sec:credal:guide:steps}

\begin{enumerate}
\item \emph{Define Research Goals and Understand the Data Model.} Set clear objectives for your analysis, whether it be model-driven or, if data is available,  data-driven.
Understand your data and data model, paying attention to its structure, domain, and other details, such as missing relevant details.

\item \emph{Evaluate the Context, Domain Knowledge and Sources.} 
(a) Gain domain-specific insights from additional sources. 
(b) Consider ethical, privacy, and other concerns, including the absence of sensitive but relevant data. 
(c) Evaluate data sources for their biases and omissions.

\item \emph{Exploratory Data Analysis (EDA) and Schema Analysis.}
\begin{itemize}
    \item For data-driven close reading: Identify general patterns in the data, including outliers and unexpected features (for more practical tips on EDA, see \cite{abedjan,downey}). 
    \item For model-driven close reading: Conduct a detailed schema analysis, focusing on the following aspects:
        (a) Identifying entities and relationships;
        (b) Reviewing constraints such as primary keys, foreign keys, unique constraints, and field formats; and,
        (c) Identifying indices and design patterns.
\end{itemize}

\item \emph{Related Schemas Exploration.}
    (a) Identify related schemas and assess their content in comparison to the primary schema;
    (b) Note any elements or relationships that are included or excluded in the related schemas; and,
    (c) Analyze how the comparison can inform improvements or enhancements to the target schema.

\item \emph{Assumptions Loop.}
\begin{enumerate}
    \item Select a small portion of the data model (e.g., an entity, its attribute, or a relationship between entities) where potential interesting bias may be present.
    \item Establish criteria for determining fairness and objectivity before identifying negative bias or inequality. It is essential to distinguish between assumptions and verifiable bias.
    \begin{itemize}
        \item For data-driven close reading: Define metrics to quantify assumed bias or inequality. These may include skewness in data distribution, the presence of empty fields, or other quantifiable characteristics.
        \item For model-driven close reading: Compare the schema with related models, analyze similar fields, and review restrictive types or irregular relationships to validate assumptions. Conduct thorough domain research to substantiate any findings.
    \end{itemize}
    \item Identify potential risks or issues arising from the assumed bias. Consider specific scenarios in which the model may fail or produce unintended consequences.
        i) Assess how sensitive attributes (e.g., gender) influence decision boundaries and overall fairness when legally and ethically permissible;
        ii) Evaluate fairness by analyzing causal pathways and understanding how different factors contribute to model outcomes; and
        iii) The "5 Whys" method is recommended to trace the root cause of bias and understand its origin (details in Section~\ref{sec:credal:supp}). 
    \item Propose solutions to mitigate identified harmful biases and/or enhance the comprehensiveness of the data or schema.
\end{enumerate}

\item \emph{Compile Findings.}
    Generate a conclusion summarizing your observations on the analyzed model. This report should be easily comprehensible for the target audience and ideally provide valuable insights to aid the model developer in its revision.
\end{enumerate}


The overall pipeline of the data close reading methodology is provided in Figure~\ref{fig:close_reading_pipeline}.

\begin{figure}[t]
  \centering
  \includegraphics[width=0.9\linewidth]{imgs/diagram_methodology.png}
  \caption{Visual Representation of \credal}
  \Description{Directed graph depicting the main steps to take during data model close reading}
  \label{fig:close_reading_pipeline}
\end{figure}


\subsection{Supporting Materials}
\label{sec:credal:supp}

\paragraph{Video tutorial.} As a result of the iterative improvements to the methodology, we identified the need to enhance the visual presentation of the methodology to improve comprehension. In response, we developed a video tutorial that presents and explains the key terminology and main stages of the methodology, reinforcing each step with illustrative examples. 

\paragraph{Example of Reading.} To demonstrate the application of the methodology, we developed a comprehensive example reading that describes the entire process from start to finish, illustrating the nuances involved in applying the methodology to a real-world scenario. The selected data model, ``A Secure Students’ Attendance Monitoring System'' (Appendix \ref{sec:example_reading}), was chosen due to a reasonable level of complexity. It provides sufficient depth to thoroughly illustrate \credal while remaining accessible to the target audience. This step-by-step walk-through offered clear instructions, making the methodology application more straightforward for users.

\paragraph{Suggestions and Best Practices.} We identified several key techniques that enhance both the perception and application of the methodology. These insights are organized into practical recommendations:

\begin{itemize} 
    \item \textbf{Use the ``5 Whys'' Method.} This technique can help identify bias and uncover its root causes. The process involves asking ``why'' multiple times to drill down into the underlying issues. The steps are as follows:
        \begin{itemize} 
            \item Define the problem: Clearly articulate the issue to ensure an accurate understanding, as solving an ill-defined problem is challenging. 
            \item Ask the first ``Why''. Begin by questioning why the problem occurred to uncover initial insights. 
            \item Continue asking ``Why''. For each identified reason, continue asking ``why'' until you reach the fundamental cause, repeating this process until a clear solution emerges. 
            \item Practical Tips: Move swiftly from one ``why'' to the next to maintain focus and avoid distractions. Know when to stop, as continuing beyond meaningful responses may lead to diminishing returns. 
    \end{itemize}
    
    \item \textbf{Incorporate Brainstorming Sessions.}  Brainstorming encourages creative thinking and diverse viewpoints, which helps uncover biases that may not be evident in a solo analysis. Key steps include:
        \begin{itemize}
            \item Visualize the goal of the session.
            \item Document all discussions and ideas.
            \item Encourage participants to think aloud and propose varied ideas.
            \item Foster an environment where all ideas are welcomed without immediate criticism.
            \item Collaborate and ask clarifying questions.
            \item Organize the outcomes of the brainstorming session for further analysis.
        \end{itemize}
        
    \item \textbf{Apply the Principles of Literary Close Reading.}  Close reading, rooted in literary analysis, offers a systematic approach to examining textual content in depth. Its principles can be adapted for data model reading:
        \begin{itemize}
            \item Focus on explicit content, avoiding speculative interpretations.
            \item Read slowly and attentively to capture nuanced details that might otherwise be overlooked.
            \item Perform multiple readings, allowing for a comprehensive understanding of the material.
            \item Analyze each element thoroughly to ensure a detailed and accurate interpretation.
        \end{itemize}
    \end{itemize}
    
    \paragraph{Common Pitfalls and Considerations.} When applying the methodology, users may encounter several potential challenges, summarized below, along with advice for mitigating them:
    \begin{itemize} 
        \item \textbf{Overlooking Entity Relationships.}
        Thoroughly explore the relationships between entities within the schema. Neglecting these connections may lead to incomplete or inaccurate conclusions.
        \item \textbf{Managing Large Data Volumes.}  
        A data-driven approach may introduce complexity when dealing with large datasets. Focus on specific areas of interest and relevant data points to avoid becoming overwhelmed by the volume of information.
        \item \textbf{Bias Below the Surface.}  
        Bias is not always apparent at the top levels of the schema. Take multiple perspectives, playing the roles of both the creator and reviewer, to identify potential biases or controversies that may lie deeper in the structure.
        \item \textbf{Working with Unfamiliar Data or Schemas.}  
        When dealing with an unfamiliar domain, consider consulting specialists or leveraging external resources to gain a better understanding before proceeding with bias analysis.
        \item \textbf{Curiosity and Questioning.}  
        Embrace curiosity and do not hesitate to ask fundamental questions. Seemingly basic inquiries can often lead to important discoveries and unveil hidden aspects of the data model.
        \item \textbf{Concluding the Analysis.}  
        After identifying biases, it is essential to clearly articulate your findings and suggest actionable solutions. This ensures that the issues are addressed and mitigated in future iterations of the data model.

    \end{itemize}




\section{Implementation and Evaluation}
\label{sec:evaluation}

We prototype our proposal into a tool \toolName, using approximately 5K lines of OCaml (for the program analysis) and 5K lines of Python code (for the repair). 
In particular, we employ Z3~\cite{DBLP:conf/tacas/MouraB08} as the SMT solver, clingo~\cite{DBLP:books/sp/Lifschitz19} as the ASP solver, and Souffle~\cite{scholz2016fast} as the Datalog engine. %, respectively.
To show the effectiveness, 
we design the experimental evaluation to answer the 
following research questions (RQ):
(Experiments ran on a server with an Intel® Xeon® Platinum 8468V, 504GB RAM, and 192 cores. All the dataset are publicly available from \cite{zenodo_benchmark})

\begin{itemize}[align=left, leftmargin=*,labelindent=0pt]
\item \textbf{RQ1:} How effective is \toolName in verifying CTL properties for relatively small but complex programs, compared to the state-of-the-art tool  \function~\cite{DBLP:conf/sas/UrbanU018}?


\item \textbf{RQ2:} What is the effectiveness of \toolName in detecting real-world bugs, which can be encoded using both CTL and linear temporal logic (LTL), such as non-termination gathered from GitHub \cite{DBLP:conf/sigsoft/ShiXLZCL22} and unresponsive behaviours in protocols  \cite{DBLP:conf/icse/MengDLBR22}, compared with \ultimate~\cite{DBLP:conf/cav/DietschHLP15}?

\item \textbf{RQ3:} How effective is \toolName in repairing CTL violations identified in RQ1 and RQ2? which has not been achieved by any existing tools. 


 

\end{itemize}



% \begin{itemize}[align=left, leftmargin=*,labelindent=0pt]
% \item \textbf{RQ1:} What is the effectiveness of \toolName in verifying CTL properties in a set of relatively small yet challenging programs, compared to the state-of-the-art tools, T2~\cite{DBLP:conf/fmcad/CookKP14},  \function~\cite{DBLP:conf/sas/UrbanU018}, and \ultimate~\cite{DBLP:conf/cav/DietschHLP15}?


% \item \textbf{RQ2:} What is the effectiveness of \toolName in finding  real-world bugs, which can be encoded using CTL properties, such as non-termination 
% gathered from GitHub \cite{DBLP:conf/sigsoft/ShiXLZCL22} and unresponsive behaviours in protocol implementations \cite{DBLP:conf/icse/MengDLBR22}?

% \item \textbf{RQ3:} What is the effectiveness of \toolName in repairing CTL bugs from RQ1--2?

% \end{itemize}

%The benchmark programs are from various sources. More specifically, termination bugs from real-world projects \cite{DBLP:conf/sigsoft/ShiXLZCL22} and CTL analysis \cite{DBLP:conf/fmcad/CookKP14} \cite{DBLP:conf/sas/UrbanU018}, and temporal bugs in real-world protocol implementations \cite{DBLP:conf/icse/MengDLBR22}. 



% \ly{are termination bugs ok? Do we need to add new CTL bugs?}
\subsection{RQ1: Verifying CTL Properties}

% Please add the following required packages to your document preamble:
%  \Xhline{1.5\arrayrulewidth}

\hide{\begin{figure}[!h]
\vspace{-8mm}
\begin{lstlisting}[xleftmargin=0.2em,numbersep=6pt,basicstyle=\footnotesize\ttfamily]
(*@\textcolor{mGray}{//$EF(\m{resp}{\geq}5)$}@*)
int c = *; int resp = 0;
int curr_serv = 5; 
while (curr_serv > 0){ 
 if (*) {  
   c--; 
   curr_serv--;
   resp++;} 
 else if (c<curr_serv){
   curr_serv--; }}
\end{lstlisting} 
\vspace{-2mm}
\caption{A possibly terminating loop} 
\label{fig:terminating_loop}
\vspace{-2mm}
\end{figure}}


%loses precision due to a \emph{dual widening} \cite{DBLP:conf/tacas/CourantU17}, and 

The programs listed in \tabref{tab:comparewithFuntionT2} were obtained from the evaluation benchmark of \function, which includes a total of 83 test cases across over 2,000 lines of code. We categorize these test cases into six groups, labeled according to the types of CTL properties. 
These programs are short but challenging, as they often involve complex loops or require a more precise analysis of the target properties. The \function tends to be conservative, often leading it to return ``unknown" results, resulting in an accuracy rate of 27.7\%. In contrast, \toolName demonstrates advantages with improved accuracy, particularly in \ourToolSmallBenchmark. 
%achieved by the novel loop summaries. 
The failure cases faced by \toolName highlight our limitations when loop guards are not explicitly defined or when LRFs are inadequate to prove termination. 
Although both \function and \toolName struggle to obtain meaningful invariances for infinite loops, the benefits of our loop summaries become more apparent when proving properties related to termination, such as reachability and responsiveness.  




\begin{table}[!t]
\vspace{1.5mm}
\caption{Detecting real-world CTL bugs.}
\normalsize
\label{tab:comparewithCook}
\renewcommand{\arraystretch}{0.95}
\setlength{\tabcolsep}{4pt}  
\begin{tabular}{c|l|c|cc|cc}
\Xhline{1.5\arrayrulewidth}
\multicolumn{1}{l|}{\multirow{2}{*}{\textbf{}}} & \multirow{2}{*}{\textbf{Program}}        & \multirow{2}{*}{\textbf{LoC}} & \multicolumn{2}{c|}{\textbf{\ultimateshort}}   & \multicolumn{2}{c}{\textbf{\toolName}}             \\ \cline{4-7} 
  \multicolumn{1}{l|}{}                           &                                          &                               & \multicolumn{1}{c|}{\textbf{Res.}} & \textbf{Time} & \multicolumn{1}{c|}{\textbf{Res.}} & \textbf{Time} \\ \hline
  1 \xmark                                      & \multirow{2}{*}{\makecell[l]{libvncserver\\(c311535)}}   & 25                            & \multicolumn{1}{c|}{\xmark}      & 2.845         & \multicolumn{1}{c|}{\xmark}      & 0.855         \\  
  1 \cmark                                      &                                          & 27                            & \multicolumn{1}{c|}{\cmark}      & 3.743         & \multicolumn{1}{c|}{\cmark}      & 0.476         \\ \hline
  2 \xmark                                      & \multirow{2}{*}{\makecell[l]{Ffmpeg\\(a6cba06)}}         & 40                            & \multicolumn{1}{c|}{\xmark}      & 15.254        & \multicolumn{1}{c|}{\xmark}      & 0.606         \\  
  2 \cmark                                      &                                          & 44                            & \multicolumn{1}{c|}{\cmark}      & 40.176        & \multicolumn{1}{c|}{\cmark}      & 0.397         \\ \hline
  3 \xmark                                      & \multirow{2}{*}{\makecell[l]{cmus\\(d5396e4)}}           & 87                            & \multicolumn{1}{c|}{\xmark}      & 6.904         & \multicolumn{1}{c|}{\xmark}      & 0.579         \\  
  3 \cmark                                      &                                          & 86                            & \multicolumn{1}{c|}{\cmark}      & 33.572        & \multicolumn{1}{c|}{\cmark}      & 0.986         \\ \hline
  4 \xmark                                      & \multirow{2}{*}{\makecell[l]{e2fsprogs\\(caa6003)}}      & 58                            & \multicolumn{1}{c|}{\xmark}      & 5.952         & \multicolumn{1}{c|}{\xmark}      & 0.923         \\  
  4 \cmark                                      &                                          & 63                            & \multicolumn{1}{c|}{\cmark}      & 4.533         & \multicolumn{1}{c|}{\cmark}      & 0.842         \\ \hline
  5 \xmark                                      & \multirow{2}{*}{\makecell[l]{csound-an...\\(7a611ab)}} & 43                            & \multicolumn{1}{c|}{\xmark}      & 3.654         & \multicolumn{1}{c|}{\xmark}      & 0.782         \\  
  5 \cmark                                      &                                          & 45                            & \multicolumn{1}{c|}{TO}          & -             & \multicolumn{1}{c|}{\cmark}      & 0.648         \\ \hline
  6 \xmark                                      & \multirow{2}{*}{\makecell[l]{fontconfig\\(fa741cd)}}     & 25                            & \multicolumn{1}{c|}{\xmark}      & 3.856         & \multicolumn{1}{c|}{\xmark}      & 0.769         \\  
  6 \cmark                                      &                                          & 25                            & \multicolumn{1}{c|}{Error}       & -             & \multicolumn{1}{c|}{\cmark}      & 0.651         \\ \hline
  7 \xmark                                      & \multirow{2}{*}{\makecell[l]{asterisk\\(3322180)}}       & 22                            & \multicolumn{1}{c|}{\unk}        & 12.687        & \multicolumn{1}{c|}{\unk}        & 0.196         \\  
  7 \cmark                                      &                                          & 25                            & \multicolumn{1}{c|}{\unk}        & 11.325        & \multicolumn{1}{c|}{\unk}        & 0.34          \\ \hline
  8 \xmark                                      & \multirow{2}{*}{\makecell[l]{dpdk\\(cd64eeac)}}          & 45                            & \multicolumn{1}{c|}{\xmark}      & 3.712         & \multicolumn{1}{c|}{\xmark}      & 0.447         \\  
  8 \cmark                                      &                                          & 45                            & \multicolumn{1}{c|}{\cmark}      & 2.97          & \multicolumn{1}{c|}{\unk}        & 0.481         \\ \hline
  9 \xmark                                      & \multirow{2}{*}{\makecell[l]{xorg-server\\(930b9a06)}}   & 19                            & \multicolumn{1}{c|}{\xmark}      & 3.111         & \multicolumn{1}{c|}{\xmark}      & 0.581         \\  
  9 \cmark                                      &                                          & 20                            & \multicolumn{1}{c|}{\cmark}      & 3.101         & \multicolumn{1}{c|}{\cmark}      & 0.409         \\ \hline
  10 \xmark                                      & \multirow{2}{*}{\makecell[l]{pure-ftpd\\(37ad222)}}      & 42                            & \multicolumn{1}{c|}{\cmark}      & 2.555         & \multicolumn{1}{c|}{\xmark}      & 0.933         \\  
  10 \cmark                                      &                                          & 49                            & \multicolumn{1}{c|}{\cmark}        & 2.286         & \multicolumn{1}{c|}{\cmark}      & 0.383         \\ \hline
  11 \xmark  & \multirow{2}{*}{\makecell[l]{live555$_a$\\(181126)}} & 34  & \multicolumn{1}{c|}{\cmark} &  2.715         & \multicolumn{1}{c|}{\xmark}    & 0.513   \\  
  11 \cmark  &     &   37    & \multicolumn{1}{c|}{\cmark} &  2.837         & \multicolumn{1}{c|}{\cmark}      & 0.341 \\ \hline
  12 \xmark  & \multirow{2}{*}{\makecell[l]{openssl\\(b8d2439)}} & 88  & \multicolumn{1}{c|}{\xmark} &  4.15          & \multicolumn{1}{c|}{\xmark}    & 0.78   \\
  12 \cmark  &     &  88     & \multicolumn{1}{c|}{\cmark} &  3.809         & \multicolumn{1}{c|}{\cmark}      & 0.99 \\ \hline
  13 \xmark  & \multirow{2}{*}{\makecell[l]{live555$_b$\\(131205)}} & 83  & \multicolumn{1}{c|}{\xmark} & 2.838         & \multicolumn{1}{c|}{\xmark}    & 0.602     \\  
  13 \cmark  &    &   84     & \multicolumn{1}{c|}{\cmark} &  2.393         & \multicolumn{1}{c|}{\cmark}      & 0.565 \\ \Xhline{1.5\arrayrulewidth}
                                                   & {\bf{Total}}                                  & 1249  & \multicolumn{1}{c|}{\bestBaseLineReal}          & $>$180       & \multicolumn{1}{c|}{\ourToolRealBenchmark}              & 16.01        \\ \Xhline{1.5\arrayrulewidth}
  \end{tabular}
  \end{table}

\subsection{RQ2: CTL Analysis on  Real-world Projects}




Programs in \tabref{tab:comparewithCook} are from real-world repositories, each associated with a Git commit number where developers identify and fix the bug manually. 
In particular, the property used for programs 1-9 (drawn from \cite{DBLP:conf/sigsoft/ShiXLZCL22}) is  \code{AF(Exit())}, capturing non-termination bugs. The properties used for programs 10-13 (drawn from \cite{DBLP:conf/icse/MengDLBR22}) are of the form \code{AG(\phi_1{\rightarrow}AF(\phi_2))}, capturing unresponsive behaviours from the protocol implementation. 
We extracted the main segments of these real-world bugs into smaller programs (under 100 LoC each), preserving features like data structures and pointer arithmetic. Our evaluation includes both buggy (\eg 1\,\xmark) and developer-fixed (\eg 1\,\cmark) versions.
After converting the CTL properties to LTL formulas, we compared our tool with the latest release of UltimateLTL (v0.2.4), a regular participant in SV-COMP \cite{svcomp} with competitive performance. 
Both tools demonstrate high accuracy in bug detection, while \ultimateshort often requires longer processing time. 
This experiment indicates that LRFs can effectively handle commonly seen real-world loops, and \toolName performs a more lightweight summary computation without compromising accuracy. 



%Following the convention in \cite{DBLP:conf/sigsoft/ShiXLZCL22}, t
%Prior works \cite{DBLP:conf/sigsoft/ShiXLZCL22} gathered such examples by extracting 
%\toolName successfully identifies the majority of buggy and correct programs, with the exception of programs 7 and 8. 







{
\begin{table*}[!h]
  \centering
\caption{\label{tab:repair_benchmark}
{Experimental results for repairing CTL bugs. Time spent by the ASP solver is separately recorded. 
}
}
\small
\renewcommand{\arraystretch}{0.95}
  \setlength{\tabcolsep}{9pt}
\begin{tabular}{l|c|c|c|c|c|c|c|c}
  \Xhline{1.5\arrayrulewidth}
  \multicolumn{1}{c|}{\multirow{2}{*}{\textbf{Program}}} & \multicolumn{1}{c|}{\multirow{2}{*}{\shortstack{\textbf{LoC}\\\textbf{(Datalog)}}}} & \multicolumn{3}{c|}{\textbf{Configuration}}                                 & \multicolumn{1}{c|}{\multirow{2}{*}{\textbf{Fixed}}} & \multicolumn{1}{c|}{\multirow{2}{*}{\textbf{\#Patch}}} & \multicolumn{1}{c|}{\multirow{2}{*}{\textbf{ASP(s)}}} & \multirow{2}{*}{\textbf{Total(s)}} \\ \cline{3-5}

  \multicolumn{1}{c|}{}                                  & \multicolumn{1}{c|}{}                              & \multicolumn{1}{c|}{\textbf{Symbols}} & \multicolumn{1}{c|}{\textbf{Facts}} & \multicolumn{1}{c|}{\textbf{Template}} & \multicolumn{1}{c|}{} & \multicolumn{1}{c|}{} & \multicolumn{1}{c|}{}  &                                      \\ \hline

AF\_yEQ5 (\figref{fig:first_Example})                                           & 115                           & 3+0                   & 0+1                & Add                & \cmark     & 1                   & 0.979                              & 1.593                                \\
test\_until.c                                         & 101                            & 0+3                   & 1+0                & Delete                & \cmark     & 1                   & 0.023                              & 0.498                                \\
next.c                                                & 87                            & 0+4                   & 1+0                & Delete                & \cmark     & 1                   & 0.023                              & 0.472                                \\
libvncserver                                          & 118                            & 0+6                   & 1+0                & Delete                & \cmark     & 3                   & 0.049                              & 1.081                                \\
Ffmpeg                                                & 227                           & 0+12                  & 1+0                & Delete                & \cmark     & 4                   & 13.113                              & 13.335                                \\
cmus                                                  & 145                           & 0+12                  & 1+0                & Delete                & \cmark     & 4                   & 0.098                              & 2.052                                \\
e2fsprogs                                             & 109                           & 0+8                   & 1+0                & Delete                & \cmark     & 2                   & 0.075                              & 1.515                                \\
csound-android                                        & 183                           & 0+8                   & 1+0                & Delete                & \cmark     & 4                   & 0.076                              & 1.613                                \\
fontconfig                                            & 190                           & 0+11                  & 1+0                & Delete                & \cmark     & 6                   & 0.098                              & 2.507                                \\
dpdk                                                  & 196                           & 0+12                  & 1+0                & Delete                & \cmark     & 1                   & 0.091                              & 2.006                                \\
xorg-server                                           & 118                            & 0+2                   & 1+0                & Delete                & \cmark     & 2                   & 0.026                              & 0.605                                \\
pure-ftpd                                             & 258                           & 0+21                  & 1+0                & Delete                & \cmark     & 2                   & 0.069                              & 3.590                               \\
live$_a$                                              & 112                            & 3+4                   & 1+1                & Update                & \cmark     & 1                   & 0.552                              & 0.816                                \\
openssl                                               & 315                           & 1+0                   & 0+1                & Add.                & \cmark     & 1                   & 1.188                              & 2.277                                \\
live$_b$                                              & 217                           & 1+0                   & 0+1                & Add                & \cmark     & 1                   & 0.977                              & 1.494                                 \\
  \Xhline{1.5\arrayrulewidth}
\textbf{Total}                                                 & 2491                          &                       &                    &                   &           &                     & 17.437                              & 35.454                               \\ 
  \Xhline{1.5\arrayrulewidth}           
\end{tabular}

\vspace{-2mm}
\end{table*}
}


\subsection{RQ3: Repairing CTL Property Violations} 


\tabref{tab:repair_benchmark} gathers all the program instances (from \tabref{tab:comparewithFuntionT2} and \tabref{tab:comparewithCook}) that violate their specified CTL properties and are sent to \toolName for repair.   
The \textbf{Symbols} column records the number of symbolic constants + symbolic signs, while the \textbf{Facts} column records the number of facts allowed to be removed + added. 
We gradually increase the number of symbols and the maximum number of facts that can be added or deleted. 
The \textbf{Configuration} column shows the first successful configuration that led to finding patches, and we record the total searching time till reaching such configurations. 
We configure \toolName to apply three atomic templates in a breadth-first manner with a depth limit of 1, \ie, \tabref{tab:repair_benchmark} records the patch result after one iteration of the repair. 
The templates are applied sequentially in the order: delete, update, and add. The repair process stops when a correct patch is found or when all three templates have been attempted. 
%without success. 
% Because of this configuration, \toolName only finds one patch for Program 1 (AF\_yEQ5). 
% The patch inserting \plaincode{if (i>10||x==y) \{y=5; return;\}} mentioned in \figref{fig:Patched-program} cannot be found in current configuration, as it requires deleting facts then adding new facts on the updated program.
% The `Configuration' column in \tabref{tab:repair_benchmark} shows the number of symbolic constants and signs, the number of facts allowed to be removed and added, and the template used when a patch is found.

Due to the current configuration, \toolName only finds patch (b) for Program 1 (AF\_yEQ5), while the patch (a) shown in \figref{fig:Patched-program} can be obtained by allowing two iterations of the repair: the first iteration adds the conditional than a second iteration to add a new assignment on the updated program. 
Non-termination bugs are resolved within a single iteration by adding a conditional statement that provides an earlier exit. 
For instance, \figref{fig:term-Patched-program} illustrates the main logic of 1\,\xmark, which enters an infinite loop when \code{\m{linesToRead}{\leq}0}. 
\toolName successfully 
provides a fix that prevents \code{\m{linesToRead}{\leq}0} from occurring before entering the loop. Note that such patches are more desirable which fix the non-termination bug without dropping the loops completely. 
%much like the example shown in  \figref{fig:term-Patched-program}. At the same time, 
Unresponsive bugs involve adding more function calls or assignment modifications. 
%Most repairs were completed within seconds. 

On average, the time taken to solve ASP accounts for 49.2\% (17.437/35.454) of the total repair time. We also keep track of the number of patches that successfully eliminate the CTL violations. More than one patch is available for non-termination bugs, as some patches exit the entire program without entering the loop. 
While all the patches listed are valid, those that intend to cut off the main program logic can be excluded based on the minimum change criteria. 
After a manual inspection of each buggy program shown in \tabref{tab:repair_benchmark}, we confirmed that at least one generated patch is semantically equivalent to the fix provided by the developer. 
As the first tool to achieve automated repair of CTL violations, \toolName successfully resolves all reported bugs. 



\begin{figure}[!t]
\begin{lstlisting}[xleftmargin=6em,numbersep=6pt,basicstyle=\footnotesize\ttfamily]
void main(){ //AF(Exit())
  int lines ToRead = *;
  int h = *;
  (*@\repaircode{if ( linesToRead <= 0 )  return;}@*)
  while(h>0){
    if(linesToRead>h)  
        linesToRead=h; 
    h-=linesToRead;} 
  return;}
\end{lstlisting}
\caption{Fixing a Possible Hang Found in libvncserver \cite{LibVNCClient}}
\label{fig:term-Patched-program}
\end{figure}

  
\section{Discussion and Future Work} \label{sec:disc}

This paper introduces the Flexible Bivariate Beta Mixture Model (FBBMM), a novel probabilistic clustering model leveraging the flexibility of the bivariate beta distribution. Experimental results show that FBBMM outperforms popular clustering algorithms such as $k$-means, MeanShift, DBSCAN, Gaussian Mixture Models, and MBMM, particularly on nonconvex clusters. Its ability to handle a wide range of cluster shapes and correlations makes it highly effective.

FBBMM offers several advantages. Its use of the beta distribution allows for flexible cluster shapes, capturing complex structures more accurately than traditional models. It supports soft clustering, assigning probabilities to data points for belonging to clusters, which is versatile for overlapping clusters. Additionally, FBBMM is generative, capable of producing new data resembling the original dataset, useful for tasks like data augmentation and simulation.

However, FBBMM has limitations, including higher computational complexity due to iterative parameter estimation. Future work could focus on improving efficiency through parallelization or better optimization strategies, extending FBBMM to multivariate data, and enhancing robustness to noise and outliers. Applying FBBMM in diverse domains such as bioinformatics and image analysis could further validate its versatility and impact.


\section{Conclusion}


\sdeni{}{We introduced convex-concave generative-adversarial characterization of inverse Nash equilibria for a large class of games, including normal-form, finite state and action Markov games, and a number of continuous state and action Markov games. This novel formulation then allowed us to obtain polynomial-time computation guarantees for inverse equilibria in these games, a rather surprising result since the computation of a Nash equilibrium is in general PPAD-complete. Our result can be thus seen as a positive computation result for game theory. We then extended our characterization to a multiagent apprenticeship learning setting, where we souught to not only rationalize the observed behavior as an inverse Nash equilibrium but also make predictions based off the inverse Nash equilibrium, and have shown in experiments on prices in Spanish electricity markets that our approach to solving multiagent apprenticship learning can be effective at predicting behavior in multiagent systems. The approach to inverse game theory that we provided in this paper is a highly flexible one and thus can be used to solve inverse equilibrium beyond inverse Nash equilibria and future work could explore ways to extend our approach to other game-theoretic settings and equilibrium concepts.}


\amy{discuss extenstion to other eqm concepts? CE, CCE, etc.}

\amy{and other idea from yesterday. check text thread?}



% \smallskip
% \myparagraph{Acknowledgments} We thank the reviewers for their comments.
% The work by Moshe Tennenholtz was supported by funding from the
% European Research Council (ERC) under the European Union's Horizon
% 2020 research and innovation programme (grant agreement 740435).


\bibliographystyle{ACM-Reference-Format}
\bibliography{main}

\newpage 
\appendix
\subsection{Lloyd-Max Algorithm}
\label{subsec:Lloyd-Max}
For a given quantization bitwidth $B$ and an operand $\bm{X}$, the Lloyd-Max algorithm finds $2^B$ quantization levels $\{\hat{x}_i\}_{i=1}^{2^B}$ such that quantizing $\bm{X}$ by rounding each scalar in $\bm{X}$ to the nearest quantization level minimizes the quantization MSE. 

The algorithm starts with an initial guess of quantization levels and then iteratively computes quantization thresholds $\{\tau_i\}_{i=1}^{2^B-1}$ and updates quantization levels $\{\hat{x}_i\}_{i=1}^{2^B}$. Specifically, at iteration $n$, thresholds are set to the midpoints of the previous iteration's levels:
\begin{align*}
    \tau_i^{(n)}=\frac{\hat{x}_i^{(n-1)}+\hat{x}_{i+1}^{(n-1)}}2 \text{ for } i=1\ldots 2^B-1
\end{align*}
Subsequently, the quantization levels are re-computed as conditional means of the data regions defined by the new thresholds:
\begin{align*}
    \hat{x}_i^{(n)}=\mathbb{E}\left[ \bm{X} \big| \bm{X}\in [\tau_{i-1}^{(n)},\tau_i^{(n)}] \right] \text{ for } i=1\ldots 2^B
\end{align*}
where to satisfy boundary conditions we have $\tau_0=-\infty$ and $\tau_{2^B}=\infty$. The algorithm iterates the above steps until convergence.

Figure \ref{fig:lm_quant} compares the quantization levels of a $7$-bit floating point (E3M3) quantizer (left) to a $7$-bit Lloyd-Max quantizer (right) when quantizing a layer of weights from the GPT3-126M model at a per-tensor granularity. As shown, the Lloyd-Max quantizer achieves substantially lower quantization MSE. Further, Table \ref{tab:FP7_vs_LM7} shows the superior perplexity achieved by Lloyd-Max quantizers for bitwidths of $7$, $6$ and $5$. The difference between the quantizers is clear at 5 bits, where per-tensor FP quantization incurs a drastic and unacceptable increase in perplexity, while Lloyd-Max quantization incurs a much smaller increase. Nevertheless, we note that even the optimal Lloyd-Max quantizer incurs a notable ($\sim 1.5$) increase in perplexity due to the coarse granularity of quantization. 

\begin{figure}[h]
  \centering
  \includegraphics[width=0.7\linewidth]{sections/figures/LM7_FP7.pdf}
  \caption{\small Quantization levels and the corresponding quantization MSE of Floating Point (left) vs Lloyd-Max (right) Quantizers for a layer of weights in the GPT3-126M model.}
  \label{fig:lm_quant}
\end{figure}

\begin{table}[h]\scriptsize
\begin{center}
\caption{\label{tab:FP7_vs_LM7} \small Comparing perplexity (lower is better) achieved by floating point quantizers and Lloyd-Max quantizers on a GPT3-126M model for the Wikitext-103 dataset.}
\begin{tabular}{c|cc|c}
\hline
 \multirow{2}{*}{\textbf{Bitwidth}} & \multicolumn{2}{|c|}{\textbf{Floating-Point Quantizer}} & \textbf{Lloyd-Max Quantizer} \\
 & Best Format & Wikitext-103 Perplexity & Wikitext-103 Perplexity \\
\hline
7 & E3M3 & 18.32 & 18.27 \\
6 & E3M2 & 19.07 & 18.51 \\
5 & E4M0 & 43.89 & 19.71 \\
\hline
\end{tabular}
\end{center}
\end{table}

\subsection{Proof of Local Optimality of LO-BCQ}
\label{subsec:lobcq_opt_proof}
For a given block $\bm{b}_j$, the quantization MSE during LO-BCQ can be empirically evaluated as $\frac{1}{L_b}\lVert \bm{b}_j- \bm{\hat{b}}_j\rVert^2_2$ where $\bm{\hat{b}}_j$ is computed from equation (\ref{eq:clustered_quantization_definition}) as $C_{f(\bm{b}_j)}(\bm{b}_j)$. Further, for a given block cluster $\mathcal{B}_i$, we compute the quantization MSE as $\frac{1}{|\mathcal{B}_{i}|}\sum_{\bm{b} \in \mathcal{B}_{i}} \frac{1}{L_b}\lVert \bm{b}- C_i^{(n)}(\bm{b})\rVert^2_2$. Therefore, at the end of iteration $n$, we evaluate the overall quantization MSE $J^{(n)}$ for a given operand $\bm{X}$ composed of $N_c$ block clusters as:
\begin{align*}
    \label{eq:mse_iter_n}
    J^{(n)} = \frac{1}{N_c} \sum_{i=1}^{N_c} \frac{1}{|\mathcal{B}_{i}^{(n)}|}\sum_{\bm{v} \in \mathcal{B}_{i}^{(n)}} \frac{1}{L_b}\lVert \bm{b}- B_i^{(n)}(\bm{b})\rVert^2_2
\end{align*}

At the end of iteration $n$, the codebooks are updated from $\mathcal{C}^{(n-1)}$ to $\mathcal{C}^{(n)}$. However, the mapping of a given vector $\bm{b}_j$ to quantizers $\mathcal{C}^{(n)}$ remains as  $f^{(n)}(\bm{b}_j)$. At the next iteration, during the vector clustering step, $f^{(n+1)}(\bm{b}_j)$ finds new mapping of $\bm{b}_j$ to updated codebooks $\mathcal{C}^{(n)}$ such that the quantization MSE over the candidate codebooks is minimized. Therefore, we obtain the following result for $\bm{b}_j$:
\begin{align*}
\frac{1}{L_b}\lVert \bm{b}_j - C_{f^{(n+1)}(\bm{b}_j)}^{(n)}(\bm{b}_j)\rVert^2_2 \le \frac{1}{L_b}\lVert \bm{b}_j - C_{f^{(n)}(\bm{b}_j)}^{(n)}(\bm{b}_j)\rVert^2_2
\end{align*}

That is, quantizing $\bm{b}_j$ at the end of the block clustering step of iteration $n+1$ results in lower quantization MSE compared to quantizing at the end of iteration $n$. Since this is true for all $\bm{b} \in \bm{X}$, we assert the following:
\begin{equation}
\begin{split}
\label{eq:mse_ineq_1}
    \tilde{J}^{(n+1)} &= \frac{1}{N_c} \sum_{i=1}^{N_c} \frac{1}{|\mathcal{B}_{i}^{(n+1)}|}\sum_{\bm{b} \in \mathcal{B}_{i}^{(n+1)}} \frac{1}{L_b}\lVert \bm{b} - C_i^{(n)}(b)\rVert^2_2 \le J^{(n)}
\end{split}
\end{equation}
where $\tilde{J}^{(n+1)}$ is the the quantization MSE after the vector clustering step at iteration $n+1$.

Next, during the codebook update step (\ref{eq:quantizers_update}) at iteration $n+1$, the per-cluster codebooks $\mathcal{C}^{(n)}$ are updated to $\mathcal{C}^{(n+1)}$ by invoking the Lloyd-Max algorithm \citep{Lloyd}. We know that for any given value distribution, the Lloyd-Max algorithm minimizes the quantization MSE. Therefore, for a given vector cluster $\mathcal{B}_i$ we obtain the following result:

\begin{equation}
    \frac{1}{|\mathcal{B}_{i}^{(n+1)}|}\sum_{\bm{b} \in \mathcal{B}_{i}^{(n+1)}} \frac{1}{L_b}\lVert \bm{b}- C_i^{(n+1)}(\bm{b})\rVert^2_2 \le \frac{1}{|\mathcal{B}_{i}^{(n+1)}|}\sum_{\bm{b} \in \mathcal{B}_{i}^{(n+1)}} \frac{1}{L_b}\lVert \bm{b}- C_i^{(n)}(\bm{b})\rVert^2_2
\end{equation}

The above equation states that quantizing the given block cluster $\mathcal{B}_i$ after updating the associated codebook from $C_i^{(n)}$ to $C_i^{(n+1)}$ results in lower quantization MSE. Since this is true for all the block clusters, we derive the following result: 
\begin{equation}
\begin{split}
\label{eq:mse_ineq_2}
     J^{(n+1)} &= \frac{1}{N_c} \sum_{i=1}^{N_c} \frac{1}{|\mathcal{B}_{i}^{(n+1)}|}\sum_{\bm{b} \in \mathcal{B}_{i}^{(n+1)}} \frac{1}{L_b}\lVert \bm{b}- C_i^{(n+1)}(\bm{b})\rVert^2_2  \le \tilde{J}^{(n+1)}   
\end{split}
\end{equation}

Following (\ref{eq:mse_ineq_1}) and (\ref{eq:mse_ineq_2}), we find that the quantization MSE is non-increasing for each iteration, that is, $J^{(1)} \ge J^{(2)} \ge J^{(3)} \ge \ldots \ge J^{(M)}$ where $M$ is the maximum number of iterations. 
%Therefore, we can say that if the algorithm converges, then it must be that it has converged to a local minimum. 
\hfill $\blacksquare$


\begin{figure}
    \begin{center}
    \includegraphics[width=0.5\textwidth]{sections//figures/mse_vs_iter.pdf}
    \end{center}
    \caption{\small NMSE vs iterations during LO-BCQ compared to other block quantization proposals}
    \label{fig:nmse_vs_iter}
\end{figure}

Figure \ref{fig:nmse_vs_iter} shows the empirical convergence of LO-BCQ across several block lengths and number of codebooks. Also, the MSE achieved by LO-BCQ is compared to baselines such as MXFP and VSQ. As shown, LO-BCQ converges to a lower MSE than the baselines. Further, we achieve better convergence for larger number of codebooks ($N_c$) and for a smaller block length ($L_b$), both of which increase the bitwidth of BCQ (see Eq \ref{eq:bitwidth_bcq}).


\subsection{Additional Accuracy Results}
%Table \ref{tab:lobcq_config} lists the various LOBCQ configurations and their corresponding bitwidths.
\begin{table}
\setlength{\tabcolsep}{4.75pt}
\begin{center}
\caption{\label{tab:lobcq_config} Various LO-BCQ configurations and their bitwidths.}
\begin{tabular}{|c||c|c|c|c||c|c||c|} 
\hline
 & \multicolumn{4}{|c||}{$L_b=8$} & \multicolumn{2}{|c||}{$L_b=4$} & $L_b=2$ \\
 \hline
 \backslashbox{$L_A$\kern-1em}{\kern-1em$N_c$} & 2 & 4 & 8 & 16 & 2 & 4 & 2 \\
 \hline
 64 & 4.25 & 4.375 & 4.5 & 4.625 & 4.375 & 4.625 & 4.625\\
 \hline
 32 & 4.375 & 4.5 & 4.625& 4.75 & 4.5 & 4.75 & 4.75 \\
 \hline
 16 & 4.625 & 4.75& 4.875 & 5 & 4.75 & 5 & 5 \\
 \hline
\end{tabular}
\end{center}
\end{table}

%\subsection{Perplexity achieved by various LO-BCQ configurations on Wikitext-103 dataset}

\begin{table} \centering
\begin{tabular}{|c||c|c|c|c||c|c||c|} 
\hline
 $L_b \rightarrow$& \multicolumn{4}{c||}{8} & \multicolumn{2}{c||}{4} & 2\\
 \hline
 \backslashbox{$L_A$\kern-1em}{\kern-1em$N_c$} & 2 & 4 & 8 & 16 & 2 & 4 & 2  \\
 %$N_c \rightarrow$ & 2 & 4 & 8 & 16 & 2 & 4 & 2 \\
 \hline
 \hline
 \multicolumn{8}{c}{GPT3-1.3B (FP32 PPL = 9.98)} \\ 
 \hline
 \hline
 64 & 10.40 & 10.23 & 10.17 & 10.15 &  10.28 & 10.18 & 10.19 \\
 \hline
 32 & 10.25 & 10.20 & 10.15 & 10.12 &  10.23 & 10.17 & 10.17 \\
 \hline
 16 & 10.22 & 10.16 & 10.10 & 10.09 &  10.21 & 10.14 & 10.16 \\
 \hline
  \hline
 \multicolumn{8}{c}{GPT3-8B (FP32 PPL = 7.38)} \\ 
 \hline
 \hline
 64 & 7.61 & 7.52 & 7.48 &  7.47 &  7.55 &  7.49 & 7.50 \\
 \hline
 32 & 7.52 & 7.50 & 7.46 &  7.45 &  7.52 &  7.48 & 7.48  \\
 \hline
 16 & 7.51 & 7.48 & 7.44 &  7.44 &  7.51 &  7.49 & 7.47  \\
 \hline
\end{tabular}
\caption{\label{tab:ppl_gpt3_abalation} Wikitext-103 perplexity across GPT3-1.3B and 8B models.}
\end{table}

\begin{table} \centering
\begin{tabular}{|c||c|c|c|c||} 
\hline
 $L_b \rightarrow$& \multicolumn{4}{c||}{8}\\
 \hline
 \backslashbox{$L_A$\kern-1em}{\kern-1em$N_c$} & 2 & 4 & 8 & 16 \\
 %$N_c \rightarrow$ & 2 & 4 & 8 & 16 & 2 & 4 & 2 \\
 \hline
 \hline
 \multicolumn{5}{|c|}{Llama2-7B (FP32 PPL = 5.06)} \\ 
 \hline
 \hline
 64 & 5.31 & 5.26 & 5.19 & 5.18  \\
 \hline
 32 & 5.23 & 5.25 & 5.18 & 5.15  \\
 \hline
 16 & 5.23 & 5.19 & 5.16 & 5.14  \\
 \hline
 \multicolumn{5}{|c|}{Nemotron4-15B (FP32 PPL = 5.87)} \\ 
 \hline
 \hline
 64  & 6.3 & 6.20 & 6.13 & 6.08  \\
 \hline
 32  & 6.24 & 6.12 & 6.07 & 6.03  \\
 \hline
 16  & 6.12 & 6.14 & 6.04 & 6.02  \\
 \hline
 \multicolumn{5}{|c|}{Nemotron4-340B (FP32 PPL = 3.48)} \\ 
 \hline
 \hline
 64 & 3.67 & 3.62 & 3.60 & 3.59 \\
 \hline
 32 & 3.63 & 3.61 & 3.59 & 3.56 \\
 \hline
 16 & 3.61 & 3.58 & 3.57 & 3.55 \\
 \hline
\end{tabular}
\caption{\label{tab:ppl_llama7B_nemo15B} Wikitext-103 perplexity compared to FP32 baseline in Llama2-7B and Nemotron4-15B, 340B models}
\end{table}

%\subsection{Perplexity achieved by various LO-BCQ configurations on MMLU dataset}


\begin{table} \centering
\begin{tabular}{|c||c|c|c|c||c|c|c|c|} 
\hline
 $L_b \rightarrow$& \multicolumn{4}{c||}{8} & \multicolumn{4}{c||}{8}\\
 \hline
 \backslashbox{$L_A$\kern-1em}{\kern-1em$N_c$} & 2 & 4 & 8 & 16 & 2 & 4 & 8 & 16  \\
 %$N_c \rightarrow$ & 2 & 4 & 8 & 16 & 2 & 4 & 2 \\
 \hline
 \hline
 \multicolumn{5}{|c|}{Llama2-7B (FP32 Accuracy = 45.8\%)} & \multicolumn{4}{|c|}{Llama2-70B (FP32 Accuracy = 69.12\%)} \\ 
 \hline
 \hline
 64 & 43.9 & 43.4 & 43.9 & 44.9 & 68.07 & 68.27 & 68.17 & 68.75 \\
 \hline
 32 & 44.5 & 43.8 & 44.9 & 44.5 & 68.37 & 68.51 & 68.35 & 68.27  \\
 \hline
 16 & 43.9 & 42.7 & 44.9 & 45 & 68.12 & 68.77 & 68.31 & 68.59  \\
 \hline
 \hline
 \multicolumn{5}{|c|}{GPT3-22B (FP32 Accuracy = 38.75\%)} & \multicolumn{4}{|c|}{Nemotron4-15B (FP32 Accuracy = 64.3\%)} \\ 
 \hline
 \hline
 64 & 36.71 & 38.85 & 38.13 & 38.92 & 63.17 & 62.36 & 63.72 & 64.09 \\
 \hline
 32 & 37.95 & 38.69 & 39.45 & 38.34 & 64.05 & 62.30 & 63.8 & 64.33  \\
 \hline
 16 & 38.88 & 38.80 & 38.31 & 38.92 & 63.22 & 63.51 & 63.93 & 64.43  \\
 \hline
\end{tabular}
\caption{\label{tab:mmlu_abalation} Accuracy on MMLU dataset across GPT3-22B, Llama2-7B, 70B and Nemotron4-15B models.}
\end{table}


%\subsection{Perplexity achieved by various LO-BCQ configurations on LM evaluation harness}

\begin{table} \centering
\begin{tabular}{|c||c|c|c|c||c|c|c|c|} 
\hline
 $L_b \rightarrow$& \multicolumn{4}{c||}{8} & \multicolumn{4}{c||}{8}\\
 \hline
 \backslashbox{$L_A$\kern-1em}{\kern-1em$N_c$} & 2 & 4 & 8 & 16 & 2 & 4 & 8 & 16  \\
 %$N_c \rightarrow$ & 2 & 4 & 8 & 16 & 2 & 4 & 2 \\
 \hline
 \hline
 \multicolumn{5}{|c|}{Race (FP32 Accuracy = 37.51\%)} & \multicolumn{4}{|c|}{Boolq (FP32 Accuracy = 64.62\%)} \\ 
 \hline
 \hline
 64 & 36.94 & 37.13 & 36.27 & 37.13 & 63.73 & 62.26 & 63.49 & 63.36 \\
 \hline
 32 & 37.03 & 36.36 & 36.08 & 37.03 & 62.54 & 63.51 & 63.49 & 63.55  \\
 \hline
 16 & 37.03 & 37.03 & 36.46 & 37.03 & 61.1 & 63.79 & 63.58 & 63.33  \\
 \hline
 \hline
 \multicolumn{5}{|c|}{Winogrande (FP32 Accuracy = 58.01\%)} & \multicolumn{4}{|c|}{Piqa (FP32 Accuracy = 74.21\%)} \\ 
 \hline
 \hline
 64 & 58.17 & 57.22 & 57.85 & 58.33 & 73.01 & 73.07 & 73.07 & 72.80 \\
 \hline
 32 & 59.12 & 58.09 & 57.85 & 58.41 & 73.01 & 73.94 & 72.74 & 73.18  \\
 \hline
 16 & 57.93 & 58.88 & 57.93 & 58.56 & 73.94 & 72.80 & 73.01 & 73.94  \\
 \hline
\end{tabular}
\caption{\label{tab:mmlu_abalation} Accuracy on LM evaluation harness tasks on GPT3-1.3B model.}
\end{table}

\begin{table} \centering
\begin{tabular}{|c||c|c|c|c||c|c|c|c|} 
\hline
 $L_b \rightarrow$& \multicolumn{4}{c||}{8} & \multicolumn{4}{c||}{8}\\
 \hline
 \backslashbox{$L_A$\kern-1em}{\kern-1em$N_c$} & 2 & 4 & 8 & 16 & 2 & 4 & 8 & 16  \\
 %$N_c \rightarrow$ & 2 & 4 & 8 & 16 & 2 & 4 & 2 \\
 \hline
 \hline
 \multicolumn{5}{|c|}{Race (FP32 Accuracy = 41.34\%)} & \multicolumn{4}{|c|}{Boolq (FP32 Accuracy = 68.32\%)} \\ 
 \hline
 \hline
 64 & 40.48 & 40.10 & 39.43 & 39.90 & 69.20 & 68.41 & 69.45 & 68.56 \\
 \hline
 32 & 39.52 & 39.52 & 40.77 & 39.62 & 68.32 & 67.43 & 68.17 & 69.30  \\
 \hline
 16 & 39.81 & 39.71 & 39.90 & 40.38 & 68.10 & 66.33 & 69.51 & 69.42  \\
 \hline
 \hline
 \multicolumn{5}{|c|}{Winogrande (FP32 Accuracy = 67.88\%)} & \multicolumn{4}{|c|}{Piqa (FP32 Accuracy = 78.78\%)} \\ 
 \hline
 \hline
 64 & 66.85 & 66.61 & 67.72 & 67.88 & 77.31 & 77.42 & 77.75 & 77.64 \\
 \hline
 32 & 67.25 & 67.72 & 67.72 & 67.00 & 77.31 & 77.04 & 77.80 & 77.37  \\
 \hline
 16 & 68.11 & 68.90 & 67.88 & 67.48 & 77.37 & 78.13 & 78.13 & 77.69  \\
 \hline
\end{tabular}
\caption{\label{tab:mmlu_abalation} Accuracy on LM evaluation harness tasks on GPT3-8B model.}
\end{table}

\begin{table} \centering
\begin{tabular}{|c||c|c|c|c||c|c|c|c|} 
\hline
 $L_b \rightarrow$& \multicolumn{4}{c||}{8} & \multicolumn{4}{c||}{8}\\
 \hline
 \backslashbox{$L_A$\kern-1em}{\kern-1em$N_c$} & 2 & 4 & 8 & 16 & 2 & 4 & 8 & 16  \\
 %$N_c \rightarrow$ & 2 & 4 & 8 & 16 & 2 & 4 & 2 \\
 \hline
 \hline
 \multicolumn{5}{|c|}{Race (FP32 Accuracy = 40.67\%)} & \multicolumn{4}{|c|}{Boolq (FP32 Accuracy = 76.54\%)} \\ 
 \hline
 \hline
 64 & 40.48 & 40.10 & 39.43 & 39.90 & 75.41 & 75.11 & 77.09 & 75.66 \\
 \hline
 32 & 39.52 & 39.52 & 40.77 & 39.62 & 76.02 & 76.02 & 75.96 & 75.35  \\
 \hline
 16 & 39.81 & 39.71 & 39.90 & 40.38 & 75.05 & 73.82 & 75.72 & 76.09  \\
 \hline
 \hline
 \multicolumn{5}{|c|}{Winogrande (FP32 Accuracy = 70.64\%)} & \multicolumn{4}{|c|}{Piqa (FP32 Accuracy = 79.16\%)} \\ 
 \hline
 \hline
 64 & 69.14 & 70.17 & 70.17 & 70.56 & 78.24 & 79.00 & 78.62 & 78.73 \\
 \hline
 32 & 70.96 & 69.69 & 71.27 & 69.30 & 78.56 & 79.49 & 79.16 & 78.89  \\
 \hline
 16 & 71.03 & 69.53 & 69.69 & 70.40 & 78.13 & 79.16 & 79.00 & 79.00  \\
 \hline
\end{tabular}
\caption{\label{tab:mmlu_abalation} Accuracy on LM evaluation harness tasks on GPT3-22B model.}
\end{table}

\begin{table} \centering
\begin{tabular}{|c||c|c|c|c||c|c|c|c|} 
\hline
 $L_b \rightarrow$& \multicolumn{4}{c||}{8} & \multicolumn{4}{c||}{8}\\
 \hline
 \backslashbox{$L_A$\kern-1em}{\kern-1em$N_c$} & 2 & 4 & 8 & 16 & 2 & 4 & 8 & 16  \\
 %$N_c \rightarrow$ & 2 & 4 & 8 & 16 & 2 & 4 & 2 \\
 \hline
 \hline
 \multicolumn{5}{|c|}{Race (FP32 Accuracy = 44.4\%)} & \multicolumn{4}{|c|}{Boolq (FP32 Accuracy = 79.29\%)} \\ 
 \hline
 \hline
 64 & 42.49 & 42.51 & 42.58 & 43.45 & 77.58 & 77.37 & 77.43 & 78.1 \\
 \hline
 32 & 43.35 & 42.49 & 43.64 & 43.73 & 77.86 & 75.32 & 77.28 & 77.86  \\
 \hline
 16 & 44.21 & 44.21 & 43.64 & 42.97 & 78.65 & 77 & 76.94 & 77.98  \\
 \hline
 \hline
 \multicolumn{5}{|c|}{Winogrande (FP32 Accuracy = 69.38\%)} & \multicolumn{4}{|c|}{Piqa (FP32 Accuracy = 78.07\%)} \\ 
 \hline
 \hline
 64 & 68.9 & 68.43 & 69.77 & 68.19 & 77.09 & 76.82 & 77.09 & 77.86 \\
 \hline
 32 & 69.38 & 68.51 & 68.82 & 68.90 & 78.07 & 76.71 & 78.07 & 77.86  \\
 \hline
 16 & 69.53 & 67.09 & 69.38 & 68.90 & 77.37 & 77.8 & 77.91 & 77.69  \\
 \hline
\end{tabular}
\caption{\label{tab:mmlu_abalation} Accuracy on LM evaluation harness tasks on Llama2-7B model.}
\end{table}

\begin{table} \centering
\begin{tabular}{|c||c|c|c|c||c|c|c|c|} 
\hline
 $L_b \rightarrow$& \multicolumn{4}{c||}{8} & \multicolumn{4}{c||}{8}\\
 \hline
 \backslashbox{$L_A$\kern-1em}{\kern-1em$N_c$} & 2 & 4 & 8 & 16 & 2 & 4 & 8 & 16  \\
 %$N_c \rightarrow$ & 2 & 4 & 8 & 16 & 2 & 4 & 2 \\
 \hline
 \hline
 \multicolumn{5}{|c|}{Race (FP32 Accuracy = 48.8\%)} & \multicolumn{4}{|c|}{Boolq (FP32 Accuracy = 85.23\%)} \\ 
 \hline
 \hline
 64 & 49.00 & 49.00 & 49.28 & 48.71 & 82.82 & 84.28 & 84.03 & 84.25 \\
 \hline
 32 & 49.57 & 48.52 & 48.33 & 49.28 & 83.85 & 84.46 & 84.31 & 84.93  \\
 \hline
 16 & 49.85 & 49.09 & 49.28 & 48.99 & 85.11 & 84.46 & 84.61 & 83.94  \\
 \hline
 \hline
 \multicolumn{5}{|c|}{Winogrande (FP32 Accuracy = 79.95\%)} & \multicolumn{4}{|c|}{Piqa (FP32 Accuracy = 81.56\%)} \\ 
 \hline
 \hline
 64 & 78.77 & 78.45 & 78.37 & 79.16 & 81.45 & 80.69 & 81.45 & 81.5 \\
 \hline
 32 & 78.45 & 79.01 & 78.69 & 80.66 & 81.56 & 80.58 & 81.18 & 81.34  \\
 \hline
 16 & 79.95 & 79.56 & 79.79 & 79.72 & 81.28 & 81.66 & 81.28 & 80.96  \\
 \hline
\end{tabular}
\caption{\label{tab:mmlu_abalation} Accuracy on LM evaluation harness tasks on Llama2-70B model.}
\end{table}

%\section{MSE Studies}
%\textcolor{red}{TODO}


\subsection{Number Formats and Quantization Method}
\label{subsec:numFormats_quantMethod}
\subsubsection{Integer Format}
An $n$-bit signed integer (INT) is typically represented with a 2s-complement format \citep{yao2022zeroquant,xiao2023smoothquant,dai2021vsq}, where the most significant bit denotes the sign.

\subsubsection{Floating Point Format}
An $n$-bit signed floating point (FP) number $x$ comprises of a 1-bit sign ($x_{\mathrm{sign}}$), $B_m$-bit mantissa ($x_{\mathrm{mant}}$) and $B_e$-bit exponent ($x_{\mathrm{exp}}$) such that $B_m+B_e=n-1$. The associated constant exponent bias ($E_{\mathrm{bias}}$) is computed as $(2^{{B_e}-1}-1)$. We denote this format as $E_{B_e}M_{B_m}$.  

\subsubsection{Quantization Scheme}
\label{subsec:quant_method}
A quantization scheme dictates how a given unquantized tensor is converted to its quantized representation. We consider FP formats for the purpose of illustration. Given an unquantized tensor $\bm{X}$ and an FP format $E_{B_e}M_{B_m}$, we first, we compute the quantization scale factor $s_X$ that maps the maximum absolute value of $\bm{X}$ to the maximum quantization level of the $E_{B_e}M_{B_m}$ format as follows:
\begin{align}
\label{eq:sf}
    s_X = \frac{\mathrm{max}(|\bm{X}|)}{\mathrm{max}(E_{B_e}M_{B_m})}
\end{align}
In the above equation, $|\cdot|$ denotes the absolute value function.

Next, we scale $\bm{X}$ by $s_X$ and quantize it to $\hat{\bm{X}}$ by rounding it to the nearest quantization level of $E_{B_e}M_{B_m}$ as:

\begin{align}
\label{eq:tensor_quant}
    \hat{\bm{X}} = \text{round-to-nearest}\left(\frac{\bm{X}}{s_X}, E_{B_e}M_{B_m}\right)
\end{align}

We perform dynamic max-scaled quantization \citep{wu2020integer}, where the scale factor $s$ for activations is dynamically computed during runtime.

\subsection{Vector Scaled Quantization}
\begin{wrapfigure}{r}{0.35\linewidth}
  \centering
  \includegraphics[width=\linewidth]{sections/figures/vsquant.jpg}
  \caption{\small Vectorwise decomposition for per-vector scaled quantization (VSQ \citep{dai2021vsq}).}
  \label{fig:vsquant}
\end{wrapfigure}
During VSQ \citep{dai2021vsq}, the operand tensors are decomposed into 1D vectors in a hardware friendly manner as shown in Figure \ref{fig:vsquant}. Since the decomposed tensors are used as operands in matrix multiplications during inference, it is beneficial to perform this decomposition along the reduction dimension of the multiplication. The vectorwise quantization is performed similar to tensorwise quantization described in Equations \ref{eq:sf} and \ref{eq:tensor_quant}, where a scale factor $s_v$ is required for each vector $\bm{v}$ that maps the maximum absolute value of that vector to the maximum quantization level. While smaller vector lengths can lead to larger accuracy gains, the associated memory and computational overheads due to the per-vector scale factors increases. To alleviate these overheads, VSQ \citep{dai2021vsq} proposed a second level quantization of the per-vector scale factors to unsigned integers, while MX \citep{rouhani2023shared} quantizes them to integer powers of 2 (denoted as $2^{INT}$).

\subsubsection{MX Format}
The MX format proposed in \citep{rouhani2023microscaling} introduces the concept of sub-block shifting. For every two scalar elements of $b$-bits each, there is a shared exponent bit. The value of this exponent bit is determined through an empirical analysis that targets minimizing quantization MSE. We note that the FP format $E_{1}M_{b}$ is strictly better than MX from an accuracy perspective since it allocates a dedicated exponent bit to each scalar as opposed to sharing it across two scalars. Therefore, we conservatively bound the accuracy of a $b+2$-bit signed MX format with that of a $E_{1}M_{b}$ format in our comparisons. For instance, we use E1M2 format as a proxy for MX4.

\begin{figure}
    \centering
    \includegraphics[width=1\linewidth]{sections//figures/BlockFormats.pdf}
    \caption{\small Comparing LO-BCQ to MX format.}
    \label{fig:block_formats}
\end{figure}

Figure \ref{fig:block_formats} compares our $4$-bit LO-BCQ block format to MX \citep{rouhani2023microscaling}. As shown, both LO-BCQ and MX decompose a given operand tensor into block arrays and each block array into blocks. Similar to MX, we find that per-block quantization ($L_b < L_A$) leads to better accuracy due to increased flexibility. While MX achieves this through per-block $1$-bit micro-scales, we associate a dedicated codebook to each block through a per-block codebook selector. Further, MX quantizes the per-block array scale-factor to E8M0 format without per-tensor scaling. In contrast during LO-BCQ, we find that per-tensor scaling combined with quantization of per-block array scale-factor to E4M3 format results in superior inference accuracy across models. 
 

\end{document}
