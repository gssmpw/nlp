\section{Interview questions}
\label{sec:questions}

In this section, we list all interview questions, categorized into four themes, as described in Section~\ref{sec:eval}.

\paragraph{Background and experience with the \credal methodology}
    \begin{itemize}
        \item [1.] Please describe your familiarity and experience with data modeling before the workshop.
        \item [2.] Did your understanding of data modeling change as a result of the workshop?  Do you feel more or less confident in your data modeling skills after the workshop? Please explain.
        \item [3.] Follow-up: Please describe your overall experience in learning and applying the \credal methodology during the workshop. 
        \item [4.] Follow-up: Which aspects of the methodology were particularly helpful? Which aspects of the methodology were particularly challenging?
    \end{itemize}

\paragraph{Feedback on supplemental materials}
    \begin{itemize}
        \item [5.] On a scale of 1 to 5, how helpful was the tutorial for understanding the methodology? (1- Not helpful, 5- Very helpful)
        \item [6.] How helpful was the example of close reading, we provided in helping for your understanding of the methodology? (1- Not helpful, 5- Very helpful)
        \item [7.] Are you familiar with the literary close reading technique? (Yes/No)
        \item [8.] If the answer was ``Yes'', did you find the comparison in the tutorial helpful for understanding our methodology?
    \end{itemize}
    
\paragraph{Perceived effectiveness of \credal}
    \begin{itemize}
        \item [9.] Were you able to apply the methodology during the reading session? (Yes/No)
        \item [10.] If the answer was ``No'', what challenges did you face?
        \item [11.] Please give an example of how you might apply \credal in your work or studies.
        \item [12.] Would you consider using the \credal methodology for ``reading of data models'' as a preliminary step for future projects? (Yes/No)
        \item [13.] If the answer was ``Yes'', at what point during your project would you use the methodology?  Will you use it as a checker before starting work with some sort of datasets or data models?
    \end{itemize}

\paragraph{Suggestions for improving \credal}
    \begin{itemize}
        \item [14.] Please suggest ways to improve the methodology itself or how we explain it.
        
    \end{itemize}

\section{Data model for example reading}
\label{sec:example_reading}

\begin{figure}[H]
    \centering
    \includegraphics[width=1\linewidth]{imgs/high-level-competition_model.png}
    \caption{Data Model for Example Reading, based on~\citet{erd}}
    \label{fig:enter-label}
\end{figure}
\newpage

\section{Intreview codebook}
\label{sec:Codebook}

%Codebook used for the interview encoding.  

\begin{table}[htbp]
\centering
\small
\begin{tabular}{|>{\raggedright\arraybackslash}p{3cm}|>{\raggedright\arraybackslash}p{4cm}|>{\raggedright\arraybackslash}p{7cm}|}
\hline
\textbf{Groups} & \textbf{Codes} & \textbf{Explanation} \\ \hline

\cellcolor{blue!10}{A: Participant's background} 
& \cellcolor{blue!10}1: Basic data modeling & \cellcolor{blue!10}General feedback on respondent's previous data model experience \\ \cline{2-3}
\cellcolor{blue!10} & \cellcolor{blue!10}2: Intermediate data modeling & \cellcolor{blue!10}General feedback on respondent's previous data model experience \\ \cline{2-3}
\cellcolor{blue!10} & \cellcolor{blue!10}3: Advanced data modeling & \cellcolor{blue!10}General feedback on respondent's previous data model experience \\ \hline

\cellcolor{yellow!10}{B: Methodology effectiveness} & \cellcolor{yellow!10}1: Improved participant's data modeling proficiency& \cellcolor{yellow!10}Anytime respondents mentioned that methodology improved their data modeling skills \\ \cline{2-3}
\cellcolor{yellow!10} & \cellcolor{yellow!10}2: Helped participant understand a specific model & \cellcolor{yellow!10}Anytime respondents mentioned that methodology helped them understand a data model \\ \cline{2-3}
\cellcolor{yellow!10} & \cellcolor{yellow!10}3: Participant likely to use the methodology in the future & \cellcolor{yellow!10}Whenever respondents claimed they would use the methodology for future work, studies, or personal projects \\ \hline

\cellcolor{red!10}{C: Methodology improvement} & \cellcolor{red!10}1: Opportunity to engage domain experts & \cellcolor{red!10}When scaling methodology for teams, domain expert engagement might improve overall close reading results \\ \cline{2-3}
\cellcolor{red!10}& \cellcolor{red!10}2: Opportunity to automate & \cellcolor{red!10}When respondents identified opportunities to automate or semi-automate application of the methodology \\ \cline{2-3} 
\cellcolor{red!10} & \cellcolor{red!10}3: Need to add examples& \cellcolor{red!10}When respondents mentioned the need for more examples (in video or the example close reading) \\ \cline{2-3} 
\cellcolor{red!10} & \cellcolor{red!10}4: Need to improve supplemental materials & \cellcolor{red!10}When a responded states a need to improve supplemental materials, e.g., tutorial, examples, etc) \\ \cline{2-3}
\cellcolor{red!10} & \cellcolor{red!10}5: Need to improve presentation of bias & \cellcolor{red!10}Whenever respondents mentioned the need for clearer bias definition \\ \cline{2-3} 
\cellcolor{red!10} & \cellcolor{red!10}6: Opportunity for case-by-case customization & \cellcolor{red!10}When respondents asked for shorter and longer versions of the methodology for different application cases \\ \hline

\cellcolor{green!10}{D: Methodology strengths} 
& \cellcolor{green!10}1: Helpful structure & \cellcolor{green!10}Respondents mentioning they liked the steps in the methodology (e.g., assumption loop), as well as its general structure: divide and conquer, and iterative deliberation \\ \cline{2-3}
\cellcolor{green!10} & \cellcolor{green!10}2: Helpful supplemental materials & \cellcolor{green!10}Respondents liked the video, tips \& tricks, and example, which were in addition to the methodology \\ \cline{2-3}
\cellcolor{green!10} & \cellcolor{green!10}3: Encourages analysis and reflection & \cellcolor{green!10}Respondents mentioned that the methodology is time-consuming or requires deep data model analysis (which is what we aim for) \\ \hline

\end{tabular}
\caption{Groups, Codes, and Explanation}
\end{table}