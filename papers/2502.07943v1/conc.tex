\section{Conclusion, Limitations, and Future Work}
\label{sec:conc}
In our work with and within data and data systems, close readings of data models reconnect us with the materiality, the genealogies, the techne, the closed nature, and the design of data models.  
We presented the \credal methodology for close readings of data models, along with the results of a qualitative study demonstrating its usability, usefulness, and effectiveness in the critical study of data.   \credal is the first systematic method for this important activity.

We conclude with a discussion of the limitations of this study and pointers for future work.   (1) The development of \credal and our qualitative study of the methodology was limited to undergraduate and graduate students, some with professional work experience in software engineering and data science roles.  Furthermore, our primary focus was on data models for relational data systems.  Future work to address these limitations can include larger qualitative and quantitative studies in contexts beyond university students and with non-relational data modeling paradigms such as knowledge graphs and semantic web ontologies. (2) Our interview study highlighted several limitations of the current version of \credal as well as points for extension: extending and improving the supplemental materials, support and guidance for engaging with domain experts, case-by-case customization of the methodology, and opportunities to automate repetitive and data-intensive aspects of the methodology.  Addressing each of these are important research and development topics for future iterations of the methodology, and its supporting materials and technologies.