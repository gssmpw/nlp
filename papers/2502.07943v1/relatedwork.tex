\section{Background and Related Work}
\label{sec:background}

In the previous section, we have highlighted data models as imbricated in making possible and driving the social, material, and genealogical facets of the work of data.  We also positioned close readings of data models with respect to the classical notion of close reading of texts.  In this section, we place our work in the context of closely related work in reading and modeling data.

\subsection{Reading data}

We first highlight work on reading data and data models.
Lindsay Poirier identifies three modes of \emph{close reading for datasets}: denotative (understanding technical aspects), connotative (understanding cultural contexts of production and change), and deconstructive (understanding representational limits) \cite{poirier}.
Through case studies, Poirier's work shows how students learn to read datasets through these modes to better understand the assumptions and politics of a given dataset.  

Melanie Feinberg has proposed reading data models as a ``slow'' data interaction paradigm which is complementary to the dominant ``fast'' outcome-oriented search and retrieval paradigms \cite{melanieSlow}.   Slow readings are shown to create opportunities for teasing out and reflecting on the interpretative facets of data model interaction.
While highlighting the vital importance of reading data, the work of both Poirier and Feinberg does not introduce a systematic methodology to jumpstart and guide close readings, which is important for beginning readers (especially those from STEM backgrounds). 

Karen Wickett and Nikki Lane Stevens have recently introduced \emph{critical data modeling} as a focus within critical information and data studies.
Wickett's work \cite{wickett} identifies a basic representation model to highlight the propositional, symbolic, and material aspects of data models. This model aims to support \emph{close readings of data collections} through the lens of these aspects as well as the transformations between the modeling domain and models of the domain in information systems.  The goal of this work is to elevate models as first-class citizens in the field of critical data studies, thereby making the biases and assumptions latent in the models under which data collections are created visible and open to critique and contestation.  A notable aspect of Wickett's model is 
the focus on the propositional content of data collection---on the truth claims of data relative to a posited ``real world''---rather than the relational aspects of data collections.  Here, people and modeling (and iterative remodeling) activities are secondary; both the modelers themselves and those being modeled are essentially absent.  This is in contrast with a relational understanding of data and data models, which highlights the hermeneutic nature of data-driven knowledge making (i.e., data and data models are only ever such in relation to human storytelling).

Lane Stevens also aims to elevate data models as first class citizens in the critical study of data, through several autoethnographic investigations that uncover the fundamental ways in which data models make algorithmic decision and knowledge-making possible~\cite{stevens}. Lane Stevens particularly unpacks the ways in which modeling choices are intimately tied up in social power and empowerment.  This work demonstrates how decisions and knowledge embody and perpetuate the social norms and power structures coded in data models.  The overarching contribution of this work is to identify data models as sites for intervention and subversion towards more just and equitable futures.  

We also highlight qualitative studies of troubling data models with similar aims to ours, including grounded theory investigations of Michael Muller et al. on how data scientists work \cite{muller}, and the recent ethnographic ``autospeculation'' method of Brian Kinnee et al. \cite{kinnee}.  
Finally, we note the broad literature on \emph{critical data literacy}, which aims in part to help students understand the complex social factors behind data collections  
\cite{dangol}.  Our work is complementary with several of the aims of this area of scholarship.


\subsection{Modeling data}

Conceptual and knowledge modeling of data has been studied in the data management, business information systems, and knowledge representation communities for over 50 years.  For overviews of these rich literatures, see the surveys of Veda Storey et al. \cite{akoka,storey} and Wei Yun et al. \cite{yun2021}.  Apart from a rich sub-literature on model validation (i.e., tools and methods to ensure that a model is fit for purpose), we find that the overwhelming majority of work has overlooked broader social aspects of data modeling.   Notable exceptions include the qualitative investigation of Graeme Simsion and colleagues that highlights the constructive consensus-making nature of modeling in business information systems \cite{simsion}, and recent efforts in the conceptual modeling community on inclusiveness in modeling, see  Lukyanenko et al. \cite{LukyanenkoBSP023}.