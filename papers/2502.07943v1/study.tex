\subsection{Iterative refinement}
\label{sec:credal:refine}

We needed a systematic approach to gathering deeper feedback on \credal to iteratively refine the methodology. Moreover, systematic feedback was necessary to answer the research questions guiding our study. Consequently, we developed a set of semi-structured interview questions to gather feedback from individuals working with data models.  We engaged with three groups of people, of increasing size, first explaining the \credal methodology to them and then collecting their feedback.

\subsubsection{Initial Interviews with Undergraduate Students}
\label{sec:credal:refine:1}

Each team member recruited a volunteer Computer Science student as a reviewer. The students were provided unlimited time to familiarize themselves with \credal and its supplemental materials. They were also given unlimited time to apply the methodology to a relational data model, which was provided to them. The application of \credal was carried out in an unconstrained written form according to the student's preferences.

Once the students reported that they had completed the application of the methodology, they were invited to participate in a 30-minute semi-structured interview. These interviews were recorded using an audio recorder, and the anonymized recordings were subsequently transcribed manually by team members. The text-based transcriptions were then used for further analysis, enabling us to gather detailed feedback. Based on this feedback, we identified several areas for improvement and addressed key gaps in the methodology, enhancing its clarity and making it more intuitive and accessible for users.

\subsubsection{Organizing a Workshop for Peer Feedback}
\label{sec:credal:refine:2}

After refining the methodology, we conducted a workshop with 6 graduate students in computer science and data science at New York University. Prior to the workshop, participants were invited to review the \credal methodology and its supplemental materials to familiarize themselves with the process. The workshop itself was one hour long, during which participants collectively applied the methodology to the same relational data model as the previous respondents. Following the application of the methodology, feedback was gathered using the same set of interview questions.

Using the same data model and identical interview questions ensured consistency and allowed for a comparison of different versions of the methodology. This approach also minimized variability in the feedback that might have arisen due to differences in the size, complexity, or format of the data models. Additionally, using the same interview questions enabled us to directly track the progress of \credal's development and evaluate the effectiveness of the improvements we made.

From the feedback collected during the workshop, we gained valuable insights into how to make \credal easier to understand and use.
One key recommendation was to increase the use of visual aids to accompany the methodology, to help participants familiarize themselves with the methodology more quickly.

\subsubsection{Conducting a Pilot Study}  
\label{sec:credal:refine:3}

To address these recommendations, the next group of respondents, comprising 4 graduate students in data science and 7 undergraduate students in computer science, were provided with a video guide. This guide presented the key points of the methodology in a visual format, with examples illustrating its application. Respondents were also given an improved version of the methodology guide, which included practical tips to facilitate the application process, along with a completed example of applying \credal to a data model.  We will provide additional details about the pilot study in Section~\ref{sec:eval}, and will discuss results of evaluation in Section~\ref{sec:discussion}.