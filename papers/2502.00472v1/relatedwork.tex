\section{Related Works}
The challenges of learning the evolution of high frequency structures in spatiotemporal dynamical systems has been given considerable study in recent literature \citep{karniadakis2021physics,lai2024machine,chakraborty2024divide,chen2024physics}. Researchers have tried to solve this problem through various instruments, in particular, by modifying the deep neural network architecture. \citep{liu2024mitigating} proposed a Hierarchical Attention Neural Operator (HANO) inspired by multilevel matrix methods, featuring hierarchical self-attentions and local aggregations to effectively capture multi-scale features. By leveraging insights from diffusion models, \citep{lippe2023modeling} proposed the PDE-Refiner which iteratively refines predictions, focusing on modeling both dominant and low-amplitude spatial frequency components. Hybrid methods combining classical numerical methods and deep learning has also been used to capture both lower and higher modes of the energy spectrum accurately \citep{shankar2023differentiable,zhang2024blending}. Other techniques like multiscale networks \citep{wang2020multi,liu2020multi} and diffusion models \citep{oommen2024integrating} have also been explored for the same. However, these techniques heavily exploit architectural modifications that are difficult to devise and frequently require significant computational overhead. 
% For example, diffusion models come with additional cost both during training and inference.

Another intuitive solution to the problem of capturing the fine scales can be to penalize the mismatch of the Fourier transform of the model outputs from the ground truth \citep{chattopadhyay2024oceannet,guan2024lucie,kochkov2023neural}. This is typically done by a mean absolute error loss in the Fourier space :
\begin{equation}\label{MSE-Fspace}
L_f = \frac{1}{N}\sum_{j=0}^Nw_j \left\|\mathcal{F}(F_{\phi}(x_j)) - \mathcal{F}(G(x_j))\right\|.
\end{equation}
where $\mathcal{F}$ is the Fourier transform, and $w_j$ is a hyperparameter used to weigh or cut-off some modes. 
% Although this seems promising in their applications, the effect of Equation \ref{MSE-Fspace} is same as the loss function given in Equation \ref{MSE-1-act} since Fourier transforms are \textcolor{red}{$L^2$} isometries. Consequently, it will also be heavily biased to the larger values in the Fourier spectrum which correspond to lower frequency modes.
It is evident that Equation \ref{MSE-Fspace} will also be heavily biased towards the larger values in the Fourier spectrum which typically correspond to the lower frequency modes. Consequently, the effect of Equation \ref{MSE-Fspace} is same as the loss function in Equation \ref{MSE-1-act}. To overcome this, \citep{chattopadhyay2024oceannet} used a cutoff to empirically ignore some of the lower frequencies. However, for the higher frequencies with extremely low values, it is not judicious to try to match them exactly in a point wise manner. In the following section we come up a new strategy to solve the mentioned problems without modifying the network architecture or incurring a heavy cost during training and inference.