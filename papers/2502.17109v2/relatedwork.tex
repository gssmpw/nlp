\section{Related Works}
%\label{subsec:related_works}

%Strength estimation is also related to ranking problems, which typically involve sorting items into a specific order based on a given query. 
%This approach is commonly applied to information retrieval, particularly for organizing documents, the listwise approaches \citep{cao2007learning,xia2008listwise,li2011short} are currently the state-of-the-art approach for this problem.
%Strength estimation can also be regarded as ranking problems as only the query was "strength" to sort the players.

%Early research of rformulated the ranking problem as a regression or classification task applied to individual items, referred to pointwise approaches, treating each item independently \citep{shashua2002ranking,cossock2006subset}. 
%Subsequently, the pairwise approach shifted the focus to the relative order between pairs of items, transforming the ranking problem into a pairwise classification or regression task, which have been more effective than pointwise approaches \citep{herbrich2000large, freund2003efficient}.
%However, both the pointwise and pairwise approaches fail to consider the group structure inherent in ranking.

%To address this limitation, the listwise approach was later proposed \citep{cao2007learning, xia2008listwise}. 
%These methods preserve the group structure of ranking and directly models and optimizes the overall ranking performance for an entire list of items, achieving better results than the earlier pointwise and pairwise approaches \citep{li2011short}.

%and can be categorized into three main approaches: (1) Pointwise Approach: This approach formulates the ranking problem as a regression or classification task on individual items, treating each query-document pair independently. 
%Notable methods include Subset Ranking, McRank, Prank, and OC SVM. (2) Pairwise Approach: This approach focuses on the relative order between pairs of items, transforming the ranking problem into a pairwise classification or regression task. Representative methods are Ranking SVM, RankBoost, RankNet, GBRank, IR SVM, Lambda Rank, and LambdaMART. (3) Listwise Approach: This approach directly models and optimizes the overall ranking performance for an entire list of items, considering global ranking metrics such as NDCG or MAP. Prominent methods in this category include ListNet, ListMLE, AdaRank, SVM MAP, and Soft Rank.