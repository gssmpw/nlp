
\documentclass{article} % For LaTeX2e
\usepackage{iclr2025_conference,times}

% Optional math commands from https://github.com/goodfeli/dlbook_notation.
%%%%% NEW MATH DEFINITIONS %%%%%

\usepackage{amsmath,amsfonts,bm}
\usepackage{derivative}
% Mark sections of captions for referring to divisions of figures
\newcommand{\figleft}{{\em (Left)}}
\newcommand{\figcenter}{{\em (Center)}}
\newcommand{\figright}{{\em (Right)}}
\newcommand{\figtop}{{\em (Top)}}
\newcommand{\figbottom}{{\em (Bottom)}}
\newcommand{\captiona}{{\em (a)}}
\newcommand{\captionb}{{\em (b)}}
\newcommand{\captionc}{{\em (c)}}
\newcommand{\captiond}{{\em (d)}}

% Highlight a newly defined term
\newcommand{\newterm}[1]{{\bf #1}}

% Derivative d 
\newcommand{\deriv}{{\mathrm{d}}}

% Figure reference, lower-case.
\def\figref#1{figure~\ref{#1}}
% Figure reference, capital. For start of sentence
\def\Figref#1{Figure~\ref{#1}}
\def\twofigref#1#2{figures \ref{#1} and \ref{#2}}
\def\quadfigref#1#2#3#4{figures \ref{#1}, \ref{#2}, \ref{#3} and \ref{#4}}
% Section reference, lower-case.
\def\secref#1{section~\ref{#1}}
% Section reference, capital.
\def\Secref#1{Section~\ref{#1}}
% Reference to two sections.
\def\twosecrefs#1#2{sections \ref{#1} and \ref{#2}}
% Reference to three sections.
\def\secrefs#1#2#3{sections \ref{#1}, \ref{#2} and \ref{#3}}
% Reference to an equation, lower-case.
\def\eqref#1{equation~\ref{#1}}
% Reference to an equation, upper case
\def\Eqref#1{Equation~\ref{#1}}
% A raw reference to an equation---avoid using if possible
\def\plaineqref#1{\ref{#1}}
% Reference to a chapter, lower-case.
\def\chapref#1{chapter~\ref{#1}}
% Reference to an equation, upper case.
\def\Chapref#1{Chapter~\ref{#1}}
% Reference to a range of chapters
\def\rangechapref#1#2{chapters\ref{#1}--\ref{#2}}
% Reference to an algorithm, lower-case.
\def\algref#1{algorithm~\ref{#1}}
% Reference to an algorithm, upper case.
\def\Algref#1{Algorithm~\ref{#1}}
\def\twoalgref#1#2{algorithms \ref{#1} and \ref{#2}}
\def\Twoalgref#1#2{Algorithms \ref{#1} and \ref{#2}}
% Reference to a part, lower case
\def\partref#1{part~\ref{#1}}
% Reference to a part, upper case
\def\Partref#1{Part~\ref{#1}}
\def\twopartref#1#2{parts \ref{#1} and \ref{#2}}

\def\ceil#1{\lceil #1 \rceil}
\def\floor#1{\lfloor #1 \rfloor}
\def\1{\bm{1}}
\newcommand{\train}{\mathcal{D}}
\newcommand{\valid}{\mathcal{D_{\mathrm{valid}}}}
\newcommand{\test}{\mathcal{D_{\mathrm{test}}}}

\def\eps{{\epsilon}}


% Random variables
\def\reta{{\textnormal{$\eta$}}}
\def\ra{{\textnormal{a}}}
\def\rb{{\textnormal{b}}}
\def\rc{{\textnormal{c}}}
\def\rd{{\textnormal{d}}}
\def\re{{\textnormal{e}}}
\def\rf{{\textnormal{f}}}
\def\rg{{\textnormal{g}}}
\def\rh{{\textnormal{h}}}
\def\ri{{\textnormal{i}}}
\def\rj{{\textnormal{j}}}
\def\rk{{\textnormal{k}}}
\def\rl{{\textnormal{l}}}
% rm is already a command, just don't name any random variables m
\def\rn{{\textnormal{n}}}
\def\ro{{\textnormal{o}}}
\def\rp{{\textnormal{p}}}
\def\rq{{\textnormal{q}}}
\def\rr{{\textnormal{r}}}
\def\rs{{\textnormal{s}}}
\def\rt{{\textnormal{t}}}
\def\ru{{\textnormal{u}}}
\def\rv{{\textnormal{v}}}
\def\rw{{\textnormal{w}}}
\def\rx{{\textnormal{x}}}
\def\ry{{\textnormal{y}}}
\def\rz{{\textnormal{z}}}

% Random vectors
\def\rvepsilon{{\mathbf{\epsilon}}}
\def\rvphi{{\mathbf{\phi}}}
\def\rvtheta{{\mathbf{\theta}}}
\def\rva{{\mathbf{a}}}
\def\rvb{{\mathbf{b}}}
\def\rvc{{\mathbf{c}}}
\def\rvd{{\mathbf{d}}}
\def\rve{{\mathbf{e}}}
\def\rvf{{\mathbf{f}}}
\def\rvg{{\mathbf{g}}}
\def\rvh{{\mathbf{h}}}
\def\rvu{{\mathbf{i}}}
\def\rvj{{\mathbf{j}}}
\def\rvk{{\mathbf{k}}}
\def\rvl{{\mathbf{l}}}
\def\rvm{{\mathbf{m}}}
\def\rvn{{\mathbf{n}}}
\def\rvo{{\mathbf{o}}}
\def\rvp{{\mathbf{p}}}
\def\rvq{{\mathbf{q}}}
\def\rvr{{\mathbf{r}}}
\def\rvs{{\mathbf{s}}}
\def\rvt{{\mathbf{t}}}
\def\rvu{{\mathbf{u}}}
\def\rvv{{\mathbf{v}}}
\def\rvw{{\mathbf{w}}}
\def\rvx{{\mathbf{x}}}
\def\rvy{{\mathbf{y}}}
\def\rvz{{\mathbf{z}}}

% Elements of random vectors
\def\erva{{\textnormal{a}}}
\def\ervb{{\textnormal{b}}}
\def\ervc{{\textnormal{c}}}
\def\ervd{{\textnormal{d}}}
\def\erve{{\textnormal{e}}}
\def\ervf{{\textnormal{f}}}
\def\ervg{{\textnormal{g}}}
\def\ervh{{\textnormal{h}}}
\def\ervi{{\textnormal{i}}}
\def\ervj{{\textnormal{j}}}
\def\ervk{{\textnormal{k}}}
\def\ervl{{\textnormal{l}}}
\def\ervm{{\textnormal{m}}}
\def\ervn{{\textnormal{n}}}
\def\ervo{{\textnormal{o}}}
\def\ervp{{\textnormal{p}}}
\def\ervq{{\textnormal{q}}}
\def\ervr{{\textnormal{r}}}
\def\ervs{{\textnormal{s}}}
\def\ervt{{\textnormal{t}}}
\def\ervu{{\textnormal{u}}}
\def\ervv{{\textnormal{v}}}
\def\ervw{{\textnormal{w}}}
\def\ervx{{\textnormal{x}}}
\def\ervy{{\textnormal{y}}}
\def\ervz{{\textnormal{z}}}

% Random matrices
\def\rmA{{\mathbf{A}}}
\def\rmB{{\mathbf{B}}}
\def\rmC{{\mathbf{C}}}
\def\rmD{{\mathbf{D}}}
\def\rmE{{\mathbf{E}}}
\def\rmF{{\mathbf{F}}}
\def\rmG{{\mathbf{G}}}
\def\rmH{{\mathbf{H}}}
\def\rmI{{\mathbf{I}}}
\def\rmJ{{\mathbf{J}}}
\def\rmK{{\mathbf{K}}}
\def\rmL{{\mathbf{L}}}
\def\rmM{{\mathbf{M}}}
\def\rmN{{\mathbf{N}}}
\def\rmO{{\mathbf{O}}}
\def\rmP{{\mathbf{P}}}
\def\rmQ{{\mathbf{Q}}}
\def\rmR{{\mathbf{R}}}
\def\rmS{{\mathbf{S}}}
\def\rmT{{\mathbf{T}}}
\def\rmU{{\mathbf{U}}}
\def\rmV{{\mathbf{V}}}
\def\rmW{{\mathbf{W}}}
\def\rmX{{\mathbf{X}}}
\def\rmY{{\mathbf{Y}}}
\def\rmZ{{\mathbf{Z}}}

% Elements of random matrices
\def\ermA{{\textnormal{A}}}
\def\ermB{{\textnormal{B}}}
\def\ermC{{\textnormal{C}}}
\def\ermD{{\textnormal{D}}}
\def\ermE{{\textnormal{E}}}
\def\ermF{{\textnormal{F}}}
\def\ermG{{\textnormal{G}}}
\def\ermH{{\textnormal{H}}}
\def\ermI{{\textnormal{I}}}
\def\ermJ{{\textnormal{J}}}
\def\ermK{{\textnormal{K}}}
\def\ermL{{\textnormal{L}}}
\def\ermM{{\textnormal{M}}}
\def\ermN{{\textnormal{N}}}
\def\ermO{{\textnormal{O}}}
\def\ermP{{\textnormal{P}}}
\def\ermQ{{\textnormal{Q}}}
\def\ermR{{\textnormal{R}}}
\def\ermS{{\textnormal{S}}}
\def\ermT{{\textnormal{T}}}
\def\ermU{{\textnormal{U}}}
\def\ermV{{\textnormal{V}}}
\def\ermW{{\textnormal{W}}}
\def\ermX{{\textnormal{X}}}
\def\ermY{{\textnormal{Y}}}
\def\ermZ{{\textnormal{Z}}}

% Vectors
\def\vzero{{\bm{0}}}
\def\vone{{\bm{1}}}
\def\vmu{{\bm{\mu}}}
\def\vtheta{{\bm{\theta}}}
\def\vphi{{\bm{\phi}}}
\def\va{{\bm{a}}}
\def\vb{{\bm{b}}}
\def\vc{{\bm{c}}}
\def\vd{{\bm{d}}}
\def\ve{{\bm{e}}}
\def\vf{{\bm{f}}}
\def\vg{{\bm{g}}}
\def\vh{{\bm{h}}}
\def\vi{{\bm{i}}}
\def\vj{{\bm{j}}}
\def\vk{{\bm{k}}}
\def\vl{{\bm{l}}}
\def\vm{{\bm{m}}}
\def\vn{{\bm{n}}}
\def\vo{{\bm{o}}}
\def\vp{{\bm{p}}}
\def\vq{{\bm{q}}}
\def\vr{{\bm{r}}}
\def\vs{{\bm{s}}}
\def\vt{{\bm{t}}}
\def\vu{{\bm{u}}}
\def\vv{{\bm{v}}}
\def\vw{{\bm{w}}}
\def\vx{{\bm{x}}}
\def\vy{{\bm{y}}}
\def\vz{{\bm{z}}}

% Elements of vectors
\def\evalpha{{\alpha}}
\def\evbeta{{\beta}}
\def\evepsilon{{\epsilon}}
\def\evlambda{{\lambda}}
\def\evomega{{\omega}}
\def\evmu{{\mu}}
\def\evpsi{{\psi}}
\def\evsigma{{\sigma}}
\def\evtheta{{\theta}}
\def\eva{{a}}
\def\evb{{b}}
\def\evc{{c}}
\def\evd{{d}}
\def\eve{{e}}
\def\evf{{f}}
\def\evg{{g}}
\def\evh{{h}}
\def\evi{{i}}
\def\evj{{j}}
\def\evk{{k}}
\def\evl{{l}}
\def\evm{{m}}
\def\evn{{n}}
\def\evo{{o}}
\def\evp{{p}}
\def\evq{{q}}
\def\evr{{r}}
\def\evs{{s}}
\def\evt{{t}}
\def\evu{{u}}
\def\evv{{v}}
\def\evw{{w}}
\def\evx{{x}}
\def\evy{{y}}
\def\evz{{z}}

% Matrix
\def\mA{{\bm{A}}}
\def\mB{{\bm{B}}}
\def\mC{{\bm{C}}}
\def\mD{{\bm{D}}}
\def\mE{{\bm{E}}}
\def\mF{{\bm{F}}}
\def\mG{{\bm{G}}}
\def\mH{{\bm{H}}}
\def\mI{{\bm{I}}}
\def\mJ{{\bm{J}}}
\def\mK{{\bm{K}}}
\def\mL{{\bm{L}}}
\def\mM{{\bm{M}}}
\def\mN{{\bm{N}}}
\def\mO{{\bm{O}}}
\def\mP{{\bm{P}}}
\def\mQ{{\bm{Q}}}
\def\mR{{\bm{R}}}
\def\mS{{\bm{S}}}
\def\mT{{\bm{T}}}
\def\mU{{\bm{U}}}
\def\mV{{\bm{V}}}
\def\mW{{\bm{W}}}
\def\mX{{\bm{X}}}
\def\mY{{\bm{Y}}}
\def\mZ{{\bm{Z}}}
\def\mBeta{{\bm{\beta}}}
\def\mPhi{{\bm{\Phi}}}
\def\mLambda{{\bm{\Lambda}}}
\def\mSigma{{\bm{\Sigma}}}

% Tensor
\DeclareMathAlphabet{\mathsfit}{\encodingdefault}{\sfdefault}{m}{sl}
\SetMathAlphabet{\mathsfit}{bold}{\encodingdefault}{\sfdefault}{bx}{n}
\newcommand{\tens}[1]{\bm{\mathsfit{#1}}}
\def\tA{{\tens{A}}}
\def\tB{{\tens{B}}}
\def\tC{{\tens{C}}}
\def\tD{{\tens{D}}}
\def\tE{{\tens{E}}}
\def\tF{{\tens{F}}}
\def\tG{{\tens{G}}}
\def\tH{{\tens{H}}}
\def\tI{{\tens{I}}}
\def\tJ{{\tens{J}}}
\def\tK{{\tens{K}}}
\def\tL{{\tens{L}}}
\def\tM{{\tens{M}}}
\def\tN{{\tens{N}}}
\def\tO{{\tens{O}}}
\def\tP{{\tens{P}}}
\def\tQ{{\tens{Q}}}
\def\tR{{\tens{R}}}
\def\tS{{\tens{S}}}
\def\tT{{\tens{T}}}
\def\tU{{\tens{U}}}
\def\tV{{\tens{V}}}
\def\tW{{\tens{W}}}
\def\tX{{\tens{X}}}
\def\tY{{\tens{Y}}}
\def\tZ{{\tens{Z}}}


% Graph
\def\gA{{\mathcal{A}}}
\def\gB{{\mathcal{B}}}
\def\gC{{\mathcal{C}}}
\def\gD{{\mathcal{D}}}
\def\gE{{\mathcal{E}}}
\def\gF{{\mathcal{F}}}
\def\gG{{\mathcal{G}}}
\def\gH{{\mathcal{H}}}
\def\gI{{\mathcal{I}}}
\def\gJ{{\mathcal{J}}}
\def\gK{{\mathcal{K}}}
\def\gL{{\mathcal{L}}}
\def\gM{{\mathcal{M}}}
\def\gN{{\mathcal{N}}}
\def\gO{{\mathcal{O}}}
\def\gP{{\mathcal{P}}}
\def\gQ{{\mathcal{Q}}}
\def\gR{{\mathcal{R}}}
\def\gS{{\mathcal{S}}}
\def\gT{{\mathcal{T}}}
\def\gU{{\mathcal{U}}}
\def\gV{{\mathcal{V}}}
\def\gW{{\mathcal{W}}}
\def\gX{{\mathcal{X}}}
\def\gY{{\mathcal{Y}}}
\def\gZ{{\mathcal{Z}}}

% Sets
\def\sA{{\mathbb{A}}}
\def\sB{{\mathbb{B}}}
\def\sC{{\mathbb{C}}}
\def\sD{{\mathbb{D}}}
% Don't use a set called E, because this would be the same as our symbol
% for expectation.
\def\sF{{\mathbb{F}}}
\def\sG{{\mathbb{G}}}
\def\sH{{\mathbb{H}}}
\def\sI{{\mathbb{I}}}
\def\sJ{{\mathbb{J}}}
\def\sK{{\mathbb{K}}}
\def\sL{{\mathbb{L}}}
\def\sM{{\mathbb{M}}}
\def\sN{{\mathbb{N}}}
\def\sO{{\mathbb{O}}}
\def\sP{{\mathbb{P}}}
\def\sQ{{\mathbb{Q}}}
\def\sR{{\mathbb{R}}}
\def\sS{{\mathbb{S}}}
\def\sT{{\mathbb{T}}}
\def\sU{{\mathbb{U}}}
\def\sV{{\mathbb{V}}}
\def\sW{{\mathbb{W}}}
\def\sX{{\mathbb{X}}}
\def\sY{{\mathbb{Y}}}
\def\sZ{{\mathbb{Z}}}

% Entries of a matrix
\def\emLambda{{\Lambda}}
\def\emA{{A}}
\def\emB{{B}}
\def\emC{{C}}
\def\emD{{D}}
\def\emE{{E}}
\def\emF{{F}}
\def\emG{{G}}
\def\emH{{H}}
\def\emI{{I}}
\def\emJ{{J}}
\def\emK{{K}}
\def\emL{{L}}
\def\emM{{M}}
\def\emN{{N}}
\def\emO{{O}}
\def\emP{{P}}
\def\emQ{{Q}}
\def\emR{{R}}
\def\emS{{S}}
\def\emT{{T}}
\def\emU{{U}}
\def\emV{{V}}
\def\emW{{W}}
\def\emX{{X}}
\def\emY{{Y}}
\def\emZ{{Z}}
\def\emSigma{{\Sigma}}

% entries of a tensor
% Same font as tensor, without \bm wrapper
\newcommand{\etens}[1]{\mathsfit{#1}}
\def\etLambda{{\etens{\Lambda}}}
\def\etA{{\etens{A}}}
\def\etB{{\etens{B}}}
\def\etC{{\etens{C}}}
\def\etD{{\etens{D}}}
\def\etE{{\etens{E}}}
\def\etF{{\etens{F}}}
\def\etG{{\etens{G}}}
\def\etH{{\etens{H}}}
\def\etI{{\etens{I}}}
\def\etJ{{\etens{J}}}
\def\etK{{\etens{K}}}
\def\etL{{\etens{L}}}
\def\etM{{\etens{M}}}
\def\etN{{\etens{N}}}
\def\etO{{\etens{O}}}
\def\etP{{\etens{P}}}
\def\etQ{{\etens{Q}}}
\def\etR{{\etens{R}}}
\def\etS{{\etens{S}}}
\def\etT{{\etens{T}}}
\def\etU{{\etens{U}}}
\def\etV{{\etens{V}}}
\def\etW{{\etens{W}}}
\def\etX{{\etens{X}}}
\def\etY{{\etens{Y}}}
\def\etZ{{\etens{Z}}}

% The true underlying data generating distribution
\newcommand{\pdata}{p_{\rm{data}}}
\newcommand{\ptarget}{p_{\rm{target}}}
\newcommand{\pprior}{p_{\rm{prior}}}
\newcommand{\pbase}{p_{\rm{base}}}
\newcommand{\pref}{p_{\rm{ref}}}

% The empirical distribution defined by the training set
\newcommand{\ptrain}{\hat{p}_{\rm{data}}}
\newcommand{\Ptrain}{\hat{P}_{\rm{data}}}
% The model distribution
\newcommand{\pmodel}{p_{\rm{model}}}
\newcommand{\Pmodel}{P_{\rm{model}}}
\newcommand{\ptildemodel}{\tilde{p}_{\rm{model}}}
% Stochastic autoencoder distributions
\newcommand{\pencode}{p_{\rm{encoder}}}
\newcommand{\pdecode}{p_{\rm{decoder}}}
\newcommand{\precons}{p_{\rm{reconstruct}}}

\newcommand{\laplace}{\mathrm{Laplace}} % Laplace distribution

\newcommand{\E}{\mathbb{E}}
\newcommand{\Ls}{\mathcal{L}}
\newcommand{\R}{\mathbb{R}}
\newcommand{\emp}{\tilde{p}}
\newcommand{\lr}{\alpha}
\newcommand{\reg}{\lambda}
\newcommand{\rect}{\mathrm{rectifier}}
\newcommand{\softmax}{\mathrm{softmax}}
\newcommand{\sigmoid}{\sigma}
\newcommand{\softplus}{\zeta}
\newcommand{\KL}{D_{\mathrm{KL}}}
\newcommand{\Var}{\mathrm{Var}}
\newcommand{\standarderror}{\mathrm{SE}}
\newcommand{\Cov}{\mathrm{Cov}}
% Wolfram Mathworld says $L^2$ is for function spaces and $\ell^2$ is for vectors
% But then they seem to use $L^2$ for vectors throughout the site, and so does
% wikipedia.
\newcommand{\normlzero}{L^0}
\newcommand{\normlone}{L^1}
\newcommand{\normltwo}{L^2}
\newcommand{\normlp}{L^p}
\newcommand{\normmax}{L^\infty}

\newcommand{\parents}{Pa} % See usage in notation.tex. Chosen to match Daphne's book.

\DeclareMathOperator*{\argmax}{arg\,max}
\DeclareMathOperator*{\argmin}{arg\,min}

\DeclareMathOperator{\sign}{sign}
\DeclareMathOperator{\Tr}{Tr}
\let\ab\allowbreak


\usepackage{hyperref}
\usepackage{url}

% added packages
\usepackage{amsmath}
\usepackage{amssymb}
\usepackage{amsthm}
\usepackage{tikz,pgfplots}
\usepackage{caption}
\usepackage{subfig}
\newtheorem{theorem}{Theorem}
\newtheorem{corollary}{Corollary}
\usepackage{comment}
%\usepackage{algorithm}
%\usepackage[noend]{algpseudocode}
\usepackage{multirow} 
\usepackage{graphicx}
\usepackage[justification=centering]{caption}

\newcommand{\revision}[1]{\textcolor{red}{#1}}
\newcommand{\delete}[1]{\textcolor{red}{\sout{#1}}}

\definecolor{indigo}{RGB}{75, 0, 130}

\usepackage[utf8]{inputenc} % allow utf-8 input
\usepackage[T1]{fontenc}    % use 8-bit T1 fonts
\usepackage{hyperref}       % hyperlinks
\usepackage{url}            % simple URL typesetting
\usepackage{booktabs}       % professional-quality tables
\usepackage{amsfonts}       % blackboard math symbols
\usepackage{nicefrac}       % compact symbols for 1/2, etc.
\usepackage{microtype}      % microtypography
\usepackage{xcolor}         % colors
\pgfplotsset{compat=1.18}

\title{Strength Estimation and Human-Like Strength Adjustment in Games}

% On Strength Estimation and Adjustment: Illustrated with the game of Go
% Strength Estimation and Human-Like Strength Adjustment in Games
% Strength Estimation and Human-Like Strength Adjustment in Games: Illustrated with the game of Go

% Strength Estimation and Strength Adjustment in Game: A Strength System in the Game of Go
% Strength Estimation and Adjustment in Games: A System in Go
% % A MCTS-based approach for Human-Like Strength Adjustment in Games
% % Human-Like MCTS-based Strength Adjustment in Games
% % Strength Estimation and Adjustment in Games
% % strength estimation / strength adjustment / human-like / MCTS / games
% Strength Estimation and Human-Like Strength Adjustment: A System in Go


% Estimating and Adjusting Strength: A Human-Like System in Go
% Estimating and Adjusting Strength with Human Behavior: A System in Go
% Estimating and Adjusting Strength Based on Human Behavior: A System in Go

% A Strength System for Estimating and Adjusting Strength: A System in Go


% "Strength Estimation and Human-Like Strength Adjustment: XXX in Go"

% "Strength Estimation and Human-Like Strength Adjustment: Adaptive Mechanisms in Go"
% "Strength Estimation and Human-Like Strength Adjustment: Behavioral Modeling in Go"
% "Strength Estimation and Human-Like Strength Adjustment: Dynamic Systems in Go"
% "Strength Estimation and Human-Like Strength Adjustment: Strategic Implementation in Go"
% "Strength Estimation and Human-Like Strength Adjustment: Intelligent Design in Go"
% "Strength Estimation and Human-Like Strength Adjustment: Integrating Tactics and Adaptability in Go"
% "Strength Estimation and Human-Like Strength Adjustment: Enhancing Gameplay with Adaptive Algorithms in Go"
% "Strength Estimation and Human-Like Strength Adjustment: Evolving Game Strategies in Go"
% "Strength Estimation and Human-Like Strength Adjustment: Crafting Competitive Edges in Go"
% "Strength Estimation and Human-Like Strength Adjustment: Shaping Player Capabilities in Go"

% Strength Estimation and Human-Like Strength Adjustment: a Study in the Game of Go
% Strength Estimation and Human-Like Strength Adjustment: Quantifying Ranks in Go
% Strength Estimation and Human-Like Strength Adjustment: Go as a Model System
% Strength Estimation and Human-Like Strength Adjustment: Methodologies and Frameworks in Go
% Strength Estimation and Human-Like Strength Adjustment: Developing A Responsive System in Go
% Strength Estimation and Human-Like Strength Adjustment: Developing A System in Go
% Strength Estimation and Human-Like Strength Adjustment: Designing A System in Go
% Strength Estimation and Strength Adjustment: Designing A Human-Like System in Go
% Strength Estimation and Human-Like Strength Adjustment: Designing an Adaptive System in Go
% Strength Estimation and Human-Like Strength Adjustment: Designing an Adaptive System in Go for Target Ranks
% Strength Estimation and Human-Like Strength Adjustment: A System in Go
% Strength Estimation and Human-Like Strength Adjustment: A Go System


% Strength Estimation and Human-Like Strength Adjustment: Designing a Go System
% Strength Estimation and Human-Like Strength Adjustment: a System Designed for Go
% % Estimating and Adjusting Strength with Human Behavior: Designing a Go System

% If the main focus of your paper is on the methodology, the challenges of design, and the innovative aspects of creating the system, then "Designing a Go System" might be more appropriate as it highlights the process and development.
% If the paper is more about the application, effectiveness, and specific functionality of the completed system within the context of Go, then "A System Designed for Go" is likely better as it emphasizes the system’s tailored utility and readiness for use.








% Strength System: Estimating Strength and Adjusting Human-like Strength in the Game of Go

% Aligning Superhuman AI with Human Behavior: Chess as a Model System
% Designing Skill-Compatible AI: Methodologies and Frameworks in Chess
% Detecting Individual Decision-Making Style: Exploring Behavioral Stylometry in Chess
% Evaluation beyond Task Performance: Analyzing Concepts in AlphaZero in Hex


% Authors must not appear in the submitted version. They should be hidden
% as long as the \iclrfinalcopy macro remains commented out below.
% Non-anonymous submissions will be rejected without review.

%\author{Antiquus S.~Hippocampus, Natalia Cerebro \& Amelie P. Amygdale \thanks{ Use footnote for providing further information
%about author (webpage, alternative address)---\emph{not} for acknowledging
%funding agencies.  Funding acknowledgements go at the end of the paper.} \\
%Department of Computer Science\\
%Cranberry-Lemon University\\
%Pittsburgh, PA 15213, USA \\
%\texttt{\{hippo,brain,jen\}@cs.cranberry-lemon.edu} \\
%\And
%Ji Q. Ren \& Yevgeny LeNet \\
%Department of Computational Neuroscience \\
%University of the Witwatersrand \\
%Joburg, South Africa \\
%\texttt{\{robot,net\}@wits.ac.za} \\
%\AND
%Coauthor \\
%Affiliation \\
%Address \\
%\texttt{email}
%}

 %\author{
 %    Chun-Jung Chen,\textsuperscript{1,2}  Chung-Chin Shih,\textsuperscript{1} Ti-Rong Wu,\textsuperscript{1}\\
%   \vspace{1pt}\\
%    \textsuperscript{\rm 1}Institute of Information Science, Academia Sinica, Taiwan\\
%     \textsuperscript{\rm 2}Department of Computer Science, National Taiwan University, Taiwan\\
%     \vspace{1pt}\\
%     \small{\texttt{gcobs104628@iis.sinica.edu.tw, rockmanray@iis.sinica.edu.tw, tirongwu@iis.sinica.edu.tw}} \\
% }

\makeatletter
%\def\@fnsymbol#1{\ifcase#1 \or \dagger\or \ast\or \ddagger\or \S\or \P\or \parallel\or \ast\ast\or \dagger\dagger \or \ddagger\ddagger \else \@ctrerr \fi}
\makeatother

\author{
    Chun-Jung Chen\textsuperscript{1,2}\thanks{These authors contributed equally.}\hspace{4pt}, Chung-Chin Shih\textsuperscript{1}$^{\ast}$, Ti-Rong Wu\textsuperscript{1}\thanks{Corresponding author: tirongwu@iis.sinica.edu.tw}\\
    \vspace{1pt}\\
    \textsuperscript{\rm 1}Institute of Information Science, Academia Sinica, Taiwan\\
    \textsuperscript{\rm 2}Department of Computer Science, National Taiwan University, Taiwan
}

%Ti-Rong Wu,\textsuperscript{1}\thanks{These authors contributed equally.}\hspace{4} Hung Guei,\textsuperscript{1}$^{\ast}$

% The \author macro works with any number of authors. There are two commands
% used to separate the names and addresses of multiple authors: \And and \AND.
%
% Using \And between authors leaves it to \LaTeX{} to determine where to break
% the lines. Using \AND forces a linebreak at that point. So, if \LaTeX{}
% puts 3 of 4 authors names on the first line, and the last on the second
% line, try using \AND instead of \And before the third author name.

\newcommand{\fix}{\marginpar{FIX}}
\newcommand{\new}{\marginpar{NEW}}

%\iclrfinalcopy % Uncomment for camera-ready version, but NOT for submission.
\iclrfinalcopy
\begin{document}




\maketitle
% % This must be in the first 5 lines to tell arXiv to use pdfLaTeX, which is strongly recommended.
\pdfoutput=1
% In particular, the hyperref package requires pdfLaTeX in order to break URLs across lines.

\documentclass[11pt]{article}

% Change "review" to "final" to generate the final (sometimes called camera-ready) version.
% Change to "preprint" to generate a non-anonymous version with page numbers.
\usepackage{acl}

% Standard package includes
\usepackage{times}
\usepackage{latexsym}

% Draw tables
\usepackage{booktabs}
\usepackage{multirow}
\usepackage{xcolor}
\usepackage{colortbl}
\usepackage{array} 
\usepackage{amsmath}

\newcolumntype{C}{>{\centering\arraybackslash}p{0.07\textwidth}}
% For proper rendering and hyphenation of words containing Latin characters (including in bib files)
\usepackage[T1]{fontenc}
% For Vietnamese characters
% \usepackage[T5]{fontenc}
% See https://www.latex-project.org/help/documentation/encguide.pdf for other character sets
% This assumes your files are encoded as UTF8
\usepackage[utf8]{inputenc}

% This is not strictly necessary, and may be commented out,
% but it will improve the layout of the manuscript,
% and will typically save some space.
\usepackage{microtype}
\DeclareMathOperator*{\argmax}{arg\,max}
% This is also not strictly necessary, and may be commented out.
% However, it will improve the aesthetics of text in
% the typewriter font.
\usepackage{inconsolata}

%Including images in your LaTeX document requires adding
%additional package(s)
\usepackage{graphicx}
% If the title and author information does not fit in the area allocated, uncomment the following
%
%\setlength\titlebox{<dim>}
%
% and set <dim> to something 5cm or larger.

\title{Wi-Chat: Large Language Model Powered Wi-Fi Sensing}

% Author information can be set in various styles:
% For several authors from the same institution:
% \author{Author 1 \and ... \and Author n \\
%         Address line \\ ... \\ Address line}
% if the names do not fit well on one line use
%         Author 1 \\ {\bf Author 2} \\ ... \\ {\bf Author n} \\
% For authors from different institutions:
% \author{Author 1 \\ Address line \\  ... \\ Address line
%         \And  ... \And
%         Author n \\ Address line \\ ... \\ Address line}
% To start a separate ``row'' of authors use \AND, as in
% \author{Author 1 \\ Address line \\  ... \\ Address line
%         \AND
%         Author 2 \\ Address line \\ ... \\ Address line \And
%         Author 3 \\ Address line \\ ... \\ Address line}

% \author{First Author \\
%   Affiliation / Address line 1 \\
%   Affiliation / Address line 2 \\
%   Affiliation / Address line 3 \\
%   \texttt{email@domain} \\\And
%   Second Author \\
%   Affiliation / Address line 1 \\
%   Affiliation / Address line 2 \\
%   Affiliation / Address line 3 \\
%   \texttt{email@domain} \\}
% \author{Haohan Yuan \qquad Haopeng Zhang\thanks{corresponding author} \\ 
%   ALOHA Lab, University of Hawaii at Manoa \\
%   % Affiliation / Address line 2 \\
%   % Affiliation / Address line 3 \\
%   \texttt{\{haohany,haopengz\}@hawaii.edu}}
  
\author{
{Haopeng Zhang$\dag$\thanks{These authors contributed equally to this work.}, Yili Ren$\ddagger$\footnotemark[1], Haohan Yuan$\dag$, Jingzhe Zhang$\ddagger$, Yitong Shen$\ddagger$} \\
ALOHA Lab, University of Hawaii at Manoa$\dag$, University of South Florida$\ddagger$ \\
\{haopengz, haohany\}@hawaii.edu\\
\{yiliren, jingzhe, shen202\}@usf.edu\\}



  
%\author{
%  \textbf{First Author\textsuperscript{1}},
%  \textbf{Second Author\textsuperscript{1,2}},
%  \textbf{Third T. Author\textsuperscript{1}},
%  \textbf{Fourth Author\textsuperscript{1}},
%\\
%  \textbf{Fifth Author\textsuperscript{1,2}},
%  \textbf{Sixth Author\textsuperscript{1}},
%  \textbf{Seventh Author\textsuperscript{1}},
%  \textbf{Eighth Author \textsuperscript{1,2,3,4}},
%\\
%  \textbf{Ninth Author\textsuperscript{1}},
%  \textbf{Tenth Author\textsuperscript{1}},
%  \textbf{Eleventh E. Author\textsuperscript{1,2,3,4,5}},
%  \textbf{Twelfth Author\textsuperscript{1}},
%\\
%  \textbf{Thirteenth Author\textsuperscript{3}},
%  \textbf{Fourteenth F. Author\textsuperscript{2,4}},
%  \textbf{Fifteenth Author\textsuperscript{1}},
%  \textbf{Sixteenth Author\textsuperscript{1}},
%\\
%  \textbf{Seventeenth S. Author\textsuperscript{4,5}},
%  \textbf{Eighteenth Author\textsuperscript{3,4}},
%  \textbf{Nineteenth N. Author\textsuperscript{2,5}},
%  \textbf{Twentieth Author\textsuperscript{1}}
%\\
%\\
%  \textsuperscript{1}Affiliation 1,
%  \textsuperscript{2}Affiliation 2,
%  \textsuperscript{3}Affiliation 3,
%  \textsuperscript{4}Affiliation 4,
%  \textsuperscript{5}Affiliation 5
%\\
%  \small{
%    \textbf{Correspondence:} \href{mailto:email@domain}{email@domain}
%  }
%}

\begin{document}
\maketitle
\begin{abstract}
Recent advancements in Large Language Models (LLMs) have demonstrated remarkable capabilities across diverse tasks. However, their potential to integrate physical model knowledge for real-world signal interpretation remains largely unexplored. In this work, we introduce Wi-Chat, the first LLM-powered Wi-Fi-based human activity recognition system. We demonstrate that LLMs can process raw Wi-Fi signals and infer human activities by incorporating Wi-Fi sensing principles into prompts. Our approach leverages physical model insights to guide LLMs in interpreting Channel State Information (CSI) data without traditional signal processing techniques. Through experiments on real-world Wi-Fi datasets, we show that LLMs exhibit strong reasoning capabilities, achieving zero-shot activity recognition. These findings highlight a new paradigm for Wi-Fi sensing, expanding LLM applications beyond conventional language tasks and enhancing the accessibility of wireless sensing for real-world deployments.
\end{abstract}

\section{Introduction}

In today’s rapidly evolving digital landscape, the transformative power of web technologies has redefined not only how services are delivered but also how complex tasks are approached. Web-based systems have become increasingly prevalent in risk control across various domains. This widespread adoption is due their accessibility, scalability, and ability to remotely connect various types of users. For example, these systems are used for process safety management in industry~\cite{kannan2016web}, safety risk early warning in urban construction~\cite{ding2013development}, and safe monitoring of infrastructural systems~\cite{repetto2018web}. Within these web-based risk management systems, the source search problem presents a huge challenge. Source search refers to the task of identifying the origin of a risky event, such as a gas leak and the emission point of toxic substances. This source search capability is crucial for effective risk management and decision-making.

Traditional approaches to implementing source search capabilities into the web systems often rely on solely algorithmic solutions~\cite{ristic2016study}. These methods, while relatively straightforward to implement, often struggle to achieve acceptable performances due to algorithmic local optima and complex unknown environments~\cite{zhao2020searching}. More recently, web crowdsourcing has emerged as a promising alternative for tackling the source search problem by incorporating human efforts in these web systems on-the-fly~\cite{zhao2024user}. This approach outsources the task of addressing issues encountered during the source search process to human workers, leveraging their capabilities to enhance system performance.

These solutions often employ a human-AI collaborative way~\cite{zhao2023leveraging} where algorithms handle exploration-exploitation and report the encountered problems while human workers resolve complex decision-making bottlenecks to help the algorithms getting rid of local deadlocks~\cite{zhao2022crowd}. Although effective, this paradigm suffers from two inherent limitations: increased operational costs from continuous human intervention, and slow response times of human workers due to sequential decision-making. These challenges motivate our investigation into developing autonomous systems that preserve human-like reasoning capabilities while reducing dependency on massive crowdsourced labor.

Furthermore, recent advancements in large language models (LLMs)~\cite{chang2024survey} and multi-modal LLMs (MLLMs)~\cite{huang2023chatgpt} have unveiled promising avenues for addressing these challenges. One clear opportunity involves the seamless integration of visual understanding and linguistic reasoning for robust decision-making in search tasks. However, whether large models-assisted source search is really effective and efficient for improving the current source search algorithms~\cite{ji2022source} remains unknown. \textit{To address the research gap, we are particularly interested in answering the following two research questions in this work:}

\textbf{\textit{RQ1: }}How can source search capabilities be integrated into web-based systems to support decision-making in time-sensitive risk management scenarios? 
% \sq{I mention ``time-sensitive'' here because I feel like we shall say something about the response time -- LLM has to be faster than humans}

\textbf{\textit{RQ2: }}How can MLLMs and LLMs enhance the effectiveness and efficiency of existing source search algorithms? 

% \textit{\textbf{RQ2:}} To what extent does the performance of large models-assisted search align with or approach the effectiveness of human-AI collaborative search? 

To answer the research questions, we propose a novel framework called Auto-\
S$^2$earch (\textbf{Auto}nomous \textbf{S}ource \textbf{Search}) and implement a prototype system that leverages advanced web technologies to simulate real-world conditions for zero-shot source search. Unlike traditional methods that rely on pre-defined heuristics or extensive human intervention, AutoS$^2$earch employs a carefully designed prompt that encapsulates human rationales, thereby guiding the MLLM to generate coherent and accurate scene descriptions from visual inputs about four directional choices. Based on these language-based descriptions, the LLM is enabled to determine the optimal directional choice through chain-of-thought (CoT) reasoning. Comprehensive empirical validation demonstrates that AutoS$^2$-\ 
earch achieves a success rate of 95–98\%, closely approaching the performance of human-AI collaborative search across 20 benchmark scenarios~\cite{zhao2023leveraging}. 

Our work indicates that the role of humans in future web crowdsourcing tasks may evolve from executors to validators or supervisors. Furthermore, incorporating explanations of LLM decisions into web-based system interfaces has the potential to help humans enhance task performance in risk control.






\section{Related Work}
\label{sec:relatedworks}

% \begin{table*}[t]
% \centering 
% \renewcommand\arraystretch{0.98}
% \fontsize{8}{10}\selectfont \setlength{\tabcolsep}{0.4em}
% \begin{tabular}{@{}lc|cc|cc|cc@{}}
% \toprule
% \textbf{Methods}           & \begin{tabular}[c]{@{}c@{}}\textbf{Training}\\ \textbf{Paradigm}\end{tabular} & \begin{tabular}[c]{@{}c@{}}\textbf{$\#$ PT Data}\\ \textbf{(Tokens)}\end{tabular} & \begin{tabular}[c]{@{}c@{}}\textbf{$\#$ IFT Data}\\ \textbf{(Samples)}\end{tabular} & \textbf{Code}  & \begin{tabular}[c]{@{}c@{}}\textbf{Natural}\\ \textbf{Language}\end{tabular} & \begin{tabular}[c]{@{}c@{}}\textbf{Action}\\ \textbf{Trajectories}\end{tabular} & \begin{tabular}[c]{@{}c@{}}\textbf{API}\\ \textbf{Documentation}\end{tabular}\\ \midrule 
% NexusRaven~\citep{srinivasan2023nexusraven} & IFT & - & - & \textcolor{green}{\CheckmarkBold} & \textcolor{green}{\CheckmarkBold} &\textcolor{red}{\XSolidBrush}&\textcolor{red}{\XSolidBrush}\\
% AgentInstruct~\citep{zeng2023agenttuning} & IFT & - & 2k & \textcolor{green}{\CheckmarkBold} & \textcolor{green}{\CheckmarkBold} &\textcolor{red}{\XSolidBrush}&\textcolor{red}{\XSolidBrush} \\
% AgentEvol~\citep{xi2024agentgym} & IFT & - & 14.5k & \textcolor{green}{\CheckmarkBold} & \textcolor{green}{\CheckmarkBold} &\textcolor{green}{\CheckmarkBold}&\textcolor{red}{\XSolidBrush} \\
% Gorilla~\citep{patil2023gorilla}& IFT & - & 16k & \textcolor{green}{\CheckmarkBold} & \textcolor{green}{\CheckmarkBold} &\textcolor{red}{\XSolidBrush}&\textcolor{green}{\CheckmarkBold}\\
% OpenFunctions-v2~\citep{patil2023gorilla} & IFT & - & 65k & \textcolor{green}{\CheckmarkBold} & \textcolor{green}{\CheckmarkBold} &\textcolor{red}{\XSolidBrush}&\textcolor{green}{\CheckmarkBold}\\
% LAM~\citep{zhang2024agentohana} & IFT & - & 42.6k & \textcolor{green}{\CheckmarkBold} & \textcolor{green}{\CheckmarkBold} &\textcolor{green}{\CheckmarkBold}&\textcolor{red}{\XSolidBrush} \\
% xLAM~\citep{liu2024apigen} & IFT & - & 60k & \textcolor{green}{\CheckmarkBold} & \textcolor{green}{\CheckmarkBold} &\textcolor{green}{\CheckmarkBold}&\textcolor{red}{\XSolidBrush} \\\midrule
% LEMUR~\citep{xu2024lemur} & PT & 90B & 300k & \textcolor{green}{\CheckmarkBold} & \textcolor{green}{\CheckmarkBold} &\textcolor{green}{\CheckmarkBold}&\textcolor{red}{\XSolidBrush}\\
% \rowcolor{teal!12} \method & PT & 103B & 95k & \textcolor{green}{\CheckmarkBold} & \textcolor{green}{\CheckmarkBold} & \textcolor{green}{\CheckmarkBold} & \textcolor{green}{\CheckmarkBold} \\
% \bottomrule
% \end{tabular}
% \caption{Summary of existing tuning- and pretraining-based LLM agents with their training sample sizes. "PT" and "IFT" denote "Pre-Training" and "Instruction Fine-Tuning", respectively. }
% \label{tab:related}
% \end{table*}

\begin{table*}[ht]
\begin{threeparttable}
\centering 
\renewcommand\arraystretch{0.98}
\fontsize{7}{9}\selectfont \setlength{\tabcolsep}{0.2em}
\begin{tabular}{@{}l|c|c|ccc|cc|cc|cccc@{}}
\toprule
\textbf{Methods} & \textbf{Datasets}           & \begin{tabular}[c]{@{}c@{}}\textbf{Training}\\ \textbf{Paradigm}\end{tabular} & \begin{tabular}[c]{@{}c@{}}\textbf{\# PT Data}\\ \textbf{(Tokens)}\end{tabular} & \begin{tabular}[c]{@{}c@{}}\textbf{\# IFT Data}\\ \textbf{(Samples)}\end{tabular} & \textbf{\# APIs} & \textbf{Code}  & \begin{tabular}[c]{@{}c@{}}\textbf{Nat.}\\ \textbf{Lang.}\end{tabular} & \begin{tabular}[c]{@{}c@{}}\textbf{Action}\\ \textbf{Traj.}\end{tabular} & \begin{tabular}[c]{@{}c@{}}\textbf{API}\\ \textbf{Doc.}\end{tabular} & \begin{tabular}[c]{@{}c@{}}\textbf{Func.}\\ \textbf{Call}\end{tabular} & \begin{tabular}[c]{@{}c@{}}\textbf{Multi.}\\ \textbf{Step}\end{tabular}  & \begin{tabular}[c]{@{}c@{}}\textbf{Plan}\\ \textbf{Refine}\end{tabular}  & \begin{tabular}[c]{@{}c@{}}\textbf{Multi.}\\ \textbf{Turn}\end{tabular}\\ \midrule 
\multicolumn{13}{l}{\emph{Instruction Finetuning-based LLM Agents for Intrinsic Reasoning}}  \\ \midrule
FireAct~\cite{chen2023fireact} & FireAct & IFT & - & 2.1K & 10 & \textcolor{red}{\XSolidBrush} &\textcolor{green}{\CheckmarkBold} &\textcolor{green}{\CheckmarkBold}  & \textcolor{red}{\XSolidBrush} &\textcolor{green}{\CheckmarkBold} & \textcolor{red}{\XSolidBrush} &\textcolor{green}{\CheckmarkBold} & \textcolor{red}{\XSolidBrush} \\
ToolAlpaca~\cite{tang2023toolalpaca} & ToolAlpaca & IFT & - & 4.0K & 400 & \textcolor{red}{\XSolidBrush} &\textcolor{green}{\CheckmarkBold} &\textcolor{green}{\CheckmarkBold} & \textcolor{red}{\XSolidBrush} &\textcolor{green}{\CheckmarkBold} & \textcolor{red}{\XSolidBrush}  &\textcolor{green}{\CheckmarkBold} & \textcolor{red}{\XSolidBrush}  \\
ToolLLaMA~\cite{qin2023toolllm} & ToolBench & IFT & - & 12.7K & 16,464 & \textcolor{red}{\XSolidBrush} &\textcolor{green}{\CheckmarkBold} &\textcolor{green}{\CheckmarkBold} &\textcolor{red}{\XSolidBrush} &\textcolor{green}{\CheckmarkBold}&\textcolor{green}{\CheckmarkBold}&\textcolor{green}{\CheckmarkBold} &\textcolor{green}{\CheckmarkBold}\\
AgentEvol~\citep{xi2024agentgym} & AgentTraj-L & IFT & - & 14.5K & 24 &\textcolor{red}{\XSolidBrush} & \textcolor{green}{\CheckmarkBold} &\textcolor{green}{\CheckmarkBold}&\textcolor{red}{\XSolidBrush} &\textcolor{green}{\CheckmarkBold}&\textcolor{red}{\XSolidBrush} &\textcolor{red}{\XSolidBrush} &\textcolor{green}{\CheckmarkBold}\\
Lumos~\cite{yin2024agent} & Lumos & IFT  & - & 20.0K & 16 &\textcolor{red}{\XSolidBrush} & \textcolor{green}{\CheckmarkBold} & \textcolor{green}{\CheckmarkBold} &\textcolor{red}{\XSolidBrush} & \textcolor{green}{\CheckmarkBold} & \textcolor{green}{\CheckmarkBold} &\textcolor{red}{\XSolidBrush} & \textcolor{green}{\CheckmarkBold}\\
Agent-FLAN~\cite{chen2024agent} & Agent-FLAN & IFT & - & 24.7K & 20 &\textcolor{red}{\XSolidBrush} & \textcolor{green}{\CheckmarkBold} & \textcolor{green}{\CheckmarkBold} &\textcolor{red}{\XSolidBrush} & \textcolor{green}{\CheckmarkBold}& \textcolor{green}{\CheckmarkBold}&\textcolor{red}{\XSolidBrush} & \textcolor{green}{\CheckmarkBold}\\
AgentTuning~\citep{zeng2023agenttuning} & AgentInstruct & IFT & - & 35.0K & - &\textcolor{red}{\XSolidBrush} & \textcolor{green}{\CheckmarkBold} & \textcolor{green}{\CheckmarkBold} &\textcolor{red}{\XSolidBrush} & \textcolor{green}{\CheckmarkBold} &\textcolor{red}{\XSolidBrush} &\textcolor{red}{\XSolidBrush} & \textcolor{green}{\CheckmarkBold}\\\midrule
\multicolumn{13}{l}{\emph{Instruction Finetuning-based LLM Agents for Function Calling}} \\\midrule
NexusRaven~\citep{srinivasan2023nexusraven} & NexusRaven & IFT & - & - & 116 & \textcolor{green}{\CheckmarkBold} & \textcolor{green}{\CheckmarkBold}  & \textcolor{green}{\CheckmarkBold} &\textcolor{red}{\XSolidBrush} & \textcolor{green}{\CheckmarkBold} &\textcolor{red}{\XSolidBrush} &\textcolor{red}{\XSolidBrush}&\textcolor{red}{\XSolidBrush}\\
Gorilla~\citep{patil2023gorilla} & Gorilla & IFT & - & 16.0K & 1,645 & \textcolor{green}{\CheckmarkBold} &\textcolor{red}{\XSolidBrush} &\textcolor{red}{\XSolidBrush}&\textcolor{green}{\CheckmarkBold} &\textcolor{green}{\CheckmarkBold} &\textcolor{red}{\XSolidBrush} &\textcolor{red}{\XSolidBrush} &\textcolor{red}{\XSolidBrush}\\
OpenFunctions-v2~\citep{patil2023gorilla} & OpenFunctions-v2 & IFT & - & 65.0K & - & \textcolor{green}{\CheckmarkBold} & \textcolor{green}{\CheckmarkBold} &\textcolor{red}{\XSolidBrush} &\textcolor{green}{\CheckmarkBold} &\textcolor{green}{\CheckmarkBold} &\textcolor{red}{\XSolidBrush} &\textcolor{red}{\XSolidBrush} &\textcolor{red}{\XSolidBrush}\\
API Pack~\cite{guo2024api} & API Pack & IFT & - & 1.1M & 11,213 &\textcolor{green}{\CheckmarkBold} &\textcolor{red}{\XSolidBrush} &\textcolor{green}{\CheckmarkBold} &\textcolor{red}{\XSolidBrush} &\textcolor{green}{\CheckmarkBold} &\textcolor{red}{\XSolidBrush}&\textcolor{red}{\XSolidBrush}&\textcolor{red}{\XSolidBrush}\\ 
LAM~\citep{zhang2024agentohana} & AgentOhana & IFT & - & 42.6K & - & \textcolor{green}{\CheckmarkBold} & \textcolor{green}{\CheckmarkBold} &\textcolor{green}{\CheckmarkBold}&\textcolor{red}{\XSolidBrush} &\textcolor{green}{\CheckmarkBold}&\textcolor{red}{\XSolidBrush}&\textcolor{green}{\CheckmarkBold}&\textcolor{green}{\CheckmarkBold}\\
xLAM~\citep{liu2024apigen} & APIGen & IFT & - & 60.0K & 3,673 & \textcolor{green}{\CheckmarkBold} & \textcolor{green}{\CheckmarkBold} &\textcolor{green}{\CheckmarkBold}&\textcolor{red}{\XSolidBrush} &\textcolor{green}{\CheckmarkBold}&\textcolor{red}{\XSolidBrush}&\textcolor{green}{\CheckmarkBold}&\textcolor{green}{\CheckmarkBold}\\\midrule
\multicolumn{13}{l}{\emph{Pretraining-based LLM Agents}}  \\\midrule
% LEMUR~\citep{xu2024lemur} & PT & 90B & 300.0K & - & \textcolor{green}{\CheckmarkBold} & \textcolor{green}{\CheckmarkBold} &\textcolor{green}{\CheckmarkBold}&\textcolor{red}{\XSolidBrush} & \textcolor{red}{\XSolidBrush} &\textcolor{green}{\CheckmarkBold} &\textcolor{red}{\XSolidBrush}&\textcolor{red}{\XSolidBrush}\\
\rowcolor{teal!12} \method & \dataset & PT & 103B & 95.0K  & 76,537  & \textcolor{green}{\CheckmarkBold} & \textcolor{green}{\CheckmarkBold} & \textcolor{green}{\CheckmarkBold} & \textcolor{green}{\CheckmarkBold} & \textcolor{green}{\CheckmarkBold} & \textcolor{green}{\CheckmarkBold} & \textcolor{green}{\CheckmarkBold} & \textcolor{green}{\CheckmarkBold}\\
\bottomrule
\end{tabular}
% \begin{tablenotes}
%     \item $^*$ In addition, the StarCoder-API can offer 4.77M more APIs.
% \end{tablenotes}
\caption{Summary of existing instruction finetuning-based LLM agents for intrinsic reasoning and function calling, along with their training resources and sample sizes. "PT" and "IFT" denote "Pre-Training" and "Instruction Fine-Tuning", respectively.}
\vspace{-2ex}
\label{tab:related}
\end{threeparttable}
\end{table*}

\noindent \textbf{Prompting-based LLM Agents.} Due to the lack of agent-specific pre-training corpus, existing LLM agents rely on either prompt engineering~\cite{hsieh2023tool,lu2024chameleon,yao2022react,wang2023voyager} or instruction fine-tuning~\cite{chen2023fireact,zeng2023agenttuning} to understand human instructions, decompose high-level tasks, generate grounded plans, and execute multi-step actions. 
However, prompting-based methods mainly depend on the capabilities of backbone LLMs (usually commercial LLMs), failing to introduce new knowledge and struggling to generalize to unseen tasks~\cite{sun2024adaplanner,zhuang2023toolchain}. 

\noindent \textbf{Instruction Finetuning-based LLM Agents.} Considering the extensive diversity of APIs and the complexity of multi-tool instructions, tool learning inherently presents greater challenges than natural language tasks, such as text generation~\cite{qin2023toolllm}.
Post-training techniques focus more on instruction following and aligning output with specific formats~\cite{patil2023gorilla,hao2024toolkengpt,qin2023toolllm,schick2024toolformer}, rather than fundamentally improving model knowledge or capabilities. 
Moreover, heavy fine-tuning can hinder generalization or even degrade performance in non-agent use cases, potentially suppressing the original base model capabilities~\cite{ghosh2024a}.

\noindent \textbf{Pretraining-based LLM Agents.} While pre-training serves as an essential alternative, prior works~\cite{nijkamp2023codegen,roziere2023code,xu2024lemur,patil2023gorilla} have primarily focused on improving task-specific capabilities (\eg, code generation) instead of general-domain LLM agents, due to single-source, uni-type, small-scale, and poor-quality pre-training data. 
Existing tool documentation data for agent training either lacks diverse real-world APIs~\cite{patil2023gorilla, tang2023toolalpaca} or is constrained to single-tool or single-round tool execution. 
Furthermore, trajectory data mostly imitate expert behavior or follow function-calling rules with inferior planning and reasoning, failing to fully elicit LLMs' capabilities and handle complex instructions~\cite{qin2023toolllm}. 
Given a wide range of candidate API functions, each comprising various function names and parameters available at every planning step, identifying globally optimal solutions and generalizing across tasks remains highly challenging.



\section{Preliminaries}
\label{Preliminaries}
\begin{figure*}[t]
    \centering
    \includegraphics[width=0.95\linewidth]{fig/HealthGPT_Framework.png}
    \caption{The \ourmethod{} architecture integrates hierarchical visual perception and H-LoRA, employing a task-specific hard router to select visual features and H-LoRA plugins, ultimately generating outputs with an autoregressive manner.}
    \label{fig:architecture}
\end{figure*}
\noindent\textbf{Large Vision-Language Models.} 
The input to a LVLM typically consists of an image $x^{\text{img}}$ and a discrete text sequence $x^{\text{txt}}$. The visual encoder $\mathcal{E}^{\text{img}}$ converts the input image $x^{\text{img}}$ into a sequence of visual tokens $\mathcal{V} = [v_i]_{i=1}^{N_v}$, while the text sequence $x^{\text{txt}}$ is mapped into a sequence of text tokens $\mathcal{T} = [t_i]_{i=1}^{N_t}$ using an embedding function $\mathcal{E}^{\text{txt}}$. The LLM $\mathcal{M_\text{LLM}}(\cdot|\theta)$ models the joint probability of the token sequence $\mathcal{U} = \{\mathcal{V},\mathcal{T}\}$, which is expressed as:
\begin{equation}
    P_\theta(R | \mathcal{U}) = \prod_{i=1}^{N_r} P_\theta(r_i | \{\mathcal{U}, r_{<i}\}),
\end{equation}
where $R = [r_i]_{i=1}^{N_r}$ is the text response sequence. The LVLM iteratively generates the next token $r_i$ based on $r_{<i}$. The optimization objective is to minimize the cross-entropy loss of the response $\mathcal{R}$.
% \begin{equation}
%     \mathcal{L}_{\text{VLM}} = \mathbb{E}_{R|\mathcal{U}}\left[-\log P_\theta(R | \mathcal{U})\right]
% \end{equation}
It is worth noting that most LVLMs adopt a design paradigm based on ViT, alignment adapters, and pre-trained LLMs\cite{liu2023llava,liu2024improved}, enabling quick adaptation to downstream tasks.


\noindent\textbf{VQGAN.}
VQGAN~\cite{esser2021taming} employs latent space compression and indexing mechanisms to effectively learn a complete discrete representation of images. VQGAN first maps the input image $x^{\text{img}}$ to a latent representation $z = \mathcal{E}(x)$ through a encoder $\mathcal{E}$. Then, the latent representation is quantized using a codebook $\mathcal{Z} = \{z_k\}_{k=1}^K$, generating a discrete index sequence $\mathcal{I} = [i_m]_{m=1}^N$, where $i_m \in \mathcal{Z}$ represents the quantized code index:
\begin{equation}
    \mathcal{I} = \text{Quantize}(z|\mathcal{Z}) = \arg\min_{z_k \in \mathcal{Z}} \| z - z_k \|_2.
\end{equation}
In our approach, the discrete index sequence $\mathcal{I}$ serves as a supervisory signal for the generation task, enabling the model to predict the index sequence $\hat{\mathcal{I}}$ from input conditions such as text or other modality signals.  
Finally, the predicted index sequence $\hat{\mathcal{I}}$ is upsampled by the VQGAN decoder $G$, generating the high-quality image $\hat{x}^\text{img} = G(\hat{\mathcal{I}})$.



\noindent\textbf{Low Rank Adaptation.} 
LoRA\cite{hu2021lora} effectively captures the characteristics of downstream tasks by introducing low-rank adapters. The core idea is to decompose the bypass weight matrix $\Delta W\in\mathbb{R}^{d^{\text{in}} \times d^{\text{out}}}$ into two low-rank matrices $ \{A \in \mathbb{R}^{d^{\text{in}} \times r}, B \in \mathbb{R}^{r \times d^{\text{out}}} \}$, where $ r \ll \min\{d^{\text{in}}, d^{\text{out}}\} $, significantly reducing learnable parameters. The output with the LoRA adapter for the input $x$ is then given by:
\begin{equation}
    h = x W_0 + \alpha x \Delta W/r = x W_0 + \alpha xAB/r,
\end{equation}
where matrix $ A $ is initialized with a Gaussian distribution, while the matrix $ B $ is initialized as a zero matrix. The scaling factor $ \alpha/r $ controls the impact of $ \Delta W $ on the model.

\section{HealthGPT}
\label{Method}


\subsection{Unified Autoregressive Generation.}  
% As shown in Figure~\ref{fig:architecture}, 
\ourmethod{} (Figure~\ref{fig:architecture}) utilizes a discrete token representation that covers both text and visual outputs, unifying visual comprehension and generation as an autoregressive task. 
For comprehension, $\mathcal{M}_\text{llm}$ receives the input joint sequence $\mathcal{U}$ and outputs a series of text token $\mathcal{R} = [r_1, r_2, \dots, r_{N_r}]$, where $r_i \in \mathcal{V}_{\text{txt}}$, and $\mathcal{V}_{\text{txt}}$ represents the LLM's vocabulary:
\begin{equation}
    P_\theta(\mathcal{R} \mid \mathcal{U}) = \prod_{i=1}^{N_r} P_\theta(r_i \mid \mathcal{U}, r_{<i}).
\end{equation}
For generation, $\mathcal{M}_\text{llm}$ first receives a special start token $\langle \text{START\_IMG} \rangle$, then generates a series of tokens corresponding to the VQGAN indices $\mathcal{I} = [i_1, i_2, \dots, i_{N_i}]$, where $i_j \in \mathcal{V}_{\text{vq}}$, and $\mathcal{V}_{\text{vq}}$ represents the index range of VQGAN. Upon completion of generation, the LLM outputs an end token $\langle \text{END\_IMG} \rangle$:
\begin{equation}
    P_\theta(\mathcal{I} \mid \mathcal{U}) = \prod_{j=1}^{N_i} P_\theta(i_j \mid \mathcal{U}, i_{<j}).
\end{equation}
Finally, the generated index sequence $\mathcal{I}$ is fed into the decoder $G$, which reconstructs the target image $\hat{x}^{\text{img}} = G(\mathcal{I})$.

\subsection{Hierarchical Visual Perception}  
Given the differences in visual perception between comprehension and generation tasks—where the former focuses on abstract semantics and the latter emphasizes complete semantics—we employ ViT to compress the image into discrete visual tokens at multiple hierarchical levels.
Specifically, the image is converted into a series of features $\{f_1, f_2, \dots, f_L\}$ as it passes through $L$ ViT blocks.

To address the needs of various tasks, the hidden states are divided into two types: (i) \textit{Concrete-grained features} $\mathcal{F}^{\text{Con}} = \{f_1, f_2, \dots, f_k\}, k < L$, derived from the shallower layers of ViT, containing sufficient global features, suitable for generation tasks; 
(ii) \textit{Abstract-grained features} $\mathcal{F}^{\text{Abs}} = \{f_{k+1}, f_{k+2}, \dots, f_L\}$, derived from the deeper layers of ViT, which contain abstract semantic information closer to the text space, suitable for comprehension tasks.

The task type $T$ (comprehension or generation) determines which set of features is selected as the input for the downstream large language model:
\begin{equation}
    \mathcal{F}^{\text{img}}_T =
    \begin{cases}
        \mathcal{F}^{\text{Con}}, & \text{if } T = \text{generation task} \\
        \mathcal{F}^{\text{Abs}}, & \text{if } T = \text{comprehension task}
    \end{cases}
\end{equation}
We integrate the image features $\mathcal{F}^{\text{img}}_T$ and text features $\mathcal{T}$ into a joint sequence through simple concatenation, which is then fed into the LLM $\mathcal{M}_{\text{llm}}$ for autoregressive generation.
% :
% \begin{equation}
%     \mathcal{R} = \mathcal{M}_{\text{llm}}(\mathcal{U}|\theta), \quad \mathcal{U} = [\mathcal{F}^{\text{img}}_T; \mathcal{T}]
% \end{equation}
\subsection{Heterogeneous Knowledge Adaptation}
We devise H-LoRA, which stores heterogeneous knowledge from comprehension and generation tasks in separate modules and dynamically routes to extract task-relevant knowledge from these modules. 
At the task level, for each task type $ T $, we dynamically assign a dedicated H-LoRA submodule $ \theta^T $, which is expressed as:
\begin{equation}
    \mathcal{R} = \mathcal{M}_\text{LLM}(\mathcal{U}|\theta, \theta^T), \quad \theta^T = \{A^T, B^T, \mathcal{R}^T_\text{outer}\}.
\end{equation}
At the feature level for a single task, H-LoRA integrates the idea of Mixture of Experts (MoE)~\cite{masoudnia2014mixture} and designs an efficient matrix merging and routing weight allocation mechanism, thus avoiding the significant computational delay introduced by matrix splitting in existing MoELoRA~\cite{luo2024moelora}. Specifically, we first merge the low-rank matrices (rank = r) of $ k $ LoRA experts into a unified matrix:
\begin{equation}
    \mathbf{A}^{\text{merged}}, \mathbf{B}^{\text{merged}} = \text{Concat}(\{A_i\}_1^k), \text{Concat}(\{B_i\}_1^k),
\end{equation}
where $ \mathbf{A}^{\text{merged}} \in \mathbb{R}^{d^\text{in} \times rk} $ and $ \mathbf{B}^{\text{merged}} \in \mathbb{R}^{rk \times d^\text{out}} $. The $k$-dimension routing layer generates expert weights $ \mathcal{W} \in \mathbb{R}^{\text{token\_num} \times k} $ based on the input hidden state $ x $, and these are expanded to $ \mathbb{R}^{\text{token\_num} \times rk} $ as follows:
\begin{equation}
    \mathcal{W}^\text{expanded} = \alpha k \mathcal{W} / r \otimes \mathbf{1}_r,
\end{equation}
where $ \otimes $ denotes the replication operation.
The overall output of H-LoRA is computed as:
\begin{equation}
    \mathcal{O}^\text{H-LoRA} = (x \mathbf{A}^{\text{merged}} \odot \mathcal{W}^\text{expanded}) \mathbf{B}^{\text{merged}},
\end{equation}
where $ \odot $ represents element-wise multiplication. Finally, the output of H-LoRA is added to the frozen pre-trained weights to produce the final output:
\begin{equation}
    \mathcal{O} = x W_0 + \mathcal{O}^\text{H-LoRA}.
\end{equation}
% In summary, H-LoRA is a task-based dynamic PEFT method that achieves high efficiency in single-task fine-tuning.

\subsection{Training Pipeline}

\begin{figure}[t]
    \centering
    \hspace{-4mm}
    \includegraphics[width=0.94\linewidth]{fig/data.pdf}
    \caption{Data statistics of \texttt{VL-Health}. }
    \label{fig:data}
\end{figure}
\noindent \textbf{1st Stage: Multi-modal Alignment.} 
In the first stage, we design separate visual adapters and H-LoRA submodules for medical unified tasks. For the medical comprehension task, we train abstract-grained visual adapters using high-quality image-text pairs to align visual embeddings with textual embeddings, thereby enabling the model to accurately describe medical visual content. During this process, the pre-trained LLM and its corresponding H-LoRA submodules remain frozen. In contrast, the medical generation task requires training concrete-grained adapters and H-LoRA submodules while keeping the LLM frozen. Meanwhile, we extend the textual vocabulary to include multimodal tokens, enabling the support of additional VQGAN vector quantization indices. The model trains on image-VQ pairs, endowing the pre-trained LLM with the capability for image reconstruction. This design ensures pixel-level consistency of pre- and post-LVLM. The processes establish the initial alignment between the LLM’s outputs and the visual inputs.

\noindent \textbf{2nd Stage: Heterogeneous H-LoRA Plugin Adaptation.}  
The submodules of H-LoRA share the word embedding layer and output head but may encounter issues such as bias and scale inconsistencies during training across different tasks. To ensure that the multiple H-LoRA plugins seamlessly interface with the LLMs and form a unified base, we fine-tune the word embedding layer and output head using a small amount of mixed data to maintain consistency in the model weights. Specifically, during this stage, all H-LoRA submodules for different tasks are kept frozen, with only the word embedding layer and output head being optimized. Through this stage, the model accumulates foundational knowledge for unified tasks by adapting H-LoRA plugins.

\begin{table*}[!t]
\centering
\caption{Comparison of \ourmethod{} with other LVLMs and unified multi-modal models on medical visual comprehension tasks. \textbf{Bold} and \underline{underlined} text indicates the best performance and second-best performance, respectively.}
\resizebox{\textwidth}{!}{
\begin{tabular}{c|lcc|cccccccc|c}
\toprule
\rowcolor[HTML]{E9F3FE} &  &  &  & \multicolumn{2}{c}{\textbf{VQA-RAD \textuparrow}} & \multicolumn{2}{c}{\textbf{SLAKE \textuparrow}} & \multicolumn{2}{c}{\textbf{PathVQA \textuparrow}} &  &  &  \\ 
\cline{5-10}
\rowcolor[HTML]{E9F3FE}\multirow{-2}{*}{\textbf{Type}} & \multirow{-2}{*}{\textbf{Model}} & \multirow{-2}{*}{\textbf{\# Params}} & \multirow{-2}{*}{\makecell{\textbf{Medical} \\ \textbf{LVLM}}} & \textbf{close} & \textbf{all} & \textbf{close} & \textbf{all} & \textbf{close} & \textbf{all} & \multirow{-2}{*}{\makecell{\textbf{MMMU} \\ \textbf{-Med}}\textuparrow} & \multirow{-2}{*}{\textbf{OMVQA}\textuparrow} & \multirow{-2}{*}{\textbf{Avg. \textuparrow}} \\ 
\midrule \midrule
\multirow{9}{*}{\textbf{Comp. Only}} 
& Med-Flamingo & 8.3B & \Large \ding{51} & 58.6 & 43.0 & 47.0 & 25.5 & 61.9 & 31.3 & 28.7 & 34.9 & 41.4 \\
& LLaVA-Med & 7B & \Large \ding{51} & 60.2 & 48.1 & 58.4 & 44.8 & 62.3 & 35.7 & 30.0 & 41.3 & 47.6 \\
& HuatuoGPT-Vision & 7B & \Large \ding{51} & 66.9 & 53.0 & 59.8 & 49.1 & 52.9 & 32.0 & 42.0 & 50.0 & 50.7 \\
& BLIP-2 & 6.7B & \Large \ding{55} & 43.4 & 36.8 & 41.6 & 35.3 & 48.5 & 28.8 & 27.3 & 26.9 & 36.1 \\
& LLaVA-v1.5 & 7B & \Large \ding{55} & 51.8 & 42.8 & 37.1 & 37.7 & 53.5 & 31.4 & 32.7 & 44.7 & 41.5 \\
& InstructBLIP & 7B & \Large \ding{55} & 61.0 & 44.8 & 66.8 & 43.3 & 56.0 & 32.3 & 25.3 & 29.0 & 44.8 \\
& Yi-VL & 6B & \Large \ding{55} & 52.6 & 42.1 & 52.4 & 38.4 & 54.9 & 30.9 & 38.0 & 50.2 & 44.9 \\
& InternVL2 & 8B & \Large \ding{55} & 64.9 & 49.0 & 66.6 & 50.1 & 60.0 & 31.9 & \underline{43.3} & 54.5 & 52.5\\
& Llama-3.2 & 11B & \Large \ding{55} & 68.9 & 45.5 & 72.4 & 52.1 & 62.8 & 33.6 & 39.3 & 63.2 & 54.7 \\
\midrule
\multirow{5}{*}{\textbf{Comp. \& Gen.}} 
& Show-o & 1.3B & \Large \ding{55} & 50.6 & 33.9 & 31.5 & 17.9 & 52.9 & 28.2 & 22.7 & 45.7 & 42.6 \\
& Unified-IO 2 & 7B & \Large \ding{55} & 46.2 & 32.6 & 35.9 & 21.9 & 52.5 & 27.0 & 25.3 & 33.0 & 33.8 \\
& Janus & 1.3B & \Large \ding{55} & 70.9 & 52.8 & 34.7 & 26.9 & 51.9 & 27.9 & 30.0 & 26.8 & 33.5 \\
& \cellcolor[HTML]{DAE0FB}HealthGPT-M3 & \cellcolor[HTML]{DAE0FB}3.8B & \cellcolor[HTML]{DAE0FB}\Large \ding{51} & \cellcolor[HTML]{DAE0FB}\underline{73.7} & \cellcolor[HTML]{DAE0FB}\underline{55.9} & \cellcolor[HTML]{DAE0FB}\underline{74.6} & \cellcolor[HTML]{DAE0FB}\underline{56.4} & \cellcolor[HTML]{DAE0FB}\underline{78.7} & \cellcolor[HTML]{DAE0FB}\underline{39.7} & \cellcolor[HTML]{DAE0FB}\underline{43.3} & \cellcolor[HTML]{DAE0FB}\underline{68.5} & \cellcolor[HTML]{DAE0FB}\underline{61.3} \\
& \cellcolor[HTML]{DAE0FB}HealthGPT-L14 & \cellcolor[HTML]{DAE0FB}14B & \cellcolor[HTML]{DAE0FB}\Large \ding{51} & \cellcolor[HTML]{DAE0FB}\textbf{77.7} & \cellcolor[HTML]{DAE0FB}\textbf{58.3} & \cellcolor[HTML]{DAE0FB}\textbf{76.4} & \cellcolor[HTML]{DAE0FB}\textbf{64.5} & \cellcolor[HTML]{DAE0FB}\textbf{85.9} & \cellcolor[HTML]{DAE0FB}\textbf{44.4} & \cellcolor[HTML]{DAE0FB}\textbf{49.2} & \cellcolor[HTML]{DAE0FB}\textbf{74.4} & \cellcolor[HTML]{DAE0FB}\textbf{66.4} \\
\bottomrule
\end{tabular}
}
\label{tab:results}
\end{table*}
\begin{table*}[ht]
    \centering
    \caption{The experimental results for the four modality conversion tasks.}
    \resizebox{\textwidth}{!}{
    \begin{tabular}{l|ccc|ccc|ccc|ccc}
        \toprule
        \rowcolor[HTML]{E9F3FE} & \multicolumn{3}{c}{\textbf{CT to MRI (Brain)}} & \multicolumn{3}{c}{\textbf{CT to MRI (Pelvis)}} & \multicolumn{3}{c}{\textbf{MRI to CT (Brain)}} & \multicolumn{3}{c}{\textbf{MRI to CT (Pelvis)}} \\
        \cline{2-13}
        \rowcolor[HTML]{E9F3FE}\multirow{-2}{*}{\textbf{Model}}& \textbf{SSIM $\uparrow$} & \textbf{PSNR $\uparrow$} & \textbf{MSE $\downarrow$} & \textbf{SSIM $\uparrow$} & \textbf{PSNR $\uparrow$} & \textbf{MSE $\downarrow$} & \textbf{SSIM $\uparrow$} & \textbf{PSNR $\uparrow$} & \textbf{MSE $\downarrow$} & \textbf{SSIM $\uparrow$} & \textbf{PSNR $\uparrow$} & \textbf{MSE $\downarrow$} \\
        \midrule \midrule
        pix2pix & 71.09 & 32.65 & 36.85 & 59.17 & 31.02 & 51.91 & 78.79 & 33.85 & 28.33 & 72.31 & 32.98 & 36.19 \\
        CycleGAN & 54.76 & 32.23 & 40.56 & 54.54 & 30.77 & 55.00 & 63.75 & 31.02 & 52.78 & 50.54 & 29.89 & 67.78 \\
        BBDM & {71.69} & {32.91} & {34.44} & 57.37 & 31.37 & 48.06 & \textbf{86.40} & 34.12 & 26.61 & {79.26} & 33.15 & 33.60 \\
        Vmanba & 69.54 & 32.67 & 36.42 & {63.01} & {31.47} & {46.99} & 79.63 & 34.12 & 26.49 & 77.45 & 33.53 & 31.85 \\
        DiffMa & 71.47 & 32.74 & 35.77 & 62.56 & 31.43 & 47.38 & 79.00 & {34.13} & {26.45} & 78.53 & {33.68} & {30.51} \\
        \rowcolor[HTML]{DAE0FB}HealthGPT-M3 & \underline{79.38} & \underline{33.03} & \underline{33.48} & \underline{71.81} & \underline{31.83} & \underline{43.45} & {85.06} & \textbf{34.40} & \textbf{25.49} & \underline{84.23} & \textbf{34.29} & \textbf{27.99} \\
        \rowcolor[HTML]{DAE0FB}HealthGPT-L14 & \textbf{79.73} & \textbf{33.10} & \textbf{32.96} & \textbf{71.92} & \textbf{31.87} & \textbf{43.09} & \underline{85.31} & \underline{34.29} & \underline{26.20} & \textbf{84.96} & \underline{34.14} & \underline{28.13} \\
        \bottomrule
    \end{tabular}
    }
    \label{tab:conversion}
\end{table*}

\noindent \textbf{3rd Stage: Visual Instruction Fine-Tuning.}  
In the third stage, we introduce additional task-specific data to further optimize the model and enhance its adaptability to downstream tasks such as medical visual comprehension (e.g., medical QA, medical dialogues, and report generation) or generation tasks (e.g., super-resolution, denoising, and modality conversion). Notably, by this stage, the word embedding layer and output head have been fine-tuned, only the H-LoRA modules and adapter modules need to be trained. This strategy significantly improves the model's adaptability and flexibility across different tasks.


\section{Experiment}
\label{s:experiment}

\subsection{Data Description}
We evaluate our method on FI~\cite{you2016building}, Twitter\_LDL~\cite{yang2017learning} and Artphoto~\cite{machajdik2010affective}.
FI is a public dataset built from Flickr and Instagram, with 23,308 images and eight emotion categories, namely \textit{amusement}, \textit{anger}, \textit{awe},  \textit{contentment}, \textit{disgust}, \textit{excitement},  \textit{fear}, and \textit{sadness}. 
% Since images in FI are all copyrighted by law, some images are corrupted now, so we remove these samples and retain 21,828 images.
% T4SA contains images from Twitter, which are classified into three categories: \textit{positive}, \textit{neutral}, and \textit{negative}. In this paper, we adopt the base version of B-T4SA, which contains 470,586 images and provides text descriptions of the corresponding tweets.
Twitter\_LDL contains 10,045 images from Twitter, with the same eight categories as the FI dataset.
% 。
For these two datasets, they are randomly split into 80\%
training and 20\% testing set.
Artphoto contains 806 artistic photos from the DeviantArt website, which we use to further evaluate the zero-shot capability of our model.
% on the small-scale dataset.
% We construct and publicly release the first image sentiment analysis dataset containing metadata.
% 。

% Based on these datasets, we are the first to construct and publicly release metadata-enhanced image sentiment analysis datasets. These datasets include scenes, tags, descriptions, and corresponding confidence scores, and are available at this link for future research purposes.


% 
\begin{table}[t]
\centering
% \begin{center}
\caption{Overall performance of different models on FI and Twitter\_LDL datasets.}
\label{tab:cap1}
% \resizebox{\linewidth}{!}
{
\begin{tabular}{l|c|c|c|c}
\hline
\multirow{2}{*}{\textbf{Model}} & \multicolumn{2}{c|}{\textbf{FI}}  & \multicolumn{2}{c}{\textbf{Twitter\_LDL}} \\ \cline{2-5} 
  & \textbf{Accuracy} & \textbf{F1} & \textbf{Accuracy} & \textbf{F1}  \\ \hline
% (\rownumber)~AlexNet~\cite{krizhevsky2017imagenet}  & 58.13\% & 56.35\%  & 56.24\%& 55.02\%  \\ 
% (\rownumber)~VGG16~\cite{simonyan2014very}  & 63.75\%& 63.08\%  & 59.34\%& 59.02\%  \\ 
(\rownumber)~ResNet101~\cite{he2016deep} & 66.16\%& 65.56\%  & 62.02\% & 61.34\%  \\ 
(\rownumber)~CDA~\cite{han2023boosting} & 66.71\%& 65.37\%  & 64.14\% & 62.85\%  \\ 
(\rownumber)~CECCN~\cite{ruan2024color} & 67.96\%& 66.74\%  & 64.59\%& 64.72\% \\ 
(\rownumber)~EmoVIT~\cite{xie2024emovit} & 68.09\%& 67.45\%  & 63.12\% & 61.97\%  \\ 
(\rownumber)~ComLDL~\cite{zhang2022compound} & 68.83\%& 67.28\%  & 65.29\% & 63.12\%  \\ 
(\rownumber)~WSDEN~\cite{li2023weakly} & 69.78\%& 69.61\%  & 67.04\% & 65.49\% \\ 
(\rownumber)~ECWA~\cite{deng2021emotion} & 70.87\%& 69.08\%  & 67.81\% & 66.87\%  \\ 
(\rownumber)~EECon~\cite{yang2023exploiting} & 71.13\%& 68.34\%  & 64.27\%& 63.16\%  \\ 
(\rownumber)~MAM~\cite{zhang2024affective} & 71.44\%  & 70.83\% & 67.18\%  & 65.01\%\\ 
(\rownumber)~TGCA-PVT~\cite{chen2024tgca}   & 73.05\%  & 71.46\% & 69.87\%  & 68.32\% \\ 
(\rownumber)~OEAN~\cite{zhang2024object}   & 73.40\%  & 72.63\% & 70.52\%  & 69.47\% \\ \hline
(\rownumber)~\shortname  & \textbf{79.48\%} & \textbf{79.22\%} & \textbf{74.12\%} & \textbf{73.09\%} \\ \hline
\end{tabular}
}
\vspace{-6mm}
% \end{center}
\end{table}
% 

\subsection{Experiment Setting}
% \subsubsection{Model Setting.}
% 
\textbf{Model Setting:}
For feature representation, we set $k=10$ to select object tags, and adopt clip-vit-base-patch32 as the pre-trained model for unified feature representation.
Moreover, we empirically set $(d_e, d_h, d_k, d_s) = (512, 128, 16, 64)$, and set the classification class $L$ to 8.

% 

\textbf{Training Setting:}
To initialize the model, we set all weights such as $\boldsymbol{W}$ following the truncated normal distribution, and use AdamW optimizer with the learning rate of $1 \times 10^{-4}$.
% warmup scheduler of cosine, warmup steps of 2000.
Furthermore, we set the batch size to 32 and the epoch of the training process to 200.
During the implementation, we utilize \textit{PyTorch} to build our entire model.
% , and our project codes are publicly available at https://github.com/zzmyrep/MESN.
% Our project codes as well as data are all publicly available on GitHub\footnote{https://github.com/zzmyrep/KBCEN}.
% Code is available at \href{https://github.com/zzmyrep/KBCEN}{https://github.com/zzmyrep/KBCEN}.

\textbf{Evaluation Metrics:}
Following~\cite{zhang2024affective, chen2024tgca, zhang2024object}, we adopt \textit{accuracy} and \textit{F1} as our evaluation metrics to measure the performance of different methods for image sentiment analysis. 



\subsection{Experiment Result}
% We compare our model against the following baselines: AlexNet~\cite{krizhevsky2017imagenet}, VGG16~\cite{simonyan2014very}, ResNet101~\cite{he2016deep}, CECCN~\cite{ruan2024color}, EmoVIT~\cite{xie2024emovit}, WSCNet~\cite{yang2018weakly}, ECWA~\cite{deng2021emotion}, EECon~\cite{yang2023exploiting}, MAM~\cite{zhang2024affective} and TGCA-PVT~\cite{chen2024tgca}, and the overall results are summarized in Table~\ref{tab:cap1}.
We compare our model against several baselines, and the overall results are summarized in Table~\ref{tab:cap1}.
We observe that our model achieves the best performance in both accuracy and F1 metrics, significantly outperforming the previous models. 
This superior performance is mainly attributed to our effective utilization of metadata to enhance image sentiment analysis, as well as the exceptional capability of the unified sentiment transformer framework we developed. These results strongly demonstrate that our proposed method can bring encouraging performance for image sentiment analysis.

\setcounter{magicrownumbers}{0} 
\begin{table}[t]
\begin{center}
\caption{Ablation study of~\shortname~on FI dataset.} 
% \vspace{1mm}
\label{tab:cap2}
\resizebox{.9\linewidth}{!}
{
\begin{tabular}{lcc}
  \hline
  \textbf{Model} & \textbf{Accuracy} & \textbf{F1} \\
  \hline
  (\rownumber)~Ours (w/o vision) & 65.72\% & 64.54\% \\
  (\rownumber)~Ours (w/o text description) & 74.05\% & 72.58\% \\
  (\rownumber)~Ours (w/o object tag) & 77.45\% & 76.84\% \\
  (\rownumber)~Ours (w/o scene tag) & 78.47\% & 78.21\% \\
  \hline
  (\rownumber)~Ours (w/o unified embedding) & 76.41\% & 76.23\% \\
  (\rownumber)~Ours (w/o adaptive learning) & 76.83\% & 76.56\% \\
  (\rownumber)~Ours (w/o cross-modal fusion) & 76.85\% & 76.49\% \\
  \hline
  (\rownumber)~Ours  & \textbf{79.48\%} & \textbf{79.22\%} \\
  \hline
\end{tabular}
}
\end{center}
\vspace{-5mm}
\end{table}


\begin{figure}[t]
\centering
% \vspace{-2mm}
\includegraphics[width=0.42\textwidth]{fig/2dvisual-linux4-paper2.pdf}
\caption{Visualization of feature distribution on eight categories before (left) and after (right) model processing.}
% 
\label{fig:visualization}
\vspace{-5mm}
\end{figure}

\subsection{Ablation Performance}
In this subsection, we conduct an ablation study to examine which component is really important for performance improvement. The results are reported in Table~\ref{tab:cap2}.

For information utilization, we observe a significant decline in model performance when visual features are removed. Additionally, the performance of \shortname~decreases when different metadata are removed separately, which means that text description, object tag, and scene tag are all critical for image sentiment analysis.
Recalling the model architecture, we separately remove transformer layers of the unified representation module, the adaptive learning module, and the cross-modal fusion module, replacing them with MLPs of the same parameter scale.
In this way, we can observe varying degrees of decline in model performance, indicating that these modules are indispensable for our model to achieve better performance.

\subsection{Visualization}
% 


% % 开始使用minipage进行左右排列
% \begin{minipage}[t]{0.45\textwidth}  % 子图1宽度为45%
%     \centering
%     \includegraphics[width=\textwidth]{2dvisual.pdf}  % 插入图片
%     \captionof{figure}{Visualization of feature distribution.}  % 使用captionof添加图片标题
%     \label{fig:visualization}
% \end{minipage}


% \begin{figure}[t]
% \centering
% \vspace{-2mm}
% \includegraphics[width=0.45\textwidth]{fig/2dvisual.pdf}
% \caption{Visualization of feature distribution.}
% \label{fig:visualization}
% % \vspace{-4mm}
% \end{figure}

% \begin{figure}[t]
% \centering
% \vspace{-2mm}
% \includegraphics[width=0.45\textwidth]{fig/2dvisual-linux3-paper.pdf}
% \caption{Visualization of feature distribution.}
% \label{fig:visualization}
% % \vspace{-4mm}
% \end{figure}



\begin{figure}[tbp]   
\vspace{-4mm}
  \centering            
  \subfloat[Depth of adaptive learning layers]   
  {
    \label{fig:subfig1}\includegraphics[width=0.22\textwidth]{fig/fig_sensitivity-a5}
  }
  \subfloat[Depth of fusion layers]
  {
    % \label{fig:subfig2}\includegraphics[width=0.22\textwidth]{fig/fig_sensitivity-b2}
    \label{fig:subfig2}\includegraphics[width=0.22\textwidth]{fig/fig_sensitivity-b2-num.pdf}
  }
  \caption{Sensitivity study of \shortname~on different depth. }   
  \label{fig:fig_sensitivity}  
\vspace{-2mm}
\end{figure}

% \begin{figure}[htbp]
% \centerline{\includegraphics{2dvisual.pdf}}
% \caption{Visualization of feature distribution.}
% \label{fig:visualization}
% \end{figure}

% In Fig.~\ref{fig:visualization}, we use t-SNE~\cite{van2008visualizing} to reduce the dimension of data features for visualization, Figure in left represents the metadata features before model processing, the features are obtained by embedding through the CLIP model, and figure in right shows the features of the data after model processing, it can be observed that after the model processing, the data with different label categories fall in different regions in the space, therefore, we can conclude that the Therefore, we can conclude that the model can effectively utilize the information contained in the metadata and use it to guide the model for classification.

In Fig.~\ref{fig:visualization}, we use t-SNE~\cite{van2008visualizing} to reduce the dimension of data features for visualization.
The left figure shows metadata features before being processed by our model (\textit{i.e.}, embedded by CLIP), while the right shows the distribution of features after being processed by our model.
We can observe that after the model processing, data with the same label are closer to each other, while others are farther away.
Therefore, it shows that the model can effectively utilize the information contained in the metadata and use it to guide the classification process.

\subsection{Sensitivity Analysis}
% 
In this subsection, we conduct a sensitivity analysis to figure out the effect of different depth settings of adaptive learning layers and fusion layers. 
% In this subsection, we conduct a sensitivity analysis to figure out the effect of different depth settings on the model. 
% Fig.~\ref{fig:fig_sensitivity} presents the effect of different depth settings of adaptive learning layers and fusion layers. 
Taking Fig.~\ref{fig:fig_sensitivity} (a) as an example, the model performance improves with increasing depth, reaching the best performance at a depth of 4.
% Taking Fig.~\ref{fig:fig_sensitivity} (a) as an example, the performance of \shortname~improves with the increase of depth at first, reaching the best performance at a depth of 4.
When the depth continues to increase, the accuracy decreases to varying degrees.
Similar results can be observed in Fig.~\ref{fig:fig_sensitivity} (b).
Therefore, we set their depths to 4 and 6 respectively to achieve the best results.

% Through our experiments, we can observe that the effect of modifying these hyperparameters on the results of the experiments is very weak, and the surface model is not sensitive to the hyperparameters.


\subsection{Zero-shot Capability}
% 

% (1)~GCH~\cite{2010Analyzing} & 21.78\% & (5)~RA-DLNet~\cite{2020A} & 34.01\% \\ \hline
% (2)~WSCNet~\cite{2019WSCNet}  & 30.25\% & (6)~CECCN~\cite{ruan2024color} & 43.83\% \\ \hline
% (3)~PCNN~\cite{2015Robust} & 31.68\%  & (7)~EmoVIT~\cite{xie2024emovit} & 44.90\% \\ \hline
% (4)~AR~\cite{2018Visual} & 32.67\% & (8)~Ours (Zero-shot) & 47.83\% \\ \hline


\begin{table}[t]
\centering
\caption{Zero-shot capability of \shortname.}
\label{tab:cap3}
\resizebox{1\linewidth}{!}
{
\begin{tabular}{lc|lc}
\hline
\textbf{Model} & \textbf{Accuracy} & \textbf{Model} & \textbf{Accuracy} \\ \hline
(1)~WSCNet~\cite{2019WSCNet}  & 30.25\% & (5)~MAM~\cite{zhang2024affective} & 39.56\%  \\ \hline
(2)~AR~\cite{2018Visual} & 32.67\% & (6)~CECCN~\cite{ruan2024color} & 43.83\% \\ \hline
(3)~RA-DLNet~\cite{2020A} & 34.01\%  & (7)~EmoVIT~\cite{xie2024emovit} & 44.90\% \\ \hline
(4)~CDA~\cite{han2023boosting} & 38.64\% & (8)~Ours (Zero-shot) & 47.83\% \\ \hline
\end{tabular}
}
\vspace{-5mm}
\end{table}

% We use the model trained on the FI dataset to test on the artphoto dataset to verify the model's generalization ability as well as robustness to other distributed datasets.
% We can observe that the MESN model shows strong competitiveness in terms of accuracy when compared to other trained models, which suggests that the model has a good generalization ability in the OOD task.

To validate the model's generalization ability and robustness to other distributed datasets, we directly test the model trained on the FI dataset, without training on Artphoto. 
% As observed in Table 3, compared to other models trained on Artphoto, we achieve highly competitive zero-shot performance, indicating that the model has good generalization ability in out-of-distribution tasks.
From Table~\ref{tab:cap3}, we can observe that compared with other models trained on Artphoto, we achieve competitive zero-shot performance, which shows that the model has good generalization ability in out-of-distribution tasks.


%%%%%%%%%%%%
%  E2E     %
%%%%%%%%%%%%


\section{Conclusion}
In this paper, we introduced Wi-Chat, the first LLM-powered Wi-Fi-based human activity recognition system that integrates the reasoning capabilities of large language models with the sensing potential of wireless signals. Our experimental results on a self-collected Wi-Fi CSI dataset demonstrate the promising potential of LLMs in enabling zero-shot Wi-Fi sensing. These findings suggest a new paradigm for human activity recognition that does not rely on extensive labeled data. We hope future research will build upon this direction, further exploring the applications of LLMs in signal processing domains such as IoT, mobile sensing, and radar-based systems.

\section*{Limitations}
While our work represents the first attempt to leverage LLMs for processing Wi-Fi signals, it is a preliminary study focused on a relatively simple task: Wi-Fi-based human activity recognition. This choice allows us to explore the feasibility of LLMs in wireless sensing but also comes with certain limitations.

Our approach primarily evaluates zero-shot performance, which, while promising, may still lag behind traditional supervised learning methods in highly complex or fine-grained recognition tasks. Besides, our study is limited to a controlled environment with a self-collected dataset, and the generalizability of LLMs to diverse real-world scenarios with varying Wi-Fi conditions, environmental interference, and device heterogeneity remains an open question.

Additionally, we have yet to explore the full potential of LLMs in more advanced Wi-Fi sensing applications, such as fine-grained gesture recognition, occupancy detection, and passive health monitoring. Future work should investigate the scalability of LLM-based approaches, their robustness to domain shifts, and their integration with multimodal sensing techniques in broader IoT applications.


% Bibliography entries for the entire Anthology, followed by custom entries
%\bibliography{anthology,custom}
% Custom bibliography entries only
\bibliography{main}
\newpage
\appendix

\section{Experiment prompts}
\label{sec:prompt}
The prompts used in the LLM experiments are shown in the following Table~\ref{tab:prompts}.

\definecolor{titlecolor}{rgb}{0.9, 0.5, 0.1}
\definecolor{anscolor}{rgb}{0.2, 0.5, 0.8}
\definecolor{labelcolor}{HTML}{48a07e}
\begin{table*}[h]
	\centering
	
 % \vspace{-0.2cm}
	
	\begin{center}
		\begin{tikzpicture}[
				chatbox_inner/.style={rectangle, rounded corners, opacity=0, text opacity=1, font=\sffamily\scriptsize, text width=5in, text height=9pt, inner xsep=6pt, inner ysep=6pt},
				chatbox_prompt_inner/.style={chatbox_inner, align=flush left, xshift=0pt, text height=11pt},
				chatbox_user_inner/.style={chatbox_inner, align=flush left, xshift=0pt},
				chatbox_gpt_inner/.style={chatbox_inner, align=flush left, xshift=0pt},
				chatbox/.style={chatbox_inner, draw=black!25, fill=gray!7, opacity=1, text opacity=0},
				chatbox_prompt/.style={chatbox, align=flush left, fill=gray!1.5, draw=black!30, text height=10pt},
				chatbox_user/.style={chatbox, align=flush left},
				chatbox_gpt/.style={chatbox, align=flush left},
				chatbox2/.style={chatbox_gpt, fill=green!25},
				chatbox3/.style={chatbox_gpt, fill=red!20, draw=black!20},
				chatbox4/.style={chatbox_gpt, fill=yellow!30},
				labelbox/.style={rectangle, rounded corners, draw=black!50, font=\sffamily\scriptsize\bfseries, fill=gray!5, inner sep=3pt},
			]
											
			\node[chatbox_user] (q1) {
				\textbf{System prompt}
				\newline
				\newline
				You are a helpful and precise assistant for segmenting and labeling sentences. We would like to request your help on curating a dataset for entity-level hallucination detection.
				\newline \newline
                We will give you a machine generated biography and a list of checked facts about the biography. Each fact consists of a sentence and a label (True/False). Please do the following process. First, breaking down the biography into words. Second, by referring to the provided list of facts, merging some broken down words in the previous step to form meaningful entities. For example, ``strategic thinking'' should be one entity instead of two. Third, according to the labels in the list of facts, labeling each entity as True or False. Specifically, for facts that share a similar sentence structure (\eg, \textit{``He was born on Mach 9, 1941.''} (\texttt{True}) and \textit{``He was born in Ramos Mejia.''} (\texttt{False})), please first assign labels to entities that differ across atomic facts. For example, first labeling ``Mach 9, 1941'' (\texttt{True}) and ``Ramos Mejia'' (\texttt{False}) in the above case. For those entities that are the same across atomic facts (\eg, ``was born'') or are neutral (\eg, ``he,'' ``in,'' and ``on''), please label them as \texttt{True}. For the cases that there is no atomic fact that shares the same sentence structure, please identify the most informative entities in the sentence and label them with the same label as the atomic fact while treating the rest of the entities as \texttt{True}. In the end, output the entities and labels in the following format:
                \begin{itemize}[nosep]
                    \item Entity 1 (Label 1)
                    \item Entity 2 (Label 2)
                    \item ...
                    \item Entity N (Label N)
                \end{itemize}
                % \newline \newline
                Here are two examples:
                \newline\newline
                \textbf{[Example 1]}
                \newline
                [The start of the biography]
                \newline
                \textcolor{titlecolor}{Marianne McAndrew is an American actress and singer, born on November 21, 1942, in Cleveland, Ohio. She began her acting career in the late 1960s, appearing in various television shows and films.}
                \newline
                [The end of the biography]
                \newline \newline
                [The start of the list of checked facts]
                \newline
                \textcolor{anscolor}{[Marianne McAndrew is an American. (False); Marianne McAndrew is an actress. (True); Marianne McAndrew is a singer. (False); Marianne McAndrew was born on November 21, 1942. (False); Marianne McAndrew was born in Cleveland, Ohio. (False); She began her acting career in the late 1960s. (True); She has appeared in various television shows. (True); She has appeared in various films. (True)]}
                \newline
                [The end of the list of checked facts]
                \newline \newline
                [The start of the ideal output]
                \newline
                \textcolor{labelcolor}{[Marianne McAndrew (True); is (True); an (True); American (False); actress (True); and (True); singer (False); , (True); born (True); on (True); November 21, 1942 (False); , (True); in (True); Cleveland, Ohio (False); . (True); She (True); began (True); her (True); acting career (True); in (True); the late 1960s (True); , (True); appearing (True); in (True); various (True); television shows (True); and (True); films (True); . (True)]}
                \newline
                [The end of the ideal output]
				\newline \newline
                \textbf{[Example 2]}
                \newline
                [The start of the biography]
                \newline
                \textcolor{titlecolor}{Doug Sheehan is an American actor who was born on April 27, 1949, in Santa Monica, California. He is best known for his roles in soap operas, including his portrayal of Joe Kelly on ``General Hospital'' and Ben Gibson on ``Knots Landing.''}
                \newline
                [The end of the biography]
                \newline \newline
                [The start of the list of checked facts]
                \newline
                \textcolor{anscolor}{[Doug Sheehan is an American. (True); Doug Sheehan is an actor. (True); Doug Sheehan was born on April 27, 1949. (True); Doug Sheehan was born in Santa Monica, California. (False); He is best known for his roles in soap operas. (True); He portrayed Joe Kelly. (True); Joe Kelly was in General Hospital. (True); General Hospital is a soap opera. (True); He portrayed Ben Gibson. (True); Ben Gibson was in Knots Landing. (True); Knots Landing is a soap opera. (True)]}
                \newline
                [The end of the list of checked facts]
                \newline \newline
                [The start of the ideal output]
                \newline
                \textcolor{labelcolor}{[Doug Sheehan (True); is (True); an (True); American (True); actor (True); who (True); was born (True); on (True); April 27, 1949 (True); in (True); Santa Monica, California (False); . (True); He (True); is (True); best known (True); for (True); his roles in soap operas (True); , (True); including (True); in (True); his portrayal (True); of (True); Joe Kelly (True); on (True); ``General Hospital'' (True); and (True); Ben Gibson (True); on (True); ``Knots Landing.'' (True)]}
                \newline
                [The end of the ideal output]
				\newline \newline
				\textbf{User prompt}
				\newline
				\newline
				[The start of the biography]
				\newline
				\textcolor{magenta}{\texttt{\{BIOGRAPHY\}}}
				\newline
				[The ebd of the biography]
				\newline \newline
				[The start of the list of checked facts]
				\newline
				\textcolor{magenta}{\texttt{\{LIST OF CHECKED FACTS\}}}
				\newline
				[The end of the list of checked facts]
			};
			\node[chatbox_user_inner] (q1_text) at (q1) {
				\textbf{System prompt}
				\newline
				\newline
				You are a helpful and precise assistant for segmenting and labeling sentences. We would like to request your help on curating a dataset for entity-level hallucination detection.
				\newline \newline
                We will give you a machine generated biography and a list of checked facts about the biography. Each fact consists of a sentence and a label (True/False). Please do the following process. First, breaking down the biography into words. Second, by referring to the provided list of facts, merging some broken down words in the previous step to form meaningful entities. For example, ``strategic thinking'' should be one entity instead of two. Third, according to the labels in the list of facts, labeling each entity as True or False. Specifically, for facts that share a similar sentence structure (\eg, \textit{``He was born on Mach 9, 1941.''} (\texttt{True}) and \textit{``He was born in Ramos Mejia.''} (\texttt{False})), please first assign labels to entities that differ across atomic facts. For example, first labeling ``Mach 9, 1941'' (\texttt{True}) and ``Ramos Mejia'' (\texttt{False}) in the above case. For those entities that are the same across atomic facts (\eg, ``was born'') or are neutral (\eg, ``he,'' ``in,'' and ``on''), please label them as \texttt{True}. For the cases that there is no atomic fact that shares the same sentence structure, please identify the most informative entities in the sentence and label them with the same label as the atomic fact while treating the rest of the entities as \texttt{True}. In the end, output the entities and labels in the following format:
                \begin{itemize}[nosep]
                    \item Entity 1 (Label 1)
                    \item Entity 2 (Label 2)
                    \item ...
                    \item Entity N (Label N)
                \end{itemize}
                % \newline \newline
                Here are two examples:
                \newline\newline
                \textbf{[Example 1]}
                \newline
                [The start of the biography]
                \newline
                \textcolor{titlecolor}{Marianne McAndrew is an American actress and singer, born on November 21, 1942, in Cleveland, Ohio. She began her acting career in the late 1960s, appearing in various television shows and films.}
                \newline
                [The end of the biography]
                \newline \newline
                [The start of the list of checked facts]
                \newline
                \textcolor{anscolor}{[Marianne McAndrew is an American. (False); Marianne McAndrew is an actress. (True); Marianne McAndrew is a singer. (False); Marianne McAndrew was born on November 21, 1942. (False); Marianne McAndrew was born in Cleveland, Ohio. (False); She began her acting career in the late 1960s. (True); She has appeared in various television shows. (True); She has appeared in various films. (True)]}
                \newline
                [The end of the list of checked facts]
                \newline \newline
                [The start of the ideal output]
                \newline
                \textcolor{labelcolor}{[Marianne McAndrew (True); is (True); an (True); American (False); actress (True); and (True); singer (False); , (True); born (True); on (True); November 21, 1942 (False); , (True); in (True); Cleveland, Ohio (False); . (True); She (True); began (True); her (True); acting career (True); in (True); the late 1960s (True); , (True); appearing (True); in (True); various (True); television shows (True); and (True); films (True); . (True)]}
                \newline
                [The end of the ideal output]
				\newline \newline
                \textbf{[Example 2]}
                \newline
                [The start of the biography]
                \newline
                \textcolor{titlecolor}{Doug Sheehan is an American actor who was born on April 27, 1949, in Santa Monica, California. He is best known for his roles in soap operas, including his portrayal of Joe Kelly on ``General Hospital'' and Ben Gibson on ``Knots Landing.''}
                \newline
                [The end of the biography]
                \newline \newline
                [The start of the list of checked facts]
                \newline
                \textcolor{anscolor}{[Doug Sheehan is an American. (True); Doug Sheehan is an actor. (True); Doug Sheehan was born on April 27, 1949. (True); Doug Sheehan was born in Santa Monica, California. (False); He is best known for his roles in soap operas. (True); He portrayed Joe Kelly. (True); Joe Kelly was in General Hospital. (True); General Hospital is a soap opera. (True); He portrayed Ben Gibson. (True); Ben Gibson was in Knots Landing. (True); Knots Landing is a soap opera. (True)]}
                \newline
                [The end of the list of checked facts]
                \newline \newline
                [The start of the ideal output]
                \newline
                \textcolor{labelcolor}{[Doug Sheehan (True); is (True); an (True); American (True); actor (True); who (True); was born (True); on (True); April 27, 1949 (True); in (True); Santa Monica, California (False); . (True); He (True); is (True); best known (True); for (True); his roles in soap operas (True); , (True); including (True); in (True); his portrayal (True); of (True); Joe Kelly (True); on (True); ``General Hospital'' (True); and (True); Ben Gibson (True); on (True); ``Knots Landing.'' (True)]}
                \newline
                [The end of the ideal output]
				\newline \newline
				\textbf{User prompt}
				\newline
				\newline
				[The start of the biography]
				\newline
				\textcolor{magenta}{\texttt{\{BIOGRAPHY\}}}
				\newline
				[The ebd of the biography]
				\newline \newline
				[The start of the list of checked facts]
				\newline
				\textcolor{magenta}{\texttt{\{LIST OF CHECKED FACTS\}}}
				\newline
				[The end of the list of checked facts]
			};
		\end{tikzpicture}
        \caption{GPT-4o prompt for labeling hallucinated entities.}\label{tb:gpt-4-prompt}
	\end{center}
\vspace{-0cm}
\end{table*}
% \section{Full Experiment Results}
% \begin{table*}[th]
    \centering
    \small
    \caption{Classification Results}
    \begin{tabular}{lcccc}
        \toprule
        \textbf{Method} & \textbf{Accuracy} & \textbf{Precision} & \textbf{Recall} & \textbf{F1-score} \\
        \midrule
        \multicolumn{5}{c}{\textbf{Zero Shot}} \\
                Zero-shot E-eyes & 0.26 & 0.26 & 0.27 & 0.26 \\
        Zero-shot CARM & 0.24 & 0.24 & 0.24 & 0.24 \\
                Zero-shot SVM & 0.27 & 0.28 & 0.28 & 0.27 \\
        Zero-shot CNN & 0.23 & 0.24 & 0.23 & 0.23 \\
        Zero-shot RNN & 0.26 & 0.26 & 0.26 & 0.26 \\
DeepSeek-0shot & 0.54 & 0.61 & 0.54 & 0.52 \\
DeepSeek-0shot-COT & 0.33 & 0.24 & 0.33 & 0.23 \\
DeepSeek-0shot-Knowledge & 0.45 & 0.46 & 0.45 & 0.44 \\
Gemma2-0shot & 0.35 & 0.22 & 0.38 & 0.27 \\
Gemma2-0shot-COT & 0.36 & 0.22 & 0.36 & 0.27 \\
Gemma2-0shot-Knowledge & 0.32 & 0.18 & 0.34 & 0.20 \\
GPT-4o-mini-0shot & 0.48 & 0.53 & 0.48 & 0.41 \\
GPT-4o-mini-0shot-COT & 0.33 & 0.50 & 0.33 & 0.38 \\
GPT-4o-mini-0shot-Knowledge & 0.49 & 0.31 & 0.49 & 0.36 \\
GPT-4o-0shot & 0.62 & 0.62 & 0.47 & 0.42 \\
GPT-4o-0shot-COT & 0.29 & 0.45 & 0.29 & 0.21 \\
GPT-4o-0shot-Knowledge & 0.44 & 0.52 & 0.44 & 0.39 \\
LLaMA-0shot & 0.32 & 0.25 & 0.32 & 0.24 \\
LLaMA-0shot-COT & 0.12 & 0.25 & 0.12 & 0.09 \\
LLaMA-0shot-Knowledge & 0.32 & 0.25 & 0.32 & 0.28 \\
Mistral-0shot & 0.19 & 0.23 & 0.19 & 0.10 \\
Mistral-0shot-Knowledge & 0.21 & 0.40 & 0.21 & 0.11 \\
        \midrule
        \multicolumn{5}{c}{\textbf{4 Shot}} \\
GPT-4o-mini-4shot & 0.58 & 0.59 & 0.58 & 0.53 \\
GPT-4o-mini-4shot-COT & 0.57 & 0.53 & 0.57 & 0.50 \\
GPT-4o-mini-4shot-Knowledge & 0.56 & 0.51 & 0.56 & 0.47 \\
GPT-4o-4shot & 0.77 & 0.84 & 0.77 & 0.73 \\
GPT-4o-4shot-COT & 0.63 & 0.76 & 0.63 & 0.53 \\
GPT-4o-4shot-Knowledge & 0.72 & 0.82 & 0.71 & 0.66 \\
LLaMA-4shot & 0.29 & 0.24 & 0.29 & 0.21 \\
LLaMA-4shot-COT & 0.20 & 0.30 & 0.20 & 0.13 \\
LLaMA-4shot-Knowledge & 0.15 & 0.23 & 0.13 & 0.13 \\
Mistral-4shot & 0.02 & 0.02 & 0.02 & 0.02 \\
Mistral-4shot-Knowledge & 0.21 & 0.27 & 0.21 & 0.20 \\
        \midrule
        
        \multicolumn{5}{c}{\textbf{Suprevised}} \\
        SVM & 0.94 & 0.92 & 0.91 & 0.91 \\
        CNN & 0.98 & 0.98 & 0.97 & 0.97 \\
        RNN & 0.99 & 0.99 & 0.99 & 0.99 \\
        % \midrule
        % \multicolumn{5}{c}{\textbf{Conventional Wi-Fi-based Human Activity Recognition Systems}} \\
        E-eyes & 1.00 & 1.00 & 1.00 & 1.00 \\
        CARM & 0.98 & 0.98 & 0.98 & 0.98 \\
\midrule
 \multicolumn{5}{c}{\textbf{Vision Models}} \\
           Zero-shot SVM & 0.26 & 0.25 & 0.25 & 0.25 \\
        Zero-shot CNN & 0.26 & 0.25 & 0.26 & 0.26 \\
        Zero-shot RNN & 0.28 & 0.28 & 0.29 & 0.28 \\
        SVM & 0.99 & 0.99 & 0.99 & 0.99 \\
        CNN & 0.98 & 0.99 & 0.98 & 0.98 \\
        RNN & 0.98 & 0.99 & 0.98 & 0.98 \\
GPT-4o-mini-Vision & 0.84 & 0.85 & 0.84 & 0.84 \\
GPT-4o-mini-Vision-COT & 0.90 & 0.91 & 0.90 & 0.90 \\
GPT-4o-Vision & 0.74 & 0.82 & 0.74 & 0.73 \\
GPT-4o-Vision-COT & 0.70 & 0.83 & 0.70 & 0.68 \\
LLaMA-Vision & 0.20 & 0.23 & 0.20 & 0.09 \\
LLaMA-Vision-Knowledge & 0.22 & 0.05 & 0.22 & 0.08 \\

        \bottomrule
    \end{tabular}
    \label{full}
\end{table*}




\end{document}
 
% \onecolumn
\begin{center}
    \LARGE{
    \textbf{Computing Inconsistency Measures Under Differential Privacy} }\\
    \normalsize{Revision Letter} \\
    \textit{\normalsize{SIGMOD'24, Paper 1355}} 
\end{center}

\newcommand{\review}[1]{{\textit {\color{gray} ``#1''}}\vspace{1mm}}


We thank the reviewers for their insightful and thoughtful comments. In response to them, we have made the following major changes:

\common{
\begin{itemize}
\item {\bf Emphasize the motivation:} We have enhanced the motivation for computing inconsistency measures under differential privacy by adding a discussion about the scenario of the private data marketplace and a use case showing how such inconsistency measures can help estimate data quality in Section~\ref{sec:intro}. Such quality estimation can help assess the suitability of the private dataset before a buyer's purchase from the data marketplace.
\item {\bf Discussions and explanations of our model and results:} We have addressed nuances about our model and results in detailed discussions and proof sketches. Our discussions include possible extensions of our model to the multiple-relation scenario and explain our propositions, utility bounds, and algorithms. In particular, we have added proof sketches for all theorems/lemmas, improved clarification about utility bound for Thereom~\ref{thm:graph_general_utility}, and added discussions on multi-table synthesis in Section~\ref{sec:future}, threshold strategy for DCs in Section~\ref{sec:dc_aware} and inadequency of $\drastic$ and $\maxconsistency$ in Section~\ref{sec:hardness}. 
\item {\bf Scalability experiments:} We have added experiments that show the behavior of our algorithms in terms of scalability, examining settings with larger datasets and an increasing number of constraints. Specifically, we add three runtime analysis experiments by the varying dataset size on the Tax~\cite{nyctaxi} dataset, the number of constraints on Flight~\cite{flight} dataset, and the graph density by changing the datasets. These experiments refer to Figure~\ref{fig:runtime} in Section~\ref{sec:experiments} and Figure~\ref{fig:time_datasets_rev} in Appendix A.4. 
\end{itemize}
}

Please find our detailed response to each reviewer's comments below. In the paper, we highlight our revisions from Reviewer 1 in \reva{green}, Reviewer 2 in \revb{pink}, Reviewer 3 in \revc{red}, and all common revisions in \common{purple}. We have shortened and modified some previous text to fit the page limit.


%%%%%%%%%%%%%%%%%%%%%%%%%%%%%%%%%%%%%%%%
%%%%% REVIEWER 1
%%%%%%%%%%%%%%%%%%%%%%%%%%%%%%%%%%%%%%%%


\section*{Reviewer 1}

\paragraph{Comment R1.O1} 
\review{There is no reference to a clear real-life use-case in which inconsistency measures over a private dataset should be provided to an untrusted recipient. Could you please add a reference to at least one such real-life use-case?}

\paragraph{Response}
\reva{
{\em We have added a discussion and a use-case use case in \Cref{sec:intro} to motivate private inconsistency measures using private marketplaces and show how they can gauge the dataset quality in such scenarios.} 

``The utility of sensitive data primarily depends on its quality. Therefore, organizations that build these applications spend vast amounts of money on purchasing data from private data marketplaces~\cite{liu2021dealer, DBLP:conf/infocom/SunCLH22, DBLP:journals/corr/abs-2210-08723,DBLP:journals/tdsc/XiaoLZ23}. These marketplaces build relationships and manage monetary transactions between data owners and buyers. The buyers are often organizations that want to develop applications such as machine learning models or personalized assistants. Before the buyer purchases a dataset at a specific cost, they may want to ensure the data is suitable for their use case, adhere to particular data quality constraints, and be able to profile its quality to know if the cost reflects the quality. 


To address such scenarios, we consider the problem of assessing the quality of databases protected by DP. 
Such quality assessment will allow users to decide whether they can rely on the conclusions drawn from the data or whether the suggested data is suitable for them. ''
% To solve this problem, we must tackle several challenges. First, since DP protects the database, users can only observe noisy aggregate statistics, which can be challenging to summarize into a quality score. Second, if the number of constraints is large (e.g., if they were generated with an automatic system~\cite{BleifussKN17,DBLP:journals/pvldb/LivshitsHIK20,PenaAN21}), translating each constraint to an SQL {\tt COUNT} query and evaluating it over the database with a DP mechanism may lead to low utility since the number of queries is large, allowing for only a tiny portion of the privacy budget to be allocated to each query. 
}
% \benny{The extension from yes/no to quantity is indeed an elegant extension. But it is not enough. This paper is not just an elegant mathematical exercise. We need to emphasize the importance of extending from yes/no to measuring inconsistency. A yes/no answer 
%  does not distinguish between a database that vastly violates the user-provided constraints and one that mildly violates them and, hence, can be easily fixed or used as is. Also - one can say that data on the marketplace is clean; we can then say that it may be clean w.r.t.~the provider's constraints, but now the user can provide her own constraints and inspect the database accordingly.}

\ \\
\paragraph{Comment R1.O2} \review{ In Theorem 2, the utility bound holds with a probability of at least $1-\beta$ where $\beta=1/sqrt(2)=4.113$. Hence $1-\beta=0.757$, which is not very high. What happens when the bound does not hold? Is there a smooth decrease in utility? Can there be catastrophic failures?}

\paragraph{Response}
\reva{{\em We have clarified this in a discussion below \Cref{thm:graph_general_utility}.} 

``The $\beta$ parameter in the theorem is a controllable probability parameter. According to the accuracy requirements of a user's analysis, one may set $\beta$ as any value less than this upper bound. For example, if we set $\beta=0.01$, then our theoretical analysis of Algorithm~\ref{algo:expo_mech_basic} that says the algorithm's output being close to the true answer will hold with a probability of $1-\beta = 0.99$. We also show a plot to show the trend of the utility analysis as a function of $\beta$ in Appendix A.5 in the full version~\cite{full_paper}.''}
% \benny{I don't understand this answer. Do we we control over $\beta$? If so, why do we need $\frac{1}{e^{\sqrt{2}}}$ for? What is its purpose? Also, why do you say ``in practice'' here? Is it false in theory?}}

\ \\
\paragraph{Comment R1.O3} \review{The approaches proposed seem to be limited to a single table. What happens with tuples spread in several tables related by primary/foreign keys? Does it impact the sensitivity?}

\paragraph{Response}
\reva{
{\em We have included an in-depth discussion about the types of constraints supported by our model and a mapping of the challenges stemming from the multiple-table scenario in our new future work section (\Cref{sec:future}).} 

``Our approach can be extended to multi-relational tables as long as we can create conflict graphs representing the violations. 
However, in the multi-table setting, we must consider additional constraints that require tackling several challenges. In particular, these challenges may arise when we have non-binary or non-anti-monotonic constraints. Non-binary constraints with more than two tuples participating in a constraint lead to hypergraphs, and constraints like foreign key and inclusion constraints are non-anti-monotonic. They thus cannot be represented as conflict graphs. Such constraints are out of the scope of our work and need to be considered outside of the conflict graph realm. Furthermore, in the context of differential privacy, constraints on multi-relational tables also have implications for defining neighboring datasets and sensitivity that must be carefully considered.''}



% and their implications to differential privacy}

\ \\
\paragraph{Comment R1.O4} \review{Some lemma refer the reader to the extended version for the proof. For self-containment reasons, it would be nice to include at least a sketch for each proof in order to let busy readers easily intuite the proofs.}

\reva{
% \xh{add "$\backslash$ em" for consistent style}
Following this comment, {\em we have added 
proof sketches for all theorems and lemmas presented in the paper (\Cref{thm:privacy_proof_dc_oblivious,lemma:sensitivity,lemma:sens_quality,thm:privacy_proof_dc_aware,lemma:sens_quality_2stepEM}). }
}

\paragraph{Comment R1.O5} \review{In Theorem 5 the utility guarantees seem to be a bit overlooked. Theorem 5 claims that “ Algorithm 4 […] always outputs the size of a 2-approximate vertex cover of graph”. Am I missing something or is the Laplace perturbation of the size ignored? Also, could explainations to the following claim be given: “hence has the same utility as the original 2-approximation algorithm.”. How can that be given the Laplace perturbation? }

\paragraph{Response}
\reva{
{\em We have revised the phrasing of the theorem so that the new Theorem~\ref{thm:vertex_cover_priv_util_analysis} statement is as follows: }
``Algorithm~\ref{algo:dp_vertexcover} satisfies $\epsilon$-node DP and, prior to adding noise in line 7, obtains a 2-approximation vertex cover size."}


\paragraph{Comment R1.O6} \review{Minor remark: “fails miserably”: could more respectful words be used?}

\paragraph{Response}
\reva{
% \xh{add "$\backslash$ em" for consistent style}
We replaced this term with the phrase `falls short' {\em in \Cref{sec:results}.}
}

%%%%%%%%%%%%%%%%%%%%%%%%%%%%%%%%%%%%%%%%
%%%%% REVIEWER 2
%%%%%%%%%%%%%%%%%%%%%%%%%%%%%%%%%%%%%%%%


\section*{Reviewer 2}

\paragraph{Comment R2.O1} \review{The motivation for computing inconsistency measures under differential privacy is insufficiently articulated. It would be valuable to clarify why protecting inconsistency measures with differential privacy is a critical concern, particularly in real-world applications. For instance, how does this work benefit stakeholders or practitioners in specific fields where inconsistency measurement and data privacy intersect? Without this clarity, it is challenging to understand the practical importance or potential impact of the research.}

\paragraph{Response}
\revb{
{\em We have added a discussion and a use-case in \Cref{sec:intro} to motivate private inconsistency measures using private marketplaces and show how they can gauge the dataset quality in such scenarios.} Please see the response to R1.O1 for more details.  
% The utility of such sensitive data primarily depends on its quality. Therefore, organizations that build these applications spend vast amounts of money on purchasing data from private data marketplaces~\cite{liu2021dealer, DBLP:conf/infocom/SunCLH22, DBLP:journals/corr/abs-2210-08723,DBLP:journals/tdsc/XiaoLZ23}. These marketplaces build relationships and manage monetary transactions between data owners and buyers. These buyers are often organizations that want to develop applications such as machine learning models or personalized assistants. Before the buyer purchases a dataset at a specific cost, they may want to ensure the data is suitable for their use case, adhere to particular data quality constraints, and be able to profile its quality to know if the cost reflects the quality. 
}


\paragraph{Comment R2.O2} \review{The paper’s focus on single-relation databases is quite restrictive and does not reflect the complexity of many real-world databases that consist of multiple relational tables. It would be helpful if the authors could discuss the feasibility of extending their approach to multi-relation settings, as these are more common in practical applications. If such an extension is infeasible, a discussion on the technical challenges and limitations would provide context and help readers understand the constraints of the proposed methods.}

\paragraph{Response}
\revb{
{\em We have included an in-depth discussion about the types of constraints supported by our model and a mapping of the challenges stemming from the multiple-table scenario in a new future work section (\Cref{sec:future}). }

``Our approach can be extended to multi-relational tables as long as we can create conflict graphs representing the violations. 
However, in the multi-table setting, we must consider additional constraints that require tackling several challenges. ''
% In particular, these challenges may arise when we have non-binary or non-anti-monotonic constraints. Non-binary constraints with more than two tuples participating in a constraint lead to hypergraphs, and constraints like foreign key and inclusion constraints are non-anti-monotonic. They thus cannot be represented as conflict graphs. Such constraints are out of the scope of our work and need to be considered outside of the conflict graph realm. Furthermore, in the context of differential privacy, constraints on multi-relational tables also have implications for defining neighboring datasets and sensitivity that must be carefully considered.
Please see the response to R1.O3 for more details. 
}



%%%%%%%%%%%%%%%%%%%%%%%%%%%%%%%%%%%%%%%%
%%%%% REVIEWER 3
%%%%%%%%%%%%%%%%%%%%%%%%%%%%%%%%%%%%%%%%


\section*{Reviewer 3}

\paragraph{Comment R3.O1} \review{The optimization for computing the inconsistency measures works for Functional Dependencies, not DCs due to high sensitivity. The authors propose a fallback for dense graphs, but it seems to introduce significant noise. Developing a more robust strategy for handling DCs could improve utility and broaden the method’s applicability.}

\paragraph{Response}
\revc{
% \xh{add "$\backslash$ em" for consistent style}
{\em We have clarified and added a discussion on the importance of a more robust strategy for DCs in Section~\ref{sec:dc_aware} under the `Extension to general DCs' paragraph.}

``Despite being tailored for FDs, we show that, in practice, our approach has good performance for DCs as well. In \Cref{sec:experiments}, we show that this approach works well for the dense Adult~\cite{adult} dataset where we compute the $\problematic$ using this strategy in Figure~\ref{fig:comparing_strategies}. Developing a specific strategy for DCs is an important direction of future work.''}


% \benny{I also fail to follow the answer. We should say that the more general approach is something we develop in current and future work, and it seems to carry significant challenges that justify the separation between the graph approach (this paper) and more general approaches (future work).}


\paragraph{Comment R3.O2} \review{The discussion on the inadequacy of
I\_D and I\_MD seems somewhat flawed. Specifically for I\_MD, the fact that it is computationally hard does not justify overlooking it. It is a meaningful measure, and its release under DP remains valuable.}

% As stated in the "Inadequacy of $\drastic$ and $\maxconsistency$`` paragraph of Section 3 \ag{Reference?}, the $\maxconsistency$ measure is not only hard to compute but is hard also because of its high sensitivity. This is essentially because the $\maxconsistency$ measure requires the computation of the total number of maximal cliques which is a hard task to even approximate \sm{add citation}. We will clarify this in the next version.
\paragraph{Response}
\revc{
{\em We have clarified and emphasized the discussion on $\drastic$ and $\maxconsistency$ in the paragraph titled 'Inadequacy of $\drastic$ and $\maxconsistency$' in \Cref{sec:hardness}.}

``The \maxconsistency\ measure that computes the total number of independent sets in the conflict graph has both computational and high sensitivity issues. First, prior work~\cite{LivshitsBKS20} showed that computing \maxconsistency\ is \#P-complete and even approximating it is an NP-hard problem~\cite{DBLP:conf/ijcai/Roth93}. Even for special cases where \maxconsistency\ can be polynomially computed (when $\graph$ is $P_4$-free~\cite{KimelfeldLP20}), we show in Proposition~\ref{prop:sensitivity} that its sensitivity is exponential in the number of nodes of $\graph$. This significantly diminishes the utility of its DP estimate. Due to these challenges, we defer the study of $\drastic$ and $\maxconsistency$ to future work.''}



\paragraph{Comment R3.O3} \review{The paper doesn’t include any experiments on runtime or scalability, particularly for the graph projection approach and the exponential mechanism (EM) optimizations, which are likely the most computationally expensive. Adding this analysis with larger datasets would help show how well these methods scale to larger datasets in practical applications.}

\paragraph{Response}
\revc{

{\em We have added experiments to show the runtime for our approach as a function of dataset size and number of DCs.  
% \Cref{fig:runtime_revision} 
\Cref{fig:runtime} 
presents the runtime of our methods for each measure. Our analysis can be found in the paragraph titled `Runtime and scalability analysis' in \Cref{sec:results}. }

``We fix the privacy budget $\epsilon=1$ and run three experiments by varying the graph size, numbers of DCs, and dataset. For the first experiment, we use our largest dataset, Tax, and vary the number of nodes from $10^2$ to $10^6$. RNoise uses  $\alpha=0.005$ in the left plot and $\alpha=0.01$ in the center plot. We observe that the number of edges scales exponentially when we increase the number of nodes, and the time taken by our algorithm is proportional to the graph size. With a graph of $10^2$ nodes and $\leq 10$ edges, our algorithm takes $10^{-3}$ seconds and goes up to $4500$ seconds with $10^6$ nodes and $322$ million edges. We omit the experiment with $10^6$ nodes and $\alpha=0.01$ as the graph size for this experiment went over 30GB and was not supported by the pickle library we use to save our graph. This is not an artifact of our algorithm and can be scaled in the future using other graph libraries. 

For our second experiment, we choose a subset of $10k$ rows of the Flight dataset and vary the number of DCs to $13$ with $\alpha=0.005$. With one DC and $\alpha=0.005$, our algorithm takes approximately 5 seconds for $\mininconsistency$ and $\problematic$ and $\leq 1$ second for $\repair$, and goes up to $25$ seconds and $5$ seconds, respectively, for $\alpha=0.065$ and 13 DCs. We also notice some dips in the trend line (e.g., at 10 and 13 constraints) because the exponential mechanism chooses larger thresholds at those points, and the edge addition algorithm takes slightly longer with chosen thresholds. 

In our third experiment, we do a runtime analysis by varying the dataset and show the result in  Figure~\ref{fig:time_datasets_rev}. We fix the total number of nodes to 10k and noise to RNoise at $\alpha=0.001$ and vary the dataset. The x-axis is ordered according to the density of the dataset. We observe that the runtime is proportional to the density of the dataset and increases exponentially with the total number of edges in the graph. As the results are similar, we defer this experiment to Appendix A.4 in the full version~\cite{full_paper} due to space constraints.''
\begin{figure}
    \centering
    \includegraphics[width=0.5\linewidth]{images/runtime/time_datasets.png}
    \caption{\revc{Runtime analysis for all measures by varying datasets.}}
    \label{fig:time_datasets_rev}
\end{figure} 
}
% \begin{figure}
%     \centering
%     \includegraphics[width=\linewidth]{images/runtime/combined_plot.png}
%     % \includegraphics[width=.32\linewidth]{images/runtime/time_0.5.png}
%     % \includegraphics[width=.32\linewidth]{images/runtime/time_1.0.png}
%     % \includegraphics[width=.32\linewidth]{images/runtime/time_flight.png}
%     \includegraphics[width=0.4\linewidth]{images/runtime/legend_time.png}
%     \caption{\revc{Runtime analysis for all measures. Varying graph size on Tax dataset with RNoise $\alpha=0.005$ (left) and $\alpha=0.01$ (center) and varying \#DCs for Flight dataset (right). $\mininconsistency$ and $\problematic$ plots have y-axis in the log scale.}}
%     \label{fig:runtime_revision}
%     % Comparing different baselines at varying privacy budget.
% \end{figure}
 
%\newpage 

\setcounter{page}{1}
%%
%% This is file `sample-sigconf-authordraft.tex',
%% generated with the docstrip utility.
%%
%% The original source files were:
%%
%% samples.dtx  (with options: `all,proceedings,bibtex,authordraft')
%% 
%% IMPORTANT NOTICE:
%% 
%% For the copyright see the source file.
%% 
%% Any modified versions of this file must be renamed
%% with new filenames distinct from sample-sigconf-authordraft.tex.
%% 
%% For distribution of the original source see the terms
%% for copying and modification in the file samples.dtx.
%% 
%% This generated file may be distributed as long as the
%% original source files, as listed above, are part of the
%% same distribution. (The sources need not necessarily be
%% in the same archive or directory.)
%%
%%
%% Commands for TeXCount
% TC:macro \cite [option:text,text]
% TC:macro \citep [option:text,text]
% TC:macro \citet [option:text,text]
% TC:envir table 0 1
% TC:envir table* 0 1
% TC:envir tabular [ignore] word
% TC:envir displaymath 0 word
% TC:envir math 0 word
% TC:envir comment 0 0
%%
%%
%% The first command in your LaTeX source must be the \documentclass
%% command.
%%
%% For submission and review of your manuscript please change the
%% command to \documentclass[manuscript, screen, review]{acmart}.
%%
%% When submitting camera ready or to TAPS, please change the command
%% to \documentclass[sigconf]{acmart} or whichever template is required
%% for your publication.
%%
%%
% \documentclass[manuscript,review,anonymous]{acmart}
\documentclass[sigconf]{acmart}
%%
%% \BibTeX command to typeset BibTeX logo in the docs
\AtBeginDocument{%
  \providecommand\BibTeX{{%
    Bib\TeX}}}

\copyrightyear{2025} 
\acmYear{2025} 
\setcopyright{acmlicensed}\acmConference[IUI '25]{30th International Conference on Intelligent User Interfaces}{March 24--27, 2025}{Cagliari, Italy}
\acmBooktitle{30th International Conference on Intelligent User Interfaces (IUI '25), March 24--27, 2025, Cagliari, Italy}
\acmDOI{10.1145/3708359.3712162}
\acmISBN{979-8-4007-1306-4/25/03}

\usepackage{multirow}

%%
%% end of the preamble, start of the body of the document source.
\begin{document}

%%
%% The "title" command has an optional parameter,
%% allowing the author to define a "short title" to be used in page headers.
\title{An Exploratory Study on How AI Awareness Impacts Human-AI Design Collaboration}

%%
%% The "author" command and its associated commands are used to define
%% the authors and their affiliations.
%% Of note is the shared affiliation of the first two authors, and the
%% "authornote" and "authornotemark" commands
%% used to denote shared contribution to the research.

\author{Zhuoyi Cheng}
\orcid{0009-0003-3941-8869}
\affiliation{%
  \department{College of Software Technology}
  \institution{Zhejiang University}
  \city{Hangzhou}
  \country{China}
}
\email{zy_cheng@zju.edu.cn}

\author{Pei Chen}
\orcid{0000-0003-0962-6459}
\authornote{Corresponding author}
\affiliation{%
  \department{College of Computer Science and Technology}
  \institution{Zhejiang University}
  \city{Hangzhou}
  \country{China}
}
\email{chenpei@zju.edu.cn}

\author{Wenzheng Song}
\orcid{0009-0003-1271-7623}
\affiliation{%
  \department{College of Computer Science and Technology}
  \institution{Zhejiang University}
  \city{Hangzhou}
  \country{China}
}
\email{svvz@zju.edu.cn}

\author{Hongbo Zhang}
\orcid{0000-0001-5376-9460}
\affiliation{%
  \department{College of Computer Science and Technology}
  \institution{Zhejiang University}
  \city{Hangzhou}
  \country{China}
}
\email{hongbozhang@zju.edu.cn}

\author{Zhuoshu Li}
\orcid{0009-0000-3327-1622}
\affiliation{%
  \department{College of Computer Science and Technology}
  \institution{Zhejiang University}
  \city{Hangzhou}
  \country{China}
}
\email{lizhuoshu@zju.edu.cn}

\author{Lingyun Sun}
\orcid{0000-0002-5561-0493}
\affiliation{%
  \department{College of Computer Science and Technology}
  \institution{Zhejiang University}
  \institution{Zhejiang – Singapore Innovation and AI Joint Research}
  \city{Hangzhou}
  \country{China}
}
\email{sunly@zju.edu.cn}

\renewcommand{\shortauthors}{Cheng et al.}

\begin{abstract}
The collaborative design process is intrinsically complicated and dynamic, and researchers have long been exploring how to enhance efficiency in this process. As Artificial Intelligence technology evolves, it has been widely used as a design tool and exhibited the potential as a design collaborator. Nevertheless, problems concerning how designers should communicate with AI in collaborative design remain unsolved. To address this research gap, we referred to how designers communicate fluently in human-human design collaboration, and found awareness to be an important ability for facilitating communication by understanding their collaborators and current situation. However, previous research mainly studied and supported human awareness, the possible impact AI awareness would bring to the human-AI collaborative design process, and the way to realize AI awareness remain unknown. In this study, we explored how AI awareness will impact human-AI collaboration through a Wizard-of-Oz experiment. Both quantitative and qualitative results supported that enabling AI to have awareness can enhance the communication fluidity between human and AI, thus enhancing collaboration efficiency. We further discussed the results and concluded design implications for future human-AI collaborative design systems.
\end{abstract}

\begin{CCSXML}
<ccs2012>
   <concept>
       <concept_id>10003120.10003121.10003124.10011751</concept_id>
       <concept_desc>Human-centered computing~Collaborative interaction</concept_desc>
       <concept_significance>500</concept_significance>
       </concept>
   <concept>
       <concept_id>10003120.10003121.10011748</concept_id>
       <concept_desc>Human-centered computing~Empirical studies in HCI</concept_desc>
       <concept_significance>500</concept_significance>
       </concept>
 </ccs2012>
\end{CCSXML}

\ccsdesc[500]{Human-centered computing~Collaborative interaction}
\ccsdesc[500]{Human-centered computing~Empirical studies in HCI}

\keywords{Human-AI collaboration, Design collaboration, Generative AI, AI Awareness, Human-AI communication}

% \received{10 October 2024}  
% \received[revised]{19 December 2024}
% \received[accepted]{9 January 2025}

\maketitle

\section{Introduction}
The design process is intrinsically improvisational, open-ended, and dynamic~\cite{Rezwana2023COFI, Zhou2024Understanding}, in which designers occasionally switch between different design phases and iterate their ideas to achieve an optimal design~\cite{Hybs1992an}. To enhance efficiency in such a complicated process, researchers have long been studying how to integrate Artificial Intelligence~(AI) into the design process~\cite{Eckert2000Intelligent, Davis2016Empirically, Wang2024RoomDreaming}. Thanks to the rapid evolution in large-scale models, current AI technologies have gained higher creativity, stronger reasoning~\cite{Wang2024ModaVerse, Girdhar2023ImageBind}, and generation abilities~\cite{Shen2024PMG, Edwards2024Sketch2Prototype}, making it increasingly suitable for enhancing the efficiency of the design process. Prior endeavors have applied AI in different design stages, like ideation~\cite{Koch2019May, Xu2024Jamplate}, prototyping~\cite{Malsattar2019Designing, Dering2017an}, and evaluation~\cite{Camburn2020Computer, Zhang2017Computational}. In these cases, AI was treated as a tool to automate certain design tasks, and most interactions were initiated by designers. However, such AI tools may introduce new interaction paradigms and bring extra work in steering the system to obtain ideal feedback, which may increase designers' workload and learning cost~\cite{Gmeiner2023Exploring}. 

More recently, researchers have sought to enable AI to directly collaborate with designers. For example, \citet{muller2024group} designed an AI chatbot capable of autonomously suggesting ideas in the group brainstorming process, and \citet{Kuang2024UX} engaged UX designers to inspect UX problems with AI collaboratively. In the general area of human-AI collaboration, researchers have exploited different theories and frameworks to model human-AI collaboration, facilitating the understanding of human-AI collaboration at an abstract level. For instance, \citet{Tankelevitch2024Metacognitive} applied the descriptive framework for metacognition to understand users' metacognitive demands when using Generative AI systems, and provided cases to lower the demands. \citet{Muller2022Extending} described human-AI collaboration pattern by illustrating how human and AI initiative shifts in different scenarios. However, communication between human and AI remains to be explored extensively, which is crucial for decision-making, design information sharing, and shared understanding building in the highly dynamic design process~\cite{Chiu2002187an, Tan2023performance, Krishnakumar2022story},

To address this research gap, we first looked into existing studies about human-human collaboration to understand how designers communicate with collaborators fluently~\cite{Wardak2016Gestures, Baird2000ethnographic, Nelson2020Opening}. In various forms of human-human design collaboration, including face-to-face~\cite{Wardak2016Gestures}, virtual~\cite{Koutsabasis2012on}, and online~\cite{Vyas2013studio} collaborations, designers comprehend the status quo and leverage information in the environment and communicative cues from collaborators to initiate verbal communication. Having awareness of collaborators and the current situation is the key to such processes~\cite{Gutwin2002descriptive, Endsley2023Supporting, Vyas2013studio}. There are various definitions for the term ``awareness'' (e.g.,~\cite{Endsley2023Supporting, Gutwin2002descriptive, Gutwin2004Group}), but most of them share the core meaning concluded by~\citet[p.~107]{Dourish1992awareness}, which we used in this study: ``\textit{...awareness is an understanding of the activities of others, which provides a context for your own activity.}'' The awareness activity happens naturally in co-located collaboration, while awareness support is needed if people are working distributedly in a collaborative system~\cite{Gutwin2002descriptive}. For instance, \citet{Cidota2016HMD} evaluated different notification systems for users virtually collaborating in an augmented reality (AR) environment to develop users' awareness of each other. \citet{Dehler2011guiding} developed awareness support in a collaborative learning environment to assist learners in having awareness of others' knowledge. We propose that quipping AI with this ability can assist in efficient human-AI communication and collaboration. However, while humans can naturally be aware of others and the environment~\cite{Gutwin2002descriptive}, enabling AI to have awareness of humans is underexplored in both the impacts of doing so and the way to do so.

To this point, we explored the impacts of enabling AI awareness in human-AI design collaboration, and our main research question (RQ) focused on ``\textbf{How will AI having awareness of the designer and current situation impact communication in human-AI design collaboration?}'' As the RQ pertains to a future system that does not exist yet, we conducted user studies using the Wizard-of-Oz (WoZ) method. We first implemented a system that ensured the basic design and human-AI collaboration functions. Next, we iteratively designed a WoZ workflow with two Wizards to complement the communication and awareness functions. We then conducted a within-subject experiment involving 20 participants using this system and extracted quantitative and qualitative data from the experiment process recordings, conversation histories, and interview transcripts. The quantitative results showed that human-AI communication was significantly more fluent when AI had awareness, implying more efficient collaboration, and thematic analysis results of the interview presented detailed impacts on communication in human-AI collaboration. We further discussed the above results to build relationships between quantitative and qualitative data to reveal the causalities and inferences. Last, we concluded design implications about maintaining and leveraging awareness in human-AI collaboration and communication strategies for AI in future human-AI collaborative design systems.

Our research provides two main contributions to future human-AI collaborative design. First, we proposed a prospecting way to enhance communication fluidity in human-AI design collaboration by enabling AI to have awareness of designers and the current situation, and conducted user studies to certify this prospect. Second, we provided novel insights into future development of human-AI collaborative design systems based on the experiment results. We believe this study can shed light on how human and AI should communicate and collaborate not only in the design process, but also in other processes that share the complex and dynamic features with the design process.

\section{Related Work}
To better situate this study and provide references for system design, we conducted a thorough literature review of research on the following three topics: 1) how do researchers integrate AI in the design process, including AI-driven design tools and AI collaborators, 2) common communication channels in collaboration and how they were transferred to human-AI collaboration, and 3) mechanisms for supporting awareness in co-located and distributed human-human collaboration, and human-AI collaboration.

\subsection{Integrating AI in Design}
Design requires knowledge, information, and skills from diverse areas, and has long been a task that demands efforts beyond a single designer due to its burgeoning complexity~\cite{Rodgers1998role, Tang1996AI, Zhou2024Understanding}. The need for collaboration and assistance thus naturally appeared, and AI is one important technology that can help. In early research, AI was considered as a design support tool to assist in quantifiable or functional design tasks~\cite{Rodgers1998role}. For example, \citet{Chakrabarti1994Two} proposed a two-step approach to work out viable mechanical design solutions based on a set of known rules, and \citet{Grecu1996Learning} explored AI support for parametric design, where the product has already been defined except for combining several detailed values (e.g., color) with various options. In these tasks, AI was expected to reduce tedious work and give predictable outputs. As research on computational creativity booms, AI gained creativity and moved from the above ``routine design tasks'' to ``creative design tasks'' in the design process. Designers can now use various AI-driven tools to obtain ideas~\cite{Liu20233DALL, Koch2019May}, organize thoughts~\cite{Zhong2024Causal}, explore design space~\cite{Camburn2020Computer, Wang2024RoomDreaming}, etc. However, these applications are mainly design tools that require designers to initiate operations.

Recently, the development of AI abilities in generation and reasoning abilities~\cite{Wang2024ModaVerse, Edwards2024Sketch2Prototype} allowed AI to take an active role in design~\cite{Li2023Impression}, thus being prospective in becoming a design collaborator. Researchers are striving for better human-AI design collaboration from both theoretical and practical perspectives. \citet{Rezwana2023COFI} did a thorough literature review on human-AI co-creation and proposed a COFI framework to describe the interactions between collaborators and with the shared product in co-creative systems. \citet{Shi2023Understanding} also reviewed articles but specifically focused on designer-AI collaboration, and concluded five characteristics, including scope, access, agency, flexibility, and visibility, to help describe designer-AI collaboration and identify existing problems. Results from both works elaborated on problems in human-AI communication, with the former one presenting the lack of ``human to AI consequential communication'' and ``AI to human communication'', and the latter one pointing out the importance and a dearth of multiple communication modalities other than visual ones. Other more practice-oriented studies focus on specific design tasks, exploring interaction pathways and guidelines for human-AI collaboration. For example, \citet{Zhou2024Understanding} explored the nonlinear collaboration between graphic designers and AI in the OptiMuse system using the WoZ method, and proposed a co-design framework. \citet{Pan2023Diagrams} explored how AI can intelligently collaborate with people through a WoZ-like method, and implemented a collaborative diagram editing system guided by obtained insights.

Similar to \citet{Zhou2024Understanding} and \citet{Pan2023Diagrams}, our work seeks to explore the human-AI collaborative design process. Grounded in the problems identified by \citet{Rezwana2023COFI} and \citet{Shi2023Understanding}, we specifically focus on problems in human-AI communication and explore the potential of AI awareness to alleviate this problem.

\subsection{Communicating in Human-Human and Human-AI Collaboration}
In traditional human-human co-located design collaboration, people share expertise, ideas, resources, or responsibilities in the design process, to which communication is crucial~\cite{Chiu2002187an}. Verbal communication is the most common way that people exchange information. People leverage various linguistic strategies and non-verbal communicative cues to enhance the efficiency of verbal communication, for example, deictic words can specify the content of the current discussion in a highly contracted way~\cite{Tory2008deictic}, gestures can focus attention and represent ideas or features, and facial expressions can assist in judging one's true feelings~\cite{Ge2021emotion}. When collaboration is moved online and people work distributedly through collaborative systems, verbal communication, and its corresponding communication strategies require special support to function~\cite{Gutwin2002descriptive}. However, as remote collaboration becomes common and frequent, collaborative systems using text-based communication have matured rapidly~(e.g., Comment function in Figma\footnote{\url{https://www.figma.com/}} and Miro\footnote{\url{https://miro.com}}).

Similar to human-human distributed collaboration, communication and interaction in current human-AI collaboration is mainly based on text-based conversations. For example, \citet{muller2024group} implemented a text-based chatbot to facilitate the brainstorming process, and \citet{Kuang2024UX} designed a collaborative system for UX designers to identify usability problems with text-based conversation AI assistants. However, communicating with AI through text is not necessarily an optimal choice for human-AI communication, especially in the design process. Previous research has found communicating through voice leads to higher trust in both humans and chatbots compared to text~\cite{Burri2018, Bente2008Networking}. Additionally, the need for composing textual prompts and conducting other operations through visual channel can cause increased cognitive load~\cite{Tankelevitch2024Metacognitive, Vermeulen2008Sensory}, thus making it harder for mutual understanding in the design process~\cite{Gmeiner2023Exploring}. In the more general field of human-AI collaboration, researchers have found verbal communication to be a favored communication strategy because of its direct and efficient feature in difficult tasks~\cite{Zhang2021ideal}. In a recent research about sketching, researchers pointed out that people ``\textit{usually use language to talk about real, distant, invisible, imaginary, or conceptual objects}''~\cite[p.~4-5]{Rosenberg2024DrawTalking}, which are similar to what happens in the design process~\cite{Kokotovich2016abstraction, Ungureanu2021Analysing}. However, systems supporting designers to communicate with AI through speech are underexplored~\cite{Rezwana2023COFI, Shi2023Understanding}. Therefore, we allow designers to communicate with AI through speech in this study to gain insights for future human-AI collaborative design systems that support verbal communication.

\begin{figure*}[!th]
    \centering
    \includegraphics[width=1\linewidth]{interface.png}
    \caption{The interface example of the human-AI design collaborative system: a) Canvas. Canvas is the main interaction area where designers edit information using tools in the toolbar on the bottom right corner, and AI can also upload generated images to the canvas. b) Chat box. Communication history and speech recognition results are displayed in this area, and designers can also type here to initiate conversation in case of need. c) Camera screen. This is presented \textit{in the Aware condition only} to indicate what AI can see. \textit{Note: The interface is a translated version of the original one used for user studies.}}
    \label{fig:interface}
    \Description{This figure illustrates the system interface. The interface consists of three areas: canvas, chat box, and camera screen. The canvas, marked as area A, is on the right side, taking up about three-quarters of the interface. There is a user case from one participant, a toolbar in the bottom right corner, and a delete button in the top right corner of the canvas. The chat box and camera screen areas together take up about a quarter of the interface, marked as area B and area C respectively. The chat box takes up more than two-thirds of this area, with some communication histories of the participant and AI presenting in it and an input area for presenting speech recognition results at the bottom. The camera screen area takes up less than a third of this area, displaying the captured screen.}
\end{figure*}


\subsection{Supporting Awareness in Collaboration}
The term ``awareness'' had been combined with different words to coin a new term and specify a specific research scope, like shared awareness~\cite{Chakrabarti1994Two} and situation awareness~\cite{Endsley1988Design}. Similar as their overall meanings were, ``awareness'' has not been defined and used consistently~\cite{Gross2013Supporting}. In this paper, we adopt the definition provided by~\citet[p.~107]{Dourish1992awareness}:

\begin{quote}
    \textit{``[...]awareness is an understanding of the activities of others, which provides a context for your own activity. This context is used to ensure that individual contributions are relevant to the group's activity as a whole, and to evaluate individual actions with respect to group goals and progress. The information, then, allows groups to manage the process of collaborative working.''}
\end{quote}

Awareness is critical to successful design collaboration for providing information including shared information, group and individual activities, and coordination, to contextualize one's activity~\cite{Dourish1992awareness}. In co-located human-human collaboration, awareness activities are supported by human perceptions, which can happen naturally~\cite{Gutwin2002descriptive}. However, when collaboration is distributed, awareness activities require supporting mechanisms in the collaborative system to proceed~\cite{Gutwin2002descriptive}, and researchers have developed various mechanisms to enable people to have awareness of other collaborators and the environment in different scenarios. \citet{Gutwin1996Widgets} designed a set of interface widgets to support distributed workers having awareness of collaborators' current activities and view, including radar view, WYSIWIS (What You See Is What I See) view, WYSIWID (What You See Is What I Do) view, etc. \citet{Cidota2016HMD} compared different notification mechanisms for people collaborating through AR. \citet{Dehler2011guiding} designed a collaborative system and presented participants' knowledge levels to allow a better understanding of the group and each other. \citet{Janssen2011Group} visualized group members' contributions in online collaboration to examine how collaborative learning is impacted by group awareness.

Although studies and applications of awareness mechanisms in human-human collaboration are abundant, the idea of equipping AI with awareness ability is underexplored due to the infancy of human-AI collaboration research and the difficulty of transferring this human ability to AI. However, previous research has demonstrated that allowing users to have awareness of AI presence positively contributes to user engagement and collaborative experience~\cite{Rezwana2021penpal}, implying the potential advantages of applying the awareness theory in human-AI collaboration. Therefore, in this study, we first explored the impact of allowing AI to have awareness (i.e., AI is aware of the designer's design activities and current working content) through the WoZ method to ground future design and research. By exploring the potential of transferring the ``awareness'' concept in human-AI collaboration, we expect to guide the design of AI collaborators that have a certain level of initiative in a shared workspace~\cite{Salikutluk2024situational, He2024Canvas}, adjust their output according to real-time contextual information~\cite{Ma2023ProactiveAgent}, and infer deeper reasons and meaning behind the literal senses of obtained visual and textual information~\cite{Liu2024ComPeer, Liang2019Implicit} to support more dynamic and difficult tasks like design through human-AI collaboration.

\begin{figure*}[bthp]
    \centering
    \includegraphics[width=1\linewidth]{systemStructure.png}
    \caption{The overview of system structure and data stream. In our study, the participants \textbf{a)} interacted and communicated with the frontend (i.e., the user interface in Figure~\ref{fig:interface}) and received feedback from AI. The backend consisted of two Wizards and generative models. In the Aware condition, Wizard A would \textbf{b)} obtain awareness information from the camera screen and canvas, \textbf{c)} compose prompts based on predefined scripts and send them to Wizard B. In the Non-aware condition, the prompt sent in process \textbf{(c)} would not contain awareness information. Wizard B in both conditions supported the human-AI communication process by \textbf{d)} processing speech recognition results and input with canvas information, and \textbf{e)} sending prompts coming from the participants and Wizard A to generative models. \textbf{f)} All generation results would be uploaded to the canvas.}
    \Description{The overview of system structure and data stream. Areas from left to right are named: User area, Frontend area, Backend area. The user area presents a box representing the end user, the frontend area presents a draft of the system interface described in Figure 1, and the backend area consists of three boxes, with two boxes symbolizing two wizards in respective, and the remaining box symbolizing the generative models at the backend. The data streams are illustrated by two types of arrows, with the solid purple arrows representing data streams in both the Aware and Non-aware conditions, and the dashed gray arrows only representing data streams in the Aware condition. Data stream A describes the mutual data exchange between the user and the frontend, illustrated by a solid purple double arrow connecting the user area and the frontend area. Data stream B describes Wizard A's comprehension of awareness information, gained from the canvas and the camera screen, and is illustrated by two dashed gray arrows, pointing from the canvas and camera screen areas in the frontend area to Wizard A in the backend area. Data stream C describes the process of Wizard A sending prompts to Wizard B, illustrated by a solid purple arrow pointed from Wizard A to Wizard B. Data stream D describes the process Wizard B receives information from the frontend, illustrated by two solid purple arrows pointed from the canvas and chat box areas in the frontend area to Wizard B in the backend area. Data stream E describes the process of Wizard B sending crafted prompts to the generative models, illustrated by a solid purple arrow pointed from Wizard B to Generative Models. Data stream F describes the process of generative models sending the outputs to the frontend, illustrated by two solid purple arrows pointed from the generative model in the backend area to the canvas and chat box areas in the frontend area.}
    \label{fig:systemStructure}
\end{figure*}

\section{Human-AI Collaborative Design System}
In this section, we introduced the system used in our study, including user interface, system functions, and implementations\footnote{For source code and detailed guidelines, please refer to the supplementary material or \url{https://github.com/SongWZ3214/AI_teammate/tree/master}.}.

\subsection{User Interface and Interactions}
As an initial prototype, we kept the interface simple and consistent with common human-AI collaboration systems~(e.g.,~\cite{Kuang2024UX, Zhou2024Understanding}), and only functions served the RQ will be added apart from basic design and collaboration functions.

The interface is implemented with the React frontend framework\footnote{\url{https://react.dev}} and Ant Design UI component\footnote{\url{https://ant.design/components/overview-cn/}}. The overall interface layout is consistent with current collaborative systems containing a shared workspace/canvas and a chatbot~\cite{Kuang2024UX, Reza2024ABScribe, Robe2022PairBuddy, Ross2023Programmer}. Three main areas consist of the interface (Figure~\ref{fig:interface}): a) Canvas. Basic design functions, including adding and removing text, images, and sketches are supported. Participants can switch between different tools in the toolbar to edit the canvas. b) Chat box. Participants can communicate with AI through speech or textual input, and we encouraged participants to communicate through speech in this study unless the information can not be conveyed with speech. When using speech input, participants can naturally initiate or terminate a conversation without saying activation or termination commands. All communication histories between the participant and AI will be preserved and displayed in this area. Besides, participants can communicate with AI using deictic words like ``this'' or ``its'' to indicate what they select, which is a common linguistic strategy to simplify communication~\cite{Tory2008deictic, Bolt1980Put, Rosenberg2024DrawTalking}. c) Camera screen. We use a Hikvision 2K Computer Camera to monitor the participants. They will not interact with this area, but can see what the camera is currently capturing, which equals what the AI is seeing. This area \textbf{\textit{only}} appears in the condition where AI has awareness (henceforth the Aware condition), and in the condition where AI does not have awareness (henceforth the Non-aware condition), this area is hidden.

\subsection{System and Wizard Implementation}
\label{wizardImplementation}
The system and Wizard implementation in the two conditions (Aware and Non-aware) are illustrated in Figure~\ref{fig:systemStructure}. All contents and activities that designers see and experience exist in the frontend (Figure~\hyperref[fig:systemStructure]{\ref*{fig:systemStructure}a}), which is supported by two Wizards and generative models at the backend (Figure~\hyperref[fig:systemStructure]{\ref*{fig:systemStructure}b-f}). The system backend is built with Flask\footnote{\url{https://flask.palletsprojects.com/en/3.0.x/}}, and its functions can be broken down into three parts: awareness mechanisms fulfilled by Wizard A, communication functions supported by Wizard B, and feedback generation completed by generative models.

\textit{Awareness mechanisms.} The awareness mechanisms will only function in the Aware condition (See Figure~\hyperref[fig:systemStructure]{\ref*{fig:systemStructure}b}). In this condition, Wizard A will observe and judge participants' current design activities per 60-90 seconds from participants' shared screen (displaying the frontend/system interface) through an online meeting software according to adapted coding scheme from~\cite{Kim2011Relations} (Table~\ref{tab:coding_scheme}) to minimize bias. This information is then filled in a prompt template, and together with the screenshot of the current working process on the canvas, will be sent to Wizard B through real-time messaging software for eliciting AI feedback. In the Non-aware condition, Wizard A will only send common prompts without awareness information or screenshots. The prompt templates used in the two conditions were designed and iterated in pilot studies, and are presented as follows:

\begin{quote}
    \textbf{The Aware Condition:}\\
    \texttt{We are currently in Step [ ], designer's design activity is [ ], and the content the designer is currently working on is presented in the image.\\
    Your feedback is:}
\end{quote}

\begin{quote}
    \textbf{The Non-aware Condition:}\\
    \texttt{We are currently in Step [ ], please think divergently/convergently according to Requirement [ ].\\
    Your feedback is:}
\end{quote}

\textit{Communication between designer and AI.} Wizard B is mainly responsible for organizing participants' input into prompts, receiving prompts from Wizard A, and sending these prompts to generative models. Participants in our study are instructed to mainly use speech communication. We used the Web Speech API\footnote{\url{https://developer.mozilla.org/en-US/docs/Web/API/Web_Speech_API}} for speech recognition (i.e., STT, Speech-to-Text) and Text-to-Speech (TTS) functions. All that the participants speak will be recognized and transcribed to text, and sent to the backend. Wizard B will first judge whether the participants intend to communicate with AI, if yes, Wizard A will then assemble the transcribed speech snippets at the backend console if a complete paragraph is fragmented into several parts to form a complete prompt. Undesired inputs (e.g., think aloud, murmurs, conversations from and with the experimenters) are ignored in this process. Prompts produced by Wizard B and come from Wizard A are all sent to generative models for feedback generation. Textual inputs sent by participants will be directly sent to the generative models by Wizard A. The above operations were processed at the backend using a series of predefined short-cut commands by Wizard B.

\textit{Feedback generation.} Feedback generation is supported by dedicated generative models for generating verbal/textual content and images in respective, which is GPT-4 turbo\footnote{\url{https://platform.openai.com/docs/models/gpt-4-and-gpt-4-turbo}} for verbal/textual content, and DALL·E 3\footnote{\url{https://openai.com/index/dall-e-3/}} for images. GPT-4 turbo will be initiated by the same prompt across participants (Appendix~\ref{initiate_prompt}) to ensure the models behave consistently across participants. The generated contents are then uploaded to the frontend. Images will be directly put on the canvas, verbal content will be converted into speech, and its corresponding text will be displayed in the chat box.

\begin{figure}[ht]
    \centering
    \includegraphics[width=\linewidth]{participantPosition.png}
    \caption{The WoZ experiment setting, including the positions of the participants and experimenters (Wizards), and important devices.}
    \label{fig:participantPosition}
    \Description{This figure illustrates the Wizard-of-OZ experiment setting. The participant sits on the left short side of a rectangular table, with a touchscreen, a keyboard, a mouse, and a stylus pen put in front of the participant. The two Wizards sit at the bottom long side of the rectangular table, with Wizard A sitting closer to the participant and Wizard B sitting on the right of Wizard A. Two wizards use separate laptops that are put in front of them.}
\end{figure}

\section{User Study}
\subsection{Participants}
We recruited 20 participants (denoted as P1-P20) through an online forum, including 11 females and 9 males, ranging in age from 22 to 32~($M = 23.6$, $SD = 2.4$). They have at least three years of design experience, and knowledge of AI-driven design tools and generative AI. All participants signed a consent form before the experiment and received CNY 100 (USD 14) after completion.

\subsection{Condition Settings and Tasks}
The experiment included two conditions: AI has awareness of the designer and current situation (the Aware condition), and does not have such awareness (the Non-aware condition). Whether or not AI has awareness was controlled by the workflow of Wizard A (See \S\ref{wizardImplementation} and Figure~\ref{fig:systemStructure})

We conducted within-subject user studies to alleviate differences existing in the following aspects that can discredit the results: 1) individual hobbies and habits that can affect the metrics we used (\S\ref{measurements}), 2) Wizards' changing dexterity on WoZ workflow affected by accumulated experiences. Each participant went through both conditions. In each condition, participants were provided with materials describing a product, several target users, and possible user requirements of the product. They should produce 3-5 design concepts of the product in four predefined steps: 1) select only one type of target user and determine 1-3 main requirement(s) of the selected user, 2) break down the main requirement(s) into multiple detailed key requirements, and select the most important three key requirements, 3) explore solutions to each of the key requirement, and 4) combine the solutions to obtain 3-5 design concepts. The steps were excerpted and adapted from the product design and development steps by~\citet{Ulrich2016product}. To avoid the practice effect, participants would need to design different products in the two conditions, which were e-scooter and headphone, respectively. We determined the two products considering they are common, neither too simple nor too complex in technology and engineering, have a wide range of target users and usage scenarios, and are possible to produce various design concepts. The experiment conditions and design tasks were cross-assigned to regulate the order effect.

\subsection{Measurement}
\label{measurements}
There are no universal metrics for evaluating the communication fluidity between human and AI, but researchers have concluded communication-related analysis methods and metrics in human-human collaboration~\cite{Nyerges1998Developing, Tang1992why, Billinghurst2003communication, DalyJones1998some, OMalley1996Comparison, Convertino2011Supporting}, from which human-AI design collaboration studies often borrow. Measures used to evaluate collaboration were categorized into performance, process, and subjective measures~\cite{Billinghurst2003communication}. Performance measures are metrics pertaining to task outcome, but were found insensitive across conditions in short-term tasks, partly because ``\textit{people tend to protect their primary task of getting the work done}''[\cite{DalyJones1998some}, p.~35, \cite{Monk1996Measures}]. Using process measures is a recommended solution~\cite{Monk1996Measures}. Also, subjective measures were utilized as complementary evidence of the process measure, which corresponds to the experiment process recordings, conversation histories, and interviews in this study.

Commonly used metrics include the turn frequency, turn duration, incidence/duration of overlapping speech, number of interruptions, turn length, turn completions, backchannels, deixis, questioning behaviors etc.~\cite{Billinghurst2003communication, OMalley1996Comparison}. In this study, we mainly considered the following metrics: turn frequency, turn duration, turn length, and deixis. Other metrics were left unselected either due to the difficulty of identification or the rarity of occurrence in our experiment. The definitions of the metrics in this study are listed as follows:

\begin{itemize}
    \item \textbf{Turn frequency:} The total number of conversation turns initiated by the participant. (unit: turns)
    \item \textbf{Turn duration:} The time participant spent to complete a turn. (unit: second)
    \item \textbf{Turn length:} The number of words in each turn for the participant. As we conducted experiments in Chinese, we calculated the number of Chinese characters. (unit: words)
    \item \textbf{Deixis:} The total number of deictic words used by the participant in the whole communication process (unit: times).
\end{itemize}

\begin{figure*}[!h]
    \centering
    \includegraphics[width=1\linewidth]{analysisOverview.png}
    \caption{The overview of the data analysis process. Our raw data mainly came from three sources: The experiment process, the human-AI conversation during the experiments, and interview. These data were processed into valid data for further analysis, including four types of quantitative data corresponding to our predefined metrics, and three types of qualitative data. We analyzed these data and reported how AI awareness impacted communication between human and AI in the Result section (\S\ref{result}). Furthermore, we discussed the inferences and causalities of quantitative results and concluded design implications for future human-AI collaborative design systems in the Discussion section (\S\ref{Discussion}).}
    \label{fig:analysisOverview}
    \Description{This figure illustrates the data analysis process. The whole process includes six steps: Raw data area, data processing, data for analysis, data analysis, results, and discussion. Raw data includes experiment process recordings, full conversation histories, and interview recordings. We extracted quantitative data from the experiment process recordings and full conversation histories by calculation and counting after excluding invalid snippets. Interview recordings are transcribed into interview transcripts. After data processing, we gain four types of quantitative data and three types of qualitative data. Quantitative data include turn frequency, turn duration, turn length, and deixis. Qualitative data includes experiment process recordings, valid conversation histories, and interview transcripts. Quantitative data are then analyzed by paired sample statistical tests, which generate statistical results and are then used to draw inferences. Interview transcripts are analyzed using thematic analysis methods, which generate themes and codes. Themes and codes on the one hand are used to draw design implications, and on the other hand, are used to explain the causalities of statistical results, also exemplified by experiment process recordings and valid conversation histories.}
\end{figure*}

\subsection{Experiment Environment and Procedures}
Two Wizards experienced a training session before the formal user studies by practicing for 4 complete trials of each condition to familiarize themselves with the workflow.

The experiment setting is illustrated in Figure~\ref{fig:participantPosition}. The system interface was displayed on a Dell P2418HT touchscreen, and the participants were provided with a stylus pen, a keyboard, and a mouse to operate the system, and were encouraged to use speech as the main communication channel. The camera was fixed on the top of the touch screen, and the captured screen was displayed in the interface. The two Wizards sat beside the participant, and Wizard A could see the participant's live screen via an online meeting software. 

Each experiment included three sessions and lasted about 2 hours. All participants signed the informed consent before all sessions, and the WoZ method was not revealed to the participants until all sessions ended. The three sessions are described as follows:

\begin{enumerate}
    \item \textbf{Preparation:} The preparation session lasted about 20 minutes. This session commenced with an introduction to the experiment's background, goal, procedure, and reward. Then, one of the experimenters explained how to use the system, followed by a 5-minute practice using the system.
    \item \textbf{Formal experiments:} All participants went through both conditions. At the beginning of each condition, the experimenter will emphasize the features of this condition. Then, the participants would start the task following the four steps. Each condition took up about 40 minutes, and the participant was allowed to have a 5-minute break between the two conditions.
    \item \textbf{Post-experiment interview:} Participants underwent a 15-minute semi-structured interview to help us gain a deeper understanding of their behaviors, perceptions of AI, communication and collaboration experiences in the two conditions, and suggestions for further optimization.
\end{enumerate}

\begin{table*}[hbtp]
    \centering
    \caption{The overall statistical test results of the four metrics.}
    \begin{tabular}{lccccccc}
        \toprule
        \multicolumn{8}{l}{\textit{\textbf{Paired sample t-test}}}\\
        \midrule
        & \multicolumn{2}{c}{Aware} & \multicolumn{2}{c}{Non-aware} &  &  & \\
        \cline{2-5}
                       & \textit{M} (Mean)  & \textit{SD}    & \textit{M} (Mean)  & \textit{SD}    & $t$    & $p$ & Cohen's d\\
        Turn Frequency & 26.93 & 12.17 & 21.00 & 10.48 & 2.731  & \textbf{.012}$^{*}$ & 0.777\\
        Deixis         & 5.29 & 3.97 & 5.78 & 4.58 & -0.275 & .756 & 0.085\\
        \specialrule{0.8pt}{0.4ex}{0.4ex}
        \multicolumn{8}{l}{\textit{\textbf{Paired Wilcoxon signed-rank test}}}\\
        \midrule
        & \multicolumn{2}{c}{Aware} & \multicolumn{2}{c}{Non-aware} &  &  & \\
        \cline{2-5}
                       & \textit{M} (Mean)  & \textit{SD}    & \textit{M} (Mean)  & \textit{SD}    & $Z$    & $p$ & $r$\\
        Turn Duration  & 10.14 & 5.84  & 13.87 & 5.50  & -3.233 & \textbf{.001}$^{**}$ & 0.864\\
        Turn Length    & 38.85 & 20.82 & 46.20 & 20.81 & -3.107 & \textbf{.002}$^{**}$ & 0.830\\
        \bottomrule
        \multicolumn{8}{l}{\footnotesize *, **, *** refer to $p$ \textless .05, $p$ \textless .01, $p$ \textless .001, respectively.}\\
    \end{tabular}
    \label{tab:quantitativeResults}
\end{table*}

\subsection{Analysis Method}
The overview of the analysis process is illustrated in Figure~\ref{fig:analysisOverview}.
We obtained raw data from three sources: experiment process video recordings, full textual conversation histories stored in the system backend, and interview audio recordings. Before formal analysis, we first processed them to extract valid quantitative and qualitative data. We excluded invalid conversation snippets from the conversation histories, including speech from the experimenter, conversations that happened between the participants and experimenters when errors occurred, and meaningless snippets like ``hmm...''. Then we counted the number of turn frequencies and deixis, and calculated the turn length using the processed conversation histories. Next, we calculated the duration of each turn together with video recordings. For interview recordings, we transcribed them into texts. After the data processing processes, we obtained four types of quantitative data: turn frequency, turn duration, turn length, and the number of deixis, and three types of qualitative data: experiment recordings (not processed), valid conversation histories, and interview transcripts.

For quantitative data, we employed suitable paired sample statistical tests. First, we tested the normality of each paired sample using a Shapiro-Wilk normal distribution test. Pairs including at least one sample that is not normally distributed will be analyzed by the paired Wilcoxon signed-rank test, and normal-distributed pairs will be analyzed by paired t-test. For qualitative data, we mainly analyzed the interview transcripts using the thematic analysis method~\cite{Braun2012Thematic}. We reported the statistical results and thematic analysis results concerning both AI awareness and communication in the Result section (\S\ref{result}). In the Discussion section (\S\ref{Discussion}), we discussed the inferences and causalities of the statistical results with the help of qualitative data, and concluded design implications extracted from the interview.

\section{Result}
\label{result}
In this section, we described how AI awareness impacted the communication between human and AI through quantitative and qualitative data.

\subsection{Quantitative Results}
Data from three participants (P11, P14, P19) were dismissed for disobeying the predefined steps during the experiment task. For example, P19 ignored predefined steps and kept iterating product appearances using AI, making it difficult for Wizard A to compose prompts based on predefined scripts. Besides, we decided not to include data from another three participants (P8, P9, P17) because they used textual inputs that took up an unignorable proportion (e.g., only 52\% of P17's turns were input by speech in the Aware condition), although we encouraged them to communicate with AI using speech unless the contents could not be conveyed through such. As participants usually behave differently between speech and textual input, mixing the two data types in the analysis can discredit the results.

The Shapiro-Wilk normal distribution test results showed that data of turn duration from the Aware condition ($p = .008$) and turn length from both conditions ($p_{Aware} = .004$, $p_{Non-aware} = .007$) were not normally distributed, and the remaining were normally distributed, so we used the paired sample t-test to analyze turn frequency and deixis, paired Wilcoxon signed-rank test to analyze turn duration and turn length. The statistical results are presented in Table~\ref{tab:quantitativeResults}. Results show that participants averagely initiated more turns in the Aware condition~($M = 26.93$) compared to the Non-aware condition ($M = 21.00$, $p = .012$, $d = 0.777$). In the Aware condition, participants spent significantly less time~($M = 10.14$) on each turn compared to the Non-aware condition ($M = 13.87$, $p = .001$, $r = 0.864$). Each turn contained significantly fewer words in the Aware condition~($M = 38.85$) than in the Non-aware condition ($M = 46.20$, $p = .002$, $r = 0.830$). However, no significant difference was found in the number of deixis words between the two conditions~($p = .756$).

\subsection{Qualitative Results from Interview}
\label{interview}
We conducted the thematic analysis with an iterative coding process. First, we reviewed the transcripts to familiarize ourselves with the general contents and form an initial understanding. Then, for each transcript, we extract participant's perspectives and abstract them into codes, among which perspectives aligning with our initial understanding would be prioritized in extraction to form the first several codes. In the initial stage, all perspectives were extracted and coded, which included 13 codes about perceived impacts on communication and collaboration from AI awareness. We then iteratively merged similar codes until every code did not overlap each other, and obtained seven final codes, which were categorized into three themes: communication willingness, communication and collaboration effort, and speech style.

\subsubsection{Communication Willingness}
Ten participants mentioned their affected communication willingness brought by the change in AI awareness. Seven out of the ten felt different levels of target alignment between AI and themselves in the two conditions, and the feeling that AI in the Non-aware condition was misaligned with them lowered their willingness to communicate. When the participants were finding solutions for a certain design requirement in the process, they felt less willing to communicate with AI if it proposed solutions for another requirement, and tended to focus on their own thoughts:

\begin{quote}
    \textit{``AI seemed to not care about me and was just working on its own. So even if I heard AI speaking, I did not really get much useful information, and I tend to focus on what's in my mind instead of writing something or communicating with it.''} (P18)
\end{quote}

\noindent Another participant who was sensitive to efficiency also mentioned the alignment problem in the Non-aware condition, and misaligned contents provided by AI would likely be ignored for advancing the design process with a clear route:

\begin{quote}
    \textit{``If our intentions and target can not keep synchronized, I prefer not receiving messages from AI because it would affect our overall working efficiency.''} (P20)
\end{quote}

\noindent On the contrary, three participants explicitly mentioned a higher level of alignment in the Aware condition, which increased their communication willingness. In the Aware condition, participants could easily recognize that AI took their current working progress into consideration by commencing with \textit{``I noticed that you are doing...''} (P4), which could encourage the participants to communicate with AI:

\begin{quote}
    \textit{``...it feels good for AI to recognize my canvas and hint me [...] I was just considering that issue and the AI directly pointed it out. I was quite surprised and started to discuss this issue with it.''} (P15)
\end{quote}

Apart from the alignment issue, six participants also stated to be more comfortable communicating with AI in the Aware condition. P2 said that AI became more intelligent in the Aware condition for \textit{``giving more reliable suggestions on headphone design''}, which introduced less disagreement between AI and the participant:

\begin{quote}
    \textit{``AI kept disagreeing with me (in the Non-aware condition). But the second one (Aware condition) was more intelligent and gave more reliable suggestions, which ensured the communication to stay in a comfortable state.''} (P2)
\end{quote}

\noindent Furthermore, three participants were glad to have a feeling of being noticed in the Aware condition when AI mentioned what they were doing, making them more willing to communicate with AI to exchange their thoughts:

\begin{quote}
    \textit{``Some features (of the AI collaborator) made me feel that they were literally collaborating with me, especially the second one (Aware condition). The AI would mention what I was doing, like `I noticed that you were doing...', which was a positive signal to me and let me be more willing to communicate with AI.''} (P4)
\end{quote}

\noindent Another two participants perceived AI to act more like a collaborator in the Aware condition, and behave less cooperatively in the Non-aware condition because the feedback was to some extent unrelated to their own progress:

\begin{quote}
    \textit{``The first one (Non-aware condition) was not actively involved in my work, and made me regard it as not wanting to collaborate with me. I felt better working with the second one (Aware condition), which was more responsive and related.''} (P6)
\end{quote}

\begin{quote}
    \textit{``(In the Non-aware condition,) AI acted in a style that was not so cooperative, because our consideration diverged even in granularity.''} (P20)
\end{quote}

\noindent Besides, P7 stressed that communicating with AI in the Non-aware condition was burdensome, which gradually weakened the participant's willingness to communicate. This usually happened when the participant was working independently while AI suddenly provided feedback. In this situation, the abundant information awaits processing became the source of communication burden:

\begin{quote}
    \textit{``I felt being pushed forward in the first experiment (Non-aware condition). I was still thinking about the previous requirements, but AI has moved to the next one. I felt tired in this process and did not want to talk to it.''} (P7)
\end{quote}

Last, one participant claimed to be less likely to communicate with AI if the content was repetitive. Such situations often appear in the Non-aware condition, because when AI generates answers according to similar prompts without much context several times, repetitions are inevitable due to technical limitations. However, in the Aware condition where the prompt will be integrated with awareness information, the answers are more diverse:

\begin{quote}
    \textit{``A problem of the second AI (Non-aware condition) was that, if I asked similar questions more than two times, it would start repeating previous contents, even repeating what I said. This kind of information was useless at a glance, and I would stop listening to it and no longer wanted to ask AI.''} (P3)
\end{quote}


\subsubsection{Communication and Collaboration effort}
Nine participants expressed a different feeling of effort spent on communicating and collaborating with AI in the two conditions, supported by the reduced average turn duration ($M_{Aware} = 10.14$, $M_{Non-aware} = 13.87$, $p = .001$) and turn length ($M_{Aware} = 38.85$, $M_{Non-aware} = 46.20$, $p = .002$) in the Aware condition. Six of them implied that there was no need for them to manage and advance the design process in the Aware condition, because AI to some extent was doing this for them. For example, when the participants proceed from Step 2 to Step 3, or the previous discussion comes to an end, some of them might be in a daze and hesitate due to not knowing what to do next. In the Aware condition, AI would throw some questions that can inspire the participants:

\begin{quote}
    \textit{``...it would say what direction we can consider next, what can be dug deeper like `what can be the optimization mechanism for something'.''} (P3)
\end{quote}

\noindent Besides, even when the collaboration did not stagnate, the participants could still feel that AI was actively advancing and speeding up the design process. For instance, in the Aware condition where AI could proactively capture the canvas content and give relevant feedback, participants felt that AI could predict their intentions:

\begin{quote}
    \textit{``I remember I was marking and underlining some contents on the canvas in a certain period, and AI gave me timely feedback like `according to what you have marked' with a generated image. I think AI was actively advancing the design process.''} (P12) 
\end{quote}

Apart from reducing effort in managing the process, five participants pointed out a reduced thinking effort brought by AI awareness. In the Non-aware condition, although the participants and AI were in general performing the same task (e.g., finding solutions for the requirements), the participants needed to spend extra effort in understanding what AI was talking about in the design process by switching their thinking modes if AI proposed solutions that are unrelated to what the participants were dealing with. However, in the Aware condition, AI awareness addressed the above problem:

\begin{quote}
    \textit{``The first one (Non-aware condition) kept talking about its own idea, and I felt a little difficult to understand what the AI was saying. But the second one (Aware condition) would first follow my thoughts and then add its own suggestions. In this way, I can quickly grasp the meaning.''} (P7)
\end{quote}

In other cases, three participants complained about extra work introduced by AI in the Non-aware condition, which hindered the design process from advancing. Apart from the communication burden brought by the abundant information we mentioned in the previous section, P7 also took extra effort comparing different solutions:

\begin{quote}
    \textit{``I remained silent for quite a long time because I was thinking about, whether I should insist on my design or accept what the AI gave me. We seemed to diverge a lot. I was quite unsure what to do at that period because asking the questions one by one and discussing them with AI is time-consuming, so I let these questions remain unsolved and followed my own thoughts.''} (P7)
\end{quote}

\noindent Besides, another participant mentioned the need to always reflect on whether there was anything that AI did not know in the Non-aware condition. And AI in the Aware condition was perceived to be more reasonable because \textit{``At least I don't have to remember what I did not tell it''} (P10).

\subsubsection{Speech Style}
Four participants also mentioned and implied differences in their speech style between the two conditions. In general, their speeches were much simplified in the Aware condition, echoing the quantitative results of significantly decreased turn length in the Aware condition ($M_{Aware} = 38.85$, $M_{Non-aware} = 46.20$, $p = .002$). First, the participants' heightened expectations made them simplify their speech a lot:

\begin{quote}
    \textit{``Now that I know AI can see me and work with me in real time, I would regard AI as a powerful collaborator and expect that all information on my canvas could be captured and considered.''} (P12)
\end{quote}

\begin{quote}
    \textit{``If, in AI's eye, I am currently feeling confused, AI could just directly generate suggestions instead of me saying `I am  a bit confused, could you please do something for me?'''} (P15)
\end{quote}

\noindent When participants became familiar with how AI awareness functioned and what results were likely to appear in the Aware condition, they tended to provide less context when asking AI for answers like: \textit{``If AI could see the canvas, there's no need to provide context''} (P2). Besides, some participants even replaced explicit questions or instructions with more implicit sentences like acknowledgment \textit{``That makes sense!''} (P5), or phrases for expressing surprise and compliment:

\begin{quote}
    \textit{``I hope when I express my surprise or compliment, AI can continue generating the features it had just generated or carry them on.''} (P15)
\end{quote}

\section{Discussion}
\label{Discussion}
The quantitative and qualitative results both demonstrated that AI awareness can positively affect human-AI communication. In this section, we further discuss how these results occurred, what can we infer from the results, and what design implications we have gained from the study for the development of future human-AI collaborative design systems.

\subsection{Causalities and Inferences the Results}
The turn frequency, turn duration, and turn length were significantly different between the two conditions. Participants initiated more communications with AI in the Aware condition with an average shorter duration and length than in the Non-aware condition. We inferred possible reasons for this result from the interview results and provided evidence from the conversation histories and observations from the experiment recordings. Note that we combined the causalities for shorter turn duration and length because they were largely correlated.

\textbf{Causalities.} The interview results in \S\ref{interview} presented apparent causalities for increased turn frequency and decreased turn duration and length: Communication willingness directly affected the turns the participants initiated, and speech style affected the length and duration of what they said. Apart from these, communication and collaboration efforts can impact all of the three metrics. On the one hand, when the participants spent less effort in managing the design process, collating information, and comparing solutions in the Aware condition, the saved time allowed more turn exchanges between human and AI, thus exhibiting increased average turn frequency. On the other hand, we observed that when extra efforts were required to understand AI or obtain what they wanted, the most commonly used strategy was to add more information in the next speech, as presented in the following example:

\begin{quote}
    \textit{AI: ``I choose `Music Enthusiasts' as my target user.''\\
    P7: ``You mean you have chosen `Music Enthusiasts' as your target user? Then what are the user requirements you plan to choose?''\\
    AI: ``Hi-fi sound quality, customizable appearance, and easy to connect with other devices.''\\
    P7: ``Well, I chose `Outdoor Enthusiasts' as our target users, and am now focusing on the following user requirements such as long-term wear, high quality, and customizable appearance. What problems do you think we may face in meeting these requirements?''
    }
\end{quote}


\noindent In this case, the participants specified the current situation with longer sentences, and such cases heavily contributed to the increased turn duration and turn length in the Non-aware condition. Apart from the interview, we also observed another factor that might lead to lower turn frequency in the Non-aware condition: Technical limitations brought blank time in the experiments. In other words, AI took a longer time to process longer speech, and participants in these periods tended to simply wait for AI feedback for a while without advancing the task. Some participants explained their reasons for doing so in the interview: \textit{``I was not sure when AI would appear, or if any error occurred''} (P6), \textit{``I need AI feedback (to proceed) because that problem was beyond my knowledge''} (P7). As the total time of our experiments was fixed, longer blank time meant less time for communication, resulting in fewer turn frequency.

\textbf{Inferences.} The interpretations of these metrics in previous studies mainly include but are not limited to two aspects: fluency and efficiency. More turn exchanges and shorter turn lengths indicate higher fluency in the communication process~\cite{DalyJones1998some}, which can be further interpreted as more efficient collaboration~\cite{Kock2010Costly}. Decreased turn duration was thought to indicate increased efficiency by~\citet{Convertino2011Supporting}.

However, our observations suggested that the inferences for turn frequency should be treated with caution. Frequent turn exchanges did not always lead to efficient collaboration, and fewer exchanges do not equal low efficiency. P2 for example, had the same turn frequency for the two conditions, but in many cases, P2 was persuading, doubting, correcting, or debating with AI, which did not contribute a lot to advancing the design process:

\begin{quote}
    \textit{AI: ``I select student as my target user.''\\
    P2: ``Why do you choose student?'' \newline
    [...]\\
    P2: ``I think the white-collar worker is the more suitable target user, because [...] So I prefer white-collar worker, what do you think?''}\\
    \textbf{(Doubting and persuading)}
\end{quote}

\begin{quote}
    \textit{AI: ``For key requirement 3, we can consider: [...] network location service, anti-theft alarm system, remote control, and user community assistance.''\\
    P2: ``No! I don't care about anti-theft issues [...] The main problem is that sometimes the scooter can't connect to the Internet [...] So how can we solve the network problem so that it can send anti-theft signals?''}\\
    \textbf{(Correcting and debating)}
\end{quote}

\noindent On the other hand, we found some participants might be collaborating with AI efficiently when they exchange fewer turns with AI. In our experiment, the participants were thinking divergently most of the time. P1 and P5 mentioned in the interview that they sometimes directly combined AI feedback into their own content without initiating new conversation, especially in the Non-aware condition where AI did not necessarily give related feedback:

\begin{quote}
    \textit{``The second AI (Non-aware condition) just said some other thing, something quite strays [...] I think that makes sense because that's what I need to consider. AI was thinking outside the box, which in fact provided me with more choices, like a friend.''} (P5)
\end{quote}

In future human-AI collaborative design systems, defining different or adaptive AI behaviors in different design stages is worth considering for better fitting into the dynamic process, which we will discuss later in \S\ref{awarenessMechanism}.

\subsection{Design Implications}
The design implications mainly covered two aspects: maintaining and leveraging awareness, and communication strategies.

\subsubsection{Maintaining and Leveraging Awareness.}
\label{awarenessMechanism}
The results presented the benefit of AI having awareness of designers and the current situation. In future human-AI collaborative design systems, we should consider implementing mechanisms that help maintain and leverage awareness in the design process. Recent development of vision-language models like GPT-4o~\footnote{\url{https://openai.com/index/hello-gpt-4o/}} and Llama 3.2~\footnote{\url{https://ai.meta.com/blog/llama-3-2-connect-2024-vision-edge-mobile-devices/}} can hopefully be competent in realizing the awareness function about judging design activities from video and the canvas with proper training data. Still, our system design was thought to be flawed and participants mentioned several aspects we could optimize in future versions.

\textbf{Adjustable Awareness.}
% \hl{Add discussion about individual differences and refer to adaptive systems.} 
Despite that most participants preferred AI with awareness, some also stated that their thoughts were interrupted if AI did not come at the right time. P2, for example, responded to AI out of social etiquette, thus forgetting their original thoughts after the unexpected conversation. Some participants expressed their expectation for decreasing the frequency of AI feedback, because they were mostly thinking or reasoning in the silence moments, while AI wrongly recognized it as a stagnation that required help. P7 and P12 pointed out that AI sometimes proceeded too fast and that its feedback might cause design fixation. P12 proposed a possible solution for the above stigma:

\begin{quote}
    \textit{``Maybe I can control the frequency of its feedback, or it can be self-adjusted according to my state. Or even the abstraction level of its feedback can be adjusted in different stages. In a word, I hope the feedback content is adjustable, customizable would be better.''}
\end{quote}

\noindent Briefly speaking, participants' preferences for the timing to react differ, and future human-AI collaborative design systems whose AI has awareness ability, users' characteristics and habits should also be taken into consideration when comprehending design activities and contexts to adapt to their real needs (e.g.,~\cite{Joshi2024Crafting, Zheng2024SOAP}). Otherwise, AI will negatively influence the collaboration.

\textbf{Facilitating Mutual Awareness.}
A majority of designers mentioned the need to know more about AI. That means, apart from AI awareness of designers' activities and contexts, designers also expected awareness towards AI, which is in accord with awareness research in other contexts concerning collaboration (e.g.,~\cite{Endsley1988Design, Gross2013Supporting}). In general, participants called for awareness of AI in the following ways: 1) being able to anticipate when would AI initiate an action, 2) knowing the origin of AI's statements if they are based on existing information in the process, and 3) showing its state, such as available, occupied, or error. By providing the awareness information for the designers, they can better grasp the design process and reduce meaningless waiting, and the collaboration efficiency can thus be further enhanced.

\textbf{Exploiting unique features in speech.} One participant (P10) with relevant domain knowledge recommended we consider features in human speech like tone, speed, and volume as important awareness information. These features are common communicative cues and are crucial for human awareness in human-human collaboration~\cite{Gutwin2002descriptive}, as they can imply designers' emotions and attitudes. With this information, AI can judge the current situation and designer's intention more accurately and provide a better communication and collaboration experience.

\subsubsection{Communication Strategies}
Communicating with AI using speech was praised by most participants for its convenience. They were less likely to be distracted and easy to focus on the canvas because they could directly speak to and listen to AI instead of shifting their focus to the chat window, as well as processing audio and visual information in parallel. However, this communication method still requires optimization to become viable in real human-AI collaborative design practices.

\textbf{Supporting interruption mechanism.} Several participants pointed out that the speech output lacks interactive mechanisms. For instance, P1 wished to interrupt AI once he heard something useful, because what he needed next was to dig deeper into that useful information rather than keep listening to the remaining. Interruption is a common phenomenon in human-human collaboration, and also one of the process measures in collaboration research~\cite{Billinghurst2003communication} that serves as an indicator of fluent turn-taking and higher spontaneity~\cite{Bickmore2005social, OMalley1996Comparison}. 

\textbf{Timely feedback.} Although delay is inevitable due to technical limitations, we can still provide timely feedback to the participants to relieve their feeling of unsure when waiting for feedback. P6 explained why people feel less anxious when waiting for human collaborator's response:

\begin{quote}
    \textit{``...the delay actually makes me feel uneasy in this situation. I don't know whether AI will reply or not, and I don't know if there's something wrong with the question I asked. But in the real design process, my collaborator will definitely say something first, no matter what, after I ask a question, even if he is talking nonsense.''}
\end{quote}

\noindent In future design, we consider two ways that might meet the demand of ``say something first'': 1) preset replies. In future research, we can collect a set of replies to serve as timely and suitable initial feedback. In this way, AI would first retrieve instead of generate content, which is much faster, 2) Segmented reply. Similar to ChatGPT which presents the generated text word-by-word, we can segment the speech reply into short paragraphs, and convert the already generated content into speech first while keep generating the remaining in this process.

In conclusion, we recommend designers use speech as a main communication channel because it resembles designers' habits and assists in multitasking~\cite{Vermeulen2008Sensory, Robe2022PairBuddy}, but more efforts should be made to turn speech into a viable and usable communication channel.

\subsection{Limitation and Future Work}
\subsubsection{Methodological Limitation}
Our user studies were conducted in a partially functioning system using the WoZ method, indicating a natural difference with real AI-driven system. First, the Wizards face severe time pressure to react rapidly and accurately to alleviate (not dismiss) negative effects caused by delays and human errors~\cite{Hu2023Wizundry}. Several participants complained that they were always waiting too long for AI to react, which impacted their impression and trust in AI. Second, the awareness function was artificially fulfilled by Wizard A, which inevitably mixed personal experience and was bias-prone. We tried to lower the impact from personal experience by judging the design activity according to an existing coding scheme exclusively~\cite{Kim2011Relations}, but the accuracy of the judgment might fluctuate in the process and between different experiments. Future research using real AI-driven systems is needed to dismiss the flaws brought by humans. Besides, we excluded six participants at the data analysis stage, making our sample size limited. Future research should expand the sample size. Lastly, this study focused on the impact of AI awareness using speech modality, while the results of the impact may vary among different communication modalities, such as textual and embodied communication (e.g., haptic). Future research should also evaluate different communication modalities separately and comprehensively to complement human-AI communication styles.

\subsubsection{Ethical Concerns}
Using AI in the design process can bring ethical concerns, one of which is causing design fixation. Prior research has observed higher design fixation when ideating with Generative AI compared to the No Support condition due to participants' reliance on AI output~\cite{Wadinambiarachchi2024fixation}. We also observed similar phenomena in our user studies, for example, the feature \textit{Foldable} repeatedly ($N = 9$) appeared in participants' e-scooter concepts, no matter what condition they were in and what target users and requirements they selected, and this feature was proposed by AI without exception. Future research should explore strategies to reduce such fixation, like providing abstract and partially completed ideas~\cite{Cheng2014strategy}. Another concern frequently mentioned by academia is the privacy concern~\cite{Li2024Privacy, Staab2024memorization}, implied as well by one of our participants: ``\textit{(I would not feel uncomfortable) as long as AI runs locally, because it requires data like videos for recognition.}'' Our system monitored participants through cameras and captured their speech, which, in real-world usage scenarios, risks data leakage if data analysis is conducted remotely. Also, six participants explicitly conveyed to view AI as or expect AI to act as a real human, which can be risky in encouraging them to disclose sensitive data~\cite{Zhang2024fair}. If promoting human-AI collaboration in the future, special attention should be paid to protecting users' privacy and sensitive data.

\section{Conclusion}
This study presents empirical research contextualized in design to evaluate how AI awareness impacts human-AI collaboration. We developed a partially functioned human-AI collaborative design system and conducted user studies through the WoZ method. The quantitative demonstrated that AI awareness is beneficial to human-AI communication, thus enhancing the efficiency of human-AI collaboration: the analysis of process measures showed designers averagely initiated more conversations with shorter lengths and durations in the Aware condition than in the Non-aware condition, manifested as higher turn frequencies, shorter turn length, and shorter turn duration. The qualitative results exhibited the affected aspects of communication and collaboration in detail. Furthermore, we revealed the causalities behind the results, and discussed the inferences. The qualitative results also provided us with implications for future research, including how awareness can be better utilized and how communicating through speech can provide a better user experience. Last, we discussed the limitations existing in our methodologies and the ethical concerns, together with future work to address the limitations.

\begin{acks}
This work was supported by the National Key R\&D Program of China (2022YFB3303301).
\end{acks}

%%
%% The next two lines define the bibliography style to be used, and
%% the bibliography file.
\bibliographystyle{ACM-Reference-Format}
\bibliography{sample-base}
%%
%% If your work has an appendix, this is the place to put it.

\newpage
\appendix

\section{Coding Scheme Used by Wizard A}
\begin{table}[!h]
    \centering
    \caption{The adapted coding scheme~\cite{Kim2011Relations} for Wizard A to judge design activities and compose prompts.}
    \begin{tabular}{p{0.23\linewidth} p{0.15\linewidth} p{0.47\linewidth}}
        \toprule
        \textbf{Primary class} & \textbf{Subclass} & \textbf{Example actions}\\
        \midrule
        Problem Understanding & Understand        & Understanding design assignment and task\\
                               & Gather            & Collecting data about user or external information\\
                               & Clarify           & Defining design constraints and objectives\\
        \hline
        Idea Generation        & Generate          & Generating helpful idea for partial solution\\
                               & Judge             & Evaluating ideas and data\\
        \hline
        Design Elaboration     & Elaborate         & Finding technical solution, realizing function and shape\\
                               & Evaluate          & Assessing the solution\\
                               & Refine            & Improving the solution\\
        \hline
        Other                  & Stagnate          & Idling or hesitating for a while without progress\\
        \bottomrule
    \end{tabular}
    \label{tab:coding_scheme}
\end{table}

\section{The Prompt for Initiating the Generative Models at the Back-end}
\label{initiate_prompt}
The complete initiating prompt is the combination of the general instruction (part 1), task description, and the general instruction (part 2).

\subsection{The General Instruction (Part 1)}
You are an experienced product designer. You are collaborating with another product designer to design the product concept of headphones according to the specified steps. You will keep discussing with the designer during the collaboration process. \textbf{Your response needs to meet the following requirements:}

1. The length of the response should be consistent with the length of human conversation, and avoid overly structured answers;

2. The response needs to be concise, and contain only keywords. Information that the designer doesn't ask, such as the definition of an item, background information, etc., should not be provided;

3. If the designer doesn't explicitly ask a question and only describes background information, then just answer ``OK'' and propose 2 questions or suggestions that can induce the designer to provide more specific instructions;

4. During the conversation, use a friendly tone and be polite.

\textbf{The specified steps are as the following:}

Step 1: Select target users and determine requirements

Step 2: Determine key requirements

Step 3: Explore solutions

Step 4: Combine solutions into design concepts

\subsection{Task Description: E-scooter}
The design task is described as follows:\\
\textbf{Potential target users for e-scooters include:}

- Students

- White-collar workers

- Family members who do housework

- Urban tourists and ramblers\\
\textbf{Possible user needs include:}

- Portable

- Support long-term commuting

- Frequent usage

- Support complex and changing road conditions

- Easy to maintain and not prone to damage

- Environmentally friendly, supports sustainable development

- Customizable and transformable appearance

- Anti-theft and anti-loss

- Similar prices to fellow products on the market

\subsection{Task Description: Headphone}
The design task is described as follows:\\
\textbf{Potential target users for headphones include:}

- Music enthusiasts

- Office workers with long working hours

- Sports and fitness enthusiasts

- Long-time commuters and travel enthusiasts

- Youth trend groups\\
\textbf{Possible user needs include:}

- Hi-fi system, adaptable to a variety of different styles of music

- Suitable for long-time wear

- Suitable for quiet environments

- High quality, can withstand daily wear and tear

- Environmentally friendly, supports sustainable development

- Customizable and transformable appearance

- Can be easily connected to other electronic devices

- Similar prices to fellow products on the market

\subsection{The General Instruction (Part 2)}
During the design process, when you receive prompts in the following form, please think step-by-step according to the instructions and give reasonable feedback as a designer teammate based on the given information to promote the design process, and the generated content should not exceed 50 words.

\textbf{Prompts in the following form requires step-by-step reasoning and giving feedback according to the instruction:}

\textbf{Example 1:} We are currently in Step [ ], designer's design activity is [ ], and the content the designer is currently working on is presented in the image.

Your feedback is:

\textbf{Example 2:} The current design step is step [ ], designer's design activity is [ ].

Your feedback is:

\textbf{You need to think step-by-step according to the following instruction:}

1. Analyze the meaning of the designer's current status

2. Based on the meaning and your identity, put forward some suggestions or ideas

3. Suggestions and ideas need to be concise and easy to understand, and listed in points according to the tone and length of the conversation

\end{document}
\bibliography{iclr2025_conference}
\bibliographystyle{iclr2025_conference}

\subsection{Lloyd-Max Algorithm}
\label{subsec:Lloyd-Max}
For a given quantization bitwidth $B$ and an operand $\bm{X}$, the Lloyd-Max algorithm finds $2^B$ quantization levels $\{\hat{x}_i\}_{i=1}^{2^B}$ such that quantizing $\bm{X}$ by rounding each scalar in $\bm{X}$ to the nearest quantization level minimizes the quantization MSE. 

The algorithm starts with an initial guess of quantization levels and then iteratively computes quantization thresholds $\{\tau_i\}_{i=1}^{2^B-1}$ and updates quantization levels $\{\hat{x}_i\}_{i=1}^{2^B}$. Specifically, at iteration $n$, thresholds are set to the midpoints of the previous iteration's levels:
\begin{align*}
    \tau_i^{(n)}=\frac{\hat{x}_i^{(n-1)}+\hat{x}_{i+1}^{(n-1)}}2 \text{ for } i=1\ldots 2^B-1
\end{align*}
Subsequently, the quantization levels are re-computed as conditional means of the data regions defined by the new thresholds:
\begin{align*}
    \hat{x}_i^{(n)}=\mathbb{E}\left[ \bm{X} \big| \bm{X}\in [\tau_{i-1}^{(n)},\tau_i^{(n)}] \right] \text{ for } i=1\ldots 2^B
\end{align*}
where to satisfy boundary conditions we have $\tau_0=-\infty$ and $\tau_{2^B}=\infty$. The algorithm iterates the above steps until convergence.

Figure \ref{fig:lm_quant} compares the quantization levels of a $7$-bit floating point (E3M3) quantizer (left) to a $7$-bit Lloyd-Max quantizer (right) when quantizing a layer of weights from the GPT3-126M model at a per-tensor granularity. As shown, the Lloyd-Max quantizer achieves substantially lower quantization MSE. Further, Table \ref{tab:FP7_vs_LM7} shows the superior perplexity achieved by Lloyd-Max quantizers for bitwidths of $7$, $6$ and $5$. The difference between the quantizers is clear at 5 bits, where per-tensor FP quantization incurs a drastic and unacceptable increase in perplexity, while Lloyd-Max quantization incurs a much smaller increase. Nevertheless, we note that even the optimal Lloyd-Max quantizer incurs a notable ($\sim 1.5$) increase in perplexity due to the coarse granularity of quantization. 

\begin{figure}[h]
  \centering
  \includegraphics[width=0.7\linewidth]{sections/figures/LM7_FP7.pdf}
  \caption{\small Quantization levels and the corresponding quantization MSE of Floating Point (left) vs Lloyd-Max (right) Quantizers for a layer of weights in the GPT3-126M model.}
  \label{fig:lm_quant}
\end{figure}

\begin{table}[h]\scriptsize
\begin{center}
\caption{\label{tab:FP7_vs_LM7} \small Comparing perplexity (lower is better) achieved by floating point quantizers and Lloyd-Max quantizers on a GPT3-126M model for the Wikitext-103 dataset.}
\begin{tabular}{c|cc|c}
\hline
 \multirow{2}{*}{\textbf{Bitwidth}} & \multicolumn{2}{|c|}{\textbf{Floating-Point Quantizer}} & \textbf{Lloyd-Max Quantizer} \\
 & Best Format & Wikitext-103 Perplexity & Wikitext-103 Perplexity \\
\hline
7 & E3M3 & 18.32 & 18.27 \\
6 & E3M2 & 19.07 & 18.51 \\
5 & E4M0 & 43.89 & 19.71 \\
\hline
\end{tabular}
\end{center}
\end{table}

\subsection{Proof of Local Optimality of LO-BCQ}
\label{subsec:lobcq_opt_proof}
For a given block $\bm{b}_j$, the quantization MSE during LO-BCQ can be empirically evaluated as $\frac{1}{L_b}\lVert \bm{b}_j- \bm{\hat{b}}_j\rVert^2_2$ where $\bm{\hat{b}}_j$ is computed from equation (\ref{eq:clustered_quantization_definition}) as $C_{f(\bm{b}_j)}(\bm{b}_j)$. Further, for a given block cluster $\mathcal{B}_i$, we compute the quantization MSE as $\frac{1}{|\mathcal{B}_{i}|}\sum_{\bm{b} \in \mathcal{B}_{i}} \frac{1}{L_b}\lVert \bm{b}- C_i^{(n)}(\bm{b})\rVert^2_2$. Therefore, at the end of iteration $n$, we evaluate the overall quantization MSE $J^{(n)}$ for a given operand $\bm{X}$ composed of $N_c$ block clusters as:
\begin{align*}
    \label{eq:mse_iter_n}
    J^{(n)} = \frac{1}{N_c} \sum_{i=1}^{N_c} \frac{1}{|\mathcal{B}_{i}^{(n)}|}\sum_{\bm{v} \in \mathcal{B}_{i}^{(n)}} \frac{1}{L_b}\lVert \bm{b}- B_i^{(n)}(\bm{b})\rVert^2_2
\end{align*}

At the end of iteration $n$, the codebooks are updated from $\mathcal{C}^{(n-1)}$ to $\mathcal{C}^{(n)}$. However, the mapping of a given vector $\bm{b}_j$ to quantizers $\mathcal{C}^{(n)}$ remains as  $f^{(n)}(\bm{b}_j)$. At the next iteration, during the vector clustering step, $f^{(n+1)}(\bm{b}_j)$ finds new mapping of $\bm{b}_j$ to updated codebooks $\mathcal{C}^{(n)}$ such that the quantization MSE over the candidate codebooks is minimized. Therefore, we obtain the following result for $\bm{b}_j$:
\begin{align*}
\frac{1}{L_b}\lVert \bm{b}_j - C_{f^{(n+1)}(\bm{b}_j)}^{(n)}(\bm{b}_j)\rVert^2_2 \le \frac{1}{L_b}\lVert \bm{b}_j - C_{f^{(n)}(\bm{b}_j)}^{(n)}(\bm{b}_j)\rVert^2_2
\end{align*}

That is, quantizing $\bm{b}_j$ at the end of the block clustering step of iteration $n+1$ results in lower quantization MSE compared to quantizing at the end of iteration $n$. Since this is true for all $\bm{b} \in \bm{X}$, we assert the following:
\begin{equation}
\begin{split}
\label{eq:mse_ineq_1}
    \tilde{J}^{(n+1)} &= \frac{1}{N_c} \sum_{i=1}^{N_c} \frac{1}{|\mathcal{B}_{i}^{(n+1)}|}\sum_{\bm{b} \in \mathcal{B}_{i}^{(n+1)}} \frac{1}{L_b}\lVert \bm{b} - C_i^{(n)}(b)\rVert^2_2 \le J^{(n)}
\end{split}
\end{equation}
where $\tilde{J}^{(n+1)}$ is the the quantization MSE after the vector clustering step at iteration $n+1$.

Next, during the codebook update step (\ref{eq:quantizers_update}) at iteration $n+1$, the per-cluster codebooks $\mathcal{C}^{(n)}$ are updated to $\mathcal{C}^{(n+1)}$ by invoking the Lloyd-Max algorithm \citep{Lloyd}. We know that for any given value distribution, the Lloyd-Max algorithm minimizes the quantization MSE. Therefore, for a given vector cluster $\mathcal{B}_i$ we obtain the following result:

\begin{equation}
    \frac{1}{|\mathcal{B}_{i}^{(n+1)}|}\sum_{\bm{b} \in \mathcal{B}_{i}^{(n+1)}} \frac{1}{L_b}\lVert \bm{b}- C_i^{(n+1)}(\bm{b})\rVert^2_2 \le \frac{1}{|\mathcal{B}_{i}^{(n+1)}|}\sum_{\bm{b} \in \mathcal{B}_{i}^{(n+1)}} \frac{1}{L_b}\lVert \bm{b}- C_i^{(n)}(\bm{b})\rVert^2_2
\end{equation}

The above equation states that quantizing the given block cluster $\mathcal{B}_i$ after updating the associated codebook from $C_i^{(n)}$ to $C_i^{(n+1)}$ results in lower quantization MSE. Since this is true for all the block clusters, we derive the following result: 
\begin{equation}
\begin{split}
\label{eq:mse_ineq_2}
     J^{(n+1)} &= \frac{1}{N_c} \sum_{i=1}^{N_c} \frac{1}{|\mathcal{B}_{i}^{(n+1)}|}\sum_{\bm{b} \in \mathcal{B}_{i}^{(n+1)}} \frac{1}{L_b}\lVert \bm{b}- C_i^{(n+1)}(\bm{b})\rVert^2_2  \le \tilde{J}^{(n+1)}   
\end{split}
\end{equation}

Following (\ref{eq:mse_ineq_1}) and (\ref{eq:mse_ineq_2}), we find that the quantization MSE is non-increasing for each iteration, that is, $J^{(1)} \ge J^{(2)} \ge J^{(3)} \ge \ldots \ge J^{(M)}$ where $M$ is the maximum number of iterations. 
%Therefore, we can say that if the algorithm converges, then it must be that it has converged to a local minimum. 
\hfill $\blacksquare$


\begin{figure}
    \begin{center}
    \includegraphics[width=0.5\textwidth]{sections//figures/mse_vs_iter.pdf}
    \end{center}
    \caption{\small NMSE vs iterations during LO-BCQ compared to other block quantization proposals}
    \label{fig:nmse_vs_iter}
\end{figure}

Figure \ref{fig:nmse_vs_iter} shows the empirical convergence of LO-BCQ across several block lengths and number of codebooks. Also, the MSE achieved by LO-BCQ is compared to baselines such as MXFP and VSQ. As shown, LO-BCQ converges to a lower MSE than the baselines. Further, we achieve better convergence for larger number of codebooks ($N_c$) and for a smaller block length ($L_b$), both of which increase the bitwidth of BCQ (see Eq \ref{eq:bitwidth_bcq}).


\subsection{Additional Accuracy Results}
%Table \ref{tab:lobcq_config} lists the various LOBCQ configurations and their corresponding bitwidths.
\begin{table}
\setlength{\tabcolsep}{4.75pt}
\begin{center}
\caption{\label{tab:lobcq_config} Various LO-BCQ configurations and their bitwidths.}
\begin{tabular}{|c||c|c|c|c||c|c||c|} 
\hline
 & \multicolumn{4}{|c||}{$L_b=8$} & \multicolumn{2}{|c||}{$L_b=4$} & $L_b=2$ \\
 \hline
 \backslashbox{$L_A$\kern-1em}{\kern-1em$N_c$} & 2 & 4 & 8 & 16 & 2 & 4 & 2 \\
 \hline
 64 & 4.25 & 4.375 & 4.5 & 4.625 & 4.375 & 4.625 & 4.625\\
 \hline
 32 & 4.375 & 4.5 & 4.625& 4.75 & 4.5 & 4.75 & 4.75 \\
 \hline
 16 & 4.625 & 4.75& 4.875 & 5 & 4.75 & 5 & 5 \\
 \hline
\end{tabular}
\end{center}
\end{table}

%\subsection{Perplexity achieved by various LO-BCQ configurations on Wikitext-103 dataset}

\begin{table} \centering
\begin{tabular}{|c||c|c|c|c||c|c||c|} 
\hline
 $L_b \rightarrow$& \multicolumn{4}{c||}{8} & \multicolumn{2}{c||}{4} & 2\\
 \hline
 \backslashbox{$L_A$\kern-1em}{\kern-1em$N_c$} & 2 & 4 & 8 & 16 & 2 & 4 & 2  \\
 %$N_c \rightarrow$ & 2 & 4 & 8 & 16 & 2 & 4 & 2 \\
 \hline
 \hline
 \multicolumn{8}{c}{GPT3-1.3B (FP32 PPL = 9.98)} \\ 
 \hline
 \hline
 64 & 10.40 & 10.23 & 10.17 & 10.15 &  10.28 & 10.18 & 10.19 \\
 \hline
 32 & 10.25 & 10.20 & 10.15 & 10.12 &  10.23 & 10.17 & 10.17 \\
 \hline
 16 & 10.22 & 10.16 & 10.10 & 10.09 &  10.21 & 10.14 & 10.16 \\
 \hline
  \hline
 \multicolumn{8}{c}{GPT3-8B (FP32 PPL = 7.38)} \\ 
 \hline
 \hline
 64 & 7.61 & 7.52 & 7.48 &  7.47 &  7.55 &  7.49 & 7.50 \\
 \hline
 32 & 7.52 & 7.50 & 7.46 &  7.45 &  7.52 &  7.48 & 7.48  \\
 \hline
 16 & 7.51 & 7.48 & 7.44 &  7.44 &  7.51 &  7.49 & 7.47  \\
 \hline
\end{tabular}
\caption{\label{tab:ppl_gpt3_abalation} Wikitext-103 perplexity across GPT3-1.3B and 8B models.}
\end{table}

\begin{table} \centering
\begin{tabular}{|c||c|c|c|c||} 
\hline
 $L_b \rightarrow$& \multicolumn{4}{c||}{8}\\
 \hline
 \backslashbox{$L_A$\kern-1em}{\kern-1em$N_c$} & 2 & 4 & 8 & 16 \\
 %$N_c \rightarrow$ & 2 & 4 & 8 & 16 & 2 & 4 & 2 \\
 \hline
 \hline
 \multicolumn{5}{|c|}{Llama2-7B (FP32 PPL = 5.06)} \\ 
 \hline
 \hline
 64 & 5.31 & 5.26 & 5.19 & 5.18  \\
 \hline
 32 & 5.23 & 5.25 & 5.18 & 5.15  \\
 \hline
 16 & 5.23 & 5.19 & 5.16 & 5.14  \\
 \hline
 \multicolumn{5}{|c|}{Nemotron4-15B (FP32 PPL = 5.87)} \\ 
 \hline
 \hline
 64  & 6.3 & 6.20 & 6.13 & 6.08  \\
 \hline
 32  & 6.24 & 6.12 & 6.07 & 6.03  \\
 \hline
 16  & 6.12 & 6.14 & 6.04 & 6.02  \\
 \hline
 \multicolumn{5}{|c|}{Nemotron4-340B (FP32 PPL = 3.48)} \\ 
 \hline
 \hline
 64 & 3.67 & 3.62 & 3.60 & 3.59 \\
 \hline
 32 & 3.63 & 3.61 & 3.59 & 3.56 \\
 \hline
 16 & 3.61 & 3.58 & 3.57 & 3.55 \\
 \hline
\end{tabular}
\caption{\label{tab:ppl_llama7B_nemo15B} Wikitext-103 perplexity compared to FP32 baseline in Llama2-7B and Nemotron4-15B, 340B models}
\end{table}

%\subsection{Perplexity achieved by various LO-BCQ configurations on MMLU dataset}


\begin{table} \centering
\begin{tabular}{|c||c|c|c|c||c|c|c|c|} 
\hline
 $L_b \rightarrow$& \multicolumn{4}{c||}{8} & \multicolumn{4}{c||}{8}\\
 \hline
 \backslashbox{$L_A$\kern-1em}{\kern-1em$N_c$} & 2 & 4 & 8 & 16 & 2 & 4 & 8 & 16  \\
 %$N_c \rightarrow$ & 2 & 4 & 8 & 16 & 2 & 4 & 2 \\
 \hline
 \hline
 \multicolumn{5}{|c|}{Llama2-7B (FP32 Accuracy = 45.8\%)} & \multicolumn{4}{|c|}{Llama2-70B (FP32 Accuracy = 69.12\%)} \\ 
 \hline
 \hline
 64 & 43.9 & 43.4 & 43.9 & 44.9 & 68.07 & 68.27 & 68.17 & 68.75 \\
 \hline
 32 & 44.5 & 43.8 & 44.9 & 44.5 & 68.37 & 68.51 & 68.35 & 68.27  \\
 \hline
 16 & 43.9 & 42.7 & 44.9 & 45 & 68.12 & 68.77 & 68.31 & 68.59  \\
 \hline
 \hline
 \multicolumn{5}{|c|}{GPT3-22B (FP32 Accuracy = 38.75\%)} & \multicolumn{4}{|c|}{Nemotron4-15B (FP32 Accuracy = 64.3\%)} \\ 
 \hline
 \hline
 64 & 36.71 & 38.85 & 38.13 & 38.92 & 63.17 & 62.36 & 63.72 & 64.09 \\
 \hline
 32 & 37.95 & 38.69 & 39.45 & 38.34 & 64.05 & 62.30 & 63.8 & 64.33  \\
 \hline
 16 & 38.88 & 38.80 & 38.31 & 38.92 & 63.22 & 63.51 & 63.93 & 64.43  \\
 \hline
\end{tabular}
\caption{\label{tab:mmlu_abalation} Accuracy on MMLU dataset across GPT3-22B, Llama2-7B, 70B and Nemotron4-15B models.}
\end{table}


%\subsection{Perplexity achieved by various LO-BCQ configurations on LM evaluation harness}

\begin{table} \centering
\begin{tabular}{|c||c|c|c|c||c|c|c|c|} 
\hline
 $L_b \rightarrow$& \multicolumn{4}{c||}{8} & \multicolumn{4}{c||}{8}\\
 \hline
 \backslashbox{$L_A$\kern-1em}{\kern-1em$N_c$} & 2 & 4 & 8 & 16 & 2 & 4 & 8 & 16  \\
 %$N_c \rightarrow$ & 2 & 4 & 8 & 16 & 2 & 4 & 2 \\
 \hline
 \hline
 \multicolumn{5}{|c|}{Race (FP32 Accuracy = 37.51\%)} & \multicolumn{4}{|c|}{Boolq (FP32 Accuracy = 64.62\%)} \\ 
 \hline
 \hline
 64 & 36.94 & 37.13 & 36.27 & 37.13 & 63.73 & 62.26 & 63.49 & 63.36 \\
 \hline
 32 & 37.03 & 36.36 & 36.08 & 37.03 & 62.54 & 63.51 & 63.49 & 63.55  \\
 \hline
 16 & 37.03 & 37.03 & 36.46 & 37.03 & 61.1 & 63.79 & 63.58 & 63.33  \\
 \hline
 \hline
 \multicolumn{5}{|c|}{Winogrande (FP32 Accuracy = 58.01\%)} & \multicolumn{4}{|c|}{Piqa (FP32 Accuracy = 74.21\%)} \\ 
 \hline
 \hline
 64 & 58.17 & 57.22 & 57.85 & 58.33 & 73.01 & 73.07 & 73.07 & 72.80 \\
 \hline
 32 & 59.12 & 58.09 & 57.85 & 58.41 & 73.01 & 73.94 & 72.74 & 73.18  \\
 \hline
 16 & 57.93 & 58.88 & 57.93 & 58.56 & 73.94 & 72.80 & 73.01 & 73.94  \\
 \hline
\end{tabular}
\caption{\label{tab:mmlu_abalation} Accuracy on LM evaluation harness tasks on GPT3-1.3B model.}
\end{table}

\begin{table} \centering
\begin{tabular}{|c||c|c|c|c||c|c|c|c|} 
\hline
 $L_b \rightarrow$& \multicolumn{4}{c||}{8} & \multicolumn{4}{c||}{8}\\
 \hline
 \backslashbox{$L_A$\kern-1em}{\kern-1em$N_c$} & 2 & 4 & 8 & 16 & 2 & 4 & 8 & 16  \\
 %$N_c \rightarrow$ & 2 & 4 & 8 & 16 & 2 & 4 & 2 \\
 \hline
 \hline
 \multicolumn{5}{|c|}{Race (FP32 Accuracy = 41.34\%)} & \multicolumn{4}{|c|}{Boolq (FP32 Accuracy = 68.32\%)} \\ 
 \hline
 \hline
 64 & 40.48 & 40.10 & 39.43 & 39.90 & 69.20 & 68.41 & 69.45 & 68.56 \\
 \hline
 32 & 39.52 & 39.52 & 40.77 & 39.62 & 68.32 & 67.43 & 68.17 & 69.30  \\
 \hline
 16 & 39.81 & 39.71 & 39.90 & 40.38 & 68.10 & 66.33 & 69.51 & 69.42  \\
 \hline
 \hline
 \multicolumn{5}{|c|}{Winogrande (FP32 Accuracy = 67.88\%)} & \multicolumn{4}{|c|}{Piqa (FP32 Accuracy = 78.78\%)} \\ 
 \hline
 \hline
 64 & 66.85 & 66.61 & 67.72 & 67.88 & 77.31 & 77.42 & 77.75 & 77.64 \\
 \hline
 32 & 67.25 & 67.72 & 67.72 & 67.00 & 77.31 & 77.04 & 77.80 & 77.37  \\
 \hline
 16 & 68.11 & 68.90 & 67.88 & 67.48 & 77.37 & 78.13 & 78.13 & 77.69  \\
 \hline
\end{tabular}
\caption{\label{tab:mmlu_abalation} Accuracy on LM evaluation harness tasks on GPT3-8B model.}
\end{table}

\begin{table} \centering
\begin{tabular}{|c||c|c|c|c||c|c|c|c|} 
\hline
 $L_b \rightarrow$& \multicolumn{4}{c||}{8} & \multicolumn{4}{c||}{8}\\
 \hline
 \backslashbox{$L_A$\kern-1em}{\kern-1em$N_c$} & 2 & 4 & 8 & 16 & 2 & 4 & 8 & 16  \\
 %$N_c \rightarrow$ & 2 & 4 & 8 & 16 & 2 & 4 & 2 \\
 \hline
 \hline
 \multicolumn{5}{|c|}{Race (FP32 Accuracy = 40.67\%)} & \multicolumn{4}{|c|}{Boolq (FP32 Accuracy = 76.54\%)} \\ 
 \hline
 \hline
 64 & 40.48 & 40.10 & 39.43 & 39.90 & 75.41 & 75.11 & 77.09 & 75.66 \\
 \hline
 32 & 39.52 & 39.52 & 40.77 & 39.62 & 76.02 & 76.02 & 75.96 & 75.35  \\
 \hline
 16 & 39.81 & 39.71 & 39.90 & 40.38 & 75.05 & 73.82 & 75.72 & 76.09  \\
 \hline
 \hline
 \multicolumn{5}{|c|}{Winogrande (FP32 Accuracy = 70.64\%)} & \multicolumn{4}{|c|}{Piqa (FP32 Accuracy = 79.16\%)} \\ 
 \hline
 \hline
 64 & 69.14 & 70.17 & 70.17 & 70.56 & 78.24 & 79.00 & 78.62 & 78.73 \\
 \hline
 32 & 70.96 & 69.69 & 71.27 & 69.30 & 78.56 & 79.49 & 79.16 & 78.89  \\
 \hline
 16 & 71.03 & 69.53 & 69.69 & 70.40 & 78.13 & 79.16 & 79.00 & 79.00  \\
 \hline
\end{tabular}
\caption{\label{tab:mmlu_abalation} Accuracy on LM evaluation harness tasks on GPT3-22B model.}
\end{table}

\begin{table} \centering
\begin{tabular}{|c||c|c|c|c||c|c|c|c|} 
\hline
 $L_b \rightarrow$& \multicolumn{4}{c||}{8} & \multicolumn{4}{c||}{8}\\
 \hline
 \backslashbox{$L_A$\kern-1em}{\kern-1em$N_c$} & 2 & 4 & 8 & 16 & 2 & 4 & 8 & 16  \\
 %$N_c \rightarrow$ & 2 & 4 & 8 & 16 & 2 & 4 & 2 \\
 \hline
 \hline
 \multicolumn{5}{|c|}{Race (FP32 Accuracy = 44.4\%)} & \multicolumn{4}{|c|}{Boolq (FP32 Accuracy = 79.29\%)} \\ 
 \hline
 \hline
 64 & 42.49 & 42.51 & 42.58 & 43.45 & 77.58 & 77.37 & 77.43 & 78.1 \\
 \hline
 32 & 43.35 & 42.49 & 43.64 & 43.73 & 77.86 & 75.32 & 77.28 & 77.86  \\
 \hline
 16 & 44.21 & 44.21 & 43.64 & 42.97 & 78.65 & 77 & 76.94 & 77.98  \\
 \hline
 \hline
 \multicolumn{5}{|c|}{Winogrande (FP32 Accuracy = 69.38\%)} & \multicolumn{4}{|c|}{Piqa (FP32 Accuracy = 78.07\%)} \\ 
 \hline
 \hline
 64 & 68.9 & 68.43 & 69.77 & 68.19 & 77.09 & 76.82 & 77.09 & 77.86 \\
 \hline
 32 & 69.38 & 68.51 & 68.82 & 68.90 & 78.07 & 76.71 & 78.07 & 77.86  \\
 \hline
 16 & 69.53 & 67.09 & 69.38 & 68.90 & 77.37 & 77.8 & 77.91 & 77.69  \\
 \hline
\end{tabular}
\caption{\label{tab:mmlu_abalation} Accuracy on LM evaluation harness tasks on Llama2-7B model.}
\end{table}

\begin{table} \centering
\begin{tabular}{|c||c|c|c|c||c|c|c|c|} 
\hline
 $L_b \rightarrow$& \multicolumn{4}{c||}{8} & \multicolumn{4}{c||}{8}\\
 \hline
 \backslashbox{$L_A$\kern-1em}{\kern-1em$N_c$} & 2 & 4 & 8 & 16 & 2 & 4 & 8 & 16  \\
 %$N_c \rightarrow$ & 2 & 4 & 8 & 16 & 2 & 4 & 2 \\
 \hline
 \hline
 \multicolumn{5}{|c|}{Race (FP32 Accuracy = 48.8\%)} & \multicolumn{4}{|c|}{Boolq (FP32 Accuracy = 85.23\%)} \\ 
 \hline
 \hline
 64 & 49.00 & 49.00 & 49.28 & 48.71 & 82.82 & 84.28 & 84.03 & 84.25 \\
 \hline
 32 & 49.57 & 48.52 & 48.33 & 49.28 & 83.85 & 84.46 & 84.31 & 84.93  \\
 \hline
 16 & 49.85 & 49.09 & 49.28 & 48.99 & 85.11 & 84.46 & 84.61 & 83.94  \\
 \hline
 \hline
 \multicolumn{5}{|c|}{Winogrande (FP32 Accuracy = 79.95\%)} & \multicolumn{4}{|c|}{Piqa (FP32 Accuracy = 81.56\%)} \\ 
 \hline
 \hline
 64 & 78.77 & 78.45 & 78.37 & 79.16 & 81.45 & 80.69 & 81.45 & 81.5 \\
 \hline
 32 & 78.45 & 79.01 & 78.69 & 80.66 & 81.56 & 80.58 & 81.18 & 81.34  \\
 \hline
 16 & 79.95 & 79.56 & 79.79 & 79.72 & 81.28 & 81.66 & 81.28 & 80.96  \\
 \hline
\end{tabular}
\caption{\label{tab:mmlu_abalation} Accuracy on LM evaluation harness tasks on Llama2-70B model.}
\end{table}

%\section{MSE Studies}
%\textcolor{red}{TODO}


\subsection{Number Formats and Quantization Method}
\label{subsec:numFormats_quantMethod}
\subsubsection{Integer Format}
An $n$-bit signed integer (INT) is typically represented with a 2s-complement format \citep{yao2022zeroquant,xiao2023smoothquant,dai2021vsq}, where the most significant bit denotes the sign.

\subsubsection{Floating Point Format}
An $n$-bit signed floating point (FP) number $x$ comprises of a 1-bit sign ($x_{\mathrm{sign}}$), $B_m$-bit mantissa ($x_{\mathrm{mant}}$) and $B_e$-bit exponent ($x_{\mathrm{exp}}$) such that $B_m+B_e=n-1$. The associated constant exponent bias ($E_{\mathrm{bias}}$) is computed as $(2^{{B_e}-1}-1)$. We denote this format as $E_{B_e}M_{B_m}$.  

\subsubsection{Quantization Scheme}
\label{subsec:quant_method}
A quantization scheme dictates how a given unquantized tensor is converted to its quantized representation. We consider FP formats for the purpose of illustration. Given an unquantized tensor $\bm{X}$ and an FP format $E_{B_e}M_{B_m}$, we first, we compute the quantization scale factor $s_X$ that maps the maximum absolute value of $\bm{X}$ to the maximum quantization level of the $E_{B_e}M_{B_m}$ format as follows:
\begin{align}
\label{eq:sf}
    s_X = \frac{\mathrm{max}(|\bm{X}|)}{\mathrm{max}(E_{B_e}M_{B_m})}
\end{align}
In the above equation, $|\cdot|$ denotes the absolute value function.

Next, we scale $\bm{X}$ by $s_X$ and quantize it to $\hat{\bm{X}}$ by rounding it to the nearest quantization level of $E_{B_e}M_{B_m}$ as:

\begin{align}
\label{eq:tensor_quant}
    \hat{\bm{X}} = \text{round-to-nearest}\left(\frac{\bm{X}}{s_X}, E_{B_e}M_{B_m}\right)
\end{align}

We perform dynamic max-scaled quantization \citep{wu2020integer}, where the scale factor $s$ for activations is dynamically computed during runtime.

\subsection{Vector Scaled Quantization}
\begin{wrapfigure}{r}{0.35\linewidth}
  \centering
  \includegraphics[width=\linewidth]{sections/figures/vsquant.jpg}
  \caption{\small Vectorwise decomposition for per-vector scaled quantization (VSQ \citep{dai2021vsq}).}
  \label{fig:vsquant}
\end{wrapfigure}
During VSQ \citep{dai2021vsq}, the operand tensors are decomposed into 1D vectors in a hardware friendly manner as shown in Figure \ref{fig:vsquant}. Since the decomposed tensors are used as operands in matrix multiplications during inference, it is beneficial to perform this decomposition along the reduction dimension of the multiplication. The vectorwise quantization is performed similar to tensorwise quantization described in Equations \ref{eq:sf} and \ref{eq:tensor_quant}, where a scale factor $s_v$ is required for each vector $\bm{v}$ that maps the maximum absolute value of that vector to the maximum quantization level. While smaller vector lengths can lead to larger accuracy gains, the associated memory and computational overheads due to the per-vector scale factors increases. To alleviate these overheads, VSQ \citep{dai2021vsq} proposed a second level quantization of the per-vector scale factors to unsigned integers, while MX \citep{rouhani2023shared} quantizes them to integer powers of 2 (denoted as $2^{INT}$).

\subsubsection{MX Format}
The MX format proposed in \citep{rouhani2023microscaling} introduces the concept of sub-block shifting. For every two scalar elements of $b$-bits each, there is a shared exponent bit. The value of this exponent bit is determined through an empirical analysis that targets minimizing quantization MSE. We note that the FP format $E_{1}M_{b}$ is strictly better than MX from an accuracy perspective since it allocates a dedicated exponent bit to each scalar as opposed to sharing it across two scalars. Therefore, we conservatively bound the accuracy of a $b+2$-bit signed MX format with that of a $E_{1}M_{b}$ format in our comparisons. For instance, we use E1M2 format as a proxy for MX4.

\begin{figure}
    \centering
    \includegraphics[width=1\linewidth]{sections//figures/BlockFormats.pdf}
    \caption{\small Comparing LO-BCQ to MX format.}
    \label{fig:block_formats}
\end{figure}

Figure \ref{fig:block_formats} compares our $4$-bit LO-BCQ block format to MX \citep{rouhani2023microscaling}. As shown, both LO-BCQ and MX decompose a given operand tensor into block arrays and each block array into blocks. Similar to MX, we find that per-block quantization ($L_b < L_A$) leads to better accuracy due to increased flexibility. While MX achieves this through per-block $1$-bit micro-scales, we associate a dedicated codebook to each block through a per-block codebook selector. Further, MX quantizes the per-block array scale-factor to E8M0 format without per-tensor scaling. In contrast during LO-BCQ, we find that per-tensor scaling combined with quantization of per-block array scale-factor to E4M3 format results in superior inference accuracy across models. 




\end{document}
