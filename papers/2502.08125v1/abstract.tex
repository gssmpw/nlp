\begin{abstract}
The algorithms-with-predictions framework has been used extensively to develop online algorithms with improved beyond-worst-case competitive ratios.  Recently, there is growing interest in leveraging predictions for designing data structures with improved beyond-worst-case running times.
In this paper, we study the fundamental data structure problem of maintaining approximate shortest paths in incremental graphs in the algorithms-with-predictions model.
Given a sequence $\sigma$ of edges that are inserted one at a time,  the goal is to maintain approximate shortest paths from the source to each vertex in the graph at each time step.   
Before any edges arrive, the data structure is given a prediction of the online edge sequence $\hat{\sigma}$ which is used to ``warm start'' its state.   

As our main result, we design a learned algorithm that maintains $(1+\epsilon)$-approximate single-source shortest paths, which runs in $\tilde{O}(m \eta \log W/\epsilon)$ time, where $W$ is the weight of the heaviest edge and $\eta$ is the prediction error.  We show these techniques immediately extend to the all-pairs shortest-path setting as well.
Our algorithms are consistent (performing nearly as fast as the offline algorithm) when predictions are nearly perfect, have a smooth degradation in performance with respect to the prediction error and, in the worst case, match the best offline algorithm up to logarithmic factors. That is, the algorithms are ``ideal'' in the algorithms-with-predictions model.

As a building block, we study the \emph{offline incremental} approximate single-source shortest-path (SSSP) problem.   In the offline incremental SSSP problem, the edge sequence $\sigma$ is known a priori and the goal is to construct a data structure that can efficiently return the length of the shortest paths in the intermediate graph $G_t$ consisting of the first $t$ edges, for all $t$.
Note that the offline incremental problem is defined in the worst-case setting (without predictions) and is of independent interest. 
\end{abstract}