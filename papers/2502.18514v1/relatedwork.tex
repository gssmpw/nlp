\section{Related Work and Enhancements}
The research "Ciphertext Face Recognition System Based on Secure Inner Product Protocol" \cite{related2} combines Pallier homomorphic encryption with an inner product protocol to protect user privacy during face recognition. Their system achieves 98.78\% accuracy for both plaintext and ciphertext face recognition, demonstrating that encryption does not compromise performance. While this work provides a robust solution for privacy-preserving face recognition, we extend their approach by using fully homomorphic encryption (FHE), which provides enhanced flexibility and computational power compared to Pallier encryption. In our study, we conduct experiments using 128d, 512d, and 4096d embeddings, offering a more detailed analysis of how embedding dimensions impact recognition accuracy and efficiency. Unlike their approach, which primarily focuses on a fixed encryption protocol, we provide a complete Python-based framework that simplifies the integration of secure face recognition into real-world applications. This makes our system not only more versatile but also easier to implement and adopt by users, thus making privacy-preserving face recognition more accessible to the broader research and developer community.

The paper "Secure Face Matching Using Fully Homomorphic Encryption" \cite{related1} focuses on utilizing fully homomorphic encryption (FHE) for secure face recognition, allowing template matching to be performed directly on encrypted data. This work demonstrates the feasibility of secure face matching by reducing face templates to a 16 KB size, achieving a matching time of 0.01 seconds per face. They primarily use cosine distance for matching and conduct experiments with 128d and 512d embeddings. In contrast, our study not only adopts FHE for secure face recognition but extends it by testing with embeddings of 128d, 512d, and 4096d, which offers a more comprehensive evaluation of how different embedding dimensions affect performance. We also employ Euclidean distance for face matching, which provides a competitive alternative distance metric compared to their use of cosine distance. Most notably, our work offers a Python-based framework, which allows users to easily implement secure face recognition. This user-friendly framework is designed to facilitate the adoption of privacy-preserving face recognition systems, making our approach more accessible and practical for a broader audience.