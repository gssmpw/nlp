\documentclass[lettersize,journal]{IEEEtran}
\usepackage{amsmath,amsfonts}
\usepackage{algorithmic}
\usepackage{algorithm}
\usepackage{array}
\usepackage[caption=false,font=normalsize,labelfont=sf,textfont=sf]{subfig}
\usepackage{textcomp}
\usepackage{stfloats}
\usepackage{url}
\usepackage{verbatim}
\usepackage{graphicx}
\usepackage{cite}
\hyphenation{op-tical net-works semi-conduc-tor IEEE-Xplore}
% updated with editorial comments 8/9/2021

% Custom added
\usepackage{multirow}
\usepackage{url}
\usepackage{hyperref}

\begin{document}

\title{LISArD: Learning Image Similarity to Defend Against Gray-box Adversarial Attacks}

\author{Joana C. Costa,
        Tiago Roxo,
        Hugo Proença,~\IEEEmembership{Senior Member,~IEEE,}
        Pedro R. M. Inácio,~\IEEEmembership{Senior Member,~IEEE}
        % <-this % stops a space
\thanks{Manuscript received February XX, 2025; revised XX XX, 2025. This work was supported in part by the Portuguese Fundação para a Ciência e Tecnologia (FCT)/Ministério da Ciência, Tecnologia e Ensino Superior (MCTES) through National Funds and co-funded by EU funds under Project UIDB/50008/2020; in part by the FCT Doctoral Grant 2020.09847.BD and Grant 2021.04905.BD.}% <-this % stops a space
\thanks{Joana C. Costa, Tiago Roxo, Hugo Proença and Pedro R. M. Inácio are with the Instituto de Telecomunicações, sins-lab, and Department of Computer Science, Universidade da Beira Interior, Portugal (corresponding author e-mail: \href{mailto:joana.cabral.costa@ubi.pt}{joana.cabral.costa@ubi.pt}).}}

% The paper headers
\markboth{Submitted to IEEE Transactions on Information Forensics and Security, Vol. XX, XXXX}%
{Costa \MakeLowercase{\textit{et al.}}: LISArD: Learning Image Similarity to Defend Against Gray-box Adversarial Attacks}

\IEEEpubid{0000--0000/00\$00.00~\copyright~2021 IEEE}
% Remember, if you use this you must call \IEEEpubidadjcol in the second
% column for its text to clear the IEEEpubid mark.

\maketitle

\begin{abstract}  
Test time scaling is currently one of the most active research areas that shows promise after training time scaling has reached its limits.
Deep-thinking (DT) models are a class of recurrent models that can perform easy-to-hard generalization by assigning more compute to harder test samples.
However, due to their inability to determine the complexity of a test sample, DT models have to use a large amount of computation for both easy and hard test samples.
Excessive test time computation is wasteful and can cause the ``overthinking'' problem where more test time computation leads to worse results.
In this paper, we introduce a test time training method for determining the optimal amount of computation needed for each sample during test time.
We also propose Conv-LiGRU, a novel recurrent architecture for efficient and robust visual reasoning. 
Extensive experiments demonstrate that Conv-LiGRU is more stable than DT, effectively mitigates the ``overthinking'' phenomenon, and achieves superior accuracy.
\end{abstract}  

\section{Introduction}


\begin{figure}[t]
\centering
\includegraphics[width=0.6\columnwidth]{figures/evaluation_desiderata_V5.pdf}
\vspace{-0.5cm}
\caption{\systemName is a platform for conducting realistic evaluations of code LLMs, collecting human preferences of coding models with real users, real tasks, and in realistic environments, aimed at addressing the limitations of existing evaluations.
}
\label{fig:motivation}
\end{figure}

\begin{figure*}[t]
\centering
\includegraphics[width=\textwidth]{figures/system_design_v2.png}
\caption{We introduce \systemName, a VSCode extension to collect human preferences of code directly in a developer's IDE. \systemName enables developers to use code completions from various models. The system comprises a) the interface in the user's IDE which presents paired completions to users (left), b) a sampling strategy that picks model pairs to reduce latency (right, top), and c) a prompting scheme that allows diverse LLMs to perform code completions with high fidelity.
Users can select between the top completion (green box) using \texttt{tab} or the bottom completion (blue box) using \texttt{shift+tab}.}
\label{fig:overview}
\end{figure*}

As model capabilities improve, large language models (LLMs) are increasingly integrated into user environments and workflows.
For example, software developers code with AI in integrated developer environments (IDEs)~\citep{peng2023impact}, doctors rely on notes generated through ambient listening~\citep{oberst2024science}, and lawyers consider case evidence identified by electronic discovery systems~\citep{yang2024beyond}.
Increasing deployment of models in productivity tools demands evaluation that more closely reflects real-world circumstances~\citep{hutchinson2022evaluation, saxon2024benchmarks, kapoor2024ai}.
While newer benchmarks and live platforms incorporate human feedback to capture real-world usage, they almost exclusively focus on evaluating LLMs in chat conversations~\citep{zheng2023judging,dubois2023alpacafarm,chiang2024chatbot, kirk2024the}.
Model evaluation must move beyond chat-based interactions and into specialized user environments.



 

In this work, we focus on evaluating LLM-based coding assistants. 
Despite the popularity of these tools---millions of developers use Github Copilot~\citep{Copilot}---existing
evaluations of the coding capabilities of new models exhibit multiple limitations (Figure~\ref{fig:motivation}, bottom).
Traditional ML benchmarks evaluate LLM capabilities by measuring how well a model can complete static, interview-style coding tasks~\citep{chen2021evaluating,austin2021program,jain2024livecodebench, white2024livebench} and lack \emph{real users}. 
User studies recruit real users to evaluate the effectiveness of LLMs as coding assistants, but are often limited to simple programming tasks as opposed to \emph{real tasks}~\citep{vaithilingam2022expectation,ross2023programmer, mozannar2024realhumaneval}.
Recent efforts to collect human feedback such as Chatbot Arena~\citep{chiang2024chatbot} are still removed from a \emph{realistic environment}, resulting in users and data that deviate from typical software development processes.
We introduce \systemName to address these limitations (Figure~\ref{fig:motivation}, top), and we describe our three main contributions below.


\textbf{We deploy \systemName in-the-wild to collect human preferences on code.} 
\systemName is a Visual Studio Code extension, collecting preferences directly in a developer's IDE within their actual workflow (Figure~\ref{fig:overview}).
\systemName provides developers with code completions, akin to the type of support provided by Github Copilot~\citep{Copilot}. 
Over the past 3 months, \systemName has served over~\completions suggestions from 10 state-of-the-art LLMs, 
gathering \sampleCount~votes from \userCount~users.
To collect user preferences,
\systemName presents a novel interface that shows users paired code completions from two different LLMs, which are determined based on a sampling strategy that aims to 
mitigate latency while preserving coverage across model comparisons.
Additionally, we devise a prompting scheme that allows a diverse set of models to perform code completions with high fidelity.
See Section~\ref{sec:system} and Section~\ref{sec:deployment} for details about system design and deployment respectively.



\textbf{We construct a leaderboard of user preferences and find notable differences from existing static benchmarks and human preference leaderboards.}
In general, we observe that smaller models seem to overperform in static benchmarks compared to our leaderboard, while performance among larger models is mixed (Section~\ref{sec:leaderboard_calculation}).
We attribute these differences to the fact that \systemName is exposed to users and tasks that differ drastically from code evaluations in the past. 
Our data spans 103 programming languages and 24 natural languages as well as a variety of real-world applications and code structures, while static benchmarks tend to focus on a specific programming and natural language and task (e.g. coding competition problems).
Additionally, while all of \systemName interactions contain code contexts and the majority involve infilling tasks, a much smaller fraction of Chatbot Arena's coding tasks contain code context, with infilling tasks appearing even more rarely. 
We analyze our data in depth in Section~\ref{subsec:comparison}.



\textbf{We derive new insights into user preferences of code by analyzing \systemName's diverse and distinct data distribution.}
We compare user preferences across different stratifications of input data (e.g., common versus rare languages) and observe which affect observed preferences most (Section~\ref{sec:analysis}).
For example, while user preferences stay relatively consistent across various programming languages, they differ drastically between different task categories (e.g. frontend/backend versus algorithm design).
We also observe variations in user preference due to different features related to code structure 
(e.g., context length and completion patterns).
We open-source \systemName and release a curated subset of code contexts.
Altogether, our results highlight the necessity of model evaluation in realistic and domain-specific settings.







\section{Related Work}
\label{sec:related-work}

\noindent\textbf{White-box Adversarial Attacks}.
L-BFGS~\cite{szegedy2014intriguing} was the first proposed adversarial attack that demonstrated how simple perturbations could affect the DNNs performance.
Fast Gradient Sign Method (FGSM)~\cite{goodfellow2015explaining} is a one-step method that uses the model cost function, the gradient, and the radius epsilon to search for perturbations.
Jacobian-based Saliency Maps (JSM)~\cite{papernot2016limitations} explore the forward derivatives and construct the adversarial saliency maps.
Gradient Aligned Adversarial Subspace (GAAS)~\cite{tramer2017space} estimates the dimensionality of the adversarial subspace using the first-order approximation of the loss function.
Sparse and Imperceivable Adversarial Attacks (SIAA)~\cite{croce2019sparse} create sporadic and imperceptible perturbations by applying the standard deviation of each color channel in both axis directions.
DeepFool~\cite{moosavi2016deepfool} is an iterative attack that stops when the minimal vector orthogonal to the hyperplane representing the decision boundary is found.
SmoothFool (SF)~\cite{dabouei2020smoothfool} is an iterative algorithm that uses DeepFool to calculate the initial perturbation and smoothly rectifies the resulting perturbation until the adversarial example fools the classifier.
Projected Gradient Descent (PGD)~\cite{madry2018towards} is an iterative attack that uses saddle point formulation to find a strong perturbation.
Momentum Iterative FGSM (MI-FGSM)~\cite{dong2018boosting} introduces momentum into the Iterative FGSM (I-FGSM).
Auto-Attack~\cite{croce2020reliable} is a set of attacks to evaluate the networks, proposing the APGD-CE (i.e., PGD using Cross-Entropy (CE)), and APGD-DLR (i.e., PGD using Difference of Logits Ratio (DLR)) attacks. These techniques are are combined with Fast Adaptive Boundary (FAB)~\cite{croce2020minimally}, used to minimize the norm of the adversarial perturbations, and the Square Attack~\cite{andriushchenko2020square}, a query-efficient black-box attack. LISArD proposes using white-box attacks against models with the same architecture and data as the target, but without assuming the attacker can access this target model, making our approach more suitable to deal with realistic scenarios.



\textbf{Adversarial Distillation}.
Defensive Distillation (DD)~\cite{papernot2016distillation}, and its extension~\cite{papernot2017extending}, were the first methods to demonstrate the usefulness of distillation to defend against adversarial examples.
Robust Self-Training (RST)~\cite{carmon2019unlabeled} uses a standard supervised approach to obtain pseudo-labels and feed them into another network that targets adversarial robustness.
Adversarially Robust Distillation (ARD)~\cite{goldblum2020adversarially} performs distillation using an adversarially trained network as the teacher.
Introspective Adversarial Distillation (IAD)~\cite{zhu2021reliable} evaluates the robustness of the teacher network considering both the student and teacher labels.
Robust Soft Label Adversarial Distillation (RSLAD)~\cite{zi2021revisiting} uses robust soft labels produced by a teacher network to supervise the student training on natural and adversarial examples.
Low Temperature Distillation (LTD)~\cite{chen2021ltd} considers low temperature in the teacher network and generates soft labels that can be integrated into existing works.
Robustness Critical Fine-Tuning (RiFT)~\cite{zhu2023improving} introduces the module robust criticality metric to fine-tune the less robust modules to adversarial perturbations.
Adaptive Adversarial Distillation (AdaAD)~\cite{huang2023boosting} involves the teacher model in the optimization process by interacting
with the student model to search for the inner
results adaptively.
Information Bottleneck Distillation (IBD)~\cite{kuang2024improving} uses soft-label distillation to increase the mutual information between latent features and predictions and transfers relevant knowledge from the teacher
to the student to reduce the mutual information between the input and latent features.
Fair Adversarial Robustness Distillation (FairARD)~\cite{yue2024revisiting} ensures robust fairness of the student by increasing the weights for naturally more difficult classes.
PeerAiD~\cite{jung2024peeraid} trains a peer network on
the adversarial examples generated for the student network, simultaneously training the student and peer network.
Dynamic Guidance Adversarial Distillation (DGAD)~\cite{park2025dynamic} corrects teacher and student misclassification on clean and adversarially perturbed images. LISArD also does not include additional models during the inference phase, without involving Adversarial Training (AT) and larger previously trained models, thus being a more reliable approach for various domains.

\begin{figure*}[!t]
    \centering
    \includegraphics[width=0.9\linewidth]{imgs/approaches_v2.png}
    \caption{Types of approaches commonly used to defend against adversarial attacks. The Teacher Model refers to a previously trained model, usually bigger than the Student Model, that aids the latter by providing soft labels. The DDPM refers to a Denoising Diffusion Probabilistic Model (a generative model) that uses noise and denoise to produce a ``purified'' image.}
    \label{fig:approaches}
\end{figure*}



\textbf{Adversarial Purification}.
Yoon \textit{et al.}~\cite{yoon2021adversarial} propose using an Energy-Based Model with Denoising Score-Matching to purify perturbed images quickly.
For the first time, diffPure~\cite {nie2022diffusion} uses DDPM to remove the adversarial perturbations from the input images.
Guided Diffusion Model for Adversarial Purification (GDMAP)~\cite{wu2022guided} gradually denoises pure Gaussian noise with guidance to an adversarial image.
APuDAE~\cite{kalaria2022towards} uses Denoising AutoEncoders~\cite{vincent2008extracting} to purify the adversarial examples in an adaptive way, improving the accuracy of target networks.
DensePure~\cite{chen2022densepure} uses different random seeds to get multiple purified images, which are fed to the classifier, and its final prediction is based on majority voting.
Wang \textit{et al.}~\cite{wang2023better} uses better diffusion models~\cite{karras2022elucidating} to demonstrate that higher efficiency and quality diffusion models translate into better robust accuracy.
Lee \textit{et al.}~\cite{lee2023robust} propose a gradual noise-scheduling strategy that improves the robustness of diffusion-based purification.
Feature Purification Network (FePN)~\cite{cao2023fepn} is an adversarial learning mechanism that learns robust features by removing non-robust features from inputs while reconstructing high-quality clean images.
DifFilter~\cite{chen2024diffilter} uses a score-based method to improve the data distribution of the clean samples.
DiffAP~\cite{zhang2024random} uses conditional guidance to ensure prediction consistency between the purified and clean images. 
MimicDiffusion~\cite{song2024mimicdiffusion} approximates the purification process of adversarial examples and clean images by using Manhattan distance and two guidances. Adversarial Purification is the most efficient defense approach for DNNs, but it comes at the cost of high computational resources, while LISArD is able to protect different architectures in various setups without requiring additional training overhead.



\section{LISArD Methodology}
\label{sec:proposed_model}

\subsection{Adversarial Context and Preliminary}

\noindent\textbf{White-box Issues}. The white-box attacks are the strongest attacks for a specific model, yet if its training method slightly diverges, the same perturbations no longer have the identical effect as the model that was used to generate the adversarial samples. Furthermore, the white-box scenario requires that the attacker has access to the implementation/code of the model, which might not be realistic in most cases, since the attacker will rarely have access to the code of deployed models.


\textbf{Black-box Problems}. The black-box attacks are mainly focused on generating perturbations based on a low amount of knowledge, reducing the effect of the adversarial samples when compared to the white-box. However, the former attacks are more viable since the attacker only needs to know pairs of images and answers given by the target model to generate the perturbations. The black-box scenario can be considered as the most generic nowadays, since it does not require any information about the model, being potentially applicable to any available system exposing an DNN. Nevertheless, in this scenario, the attacker does not benefit from additional details of the target model, which hinders the probability of success.


\textbf{Proposed Solution}. We propose an alternative scenario in which the attacker knows the architecture of the model and dataset used to train it but does not have access to the gradients of the model. This information is usually accessible in papers or descriptive pages of the model, which can help the attacker to achieve stronger perturbations. This scenario is more realistic than white-box by compromising the amount of knowledge needed and the influence of the perturbations on the target model. LISArD considers using white-box attacks to generate adversarial samples against a model that uses the same architecture and dataset as the target model.


\textbf{Types of Approaches}. The two approaches described in the specialized literature and the approach proposed herein are summarized in Figure~\ref{fig:approaches}, which highlights the increased resources for performing \textit{Adversarial Distillation} and \textit{Purification} compared to LISArD. \textit{Adversarial Distillation} requires the usage of an additional previously trained model (Teacher) to aid in teaching the resilient network (Student), and \textit{Adversarial Purification} involves the training of a generative model (DDPM) to remove the adversarial noise during the inference phase (Purification). LISArD considers its attack scenario as a gray-box, meaning that the attacker only has partial knowledge about the target model. Thus, training to perceive noisy images (created by adding random Gaussian noise) similar to clean images can aid in defending against this type of attack.

\begin{figure}[!t]
    \centering
    \includegraphics[width=0.9\linewidth]{imgs/learning_image_similarity.png}
    \caption{Overview of the conversion from embeddings to a matrix in the Learning Image Similarity component. $E$ refers to the size of the embeddings, which vary depending on the selected model.}
    \label{fig:learning_image_similarity}
\end{figure}



\subsection{Image Similarity and Importance}

\noindent\textbf{Motivation}. Learning Image Similarity (LIS) is based on the idea that an image containing a reduced amount of noise does not affect the object represented in that image. Barlow Twins~\cite{zbontar2021barlow} proposes a procedure to reduce the redundancy between a pair of identical networks in the context of self-supervised learning. LISArD utilizes the redundancy reduction approach to teach the model to identify the noisy and clean images as similar and improve robustness against gray-box and white-box attacks. 

\textbf{Embeddings to Matrix Conversion}. An overview of the LIS component, explaining the conversion process from embeddings to a matrix, which is used to achieve redundancy reduction between images is provided in Figure~\ref{fig:learning_image_similarity}. The embeddings with size $E$ are extracted before being fed to the classification layer, and each clean embedding is multiplied by each noisy embedding to obtain the cross-correlation matrix. Then, this cross-correlation matrix is approximated to the diagonal matrix to achieve a perfect correlation.

\textbf{Weighted Training}. LISArD focuses on two main approaches: learning that two images are similar and simultaneously learning to classify the images, which motivates the usage of weighted training. Figure~\ref{fig:weighted_loss} explains how LISArD relates the LIS component with the classification one. As previously explained, the embeddings obtained from clean and noisy images are used in LIS while simultaneously being forwarded to the classification layer. The predictions are used in the (losses) $\mathcal{L}_{C}$ and $\mathcal{L}_{R}$ for the clean and noisy images, respectively, to train the classification component (the losses are better explained below). With this approach, we intend that the model initially concentrates on learning that two images represent the same object, but the final task is the classification of that object, justifying the initially increased importance of LIS and the gradually increasing importance of classification toward the end of training.

\begin{figure}[!t]
    \centering
    \includegraphics[width=0.9\linewidth]{imgs/weighted_loss.png}
    \caption{Overview of the LISArD~architecture. The clean and noisy images are fed to the model, and the inner product is calculated using their respective embeddings. Both clean (orange) and noisy embeddings (green) are used to predict each class using an adaptive weight loss between $\mathcal{L}_{C}$ and $\mathcal{L}_{R}$ and $\mathcal{L}_{S}$.}
    \label{fig:weighted_loss}
\end{figure}


\subsection{Loss Function}

\noindent The LISArD consists of a new defense mechanism that does not cost significantly more than the standard training but provides the networks with robustness against gray-box and white-box adversarial attacks. It starts by generating random images for every batch, according to the following equation:

\begin{equation}
x_{R} = x_{C} + \sqrt{\mu} \cdot x_{N},
\end{equation}
%
where $x_{R}$ refers to the random image, $x_{C}$ refers to the clean image, and $\mu$ is the maximum amount of perturbation to be added to the image (simulating the $\epsilon$ from adversarial attacks). $x_{N}$ refers to the Gaussian noise with the same size as the clean image. Since we have two images that are given as input to the model, we have a classification loss for each of them. Formally, this loss is defined as the comparison between the predicted label and the ground truth via Cross-Entropy:

\begin{equation}
\mathcal{L}_{\{C,R\}} = (y~\text{log}(p) + (1 - y)~\text{log}(1 - p)),
\end{equation}
%
where $\mathcal{L}_{\{C,R\}}$ refers to either the clean image loss or random image loss, $p$ are the predicted labels for the images batch, and $y$ are the ground truth labels for the images batch. Another part of the loss function consists of the approximation between the embeddings of each input image. The following equation translates this process:

\begin{equation}
    \mathcal{L}_{S} = \sum_i (1 - M_{ii})^2 + \lambda \sum_i \sum_{j \neq i} M_{ij}^2,
\end{equation}
%
where $\lambda$ is a positive constant that balances the importance of the terms and $M$ is the cross-correlation matrix obtained by the embeddings of the two images along the batch:

\begin{equation}
    M_{ij} = \frac{ \sum_b z^A_{b,i} z^B_{b,j} }{ \sqrt{\sum_b (z^A_{b,i})^2} \sqrt{\sum_b (z^B_{b,j})^2}},
\end{equation}
%
where $b$ is the index for the batch samples and $i$, $j$ are the indexes for the elements of the matrix. $M$ is a square matrix with a size equal to the network output. Finally, the complete loss function is expressed by:

\begin{equation}
\begin{aligned}
\mathcal{L}_{} = \alpha~(\mathcal{L}_{C} + \mathcal{L}_{R}) + (1-\alpha)~\left(\frac{\mathcal{L}_{S}}{\tau}\right), 
\end{aligned}
\label{eq:final_loss}
\end{equation}
%
where $\mathcal{L}_{C}$, $\mathcal{L}_{R}$, and $\mathcal{L}_{S}$ refer to the losses for clean images, random images (defined in equation 2), and similarity approximation (defined in equation 3). $\tau$ refers to the temperature and $\alpha$ refers to the weight for classification, with $\alpha$ starting at 0.5 and incrementing to 1 throughout training, as follows:

\begin{equation}
\begin{aligned}
\alpha = \alpha_{0} + \delta(\varepsilon-1),
\end{aligned}
\end{equation}
%
where $\alpha_{0}$ is the starting coefficient, defined as 0.5, $\delta$ is the decay degree, set to $\frac{1}{400}$ and $\varepsilon$ refers to the training epoch.



\subsection{Selected Attacks and State-of-the-art}

\noindent FGSM~\cite{goodfellow2015explaining}, PGD~\cite{madry2018towards}, and AA~\cite{croce2020reliable} attacks are selected to evaluate LISArD and compare it with state-of-the-art. FGSM is a one-step adversarial attack that uses the gradients of the model, being a weaker white-box adversarial attack. PGD is a strong attack that many defenses still fail to overcome and has multiple iterations that increase its strength. AA consists of an ensemble of attacks containing white-box and black-box variants, allowing an evaluation in both settings, which increases the scope of our evaluation. AdaAD~\cite{huang2023boosting}, PeerAiD~\cite{jung2024peeraid}, and DGAD~\cite{park2025dynamic} are the approaches selected to compare with LISArD since these \textit{Adversarial Distillation} models achieve state-of-the-art performance in white-box settings and have available implementations.



\subsection{Implementation Details}

\noindent\textbf{Hardware.} The experiments were performed in a multi-GPU server containing seven NVIDIA A40 and an Intel Xeon Silver 4310 \symbol{`@} 2.10 GHz, with the Pop!\_OS 22.04 LTS operating system. The models were trained using a single NVIDIA A40 GPU without additional models running on the same GPU when presenting the total time or time per epoch results.

\textbf{Models}. In order to be as comprehensive as possible regarding the multiple proposal of architectures, we selected ResNet18~\cite{he2016deep}, ResNet50~\cite{he2016deep}, ResNet101~\cite{he2016deep}, WideResNet28-10~\cite{zagoruyko2016wide}, VGG19~\cite{simonyan2014very}, MobileNetv2~\cite{sandler2018mobilenetv2}, and EfficientNetB2~\cite{tan2019efficientnet} as our backbones. 
For all the datasets, the networks were trained using an SGD optimizer with a learning rate of 0.001, a momentum of 0.9, and a weight decay of 0.0005 during 200 epochs. 
We disregarded the training of Inceptionv3~\cite{szegedy2016rethinking} due to its need to increase the image size to 299x299, which would not be the same training and evaluation settings as other models.

\textbf{Ablation Studies}. Models were trained for 200 epochs using ResNet18 as the backbone architecture for all ablation studies and evaluated on the CIFAR-10 clean, FGSM, PGD, and AA datasets. The last three datasets were generated by applying the respective attack to a previously trained ResNet18 on CIFAR-10 clean.
\section{Experiments}
\label{sec:experiments}
The experiments are designed to address two key research questions.
First, \textbf{RQ1} evaluates whether the average $L_2$-norm of the counterfactual perturbation vectors ($\overline{||\perturb||}$) decreases as the model overfits the data, thereby providing further empirical validation for our hypothesis.
Second, \textbf{RQ2} evaluates the ability of the proposed counterfactual regularized loss, as defined in (\ref{eq:regularized_loss2}), to mitigate overfitting when compared to existing regularization techniques.

% The experiments are designed to address three key research questions. First, \textbf{RQ1} investigates whether the mean perturbation vector norm decreases as the model overfits the data, aiming to further validate our intuition. Second, \textbf{RQ2} explores whether the mean perturbation vector norm can be effectively leveraged as a regularization term during training, offering insights into its potential role in mitigating overfitting. Finally, \textbf{RQ3} examines whether our counterfactual regularizer enables the model to achieve superior performance compared to existing regularization methods, thus highlighting its practical advantage.

\subsection{Experimental Setup}
\textbf{\textit{Datasets, Models, and Tasks.}}
The experiments are conducted on three datasets: \textit{Water Potability}~\cite{kadiwal2020waterpotability}, \textit{Phomene}~\cite{phomene}, and \textit{CIFAR-10}~\cite{krizhevsky2009learning}. For \textit{Water Potability} and \textit{Phomene}, we randomly select $80\%$ of the samples for the training set, and the remaining $20\%$ for the test set, \textit{CIFAR-10} comes already split. Furthermore, we consider the following models: Logistic Regression, Multi-Layer Perceptron (MLP) with 100 and 30 neurons on each hidden layer, and PreactResNet-18~\cite{he2016cvecvv} as a Convolutional Neural Network (CNN) architecture.
We focus on binary classification tasks and leave the extension to multiclass scenarios for future work. However, for datasets that are inherently multiclass, we transform the problem into a binary classification task by selecting two classes, aligning with our assumption.

\smallskip
\noindent\textbf{\textit{Evaluation Measures.}} To characterize the degree of overfitting, we use the test loss, as it serves as a reliable indicator of the model's generalization capability to unseen data. Additionally, we evaluate the predictive performance of each model using the test accuracy.

\smallskip
\noindent\textbf{\textit{Baselines.}} We compare CF-Reg with the following regularization techniques: L1 (``Lasso''), L2 (``Ridge''), and Dropout.

\smallskip
\noindent\textbf{\textit{Configurations.}}
For each model, we adopt specific configurations as follows.
\begin{itemize}
\item \textit{Logistic Regression:} To induce overfitting in the model, we artificially increase the dimensionality of the data beyond the number of training samples by applying a polynomial feature expansion. This approach ensures that the model has enough capacity to overfit the training data, allowing us to analyze the impact of our counterfactual regularizer. The degree of the polynomial is chosen as the smallest degree that makes the number of features greater than the number of data.
\item \textit{Neural Networks (MLP and CNN):} To take advantage of the closed-form solution for computing the optimal perturbation vector as defined in (\ref{eq:opt-delta}), we use a local linear approximation of the neural network models. Hence, given an instance $\inst_i$, we consider the (optimal) counterfactual not with respect to $\model$ but with respect to:
\begin{equation}
\label{eq:taylor}
    \model^{lin}(\inst) = \model(\inst_i) + \nabla_{\inst}\model(\inst_i)(\inst - \inst_i),
\end{equation}
where $\model^{lin}$ represents the first-order Taylor approximation of $\model$ at $\inst_i$.
Note that this step is unnecessary for Logistic Regression, as it is inherently a linear model.
\end{itemize}

\smallskip
\noindent \textbf{\textit{Implementation Details.}} We run all experiments on a machine equipped with an AMD Ryzen 9 7900 12-Core Processor and an NVIDIA GeForce RTX 4090 GPU. Our implementation is based on the PyTorch Lightning framework. We use stochastic gradient descent as the optimizer with a learning rate of $\eta = 0.001$ and no weight decay. We use a batch size of $128$. The training and test steps are conducted for $6000$ epochs on the \textit{Water Potability} and \textit{Phoneme} datasets, while for the \textit{CIFAR-10} dataset, they are performed for $200$ epochs.
Finally, the contribution $w_i^{\varepsilon}$ of each training point $\inst_i$ is uniformly set as $w_i^{\varepsilon} = 1~\forall i\in \{1,\ldots,m\}$.

The source code implementation for our experiments is available at the following GitHub repository: \url{https://anonymous.4open.science/r/COCE-80B4/README.md} 

\subsection{RQ1: Counterfactual Perturbation vs. Overfitting}
To address \textbf{RQ1}, we analyze the relationship between the test loss and the average $L_2$-norm of the counterfactual perturbation vectors ($\overline{||\perturb||}$) over training epochs.

In particular, Figure~\ref{fig:delta_loss_epochs} depicts the evolution of $\overline{||\perturb||}$ alongside the test loss for an MLP trained \textit{without} regularization on the \textit{Water Potability} dataset. 
\begin{figure}[ht]
    \centering
    \includegraphics[width=0.85\linewidth]{img/delta_loss_epochs.png}
    \caption{The average counterfactual perturbation vector $\overline{||\perturb||}$ (left $y$-axis) and the cross-entropy test loss (right $y$-axis) over training epochs ($x$-axis) for an MLP trained on the \textit{Water Potability} dataset \textit{without} regularization.}
    \label{fig:delta_loss_epochs}
\end{figure}

The plot shows a clear trend as the model starts to overfit the data (evidenced by an increase in test loss). 
Notably, $\overline{||\perturb||}$ begins to decrease, which aligns with the hypothesis that the average distance to the optimal counterfactual example gets smaller as the model's decision boundary becomes increasingly adherent to the training data.

It is worth noting that this trend is heavily influenced by the choice of the counterfactual generator model. In particular, the relationship between $\overline{||\perturb||}$ and the degree of overfitting may become even more pronounced when leveraging more accurate counterfactual generators. However, these models often come at the cost of higher computational complexity, and their exploration is left to future work.

Nonetheless, we expect that $\overline{||\perturb||}$ will eventually stabilize at a plateau, as the average $L_2$-norm of the optimal counterfactual perturbations cannot vanish to zero.

% Additionally, the choice of employing the score-based counterfactual explanation framework to generate counterfactuals was driven to promote computational efficiency.

% Future enhancements to the framework may involve adopting models capable of generating more precise counterfactuals. While such approaches may yield to performance improvements, they are likely to come at the cost of increased computational complexity.


\subsection{RQ2: Counterfactual Regularization Performance}
To answer \textbf{RQ2}, we evaluate the effectiveness of the proposed counterfactual regularization (CF-Reg) by comparing its performance against existing baselines: unregularized training loss (No-Reg), L1 regularization (L1-Reg), L2 regularization (L2-Reg), and Dropout.
Specifically, for each model and dataset combination, Table~\ref{tab:regularization_comparison} presents the mean value and standard deviation of test accuracy achieved by each method across 5 random initialization. 

The table illustrates that our regularization technique consistently delivers better results than existing methods across all evaluated scenarios, except for one case -- i.e., Logistic Regression on the \textit{Phomene} dataset. 
However, this setting exhibits an unusual pattern, as the highest model accuracy is achieved without any regularization. Even in this case, CF-Reg still surpasses other regularization baselines.

From the results above, we derive the following key insights. First, CF-Reg proves to be effective across various model types, ranging from simple linear models (Logistic Regression) to deep architectures like MLPs and CNNs, and across diverse datasets, including both tabular and image data. 
Second, CF-Reg's strong performance on the \textit{Water} dataset with Logistic Regression suggests that its benefits may be more pronounced when applied to simpler models. However, the unexpected outcome on the \textit{Phoneme} dataset calls for further investigation into this phenomenon.


\begin{table*}[h!]
    \centering
    \caption{Mean value and standard deviation of test accuracy across 5 random initializations for different model, dataset, and regularization method. The best results are highlighted in \textbf{bold}.}
    \label{tab:regularization_comparison}
    \begin{tabular}{|c|c|c|c|c|c|c|}
        \hline
        \textbf{Model} & \textbf{Dataset} & \textbf{No-Reg} & \textbf{L1-Reg} & \textbf{L2-Reg} & \textbf{Dropout} & \textbf{CF-Reg (ours)} \\ \hline
        Logistic Regression   & \textit{Water}   & $0.6595 \pm 0.0038$   & $0.6729 \pm 0.0056$   & $0.6756 \pm 0.0046$  & N/A    & $\mathbf{0.6918 \pm 0.0036}$                     \\ \hline
        MLP   & \textit{Water}   & $0.6756 \pm 0.0042$   & $0.6790 \pm 0.0058$   & $0.6790 \pm 0.0023$  & $0.6750 \pm 0.0036$    & $\mathbf{0.6802 \pm 0.0046}$                    \\ \hline
%        MLP   & \textit{Adult}   & $0.8404 \pm 0.0010$   & $\mathbf{0.8495 \pm 0.0007}$   & $0.8489 \pm 0.0014$  & $\mathbf{0.8495 \pm 0.0016}$     & $0.8449 \pm 0.0019$                    \\ \hline
        Logistic Regression   & \textit{Phomene}   & $\mathbf{0.8148 \pm 0.0020}$   & $0.8041 \pm 0.0028$   & $0.7835 \pm 0.0176$  & N/A    & $0.8098 \pm 0.0055$                     \\ \hline
        MLP   & \textit{Phomene}   & $0.8677 \pm 0.0033$   & $0.8374 \pm 0.0080$   & $0.8673 \pm 0.0045$  & $0.8672 \pm 0.0042$     & $\mathbf{0.8718 \pm 0.0040}$                    \\ \hline
        CNN   & \textit{CIFAR-10} & $0.6670 \pm 0.0233$   & $0.6229 \pm 0.0850$   & $0.7348 \pm 0.0365$   & N/A    & $\mathbf{0.7427 \pm 0.0571}$                     \\ \hline
    \end{tabular}
\end{table*}

\begin{table*}[htb!]
    \centering
    \caption{Hyperparameter configurations utilized for the generation of Table \ref{tab:regularization_comparison}. For our regularization the hyperparameters are reported as $\mathbf{\alpha/\beta}$.}
    \label{tab:performance_parameters}
    \begin{tabular}{|c|c|c|c|c|c|c|}
        \hline
        \textbf{Model} & \textbf{Dataset} & \textbf{No-Reg} & \textbf{L1-Reg} & \textbf{L2-Reg} & \textbf{Dropout} & \textbf{CF-Reg (ours)} \\ \hline
        Logistic Regression   & \textit{Water}   & N/A   & $0.0093$   & $0.6927$  & N/A    & $0.3791/1.0355$                     \\ \hline
        MLP   & \textit{Water}   & N/A   & $0.0007$   & $0.0022$  & $0.0002$    & $0.2567/1.9775$                    \\ \hline
        Logistic Regression   &
        \textit{Phomene}   & N/A   & $0.0097$   & $0.7979$  & N/A    & $0.0571/1.8516$                     \\ \hline
        MLP   & \textit{Phomene}   & N/A   & $0.0007$   & $4.24\cdot10^{-5}$  & $0.0015$    & $0.0516/2.2700$                    \\ \hline
       % MLP   & \textit{Adult}   & N/A   & $0.0018$   & $0.0018$  & $0.0601$     & $0.0764/2.2068$                    \\ \hline
        CNN   & \textit{CIFAR-10} & N/A   & $0.0050$   & $0.0864$ & N/A    & $0.3018/
        2.1502$                     \\ \hline
    \end{tabular}
\end{table*}

\begin{table*}[htb!]
    \centering
    \caption{Mean value and standard deviation of training time across 5 different runs. The reported time (in seconds) corresponds to the generation of each entry in Table \ref{tab:regularization_comparison}. Times are }
    \label{tab:times}
    \begin{tabular}{|c|c|c|c|c|c|c|}
        \hline
        \textbf{Model} & \textbf{Dataset} & \textbf{No-Reg} & \textbf{L1-Reg} & \textbf{L2-Reg} & \textbf{Dropout} & \textbf{CF-Reg (ours)} \\ \hline
        Logistic Regression   & \textit{Water}   & $222.98 \pm 1.07$   & $239.94 \pm 2.59$   & $241.60 \pm 1.88$  & N/A    & $251.50 \pm 1.93$                     \\ \hline
        MLP   & \textit{Water}   & $225.71 \pm 3.85$   & $250.13 \pm 4.44$   & $255.78 \pm 2.38$  & $237.83 \pm 3.45$    & $266.48 \pm 3.46$                    \\ \hline
        Logistic Regression   & \textit{Phomene}   & $266.39 \pm 0.82$ & $367.52 \pm 6.85$   & $361.69 \pm 4.04$  & N/A   & $310.48 \pm 0.76$                    \\ \hline
        MLP   &
        \textit{Phomene} & $335.62 \pm 1.77$   & $390.86 \pm 2.11$   & $393.96 \pm 1.95$ & $363.51 \pm 5.07$    & $403.14 \pm 1.92$                     \\ \hline
       % MLP   & \textit{Adult}   & N/A   & $0.0018$   & $0.0018$  & $0.0601$     & $0.0764/2.2068$                    \\ \hline
        CNN   & \textit{CIFAR-10} & $370.09 \pm 0.18$   & $395.71 \pm 0.55$   & $401.38 \pm 0.16$ & N/A    & $1287.8 \pm 0.26$                     \\ \hline
    \end{tabular}
\end{table*}

\subsection{Feasibility of our Method}
A crucial requirement for any regularization technique is that it should impose minimal impact on the overall training process.
In this respect, CF-Reg introduces an overhead that depends on the time required to find the optimal counterfactual example for each training instance. 
As such, the more sophisticated the counterfactual generator model probed during training the higher would be the time required. However, a more advanced counterfactual generator might provide a more effective regularization. We discuss this trade-off in more details in Section~\ref{sec:discussion}.

Table~\ref{tab:times} presents the average training time ($\pm$ standard deviation) for each model and dataset combination listed in Table~\ref{tab:regularization_comparison}.
We can observe that the higher accuracy achieved by CF-Reg using the score-based counterfactual generator comes with only minimal overhead. However, when applied to deep neural networks with many hidden layers, such as \textit{PreactResNet-18}, the forward derivative computation required for the linearization of the network introduces a more noticeable computational cost, explaining the longer training times in the table.

\subsection{Hyperparameter Sensitivity Analysis}
The proposed counterfactual regularization technique relies on two key hyperparameters: $\alpha$ and $\beta$. The former is intrinsic to the loss formulation defined in (\ref{eq:cf-train}), while the latter is closely tied to the choice of the score-based counterfactual explanation method used.

Figure~\ref{fig:test_alpha_beta} illustrates how the test accuracy of an MLP trained on the \textit{Water Potability} dataset changes for different combinations of $\alpha$ and $\beta$.

\begin{figure}[ht]
    \centering
    \includegraphics[width=0.85\linewidth]{img/test_acc_alpha_beta.png}
    \caption{The test accuracy of an MLP trained on the \textit{Water Potability} dataset, evaluated while varying the weight of our counterfactual regularizer ($\alpha$) for different values of $\beta$.}
    \label{fig:test_alpha_beta}
\end{figure}

We observe that, for a fixed $\beta$, increasing the weight of our counterfactual regularizer ($\alpha$) can slightly improve test accuracy until a sudden drop is noticed for $\alpha > 0.1$.
This behavior was expected, as the impact of our penalty, like any regularization term, can be disruptive if not properly controlled.

Moreover, this finding further demonstrates that our regularization method, CF-Reg, is inherently data-driven. Therefore, it requires specific fine-tuning based on the combination of the model and dataset at hand.


\section{Conclusion}
\label{sec:conclusion}

\noindent This paper describes an evaluation framework based on the \emph{gray-box setting} that is more realistic than the typically used \emph{white-box} scenario, where the existing models do not perform reliably. We propose an adversarial defense mechanism for this setting (LISArD), which is simultaneously robust against white-box attacks, and does not depend on the inclusion of adversarial samples. This mechanism uses image similarity to instruct the model to recognize that images pairs regard the same object, while simultaneously inferring class information. The experiments show the vulnerability of pre-trained and scratch-trained networks to gray-box adversarial samples and point to the effectiveness of LISArD in increasing resilience against this type of samples. Also, state-of-the-art \textit{Adversarial Distillation} models cannot perform in white-box settings without the inclusion of AT. In the future, the injection of other types of noise (\textit{e.g.}, using fractional Gaussian noise with persistence, instead of white noise) will be subject of our further analysis. Finally, for realism purposes, we suggest evaluating the models in scenarios in which the attacker only knows the training data and the type of architecture.



% Generated by IEEEtran.bst, version: 1.14 (2015/08/26)
\begin{thebibliography}{10}
\providecommand{\url}[1]{#1}
\csname url@samestyle\endcsname
\providecommand{\newblock}{\relax}
\providecommand{\bibinfo}[2]{#2}
\providecommand{\BIBentrySTDinterwordspacing}{\spaceskip=0pt\relax}
\providecommand{\BIBentryALTinterwordstretchfactor}{4}
\providecommand{\BIBentryALTinterwordspacing}{\spaceskip=\fontdimen2\font plus
\BIBentryALTinterwordstretchfactor\fontdimen3\font minus \fontdimen4\font\relax}
\providecommand{\BIBforeignlanguage}[2]{{%
\expandafter\ifx\csname l@#1\endcsname\relax
\typeout{** WARNING: IEEEtran.bst: No hyphenation pattern has been}%
\typeout{** loaded for the language `#1'. Using the pattern for}%
\typeout{** the default language instead.}%
\else
\language=\csname l@#1\endcsname
\fi
#2}}
\providecommand{\BIBdecl}{\relax}
\BIBdecl

\bibitem{thirunavukarasu2023large}
A.~J. Thirunavukarasu, D.~S.~J. Ting, K.~Elangovan, L.~Gutierrez, T.~F. Tan, and D.~S.~W. Ting, ``Large language models in medicine,'' \emph{Nature medicine}, vol.~29, no.~8, pp. 1930--1940, 2023.

\bibitem{patricio2023coherent}
C.~Patr{\'\i}cio, J.~C. Neves, and L.~F. Teixeira, ``Coherent concept-based explanations in medical image and its application to skin lesion diagnosis,'' in \emph{Proceedings of the IEEE/CVF Conference on Computer Vision and Pattern Recognition}, 2023, pp. 3799--3808.

\bibitem{touvron2023llama2openfoundation}
\BIBentryALTinterwordspacing
H.~Touvron and et~al., ``Llama 2: Open foundation and fine-tuned chat models,'' 2023. [Online]. Available: \url{https://arxiv.org/abs/2307.09288}
\BIBentrySTDinterwordspacing

\bibitem{costa2022predicting}
J.~C. Costa, T.~Roxo, J.~B. Sequeiros, H.~Proenca, and P.~R. Inacio, ``Predicting cvss metric via description interpretation,'' \emph{IEEE Access}, vol.~10, pp. 59\,125--59\,134, 2022.

\bibitem{roxo2023exploring}
T.~Roxo, J.~C. Costa, P.~R. In{\'a}cio, and H.~Proen{\c{c}}a, ``On exploring audio anomaly in speech,'' in \emph{2023 IEEE International Workshop on Information Forensics and Security (WIFS)}.\hskip 1em plus 0.5em minus 0.4em\relax IEEE, 2023, pp. 1--6.

\bibitem{roxo2024bias}
------, ``Bias: A body-based interpretable active speaker approach,'' \emph{IEEE Transactions on Biometrics, Behavior, and Identity Science}, 2024.

\bibitem{roxo2024asdnb}
T.~Roxo, J.~C. Costa, P.~In{\'a}cio, and H.~Proen{\c{c}}a, ``Asdnb: Merging face with body cues for robust active speaker detection,'' \emph{arXiv preprint arXiv:2412.08594}, 2024.

\bibitem{szegedy2014intriguing}
C.~Szegedy, W.~Zaremba, I.~Sutskever, J.~Bruna, D.~Erhan, I.~Goodfellow, and R.~Fergus, ``Intriguing properties of neural networks,'' in \emph{2nd International Conference on Learning Representations, ICLR 2014}, 2014.

\bibitem{papernot2016distillation}
N.~Papernot, P.~McDaniel, X.~Wu, S.~Jha, and A.~Swami, ``Distillation as a defense to adversarial perturbations against deep neural networks,'' in \emph{2016 IEEE symposium on security and privacy (SP)}.\hskip 1em plus 0.5em minus 0.4em\relax IEEE, 2016, pp. 582--597.

\bibitem{goldblum2020adversarially}
M.~Goldblum, L.~Fowl, S.~Feizi, and T.~Goldstein, ``Adversarially robust distillation,'' in \emph{Proceedings of the AAAI conference on artificial intelligence}, vol.~34, no.~04, 2020, pp. 3996--4003.

\bibitem{zhu2021reliable}
J.~Zhu, J.~Yao, B.~Han, J.~Zhang, T.~Liu, G.~Niu, J.~Zhou, J.~Xu, and H.~Yang, ``Reliable adversarial distillation with unreliable teachers,'' in \emph{International Conference on Learning Representations}, 2021.

\bibitem{yoon2021adversarial}
J.~Yoon, S.~J. Hwang, and J.~Lee, ``Adversarial purification with score-based generative models,'' in \emph{International Conference on Machine Learning}.\hskip 1em plus 0.5em minus 0.4em\relax PMLR, 2021, pp. 12\,062--12\,072.

\bibitem{wu2022guided}
Q.~Wu, H.~Ye, and Y.~Gu, ``Guided diffusion model for adversarial purification from random noise,'' \emph{arXiv preprint arXiv:2206.10875}, 2022.

\bibitem{chen2022densepure}
Z.~Chen, K.~Jin, J.~Wang, W.~Nie, M.~Liu, A.~Anandkumar, B.~Li, and D.~Song, ``Densepure: Understanding diffusion models towards adversarial robustness,'' in \emph{Workshop on Trustworthy and Socially Responsible Machine Learning, NeurIPS 2022}, 2022.

\bibitem{ho2020denoising}
J.~Ho, A.~Jain, and P.~Abbeel, ``Denoising diffusion probabilistic models,'' \emph{Advances in neural information processing systems}, vol.~33, pp. 6840--6851, 2020.

\bibitem{katzir2020gradients}
Z.~Katzir and Y.~Elovici, ``Gradients cannot be tamed: Behind the impossible paradox of blocking targeted adversarial attacks,'' \emph{IEEE Transactions on Neural Networks and Learning Systems}, vol.~32, no.~1, pp. 128--138, 2020.

\bibitem{goodfellow2015explaining}
I.~J. Goodfellow, J.~Shlens, and C.~Szegedy, ``Explaining and harnessing adversarial examples,'' \emph{stat}, vol. 1050, p.~20, 2015.

\bibitem{papernot2016limitations}
N.~Papernot, P.~McDaniel, S.~Jha, M.~Fredrikson, Z.~B. Celik, and A.~Swami, ``The limitations of deep learning in adversarial settings,'' in \emph{2016 IEEE European symposium on security and privacy (EuroS\&P)}.\hskip 1em plus 0.5em minus 0.4em\relax IEEE, 2016, pp. 372--387.

\bibitem{tramer2017space}
F.~Tram{\`e}r, N.~Papernot, I.~Goodfellow, D.~Boneh, and P.~McDaniel, ``The space of transferable adversarial examples,'' \emph{stat}, vol. 1050, p.~23, 2017.

\bibitem{croce2019sparse}
F.~Croce and M.~Hein, ``Sparse and imperceivable adversarial attacks,'' in \emph{Proceedings of the IEEE/CVF international conference on computer vision}, 2019, pp. 4724--4732.

\bibitem{moosavi2016deepfool}
S.-M. Moosavi-Dezfooli, A.~Fawzi, and P.~Frossard, ``Deepfool: a simple and accurate method to fool deep neural networks,'' in \emph{Proceedings of the IEEE conference on computer vision and pattern recognition}, 2016, pp. 2574--2582.

\bibitem{dabouei2020smoothfool}
A.~Dabouei, S.~Soleymani, F.~Taherkhani, J.~Dawson, and N.~Nasrabadi, ``Smoothfool: An efficient framework for computing smooth adversarial perturbations,'' in \emph{Proceedings of the IEEE/CVF Winter Conference on Applications of Computer Vision}, 2020, pp. 2665--2674.

\bibitem{madry2018towards}
A.~Madry, A.~Makelov, L.~Schmidt, D.~Tsipras, and A.~Vladu, ``Towards deep learning models resistant to adversarial attacks,'' in \emph{International Conference on Learning Representations}, 2018.

\bibitem{dong2018boosting}
Y.~Dong, F.~Liao, T.~Pang, H.~Su, J.~Zhu, X.~Hu, and J.~Li, ``Boosting adversarial attacks with momentum,'' in \emph{Proceedings of the IEEE conference on computer vision and pattern recognition}, 2018, pp. 9185--9193.

\bibitem{croce2020reliable}
F.~Croce and M.~Hein, ``Reliable evaluation of adversarial robustness with an ensemble of diverse parameter-free attacks,'' in \emph{International conference on machine learning}.\hskip 1em plus 0.5em minus 0.4em\relax PMLR, 2020, pp. 2206--2216.

\bibitem{croce2020minimally}
------, ``Minimally distorted adversarial examples with a fast adaptive boundary attack,'' in \emph{International Conference on Machine Learning}.\hskip 1em plus 0.5em minus 0.4em\relax PMLR, 2020, pp. 2196--2205.

\bibitem{andriushchenko2020square}
M.~Andriushchenko, F.~Croce, N.~Flammarion, and M.~Hein, ``Square attack: a query-efficient black-box adversarial attack via random search,'' in \emph{European conference on computer vision}.\hskip 1em plus 0.5em minus 0.4em\relax Springer, 2020, pp. 484--501.

\bibitem{papernot2017extending}
N.~Papernot and P.~McDaniel, ``Extending defensive distillation,'' \emph{arXiv preprint arXiv:1705.05264}, 2017.

\bibitem{carmon2019unlabeled}
Y.~Carmon, A.~Raghunathan, L.~Schmidt, J.~C. Duchi, and P.~S. Liang, ``Unlabeled data improves adversarial robustness,'' \emph{Advances in neural information processing systems}, vol.~32, 2019.

\bibitem{zi2021revisiting}
B.~Zi, S.~Zhao, X.~Ma, and Y.-G. Jiang, ``Revisiting adversarial robustness distillation: Robust soft labels make student better,'' in \emph{Proceedings of the IEEE/CVF International Conference on Computer Vision}, 2021, pp. 16\,443--16\,452.

\bibitem{chen2021ltd}
E.-C. Chen and C.-R. Lee, ``Ltd: Low temperature distillation for robust adversarial training,'' \emph{arXiv preprint arXiv:2111.02331}, 2021.

\bibitem{zhu2023improving}
K.~Zhu, X.~Hu, J.~Wang, X.~Xie, and G.~Yang, ``Improving generalization of adversarial training via robust critical fine-tuning,'' in \emph{Proceedings of the IEEE/CVF International Conference on Computer Vision}, 2023, pp. 4424--4434.

\bibitem{huang2023boosting}
B.~Huang, M.~Chen, Y.~Wang, J.~Lu, M.~Cheng, and W.~Wang, ``Boosting accuracy and robustness of student models via adaptive adversarial distillation,'' in \emph{Proceedings of the IEEE/CVF Conference on Computer Vision and Pattern Recognition}, 2023, pp. 24\,668--24\,677.

\bibitem{kuang2024improving}
H.~Kuang, H.~Liu, Y.~Wu, S.~Satoh, and R.~Ji, ``Improving adversarial robustness via information bottleneck distillation,'' \emph{Advances in Neural Information Processing Systems}, vol.~36, 2024.

\bibitem{yue2024revisiting}
X.~Yue, M.~Ningping, Q.~Wang, and L.~Zhao, ``Revisiting adversarial robustness distillation from the perspective of robust fairness,'' \emph{Advances in Neural Information Processing Systems}, vol.~36, 2024.

\bibitem{jung2024peeraid}
J.~Jung, H.~Jang, J.~Song, and J.~Lee, ``Peeraid: Improving adversarial distillation from a specialized peer tutor,'' in \emph{Proceedings of the IEEE/CVF Conference on Computer Vision and Pattern Recognition}, 2024, pp. 24\,482--24\,491.

\bibitem{park2025dynamic}
H.~Park and D.~Min, ``Dynamic guidance adversarial distillation with enhanced teacher knowledge,'' in \emph{European Conference on Computer Vision}.\hskip 1em plus 0.5em minus 0.4em\relax Springer, 2025, pp. 204--219.

\bibitem{nie2022diffusion}
W.~Nie, B.~Guo, Y.~Huang, C.~Xiao, A.~Vahdat, and A.~Anandkumar, ``Diffusion models for adversarial purification,'' in \emph{International Conference on Machine Learning}.\hskip 1em plus 0.5em minus 0.4em\relax PMLR, 2022, pp. 16\,805--16\,827.

\bibitem{kalaria2022towards}
D.~Kalaria, A.~Hazra, and P.~P. Chakrabarti, ``Towards adversarial purification using denoising autoencoders,'' \emph{arXiv preprint arXiv:2208.13838}, 2022.

\bibitem{vincent2008extracting}
P.~Vincent, H.~Larochelle, Y.~Bengio, and P.-A. Manzagol, ``Extracting and composing robust features with denoising autoencoders,'' in \emph{Proceedings of the 25th international conference on Machine learning}, 2008, pp. 1096--1103.

\bibitem{wang2023better}
Z.~Wang, T.~Pang, C.~Du, M.~Lin, W.~Liu, and S.~Yan, ``Better diffusion models further improve adversarial training,'' in \emph{International Conference on Machine Learning}.\hskip 1em plus 0.5em minus 0.4em\relax PMLR, 2023, pp. 36\,246--36\,263.

\bibitem{karras2022elucidating}
T.~Karras, M.~Aittala, T.~Aila, and S.~Laine, ``Elucidating the design space of diffusion-based generative models,'' \emph{Advances in neural information processing systems}, vol.~35, pp. 26\,565--26\,577, 2022.

\bibitem{lee2023robust}
M.~Lee and D.~Kim, ``Robust evaluation of diffusion-based adversarial purification,'' in \emph{Proceedings of the IEEE/CVF International Conference on Computer Vision}, 2023, pp. 134--144.

\bibitem{cao2023fepn}
D.~Cao, K.~Wei, Y.~Wu, J.~Zhang, B.~Feng, and J.~Chen, ``Fepn: A robust feature purification network to defend against adversarial examples,'' \emph{Computers \& Security}, vol. 134, p. 103427, 2023.

\bibitem{chen2024diffilter}
Y.~Chen, X.~Li, X.~Wang, P.~Hu, and D.~Peng, ``Diffilter: Defending against adversarial perturbations with diffusion filter,'' \emph{IEEE Transactions on Information Forensics and Security}, 2024.

\bibitem{zhang2024random}
J.~Zhang, P.~Dong, Y.~Chen, Y.-P. Zhao, and S.~Guo, ``Random sampling for diffusion-based adversarial purification,'' \emph{arXiv preprint arXiv:2411.18956}, 2024.

\bibitem{song2024mimicdiffusion}
K.~Song, H.~Lai, Y.~Pan, and J.~Yin, ``Mimicdiffusion: Purifying adversarial perturbation via mimicking clean diffusion model,'' in \emph{Proceedings of the IEEE/CVF Conference on Computer Vision and Pattern Recognition}, 2024, pp. 24\,665--24\,674.

\bibitem{zbontar2021barlow}
J.~Zbontar, L.~Jing, I.~Misra, Y.~LeCun, and S.~Deny, ``Barlow twins: Self-supervised learning via redundancy reduction,'' in \emph{International conference on machine learning}.\hskip 1em plus 0.5em minus 0.4em\relax PMLR, 2021, pp. 12\,310--12\,320.

\bibitem{he2016deep}
K.~He, X.~Zhang, S.~Ren, and J.~Sun, ``Deep residual learning for image recognition,'' in \emph{Proceedings of the IEEE conference on computer vision and pattern recognition}, 2016, pp. 770--778.

\bibitem{zagoruyko2016wide}
S.~Zagoruyko, ``Wide residual networks,'' \emph{arXiv preprint arXiv:1605.07146}, 2016.

\bibitem{simonyan2014very}
K.~Simonyan and A.~Zisserman, ``Very deep convolutional networks for large-scale image recognition,'' \emph{arXiv preprint arXiv:1409.1556}, 2014.

\bibitem{sandler2018mobilenetv2}
M.~Sandler, A.~Howard, M.~Zhu, A.~Zhmoginov, and L.-C. Chen, ``Mobilenetv2: Inverted residuals and linear bottlenecks,'' in \emph{Proceedings of the IEEE conference on computer vision and pattern recognition}, 2018, pp. 4510--4520.

\bibitem{tan2019efficientnet}
M.~Tan and Q.~Le, ``Efficientnet: Rethinking model scaling for convolutional neural networks,'' in \emph{International conference on machine learning}.\hskip 1em plus 0.5em minus 0.4em\relax PMLR, 2019, pp. 6105--6114.

\bibitem{szegedy2016rethinking}
C.~Szegedy, V.~Vanhoucke, S.~Ioffe, J.~Shlens, and Z.~Wojna, ``Rethinking the inception architecture for computer vision,'' in \emph{Proceedings of the IEEE conference on computer vision and pattern recognition}, 2016, pp. 2818--2826.

\bibitem{krizhevsky2009learning}
A.~Krizhevsky, G.~Hinton \emph{et~al.}, ``Learning multiple layers of features from tiny images,'' \emph{Master's thesis, University of Tront}, 2009.

\bibitem{le2015tiny}
Y.~Le and X.~Yang, ``Tiny imagenet visual recognition challenge,'' \emph{CS 231N}, vol.~7, no.~7, p.~3, 2015.

\bibitem{daugman2000biometric}
J.~Daugman, ``Biometric decision landscapes,'' University of Cambridge, Computer Laboratory, Tech. Rep., 2000.
\end{thebibliography}



\section{Biography Section}

\vspace{11pt}

\begin{IEEEbiography}[{\includegraphics[width=1in,height=1.25in,clip,
    keepaspectratio]{figures/bioPics/clemot_bioPic_bw.jpg}}]{Mattéo Clémot}
    is a Ph.D.\ student at Université Lyon~1.
    He received an M.Sc. degree in Computer Science from the \'Ecole Normale Sup\'erieure de Lyon in 2022.
    His research lies in geometry and topological data analysis.
\end{IEEEbiography}

\begin{IEEEbiography}[{\includegraphics[width=1in,height=1.25in,clip,
    keepaspectratio]{figures/bioPics/digne_bioPic_bw.jpg}}]{Julie Digne}
    received the PhD degree in Applied Mathematics from the \'Ecole Normale Sup\'erieure de Cachan in 2010.
    She is a senior CNRS researcher at Université Lyon 1. She specializes in Geometry Processing, and in particular shape analysis, leveraging local differential properties, nonlocal analysis or statistical learning. She served as program chair of the Symposium on Geometry Processing in 2021.
\end{IEEEbiography}

\begin{IEEEbiography}[{\includegraphics[width=1in,height=1.25in,clip,
    keepaspectratio]{figures/bioPics/tierny_bioPic_bw.jpg}}]{Julien Tierny}
    received the Ph.D. degree in Computer Science from the University
    of Lille in 2008. He is a CNRS research director at
    Sorbonne University. Prior to his CNRS tenure, he held a
    Fulbright fellowship and was a post-doctoral
    researcher at the University of Utah.
    His research expertise lies in topological methods for data analysis
    and visualization.
    He is the founder and lead developer of the Topology ToolKit
    (TTK), an open source library for topological data analysis.
\end{IEEEbiography}
\vfill

\end{document}


