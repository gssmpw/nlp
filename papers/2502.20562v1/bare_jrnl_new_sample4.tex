\documentclass[lettersize,journal]{IEEEtran}
\usepackage{amsmath,amsfonts}
\usepackage{algorithmic}
\usepackage{algorithm}
\usepackage{array}
\usepackage[caption=false,font=normalsize,labelfont=sf,textfont=sf]{subfig}
\usepackage{textcomp}
\usepackage{stfloats}
\usepackage{url}
\usepackage{verbatim}
\usepackage{graphicx}
\usepackage{cite}
\hyphenation{op-tical net-works semi-conduc-tor IEEE-Xplore}
% updated with editorial comments 8/9/2021

% Custom added
\usepackage{multirow}
\usepackage{url}
\usepackage{hyperref}

\begin{document}

\title{LISArD: Learning Image Similarity to Defend Against Gray-box Adversarial Attacks}

\author{Joana C. Costa,
        Tiago Roxo,
        Hugo Proença,~\IEEEmembership{Senior Member,~IEEE,}
        Pedro R. M. Inácio,~\IEEEmembership{Senior Member,~IEEE}
        % <-this % stops a space
\thanks{Manuscript received February XX, 2025; revised XX XX, 2025. This work was supported in part by the Portuguese Fundação para a Ciência e Tecnologia (FCT)/Ministério da Ciência, Tecnologia e Ensino Superior (MCTES) through National Funds and co-funded by EU funds under Project UIDB/50008/2020; in part by the FCT Doctoral Grant 2020.09847.BD and Grant 2021.04905.BD.}% <-this % stops a space
\thanks{Joana C. Costa, Tiago Roxo, Hugo Proença and Pedro R. M. Inácio are with the Instituto de Telecomunicações, sins-lab, and Department of Computer Science, Universidade da Beira Interior, Portugal (corresponding author e-mail: \href{mailto:joana.cabral.costa@ubi.pt}{joana.cabral.costa@ubi.pt}).}}

% The paper headers
\markboth{Submitted to IEEE Transactions on Information Forensics and Security, Vol. XX, XXXX}%
{Costa \MakeLowercase{\textit{et al.}}: LISArD: Learning Image Similarity to Defend Against Gray-box Adversarial Attacks}

\IEEEpubid{0000--0000/00\$00.00~\copyright~2021 IEEE}
% Remember, if you use this you must call \IEEEpubidadjcol in the second
% column for its text to clear the IEEEpubid mark.

\maketitle

\begin{abstract}
Retrieval-Augmented Generation (RAG) is often used with Large Language Models (LLMs) to infuse domain knowledge or user-specific information. In RAG, given a user query, a retriever extracts chunks of relevant text from a knowledge base. These chunks are sent to an LLM as part of the input prompt. Typically, any given chunk is repeatedly retrieved across user questions. However, currently, for every question, attention-layers in LLMs fully compute the key values (KVs) repeatedly for the input chunks, as state-of-the-art methods cannot reuse KV-caches when chunks appear at arbitrary locations with arbitrary contexts. Naive reuse leads to output quality degradation.  This leads to potentially redundant computations on expensive GPUs and increases latency. In this work, we propose \sys, a system for managing and reusing precomputed KVs corresponding to the text chunks (we call \textit{chunk-caches}) in RAG-based systems. We present how to identify \hl{\textit{chunk-caches} that are reusable}, how to efficiently perform a small fraction of recomputation to \textit{fix} the cache to maintain output quality, and how to efficiently store and evict \textit{chunk-caches} in the hardware for maximizing reuse while masking any overheads. With real production workloads as well as synthetic datasets, we show that \sys reduces redundant computation by \textbf{51\%} over SOTA prefix-caching and \textbf{75\%} over full recomputation.
\hl{Additionally, with continuous batching on a real production workload, we get a \textbf{1.6$\times$} speedup in throughput and a \textbf{2$\times$} reduction in end-to-end response latency over prefix-caching while maintaining quality, for both the \llama-3-8B and \llama-3-70B models. 
}
\end{abstract}






\section{Introduction}
\label{sec:intro}

\begin{figure*}[tb]
    \centering
    \includegraphics[width=0.848\linewidth]{figs/circuitnn.pdf} 
    \caption{Illustration of differentiable CircuitNN. CircuitNN is designed based on differentiable NAND gates. After DAS is guided by PI and PO pairs of the truth table, CircuitNN can get the precise circuit architecture logic equivalent to the truth table.}
    \label{fig:circuitnn}
\end{figure*}

% 1. Describe the importance of logic synthesis
% 2. Existing Problems
% (a) Neural Architecture Search: Unstable, Predefined Setting, etc.
% (b) Circuit Generation: Probabilistic Model, Logic Equivalence

With the rapid advancement of technology, the scale of integrated circuits (ICs) has expanded exponentially. 
This expansion has introduced significant challenges in chip manufacturing, particularly concerning power and area metrics.
A primary objective in IC design is achieving the same circuit function with fewer transistors, thereby reducing power usage and area occupancy.

Logic synthesis~\cite{hachtel2005logicsynth}, a critical step in electronic design automation (EDA), transforms behavioral-level circuit designs into optimized gate-level circuits, ultimately yielding the final IC layout. 
The primary goal of logic synthesis is to identify the physical implementation with the fewest gates for a given circuit function. 
This task constitutes a challenging NP-hard combinatorial optimization problem. 
Current logic synthesis tools~\cite{brayton2010abc, wolf2013yosys} rely on human-designed heuristics, often leading to sub-optimal outcomes.

Differentiable architecture search (DAS) techniques~\cite{liu2018darts, chu2020darts} offer novel perspectives on addressing challenges in this problem.
Circuit functions can be represented through truth tables, which map binary inputs to their corresponding outputs. 
Truth tables provide a precise representation of input-output relationships, ensuring the design of functionally equivalent circuits.
Inspired by this, researchers~\cite{deepmind2024ai4sys, wang2024tnet} have begun exploring the application of DAS to synthesize circuits directly from truth tables.
Specifically, \citet{deepmind2024ai4sys} proposed CircuitNN, a framework that learns differentiable connection structures with logic gates, enabling the automatic generation of logic circuits from truth tables.
This approach significantly reduces the complexity of traditional circuit generation. 
Building on this, \citet{wang2024tnet} introduced T-Net, a triangle-shaped variant of CircuitNN, incorporating regularization techniques to enhance the efficiency of DAS.

Despite these advancements, several challenges remain. 
The computational complexity of DAS grows quadratically with the number of gates, posing scalability issues.
Although triangle-shaped architecture~\cite{wang2024tnet} partially mitigates this problem, redundancy persists. 
%Additionally, DAS is susceptible to converging to local optima, limiting the ability to search architectures that satisfy the given truth tables~\cite{liu2018darts}. 
%Furthermore, hyperparameters (network depth and layer width) require extensive searches, introducing complexity and prolonging the synthesis process. 
Additionally, DAS is susceptible to converging to local optima~\cite{liu2018darts} and hyperparameters (network depth and layer width) require extensive searches. 
The challenges arise from the vast search space in DAS. 
% Even with predefined settings for CircuitNN, finding a configuration that meets the truth table requires extensive trial and error during the DAS process. 
Intuitively, limiting the search space through predefined parameters (network depth, gates per layer, and connection probabilities) can significantly reduce the complexity.

Recent advances~\cite{openai2023gpt4, abramson2024alphafold3, esser2024sd3, li2024mar} in conditional generative models have demonstrated remarkable performance across language, vision, and graph generation tasks. 
Motivated by these developments, we propose a novel approach to circuit generation that generates preliminary circuit structures to guide DAS in generating refined circuits matching specified truth tables. 
Firstly, we introduce CircuitVQ, a tokenizer with a discrete codebook for circuit tokenization. 
Built upon our Circuit AutoEncoder framework~\cite{hou2022graphmae,li2023maskgae,wu2025mgvga}, CircuitVQ is trained through a circuit reconstruction task. 
Specifically, the CircuitVQ encoder encodes input circuits into discrete tokens using a learnable codebook, while the decoder reconstructs the circuit adjacency matrix based on these tokens.
Subsequently, the CircuitVQ encoder serves as a circuit tokenizer for CircuitAR pretraining, which employs a masked autoregressive modeling paradigm~\cite{chang2022maskgit, li2023mage}. 
In this process, the discrete codes function as supervision signals. 
After training, CircuitAR can generate discrete tokens progressively, which can be decoded into initial circuit structures by the decoder of the CircuitVQ. 
These prior insights can guide DAS in producing refined circuits that match the target truth tables precisely.

Our key contributions can be summarized as follows:
\begin{itemize}
\item We introduce CircuitVQ, a circuit tokenizer that facilitates graph autoregressive modeling for circuit generation, based on our Circuit AutoEncoder framework;
\item Develop CircuitAR, a model trained using masked autoregressive modeling, which generates initial circuit structures conditioned on given truth tables;
\item Propose a refinement framework that integrates differentiable architecture search to produce functionally equivalent circuits guided by target truth tables;
\item Comprehensive experiments demonstrating the scalability and capability emergence of our CircuitAR and the superior performance of the proposed circuit generation approach.
\end{itemize}

% Motivation
% (a) Diffusion (Vision, Graph), Autoregressive (Language, Vision)
% (b) Circuit Generation for Predefined Setting
% (c) Neural Architecture Search for Strict Logic Equivalence

% Contribution
% (a) Circuit Tokenizer (new transformer arch, training strategy)
% (b) CircuitAR (train and gen strategies, post-ar strategy)
% (c) Extensive Evaluation including BitD (Bit Distance) for Scalability



\section{Related Work}
\label{sec:related-work}

\noindent\textbf{White-box Adversarial Attacks}.
L-BFGS~\cite{szegedy2014intriguing} was the first proposed adversarial attack that demonstrated how simple perturbations could affect the DNNs performance.
Fast Gradient Sign Method (FGSM)~\cite{goodfellow2015explaining} is a one-step method that uses the model cost function, the gradient, and the radius epsilon to search for perturbations.
Jacobian-based Saliency Maps (JSM)~\cite{papernot2016limitations} explore the forward derivatives and construct the adversarial saliency maps.
Gradient Aligned Adversarial Subspace (GAAS)~\cite{tramer2017space} estimates the dimensionality of the adversarial subspace using the first-order approximation of the loss function.
Sparse and Imperceivable Adversarial Attacks (SIAA)~\cite{croce2019sparse} create sporadic and imperceptible perturbations by applying the standard deviation of each color channel in both axis directions.
DeepFool~\cite{moosavi2016deepfool} is an iterative attack that stops when the minimal vector orthogonal to the hyperplane representing the decision boundary is found.
SmoothFool (SF)~\cite{dabouei2020smoothfool} is an iterative algorithm that uses DeepFool to calculate the initial perturbation and smoothly rectifies the resulting perturbation until the adversarial example fools the classifier.
Projected Gradient Descent (PGD)~\cite{madry2018towards} is an iterative attack that uses saddle point formulation to find a strong perturbation.
Momentum Iterative FGSM (MI-FGSM)~\cite{dong2018boosting} introduces momentum into the Iterative FGSM (I-FGSM).
Auto-Attack~\cite{croce2020reliable} is a set of attacks to evaluate the networks, proposing the APGD-CE (i.e., PGD using Cross-Entropy (CE)), and APGD-DLR (i.e., PGD using Difference of Logits Ratio (DLR)) attacks. These techniques are are combined with Fast Adaptive Boundary (FAB)~\cite{croce2020minimally}, used to minimize the norm of the adversarial perturbations, and the Square Attack~\cite{andriushchenko2020square}, a query-efficient black-box attack. LISArD proposes using white-box attacks against models with the same architecture and data as the target, but without assuming the attacker can access this target model, making our approach more suitable to deal with realistic scenarios.



\textbf{Adversarial Distillation}.
Defensive Distillation (DD)~\cite{papernot2016distillation}, and its extension~\cite{papernot2017extending}, were the first methods to demonstrate the usefulness of distillation to defend against adversarial examples.
Robust Self-Training (RST)~\cite{carmon2019unlabeled} uses a standard supervised approach to obtain pseudo-labels and feed them into another network that targets adversarial robustness.
Adversarially Robust Distillation (ARD)~\cite{goldblum2020adversarially} performs distillation using an adversarially trained network as the teacher.
Introspective Adversarial Distillation (IAD)~\cite{zhu2021reliable} evaluates the robustness of the teacher network considering both the student and teacher labels.
Robust Soft Label Adversarial Distillation (RSLAD)~\cite{zi2021revisiting} uses robust soft labels produced by a teacher network to supervise the student training on natural and adversarial examples.
Low Temperature Distillation (LTD)~\cite{chen2021ltd} considers low temperature in the teacher network and generates soft labels that can be integrated into existing works.
Robustness Critical Fine-Tuning (RiFT)~\cite{zhu2023improving} introduces the module robust criticality metric to fine-tune the less robust modules to adversarial perturbations.
Adaptive Adversarial Distillation (AdaAD)~\cite{huang2023boosting} involves the teacher model in the optimization process by interacting
with the student model to search for the inner
results adaptively.
Information Bottleneck Distillation (IBD)~\cite{kuang2024improving} uses soft-label distillation to increase the mutual information between latent features and predictions and transfers relevant knowledge from the teacher
to the student to reduce the mutual information between the input and latent features.
Fair Adversarial Robustness Distillation (FairARD)~\cite{yue2024revisiting} ensures robust fairness of the student by increasing the weights for naturally more difficult classes.
PeerAiD~\cite{jung2024peeraid} trains a peer network on
the adversarial examples generated for the student network, simultaneously training the student and peer network.
Dynamic Guidance Adversarial Distillation (DGAD)~\cite{park2025dynamic} corrects teacher and student misclassification on clean and adversarially perturbed images. LISArD also does not include additional models during the inference phase, without involving Adversarial Training (AT) and larger previously trained models, thus being a more reliable approach for various domains.

\begin{figure*}[!t]
    \centering
    \includegraphics[width=0.9\linewidth]{imgs/approaches_v2.png}
    \caption{Types of approaches commonly used to defend against adversarial attacks. The Teacher Model refers to a previously trained model, usually bigger than the Student Model, that aids the latter by providing soft labels. The DDPM refers to a Denoising Diffusion Probabilistic Model (a generative model) that uses noise and denoise to produce a ``purified'' image.}
    \label{fig:approaches}
\end{figure*}



\textbf{Adversarial Purification}.
Yoon \textit{et al.}~\cite{yoon2021adversarial} propose using an Energy-Based Model with Denoising Score-Matching to purify perturbed images quickly.
For the first time, diffPure~\cite {nie2022diffusion} uses DDPM to remove the adversarial perturbations from the input images.
Guided Diffusion Model for Adversarial Purification (GDMAP)~\cite{wu2022guided} gradually denoises pure Gaussian noise with guidance to an adversarial image.
APuDAE~\cite{kalaria2022towards} uses Denoising AutoEncoders~\cite{vincent2008extracting} to purify the adversarial examples in an adaptive way, improving the accuracy of target networks.
DensePure~\cite{chen2022densepure} uses different random seeds to get multiple purified images, which are fed to the classifier, and its final prediction is based on majority voting.
Wang \textit{et al.}~\cite{wang2023better} uses better diffusion models~\cite{karras2022elucidating} to demonstrate that higher efficiency and quality diffusion models translate into better robust accuracy.
Lee \textit{et al.}~\cite{lee2023robust} propose a gradual noise-scheduling strategy that improves the robustness of diffusion-based purification.
Feature Purification Network (FePN)~\cite{cao2023fepn} is an adversarial learning mechanism that learns robust features by removing non-robust features from inputs while reconstructing high-quality clean images.
DifFilter~\cite{chen2024diffilter} uses a score-based method to improve the data distribution of the clean samples.
DiffAP~\cite{zhang2024random} uses conditional guidance to ensure prediction consistency between the purified and clean images. 
MimicDiffusion~\cite{song2024mimicdiffusion} approximates the purification process of adversarial examples and clean images by using Manhattan distance and two guidances. Adversarial Purification is the most efficient defense approach for DNNs, but it comes at the cost of high computational resources, while LISArD is able to protect different architectures in various setups without requiring additional training overhead.
\subsection{Motivation and model description} 

We propose a new model, which bridges the gap between the simple SLMM, and the rich, but complicated ELMM. The proposed model, which we call the \textbf{two-step linear mixing model} (2LMM), is a physically motivated model that uses reference EMs extracted from the image or provided in a spectral library.

The 2LMM balances the computational ease of the SLMM and the model complexity of the ELMM. The model is not as complicated as the ELMM, leading to better-posed optimization problems. On the other hand, it is richer than the SLMM, so it can model more diverse scenes.

Using the reference EMs $\E$, the model is constructed as follows. As a first scaling step, the EMs are scaled independently of each other, but in the same way across the entire image. Then, the EMs are  linearly combined to form unscaled pixels. The second scaling step then consists of scaling each mixed pixel independently. See Fig. \ref{fig: 2lmm concept} for a conceptual representation of the 2LMM mixing process.  The 2LMM can still model material-specific variability, but in a more constrained way than the ELMM. 

The 2LMM is constructed with the following acquisition scenario in mind. Assume that reference EMs have been obtained, either from the image itself, or from some spectral library. The acquisition conditions to generate  the reference signatures might differ significantly from the acquisition conditions of  the image, such that the reference EMs will be scaled versions of the actual EMs in the image. If the EEA extracts an EM from a heavily illuminated region, it will have to be scaled to obtain the actual EM in standard conditions. This is especially true when reference EMs are obtained from a spectral library. The first scaling step of the EMs  corrects for this effect. The pixel scaling step  then further corrects for any pixel-wise illumination differences. 

\begin{figure}
    \centering
    \includegraphics[width=\linewidth]{Figs/2LMM.pdf}
    \caption{A graphical representation of the 2LMM model assumption. \textbf{1.} The reference EMs (blue lines) are scaled independently (red lines). \textbf{2.} The scaled EMs are mixed to form the unscaled pixels in the image. \textbf{3.} Each pixel is scaled independently to form the final image.}
    \label{fig: 2lmm concept}
\end{figure}

\subsection{Mathematical formulation}

Let $\s_\E \in \real^K$ be the EM scaling vector representing the first scaling step, and $s_{\x_n}$ a pixel-dependent scaling factor representing the second scaling step. Then the $n$-th pixel in the 2LMM is given by:
\begin{equation*}
    \x_n = \E\mathrm{diag}(\s_\E)\ba_n s_{\x_n}.
\end{equation*}
We can combine this for all pixels. Let 
\[
\s_\X = [s_{\x_1}~s_{\x_2}~\cdots~s_{\x_N}]^\top
\]
denote the pixel scaling vector. Then the 2LMM at the image level is given by:

\begin{equation} \label{eq: 2lmm cost function}
    \X = \E\mathrm{diag}(\s_\E)\A\mathrm{diag}(\s_\X).
\end{equation}



\subsection{Performing unmixing with the 2LMM}
Consider a non-convex FCLSU problem for the 2LMM:
\begin{equation}
\begin{aligned}
    \min_{\A, \s_\E, \s_\X} &~\|\Hat{\X} - \E \diag(\s_\E)\A\diag(\s_\X)\|_F^2 \\
    \text{s.t.} &~ 0 \leq \A \leq 1, \mathbf{1}^\top \A = \mathbf{1} \\
    &~ \underline{S} \leq \s_\E \leq \overline{S}, \quad \underline{S} \leq \s_\X \leq \overline{S}
\end{aligned}
\end{equation}
where the box bounds $\underline{S}, \overline{S} > 0$ can be used to constrain the scaling variables to a user-specified interval. Naturally, $\underline{S} < \overline{S}$. The motivation for introducing these box bounds is both physical and mathematical. First, it allows us to constrain the scaling factors to a physically meaningful range, since in many cases credible assumptions can be made about the magnitude of the scaling factors. Secondly, it makes the problem easier to solve mathematically, since it reduces the size of the search space, and therefore reduces the probability of finding a sub-optimal solution to the non-convex cost function.

The quantity in which we are interested is the abundance matrix $\A$. The estimates of the scaling variables $\s_\E$ and $\s_\X$ are less important. However, this increase in variables might lead to worse optimization results because the problem becomes more complex. Therefore, we propose the following optimization strategies to (partially) avoid the need to estimate the scaling factors.

\subsubsection{Scaling-independent optimization}

The cost function for the model (\ref{eq: 2lmm cost function}) can be written as a sum over the different pixels:
\[
J(\A, \s_\E, \s_\X) = \sum_{n=1}^N \| \hat{\x}_n - \E \diag(\s_\E) \ba_n s_{\x_n}\|_2^2.
\]
To remove the need to estimate the pixel scaling factors, we may now divide both terms by their norm, to obtain the norm-divided cost function:
\begin{equation} \label{eq: norm division cost function}
    \Tilde{J}(\A, \s_\E) = \sum_{n=1}^N \left\| \frac{\hat{\x}_n}{\|\hat{\x}_n\|_2} - \frac{\E \diag(\s_\E) \ba_n}{\|\E \diag(\s_\E) \ba_n\|_2}\right\|_2^2
\end{equation}
This cost function defines the \textit{norm division approach}:
\begin{equation} \label{eq: norm division}
   (\mathrm{2LMM}_\mathrm{norm}):~
   \begin{aligned}
        \min_{\A, \s_\E} &~\sum_{n=1}^N \left\| \frac{\hat{\x}_n}{\|\hat{\x}_n\|_2} - \frac{\E \diag(\s_\E) \ba_n}{\|\E \diag(\s_\E) \ba_n\|_2}\right\|_2^2 \\
    \text{s.t.} &~ 0 \leq \A \leq 1,~\mathbf{1}^\top \A = \mathbf{1}  \\
    &~ \underline{S} \leq \s_\E \leq \overline{S}
\end{aligned}
\end{equation}
By normalizing the two terms in Eq. (\ref{eq: norm division cost function}), we discard the length of the vectors in order to minimize the \textit{angle} between them. We can do this more explicitly. Using the standard inner product in $\real^D$:
\[
\langle \mathbf{u}, \mathbf{v} \rangle = \sum_{d=1}^D u_d v_d = \mathbf{u}^\top \mathbf{v},
\]
the angle between two vectors $\mathbf{u}$ and $\mathbf{v}$ is given by:
\[
\angle(\mathbf{u},\mathbf{v}) = \arccos \left( \frac{\mathbf{u}^\top \mathbf{v}}{\|\mathbf{u}\|_2 \|\mathbf{v}\|_2}\right).
\]
This suggests using the following cost function:
\begin{equation}
J(\A, \s_\E) = \sum_{n=1}^N \arccos \left( \frac{(\E \diag (\s_\E) \ba_n)^\top \hat{\x}_n}{\|\E \diag (\s_\E) \ba_n)\|_2 \|\hat{\x_n}\|_2 }\right)    
\end{equation}
where the factor $s_{\x_n}$ is canceled because it is a scalar. The arc cosine makes this a highly nonlinear function. By removing the arc cosine, the negative of the argument needs to be minimized (since the derivative of the arc cosine is always negative). Therefore we define the new cost function as:
\begin{equation} \label{eq: angle cost function}
    \Tilde{J}(\A, \s_\E) = - \sum_{n=1}^N \frac{(\E \diag (\s_\E) \ba_n)^\top \hat{\x}_n}{\|\E \diag (\s_\E) \ba_n)\|_2 \|\hat{\x}_n\|_2 }
\end{equation}
and the corresponding optimization problem, which we will call the \textit{angle approach}:
\begin{equation}
    (\mathrm{2LMM}_\mathrm{angle}):~
    \begin{aligned}
            \min_{\A, \s_\E} &~- \sum_{n=1}^N \frac{(\E \diag (\s_\E) \ba_n)^\top \hat{\x}_n}{\|\E \diag (\s_\E) \ba_n)\|_2 \|\hat{\x}_n\|_2 } \\
            \text{s.t.} &~ 0 \leq \A \leq 1,~\mathbf{1}^\top \A = \mathbf{1}  \\
                        &~ \underline{S} \leq \s_\E \leq \overline{S}.
    \end{aligned}
\end{equation}
Conceptually, this makes the optimization problem simpler, since we reduce the parameter space by $N$ dimensions. However, there are some drawbacks, especially from a computational perspective. Since both approaches have variables in both the numerator and denominator, every function evaluation will involve a very expensive division operation. When a division is performed on a computer, it involves an iterative process of subtractions and comparisons, which is notoriously slow \cite[Ch. 3.4]{patterson_computer_2017} and can produce considerable numerical errors. Additionally, this can lead to excessive memory requirements. Therefore, we propose a third approach, which avoids expensive divisions, and is inspired by CLSU.
\subsubsection{Two scaling factor approach}
A third approach combines the matrices $\A$ and $\s_\X$ of the cost function (\ref{eq: 2lmm cost function}) into one matrix $\A_\s$, and drops the ASC. This leads to the optimization problem
\begin{equation} \label{eq: two scaling factor}
  (\mathrm{2LMM}):~ \begin{aligned}
        \min_{\A_\s, \s_\E} &~\|\Hat{\X} - \E \diag(\s_\E)\A_\s\|_F^2 \\
    \text{s.t.} &~ 0 \leq \A_\s \leq \overline{S},  \\
    &~ \underline{S} \leq \s_\E \leq \overline{S}
\end{aligned}
\end{equation}
The actual abundances and pixel scaling factors are easily recovered using the normalization step (\ref{eq: normalization}). We will refer to this approach by the model name, 2LMM, or by calling it the \emph{two scaling factor approach}.

\subsection{Optimization algorithm}
To perform spectral unmixing using the 2LMM, we will use an interior-point (IP) method. Interior-point methods can solve general nonlinear constrained minimization problems \cite{geletu_introduction_nodate}. 

\subsubsection{Concept of IP methods}
In what follows we give a high-level overview of the IP method. Formal assumptions and discussions which are not essential to understanding the concept of the algorithm are deferred to the appendices. Consider the general problem:
\begin{equation} \label{eq: interior point problem}
   \begin{aligned}
    \min_{\x \in \real^D} &~f(\x) \\ \mathrm{s.t.}&~g_i(\x) \geq 0,  &i=1,2,\ldots, I \\ &~\mathbf{C}\x = \mathbf{b}\\ &~\x\geq0
    \end{aligned} 
\end{equation}
where $f$ need not be convex. Furthermore, define the feasible set $\mathcal{X}$ as the set of all points that satisfy the constraints:
\[
\mathcal{X} = \left\{ \x \in \real^D \Bigg| \begin{array}{c}
     g_i(\x) \geq 0,~i=1,2,\ldots, I \\ \mathbf{C}\x = \mathbf{b}\\ \x \geq 0
\end{array}\right\}.
\]
 The main idea of the IP method is to replace  the inequality constraints in the cost function with penalty terms  that approach infinity as the argument approaches the boundary of the feasible set, and that are very small when the argument falls within the feasible set. For the IP algorithm to work, we require the feasible set $\mathcal{X}$ to be \textit{large enough} (see App. A for a formal assumption), so that it allows the definition of a \textbf{barrier function}, in this case a log barrier function:
\[
B(\x, \mu) = f(\x) - \mu \left( \sum_{i=1}^I \log(g_i(\x)) + \sum_{d = 1}^D \log(x_d)\right)
\]
A {barrier function} is a function that approaches $f(\x)$ as $\mu$ decreases but still approaches $+\infty$ as $\x$ approaches the boundary of the feasible set (see Fig. \ref{fig:barrier fct}). Instead of the original problem (\ref{eq: interior point problem}), one can now consider the problem:
\begin{equation} \label{eq: barrier problem}
\begin{aligned}
    \min_{\x \in \real^D} &~B(\x, \mu) \\
    \mathrm{s.t.}&~\mathbf{C}\x = \mathbf{b}.
\end{aligned}
\end{equation}

\begin{figure}
    \centering
    \includegraphics[width=0.9\linewidth]{Figs/barrier_fct.jpg}
    \caption{A barrier function for $f(x) = 2(x - 0.7)^2$ and $\mathcal{X} = [0, 1]$. As the value of $\mu$ decreases, the barrier function approaches $f(x)$ while still approaching infinity at the boundaries.}
    \label{fig:barrier fct}
\end{figure}

The interior-point method is an extension to the barrier method, where the problem (\ref{eq: barrier problem}) is solved several times using Newton's method for a decreasing sequence of nonnegative numbers $\{\mu_k\}_{k \geq 0}$. This way, we obtain a sequence $\{(\x_{\mu_k}, \bm{\lambda}_{\mu_k})\}_{k \geq 0}$ of optimal solutions, where $\bm{\lambda}_k$ is the dual variable or Lagrange multiplier (see App. B). This sequence is called the \emph{primal-dual path}. Under mild assumptions, the primal-dual path converges to a locally optimal solution of the problem (\ref{eq: interior point problem}) (see App. C for a more elaborate convergence discussion). The full conceptual algorithm is shown in Algorithm \ref{alg: interior point}. For more information on interior-point and related methods, see \cite[Ch. 11]{boyd_convex_2004}.

\begin{algorithm}[t]
\caption{Interior-point algorithm}\label{alg: interior point}
\begin{algorithmic}
\FOR {$k = 0, 1, 2, \ldots$}
\STATE {Construct the barrier function $B(\x, \mu_k)$}
\STATE {Solve the problem (\ref{eq: barrier problem}) using, e.g., Newton's method}
\STATE {Call the solutions $(\x_{\mu_k}, \bm{\lambda}_{\mu_k})$}
\IF {some termination criterion is met}
\STATE{Terminate and return $\x^\star = \x_{\mu_k}$}
\ENDIF
\ENDFOR
\end{algorithmic}
\end{algorithm}



\subsubsection{The IP method for 2LMM}
Consider the two scaling factor approach (\ref{eq: two scaling factor}), with the cost function
\[
J(\A_\s, \s_\E) = \|\hat{\X} - \E \diag(\s_\E)\A_\s\|_F^2
\]
and the constraints $0 \leq \A_\s \leq \overline{S}$ and $\underline{S} \leq \s_\E \leq \overline{S}$. There are no equality constraints. The inequality constraints can be written as:
\begin{equation}
    \begin{aligned}
    s_{\e_k} - \underline{S} & \geq 0, \quad \overline{S} - s_{\e_k} \geq 0, & \quad \forall k  \\
    a_{nk} & \geq 0, \quad  \overline{S} - a_{nk} \geq 0, & \quad \forall n, k
    \end{aligned}
\end{equation}
leading to the barrier function
\begin{multline*}
    B(\A_\s, \s_\E, \mu) = J(\A_\s, \s_\E) -   \\
    \mu \Big(\sum_{k=1}^K (\log (s_{\e_k} - \underline{S}) + \log (\overline{S} - s_{\e_k})) + \\
    \sum_{k=1}^K \sum_{n=1}^N (\log (\overline{S} - a_{nk}) + \log (a_{nk}))\Big).
\end{multline*}
We have omitted the terms $\log (s_{\e_k})$ since they are redundant. The barrier function for the other approaches is similar, and given by
\begin{multline*}
    \Tilde{B}(\A, \s_\E, \mu) = \Tilde{J}(\A, \s_\E) -   \\
    \mu \Big(\sum_{k=1}^K (\log (s_{\e_k} - \underline{S}) + \log (\overline{S} - s_{\e_k})) + \\
    \sum_{k=1}^K \sum_{n=1}^N (\log (1 - a_{nk}) + \log (a_{nk}))\Big)
\end{multline*}
with $\Tilde{J}(\A, \s_\E)$ either the norm division cost function (\ref{eq: norm division cost function}) or the angle cost function (\ref{eq: angle cost function}). To improve the numerical behavior of the algorithm, slack variables are often introduced in the barrier function (see App. D for the modified cost function in case of the two scaling factor approach). 

\subsubsection{Implementation}
Several general-purpose software implementations of the interior-point algorithm exist. For the two scaling factor approach 2LMM, we use the Ipopt solver \cite{wachter_line_2005} interfaced through the Julia programming language. For the norm division approach $\mathrm{2LMM}_\mathrm{norm}$ and the angle approach $\mathrm{2LMM}_\mathrm{angle}$, we use \textsc{Matlab}'s interior-point solver, implemented in the $\texttt{lsqnonlin}$ and $\texttt{fmincon}$ functions from the \textsc{Matlab} Optimization Toolbox. All experiments were run on a desktop computer with a 32-core Intel i9 CPU with a 3-level cache and 64 GiB RAM (DIMM).
\section{Experiments: Planning outperforms Heuristics}
\label{sec:experiment}

We begin our empirical demonstrations by showcasing the effectiveness of our planning framework on both synthetic and real datasets. We focus on the simplest planning algorithm, 1-step lookaheads (Algorithm~\ref{alg:complete}), and show that even basic planning can hold great promise. 
We illustrate our framework using two uncertainty quantification modules---GPs and 
\ensembles/ \ensembleplus. 

Throughout this section, we focus on evaluating the mean squared error of 
a regression model $\model$,  and develop adaptive policies that minimize uncertainty on $g(f)$ defined in~\eqref{eqn:l2-g-f}.
When GPs provide a valid model of uncertainty, 
our experiments show that our planning framework significantly outperforms other baselines. 
We further demonstrate that our conceptual framework extends to deep learning-based uncertainty quantification methods such as  \ensembleplus while highlighting computational challenges that need to be resolved in order to scale our ideas. 
For simplicity, we assume a naive predictor, i.e., $\psi(\cdot) \equiv 0$. However, we emphasize that this problem is just as complex as if we were using a sophisticated model $\psi(.)$. The performance gap between the algorithms 
primarily depends
on the level  of uncertainty in our prior beliefs.

To evaluate the performance of our algorithm, we benchmark it against several baselines. 
%Active learning baselines use an acquisition function $\ac$ to select points that have the highest   function value: $X\opt_t \in \argmax_{X \in \xpoolj{t}} \ac({X})$ at every step $t$. These methods may also need an UQ module, which we simply use the same UQ module as in our algorithm, and it  outputs $V(X)$ that measures the the uncertainty of each point $X \in \xpoolj{t}$.
Our first set of baselines are from active learning~\citep{AggarwalKoGuHaPh14}:
\\ % \noindent\textbf{Active Learning Heuristics:} 
\textbf{(1)} 
\textsf{Uncertainty Sampling (Static):}  In this approach, we query the samples for which the model is least certain about. Specifically, we estimate the variance of the latent output $f(X)$ for each $X \in \xpool$ using the UQ module and select the top-$K$ points with the highest uncertainty. \\
\textbf{(2)} \textsf{Uncertainty Sampling (Sequential):} This is a greedy heuristic that sequentially selects the points with the highest uncertainty within a batch, while updating the posterior beliefs using pseudo labels from the current posterior state. Unlike \textsf{Uncertainty Sampling (Static)}, this method takes into account the information gained from each point within batch, and hence tries to diversify the selected points within a batch. 

 
We also compare our approach to the  \textbf{(3)} \textsf{Random Sampling}, which selects each batch uniformly at random from the pool. Additionally, we compare solving the planning problem using  \textsf{REINFORCE}-based policy gradients with   $\mathsf{Smoothed\text{-}Autodiff}$ policy gradients.\footnote{Our code repository is available at
  \url{https://github.com/namkoong-lab/adaptive-labeling}.}
%Detailed experimental setups are provided in Section \ref{sec:details-experiments}.

%We repeat all experiments with 10 random seeds.




\begin{figure}[t]
\centering
\begin{minipage}[b]{0.49\textwidth}
\centering
\includegraphics[width=\textwidth, height=5cm]{figures/original_scale/Var_of_l_2_loss.pdf}
\caption{(Synthetic data) Variance of mean squared loss evaluated through the posterior belief $\mu_t$ at each horizon $t$. This is the objective that policy gradient methods like \textsf{REINFORCE} and $\ouralgo$ optimizes. 1-step lookaheads are surprisingly effective even in long horizons.}
\label{fig:var-l2-sim}
\end{minipage}
\hfill
\begin{minipage}[b]{0.49\textwidth}
\centering \includegraphics[width=\textwidth, height=5cm]{figures/original_scale/Error_of_estimated_model_l_2_loss.pdf}
\caption{(Synthetic data) Error between MSE calculated based on collected data $\mc{D}^{0:T}$ vs. population oracle MSE over $\mc{D}_{\rm eval} \sim P_X$. Reducing uncertainty over posteriors directly leads to better OOD evaluations. 1-step lookaheads significantly outperform active learning heuristics in small horizons.}
\label{fig:mean-l2-sim}
\end{minipage}
%\caption{Simulated data for GPs}
%\label{fig:both_plots}
\end{figure}

\subsection{Planning with Gaussian processes}
\label{sec:experiment-plan-GP}
We now briefly describe the data generation process for the GP experiments,  deferring a more detailed discussion of the dataset generation to Section~\ref{sec:details-experiments}. 
We use both the synthetic data and the real data to test our methodology.
For the \emph{simulated data},  we construct a setting where the general population is distributed across \emph{51 non-overlapping clusters} while the initial labeled data $\dtrain$ just comes from one cluster. In contrast, both $\dpool \defeq (\xpool,\ypool),\deval \defeq (\xeval,\yeval)$ are generated   from all the clusters. 
We begin with a low-dimensional scenario, generating a one-dimensional regression setting using a GP. %Gaussian Process (GP).
Although the data-generating process is not known to the algorithms,  we assume that the GP hyperparameters are known to all the algorithms
to ensure fair comparisons. This can be viewed as a setting where our prior is well-specified, allowing us to isolate the effects
of different policy optimization approaches
 without any concerns about the misspecified priors. We select $10$ batches, each of size $K=5$ across $T = 10$ time horizons.

To examine the robustness of our method against the distributional assumptions made  in the simulated case, we then move to a real dataset where the correct prior is not known. We simulate selection bias from the eICU dataset~\citep{PollardJoRaCeMaBa18}, which contains real-world patient data with in-hospital mortality outcomes. 
We conduct a $k$-means clustering to generate 51 clusters and then select data from those clusters. We view this to be a credible replication of practice, as severe distribution shifts are common due to selection bias in clinical labels.  To convert the binary mortality labels into a regression setting, we train a  random forest classifier and fit a GP on predicted scores, which serves as the UQ module for all the algorithms. As before, the task is to select 10 batches, each consisting of 5 samples, across 10 time horizons.

 In Figures~\ref{fig:var-l2-sim} and~\ref{fig:mean-l2-sim}, we present results for the simulated data. 
Figure~\ref{fig:var-l2-sim} shows the variance of $\ell_2$ loss, and Figure~\ref{fig:mean-l2-sim} presents the error in the estimated $\ell_2$ loss using $\mu_t$ (relative to true $\ell_2$ loss, that is unknown to the algorithm). 
As we can see from these plots, our method one-step lookahead  gives substantial improvements  over active learning baselines and random sampling. In addition,
compared to the one-step lookahead planning approach using \textsf{REINFORCE}-based policy gradients, 
we observe that $\mathsf{Smoothed\text{-}Autodiff}$-based policy gradients provide significantly more robust performance over all horizons.

In Figures~\ref{fig:var-l2-real}~and~\ref{fig:mean-l2-real}, we observe similar findings on the eICU data. We see that planning policies (\textsf{REINFORCE} and $\mathsf{Smoothed\text{-}Autodiff}$) consistently outperform other heuristics by a large margin.  Active learning baselines perform poorly in these small-horizon batched problems and can sometimes be even worse than the random search baselines.  Overall, our results show the importance of careful planning in adaptive labeling for reliable model evaluation. 

We offer some intuition as to why one-step lookahead planning may outperform other heuristic algorithms. 
 First,  \textsf{Uncertainty sampling (Static)} while myopically selects the
 top-$K$ inputs with the highest uncertainty, it fails to consider 
the overlap in information content among the ``best” instances; see \citep{AggarwalKoGuHaPh14} for more details. 
In other words,  it might acquire points from the same region with high uncertainty while failing to induce diversity among the batch.
Although \textsf{Uncertainty Sampling (Sequential)} somewhat addresses the issue of information overlap, a significant drawback of 
this algorithm
is the disconnect between the objective we aim to optimize and the algorithm. For example, it might sample from a region with high uncertainty but very low density. 

\begin{figure}[t]
\centering
\begin{minipage}[b]{0.48\textwidth}
\centering
\includegraphics[width=\textwidth, height=5cm]{figures/original_scale/Var_of_l_2_loss_real.pdf}
\caption{(Real-world eICU data) Variance of mean squared loss evaluated through the posterior belief $\mu_t$ at each horizon $t$. Even 1-step lookaheads are extremely effective planners, and auto-differentiation-based pathwise policy gradients provide a reliable optimization algorithm based on low-variance gradient estimates.}
\label{fig:var-l2-real}
\end{minipage}
\hfill
\begin{minipage}[b]{0.48\textwidth}
\centering \includegraphics[width=\textwidth, height=5cm]{figures/original_scale/Error_of_estimated_model_l_2_loss_real.pdf}
\caption{(Real-world eICU data) Error between MSE calculated based on collected data $\mc{D}^{0:T}$ vs. population oracle MSE over $\mc{D}_{\rm eval} \sim P_X$. Reducing uncertainty over posteriors directly leads to better OOD evaluations. Our method significantly outperforms active learning-based heuristics, and random sampling.}
\label{fig:mean-l2-real}
\end{minipage}
%\caption{Real data for GPs}
\end{figure}
 
%\vspace{-1.5cm}
% \begin{wrapfigure}{r}{.32\columnwidth}
%   \vspace{-.5cm} 
%   \centering
% \includegraphics[scale=.29]{figures/Var of l2l_2 loss.pdf}
%   \vspace{-0.2cm}
%   \caption{Results of GP}
% \label{fig:var-l2-gp}
%   \vspace{-0.1cm}
% \end{wrapfigure}


% Attempts have been made  in the past to address these  drawbacks heuristically  (see \citep{AggarwalKoGuHaPh14}). We give a unified computational framework while approaching the problem in a more principled manner and solving it more optimally.




\subsection{Planning with  neural network-based uncertainty quantification methods ($\ensembleplus$)}


We now provide a proof-of-concept that shows the generalizability of our conceptual framework  to the deep learning-based UQ modules, specifically focusing on $\ensembleplus$ due to their previously observed superior performance~\citep{OsbandWenAsDwIbLuRo23}. Recall that implementing our framework with deep learning-based UQ modules  requires us to retrain the model across multiple possible random actions $\bm{a}(\theta)$ sampled from the current policy $\pi_\theta$.
This requires significant computational resources, in sharp contrast to the GPs where the posteriors are in closed form and can be readily updated and differentiated. 

Due to the computational constraints, we test $\ensembleplus$ on a toy setting to demonstrate the generalizability of our framework. We consider a setting where the general population consists of four clusters, while the initial labeled data only comes from one cluster. Again we generate data using GPs.  The task is to select a batch of 2 points in one horizon. We detail the $\ensembleplus$ architecture in Section \ref{sec:details-experiments}, and we assume prior uncertainty to be large (depends on the scaling of the prior generating functions). 
The results are summarized in the Table~\ref{tab:UQ_ensemble}.

% \begin{table}[H]
% \vspace{-10pt}
% \caption{Performance under \ensembleplus as UQ module}
%     \centering
%     \begin{tabular}{|m{3cm}|m{2.5cm}|m{2cm}|} 
%     \hline
%       Algorithm   & Variance of $\loss_2$ loss estimate & Error of $\loss_2$ loss estimate  \\ \hline Random Sampling 
%          & $1710.9 \pm 1352.1$ & $8.67\pm6.62$ 
%       \\ \hline \ouralgo & $1.30 \pm 0.68$ & $0.91\pm0.25$ \\ \hline
%     \end{tabular}
%     \label{tab:UQ_ensemble}
%     %\vspace{-10pt}
% \end{table}




\begin{table}[h]
\vspace{-10pt}
\caption{Performance under \ensembleplus as the UQ module}
\centering
\begin{tabular}{|l|l|l|}
\hline
Algorithm   & Variance of $\loss_2$ loss estimate & Error of $\loss_2$ loss estimate  \\
\hline
\textsf{Random sampling} & 7129.8 $\pm$ 1027.0 & 136.2 $\pm$ 8.28 \\ \hline
\textsf{Uncertainty sampling (Static)} & 10852 $\pm$ 0.0 & 162.156 $\pm$ 0.0 \\ \hline
\textsf{Uncertainty sampling (Sequential)} & 8585.5 $\pm$ 898.9 & 144 $\pm$ 6.93 \\ \hline
\textsf{REINFORCE} & 1697.1 $\pm$ 0.0 & 45.27 $\pm$ 0.0 \\ \hline
\ouralgo & 1697.1 $\pm$ 0.0 & 45.27 $\pm$ 0.0 \\ \hline
\end{tabular}
%\caption{Comparison of different algorithms based on variance   and   error in $\ell_2$ loss estimation with Ensemble $+$ as the UQ module. Our results demonstrate that {\ouralgo} and REINFORCE outperformthe other active learning based heuristics, confirming the benefits of our MDP formulation for the adaptive labeling problem, as also demonstrated in Section 4.\\
%\footnotesize{Experimental details: We use Gaussian Processes as our data generating process, GP parameters are the same as in Section D.3.  The task is to select a batch of 2 points along one horizon.The marginal distribution $p_X$ has 4 \textit{non-overlapping} clusters. Initial data comes from one cluster, while pool and evaluation points comes from all the clusters. We have $20$ initial labeled data points, $10$ pool points, and $252$ evaluation points.  Training procedures are similar to the one in Section D.3.} }
\label{tab:UQ_ensemble}
\end{table}



% We faced  issues in scaling up these experiments which will be our focus in the future. 





% \begin{itemize}
%     \item Posteriors should be consistent. Two dimensions: even with less training,  
%     \item the inference should be  fast enough
% \end{itemize}


% Potential research directions for uncertainty quantification

% In this section we consider a simple setting We consider a simpler setting and 


% For synthetic dataset generation, we use ...... For real datasets, we use ...... We compare our methodolgy to several baselines ()    This Section is structured as follows:
% \begin{itemize}
%     \item \textbf{GPs, square loss objective} (Section \ref{}): 
%     %the broad aim of the experiments  in this section is to isolate the performance of our methodology without any concerns for the inefficiencies induced due to a mis-specified prior or imperfect posterior inference. To accomplish this we generate synthetic datasets using GPs (detailed later). We use the well specified prior (GPs - with same hyperparameter setting) as our UQ module.   
%      As GPs provide differentaible posterior inference - any errors induced due to imperfect posterior updates are also isolated. We note that under this setting
%      \item In Section\ref{} we demonstrate why our methodology performs better than other baselines - by devising various synthetic experiments ()
%     \item  \textbf{UQ Benchmarking }(Section \ref{}): Before diving into the experiments using $\ensembleplus$ and ENNs,  we showcase our benchmarking experiments in Section \ref{}. We use real datasets We observe that ENNs perform better
%      \item \textbf{Ensemble $+$}, objective: recall, accuracy
%     \item \textbf{ENN}, objective: recall, accuracy
% \end{itemize}




% In Section {}, we test 
% \subsection{Experimental details}

% \begin{itemize}
%     \item UQ methodologies - GPs, ENNs
%     \item Objectives - Recall,  ATE
%     \item Datasets - ATE-synthetic datasets, Recall-synthetic, real datasets
%     \item Baselines - 
%     \begin{itemize}
%         \item Random sampling
%         \item Active learning - Uncertainty based sampling - In regression setting almost all of the 
%         \item Myopic greedy - Greedy Batch based sampling
%         \item Policy Gradient
%     \end{itemize}
    
% \end{itemize}

% \subsection{Experiments}
%     \begin{itemize}
%     \item GPs with square loss
%     \item Benchmarking ENN
%         \item ENNs with ATE
%         \item ENNs with Recall
%     \end{itemize}

% \subsection{Benefits over other algorithms - intuition and experiments}

%Active learning - Myopic greedy / Don't rely on the objective rather some entropy version.


%%% Local Variables:
%%% mode: latex
%%% TeX-master: "main"
%%% End:



\section{Conclusion}
\label{sec:conclusion}

\noindent This paper describes an evaluation framework based on the \emph{gray-box setting} that is more realistic than the typically used \emph{white-box} scenario, where the existing models do not perform reliably. We propose an adversarial defense mechanism for this setting (LISArD), which is simultaneously robust against white-box attacks, and does not depend on the inclusion of adversarial samples. This mechanism uses image similarity to instruct the model to recognize that images pairs regard the same object, while simultaneously inferring class information. The experiments show the vulnerability of pre-trained and scratch-trained networks to gray-box adversarial samples and point to the effectiveness of LISArD in increasing resilience against this type of samples. Also, state-of-the-art \textit{Adversarial Distillation} models cannot perform in white-box settings without the inclusion of AT. In the future, the injection of other types of noise (\textit{e.g.}, using fractional Gaussian noise with persistence, instead of white noise) will be subject of our further analysis. Finally, for realism purposes, we suggest evaluating the models in scenarios in which the attacker only knows the training data and the type of architecture.



% Generated by IEEEtran.bst, version: 1.14 (2015/08/26)
\begin{thebibliography}{10}
\providecommand{\url}[1]{#1}
\csname url@samestyle\endcsname
\providecommand{\newblock}{\relax}
\providecommand{\bibinfo}[2]{#2}
\providecommand{\BIBentrySTDinterwordspacing}{\spaceskip=0pt\relax}
\providecommand{\BIBentryALTinterwordstretchfactor}{4}
\providecommand{\BIBentryALTinterwordspacing}{\spaceskip=\fontdimen2\font plus
\BIBentryALTinterwordstretchfactor\fontdimen3\font minus \fontdimen4\font\relax}
\providecommand{\BIBforeignlanguage}[2]{{%
\expandafter\ifx\csname l@#1\endcsname\relax
\typeout{** WARNING: IEEEtran.bst: No hyphenation pattern has been}%
\typeout{** loaded for the language `#1'. Using the pattern for}%
\typeout{** the default language instead.}%
\else
\language=\csname l@#1\endcsname
\fi
#2}}
\providecommand{\BIBdecl}{\relax}
\BIBdecl

\bibitem{thirunavukarasu2023large}
A.~J. Thirunavukarasu, D.~S.~J. Ting, K.~Elangovan, L.~Gutierrez, T.~F. Tan, and D.~S.~W. Ting, ``Large language models in medicine,'' \emph{Nature medicine}, vol.~29, no.~8, pp. 1930--1940, 2023.

\bibitem{patricio2023coherent}
C.~Patr{\'\i}cio, J.~C. Neves, and L.~F. Teixeira, ``Coherent concept-based explanations in medical image and its application to skin lesion diagnosis,'' in \emph{Proceedings of the IEEE/CVF Conference on Computer Vision and Pattern Recognition}, 2023, pp. 3799--3808.

\bibitem{touvron2023llama2openfoundation}
\BIBentryALTinterwordspacing
H.~Touvron and et~al., ``Llama 2: Open foundation and fine-tuned chat models,'' 2023. [Online]. Available: \url{https://arxiv.org/abs/2307.09288}
\BIBentrySTDinterwordspacing

\bibitem{costa2022predicting}
J.~C. Costa, T.~Roxo, J.~B. Sequeiros, H.~Proenca, and P.~R. Inacio, ``Predicting cvss metric via description interpretation,'' \emph{IEEE Access}, vol.~10, pp. 59\,125--59\,134, 2022.

\bibitem{roxo2023exploring}
T.~Roxo, J.~C. Costa, P.~R. In{\'a}cio, and H.~Proen{\c{c}}a, ``On exploring audio anomaly in speech,'' in \emph{2023 IEEE International Workshop on Information Forensics and Security (WIFS)}.\hskip 1em plus 0.5em minus 0.4em\relax IEEE, 2023, pp. 1--6.

\bibitem{roxo2024bias}
------, ``Bias: A body-based interpretable active speaker approach,'' \emph{IEEE Transactions on Biometrics, Behavior, and Identity Science}, 2024.

\bibitem{roxo2024asdnb}
T.~Roxo, J.~C. Costa, P.~In{\'a}cio, and H.~Proen{\c{c}}a, ``Asdnb: Merging face with body cues for robust active speaker detection,'' \emph{arXiv preprint arXiv:2412.08594}, 2024.

\bibitem{szegedy2014intriguing}
C.~Szegedy, W.~Zaremba, I.~Sutskever, J.~Bruna, D.~Erhan, I.~Goodfellow, and R.~Fergus, ``Intriguing properties of neural networks,'' in \emph{2nd International Conference on Learning Representations, ICLR 2014}, 2014.

\bibitem{papernot2016distillation}
N.~Papernot, P.~McDaniel, X.~Wu, S.~Jha, and A.~Swami, ``Distillation as a defense to adversarial perturbations against deep neural networks,'' in \emph{2016 IEEE symposium on security and privacy (SP)}.\hskip 1em plus 0.5em minus 0.4em\relax IEEE, 2016, pp. 582--597.

\bibitem{goldblum2020adversarially}
M.~Goldblum, L.~Fowl, S.~Feizi, and T.~Goldstein, ``Adversarially robust distillation,'' in \emph{Proceedings of the AAAI conference on artificial intelligence}, vol.~34, no.~04, 2020, pp. 3996--4003.

\bibitem{zhu2021reliable}
J.~Zhu, J.~Yao, B.~Han, J.~Zhang, T.~Liu, G.~Niu, J.~Zhou, J.~Xu, and H.~Yang, ``Reliable adversarial distillation with unreliable teachers,'' in \emph{International Conference on Learning Representations}, 2021.

\bibitem{yoon2021adversarial}
J.~Yoon, S.~J. Hwang, and J.~Lee, ``Adversarial purification with score-based generative models,'' in \emph{International Conference on Machine Learning}.\hskip 1em plus 0.5em minus 0.4em\relax PMLR, 2021, pp. 12\,062--12\,072.

\bibitem{wu2022guided}
Q.~Wu, H.~Ye, and Y.~Gu, ``Guided diffusion model for adversarial purification from random noise,'' \emph{arXiv preprint arXiv:2206.10875}, 2022.

\bibitem{chen2022densepure}
Z.~Chen, K.~Jin, J.~Wang, W.~Nie, M.~Liu, A.~Anandkumar, B.~Li, and D.~Song, ``Densepure: Understanding diffusion models towards adversarial robustness,'' in \emph{Workshop on Trustworthy and Socially Responsible Machine Learning, NeurIPS 2022}, 2022.

\bibitem{ho2020denoising}
J.~Ho, A.~Jain, and P.~Abbeel, ``Denoising diffusion probabilistic models,'' \emph{Advances in neural information processing systems}, vol.~33, pp. 6840--6851, 2020.

\bibitem{katzir2020gradients}
Z.~Katzir and Y.~Elovici, ``Gradients cannot be tamed: Behind the impossible paradox of blocking targeted adversarial attacks,'' \emph{IEEE Transactions on Neural Networks and Learning Systems}, vol.~32, no.~1, pp. 128--138, 2020.

\bibitem{goodfellow2015explaining}
I.~J. Goodfellow, J.~Shlens, and C.~Szegedy, ``Explaining and harnessing adversarial examples,'' \emph{stat}, vol. 1050, p.~20, 2015.

\bibitem{papernot2016limitations}
N.~Papernot, P.~McDaniel, S.~Jha, M.~Fredrikson, Z.~B. Celik, and A.~Swami, ``The limitations of deep learning in adversarial settings,'' in \emph{2016 IEEE European symposium on security and privacy (EuroS\&P)}.\hskip 1em plus 0.5em minus 0.4em\relax IEEE, 2016, pp. 372--387.

\bibitem{tramer2017space}
F.~Tram{\`e}r, N.~Papernot, I.~Goodfellow, D.~Boneh, and P.~McDaniel, ``The space of transferable adversarial examples,'' \emph{stat}, vol. 1050, p.~23, 2017.

\bibitem{croce2019sparse}
F.~Croce and M.~Hein, ``Sparse and imperceivable adversarial attacks,'' in \emph{Proceedings of the IEEE/CVF international conference on computer vision}, 2019, pp. 4724--4732.

\bibitem{moosavi2016deepfool}
S.-M. Moosavi-Dezfooli, A.~Fawzi, and P.~Frossard, ``Deepfool: a simple and accurate method to fool deep neural networks,'' in \emph{Proceedings of the IEEE conference on computer vision and pattern recognition}, 2016, pp. 2574--2582.

\bibitem{dabouei2020smoothfool}
A.~Dabouei, S.~Soleymani, F.~Taherkhani, J.~Dawson, and N.~Nasrabadi, ``Smoothfool: An efficient framework for computing smooth adversarial perturbations,'' in \emph{Proceedings of the IEEE/CVF Winter Conference on Applications of Computer Vision}, 2020, pp. 2665--2674.

\bibitem{madry2018towards}
A.~Madry, A.~Makelov, L.~Schmidt, D.~Tsipras, and A.~Vladu, ``Towards deep learning models resistant to adversarial attacks,'' in \emph{International Conference on Learning Representations}, 2018.

\bibitem{dong2018boosting}
Y.~Dong, F.~Liao, T.~Pang, H.~Su, J.~Zhu, X.~Hu, and J.~Li, ``Boosting adversarial attacks with momentum,'' in \emph{Proceedings of the IEEE conference on computer vision and pattern recognition}, 2018, pp. 9185--9193.

\bibitem{croce2020reliable}
F.~Croce and M.~Hein, ``Reliable evaluation of adversarial robustness with an ensemble of diverse parameter-free attacks,'' in \emph{International conference on machine learning}.\hskip 1em plus 0.5em minus 0.4em\relax PMLR, 2020, pp. 2206--2216.

\bibitem{croce2020minimally}
------, ``Minimally distorted adversarial examples with a fast adaptive boundary attack,'' in \emph{International Conference on Machine Learning}.\hskip 1em plus 0.5em minus 0.4em\relax PMLR, 2020, pp. 2196--2205.

\bibitem{andriushchenko2020square}
M.~Andriushchenko, F.~Croce, N.~Flammarion, and M.~Hein, ``Square attack: a query-efficient black-box adversarial attack via random search,'' in \emph{European conference on computer vision}.\hskip 1em plus 0.5em minus 0.4em\relax Springer, 2020, pp. 484--501.

\bibitem{papernot2017extending}
N.~Papernot and P.~McDaniel, ``Extending defensive distillation,'' \emph{arXiv preprint arXiv:1705.05264}, 2017.

\bibitem{carmon2019unlabeled}
Y.~Carmon, A.~Raghunathan, L.~Schmidt, J.~C. Duchi, and P.~S. Liang, ``Unlabeled data improves adversarial robustness,'' \emph{Advances in neural information processing systems}, vol.~32, 2019.

\bibitem{zi2021revisiting}
B.~Zi, S.~Zhao, X.~Ma, and Y.-G. Jiang, ``Revisiting adversarial robustness distillation: Robust soft labels make student better,'' in \emph{Proceedings of the IEEE/CVF International Conference on Computer Vision}, 2021, pp. 16\,443--16\,452.

\bibitem{chen2021ltd}
E.-C. Chen and C.-R. Lee, ``Ltd: Low temperature distillation for robust adversarial training,'' \emph{arXiv preprint arXiv:2111.02331}, 2021.

\bibitem{zhu2023improving}
K.~Zhu, X.~Hu, J.~Wang, X.~Xie, and G.~Yang, ``Improving generalization of adversarial training via robust critical fine-tuning,'' in \emph{Proceedings of the IEEE/CVF International Conference on Computer Vision}, 2023, pp. 4424--4434.

\bibitem{huang2023boosting}
B.~Huang, M.~Chen, Y.~Wang, J.~Lu, M.~Cheng, and W.~Wang, ``Boosting accuracy and robustness of student models via adaptive adversarial distillation,'' in \emph{Proceedings of the IEEE/CVF Conference on Computer Vision and Pattern Recognition}, 2023, pp. 24\,668--24\,677.

\bibitem{kuang2024improving}
H.~Kuang, H.~Liu, Y.~Wu, S.~Satoh, and R.~Ji, ``Improving adversarial robustness via information bottleneck distillation,'' \emph{Advances in Neural Information Processing Systems}, vol.~36, 2024.

\bibitem{yue2024revisiting}
X.~Yue, M.~Ningping, Q.~Wang, and L.~Zhao, ``Revisiting adversarial robustness distillation from the perspective of robust fairness,'' \emph{Advances in Neural Information Processing Systems}, vol.~36, 2024.

\bibitem{jung2024peeraid}
J.~Jung, H.~Jang, J.~Song, and J.~Lee, ``Peeraid: Improving adversarial distillation from a specialized peer tutor,'' in \emph{Proceedings of the IEEE/CVF Conference on Computer Vision and Pattern Recognition}, 2024, pp. 24\,482--24\,491.

\bibitem{park2025dynamic}
H.~Park and D.~Min, ``Dynamic guidance adversarial distillation with enhanced teacher knowledge,'' in \emph{European Conference on Computer Vision}.\hskip 1em plus 0.5em minus 0.4em\relax Springer, 2025, pp. 204--219.

\bibitem{nie2022diffusion}
W.~Nie, B.~Guo, Y.~Huang, C.~Xiao, A.~Vahdat, and A.~Anandkumar, ``Diffusion models for adversarial purification,'' in \emph{International Conference on Machine Learning}.\hskip 1em plus 0.5em minus 0.4em\relax PMLR, 2022, pp. 16\,805--16\,827.

\bibitem{kalaria2022towards}
D.~Kalaria, A.~Hazra, and P.~P. Chakrabarti, ``Towards adversarial purification using denoising autoencoders,'' \emph{arXiv preprint arXiv:2208.13838}, 2022.

\bibitem{vincent2008extracting}
P.~Vincent, H.~Larochelle, Y.~Bengio, and P.-A. Manzagol, ``Extracting and composing robust features with denoising autoencoders,'' in \emph{Proceedings of the 25th international conference on Machine learning}, 2008, pp. 1096--1103.

\bibitem{wang2023better}
Z.~Wang, T.~Pang, C.~Du, M.~Lin, W.~Liu, and S.~Yan, ``Better diffusion models further improve adversarial training,'' in \emph{International Conference on Machine Learning}.\hskip 1em plus 0.5em minus 0.4em\relax PMLR, 2023, pp. 36\,246--36\,263.

\bibitem{karras2022elucidating}
T.~Karras, M.~Aittala, T.~Aila, and S.~Laine, ``Elucidating the design space of diffusion-based generative models,'' \emph{Advances in neural information processing systems}, vol.~35, pp. 26\,565--26\,577, 2022.

\bibitem{lee2023robust}
M.~Lee and D.~Kim, ``Robust evaluation of diffusion-based adversarial purification,'' in \emph{Proceedings of the IEEE/CVF International Conference on Computer Vision}, 2023, pp. 134--144.

\bibitem{cao2023fepn}
D.~Cao, K.~Wei, Y.~Wu, J.~Zhang, B.~Feng, and J.~Chen, ``Fepn: A robust feature purification network to defend against adversarial examples,'' \emph{Computers \& Security}, vol. 134, p. 103427, 2023.

\bibitem{chen2024diffilter}
Y.~Chen, X.~Li, X.~Wang, P.~Hu, and D.~Peng, ``Diffilter: Defending against adversarial perturbations with diffusion filter,'' \emph{IEEE Transactions on Information Forensics and Security}, 2024.

\bibitem{zhang2024random}
J.~Zhang, P.~Dong, Y.~Chen, Y.-P. Zhao, and S.~Guo, ``Random sampling for diffusion-based adversarial purification,'' \emph{arXiv preprint arXiv:2411.18956}, 2024.

\bibitem{song2024mimicdiffusion}
K.~Song, H.~Lai, Y.~Pan, and J.~Yin, ``Mimicdiffusion: Purifying adversarial perturbation via mimicking clean diffusion model,'' in \emph{Proceedings of the IEEE/CVF Conference on Computer Vision and Pattern Recognition}, 2024, pp. 24\,665--24\,674.

\bibitem{zbontar2021barlow}
J.~Zbontar, L.~Jing, I.~Misra, Y.~LeCun, and S.~Deny, ``Barlow twins: Self-supervised learning via redundancy reduction,'' in \emph{International conference on machine learning}.\hskip 1em plus 0.5em minus 0.4em\relax PMLR, 2021, pp. 12\,310--12\,320.

\bibitem{he2016deep}
K.~He, X.~Zhang, S.~Ren, and J.~Sun, ``Deep residual learning for image recognition,'' in \emph{Proceedings of the IEEE conference on computer vision and pattern recognition}, 2016, pp. 770--778.

\bibitem{zagoruyko2016wide}
S.~Zagoruyko, ``Wide residual networks,'' \emph{arXiv preprint arXiv:1605.07146}, 2016.

\bibitem{simonyan2014very}
K.~Simonyan and A.~Zisserman, ``Very deep convolutional networks for large-scale image recognition,'' \emph{arXiv preprint arXiv:1409.1556}, 2014.

\bibitem{sandler2018mobilenetv2}
M.~Sandler, A.~Howard, M.~Zhu, A.~Zhmoginov, and L.-C. Chen, ``Mobilenetv2: Inverted residuals and linear bottlenecks,'' in \emph{Proceedings of the IEEE conference on computer vision and pattern recognition}, 2018, pp. 4510--4520.

\bibitem{tan2019efficientnet}
M.~Tan and Q.~Le, ``Efficientnet: Rethinking model scaling for convolutional neural networks,'' in \emph{International conference on machine learning}.\hskip 1em plus 0.5em minus 0.4em\relax PMLR, 2019, pp. 6105--6114.

\bibitem{szegedy2016rethinking}
C.~Szegedy, V.~Vanhoucke, S.~Ioffe, J.~Shlens, and Z.~Wojna, ``Rethinking the inception architecture for computer vision,'' in \emph{Proceedings of the IEEE conference on computer vision and pattern recognition}, 2016, pp. 2818--2826.

\bibitem{krizhevsky2009learning}
A.~Krizhevsky, G.~Hinton \emph{et~al.}, ``Learning multiple layers of features from tiny images,'' \emph{Master's thesis, University of Tront}, 2009.

\bibitem{le2015tiny}
Y.~Le and X.~Yang, ``Tiny imagenet visual recognition challenge,'' \emph{CS 231N}, vol.~7, no.~7, p.~3, 2015.

\bibitem{daugman2000biometric}
J.~Daugman, ``Biometric decision landscapes,'' University of Cambridge, Computer Laboratory, Tech. Rep., 2000.
\end{thebibliography}




\begin{IEEEbiography}[{\includegraphics[width=1in,height=1.25in,clip,keepaspectratio]{photos/joana_costa.png}}]{Joana C. Costa} obtained her bachelor's and master's degree in Computer Science and Engineering from Universidade da Beira Interior (UBI) in 2019 and 2021, respectively. She is currently pursuing a Ph.D. degree, with an FCT (\textit{Funda\c{c}\~{a}o para a Ciência e a Tecnologia}) scholarship, in the field of Computer Vision and Adversarial Attacks.
\end{IEEEbiography}

\begin{IEEEbiography}[{\includegraphics[width=1in,height=1.25in,clip,keepaspectratio]{photos/Tiago_Roxo.png}}]{Tiago Roxo} obtained a bachelor's degree in Computer Science and Engineering from Universidade da Beira Interior (UBI) in 2019 and is currently pursuing a Ph.D. degree, with an FCT (\textit{Funda\c{c}\~{a}o para a Ciência e a Tecnologia}) scholarship, in the field of Computer Vision and Artificial Intelligence.
\end{IEEEbiography}

\begin{IEEEbiography}[{\includegraphics[width=1in,height=1.25in,clip,keepaspectratio]{photos/Hugo_Proenca.pdf}}]{Hugo Proen\c{c}a} (SM'12), B.Sc. (2001), M.Sc. (2004) and Ph.D. (2007) is a Full Professor in the Department of Computer Science, University of Beira Interior and has been researching mainly about biometrics and visual-surveillance. He was the coordinating editor of the IEEE Biometrics Council Newsletter and the area editor (ocular biometrics) of the IEEE Biometrics Compendium Journal. He is a member of the Editorial Boards of the Image and Vision Computing, IEEE Access and International Journal of Biometrics. Also, he served as Guest Editor of special issues of the Pattern Recognition Letters, Image and Vision Computing and Signal, Image and Video Processing journals. 
\end{IEEEbiography}

\begin{IEEEbiography}[{\includegraphics[width=1in,height=1.25in,clip,keepaspectratio]{photos/Pedro_Inacio.png}}]{Pedro R. M. In\'{a}cio}  (SM'15), B.Sc. in Mathematics/Computer Science (2005), and Ph.D. in Computer Science and Engineering (2009) is an associate professor of the Department of Computer Science at the University of Beira Interior (UBI), which he joined in 2010 and where he lectures subjects related with information assurance and (cyber)security. The Ph.D. work was performed in the enterprise environment of Nokia Siemens Networks Portugal S.A.

He is an IEEE senior member, an ACM professional member and a researcher of the Instituto de Telecomunicações (IT). His main research topics are information assurance and security, computer based simulation, and network traffic monitoring, analysis and classification. He frequently reviews papers for IEEE, Springer, Wiley and Elsevier journals. He is a member of the Technical Program Committees of flagship national and international workshops and conferences, such as ACM SAC, IEEE NCA, IFIPSEC or ARES. He is also a Senior Editor for IEEE Access.

\end{IEEEbiography}
\vfill

\end{document}


