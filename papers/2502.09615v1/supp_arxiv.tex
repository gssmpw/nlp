\section{Final Training Objective}

We train our entire model end-to-end, ensuring that joint positions, connectivity, and skinning weights are learned in a mutually reinforcing manner. Specifically, we combine the losses from the joint diffusion, connectivity, and skinning prediction modules into a single objective. The integrated objective allows the network to learn coherent skeleton structures and accurate skinning assignments simultaneously.

\begin{equation*}
    \mathcal{L} = \mathcal{L}_{\text{joint}} + \mathcal{L}_{\text{connect}} + \mathcal{L}_{\text{skinning}}.
\end{equation*}


\section{Skeleton Ambiguity}

We elaborate on the two types of skeleton ambiguity introduced in the main paper, \textit{i.e.}, the sibling ambiguity and the topology ambiguity. During the BFS ordering process used to traverse and sequentialize the skeleton tree, the ordering of nodes at the same depth in the BFS tree remains undefined. This can lead to the sibling ambiguity, as illustrated in \cref{fig:sibling_ambiguity}. For example, when the preceding skeleton tokens are 1, 2, and 3, the next joint could either be 4 or 5, each with an equal probability of occurrence.
Additionally, an object can have multiple valid skeleton topologies, as shown in \cref{fig: diffusion_sampling}, and the model must be capable of capturing these multiple valid configurations.
Our method naturally addresses these by modeling the distribution of the next joint based on preceding predictions, which offers a distinct advantage over deterministic approaches in capturing the inherent uncertainty in joint positions.

% \section{Sibling Ambiguity}

% We elaborate on the sibling ambiguity issue introduced in the main paper. During the BFS ordering process used to traverse and sequentialize the skeleton tree, the ordering of nodes at the same depth in the BFS tree remains undefined. This can lead to ambiguity, as illustrated in \cref{fig:sibling_ambiguity}. For example, when the preceding skeleton tokens are 1, 2, and 3, the next joint could either be 4 or 5, each with an equal probability of occurrence. Our method naturally addresses this by modeling the distribution of the next joint based on preceding predictions, which offers a distinct advantage over deterministic approaches in capturing the inherent uncertainty in joint positions.


\begin{figure}[h]
\includegraphics[width=0.8\linewidth]{Figures/sibling_ambiguity.pdf}
\caption{Illustration of sibling ambiguity during BFS ordering in skeletons.}
\label{fig:sibling_ambiguity}
\end{figure}



\begin{figure}[h]
\includegraphics[width=0.7\linewidth]{Figures/riganything-diffusion-sampling.pdf}
\caption{Two skeleton examples with different skeleton topologies generated on the same shape input.}
\label{fig: diffusion_sampling}
\end{figure}

\section{Dataset Details}

Our dataset consists of a total of 12,040 shapes, combining 2,354 samples from the RigNet dataset and 9,686 shapes from the Objaverse dataset. From this collection, we utilize 11,217 samples for training and reserve 823 samples for validation and testing. The dataset covers a wide range of object categories and poses, ensuring diversity and robustness in downstream tasks.

\subsection{Data Filtering}

We apply rigorous data filtering processes to ensure the quality and validity of the dataset. The filtering criteria differ slightly between the RigNet and Objaverse datasets due to their varying quality and size. For the RigNet dataset, which is already high-quality, we apply simple filtering based on the following principles:

\begin{itemize}
    \item Shapes with more than 64 joints in their rigging are excluded.
    \item Shape are excluded if their skeleton is invalid (e.g. skeleton hierarchy does not form a proper tree structure).
\end{itemize}

For the Objaverse dataset, which is significantly larger, we employ both manual labeling and automated filtering scripts to ensure consistency and quality. The filtering is performed based on the following principles:

\begin{itemize}
    \item Shapes with more than 64 joints in their rigging are excluded.
    \item Shapes are excluded if their skeleton is invalid (e.g. skeleton hierarchy does not form a proper 
    tree structure).
    \item Shapes are excluded if the skeleton does not properly align with the corresponding geometry.
    \item Overly simplified or indistinguishable shapes (i.e. shapes consisting of only a few vertices and faces) are excluded.
\end{itemize}

The Objaverse dataset originally includes 21,622 shapes with rigging annotations. During filtering, we excluded 811 shapes with overly complicated skeletons containing more than 64 joints, which were usually associated with facial or hair rigs. Additionally, 10,471 shapes were filtered out due to low-quality rigging or skinning annotations, leaving us with a refined set of shapes with reliable rigging information.

\begin{figure}[t]
\includegraphics[width=\linewidth]{Figures/joint-distribution.png}
\caption{Distribution of joint numbers across shapes in our dataset.}
\label{fig:joint_number_distribution}
\end{figure}

\subsection{Category statistics}

For the Objaverse dataset, we also collect the Objaverse category labels and classify the shapes into six categories: humanoid/bipedal, quadruped, insectoid, avian, marine, and other. The "other" category typically includes manipulable articulated rigid objects found in the dataset (e.g. suitcases, cabinets, etc). Below, we provide the statistics for the number of shapes in each category in \cref{tab:dataset_stat}:

\begin{table}[H]
    \caption{Category statistics of the filtered Objaverse dataset.}
    \label{tab:dataset_stat}
    \small
    \centering
    \setlength{\tabcolsep}{5pt}
    \scalebox{0.92}{
    {\fontsize{9pt}{10pt}\selectfont
    \begin{tabular}{@{}lc@{}}
    \toprule
        & Number\\ \midrule
    Humanoid/Bipedal  & 7459\\
    Quadruped  &  543 \\
    Insectoid & 129 \\
    Avian & 176 \\
    Marine & 251 \\
    Other & 830 \\ \midrule
    \textbf{Total} & 9388 \\
    \bottomrule
    \end{tabular}}}
\end{table}

\subsection{Skeleton Distribution}

We collect the distribution of joint numbers in each shape and provide the corresponding distribution plot below. Our dataset contains more shapes with joint numbers in the intervals $[25,30]$, $[50,55]$, and $[60,64]$.


