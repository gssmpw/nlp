%%
%% This is file `sample-acmtog.tex',
%% generated with the docstrip utility.
%%
%% The original source files were:
%%
%% samples.dtx  (with options: `all,journal,bibtex,acmtog')
%%
%% IMPORTANT NOTICE:
%%
%% For the copyright see the source file.
%%
%% Any modified versions of this file must be renamed
%% with new filenames distinct from sample-acmtog.tex.
%%
%% For distribution of the original source see the terms
%% for copying and modification in the file samples.dtx.
%%
%% This generated file may be distributed as long as the
%% original source files, as listed above, are part of the
%% same distribution. (The sources need not necessarily be
%% in the same archive or directory.)
%%
%%
%% Commands for TeXCount
%TC:macro \cite [option:text,text]
%TC:macro \citep [option:text,text]
%TC:macro \citet [option:text,text]
%TC:envir table 0 1
%TC:envir table* 0 1
%TC:envir tabular [ignore] word
%TC:envir displaymath 0 word
%TC:envir math 0 word
%TC:envir comment 0 0
%%
%% The first command in your LaTeX source must be the \documentclass
%% command.
%%
%% For submission and review of your manuscript please change the
%% command to \documentclass[manuscript, screen, review]{acmart}.
%%
%% When submitting camera ready or to TAPS, please change the command
%% to \documentclass[sigconf]{acmart} or whichever template is required
%% for your publication.
%%
%%
% \documentclass[acmtog,anonymous,review]{acmart}

\documentclass[sigconf]{acmart}

% -- Remove ACM-specific formatting elements --
\settopmatter{printacmref=false, authorsperrow=4} % Remove reference format block
\renewcommand\footnotetextcopyrightpermission[1]{} % Remove permission footnote
\pagestyle{plain} % Use plain page style
\acmConference[ArXiv]{}{Feb}{2025}
% -- Remove ACM-specific formatting elements --

%%
%% \BibTeX command to typeset BibTeX logo in the docs
\AtBeginDocument{%
  \providecommand\BibTeX{{%
    Bib\TeX}}}



% New added features
\usepackage{multirow}
\usepackage{array}
\usepackage{pifont}
\newcommand*\colourcheck[1]{%
  \expandafter\newcommand\csname #1check\endcsname{\textcolor{#1}{\ding{52}}}%
}
\newcommand*\colourmark[1]{%
  \expandafter\newcommand\csname #1mark\endcsname{\textcolor{#1}{\ding{55}}}%
}
\newcommand{\boldstartspace}[1]{\vspace{0.05in}\noindent\textbf{#1}}
\colourcheck{blue}
\colourcheck{green}
\colourcheck{red}
\colourcheck{black}

\colourmark{blue}
\colourmark{green}
\colourmark{red}
\colourmark{black}

\usepackage{xcolor} % Ensure xcolor is loaded
\PassOptionsToPackage{pagebackref,breaklinks,colorlinks,citecolor=citeblue,bookmarks=false,linkcolor=blue,urlcolor=blue}{hyperref}
\usepackage{hyperref}
\usepackage[capitalize]{cleveref}  % Should be loaded after 'hyperref', and works perfectly with 'subfigure'.
\crefname{section}{Sec.}{Secs.}
\Crefname{section}{Section}{Sections}
\crefname{table}{Tab.}{Tabs.}
\Crefname{table}{Table}{Tables}
\crefname{figure}{Fig.}{Figs.}
\Crefname{figure}{Figure}{Figures}
\crefname{equation}{Eq.}{Eqs.}
\Crefname{equation}{Equation}{Equations}


\newcommand{\todo}[1]{\textcolor{red}{[TODO: #1]}}
\newcommand{\tocite}[1]{\textcolor{red}{[TO CITE]}}
\newcommand{\toref}[1]{\textcolor{red}{[TO REF]}}

\newcommand{\zf}[1]{\textcolor{cyan}{[Zifan: #1]}}
\newcommand{\yf}[1]{\textcolor{purple}{[Yifan: #1]}}
\newcommand{\zhan}[1]{\textcolor{teal}{[Zhan: #1]}}
\newcommand{\isa}[1]{\textcolor{red}{[Isa: #1]}}
\newcommand{\hao}[1]{\textcolor{brown}{[Hao: #1]}}

%\renewcommand{\paragraph}[1]{\vspace{0.3em}\noindent\textit{#1}}

\usepackage{xcolor}

%%%%%%%%%%%%%%%%%%%%%%%%%%%%%%%%%%%%%%%%%%%%%%%%%%%%%%%%%%%%%%%%%%%
%%%%%%%%%%%%%%%%%%%%%%% SYMBOLS %%%%%%%%%%%%%%%%%%%%%%%%%%%%%%%
%%%%%%%%%%%%%%%%%%%%%%%%%%%%%%%%%%%%%%%%%%%%%%%%%%%%%%%%%%%%%%%%%%%
\newcommand{\ba}{\mathbf{a}}
\newcommand{\bb}{\mathbf{b}}
\newcommand{\bc}{\mathbf{c}}
\newcommand{\bd}{\mathbf{d}}
\newcommand{\be}{\mathbf{e}}
\newcommand{\bff}{\mathbf{f}}
\newcommand{\bg}{\mathbf{g}}
\newcommand{\bh}{\mathbf{h}}
\newcommand{\bi}{\mathbf{i}}
\newcommand{\bj}{\mathbf{j}}
\newcommand{\bk}{\mathbf{k}}
\newcommand{\bl}{\mathbf{l}}
\newcommand{\bm}{\mathbf{m}}
\newcommand{\bn}{\mathbf{n}}
\newcommand{\bo}{\mathbf{o}}
\newcommand{\bp}{\mathbf{p}}
\newcommand{\bq}{\mathbf{q}}
\newcommand{\br}{\mathbf{r}}
\newcommand{\bs}{\mathbf{s}}
\newcommand{\bt}{\mathbf{t}}
\newcommand{\bu}{\mathbf{u}}
\newcommand{\bv}{\mathbf{v}}
\newcommand{\bw}{\mathbf{w}}
\newcommand{\bx}{\mathbf{x}}
\newcommand{\by}{\mathbf{y}}
\newcommand{\bz}{\mathbf{z}}
\newcommand{\bA}{\mathbf{A}}
\newcommand{\bB}{\mathbf{B}}
\newcommand{\bC}{\mathbf{C}}
\newcommand{\bD}{\mathbf{D}}
\newcommand{\bE}{\mathbf{E}}
\newcommand{\bF}{\mathbf{F}}
\newcommand{\bG}{\mathbf{G}}
\newcommand{\bH}{\mathbf{H}}
\newcommand{\bI}{\mathbf{I}}
\newcommand{\bJ}{\mathbf{J}}
\newcommand{\bK}{\mathbf{K}}
\newcommand{\bL}{\mathbf{L}}
\newcommand{\bM}{\mathbf{M}}
\newcommand{\bN}{\mathbf{N}}
\newcommand{\bO}{\mathbf{O}}
\newcommand{\bP}{\mathbf{P}}
\newcommand{\bQ}{\mathbf{Q}}
\newcommand{\bR}{\mathbf{R}}
\newcommand{\bS}{\mathbf{S}}
\newcommand{\bT}{\mathbf{T}}
\newcommand{\bU}{\mathbf{U}}
\newcommand{\bV}{\mathbf{V}}
\newcommand{\bW}{\mathbf{W}}
\newcommand{\bX}{\mathbf{X}}
\newcommand{\bY}{\mathbf{Y}}
\newcommand{\bZ}{\mathbf{Z}}
\newcommand{\balpha}{\mbox{\boldmath$\alpha$}}
\newcommand{\bgamma}{\mbox{\boldmath$\gamma$}}
\newcommand{\bGamma}{\mbox{\boldmath$\Gamma$}}
\newcommand{\bmu}{\mbox{\boldmath$\mu$}}
\newcommand{\bphi}{\mbox{\boldmath$\phi$}}
\newcommand{\bPhi}{\mbox{\boldmath$\Phi$}}
\newcommand{\bSigma}{\mbox{\boldmath$\Sigma$}}
\newcommand{\bsigma}{\mbox{\boldmath$\sigma$}}
\newcommand{\btheta}{\mbox{\boldmath$\theta$}}

\newcommand{\mE}{\mathcal{E}}
\newcommand{\mV}{\mathcal{V}}
\newcommand{\mM}{\mathcal{M}}
\newcommand{\mH}{\mathcal{H}}
\newcommand{\mL}{\mathcal{L}}
\newcommand{\mU}{\mathcal{U}}
\newcommand{\mC}{\mathcal{C}}
\newcommand{\mS}{\mathcal{S}}
\newcommand{\mR}{\mathcal{R}}
\newcommand{\mD}{\mathcal{D}}
\newcommand{\mO}{\mathcal{O}}
\newcommand{\mP}{\mathcal{P}}
\newcommand{\mT}{\mathcal{T}}
\newcommand{\mSl}{\mathcal{S}_l}
\newcommand{\mN}{\mathcal{N}}
\newcommand{\mDll}{\mathcal{D}_{l,l'}}

\newcommand{\ra}{\rightarrow}
\newcommand{\la}{\leftarrow}

\def\A{{\cal A}}
\def\B{{\cal B}}
\def\C{{\cal C}}
\def\D{{\cal D}}
\def\E{{\cal E}}
\def\F{{\cal F}}
\def\G{{\cal G}}
\def\H{{\cal H}}
\def\I{{\cal I}}
\def\J{{\cal J}}
\def\K{{\cal K}}
\def\L{{\cal L}}
\def\M{{\cal M}}
\def\N{{\cal N}}
\def\O{{\cal O}}
\def\P{{\cal P}}
\def\Q{{\cal Q}}
\def\R{{\cal R}}
\def\S{{\cal S}}
\def\T{{\cal T}}
\def\U{{\cal U}}
\def\V{{\cal V}}
\def\W{{\cal W}}
\def\X{{\cal X}}
\def\Y{{\cal Y}}
\def\Z{{\cal Z}}
\def\Re{{\mathbb R}}
\def\Cx{{\mathbb C}}
\def\Ze{{\mathbb Z}}
\def\Na{{\mathbb N}}
\def\ud{\mathrm{d}}
\def\eps{\varepsilon}
\def\dist{\textrm{dist}}

% custom symbols
\newcommand{\Shape}{\mathcal{S}}

%% Rights management information.  This information is sent to you
%% when you complete the rights form.  These commands have SAMPLE
%% values in them; it is your responsibility as an author to replace
%% the commands and values with those provided to you when you
%% complete the rights form.
% \setcopyright{acmlicensed}
% \copyrightyear{2025}
% \acmYear{2025}
% \acmDOI{XXXXXXX.XXXXXXX}

%%
%% These commands are for a JOURNAL article.
% \acmJournal{TOG}
% \acmVolume{37}
% \acmNumber{4}
% \acmArticle{111}
% \acmMonth{8}

%%
%% Submission ID.
%% Use this when submitting an article to a sponsored event. You'll
%% receive a unique submission ID from the organizers
%% of the event, and this ID should be used as the parameter to this command.
% \acmSubmissionID{}

%%
%% For managing citations, it is recommended to use bibliography
%% files in BibTeX format.
%%
%% You can then either use BibTeX with the ACM-Reference-Format style,
%% or BibLaTeX with the acmnumeric or acmauthoryear sytles, that include
%% support for advanced citation of software artefact from the
%% biblatex-software package, also separately available on CTAN.
%%
%% Look at the sample-*-biblatex.tex files for templates showcasing
%% the biblatex styles.
%%

%%
%% The majority of ACM publications use numbered citations and
%% references.  The command \citestyle{authoryear} switches to the
%% "author year" style.
%%
%% If you are preparing content for an event
%% sponsored by ACM SIGGRAPH, you must use the "author year" style of
%% citations and references.
\citestyle{acmauthoryear}

%%
%% end of the preamble, start of the body of the document source.
\begin{document}

%%
%% The "title" command has an optional parameter,
%% allowing the author to define a "short title" to be used in page headers.
\title[RigAnything]{RigAnything: Template-Free Autoregressive Rigging \\ for Diverse 3D Assets}

%%
%% The "author" command and its associated commands are used to define
%% the authors and their affiliations.
%% Of note is the shared affiliation of the first two authors, and the
%% "authornote" and "authornotemark" commands
%% used to denote shared contribution to the research.
\author{Isabella Liu}
% \email{lal005@ucsd.edu}
\affiliation{%
  \institution{UC San Diego}
  % \country{USA}
}

\author{Zhan Xu}
% \email{zhaxu@adobe.com}
\affiliation{%
  \institution{Adobe Research}
  % \country{USA}
}

\author{Wang Yifan}
% \email{yifwang@adobe.com}
\affiliation{%
  \institution{Adobe Research}
  % \country{USA}
}

\author{Hao Tan}
% \email{hatan@adobe.com}
\affiliation{%
  \institution{Adobe Research}
  % \country{USA}
}

\author{Zexiang Xu}
% \email{zexiangxu@gmail.com}
\affiliation{%
  \institution{Hillbot Inc.}
  % \country{USA}
}

\author{Xiaolong Wang}
% \email{xiw012@ucsd.edu}
\affiliation{%
  \institution{UC San Diego}
  % \country{USA}
}

\author{Hao Su}
% \email{haosu@ucsd.edu}
\affiliation{%
  \institution{UC San Diego}
  % \country{USA}
}
\affiliation{%
  \institution{Hillbot Inc.}
  % \country{USA}
}

\author{Zifan Shi}
% \email{vivianszf9@gmail.com}
\affiliation{%
  \institution{Adobe Research}
  % \country{USA}
}
%%
%% By default, the full list of authors will be used in the page
%% headers. Often, this list is too long, and will overlap
%% other information printed in the page headers. This command allows
%% the author to define a more concise list
%% of authors' names for this purpose.
\renewcommand{\shortauthors}{Liu et al.}

%%
%% The abstract is a short summary of the work to be presented in the
%% article.
\begin{abstract}
We present \textbf{\emph{RigAnything}}, a novel autoregressive transformer-based model, which makes 3D assets rig-ready by probabilistically generating joints, skeleton topologies, and assigning skinning weights in a template-free manner.
Unlike most existing auto-rigging methods, which rely on predefined skeleton template and are limited to specific categories like humanoid,
RigAnything approaches the rigging problem in an autoregressive manner, iteratively predicting the next joint based on the global input shape and the previous prediction.
While autoregressive models are typically used to generate sequential data, RigAnything extends their application to effectively learn and represent skeletons, which are inherently tree structures. To achieve this, we organize the joints in a breadth-first search (BFS) order, enabling the skeleton to be defined as a sequence of 3D locations and the parent index.
Furthermore, our model improves the accuracy of position prediction by leveraging diffusion modeling, ensuring precise and consistent placement of joints within the hierarchy. This formulation allows the autoregressive model to efficiently capture both spatial and hierarchical relationships within the skeleton.
Trained end-to-end on both RigNet and Objaverse datasets, RigAnything demonstrates state-of-the-art performance across diverse object types, including humanoids, quadrupeds, marine creatures, insects, and many more, surpassing prior methods in quality, robustness, generalizability, and efficiency. Please check our website for more details: \href{https://www.liuisabella.com/RigAnything}{\textcolor{blue}{\textsf{https://www.liuisabella.com/RigAnything}}}.
\end{abstract}


%%
%% The code below is generated by the tool at http://dl.acm.org/ccs.cfm.
%% Please copy and paste the code instead of the example below.
%%
\begin{CCSXML}
<ccs2012>
   <concept>
       <concept_id>10010147.10010371.10010352</concept_id>
       <concept_desc>Computing methodologies~Animation</concept_desc>
       <concept_significance>500</concept_significance>
       </concept>
   <concept>
       <concept_id>10010147.10010257.10010293.10010294</concept_id>
       <concept_desc>Computing methodologies~Neural networks</concept_desc>
       <concept_significance>300</concept_significance>
       </concept>
 </ccs2012>
\end{CCSXML}

\ccsdesc[500]{Computing methodologies~Animation}
\ccsdesc[300]{Computing methodologies~Neural networks}

%%
%% Keywords. The author(s) should pick words that accurately describe
%% the work being presented. Separate the keywords with commas.
\keywords{Animation Skeleton, Automatic Rigging, Skinning, Autoregressive Modeling, Transformer-Based Models}

% \received{23 January 2025}
% \received[revised]{12 March 2009}
% \received[accepted]{5 June 2009}

%%
%% This command processes the author and affiliation and title
%% information and builds the first part of the formatted document.
\begin{teaserfigure}
    \includegraphics[width=\textwidth]{Figures/riganything-teaser.pdf}
    % \includegraphics[width=\textwidth]{Figures/Asset 2.png}
    \caption{RigAnything is an autoregressive transformer-based approach for automatic rigging. From an arbitrarily posed shape (shown on the left), it can generate a skeleton and skinning weights that adapt seamlessly to the input's global structure (shown on the right), enabling articulation into new poses.
    }
    % \Description{teaser figure}
    \label{fig:teaser}
\end{teaserfigure}
\maketitle

%  Sections
\section{Introduction}\label{sec:intro}

In computational finance, Monte Carlo simulations are used extensively to estimate the expected value of financial payoffs based on the solution of stochastic differential equations (SDEs) which model the evolution of stock prices, interest rates, exchange rates and other quantities \cite{glasserman04}.  Monte Carlo methods are very general and flexible, but for high accuracy it requires generating a large number of costly SDE path approximations, which has motivated research into a number of variance reduction or, equivalently, cost reduction techniques. One such method is
Multilevel Monte Carlo (MLMC), which was proposed in \cite{GILES2008} and was adapted for various applications that are summarised in \cite{Giles_overview17} and successfully combined with other methods such as quasi-Monte Carlo methods. The main idea of MLMC is to approximate the payoff using different time stepping resolutions when numerically solving the underlying SDE and to generate an optimal number of samples on each level, such that the overall computational cost is minimised subject to the desired bound on the variance. %, such that the total computational cost is minimised. 
The computational savings come from the fact that most samples are computed on the coarser levels and hence are less expensive while only a few samples from the finest levels are required \cite{GILES2008}.


Among the directions in which the computational cost 
of MLMC methods could further be reduced, an important avenue is the use of lower precision calculations, especially for the first Monte Carlo levels where the targeted accuracy is relatively low. 
 An overview of the research on mixed precision for the standard Monte Carlo (MC) framework is provided in \cite{ChowMixedPrecisionStandardMC} but only a few references study the potential of low precision computation in the MLMC framework \cite{Rounding_error_oliver}. To the best of our knowledge, the only MLMC framework with customised precision in the literature is \cite{brugger2014mixed}, but they use a uniform precision for all operations on each Monte Carlo level instead of optimising 
 the precision of each intermediary variable to reduce as much as possible the cost of path generation.
 
An important motivation for an MLMC framework with variable precision would be performing the low precision computations on reconfigurable hardware devices such as Field Programmable Gate Arrays (FPGAs). FPGAs contain customizable logic blocks and connectors that make it easy to adapt the digital circuit architecture for a specific application, leading to a highly parallel and optimised implementation. Therefore they are successfully exploited in applications that require high speed and have high computational workload, such as signal processing \cite{woods2008fpga}, and real time applications like high frequency trading \cite{HFT1,HFT2}. That is why a number of previous works in hardware architecture design implemented the MLMC algorithm to price financial options using FPGAs as accelerators, which resulted in improved speed and power efficiency compared to full CPU architectures \cite{Schryver2013AMM}. The paper \cite{lindsey2016domain} also proposed 
a Domain Specific Language to automate the configuration of FPGAs for this specific application. However, only \cite{brugger2014mixed} proposed a heuristic to reduce the precision in calculations.

In addition, all aforementioned works considered that the random number generation (RNG) is performed in single or double precision. Yet in most cases an important portion of the workload in the overall MLMC simulation comes from the RNG and in \cite{brugger2014mixed} this limited the total computational savings.
To reduce the cost of MLMC simulations in particular those based on the Geometric Brownian Motion (GBM), \cite{approximateICDF_Oliver, NestedOliver} have proposed to use approximate random numbers that are generated by applying an approximation of the inverse CDF to uniform random numbers. In \cite{NestedOliver}, the authors proposed a way to integrate these lower precision random variables into a \textit{nested} MLMC framework and completed a numerical analysis to bound the resulting error at each MC level by a product of the time step and the error in the random number approximation. The same authors show in \cite{approximateICDF_Oliver} that using approximate random variables reduces the cost of path generation by a factor 7.


In this paper we propose a nested MLMC framework that combines the use of approximate random normal variables and lower precision calculations to reduce the computational cost of MLMC even further than \cite{brugger2014mixed,NestedOliver}. We illustrate the efficiency of our framework in Matlab, after making several assumptions on the cost of operations and size of the errors that we carefully justify. We focus on the case of GBM and use the approximate RNG methods presented in \cite{approximateICDF_Oliver} as well as a new slightly modified method that combines CDF inversion and the central limit theorem. To choose the precision of the variables in the low precision path generation, we introduce a novel method to optimise the bit-widths. This optimisation is performed before the main path generation loop is executed and is based on a linear model of the payoff error  
due to rounding when computing in low precision. The error model relies on algorithmic differentiation in a similar manner to \cite{unifying-bwoptim,bitwidth-AD,ADAPT}. The bit-width optimisation procedure can be performed off-line, so this stage can be excluded from the on-line time complexity of our framework. The user specified desired accuracy is then enforced by calculating on-line the number of samples that need to be generated.

In terms of hardware design, we suggest implementing the low precision path generation on FPGAs and the full-precision ones on a CPU or GPU. 
The FPGA offers enough flexibility to define a separate bit-width for every variable in the low precision path generation, and can be reconfigured periodically to update the bit-widths when the market parameters have changed considerably. 


The paper is organized as follows : \Cref{sec:MLMC} introduces MLMC and nested MLMC to make clear the estimator that is implemented in our framework. Then in \Cref{sec:RNG} we detail the methods that could be used to obtain approximate random normally distributed numbers very cheaply for the low precision path generation. In \Cref{sec:error_model} and \Cref{sec:costModel} we propose an error model and a cost model (resp.) that we then use to formulate the optimisation problem that is solved to obtain the optimal bit-widths of fixed point variables in \Cref{sec:optimisation}. Finally we summarise our results and future directions in \Cref{sec:conclusion}.




\section{Related Work}
\label{sec:related_work}

The original investigation \cite{gibson1979ecological} on the relationship between visual perception and human action defines \emph{affordance} as the opportunities for interaction with the surrounding environment. Behavioral studies on regular and cognitively impaired persons have shown evidence that perception results in both visual and motor signals in the human brain. An extended study \cite{anderson2002attentional} shows that visual attention to the spatial characteristics of the perceived objects initiates automatic motor signals for different actions. In computer vision, human affordance learning involves novel pose prediction such that the estimated pose represents a valid human action within the scene context. The task is fundamental to many problems requiring robust semantic reasoning about the environment, such as human motion synthesis \cite{wang2021scene} and scene-aware human pose generation \cite{wang2017binge, roy2016multi, zhang2022inpaint, yao2023scene}.

Earlier methods of affordance learning have explored knowledge mining \cite{zhu2014reasoning} and multimodal feature cues \cite{roy2016multi} to address the problem. In \cite{zhu2014reasoning}, the authors use a Markov Logic Network for constructing a knowledge base by extracting several object attributes from different image and metadata sources, which can perform various downstream visual inference tasks without any additional classifier, including zero-shot affordance prediction. In \cite{roy2016multi}, the authors use depth map, surface normals, and segmentation map as multimodal cues to train a multi-scale convolutional neural network (CNN) for scene-level semantic label assignment associated with specific human actions. In \cite{do2018affordancenet}, the authors design a multi-branch end-to-end CNN with two separate pathways for object detection and affordance label assignment to achieve high real-time inference throughput. Researchers \cite{chuang2018learning} have also explored socially imposed constraints for affordance learning. In \cite{chuang2018learning}, the authors propose a graph neural network (GNN) to propagate contextual scene information from egocentric views for action-object affordance reasoning.

Probabilistic modeling of scene-aware human motion generation also involves semantic reasoning of human interaction with the environment. Initial works on human motion synthesis have taken different architectural approaches, such as sequence-to-sequence models \cite{barsoum2018hp}, generative adversarial networks (GAN) \cite{barsoum2018hp, cai2018deep, yang2018pose}, graph convolutional networks (GCN) \cite{yan2019convolutional}, and variational autoencoders (VAE) \cite{guo2020action2motion}. However, these methods have mostly ignored the role of environmental semantics. Due to potential uncertainty in human motion, in a recent approach \cite{wang2021scene}, the authors address such motion synthesis with a GAN conditioned on scene attributes and motion trajectory to predict probable body pose dynamics.

One key challenge of human affordance generation in 2D scenes is the lack of large-scale datasets with rich pose annotations. In \cite{wang2017binge}, the authors compile the only public dataset of annotated human body poses in complex 2D indoor scenes by extracting frames from sitcom videos. Aiming to generate a contextually valid human affordance at a user-defined location, the authors propose sampling the scale and deformation parameters for an existing human pose template using a VAE conditioned on the localized image patches as scene context. In \cite{zhang2022inpaint}, the authors introduce a two-stage GAN architecture for achieving a similar goal by estimating the affine bounding box parameters to localize a probable human in the scene and then generating a potential body pose at that location. The method uses the input scene, corresponding depth, and segmentation maps as semantic guidance. In \cite{yao2023scene}, the authors propose a transformer-based approach with knowledge distillation for generating human affordances in 2D indoor scenes.





\section{Methodology}
\paragraph{Preliminaries.}
We primarily focus on the homologous model merging, in which $\boldsymbol{\theta}_i$ all come from the same base model $\boldsymbol{\theta}_{\rm{base}}$. Given $K$ tasks $\{T_1,T_2,\cdots,T_K\}$ and $K$ corresponding fine-tuned models with parameters $\{\boldsymbol{\theta}_1,\boldsymbol{\theta}_2,\cdots,\boldsymbol{\theta}_K\}$, model merging aims to combine $K$ fine-tuned models into one single model simultaneously performing on $\{T_1,T_2,\cdots,T_K\}$ without post-training~\cite{method_p1_1,method_p1_2}.
Task vector~\cite{ilharco2023editing,yang2024adamerging} is a key element in merging method which could enhances the base model‘s ability or enable the model to handle other tasks. Specifically, for task $T_i$, the task vector $\boldsymbol\tau_i\in \mathbb{R}^D$ is defined as the vector obtained by subtracting the SFT weights $\boldsymbol{\theta}_i$ from the base model weight
$\boldsymbol{\theta}_{\rm{base}}$, \emph{i.e.}, $\boldsymbol\tau_i=\boldsymbol{\theta}_i-\boldsymbol{\theta}_{\rm{base}}$. The merged model could be denoted as $\boldsymbol{\theta}_m=\boldsymbol{\theta}_{\rm{base}}+\sum_i \lambda_i\boldsymbol{\tau}_i$, which $\lambda_i$ is the scaling factor measuring the importance of task vector. For clarification, we also denote the neuron set in $\boldsymbol{\theta}_i$ as $\mathcal{N}_i$, the neuron set in $\boldsymbol{\tau}_i$ as $\mathcal{T}_i$.



\begin{algorithm}[!ht]
    \caption{LED-Merging}
    \label{alg1}
    \begin{algorithmic}[1]
        \REQUIRE  base model $\boldsymbol{\theta}_{\rm{base}}$, SFT models $\{\boldsymbol{\theta}_{i}\mid i\in [K]\}$, mask ratios \{$r_{i} \mid i\in [K]\}$, scaling factors $\{\lambda_i\mid i\in[K]\}$, location datasets $\{\mathcal{X}_{i}\mid i\in[K]\}$
        \ENSURE merged parameter $\boldsymbol{\theta}_{m}$
        \STATE $\mathcal{M}\leftarrow\phi$
        \STATE $\boldsymbol{\theta}_{m}\leftarrow \boldsymbol{\theta}_{\rm{base}}$
        \FOR{$i\in [K]$}
        \STATE $I(\boldsymbol{\theta}_i)=\mathbb{E}_{x\sim \mathcal{X}_i}|\boldsymbol{\theta}_{i}\odot \nabla_{\boldsymbol{\theta}_i}\mathcal{L}(x)|$
        \STATE $I(\boldsymbol{\theta}_{\rm{base}})=\mathbb{E}_{x\sim \mathcal{X}_i}|\boldsymbol{\theta}_{\rm{base}}\odot \nabla_{\boldsymbol{\theta}_{\rm{base}}}\mathcal{L}(x)|$
        
        \STATE calculate $\mathcal{T}^{r_i}_{i}$ following Equation \ref{vote}
        \STATE  $\mathcal{M}\leftarrow \mathcal{M}\cup\{\mathcal{T}^{r_i}_i\}$
       
        
   
        
        
        \ENDFOR  
        \FOR{$i\in [K]$}
        
        \STATE calculate $\text{Disjoint}(\mathcal{T}_i^{r_i})$ use Equation~\ref{disjoint_safety}
        \STATE $\boldsymbol{m}_i \leftarrow \boldsymbol{0}$
        \FOR{$d\in \mathcal{T}_i^{r_i}$}
        \STATE $\boldsymbol{m}_{i,d}=1$
        \ENDFOR
        \STATE $\boldsymbol{\theta}_{m}\leftarrow \boldsymbol{\theta}_{m}+\lambda_i \boldsymbol{\tau}_i\odot \boldsymbol{m}_{i}$
        \ENDFOR
    \end{algorithmic}
\end{algorithm}
    %\vspace{-5pt}
\begin{figure*}[h!]
    \centering
    \includegraphics[width=\linewidth]{figs/pipeline_v2.pdf}
    \vspace{-40mm}
    \caption{Overview of our two-stage training pipeline {\ours}.}
    \label{fig:pipeline}
\end{figure*}


\paragraph{LED-Merging: Location, Election, and Disjoint Merging}
To address the neuron misidentification and interference issues in existing model merging methods, we propose LED-Merging (Location, Election, and Disjoint Merging). Specifically, previous studies \cite{modelstock, ilharco2023editing, tiesmerging} fail to accurately identify safety-related neurons in task vectors with a single magnitude score, namely \textit{neuron misidentification}. Meanwhile, there exists an interference between safety-related and utility-related task vector neurons during the merging process, namely \textit{neuron interference}. To address neuron misidentification, we first locate important neurons both in the base and fine-tuned models and then elect neurons from the task vector considering these two scores together. Subsequently, to mitigate the interference, we introduce a disjoint step, isolating these important neurons so that they influence different base neurons. The whole process is illustrated in Figure~\ref{fig:method}. 




In the location and election step, we consider the importance score from base and fine-tuned models simultaneously to locate task-specific neurons. In this way, it is more accurate than relying on the magnitude score alone because task-specific neurons with high importance score in the fine-tuned model may not necessarily score high in the base model, and vice versa.

{\textbf{Location}}.  We first calculate importance scores for each neuron in a base/fine-tuned model. Given a location dataset $\mathcal{X}_i=\{(x,y)_k\}$, where $x$ is the question and $y$ is the answer, we calculate the importance scores for the weight $\boldsymbol{\theta}_i\in\mathbb{R}^D$ in any  layer as follows~\cite{snip,spareseGPT,sun2024a}:
\begin{equation}
    I(\boldsymbol{\theta}_i)=\mathbb{E}_{x\sim \mathcal{X}_i}[\boldsymbol{\theta}_i\odot \nabla _{\boldsymbol{\theta}_i}\mathcal{L}(x)],
    \label{location}
\end{equation}
which $\mathcal{L}(x)=-\log p(y\mid x)$ is the conditional negative log-likelihood loss. We choose the SNIP score~\cite{snip} because it balances computational efficiency and performance~\cite{cq}. Please refer to Sec.~\ref{sec:ablation} for the comparison between different location methods. After computing importance scores, we choose top-$r_i$ neurons as the important neuron subset $\mathcal{N}_{i}^{r_i}$ from $I(\boldsymbol{\theta}_i)$.
 
 % After computing locating scores, we select the neurons scoring both high in base and fine-tuned models as important neurons in task vectors. Then in the disjoint step,  with preventing  polysemantic neurons  from receiving gradient updates towards different directions,
 % we use set difference to isolate the safety   and utility-related neurons  and construct corresponding masks for merging process,

{\textbf{Election}}. A natural question is how to select important neurons in the task vector $\boldsymbol{\tau}_i$ based on $I(\boldsymbol{\theta}_{\rm{base}})$ and $I(\boldsymbol{\theta}_{i})$. The important neurons in the base model may be different from neurons in the fine-tuned model. Therefore, we introduce the following election strategy to select neurons with high scores in both base and fine-tuned models:
\begin{equation}
    \mathcal{T}_i^{r_i}=\mathcal{N}_i^{r_i}\cap \mathcal{N}_{\rm{base}}^{r_i}.
    \label{vote}
\end{equation}
\emph{Remark}. We compare different choosing methods, including scoring low or high in base or fine-tuned model in Section~\ref{sec:ablation} and find that Equation \ref{vote} achieves the best performance.





{\textbf{Disjoint}}. As important neurons from different task vectors may conflict with each other at the same position, we use the set difference to disjoint the neurons from others to prevent interference:
\begin{equation}
    \text{Disjoint}(\mathcal{T}^{r_i}_{i})=\mathcal{T}^{r_i}_{i}-\mathop{\cup}\limits_{{J}\subsetneqq [K],|J|\geq 2}\mathop{\cap}\limits_{j\in {J}}\mathcal{T}^{r_j}_{j}.
    \label{disjoint_safety}
\end{equation}

Next, we construct a mask $\boldsymbol{m}_i\in\mathbb{R}^D$ to implement disjoint in the merging process. Specifically, this mask $\boldsymbol{m}_i$ is used to select neurons from $\mathcal{T}_i$. The mask ratio is $r_i$, where $r\in(0,1]$. The mask $\boldsymbol{m}_i$ can be derived from:
\begin{equation}
    \boldsymbol{m}_{i,d}=\begin{aligned} &\left\{ \begin{array}{ll} 1, & \text{if } d\in \text{Disjoint}(\mathcal{T}_{i}^{r_i}), \\ 0, & \text{otherwise}. \end{array} \right. \end{aligned}
    \label{mask_safety}
\end{equation}


% \subsection{Merging Models with Masks}
{\textbf{Merging}}. The final
merged task vector $\boldsymbol{\tau}_m$ is as follows:
\begin{equation}
    \boldsymbol{\tau}_m= \sum_i \lambda_i\boldsymbol{\tau}_{i}\odot\boldsymbol{m}_i.
    \label{merged_task_vector}
\end{equation}
We summarize the workflow in Algorithm \ref{alg1}.




\section{Experiments}
\label{sec:experiments}

\begin{figure*}[t]
\vspace{-6mm}
    \centering
    \includegraphics[width=0.8\linewidth]{figs/compare.pdf}
    \vspace{-4mm}
    \caption{\textbf{Qualitative comparison} with the baseline for generating a sequence of novel view images.  
    The results demonstrate that our method synthesizes more consistent multi-view images compared to our baseline model (Zero123). In addition, compared to SyncDreamer, our method visually maintains better similarity to the conditioned image and appears more natural.}
    \label{fig:sota_compare}
\vspace{-5mm}
\end{figure*}

\subsection{Experimental Setups}
\textbf{Dataset.}
Following previous work~\cite{zero123, SyncDreamer}, we evaluate our work on the Google Scanned Object (GSO)~\cite{GSO} dataset to verify the zero-shot novel view image synthesis capability. 
We also provide results for additional datasets in the Supplementary Material.
Specifically, we randomly select 30 objects from the GSO dataset with various object categories. 
Unlike recent approaches~\cite{mvdream, SyncDreamer} that aim to enhance the consistency of novel view synthesis models by generating multiple fixed-view images, our method can generate images from any camera pose and any number of views. Therefore, we conduct experiments under different camera pose settings to validate our approach:
specifically, 
1) \textit{16-views with free camera pose}: for each object, we circularly render 16 views with the elevation angles ranging in $[-10\degree, 40\degree]$ and the azimuth angles are evenly distributed in $[0\degree, 360\degree]$. 
2) \textit{16-views with fixed camera pose}: We maintain a constant elevation angle of $30\degree$ and uniformly sample azimuth angles (same as SyncDreamer~\cite{SyncDreamer}).
3) \textit{32-views with free camera pose}: Similar to the first setting, but we sample 32 views.
It's important to note that our method does not require additional training or fine-tuning on any datasets.

\noindent\textbf{Metrics.}
To validate the effectiveness of our method, we mainly evaluate it based on three criteria:
1) \textit{Quality Score}. We evaluate the image quality of synthesized multi-view images by measuring their similarity with ground truth images. Following prior research~\cite{zero123, sparsefusion}, we report the similarity between the synthesized images and the ground truth images with standard metrics: PSNR, SSIM~\cite{ssim}, and LPIPS~\cite{lpips}.
2) \textit{Multi-view Consistency Score}. As the primary goal of our work is to improve the consistency of generated images, we also employ the 3D consistency score~\cite{3dim} to verify the consistency among the synthesized images. Specifically, we train an Instant-NGP~\cite{instant_ngp} with the input image and part of the synthesized novel view images of our model and evaluate the similarity between the remaining synthesized images and the rendered images of Instant-NGP. For the synthesized multi-view images of each object, we allocate $3/4$ for training and reserve the remaining $1/4$ for validation.
Intuitively, if the consistency of synthesized images is improved, the NeRF-like model will train a better object representation, and the re-rendered images will agree more with the validation images.
3) \textit{Input Consistency Score}. To assess the faithfulness of synthesized images in preserving the identity of the input condition image, we introduce the input consistency score. This score calculates the similarity of each synthesized image with the input condition image, utilizing the LPIPS metric.

In addition, we use synthesized multi-view images to train a neural 3D reconstruction model (NeuS~\cite{neus}) and report commonly used Chamfer Distances (CD) and Volume IoUs between the trained 3D model and the ground truth.

\noindent\textbf{Baselines.}
Given that our main goal is to improve the consistency of the trained baseline model without further fine-tuning, we mainly compare our approach with the used baseline model Zero123~\cite{zero123}. Additionally, we compare our method to the SOTA approaches such as PGD~\cite{tseng2023consistent} and SyncDreamer~\cite{SyncDreamer} using the same Zero123 base model.

\noindent\textbf{Implementation Details.}
We use the official checkpoint provided by Zero123~\cite{zero123}, which is trained on objaverse~\cite{objaverse} for 165,000 steps. We inject our epipolar attention layer after step $T=4$ and layer $L=10$ by default. We find that feature fusion weight $\alpha=0.5$, and the number of context views $M=2$ work better.

\begin{table}[t]
\centering
\caption{Comparison of multi-view consistency, image quality, and input consistency of synthesized multi-view images at the 16-view setting with free camera pose.}
\label{tab:view16_free_compare}
\vspace{-2mm}
\scalebox{0.6}{
\begin{tabular}{c ccc ccc c}
\toprule
              & \multicolumn{3}{c}{Multi-view Consistency} & \multicolumn{3}{c}{Quality Score} & \multicolumn{1}{c}{Input Consis.} \\
              \cmidrule(lr){2-4} \cmidrule(lr){5-7} \cmidrule(lr){8-8}
              & PSNR$\uparrow$  & SSIM$\uparrow$ & LPIPS$\downarrow$ 
              & PSNR$\uparrow$  & SSIM$\uparrow$ & LPIPS$\downarrow$ 
              & LPIPS$\downarrow$ 
              \\ \midrule

Zero123
& 15.225        & 0.645       & 0.408
& 14.255        & 0.747       &	0.208
& 0.303         
\\
SyncDreamer
& 14.830        & 0.626       & 0.434
& 12.650        & 0.713       &	0.254
& 0.317         
\\
Ours 
& \best{18.300}	& \best{0.734}	& \best{0.355}
& \best{14.947}	& \best{0.763}	& \best{0.191}
& \best{0.282}
\\

\bottomrule
\end{tabular}
}
\end{table}

\begin{table}[t]
\vspace{-1mm}
\centering
\caption{Comparison of multi-view consistency, image quality, and input consistency at the 16-view setting with fixed camera pose as SyncDreamer~\cite{SyncDreamer}.}
\label{tab:view16_fxied_compare}
\vspace{-3mm}
\scalebox{0.6}{
\begin{tabular}{c ccc ccc c}
\toprule
              & \multicolumn{3}{c}{Multi-view Consistency} & \multicolumn{3}{c}{Quality Score} & \multicolumn{1}{c}{Input Consis.} \\
              \cmidrule(lr){2-4} \cmidrule(lr){5-7} \cmidrule(lr){8-8}
              & PSNR$\uparrow$  & SSIM$\uparrow$ & LPIPS$\downarrow$ 
              & PSNR$\uparrow$  & SSIM$\uparrow$ & LPIPS$\downarrow$ 
              & LPIPS$\downarrow$ 
              \\ \midrule

Zero123
& 16.556        & 0.682       & 0.378
& 14.592        & 0.750       &	0.207
& 0.305         
\\
SyncDreamer
& \best{22.424}        & \best{0.812}       & \best{0.268}
& 15.269        & 0.749       &	0.196
& 0.300         
\\
Ours 
& 21.151	& 0.780	& 0.302
& \best{15.293}	& \best{0.764}	& \best{0.184}
& \best{0.287}
\\

\bottomrule
\end{tabular}
}
\vspace{-4mm}
\end{table}


\subsection{Comparison With Baseline Models}
The quantitative comparison on three settings are shown in Tab.~\ref{tab:view16_free_compare}, Tab.~\ref{tab:view16_fxied_compare}, and Tab.~\ref{tab:view32_free_compare}. The qualitative comparison is shown in Fig.~\ref{fig:sota_compare}.

\begin{table}[t]
\centering
\caption{Comparison of multi-view consistency and image quality scores of synthesized multi-view images at the 32-view setting with free camera pose.}
\vspace{-3mm}
\label{tab:view32_free_compare}
\scalebox{0.7}{
\begin{tabular}{c ccc ccc}
\toprule
              & \multicolumn{3}{c}{Multi-view Consistency} & \multicolumn{3}{c}{Quality Score} \\
              \cmidrule(lr){2-4} \cmidrule(lr){5-7}
              & PSNR$\uparrow$  & SSIM$\uparrow$ & LPIPS$\downarrow$ 
              & PSNR$\uparrow$  & SSIM$\uparrow$ & LPIPS$\downarrow$ 
              \\ \midrule

Zero123
& 16.515        & 0.694       & 0.378
& 15.142        & 0.733       &	0.211
\\
PGD~\cite{tseng2023consistent}
& 18.481        & 0.720       & 0.343
& 15.281        & 0.739       &	0.205
\\
Ours 
& \best{20.655}	& \best{0.792}	& \best{0.305}
& \best{15.268}	& \best{0.742}	& \best{0.203}
\\

\bottomrule
\end{tabular}
}
\vspace{-3mm}
\end{table}

\begin{table*}
  [t]
  \centering
  \resizebox{\textwidth}{!}{%
  \begin{tabular}{cccccccccccc}
    \toprule \multicolumn{2}{c}{Components}                                                             & \multicolumn{5}{c}{Re-executability Rate (\%)} & \multicolumn{5}{c}{Readability (\#)} \\
    \cmidrule(lr){1-2} \cmidrule(lr){3-7} \cmidrule(lr){8-12}        \hspace{8pt}\labelemoji\hspace{8pt}                                                                & \hspace{8pt}\toolemoji\hspace{8pt}                                      & O0                                 & O1             & O2             & O3             & AVG            & O0             & O1             & O2             & O3             & AVG            \\
    \hline
    \rowcolor[rgb]{0.93,0.93,0.93}\multicolumn{12}{c}{\textbf{Initialize with LLM4Decompile-End-6.7B~\citep{llm4decompile}}}   \\
    \xmark                                                                                              & \xmark                                    & 69.51                              & 46.95          & 50.61          & 46.34          & 53.35          & 3.98 & 3.41 & 3.44 & 3.38 & 3.55 \\
    \cmark                                                                                              & \xmark                                    & 75.61                              & 50.61          & 50.00          & 50.00          & 56.55          & 4.01 & 3.44 & 3.39 & \textbf{3.49} & 3.58 \\
    \xmark                                                                                              & \cmark                                    & 83.54                     & \textbf{56.10}          & 51.22          & 50.61 & 60.37 & 4.05 & 3.51 & 3.51 & 3.42 & 3.62 \\
    \cmark                                                                                              & \cmark                                    & \textbf{85.37}                            & \textbf{56.10}                     & \textbf{51.83} & \textbf{52.43}          & \textbf{61.43} & \textbf{4.13} & \textbf{3.60} & \textbf{3.54} & \textbf{3.49} & \textbf{3.69} \\

    \rowcolor[rgb]{0.93,0.93,0.93}\multicolumn{12}{c}{\textbf{Initialize with Deepseek-Coder-6.7B-base~\citep{deepseekcoder}}} \\
    \xmark                                                                                              & \xmark                                    & 59.15                              & 35.98          & 39.02          & 37.80          & 42.99          & 3.71 & 3.05 & 3.16 & 3.05 & 3.24 \\
    \cmark                                                                                              & \xmark                                    & 66.46                              & 41.46          & 38.41          & 36.59          & 45.73          & 3.76 & 3.17 & \textbf{3.21} & 3.08 & 3.31 \\
    \xmark                                                                                              & \cmark                                    & 70.73                              & 39.63          & 39.02          & 40.24          & 47.41          & 3.90 & 3.17 & 3.08 & 3.11 & 3.31 \\
    \cmark                                                                                              & \cmark                                    & \textbf{79.88}                     & \textbf{45.73} & \textbf{43.90} & \textbf{42.68} & \textbf{53.05} & \textbf{3.96} & \textbf{3.21} & 3.18 & \textbf{3.19} & \textbf{3.38} \\
    \bottomrule
  \end{tabular}%
  }
  \caption{The ablation study of different methods across four optimization levels
  (O0, O1, O2, O3), as well as their average scores (AVG). The results in bold represent the optimal performance. The ~\labelemoji~ and ~\toolemoji~ means Relabedling and Function Call. \textbf{Bold} denotes the best performance.}
  \label{tab:ablation}
\end{table*}



\begin{figure*}[ht]
    \centering
    \begin{minipage}{0.65\textwidth}
        \centering
        \includegraphics[width=0.95\linewidth]{figs/ablation.pdf}
        \vspace{-2mm}
        \captionof{figure}{Qualitative Comparison for different design choices. Our method, employing multi-view epipolar attention, demonstrates the best consistency.}
        \label{fig:ablation}
    \end{minipage}\hfill
    \begin{minipage}{0.33\textwidth}
        \centering
        \includegraphics[width=0.8\linewidth]{figs/neus_ver.pdf}
        \vspace{-3mm}
        \caption{Our method shows better direct 3D reconstruction~\cite{neus}.}
        \label{fig:neus}
    \end{minipage}
    \vspace{-5mm}
\end{figure*}

\noindent\textbf{Multi-view Consistency.}
Tab.~\ref{tab:view16_fxied_compare} presents the 3D consistency scores compared to our baseline model (Zero123) and SyncDreamer. The results indicate a significant improvement across all three metrics achieved by our method when compared with Zero123.
While our method exhibits a marginally lower numerical consistency score compared to SyncDreamer, it enables the synthesis of images with arbitrary camera poses.	
This capability is illustrated in Tab.~\ref{tab:view16_free_compare}, where our method consistently enhances consistency with changes in camera pose settings, whereas SyncDreamer fails to do so and exhibits inferior results compared to Zero123.
Furthermore, our method facilitates the synthesis of multi-view images with any number of camera views. This versatility is demonstrated in Tab.~\ref{tab:view32_free_compare}, where our method continues to achieve significant improvements in consistency scores, while SyncDreamer is unable to operate under such conditions.	

Meanwhile, Fig.~\ref{fig:sota_compare} provides a qualitative comparison with the baseline. While both our method and SyncDreamer enhance consistency, our method visually preserves better similarity to the input image, including color and texture details. The input consistency score further corroborates this.

\noindent\textbf{Image Quality.}
While our primary goal centers around enhancing the consistency of synthesized multi-view images, we also evaluate the image quality by comparing the similarity with the ground truth images. The results shown in Tab.~\ref{tab:view16_free_compare}, Tab.~\ref{tab:view16_fxied_compare}, and Tab.~\ref{tab:view32_free_compare} indicate that our method also enhances the image quality under different settings besides improving the consistency.
Moreover, our method shows better image quality compared with SyncDreamer even in the 16-view setting with fixed camera pose.

\noindent\textbf{Input Consistency.}
Input consistency terms whether the results align with the input image.
Fig.~\ref{fig:sota_compare} illustrates that both our method and SyncDreamer enhance multi-view consistency. However, the color and texture details of SyncDreamer's results diverge from the input image and appear visually unnatural.
This discrepancy is evident in the input consistency score presented in Tab.~\ref{tab:view16_fxied_compare}, indicating lower similarity with the condition image in the SyncDreamer results.	

\subsection{Ablation Study}
The overall quantitative results are shown in Tab.~\ref{tab:ablation}, and the qualitative comparisons are shown in Fig.~\ref{fig:ablation}.

\noindent \textbf{Full Attention \vs Epipolar Attention.}
The results presented in Tab.\ref{tab:ablation} and Fig.\ref{fig:ablation} demonstrate that our epipolar attention mechanism can synthesize more consistent multi-view images compared with full attention. Furthermore, our epipolar attention achieves a greater performance improvement compared to full attention when using multiple reference images. This could be attributed to the fact that our epipolar attention more effectively localizes target information, as depicted in Fig.~\ref{fig:full_attn_compare}, thereby reducing noise from the reference images. In the multi-view setting, where multiple reference images are utilized, this noise reduction becomes particularly crucial.
Moreover, it is noteworthy that the epipolar attention mechanism consumes less GPU memory compared to our baseline, as discussed in Sec.~\ref{sec:attn_analysis}.

\noindent \textbf{Attending Single-View \vs Multi-View.}
Applying the epipolar attention significantly improves the consistency between the input and target views. However, the consistency between different views in the unobserved regions of the input view is not well preserved.
After implementing our epipolar attention in the multi-view setting, the consistency across the generated multi-view images is further improved. The last row in Tab.~\ref{tab:ablation} shows that after applying our multi-view epipolar attention, the consistency score is further improved compared with the single-view setting. Besides, the qualitative result in Fig.~\ref{fig:ablation} also shows better consistency among different target views.



\begin{table}[t]
\centering
\vspace{-1mm}
\caption{Comparison of 3D reconstruction results. Our method significantly improves the reconstruction quality.}
\vspace{-3mm}
\label{tab:neus}
\scalebox{0.7}{
\begin{tabular}{c cc}
\toprule
              &  Chamfer Dist.$\downarrow$  & Volume IoU$\uparrow$
\\ \midrule

            Zero123         & 0.017         & 0.819    \\
            SyncDreamer     & \best{0.013}         & \best{0.847}    \\
            Ours            & 0.014	& 0.842 \\

\bottomrule
\end{tabular}
}
\vspace{-5mm}
\end{table}


\vspace{-2mm}
\subsection{Downstream Application}
\vspace{-2mm}
To demonstrate the effectiveness of our method, we also applied it to the downstream 3D reconstruction task. Specifically, we trained the NeuS model~\cite{neus} directly using images synthesized by our method, Zero123, and SyncDreamer, respectively.
The quantitative results in Tab.~\ref{tab:neus} show that the consistent multi-view images synthesized by our method can significantly improve the 3D reconstruction quality.
Additionally, our method exhibits similar performance to SyncDreamer which requires time-consuming re-training.
The qualitative results in Fig.~\ref{fig:neus} show that it is challenging to train the NeuS model directly due to the lack of consistency in the images generated by Zero123. In contrast, our method generates more consistent multi-view images and, therefore, better reconstructs the geometry and texture details.
We show improvements on other downstream applications such as image-to-3D in the Supplementary Material.



\section{Discussion}
\label{section:discussion}


\subsection{Practical Implications for Feedforward Prompting}

Of course, prompting an LLM continuously before the user submits their prompt is significantly most costly over submitting the prompt just once, once the user is ready.

% But user might not be ready, and the cognitive costs is pretty heavy.


\subsection{}


% Does this work well with Chain of Thought actually?
% Maybe this approach will actually incentivize self-prompt-chaining???
% What are the implications of this?


% A benefit of this is certainly more transparency in the LLM
% LLM is so flexible that adding this kind of structure is still okay for the LLM



% What's more costly, entering a prompt, then responding and saying, no i want this, or typing a prompt, and tuning the prompt/expected output to reduce message exchanges?

% Learning to become a better prompter. One is by trial and error experience. Perhaps another is through this feedforward that tells you what you might be able to anticipate.

\section{Conclusion}

In this work, we present RigAnything, an autoregressive transformer-based method that automatically predicts rigs for 3D assets. 
% 
To address the challenges posed by objects with diverse topologies and eliminate the inherent ambiguities in rigging, RigAnything probabilistically predicts the skeletons and assigns skinning weights, eliminating the need for any templates.
% 
This approach allows RigAnything to be trained end-to-end on both RigNet and the diverse Objaverse dataset, ensuring its versatility.
% 
Extensive experiments highlight the superiority of RigAnything across a wide range of object categories, showcasing its effectiveness and generalizability.


\bibliographystyle{ACM-Reference-Format}
\bibliography{reference}

\clearpage

\appendix
\clearpage
\setcounter{page}{1}
\setcounter{figure}{0}
\setcounter{table}{0}
\setcounter{equation}{0}

\appendix


\begin{center}
    {\Large \bfseries Supplementary Materials}\\[1em]
\end{center}




\begin{table*}[t]
    \centering
    \resizebox{1.0\linewidth}{!}{ %< auto-adjusts font size to fill line
    \setlength{\tabcolsep}{4.5pt}
         \begin{tabular}{l|ccccc}
        \toprule
        \textbf{Training Data \textbackslash~Test}  & \textbf{X\textsubscript{test}} & \textbf{M\textsubscript{test}} & \textbf{GC\textsubscript{test}} & \textbf{E\textsubscript{test}} & \textbf{G360\textsubscript{test}} \\
        \midrule
        \rowcolor{LightGray} \multicolumn{6}{l}{ResNet-50}  \\
         \textit{same-domain}       & 5.25  & 5.11   & 3.49  &  8.51 & 11.87\\
         \textit{leave-one-dataset-out}    & 16.31 (\increase{210.7}) & 6.23 (\increase{21.9}) & 6.35 (\increase{82.0}) & 8.25 (\reduce{3.1}) & 20.38 (\increase{71.7}) \\
         \textit{joint-dataset} & \textbf{5.04} (\reduce{4.0}) & 5.88 (\increase{15.1}) & 3.59 (\increase{2.9}) & \textbf{6.04} (\reduce{29.0}) & \textbf{10.55} (\reduce{11.1}) \\
         \midrule
        \rowcolor{LightGray}
        \multicolumn{6}{l}{\methodname-H} \\
        \textit{same-domain}   & 4.62 & 5.19 & 3.01 & 6.11 & 9.44  \\
         \textit{leave-one-dataset-out} & 11.29 (\increase{144.4}) & 5.22 (\increase{0.6}) & 5.13 (\increase{70.4}) & 6.14 (\increase{0.5}) & 13.12 (\increase{39.0}) \\
          \textit{joint-dataset} & \textbf{4.46} (\reduce{3.5}) & \textbf{5.08} (\reduce{2.1}) & 3.20 (\increase{6.3}) & \textbf{5.16} (\reduce{15.6}) & \textbf{9.07} (\reduce{3.9})  \\
        \bottomrule
        \end{tabular}
        }
    \caption{
        Comparison of different training data configurations for gaze estimation.
        Each column represents a specific test dataset: XGaze Test, MPIIFaceGaze Test, GazeCapture Test, EYEDIAP Test, and Gaze360 Test.
        Each row corresponds to a training configuration:
        \textit{Same-domain} means training on the same domain as the test set,
        \textit{leave-one-dataset-out} means training on the remaining four datasets other than the test set, and \textit{joint-dataset} means training on the aggregated Train split of all five datasets.
        The percentages in parentheses indicate the reduction or increment compared to the \textit{same-domain} results, where lower errors indicate better performance.
        For the \textit{leave-one-dataset-out} configuration, the errors reported here are on the Test splits, while the main paper reports errors on the entire dataset.
    }
\label{table:comprehensive}
\end{table*}



\noindent In this supplementary material, we first provide an analysis of the effect of combining multiple domains.
Then, we include additional ablations to investigate the effects of color-jitter augmentation and pixel normalization during the pre-training.
Finally, we present qualitative results, highlighting images captured under diverse and challenging conditions.



\section{Analysis on Combining Multiple Domains}\label{sec:supp_data}

We analyze the effect of different training data configurations on gaze estimation performance.
Specifically, we compare three configurations: training on the same domain (\textit{same-domain}), training on multiple domains excluding the testing domain (\textit{leave-one-dataset-out}), and training on multiple domains including the testing domain (\textit{joint-dataset}).

\Cref{table:comprehensive} shows the comparison of gaze errors for these configurations.
Each column corresponds to a specific test dataset: XGaze Test, MPIIFaceGaze Test, GazeCapture Test, EYEDIAP Test, and Gaze360 Test, while each row represents a training configuration.
This \textit{same-domain} setting is different from the \textit{within-dataset} in the main paper. 
We use the splits defined in Sec.~4.1 of the main paper.
Especially, please note that for MPIIFaceGaze dataset, we train the model on the first 10 subjects and test on the remaining five subjects, different from the typical leave-one-subject-out protocol~\cite{abdelrahman2023l2cs,shi2024agent,ververas20253dgazenet}.

The percentages in parentheses indicate the reduction or increment compared to the \textit{same-domain} results, where lower errors indicate better performance.
Note that, for the \textit{leave-one-dataset-out} configuration, errors on the entire left-out dataset are reported in the main paper, but here we present errors on the Test split to align with the other configurations that require dataset splits.



\paragraph{\textit{Same-domain}}
In general, training and testing on the same domain (\textit{same-domain}) yields the best results, even though datasets combined from multiple domains have the potential to be more diverse.
This emphasizes the persistent challenge of achieving optimal performance when using data from different domains.
The exception observed for E\textsubscript{test} with the ResNet-50 backbone may be attributed to the limited number of samples in the EYEDIAP Train split.

\paragraph{\textit{Leave-one-dataset-out}}
In the \textit{leave-one-dataset-out} configuration, we observe varying tendencies across different test datasets. 
Some datasets achieve errors comparable to the \textit{same-domain} results, while others remain challenging. 
For instance, for M\textsubscript{test} and E\textsubscript{test}, which are relatively less complex, the remaining four datasets provide sufficient information to achieve good performance.
In contrast, for X\textsubscript{test}, GC\textsubscript{test}, and G360\textsubscript{test}, the remaining four datasets fail to fully capture the critical factors required for optimal performance.
This variation highlights the strong dependence of performance on the attributes of the training data.

Importantly, our \methodname-H demonstrates a smaller performance gap compared to ResNet-50 in most cases, with the only exception being EYEDIAP, where the difference is marginal. 
This suggests that \methodname-H is better equipped to learn gaze representations from out-of-domain data with less overfitting, underscoring its enhanced generalization capability.


\paragraph{\textit{Joint-dataset}}

Overall, the \textit{joint-dataset} configuration demonstrates significant promise, creating a single model robust across multiple test domains. 
For \methodname-H, the only exception is GC\textsubscript{test}, where the \textit{joint-dataset} configuration produces a slightly higher error (3.01$\rightarrow$3.20). 
Although this suggests some negative effects from the other four datasets, the effects remain marginal.
While the improvement percentages for \methodname-H are smaller compared to ResNet-50, the absolute errors are consistently lower.


\begin{table}[t]
    \centering
    \resizebox{0.8\linewidth}{!}{ %< auto-adjusts font size to fill line
    \centering
    \setlength{\tabcolsep}{6pt}
        \begin{tabular}{ccc|cccc}
        \toprule
        \textbf{\textit{Real}} & \textbf{\textit{Syn.}} & \textbf{\textit{NV.}}  & \textbf{M} & \textbf{GC} & \textbf{E} & \textbf{G360} \\
        \midrule
        \checkmark &            &            & 6.79 & 7.81 & 6.86 & 12.93 \\
        \checkmark & \checkmark &            & 6.57 & 7.37 & \textbf{6.51} & 13.23 \\
        \checkmark & \checkmark & \checkmark & \textbf{6.21} & \textbf{7.35} & 6.64 & \textbf{12.18} \\
        \bottomrule
        \end{tabular}
    }
    \caption{
    We ablate the pre-training facial datasets by comparing real, synthetic, and novel-rendered images.
    The comparison is performed on the \methodname-B network, followed by training on XGaze.
    The last row represents the full-dataset setting.
    }
\label{table:ablation_data_type}
\end{table}


\begin{figure*}[t]
  \centering
  \begin{subfigure}{0.99\linewidth}
    \centering
    \includegraphics[width=0.99\linewidth]{figs/supp/MAE_without_pixel_norm.pdf}
    \caption{ MAE reconstruction examples without pixel normalization.}
    \label{fig:no_pixel_norm}
  \end{subfigure}
  % \hfill
  \begin{subfigure}{0.99\linewidth}
    \centering
    \includegraphics[width=0.99\linewidth]{figs/supp/MAE_with_pixel_norm.pdf}
    \caption{ MAE reconstruction examples with pixel normalization (Proposed).}
    \label{fig:w_pixel_norm}
  \end{subfigure}
  \caption{Examples comparison of the pixel normalization during the MAE pre-training.
  The left, middle, and right columns show the original image, masked input, and the reconstructed image, respectively.
    }
  \label{fig:pixel_norm_samples}
\end{figure*}


\begin{figure}[t]
  \centering
    \includegraphics[width=0.9\linewidth]{figs/data_diversity/leave_perc_change.pdf}
  \caption{
        Effect of MAE pre-training dataset composition on downstream gaze estimation performance.  
        The horizontal axis represents the incremental accumulation of datasets, while the vertical axis shows the percentage reduction in error relative to the first CelebV-Text dataset~\cite{yu2023celebv}.
    }
  \label{fig:plot_err_vs_data_composition}
\end{figure}


\section{Additional Ablation Studies on Pre-Training}


\paragraph{Effect of Pre-Training Dataset Composition}


Beyond the overall pre-training dataset size, the composition of the dataset also plays a critical role in learning effective gaze representations.
To investigate the impact of different facial dataset components, we conduct an experiment where we incrementally accumulate datasets during the MAE pre-training stage and analyze their effect on the downstream gaze estimation performance.
Starting with CelebV-Text~\cite{yu2023celebv}, we progressively add datasets for pre-training and evaluate model separately on gaze estimation.
Each pre-trained model is subsequently trained on gaze datasets using the same \textit{leave-one-dataset-out} protocol.
\Cref{fig:plot_err_vs_data_composition} illustrates the error change across different test sets as more datasets are included in pre-training.



Overall, the results indicate that adding more diverse data during pre-training generally enhances gaze generalization.
However, there are exceptions that adding a model can result in increased error for specific test sets.
For example, adding VFHQ increases the error on the XGaze Test set from 12.65 to 13.32, while including SFHQ-T2I causes performance fluctuations across different benchmarks.
This suggests that certain dataset attributes may not align well with particular test distributions, leading to suboptimal transferability.
On the other hand, datasets such as VGGFace2 and XGaze-Dense provide performance improvements on most test sets.
Additionally, performance gains becomes marginal as the dataset number increases, aligning with the analysis of pre-training data size in main paper.


In conclusion, dataset diversity plays a crucial role in improving MAE pre-training for gaze estimation.
A more detailed analysis of dataset attributes and their impact on gaze estimation remains an open research question, which we leave for future work.
Nonetheless, our empirical results suggest that increasing data diversity in pre-training tends to improve model performance across various test domains.



\paragraph{Effect of Novel-View Synthesis Data in Pre-Training}\label{sec:ablation_novel_view_data}
To examine the effect of novel-view synthesis in pre-training data, we conduct further experiments separating these two elements.
In \cref{table:ablation_data_type}, we conduct an ablation study by varying data subsets during the pre-training: real datasets (CelebV-Text, VGGFace2, and VFHQ), synthetic datasets (FaceSynthetics and SFHQ-T2I), and novel-view-rendered datasets (FFHQ-NV and XGaze-Dense).
We use the \methodname-B to conduct the experiment due to its time efficiency.
After pre-training, we train on XGaze and test on the rest of the four datasets. 

The results further clarify the effect of different data types on the model's generalizability.
Adding synthetic data (\textit{Real + Syn.}) reduces errors in several test domains compared to using only real data, suggesting the variability of the synthetic data contributes to generalization.
Further incorporating novel-view data (\textit{Real + Syn. + NV}) provides additional performance gains, especially in head-pose generalization, likely due to the expanded range of facial orientations.
This finding supports the idea that a mix of real, synthetic, and novel-view data in MAE pre-training strengthens ViT’s representation learning.



\paragraph{Effect of Pixel Normalization}\label{sec:ablation_pixel_norm}
The patch normalization technique is applied during the MAE pre-training as suggested in~\cite{he2022masked} which is different from reconstructing the natural image, as shown in \cref{fig:pixel_norm_samples}.
We compare models pre-trained with and without patch normalization to investigate its impact.

\paragraph{Effect of Color-Jitter Augmentation}\label{sec:ablation_augmentation}
Color jittering introduces randomness in brightness, contrast, saturation, and hue to simulate diverse lighting conditions, enhancing the robustness of learned features.
We compare models pre-trained with and without color-jitter augmentation to investigate its impact.

\paragraph{Results}
We use the \methodname-B model as the backbone and compare different pre-training settings, followed by training on XGaze and testing on the remaining four datasets.
\Cref{table:ablation_colorjitter_pixelnorm} demonstrates that both color-jitter augmentation and pixel normalization contribute to improved gaze estimation performance, highlighting their benefits for the generalization of the pre-trained model.
Notably, pixel normalization consistently improves performance across all test datasets, aligning with the observations in the original MAE paper~\cite{he2022masked}, which showed that pixel normalization enhances representation learning.



\begin{table}[t]
    \centering
    \resizebox{0.97\linewidth}{!}{ %< auto-adjusts font size to fill line
    \centering
    \setlength{\tabcolsep}{6pt}
        \begin{tabular}{cc|cccc}
        \toprule
        \textbf{Color-Jitter} & \textbf{Pixel Norm.} & \textbf{M} & \textbf{GC} & \textbf{E} & \textbf{G360} \\
        \midrule
         \XSolidBrush & \XSolidBrush     & 7.52 & 8.01 &  8.56 & 14.14 \\
        \checkmark & \XSolidBrush   &  7.17 & 8.23 & 8.03 & 14.03 \\
        \XSolidBrush   & \checkmark &  7.18 & 7.94 & 8.05 & 13.66 \\
        \checkmark & \checkmark & \textbf{6.21} & \textbf{7.35} & \textbf{6.64} & \textbf{12.18} \\
        \bottomrule
        \end{tabular}
    }
    \caption{
        Ablation studies on the pre-training, comparing the effect of the color-jitter augmentation and the pixel normalization.
        During the gaze estimation training, we train the model using XGaze and test on the other four datasets to evaluate the generalizability.
    }
\label{table:ablation_colorjitter_pixelnorm}
\end{table}




\section{Comparison with the SOTAs}
3DGazeNet~\cite{ververas20253dgazenet} collects in-the-wild face images with pseudo gaze labels and applies multi-view synthesis to obtain an augmented dataset ITWG-MV.
To account for the difference in test data settings, we compare 3DGazeNet with \methodname-H separately in \cref{table:supp_sota_3dgn}.
The results demonstrate that \methodname-H outperforms 3DGazeNet in all domain generalization settings.

\paragraph{Re-implementation}
In the main paper, we compared our \methodname-H model with state-of-the-art (SOTA) methods using their reported results. 
It is important to note that minor discrepancies may arise due to differences in our data pre-processing compared to prior work~\cite{cheng2022puregaze,xu2023learning,zhao2024improving}. 
To ensure a fair comparison, we re-implemented ResNet-18 and PureGaze~\cite{cheng2022puregaze} using our pre-processed datasets, aligning them with the reported results~\cite{cheng2022puregaze,zhao2024improving}. 
The re-implementation results, alongside the reported values, are summarized in \cref{table:supp_sota}.

While minor differences exist between our re-implementation and the reported values, the improvements achieved by our \methodname-H model remain significant, demonstrating its superior performance across all domain generalization tasks. 



\begin{table}[t]
\begin{center}
    \resizebox{0.99\linewidth}{!}{ %< auto-adjusts font size to fill line
    \setlength{\tabcolsep}{4pt}
        \begin{tabular}{l|cccc}
        \toprule
        \textbf{Models}  & \textbf{X}$\rightarrow$\textbf{M} & \textbf{X}$\rightarrow$\textbf{GC}& \textbf{G360}$\rightarrow$\textbf{M} & \textbf{G360}$\rightarrow$\textbf{GC} \\
        \midrule
        \rowcolor{LightGray}
        3DGazeNet$^{\dagger}$~\cite{ververas20253dgazenet} & 6.0 & 7.8 & 6.3 & 8.0  \\
        \midrule
        \methodname-H & \textbf{5.57} &  \textbf{6.56} & \textbf{5.43} & \textbf{6.48} \\
        \bottomrule
        \end{tabular}
}
\end{center}
\caption{
    Domain generalization compared with SOTA methods. 
    The results marked with $^{\dagger}$ are directly cited from previous studies~\cite{ververas20253dgazenet}.
}
\label{table:supp_sota_3dgn}
\end{table}


\begin{table}[t]
\begin{center}
    \resizebox{0.99\linewidth}{!}{ %< auto-adjusts font size to fill line
    \setlength{\tabcolsep}{4pt}
        \begin{tabular}{l|cccc}
        \toprule
        \textbf{Models}  & \textbf{X}$\rightarrow$\textbf{M} & \textbf{X}$\rightarrow$\textbf{E\textsubscript{CS}}& \textbf{G360}$\rightarrow$\textbf{M} & \textbf{G360}$\rightarrow$\textbf{E\textsubscript{CS}} \\
        \midrule
        ResNet-18   & 7.57 & 9.54 & 9.24  & 8.07  \\
        \rowcolor{LightGray}
        ResNet-18$^{\dagger}$~\cite{zhao2024improving}  & 8.02 & 9.11 & 8.04 & 9.20 \\
        PureGaze & 6.68 & 7.62 & 8.87 & 10.53 \\ %% the paper used pure50 for XGaze, pure18 for Gaze360
        \rowcolor{LightGray}
        PureGaze$^{\dagger}$~\cite{cheng2022puregaze}  & 7.08 & 7.48 & 9.28 & 9.32 \\
        \midrule
        \methodname-H & \textbf{5.57} &  \textbf{4.65} & \textbf{5.43} & \textbf{5.35} \\
        \bottomrule
        \end{tabular}

}
\end{center}
\caption{
    Domain generalization compared with SOTA methods and their re-implementations. 
    The results marked with $^{\dagger}$ are cited from previous studies~\cite{cheng2022puregaze,zhao2024improving}, and the rest of the results are based on our implementation.
}
\label{table:supp_sota}
\end{table}




\section{Implementation Details}

\paragraph{Novel-Rendered Data Preparation}

To render images from novel views, we follow the rendering approach described in~\cite{qin2022learning}.
To control the head pose, we randomly generate target head poses and compute the corresponding rotation matrices to apply to the 3D face models. 
During the rendering process, 40\% of the images are assigned a random background color, while the remaining 60\% use random scene images from the Places365 dataset~\cite{zhou2017places} as background. 
Additionally, to simulate varied lighting conditions, half of the rendered images are adjusted to have lower ambient light intensity, ranging from $0.2$ to $0.75$.

All face images in our method are in the size of \num{224} $\times$ \num{224} after the data normalization process~\cite{zhang2018revisiting}.
When the camera parameters are unknown, we use a camera matrix with focal length $f$ set to the image width and principal point $(c_x, c_y)$ set to half the image height and width.

\paragraph{Pre-Training}
We apply random color jitter augmentation with a probability of 0.5 and the following parameters: hue in the range $[-0.15, 0.15]$, saturation in $[0.8, 1.2]$, contrast in $[0.4, 1.8]$, and brightness in $[0.7, 1.3]$.
We apply random grayscale with a probability of 0.05 on all images.

\paragraph{Gaze Estimation Training}

We use the Adam optimizer~\cite{kingma2014adam} with a learning rate of \num{1e-4} and a weight decay of \num{1e-6} for all experiments.
For experiments with ResNet-50 and GazeTR-50, we set the batch size to 128 and decay the learning rate by 0.1 every five epochs, with a total of 12 epochs.
For cross-dataset evaluation with \methodname-H, we use a batch size of 128 and train the model for eight epochs with the one-cycle learning rate schedule~\cite{smith2019super}.
For \textit{leave-one-dataset-out} and \textit{joint-dataset} evaluations, we set the batch size to 160 with 12 epochs.

\section{Qualitative Results}
In this section, we present additional qualitative results using the \methodname-H model trained on the aggregated datasets under the \textit{joint-dataset} setting. 
We employ an off-the-shelf facial landmark detector~\cite{bulat2017far} to extract landmarks and perform data normalization.
Gaze estimation is conducted on the normalized images, and the results are de-normalized back to the original image for visualization.
For reference, we also include the normalized faces alongside the original images.

\Cref{fig:results_qual_wild_one_person} and \cref{fig:results_qual_wild} showcase examples from various in-the-wild videos captured under challenging conditions, including large head poses and diverse lighting environments.
Notably, we also include a synthetic example from URAvatar~\cite{li2024uravatar} (bottom row in \cref{fig:results_qual_wild}), which generates faces with controlled viewpoints and lighting.
Furthermore, \cref{fig:results_qual_VAT} presents examples from the gaze-following dataset VideoAttentionTarget~\cite{chong2020detecting}, a collection of diverse samples extracted from movies. 
This dataset provides annotated gaze targets, which are visualized when annotated within the image frame, as some targets may be out of frame.

These examples highlight the model's ability to predict gaze direction accurately in unseen environments, even under extreme head poses, challenging lighting conditions, and synthetic appearances.



\begin{figure*}[t]
    \centering
    \begin{subfigure}{\textwidth}
        \centering
        \includegraphics[width=\linewidth]{fig/qualitative_1_0.png}
    \end{subfigure}
    \hfill
    \vspace{0.2cm}
    \begin{subfigure}{\textwidth}
        \centering
        \includegraphics[width=\linewidth]{fig/qualitative_1_30.png}
    \end{subfigure}
    \begin{subfigure}{\textwidth}
        \centering
        \includegraphics[width=\linewidth]{fig/bar.jpg}
    \end{subfigure}
    \caption{\textbf{Qualitative predictions of a SurroundOcc \cite{wei2023surroundocc} model trained with \method{} on the SurroundOcc-nuScenes \cite{wei2023surroundocc} dataset.} We display the six input camera images (top left), the rendered predictions (bottom left), the BeV ground-truth (top right) and BeV prediction (bottom left). The scene is randomly selected from the validation set and we show predictions at two different timesteps.}
    \label{fig:qualitative_0}
\end{figure*}


\begin{figure*}[t]
    \centering
    \begin{subfigure}{\textwidth}
        \centering
        \includegraphics[width=\linewidth]{fig/qualitative_0_0.png}
    \end{subfigure}
    \hfill
    \vspace{0.2cm}
    \begin{subfigure}{\textwidth}
        \centering
        \includegraphics[width=\linewidth]{fig/qualitative_0_30.png}
    \end{subfigure}
    \begin{subfigure}{\textwidth}
        \centering
        \includegraphics[width=\linewidth]{fig/bar.jpg}
    \end{subfigure}
    \caption{\textbf{Qualitative predictions of a TPVFormer \cite{huang2023tpv} model trained with \method{} on the Occ3d-nuScenes \citep{tian2023occ3d} dataset.} We display the six input camera images (top left), the rendered predictions (bottom left), the BeV ground-truth (top right) and BeV prediction (bottom left). The scene is randomly selected from the validation set and we show predictions at two different timesteps.}
    \label{fig:qualitative_1}
\end{figure*}


\section{Ethical Considerations}

Our research involves the use of existing facial and gaze datasets.
In accordance with ethical guidelines, we rely on the fact that these datasets were originally collected and published following relevant ethical and data protection standards, including obtaining consent, and we do not generate or collect additional new data.
Our experimental protocols involve only image content, with no identifiable personal information or links to other personal data.





\end{document}
%% End of file `sample-acmtog.tex'.
