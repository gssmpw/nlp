% This must be in the first 5 lines to tell arXiv to use pdfLaTeX, which is strongly recommended.
\pdfoutput=1
% In particular, the hyperref package requires pdfLaTeX in order to break URLs across lines.

\documentclass[11pt]{article}

% Change "review" to "final" to generate the final (sometimes called camera-ready) version.
% Change to "preprint" to generate a non-anonymous version with page numbers.
\usepackage[preprint]{acl}

% Standard package includes
\usepackage{times}
\usepackage{latexsym}

% For proper rendering and hyphenation of words containing Latin characters (including in bib files)
\usepackage[T1]{fontenc}
% For Vietnamese characters
% \usepackage[T5]{fontenc}
% See https://www.latex-project.org/help/documentation/encguide.pdf for other character sets

% This assumes your files are encoded as UTF8
\usepackage[utf8]{inputenc}

% This is not strictly necessary, and may be commented out,
% but it will improve the layout of the manuscript,
% and will typically save some space.
\usepackage{microtype}

% This is also not strictly necessary, and may be commented out.
% However, it will improve the aesthetics of text in
% the typewriter font.
\usepackage{inconsolata}

%Including images in your LaTeX document requires adding
%additional package(s)
\usepackage{graphicx}
\usepackage{microtype}
\usepackage{graphicx}
\usepackage{subfigure}
\usepackage{booktabs} % for professional tables

\usepackage{hyperref}
\usepackage{tabularx}

% Attempt to make hyperref and algorithmic work together better:
\newcommand{\theHalgorithm}{\arabic{algorithm}}

% For theorems and such
\usepackage{amsmath}
\usepackage{amssymb}
\usepackage{mathtools}
\usepackage{amsthm}

% if you use cleveref..
\usepackage[capitalize,noabbrev]{cleveref}

% Custom Packages
% \usepackage{xspace}

%%%%%%%%%%%%%%%%%%%%%%%%%%%%%%%%
% THEOREMS
%%%%%%%%%%%%%%%%%%%%%%%%%%%%%%%%
\theoremstyle{plain}
\newtheorem{theorem}{Theorem}[section]
\newtheorem{proposition}[theorem]{Proposition}
\newtheorem{lemma}[theorem]{Lemma}
\newtheorem{corollary}[theorem]{Corollary}
\theoremstyle{definition}
\newtheorem{definition}[theorem]{Definition}
\newtheorem{assumption}[theorem]{Assumption}
\theoremstyle{remark}
\newtheorem{remark}[theorem]{Remark}

% Todonotes is useful during development; simply uncomment the next line
%    and comment out the line below the next line to turn off comments
%\usepackage[disable,textsize=tiny]{todonotes}
\usepackage[textsize=tiny]{todonotes}

% Custom packages
\usepackage[utf8]{inputenc} % allow utf-8 input
\usepackage[T1]{fontenc}    % use 8-bit T1 fonts
\usepackage{hyperref}       % hyperlinks
\usepackage{url}            % simple URL typesetting
\usepackage{booktabs}       % professional-quality tables
\usepackage{amsfonts}       % blackboard math symbols
\usepackage{nicefrac}       % compact symbols for 1/2, etc.
\usepackage{microtype}      % microtypography
\usepackage{xcolor}         % colors
\definecolor{darkgreen}{rgb}{0,0.6,0} 
\usepackage{longtable}
\usepackage{graphicx}
\usepackage{array}
\usepackage{multirow}
% \usepackage[svgnames]{xcolor}
\usepackage{xcolor,colortbl}
\usepackage{xspace}
\usepackage{makecell}
\usepackage{boldline}
\usepackage{pifont}
\usepackage{tabularx}
\usepackage{graphicx}
\usepackage{wrapfig}
\usepackage[export]{adjustbox}
\usepackage{fancyvrb}
\usepackage{fvextra}
\usepackage[most,breakable]{tcolorbox}
% \usepackage[rawfloats]{floatrow}
\usepackage{caption}
\usepackage{pgfplots}
\pgfplotsset{compat=newest}
\usepgfplotslibrary{units}
\usepackage{placeins}
\usepackage{lineno}
\usepackage{enumitem}

\usepackage{listings}

\definecolor{codegreen}{rgb}{0,0.6,0}
\definecolor{codegray}{rgb}{0.5,0.5,0.5}
\definecolor{codepurple}{rgb}{0.58,0,0.82}

\lstdefinestyle{pddl}{
    language=Lisp, 
    basicstyle=\ttfamily\footnotesize,
    commentstyle=\color{codegreen},
    keywordstyle=\color{magenta},
    numberstyle=\tiny\color{codegray},
    stringstyle=\color{codepurple},
    breakatwhitespace=false,         
    breaklines=true,                 
    captionpos=b,                    
    keepspaces=true,                 
    % numbers=left,                    
    numbersep=5pt,                  
    showspaces=false,                
    showstringspaces=false,
    showtabs=false,                  
    tabsize=2,
    morekeywords={define, domain, requirements, predicates, parameters, precondition, effect, action, :, and, not},
    deletekeywords={}, 
    morecomment=[l]{;} 
}

\newcommand{\benchmark}[0]{\textsc{Text2World}\xspace}
\newcommand{\brieftitle}[0]{\textsc{Text2World}\xspace}
\newcommand{\exec}[0]{\textsc{Exec.}\xspace}
\newcommand{\simmetric}[0]{\textsc{Sim.}\xspace}
\newcommand{\fonepred}[0]{\textsc{F1\textsubscript{pred}}\xspace}
\newcommand{\foneparam}[0]{\textsc{F1\textsubscript{param}}\xspace}
\newcommand{\foneprecond}[0]{\textsc{F1\textsubscript{precond}}\xspace}
\newcommand{\foneeff}[0]{\textsc{F1\textsubscript{eff}}\xspace}

\begin{document}

% \twocolumn[
% \title{%
%   \raisebox{-0.3ex}{\includegraphics[width=0.5cm]{images/world.pdf}}%
%    \hspace{2px}\brieftitle: Benchmarking World Modeling Capabilities of Large Language Models via Program Synthesis%
% }]


\title{%
  \raisebox{-0.35ex}{\includegraphics[width=0.55cm]{images/world.pdf}}%
   % \hspace{2px}\brieftitle: Benchmarking World Modeling Capabilities of Large Language Models via Program Synthesis%
  \hspace{3.5px}\brieftitle: Benchmarking Large Language Models for \\ Symbolic World Model Generation%
}

% Author information can be set in various styles:
% For several authors from the same institution:
% \author{Author 1 \and ... \and Author n \\
%         Address line \\ ... \\ Address line}
% if the names do not fit well on one line use
%         Author 1 \\ {\bf Author 2} \\ ... \\ {\bf Author n} \\
% For authors from different institutions:
% \author{Author 1 \\ Address line \\  ... \\ Address line
%         \And  ... \And
%         Author n \\ Address line \\ ... \\ Address line}
% To start a separate ``row'' of authors use \AND, as in
% \author{Author 1 \\ Address line \\  ... \\ Address line
%         \AND
%         Author 2 \\ Address line \\ ... \\ Address line \And
%         Author 3 \\ Address line \\ ... \\ Address line}

% \author{
% Mengkang Hu \\
% % \\\And
% \\\And
% Tianxing Chen \\
% \\\And
% Yude Zou \\
% \\\And
% Yuheng Lei \\
% \\\And
% Qiguang Chen \\
% \\\And
% Ming Li \\
% \\\And
% Yao Mu \\
% \\\And
% Hongyuan Zhang \\
% \\\And
% Wenqi Shao \\
% \\\And
% Ping Luo \\
% }

\author{Mengkang Hu\footnotemark[1]$^{\spadesuit}$ \quad Tianxing Chen\footnotemark[1]$^{\spadesuit}$ \quad Yude Zou\thanks{~Equal Contribution. $\dagger$~Corresponding Author}$^{\clubsuit}$ \quad Yuheng Lei$^{\spadesuit}$ \quad Qiguang Chen$^{\heartsuit}$ \\
	\textbf{Ming Li$^{\spadesuit}$ \quad Yao Mu$^\spadesuit$ \quad Hongyuan Zhang$^\spadesuit$ \quad Wenqi Shao$^\diamondsuit$ \quad Ping Luo$^\dagger$$^\spadesuit$} \\
	$^{\spadesuit}$ The University of Hong Kong \quad $^{\clubsuit}$  Shenzhen University \\
        $^{\heartsuit}$ Harbin Institute of Technology \quad
	$^{\diamondsuit}$ Shanghai AI Laboratory\\
	\texttt{mkhu@connect.hku.hk}, \quad \texttt{pluo.lhi@gmail.com}\\
    \href{https://text-to-world.github.io/}{text-to-world.github.io}  
    }

%\author{
%  \textbf{First Author\textsuperscript{1}},
%  \textbf{Second Author\textsuperscript{1,2}},
%  \textbf{Third T. Author\textsuperscript{1}},
%  \textbf{Fourth Author\textsuperscript{1}},
%\\
%  \textbf{Fifth Author\textsuperscript{1,2}},
%  \textbf{Sixth Author\textsuperscript{1}},
%  \textbf{Seventh Author\textsuperscript{1}},
%  \textbf{Eighth Author \textsuperscript{1,2,3,4}},
%\\
%  \textbf{Ninth Author\textsuperscript{1}},
%  \textbf{Tenth Author\textsuperscript{1}},
%  \textbf{Eleventh E. Author\textsuperscript{1,2,3,4,5}},
%  \textbf{Twelfth Author\textsuperscript{1}},
%\\
%  \textbf{Thirteenth Author\textsuperscript{3}},
%  \textbf{Fourteenth F. Author\textsuperscript{2,4}},
%  \textbf{Fifteenth Author\textsuperscript{1}},
%  \textbf{Sixteenth Author\textsuperscript{1}},
%\\
%  \textbf{Seventeenth S. Author\textsuperscript{4,5}},
%  \textbf{Eighteenth Author\textsuperscript{3,4}},
%  \textbf{Nineteenth N. Author\textsuperscript{2,5}},
%  \textbf{Twentieth Author\textsuperscript{1}}
%\\
%\\
%  \textsuperscript{1}Affiliation 1,
%  \textsuperscript{2}Affiliation 2,
%  \textsuperscript{3}Affiliation 3,
%  \textsuperscript{4}Affiliation 4,
%  \textsuperscript{5}Affiliation 5
%\\
%  \small{
%    \textbf{Correspondence:} \href{mailto:email@domain}{email@domain}
%  }
%}


\maketitle
\begin{abstract}
Recently, there has been growing interest in leveraging large language models (LLMs) to generate symbolic world models from textual descriptions. 
% However, despite extensive exploration, no comprehensive benchmark exists to assess the effectiveness of LLMs in world modeling.
% Prior studies encounter several challenges, including evaluation randomness, dependence on indirect metrics, and a limited domain scope.
Although LLMs have been extensively explored in the context of world modeling, prior studies encountered several challenges, including evaluation randomness, dependence on indirect metrics, and a limited domain scope.
To address these limitations, we introduce a novel benchmark, \benchmark, based on planning domain definition language (PDDL), featuring hundreds of diverse domains and employing multi-criteria, execution-based metrics for a more robust evaluation.
We benchmark current LLMs using \benchmark and find that reasoning models trained with large-scale reinforcement learning outperform others.
However, even the best-performing model still demonstrates limited capabilities in world modeling.
Building on these insights, we examine several promising strategies to enhance the world modeling capabilities of LLMs, including test-time scaling, agent training, and more.
We hope that \benchmark can serve as a crucial resource, laying the groundwork for future research in leveraging LLMs as world models.
\end{abstract}

\section{Introduction}
\label{sec:intro}

% 1. Background
The significance of world models for intelligent behavior has been historically acknowledged in early psychological theories, which posited that organisms employ internal representations of the external world for prediction and planning~\citep{craik1967nature}.
Furthermore, ~\citet{lecun2022path} extends this concept by highlighting world modeling as a core component of autonomous machine intelligence.
In this paper, we primarily study \textit{symbolic world models} (also known as domain models), which are formal representations of an environment’s dynamics and constraints.
In recent years, Large Language Models (LLMs)~\cite{opeiai2022gpt,yang2024qwen2,meta2024llama3} have showcased their understanding of common-world knowledge, making them promising candidates for generating symbolic world models, which requires inferring action dynamics and constraints from solely natural language description.
Some works have already explored this across numerous tasks, including planning~\citep{hu2024agentgen, guan2023leveraging}, game design~\citep{wang2023bytesized32,wang2024can}, reinforcement learning~\cite{tang2024worldcoder} among others.

% embodied agents~\citep{xiang2024language},
% In this paper, a \textit{world model} refers to a computational framework that allows an agent to represent and predict the dynamics of its environment~\cite{ha2018world}.
% In recent years, Large Language Models (LLMs) have demonstrated their ability to capture a vast amount of common-world knowledge by training on large text corpora~\cite{opeiai2022gpt,openai2023gpt4,meta2024llama3}. 
% This extensive world knowledge makes LLMs promising candidates for generating symbolic world models.

% 2. Prior works
% Two predominant paradigms have emerged in leveraging LLMs as world models:
% \textbf{(1)} \textbf{Direct State Prediction}: direct application of LLMs for state prediction during planning~\cite{hao2023reasoning,wang2023promptagent}, where benchmarks like ByteSized32-State-Prediction~\cite{wang2024can} have been developed to evaluate LLMs' predictive capabilities. However, this evaluation approach faces significant challenges: exhaustively evaluating all possible state transitions becomes intractable in complex environments, and limited interpretability and high inference latency constrain deeper analysis of LLMs' world modeling capabilities.
% \textbf{(2)} \textbf{Program Synthesis}: Along this direction, other works leverage \textit{program synthesis}, which transforms natural language descriptions into explainable and verifiable formal world specifications. 
% Despite extensive exploration, no comprehensive benchmark for generating symbolic world models with LLMs is currently available.
% Previous work suffers from several key limitations:
Despite extensive exploration, previous work for evaluating symbolic world model generation suffers from several key limitations:
% no comprehensive benchmark for generating symbolic world models with LLMs is currently available.
\textit{(i)} \textit{\textbf{Limited Domain Scope}}: These studies are often confined to a narrow set of domains (typically fewer than 20), which limits the generalizability and applicability of their findings~\cite{oswald2024large,silver2024generalized,wong2023word}.
\textit{(ii)} \textit{\textbf{Evaluation Randomness}}: Some works rely on LLM-based evaluation methods, which may introduce additional margins of error~\cite{wang2023bytesized32}. 
% , thereby affecting the accuracy of assessment~\cite{wang2023bytesized32}. 
Preliminary experiments in Section~\ref{sec:llmself} demonstrate that the LLM-based evaluation exhibits a low inter-annotator agreement with human annotators (Cohen’s $\kappa=0.10$).
% (ii) \textit{\textbf{Indirect Evaluation Metrics}}: Other studies generate world models for use in model-based planning and then evaluate the quality of these models based solely on the end-to-end success rate. This approach does not directly assess the world model itself, making it difficult to pinpoint specific failure modes in state prediction or reasoning~\cite{guan2023leveraging,dainese2024generating}.
\textit{(iii)} \textit{\textbf{Indirect Evaluation}}: Some studies evaluate world models based on end-to-end success rates in model-based planning, making it difficult to identify specific failure modes~\cite{guan2023leveraging,dainese2024generating}.


% 3. Ours
Motivated by these issues, this paper introduces a novel benchmark \benchmark based on the Planning Domain Definition Language (PDDL) as illustrated in Figure~\ref{fig:main}.
% As illustrated in Figure~\ref{fig:main}, the task involves an LLM mapping natural language descriptions to symbolic world models, where the model must infer the environment's dynamics and constraints from textual descriptions alone.
Specifically, to address the first issue, we initially gathered a broad set of domains, which were then filtered through an automated pipeline and manually curated to ensure their quality, ultimately resulting in a collection of hundreds of diverse domains.
% Furthermore, to mitigate the randomness associated with LLM-based evaluations, we designed multi-criteria, execution-based metrics that allow for a more robust assessment of world modeling capabilities.
% Furthermore, to mitigate the randomness associated with LLM-based evaluations and overcome the limitations of indirect evaluation metrics, we designed multi-criteria, execution-based metrics that directly assess the generated world model, ensuring a more robust evaluation.
Furthermore, to tackle the second issue, we designed multi-criteria, execution-based metrics to ensure a more robust assessment. Specifically, we not only employed structural similarity for an overall evaluation but also designed component-wise F1 scores to assess finer-grained aspects such as action dynamics.
Moreover, to overcome the third issue, we systematically apply these metrics to assess the generated world model directly, eliminating reliance on indirect feedback mechanisms.
% These metrics directly assess the generated world model, overcoming the limitations of the third issue.
% We also performed data contamination analysis using n-gram matching. The results showed a lower contamination rate ($\mu$ = 0.04) compared to prior works~\citep{guan2023leveraging,smirnov2024generating}, suggesting the benchmark effectively tests models' world modeling capabilities rather than pattern memorization.

We also performed data contamination analysis using n-gram matching~\cite{touvron2023llama}, revealing a lower contamination rate ($\mu$ = 0.04) compared to prior works~\citep{guan2023leveraging,smirnov2024generating}, indicating that \benchmark effectively evaluates LLMs' world modeling capabilities rather than pattern memorization.

% with their corresponding natural language descriptions.
% To construct \benchmark, we first collected thousands of PDDL files from public repositories and competitions, which were then filtered through a comprehensive pipeline including syntax validation, similarity deduplication, and etc. 
% We also applied manual curation to ensure the quality of the gathered domains.
% To construct \benchmark, we collected thousands of PDDL domain files from public repositories and competitions, which were then filtered through a comprehensive automatic pipeline and manually curated to ensure the quality of the gathered domains.
% Subsequently, annotators with PDDL expertise created natural language descriptions for each domain with regular inspection by senior researchers.
% We designed action descriptions to be high-level natural language explanations without explicit preconditions and effects, focusing on evaluating the models' ability to infer world dynamics from descriptive text. 


%% 4. Experiments
We used \benchmark to benchmark the world modeling capabilities of 16 different LLMs from 9 model families. 
Experimental results in Table~\ref{tab:main} highlight several key findings: 
% \textit{(i)} open-source models show a notable performance gap compared to proprietary models; 
\textit{(i)} \textit{The most advanced LLMs still struggle with \benchmark;}
\textit{(ii)} \textit{large reasoning models trained by reinforcement learning show stronger world modeling capabilities;}
and \textit{(iii)} \textit{error correction significantly improves model performance.}
% Benefiting from the interpretable nature of \benchmark, we also performed a deeper analysis of the failure modes when generating symbolic world models and found that the majority of the failure modes were due to the LLMs' inability to include essential preconditions or effects. 
To gain a deeper understanding, we performed a manual error analysis and found that the majority were due to the LLMs' inability to include essential preconditions or effects. 
% We also explored various strategies to enhance LLM's world modeling performance. 
We also explored several strategies to enhance the world modeling capabilities of LLMs. 
Specifically, we initially experimented with scaling the test-time budget and observed consistent improvements as the test-time budget increased.  
Additionally, methods like fine-tuning and in-context learning contributed positively to model effectiveness.
Moreover, we found that supervised fine-tuning on agent trajectory data yielded unexpected gains, underscoring the importance of robust world modeling for developing high-performing agents.

%% 5. Our contributions? + Conclusion
% We hope that \benchmark will catalyze future research endeavors in world modeling capabilities of LLMs, advancing the field toward more robust and autonoumous artificial intelligence systems. 
To facilitate further research, benchmark and code are available at  \href{https://text-to-world.github.io/}{this URL}.

\begin{figure*}[htbp]
    \centering
    \includegraphics[width=\linewidth]{images/main.pdf}
    \caption{
    Overview of \benchmark.
    }
    \label{fig:main}
\end{figure*}

\section{Preliminary}


% \subsection{Classical Planning Problem (optional?)}
% We first formally define classical planning problems, which form the foundation for understanding PDDL domain specifications. A classical planning problem is defined as a tuple $P = \langle D, I, G \rangle$, where $I$ represents the initial state and $G$ specifies the goal conditions. $D = \langle F, A \rangle$ is the domain specification, where $F$ is the set of fluents (state variables represented as predicates) and $A$ is the set of possible actions.
% A solution to a planning problem is a sequence of actions that transforms the initial state $I$ into a state satisfying the goal specification $G$.

\subsection{World Model}

We formally define a symbolic world model as $D = \langle F, A \rangle$, where $F$ represents the set of fluents (state variables represented as predicates) and $A$ is the set of possible actions. 
Each fluent $f \in F$ is a predicate of the form $p(x_1, ..., x_n)$, where $p$ is the predicate name and $x_1, ..., x_n$ are typed variables. 
% The state of the world at any given time is described by the truth values assigned to all ground instances of these fluents. 
Each action $a \in A$ is defined as a tuple $a = \langle \alpha, \mathcal{P}, \varphi, \mathcal{E} \rangle$ where: 
i) $\alpha$ denotes the action signature (identifier); 
ii) $\mathcal{P}$ represents a list of typed parameters $(p_1, ..., p_k)$; 
iii) $\varphi$ specifies the preconditions: a logical formula over fluents that must hold for the action to be applicable; 
and iv) $\mathcal{E}$ defines the effects: a set of fluent literals describing how the action changes the world state. 
% The world model $D$ thus provides a complete formal specification of: i) the state space, defined by all possible assignments to the fluents in $F$; ii) the transition dynamics, encoded by the actions in $A$ that specify how the world state can change; and iii) the valid configurations and constraints of the domain through the typing system and predicate definitions. \
% This formalization allows for representing complex planning domains while maintaining the classical assumptions of deterministic actions, fully observable states, and static world dynamics outside of explicit action effects.

\subsection{Task Definition}
\label{sec:task_define}

% Program synthesis aims to automatically generate a program that implements a natural language specification.
% In \benchmark, we aim to automatically generate a formal world model $D$ given a natural language description $\mathcal{N}$ that accurately captures the described world dynamics.
The task is formally defined as: $\mathcal{M}: \mathcal{N} \rightarrow D, D \models \mathcal{N}$, where $\mathcal{M}$ is a mapping function (implemented by an LLM) that generates world model $D$ from the natural language description $\mathcal{N}$. $\models$ denotes semantic satisfaction.
% , requiring that $D$ correctly models the world dynamics described in $\mathcal{N}$.
Each $\mathcal{N}$ contains the following components: 
i) A general description describing the overall objective of the domain; 
ii) A set of predicates $\mathcal{N}_{F} = \{f_1, ..., f_n\}$ where each predicate is described with its signature (e.g., ``\textit{(conn ?x ?y)}'') and an explanation (e.g., ``\textit{Indicates a connection between two places ?x and ?y}''); 
iii) A set of actions $\mathcal{N}_{A} = \{a_1, ..., a_m\}$ where each action is described with: its signature (e.g., ``\textit{move <?curpos> <?nextpos>}'') and an explanation (e.g., ``\textit{Allows the robot to move from place <?curpos> to place <?nextpos>}'').
% An overview of the process of \benchmark is presented in Figure~\ref{fig:main}.
Note that to evaluate LLMs' inherent world modeling capabilities, action descriptions in $\mathcal{N}_{A}$ are intentionally kept at a high level, without explicit specifications of preconditions $\varphi$ and effects $\mathcal{E}$. This design choice allows us to assess how well LLMs can infer the underlying world dynamics and constraints from purely descriptive text.
% providing only natural language explanations of what each action does
% We present experimental results comparing model performance conditioned on different description styles in Section~\ref{sec:concrete}.
A comparative analysis of model performance conditioned on different description styles is presented in Section~\ref{sec:concrete}.


\subsection{Evaluation Metrics}

We directly evaluate generated world models, addressing the ambiguity associated with indirect evaluations~\cite{guan2023leveraging,dainese2024generating}. 
In addition, we propose using execution-based metrics, overcoming the randomness of LLM-based evaluation~\cite{wang2023bytesized32}.
Specifically, we established the following evaluation metrics: 
\textit{(i)} \textit{\textbf{Executability (\exec):}} Measures whether the generated PDDL can be successfully parsed and validated by standard PDDL validators. 
\textit{(ii)} \textit{\textbf{Structural Similarity (\simmetric):}} Quantifies the textual similarity between the generated and ground truth PDDL using normalized Levenshtein ratio. 
\textit{(iii)} \textit{\textbf{Component-wise F1 Scores:}} When generated PDDL achieves executability (\exec = 1), we perform fine-grained analysis by calculating the macro-averaged F1 score for each component type (predicates, actions, etc.). More specifically, we compute F1 scores for predicates (\textbf{\fonepred}), parameters (\textbf{\foneparam}), preconditions (\textbf{\foneprecond}), and effects (\textbf{\foneeff}) by parsing both generated and ground truth PDDL into structured representations. 
% Detailed explanations of the evaluation metrics is presented in Appendix~\ref{app:eval_metric};


\section{Benchmark Construction}

The overall process of benchmark construction is shown in Figure~\ref{fig:combined}. In this section, we provide a detailed explanation of each stage.

\begin{figure*}[htbp]
    % 左侧图片
    \begin{minipage}{0.77\linewidth}  % 调整宽度
        \centering
        \includegraphics[width=\linewidth]{images/benchmark_construction.pdf}
    \end{minipage}%
    % 间隔
    \hfill
    % 右侧表格
    \begin{minipage}{0.23\linewidth}  % 调整宽度
        \centering
        \resizebox{\linewidth}{!}{  % 调整表格至合适的宽度
            \begin{tabular}{lcc}
                \toprule
                \textbf{Statistic} & \textbf{Number} \\
                \midrule
                \rowcolor[HTML]{F2F2F2} 
                \textit{Domain Count} &  \\
                \midrule
                Domain & 103 \\
                - Train & 2 \\
                - Test & 101 \\
                \midrule
                \rowcolor[HTML]{F2F2F2} 
                \textit{Token Count} &  \\
                \midrule
                Description & 851.6 $\pm$ 515.2 \\
                - Min/Max & [159, 2814] \\
                Domain & 1187.2 $\pm$ 1212.1 \\
                - Min/Max & [85, 7514] \\
                \midrule
                \rowcolor[HTML]{F2F2F2} 
                \textit{Line Count} &  \\
                \midrule
                Domain & 75.4 $\pm$ 62.9 \\
                - Min/Max & [9, 394] \\
                \midrule
                \rowcolor[HTML]{F2F2F2} 
                \textit{Component Count} &  \\
                \midrule
                Actions & 4.5 $\pm$ 2.8 \\
                - Min/Max & [1, 16] \\
                Predicates & 8.1 $\pm$ 4.8 \\
                - Min/Max & [1, 25] \\
                Types & 1.1 $\pm$ 1.3 \\
                - Min/Max & [1, 8] \\
                \bottomrule
            \end{tabular}
        }
    \end{minipage}
    % 公共标题
    \caption{\textit{Left}: Dataset construction process including: (a) \textit{Data Acquisition} (\S\ref{sec:data_acquisition}); (b) \textit{Data Filtering and Manual Selection} (\S\ref{sec:data_filtering}); (c) \textit{Data Annotation and Quality Assurance} (\S\ref{sec:data_annotation} and \S\ref{sec:quality_assurance}). \textit{Right}: Key statistics of \texttt{\benchmark}. Tokens are counted by GPT-2~\cite{openai2019gpt2} tokenizer. The style is referenced from \citet{chen-etal-2024-m3cot}.}
    \label{fig:combined}
\end{figure*}



\subsection{Data Acquisition}
\label{sec:data_acquisition}

Our benchmark construction process began with collecting PDDL files from various public repositories and planning competitions. 
Through this initial collection phase, we accumulated 1,801 raw PDDL files.
We performed several preprocessing steps to standardize the data format (e.g., convert files with BOM encoding to standard UTF-8).
% These preprocessing steps ensured consistent file formatting and eliminated potential parsing issues in subsequent stages of benchmark construction. 
The processed files served as the foundation for our dataset construction.

\subsection{Data Filtering and Manual Selection}
\label{sec:data_filtering}

To ensure the quality and reliability of \benchmark, we implemented a comprehensive filtering pipeline: 
\textit{(i)} \noindent \textit{\textbf{Validation:}} 
We employed a PDDL domain parser to perform syntax validation on each file;
% , retaining only those that could be successfully parsed as valid PDDL domains.
\textit{(ii)} \noindent \textit{\textbf{Similarity Deduplication:}} 
We eliminated duplicate entries by computing pairwise cosine similarity on TF-IDF vectorized PDDL content, removing files with similarity scores exceeding 0.9;
% To maintain diversity within \benchmark, we eliminated duplicate entries by computing pairwise cosine similarity between files using TF-IDF vectorized PDDL content, removing pairs with similarity scores exceeding 0.9.
\textit{(iii)} \noindent \textit{\textbf{Complexity Control:}} 
% For complexity control, we established thresholds by removing domains with over 40 predicates or 20 actions to balance expressiveness and practical utility.
We removed domains with over 40 predicates or 20 actions to balance expressiveness with practical utility.
\textit{(iv)} \noindent \textit{\textbf{Token Length Filtering:}} We removed files exceeding 5,000 tokens using GPT-2~\cite{openai2019gpt2} tokenizer to ensure compatibility with model context windows.
Additionally, we conducted manual selection to eliminate domains that were not designed for world modeling (such as blocksworld-mystery) and low-quality cases that were not captured by the automated filtering methods.
After this process, we obtained 264 high-quality PDDL domain specifications.

% 例如把用词完全不同但是语义相同的case给筛选掉。

\subsection{Data Annotation}
\label{sec:data_annotation}

After obtaining the high-quality PDDL domains, we manually annotated natural language descriptions for each domain. 
To ensure the quality of annotations, we recruited 6 computer science graduates as annotators.
The annotated description followed the structured format described in Section~\ref{sec:task_define}, and annotators were required to follow the annotation criteria:
\textit{(i)} \textit{\textbf{Descriptive Completeness:}} Annotations must contain all required components;
\textit{(ii)} \textit{\textbf{Action Abstraction:}} Action descriptions should avoid explicit references to formal preconditions and effects;
\textit{(iii)} \textit{\textbf{Inference-Enabling:}} Descriptions should contain sufficient contextual information to allow models to infer the underlying dynamics;
\textit{(iv)} \textit{\textbf{Natural Language Priority:}} Technical terminology should be minimized in favor of natural language explanations.
% These descriptions included a general overview of the domain's objectives and environment and explanations of each predicate and action.
% We also required the annotators to annotate the concrete description used in Section~\ref{sec:concrete} in this stage.
Examples of \benchmark can be found in Appendix~\ref{app:dataset-example}.

\subsection{Quality Assurance}
\label{sec:quality_assurance}

\noindent \textbf{Manual Recheck}
To maintain rigorous quality standards throughout the annotation process, we established a review system supervised by two senior experts. 
These experts conducted regular inspections of the annotations, ensuring accuracy and consistency. 
Inspectors must verify all data twice to determine if the annotated examples meet the specified annotation standards. Examples are accepted only if both inspectors approve them. The verification results showed "almost perfect agreement" with a Fleiss Kappa~\cite{10.2307/2529310} score of 0.82.
Through this comprehensive quality control process, we compiled a final curated dataset of 103 domains with gold-standard descriptions. 

\noindent \textbf{Data Contamination}
% As studied in~\citet{carlini2021extracting}, LLMs may memorize content from their training datasets, which could result in mere memorization rather than developing genuine world modeling capabilities.
% Therefore, we conducted a preliminary systematic analysis to assess potential data contamination between LLMs' training data and our benchmark. Similar to~\citet{carlini2021extracting}, we generated complete PDDL domains based on the first 20 tokens of each domain using GPT-4~\cite{openai2023gpt4}. Subsequently, we computed the contamination rate using a token-based approach that matches tokenized 10-grams between gold domains and generated domains, allowing up to 4 token mismatches to account for minor variations~\cite{touvron2023llama}. To focus on actual content overlap rather than structural similarities, we excluded PDDL-specific keywords (e.g., :parameters, :precondition), syntax tokens, and parameter variables from the n-gram computation.
% For comparison, we also analyzed domains used in prior world modeling works with LLMs~\cite{guan2023leveraging,smirnov2024generating}.
% As shown in Figure~\ref{fig:cont}, \benchmark demonstrates a lower contamination score with a mean value of $\mu = 0.04$, suggesting that the models' performance on \benchmark is more likely attributed to their ability to understand and formalize domain dynamics rather than simple memorization of PDDL patterns. However, we acknowledge that complete elimination of contamination is challenging due to the widespread use of PDDL in various contexts.
% As demonstrated by~\citet{carlini2021extracting}, LLMs can memorize content from their training data, leading to memorization rather than true world modeling. 
% To assess potential contamination between LLMs' training data and our benchmark, we followed a similar approach to~\citet{carlini2021extracting} by generating complete PDDL domains from the first 20 tokens using GPT-4~\cite{openai2023gpt4}, then calculating the contamination rate based on tokenized 10-grams with up to 4 token mismatches~\cite{touvron2023llama}. 
% We excluded PDDL-specific keywords and variables to focus on content overlap. We also compared these results with domains from previous studies~\cite{guan2023leveraging,smirnov2024generating}.
% Figure~\ref{fig:cont} shows that \benchmark has a lower contamination rate (mean $\mu = 0.04$), suggesting its performance reflects the models' ability to understand and formalize domain dynamics rather than memorize PDDL patterns. However, we acknowledge that complete elimination of contamination is difficult due to PDDL's widespread use.
As shown by~\citet{carlini2021extracting}, LLMs can memorize training data rather than truly model the world. 
To assess potential contamination between LLMs' training data and \benchmark, we generated complete PDDL domains from the first 20 tokens using GPT-4~\cite{openai2023gpt4} and calculated contamination rates based on tokenized 10-grams with up to 4 mismatches~\cite{touvron2023llama}, excluding PDDL-specific keywords and variables. 
We also compared these results with previous studies~\cite{guan2023leveraging,smirnov2024generating}. 
Figure~\ref{fig:cont} shows that \benchmark has a lower contamination rate ($\mu = 0.04$ vs. $\mu = 0.47$), suggesting its performance reflects domain understanding rather than memorization. However, the complete elimination of contamination remains challenging due to PDDL's widespread use.

\begin{figure}[t]
    \centering
    \includegraphics[width=0.8\linewidth]{images/boxplot.pdf}
    \caption{
    n-gram contamination rate of \benchmark and prior works.
    }
    \label{fig:cont}
\end{figure}



\subsection{Data Analysis}

This section provides some detailed data analysis to better understand \benchmark.

\noindent \textbf{Core Statistics} We designated 2 domains as in-context exemplars (train set), with the remaining 101 samples forming our test set.
% We present the core statistical properties of the dataset \benchmark in Figure~\ref{fig:combined} (right), focusing on key metrics such as token length, predicate and action counts, etc.

% \begin{figure*}[htbp]
%     % 左侧图片
%     \begin{minipage}{0.77\linewidth}  % 调整宽度
%         \centering
%         \includegraphics[width=\linewidth]{images/benchmark_construction.pdf}
%     \end{minipage}%
%     % 间隔
%     \hfill
%     % 右侧表格
%     \begin{minipage}{0.23\linewidth}  % 调整宽度
%         \centering
%         \resizebox{\linewidth}{!}{  % 调整表格至合适的宽度
%             \begin{tabular}{lcc}
%                 \toprule
%                 \textbf{Statistic} & \textbf{Number} \\
%                 \midrule
%                 \rowcolor[HTML]{F2F2F2} 
%                 \textit{Domain Count} &  \\
%                 \midrule
%                 Domain & 103 \\
%                 Requirement & 8 \\
%                 \midrule
%                 \rowcolor[HTML]{F2F2F2} 
%                 \textit{Token Count} &  \\
%                 \midrule
%                 Description & 851.6 $\pm$ 515.2 \\
%                 - Min/Max & [159, 2814] \\
%                 Domain & 1187.2 $\pm$ 1212.1 \\
%                 - Min/Max & [85, 7514] \\
%                 \midrule
%                 \rowcolor[HTML]{F2F2F2} 
%                 \textit{Line Count} &  \\
%                 \midrule
%                 Domain & 75.4 $\pm$ 62.9 \\
%                 - Min/Max & [9, 394] \\
%                 \midrule
%                 \rowcolor[HTML]{F2F2F2} 
%                 \textit{Component Count} &  \\
%                 \midrule
%                 Actions & 4.5 $\pm$ 2.8 \\
%                 - Min/Max & [1, 16] \\
%                 Predicates & 8.1 $\pm$ 4.8 \\
%                 - Min/Max & [1, 25] \\
%                 Types & 1.1 $\pm$ 1.3 \\
%                 - Min/Max & [1, 8] \\
%                 \bottomrule
%             \end{tabular}
%         }
%     \end{minipage}
%     % 公共标题
%     \caption{Dataset construction process (left) and key statistics (right) of the \texttt{\benchmark} dataset.     Dataset construction process including: (a) \textit{Data Acquisition} (\S\ref{sec:data_acquisition}); (b) \textit{Data Filtering and Manual Selection} (\S\ref{sec:data_filtering}); (c) \textit{Data Annotation and Quality Assurance}(\S\ref{sec:data_annotation} and \S\ref{sec:quality_assurance}). Tokens are counted by GPT-2~\cite{openai2019gpt2} tokenizer.}
%     \label{fig:combined}
% \end{figure*}


\noindent \textbf{Semantic Analysis} 
\label{sec:wordcloud}
We use LLMs to extract high-level domain characteristics to better understand the conceptual distribution of \benchmark,
% To better understand the conceptual distribution and thematic coverage of \benchmark domain collection, we conducted a semantic analysis using LLMs to extract high-level domain characteristics. 
As shown in Figure~\ref{fig:three_plots} (Bottom), common themes such as \textit{path planning}, \textit{constraint satisfaction}, and \textit{task allocation}, among others, emerge. 

% \begin{figure}[htbp]
%     \centering
%     \includegraphics[width=\linewidth]{images/pddl_domain_wordcloud_filtered.pdf}
%     \caption{
%     A semantic word cloud visualization of concepts.
%     }
%     \label{fig:wordcloud}
% \end{figure}

\noindent \textbf{Requirements Analysis} 
\label{sec:requirements_analysis}
A PDDL requirement specifies a formal capability needed to express a domain, often reflecting its complexity. For instance, \texttt{:typing} stands for allowing the usage of typing for objects.
As shown in Figure~\ref{fig:three_plots} (Top), there are eight different requirement type in \benchmark. We also provide an in-depth analysis of requirement type in Appendix~\ref{app:detailed_data_analysis}.

\begin{figure}[t]
    % 上面第一行:boxplot 和 requirements_piechart 并列
    % \begin{minipage}{0.40\columnwidth}  % 左侧图片
    %     \centering
    %     \includegraphics[width=\linewidth]{images/boxplot.pdf}
    %     % \\  % 换行以便放置子描述
    %     % \textbf{(a)} n-gram contamination rate compared to prior works (\S\ref{sec:contamination})
    % \end{minipage}
    % \hfill  % 添加间隔
    \begin{minipage}{\columnwidth}  % 右侧图片
        \centering
        \includegraphics[width=\linewidth]{images/requirements_piechart.pdf}
        % \\  % 换行以便放置子描述
        % \textbf{(b)} the frequency of requirements distribution (\S\ref{sec:requirements})
    \end{minipage}
    
    % 下面第二行:pddl_domain_wordcloud_filtered 单独放置
    \begin{minipage}{\columnwidth}
        \centering
        \includegraphics[width=\linewidth]{images/pddl_domain_wordcloud_filtered.pdf}
        % \\  % 换行以便放置子描述
        % \textbf{(c)} a semantic word cloud visualization of concepts (\S\ref{sec:wordcloud})
    \end{minipage}
    
    % 公共标题
    \caption{
    \textit{Top}: The frequency of requirements distribution. \textit{Bottom}: Word cloud of concepts in \benchmark.}
    \label{fig:three_plots}
\end{figure}


% \begin{figure}[htbp]
%     \centering
%     \includegraphics[width=\linewidth]{images/heatmap.pdf}
%     \caption{
%     The co-occurrence matrix of requirements of \benchmark.
%     }
%     \label{fig:heatmap}
% \end{figure}

\subsection{Preliminary Experiment}
\label{sec:llmself}
In previous works, LLMs have been employed to evaluate the action dynamics of world models generated by LLMs themselves~\cite{wang2023bytesized32}. 
To further assess the ability of LLMs to detect errors in world models, we conducted a preliminary experiment where we first used \texttt{claude-3.5-sonnect} for \benchmark. Subsequently, human annotators and the LLM independently evaluated the generated action dynamics to identify potential errors.
The inter-annotator agreement between human ratings and LLM ratings, measured using Cohen’s $\kappa$, was 0.10, indicating a low level of agreement. This suggests that predicting the correctness of PDDL domains using an LLM is particularly challenging, highlighting the need for more discriminative evaluation metrics.
% Notably, the LLM exhibited high confidence in its evaluations, assigning "high" confidence to 97.5\% of cases. This demonstrates that the model was highly self-assured in its judgments, despite the low agreement with human annotators.
% These results highlight the need for more descriptive and discriminative evaluation metrics to gain a more reliable understanding of how effectively LLMs model the world.
Prompting examples and more results can be found in Appendix~\ref{app:self_eval}.


\section{Experiments}
\section{Experiments}
\label{sec:exp}
Following the settings in Section \ref{sec:existing}, we evaluate \textit{NovelSum}'s correlation with the fine-tuned model performance across 53 IT datasets and compare it with previous diversity metrics. Additionally, we conduct a correlation analysis using Qwen-2.5-7B \cite{yang2024qwen2} as the backbone model, alongside previous LLaMA-3-8B experiments, to further demonstrate the metric's effectiveness across different scenarios. Qwen is used for both instruction tuning and deriving semantic embeddings. Due to resource constraints, we run each strategy on Qwen for two rounds, resulting in 25 datasets. 

\subsection{Main Results}

\begin{table*}[!t]
    \centering
    \resizebox{\linewidth}{!}{
    \begin{tabular}{lcccccccccc}
    \toprule
    \multirow{3}*{\textbf{Diversity Metrics}} & \multicolumn{10}{c}{\textbf{Data Selection Strategies}} \\
    \cmidrule(lr){2-11}
    & \multirow{2}*{\textbf{K-means}} & \multirow{2}*{\vtop{\hbox{\textbf{K-Center}}\vspace{1mm}\hbox{\textbf{-Greedy}}}}  & \multirow{2}*{\textbf{QDIT}} & \multirow{2}*{\vtop{\hbox{\textbf{Repr}}\vspace{1mm}\hbox{\textbf{Filter}}}} & \multicolumn{5}{c}{\textbf{Random}} & \multirow{2}{*}{\textbf{Duplicate}} \\ 
    \cmidrule(lr){6-10}
    & & & & & \textbf{$\mathcal{X}^{all}$} & ShareGPT & WizardLM & Alpaca & Dolly &  \\
    \midrule
    \rowcolor{gray!15} \multicolumn{11}{c}{\textit{LLaMA-3-8B}} \\
    Facility Loc. $_{\times10^5}$ & \cellcolor{BLUE!40} 2.99 & \cellcolor{ORANGE!10} 2.73 & \cellcolor{BLUE!40} 2.99 & \cellcolor{BLUE!20} 2.86 & \cellcolor{BLUE!40} 2.99 & \cellcolor{BLUE!0} 2.83 & \cellcolor{BLUE!30} 2.88 & \cellcolor{BLUE!0} 2.83 & \cellcolor{ORANGE!20} 2.59 & \cellcolor{ORANGE!30} 2.52 \\    
    DistSum$_{cosine}$  & \cellcolor{BLUE!30} 0.648 & \cellcolor{BLUE!60} 0.746 & \cellcolor{BLUE!0} 0.629 & \cellcolor{BLUE!50} 0.703 & \cellcolor{BLUE!10} 0.634 & \cellcolor{BLUE!40} 0.656 & \cellcolor{ORANGE!30} 0.578 & \cellcolor{ORANGE!10} 0.605 & \cellcolor{ORANGE!20} 0.603 & \cellcolor{BLUE!10} 0.634 \\
    Vendi Score $_{\times10^7}$ & \cellcolor{BLUE!30} 1.70 & \cellcolor{BLUE!60} 2.53 & \cellcolor{BLUE!10} 1.59 & \cellcolor{BLUE!50} 2.23 & \cellcolor{BLUE!20} 1.61 & \cellcolor{BLUE!30} 1.70 & \cellcolor{ORANGE!10} 1.44 & \cellcolor{ORANGE!20} 1.32 & \cellcolor{ORANGE!10} 1.44 & \cellcolor{ORANGE!30} 0.05 \\
    \textbf{NovelSum (Ours)} & \cellcolor{BLUE!60} 0.693 & \cellcolor{BLUE!50} 0.687 & \cellcolor{BLUE!30} 0.673 & \cellcolor{BLUE!20} 0.671 & \cellcolor{BLUE!40} 0.675 & \cellcolor{BLUE!10} 0.628 & \cellcolor{BLUE!0} 0.591 & \cellcolor{ORANGE!10} 0.572 & \cellcolor{ORANGE!20} 0.50 & \cellcolor{ORANGE!30} 0.461 \\
    \midrule    
    \textbf{Model Performance} & \cellcolor{BLUE!60}1.32 & \cellcolor{BLUE!50}1.31 & \cellcolor{BLUE!40}1.25 & \cellcolor{BLUE!30}1.05 & \cellcolor{BLUE!20}1.20 & \cellcolor{BLUE!10}0.83 & \cellcolor{BLUE!0}0.72 & \cellcolor{ORANGE!10}0.07 & \cellcolor{ORANGE!20}-0.14 & \cellcolor{ORANGE!30}-1.35 \\
    \midrule
    \midrule
    \rowcolor{gray!15} \multicolumn{11}{c}{\textit{Qwen-2.5-7B}} \\
    Facility Loc. $_{\times10^5}$ & \cellcolor{BLUE!40} 3.54 & \cellcolor{ORANGE!30} 3.42 & \cellcolor{BLUE!40} 3.54 & \cellcolor{ORANGE!20} 3.46 & \cellcolor{BLUE!40} 3.54 & \cellcolor{BLUE!30} 3.51 & \cellcolor{BLUE!10} 3.50 & \cellcolor{BLUE!10} 3.50 & \cellcolor{ORANGE!20} 3.46 & \cellcolor{BLUE!0} 3.48 \\ 
    DistSum$_{cosine}$ & \cellcolor{BLUE!30} 0.260 & \cellcolor{BLUE!60} 0.440 & \cellcolor{BLUE!0} 0.223 & \cellcolor{BLUE!50} 0.421 & \cellcolor{BLUE!10} 0.230 & \cellcolor{BLUE!40} 0.285 & \cellcolor{ORANGE!20} 0.211 & \cellcolor{ORANGE!30} 0.189 & \cellcolor{ORANGE!10} 0.221 & \cellcolor{BLUE!20} 0.243 \\
    Vendi Score $_{\times10^6}$ & \cellcolor{ORANGE!10} 1.60 & \cellcolor{BLUE!40} 3.09 & \cellcolor{BLUE!10} 2.60 & \cellcolor{BLUE!60} 7.15 & \cellcolor{ORANGE!20} 1.41 & \cellcolor{BLUE!50} 3.36 & \cellcolor{BLUE!20} 2.65 & \cellcolor{BLUE!0} 1.89 & \cellcolor{BLUE!30} 3.04 & \cellcolor{ORANGE!30} 0.20 \\
    \textbf{NovelSum (Ours)}  & \cellcolor{BLUE!40} 0.440 & \cellcolor{BLUE!60} 0.505 & \cellcolor{BLUE!20} 0.403 & \cellcolor{BLUE!50} 0.495 & \cellcolor{BLUE!30} 0.408 & \cellcolor{BLUE!10} 0.392 & \cellcolor{BLUE!0} 0.349 & \cellcolor{ORANGE!10} 0.336 & \cellcolor{ORANGE!20} 0.320 & \cellcolor{ORANGE!30} 0.309 \\
    \midrule
    \textbf{Model Performance} & \cellcolor{BLUE!30} 1.06 & \cellcolor{BLUE!60} 1.45 & \cellcolor{BLUE!40} 1.23 & \cellcolor{BLUE!50} 1.35 & \cellcolor{BLUE!20} 0.87 & \cellcolor{BLUE!10} 0.07 & \cellcolor{BLUE!0} -0.08 & \cellcolor{ORANGE!10} -0.38 & \cellcolor{ORANGE!30} -0.49 & \cellcolor{ORANGE!20} -0.43 \\
    \bottomrule
    \end{tabular}
    }
    \caption{Measuring the diversity of datasets selected by different strategies using \textit{NovelSum} and baseline metrics. Fine-tuned model performances (Eq. \ref{eq:perf}), based on MT-bench and AlpacaEval, are also included for cross reference. Darker \colorbox{BLUE!60}{blue} shades indicate higher values for each metric, while darker \colorbox{ORANGE!30}{orange} shades indicate lower values. While data selection strategies vary in performance on LLaMA-3-8B and Qwen-2.5-7B, \textit{NovelSum} consistently shows a stronger correlation with model performance than other metrics. More results are provided in Appendix \ref{app:results}.}
    \label{tbl:main}
    \vspace{-4mm}
\end{table*}


\begin{table}[t!]
\centering
\resizebox{\linewidth}{!}{
\begin{tabular}{lcccc}
\toprule
\multirow{2}*{\textbf{Diversity Metrics}} & \multicolumn{3}{c}{\textbf{LLaMA}} & \textbf{Qwen}\\
\cmidrule(lr){2-4} \cmidrule(lr){5-5} 
& \textbf{Pearson} & \textbf{Spearman} & \textbf{Avg.} & \textbf{Avg.} \\
\midrule
TTR & -0.38 & -0.16 & -0.27 & -0.30 \\
vocd-D & -0.43 & -0.17 & -0.30 & -0.31 \\
\midrule
Facility Loc. & 0.86 & 0.69 & 0.77 & 0.08 \\
Entropy & 0.93 & 0.80 & 0.86 & 0.63 \\
\midrule
LDD & 0.61 & 0.75 & 0.68 & 0.60 \\
KNN Distance & 0.59 & 0.80 & 0.70 & 0.67 \\
DistSum$_{cosine}$ & 0.85 & 0.67 & 0.76 & 0.51 \\
Vendi Score & 0.70 & 0.85 & 0.78 & 0.60 \\
DistSum$_{L2}$ & 0.86 & 0.76 & 0.81 & 0.51 \\
Cluster Inertia & 0.81 & 0.85 & 0.83 & 0.76 \\
Radius & 0.87 & 0.81 & 0.84 & 0.48 \\
\midrule
NovelSum & \textbf{0.98} & \textbf{0.95} & \textbf{0.97} & \textbf{0.90} \\
\bottomrule
\end{tabular}
}
\caption{Correlations between different metrics and model performance on LLaMA-3-8B and Qwen-2.5-7B.  “Avg.” denotes the average correlation (Eq. \ref{eq:cor}).}
\label{tbl:correlations}
\vspace{-2mm}
\end{table}

\paragraph{\textit{NovelSum} consistently achieves state-of-the-art correlation with model performance across various data selection strategies, backbone LLMs, and correlation measures.}
Table \ref{tbl:main} presents diversity measurement results on datasets constructed by mainstream data selection methods (based on $\mathcal{X}^{all}$), random selection from various sources, and duplicated samples (with only $m=100$ unique samples). 
Results from multiple runs are averaged for each strategy.
Although these strategies yield varying performance rankings across base models, \textit{NovelSum} consistently tracks changes in IT performance by accurately measuring dataset diversity. For instance, K-means achieves the best performance on LLaMA with the highest NovelSum score, while K-Center-Greedy excels on Qwen, also correlating with the highest NovelSum. Table \ref{tbl:correlations} shows the correlation coefficients between various metrics and model performance for both LLaMA and Qwen experiments, where \textit{NovelSum} achieves state-of-the-art correlation across different models and measures.

\paragraph{\textit{NovelSum} can provide valuable guidance for data engineering practices.}
As a reliable indicator of data diversity, \textit{NovelSum} can assess diversity at both the dataset and sample levels, directly guiding data selection and construction decisions. For example, Table \ref{tbl:main} shows that the combined data source $\mathcal{X}^{all}$ is a better choice for sampling diverse IT data than other sources. Moreover, \textit{NovelSum} can offer insights through comparative analyses, such as: (1) ShareGPT, which collects data from real internet users, exhibits greater diversity than Dolly, which relies on company employees, suggesting that IT samples from diverse sources enhance dataset diversity \cite{wang2024diversity-logD}; (2) In LLaMA experiments, random selection can outperform some mainstream strategies, aligning with prior work \cite{xia2024rethinking,diddee2024chasing}, highlighting gaps in current data selection methods for optimizing diversity.



\subsection{Ablation Study}


\textit{NovelSum} involves several flexible hyperparameters and variations. In our main experiments, \textit{NovelSum} uses cosine distance to compute $d(x_i, x_j)$ in Eq. \ref{eq:dad}. We set $\alpha = 1$, $\beta = 0.5$, and $K = 10$ nearest neighbors in Eq. \ref{eq:pws} and \ref{eq:dad}. Here, we conduct an ablation study to investigate the impact of these settings based on LLaMA-3-8B.

\begin{table}[ht!]
\centering
\resizebox{\linewidth}{!}{
\begin{tabular}{lccc}
\toprule
\textbf{Variants} & \textbf{Pearson} & \textbf{Spearman} & \textbf{Avg.} \\
\midrule
NovelSum & 0.98 & 0.96 & 0.97 \\
\midrule
\hspace{0.10cm} - Use $L2$ distance & 0.97 & 0.83 & 0.90\textsubscript{↓ 0.08} \\
\hspace{0.10cm} - $K=20$ & 0.98 & 0.96 & 0.97\textsubscript{↓ 0.00} \\
\hspace{0.10cm} - $\alpha=0$ (w/o proximity) & 0.79 & 0.31 & 0.55\textsubscript{↓ 0.42} \\
\hspace{0.10cm} - $\alpha=2$ & 0.73 & 0.88 & 0.81\textsubscript{↓ 0.16} \\
\hspace{0.10cm} - $\beta=0$ (w/o density) & 0.92 & 0.89 & 0.91\textsubscript{↓ 0.07} \\
\hspace{0.10cm} - $\beta=1$ & 0.90 & 0.62 & 0.76\textsubscript{↓ 0.21} \\
\bottomrule
\end{tabular}
}
\caption{Ablation Study for \textit{NovelSum}.}
\label{tbl:ablation}
\vspace{-2mm}
\end{table}

In Table \ref{tbl:ablation}, $\alpha=0$ removes the proximity weights, and $\beta=0$ eliminates the density multiplier. We observe that both $\alpha=0$ and $\beta=0$ significantly weaken the correlation, validating the benefits of the proximity-weighted sum and density-aware distance. Additionally, improper values for $\alpha$ and $\beta$ greatly reduce the metric's reliability, highlighting that \textit{NovelSum} strikes a delicate balance between distances and distribution. Replacing cosine distance with Euclidean distance and using more neighbors for density approximation have minimal impact, particularly on Pearson's correlation, demonstrating \textit{NovelSum}'s robustness to different distance measures.







\begin{table*}[htbp]
\centering
% \small
\caption{Performance comparison of different LLMs on \benchmark. EC\textsubscript{k} denotes the setting where models are allowed k correction attempts (EC\textsubscript{0}: zero-shot without correction, EC\textsubscript{3}: with 3 correction attempts).}
\label{tab:main}
\resizebox{\textwidth}{!}{%
\begin{tabular}{l|c|cc|cc|cc|cc|cc|cc}
\toprule
\multirow{2}{*}{\textbf{Model Family}} & \multirow{2}{*}{\textbf{Version}} & \multicolumn{2}{c|}{\textbf{\exec} $\uparrow$} & \multicolumn{2}{c|}{\textbf{\simmetric} $\uparrow$} & \multicolumn{2}{c|}{\textbf{\fonepred} $\uparrow$} & \multicolumn{2}{c|}{\textbf{\foneparam} $\uparrow$} & \multicolumn{2}{c|}{\textbf{\foneprecond} $\uparrow$} & \multicolumn{2}{c}{\textbf{\foneeff} $\uparrow$} \\
\cmidrule{3-14} & & \textbf{EC\textsubscript{0}} & \textbf{EC\textsubscript{3}} & \textbf{EC\textsubscript{0}} & \textbf{EC\textsubscript{3}} & \textbf{EC\textsubscript{0}} & \textbf{EC\textsubscript{3}} & \textbf{EC\textsubscript{0}} & \textbf{EC\textsubscript{3}} & \textbf{EC\textsubscript{0}} & \textbf{EC\textsubscript{3}} & \textbf{EC\textsubscript{0}} & \textbf{EC\textsubscript{3}} \\
\midrule
% \multirow{3}{*}{\textsc{OpenAI-o1}} & {\small\texttt{o1-preview}} &  &  &  &  &  &  &  &  &  &  &  &  \\
\multirow{1}{*}{\textsc{OpenAI-o1}}  & {\small\texttt{o1-mini}} & 49.5 & 69.3 & 82.5 & 82.2 & 48.4 & 66.3 & 36.4 & 49.7 & 28.9 & 38.0 & 31.7 & 42.1 \\
\midrule
\textsc{OpenAI-o3} & {\small\texttt{o3-mini}} & 54.5 & 84.2 & 83.0 & 81.9 & 53.9 & 81.1 & 43.7 & 63.0 & 36.8 & 50.4 & 39.4 & 53.8 \\
\midrule
\multirow{2}{*}{\textsc{GPT-4}} & {\small\texttt{gpt-4o}} & 60.4 & 75.2 & 84.5 & 84.1 & 59.6 & 72.1 & 56.5 & 68.1 & 49.3 & 56.4 & 47.8 & 56.7 \\
 & {\small\texttt{gpt-4o-mini}} & 48.5 & 72.3 & 82.6 & 82.2 & 48.1 & 70.1 & 47.1 & 67.3 & 34.9 & 47.5 & 38.2 & 52.7 \\
% chatgpt-4o-latest
% & {\small\texttt{chatgpt}} &  &  &  &  &  &  &  &  &  &  &  &  \\
\midrule
% \multirow{2}{*}{\textsc{GPT-3.5}} & {\small\texttt{turbo-instruct}} &  &  &  &  &  &  &  &  &  &  &  &  \\
\multirow{1}{*}{\textsc{GPT-3.5}}  & {\small\texttt{turbo-0125}} & 41.6 & 56.4 & 81.9 & 81.6 & 41.2 & 55.8 & 39.6 & 53.8 & 30.2 & 39.2 & 27.5 & 37.7 \\
\midrule
\textsc{Claude-3.5} & {\small\texttt{sonnet}} & 45.5 & 64.4 & 73.2 & 66.8 & 45.5 & 62.5 & 41.5 & 48.8 & 37.4 & 44.0 & 38.4 & 45.0 \\
\midrule
\multirow{2}{*}{\textsc{LLaMA-2}} & {\small\texttt{7b-instruct}} & 0.0 & 0.0 & 45.5 & 33.9 & 0.0 & 0.0 & 0.0 & 0.0 & 0.0 & 0.0 & 0.0 & 0.0 \\
 & {\small\texttt{70b-instruct}} & 0.0 & 0.0 & 48.7 & 48.6 & 0.0 & 0.0 & 0.0 & 0.0 & 0.0 & 0.0 & 0.0 & 0.0 \\
\midrule
\multirow{2}{*}{\textsc{LLaMA-3.1}} & {\small\texttt{8b-instruct}} & 0.0 & 0.0 & 74.3 & 74.9 & 0.0 & 0.0 & 0.0 & 0.0 & 0.0 & 0.0 & 0.0 & 0.0 \\
 & {\small\texttt{70b-instruct}} & 0.0 & 0.0 & 83.6 & 79.2 & 0.0 & 0.0 & 0.0 & 0.0 & 0.0 & 0.0 & 0.0 & 0.0 \\
 % & {\small\texttt{405b-instruct}} &  &  &  &  &  &  &  &  &  &  &  &  \\
% \midrule
% \multirow{1}{*}{\textsc{Mistral}} & {\small\texttt{7b-instruct}} & & & & & & &  &  &  &  &  &  \\
 % & {\small\texttt{8x22b-instruct}} &  &  &  &  &  &  &  &  &  &  &  &  \\
\midrule
\multirow{2}{*}{\textsc{DeepSeek}} & {\small\texttt{deepseek-v3}} & 56.4 & 79.2 & \cellcolor[HTML]{F2F2F2}\textbf{84.7} & \cellcolor[HTML]{F2F2F2}\textbf{84.2} & 55.9 & 75.6 & 53.7 & 74.4 & 45.1 & 58.6 & 46.7 & 61.5 \\ 
& {\small\texttt{deepseek-r1}} & \cellcolor[HTML]{F2F2F2}\textbf{72.3} & \cellcolor[HTML]{F2F2F2}\textbf{89.1} & 84.3 & 84.0 & \cellcolor[HTML]{F2F2F2}\textbf{71.7} & \cellcolor[HTML]{F2F2F2}\textbf{86.7} & \cellcolor[HTML]{F2F2F2}\textbf{64.0} & \cellcolor[HTML]{F2F2F2}\textbf{76.3} & \cellcolor[HTML]{F2F2F2}\textbf{57.6} & \cellcolor[HTML]{F2F2F2}\textbf{65.0} & \cellcolor[HTML]{F2F2F2}\textbf{58.8} & \cellcolor[HTML]{F2F2F2}\textbf{67.3} \\ 
\midrule
\multirow{4}{*}{\textsc{CodeLLaMA}} & {\small\texttt{7b-instruct}} & 17.8 & 22.8 & 60.2 & 57.6 & 17.8 & 18.8 & 17.2 & 18.2 & 11.3 & 12.2 & 10.7 & 11.1 \\
 & {\small\texttt{13b-instruct}} & 7.9 & 8.9 & 57.6 & 55.0 & 7.9 & 8.9 & 7.9 & 8.9 & 4.9 & 5.9 & 5.2 & 6.1 \\
 & {\small\texttt{34b-instruct}} & 7.9 & 8.9 & 34.2 & 7.6 & 7.9 & 8.6 & 7.9 & 8.4 & 5.0 & 5.0 & 5.4 & 5.4 \\
 & {\small\texttt{70b-instruct}} & 16.8 & 16.8 & 54.0 & 14.0 & 16.4 & 16.4 & 16.8 & 16.8 & 10.7 & 10.7 & 14.1 & 14.1 \\
% \midrule
% \multirow{1}{*}{\textsc{AgentLLM}} & {\small\texttt{70b}} & 7.9 & 9.9 & 65.6 & 47.9 & 7.3 & 8.8 & 7.3 & 9.1 & 6.1  & 6.5 & 5.7 & 6.1 \\
\bottomrule
\end{tabular}%
}
\end{table*}


\subsection{Experimental Setup}



% \subsubsection{Implementation Details}

We evaluate several state-of-the-art LLMs, including \textit{GPT-4}~\cite{openai2023gpt4}, \textit{GPT-3.5}~\cite{opeiai2022gpt}, \textit{Claude-3.5}~\cite{claude_3.5_sonnet}, and \textit{LLaMA-3.1}~\cite{meta_llama_3_1}, \textit{DeepSeek-v3}~\cite{liu2024deepseek}, \textit{CodeLlaMA}~\cite{roziere2023code}, \textit{LlaMA-2}~\cite{touvron2023llama}, etc.
% Mistral~\cite{jiang2023mistral},
We also evaluated Large Reasoning Models (LRMs) trained using reinforcement learning, such as \textit{DeepSeek-R1}~\cite{deepseekai2025deepseekr1incentivizingreasoningcapability}, \textit{OpenAI-o1}~\cite{openai2024o1} and \textit{OpenAI-o3}~\cite{openai2025o3}.
We set temperature = 0 for each model for all experiments to maintain reproducibility.
We employ tarski~\footnote{\url{https://github.com/aig-upf/tarski}} library to check syntactic correctness and executability.
% For PDDL validation, we employ the VAL PDDL validator~\cite{howey2004val} to check syntactic correctness and executability.
We prompt LLMs to generate symbolic world models under a zero-shot setting with chain-of-thought reasoning~\cite{wei2022chain}. 
In error-correction experiments, LLMs refine outputs based on validator-reported syntax errors, denoted as $\text{EC}_{3}$ for $k$ attempts.
Evaluation of open-sourced models were conducted on NVIDIA A100 GPUs with 80GB memory.
We access proprietary models through their official API platform.
Prompt examples can be found in Appendix~\ref{app:prompt}.

% \subsubsection{Experiment Configuration}
% For main experiments in Section~\ref{sec:main_exp}, we prompt LLMs to generate symbolic world model under a zero-shot setting with chain-of-thought reasoning~\cite{wei2022chain}. 
% Additionally, we conduct experiments on error correction where LLMs refine their outputs based on syntax errors reported by the validator. 
% We denote these settings as $\text{EC}_{k}$, where k represents the number of allowed correction attempts (e.g., $\text{EC}_{3}$ for setting with 3 correction attempts). 
% Further experiments with varying numbers of correction attempts are presented in Section~\ref{sec:more_correction}.
% More details and examples about the experiment configuration are presented in Appendix~\ref{app:exp_config}.

\subsection{Experimental Results}
\label{sec:main_exp}

% Table~\ref{tab:main} shows the performance of 16 different LLMs on \benchmark. 
Several conclusions can be drawn from Table~\ref{tab:main}:
% \textit{(i)} \textbf{\textit{A significant performance gap exists between open-source models and proprietary models.}} 
% In terms of the structural similarity, there is at least a 9.2\% gap between open-source and proprietary models.
\textit{(i)} \textbf{\textit{The most advanced LLMs still struggle with \benchmark.}}
% For example, the best-performing model, \textit{DeepSeek-R1}, achieves F1 scores for both precondition and effect below 60\% under the without error correction setting.
For example, the best-performing model, \textit{DeepSeek-R1}, achieves F1 scores below 60\% for both preconditions (\foneprecond) and effects (\foneeff) under the without error correction setting.
This highlights the limitations of current LLMs in world modeling tasks.
\textit{(ii)} \textit{\textbf{Large reasoning models trained with reinforcement learning exhibit superior world modeling capabilities.}} These models, such as \textit{DeepSeek-R1}~\cite{deepseekai2025deepseekr1incentivizingreasoningcapability}, outperform others in executability, structural similarity, and component-wise performance, indicating that RL-based training enhances the ability of models to generate structured and valid world models.
\textit{(iii)} \textit{\textbf{The ability of models to benefit from error correction is evident.}} For instance, \textit{GPT-4} (\texttt{gpt-4o-mini}) demonstrates a notable improvement in executability, increasing from 48.5\% to 72.3\% after three correction attempts. 
% This shows that error correction plays a crucial role in refining the models' world modeling capability. 
% We also conduct statistical analysis to verify this improvement, which confirms the significant enhancement in performance in Section~\ref{sec:stat_ana}.

\section{Analysis}

\subsection{Statistical Analysis}
\label{sec:stat_ana}

We conducted a one-way ANOVA~\cite{girden1992anova} to evaluate the impact of correction attempts on model performance, excluding anomalous zero values. The results showed a significant improvement with three correction attempts ($F = 27.48, p = 0.00012$), indicating that correction attempts lead to a notable enhancement in model performance.

\subsection{Error Analysis}
\label{sec:analysis}

The interpretable nature of generating symbolic world models can be utilized for a deeper manual analysis of the failure modes. 
We select the results from \texttt{claude-3.5-sonnect} under the few-shot setting for manual error analysis. 
Errors are categorized into syntax and semantic errors, where syntax errors occur when the generated domain cannot be validated ($\exec=0$), and semantic errors arise when the generated world model does not align with action dynamics or fails to follow the natural language description.
The distribution for each error type and detailed explanations are presented in Appendix~\ref{app:analysis_detail}.

\noindent \textbf{Syntax Errors}
Figure~\ref{fig:sankey_venn} (\textit{Left}) shows the distribution of syntax errors during correction. 
Common errors like \texttt{UndefinedConstant} and \texttt{IncorrectParentheses} decrease over correction steps, indicating improvements in syntax validation, though errors like \texttt{UndefinedDomainName} and \texttt{UndefinedType} persist.

\begin{figure*}[htbp]
    \centering
    \includegraphics[width=\linewidth]{images/sankey_venn.pdf}
    \caption{\textit{Left}: The distribution of syntax error types during the progression of correction. \textit{Right}: The distribution of semantic error types.}
    \label{fig:sankey_venn}
\end{figure*}


\noindent \textbf{Semantic Error}
Figure~\ref{fig:sankey_venn} (\textit{Right}) illustrates the distribution of semantic errors. 
Semantic errors are categorized into four types: 
\textit{(i)} \texttt{DisobeyDescription} involves direct violations of descriptions.
\textit{(ii)} \texttt{IncompleteModeling}, where the world model lacks necessary components. 
\textit{(iii)} \texttt{RedundantSpecifications} refers to superfluous preconditions or effects;
and \textit{(iv)} \texttt{SurfaceDivergence} involves surface-level variations that preserve semantic equivalence to gold domain.
In addition, since a domain may encompass various action dynamics, different error types can occur simultaneously. 
For instance, nearly 10\% of cases exhibited both \texttt{IncompleteModeling} and \texttt{RedundantSpecifications} concurrently.

% \subsection{Qualitatitive Analysis}
% % qualitatitive analysis of LLM的internal world model

\vspace{-5pt}

\section{Exploration}

In addition to the zero-shot CoT evaluation in Section~\ref{sec:main_exp}, we further evaluate the models on \benchmark with five different strategies: (1) \textit{Test-time Scaling}; (2) \textit{In-Context Learning}; (3) \textit{Fine-tuning}; (4) \textit{Agent Training}; (5) \textit{Inference with Concrete Description}.

\subsection{Test-time Scaling}
\label{sec:more_correction}
Recently, test-time scaling has demonstrated remarkable potential~\cite{openai2024o1,deepseekai2025deepseekr1incentivizingreasoningcapability}.
We use the error information from the syntax parser as feedback and assess whether increasing the test-time compute budget can enhance the LLM's performance.
As shown in Figure~\ref{fig:20correction}, the model exhibits consistent improvement with increased test-time computation.
More advanced test-time scaling strategies may serve as a viable approach to enhancing the model's world modeling ability~\cite{chen2025ecm}.

\begin{figure}[t]
    \centering
    \includegraphics[width=\linewidth]{images/20_correction_combined.pdf}
    \caption{
    The performance of \texttt{gpt-4o-mini} (left) and \texttt{deepseek-v3} (right) under different test-time compute budgets, showing consistent improvement with increased compute.
    }
    \label{fig:20correction}
\end{figure}

\vspace{-2pt}

\subsection{In-Context Learning} % GPT-4o
\label{sec:few-shot}

We also perform a few-shot evaluation in Section~\ref{sec:few-shot}, where we carefully select demonstration ``\textit{gripper}'' and ``\textit{blocks}'' that are structurally similar but semantically distinct from the test cases to prevent data leakage.
As shown in Table~\ref{tab:optimization}, we observe that different models exhibit varying degrees of improvement from in-context learning. For instance, \texttt{claude-3.5-sonnect} demonstrates a substantial enhancement, achieving over a 20\% increase in the component-wise F1 score. However, for \texttt{gpt-4o-mini}, incorporating few-shot examples resulted in a decrease in model performance.

\subsection{Fine-tuning} 
\label{sec:finetuning}

We leverage the AgentGen~\cite{hu2024agentgen} framework to synthesize 601 PDDL domains and their corresponding descriptions for fine-tuning \textit{LLaMA-3.1}~\cite{meta_llama_3_1} to investigate potential improvements in their world modeling capabilities.
As shown in Table~\ref{tab:optimization}, fine-tuning can lead to significant improvements in model performance. For instance, the fine-tuned Llama-3.1-70B demonstrated performance comparable to GPT-4o-mini, highlighting that supervised fine-tuning is an effective method for bridging the gap between open-source and proprietary models.
Moreover, larger models tend to benefit more from supervised fine-tuning, with the 70B LLaMA-3.1 showing greater improvement than the 8B model.

\subsection{Agent Training}
\label{sec:agent_training}
Many studies have demonstrated that supervised fine-tuning on agent trajectories can enhance a model's performance on agentic tasks~\cite{hu2024agentgen,zeng2023agenttuning} (i.e., agent training). 
Some previous works also discussed that a good agent model requires a sufficiently strong internal world representation~\cite{lecun2022path}. 
Therefore, in this section, we explore whether agent training can improve the model's world modeling capabilities. 
More specifically, we trained \textit{LLaMA-2-70B} model on AgentInstruct~\cite{zeng2023agenttuning}. 
As shown in Table~\ref{tab:optimization}, the model's world modeling capabilities are enhanced post-agent training, indicating a positive correlation between performance on agentic tasks and the model's world modeling abilities.


\begin{table*}[htbp]
\centering
% \small
\caption{The experimental results of models under different settings: (1) In-context learning (\S\ref{sec:few-shot}); (2) Fine-tuning, and fine-tuning with LoRA~\cite{hu2021lora} (\S\ref{sec:finetuning}); (3) Agent training (\S\ref{sec:agent_training}).}
\label{tab:optimization}
\resizebox{\textwidth}{!}{%
\begin{tabular}{l|ll|ll|ll|ll|ll|ll}
\toprule
\multirow{1}{*}{\textbf{Model Family}} & \multicolumn{2}{c|}{\textbf{\exec}} & \multicolumn{2}{c|}{\textbf{\simmetric}} & \multicolumn{2}{c|}{\textbf{\fonepred}} & \multicolumn{2}{c|}{\textbf{\foneparam}} & \multicolumn{2}{c|}{\textbf{\foneprecond}} & \multicolumn{2}{c}{\textbf{\foneeff}} \\
\cmidrule{2-13} & \textbf{EC\textsubscript{0}} & \textbf{EC\textsubscript{3}} & \textbf{EC\textsubscript{0}} & \textbf{EC\textsubscript{3}} & \textbf{EC\textsubscript{0}} & \textbf{EC\textsubscript{3}} & \textbf{EC\textsubscript{0}} & \textbf{EC\textsubscript{3}} & \textbf{EC\textsubscript{0}} & \textbf{EC\textsubscript{3}} & \textbf{EC\textsubscript{0}} & \textbf{EC\textsubscript{3}} \\
\midrule
\rowcolor[HTML]{F2F2F2} 
\multicolumn{13}{c}{\textit{In-Context Learning}} \\
\midrule
\textsc{Claude-3.5-sonnet} & 45.5 & 64.4 & 73.2 & 66.8 & 45.5 & 62.5 & 41.5 & 48.8 & 37.4 & 44.0 & 38.4 & 45.0 \\
~$w.$ \textsc{2-shot} & 
78.2\textsubscript{\textcolor{darkgreen}{+32.7}} & 88.1\textsubscript{\textcolor{darkgreen}{+23.7}} & 83.9\textsubscript{\textcolor{darkgreen}{+10.7}} & 82.3\textsubscript{\textcolor{darkgreen}{+15.5}} & 77.0\textsubscript{\textcolor{darkgreen}{+31.5}} & 86.1\textsubscript{\textcolor{darkgreen}{+23.6}} & 75.2\textsubscript{\textcolor{darkgreen}{+33.7}} & 82.1\textsubscript{\textcolor{darkgreen}{+33.3}} & 65.6\textsubscript{\textcolor{darkgreen}{+28.2}} & 71.3\textsubscript{\textcolor{darkgreen}{+27.3}} & 67.2\textsubscript{\textcolor{darkgreen}{+28.8}} & 73.4\textsubscript{\textcolor{darkgreen}{+28.4}} \\
% ~$\Delta$& 87.5 & 94.2 & 85.3 & 88.7 & 89.2 & 91.5 & 88.7 & 90.8 & 84.5 & 87.3 & 83.1 & 86.2 \\
\midrule
\textsc{Deepseek-r1} & 72.3 & 89.1 & 84.3 & 84.0 & 71.7 & 86.7 & 64.0 & 76.3 & 57.6 & 65.0 & 58.8 & 67.3 \\ 
~$w.$ \textsc{2-shot} & 69.3\textsubscript{\textcolor{red}{-3.0}} & 90.1\textsubscript{\textcolor{darkgreen}{+1.0}} & 83.8\textsubscript{\textcolor{red}{-0.5}} & 83.5\textsubscript{\textcolor{red}{-0.5}} & 68.4\textsubscript{\textcolor{red}{-3.3}} & 87.7\textsubscript{\textcolor{darkgreen}{+1.0}} & 64.6\textsubscript{\textcolor{darkgreen}{+0.6}} & 79.1\textsubscript{\textcolor{darkgreen}{+2.8}} & 56.0\textsubscript{\textcolor{red}{-1.6}} & 66.9\textsubscript{\textcolor{darkgreen}{+1.9}} & 57.6\textsubscript{\textcolor{red}{-1.2}} & 68.9\textsubscript{\textcolor{darkgreen}{+1.6}} \\
\midrule
\textsc{GPT-4o-mini}  & 48.5 & 72.3 & 82.6 & 82.2 & 48.1 & 70.1 & 47.1 & 67.3 & 34.9 & 47.5 & 38.2 & 52.7 \\
~$w.$ \textsc{2-shot} & 40.6\textsubscript{\textcolor{red}{-7.9}} & 69.3\textsubscript{\textcolor{red}{-3}} & 82.9\textsubscript{\textcolor{darkgreen}{+0.3}} & 82.4\textsubscript{\textcolor{darkgreen}{+0.2}} & 40.3\textsubscript{\textcolor{red}{-7.8}} & 67.2\textsubscript{\textcolor{red}{-2.9}} & 40.1\textsubscript{\textcolor{red}{-7}} & 67.0\textsubscript{\textcolor{red}{-0.3}} & 31.6\textsubscript{\textcolor{red}{-3.3}} & 49.3\textsubscript{\textcolor{darkgreen}{+1.8}} & 32.5\textsubscript{\textcolor{red}{-5.7}} & 54.8\textsubscript{\textcolor{darkgreen}{+2.1}} \\
% ~$\Delta$& 87.5 & 94.2 & 85.3 & 88.7 & 89.2 & 91.5 & 88.7 & 90.8 & 84.5 & 87.3 & 83.1 & 86.2 \\
\midrule
\rowcolor[HTML]{F2F2F2} 
\multicolumn{13}{c}{\textit{Fine-tuning (FT)}} \\
\midrule
\textsc{LLaMA-3.1-8B} & 0.0 & 0.0 & 74.3 & 74.9 & 0.0 & 0.0 & 0.0 & 0.0 & 0.0 & 0.0 & 0.0 & 0.0 \\
~$w.$ \textsc{FT} & 52.5\textsubscript{\textcolor{darkgreen}{+52.5}} & 68.3\textsubscript{\textcolor{darkgreen}{+68.3}} & 80.8\textsubscript{\textcolor{darkgreen}{+6.5}} & 80.6\textsubscript{\textcolor{darkgreen}{+5.7}} & 51.4\textsubscript{\textcolor{darkgreen}{+51.4}} & 65.4\textsubscript{\textcolor{darkgreen}{+65.4}} & 48.5\textsubscript{\textcolor{darkgreen}{+48.5}} & 60.6\textsubscript{\textcolor{darkgreen}{+60.6}} & 31.5\textsubscript{\textcolor{darkgreen}{+31.5}} & 38.1\textsubscript{\textcolor{darkgreen}{+38.1}} & 32.4\textsubscript{\textcolor{darkgreen}{+32.4}} & 40.2\textsubscript{\textcolor{darkgreen}{+40.2}} \\
% ~$\Delta$& 87.5 & 94.2 & 85.3 & 88.7 & 89.2 & 91.5 & 88.7 & 90.8 & 84.5 & 87.3 & 83.1 & 86.2 \\
\midrule
\textsc{LLaMA-3.1-70B} & 0.0 & 0.0 & 83.6 & 79.2 & 0.0 & 0.0 & 0.0 & 0.0 & 0.0 & 0.0 & 0.0 & 0.0 \\
~$w.$ \textsc{LoRA} & 48.5\textsubscript{\textcolor{darkgreen}{+48.5}} & 70.3\textsubscript{\textcolor{darkgreen}{+70.3}} & 83.8\textsubscript{\textcolor{darkgreen}{+0.2}} & 82.3\textsubscript{\textcolor{darkgreen}{+3.1}} & 47.9\textsubscript{\textcolor{darkgreen}{+47.9}} & 68.5\textsubscript{\textcolor{darkgreen}{+68.5}} & 48.5\textsubscript{\textcolor{darkgreen}{+48.5}} & 66.4\textsubscript{\textcolor{darkgreen}{+66.4}} & 39.9\textsubscript{\textcolor{darkgreen}{+39.9}} & 52.8\textsubscript{\textcolor{darkgreen}{+52.8}} & 40.6\textsubscript{\textcolor{darkgreen}{+40.6}} & 52.1\textsubscript{\textcolor{darkgreen}{+52.1}} \\
% ~$\Delta$& 87.5 & 94.2 & 85.3 & 88.7 & 89.2 & 91.5 & 88.7 & 90.8 & 84.5 & 87.3 & 83.1 & 86.2 \\
\midrule
\rowcolor[HTML]{F2F2F2} 
\multicolumn{13}{c}{\textit{Agent Training (AT)}} \\
\midrule
\textsc{LLaMA-2-70B} & 0.0 & 0.0 & 48.7 & 48.6 & 0.0 & 0.0 & 0.0 & 0.0 & 0.0 & 0.0 & 0.0 & 0.0 \\
% ~$w.$ \textsc{AT}  & 7.9 & 9.9 & 65.6 & 47.9 & 7.3 & 8.8 & 7.3 & 9.1 & 6.1 & 6.5 & 5.7 & 6.1 \\
\text{~$w.$ \textsc{AT}} & 7.9\textsubscript{\textcolor{darkgreen}{+7.9}} & 9.9\textsubscript{\textcolor{darkgreen}{+9.9}} & 65.6\textsubscript{\textcolor{darkgreen}{+16.9}} & 47.9\textsubscript{\textcolor{red}{-0.7}} & 7.3\textsubscript{\textcolor{darkgreen}{+7.3}} & 8.8\textsubscript{\textcolor{darkgreen}{+8.8}} & 7.3\textsubscript{\textcolor{darkgreen}{+7.3}} & 9.1\textsubscript{\textcolor{darkgreen}{+9.1}} & 6.1\textsubscript{\textcolor{darkgreen}{+6.1}} & 6.5\textsubscript{\textcolor{darkgreen}{+6.5}} & 5.7\textsubscript{\textcolor{darkgreen}{+5.7}} & 6.1\textsubscript{\textcolor{darkgreen}{+6.1}} \\
\bottomrule
\end{tabular}%
}
\end{table*}

\subsection{Inference with Concrete Description}
\label{sec:concrete}
As is discussed in Section~\ref{sec:task_define}, we intentionally make the natural language description of a world model at a high level.
We refer to these high-level descriptions as "\textit{abstract descriptions}," in contrast to more detailed "\textit{concrete descriptions}" that explicitly specify preconditions and effects. 
Examples of both description types can be found in the Appendix~\ref{app:diffdesc}. 
Using concrete descriptions simplifies the task by requiring the model to directly map the provided text to a world specification, bypassing the need to infer symbolic action dynamics. 
The observed consistent improvement (as shown in Figure~\ref{fig:concrete}) supports the claim that the model's ability to deduce action dynamics from abstract descriptions is still lacking.
We also provide more detailed experimental results in Appendix~\ref{app:concrete_results}.

\begin{figure}[t]
    \centering
    \includegraphics[width=\linewidth]{images/concrete_vs_abstract.pdf}
    \caption{
    % Comparison of model performance on abstract versus concrete domain descriptions, showing the base score for abstract descriptions (blue) and the improvement gained from concrete descriptions (green).
    Comparison of model performance on abstract versus concrete domain descriptions, showing the base score for abstract descriptions (\textcolor[rgb]{0.663,0.773,0.906}{blue}) and the improvement gained from concrete descriptions (\textcolor[rgb]{0.663,0.867,0.655}{green}).
    }
    \label{fig:concrete}
\end{figure}

\vspace{-5pt}

\section{Related Work}

% \paragraph{World Modeling with LLMs.}
% Neural World Modeling
% https://arxiv.org/pdf/2412.03572
% merler2024generating 基于mcts生成code world model
% wong2023word
% li2022emergent leverage transformer model to predict next legal step on a board game...
% tang2024worldcoder construct a code world model via interacting with the environments
% direct application of LLMs for state prediction during planning~\cite{hao2023reasoning,wang2023promptagent}, where benchmarks like ByteSized32-State-Prediction~\cite{wang2024can} have been developed to evaluate LLMs' predictive capabilities. However, this evaluation approach faces significant challenges: exhaustively evaluating all possible state transitions becomes intractable in complex environments, and limited interpretability and high inference latency constrain deeper analysis of LLMs' world modeling capabilities.
Neural world modeling is a long-standing research topic with widespread applications across various fields, including reinforcement learning~\citep{ha2018world, ha2018recurrent}, robotics~\citep{wu2023daydreamer}, and autonomous driving~\citep{guan2024world}, among others.
In recent years, LLMs trained on massive datasets have demonstrated zero-shot capabilities across a variety of tasks, including planning~\cite{zhao2023survey,qin2024large,huang2022language,hu2024hiagent}, robotics~\cite{mu2024robocodex,chen2024textbfemostextbfembodimentawareheterogeneoustextbfmultirobot}, analog design~\cite{lai2024analogcoder}, and more.
% Preliminary studies propose directly using LLMs as world models~\citep{hao2023reasoning,wang2024can,wang2023promptagent,li2022emergent}, where the input consists of the state and action, and the LLM outputs the predicted next state. 
% However, due to the unreliability of LLM outputs and their limited interpretability, this approach can lead to the problem of accumulating errors. 
Preliminary studies propose directly using LLMs as world models~\citep{hao2023reasoning,wang2024can,wang2023promptagent,li2022emergent}, by taking the state and action as input and predicting the next state, but the unreliability and limited interpretability of LLM outputs can lead to accumulating errors.
Moreover, some studies have shown that autoregressive models perform poorly in predicting action effects~\cite{banerjee2020can,luo2023towards}.
Tree-planner~\cite{hu2023tree} instead proposes to constructing the possible action space using LLMs before executing.
Another line of work focuses on leveraging LLMs to construct symbolic world models~\citep{oswald2024large,silver2024generalized,smirnov2024generating,zhu2024language,wang2023bytesized32,wong2023word,vafa2024evaluating}. 
For example, ~\citet{guan2023leveraging} uses LLMs to generate a PDDL domain model and relies on human feedback to correct errors. 
AgentGen~\citep{hu2024agentgen} synthesizes diverse PDDL domains, aiming to create high-quality planning data. 
~\citet{xie2024making} propose to finetune LLMs for predicting precondition and effect of actions.
% ~\citet{vafa2024evaluating} proposes novel evaluation metrics inspired by the Myhill-Nerode theorem to assess the implicit world models within generative models.
% 
Despite the growing interest in this research direction, there is currently a lack of a comprehensive benchmark in this area. 


% \vspace{-10pt}

\section{Conclusion}

We present \benchmark, a novel benchmark consisting of hundreds of domains designed to evaluate the world modeling capabilities of large language models (LLMs). 
Developed through a meticulous and thorough process, \benchmark provides a robust foundation for analysis. 
Additionally, we conducted an extensive evaluation involving 16 different LLMs from 9 model families based on \benchmark.
We hope that \benchmark will inspire future research in leveraging LLMs as world models.


\section*{Ethical Considerations}

\textbf{Data Access.} We collected the \benchmark data from open-source repositories and ensured that these repositories are available for academic research in accordance with our commitment to ethical data use.

\noindent \textbf{Participant Recruitment.} 
We recruited graduate students as annotators and required all participants to achieve an IELTS score of 6 or above. 
To mitigate potential biases stemming from participants’ geographical backgrounds, we minimized national differences in the dataset by focusing on human commonsense. 
All annotators provided informed consent and were compensated above the local minimum wage—\$10 per hour for standard annotators and \$20 per hour for senior annotators.

\noindent \textbf{Potential Risk.} After careful examination, we confirmed that our dataset does not contain any personal data (e.g., names, contacting information), and our data collection procedures adhere to ethical guidelines.

\section*{Limitation}

Due to the limited number of available domains online, we did not construct a large-scale training set. 
Future work should focus on expanding the dataset by incorporating additional data sources, such as synthesized data~\cite{hu2024agentgen}, to cover a broader range of domains.
Furthermore, although we conducted regular inspections to minimize the introduction of subjectivity into the dataset, the unavoidable influence of human subjectivity during manual annotation may introduce potential biases.


% \clearpage
% \newpage

\nocite{langley00}
% \begin{abstract}
% This document is a supplement to the general instructions for *ACL authors. It contains instructions for using the \LaTeX{} style files for ACL conferences.
% The document itself conforms to its own specifications, and is therefore an example of what your manuscript should look like.
% These instructions should be used both for papers submitted for review and for final versions of accepted papers.
% \end{abstract}

% \section{Introduction}

% These instructions are for authors submitting papers to *ACL conferences using \LaTeX. They are not self-contained. All authors must follow the general instructions for *ACL proceedings,\footnote{\url{http://acl-org.github.io/ACLPUB/formatting.html}} and this document contains additional instructions for the \LaTeX{} style files.

% The templates include the \LaTeX{} source of this document (\texttt{acl\_latex.tex}),
% the \LaTeX{} style file used to format it (\texttt{acl.sty}),
% an ACL bibliography style (\texttt{acl\_natbib.bst}),
% an example bibliography (\texttt{custom.bib}),
% and the bibliography for the ACL Anthology (\texttt{anthology.bib}).

% \section{Engines}

% To produce a PDF file, pdf\LaTeX{} is strongly recommended (over original \LaTeX{} plus dvips+ps2pdf or dvipdf). Xe\LaTeX{} also produces PDF files, and is especially suitable for text in non-Latin scripts.

% \section{Preamble}

% The first line of the file must be
% \begin{quote}
% \begin{verbatim}
% \documentclass[11pt]{article}
% \end{verbatim}
% \end{quote}

% To load the style file in the review version:
% \begin{quote}
% \begin{verbatim}
% \usepackage[review]{acl}
% \end{verbatim}
% \end{quote}
% For the final version, omit the \verb|review| option:
% \begin{quote}
% \begin{verbatim}
% \usepackage{acl}
% \end{verbatim}
% \end{quote}

% To use Times Roman, put the following in the preamble:
% \begin{quote}
% \begin{verbatim}
% \usepackage{times}
% \end{verbatim}
% \end{quote}
% (Alternatives like txfonts or newtx are also acceptable.)

% Please see the \LaTeX{} source of this document for comments on other packages that may be useful.

% Set the title and author using \verb|\title| and \verb|\author|. Within the author list, format multiple authors using \verb|\and| and \verb|\And| and \verb|\AND|; please see the \LaTeX{} source for examples.

% By default, the box containing the title and author names is set to the minimum of 5 cm. If you need more space, include the following in the preamble:
% \begin{quote}
% \begin{verbatim}
% \setlength\titlebox{<dim>}
% \end{verbatim}
% \end{quote}
% where \verb|<dim>| is replaced with a length. Do not set this length smaller than 5 cm.

% \section{Document Body}

% \subsection{Footnotes}

% Footnotes are inserted with the \verb|\footnote| command.\footnote{This is a footnote.}

% \subsection{Tables and figures}

% See Table~\ref{tab:accents} for an example of a table and its caption.
% \textbf{Do not override the default caption sizes.}

% \begin{table}
%   \centering
%   \begin{tabular}{lc}
%     \hline
%     \textbf{Command} & \textbf{Output} \\
%     \hline
%     \verb|{\"a}|     & {\"a}           \\
%     \verb|{\^e}|     & {\^e}           \\
%     \verb|{\`i}|     & {\`i}           \\
%     \verb|{\.I}|     & {\.I}           \\
%     \verb|{\o}|      & {\o}            \\
%     \verb|{\'u}|     & {\'u}           \\
%     \verb|{\aa}|     & {\aa}           \\\hline
%   \end{tabular}
%   \begin{tabular}{lc}
%     \hline
%     \textbf{Command} & \textbf{Output} \\
%     \hline
%     \verb|{\c c}|    & {\c c}          \\
%     \verb|{\u g}|    & {\u g}          \\
%     \verb|{\l}|      & {\l}            \\
%     \verb|{\~n}|     & {\~n}           \\
%     \verb|{\H o}|    & {\H o}          \\
%     \verb|{\v r}|    & {\v r}          \\
%     \verb|{\ss}|     & {\ss}           \\
%     \hline
%   \end{tabular}
%   \caption{Example commands for accented characters, to be used in, \emph{e.g.}, Bib\TeX{} entries.}
%   \label{tab:accents}
% \end{table}

% As much as possible, fonts in figures should conform
% to the document fonts. See Figure~\ref{fig:experiments} for an example of a figure and its caption.

% Using the \verb|graphicx| package graphics files can be included within figure
% environment at an appropriate point within the text.
% The \verb|graphicx| package supports various optional arguments to control the
% appearance of the figure.
% You must include it explicitly in the \LaTeX{} preamble (after the
% \verb|\documentclass| declaration and before \verb|\begin{document}|) using
% \verb|\usepackage{graphicx}|.

% \begin{figure}[t]
%   \includegraphics[width=\columnwidth]{example-image-golden}
%   \caption{A figure with a caption that runs for more than one line.
%     Example image is usually available through the \texttt{mwe} package
%     without even mentioning it in the preamble.}
%   \label{fig:experiments}
% \end{figure}

% \begin{figure*}[t]
%   \includegraphics[width=0.48\linewidth]{example-image-a} \hfill
%   \includegraphics[width=0.48\linewidth]{example-image-b}
%   \caption {A minimal working example to demonstrate how to place
%     two images side-by-side.}
% \end{figure*}

% \subsection{Hyperlinks}

% Users of older versions of \LaTeX{} may encounter the following error during compilation:
% \begin{quote}
% \verb|\pdfendlink| ended up in different nesting level than \verb|\pdfstartlink|.
% \end{quote}
% This happens when pdf\LaTeX{} is used and a citation splits across a page boundary. The best way to fix this is to upgrade \LaTeX{} to 2018-12-01 or later.

% \subsection{Citations}

% \begin{table*}
%   \centering
%   \begin{tabular}{lll}
%     \hline
%     \textbf{Output}           & \textbf{natbib command} & \textbf{ACL only command} \\
%     \hline
%     \citep{Gusfield:97}       & \verb|\citep|           &                           \\
%     \citealp{Gusfield:97}     & \verb|\citealp|         &                           \\
%     \citet{Gusfield:97}       & \verb|\citet|           &                           \\
%     \citeyearpar{Gusfield:97} & \verb|\citeyearpar|     &                           \\
%     \citeposs{Gusfield:97}    &                         & \verb|\citeposs|          \\
%     \hline
%   \end{tabular}
%   \caption{\label{citation-guide}
%     Citation commands supported by the style file.
%     The style is based on the natbib package and supports all natbib citation commands.
%     It also supports commands defined in previous ACL style files for compatibility.
%   }
% \end{table*}

% Table~\ref{citation-guide} shows the syntax supported by the style files.
% We encourage you to use the natbib styles.
% You can use the command \verb|\citet| (cite in text) to get ``author (year)'' citations, like this citation to a paper by \citet{Gusfield:97}.
% You can use the command \verb|\citep| (cite in parentheses) to get ``(author, year)'' citations \citep{Gusfield:97}.
% You can use the command \verb|\citealp| (alternative cite without parentheses) to get ``author, year'' citations, which is useful for using citations within parentheses (e.g. \citealp{Gusfield:97}).

% A possessive citation can be made with the command \verb|\citeposs|.
% This is not a standard natbib command, so it is generally not compatible
% with other style files.

% \subsection{References}

% \nocite{Ando2005,andrew2007scalable,rasooli-tetrault-2015}

% The \LaTeX{} and Bib\TeX{} style files provided roughly follow the American Psychological Association format.
% If your own bib file is named \texttt{custom.bib}, then placing the following before any appendices in your \LaTeX{} file will generate the references section for you:
% \begin{quote}
% \begin{verbatim}
% \bibliography{custom}
% \end{verbatim}
% \end{quote}

% You can obtain the complete ACL Anthology as a Bib\TeX{} file from \url{https://aclweb.org/anthology/anthology.bib.gz}.
% To include both the Anthology and your own .bib file, use the following instead of the above.
% \begin{quote}
% \begin{verbatim}
% \bibliography{anthology,custom}
% \end{verbatim}
% \end{quote}

% Please see Section~\ref{sec:bibtex} for information on preparing Bib\TeX{} files.

% \subsection{Equations}

% An example equation is shown below:
% \begin{equation}
%   \label{eq:example}
%   A = \pi r^2
% \end{equation}

% Labels for equation numbers, sections, subsections, figures and tables
% are all defined with the \verb|\label{label}| command and cross references
% to them are made with the \verb|\ref{label}| command.

% This an example cross-reference to Equation~\ref{eq:example}.

% \subsection{Appendices}

% Use \verb|\appendix| before any appendix section to switch the section numbering over to letters. See Appendix~\ref{sec:appendix} for an example.

% \section{Bib\TeX{} Files}
% \label{sec:bibtex}

% Unicode cannot be used in Bib\TeX{} entries, and some ways of typing special characters can disrupt Bib\TeX's alphabetization. The recommended way of typing special characters is shown in Table~\ref{tab:accents}.

% Please ensure that Bib\TeX{} records contain DOIs or URLs when possible, and for all the ACL materials that you reference.
% Use the \verb|doi| field for DOIs and the \verb|url| field for URLs.
% If a Bib\TeX{} entry has a URL or DOI field, the paper title in the references section will appear as a hyperlink to the paper, using the hyperref \LaTeX{} package.

% \section*{Acknowledgments}

% This document has been adapted
% by Steven Bethard, Ryan Cotterell and Rui Yan
% from the instructions for earlier ACL and NAACL proceedings, including those for
% ACL 2019 by Douwe Kiela and Ivan Vuli\'{c},
% NAACL 2019 by Stephanie Lukin and Alla Roskovskaya,
% ACL 2018 by Shay Cohen, Kevin Gimpel, and Wei Lu,
% NAACL 2018 by Margaret Mitchell and Stephanie Lukin,
% Bib\TeX{} suggestions for (NA)ACL 2017/2018 from Jason Eisner,
% ACL 2017 by Dan Gildea and Min-Yen Kan,
% NAACL 2017 by Margaret Mitchell,
% ACL 2012 by Maggie Li and Michael White,
% ACL 2010 by Jing-Shin Chang and Philipp Koehn,
% ACL 2008 by Johanna D. Moore, Simone Teufel, James Allan, and Sadaoki Furui,
% ACL 2005 by Hwee Tou Ng and Kemal Oflazer,
% ACL 2002 by Eugene Charniak and Dekang Lin,
% and earlier ACL and EACL formats written by several people, including
% John Chen, Henry S. Thompson and Donald Walker.
% Additional elements were taken from the formatting instructions of the \emph{International Joint Conference on Artificial Intelligence} and the \emph{Conference on Computer Vision and Pattern Recognition}.

% % Bibliography entries for the entire Anthology, followed by custom entries
% %\bibliography{anthology,custom}
% % Custom bibliography entries only

\bibliography{custom}

\newpage
\appendix
\onecolumn

\section{Benchmark Construction}
\subsection{Example}
\label{app:dataset-example}

\subsubsection{Domain Example}

\begin{lstlisting}[style=pddl,caption={Grid PDDL},label=lst:grid-domain]
(define (domain grid)
  (:requirements :strips)
  (:predicates (conn ?x ?y) (key-shape ?k ?s) (lock-shape ?x ?s)
	       (at ?r ?x ) (at-robot ?x) (place ?p) (key ?k) (shape ?s)
	       (locked ?x) (holding ?k)  (open ?x)  (arm-empty ))

  (:action unlock
    :parameters (?curpos ?lockpos ?key ?shape)
    :precondition (and (place ?curpos) (place ?lockpos) (key ?key)
		       (shape ?shape) (conn ?curpos ?lockpos)
		       (key-shape ?key ?shape) (lock-shape ?lockpos ?shape)
		       (at-robot ?curpos) (locked ?lockpos) (holding ?key))
    :effect (and (open ?lockpos) (not (locked ?lockpos))))

  (:action move
    :parameters (?curpos ?nextpos)
    :precondition (and (place ?curpos) (place ?nextpos) (at-robot ?curpos)
		       (conn ?curpos ?nextpos) (open ?nextpos))
    :effect (and (at-robot ?nextpos) (not (at-robot ?curpos))))

  (:action pickup
    :parameters (?curpos ?key)
    :precondition (and (place ?curpos) (key ?key) (at-robot ?curpos)
		       (at ?key ?curpos) (arm-empty ))
    :effect (and (holding ?key) (not (at ?key ?curpos)) (not (arm-empty ))))

  (:action pickup-and-loose
    :parameters (?curpos ?newkey ?oldkey)
    :precondition (and (place ?curpos) (key ?newkey) (key ?oldkey)
		       (at-robot ?curpos) (holding ?oldkey)
		       (at ?newkey ?curpos))
    :effect (and (holding ?newkey) (at ?oldkey ?curpos)
		 (not (holding ?oldkey)) (not (at ?newkey ?curpos))))

  (:action putdown
    :parameters (?curpos ?key)
    :precondition (and (place ?curpos) (key ?key) (at-robot ?curpos)
		       (holding ?key))
    :effect (and (arm-empty ) (at ?key ?curpos) (not (holding ?key))))
  )
\end{lstlisting}

\subsubsection{Abstract Description}
\label{app:diffdesc}

\noindent \textbf{General.}
This domain models a robot navigating a grid environment with the objective of unlocking doors and moving through the grid. The robot can carry keys that match the shape of locks to unlock doors. The environment includes places, keys with specific shapes, and doors (locks) with corresponding shapes that need to be unlocked.

\noindent \textbf{Predicates.}
The following predicates are used in the domain:
\begin{itemize}
    \item \texttt{(conn ?x ?y)}: Indicates a connection between two places \texttt{?x} and \texttt{?y}, allowing movement between them.
    \item \texttt{(key-shape ?k ?s)}: Indicates that key \texttt{?k} has shape \texttt{?s}.
    \item \texttt{(lock-shape ?x ?s)}: Indicates that lock (or door) at place \texttt{?x} has shape \texttt{?s}.
    \item \texttt{(at ?r ?x)}: Indicates that key \texttt{?r} is at place \texttt{?x}.
    \item \texttt{(at-robot ?x)}: Indicates that the robot is at place \texttt{?x}.
    \item \texttt{(place ?p)}: Indicates that \texttt{?p} is a place in the grid.
    \item \texttt{(key ?k)}: Indicates that \texttt{?k} is a key.
    \item \texttt{(shape ?s)}: Indicates that \texttt{?s} is a shape.
    \item \texttt{(locked ?x)}: Indicates that the place \texttt{?x} is locked.
    \item \texttt{(holding ?k)}: Indicates that the robot is holding key \texttt{?k}.
    \item \texttt{(open ?x)}: Indicates that the place \texttt{?x} is open.
    \item \texttt{(arm-empty)}: Indicates that the robot's arm is empty.
\end{itemize}

\noindent \textbf{Actions.}
The following actions are available in the domain:
\begin{itemize} 
    \item \texttt{unlock <?curpos> <?lockpos> <?key> <?shape>}: Allows the robot to unlock a door at place \texttt{<?lockpos>} using a key of a specific shape.
    \item \texttt{move <?curpos> <?nextpos>}: Allows the robot to move from place \texttt{<?curpos>} to place \texttt{<?nextpos>}.
    \item \texttt{pickup <?curpos> <?key>}: Allows the robot to pick up a key at its current location.
    \item \texttt{pickup-and-loose <?curpos> <?newkey> <?oldkey>}: Allows the robot to pick up a new key while dropping the one it was holding.
    \item \texttt{putdown <?curpos> <?key>}: Allows the robot to put down a key it is holding.
\end{itemize}

\subsubsection{Concrete Description}
\label{app:concrete_desc}

\noindent \textbf{General.}
This domain models a robot navigating a grid environment with the objective of unlocking doors and moving through the grid. The robot can carry keys that match the shape of locks to unlock doors. The environment includes places, keys with specific shapes, and doors (locks) with corresponding shapes that need to be unlocked.

\noindent \textbf{Predicates.}
The following predicates are used in the domain:
\begin{itemize}
    \item \texttt{(conn ?x ?y)}: Indicates a connection between two places \texttt{?x} and \texttt{?y}, allowing movement between them.
    \item \texttt{(key-shape ?k ?s)}: Indicates that key \texttt{?k} has shape \texttt{?s}.
    \item \texttt{(lock-shape ?x ?s)}: Indicates that lock (or door) at place \texttt{?x} has shape \texttt{?s}.
    \item \texttt{(at ?r ?x)}: Indicates that key \texttt{?r} is at place \texttt{?x}.
    \item \texttt{(at-robot ?x)}: Indicates that the robot is at place \texttt{?x}.
    \item \texttt{(place ?p)}: Indicates that \texttt{?p} is a place in the grid.
    \item \texttt{(key ?k)}: Indicates that \texttt{?k} is a key.
    \item \texttt{(shape ?s)}: Indicates that \texttt{?s} is a shape.
    \item \texttt{(locked ?x)}: Indicates that the place \texttt{?x} is locked.
    \item \texttt{(holding ?k)}: Indicates that the robot is holding key \texttt{?k}.
    \item \texttt{(open ?x)}: Indicates that the place \texttt{?x} is open.
    \item \texttt{(arm-empty)}: Indicates that the robot's arm is empty.
\end{itemize}

\noindent \textbf{Actions.} The following actions are available in the domain:
\begin{itemize}
\item \texttt{unlock <?curpos> <?lockpos> <?key> <?shape>}: Allows the robot to unlock a door at place \texttt{<?lockpos>} using a key of a specific shape if the robot is at place \texttt{<?curpos>}, the key matches the lock's shape, the robot is holding the key, there is a connection between \texttt{<?curpos>} and \texttt{<?lockpos>}, and the destination is locked. After the action, the lock is no longer locked.
\item \texttt{move <?curpos> <?nextpos>}: Allows the robot to move from place \texttt{<?curpos>} to place \texttt{<?nextpos>} if there is a connection between them and the destination is open. After the move, the robot is no longer at the original place.
\item \texttt{pickup <?curpos> <?key>}: Allows the robot to pick up a key at its current location if the robot's arm is empty and it is at the same place as the new key. After the action, the robot is holding the key, and the key is no longer at that location.
\item \texttt{pickup-and-loose <?curpos> <?newkey> <?oldkey>}: Allows the robot to pick up a new key while dropping the one it was holding if it is at the same place as the new key. After the action, the robot is holding the new key, and the old key is at the robot's current location.
\item \texttt{putdown <?curpos> <?key>}: Allows the robot to put down a key it is holding if it is at a specific place. After the action, the robot's arm is empty, and the key is at that location.
\end{itemize}

\subsection{Preliminary Experiment}
\label{app:self_eval}

The experimental results show that LLM's effectiveness in detecting PDDL semantic errors is limited, with an accuracy of 55.0\%, a precision of 56.2\%, a recall rate of 45.0\%, an F1 score of 50.0\%, and a ROC AUC of 55.0. ROC AUC indicates that the model is close to random performance, making it difficult to reliably distinguish between correct and incorrect PDDL domains. 
% The low recall rate suggests significant deficiencies in the detection of actual semantic errors, while moderate precision reveals frequent false positives in negative cases. 
% This performance gap may arise from the limitations of LLM in strict logical reasoning and specific domain semantic understanding, particularly regarding predicates consistency, action preconditions completeness, and action effects logical consistency, calling for the need for formal verification methods to improve evaluation capabilities.\\
Below is the prompt used for LLMs to detect semantic errors in generated PDDL domains:\\
\lstset{
    basicstyle=\small\ttfamily, % Use a monospace font and smaller size
    breaklines=true, % Enable automatic line breaking
    % backgroundcolor=\color{gray!10}, % Optional: light gray background
    % frame=single, % Optional: frame around the code block
    % numbers=left, % Optional: line numbers
    % numbersep=5pt, % Optional: line number separation
    % xleftmargin=20pt, % Optional: left margin
    % xrightmargin=20pt % Optional: right margin
}
\begin{lstlisting}
You are an expert in automated planning systems and PDDL semantics. Your task is to evaluate whether the LLM are physically accurate models of the world or whether they don't make sense by detecting semantic errors in generated PDDL domain.
You need carefully analyze the following PDDL domain by comparing it to the pddl domain description, evaluate whether the generated pddl domain contains SEMANTIC ERRORS in these key aspects:
1. Predicates consistency.
2. Action parameters validity.
3. Action preconditions completeness.
4. Action effects logical consistency.
5. Consistency with the description.

An example of semantic error would be:
1. Missing precondition constraints (e.g. executing "unlock-door" without holding a key).
2. Contradictory effects (e.g. both adding and deleting the same predicate).
3. Incorrect predicate arguments (e.g. reversed parameter order).

Output Format:
{
"evaluation": "yes/no",
"error_type": "[MissingPrecond|IncorrectEffect|MissingPredicate|...]",
"confidence": "high/medium/low",
"evidence": "<specific code segment>",
"justification": "<short justification>"
}

PDDL Description:
{PDDL_DESCRIPTION}

Generated PDDL:
{PDDL_DOMAIN}
\end{lstlisting}

\subsection{More Details on Data Analysis}
\label{app:detailed_data_analysis}

% \subsection{PDDL requirements}
Figure~\ref{fig:heatmap} shows the co-occurrence of PDDL requirements across domains, highlighting that \texttt{:typing} and \texttt{:strips} are the most prevalent features.

\begin{figure}[htbp]
    \centering
    \includegraphics[width=\linewidth]{images/heatmap.pdf}
    \caption{
    The co-occurrence matrix of requirements of \benchmark.
    }
    \label{fig:heatmap}
\end{figure}


\section{More Details on Experiments}
\label{app:exp}

\subsection{Evaluation Metrics}
\label{app:eval_metric}

\noindent \textbf{Levenshtein Ratio.} 
The Levenshtein Ratio is a value between 0 and 1 that quantifies the similarity between two strings, such as a predicted PDDL domain and a golden PDDL domain. It is derived from the Levenshtein distance, which calculates the minimum number of character-level operations—insertions, deletions, or substitutions—needed to convert one string into the other. The ratio is then computed by dividing the Levenshtein distance by the length of the longer string, providing a measure of how closely the two strings match, where a value closer to 1 indicates high similarity and a value closer to 0 indicates significant differences.

\noindent \textbf{Component-wise F1 Scores.} 
The F1 score is mainly used to measure the similarity between the predicted PDDL domain and the golden PDDL domain, specifically including predicate F1 and action F1. The range of this score is from 0 to 1, which is the harmonic mean of precision and recall. 


% \subsection{Experiment Configuration}
% \label{app:exp_config}

\subsection{Prompt Examples}
\label{app:prompt}


\subsubsection{Error Correction}
\label{app:error}

\begin{lstlisting}
I would like you to serve as an expert in PDDL, assisting me in correcting erroneous PDDL code. I will provide you with the incorrect PDDL along with the error messages returned by the system. You should output your thought process firstly. You MUST enclose the COMPLETE corrected PDDL within ```pddl```.
Here are some hints to help you debug the pddl domain file:
1. You should start by checking if all the essential domain constructs or domain definition constructs are present. Commonly included domains comprise:
    a. :domain declaration to name the domain.
    b. :requirements to specify the PDDL features used in the domain.
    c. :types to define any object types for categorizing entities in the planning problem.
    d. :constants (if necessary) to declare any objects that remain static throughout the planning problems.
    e. :predicates to define the properties and relations between objects that can change over time.
    f. :functions (if necessary) to define numeric functions for more complex domains.
    g. :action definitions for each action that agents can perform, including parameters, preconditions, and effects.
2. You need to check the number of parameters of each actions.
3. Having :typing in the domain indicates that it uses a hierarchy to organize objects. Therefore, it's crucial to clearly list all object types related to the planning task in a :types section.
4. '-' should not appear in :types.



Round 0
Incorrect PDDL:
(:action clean-up
    :parameters (?robot - robot ?robotTile - tile ?tileToBeCleaned - tile)
    :precondition (and 
        (robot-at ?robot ?robotTile)
        (up ?tileToBeCleaned ?robotTile)
        (clear ?tileToBeCleaned)
        (not (cleaned ?tileToBeCleaned))
    )
    :effect (and 
        (cleaned ?tileToBeCleaned)
    )
)

(:action clean-down
    :parameters (?robot - robot ?robotTile - tile ?tileToBeCleaned - tile)
    :precondition (and 
        (robot-at ?robot ?robotTile)
        (down ?tileToBeCleaned ?robotTile)
        (clear ?tileToBeCleaned)
        (not (cleaned ?tileToBeCleaned))
    )
    :effect (and 
        (cleaned ?tileToBeCleaned)
    )
)

(:action up
    :parameters (?robot - robot ?robotTile - tile ?moveToNextTile - tile)
    :precondition (and 
        (robot-at ?robot ?robotTile)
        (up ?moveToNextTile ?robotTile)
        (clear ?moveToNextTile)
    )
    :effect (and 
        (not (robot-at ?robot ?robotTile))
        (robot-at ?robot ?moveToNextTile)
    )
)

(:action down
    :parameters (?robot - robot ?robotTile - tile ?moveToNextTile - tile)
    :precondition (and 
        (robot-at ?robot ?robotTile)
        (down ?moveToNextTile ?robotTile)
        (clear ?moveToNextTile)
    )
    :effect (and 
        (not (robot-at ?robot ?robotTile))
        (robot-at ?robot ?moveToNextTile)
    )
)

(:action right
    :parameters (?robot - robot ?robotTile - tile ?moveToNextTile - tile)
    :precondition (and 
        (robot-at ?robot ?robotTile)
        (right ?moveToNextTile ?robotTile)
        (clear ?moveToNextTile)
    )
    :effect (and 
        (not (robot-at ?robot ?robotTile))
        (robot-at ?robot ?moveToNextTile)
    )
)

(:action left
    :parameters (?robot - robot ?robotTile - tile ?moveToNextTile - tile)
    :precondition (and 
        (robot-at ?robot ?robotTile)
        (left ?moveToNextTile ?robotTile)
        (clear ?moveToNextTile)
    )
    :effect (and 
        (not (robot-at ?robot ?robotTile))
        (robot-at ?robot ?moveToNextTile)
    )
)
Error Information:
ParsingError: line 1:1 mismatched input ':action' expecting 'define'
Corrected PDDL:
\end{lstlisting}
% \subsubsection{Prompt Examples}

% \noindent \textbf{Zero-Shot.}
\subsubsection{Zero-Shot Prompt}

\begin{lstlisting}
You are tasked with converting a given Planning Domain Definition Language (PDDL) domain description into its corresponding formal PDDL domain. The description will outline the essential components of the domains. 
Your output should be a well-structured PDDL domain that accurately represents the given description, adhering to the syntax and semantics of PDDL.
Your output pddl domain must be enclosed in ```pddl```.

You need to generate the corresponding domain pddl for the following description.
    
PDDL Domain Description:
### General
This domain is designed for a robot tasked with cleaning floor tiles. The robot can move in four directions (up, down, right, left) relative to its current position on a grid of tiles. The goal is to clean all the specified tiles by moving to them and performing a cleaning action.

### Types
- **robot**: Represents the robot that performs the cleaning.
- **tile**: Represents the individual tiles on the floor that may need to be cleaned.

### Predicates
- **(robot-at ?robot - robot ?robotTile - tile)**: Indicates that the robot is currently at a specific tile.
- **(up ?tileAbove - tile ?tileBelow - tile)**: Indicates that one tile is directly above another.
- **(down ?tileBelow - tile ?tileAbove - tile)**: Indicates that one tile is directly below another.
- **(right ?tileOnRight - tile ?tileOnLeft - tile)**: Indicates that one tile is directly to the right of another.
- **(left ?tileOnLeft - tile ?tileOnRight - tile)**: Indicates that one tile is directly to the left of another.
- **(clear ?clearedTile - tile)**: Indicates that a tile is clear and robot can move there.
- **(cleaned ?cleanedTile - tile)**: Indicates that a tile has been cleaned.

### Actions
- **clean-up <?robot> <?robotTile> <?tileToBeCleaned>**: Allows the robot (?robot) to clean a tile (?tileToBeCleaned) that is directly above its current position (?robotTile).  

- **clean-down <?robot> <?robotTile> <?tileToBeCleaned>**: Allows the robot (?robot) to clean a tile (?tileToBeCleaned) that is directly below its current position (?robotTile).  

- **up <?robot> <?robotTile> <?moveToNextTile>**: Moves the robot (?robot) to a tile (?moveToNextTile) directly above its current position (?robotTile).  

- **down <?robot> <?robotTile> <?moveToNextTile>**: Moves the robot (?robot) to a tile (?moveToNextTile) directly below its current position (?robotTile).  

- **right <?robot> <?robotTile> <?moveToNextTile>**: Moves the robot (?robot) to a tile (?moveToNextTile) directly to the right of its current position (?robotTile).  

- **left <?robot> <?robotTile> <?moveToNextTile>**: Moves the robot (?robot) to a tile (?moveToNextTile) directly to the left of its current position (?robotTile).
PDDL Domain:
Let's think step by step.
\end{lstlisting}

% \noindent \textbf{Few-Shot.}
\subsubsection{Few-Shot Prompt}

\begin{lstlisting}
You are tasked with converting a given Planning Domain Definition Language (PDDL) domain description into its corresponding formal PDDL domain. The description will outline the essential components of the domains. Your output should be a well-structured PDDL domain that accurately represents the given description, adhering to the syntax and semantics of PDDL.
Your output must strictly adhere to the format exemplified below. 
Here are some examples:

Example 0:
## PDDL Domain Description
### General
You are a robot equipped with a gripper mechanism, designed to move and manipulate balls between different rooms. The domain focuses on the robot's ability to navigate rooms, pick up balls, and drop them in designated locations.
### Types
- **room**: Represents the different rooms within the environment.
- **ball**: Represents the objects that the robot can pick up and move.
- **gripper**: Represents the robot's mechanism for holding balls.
### Predicates
- **(at-robby ?r - room)**: Indicates that Robby, the robot, is currently in room ?r.
- **(at ?b - ball ?r - room)**: Indicates that ball ?b is located in room ?r.
- **(free ?g - gripper)**: Indicates that the gripper ?g is not currently holding any ball.
- **(carry ?o - ball ?g - gripper)**: Indicates that the gripper ?g is carrying ball ?o.
### Actions
- **move <?from> <?to>**: Allows Robby to move from one room to another.  
- **pick <?obj> <?room> <?gripper>**: Enables Robby to pick up a ball in a room using its gripper.  
- **drop <?obj> <?room> <?gripper>**: Allows Robby to drop a ball it is carrying into a room.

## PDDL Domain
```pddl
(define (domain gripper-strips)
	(:types 
		room - object
		ball - object
		gripper - object
		)
   (:predicates
		(at-robby ?r - room)
		(at ?b - ball ?r - room)
		(free ?g - gripper)
		(carry ?o - ball ?g - gripper))
   (:action move
       :parameters  (?from - room ?to - room)
       :precondition (and (at-robby ?from))
       :effect (and  (at-robby ?to)
		     (not (at-robby ?from))))
   (:action pick
       :parameters (?obj - ball ?room - room ?gripper - gripper)
       :precondition  (and 
			    (at ?obj ?room) (at-robby ?room) (free ?gripper))
       :effect (and (carry ?obj ?gripper)
		    (not (at ?obj ?room)) 
		    (not (free ?gripper))))
   (:action drop
       :parameters  (?obj - ball ?room - room ?gripper - gripper)
       :precondition  (and 
			    (carry ?obj ?gripper) (at-robby ?room))
       :effect (and (at ?obj ?room)
		    (free ?gripper)
		    (not (carry ?obj ?gripper)))))
```


Example 1:
## PDDL Domain Description
### General
This domain represents a simplified version of the classic "blocks world" problem, where a robot arm can stack and unstack blocks. The domain includes actions for picking up blocks from the table, putting down blocks onto the table, stacking blocks on top of each other, and unstacking them. The goal is to manipulate the blocks to achieve a specified configuration.
### Predicates
(clear ?x): Indicates that there is no block on top of block ?x, making it accessible for stacking or picking up.
(on-table ?x): Indicates that block ?x is directly on the table.
(arm-empty): Indicates that the robot's arm is not holding any block.
(holding ?x): Indicates that the robot's arm is currently holding block ?x.
(on ?x ?y): Indicates that block ?x is directly on top of block ?y.
### Actions
- **pickup <?ob>**: Picks up an object (?ob) from the table.  
- **putdown <?ob>**: Puts down an object (?ob) onto the table.  
- **stack <?ob> <?underob>**: Stacks an object (?ob) on top of another object (?underob), making the robot arm empty.  
- **unstack <?ob> <?underob>**: Unstacks an object (?ob) from another object (?underob), making the robot arm no longer empty.

## PDDL Domain:
```pddl
(define (domain blocksworld)
  (:requirements :strips)
(:predicates (clear ?x)
             (on-table ?x)
             (arm-empty)
             (holding ?x)
             (on ?x ?y))

(:action pickup
  :parameters (?ob)
  :precondition (and (clear ?ob) (on-table ?ob) (arm-empty))
  :effect (and (holding ?ob) (not (clear ?ob)) (not (on-table ?ob)) 
               (not (arm-empty))))

(:action putdown
  :parameters  (?ob)
  :precondition (holding ?ob)
  :effect (and (clear ?ob) (arm-empty) (on-table ?ob) 
               (not (holding ?ob))))

(:action stack
  :parameters  (?ob ?underob)
  :precondition (and (clear ?underob) (holding ?ob))
  :effect (and (arm-empty) (clear ?ob) (on ?ob ?underob)
               (not (clear ?underob)) (not (holding ?ob))))

(:action unstack
  :parameters  (?ob ?underob)
  :precondition (and (on ?ob ?underob) (clear ?ob) (arm-empty))
  :effect (and (holding ?ob) (clear ?underob)
               (not (on ?ob ?underob)) (not (clear ?ob)) (not (arm-empty)))))
```

You need to generate the corresponding domain pddl for the following description.
    
## PDDL Domain Description
### General
This domain is designed for a robot tasked with cleaning floor tiles. The robot can move in four directions (up, down, right, left) relative to its current position on a grid of tiles. The goal is to clean all the specified tiles by moving to them and performing a cleaning action.

### Types
- **robot**: Represents the robot that performs the cleaning.
- **tile**: Represents the individual tiles on the floor that may need to be cleaned.

### Predicates
- **(robot-at ?robot - robot ?robotTile - tile)**: Indicates that the robot is currently at a specific tile.
- **(up ?tileAbove - tile ?tileBelow - tile)**: Indicates that one tile is directly above another.
- **(down ?tileBelow - tile ?tileAbove - tile)**: Indicates that one tile is directly below another.
- **(right ?tileOnRight - tile ?tileOnLeft - tile)**: Indicates that one tile is directly to the right of another.
- **(left ?tileOnLeft - tile ?tileOnRight - tile)**: Indicates that one tile is directly to the left of another.
- **(clear ?clearedTile - tile)**: Indicates that a tile is clear and robot can move there.
- **(cleaned ?cleanedTile - tile)**: Indicates that a tile has been cleaned.

### Actions
- **clean-up <?robot> <?robotTile> <?tileToBeCleaned>**: Allows the robot (?robot) to clean a tile (?tileToBeCleaned) that is directly above its current position (?robotTile).  

- **clean-down <?robot> <?robotTile> <?tileToBeCleaned>**: Allows the robot (?robot) to clean a tile (?tileToBeCleaned) that is directly below its current position (?robotTile).  

- **up <?robot> <?robotTile> <?moveToNextTile>**: Moves the robot (?robot) to a tile (?moveToNextTile) directly above its current position (?robotTile).  

- **down <?robot> <?robotTile> <?moveToNextTile>**: Moves the robot (?robot) to a tile (?moveToNextTile) directly below its current position (?robotTile).  

- **right <?robot> <?robotTile> <?moveToNextTile>**: Moves the robot (?robot) to a tile (?moveToNextTile) directly to the right of its current position (?robotTile).  

- **left <?robot> <?robotTile> <?moveToNextTile>**: Moves the robot (?robot) to a tile (?moveToNextTile) directly to the left of its current position (?robotTile).
## PDDL Domain
\end{lstlisting}

\section{More Details on Analysis}
\label{app:analysis_detail}

\subsection{Overall}

% \begin{table*}[htbp]
% \centering
% \caption{Distribution of Error Types in PDDL Generation. exec=1 indicates executable code, semantic=1 indicates semantically correct code.}
% \label{tab:error_distribution}
% \begin{tabular}{l|c|c|c}
% \toprule
%  & \textbf{Proportion (\%)} & \textbf{Number} & \textbf{Condition} \\ \midrule
% \textbf{Correct} & 23.76 & 24 & exec=1, semantic=1 \\ 
% \textbf{Syntax Error} & 11.88 & 12 & exec=0 \\ 
% \textbf{Semantic Error} & 64.36 & 65 & exec=1, semantic=0 \\ \midrule
% \textbf{All} & 100.00 & 101 &  \\ 
% \bottomrule
% \end{tabular}
% \end{table*}

\begin{table}[htbp]
\centering
\caption{Distribution of error types of \texttt{claude-3.5-sonnect} on \benchmark under few-shot setting.}
\label{tab:error_distribution}
\begin{tabular}{l|c|c}
\toprule
 & \textbf{Proportion (\%)} & \textbf{Number} \\ \midrule
\textbf{Correct} & 23.76 & 24 \\ 
\textbf{Syntax Error} & 11.88 & 12 \\ 
\textbf{Semantic Error} & 64.36 & 65 \\ \midrule
\textbf{All} & 100.00 & 101 \\ 
\bottomrule
\end{tabular}
\end{table}

The overall distribution for syntax errors and semantic errors is presented in Table~\ref{tab:error_distribution}.

\subsection{Syntax Error}

\begin{table*}[htbp]
\centering
\caption{Distribution of Syntax Errors in PDDL Generation (Total Samples: 66, a task may have 1 to 4 samples.)}
\label{tab:syntax_errors}
\resizebox{\textwidth}{!}{
\begin{tabular}{l|l|c}
\toprule
\textbf{Syntax Error} & \textbf{Explanation} & \textbf{Proportion (\%)} \\ 
\midrule
UndefinedDomainName & Missing mandatory \texttt{(define (domain ...))} declaration in PDDL header & 33.33 \\ 
IncorrectParentheses & Invalid empty/mismatched parentheses & 3.03 \\ 
UndefinedConstant & Reference to undeclared constants in predicates or actions & 13.64 \\ 
MissingRequirements & Absence of required PDDL extension declarations (e.g., \texttt{:action-costs}) & 22.73 \\ 
UndefinedType & Undeclared parent type in hierarchical type definitions & 18.18 \\ 
UnsupportedFeature & Use of parser-incompatible language features (e.g., \texttt{either} types) & 3.03 \\ 
TypeMismatch & Parameter type conflict with declared type constraints & 1.52 \\ 
UndefinedVariable & Undeclared variables in action preconditions/effects & 1.52 \\ 
DuplicateDefinition & Multiple declarations of identical domain elements & 3.03 \\ 
% \midrule
% \textbf{All} & & 100.00 \\ 
\bottomrule
\end{tabular}
}
\end{table*}

The distribution and detailed explanation of each syntax error type are presented in Table~\ref{tab:syntax_errors}.

\subsection{Semantic Error}
\begin{table*}[htbp]
\centering
\caption{Distribution of Semantic Errors in PDDL Generation (Total Samples: 91, a task may have multiple semantic errors.)}
\label{tab:semantic_errors}
\resizebox{\textwidth}{!}{
\begin{tabular}{llc}
\toprule
\textbf{Semantic Error} & \textbf{Explanation} & \textbf{Proportion (\%)} \\ 
\midrule
\textbf{DisobeyDescription} & Direct violation of semantic requirements explicitly stated in the task description. & \textbf{14.29} \\ 
\quad IncorrectPredicate & Incorrect or missing the declaration of predicates. & 6.59 \\ 
\quad IncorrectAction & Incorrect or missing the declaration of actions. & 7.69 \\ 
\midrule
\textbf{IncompleteModeling} & Incomplete world modeling compared to basic requirements. & \textbf{58.24} \\ 
\quad IncorrectPrecondition & The precondition of the action is deficient or incorrect. & 29.67 \\ 
\quad IncorrectEffect & The effect of the action is deficient or incorrect. & 28.57 \\ 
\midrule
\textbf{RedundantSpecifications} & Predicted domain includes superfluous preconditions or effects. & \textbf{17.58} \\ 
\quad RedundantPrecondition & Predicted domain includes superfluous preconditions. & 10.99 \\ 
\quad RedundantEffect & Predicted domain includes superfluous effects. & 6.59 \\ 
\midrule
\textbf{SurfaceDivergence} & Surface variations preserving semantic equivalence with ground truth. & \textbf{9.89} \\ 
\bottomrule
\end{tabular}
}
\end{table*}

The distribution and detailed explanation of each semantic error type are presented in Table~\ref{tab:semantic_errors}.


\section{More Experimental Results}

\subsection{Experimental Results with Concrete Description}
\label{app:concrete_results}

\begin{table*}[htbp]
\centering
% \small
\caption{Performance comparison of different LLMs on \benchmark using concrete domain description. EC\textsubscript{k} denotes the setting where models are allowed k correction attempts (EC\textsubscript{0}: zero-shot without correction, EC\textsubscript{3}: with 3 correction attempts).}
\label{tab:main-concrete}
\resizebox{\textwidth}{!}{%
\begin{tabular}{l|c|cc|cc|cc|cc|cc|cc}
\toprule
\multirow{2}{*}{\textbf{Model Family}} & \multirow{2}{*}{\textbf{Version}} & \multicolumn{2}{c|}{\textbf{\exec} $\uparrow$} & \multicolumn{2}{c|}{\textbf{\simmetric} $\uparrow$} & \multicolumn{2}{c|}{\textbf{\fonepred} $\uparrow$} & \multicolumn{2}{c|}{\textbf{\foneparam} $\uparrow$} & \multicolumn{2}{c|}{\textbf{\foneprecond} $\uparrow$} & \multicolumn{2}{c}{\textbf{\foneeff} $\uparrow$} \\
\cmidrule{3-14} & & \textbf{EC\textsubscript{0}} & \textbf{EC\textsubscript{3}} & \textbf{EC\textsubscript{0}} & \textbf{EC\textsubscript{3}} & \textbf{EC\textsubscript{0}} & \textbf{EC\textsubscript{3}} & \textbf{EC\textsubscript{0}} & \textbf{EC\textsubscript{3}} & \textbf{EC\textsubscript{0}} & \textbf{EC\textsubscript{3}} & \textbf{EC\textsubscript{0}} & \textbf{EC\textsubscript{3}} \\
\midrule
\multirow{1}{*}{\textsc{GPT-4}} & {\small\texttt{gpt-4o}} & 60.4 & 75.2 & 90.7 & 90.3 & 59.4 & 71.8 & 57.1 & 69.1 & 55.3 & 65.1 & 54.1 & 65.2 \\
\midrule
\multirow{1}{*}{\textsc{GPT-3.5}} & {\small\texttt{turbo-0125}} & 53.5 & 68.3 & 89.0 & 88.7 & 52.9 & 66.7 & 50.3 & 64.6 & 45.1 & 58.0 & 46.5 & 59.9 \\
\midrule
\textsc{Claude-3.5} & {\small\texttt{sonnet}} & 64.4 & 84.2 & 84.7 & 77.6 & 64.4 & 80.7 & 55.0 & 67.5 & 53.3 & 65.0 & 53.3 & 64.8\\
\midrule
\multirow{2}{*}{\textsc{LLaMA-2}} & {\small\texttt{7b-instruct}} & 0.0 & 0.0 & 48.4 & 32.4 & 0.0 & 0.0 & 0.0 & 0.0 & 0.0 & 0.0 & 0.0 & 0.0 \\
 & {\small\texttt{70b-instruct}} & 0.0 & 0.0 & 53.5 & 52.5 & 0.0 & 0.0 & 0.0 & 0.0 & 0.0 & 0.0 & 0.0 & 0.0 \\
\midrule
\multirow{2}{*}{\textsc{LLaMA-3.1}} & {\small\texttt{8b-instruct}} & 0.0 & 1.0 & 84.1 & 83.2 & 0.0 & 0.0 & 0.0 & 0.0 & 0.0 & 0.0 & 0.0 & 0.0 \\
 & {\small\texttt{70b-instruct}} & 1.0 & 1.0 & 89.7 & 85.4 & 1.0 & 1.0 & 1.0 & 1.0 & 1.0 & 1.0 & 1.0 & 1.0 \\
\midrule
\multirow{1}{*}{\textsc{DeepSeek}} & {\small\texttt{deepseek-v3}} & 58.4 & 80.2 & 90.1 & 89.3 & 58.1 & 76.4 & 56.2 & 73.5 & 53.4 & 66.0 & 53.5 & 67.6 \\ 
\bottomrule
\end{tabular}%
}
\end{table*}

% \subsection{Experimental Results by Requirements}

\end{document}


% This document was modified from the file originally made available by
% Pat Langley and Andrea Danyluk for ICML-2K. This version was created
% by Iain Murray in 2018, and modified by Alexandre Bouchard in
% 2019 and 2021 and by Csaba Szepesvari, Gang Niu and Sivan Sabato in 2022.
% Modified again in 2023 and 2024 by Sivan Sabato and Jonathan Scarlett.
% Previous contributors include Dan Roy, Lise Getoor and Tobias
% Scheffer, which was slightly modified from the 2010 version by
% Thorsten Joachims & Johannes Fuernkranz, slightly modified from the
% 2009 version by Kiri Wagstaff and Sam Roweis's 2008 version, which is
% slightly modified from Prasad Tadepalli's 2007 version which is a
% lightly changed version of the previous year's version by Andrew
% Moore, which was in turn edited from those of Kristian Kersting and
% Codrina Lauth. Alex Smola contributed to the algorithmic style files.

