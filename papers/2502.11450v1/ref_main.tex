%%%%%%%% ICML 2025 EXAMPLE LATEX SUBMISSION FILE %%%%%%%%%%%%%%%%%

\documentclass{article}

% Recommended, but optional, packages for figures and better typesetting:
\usepackage{microtype}
\usepackage{graphicx}
\usepackage{subfigure}
\usepackage{booktabs} % for professional tables

% hyperref makes hyperlinks in the resulting PDF.
% If your build breaks (sometimes temporarily if a hyperlink spans a page)
% please comment out the following usepackage line and replace
% \usepackage{icml2025} with \usepackage[nohyperref]{icml2025} above.
\usepackage{hyperref}


% Attempt to make hyperref and algorithmic work together better:
\newcommand{\theHalgorithm}{\arabic{algorithm}}

% Use the following line for the initial blind version submitted for review:
\usepackage{icml2025}
% \usepackage[nohyperref]{icml2025}

% If accepted, instead use the following line for the camera-ready submission:
% \usepackage[accepted]{icml2025}

% For theorems and such
\usepackage{amsmath}
\usepackage{amssymb}
\usepackage{mathtools}
\usepackage{amsthm}

% if you use cleveref..
\usepackage[capitalize,noabbrev]{cleveref}

%%%%%%%%%%%%%%%%%%%%%%%%%%%%%%%%
% THEOREMS
%%%%%%%%%%%%%%%%%%%%%%%%%%%%%%%%
\theoremstyle{plain}
\newtheorem{theorem}{Theorem}[section]
\newtheorem{proposition}[theorem]{Proposition}
\newtheorem{lemma}[theorem]{Lemma}
\newtheorem{corollary}[theorem]{Corollary}
\theoremstyle{definition}
\newtheorem{definition}[theorem]{Definition}
\newtheorem{assumption}[theorem]{Assumption}
\theoremstyle{remark}
\newtheorem{remark}[theorem]{Remark}

% Todonotes is useful during development; simply uncomment the next line
%    and comment out the line below the next line to turn off comments
%\usepackage[disable,textsize=tiny]{todonotes}
\usepackage[textsize=tiny]{todonotes}


% The \icmltitle you define below is probably too long as a header.
% Therefore, a short form for the running title is supplied here:
\icmltitlerunning{Submission and Formatting Instructions for ICML 2025}

\begin{document}

\twocolumn[
\icmltitle{Submission and Formatting Instructions for \\
           International Conference on Machine Learning (ICML 2025)}

% It is OKAY to include author information, even for blind
% submissions: the style file will automatically remove it for you
% unless you've provided the [accepted] option to the icml2025
% package.

% List of affiliations: The first argument should be a (short)
% identifier you will use later to specify author affiliations
% Academic affiliations should list Department, University, City, Region, Country
% Industry affiliations should list Company, City, Region, Country

% You can specify symbols, otherwise they are numbered in order.
% Ideally, you should not use this facility. Affiliations will be numbered
% in order of appearance and this is the preferred way.
\icmlsetsymbol{equal}{*}

\begin{icmlauthorlist}
\icmlauthor{Firstname1 Lastname1}{equal,yyy}
\icmlauthor{Firstname2 Lastname2}{equal,yyy,comp}
\icmlauthor{Firstname3 Lastname3}{comp}
\icmlauthor{Firstname4 Lastname4}{sch}
\icmlauthor{Firstname5 Lastname5}{yyy}
\icmlauthor{Firstname6 Lastname6}{sch,yyy,comp}
\icmlauthor{Firstname7 Lastname7}{comp}
%\icmlauthor{}{sch}
\icmlauthor{Firstname8 Lastname8}{sch}
\icmlauthor{Firstname8 Lastname8}{yyy,comp}
%\icmlauthor{}{sch}
%\icmlauthor{}{sch}
\end{icmlauthorlist}

\icmlaffiliation{yyy}{Department of XXX, University of YYY, Location, Country}
\icmlaffiliation{comp}{Company Name, Location, Country}
\icmlaffiliation{sch}{School of ZZZ, Institute of WWW, Location, Country}

\icmlcorrespondingauthor{Firstname1 Lastname1}{first1.last1@xxx.edu}
\icmlcorrespondingauthor{Firstname2 Lastname2}{first2.last2@www.uk}

% You may provide any keywords that you
% find helpful for describing your paper; these are used to populate
% the "keywords" metadata in the PDF but will not be shown in the document
\icmlkeywords{Machine Learning, ICML}

\vskip 0.3in
]

% this must go after the closing bracket ] following \twocolumn[ ...

% This command actually creates the footnote in the first column
% listing the affiliations and the copyright notice.
% The command takes one argument, which is text to display at the start of the footnote.
% The \icmlEqualContribution command is standard text for equal contribution.
% Remove it (just {}) if you do not need this facility.

%\printAffiliationsAndNotice{}  % leave blank if no need to mention equal contribution
\printAffiliationsAndNotice{\icmlEqualContribution} % otherwise use the standard text.

\begin{abstract}
  In this work, we present a novel technique for GPU-accelerated Boolean satisfiability (SAT) sampling. Unlike conventional sampling algorithms that directly operate on conjunctive normal form (CNF), our method transforms the logical constraints of SAT problems by factoring their CNF representations into simplified multi-level, multi-output Boolean functions. It then leverages gradient-based optimization to guide the search for a diverse set of valid solutions. Our method operates directly on the circuit structure of refactored SAT instances, reinterpreting the SAT problem as a supervised multi-output regression task. This differentiable technique enables independent bit-wise operations on each tensor element, allowing parallel execution of learning processes. As a result, we achieve GPU-accelerated sampling with significant runtime improvements ranging from $33.6\times$ to $523.6\times$ over state-of-the-art heuristic samplers. We demonstrate the superior performance of our sampling method through an extensive evaluation on $60$ instances from a public domain benchmark suite utilized in previous studies. 


  
  % Generating a wide range of diverse solutions to logical constraints is crucial in software and hardware testing, verification, and synthesis. These solutions can serve as inputs to test specific functionalities of a software program or as random stimuli in hardware modules. In software verification, techniques like fuzz testing and symbolic execution use this approach to identify bugs and vulnerabilities. In hardware verification, stimulus generation is particularly vital, forming the basis of constrained-random verification. While generating multiple solutions improves coverage and increases the chances of finding bugs, high-throughput sampling remains challenging, especially with complex constraints and refined coverage criteria. In this work, we present a novel technique that enables GPU-accelerated sampling, resulting in high-throughput generation of satisfying solutions to Boolean satisfiability (SAT) problems. Unlike conventional sampling algorithms that directly operate on conjunctive normal form (CNF), our method refines the logical constraints of SAT problems by transforming their CNF into simplified multi-level Boolean expressions. It then leverages gradient-based optimization to guide the search for a diverse set of valid solutions.
  % Our method specifically takes advantage of the circuit structure of refined SAT instances by using GD to learn valid solutions, reinterpreting the SAT problem as a supervised multi-output regression task. This differentiable technique enables independent bit-wise operations on each tensor element, allowing parallel execution of learning processes. As a result, we achieve GPU-accelerated sampling with significant runtime improvements ranging from $10\times$ to $1000\times$ over state-of-the-art heuristic samplers. Specifically, we demonstrate the superior performance of our sampling method through an extensive evaluation on $60$ instances from a public domain benchmark suite utilized in previous studies.

\end{abstract}

\begin{IEEEkeywords}
Boolean Satisfiability, Gradient Descent, Multi-level Circuits, Verification, and Testing.
\end{IEEEkeywords}
\section{Introduction}
\label{section:introduction}

% redirection is unique and important in VR
Virtual Reality (VR) systems enable users to embody virtual avatars by mirroring their physical movements and aligning their perspective with virtual avatars' in real time. 
As the head-mounted displays (HMDs) block direct visual access to the physical world, users primarily rely on visual feedback from the virtual environment and integrate it with proprioceptive cues to control the avatar’s movements and interact within the VR space.
Since human perception is heavily influenced by visual input~\cite{gibson1933adaptation}, 
VR systems have the unique capability to control users' perception of the virtual environment and avatars by manipulating the visual information presented to them.
Leveraging this, various redirection techniques have been proposed to enable novel VR interactions, 
such as redirecting users' walking paths~\cite{razzaque2005redirected, suma2012impossible, steinicke2009estimation},
modifying reaching movements~\cite{gonzalez2022model, azmandian2016haptic, cheng2017sparse, feick2021visuo},
and conveying haptic information through visual feedback to create pseudo-haptic effects~\cite{samad2019pseudo, dominjon2005influence, lecuyer2009simulating}.
Such redirection techniques enable these interactions by manipulating the alignment between users' physical movements and their virtual avatar's actions.

% % what is hand/arm redirection, motivation of study arm-offset
% \change{\yj{i don't understand the purpose of this paragraph}
% These illusion-based techniques provide users with unique experiences in virtual environments that differ from the physical world yet maintain an immersive experience. 
% A key example is hand redirection, which shifts the virtual hand’s position away from the real hand as the user moves to enhance ergonomics during interaction~\cite{feuchtner2018ownershift, wentzel2020improving} and improve interaction performance~\cite{montano2017erg, poupyrev1996go}. 
% To increase the realism of virtual movements and strengthen the user’s sense of embodiment, hand redirection techniques often incorporate a complete virtual arm or full body alongside the redirected virtual hand, using inverse kinematics~\cite{hartfill2021analysis, ponton2024stretch} or adjustments to the virtual arm's movement as well~\cite{li2022modeling, feick2024impact}.
% }

% noticeability, motivation of predicting a probability, not a classification
However, these redirection techniques are most effective when the manipulation remains undetected~\cite{gonzalez2017model, li2022modeling}. 
If the redirection becomes too large, the user may not mitigate the conflict between the visual sensory input (redirected virtual movement) and their proprioception (actual physical movement), potentially leading to a loss of embodiment with the virtual avatar and making it difficult for the user to accurately control virtual movements to complete interaction tasks~\cite{li2022modeling, wentzel2020improving, feuchtner2018ownershift}. 
While proprioception is not absolute, users only have a general sense of their physical movements and the likelihood that they notice the redirection is probabilistic. 
This probability of detecting the redirection is referred to as \textbf{noticeability}~\cite{li2022modeling, zenner2024beyond, zenner2023detectability} and is typically estimated based on the frequency with which users detect the manipulation across multiple trials.

% version B
% Prior research has explored factors influencing the noticeability of redirected motion, including the redirection's magnitude~\cite{wentzel2020improving, poupyrev1996go}, direction~\cite{li2022modeling, feuchtner2018ownershift}, and the visual characteristics of the virtual avatar~\cite{ogawa2020effect, feick2024impact}.
% While these factors focus on the avatars, the surrounding virtual environment can also influence the users' behavior and in turn affect the noticeability of redirection.
% One such prominent external influence is through the visual channel - the users' visual attention is constantly distracted by complex visual effects and events in practical VR scenarios.
% Although some prior studies have explored how to leverage user blindness caused by visual distractions to redirect users' virtual hand~\cite{zenner2023detectability}, there remains a gap in understanding how to quantify the noticeability of redirection under visual distractions.

% visual stimuli and gaze behavior
Prior research has explored factors influencing the noticeability of redirected motion, including the redirection's magnitude~\cite{wentzel2020improving, poupyrev1996go}, direction~\cite{li2022modeling, feuchtner2018ownershift}, and the visual characteristics of the virtual avatar~\cite{ogawa2020effect, feick2024impact}.
While these factors focus on the avatars, the surrounding virtual environment can also influence the users' behavior and in turn affect the noticeability of redirection.
This, however, remains underexplored.
One such prominent external influence is through the visual channel - the users' visual attention is constantly distracted by complex visual effects and events in practical VR scenarios.
We thus want to investigate how \textbf{visual stimuli in the virtual environment} affect the noticeability of redirection.
With this, we hope to complement existing works that focus on avatars by incorporating environmental visual influences to enable more accurate control over the noticeability of redirected motions in practical VR scenarios.
% However, in realistic VR applications, the virtual environment often contains complex visual effects beyond the virtual avatar itself. 
% We argue that these visual effects can \textbf{distract users’ visual attention and thus affect the noticeability of redirection offsets}, while current research has yet taken into account.
% For instance, in a VR boxing scenario, a user’s visual attention is likely focused on their opponent rather than on their virtual body, leading to a lower noticeability of redirection offsets on their virtual movements. 
% Conversely, when reaching for an object in the center of their field of view, the user’s attention is more concentrated on the virtual hand’s movement and position to ensure successful interaction, resulting in a higher noticeability of offsets.

Since each visual event is a complex choreography of many underlying factors (type of visual effect, location, duration, etc.), it is extremely difficult to quantify or parameterize visual stimuli.
Furthermore, individuals respond differently to even the same visual events.
Prior neuroscience studies revealed that factors like age, gender, and personality can influence how quickly someone reacts to visual events~\cite{gillon2024responses, gale1997human}. 
Therefore, aiming to model visual stimuli in a way that is generalizable and applicable to different stimuli and users, we propose to use users' \textbf{gaze behavior} as an indicator of how they respond to visual stimuli.
In this paper, we used various gaze behaviors, including gaze location, saccades~\cite{krejtz2018eye}, fixations~\cite{perkhofer2019using}, and the Index of Pupil Activity (IPA)~\cite{duchowski2018index}.
These behaviors indicate both where users are looking and their cognitive activity, as looking at something does not necessarily mean they are attending to it.
Our goal is to investigate how these gaze behaviors stimulated by various visual stimuli relate to the noticeability of redirection.
With this, we contribute a model that allows designers and content creators to adjust the redirection in real-time responding to dynamic visual events in VR.

To achieve this, we conducted user studies to collect users' noticeability of redirection under various visual stimuli.
To simulate realistic VR scenarios, we adopted a dual-task design in which the participants performed redirected movements while monitoring the visual stimuli.
Specifically, participants' primary task was to report if they noticed an offset between the avatar's movement and their own, while their secondary task was to monitor and report the visual stimuli.
As realistic virtual environments often contain complex visual effects, we started with simple and controlled visual stimulus to manage the influencing factors.

% first user study, confirmation study
% collect data under no visual stimuli, different basic visual stimuli
We first conducted a confirmation study (N=16) to test whether applying visual stimuli (opacity-based) actually affects their noticeability of redirection. 
The results showed that participants were significantly less likely to detect the redirection when visual stimuli was presented $(F_{(1,15)}=5.90,~p=0.03)$.
Furthermore, by analyzing the collected gaze data, results revealed a correlation between the proposed gaze behaviors and the noticeability results $(r=-0.43)$, confirming that the gaze behaviors could be leveraged to compute the noticeability.

% data collection study
We then conducted a data collection study to obtain more accurate noticeability results through repeated measurements to better model the relationship between visual stimuli-triggered gaze behaviors and noticeability of redirection.
With the collected data, we analyzed various numerical features from the gaze behaviors to identify the most effective ones. 
We tested combinations of these features to determine the most effective one for predicting noticeability under visual stimuli.
Using the selected features, our regression model achieved a mean squared error (MSE) of 0.011 through leave-one-user-out cross-validation. 
Furthermore, we developed both a binary and a three-class classification model to categorize noticeability, which achieved an accuracy of 91.74\% and 85.62\%, respectively.

% evaluation study
To evaluate the generalizability of the regression model, we conducted an evaluation study (N=24) to test whether the model could accurately predict noticeability with new visual stimuli (color- and scale-based animations).
Specifically, we evaluated whether the model's predictions aligned with participants' responses under these unseen stimuli.
The results showed that our model accurately estimated the noticeability, achieving mean squared errors (MSE) of 0.014 and 0.012 for the color- and scale-based visual stimili, respectively, compared to participants' responses.
Since the tested visual stimuli data were not included in the training, the results suggested that the extracted gaze behavior features capture a generalizable pattern and can effectively indicate the corresponding impact on the noticeability of redirection.

% application
Based on our model, we implemented an adaptive redirection technique and demonstrated it through two applications: adaptive VR action game and opportunistic rendering.
We conducted a proof-of-concept user study (N=8) to compare our adaptive redirection technique with a static redirection, evaluating the usability and benefits of our adaptive redirection technique.
The results indicated that participants experienced less physical demand and stronger sense of embodiment and agency when using the adaptive redirection technique. 
These results demonstrated the effectiveness and usability of our model.

In summary, we make the following contributions.
% 
\begin{itemize}
    \item 
    We propose to use users' gaze behavior as a medium to quantify how visual stimuli influences the noticebility of redirection. 
    Through two user studies, we confirm that visual stimuli significantly influences noticeability and identify key gaze behavior features that are closely related to this impact.
    \item 
    We build a regression model that takes the user's gaze behavioral data as input, then computes the noticeability of redirection.
    Through an evaluation study, we verify that our model can estimate the noticeability with new participants under unseen visual stimuli.
    These findings suggest that the extracted gaze behavior features effectively capture the influence of visual stimuli on noticeability and can generalize across different users and visual stimuli.
    \item 
    We develop an adaptive redirection technique based on our regression model and implement two applications with it.
    With a proof-of-concept study, we demonstrate the effectiveness and potential usability of our regression model on real-world use cases.

\end{itemize}

% \delete{
% Virtual Reality (VR) allows the user to embody a virtual avatar by mirroring their physical movements through the avatar.
% As the user's visual access to the physical world is blocked in tasks involving motion control, they heavily rely on the visual representation of the avatar's motions to guide their proprioception.
% Similar to real-world experiences, the user is able to resolve conflicts between different sensory inputs (e.g., vision and motor control) through multisensory integration, which is essential for mitigating the sensory noise that commonly arises.
% However, it also enables unique manipulations in VR, as the system can intentionally modify the avatar's movements in relation to the user's motions to achieve specific functional outcomes,
% for example, 
% % the manipulations on the avatar's movements can 
% enabling novel interaction techniques of redirected walking~\cite{razzaque2005redirected}, redirected reaching~\cite{gonzalez2022model}, and pseudo haptics~\cite{samad2019pseudo}.
% With small adjustments to the avatar's movements, the user can maintain their sense of embodiment, due to their ability to resolve the perceptual differences.
% % However, a large mismatch between the user and avatar's movements can result in the user losing their sense of embodiment, due to an inability to resolve the perceptual differences.
% }

% \delete{
% However, multisensory integration can break when the manipulation is so intense that the user is aware of the existence of the motion offset and no longer maintains the sense of embodiment.
% Prior research studied the intensity threshold of the offset applied on the avatar's hand, beyond which the embodiment will break~\cite{li2022modeling}. 
% Studies also investigated the user's sensitivity to the offsets over time~\cite{kohm2022sensitivity}.
% Based on the findings, we argue that one crucial factor that affects to what extent the user notices the offset (i.e., \textit{noticeability}) that remains under-explored is whether the user directs their visual attention towards or away from the virtual avatar.
% Related work (e.g., Mise-unseen~\cite{marwecki2019mise}) has showcased applications where adjustments in the environment can be made in an unnoticeable manner when they happen in the area out of the user's visual field.
% We hypothesize that directing the user's visual attention away from the avatar's body, while still partially keeping the avatar within the user's field-of-view, can reduce the noticeability of the offset.
% Therefore, we conduct two user studies and implement a regression model to systematically investigate this effect.
% }

% \delete{
% In the first user study (N = 16), we test whether drawing the user's visual attention away from their body impacts the possibility of them noticing an offset that we apply to their arm motion in VR.
% We adopt a dual-task design to enable the alteration of the user's visual attention and a yes/no paradigm to measure the noticeability of motion offset. 
% The primary task for the user is to perform an arm motion and report when they perceive an offset between the avatar's virtual arm and their real arm.
% In the secondary task, we randomly render a visual animation of a ball turning from transparent to red and becoming transparent again and ask them to monitor and report when it appears.
% We control the strength of the visual stimuli by changing the duration and location of the animation.
% % By changing the time duration and location of the visual animation, we control the strengths of attraction to the users.
% As a result, we found significant differences in the noticeability of the offsets $(F_{(1,15)}=5.90,~p=0.03)$ between conditions with and without visual stimuli.
% Based on further analysis, we also identified the behavioral patterns of the user's gaze (including pupil dilation, fixations, and saccades) to be correlated with the noticeability results $(r=-0.43)$ and they may potentially serve as indicators of noticeability.
% }

% \delete{
% To further investigate how visual attention influences the noticeability, we conduct a data collection study (N = 12) and build a regression model based on the data.
% The regression model is able to calculate the noticeability of the offset applied on the user's arm under various visual stimuli based on their gaze behaviors.
% Our leave-one-out cross-validation results show that the proposed method was able to achieve a mean-squared error (MSE) of 0.012 in the probability regression task.
% }

% \delete{
% To verify the feasibility and extendability of the regression model, we conduct an evaluation study where we test new visual animations based on adjustments on scale and color and invite 24 new participants to attend the study.
% Results show that the proposed method can accurately estimate the noticeability with an MSE of 0.014 and 0.012 in the conditions of the color- and scale-based visual effects.
% Since these animations were not included in the dataset that the regression model was built on, the study demonstrates that the gaze behavioral features we extracted from the data capture a generalizable pattern of the user's visual attention and can indicate the corresponding impact on the noticeability of the offset.
% }

% \delete{
% Finally, we demonstrate applications that can benefit from the noticeability prediction model, including adaptive motion offsets and opportunistic rendering, considering the user's visual attention. 
% We conclude with discussions of our work's limitations and future research directions.
% }

% \delete{
% In summary, we make the following contributions.
% }
% % 
% \begin{itemize}
%     \item 
%     \delete{
%     We quantify the effects of the user's visual attention directed away by stimuli on their noticeability of an offset applied to the avatar's arm motion with respect to the user's physical arm. 
%     Through two user studies, we identified gaze behavioral features that are indicative of the changes in noticeability.
%     }
%     \item 
%     \delete{We build a regression model that takes the user's gaze behavioral data and the offset applied to the arm motion as input, then computes the probability of the user noticing the offset.
%     Through an evaluation study, we verified that the model needs no information about the source attracting the user's visual attention and can be generalizable in different scenarios.
%     }
%     \item 
%     \delete{We demonstrate two applications that potentially benefit from the regression model, including adaptive motion offsets and opportunistic rendering.
%     }

% \end{itemize}

\begin{comment}
However, users will lose the sense of embodiment to the virtual avatars if they notice the offset between the virtual and physical movements.
To address this, researchers have been exploring the noticing threshold of offsets with various magnitudes and proposing various redirection techniques that maintain the sense of embodiment~\cite{}.

However, when users embody virtual avatars to explore virtual environments, they encounter various visual effects and content that can attract their attention~\cite{}.
During this, the user may notice an offset when he observes the virtual movement carefully while ignoring it when the virtual contents attract his attention from the movements.
Therefore, static offset thresholds are not appropriate in dynamic scenarios.

Past research has proposed dynamic mapping techniques that adapted to users' state, such as hand moving speed~\cite{frees2007prism} or ergonomically comfortable poses~\cite{montano2017erg}, but not considering the influence of virtual content.
More specifically, PRISM~\cite{frees2007prism} proposed adjusting the C/D ratio with a non-linear mapping according to users' hand moving speed, but it might not be optimal for various virtual scenarios.
While Erg-O~\cite{montano2017erg} redirected users' virtual hands according to the virtual target's relative position to reduce physical fatigue, neglecting the change of virtual environments. 

Therefore, how to design redirection techniques in various scenarios with different visual attractions remains unknown.
To address this, we investigate how visual attention affects the noticing probability of movement offsets.
Based on our experiments, we implement a computational model that automatically computes the noticing probability of offsets under certain visual attractions.
VR application designers and developers can easily leverage our model to design redirection techniques maintaining the sense of embodiment adapt to the user's visual attention.
We implement a dynamic redirection technique with our model and demonstrate that it effectively reduces the target reaching time without reducing the sense of embodiment compared to static redirection techniques.

% Need to be refined
This paper offers the following contributions.
\begin{itemize}
    \item We investigate how visual attractions affect the noticing probability of redirection offsets.
    \item We construct a computational model to predict the noticing probability of an offset with a given visual background.
    \item We implement a dynamic redirection technique adapting to the visual background. We evaluate the technique and develop three applications to demonstrate the benefits. 
\end{itemize}



First, we conducted a controlled experiment to understand how users perceived the movement offset while subjected to various distractions.
Since hand redirection is one of the most frequently used redirections in VR interactions, we focused on the dynamic arm movements and manually added angular offsets to the' elbow joint~\cite{li2022modeling, gonzalez2022model, zenner2019estimating}. 
We employed flashing spheres in the user's field of view as distractions to attract users' visual attention.
Participants were instructed to report the appearing location of the spheres while simultaneously performing the arm movements and reporting if they perceived an offset during the movement. 
(\zhipeng{Add the results of data collection. Analyze the influence of the distance between the gaze map and the offset.}
We measured the visual attraction's magnitude with the gaze distribution on it.
Results showed that stronger distractions made it harder for users to notice the offset.)
\zhipeng{Need to rewrite. Not sure to use gaze distribution or a metric obtained from the visual content.}
Secondly, we constructed a computational model to predict the noticing probability of offsets with given visual content.
We analyzed the data from the user studies to measure the influence of visual attractions on the noticing probability of offsets.
We built a statistical model to predict the offset's noticing probability with a given visual content.
Based on the model, we implement a dynamic redirection technique to adjust the redirection offset adapted to the user's current field of view.
We evaluated the technique in a target selection task compared to no hand redirection and static hand redirection.
\zhipeng{Add the results of the evaluation.}
Results showed that the dynamic hand redirection technique significantly reduced the target selection time with similar accuracy and a comparable sense of embodiment.
Finally, we implemented three applications to demonstrate the potential benefits of the visual attention adapted dynamic redirection technique.
\end{comment}

% This one modifies arm length, not redirection
% \citeauthor{mcintosh2020iteratively} proposed an adaptation method to iteratively change the virtual avatar arm's length based on the primary tasks' performance~\cite{mcintosh2020iteratively}.



% \zhipeng{TO ADD: what is redirection}
% Redirection enables novel interactions in Virtual Reality, including redirected walking, haptic redirection, and pseudo haptics by introducing an offset to users' movement.
% \zhipeng{TO ADD: extend this sentence}
% The price of this is that users' immersiveness and embodiment in VR can be compromised when they notice the offset and perceive the virtual movement not as theirs~\cite{}.
% \zhipeng{TO ADD: extend this sentence, elaborate how the virtual environment attracts users' attention}
% Meanwhile, the visual content in the virtual environment is abundant and consistently captures users' attention, making it harder to notice the offset~\cite{}.
% While previous studies explored the noticing threshold of the offsets and optimized the redirection techniques to maintain the sense of embodiment~\cite{}, the influence of visual content on the probability of perceiving offsets remains unknown.  
% Therefore, we propose to investigate how users perceive the redirection offset when they are facing various visual attractions.


% We conducted a user study to understand how users notice the shift with visual attractions.
% We used a color-changing ball to attract the user's attention while instructing users to perform different poses with their arms and observe it meanwhile.
% \zhipeng{(Which one should be the primary task? Observe the ball should be the primary one, but if the primary task is too simple, users might allocate more attention on the secondary task and this makes the secondary task primary.)}
% \zhipeng{(We need a good and reasonable dual-task design in which users care about both their pose and the visual content, at least in the evaluation study. And we need to be able to control the visual content's magnitude and saliency maybe?)}
% We controlled the shift magnitude and direction, the user's pose, the ball's size, and the color range.
% We set the ball's color-changing interval as the independent factor.
% We collect the user's response to each shift and the color-changing times.
% Based on the collected data, we constructed a statistical model to describe the influence of visual attraction on the noticing probability.
% \zhipeng{(Are we actually controlling the attention allocation? How do we measure the attracting effect? We need uniform metrics, otherwise it is also hard for others to use our knowledge.)}
% \zhipeng{(Try to use eye gaze? The eye gaze distribution in the last five seconds to decide the attention allocation? Basically constructing a model with eye gaze distribution and noticing probability. But the user's head is moving, so the eye gaze distribution is not aligned well with the current view.)}

% \zhipeng{Saliency and EMD}
% \zhipeng{Gaze is more than just a point: Rethinking visual attention
% analysis using peripheral vision-based gaze mapping}

% Evaluation study(ideal case): based on the visual content, adjusting the redirection magnitude dynamically.

% \zhipeng{(The risk is our model's effect is trivial.)}

% Applications:
% Playing Lego while watching demo videos, we can accelerate the reaching process of bricks, and forbid the redirection during the manipulation.

% Beat saber again: but not make a lot of sense? Difficult game has complicated visual effects, while allows larger shift, but do not need large shift with high difficulty




\section{Electronic Submission}
\label{submission}

Submission to ICML 2025 will be entirely electronic, via a web site
(not email). Information about the submission process and \LaTeX\ templates
are available on the conference web site at:
\begin{center}
\textbf{\texttt{http://icml.cc/}}
\end{center}

The guidelines below will be enforced for initial submissions and
camera-ready copies. Here is a brief summary:
\begin{itemize}
\item Submissions must be in PDF\@. 
\item If your paper has appendices, submit the appendix together with the main body and the references \textbf{as a single file}. Reviewers will not look for appendices as a separate PDF file. So if you submit such an extra file, reviewers will very likely miss it.
\item Page limit: The main body of the paper has to be fitted to 8 pages, excluding references and appendices; the space for the latter two is not limited in pages, but the total file size may not exceed 10MB. For the final version of the paper, authors can add one extra page to the main body.
\item \textbf{Do not include author information or acknowledgements} in your
    initial submission.
\item Your paper should be in \textbf{10 point Times font}.
\item Make sure your PDF file only uses Type-1 fonts.
\item Place figure captions \emph{under} the figure (and omit titles from inside
    the graphic file itself). Place table captions \emph{over} the table.
\item References must include page numbers whenever possible and be as complete
    as possible. Place multiple citations in chronological order.
\item Do not alter the style template; in particular, do not compress the paper
    format by reducing the vertical spaces.
\item Keep your abstract brief and self-contained, one paragraph and roughly
    4--6 sentences. Gross violations will require correction at the
    camera-ready phase. The title should have content words capitalized.
\end{itemize}

\subsection{Submitting Papers}

\textbf{Anonymous Submission:} ICML uses double-blind review: no identifying
author information may appear on the title page or in the paper
itself. \cref{author info} gives further details.

\medskip

Authors must provide their manuscripts in \textbf{PDF} format.
Furthermore, please make sure that files contain only embedded Type-1 fonts
(e.g.,~using the program \texttt{pdffonts} in linux or using
File/DocumentProperties/Fonts in Acrobat). Other fonts (like Type-3)
might come from graphics files imported into the document.

Authors using \textbf{Word} must convert their document to PDF\@. Most
of the latest versions of Word have the facility to do this
automatically. Submissions will not be accepted in Word format or any
format other than PDF\@. Really. We're not joking. Don't send Word.

Those who use \textbf{\LaTeX} should avoid including Type-3 fonts.
Those using \texttt{latex} and \texttt{dvips} may need the following
two commands:

{\footnotesize
\begin{verbatim}
dvips -Ppdf -tletter -G0 -o paper.ps paper.dvi
ps2pdf paper.ps
\end{verbatim}}
It is a zero following the ``-G'', which tells dvips to use
the config.pdf file. Newer \TeX\ distributions don't always need this
option.

Using \texttt{pdflatex} rather than \texttt{latex}, often gives better
results. This program avoids the Type-3 font problem, and supports more
advanced features in the \texttt{microtype} package.

\textbf{Graphics files} should be a reasonable size, and included from
an appropriate format. Use vector formats (.eps/.pdf) for plots,
lossless bitmap formats (.png) for raster graphics with sharp lines, and
jpeg for photo-like images.

The style file uses the \texttt{hyperref} package to make clickable
links in documents. If this causes problems for you, add
\texttt{nohyperref} as one of the options to the \texttt{icml2025}
usepackage statement.


\subsection{Submitting Final Camera-Ready Copy}

The final versions of papers accepted for publication should follow the
same format and naming convention as initial submissions, except that
author information (names and affiliations) should be given. See
\cref{final author} for formatting instructions.

The footnote, ``Preliminary work. Under review by the International
Conference on Machine Learning (ICML). Do not distribute.'' must be
modified to ``\textit{Proceedings of the
$\mathit{42}^{nd}$ International Conference on Machine Learning},
Vancouver, Canada, PMLR 267, 2025.
Copyright 2025 by the author(s).''

For those using the \textbf{\LaTeX} style file, this change (and others) is
handled automatically by simply changing
$\mathtt{\backslash usepackage\{icml2025\}}$ to
$$\mathtt{\backslash usepackage[accepted]\{icml2025\}}$$
Authors using \textbf{Word} must edit the
footnote on the first page of the document themselves.

Camera-ready copies should have the title of the paper as running head
on each page except the first one. The running title consists of a
single line centered above a horizontal rule which is $1$~point thick.
The running head should be centered, bold and in $9$~point type. The
rule should be $10$~points above the main text. For those using the
\textbf{\LaTeX} style file, the original title is automatically set as running
head using the \texttt{fancyhdr} package which is included in the ICML
2025 style file package. In case that the original title exceeds the
size restrictions, a shorter form can be supplied by using

\verb|\icmltitlerunning{...}|

just before $\mathtt{\backslash begin\{document\}}$.
Authors using \textbf{Word} must edit the header of the document themselves.

\section{Format of the Paper}

All submissions must follow the specified format.

\subsection{Dimensions}




The text of the paper should be formatted in two columns, with an
overall width of 6.75~inches, height of 9.0~inches, and 0.25~inches
between the columns. The left margin should be 0.75~inches and the top
margin 1.0~inch (2.54~cm). The right and bottom margins will depend on
whether you print on US letter or A4 paper, but all final versions
must be produced for US letter size.
Do not write anything on the margins.

The paper body should be set in 10~point type with a vertical spacing
of 11~points. Please use Times typeface throughout the text.

\subsection{Title}

The paper title should be set in 14~point bold type and centered
between two horizontal rules that are 1~point thick, with 1.0~inch
between the top rule and the top edge of the page. Capitalize the
first letter of content words and put the rest of the title in lower
case.

\subsection{Author Information for Submission}
\label{author info}

ICML uses double-blind review, so author information must not appear. If
you are using \LaTeX\/ and the \texttt{icml2025.sty} file, use
\verb+\icmlauthor{...}+ to specify authors and \verb+\icmlaffiliation{...}+ to specify affiliations. (Read the TeX code used to produce this document for an example usage.) The author information
will not be printed unless \texttt{accepted} is passed as an argument to the
style file.
Submissions that include the author information will not
be reviewed.

\subsubsection{Self-Citations}

If you are citing published papers for which you are an author, refer
to yourself in the third person. In particular, do not use phrases
that reveal your identity (e.g., ``in previous work \cite{langley00}, we
have shown \ldots'').

Do not anonymize citations in the reference section. The only exception are manuscripts that are
not yet published (e.g., under submission). If you choose to refer to
such unpublished manuscripts \cite{anonymous}, anonymized copies have
to be submitted
as Supplementary Material via OpenReview\@. However, keep in mind that an ICML
paper should be self contained and should contain sufficient detail
for the reviewers to evaluate the work. In particular, reviewers are
not required to look at the Supplementary Material when writing their
review (they are not required to look at more than the first $8$ pages of the submitted document).

\subsubsection{Camera-Ready Author Information}
\label{final author}

If a paper is accepted, a final camera-ready copy must be prepared.
%
For camera-ready papers, author information should start 0.3~inches below the
bottom rule surrounding the title. The authors' names should appear in 10~point
bold type, in a row, separated by white space, and centered. Author names should
not be broken across lines. Unbolded superscripted numbers, starting 1, should
be used to refer to affiliations.

Affiliations should be numbered in the order of appearance. A single footnote
block of text should be used to list all the affiliations. (Academic
affiliations should list Department, University, City, State/Region, Country.
Similarly for industrial affiliations.)

Each distinct affiliations should be listed once. If an author has multiple
affiliations, multiple superscripts should be placed after the name, separated
by thin spaces. If the authors would like to highlight equal contribution by
multiple first authors, those authors should have an asterisk placed after their
name in superscript, and the term ``\textsuperscript{*}Equal contribution"
should be placed in the footnote block ahead of the list of affiliations. A
list of corresponding authors and their emails (in the format Full Name
\textless{}email@domain.com\textgreater{}) can follow the list of affiliations.
Ideally only one or two names should be listed.

A sample file with author names is included in the ICML2025 style file
package. Turn on the \texttt{[accepted]} option to the stylefile to
see the names rendered. All of the guidelines above are implemented
by the \LaTeX\ style file.

\subsection{Abstract}

The paper abstract should begin in the left column, 0.4~inches below the final
address. The heading `Abstract' should be centered, bold, and in 11~point type.
The abstract body should use 10~point type, with a vertical spacing of
11~points, and should be indented 0.25~inches more than normal on left-hand and
right-hand margins. Insert 0.4~inches of blank space after the body. Keep your
abstract brief and self-contained, limiting it to one paragraph and roughly 4--6
sentences. Gross violations will require correction at the camera-ready phase.

\subsection{Partitioning the Text}

You should organize your paper into sections and paragraphs to help
readers place a structure on the material and understand its
contributions.

\subsubsection{Sections and Subsections}

Section headings should be numbered, flush left, and set in 11~pt bold
type with the content words capitalized. Leave 0.25~inches of space
before the heading and 0.15~inches after the heading.

Similarly, subsection headings should be numbered, flush left, and set
in 10~pt bold type with the content words capitalized. Leave
0.2~inches of space before the heading and 0.13~inches afterward.

Finally, subsubsection headings should be numbered, flush left, and
set in 10~pt small caps with the content words capitalized. Leave
0.18~inches of space before the heading and 0.1~inches after the
heading.

Please use no more than three levels of headings.

\subsubsection{Paragraphs and Footnotes}

Within each section or subsection, you should further partition the
paper into paragraphs. Do not indent the first line of a given
paragraph, but insert a blank line between succeeding ones.

You can use footnotes\footnote{Footnotes
should be complete sentences.} to provide readers with additional
information about a topic without interrupting the flow of the paper.
Indicate footnotes with a number in the text where the point is most
relevant. Place the footnote in 9~point type at the bottom of the
column in which it appears. Precede the first footnote in a column
with a horizontal rule of 0.8~inches.\footnote{Multiple footnotes can
appear in each column, in the same order as they appear in the text,
but spread them across columns and pages if possible.}

\begin{figure}[ht]
\vskip 0.2in
\begin{center}
\centerline{\includegraphics[width=\columnwidth]{icml_numpapers}}
\caption{Historical locations and number of accepted papers for International
Machine Learning Conferences (ICML 1993 -- ICML 2008) and International
Workshops on Machine Learning (ML 1988 -- ML 1992). At the time this figure was
produced, the number of accepted papers for ICML 2008 was unknown and instead
estimated.}
\label{icml-historical}
\end{center}
\vskip -0.2in
\end{figure}

\subsection{Figures}

You may want to include figures in the paper to illustrate
your approach and results. Such artwork should be centered,
legible, and separated from the text. Lines should be dark and at
least 0.5~points thick for purposes of reproduction, and text should
not appear on a gray background.

Label all distinct components of each figure. If the figure takes the
form of a graph, then give a name for each axis and include a legend
that briefly describes each curve. Do not include a title inside the
figure; instead, the caption should serve this function.

Number figures sequentially, placing the figure number and caption
\emph{after} the graphics, with at least 0.1~inches of space before
the caption and 0.1~inches after it, as in
\cref{icml-historical}. The figure caption should be set in
9~point type and centered unless it runs two or more lines, in which
case it should be flush left. You may float figures to the top or
bottom of a column, and you may set wide figures across both columns
(use the environment \texttt{figure*} in \LaTeX). Always place
two-column figures at the top or bottom of the page.

\subsection{Algorithms}

If you are using \LaTeX, please use the ``algorithm'' and ``algorithmic''
environments to format pseudocode. These require
the corresponding stylefiles, algorithm.sty and
algorithmic.sty, which are supplied with this package.
\cref{alg:example} shows an example.

\begin{algorithm}[tb]
   \caption{Bubble Sort}
   \label{alg:example}
\begin{algorithmic}
   \STATE {\bfseries Input:} data $x_i$, size $m$
   \REPEAT
   \STATE Initialize $noChange = true$.
   \FOR{$i=1$ {\bfseries to} $m-1$}
   \IF{$x_i > x_{i+1}$}
   \STATE Swap $x_i$ and $x_{i+1}$
   \STATE $noChange = false$
   \ENDIF
   \ENDFOR
   \UNTIL{$noChange$ is $true$}
\end{algorithmic}
\end{algorithm}

\subsection{Tables}

You may also want to include tables that summarize material. Like
figures, these should be centered, legible, and numbered consecutively.
However, place the title \emph{above} the table with at least
0.1~inches of space before the title and the same after it, as in
\cref{sample-table}. The table title should be set in 9~point
type and centered unless it runs two or more lines, in which case it
should be flush left.

% Note use of \abovespace and \belowspace to get reasonable spacing
% above and below tabular lines.

\begin{table}[t]
\caption{Classification accuracies for naive Bayes and flexible
Bayes on various data sets.}
\label{sample-table}
\vskip 0.15in
\begin{center}
\begin{small}
\begin{sc}
\begin{tabular}{lcccr}
\toprule
Data set & Naive & Flexible & Better? \\
\midrule
Breast    & 95.9$\pm$ 0.2& 96.7$\pm$ 0.2& $\surd$ \\
Cleveland & 83.3$\pm$ 0.6& 80.0$\pm$ 0.6& $\times$\\
Glass2    & 61.9$\pm$ 1.4& 83.8$\pm$ 0.7& $\surd$ \\
Credit    & 74.8$\pm$ 0.5& 78.3$\pm$ 0.6&         \\
Horse     & 73.3$\pm$ 0.9& 69.7$\pm$ 1.0& $\times$\\
Meta      & 67.1$\pm$ 0.6& 76.5$\pm$ 0.5& $\surd$ \\
Pima      & 75.1$\pm$ 0.6& 73.9$\pm$ 0.5&         \\
Vehicle   & 44.9$\pm$ 0.6& 61.5$\pm$ 0.4& $\surd$ \\
\bottomrule
\end{tabular}
\end{sc}
\end{small}
\end{center}
\vskip -0.1in
\end{table}

Tables contain textual material, whereas figures contain graphical material.
Specify the contents of each row and column in the table's topmost
row. Again, you may float tables to a column's top or bottom, and set
wide tables across both columns. Place two-column tables at the
top or bottom of the page.

\subsection{Theorems and such}
The preferred way is to number definitions, propositions, lemmas, etc. consecutively, within sections, as shown below.
\begin{definition}
\label{def:inj}
A function $f:X \to Y$ is injective if for any $x,y\in X$ different, $f(x)\ne f(y)$.
\end{definition}
Using \cref{def:inj} we immediate get the following result:
\begin{proposition}
If $f$ is injective mapping a set $X$ to another set $Y$, 
the cardinality of $Y$ is at least as large as that of $X$
\end{proposition}
\begin{proof} 
Left as an exercise to the reader. 
\end{proof}
\cref{lem:usefullemma} stated next will prove to be useful.
\begin{lemma}
\label{lem:usefullemma}
For any $f:X \to Y$ and $g:Y\to Z$ injective functions, $f \circ g$ is injective.
\end{lemma}
\begin{theorem}
\label{thm:bigtheorem}
If $f:X\to Y$ is bijective, the cardinality of $X$ and $Y$ are the same.
\end{theorem}
An easy corollary of \cref{thm:bigtheorem} is the following:
\begin{corollary}
If $f:X\to Y$ is bijective, 
the cardinality of $X$ is at least as large as that of $Y$.
\end{corollary}
\begin{assumption}
The set $X$ is finite.
\label{ass:xfinite}
\end{assumption}
\begin{remark}
According to some, it is only the finite case (cf. \cref{ass:xfinite}) that is interesting.
\end{remark}
%restatable

\subsection{Citations and References}

Please use APA reference format regardless of your formatter
or word processor. If you rely on the \LaTeX\/ bibliographic
facility, use \texttt{natbib.sty} and \texttt{icml2025.bst}
included in the style-file package to obtain this format.

Citations within the text should include the authors' last names and
year. If the authors' names are included in the sentence, place only
the year in parentheses, for example when referencing Arthur Samuel's
pioneering work \yrcite{Samuel59}. Otherwise place the entire
reference in parentheses with the authors and year separated by a
comma \cite{Samuel59}. List multiple references separated by
semicolons \cite{kearns89,Samuel59,mitchell80}. Use the `et~al.'
construct only for citations with three or more authors or after
listing all authors to a publication in an earlier reference \cite{MachineLearningI}.

Authors should cite their own work in the third person
in the initial version of their paper submitted for blind review.
Please refer to \cref{author info} for detailed instructions on how to
cite your own papers.

Use an unnumbered first-level section heading for the references, and use a
hanging indent style, with the first line of the reference flush against the
left margin and subsequent lines indented by 10 points. The references at the
end of this document give examples for journal articles \cite{Samuel59},
conference publications \cite{langley00}, book chapters \cite{Newell81}, books
\cite{DudaHart2nd}, edited volumes \cite{MachineLearningI}, technical reports
\cite{mitchell80}, and dissertations \cite{kearns89}.

Alphabetize references by the surnames of the first authors, with
single author entries preceding multiple author entries. Order
references for the same authors by year of publication, with the
earliest first. Make sure that each reference includes all relevant
information (e.g., page numbers).

Please put some effort into making references complete, presentable, and
consistent, e.g. use the actual current name of authors.
If using bibtex, please protect capital letters of names and
abbreviations in titles, for example, use \{B\}ayesian or \{L\}ipschitz
in your .bib file.

\section*{Accessibility}
Authors are kindly asked to make their submissions as accessible as possible for everyone including people with disabilities and sensory or neurological differences.
Tips of how to achieve this and what to pay attention to will be provided on the conference website \url{http://icml.cc/}.

\section*{Software and Data}

If a paper is accepted, we strongly encourage the publication of software and data with the
camera-ready version of the paper whenever appropriate. This can be
done by including a URL in the camera-ready copy. However, \textbf{do not}
include URLs that reveal your institution or identity in your
submission for review. Instead, provide an anonymous URL or upload
the material as ``Supplementary Material'' into the OpenReview reviewing
system. Note that reviewers are not required to look at this material
when writing their review.

% Acknowledgements should only appear in the accepted version.
\section*{Acknowledgements}

\textbf{Do not} include acknowledgements in the initial version of
the paper submitted for blind review.

If a paper is accepted, the final camera-ready version can (and
usually should) include acknowledgements.  Such acknowledgements
should be placed at the end of the section, in an unnumbered section
that does not count towards the paper page limit. Typically, this will 
include thanks to reviewers who gave useful comments, to colleagues 
who contributed to the ideas, and to funding agencies and corporate 
sponsors that provided financial support.

\section*{Impact Statement}

Authors are \textbf{required} to include a statement of the potential 
broader impact of their work, including its ethical aspects and future 
societal consequences. This statement should be in an unnumbered 
section at the end of the paper (co-located with Acknowledgements -- 
the two may appear in either order, but both must be before References), 
and does not count toward the paper page limit. In many cases, where 
the ethical impacts and expected societal implications are those that 
are well established when advancing the field of Machine Learning, 
substantial discussion is not required, and a simple statement such 
as the following will suffice:

``This paper presents work whose goal is to advance the field of 
Machine Learning. There are many potential societal consequences 
of our work, none which we feel must be specifically highlighted here.''

The above statement can be used verbatim in such cases, but we 
encourage authors to think about whether there is content which does 
warrant further discussion, as this statement will be apparent if the 
paper is later flagged for ethics review.


% In the unusual situation where you want a paper to appear in the
% references without citing it in the main text, use \nocite
\nocite{langley00}

\bibliography{references}
\bibliographystyle{icml2025}


%%%%%%%%%%%%%%%%%%%%%%%%%%%%%%%%%%%%%%%%%%%%%%%%%%%%%%%%%%%%%%%%%%%%%%%%%%%%%%%
%%%%%%%%%%%%%%%%%%%%%%%%%%%%%%%%%%%%%%%%%%%%%%%%%%%%%%%%%%%%%%%%%%%%%%%%%%%%%%%
% APPENDIX
%%%%%%%%%%%%%%%%%%%%%%%%%%%%%%%%%%%%%%%%%%%%%%%%%%%%%%%%%%%%%%%%%%%%%%%%%%%%%%%
%%%%%%%%%%%%%%%%%%%%%%%%%%%%%%%%%%%%%%%%%%%%%%%%%%%%%%%%%%%%%%%%%%%%%%%%%%%%%%%
\newpage
\appendix
\onecolumn
\section{You \emph{can} have an appendix here.}

You can have as much text here as you want. The main body must be at most $8$ pages long.
For the final version, one more page can be added.
If you want, you can use an appendix like this one.  

The $\mathtt{\backslash onecolumn}$ command above can be kept in place if you prefer a one-column appendix, or can be removed if you prefer a two-column appendix.  Apart from this possible change, the style (font size, spacing, margins, page numbering, etc.) should be kept the same as the main body.
%%%%%%%%%%%%%%%%%%%%%%%%%%%%%%%%%%%%%%%%%%%%%%%%%%%%%%%%%%%%%%%%%%%%%%%%%%%%%%%
%%%%%%%%%%%%%%%%%%%%%%%%%%%%%%%%%%%%%%%%%%%%%%%%%%%%%%%%%%%%%%%%%%%%%%%%%%%%%%%


\end{document}


% This document was modified from the file originally made available by
% Pat Langley and Andrea Danyluk for ICML-2K. This version was created
% by Iain Murray in 2018, and modified by Alexandre Bouchard in
% 2019 and 2021 and by Csaba Szepesvari, Gang Niu and Sivan Sabato in 2022.
% Modified again in 2023 and 2024 by Sivan Sabato and Jonathan Scarlett.
% Previous contributors include Dan Roy, Lise Getoor and Tobias
% Scheffer, which was slightly modified from the 2010 version by
% Thorsten Joachims & Johannes Fuernkranz, slightly modified from the
% 2009 version by Kiri Wagstaff and Sam Roweis's 2008 version, which is
% slightly modified from Prasad Tadepalli's 2007 version which is a
% lightly changed version of the previous year's version by Andrew
% Moore, which was in turn edited from those of Kristian Kersting and
% Codrina Lauth. Alex Smola contributed to the algorithmic style files.
