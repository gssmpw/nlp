

\section{Task Details}
\label{apd:task_detail}

This appendix provides details for each task, including the number of examples, evaluation metrics, and configurable hyperparameters. The context length is fixed at 4k for almost all tasks, apart from Stateful Processing, where the context is determined by number of operation steps and set to 200 for quantity state and 100 for set state, which maps to around 1.5k context tokens.

Here is an example of number of examples calculation String search (with word):  5 (query depth) *  2 (labels)  * 5 (samples per parameter setting) = 50. 


\begin{longtable}{lp{6.5cm}cc}
\caption{Task Overview with Hyperparameters, Number of Examples, and Evaluation Metrics\label{tab:task_details}} \\

\toprule
\textbf{Task Name} & \textbf{Hyperparameters} & \textbf{\# of Examples} & \textbf{Metric} \\ 
\midrule
\endfirsthead

\toprule
\textbf{Task Name} & \textbf{Hyperparameters} & \textbf{\# of Examples} & \textbf{Metric} \\ 
\midrule
\endhead

\midrule
\multicolumn{4}{r}{\textbf{Continued on next page}} \\ 
\midrule
\endfoot

\bottomrule
\endlastfoot

\multicolumn{4}{l}{\textbf{Search}} \\ 
String search (word) & query depth = [0, 0.25, 0.5, 0.75, 1], label = [positive, negative], samples = 5 & 50 & exact\_match \\
String search (sequence) & sequence length = [8, 16, 32, 64], label = [positive, negative], samples = 10 & 80 & exact\_match \\ 
Key-value search & query depth = [0, 0.25, 0.5, 0.75, 1], samples = 10 & 50 & exact\_match \\ 
Batch search & batch size = [4, 8, 16, 32], samples = 5 & 20 & rouge-L\_recall \\
\midrule
\multicolumn{4}{r}{\textbf{Number of Entries for Category: 200}} \\ 
\midrule

\multicolumn{4}{l}{\textbf{Recall and Edit}} \\ 
Snapshot (words) & samples = 10 & 10 & rouge-L \\
Replace all & density = [0.2, 0.4, 0.6, 0.8], y = [random word, null], samples = 5 & 40 & rouge-L \\
Overwrite positions & nth = [2, 3, 4], y = [random word, null], samples = 5 & 30 & rouge-L \\ 
Snapshot (numbers) & samples = 10 & 10 & rouge-L \\ 
Functional updates & function type = [add (3), subtract (1), multiply (2)], samples = 5 & 15 & rouge-L \\ 
\midrule
\multicolumn{4}{r}{\textbf{Number of Entries for Category: 105}} \\ 
\midrule

\multicolumn{4}{l}{\textbf{Match and Compare}} \\ 
Compare positions & query 1 depth = [0, 0.25, 0.5, 0.75, 1], query 2 depth = [0, 0.25, 0.5, 0.75, 1], samples = 3 & 75 & exact\_match \\ 
Find duplicates & repetition = [2, 4, 8, 16, 32], samples = 5 & 25 & exact\_match \\ 
Count & repetition = [2, 4, 8, 16, 32], samples = 5 & 25 & exact\_match \\ 
Check association & n attribute = [2, 4, 8, 16, 32], label = [positive, negative], samples = 5 & 50 & exact\_match \\
\midrule
\multicolumn{4}{r}{\textbf{Number of Entries for Category: 175}} \\ 
\midrule

\multicolumn{4}{l}{\textbf{Spot the Differences}} \\ 
Compare two lists & num different words = [1, 5, 10, 20], chosen list = [first, second], samples = 10 & 80 & rouge-L\_recall \\ 
Identify the odd group & words per group = [25, 50, 75, 100], percentage of difference = [0, 0.25, 0.5], samples = 5 & 60 & exact\_match \\ 
Patch the difference & pattern length = [2, 15, 30],  cut off depth = [0, 0.5, 1], nth = [1, 3, 6], samples = 5 & 120\footnote{For \textit{Patch the difference} task with pattern length 2, there is only two cut off percentage options; therefore the total number of data points is 120 instead of 135.} & exact\_match \\ 
\midrule
\multicolumn{4}{r}{\textbf{Number of Entries for Category: 260}} \\ 
\midrule

\multicolumn{4}{l}{\textbf{Compute on Sets and Lists}} \\ 
Group membership & number of groups = [4, 8, 16, 32], query depth = [0, 0.25, 0.5, 0.75, 1], samples = 5 & 100 & exact\_match \\ 
Group association & number of groups = [4, 8, 16, 32], label = [positive, negative], samples = 5  & 40 & exact\_match \\ 
Group association (alternating) & number of groups = [2, 4, 8, 16, 32], number of turns = 10, label = [positive, negative], sample = 5 & 50 & exact\_match \\
Iterate & number of groups = [4, 8, 16, 32], samples = 5 & 20 & rouge-L \\ 
\midrule
\multicolumn{4}{r}{\textbf{Number of Entries for Category: 210}} \\ 
\midrule

\multicolumn{4}{l}{\textbf{Stateful Processing}} \\ 
Set state & number of steps = 100, set size = [5, 10, 15, 20], samples = 10 & 40 & jaccard\_similarity \\ 
Quantity state & number of steps = 200, samples = 10 & 10 & exact\_match \\ 
\midrule
\multicolumn{4}{r}{\textbf{Number of Entries for Category: 50}} \\ 
\midrule

\multicolumn{4}{l}{\textbf{Composite}} \\ 
Composite edits of data blocks & number of blocks = [2, 4, 8, 16, 32], number of turns = 10, samples = 5 & 50 & rouge-L \\
Theory of mind & number of steps = 100, number of agents = [2, 3, 4], samples = [10, 20] & 60 & jaccard\_similarity \\ 
\midrule
\multicolumn{4}{r}{\textbf{Number of Entries for Category: 110}} \\ 
\midrule

\multicolumn{4}{r}{\textbf{Total Number of Entries: 1110}} \\ 
\end{longtable}



\section{Evaluation Metrics}
\label{apd:eval}
In this appendix section, we provide details about the evaluation metrics we have used in the tests.

\begin{itemize} \item Exact Match: The exact match accuracy measures whether the generated answer exactly matches the reference answer. It is computed as follows: 
\[
\text{Exact Match} = 
\begin{cases} 
1 & \text{if } \text{reference\_answer} = \text{generated\_answer}, \\
0 & \text{otherwise}.
\end{cases}
\]
\item ROUGE-L / ROUGE-L-recall: ROUGE (Recall-Oriented Understudy for Gisting Evaluation) \cite{lin-2004-rouge} measures the verbatim overlap between the reference and the generated answers. ROUGE-L specifically looks for the longest common subsequence (LCS) between the two texts, which reflects the structure of the text and the longest sequence of matching words. ROUGE-L recall focuses on the ability of the model to recall the content from the reference answer, and it emphasizes matching the longest subsequences.

ROUGE-L-recall can be defined as:
\[
\text{ROUGE-L-recall} = \frac{LCS(\text{generated\_answer}, \text{reference\_answer})}{\text{length of reference\_answer}}
\]

ROUGE-L is computed as the F1-score, which combines both precision and recall to provide a more balanced measure of overlap.

\item Jaccard Similarity: Jaccard similarity measures the overlap between two sets by comparing the intersection and union of the sets. It is computed as:
\[
\text{Jaccard Similarity} = \frac{|A \cap B|}{|A \cup B|}
\]
where \(A\) and \(B\) are sets representing the elements in the generated and reference answers, respectively. This metric is used for tasks involving set-based comparisons or when the goal is to measure the similarity between two sets of elements (e.g., word sets).
\end{itemize}
