\section{Memory Tests for LLMs}

\subsection{Level 1: Look up}
This level assesses the ability of the model to look for information in the context. There are three types of tasks in this level: presence check, paired associate retrieval, and needle in a hay stack test. In this set of tasks, the model is assessed on the ability to look up and use a single point of information from varying depths of context. 

TODO: add human test reference

\subsubsection{Presence check}
In the presence check task, the model is presented with a list of items (words or numbers), the task is to identify if a query item is present in the context. The query item can be a positive example (an item sampled from the context) or negative (an item that is not present). 

\begin{tcolorbox}[colback=red!5!white,colframe=red!75!black,title=Example prompt]
Context: emerald, hook, drawing, marco, arbitrary, delivery, swimming, zope, ...

\tcblower
Instruction: Determine if the specified word is present in the context. Answer with ``yes" or ``no".
Question: Is ``arbitrary" present in the context?
Answer:
\end{tcolorbox}
    
\subsubsection{Paired associate retrieval}:
In the paired associate retrieval task, the model is presented with pairs of items (words or numbers), the task is to output the corresponding second item when queried with the first item. In this test, the model is explicitly given the information of where to look in the context, i.e. with first query item. 

\begin{tcolorbox}[colback=red!5!white,colframe=red!75!black,title=Example prompt]
Context: textbooks: obtained, our: webpage, sprint: bye, yrs: bali, mary: leaving, footwear: um, puzzles: opt, ...
\tcblower
Instruction: Given the pairs of words in the format of `word\_1: word\_2' in the context, return the second word corresponding to the given first word. For the first word `mary', the second word is:
\end{tcolorbox}

\subsubsection{Needle in a haystack}

The needle in a haystack task has been adopted by many previous work to assess the ability of the model to leverage context \cite{}. Similar to the other tasks in this category, the standard needle in a haystack task assesses the ability of the model to answer a question using a single piece of information injected in the context. 

\subsection{Level 2: Verbatim Recall}
This level assesses the ability of the model to recall the context. At this level, the model is instructed to  repeat the context as it is, without any modification. This is the most simple form of a recall test, and it does not require the model to understand the context or manipulate the context in any form. In this test, the context is comprised of lists of words or numbers.

\begin{tcolorbox}[colback=red!5!white,colframe=red!75!black,title=Example prompt]
Context: 573, 849, 999, 329, 966, 209, 970, 777, ...
\tcblower
Instruction: Repeat the previous context exactly as it is, without making any additions or deletions.
\end{tcolorbox}



\subsection{Level 3: Recall with Manipulation}
This level further assesses the ability of the model to recall the context while applying simple manipulations to the items in the context. At this level, the manipulation is applied to each item of the context independently, and it does not require inter-item information to be able to perform the manipulation, apart from keeping the order in the output.

\subsubsection{Basic Computation}
In this task, the model is presented with a list of numbers, and instead of directly copying the context, the model is instructed to apply a simple arithmetic computation to the numbers. 

\begin{tcolorbox}[colback=red!5!white,colframe=red!75!black,title=Example prompt]
Context: 573, 849, 999, 329, 966, 209, 970, 777, 964, 144, ...
\tcblower
Instruction: Add 7 to every number in the previous context.
\end{tcolorbox}

\subsubsection{Replace and Skip}
In this task, the model is instructed to replace or skip items in the context of lists of words or numbers. There are multiple variants in this task: (1) Replace/skip specified item: The model is instructed to skip a specified item, or replace it with another specified item (2) Replace/skip nth item: The model is instructed to skip every nth item or replace it with a specified item. This variant introduces a flavor of counting into the task, which also challenges the language model's ability to count. (3) Replace nth letter: The model is instructed to replace nth letter of every word with a specified letter. This particular variant is designed to stress test the tokenization mechanism of the language models. 

\begin{tcolorbox}[colback=red!5!white,colframe=red!75!black,title=Example prompt]
Context: distinction, late, activated, apple, egyptian, apple, hwy, rating, ...
\tcblower
Instruction: Repeat the previous context and substitute the word ``apple" with ``kiwi" each time it appears.
\end{tcolorbox}

\subsubsection{Reverse}
In this task, the model is instructed to output the items in the context in a reverse order. 

\begin{tcolorbox}[colback=red!5!white,colframe=red!75!black,title=Example prompt]
Context: emerald, hook, drawing, marco, arbitrary, delivery, ...
\tcblower
Instruction: Repeat the words in the previous context in reverse order.
\end{tcolorbox}



\subsubsection{Compute Expressions}

In this task, the model is presented with a sequence of simple arithmetic expressions, including addition, subtraction and multiplication. It is then instructed to calculate the results of the expressions.

\begin{tcolorbox}[colback=red!5!white,colframe=red!75!black,title=Example prompt]
Context: 85 + 81, 98 - 39, 49 + 2, 45 - 78, 10 * 10, ...
\tcblower
Instruction: Calculate the results of the expressions in the previous context. Output the results in the same order as the expressions.
\end{tcolorbox}


\subsection{Level 4: Combining and Linking}
This level assesses the model's ability to combine and link information in the context. TODO: rephrase the title.

\subsubsection{Detect Repetition}
In this task, the model is presented with a list of items (words and numbers) and a single item has been repeated a number of times. The model is asked to detect which item has been repeated.

\begin{tcolorbox}[colback=red!5!white,colframe=red!75!black,title=Example prompt]
Context: face, kaleidoscope, bbc, xl, kaleidoscope, produced, curve, kaleidoscope, kaleidoscope, ...
\tcblower
Instruction: A word was repeated multiple times in the context. Identify the word that was repeated.
\end{tcolorbox}

\subsubsection{Count Repetition}

In this task, the model is presented with a list of items with a repeated item. It is instructed to count how many times a specified item has appeared in the context.

\begin{tcolorbox}[colback=red!5!white,colframe=red!75!black,title=Example prompt]
Context: 2579, 12345, 6515, 4134, 7311, 12345, 5195, 6197, 12345, ...
\tcblower
Instruction: Count the number of times the number ``12345" appears in the context.
\end{tcolorbox}

\subsubsection{Relative Position}

In this task, the model is presented with a list of items, and instead of checking the presence of a single item, at this level it is instructed to decide if the relative position of two items is correct.

\begin{tcolorbox}[colback=red!5!white,colframe=red!75!black,title=Example prompt]
Context: 2579, 5344, 6515, 4134, 7311, 7409, 5195, 6197, 4843, 5539, ...
\tcblower
Instruction: Given the list of numbers in the context, determine the relative positions of two numbers. Does the number ``7409" appear after ``6515"? Answer `yes' or `no'.
\end{tcolorbox}

\subsubsection{Sequential Paired Associate Retrieval}

Similar to the paired associate retrieval task, the model is presented with a context of paired items. It is then asked to output a sequence of associated items given a sequence of first items.

\begin{tcolorbox}[colback=red!5!white,colframe=red!75!black,title=Example prompt]
Context: cultures: rt, textbooks: obtained, our: webpage, sprint: bye, ...
\tcblower
Instruction: Given the pairs of words in the format of `word\_1: word\_2' in the context, return the second words that correspond to the given first words. For the words 'sprint, textbooks, puzzles', the corresponding second words are:\end{tcolorbox}

\subsubsection{Sort}
In this task, the model is instructed to sort the items in the context in an alphabetical or numerical order.

\begin{tcolorbox}[colback=red!5!white,colframe=red!75!black,title=Example prompt]
Context: 2579, 5344, 6515, 4134, 7311, 7409, 5195, ...
\tcblower
Instruction: Sort the numbers in the previous context in ascending order.\end{tcolorbox}

\subsection{Level 5: Multi-step manipulation}
The previous levels mainly focuses on the model's ability to use or manipulate the context with a straightforward single step manipulation. At this level, we increase the difficulty of context manipulation by combining some of the previous manipulation tasks to form a multi-step manipulation task.

\subsubsection{Detect and Count Repetition}

In this task, the model is asked to detect the repeated item in the context and count the number of times it has repeated.

\subsubsection{Detect Repetition and Replace}

In this task, the model is asked to detect the repeated item in the context and replace it with another item.

\subsubsection{Estimate Expression}

This task is a combination of compute expression and presence check. The model is presented with a context of simple arithmetic expressions, and it is asked to decide if any of the expressions yeilds a particular result.

\subsection{Level 6: Partitioning and Theory of Mind}