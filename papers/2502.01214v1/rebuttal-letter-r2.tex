\documentclass[a4paper,10pt]{letter}
\usepackage[utf8]{inputenc}
\usepackage[english]{babel}
\usepackage{xcolor}
%\usepackage{xr}
%\externaldocument[R-]{main}

%\usepackage{xc}
%\externalcitedocument[C-]{main}


% Some useful macros
\newcommand{\dflys}{Dragonflies}
\newcommand{\dfly}{Dragonfly}
\newcommand{\iba}{InfiniBand}
\newcommand{\ib}{IB}



\signature{The authors}
\hyphenation{tra-ffic}

\begin{document}
% If you want headings on subsequent pages,
% remove the ``%'' on the next line:
% \pagestyle{headings}

\clearpage
\thispagestyle{empty}

\begin{letter}
\address{German~Maglione-Mathey\\ Jesus~Escudero-Sahuquillo\\ Pedro~Javier~Garcia\\Francisco~J.~Quiles\\Eitan Zahavi\\}

May 26, 2020\\

Dear JPDC reviewers and editors, \\

The current submission titled ``Leveraging InfiniBand Controller to Configure Deadlock-Free Routing Engines
for Dragonflies'' is a minor revision
of the submission with Manuscript ID: JPDC\_2018\_378 that was originally submitted by September 19, 2018, and later resubmitted after a major revision by December 25, 2019.
In this letter we explain in detail how we have addressed the concerns raised by the editor and reviewers,
also describing the main modifications and the new content included in the paper.
We are very grateful to the editor and reviewers for their suggestions, which have been very helpful to improve the paper.
Indeed, we think that this new version of the paper satisfactorily addresses the concerns raised by both the editor and  reviewers.

In the following, we provide specific explanations and answers to the reviewers comments.

(Please note C = reviewer/editor comment, A = authors' reply)

{\bf Reviewer \#1}

\fcolorbox{white}{lightgray}{\parbox{\textwidth}{%
\color{black}%
C- ``The authors do not need to reference other articles in the conclusion of their paper. I suggest that the authors rewrite the conclusion of the paper concisely and coherently in which their main motivation of the research and the principal contributions of their paper can be easily understood.''}}

A- We completely agree with the reviewer. We have rewritten most of the conclusion section in order to explain better the main motivation and the contributions of this paper. In addition, we have removed all the references from that section.


\fcolorbox{white}{lightgray}{\parbox{\textwidth}{%
\color{black}%
C- ``I also insist on the addition of the definition of “circular dependency” to the manuscript if it can lead to a better understanding of the work.''}}

A-  We agree with the reviewer. Moreover, in the previous version of the paper we used both ``circular dependencies'' and ``cyclic dependencies'' to refer actually to the same concept, which may be confusing. In order to solve all this, in the new version of the paper we mention only ``cyclic dependencies'', that we have defined as a concept strongly related to deadlocks, whose definition we have also added. We think that with these changes the paper is easier to understand and more self-contained.


\clearpage

{\bf Reviewer \#2}

\fcolorbox{white}{lightgray}{\parbox{\textwidth}{%
\color{black}%
C- ``Some graphs may need more runs of simulation because the shape of the graph in some cases cannot be analyzed. For example in:
\begin{itemize}
    \item Figure 16(a), the throughput results of ‘lash’ algorithm.
    \item Figure 16(b) , the throughput results of ‘dla’ algorithm.
    \item Figure 16(d) , the throughput results of ‘lash’ algorithm in 0.7 load.
    \item Figure 19 (a) , the throughput results of ‘lash’ algorithm.
\end{itemize}
In my opinion, adding more simulation runs may help in achieving more analyzable results.''}}

A- The reviewer is right, so we have increased the number of runs per point in each mentioned simulation.
Where each simulation point is the average of five runs, each one using a different random seed.
We hope that these graphs can be analyzed now more easily.



\closing{Thank you for your consideration,}


\end{letter}
\end{document}

