\documentclass[10pt,journal,compsoc]{IEEEtran}
\usepackage{graphicx}

\begin{document}

\begin{IEEEbiography}[{\includegraphics[width=1in,height=1.25in,clip,keepaspectratio]{german-maglione.png}}]
{German Maglione-Mathey} received the BS and MS degrees in Computer Science from the
University of Castilla-La Mancha (UCLM), Spain, in 2015 and 2016 respectively. He
began his research career in 2015 as a PhD Student at the University of Castilla-La Mancha in Spain,
when he was recruited by the Computer Architecture Department of that University.
His research interests include High Performance Computing interconnects and Data Center Networks
and all the strategies related to improve them, especially network topologies, routing algorithms and congestion management.
\end{IEEEbiography}


\begin{IEEEbiography}[{\includegraphics[width=1in,height=1.25in,clip,keepaspectratio]{jesus-escudero-sauquillo.pdf}}]
{Jesus Escudero-Sahuquillo} received the MS and PhD degrees in Computer Science from the University of Castilla-La Mancha (UCLM), Spain, in 2008 and 2011, respectively. His research interests include high-performance computing and Big-Data services for cluster and datacenters, interconnection networks and all the strategies related to improve them, such as network topologies, routing algorithms, congestion management, and power saving. He has published more than 30 papers in national and
international peer-reviewed conferences and journals. In 2006 he joined the Computer Systems Department (DSI), UCLM, Spain. He performed several pre- and post-doc research stays in Simula Research Labs (Norway) and Heidelberg University (Gemany). In 2014 he moved to the industry and worked for Oracle Corporation (Norway), as a PhD Senior Engineer. In 2015 he moved to the Technical University of Valencia (Spain), as a PostDoc research assistant granted with a national-competitive grant "Juan de La Cierva". In 2016 he joined again the DSI, UCLM (Spain), with a 5-years PostDoc position funded by the UCLM research program and the European Commission (FSE funds). He has participated in several research projects funded by the European Commission and the Spanish Government. He has served as program committee, guest editor and reviewer in several conferences, such as ICPP, CCGrid, HoTI or EuroPar, and journals, such as TPDS, JPDC, CCPE or JSC. He is co-organizer of the IEEE International Workshop on High-Performance Interconnection Networks in the Exascale and Big-Data Era (HiPINIEB).
\end{IEEEbiography}

\begin{IEEEbiography}[{\includegraphics[width=1in,height=1.25in,clip,keepaspectratio]{pedro-garcia.png}}]
{Pedro J. Garcia} received a degree in communication engineering from the
Technical University of Valencia, Spain, in 1996, and the PhD degree in computer
science from the University of Castilla-La Mancha (UCLM), Spain, in 2006.
In 1999, he joined the Computing Systems Department (DSI), UCLM, Spain, where he
is currently an assistant professor of computer architecture and technology. His
main research interests are the design and implementation of strategies to
improve several aspects of high-performance interconnection networks, especially
congestion management schemes and routing algorithms. He has published more than
50 refereed papers in ranked journals and conferences. He has guided two doctoral
thesis and is guiding currently three more. He has been the coordinator of three
research projects supported respectively by the Spanish Government and by the
Government of Castilla-La Mancha. He has been also the coordinator of four
Research \& Development Agreements between UCLM and different companies. In addition, he has participated in other (more than 30) research projects, supported by the
European Commission and the Spanish Government. He has served as organizer
committee member and program committee member in several international
conferences and workshops, such as ICPP, HotI, CCGrid, ISC, HiPINEB. He has been
also a guest editor of several journals.
\end{IEEEbiography}

\begin{IEEEbiography}[{\includegraphics[width=1in,height=1.25in,clip,keepaspectratio]{francisco-quiles.png}}]
{Francisco J. Quiles} is a Full Professor of Computer Architecture and Technology
at the Computing Systems Department of UCLM. His research interests include:
high-performance interconnection networks for multiprocessor systems and
clusters, parallel algorithms for video compression and video transmission.
He has served as Program Committee member in several conferences. He has
published over 200 papers in international journals and conferences and
participated in 68 research projects supported by the NFS, European Commission,
the Spanish Government and Research \& Development Agreements with different
companies. Also, he has guided 9 doctoral theses.
\end{IEEEbiography}


\begin{IEEEbiography}[{\includegraphics[width=1in,height=1.25in,clip,keepaspectratio]{eitan-zahavi.png}}]
{Eitan Zahavi} manages the Mellanox end-to-end performance architecture group which
focuses on features that improve the overall system performance for both Ethernet
and InfiniBand, lossy and lossless. We also study Optical Data Center networks.
Example fields of research are Application performance, Congestion Control,
Adaptive Routing, Tenants Isolation, and Topologies. The group employs large system
simulation and lab experiments to validate our hypothesis and test new features
implementations.
\end{IEEEbiography}

\end{document}


