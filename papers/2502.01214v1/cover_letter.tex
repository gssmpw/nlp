\documentclass[a4paper,10pt]{letter}
\usepackage[utf8]{inputenc}

\signature{The authors}

\begin{document}
% If you want headings on subsequent pages,
% remove the ``%'' on the next line:
% \pagestyle{headings}

\clearpage
\thispagestyle{empty}

\begin{letter}
\address{German Maglione-Mathey\\ Jesus Escudero-Sahuquillo\\ Pedro Javier Garcia\\ Francisco J. Quiles\\ Eitan Zahavi}

September 13, 2018\\

Dear JPDC Editor, \\

The paper titled ``Leveraging InfiniBand Controller to Configure Deadlock-Free Routing Engines for Dragonflies''
that we are submitting to the ``Journal on Parallel and Distributed Computing'' is a completely original work.
In this paper, we propose a straightforward method to implement the deterministic, minimal-path routing
algorithm proposed originally by Kim and Dally for Dragonfly networks as a new routing engine in the InfiniBand control software.
Also, we analyze the requirements, pros and cons of all the routing engines
suitable for Dragonfly networks that are currently available in OpenSM, including the one proposed in the paper. We consider the impact
of several switch features and constraints in their implementation and performance.
That said, and bearing in mind the novelty requirements for new publications, in the following points we describe in more detail the main contributions of this paper:

\begin{itemize}
\item The method we propose in this paper to implement the routing algorithm is a completely new approach to configure for that purpose the control software stack of InfiniBand-based Dragonfly networks.

\item This method uses Virtual Lanes (VLs) as escape ways so that no circular dependencies among channels appear in the network. As far as we know, this is the first InfiniBand routing engine that uses the ``VL shifting'' technique (i.e. packets may change their assigned VL along their path) proposed for Dragonfly networks in order to prevent deadlocks.

\item We explain in depth the difficulties that present the implementation of the routing algorithms that are the base of our proposal.

\item We analyze the requirements, pros and cons, of all the routing engines suitable for Dragonfly networks and available in OpenSM, including the one proposed in the paper. We consider different buffer sizes and the use (or not) of a Virtual Output Queuing (VOQ) architecture, in order to measure their impact on the routing engines implementation and performance.

\item We also cover the implementation details of the proposed routing for Dragonflies as a routing engine in the open-source implementation of the InfiniBand subnet manager (OpenSM).

\item Finally, we evaluate the implementation of this routing engine, in comparison with existing ones, based on both simulation experiments and execution of real applications and benchmarks in a real InfiniBand-based cluster.


\end{itemize}

\closing{Thank you for your consideration,}

% enclosure listing
%\encl{}

\end{letter}
\end{document}
