\documentclass[a4paper,10pt]{letter}
\usepackage[utf8]{inputenc}
\usepackage[english]{babel}
\usepackage{xcolor}
%\usepackage{xr}
%\externaldocument[R-]{main}

%\usepackage{xc}
%\externalcitedocument[C-]{main}


% Some useful macros
\newcommand{\dflys}{Dragonflies}
\newcommand{\dfly}{Dragonfly}
\newcommand{\iba}{InfiniBand}
\newcommand{\ib}{IB}



\signature{The authors}
\hyphenation{tra-ffic}

\begin{document}
% If you want headings on subsequent pages,
% remove the ``%'' on the next line:
% \pagestyle{headings}

\clearpage
\thispagestyle{empty}

\begin{letter}
\address{German~Maglione-Mathey\\ Jesus~Escudero-Sahuquillo\\ Pedro~Javier~Garcia\\Francisco~J.~Quiles\\Eitan Zahavi\\}

December 25, 2019\\

Dear JPDC reviewers and editors, \\

The current submission titled ``Leveraging InfiniBand Controller to Configure Deadlock-Free Routing Engines
for Dragonflies'' is a major revision
of the previous submission with Manuscript ID: JPDC\_2018\_378 that was submitted in September 19, 2018.
In this letter we explain in detail how we have addressed each concern raised by the editor and reviewers,
also describing the main modifications and the new content included in the paper.
We are very grateful to the editor and reviewers for their suggestions, which have been very helpful to improve the paper.
Indeed, we think that this new version of the paper satisfactorily addresses the concerns raised by both the editor and  reviewers.

In the following, we provide specific explanations and answers to the reviewers comments.

(Please note C = reviewer/editor comment, A = authors' reply)

{\bf Reviewer \#1}

\fcolorbox{white}{lightgray}{\parbox{\textwidth}{%
\color{black}%
C- ``1- The simulation results are nice, but why did not the authors investigate and compare the
performance of the proposed work under other traffic patterns (e.g. bursty, Incast and etc.),
in addition to the conducted random scenarios in the simulations?''}}

A- We completely agree with the reviewer that more simulations were needed. Hence, in this major revision We have added new simulation results, specifically obtained with two synthetic traffic patterns commonly used to evaluate the interconnection networks of HPC systems, but not considered in the previous version of the paper.

\fcolorbox{white}{lightgray}{\parbox{\textwidth}{%
\color{black}%
C- ``2- Authors did not mention the weaknesses of their approach. I am sure there is a number
of trade-offs involved in deciding to employ your proposal instead of the existing works. It
would be interesting to discuss them.''}}

A- The reviewer points to an interesting question that requires some clarifications. First of all, note that our actual proposal is an implementation within the standard \iba{} specification of the MIN routing algorithm proposed in [1] for \dflys{}.
In that sense, we should not explain the weaknesses and trade-offs of the \dfly{} minimal routing algorithm, as they are already discussed in the corresponding paper [1]. On the contrary, we should focus on the weaknesses and trade-offs of our \iba{} implementation, mainly derived from the restrictions imposed by the \iba{} specification. Although some comment in that sense was included in the previous version, we have added deeper explanations in order to make the weaknesses and trade-offs of our proposal much clearer.

\textbf{References}:

[1]
J.~Kim, W.~J. Dally, S.~Scott, D.~Abts, Technology-{Driven},
{Highly}-{Scalable} {Dragonfly} {Topology}, in: 2008 {International}
{Symposium} on {Computer} {Architecture}, 2008, pp. 77--88.
  


\fcolorbox{white}{lightgray}{\parbox{\textwidth}{%
\color{black}%
C- ``3- It would be also helpful to examine the load distribution on various network resources,
such as switches and links, in order to investigate the load-balancing issue in your
proposal.''}}

A- As we have mentioned above, our proposal is focused actually on the implementation side of the \dfly{} routing algorithm, while the load distribution is determined by the algorithm itself.
However, we agree with the reviewer that a discussion in that sense may be interesting for many paper readers.
For that reason, we have included more insights about the original algorithm in Section 2.1.1, including the load distribution.

\fcolorbox{white}{lightgray}{\parbox{\textwidth}{%
\color{black}%
C- ``4- If all the packets are assumed to have the same SL (SL 0), then what is the role of SL in
SL2VL mapping?''}}

A- The reviewer's comment is very pertinent, as it points to a sentence in the previous version of the paper that was confusing. Indeed, this sentence suggested that a specific SL had to be used, while actually our criteria to map packet to VLs is independent of the packet's SL.
Specifically, regardless of the packet's SL, a packet is mapped to VL 1 when it traverses a local channel after traversing a global one, otherwise that packet being mapped to VL 0.
We have rewritten this part of the text in order to explain better all this and so to avoid misunderstandings.

\fcolorbox{white}{lightgray}{\parbox{\textwidth}{%
\color{black}%
C- ``5- Does the proposed routing engine perform efficiently on the other variations of Dragonfly
network? Or its good performance is just limited to the fully-connected Dragonfly
architectures?''}}

A- As mentioned in the paper, our proposal is
restricted to the fully-connected \dfly{} pattern, both in the inter-group and the intra-group network, due to the limitations imposed by the \iba{} specification (see Section 3). Nevertheless, we have made an effort to state this more clearly in the text of the new version of the paper.


\fcolorbox{white}{lightgray}{\parbox{\textwidth}{%
\color{black}%
C- ``6- The concept of ``circular dependency among channels/ buffers'' might not be clear to all. I
suggest explaining the meaning in Motivation section.''}}

A- The reviewer is right, as the term ``channels/buffers'' in that sentence was very confusing. We have corrected and improved this explanation in the paper.

\clearpage

{\bf Reviewer \#2}

\fcolorbox{white}{lightgray}{\parbox{\textwidth}{%
\color{black}%
C- ``The main novelty of the manuscript is not clear, the Dragonfly paper proposed by Kim and
Dally includes the simulation results of a Dragonfly topology and related routing algorithm,
so can I say that the novelty of this manuscript is only integration to InfiniBand network
controller (OpenSM)?''}}

A- The reviewer is right, in the sense that our actual proposal is an implementation of the \dfly{} minimal
routing algorithm in the \iba{} OpenSM. However, note that, due to the limitations imposed by the\iba{} specification, this implementation is not trivial at all.
We have made an effort to state this much more clearly in the new version of the paper. Moreover, note that our experiments focus on comparing the implemented algorithm with the routing engines currently implemented in the \iba{} OpensM, including comparative experiments performed in a real \iba{}-based cluster. As far as we know, this comparison had never been done before, which increases the degree of novelty of the paper, in our opinion.   

\fcolorbox{white}{lightgray}{\parbox{\textwidth}{%
\color{black}%
C- ``Moreover, it will be better if the experimental results are based on the known
interconnection networks parameters, for example, average network latency and
throughput that present the performance of the proposed idea.''}}

A- We agree with the reviewer on the need of clarifying the metrics of the experimental results. In that sense, note that each benchmark, such as \textit{Graph500}, \textit{HPCC} and \textit{NAMD}, reports its  own  metric. One of the reported metrics is bandwidth, but actually it means throughput. In the previous version of the paper we referred to bandwidth in the simulation results just to keep the same notation as in the benchmark results. However, we agree that throughput is a clearer notation. Therefore, we have modified the Y-axis label of the figures previously referring to bandwidth, so that they refer now to throughput instead.
On the other hand, we have chosen to include only throughput results from the simulation experiments due to space constrains, in order not to make Section 5 longer. 

\fcolorbox{white}{lightgray}{\parbox{\textwidth}{%
\color{black}%
C- ``Simulating only by uniform traffic is not adequate and evaluating the
performance under other known synthetic traffics is recommended.''}}

A- We completely agree on this concern. As we have already mentioned in the answers to Reviewer \#1, we have extended the simulation experiments in the paper, using two
additional synthetic traffic patterns commonly used to evaluate the interconnection networks of HPC systems.

\fcolorbox{white}{lightgray}{\parbox{\textwidth}{%
\color{black}%
C- ``The hotspot traffic used in the manuscript is many to one and the
hotspot node is not mentioned in the text, I propose considering more than one hotspot
nodes with determining the percentage of total traffic directed to those nodes. This
scenario will show the efficiency of routing algorithm implementation.''}}

A- First of all, note that the only hot-spot traffic considered in the previous version of the paper is part of the Netgauge benchmark [40] (Nto1), so
we have no control over the hot-spot selection process.
Nevertheless, we agree that the traffic pattern proposed by the reviewer is very interesting. Indeed, as mentioned above, in the new version of the paper we have extended the simulation results using additional synthetic traffic patterns, and one of these patterns corresponds to the hot-spot pattern suggested by the reviewer.


\textbf{References}:

[40]
T.~Hoefler, T.~Mehlan, A.~Lumsdaine, W.~Rehm, {Netgauge: A Network Performance
  Measurement Framework}, in: Proceedings of High Performance Computing and
  Communications, HPCC'07, Vol. 4782, Springer, 2007, pp. 659--671.


\fcolorbox{white}{lightgray}{\parbox{\textwidth}{%
\color{black}%
C- ``Furthermore, the switching mechanism is not reported, Is it a wormhole switching?''}}

A- As we focus on \iba{}-based systems, the switching mechanism must be Virtual Cut-Through (VCT), as described in the \iba{} specification. In the new version of the paper, we have indicated this in Section 2.2. 

\closing{Thank you for your consideration,}


\end{letter}
\end{document}
