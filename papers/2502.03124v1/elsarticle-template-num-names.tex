%% 
%% Copyright 2007-2020 Elsevier Ltd
%% 
%% This file is part of the 'Elsarticle Bundle'.
%% ---------------------------------------------
%% 
%% It may be distributed under the conditions of the LaTeX Project Public
%% License, either version 1.2 of this license or (at your option) any
%% later version.  The latest version of this license is in
%%    http://www.latex-project.org/lppl.txt
%% and version 1.2 or later is part of all distributions of LaTeX
%% version 1999/12/01 or later.
%% 
%% The list of all files belonging to the 'Elsarticle Bundle' is
%% given in the file `manifest.txt'.
%% 
%% Template article for Elsevier's document class `elsarticle'
%% with harvard style bibliographic references

%\documentclass[preprint,12pt]{elsarticle}
%% \documentclass[final,5p]{elsarticle}
\documentclass[final,3p,times]{elsarticle}
%\usepackage{wrapfig}

% If you use the following package, be sure to comment out \usepackate{xcolor}

\usepackage{multirow,mathtools } \usepackage{algorithm,algpseudocode}

% Very useful for various special symbols
\usepackage{pifont}
% For \rowcolor
\usepackage{color, colortbl}

\usepackage{blindtext}
\usepackage{lipsum}

\usepackage{multirow}
\usepackage{graphicx}
\usepackage{listings}

\usepackage{bbm}
\usepackage{dsfont}

\usepackage[most]{tcolorbox}

\newcommand{\JH}[1]{\textcolor{blue}{JH: #1}}
\newcommand{\SL}[1]{\textcolor{purple}{SL: #1}}
\newcommand{\JC}[1]{\textcolor{cyan}{JC: #1}}
\newcommand{\SM}[1]{\textcolor{red}{SM: #1}}
\newcommand{\CF}[1]{\textcolor{orange}{CF: #1}}
\newcommand{\RZ}[1]{\textcolor{red}{RZ: #1}}
\newcommand{\LL}[1]{\textcolor{red}{LL: #1}}
\newcommand{\YY}[1]{\textcolor{red}{YY: #1}}


\usepackage{filecontents}

\begin{filecontents}{cas-refs.bib}

@article{strbac2008demand,
  title={Demand side management: Benefits and challenges},
  author={Strbac, Goran},
  journal={Energy Policy},
  volume={36},
  number={12},
  pages={4419--4426},
  year={2008},
  publisher={Elsevier}
}

@article{pinson2014benefits,
  title={Benefits and challenges of electrical demand response: A critical review},
  author={O'Connell, Niamh and Pinson, Pierre and Madsen, Henrik and O'Malley, Mark},
  journal={Renewable and Sustainable Energy Reviews},
  volume={39},
  pages={686--699},
  year={2014},
  publisher={Elsevier}
}

@article{geng2024assessment,
  title={Assessment of vehicle-side costs and profits of providing vehicle-to-grid services},
  author={Geng, Jingxuan and Bai, Bo and Hao, Han and Sun, Xin and Liu, Ming and Liu, Zongwei and Zhao, Fuquan},
  journal={eTransportation},
  volume={19},
  pages={100303},
  year={2024},
  publisher={Elsevier}
}

@article{geng4872108techno,
  title={Techno-Economic Comparison of Vehicle-to-Grid and Stationary Battery Energy Storage System: Insights for the Technology Roadmap of Electric Vehicle Batteries},
  author={Geng, Jingxuan and Hao, Han and Hao, Xu and Sun, Xin and Liu, Ming and Dou, Hao and Liu, Zongwei and Zhao, Fuquan},
  year={2024},
  journal={Available at SSRN 4872108}
}

@article{rahman2023development,
  title={The development of a techno-economic model for assessment of cost of energy storage for vehicle-to-grid applications in a cold climate},
  author={Rahman, Md Mustafizur and Gemechu, Eskinder and Oni, Abayomi Olufemi and Kumar, Amit},
  journal={Energy},
  volume={262},
  pages={125398},
  year={2023},
  publisher={Elsevier}
}

@article{schmidt2019projecting,
  title={Projecting the future levelized cost of electricity storage technologies},
  author={Schmidt, Oliver and Melchior, Sylvain and Hawkes, Adam and Staffell, Iain},
  journal={Joule},
  volume={3},
  number={1},
  pages={81--100},
  year={2019},
  publisher={Elsevier}
}

@article{simpson2020cost,
  title={Cost of Valued Energy for design of renewable energy systems},
  author={Simpson, Juliet and Loth, Eric and Dykes, Katherine},
  journal={Renewable Energy},
  volume={153},
  pages={290--300},
  year={2020},
  publisher={Elsevier}
}

@article{hirth2013market,
  title={The market value of variable renewables: The effect of solar wind power variability on their relative price},
  author={Hirth, Lion},
  journal={Energy Economics},
  volume={38},
  pages={218--236},
  year={2013},
  publisher={Elsevier}
}

@article{siano2014demand,
  title={Demand response and smart grids—A survey},
  author={Siano, Pierluigi},
  journal={Renewable and Sustainable Energy Reviews},
  volume={30},
  pages={461--478},
  year={2014},
  publisher={Elsevier}
}

@article{gils2016economic,
  title={Economic potential for future demand response in {G}ermany--Modeling approach and case study},
  author={Gils, Hans Christian},
  journal={Applied Energy},
  volume={162},
  pages={401--415},
  year={2016},
  publisher={Elsevier}
}

@article{fischer2017heat,
  title={On heat pumps in smart grids: A review},
  author={Fischer, David and Madani, Hatef},
  journal={Renewable and Sustainable Energy Reviews},
  volume={70},
  pages={342--357},
  year={2017},
  publisher={Elsevier}
}

@article{klaassen2017methodology,
  title={A methodology to assess demand response benefits from a system perspective: A {D}utch case study},
  author={Klaassen, EAM and Van Gerwen, RJF and Frunt, J and Slootweg, JG},
  journal={Utilities Policy},
  volume={44},
  pages={25--37},
  year={2017},
  publisher={Elsevier}
}

@article{good2017review,
  title={Review and classification of barriers and enablers of demand response in the smart grid},
  author={Good, Nicholas and Ellis, Keith A and Mancarella, Pierluigi},
  journal={Renewable and Sustainable Energy Reviews},
  volume={72},
  pages={57--72},
  year={2017},
  publisher={Elsevier}
}

@article{parrish2019demand,
  title={On demand: Can demand response live up to expectations in managing electricity systems?},
  author={Parrish, Bryony and Gross, Rob and Heptonstall, Phil},
  journal={Energy Research \& Social Science},
  volume={51},
  pages={107--118},
  year={2019},
  publisher={Elsevier}
}

@article{faruqui2013arcturus,
  title={Arcturus: international evidence on dynamic pricing},
  author={Faruqui, Ahmad and Sergici, Sanem},
  journal={The Electricity Journal},
  volume={26},
  number={7},
  pages={55--65},
  year={2013},
  publisher={Elsevier}
}

@article{kubli2022ev,
  title={{EV} drivers’ willingness to accept smart charging: Measuring preferences of potential adopters},
  author={Kubli, Merla},
  journal={Transportation Research Part D: Transport and Environment},
  volume={109},
  pages={103396},
  year={2022},
  publisher={Elsevier}
}

@article{ensslen2018incentivizing,
  title={Incentivizing smart charging: Modeling charging tariffs for electric vehicles in {G}erman and {F}rench electricity markets},
  author={Ensslen, Axel and Ringler, Philipp and D{\"o}rr, Lasse and Jochem, Patrick and Zimmermann, Florian and Fichtner, Wolf},
  journal={Energy Research \& Social Science},
  volume={42},
  pages={112--126},
  year={2018},
  publisher={Elsevier}
}

@article{baumgartner2022does,
  title={Does experience matter? {A}ssessing user motivations to accept a vehicle-to-grid charging tariff},
  author={Baumgartner, Nora and Kellerer, Franziska and Ruppert, Manuel and Hirsch, Sebastian and Mang, Stefan and Fichtner, Wolf},
  journal={Transportation Research Part D: Transport and Environment},
  volume={113},
  pages={103528},
  year={2022},
  publisher={Elsevier}
}

@article{parsons2014willingness,
  title={Willingness to pay for vehicle-to-grid ({V2G}) electric vehicles and their contract terms},
  author={Parsons, George R and Hidrue, Michael K and Kempton, Willett and Gardner, Meryl P},
  journal={Energy Economics},
  volume={42},
  pages={313--324},
  year={2014},
  publisher={Elsevier}
}

@article{mehdizadeh2023estimating,
  title={Estimating financial compensation and minimum guaranteed charge for vehicle-to-grid technology},
  author={Mehdizadeh, Milad and Nordfjaern, Trond and Kl{\"o}ckner, Christian A},
  journal={Energy Policy},
  volume={180},
  pages={113649},
  year={2023},
  publisher={Elsevier}
}

@article{lim2016assessment,
  title={Assessment of the Potential Consumers' Preference for the {V2G} System},
  author={Lim, Seul-Ye and Kim, Hee-Hoon and Yoo, Seung-Hoon},
  journal={Journal of Energy Engineering},
  volume={25},
  number={4},
  pages={93--102},
  year={2016},
  publisher={The Korean Society for Energy}
}

@article{lee2020willingness,
  title={Willingness to accept values for vehicle-to-grid service in {S}outh {K}orea},
  author={Lee, Chul-Yong and Jang, Jung-Woo and Lee, Min-Kyu},
  journal={Transportation Research Part D: Transport and Environment},
  volume={87},
  pages={102487},
  year={2020},
  publisher={Elsevier}
}

@article{noel2019willingness,
  title={Willingness to pay for electric vehicles and vehicle-to-grid applications: A Nordic choice experiment},
  author={Noel, Lance and Carrone, Andrea Papu and Jensen, Anders Fjendbo and de Rubens, Gerardo Zarazua and Kester, Johannes and Sovacool, Benjamin K},
  journal={Energy Economics},
  volume={78},
  pages={525--534},
  year={2019},
  publisher={Elsevier}
}

@article{huang2021electric,
  title={Are electric vehicle drivers willing to participate in vehicle-to-grid contracts? A context-dependent stated choice experiment},
  author={Huang, Bing and Meijssen, Aart Gerard and Annema, Jan Anne and Lukszo, Zofia},
  journal={Energy Policy},
  volume={156},
  pages={112410},
  year={2021},
  publisher={Elsevier}
}

@article{ma2022research,
  title={Research on the valley-filling pricing for {EV} charging considering renewable power generation},
  author={Ma, Shao-Chao and Yi, Bo-Wen and Fan, Ying},
  journal={Energy Economics},
  volume={106},
  pages={105781},
  year={2022},
  publisher={Elsevier}
}


@article{will2016understanding,
  title={Understanding user acceptance factors of electric vehicle smart charging},
  author={Will, Christian and Schuller, Alexander},
  journal={Transportation Research Part C: Emerging Technologies},
  volume={71},
  pages={198--214},
  year={2016},
  publisher={Elsevier}
}

@article{schmalfuss2015user,
  title={User responses to a smart charging system in {G}ermany: Battery electric vehicle driver motivation, attitudes and acceptance},
  author={Schmalfuss, Franziska and Mair, Claudia and D{\"o}belt, Susen and Kaempfe, Bettina and Wuestemann, Ramona and Krems, Josef F and Keinath, Andreas},
  journal={Energy Research \& Social Science},
  volume={9},
  pages={60--71},
  year={2015},
  publisher={Elsevier}
}

@article{delmonte2020consumers,
  title={What do consumers think of smart charging? Perceptions among actual and potential plug-in electric vehicle adopters in the United Kingdom},
  author={Delmonte, Emma and Kinnear, Neale and Jenkins, Becca and Skippon, Stephen},
  journal={Energy Research \& Social Science},
  volume={60},
  pages={101318},
  year={2020},
  publisher={Elsevier}
}

@article{van2021factors,
  title={Factors influencing consumer acceptance of vehicle-to-grid by electric vehicle drivers in the Netherlands},
  author={van Heuveln, Koen and Ghotge, Rishabh and Annema, Jan Anne and van Bergen, Esther and van Wee, Bert and Pesch, Udo},
  journal={Travel Behaviour and Society},
  volume={24},
  pages={34--45},
  year={2021},
  publisher={Elsevier}
}

@article{ghotge2022use,
  title={Use before you choose: What do {EV} drivers think about {V2G} after experiencing it?},
  author={Ghotge, Rishabh and Nijssen, Koen Philippe and Annema, Jan Anne and Lukszo, Zofia},
  journal={Energies},
  volume={15},
  number={13},
  pages={4907},
  year={2022},
  publisher={MDPI}
}

@article{geske2018willing,
  title={Willing to participate in vehicle-to-grid ({V2G})? Why not!},
  author={Geske, Joachim and Schumann, Diana},
  journal={Energy Policy},
  volume={120},
  pages={392--401},
  year={2018},
  publisher={Elsevier}
}

@article{gong2021incentives,
  title={Incentives and concerns on vehicle-to-grid technology expressed by Australian employees and employers},
  author={Gong, Shuangqing and Cheng, Vincent Hin Sing and Ardeshiri, Ali and Rashidi, Taha Hossein},
  journal={Transportation Research Part D: Transport and Environment},
  volume={98},
  pages={102986},
  year={2021},
  publisher={Elsevier}
}

@article{bohnsack2015deriving,
  title={Deriving vehicle-to-grid business models from consumer preferences},
  author={Bohnsack, Ren{\'e} and Van den Hoed, Robert and Oude Reimer, Hugo},
  journal={World Electric Vehicle Journal},
  volume={7},
  number={4},
  pages={621--630},
  year={2015},
  publisher={Multidisciplinary Digital Publishing Institute}
}

@article{bailey2015anticipating,
  title={Anticipating {PEV} buyers’ acceptance of utility controlled charging},
  author={Bailey, Joseph and Axsen, Jonn},
  journal={Transportation Research Part A: Policy and Practice},
  volume={82},
  pages={29--46},
  year={2015},
  publisher={Elsevier}
}

@article{kubli2018flexible,
  title={The flexible prosumer: Measuring the willingness to co-create distributed flexibility},
  author={Kubli, Merla and Loock, Moritz and W{\"u}stenhagen, Rolf},
  journal={Energy Policy},
  volume={114},
  pages={540--548},
  year={2018},
  publisher={Elsevier}
}

@article{marxen2023empirical,
  title={Empirical evaluation of behavioral interventions to enhance flexibility provision in smart charging},
  author={Marxen, Hanna and Ansarin, Mohammad and Chemudupaty, Raviteja and Fridgen, Gilbert},
  journal={Transportation Research Part D: Transport and Environment},
  volume={123},
  pages={103897},
  year={2023},
  publisher={Elsevier}
}

@article{hidrue2015there,
  title={Is there a near-term market for vehicle-to-grid electric vehicles?},
  author={Hidrue, Michael K and Parsons, George R},
  journal={Applied Energy},
  volume={151},
  pages={67--76},
  year={2015},
  publisher={Elsevier}
}

@article{julch2016comparison,
  title={Comparison of electricity storage options using levelized cost of storage ({LCOS}) method},
  author={J{\"u}lch, Verena},
  journal={Applied Energy},
  volume={183},
  pages={1594--1606},
  year={2016},
  publisher={Elsevier}
}

@article{kaczmarski2022determinants,
  title={Determinants of demand response program participation: Contingent valuation evidence from a smart thermostat program},
  author={Kaczmarski, Jesse and Jones, Benjamin and Chermak, Janie},
  journal={Energies},
  volume={15},
  number={2},
  pages={590},
  year={2022},
  publisher={MDPI}
}

@article{joskow2011comparing,
  title={Comparing the costs of intermittent and dispatchable electricity generating technologies},
  author={Joskow, Paul L},
  journal={American Economic Review},
  volume={101},
  number={3},
  pages={238--241},
  year={2011},
  publisher={American Economic Association}
}

@article{ueckerdt2013system,
  title={System {LCOE}: What are the costs of variable renewables?},
  author={Ueckerdt, Falko and Hirth, Lion and Luderer, Gunnar and Edenhofer, Ottmar},
  journal={Energy},
  volume={63},
  pages={61--75},
  year={2013},
  publisher={Elsevier}
}

@article{loth2022we,
  title={Why we must move beyond {LCOE} for renewable energy design},
  author={Loth, Eric and Qin, Chris and Simpson, Juliet G and Dykes, Katherine},
  journal={Advances in Applied Energy},
  volume={8},
  pages={100112},
  year={2022},
  publisher={Elsevier}
}

@article{heilmann2021much,
  title={How much smart charging is smart?},
  author={Heilmann, Christoph and Wozabal, David},
  journal={Applied Energy},
  volume={291},
  pages={116813},
  year={2021},
  publisher={Elsevier}
}

@article{luerssen2021global,
  title={Global sensitivity and uncertainty analysis of the levelised cost of storage ({LCOS}) for solar-{PV}-powered cooling},
  author={Luerssen, Christoph and Verbois, Hadrien and Gandhi, Oktoviano and Reindl, Thomas and Sekhar, Chandra and Cheong, David},
  journal={Applied Energy},
  volume={286},
  pages={116533},
  year={2021},
  publisher={Elsevier}
}

@misc{housing2024,
  title={Size of {E}nglish homes - {E}nglish housing survey 2018-19},
  author={{Ministry of Housing, Communities and Local Government}},
  howpublished = "\url{https://assets.publishing.service.gov.uk/media/5f047a01d3bf7f2be8350262/Size_of_English_Homes_Fact_Sheet_EHS_2018.pdf}",
  publisher = {{Office for National Statistics}},
  year         = 2019,
  note         = "Accessed: 2024-09-04"}

@misc{team_2024,
  title={Private rent and house prices, UK: August 2024},
  author={{Office for National Statistics}},
  howpublished = "\url{https://www.ons.gov.uk/economy/inflationandpriceindices/bulletins/privaterentandhousepricesuk/august2024}",
  publisher = {{Office for National Statistics}},
  year         = 2024,
  note         = "Accessed: 2024-09-04"}

@article{schauble2020conditions,
  title={Conditions for a cost-effective application of smart thermostat systems in residential buildings},
  author={Sch{\"a}uble, Dominik and Marian, Adela and Cremonese, Lorenzo},
  journal={Applied Energy},
  volume={262},
  pages={114526},
  year={2020},
  publisher={Elsevier}
}

@article{li2020estimation,
  title={Estimation of building heat transfer coefficients from in-use data: impacts of unmonitored energy flows},
  author={Li, Matthew and Allinson, David and Lomas, Kevin},
  journal={International Journal of Building Pathology and Adaptation},
  volume={38},
  number={1},
  pages={38--50},
  year={2020},
  publisher={Emerald Publishing Limited}
}

@misc{lowe2017,
author = {Robert Lowe},
title = {Renewable Heat Premium Payment Scheme: Heat Pump Monitoring: Cleaned Data, 2013-2015},
doi = {10.5255/UKDA-SN-8151-1},
howpublished= {\url{http://doi.org/10.5255/UKDA-SN-8151-1}},
year         = 2016,
institution = {{Department of Energy and Climate Change}},
publisher = {{UK Data Service}}
}

@misc{dftEVdata,
    author = {{Department for Transport}},
    title        = "Electric Chargepoint Analysis 2017: Domestics",
    howpublished = "\url{https://www.data.gov.uk/dataset/5438d88d-695b-4381-a5f2-6ea03bf3dcf0/electric-chargepoint-analysis-2017-domestics}",
    year         = 2018,
    note         = "Accessed: 2023-09-15"
}

@misc{dftBATdata,
    author = {{Department for Transport}},
    title        = "Vehicle licensing statistics data tables",
    howpublished = "\url{https://www.gov.uk/government/statistical-data-sets/vehicle-licensing-statistics-data-tables#ultra-low-emission-vehicles}",
    year         = 2024,
    note         = "Accessed: 2025-01-14"
}

https://www.gov.uk/government/statistical-data-sets/vehicle-licensing-statistics-data-tables#ultra-low-emission-vehicles

@misc{elecpricedata,
    author = {{Elexon}},
    title        = "Balancing Mechanism Reporting Service ({BMRS}): Market Index Data",
    howpublished = "\url{https://www.bmreports.com/bmrs/?q=balancing/marketindex/historic}",
    year         = 2023,
    note         = "Accessed: 2023-09-15"
}

@misc{octopus_powerpack,
    title = {Octopus Power Pack: the {UK}’s first Vehicle-to-Grid tariff},
    author = {{Octopus Energy UK}}, 
    note = "Accessed: 2024-09-05",
    howpublished = "\url{https://octopus.energy/power-pack/}",
    year = 2024
}

@misc{edf_ev,
    title = {{EV} tariffs - Designed specifically for {EV} drivers },
    author = {{EdF Energy}}, 
    note = "Accessed: 2024-09-05",
    howpublished = "\url{https://www.edfenergy.com/electric-cars/ev-tariffs}",
    year = 2024
}

@misc{ovo_ev,
    title = {Charge Anytime - The UK's cheapest home {EV} charging},
    author = {{OVO Energy}}, 
    note = "Accessed: 2024-09-05",
    howpublished = "\url{https://www.ovoenergy.com/electric-cars/charge-anytime}",
    year = 2023
}

@misc{octopus_ev,
    title = {How {EV} tariffs work for electric cars},
    author = {{Octopus Energy UK}}, 
    note = "Accessed: 2024-09-05",
    howpublished = "\url{https://octopusev.com/ev-hub/guide-to-energy-tariffs-for-electric-cars}",
    year = 2023
}

@inproceedings{fabianek2024willing,
  title={Willing to Wait? Acceptance for Load Management at e-Vehicle Charging Stations in Germany},
  author={Fabianek, Paul and Madlener, Reinhard},
  booktitle={2024 20th International Conference on the European Energy Market (EEM)},
  pages={1--5},
  year={2024},
  organization={IEEE}
}

@article{yilmaz2020analysis,
  title={Analysis of demand-side response preferences regarding electricity tariffs and direct load control: Key findings from a Swiss survey},
  author={Yilmaz, Selin and Xu, Xiaojing and Cabrera, Daniel and Chanez, C{\'e}dric and Cuony, Peter and Patel, Martin K},
  journal={Energy},
  volume={212},
  pages={118712},
  year={2020},
  publisher={Elsevier}
}

@article{huber2021vehicle,
  title={Vehicle to grid impacts on the total cost of ownership for electric vehicle drivers},
  author={Huber, Dominik and De Clerck, Quentin and De Cauwer, Cedric and Sapountzoglou, Nikolaos and Coosemans, Thierry and Messagie, Maarten},
  journal={World Electric Vehicle Journal},
  volume={12},
  number={4},
  pages={236},
  year={2021},
  publisher={MDPI}
}

@misc{smartmeter,
    title = {Smart meter transition and the Data Communications Company ({DCC})},
    author = {{Office of Gas and Electricity Markets (Ofgem)}}, 
    note = "Accessed: 2024-09-18",
    howpublished = "\url{https://www.ofgem.gov.uk/energy-policy-and-regulation/policy-and-regulatory-programmes/smart-meter-transition-and-data-communications-company-dcc}",
    year = 2024
}

@article{lee2020providing,
  title={Providing grid services with heat pumps: A review},
  author={Lee, Zachary E and Sun, Qingxuan and Ma, Zhao and Wang, Jiangfeng and MacDonald, Jason S and Max Zhang, K},
  journal={Journal of Engineering for Sustainable Buildings and Cities},
  volume={1},
  number={1},
  pages={011007},
  year={2020},
  publisher={American Society of Mechanical Engineers}
}

@article{Zhang2022, 
author = {Menglin Zhang and Qiuwei Wu and Theis Bo Harild Rasmussen and Xiaodong Yang and Jinyu Wen},
title = {Heat Pumps in {D}enmark: Current Situation in Providing Frequency Control Ancillary Services},
year = {2022},
journal = {CSEE Journal of Power and Energy Systems},
volume = {8},
number = {3},
pages = {769-779},
keywords = {Danish energy system, frequency control ancillary service, individual heat pumps, integrated electricity and heat system, large-scale heat pumps, power to heat},
url = {https://www.sciopen.com/article/10.17775/CSEEJPES.2020.06070},
doi = {10.17775/CSEEJPES.2020.06070},
abstract = {The Danish government set an ambitious goal to achieve a fully renewable-based energy system by 2050. In this context, the integrated electricity and heating system is undergoing rapid development in Denmark as a promising way to accommodate the ever-growing renewable energy sources (RESs). The electric heat pumps (HPs), coupled with the power and heat sectors, can propagate the flexibility on the heat consumer side to the power system operations, playing an important role of improving system flexibility and balancing the variability of the RES. In this paper, the current development situation of HPs in Denmark is analyzed, including both the large-scale HPs in the district heating system and individual HPs on the residential side. The possibility of using HPs to provide frequency control ancillary service (FCAS) is analyzed according to the market and technical requirements of the FCAS in the Danish transmission system and experimental results of representative demonstration projects of HPs. A comprehensive analysis of the advantages, barriers, future prospects, and challenges for using HPs to provide the FCAS are carried out from the perspectives of different entities.}
}

@article{thran2024reserve,
  title={Reserve Provision from Electric Vehicles: Aggregate Boundaries and Stochastic Model Predictive Control},
  author={Thr{\"a}n, Jacob and Mare{\v{c}}ek, Jakub and Shorten, Robert N and Green, Timothy C},
  journal={arXiv preprint arXiv:2406.07454},
  year={2024}
}

@misc{dft_mileage,
    title = {Vehicle mileage and occupancy},
    author = {{Department for Transport}}, 
    note = "Accessed: 2024-09-20",
    howpublished = "\url{https://www.gov.uk/government/statistical-data-sets/nts09-vehicle-mileage-and-occupancy#car-mileage}",
    year = 2013
}

@misc{kwhpermile,
    title = {Energy consumption of electric vehicles},
    author = {{EV} Database}, 
    note = "Accessed: 2024-09-20",
    howpublished = "\url{https://ev-database.org/uk/cheatsheet/energy-consumption-electric-car}",
    year = 2024
}


@article{lai2017levelized,
  title={Levelized cost of electricity for solar photovoltaic and electrical energy storage},
  author={Lai, Chun Sing and McCulloch, Malcolm D},
  journal={Applied Energy},
  volume={190},
  pages={191--203},
  year={2017},
  publisher={Elsevier}
}

@article{branker2011review,
  title={A review of solar photovoltaic levelized cost of electricity},
  author={Branker, Kadra and Pathak, MJM and Pearce, Joshua M},
  journal={Renewable and Sustainable Energy Reviews},
  volume={15},
  number={9},
  pages={4470--4482},
  year={2011},
  publisher={Elsevier}
}


@book{schmidt2024monetizing,
  title={Monetizing energy storage: a toolkit to assess future cost and value},
  author={Schmidt, Oliver and Staffell, Iain},
  year={2024},
  publisher={Oxford University Press}
}

@article{vilen2024seasonal,
  title={Seasonal large-scale thermal energy storage in an evolving district heating system--Long-term modeling of interconnected supply and demand},
  author={Vil{\'e}n, Karl and Ahlgren, Erik O},
  journal={Smart Energy},
  volume={15},
  pages={100156},
  year={2024},
  publisher={Elsevier}
}

@article{zhang2016evaluation,
  title={Evaluation of achievable vehicle-to-grid capacity using aggregate PEV model},
  author={Zhang, Hongcai and Hu, Zechun and Xu, Zhiwei and Song, Yonghua},
  journal={IEEE Transactions on Power Systems},
  volume={32},
  number={1},
  pages={784--794},
  year={2016},
  publisher={IEEE}
}

@misc{dft_homecharge,
    title = {Electric Vehicle Charging Research: Survey with electric vehicle drivers},
    author = {{BritainThinks} and {Department for Transport}}, 
    note = "Accessed: 2025-01-14",
    howpublished = "\url{https://assets.publishing.service.gov.uk/media/628f5603d3bf7f037097bd73/dft-ev-driver-survey-summary-report.pdf}",
    year = 2022}

@article{panao2019measured,
  title={Measured and modeled performance of internal mass as a thermal energy battery for energy flexible residential buildings},
  author={Pan{\~a}o, Marta JN Oliveira and Mateus, Nuno M and Da Gra{\c{c}}a, G Carrilho},
  journal={Applied Energy},
  volume={239},
  pages={252--267},
  year={2019},
  publisher={Elsevier}
}

@article{odukomaiya2021addressing,
  title={Addressing energy storage needs at lower cost via on-site thermal energy storage in buildings},
  author={Odukomaiya, Adewale and Woods, Jason and James, Nelson and Kaur, Sumanjeet and Gluesenkamp, Kyle R and Kumar, Navin and Mumme, Sven and Jackson, Roderick and Prasher, Ravi},
  journal={Energy \& Environmental Science},
  volume={14},
  number={10},
  pages={5315--5329},
  year={2021},
  publisher={Royal Society of Chemistry}
}

@article{CHANDRA2019110,
title = {Stratification analysis of domestic hot water storage tanks: A comprehensive review},
journal = {Energy and Buildings},
volume = {187},
pages = {110-131},
year = {2019},
issn = {0378-7788},
doi = {https://doi.org/10.1016/j.enbuild.2019.01.052},
url = {https://www.sciencedirect.com/science/article/pii/S0378778818316669},
author = {Yogender Pal Chandra and Tomas Matuska},
keywords = {Thermal stratification, Thermal energy storage, Domestic hot water tank (DHWT)},
abstract = {To assure high quality thermal storage and high efficiency of its acquisition, thermal stratification is often employed in domestic hot water tanks. The whole motivation of stratification lies in the fact that mixing effect can be minimized during operational cycle of the tank so that high temperature water could be taken at the load end, thus maintaining high thermal efficiency at demand side, while low-temperature water can be drawn at lower bottom, thus maintaining the high efficiency at energy collection side. The study of stratification entails the assessment of a wide variety of concepts to be embodied around the central theme of the tank – its design and modelling. This paper presents a systematic review pertaining to various such concepts. For instance, multi-node and plug-flow approach to model various temperature distribution models are considered. These models are categorized in paper as linear, stepped, continuous-linear and general three-zone temperature distribution models. Subsequently, the dynamics of thermocline decay and influencing parameters both during standby and dynamic mode will be demonstrated. In addition, a survey of state of the art methods and practices to ascertain the performance improvement and its quantification will be illustrated. This includes geometrical parameters – such as, structural design incorporation, essentially – inlet design, tank aspect ratio and wall material specification, and also, operational parameters to curb down the inlet mixing. Practice techniques and methods which are presented here in a novel way, extend towards the ground of practical application and research procedures.}
}

@article{CRUICKSHANK20101703,
title = {Heat loss characteristics for a typical solar domestic hot water storage},
journal = {Energy and Buildings},
volume = {42},
number = {10},
pages = {1703-1710},
year = {2010},
issn = {0378-7788},
doi = {https://doi.org/10.1016/j.enbuild.2010.04.013},
url = {https://www.sciencedirect.com/science/article/pii/S0378778810001544},
author = {Cynthia A. Cruickshank and Stephen J. Harrison},
keywords = {Thermal stratification, Thermal energy storage, Solar heating systems, Multi-tank, Solar hot water, Storage heat loss},
abstract = {It is common practice to predict the performance of solar domestic hot water (SDHW) systems by computer simulation. This process relies on the accurate specification of the system's physical and thermal characteristics, and is often based on a number of simplifying assumptions. An important aspect of system performance is storage heat loss characteristics; however, these are often represented by an average heat loss coefficient or U-value that does not account for the complex geometry of the thermal storage or the interaction of the various inlet and outlet ports that may act as thermal conduits. In addition, most solar storage models assume that the tank temperature profile is one-dimensional and that conduction within the tank wall is negligible. To investigate these effects, tests were conducted on a typical thermal storage used in SDHW applications and included a cool-down test and a heat diffusion test sequence. The values derived from these test sequences were then compared to computer predictions based on estimated thermal properties. In addition, the basic assumptions typically used in the computer modelling of solar storage heat losses (e.g., one-dimensional temperature profiles, minimal tank wall conduction, uniform wall heat loss) were investigated, particularly in the context of a thermally stratified thermal storage.}
}

@article{ARMSTRONG2014128,
title = {Improving the energy storage capability of hot water tanks through wall material specification},
journal = {Energy},
volume = {78},
pages = {128-140},
year = {2014},
issn = {0360-5442},
doi = {https://doi.org/10.1016/j.energy.2014.09.061},
url = {https://www.sciencedirect.com/science/article/pii/S0360544214011189},
author = {P. Armstrong and D. Ager and I. Thompson and M. McCulloch},
keywords = {Thermal stratification, Domestic hot water tank, Demand side management, Computational fluid dynamics, Material selection},
abstract = {Domestic hot water tanks represent a significant potential demand side management asset within energy systems. To operate effectively as energy storage devices, it is crucial that a stratified temperature distribution is maintained during operation; this paper details experimental and numerical work conducted to understand the influence that wall material specification has on de-stratification within domestic hot water tanks. A 2d axisymmetric CFD (Computational Fluid Dynamic) model was consistent with experiments which showed that switching from copper to stainless steel resulted in a 2.7 fold reduction in useable hot water loss through reduced de-stratification for a 74 L UK domestic hot water tank over a 48 h period. During simulation, a counter rotating convection system, with peak velocities of 0.005 m/s, was observed above and below the thermocline. Minimizing de-stratification, through appropriate wall material selection, increases the performance of hot water tanks and scope for their use in demand side management applications. Given the inconclusive evidence surrounding copper's efficacy as a sanitizing agent, along with the low tensile strength of polyethylene, this paper advocates the use of stainless steel in hot water tank walls and further exploration of alternative materials and composites which have low cost and low thermal conductivity along with high strength and manufacturability.}
}


@misc{dclg_ceilingheight,
    title = {Technical housing standards – nationally described space standard},
    author = {{Department for Communities and Local Government}}, 
    note = "Accessed: 2025-01-16",
    howpublished = "\url{https://assets.publishing.service.gov.uk/government/uploads/system/uploads/attachment_data/file/1012976/160519_Nationally_Described_Space_Standard.pdf}",
    year = 2015}

@article{zang2025levelized,
  title={Levelized cost quantification of energy flexibility in high-density cities and evaluation of demand-side technologies for providing grid services},
  author={Zang, Xingyu and Li, Hangxin and Wang, Shengwei},
  journal={Renewable and Sustainable Energy Reviews},
  volume={211},
  pages={115290},
  year={2025},
  publisher={Elsevier}
}



@misc{ofgem_pricecap,
    title = {Energy price cap},
    author = {{Office of Gas and Electricity Markets (ofgem)}}, 
    note = "Accessed: 2025-01-22",
    howpublished = "\url{https://www.ofgem.gov.uk/energy-price-cap}",
    year = 2025}





@misc{ukevse,
    title = {A guide on electric vehicle charging and DNO engagement for local authorities},
    author = {{UK Electric Vehicle Supply Equipment Association} and {Western Power Distribution}}, 
    note = "Accessed: 2025-01-24",
    howpublished = "\url{https://www.nationalgrid.co.uk/downloads/29134}",
    year = 2022}

@article{samuelson1948consumption,
  title={Consumption theory in terms of revealed preference},
  author={Samuelson, Paul A},
  journal={Economica},
  volume={15},
  number={60},
  pages={243--253},
  year={1948},
  publisher={JSTOR}
}

@article{mangham2009or,
  title={How to do (or not to do)… Designing a discrete choice experiment for application in a low-income country},
  author={Mangham, Lindsay J and Hanson, Kara and McPake, Barbara},
  journal={Health policy and planning},
  volume={24},
  number={2},
  pages={151--158},
  year={2009},
  publisher={Oxford University Press}
}

@article{lynch2019impacts,
  title={The impacts of demand response participation in capacity markets},
  author={Lynch, Muireann {\'A} and Nolan, Sheila and Devine, Mel T and O’Malley, Mark},
  journal={Applied Energy},
  volume={250},
  pages={444--451},
  year={2019},
  publisher={Elsevier}
}

\end{filecontents}

\begin{filecontents}{elsarticle-template-num-names.nls}
\begin{thenomenclature} 
\nomgroup{Symbols}
  \item [{$A$}]\begingroup Area rented for hot water storage in $m^2$\nomeqref {0}\nompageref{3}
  \item [{$COP$}]\begingroup Heat pump’s coefficient of performance\nomeqref {0}\nompageref{3}
  \item [{$E^{daily}_{drive} $}]\begingroup Energy used by EV daily for driving in $kWh$\nomeqref {0}\nompageref{3}
  \item [{$E_{EV}^{arr}$}]\begingroup EV battery’s energy level upon arrival in $kWh$\nomeqref {0}\nompageref{3}
  \item [{$E_{EV}^{max}$}]\begingroup EV battery’s energy capacity in $kWh$\nomeqref {0}\nompageref{3}
  \item [{$E_{EV}^{min}$}]\begingroup EV battery’s minimum energy level in $kWh$\nomeqref {0}\nompageref{3}
  \item [{$E_{app}$}]\begingroup Application's shifted/discharged energy in $kWh$\nomeqref {0}\nompageref{3}
  \item [{$E_{bat}^{max}$}]\begingroup Battery’s energy capacity in $kWh$\nomeqref {0}\nompageref{3}
  \item [{$E_{bat}^{min}$}]\begingroup Battery’s minimum energy level in $kWh$\nomeqref {0}\nompageref{3}
  \item [{$H$}]\begingroup Residential ceiling height in $m$\nomeqref {0}\nompageref{3}
  \item [{$L$}]\begingroup Wall thickness of hot water storage in $m$\nomeqref {0}\nompageref{3}
  \item [{$N^{E}_{cha} $}]\begingroup Number of required V2G chargers for energy\nomeqref {0}\nompageref{3}
  \item [{$N^{P}_{cha} $}]\begingroup Number of required V2G chargers for power\nomeqref {0}\nompageref{3}
  \item [{$N^{adj}_{HP} $}]\begingroup Number of required HPs adjusted for activation frequency constraints\nomeqref {0}\nompageref{3}
  \item [{$N^{avail}_{cha}$}]\begingroup Number of available V2G chargers required\nomeqref {0}\nompageref{3}
  \item [{$N^{contr}_{cha}$}]\begingroup Number of contracted V2G chargers required\nomeqref {0}\nompageref{3}
  \item [{$N_{1D}$}]\begingroup Number of required unidirectional DR assets\nomeqref {0}\nompageref{3}
  \item [{$N_{app}^{cyc}$}]\begingroup Application’s number of annual cycles\nomeqref {0}\nompageref{3}
  \item [{$P^{DR}_{t} $}]\begingroup Available DR power reduction at time $t$ in $kW$\nomeqref {0}\nompageref{3}
  \item [{$P^{cap}_{app}$}]\begingroup Application’s required power capacity in $kW$\nomeqref {0}\nompageref{3}
  \item [{$P^{cap}_{cha}$}]\begingroup Power capacity of EV charger in $kW$\nomeqref {0}\nompageref{3}
  \item [{$P_{1D}^{avg}$}]\begingroup Average shiftable power consumption of a unidirectional DR asset in $kW$\nomeqref {0}\nompageref{3}
  \item [{$P_{1D}^{cap}$}]\begingroup Power consumption capacity of a unidirectional DR asset in $kW$\nomeqref {0}\nompageref{3}
  \item [{$P_{EV}^{max}$}]\begingroup EV charger’s maximum power in $kW$\nomeqref {0}\nompageref{3}
  \item [{$P_{HP}^{act,avg}$}]\begingroup Average power use of activated HPs in $kW$\nomeqref {0}\nompageref{3}
  \item [{$P_{bat}^{max}$}]\begingroup Battery’s maximum power in $kW$\nomeqref {0}\nompageref{3}
  \item [{$Q_w$}]\begingroup Thermal energy in hot water storage in  $kWh$\nomeqref {0}\nompageref{3}
  \item [{$SPF$}]\begingroup Heat pump’s seasonal performance factor\nomeqref {0}\nompageref{3}
  \item [{$V_{w}$}]\begingroup Hot water storage volume in $m^3$\nomeqref {0}\nompageref{3}
  \item [{$\Delta E^{elec}_{HP}$}]\begingroup HP’s electrical energy needs for heating thermal storage in $kWh$\nomeqref {0}\nompageref{3}
  \item [{$\Delta P_{HP}^{red}$}]\begingroup DR-related reduction in HP power in $kW$\nomeqref {0}\nompageref{3}
  \item [{$\Delta Q_{b}$}]\begingroup Building’s permitted heat divergence $kWh$\nomeqref {0}\nompageref{3}
  \item [{$\Delta T_w$}]\begingroup Temperature range of hot water storage in $K$\nomeqref {0}\nompageref{3}
  \item [{$\Delta T_{b}$}]\begingroup Maximum permitted temperature divergence in heat-pump-only DLC contract in $K$\nomeqref {0}\nompageref{3}
  \item [{$\Delta t^{day}_{cha}$}]\begingroup Daily time required for charging EV in $h$\nomeqref {0}\nompageref{3}
  \item [{$\Delta t_{DD}$}]\begingroup Application’s discharge duration in $h$\nomeqref {0}\nompageref{3}
  \item [{$\Delta t_{PD}$}]\begingroup Average plug-in duration in $h$\nomeqref {0}\nompageref{3}
  \item [{$\Delta t_{RP}$}]\begingroup Daily required plug-in time (RPT) in $h$\nomeqref {0}\nompageref{3}
  \item [{$\Delta t_{base}^{plug-in}$}]\begingroup RPT for $p^{WTA}_{base}$ in $h$\nomeqref {0}\nompageref{3}
  \item [{$\delta_w$}]\begingroup Hot water storage density in $kg/m^3$\nomeqref {0}\nompageref{3}
  \item [{$\eta_{cha}$}]\begingroup Efficiency of charger\nomeqref {0}\nompageref{3}
  \item [{$a^{V2G}_{f} $}]\begingroup Availability factor for V2G (fraction of time that an EV is plugged in for)\nomeqref {0}\nompageref{3}
  \item [{$a_{GMC}$}]\begingroup Guaranteed minimum charge in $\%$\nomeqref {0}\nompageref{3}
  \item [{$a_{HC} $}]\begingroup Fraction of EV’s driving energy charged at home\nomeqref {0}\nompageref{3}
  \item [{$c_{p,b}$}]\begingroup Building’s specific heat capacity in $kJ/(kg\cdot K)$\nomeqref {0}\nompageref{3}
  \item [{$c_{p,w}$}]\begingroup Hot water specific heat capacity in $kJ/(kg\cdot K)$\nomeqref {0}\nompageref{3}
  \item [{$f_{act}^{max}$}]\begingroup Maximum frequency of activations in heat-pump-only DLC contract per month\nomeqref {0}\nompageref{3}
  \item [{$m_b$}]\begingroup Building’s mass in $kg$\nomeqref {0}\nompageref{3}
  \item [{$m_w$}]\begingroup Hot water storage mass in $kg$\nomeqref {0}\nompageref{3}
  \item [{$p^E_t$}]\begingroup Electricity spot price in time interval $t$ in $\$/kWh$\nomeqref {0}\nompageref{3}
  \item [{$p^{WTA}$}]\begingroup Consumers' median willingness-to-accept for DR heating schemes in $\$/month$\nomeqref {0}\nompageref{3}
  \item [{$p^{WTA}_{base}$}]\begingroup Consumers' median willingness-to-accept for base contract of 11.5 hours of RPT in $\$/month$\nomeqref {0}\nompageref{3}
  \item [{$p^{WTA}_{hour}$}]\begingroup Consumers' median willingness-to-accept per hour of RPT in $\$/month$\nomeqref {0}\nompageref{3}
  \item [{$r$}]\begingroup Discount rate\nomeqref {0}\nompageref{3}
  \item [{$t_{EV}^{arr}$}]\begingroup EV’s arrival time\nomeqref {0}\nompageref{3}
  \item [{$t_{EV}^{dep}$}]\begingroup EV’s departure time\nomeqref {0}\nompageref{3}
  \item [{$t_{act}$}]\begingroup Time of DR activation\nomeqref {0}\nompageref{3}
  \item [{$t/T$}]\begingroup Index/Set of time intervals\nomeqref {0}\nompageref{3}

\end{thenomenclature}


\end{filecontents}

\usepackage{tikz}
\def\checkmark{\tikz\fill[scale=0.4](0,.35) -- (.25,0) -- (1,.7) -- (.25,.15) -- cycle;}
\usepackage{makecell}
\usepackage{hhline}
\usepackage{amsmath}
\usepackage{xurl}
\usepackage{enumitem}

\usepackage [english]{babel}
\usepackage [autostyle, english = american]{csquotes}
\MakeOuterQuote{"}

\usepackage{gensymb}

\usepackage{framed} % Framing content

\usepackage{multicol} % Multiple columns environment

\usepackage{nomencl} % Nomenclature package

\makenomenclature

\setlength{\nomitemsep}{-\parskip} % Baseline skip between items

\renewcommand*\nompreamble{\begin{multicols}{2}}

\renewcommand*\nompostamble{\end{multicols}}

\usepackage{etoolbox}
%\renewcommand\nomgroup[1]{%
%  \item[\bfseries
%  \ifstrequal{#1}{N}{Abbreviations}{%
%  \ifstrequal{#1}{S}{Sets/Indexes}{%
%  \ifstrequal{#1}{V}{Variables}{%
%  \ifstrequal{#1}{P}{Parameters}{}}}}%
%]}

%% Use the option review to obtain double line spacing
%% \documentclass[preprint,review,12pt]{elsarticle}

%% Use the options 1p,twocolumn; 3p; 3p,twocolumn; 5p; or 5p,twocolumn
%% for a journal layout:
%% \documentclass[final,1p,times]{elsarticle}
%% \documentclass[final,1p,times,twocolumn]{elsarticle}
%% \documentclass[final,3p,times]{elsarticle}
%% \documentclass[final,3p,times,twocolumn]{elsarticle}
%% \documentclass[final,5p,times]{elsarticle}
%% \documentclass[final,5p,times,twocolumn]{elsarticle}

%% For including figures, graphicx.sty has been loaded in
%% elsarticle.cls. If you prefer to use the old commands
%% please give \usepackage{epsfig}

%% The amssymb package provides various useful mathematical symbols
\usepackage{amssymb}
%% The amsthm package provides extended theorem environments
 \usepackage{amsthm}
\usepackage{cuted}
%% The lineno packages adds line numbers. Start line numbering with
%% \begin{linenumbers}, end it with \end{linenumbers}. Or switch it on
%% for the whole article with \linenumbers.
%% \usepackage{lineno}

%\usepackage{microtype}

\journal{Applied Energy}

\begin{document}

\begin{frontmatter}

%% Title, authors and addresses

%% use the tnoteref command within \title for footnotes;
%% use the tnotetext command for theassociated footnote;
%% use the fnref command within \author or \address for footnotes;
%% use the fntext command for theassociated footnote;
%% use the corref command within \author for corresponding author footnotes;
%% use the cortext command for theassociated footnote;
%% use the ead command for the email address,
%% and the form \ead[url] for the home page:
%% \title{Title\tnoteref{label1}}
%% \tnotetext[label1]{}
%% \author{Name\corref{cor1}\fnref{label2}}
%% \ead{email address}
%% \ead[url]{home page}
%% \fntext[label2]{}
%% \cortext[cor1]{}
%% \affiliation{organization={},
%%             addressline={},
%%             city={},
%%             postcode={},
%%             state={},
%%             country={}}
%% \fntext[label3]{}

\title{Levelised Cost of Demand Response: Estimating the Cost-Competitiveness of Flexible Demand}

%% use optional labels to link authors explicitly to addresses:
%% \author[label1,label2]{}
%% \affiliation[label1]{organization={},
%%             addressline={},
%%             city={},
%%             postcode={},
%%             state={},
%%             country={}}
%%
%% \affiliation[label2]{organization={},
%%             addressline={},
%%             city={},
%%             postcode={},
%%             state={},
%%             country={}}

\author[inst1,inst2]{Jacob Thrän\corref{cor1}}
\ead{j.thran22@imperial.ac.uk}
\cortext[cor1]{Corresponding author}


\affiliation[inst1]{organization={Department of Electrical and Electronic Engineering, Imperial College London},%Department and Organization
            addressline={South Kensington Campus}, 
            city={London},
            postcode={SW7 2AZ}, 
            %state={State One},
            country={United Kingdom}}

\author[inst1]{Tim C. Green}
\author[inst2]{Robert Shorten}

\affiliation[inst2]{organization={Dyson School of Design Engineering, Imperial College London},%Department and Organization
            addressline={South Kensington Campus}, 
            city={London},
            postcode={SW7 2AZ}, 
            %state={State One},
            country={United Kingdom}}

\begin{abstract}
%% Text of abstract
To make well-informed investment decisions, energy system stakeholders require reliable cost frameworks for demand response (DR) and storage technologies. While the levelised cost of storage (LCOS) permits comprehensive cost comparisons between different storage technologies, no generic cost measure for the comparison of different DR schemes exists. This paper introduces the levelised cost of demand response (LCODR) which is an analogous measure to the LCOS but crucially differs from it by considering consumer reward payments. Additionally, the value factor from cost estimations of variable renewable energy is adapted to account for the variable availability of DR. The LCODRs for four direct load control (DLC) schemes and twelve storage applications are estimated and contrasted against LCOS literature values for the most competitive storage technologies. The DLC schemes are vehicle-to-grid, smart charging, smart heat pumps, and heat pumps with thermal storage. The results show that only heat pumps with thermal storage consistently outcompete storage technologies with EV-based DR schemes being competitive for some applications. The results and the underlying methodology offer a tool for energy system stakeholders to assess the competitiveness of DR schemes even with limited user data.
\end{abstract}

%%Graphical abstract
\iffalse
\begin{graphicalabstract}
\includegraphics{grabs}
\end{graphicalabstract}
\fi
%%Research highlights
\begin{highlights}
\item The LCODR permits the comparison of storage and demand response solutions
\item Four DR schemes: V2G, smart charging, smart heat pump, heat pump + thermal storage
\item Heat pumps with thermal storage are cheaper than any other DR or storage technology
\item V2G and smart charging are competitive with storage for some applications
\end{highlights}

\begin{keyword}
%% keywords here, in the form: keyword \sep keyword
Levelised cost of storage \sep Demand response \sep Electric vehicles \sep Vehicle-to-grid \sep Heat pumps
%% PACS codes here, in the form: \PACS code \sep code
\PACS 0000 \sep 1111
%% MSC codes here, in the form: \MSC code \sep code
%% or \MSC[2008] code \sep code (2000 is the default)
\MSC 0000 \sep 1111
\end{keyword}

\end{frontmatter}

%% \linenumbers

%% main text
\section{Introduction}
\label{sec:introduction}

%% For citations use: 
%%       \citet{<label>} ==> Jones et al. [21]
%%       \citep{<label>} ==> [21]
%%


Demand response (DR) has long been praised for its potential benefits for power system operation \citep{strbac2008demand}. The energy transition, with its need for the integration of increasing shares of variable renewable energy (VRE), has only amplified these potential benefits \citep{gils2016economic}. The addition of large domestic loads, like electric vehicles (EVs) and heat pumps (HPs) has simultaneously increased the effectiveness of DR \citep{fischer2017heat}, and advances in digital control have reduced the perceived inconvenience associated with it \citep{faruqui2013arcturus}. Barriers to DR remain, though, with two regularly mentioned challenges being the fragmented business case of DR and the regulatory framework around it \citep{pinson2014benefits}. The business case for DR is described as fragmented because it has the potential to create benefits for various stakeholders across the energy system. This fragmentation of DR benefits means that no single agent reaps all the benefits of an investment in DR schemes, potentially making it harder to establish a business model \citep{pinson2014benefits}. Various aspects of the regulatory framework have been described as barriers to DR uptake, with the exclusion of DR from certain revenue streams (e.g. capacity markets) being a frequently mentioned example \citep{lynch2019impacts}. 

For policymakers and stakeholders to reduce barriers to DR, it is imperative that they have detailed information about the competitiveness of different DR technologies. To understand the competitiveness of different consumer flexibility technologies, their costs and benefits have to be estimated and compared. To further assess their competitiveness against non-DR flexibility assets, cost and benefit estimations need to be in a format that allows for their comparison with storage, or flexible generation assets. The benefits of DR have been appraised both qualitatively \citep{pinson2014benefits}, and quantitatively \citep{klaassen2017methodology}. A number of reviews have also qualitatively discussed the barriers and costs of DR \citep{good2017review}. For quantitative cost estimations, however, existing studies either only consider consumer payments \citep{faruqui2013arcturus,parrish2019demand,baumgartner2022does} or techno-economic costs \citep{geng2024assessment,geng4872108techno,rahman2023development}. To the best of the author's knowledge, no measure exists that combines consumer payments and techno-economic costs into a comprehensive framework which can also be used to compare costs to other flexibility assets, like energy storage.
%\begin{strip}[h]

\begin{table*}[htbp]

\begin{framed}

\nomenclature[]{$r$}{Discount rate}
\nomenclature[]{$\Delta t_{PD}$}{Average plug-in duration in $h$}
\nomenclature[]{$P^{cap}_{app}$}{Application’s required power capacity in $kW$}
%\nomenclature{$\Delta t_{dd}^{max}$}{DR scheme’s maximum discharge duration in $h$}
\nomenclature[]{$\Delta t_{DD}$}{Application’s discharge duration in $h$}
%\nomenclature[]{$\Delta t^{DD}_{app}$}{Application’s discharge duration in $h$}
\nomenclature[]{$N_{app}^{cyc}$}{Application’s number of annual cycles}

\nomenclature[]{$N^{avail}_{cha}$}{Number of available V2G chargers required}
\nomenclature[]{$N^{contr}_{cha}$}{Number of contracted V2G chargers required}
\nomenclature[]{$N^{P}_{cha} $}{Number of required V2G chargers for power}
\nomenclature[]{$N^{E}_{cha} $}{Number of required V2G chargers for energy}

\nomenclature[]{$N^{adj}_{HP} $}{Number of required HPs adjusted for activation frequency constraints}
\nomenclature[]{$P^{DR}_{t} $}{Available DR power reduction at time $t$ in $kW$}
\nomenclature[56]{$t/T$}{Index/Set of time intervals}
%\nomenclature[]{$n/N$}{Index/Set of years in LCODR project}


\nomenclature[]{$a_{GMC}$}{Guaranteed minimum charge in $\%$}
\nomenclature[]{$p^{WTA}_{base}$}{Consumers' median willingness-to-accept for base contract of 11.5 hours of RPT in $\$/month$}
\nomenclature[]{$p^{WTA}_{hour}$}{Consumers' median willingness-to-accept per hour of RPT in $\$/month$}
\nomenclature[]{$p^{WTA}$}{Consumers' median willingness-to-accept for DR heating schemes in $\$/month$}

\nomenclature[]{$P^{cap}_{cha}$}{Power capacity of EV charger in $kW$}
\nomenclature[]{$\eta_{cha}$}{Efficiency of charger}
\nomenclature[]{$P_{EV}^{max}$}{EV charger’s maximum power in $kW$}
\nomenclature[]{$P_{bat}^{max}$}{Battery’s maximum power in $kW$}
\nomenclature[]{$E_{app}$}{Application's shifted/discharged energy in $kWh$}
\nomenclature[]{$E_{EV}^{max}$}{EV battery’s energy capacity in $kWh$}
\nomenclature[]{$E_{EV}^{min}$}{EV battery’s minimum energy level in $kWh$}
\nomenclature[]{$E_{EV}^{arr}$}{EV battery’s energy level upon arrival in $kWh$}
\nomenclature[]{$E_{bat}^{max}$}{Battery’s energy capacity in $kWh$}
\nomenclature[]{$E_{bat}^{min}$}{Battery’s minimum energy level in $kWh$}
\nomenclature[]{$t_{EV}^{arr}$}{EV’s arrival time}
\nomenclature[]{$t_{EV}^{dep}$}{EV’s departure time}
\nomenclature[]{$f_{act}^{max}$}{Maximum frequency of activations in heat-pump-only DLC contract per month}
\nomenclature[]{$\Delta T_{b}$}{Maximum permitted temperature divergence in heat-pump-only DLC contract in $K$}
\nomenclature[]{$a^{V2G}_{f} $}{Availability factor for V2G (fraction of time that an EV is plugged in for)}
\nomenclature[]{$\Delta t^{day}_{cha}$}{Daily time required for charging EV in $h$}
\nomenclature[]{$N_{1D}$}{Number of required unidirectional DR assets}
\nomenclature[]{$P_{1D}^{avg}$}{Average shiftable power consumption of a unidirectional DR asset in $kW$}
\nomenclature[]{$\Delta Q_{b}$}{Building’s permitted heat divergence $kWh$}

\nomenclature[]{$Q_w$}{Thermal energy in hot water storage in  $kWh$}
%\nomenclature[]{$Q^{max}_w$}{Maximum heat level of hot water storage in  $kWh$}
%\nomenclature[]{$Q^{min}_w$}{Minimum heat level of hot water storage in  $kWh$}

\nomenclature[]{$P_{1D}^{cap}$}{Power consumption capacity of a unidirectional DR asset in $kW$}
\nomenclature[]{$a_{HC} $}{Fraction of EV’s driving energy charged at home}
\nomenclature[]{$E^{daily}_{drive} $}{Energy used by EV daily for driving in $kWh$}
\nomenclature[]{$t_{act}$}{Time of DR activation}
\nomenclature[]{$COP$}{Heat pump’s coefficient of performance}
\nomenclature[]{$SPF$}{Heat pump’s seasonal performance factor}
\nomenclature[]{$m_b$}{Building’s mass in $kg$}
\nomenclature[]{$c_{p,b}$}{Building’s specific heat capacity in $kJ/(kg\cdot K)$}
\nomenclature[]{$m_w$}{Hot water storage mass in $kg$}
\nomenclature[]{$c_{p,w}$}{Hot water specific heat capacity in $kJ/(kg\cdot K)$}
%\nomenclature[]{$\Delta P^{red}_{HP}$}{Possible reduction in heat pump’s power consumption in $kW$}
\nomenclature[]{$\Delta E^{elec}_{HP}$}{HP’s electrical energy needs for heating thermal storage in $kWh$}
\nomenclature[]{$\delta_w$}{Hot water storage density in $kg/m^3$}
\nomenclature[]{$V_{w}$}{Hot water storage volume in $m^3$}
\nomenclature[]{$\Delta T_w$}{Temperature range of hot water storage in $K$}
\nomenclature[]{$L$}{Wall thickness of hot water storage in $m$}
\nomenclature[]{$H$}{Residential ceiling height in $m$}
\nomenclature[]{$A$}{Area rented for hot water storage in $m^2$}

\nomenclature[]{$P_{HP}^{act,avg}$}{Average power use of activated HPs in $kW$}
%\nomenclature[]{$P_{HP}^{act}$}{Power consumption of single activated heat pump in $kW$}
\nomenclature[]{$\Delta P_{HP}^{red}$}{DR-related reduction in HP power in $kW$}
\nomenclature[]{$\Delta t_{RP}$}{Daily required plug-in time (RPT) in $h$}
\nomenclature[]{$\Delta t_{base}^{plug-in}$}{RPT for $p^{WTA}_{base}$ in $h$}

%\nomenclature[]{$P^{cap}_{app}$}{Area rented for hot water storage in $m^2$}
%\nomenclature[]{$N^{cap}_{app}$}{Area rented for hot water storage in $m^2$}
%\nomenclature[]{$A$}{Area rented for hot water storage in $m^2$}


\nomenclature[]{$p^E_t$}{Electricity spot price in time interval $t$ in $\$/kWh$}


%\nomenclature[]{$t/T$}{Set of time intervals}

\printnomenclature



\end{framed}

\end{table*}


DR schemes are often framed as a cost-efficient alternative to investment in energy storage because they promise lower capital costs \citep{siano2014demand}. This advantage may be counteracted by DR's requirement for continuous reward payments which increases operational costs \citep{parrish2019demand}. For cost assessments of electricity generation and storage, capital and operational expenditure are often combined in the levelised cost of electricity (LCOE) or storage (LCOS). These represent the average annuitised cost per unit of electricity (generated or stored). DR, by definition, shifts electricity demand temporally, thereby creating a similar effect on the electricity grid as grid-connected energy storage. As such, a measure for the levelised cost of demand response (LCODR), analogous to the LCOS is proposed in this paper, providing a combined measure of investment and operational expenses for DR. %This will enable the cost-comparison between different DR schemes as well as between DR and storage.

\begin{figure*}
\begin{equation}
    LCODR[\frac{\$}{MWh}] = \frac{Investment + \sum^{T}_{t}\frac{O\&M}{(1+r)^t} + \sum^{T}_{t}\frac{Rewards}{(1+r)^t} + 
    \sum^{T}_{t}\frac{Rebound}{(1+r)^t} +
    \frac{End-of-life\; costs}{(1+r)^{T+1}}}{\sum^{T}_{t}\frac{E_{app}}{(1+r)^t}}\label{eqb1}
\end{equation}
\end{figure*}
%\end{strip}

Vehicle-to-grid (V2G) technology, which will be considered a type of DR in this paper, has recently been included in extended LCOS estimations \citep{geng2024assessment,geng4872108techno,rahman2023development}. These estimations simply extend the LCOS methodology to the techno-economic features of V2G, though, and do not consider that DR schemes like V2G exhibit crucial differences to storage technologies. The recently proposed levelised cost of energy flexibility (LCOEF) \citep{zang2025levelized} is, to the best of the authors' knowledge, the only measure that aims to compare the annuitised lifetime per-unit costs of different DR schemes. However, it also focuses mostly on the techno-economic features and does not consider survey-informed consumer reward payments in detail. None of the studies mentioned above \citep{geng2024assessment,geng4872108techno,rahman2023development,zang2025levelized} develop a methodology to address the availability profiles of different DR schemes and how they may affect their value. The LCODR proposed in this work improves upon this by adjusting the LCOS methodology to account for two key differences with DR: firstly, DR schemes require consumer participation which necessitates paying participation rewards, something that grid-scale storage naturally does not require. This difference is addressed in this paper by including an additional "reward payments" term in the LCODR equation. The second key difference is that DR exhibits variations in availability whereas storage technologies tend to be continuously connected. Flexibility from V2G for example will only be available when the corresponding EV is plugged in. This difference is analogous to the difference in availability between VRE sources like wind or solar and continuously available conventional generators like gas turbines. Numerous studies have developed methods to adjust the LCOE for this difference in availability \citep{simpson2020cost}. One of these methods, the value factor from \citet{hirth2013market} is adapted in this paper to address the intermittent availability of DR assets.

%third difference: the value of charging (disregarded in LCOS)

The remainder of this paper is organised as follows: Section \ref{sec:LCODR} formally introduces the LCODR as well as the DR schemes and applications for which it is calculated. Section \ref{sec:method} presents the methodology for estimating the LCODR of various pairings of DR schemes and applications, including the estimation of the value factor to adjust for variable availability. Section \ref{sec:data} describes the data that was used for a case study of the levelised DR costs in the context of the British electricity system. Section \ref{sec:res_dis} presents the results, contrasts them with storage costs and discusses their limitations. Section \ref{sec:concl} concludes the paper.

\section{Levelised cost of demand response} \label{sec:LCODR}



Borrowing from the definition of the LCOS \citep{schmidt2019projecting}, the levelised cost of demand response is the total lifetime cost of a DR scheme per unit of cumulative shifted electricity demand. Shifted electricity is always measured as the delivered reduction in electricity demand rather than the increase in demand at other times. This is in line with the definition of the LCOS which measures costs per unit of discharged electrical energy \citep{julch2016comparison}. In the case of V2G, shifted electricity can also extend to electricity delivered back to the grid. The computation of the LCODR in energy terms is shown in Equation (\ref{eqb1}). For applications that value available power rather than delivered energy, the LCODR in power terms, shown in \ref{app1}, may be preferred. The LCODR terms in the numerator are identical to that of the LCOS, except for two differences. Firstly, the charging costs are replaced by rebound costs which describe the cost of supplying electricity for the shifted electricity demand. Secondly, an additional reward payments term is introduced in Equation (\ref{eqb1}) to reflect the need for financial compensation to consumers.\\

Figure~\ref{fig:ex} shows how a single discharge cycle can be provided using a battery, a V2G-capable vehicle, or a unidirectional smart charger. Storage and DR assets can be used to service a variety of applications, such as energy arbitrage or various ancillary services. The LCOS is normally estimated individually for each potential application. Any application that is centred on active power delivery, can be characterised by four main features: response time, rated power, discharge duration, and the annual number of discharges \citep{schmidt2019projecting}. Response time is a prerequisite for a technology to be able to service a certain application. The other three features strongly impact the required design of a storage or DR asset, and thereby greatly influence the resulting levelised costs. 

\begin{figure}[t]
\centering
\includegraphics[width=0.5\linewidth]{output-figure0.pdf} %[width=0.5\linewidth]
\caption{Energy and power profiles for servicing an energy storage application with different technologies. Note that the minimum energy level for EVs ($E_{EV}^{min}$) is their energy capacity ($E_{EV}^{max}$) multiplied by the guaranteed minimum charge ($f_{GMC}$): $E^{min}_{EV} = E^{max}_{EV} \times f_{GMC}$}
\label{fig:ex}
\end{figure}

The power and energy capacities of storage technologies can be designed relatively independently to meet the characteristics of a specific application. The domestic DR assets that are considered in this paper, on the other hand, are often much more constrained in their power-to-energy capacity ratio. For V2G and smart charging, for example, the energy capacity is very difficult to change for a single vehicle/charger, because this would imply redesigning the EV's battery. The power capacity of the charger would be easier to alter but is usually limited by the power rating of a household's grid connection. Charger power and the energy capacity of an EV's battery are therefore considered unchangeable in the rest of this paper. The only DR scheme where the power-to-energy capacity ratio is not considered set is the combination of heat pump and hot water tank (HP + thermal storage) where the tank's volume is independent of the power capacity.

With DR's fragmented business case, different elements of the costs are incurred by different stakeholders, meaning it is important to specify the perspective of the cost assessment. The proposed LCODR approach takes the perspective of the beneficiary from responsive consumers, rather than the consumers themselves. In recent literature, this beneficiary is often referred to as an "aggregator" which could be an energy supplier or an intermediary that sells flexibility from DR to grid operators, generators or suppliers. The proposed DR schemes all fall under the DR category of direct load control (DLC), meaning they do not require consumers to directly engage with a price signal. Instead, consumers are paid monthly compensation for giving away some of the control over one of their assets. In the assessed DR scenarios,  installing the required infrastructure is considered a cost to the aggregator, meaning consumers have it installed at no extra cost to them as part of their DR contract.

\subsection{Demand response schemes and applications}

\begin{figure}
    \centering
    \includegraphics[width=0.5\linewidth]{scheme_illustration_cropped.pdf} %[width=0.5\linewidth]
    \caption{Illustration of the components and estimation process of the availability profile-adjusted LCODR ($LCODR_{VF}$)}
    \label{fig:illu_schm}
\end{figure}

The levelised costs for four different DR schemes are considered in this paper: V2G, EV smart charging, smart heat pumps and heat pumps with thermal storage. They are illustrated in Figure~\ref{fig:illu_schm}  and their details are listed below.\begin{enumerate}[leftmargin=*]
    \item  \textbf{V2G} describes the delivery of power from EVs to the grid and requires the installation of a bidirectional charger. Consumers on this contract indicate their departure time at which the aggregator ensures a fully charged battery. Before departure, the aggregator is free to discharge the battery to a specified guaranteed minimum charge (GMC) which the consumer reserves for unexpected journeys. Consumers commit to fulfilling a daily required plug-in time (RPT) and receive a monthly reward for making their vehicle available for V2G. 
    \item  \textbf{Smart charging} contracts work much in the same way as V2G except that they work with unidirectional chargers. Rather than discharging, smart charging allows the aggregator to shift the charging time within the plugin period (again while ensuring a full battery at the indicated departure).
    \item The heat pump schemes also work by shifting electricity consumption temporally. The  \textbf{smart heat pump} scheme uses consumers' homes as a thermal battery while the heat pump with thermal storage uses home-installed hot water storage tanks instead. Consumers on the smart heat pump scheme specify a maximum temperature setting divergence ($\Delta T_b$) which allows the aggregator to temporarily increase their house's temperature above or below the specified setting. The aggregator can only do this for a specified maximum number of activations per month ($f_{act}^{max}$). 
    \item The  \textbf{heat pump with thermal storage} tariff installs a heat storage tank within the consumer's home which will be entirely controlled by the aggregator while ensuring the heating preferences of the consumer are met without exception. This scheme is somewhat innovative in that it potentially does not impact consumers' lives at all (not even having to indicate preferences like departure times). For that reason, no conventional reward payments are considered for this tariff. Instead, consumers are compensated for the space that the hot water tank takes up (the installation of which is already paid for by the aggregator).
\end{enumerate}

The 12 storage applications from \citep{schmidt2019projecting} are considered in this paper as potential applications for DR. Table \ref{tab:my_label} shows the suitability of the DR schemes for the 12 applications. Response times should not be an issue for any of the DR technologies with \citet{Zhang2022} and \citet{lee2020providing} showing that even the pairing of heat pumps with primary response is possible.
Black start, power reliability and power quality services require active power delivery rather than just load shifting, so unidirectional DR assets are not suitable for this. Seasonal storage requires long discharge durations that cannot be serviced by any of the DR assets.
\begin{table}[htbp]
    \centering
    \resizebox{0.5\linewidth}{!}{%
    \begin{tabular}{l|c c c | c c c c}
         \hhline{========}
          & \makecell[tl]{$P_{app}^{cap}$ \\ MW}& \makecell[tl]{$\Delta t_{DD}$ \\ h} & $N^{cyc}_{app}$ & \makecell[tl]{V2G} & \makecell[tl]{Smart\\charging} & \makecell[tl]{Heat\\pump} & \makecell[tl]{HP \& \\thermal\\storage}\\
         \hline
         Energy arbitrage & 100 & 4 &300&\checkmark & \checkmark & \checkmark & \checkmark \\
         Primary response & 10 & 0.5 &5000& \checkmark & \checkmark & \checkmark & \checkmark \\
         Secondary response & 100 & 1 &1000& \checkmark & \checkmark & \checkmark & \checkmark \\
         Tertiary response & 100 & 4& 10 & \checkmark & \checkmark & \checkmark & \checkmark \\
         Peaker replacement & 100 &4& 50& \checkmark & \checkmark & \checkmark & \checkmark \\
         Black start & 10 & 1 & 10 &\checkmark & & & \\
         Seasonal storage & 100 & 700 & 3 & & & \\
         T\&D invest. deferral & 100 & 8 & 300 & \checkmark & \checkmark & \checkmark & \checkmark \\
         Congest. management &100& 1& 300 & \checkmark & \checkmark & \checkmark & \checkmark \\
         Bill management &1& 4 & 500 & \checkmark & \checkmark & \checkmark & \checkmark \\
         Power quality &1&0.5& 100& \checkmark &  & & \\
         Power reliability &1&8& 50&\checkmark &  &  &  \\
         \hhline{========}
    \end{tabular}}
    \caption{Suitability of DR schemes for storage applications including application requirements from \citep{schmidt2019projecting}}
    \label{tab:my_label}
\end{table}

\section{Methodology} \label{sec:method}
In the following, a consumer who is enrolled in a DLC scheme and has the required infrastructure installed will be labelled a "DR asset". To identify the cost of delivering a certain service, it is crucial to identify the number of DR assets that are required to be available to deliver said service (Section~\ref{num_ass}). The number of assets and their reward payments can further be affected by the required discharge duration (Section~\ref{dd_cons}) and the number of annual activations (Section~\ref{cons_nc}). To adjust the LCODR for differences in availability profile, it is divided by the value factor (Section~\ref{Availability adjustments}). To account for uncertainties in all of the input data, a Monte-Carlo simulation is carried out (Section~\ref{mc_sim}). Figure~\ref{fig:illu_est} illustrates the estimation process for the availability profile-adjusted LCODR. It differentiates input parameters and intermediate variables. Input parameters are either from the data sources detailed in Section~\ref{sec:data} or dictated by the requirements of the application for which the LCODR is being estimated (see Table~\ref{tab:my_label}). 


\begin{figure}
    \centering
    \includegraphics[width=0.5\linewidth]{calc_illustration_cropped.pdf}
    \caption{Illustration of the components and estimation process of the availability profile-adjusted LCODR ($LCODR_{VF}$)}
    \label{fig:illu_est}
\end{figure}

\subsection{Estimating the number of required demand response assets} \label{num_ass}
 The methodology for identifying the number of required assets varies between V2G and unidirectional DR schemes.

\subsubsection{Vehicle-to-grid}
Given that the chargers' power capacity and the EVs' energy capacity are considered immutable, the number of available chargers required, $N^{avail}_{cha}$, is either dictated by the power, $P_{app}^{cap}$, or the energy, $E_{app}$, requirements, depending on the application. To make sure that both energy and power requirements are met, the higher number between the energy-mandated ($N_{cha}^{E}$) and power-mandated ($N_{cha}^{P}$) number of chargers has to be chosen.

\begin{equation}
    N^{avail}_{cha} = max(N_{cha}^{P},N_{cha}^{E}) \label{eq2}
\end{equation}

The number of available chargers required to provide the needed power ($N_{cha}^{P}$) is simply the power capacity required by the application ($P_{app}^{cap}$) divided by the effective power capacity of an individual bidirectional charger($P^{cap}_{cha} \times \eta_{cha}$).

\begin{equation}
    N_{cha}^{P} = \frac{P^{cap}_{app}}{P^{cap}_{cha} \times \eta_{cha}} \label{eq3}
\end{equation}

Similarly, the number of available chargers needed to provide the required energy capacity ($N_{cha}^{E}$) is obtained by dividing an application's required energy capacity ($E_{app}^{cap}$) by the available energy capacity of a singular EV ($E_{EV}^{max} - E_{EV}^{min}$). The available energy capacity of a single EV is dictated by its battery's capacity ($E_{EV}^{max}$) and the guaranteed minimum charge ($a_{GMC}$) which is the minimum level of energy that a V2G contract commits to always leaving in a vehicle's battery. The needed energy capacity for an application is the product of power capacity ($P_{app}^{cap}$) and discharge duration ($\Delta t_{DD}$).

\begin{equation}
    N_{cha}^{E} = \frac{E_{app}}{E_{EV}^{max} - E_{EV}^{min}} = \frac{P_{app}^{cap} \times \Delta t_{DD}}{E_{EV}^{max} \times (1-a_{GMC})}
\end{equation}

The estimations above are for the required number of available chargers. A key difference between DR and storage is that DR assets are not always available (e.g. V2G-capable EVs are not always plugged in). To obtain a cost estimate, it is crucial to obtain the number of contracted chargers that are required to ensure that on average there are enough available chargers. The number of available chargers is a product of the number of contracted chargers and the availability factor ($a^{V2G}_f$) which is the fraction of time that each charger is available for. This relationship can be rearranged to calculate the number of contracted chargers required

\begin{equation}
    N^{contr}_{cha} = \frac{N^{avail}_{cha}}{a^{V2G}_f}
\end{equation}

The available time per day for bidirectional charging is simply the average daily time that the vehicle is plugged in for, and not charging. The amount of time that EVs will be plugged in for, if drivers get paid for plugging in, is difficult to estimate because few such schemes exist today. Many V2G choice experiments \citep{huang2021electric,lee2020willingness} include a minimum daily RPT term, though. It is assumed that consumers plug in for as long as they are required by their contracts ($\Delta t_{RP}$), and no longer. The daily charging time ($\Delta t^{day}_{cha}$) can be determined from the effective charger power ($P^{cap}_{cha} \times \eta_{cha}$) and the average daily energy used for driving ($E^{daily}_{drive}$). $f_{hc}$ is the fraction of the latter which is charged at the domestic V2G charger.

\begin{equation}
    a^{V2G}_f = \frac{\Delta t_{RP} - \Delta t^{day}_{cha}}{24h} = \frac{\Delta t_{RP} - \frac{E_{drive}^{day} \times a_{HC}}{\eta_{cha} \times P_{cha}^{cap}}}{24h}
\end{equation}

%unidirectional DR assets are only available for flexibility when controlling their consumption creates a net reduction in electricity demand. This is only the case when the assets would be "on" if uncontrolled. Assuming assets run at full power, the availability factor for unidirectional DR assets can be calculated as follows.

%\begin{equation}
%    a^{1d}_{factor} = \frac{t^{daily}_{on}}{24h} = \frac{\frac{E_{consumption}^{daily}}{P^{asset}_{cap}}}{24h}
%\end{equation}

Dividing by the availability factors only means that the required capacity will be available on average. While this ensures that on average the DR assets can provide the same service as storage, their profile may vary quite significantly from the flat profile of a continuously available storage asset. Section~\ref{Availability adjustments} introduces a methodology to adjust for an uneven availability profile.

\subsubsection{Unidirectional demand response} 
In contrast to V2G, unidirectional DR schemes have the number of required available assets dictated only by power requirements. This is because the available power is dictated by the uncontrolled power consumption that can be moved by DR. This consumption also dictates the energy that can be moved, meaning that meeting the power requirements also ensures that the energy requirements are met.
Equation (\ref{eq5}) gives the number of assets ($N_{1D}$) required to fulfil the power requirements which is similar to that of V2G from Equation (\ref{eq3}).

\begin{equation}
    N_{1D} = \frac{P_{app}^{cap}}{P_{1D}^{avg}} \label{eq5}
\end{equation}

Unidirectional DR assets are only available when they would otherwise be consuming in an uncontrolled scenario, meaning usually more need to be contracted than in the V2G scenario to ensure enough of them are available. No availability factor has to be applied to Equation (\ref{eq5}) because utilising the average power is equivalent to adjusting the power capacity for its availability (i.e. $P_{1D}^{avg} = P_{1D}^{cap} \times a^{1D}_f$).



%The time for which an EV is charging (and therefore available) is dictated by its energy consumption. The energy capacity of a smart charging EV is also dependent on the energy consumption, meaning that the number of chargers that are required to fulfil the energy requirements is always the same as the number of chargers required to fulfil the power requirements.

%This requires some more explanation

%\begin{equation}
%    N_{chargers}^{Energy} = \frac{E^{app}_{cap}}{\frac{E_{drive}^{daily}}{\eta_{charger}}} = \frac{P^{app}_{cap} \times DD}{\frac{E_{drive}^{daily}}{\eta_{charger}}}
%\end{equation}

%\begin{equation}
%    N_{Chargers}^{Power} = \frac{P^{app}_{cap}}{P^{Charger}_{cap}}
%\end{equation}

%\subsection{Availability fraction} \label{avail frac}

\subsection{Discharge duration constraints}  \label{dd_cons}
Even though unidirectional DR assets do not discharge energy, the term discharge duration will be used throughout this paper to denote the duration of activation, i.e. duration of lowered electricity consumption. The discharge duration is limited by the rebound for unidirectional DR assets (recharge for V2G). This is because the energy that is required to fulfil the rebound requirements would count against the DR activation and thereby annul it, even if other assets were continuing the demand reduction. Figure~\ref{fig:exDD} shows how the rebound limits the discharge duration for the different DR assets.

\begin{figure}[t]
\centering
\includegraphics[width=0.5\linewidth]{output-figure36.pdf}
\caption{Discharge duration limits illustration on the energy profiles of different DR assets. Note that the heat pump-only profile (3\textsuperscript{rd} graph) shows an activation that reduces the heat pump's power consumption to zero ($ \Delta P^{red}_{HP} = P^{act,avg}_{HP}$)}
\label{fig:exDD}
\end{figure}

\subsubsection{Vehicle-to-grid}
For both EV-based DR schemes, the discharge duration depends on the average duration for which vehicles are plugged in. While data exists for current plug-in durations \citep{dftEVdata}, this does not take into account that plug-in durations will likely be longer when consumers are financially incentivised to plug in for longer. Consumer survey data for the effect of financial incentives on daily RPT is available both for V2G (e.g. \citep{huang2021electric,parsons2014willingness,lim2016assessment}) and smart charging \citep{fabianek2024willing}. RPT states the number of hours per day that an EV has to be plugged in for and can only be translated into the average plug-in duration by assuming a plug-in frequency. An EV that has an average daily plug-in time of 20 hours, for example, will have an average plug-in duration of 10 hours if its plug-in frequency is twice per day. In the following, an average plug-in frequency of once per day is assumed, resulting in the average plug-in duration being equal to the daily RPT ($\Delta t_{PD} =\Delta t_{RP}$). \\

For V2G, the discharge may start at any point during the time that an EV is available for V2G services which will be simulated by assuming that discharge starts halfway through the plug-in period. This means that, during actual operation, the discharge duration may be constrained to be shorter or allowed to be longer, but on average it will be constrained by Equation (\ref{eq9}).

\begin{equation}
    \Delta t_{DD}^{max} = \frac{\Delta t_{PD} - \frac{E^{daily}_{drive} \times a_{HC}}{\eta_{cha}\times P^{cap}_{cha}}}{2} - \frac{E^{max}_{EV} - E^{min}_{EV}}{\eta_{cha}\times P^{cap}_{cha}} \label{eq9}
\end{equation}

 The time needed for charging driving energy needs ($E^{daily}_{drive} \times a_{HC}$) and recharging any V2G-discharged energy ($E^{max}_{EV} - E^{min}_{EV}$) cannot be used for V2G, and therefore has to be subtracted.

%$E^{daily}_{drive}$ denotes the daily energy that the EV uses for driving and $a_{hc}$ is the fraction of that which is charged at the domestic V2G charger, meaning their product returns the total home-charged energy. Dividing this by the effective charging power ($P^{cap}_{cha} \times \eta_{cha}$) produces the time spent on charging the energy needed for driving which naturally cannot be used for V2G discharging. The energy that is discharged via V2G ($E^{max}_{EV} - E^{min}_{EV}$) also has to be recharged and the time for this is also unavailable for V2G discharging while being similarly calculated by dividing by the effective charging power ($P^{cap}_{cha} \times \eta_{cha}$).

\subsubsection{Smart charging}
The activation of flexibility from smart charging can only occur when the EV would otherwise be charging, which, in an uncontrolled charging scenario, is only the case for a limited time window after plugging in. The remaining time therefore acts as an upper constraint for the discharge duration, because afterwards, the previously delayed energy will have to be charged. 

\begin{align}
    %DD < RPT - t_{charge}\\
    \Delta t_{DD}^{max} = \Delta t_{PD} - \frac{E^{daily}_{drive} \times a_{HC}}{P_{cha}^{cap} \times \eta_{cha}}
\end{align}


Note that several EVs are needed to achieve a discharge duration that is longer than the uncontrolled charging time. Once an EV is forced to recharge because its departure is near, however, the recharge would work against the effect of the next EV's charging delay, turning the plug-out time into an upper limit for the discharge duration.

\subsubsection{Smart heat pump}
For the heat pump-only scheme, the discharge duration is limited by the allowed temperature divergence ($\Delta T_b$) and the building's heat capacity ($m_b c_{p,b}$) which is the amount of thermal energy it takes to heat the building by one degree Celsius. Together these dictate the possible difference in the building's thermal energy ($\Delta Q_b$) which can be used to deduct the discharge duration as shown in Figure~\ref{fig:exDD}. Note that Figure~\ref{fig:exDD} shows a case in which the heat pump is used to actively heat up the building but if the heat pump only maintained the temperature to make up for heat losses, the inferred relationships would remain the same. The difference in the red and blue slopes is the reduction in the effective heating rate which is the product of the reduction in heating power and the seasonal performance factor ($\Delta P_{HP}^{red} \times SPF$). The $SPF$ is an annual power-weighted average of the Coefficient of Performance ($COP$) which measures the heat output of a heat pump per unit of electrical energy input. Dividing the difference in electrical energy input ($E^{elec}_{HP}$) by the reduction in heating power ($\Delta P_{HP}^{red}$) gives the duration associated with a certain temperature threshold ($\Delta T_b$). This duration is doubled to obtain the maximum discharge duration ($\Delta t^{max}_{dd}$) because the temperature setting can be exceeded by $\Delta T_b$ before activation and subsided by $\Delta T_b$ afterwards.

\begin{equation}
    %DD < 2 \times \frac{P_{avg}}{\frac{\Delta Q}{COP}}
    %\resizebox{.5\linewidth}{!}{
    \Delta t^{max}_{DD} = 2\times \frac{\Delta E^{elec}_{HP}}{\Delta P_{HP}^{red}} = 2\times \frac{\Delta Q_b}{\Delta P_{HP}^{red}\times {SPF}} = 2 \times \frac{m_b c_{p,b} \Delta T_b}{\Delta P_{HP}^{red} \times {SPF}}%$}
\end{equation}
 
The possible reduction in heating power decreases as discharge duration goes up because longer discharge durations require longer reductions in heating power which can only remain within the tolerated temperature divergence ($\Delta T_b$) if the heating power is reduced by less. It would be impossible for the power reduction to exceed the heat pump's active power consumption which is accounted for in Equation (\ref{eqa}).

\begin{equation}
\Delta P_{HP}^{red} \leq P^{act,avg}_{HP} \label{eqa}
\end{equation}

$P_{HP}^{act,avg}$ denotes the average power consumption of activated heat pumps, i.e. the average power consumption when a heat pump is turned on. Even though heat pumps are a unidirectional DR asset, $P_{HP}^{act,avg} \neq P_{1D}^{avg}$ because the former denotes the average non-zero power whereas the latter includes times when the heat pump is off.

%For sufficiently short discharge durations at which Equation (\ref{eqa}) applies, the heating power reduction is therefore capped at $P^{act}_{avg}$. 

\subsubsection{Heat pump and thermal storage}
For the tariff that includes thermal storage, the discharge duration is limited by how long the thermal storage can sustain the heat supply that would otherwise have been supplied by the average power consumption of the heat pump when switched on ($P^{act,avg}_{HP}$). Since the heat storage can be designed to always meet the entire heating demand, a DR activation in this scheme always reduces the power demand to zero ($\Delta P_{HP}^{red} = P^{act,avg}_{HP}$), to minimise the number of required assets under contract.

\begin{equation}
    \Delta t_{DD}^{max} = \frac{\Delta E^{elec}_{HP}}{\Delta P_{HP}^{red}} = \frac{Q_{w}^{max} - Q_{w}^{min}}{P^{act,avg}_{HP} \times {SPF}} = \frac{m_{w}c_{p,w}\Delta T_w^{max}}{P^{act,avg}_{HP} \times SPF}
\end{equation}

The mass of water within the thermal storage can be calculated from its density and the volume of the storage. It is assumed that the hot water storage tanks are cylindrical but that reward payments have to be paid for the associated square area on which the cylinder sits. Equation \ref{eq13} expresses water mass as a function of the rented area ($A$).

\begin{equation}
    m^w = \delta_wV_{st} = \delta^w((H - 2L)\times (\pi \times (\frac{\sqrt{A}}{2} - L)^2))\label{eq13} 
\end{equation}

Here, $H$ is the ceiling height and $L$ is the insulation wall thickness.

%%% explain heat losses

\subsection{Constraints for the number of cycles}  \label{cons_nc}
The smart heat pump scheme (without thermal storage) is the only one that limits the number of annual cycles. This is because an activation affects consumers directly as it influences the temperature within their home. The allowed number of weekly activations is a key contract term for the heat pump scheme and if there are more cycles than can be provided by a single contracted heat pump, more heat pumps are needed following the relation shown in Equation (\ref{eq14}).

\begin{equation}
    N^{adj}_{HP} = \frac{f_{act}^{max}\times 12}{N_{app}^{cyc}} \times N_{1D}  \label{eq14}
\end{equation}

Here, $N^{adj}_{HP} $ denotes the number of contracted heat pumps once adjusted for the number of allowed activations. This is expressed as a function of the unadjusted number of heat pumps ($N_{1D}$), the maximum number of activations per month ($f_{act}^{max}$), and the annual cycle requirement of the storage application ($N_{app}^{cyc}$).

\subsection{Value factor} \label{Availability adjustments}
The value of electricity is strongly influenced by the time at which it is made available. When assessing the cost of VRE, this issue was quickly realised and addressed by a broad body of literature \citep{simpson2020cost,hirth2013market,joskow2011comparing,ueckerdt2013system,loth2022we}. With LCOS being equivalent to LCOE for storage technologies, DR assets can be considered to behave analogous to VRE generation in that they vary in their availability. The methods that adjust for the variability of VRE can therefore be applied to address the variable availability of DR. In this paper, the value factor from \citep{hirth2013market} is adapted. To obtain this for a given VRE asset, its generation profile is multiplied by the energy price profile and then divided by the same product for a flat generation profile. The same procedure is applied for DR, with the main difference being that rather than generated electricity, the potential for electricity demand reduction (or discharge in the case of V2G) is measured, denoted in Equation (\ref{eq15}) as $P^{DR}_t$.

\begin{equation}
    VF = \frac{\sum_{t=1}^{t=T}p_t^EP^{DR}_t}{\sum_{t=1}^{t=T}p_t^E\frac{\sum_{t=1}^{t=T}P^{DR}_t}{T}} \label{eq15}
\end{equation}

%double check price (but I think it's correct)

For unidirectional DR assets $P^{DR}_t$ is simply the power that would be consumed in an uncontrolled scenario in time period $t$ and can be moved away. For V2G, $P^{DR}_t$ is the available charger power that can be discharged in time period $t$. When the energy requirements dictate the number of V2G chargers (i.e. $N_{cha}^E > N_{cha}^P$), $P^{DR}_t$ has to be replaced by $E^{DR}_t$ which is the energy available for discharging in time period $t$. To assess the power and energy profiles for V2G, the virtual battery from \citet{thran2024reserve} was used, with $P^{DR}_t$ being inferred from the power boundary and $E^{DR}_t$ being the difference between the energy boundaries. A value-adjusted LCODR ($LCODR_{VF}$) can be computed by dividing the original cost estimate by the value factor as shown in Equation (\ref{eq16}).

\begin{equation}
    LCODR_{VF} = \frac{LCODR}{VF} \label{eq16}
\end{equation}

\subsection{Monte-Carlo simulation}  \label{mc_sim}
To reflect the uncertainty of the input parameters that are presented in the following section, a Monte Carlo simulation is implemented. The methodology for this largely follows that of \citet{schmidt2024monetizing} with the number of drawn samples set to 1000. Monte Carlo samples are taken from a normal distribution that is truncated at 1.285 standard deviations above and below the mean input value. The mean input values for different parameters are given in the various tables in Section~\ref{sec:data}. Standard deviations are assumed to be at 33\% for all inputs except for the highly uncertain parameters which are assumed to be subject to a standard deviation of 67\%. The latter applies to the V2G charger cost, the investment costs required for thermal storage, and all consumer reward terms. For the estimated value factors, a standard deviation of 10\% is assumed to reflect the possible effect of increased VRE generation or a potential price cannibalism effect caused by higher penetrations of storage or DR. The uncertainty in the value factors is expected to be relatively low because some established electricity price profile patterns (e.g. expensive evening hours) would have to significantly change for them to vary. 

\section{Data}\label{sec:data}
Differences in socio-economic patterns, car usage, and climate mean that the LCODR is better computed for specific regions rather than globally. The presented data in this section is for a case study of the LCODR in Great Britain. Because of the low prevalence of domestic DLC schemes, data sources are not always available for Britain so data was also taken from studies that were conducted in regions with similar socio-economic structures. In general, the lack of accessible DLC consumer engagement data has meant that data sometimes had to be repurposed out of context, reducing the robustness of the results. 

\subsection{Investment costs}
Capital costs for the different DR schemes are given in Table~\ref{tab:inv_costs}. Note that all investment costs measure the upgrade from an existing domestic charging/heating set-up, as consumers are expected to install this regardless of DR participation. The first three  DR schemes all require investment in only one piece of infrastructure (first three rows of Table~\ref{tab:inv_costs}). Heat pumps with thermal storage, however, require both a smart thermostat and hot water storage. Advanced metering infrastructure is also a prerequisite for any of the DR tariffs and can come at a considerable cost \citep{kaczmarski2022determinants}. However, many countries, including the United Kingdom, have pledged to roll out smart meters, meaning this will likely not be a cost to the DR aggregator \citep{smartmeter} and is therefore not considered in this study.


\begin{table}[htbp]
    \centering
    \resizebox{0.5\linewidth}{!}{%
    \begin{tabular}{l l l l}
        \hhline{====}
         Parameter & Unit & Value & Source \\
         \hline
         V2G charger cost & \$/charger (7.4kW) & 3,000 & \citep{geng2024assessment,geng4872108techno} \\
         Smart charger cost & \$/charger (7.4kW) & 107 & \citep{heilmann2021much} \\
         Smart thermostat & \$/thermostat & 85 & \citep{schauble2020conditions} \\
         Hot water storage & \$/m\textsuperscript{3} & 2042 & \citep{odukomaiya2021addressing}\\ %may need better source 
         \hhline{====}
    \end{tabular}}
    \caption{Investment costs}
    \label{tab:inv_costs}
\end{table}

\subsection{Reward payments}
Accurately estimating reward payments is a difficult aspect of the LCODR data collection. Few commercial DR schemes exist for the four variants considered in this paper so consumers' preferences cannot be inferred from them, especially when considering their choices regarding specific contract settings like RPT for the EV-based schemes. When consumers' revealed preferences cannot be observed by looking at their purchasing habits (i.e. enrolment data for existing commercial DR schemes), consumers' stated preferences can instead be captured through discrete-choice experiments (DCEs) \citep{samuelson1948consumption}. DCEs ask participants to choose between hypothetical alternative scenarios of a good or service with different attributes. The relative importance of these attributes can then be inferred from their choices \citep{mangham2009or} and can be translated into the willingness-to-pay (WTP) for an attribute. For DLC schemes, where the aggregator pays the consumers, the equivalent measure for WTP is the willingness-to-accept (WTA) which states the amount of monthly financial reward participants associate with a change in the terms of the DR contract. DCEs have been carried out to estimate the WTA for EV-based DR attributes like GMC or RPT for V2G \citep{huang2021electric,lim2016assessment,bailey2015anticipating,geske2018willing} and smart charging tariffs \citep{fabianek2024willing,yilmaz2020analysis}. While giving a good estimate of, for example, the WTA per additional hour of RPT($p_{hour}^{WTA}$), they do not measure the "base-WTA" ($p_{base}^{WTA}$) that consumers need to receive to sign up for a flexible charging tariff over their base contract that allows them uncontrolled charging. There are only a couple of studies that set out to measure $p_{base}^{WTA}$ for V2G over uncontrolled charging \citep{baumgartner2022does,ensslen2018incentivizing} and there are, to the best of the authors' knowledge, no studies that measure $p_{base}^{WTA}$ of choosing a smart charging tariff over uncontrolled charging. To use consistent data sources, and since there were no academic sources for the smart charging "base-WTA", commercially available smart charging  \citep{octopus_ev,ovo_ev,edf_ev} and V2G \citep{octopus_powerpack} tariffs were used instead. These are often given as reductions on the energy price for EV charging which are converted to monthly reward payment costs by multiplying them with the daily driving energy ($E^{daily}_{drive}$), the fraction of that charged at home ($a_{HC}$), and the consumer energy price \citep{ofgem_pricecap}. Table~\ref{tab:rewards} shows the reward payment parameters that are ultimately used to estimate the LCODR. The WTA per hour of RPT is taken from DCEs in various countries. The reward payments per charger for a given RPT can be calculated by using the base cost and adding the WTA per hour of RPT multiplied by the additional number of hours. For an exemplary smart charging contract with an RPT of 15 hours, the reward payment estimation is shown in Equation (\ref{eq17}).

\begin{equation}
    Rewards = p^{WTA}_{base} + (\Delta t_{RP} - \Delta t^{plug-in}_{base}) \times p^{WTA}_{hour} = 33.1 + (15 - 11.5) \times 19.3 = 100.7 \$/DR\ asset
    \label{eq17}
\end{equation}

To ensure that Equation (\ref{eq17}) does not produce negative reward payments, a minimum monthly reward of \$5 was assumed for Smart charging and heat pumps with thermal storage. A minimum of \$10 was assumed for V2G.
Regarding reward payments for the heat pump tariff, data from a controlled air conditioning DCE in the US is repurposed under the assumption that temperature differences cause the same discomfort, be it for cooling or heating. All DCE data represents the median WTA, i.e. the reward payment at which half of the sample population is expected to participate in a DR scheme. At low penetrations of DR, WTA may be significantly cheaper because consumers with a lower WTA (e.g. early adopters) could be targeted first.


\iffalse
\begin{table}[htbp]
    \centering
    \resizebox{0.5\linewidth}{!}{%
    \begin{tabular}{l l l l}
        \hhline{====}
         DR Scheme & Unit & Value & Range & Source \\
         \hline
         V2G & \$/month/charger & 46.4 & \citep{octopus_powerpack}\\ %this includes 30% GMC and RPT of 12 hours "every couple of days"
          & \$/month/hour(RPT) & 27.2 & \citep{parsons2014willingness,huang2021electric,lim2016assessment}\\
          % & \$/Month/\%(GMC) & 6.3 & \citep{bailey2015anticipating,geske2018willing,kubli2018flexible,parsons2014willingness,huang2021electric,mehdizadeh2023estimating}\\
         Smart charging & \$/month/charger & 33.1 & \citep{octopus_ev,ovo_ev,edf_ev}\\ %octopus, OVO and EDF
          & \$/month/hour(RPT) & 19.3 & \citep{fabianek2024willing}\\ %check whether results by selin yilmaz & paul fabianek can be used here
         Heat pump & \$/month/thermostat & 9.5 &\citep{kaczmarski2022determinants}\\
         HP \& storage & \$/month/m\textsuperscript{2} & 13.6  & \citep{housing2024,team_2024}\\
         \hhline{====}
    \end{tabular}}
    \caption{Reward payments}
    \label{tab:rewards}
\end{table}
\fi

\begin{table}[htbp]
    \centering
    \resizebox{0.5\linewidth}{!}{%
    \begin{tabular}{l l l l l}
        \hhline{=====}
         DR Scheme & Measure & Unit & Value  & Source \\
         \hline
         V2G & $p^{WTA}_{base}$ &\$/month/charger & 46.4 & \citep{octopus_powerpack}\\ %this includes 30% GMC and RPT of 12 hours "every couple of days"
          & $p^{WTA}_{hour}$ & \$/month/hour(RPT) & 27.2 & \citep{huang2021electric,parsons2014willingness,lim2016assessment}\\
          % & \$/Month/\%(GMC) & 6.3 & \citep{bailey2015anticipating,geske2018willing,kubli2018flexible,parsons2014willingness,huang2021electric,mehdizadeh2023estimating}\\
         Smart charging & $p^{WTA}_{base}$ & \$/month/charger & 33.1 & \citep{octopus_ev,ovo_ev,edf_ev}\\ %octopus, OVO and EDF
          & $p^{WTA}_{hour}$ & \$/month/hour(RPT) & 19.3 & \citep{fabianek2024willing}\\ %check whether results by selin yilmaz & paul fabianek can be used here
         Heat pump & $p^{WTA}$ &\$/month/thermostat & 9.5 &\citep{kaczmarski2022determinants}\\
         HP \& storage & $p^{WTA}$ &\$/month/m\textsuperscript{2} & 13.6 & \citep{housing2024,team_2024}\\
         \hhline{=====}
    \end{tabular}}
    \caption{Reward payments}
    \label{tab:rewards}
\end{table}

\subsection{Operation and maintenance, end-of-life costs and lifetime}
In line with recent studies \citep{huber2021vehicle} the annual operation and maintenance costs ($O\&M$) are assumed to be 5\% of the capital investment and the lifetime is assumed to be 15 years. End-of-life costs are assumed to be zero for the smart charging and smart heat pump tariff, 50\$ per V2G charger, and 10\$ per m\textsuperscript{2} of hot water tank surface.

\subsection{Miscellaneous data}
The estimates for additional parameters, including their sources, are shown in Table~\ref{tab:tech_params}. Note that, the temperature difference for the hot water storage ($\Delta T_w$) can take any value, as long as the high temperature does not exceed 100\degree C. A higher value of $\Delta T_w$ results in higher losses from decreased COPs and heat dissipation while also decreasing the required storage volume and thereby area. Optimisation could be applied to $\Delta T_w$ to balance these counteracting effects but this is outside the scope of this work. Instead, $\Delta T_w$ was chosen at the relatively low value of 35 K to reduce energy losses from COP changes and heat transfer so that they can be neglected. With the hot water storage assumed to be inside the living area, heat dissipation may be considered to contribute to space heating anyway. The electricity price was assumed to be 50\$/MWh to be consistent with \citet{schmidt2019projecting}.%sentence on storage parameters?

\begin{table}[htbp]
    \centering
    \resizebox{0.5\linewidth}{!}{%
    \begin{tabular}{l l l l l l}
        \hhline{======}
         Scheme & Parameter & Symbol & Unit & Value & Source \\
         \hline
         V2G \& smart & (Dis-)charging efficiency & $\eta$ & \% & 92 & \citep{zhang2016evaluation}\\
         %& Roundtrip efficiency & $\eta_{rt}$ & \% & 86  & \\
         charging & Base plug-in time (avg) & $t^{plug-in}_{base}$ & h & 11.5 & \citep{dftEVdata}\\
         & EV battery's energy capacity & $E_{EV}^{max}$ & kWh & 60 & \citep{dftBATdata}\\ %ev database + dft registrations
         & Charger's power capacity & $P_{cha}^{cap}$ & kW & 7.4 & \citep{ukevse}\\
         & Guaranteed minimum charge & $a_{GMC} $ & \% & 30 & \citep{octopus_powerpack} \\
         & Daily driving energy & $E^{daily}_{drive}$ & kWh/day & 5.56 & \citep{dft_mileage,kwhpermile}\\
         & RPT for $p^{WTA}_{base}$ & $\Delta t^{plug-in}_{base}$ & h &  11.5 & \citep{dftEVdata,octopus_powerpack}\\
         & \makecell[tl]{Fraction of energy\\charged at home} & $a_{HC}$ & \% & 90 & \citep{dft_homecharge} \\
         Heat pump & Average power consumption & $P_{HP}^{avg}$ & kW & 0.46 & \citep{lowe2017}\\
         & Average power when on & $P^{avg,act}_{HP}$ & kW & 1.68 & \citep{lowe2017}\\
         & \makecell[tl]{Seasonal performance\\factor} & $SPF$ &  & 2.71  & \citep{lowe2017}\\ %2.50 if we include DHW
         & Building's heat capacity & $m^bc_p^b$ & kJ/K & 34,780 & \citep{housing2024,panao2019measured}\\
         Storage & Insulation thickness & $L$ & m & 0.05 & \citep{ARMSTRONG2014128}\\
         & Min. ceiling height & $H$ & m & 2.3 & \citep{dclg_ceilingheight} \\
         & Water density & $\delta^w$ & kg/m\textsuperscript{3} & 1,000 & \citep{luerssen2021global}\\
         & \makecell[tl]{Specific heat\\capacity of water} & $c_p^w$ & kJ/(kg $\cdot$ K) & 4.18 & \citep{luerssen2021global}\\
         & \makecell[tl]{Water storage\\temperature difference} & $\Delta T^w_{max}$ & K & 35 & \makecell[tl]{Design\\choice}\\
         %& \makecell[tl]{Heat transfer coefficient} & U & W/(m\textsuperscript{2}\cdot K) & \citep{luerssen2021global}\\
         \hhline{======}
    \end{tabular}}
    \caption{Additional Parameters}
    \label{tab:tech_params}
\end{table}



\subsection{Value factor} \label{vfdata}
The value factor for the EV-based tariffs is estimated based on data from the Department for Transport \citep{dftEVdata}. For the heat pump-based tariffs, the estimations are based on data from the Renewable Heat Premium Payment scheme \citep{lowe2017}. Price data in both cases comes from the Balancing Mechanism Reporting Service \citep{elecpricedata}. 


%maybe another section on third difference: LCODR supply curve

\section{Results and discussion} \label{sec:res_dis}

\subsection{Value factor}
Results for the value factors were obtained using the methodology in Section \ref{Availability adjustments} and the data sources from Section \ref{vfdata}. Table~\ref{tab:vf} shows the value factor results for the different DR schemes. The value factor compares the availability profile of a DR scheme with a continuously available flexibility asset (i.e. most storage technologies). The more a DR asset's availability coincides with high energy prices, the higher its value factor. An asset with an entirely even availability profile by definition has a value factor of one. Value factors above one denote that a DR scheme's availability generally coincides with above-average electricity prices while a value factor below one indicates prevailing availability at times when electricity prices are low. Note that value factors only measure the value of the normalised availability profile. The value factor is unaffected by a DR scheme's total availability hours. These are instead captured by V2G's availability factor ($a_f^{V2G}$) or the average shiftable power consumption of unidirectional assets ($P_{1d}^{avg}$) which, in both cases, determine the required number of assets. For unidirectional DR assets the availability profile is given by the uncontrolled load profile of the DR asset as this is the consumption that can be moved by DR. Smart charging has the highest value factor, because uncontrolled domestic charging mostly takes place in the evening hours, coinciding with high demand. Moving domestic EV demand is valuable because it would otherwise occur at times with high energy prices. Heat pump schemes with and without storage have the same value factor because in both schemes the uncontrolled demand comes from heat pumps that have the same heating demand profiles in an uncontrolled scenario. Heating demand occurs primarily in the expensive morning and evening hours ($VF>1$) but is generally spread more evenly than smart charging demand, leading to a slightly lower value factor. V2G is the only scheme where the value factor is not determined by the load profiles of uncontrolled demand but by discharge availability instead. Since the used dataset consists mostly of domestic chargers, EVs tend to plug in overnight, covering both the high-demand evening hours as well as low-demand nighttime. This explains their value factor close to unity, meaning their profile value does not vary much from a continuously available asset. Note how there are different V2G-value factors for available energy and power because, differently from other DR schemes, power and energy profiles can diverge for V2G. The power VF is applied when the number of chargers determines the required V2G chargers while the energy VF is applied when the available battery capacity sets the number of contracted V2G chargers, as determined by Equation (\ref{eq2})).

\begin{table}[htbp]
    \centering
    \resizebox{0.5\linewidth}{!}{%
    \begin{tabular}{l|c c c c}
\hhline{=====}
          & \makecell[c]{V2G} & \makecell[c]{Smart\\Charging} & \makecell[c]{Heat\\Pump} & \makecell[c]{HP \& Thermal\\Storage}\\
         \hline
         Value factor & \makecell[cc]{0.98 (Energy)\\0.99(Power)} & 1.12 & 1.05 & 1.05 \\
         \hhline{=====}
    \end{tabular}}
    \caption{Value factors for the different technologies}
    \label{tab:vf}
\end{table}


Figure~\ref{fig:vf} shows daily profiles for the parameters that were used to estimate the different value factors. Comparing the price profile with a parameter's profile gives an idea of how the value factor is estimated. However, it should be noted that seasonal fluctuations in price can naturally also affect the value factor but cannot be inferred from Figure~\ref{fig:vf}.
\begin{figure}[ht]
    \centering
    \includegraphics[width=0.5\linewidth]{vf_fig.pdf}
    \caption{Average daily profiles of value factor parameters}
    \label{fig:vf}
\end{figure}

\subsection{Levelised cost}

\begin{figure}[ht]
    \centering
    \includegraphics[width=0.5\linewidth]{LCODR_energy.pdf}
    \caption{$LCODR_{VF}$ in energy terms from Monte Carlo simulation for the twelve different applications. LCOS results from \citet{schmidt2019projecting} are included for comparison with numbers indicating the cheapest storage technology (1: Pumped hydro, 2: Flywheel, 3: Vanadium-flow, 4: Lithium-ion, 5:Hyrdogen, 6: Compressed air)}
    \label{fig:res_energy_abs}
\end{figure}

\begin{figure}[ht]
    \centering
    \includegraphics[width=0.5\linewidth]{LCODR_energy_V2G_wolog_lowhigh.pdf}
    \caption{Split-up version of Figure~\ref{fig:res_energy_abs} to allow better results identification on linear scale.}
    \label{fig:res_energy_abs2}
\end{figure}

Figure~\ref{fig:res_energy_abs} and Figure~\ref{fig:res_energy_abs2} show the LCODR in energy terms adjusted for their respective value factor ($LCODR_{VF}$) for the twelve different applications. It also contrasts the $LCODR_{VF}$ results with the cost-optimal LCOS from \citep{schmidt2019projecting}. The LCOS was taken as their predictions for the year 2025 and all included results were adjusted for inflation. DR technologies are included in Figure~\ref{fig:res_energy_abs} following their technical suitability which is indicated in Table~\ref{tab:my_label}. The results show that heat pumps combined with thermal storage, where feasible, significantly outperform other DR or storage technologies. This underlines a huge potential for cheap flexibility from residential thermal storage. While outside the scope of this study, it should be noted that district heating with large-scale thermal storage may present an even more cost-effective flexibility source and \citet{vilen2024seasonal} suggests that these could even provide seasonal load-shifting. The smart heat pump scheme cannot compete with either storage or any of the other DR technologies. The EV-based schemes are able to contend with the most cost-competitive storage technology for applications with a high total annual discharge ($N_{app}^cyc \times \Delta t_{DD}$), most prominently energy arbitrage, primary response, and transmission and distribution (T\&D) investment deferral. The reason for this is that higher discharge requirements allow the EV-based schemes to leverage their advantage over storage technologies with lower roundtrip efficiencies (e.g. hydrogen).

\begin{figure}[ht]
    \centering
    \includegraphics[width=0.5\linewidth]{LCODR_energy_MC_woHPTH.pdf}
    \caption{The probabilities of different technologies to achieve the lowest $LCODR_{VF}$ in energy terms for all twelve applications. Results for heat pumps with thermal storage were excluded because these present the cheapest option with a 100\% probability for all applications where they are feasible.}
    \label{fig:res_energy_probs}
\end{figure}

Figure~\ref{fig:res_energy_probs} shows the estimated likelihood of different technologies presenting the cheapest option in terms of energy-based LCODR. It excludes the option of heat pumps with thermal storage as these would otherwise present the cheapest technology across all applications where they can be used. The equivalent results for the power-based LCODR can be found in \ref{app1}. Figures~\ref{fig:res_energy_abs} and \ref{fig:res_energy_probs} show that DLC schemes are often unable to provide more cost-effective flexibility than storage assets. Heat pumps with thermal storage are the only one out of four trialled schemes that achieve consistently lower levelised costs than storage technologies. This insight should encourage energy system stakeholders to pursue flexibility from thermal storage in buildings with electrical heating. EV-based DR schemes cannot compete with thermal storage but may still be cost-competitive against storage in certain short-discharge applications. Deployment of EV-based DR schemes should therefore focus on those applications since these are the only ones where they are likely to outcompete storage.



Figure~\ref{fig:cost_compos} shows the fraction of the LCODR that is associated with each term in the numerator of Equation (\ref{eqb1}). It shows that reward payments make up a large part of the total cost for all studied schemes highlighting the importance of adequately portraying this cost component in cost assessments of DR. Rebound costs make up a larger part of the thermal storage scheme because of its comparatively low reward payments term.

\begin{figure}[t]
    \centering
    \includegraphics[width=0.5\linewidth]{LCODR_cost_components.pdf}
    \caption{Breakdown of average cost components for different technologies. Note that the share of rebound costs can vary significantly depending on the application and this figure only shows an average of all applications.}
    \label{fig:cost_compos}
\end{figure}

\subsection{Limitations}
Three major limitations should be considered when interpreting the results.

\begin{enumerate}[leftmargin=*]
    \item DR is not storage. The LCODR methodology builds upon the LCOS methodology because DR can provide the same service as storage in many situations. For storage, however, the times of charge and discharge are more independent of each other than for most of the DR applications, where the rebound sometimes has to occur immediately after the end of the discharge period. The proposed methodology does not account for time constraints on the rebound period, and a comprehensive value assessment should also include a value factor for the energy that is used for the rebound. The heat pump schemes, for example, may have a higher value factor because their consumption is higher in winter, but without the ability to move their consumption seasonally, this should arguably not increase their assigned value.
    \item V2G also includes a period of Smart Charging. Figure~\ref{fig:ex} shows that V2G vehicles are charged immediately to then discharge as required. However, this period of immediate charging could also be shifted like in the Smart Charging schemes, thereby potentially offering further services from V2G and lowering its costs.
    \item Better data is needed. Several assumptions had to be made due to lacking data, and existing data had to be interpreted outside of its geographic and technical context, as shown in Section~\ref{sec:data}. To get a measure of how RPT affects discharge duration, for example, it had to be assumed that vehicles plug in once a day and that consumers never plug in for longer than they are contractually obliged to. Rather than measuring the proportion of the day that drivers are plugged in for, future studies should aim to measure the plug-in durations of drivers who have been incentivised financially to plug in their EVs. More granular data may also allow the estimations to move beyond averages and take into account the spectrum of participants in each DR scheme.    
\end{enumerate}



 \section{Conclusion} \label{sec:concl}
Assessment of lifetime costs of demand response has received relatively little attention and so this paper addresses the issue by introducing a levelised cost of demand response, LCODR. A methodology for estimating the lifetime costs of storage-equivalent DR services is elaborated across twelve applications that may be provided by storage or DR. Data from academic literature and industry is applied to assess the LCODR for four different demand response schemes. The results suggest that residential heat pumps with thermal storage can provide flexibility more cost-effectively than previously studied storage technologies for most applications that can be serviced through demand reduction. Smart charging and, to a lesser extent, vehicle-to-grid can compete with storage technologies in applications that are activated for a comparatively larger proportion of the year, i.e. primary response. Results should be interpreted with caution because data was collated from various sources with simplifying assumptions required at times. Future work should therefore aim to collect consistent consumer data that is tailored towards a granular LCODR estimation. The proposed methodology and results present a versatile decision-making framework for energy system stakeholders who want to compare the costs of DR and storage.

%% The Appendices part is started with the command \appendix;
%% appendix sections are then done as normal sections
\appendix

\section{Results for power-based LCODR} \label{app1}
Equation (\ref{eqb2}) displays the estimation of the power-based LCODR. Figure~\ref{fig:power_res} shows the power-based LCODR results including a comparison with the power-based LCOS of the most competitive storage technology. The power-based LCOS is also referred to as the annuitized capacity cost (ACC) \citep{schmidt2024monetizing}. Figure~\ref{fig:power_prob} shows the probabilities of different DR technologies to achieve the lowest availability profile-adjusted ACC when the thermal storage tank scheme is excluded.

\iffalse
\begin{table*}[htbp]
    \centering
    \resizebox{\textwidth}{!}{%
    \begin{tabular}{l|lllllllllllll}
    \hhline{==============}
         \makecell[c]{\$/MWh} & \makecell[tl]{Energy\\Arbitrage} & \makecell[tl]{Primary\\Response} & \makecell[tl]{Secondary\\Response} & \makecell[tl]{Tertiary\\Response} & \makecell[tl]{Peaker\\Replacement} &\makecell[tl]{Black\\Start} & \makecell[tl]{Seasonal\\Storage} & \makecell[tl]{T\&D Invest.\\Deferral} & \makecell[tl]{Congest.\\Mgmt.} & \makecell[tl]{Bill\\Mgmt.} & \makecell[tl]{Power\\Quality} & \makecell[tl]{Power\\Reliability} \\
    \hline
    Power capacity [MW] & 100 & 10 & 100 & 100 & 100 & 10 & 100 & 100 & 100 & 1 & 1 & 1 \\
        Discharge Duration [h] & 4 & 0.5 & 1 & 4 & 4 & 1 & 700 & 8 & 1 & 4 & 0.5 & 8 \\
        Annual cycles & 300 & 5000 & 1000 & 10 & 50 & 10 & 3 & 300 & 300 & 500 & 100 & 50 \\
    %Response time & & & & & & & & & & & & \\
    \hline
         V2G & 222 & 138 & 254 & 4889 & 1676 & 19374 & - & 174 & 704 & 157 & 3923 & 1138 \\
         Smart Charging & 7758 & 3747 & 9301 & 231457 & 46327 & - & - & 3901 & 30899 & 4673 & - & - \\
         Heat Pump & - & 13866 & 6564 & - & - & - & - & -  & 6564 & - & - & - \\
         HP \& Storage & 302 & 73 & 141 & 7695 & 1576 & - & - & 286 & 359 & 202 & - & - \\
         \hline
         LCOS(2025) \citep{schmidt2019projecting} & 225\textsuperscript{1} & 165\textsuperscript{2} & 210\textsuperscript{3} & 2400\textsuperscript{1} & 900\textsuperscript{1} & 9000\textsuperscript{4} & 2300\textsuperscript{5} & 150\textsuperscript{1} & 370\textsuperscript{4} & 220\textsuperscript{3} & 1300\textsuperscript{4} & 900\textsuperscript{4}\\
    \hhline{==============}
    \end{tabular}}
    \begin{tablenotes}
    \item[]{\footnotesize Cheapest storage technology: \textsuperscript{1} Pumped Hydro, \textsuperscript{2} Flywheel, \textsuperscript{3} Vanadium Redox-Flow, \textsuperscript{4} Lithium-ion, \textsuperscript{5} Hydrogen, \textsuperscript{6} Compressed Air}
    \end{tablenotes}
    \caption{Energy-based LCODR Results compared to LCOS from \citep{schmidt2019projecting}}
    \label{tab:LCODR_results_E}
\end{table*}

\begin{table*}[htbp]
    \centering
    \resizebox{\textwidth}{!}{%
    \begin{tabular}{l|lllllllllllll}
    \hhline{==============}
         \makecell[c]{\$/kW\textsubscript{year}} & \makecell[tl]{Energy\\Arbitrage} & \makecell[tl]{Primary\\Response} & \makecell[tl]{Secondary\\Response} & \makecell[tl]{Tertiary\\Response} & \makecell[tl]{Peaker\\Replacement} &\makecell[tl]{Black\\Start} & \makecell[tl]{Seasonal\\Storage} & \makecell[tl]{T\&D Invest.\\Deferral} & \makecell[tl]{Congest.\\Mgmt.} & \makecell[tl]{Bill\\Mgmt.} & \makecell[tl]{Power\\Quality} & \makecell[tl]{Power\\Reliability} \\
    \hline
         V2G &  &  &  &  &  &  & - &  &  &  & 190 &  \\
         Smart Charging &  &  &  &  &  & - & - &  &  &  & - & - \\
         Heat Pump & - &  &  & - & - & - & - & -  &  & - & - & - \\
         HP \& Storage &  &  &  &  &  & - & - &  &  & - & - & - \\
         \hline
         LCOS(2025) \citep{schmidt2019projecting} & 180\textsuperscript{6} & 200\textsuperscript{4} & 115\textsuperscript{4} & 100\textsuperscript{6} & 110\textsuperscript{6} & 75\textsuperscript{4} & 1750\textsuperscript{5} & 265\textsuperscript{6} & 90\textsuperscript{4} & 370\textsuperscript{4} & 65\textsuperscript{4} & 310\textsuperscript{3}\\
    \hhline{==============}
    \end{tabular}}
    \begin{tablenotes}
    \item[]{\footnotesize Cheapest storage technology: \textsuperscript{1} Pumped Hydro, \textsuperscript{2} Flywheel, \textsuperscript{3} Vanadium Redox-Flow, \textsuperscript{4} Lithium-ion, \textsuperscript{5} Hydrogen, \textsuperscript{6} Compressed Air}
    \end{tablenotes}
    \caption{Power-based LCODR Results compared to LCOS from \citep{schmidt2019projecting}}
    \label{tab:LCODR_results_P}
\end{table*}
\fi

\begin{figure*}[h!]
\begin{equation}
    LCODR[\frac{\$}{kW_{year}}] = \frac{Investment + \sum^{T}_{t}\frac{O\&M}{(1+r)^t} + \sum^{T}_{t}\frac{Rewards}{(1+r)^t} + 
    \sum^{T}_{t}\frac{Rebound}{(1+r)^t} +
    \frac{End-of-life\; costs}{(1+r)^{T+1}}}{\sum^{T}_{t}\frac{P^{cap}_{app}}{(1+r)^t}} \label{eqb2}
\end{equation}
\end{figure*}

\begin{figure}[h!]
    \centering
    \includegraphics[width=0.45\linewidth]{LCODR_power.pdf}
    \caption{$LCODR/VF$ in power terms from Monte Carlo simulation for the twelve different applications. LCOS results from \citet{schmidt2019projecting} are included for comparison. Numbers indicate the cheapest storage technology (1: Pumped hydro, 2: Flywheel, 3: Vanadium-flow, 4: Lithium-ion, 5:Hyrdogen, 6: Compressed air)}
    \label{fig:power_prob}
\end{figure}

\begin{figure}[h!]
    \centering
    \includegraphics[width=0.45\linewidth]{LCODR_power_MC_woHPTH.pdf}
    \caption{The probabilities of different technologies to achieve the lowest $LCODR/VF$ in power terms. Excludes the DR scheme that combines heat pumps with thermal storage.}
    \label{fig:power_res}
\end{figure}





%\section{Sample Appendix Section}
%\label{sec:sample:appendix}
%Lorem ipsum dolor sit amet, consectetur adipiscing elit, sed do eiusmod tempor section \ref{sec:sample1} incididunt ut labore et dolore magna aliqua. Ut enim ad minim veniam, quis nostrud exercitation ullamco laboris nisi ut aliquip ex ea commodo consequat. Duis aute irure dolor in reprehenderit in voluptate velit esse cillum dolore eu fugiat nulla pariatur. Excepteur sint occaecat cupidatat non proident, sunt in culpa qui officia deserunt mollit anim id est laborum.

%% If you have bibdatabase file and want bibtex to generate the
%% bibitems, please use
%%
\bibliographystyle{elsarticle-num-names} 
\bibliography{cas-refs}

%% else use the following coding to input the bibitems directly in the
%% TeX file.

% \begin{thebibliography}{00}

% %% \bibitem[Author(year)]{label}
% %% Text of bibliographic item

% \bibitem[ ()]{}

% \end{thebibliography}
\end{document}

\endinput
%%
%% End of file `elsarticle-template-num-names.tex'.