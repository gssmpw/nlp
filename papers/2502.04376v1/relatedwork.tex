\section{Related Work}
\noindent{\textbf{Language Model Applications in Meetings.}} Considerable research has been dedicated to the summarization of meetings~\cite{Zhong2021qmsum} and other real-life dialogues~\cite{Mehdad2014summary, Tuggener2021summarysurvey}. In the context of meetings, key tasks include meeting transcript summarization and action item identification~\cite{Cohen2021action}. MeetingQA~\cite{Prasad2023meetingqa} investigated Q\&A tasks based on meeting transcripts, highlighting the challenges faced by models such as RoBERTa in handling real-world meeting data. Recent advancements in LLMs have opened new avenues for enhancing these tasks. For instance, an LLM-based meeting recap system~\cite{Asthana2023recap} has demonstrated effectiveness in generating accurate and coherent summaries and action items.

% Meeting recap including summary and actions has been a popular topic to explore in Natural Language Processing (NLP) domain \xt{refs}. MeetingQA \xt{ref} explored Q\&A tasks based on meeting transcripts, i.e. identifying answers to questions asked during discussion among meeting participants, and found it is a challenging real-world task for models such as RoBERTa and DeBERTa \xt{refs? better description?}. Recent advancements in LLMs have brought new opportunities. \cite{Asthana2023recap} present the design and implementation and evaluation in-context a meeting recap system. \xt{comment on effectiveness in one sentence.}

%A lot of great research works have focused on summarization of meetings~\cite{Zhong2021qmsum} and other real-life dialogues~\cite{Mehdad2014summary, Tuggener2021summarysurvey}. Specifically in meeting scenario, the tasks are meeting transcript summarization and action item identification~\cite{Cohen2021action}. MeetingQA~\cite{Prasad2023meetingqa} investigated Q\&A tasks based on meeting transcripts, aiming to identify answers to questions posed during discussions, highlighting the challenges faced by models like RoBERTa in handling real-world meeting data. Recent advancements in LLMs have opened new avenues for improving these tasks. For instance, a LLM-based meeting recap system~\cite{Asthana2023recap} demonstrates its effectiveness in generating accurate and coherent summaries and action items.

\noindent{\textbf{Facilitator in Multi-Participant Chat.}} MUCA \cite{mao2024muca} presents a framework that leverages LLMs to facilitate group chats by simulating users, demonstrating notable effectiveness in goal-oriented conversations. Similarly, approaches like GPT-4o demo for meetings~\cite{openaivoice} are designed to serve as facilitators in group discussions. While these studies underscore LLMs’ capabilities in managing group chats, they primarily focus on LLMs guiding the meeting process rather than representing individuals with different roles.

%designed to serve as facilitators in group discussions, but they lack the specific goal of knowledge-enabled participation. While these works highlight the capabilities of LLMs in managing group chats, they do not address LLMs' ability as a participant. Additionally, they focus on general facilitation skills rather than targeted engagement within the specific context of meetings.

\noindent{\textbf{Role-Playing with LLMs: Characters and Digital Twins.}}
% Digital twin has always been a hot topic. Recently Reid Hoffman released an interview between himself and his digital version by fine-tuning GPT4 (\xt{double check}). Though impressive and shed a preview of digital representation, the capability demonstration in this interview is only limited to the 1:1 scenario and the group discussion scenario remain unexplored. In addition, the attempt of simulating a famous person have also intrigued works such as \xt{refs}. But these works focus on try to see whether LLM can stay within character or study social activity by agent group chat social environment.
%The concept of digital twins has been a prominent topic in recent research. 
Role-play prompting~\cite{kong2024roleplay} has been shown to be a more effective trigger for the chain-of-thought process in LLMs. Additionally, efforts to simulate famous personalities~\cite{shao2023character, Sun2024persona} have garnered interest, leading to research on maintaining character consistency and studying social interactions within agent-based group chat environments. Recently, Reid Hoffman~\cite{ReidHoffman} showcased an interview between himself and his digital twin built on GPT-4. Although this demonstration highlighted the potential of digital representations, it was confined to one-on-one interactions, leaving the complexities of group discussions unexplored. Unlike previous work, our work focus on LLMs as meeting participant delegates, delivering targeted engagement tailored to multi-participant, meeting-specific objectives. Our comprehensive evaluation and real-world deployment further demonstrate the system's potential to significantly reduce the burden of meetings on individuals, thereby advancing the application of LLMs in professional environments.
%significantly differs from these previous efforts by focusing on the deployment of LLMs as meeting delegates. Unlike role-play and digital twin applications that primarily aim to replicate individual personalities or enhance chain-of-thought processes, our approach aims to enable LLMs to participate in and facilitate real-time meeting scenarios. This involves not only maintaining character consistency but also addressing specific meeting objectives, improving content accuracy, and ensuring timely responses. This focus on practical, real-world application in a dynamic, multi-participant setting makes our work a valuable contribution to the field, providing insights into the challenges and opportunities of using LLMs to alleviate the burden of meetings on individuals.

%\xt{need more and differentiation, emnlp/acl refs!}
%\xt{meeting related}
%\xt{simulated: better not to touch on this topic as we don't focus on that}
%\jz{mention the difference between group discussion vs. 1:1 as  Reidhoffman }

%