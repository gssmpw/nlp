\documentclass[12pt]{article}
\usepackage[T1]{fontenc}
\usepackage[utf8]{inputenc}
\usepackage[english]{babel}
\pdfoutput=1

\headheight 0in
\headsep 0in
\evensidemargin 0in
\oddsidemargin \evensidemargin
\textwidth 6.5in
\topmargin .25in
\textheight 8.8in

\usepackage[style=numeric]{biblatex}
\addbibresource{references.bib}
\usepackage{microtype}
\usepackage{enumitem}
\setenumerate{label=\textup{(\roman*)}}

\usepackage{amsmath, mathtools}
\usepackage{amssymb} % Remove for ACM
\usepackage{amsthm, thmtools}
\usepackage{tikz-cd}
\usepackage{braket}
\usepackage{bm}
\usepackage{csquotes}
\usepackage{xspace}
\usepackage{float}

\usepackage[colorlinks]{hyperref}
\usepackage[nameinlink]{cleveref}
\hypersetup{
  linkcolor=[rgb]{0.3,0.3,0.6},
  citecolor=[rgb]{0.2,0.6,0.2},
  urlcolor =[rgb]{0.6,0.2,0.2}
}


\declaretheorem[numberwithin=section]{theorem}
\declaretheorem[numberlike=theorem]{lemma, corollary}
\declaretheorem[numberlike=theorem,style=definition]{definition}
\declaretheorem[numberlike=theorem,style=remark]{example}
\declaretheorem[numberlike=theorem,style=remark]{remark}
\declaretheoremstyle[
notefont=\bfseries, notebraces={(}{)}
]{conjecture}
\declaretheorem[numbered=no,style=conjecture]{conjecture}

\DeclarePairedDelimiter\abs{\lvert}{\rvert}%

\newcommand{\NN}{\mathbb{N}}
\newcommand{\ZZ}{\mathbb{Z}}
\newcommand{\QQ}{\mathbb{Q}}
\newcommand{\CC}{\mathbb{C}}
\newcommand{\KK}{\mathbb{K}}
\newcommand{\FF}{\mathbb{F}}
\newcommand{\PP}{\mathbb{P}}
\newcommand{\mm}{\mathfrak{m}}

\newcommand{\NULL}{\texttt{0}}
\newcommand{\EINS}{\texttt{1}}

\usepackage{tabularx, booktabs}

\newcommand\problemBox[4][\textwidth]{
\begin{tabularx}{#1}{lX}
\toprule
\multicolumn{2}{c}{#2} \\ \midrule
\textbf{Input:}&#3\\
\textbf{Output:}&#4\\ \bottomrule
\end{tabularx}
}

\newcommand{\Landau}{\mathcal{O}}
\newcommand{\landau}{o}

\newcommand{\pclass}[1]{\textsf{\textup{#1}}}
\newcommand{\Mon}{\pclass{Mon}}
\newcommand{\Homog}{\pclass{Homog}}
\newcommand{\Univar}{\pclass{Univar}}
\newcommand{\x}{{\underline x}}
\renewcommand{\t}{{\underline t}}
\DeclareMathOperator{\ini}{in}
\DeclareMathOperator{\NF}{nf}

\newcommand{\problemText}[1]{\textsc{#1}\xspace}
\newcommand{\IdealMem}{\problemText{IdealMem}}
\newcommand{\AlgMem}{\problemText{AlgMem}}
\newcommand{\IniAlgMem}{\problemText{InAlgMem}}
\newcommand{\CMRS}{\problemText{CMRS}}

\newcommand{\OneInThreeSAT}{\problemText{1in3Sat}}
\newcommand{\SubsetSum}{\problemText{SubsetSum}}
\newcommand{\LBAhalt}{\problemText{LBAhalt}}

\newcommand{\cclass}[1]{\textsf{\textup{#1}}\xspace}
\newcommand{\TCO}{\cclass{TC\ensuremath{^0}}}
\newcommand{\DL}{\cclass{L}}
\newcommand{\NL}{\cclass{NL}}
\newcommand{\DP}{\cclass{P}}
\newcommand{\NP}{\cclass{NP}}
\newcommand{\PSPACE}{\cclass{PSPACE}}
\newcommand{\EXP}{\cclass{EXP}}
\newcommand{\EXPSPACE}{\cclass{EXPSPACE}}

\newcommand{\leqlm}{\mathrel{\leq^{\cclass{L}}_{\text{m}}}}

\newcommand{\ie}{i.\,e.\@\xspace}
\newcommand{\eg}{e.\,g.\@\xspace}


% Title Page
\title{Computational Complexity of Polynomial Subalgebras}
\author{Leonie Kayser}
\date{\today}

\begin{document}
\maketitle

\begin{abstract}
The computational complexity of polynomial ideals and Gröbner bases has been studied since the 1980s. In recent years the related notions of polynomial subalgebras and SAGBI bases have gained more and more attention in computational algebra, with a view towards effective algorithms. We investigate the computational complexity of the subalgebra membership problem and degree bounds. In particular, we place these problems in the complexity class \EXPSPACE and prove \PSPACE-completeness for homogeneous algebras. We highlight parallels and differences compared to the settings of ideals and also look at important classes of polynomials such as monomial algebras.
\end{abstract}
%%%%%%%%%%%%%%%%%%%%%%%%%%%%%%%%%%%%%%%%%%%%%%%%%%%%%%%%%%%%%%%%%%%%%%%%
\section{Introduction}
%%%%%%%%%%%%%%%%%%%%%%%%%%%%%%%%%%%%%%%%%%%%%%%%%%%%%%%%%%%%%%%%%%%%%%%%

Let $\KK$ be a field equipped with a suitable encoding over a finite alphabet, for example, the rational numbers $\QQ$ or a finite field $\FF_q$.
At the heart of many algorithms in computer algebra are efficient algorithms manipulating polynomial ideals and Gröbner bases, dating back at least to the 1960s \cite{Buchberger2006}. Since the late 1980s, algorithms for computations with subalgebras have been studied \cite{Shannon1988, Robbiano1990}, which have applications in toric degenerations and polynomial system solving \cite{betti2025}. A fundamental computational problem for ideals and subalgebras of polynomial rings $\KK[\x] = \KK[x_1,\dots,x_n]$ is deciding \emph{membership}. Of special interest are classes $\textsf{C}$ of polynomials such as homogeneous polynomials (\Homog), monomials (\Mon), or polynomials with bounded degree or number of variables.

\begin{table}[H]
\centering
\problemBox[0.48\columnwidth]{$\IdealMem_\KK(\textsf{C})$, $\IdealMem_\KK$}
{$f_1,\dots,f_s \in \textsf{C}$ (or $\KK[\x]$),\newline$g \in \KK[\x]$}
{Is $g \in \langle f_1,\dots,f_s \rangle$?} \hfill \problemBox[0.48\columnwidth]{$\AlgMem_\KK(\textsf{C})$, $\AlgMem_\KK$}
{$f_1,\dots,f_s \in \textsf{C}$ (or $\KK[\x]$),\newline$g \in \KK[\x]$}
{Is $g \in \KK[f_1,\dots,f_s]$?}
\end{table}

Recall that $g \in \langle f_1,\dots,f_s \rangle$ if and only if there is a representation $g = h_1f_1+\dots+ h_sf_s$ with $h_1,\dots,h_s \in \KK[\x]$; the \emph{representation degree} (of the tuple $f_1,\dots,f_s,g$) is the smallest possible value of $D = \max \{\deg h_1,\dots,\deg h_s\}$.
Similarly, $g \in \KK[f_1,\dots,f_s]$ if and only if $g = p(f_1,\dots,f_s)$ for some \emph{certificate} $p \in \KK[t_1,\dots,t_s]$; we call the smallest possible degree of $p$ the \emph{certification degree}.

The computational complexity and representation degree bounds of $\IdealMem_\KK$ and its variants have been studied since the 1980s, we give a brief overview in \Cref{sec:ideal_world}. The complexity of algebras in a more general sense has been studied before \cite{Kozen77}, though so far not for the concrete case of polynomial subalgebras. This paper is concerned with filling that gap and proving analogous results for $\AlgMem_\KK$ and an upper bound on the certification degree. In order to obtain reasonable upper bounds on the computational complexity, we assume that arithmetic in the base field $\KK$ can be performed efficiently, this is formalized as being \emph{well-endowed} \cite{Borodin1983}. A main theorem is the \PSPACE-completeness of the homogeneous subalgebra membership problem, combining the upper bound \Cref{thm:algmem_homog} and the complementing lower bound \Cref{thm:PSPACE_hard}.

\begin{theorem}
Over a well-endowed field the homogeneous subalgebra membership problem $\emph{\AlgMem}_\KK(\Homog)$ is \PSPACE-complete.
\end{theorem}

One of the goals of this paper is to be accessible both to commutative algebraists and computer scientists. For this reason, we introduce the relevant results from algebra and complexity theory on ideals and Gröbner bases in \Cref{sec:background}. In \Cref{sec:upper_bounds} we prove various upper bound results on variations of $\AlgMem_\KK$, as well as an upper bound on the certification degree. In \Cref{sec:lower_bounds} we present the combinatorial construction of \emph{controlled monomial replacement systems} which yields a polynomial space lower bound and is of independent interest. In \Cref{sec:monomials} we study the case of monomial subalgebras which is still \NP-complete, and discuss related questions with SAGBI bases.

%%%%%%%%%%%%%%%%%%%%%%%%%%%%%%%%%%%%%%%%%%%%%%%%%%%%%%%%%%%%%%%%%%%%%%%%
\section{Background in algebra and complexity theory}\label{sec:background}
%%%%%%%%%%%%%%%%%%%%%%%%%%%%%%%%%%%%%%%%%%%%%%%%%%%%%%%%%%%%%%%%%%%%%%%%

%%%%%%%%%%%%%%%%%%%%%%%%%%%%%%%%%%%%%%%%%%%%%%%%%%%%%%%%%%%%%%%%%%%%%%%%
\subsection{Ideals, subalgebras and their bases}\label{sec:ideals}
%%%%%%%%%%%%%%%%%%%%%%%%%%%%%%%%%%%%%%%%%%%%%%%%%%%%%%%%%%%%%%%%%%%%%%%%

%Standard references for ideals and computational algebra are the books \cite{Cox2016,Gathen2017,Robbiano1990}.
Let $\KK$ be a field, denote by $\KK[\x] = \KK[x_1,\dots,x_n]$ the polynomial ring in $n$ variables, consisting of finite linear combinations
\[
f = \sum_{\alpha \in \NN^n} c_\alpha  x^\alpha, \qquad x^\alpha = x_1^{\alpha_1}\dotsm x_n^{\alpha_n}, \quad c_\alpha \in \KK.
\]
Ideals and subalgebras arise naturally in algebra as the kernels and images of ($\KK$-linear) ring homomorphisms. We briefly recall the basic definitions for both of them in parallel.

A subset $I \subseteq \KK[\x]$ is an ideal if it is an additive subgroup (contains $0$ and is closed under addition and subtraction) and also closed under multiplication with \emph{arbitrary} polynomials $f \in \KK[\x]$. A subset $A \subseteq \KK[\x]$ is a subalgebra if it is an additive subgroup containing $\KK$ and is closed under multiplication within $A$.
Given a set $S \subseteq \KK[\x]$, we denote by
\begin{align*}
  \langle S \rangle &= \Set{ \sum_{\text{finite}} h_i f_i | \!\!\!\begin{array}{c}
f_i \in S,\\ h_i \in \KK[\x]
  \end{array}\!\!\!} = \set{\text{lin.\ comb.\ of $S$ over $\KK[\x]$}} \\
  \KK[S] &= \Set{ \sum_{\text{finite}} c_\alpha f^\alpha | \!\!\!\begin{array}{c}
   f_i \in S,\\ c_\alpha \in \KK
  \end{array}\!\!\!  } = \set{\text{polynomial expressions in $S$}}
\end{align*}
the ideal and subalgebra generated by $S$; it is the smallest ideal resp.\ subalgebra containing $S$.
\begin{example}
An important class of ideals and algebras are those generated by monomials $x^\alpha$, or binomials $x^\alpha-x^\beta$. For example, monomial algebras show up as coordinate rings of affine toric varieties, and binomial ideals are related to commutative Thue systems, capturing the combinatorial worst-case complexity of ideal membership (see \Cref{thm:idealmem_expspace}).
\end{example}

A \emph{monomial order} $\prec$ is a total order on the set of monomials $x^\alpha$ refining divisibility ($x^\alpha \mid x^\beta$ implies $x^\alpha \preceq x^\beta$) and such that $1 = x^0$ is the minimal element. In the following fix such a monomial order, for example the \emph{lexicographic order} $\prec_{\text{lex}}$; we may assume $x_1>\dots>x_n$.

The initial term $\ini_\prec(f)$ of a nonzero polynomial $f = \sum_\alpha c_\alpha x^\alpha$ is the nonzero term $c_\alpha x^\alpha$ with the largest monomial. For convenience define $\ini_\prec(0) = 0$. Given an ideal $I$ or a subalgebra $A$, its initial ideal or initial algebra respectively is
\[
\ini_\prec(I) = \langle \set{\ini_\prec(f) | f \in I} \rangle, \quad \ini_\prec(A) = \KK[\set{\ini_\prec(f) | f \in A}].
\]
A \emph{Gröbner basis} of an ideal $I$ is a \emph{finite} subset $G \subseteq I$ such that $\ini_\prec(I) = \langle \set{\ini_\prec(g) | g \in G}\rangle$. On the other hand, a \emph{SAGBI basis} of a subalgebra $A$ is \emph{any} subset $S \subseteq A$ such that $\ini_\prec(A) = \KK[\set{\ini_\prec f | f \in S}]$. In both cases such bases generate $I$ and $A$ respectively, hence the (historical) name \enquote{basis}. A Gröbner basis is \emph{reduced} if the leading coefficients are $1$ and no term of $g \in G$ is in $\ini_\prec(G\setminus\{g\})$. The reduced Gröbner basis of an ideal $I$ is unique and has the smallest number of elements and smallest degree among all Gröbner bases of $I$. For more background on Gröbner bases see for example \cite[Chapter 2]{Cox2016}, \cite[Chapter 21]{Gathen2017} or \cite[Chapter 2]{Kreuzer2000}.

By \emph{Hilbert's basis theorem} every ideal can be generated by a finite set of polynomials; in fact one can find such a finite set in any generating set. This implies the existence of (finite) Gröbner bases for any ideal. On the other hand, finite SAGBI bases may not exist, we will come back to this topic in \Cref{sec:SAGBIconjecture}.

Given an ideal $I \subseteq \KK[\x]$, any element $f \in \KK[\x]$ has a \emph{normal form} $\NF_\prec(f,I)$ against $I$, which is the unique polynomial $r \in f+I$ such that no monomial of $r$ is in $\ini_\prec(I)$ \cite[Definition 2.4.8]{Kreuzer2000}. One can compute $\NF_\prec(f,I)$ by dividing $f$ by a Gröbner basis $G$, the remainder is the normal form \cite[Proposition 2.6.1]{Cox2016}. This gives a membership test for polynomial ideals: $f \in I$ if and only if $\NF_\prec(f,I) = 0$. 

%%%%%%%%%%%%%%%%%%%%%%%%%%%%%%%%%%%%%%%%%%%%%%%%%%%%%%%%%%%%%%%%%%%%%%%%
\subsection{Complexity theory}\label{sec:complexity}
%%%%%%%%%%%%%%%%%%%%%%%%%%%%%%%%%%%%%%%%%%%%%%%%%%%%%%%%%%%%%%%%%%%%%%%%

We use the Turing model of computation, though the details of this formalism are not required to follow this paper. Here we introduce the concepts and computational problems used in the sequel.
A classical reference for computational complexity theory is the book by Hopcroft and Ullman \cite{Hopcroft1979}, a more modern treatment is the book by Arora and Barak \cite{Arora2009}, for a brief introduction with a view towards computer algebra see also \cite[Section 25.8]{Gathen2017}.

Informally, the complexity of an algorithm is the amount of resources (time or memory) used in the computation as a function in the \emph{input length}. A \emph{decision problem} $A$ is the problem of deciding whether a given (well-formed) input $x \in \Sigma^*$ has a certain property, \ie deciding membership in the set $A \subseteq \Sigma^*$.
Problems in \DP, \PSPACE and \EXPSPACE can be solved algorithmically in polynomial time, polynomial space and exponential space respectively, while the answer to problems in \NP can be \emph{verified} in polynomial time provided a certificate. One has the inclusions of complexity classes
\[
\DP \subseteq \NP \subseteq \PSPACE \subsetneq \EXPSPACE.
\]

A standard problem in \NP is \problemText{Sat}, deciding whether a given boolean formula $\varphi(x_1,\dots,x_n)$ can be satisfied by some assignment $\{x_1,\dots,x_n\} \to \{\texttt{true},\texttt{false}\}$. We will later need the useful variant \OneInThreeSAT and the \SubsetSum problem:

  \begin{table}[H]
    \centering
    \problemBox[0.48\columnwidth]{\OneInThreeSAT}
    {$S_1,\dots,S_n \subseteq \NN$, $|S_i| \leq 3$}
    {Is there a set $T \subseteq \NN$ such that $|T \cap S_i| = 1$ for all $i$?}\problemBox[0.48\columnwidth]{\SubsetSum}
    {$a_1,\dots,a_n,b \in \NN$}
    {Is $\sum_{i \in I} a_i = b$ for some $I \subseteq \{1,\dots,n\}$?}
  \end{table}

A typical example of a problem in \PSPACE is the halting problem for (deterministic) linearly bounded automata (LBA). Here a LBA $M$ consists of a finite set of states $Q$ including a starting state $q_0$ and a halting state $q_{\text{halt}}$, a tape alphabet $\Gamma = \{\NULL,\EINS,{\triangleright},{\triangleleft}\}$ containing the input alphabet $\Sigma = \{\NULL,\EINS\}$, and a transition function
\[
\delta \colon (Q\setminus\{q_{\text{halt}}\}) \times \Gamma \to Q \times \Gamma \times \{{\rm L},{\rm R}\}.
\]
A configuration is a triple $(q,i,b_0b_1\dots) \in Q \times \ZZ \times \Gamma^*$, and if $\delta(q,b_i) = (q',c,X)$, then $M$ transitions as
\[
(q,i,\dots b_{i-1}b_ib_{i+1} \dots ) \Rightarrow \begin{cases}
  (q',i-1, \dots b_{i-1}c b_{i+1} \dots) & \text{if }X = {\rm L},\\
  (q',i+1, \dots b_{i-1}c b_{i+1} \dots) & \text{if }X = {\rm R}. 
\end{cases} 
\]
On the input string $w \in \Sigma^*$, the machine $M$ starts on the starting configuration $(q_0,1,{\triangleright}w_1\dots w_n {\triangleleft})$ (here $\triangleright$ is at position $0$) and repeatedly transitions according to $\delta$. The end markers $\triangleright, \triangleleft$ may never be overwritten and the head may not pass over them (e.g. $\delta(q,{\triangleleft}) = (q',{\triangleleft},{\rm R})$ is not allowed). $M$ halts if it eventually reaches a configuration of the form $(q_{\text{halt}},i,b)$.

\begin{table}[H]
  \centering
  \problemBox[\columnwidth]{\LBAhalt}
  {A LBA $M = (Q,\delta,q_0,q_{\text{halt}})$ and an input $w \in \{\NULL,\EINS\}^*$}
  {Does $M$ halt on input $w$?}
\end{table}

\begin{example}\label{exa:binaryLBA}
  The following LBA consists of two states $q_0,q_1$ (plus a halting state), and on input $w \coloneqq \NULL\dots \NULL$ ($n$ zeros) will count in binary to $\EINS \dots \EINS$ and then halt:
  \[
  \begin{array}{c|cccc}
  b & {\triangleright}               & \NULL               & \EINS               & {\triangleleft} \\ \hline \hline
  q_0 &                                & (q_1,\EINS,{\rm L}) & (q_0,\NULL,{\rm R}) & \text{halt}      \\
  q_1 & (q_0,{\triangleright},{\rm R}) & (q_1,\NULL,{\rm L}) &                     &
  \end{array}
  \]
Transitions that do not occur (on input $w$) have been left unspecified for simplicity.
\end{example}

A \emph{complexity upper bound} for a decision problem $A$ and a complexity class $\cclass C$ is simply an algorithm solving $A$ and satisfying the resource constraints of $\cclass C$. A \emph{lower bound} or $\cclass C$\emph{-hardness} result for $A$ is a proof that the computation of any problem $A' \in \cclass C$ can be reduced to the problem $A$. We use the notion of \emph{log-space many-one reductions}, in symbols $A' \leqlm A$, this means that there is a map $f$ from inputs of $A'$ to inputs of $A$ computable in logarithmic working space, such that $x \in A'$ if and only if $f(x) \in A$. For example, \SubsetSum and \OneInThreeSAT are \NP-complete and \LBAhalt is \PSPACE-complete.

In the problems $\IdealMem_\KK$ and $\AlgMem_\KK$ the input consists of encoded polynomials over $\KK$, for example as a string of characters, so the input length is bounded below by the number of terms, the number of variables occurring and (depending on the encoding) the coefficients and the (logarithm of the) degree. All our complexity results will hold with respect to both dense and sparse encodings and binary or unary monomial representation (with the exception of \Cref{thm:Monomial_bounded}), as long as we make the reasonable assumption that the field $\KK$ is \emph{well-endowed} \cite{Borodin1983}. Commonly used fields in computer algebra such as number fields and finite fields are well-endowed. For this reason, we will not elaborate further on the encoding.

%%%%%%%%%%%%%%%%%%%%%%%%%%%%%%%%%%%%%%%%%%%%%%%%%%%%%%%%%%%%%%%%%%%%%%%%
\subsection{Brief summary of the ideal world}\label{sec:ideal_world}
%%%%%%%%%%%%%%%%%%%%%%%%%%%%%%%%%%%%%%%%%%%%%%%%%%%%%%%%%%%%%%%%%%%%%%%%

In this section, we give an overview of the complexity results regarding ideal membership, representation degree, and Gröbner basis degrees. Our upper bounds on $\AlgMem_\KK$ will rely on refined representation and Gröbner basis degree upper bounds presented below.
A more comprehensive discussion of the complexity of ideals can be found in \cite{Mayr2017}.

\begin{theorem}\label{thm:idealmem_expspace}
Let $\KK$ be a well-endowed field (the lower bounds do not require this).
\begin{enumerate}
\item \textup{(Mayr \& Mayer \cite{Mayr1982}, Mayr \cite{Mayr1989})} The problem $\emph{\IdealMem}_\KK$ is complete for \EXPSPACE. A representation can be enumerated in exponential working space.
\item \textup{(Mayr \cite{Mayr1997})} The problem $\emph{\IdealMem}_\KK(\Homog)$ is complete for \PSPACE. The general problem is also in $\PSPACE$ when bounding the number of variables.
\end{enumerate}
All hardness results already hold for ideals generated by binomials $x^\alpha-x^\beta$.
\end{theorem}

\begin{remark}\label{rmk:double_exponential}
Note that this uses the convention that in an enumeration problem we only consider the working space and not the cells on the output tape. For example, one can enumerate the binary numbers $0,1,\dots,2^n-1$ with only about $n$ bits of working memory. Similarly, here the length of a certificate may be doubly exponential in the input length, this is unavoidable in the worst case \cite{Mayr1982}.
\end{remark}

Regarding representation degree bounds are in general doubly exponential in the number of variables, which is asymptotically optimal. We also note the special case of complete intersections for later reference. For a definition of complete intersections see \eg \cite[Exercise 9.4.8]{Cox2016} or \cite[Definition 3.2.23]{Kreuzer2000}.

\begin{theorem}\label{thm:hermann_bound}
Let $g \in \langle f_1,\dots,f_s \rangle \subseteq \KK[x_1,\dots,x_n]$, $d \coloneqq \max_i \deg f_i$, and write $g = \sum_{i=1}^s h_if_i$ with $h_i \in K[\x]$ of minimal representation degree $D \coloneqq \max_i \deg h_i$.
\begin{enumerate}
\item \textup{(Hermann \cite{Hermann1926}, Mayr \& Meyer \cite{Mayr1982})} $D \leq \deg g + (ds)^{2^n}$.
\item \textup{(Dickenstein, Fitchas, Giusti \& Sessa \cite{Dickenstein1991})} If $f_1,\dots,f_s$ define a complete intersection ideal, then $D \leq \deg g + d^s$.
\end{enumerate}
\end{theorem}

Both from a theoretical and computational point of view, effective degree bounds on \emph{Gröbner bases} are also important. We recall the classical \emph{Dubé bound} as well as a dimension-refined version. 

\begin{theorem}\label{thm:gb_dim_mayr_ritscher}
Let $I \subseteq \KK[x_1,\dots,x_n]$ be an ideal generated by $f_1,\dots,f_s$ of degree $d_i \coloneqq \deg f_i$, $d_1\geq \dots \geq d_s$, and consider any monomial order.
\begin{enumerate}
\item \textup{(Dubé \cite{Dube1990})} The degree of the reduced Gröbner basis $\textit{GB}$ of $I$, \ie the largest degree of an element of $\textit{GB}$, is bounded above by
\[
\deg \textit{GB} \leq 2 \left( \frac{d_1^2}{2} + d_1\right)^{2^{n-1}}.
\]
\item \textup{(Mayr \& Ritscher \cite{Mayr2013})} If $\dim \KK[\x]/I = r$, then the degree is bounded by
\[
\deg \textit{GB} \leq 2 \left( \frac{1}{2} (d_1\dotsm d_{n-r})^{2(n-r)} + d_1\right)^{2^r} \leq 2 \left( \frac{1}{2} d_1^{2(n-r)^2} + d_1\right)^{2^r}.
\]
\end{enumerate}
These upper bounds are asymptotically sharp.
\end{theorem}


%%%%%%%%%%%%%%%%%%%%%%%%%%%%%%%%%%%%%%%%%%%%%%%%%%%%%%%%%%%%%%%%%%%%%%%%
\section{Upper bounds on subalgebra membership}\label{sec:upper_bounds}
%%%%%%%%%%%%%%%%%%%%%%%%%%%%%%%%%%%%%%%%%%%%%%%%%%%%%%%%%%%%%%%%%%%%%%%%


In this section we provide various complexity upper bounds, that is, algorithms for variants of $\AlgMem_\KK$ which place these problems in \PSPACE and \EXPSPACE.
Our complexity analysis of the membership problem for a subalgebra $A = \KK[f_1,\dots,f_s] \subseteq \KK[\x]$ relies on the following classical elimination approach using tag variables, attributed to Spear \cite{Shannon1988, Spear1977}.

Let $\t = \{t_1,\dots,t_s\}$ be additional variables, and consider an \emph{elimination order} on $\KK[\x,\t]$, i.\,e.\ a monomial order such that $x_i \succ t^\alpha$ for all $x_i$ and $t^\alpha \in \KK[\t]$, for example the lexicographical order $\prec_{\text{lex}}$.

\begin{lemma}[Shannon \& Sweedler {\cite{Shannon1988}}]\label{thm:algmem_elimination}
Let $f_1,\dots,f_s,g \in \KK[\x]$, $J\coloneqq \langle f_1-t_1,\dots,f_s-t_s\rangle$, then
\[
g \in \KK[f_1,\dots,f_s] \qquad \text{if and only if} \qquad p \coloneqq \NF_\prec(g,J) \in \KK[\t].
\]
If this is the case, then $g = p(f_1,\dots,f_s)$, interpreting $p$ as a polynomial in $t_1,\dots,t_s$.
\end{lemma}

Subalgebra membership can thus be reduced to the task of calculating the normal form of a polynomial against an ideal in a larger polynomial ring. This approach is well-known, for example as a fall-back method in the \texttt{Macaulay2} \cite{M2} package \texttt{SubalgebraBases} \cite{Burr2024}.
Kühnle \& Mayr \cite{Kühnle1996} describe an exponential space algorithm for enumerating the normal form of a polynomial over a well-endowed field. Combining these results gives the first complexity upper bound:

\begin{theorem}\label{thm:algmem_expspace}
For any well-endowed field $\KK$, $\emph{\AlgMem}_\KK$ is in \EXPSPACE. Moreover a certificate $p \in \KK[t_1,\dots,t_s]$ such that $g = p(f_1,\dots,f_s)$ can be output in exponential working space.
\end{theorem}
\begin{proof}
Given $f_1,\dots,f_s,g$, the algorithm computes the normal form $p \coloneqq \NF_\prec(g,J) \in \KK[\x,\t]$ using Kühnle \& Mayr's algorithm \cite{Kühnle1996}. If there is a nonzero term in $p$ involving some $x_i$, then $g \notin \KK[f_1,\dots,f_s]$. Otherwise, $p$ is a certificate of subalgebra membership by \Cref{thm:algmem_elimination}, which can be enumerated to the output tape using single exponential working space.

Note that it might not be possible to fit the normal form in exponential working space as mentioned in \Cref{rmk:double_exponential}. Instead one has to enumerate the terms of $p$, which fit in exponential working space using and check for occurrences of $x_i$ individually.
\end{proof}

To prove better complexity upper bounds for special classes of polynomials we outline part of Kühnle \& Mayr's upper bound constructions on normal forms \cite{Kühnle1996}.

Let $I \subseteq \KK[x_1,\dots,x_n]$ be an ideal, $g \in \KK[\x]$ and ${\prec}\coloneqq {\prec_{\text{lex}}}$ the lexicographic order (for simplicity). Let $\mathit{GB} = \mathit{GB}(I)$ be the reduced Gröbner basis of $I$, then 
\[
\deg \NF_{\prec}(g,I) \leq \deg(g) + (\deg(\mathit{GB})+1)^{n^2+1} \deg(g)^n.
\]
Let $D$ be an upper bound on the degree of the coefficients $h_i \in K[\x]$ in a representation $g-\NF_{\prec}(g,I) = \sum_{i=1}^s h_if_i$, for example the Hermann bound from \Cref{thm:hermann_bound}, then Kühnle \& Mayr reduce the normal form calculation into a linear algebra problem of size $\operatorname{poly}(D)$. Using efficient parallel algorithms for matrix rank and the parallel computation hypothesis, this yields an algorithm in space $\operatorname{polylog}(D)$. Using the Dubé bound for $\deg(\mathit{GB})$ and the Hermann bound for $D$ yields the mentioned exponential space algorithm for normal form calculation.

In our specialized setting $J = \langle t_1-f_1,\dots,t_s-f_s \rangle$ we have more refined upper bounds available regarding representation and Gröbner basis degrees. We apply this to the case of a fixed number of variables $n$ and to the case of a fixed subalgebra $A$.

\begin{theorem}\label{thm:algmem_bounded_vars}
For a fixed subalgebra $A = \KK[f_1,\dots,f_s] \subseteq \KK[\x]$ over a well-endowed field, the membership problem is in \PSPACE (with respect to the input length of $g$). In fact, bounding the number of variables $n$ already implies $\emph{\AlgMem}_\KK(\KK[x_1,\dots,x_n]) \in \PSPACE$.
\end{theorem}
\begin{proof}
The algorithm is the same as in \Cref{thm:algmem_expspace}, only the complexity analysis is slightly more elaborate. Using that normal form calculation with respect to a fixed ideal is possible in polynomial space, as remarked by Kühnle \& Mayr \cite[Section 6]{Kühnle1996}, the first assertion follows.

If only the number of variables $x_1,\dots,x_n$ is fixed, then more care has to be taken, as the computation takes place in the ring $\KK[x_1,\dots,x_n,t_1,\dots,t_s]$, whose number of variables $n+s$ still depends on the input length. However, the ideal $J = \langle t_1-f_1,\dots,t_s-f_s\rangle$ is a complete intersection of dimension $n$, hence using \Cref{thm:gb_dim_mayr_ritscher} its Gröbner basis degree is bounded above by
\[
\deg \textit{GB}(J) \leq G \coloneqq 2 \left( \frac{1}{2} d^{2s^2} + d\right)^{2^n}.
\]
Furthermore, the representation degree of $g-\NF_\prec(g,J)$ is bounded above by $D \coloneqq G + d^s$ by \Cref{thm:hermann_bound}. Since $n$ is fixed, we see that $D$ is single exponential in the input length, hence $\operatorname{polylog}(D)$ is polynomial. This shows that the Meyer \& Kühnle algorithm works in polynomial space for a bounded number of variables.
\end{proof}

\begin{corollary}
If $g \in \KK[f_1,\dots,f_s] \subseteq \KK[\x]$, then there exists a certificate $h \in \KK[t_1,\dots,t_s]$ with $g = h(f_1,\dots,f_s)$ of degree
\[
\deg h \leq \deg(g) + \left(\left( \textstyle\frac{1}{2} d^{2s^2} + d\right)^{2^n}+1\right)^{(n+s)^2+1} \deg(g)^{n+s}.
\]
\end{corollary}
\begin{proof}
This is the degree bound on the normal form by Kühnle \& Mayr evaluated for the ideal $J$ (note that the ambient polynomial ring has $n+s$ variables).
\end{proof}
\begin{remark}
This degree can probably be improved by studying the derivation of the normal form degree bounds in this particular situation. We suspect that a \enquote{nicer} upper bound akin to the Hermann bound (\Cref{thm:hermann_bound}) should hold.
\end{remark}

Of great interest is the case of subalgebras generated by homogeneous polynomials. In this case, $A$ is graded compatibly with the standard grading of the ambient ring.

\begin{theorem}\label{thm:algmem_homog}
For any well-endowed field $\KK$, $\emph{\AlgMem}_\KK(\Homog)$ is in \PSPACE.
\end{theorem}
\begin{proof}
As a consequence of the grading, we see that a certificate of lowest degree (if it exists) has degree at most $\deg g$. Thus either $\deg \NF_\prec(h,J) > \deg d$, in which case $g \notin K[f_1,\dots,f_s]$, or $\deg \NF_\prec(h,J) \leq \deg d$ and $g \in \KK[f_1,\dots,f_s]$ if and only if $\NF_\prec(h,J) \in K[\t]$. This leads to a variation of the algorithm of \Cref{thm:algmem_expspace} except we only attempt to compute the normal form up to degree $\deg g$. The complexity analysis is analogous to \Cref{thm:algmem_bounded_vars}, noting that here a representation degree bound is $D \coloneqq \deg g + d^s$.
\end{proof}

\begin{remark}
Degree bounds on certificates for elimination problems have been studied before, for example by Galligo \& Jelonek \cite{Galligo2020elimination}. Their results are not necessarily applicable here, as our ideal $J$ has dimension $n$ in a ring of $n+s$ variables, while the results of Galligo \& Jelonek only apply to intersections of an ideal with $K[x_1,\dots,x_n]$ if the ideal has dimension at least $n+1$.
\end{remark}


%%%%%%%%%%%%%%%%%%%%%%%%%%%%%%%%%%%%%%%%%%%%%%%%%%%%%%%%%%%%%%%%%%%%%%%%
\section{Lower bounds for homogeneous subalgebras}\label{sec:lower_bounds}
%%%%%%%%%%%%%%%%%%%%%%%%%%%%%%%%%%%%%%%%%%%%%%%%%%%%%%%%%%%%%%%%%%%%%%%%


In this section, we prove a matching lower bound for the homogeneous subalgebra membership problem, similar to the construction for homogeneous ideals. We will describe a combinatorial problem we call \emph{controlled monomial replacement systems} (\CMRS), which is \PSPACE-hard, and which we can embed into the homogeneous subalgebra membership problem.
The hardness results for $\IdealMem_K$ and $\IdealMem_K(\Homog)$ were constructed by embedding the word problem for commutative semigroups resp.\ its homogeneous variant into binomial ideals \cite{Mayr1982, Mayr1997}. Our construction is similar but a bit trickier since the embedding of \CMRS into subalgebras is less natural.

Consider a set of variables $\x = \mathcal{C} \mathbin{\dot\cup} \mathcal{P}$ partitioned into $\mathcal{P}$adding variables and $\mathcal{C}$ontrol variables, $\mathcal{C} \neq \emptyset$. Given a set of binomial \emph{replacement rules} $\mathcal{R}= \{x^{\alpha_i} - x^{\beta_i}\}_{i=1}^r$ we define an equivalence relation $\equiv_{\mathcal R}$ on the set of monomials $\Mon(\x)$  generated by
\begin{equation} \label{eq:controlled_replacement}
x^\gamma x^\alpha \equiv_{\mathcal R} x^\gamma x^\beta \qquad  x^\alpha-x^\beta \in \mathcal{R},\quad x^\gamma \in \Mon(\mathcal{P}).
\end{equation}
The application of a single rule $R \in \mathcal{R}$ \enquote{from left to right} will be indicated by $m \equiv^R m'$. The application in the other direction is $m' \equiv^{-R} m$, which we call a \emph{reverse step} (assuming $\mathcal{R} \cap -\mathcal{R} = \emptyset$).

We call a polynomial $f$ \emph{controlled} with respect to a partition $\mathcal{C} =  \mathcal{C}_1 \mathbin{\dot\cup} \dots \mathbin{\dot\cup} \mathcal{C}_m$ if $f$ is homogeneous of degree one in each $\mathcal{C}_i$. This means that each monomial in $f$ contains exactly one variable from each $\mathcal{C}_i$, and otherwise only padding variables.

\begin{table}[H]
\centering
\problemBox[\columnwidth]{$\CMRS$, $\CMRS(\Homog)$}
{Variables $\x = \mathcal{C}_1 \mathbin{\dot\cup} \dots \mathbin{\dot\cup} \mathcal{C}_m \mathbin{\dot\cup} \mathcal{P}$, \newline
(homogeneous) controlled replacement rules $\mathcal{R}= \{x^{\alpha_i} - x^{\beta_i}\}_{i=1}^r$, \newline
controlled monomials $x^\alpha,x^\beta \in \Mon(\x)$}
{Is $x^\alpha \equiv_{\mathcal{R}} x^\beta$?}
\end{table}

In \cite[Theorem 17]{Mayr1997} Mayr describes how to simulate linearly bounded automata (\LBAhalt) using homogeneous commutative semigroups/ideals, proving \PSPACE-hardness of this and hence the homogeneous ideal membership problem. A similar idea applies to $\CMRS(\Homog)$: We encode a configuration $(q,i,b_0\dots b_{n+1})$ as the monomial $q h_i x_{0,b_0} x_{1,b_1} \dotsm x_{n+1,b_{n+1}}$. The state $q$ and head position $h_i$ are the control variables $\mathcal{C} = \mathcal{C}_1 \mathbin{\dot\cup} \mathcal{C}_2$, while the remaining $x_{i,b}$ are padding variables.

\begin{lemma}
The problems $\emph{\CMRS}$, $\emph{\CMRS}(\Homog)$ are \PSPACE-hard, even when the replacement rules have degree $\leq 3$.
\end{lemma}
\begin{proof}
We describe a reduction $\LBAhalt \leqlm \CMRS(\Homog)$, proving hardness. Given a LBA $M$ as in \Cref{sec:complexity} with states $Q = \{q_0,q_1,\dots,q_s,q_{\text{halt}}\}$ and transition function $\delta$, and an input word $w\in\{\NULL,\EINS\}^*$ of length $n$. Construct the following set of control and padding variables
\begin{gather*}
  \mathcal{C}_1 \coloneqq Q, \quad \mathcal{C}_2 \coloneqq \{h_0,\dots,h_{n+1}\}, \quad%\\
  \mathcal{P} \coloneqq \Set{x_{i,b} | 0\leq i \leq n+1,\ b \in \Gamma = \{\NULL,\EINS,{\triangleright},{\triangleleft}\}}.
\end{gather*}
We call a monomial of the form $qh_i x_{0,{\triangleright}} x_{1,b_1} \dotsm x_{n+1,{\triangleleft}}$ \emph{correctly formed}. Using the transition function $\delta$, we build the replacement rules $\mathcal{R}$ of two kinds:
\begin{enumerate}[label=(\arabic*)]
\item For each transition $\delta(q,b) = (q',c,{\rm L})$ add the rules $qh_i x_{i,b} - q'h_{i-1}x_{i,c}$ for all $i=0,\dots,n+1$. For a right movement $\delta(q,b) = (q',c,{\rm R})$ similarly add $qh_i x_{i,b} - q'h_{i+1}x_{i,c}$.
\item Add transitions $q_{\text{halt}}h_i x_{i,b} - q_{\text{halt}}h_i x_{i,\NULL}$ and $q_{\text{halt}}h_i - q_{\text{halt}}h_1$ for all $i=0,\dots,n+1$ and $b \in \Gamma$.
\end{enumerate}
These are homogeneous replacement rules controlled with respect to $\mathcal{C}_1$ and $\mathcal{C}_2$, and rules of the first kind preserve correctly formed monomials. Let $m_w \coloneqq q_0 h_1 x_{0,{\triangleright}} x_{1,w_1} \dotsm x_{n,w_n} x_{n+1,{\triangleleft}}$ and $m_{\text{end}} \coloneqq q_{\text{halt}} h_1 x_{0,\NULL} \dotsm x_{n+1,\NULL}$. Then the logspace-computable reduction function is
\[
(M,w) \; \mapsto \; (\mathcal{C}_1 \mathbin{\dot\cup} \mathcal{C}_2 \mathbin{\dot\cup} \mathcal{P}, \mathcal{R}, m_w, m_{\text{end}}).
\]

If $M$ halts on $w$ via a sequence of transitions, then applying the same sequence of rules of the first kind to the monomials we have $m_w \equiv_{\mathcal{R}} q_{\text{halt}}h_i x_{0,b_0} \dotsm x_{n+1,b_{n+1}}$ for some $i$ and $b_0,\dots, b_{n+1} \in \Gamma$. Applying the rules of the second kind we can clean up the final configuration by replacing all symbols by $\NULL$ and moving to position $1$, so $m_w \equiv_{\mathcal{R}} m_{\text{end}}$.

Conversely, assume that given $(M,w)$ we have $m_w \equiv_{\mathcal{R}} m_{\text{end}}$, we need to show that $M$ eventually reaches $q_{\text{halt}}$.
%The start monomial $m_w$ is of the form $q h_i x_{0,b_0} \dotsm x_{n+1,b_{n+1}}$, and the replacement rules preserve this structure. Furthermore, given a monomial $m$ of this form and a rule $x^\alpha - x^\beta \in \mathcal{R}$ there is at most one monomial $m'$ such that $m \equiv_{\mathcal{R}} m'$ in one step using $R$, since at most one of $x^\alpha,x^\beta$ can divide $m$.
Denote the sequence of monomials and applied replacement rules as
\begin{equation} \label{eq:rule_sequence}
m_w \equiv^{\pm R_1} m_1 \equiv^{\pm R_2} \cdots \equiv^{\pm R_l} m_l = m_{\text{end}}, \quad R_1,\dots,R_l \in \mathcal{R},
\end{equation}
and assume that $l$ is as small as possible. Let $e$ be the smallest index such that $m_e$ involves $q_{\text{halt}}$. Since $q_{\text{halt}} \nmid m_w$ but $q_{\text{halt}} \mid m_{\text{end}}$, we see that $R_1,\dots,R_e$ must be of the first kind and hence $m_1,\dots,m_e$ are correctly formed. Once we argue that none of the replacements is a reverse step $m_{i-1} \equiv^{-R_i} m_i$, the sequence \eqref{eq:rule_sequence} translates into a valid sequence of configurations and transitions of the LBA $M$.

So assume that the $i$-th replacement is a reverse step, and assume that $i \leq e$ is maximal with this property. The replacement $R_e$ must be a forward step by choice of $e$, hence $i<e$ and we have
\[
m_{i-1} \equiv^{-R_i} m_i \equiv^{R_{i+1}} m_{i+1}.
\]
In particular $m_i \equiv^{R_i} m_{i-1}$, but since the transition function $\delta$ is deterministic (\ie the transition relation is right-unique), we have $m_{i-1}
 = m_{i+1}$, contradicting minimality of $l$, the total length \eqref{eq:rule_sequence}.\end{proof}

The key idea for reducing \CMRS to subalgebra membership is the following: If $m \equiv^R m'$, then $m - m' = R\cdot x^\gamma \in \KK[\mathcal{P} \cup \mathcal{R}]$, and using the control property one can reverse this procedure. The precise statement is the following lemma, similar to \cite[Lemma 1 \& 2]{Mayr1982}. 

\begin{lemma}\label{lem:cmrs_to_algebra}
Given a controlled monomial replacement system $(\x = \mathcal{C}_1 \mathbin{\dot\cup} \dots \mathbin{\dot\cup} \mathcal{C}_m \mathbin{\dot\cup} \mathcal{P},\,\mathcal{R})$ and controlled monomials $x^\alpha,x^\beta \in \Mon(\x)$ as in the definition of \emph{\CMRS}. The following are equivalent:
\begin{enumerate}[label=\textup{(\alph*)}]
\item $x^\alpha \equiv_{\mathcal{R}} x^\beta$;
\item $x^\alpha - x^\beta \in \ZZ[\mathcal{P} \cup \mathcal{R}] \subseteq \ZZ[\x]$;
\item $x^\alpha - x^\beta \in \KK[\mathcal{P} \cup \mathcal{R}] \subseteq \KK[\x]$.
\end{enumerate}
\end{lemma}

\begin{proof}
The implications (a)$\Rightarrow$(b)$\Rightarrow$(c) are straightforward.
For the implication (c)$\Rightarrow$(a) let $\mathcal{P} = \{x_1,\dots,x_k\}$, $\mathcal{R} = \{R_1,\dots,R_r\}$ and let $p=p(t_1,\dots,t_k,\tilde{t}_1,\dots,\tilde{t}_r)$ be a certificate for membership $g \coloneqq x^\alpha - x^\beta \in \KK[\mathcal{P} \cup \mathcal{R}]$, that is, $g = p(x_1,\dots,x_k,R_1,\dots,R_r)$.
Since $x^\alpha - x^\beta,R_1,\dots,R_r$ are controlled polynomials, setting the variables in $\mathcal{C}$ to zero yields the identity
\[
0 = g(x_1,\dots,x_k,0,\dots,0) = p(x_1,\dots,x_k,0,\dots,0).
\]
Hence every term in $p$ involves at least one $\tilde{t}_i$, so $g$ is a linear combination of multiples of elements from $\mathcal{R}$, in other words $g \in \langle \mathcal{R} \rangle$. This allows us to apply the aforementioned analogous result for ideals \cite[Lemma 2]{Mayr1982}, which says that $x^\alpha \equiv_{\mathcal{R}} x^\beta$ disregarding control variables (so $x^\gamma \in \Mon(\x)$ in \Cref{eq:controlled_replacement}). On the other hand, since $x^\alpha$ contains exactly one variable from each $\mathcal{C}_i$ and the same is true for all rules $\mathcal{R}$, no control variable can be used in $x^\gamma$ anyway, hence $x^\alpha \equiv_{\mathcal{R}} x^\beta$ in the controlled sense too.
\end{proof}

Combining these two results gives the lower bound on the complexity of $\AlgMem_\KK$.

\begin{theorem}\label{thm:PSPACE_hard}
The map
\[
(\x = \mathcal{C}_1 \mathbin{\dot\cup} \dots \mathbin{\dot\cup} \mathcal{P}, \mathcal{R}, x^\alpha, x^\beta) \; \mapsto \; (\{f_i\}_i = \mathcal{P} \cup \mathcal{R}, g=x^\alpha - x^\beta)
\]
provides reductions $\emph{\CMRS} \leqlm \emph{\AlgMem}_\KK$ and $\emph{\CMRS}(\Homog) \leqlm \emph{\AlgMem}_\KK(\Homog)$. In particular, over any field $\emph{\AlgMem}_\KK(\Homog)$ is \PSPACE-hard, even when restricting the algebra generators to single variables and binomial of degree $\leq 3$.
\end{theorem}

\begin{remark}
It remains to prove a matching exponential space complexity lower bound for the general subalgebra membership problem. In the ideal case, the construction for commutative semigroups in \cite{Mayr1982} uses a reduction from exponentially bounded counter machines, and relies crucially on the construction monomials of size $2^{2^k}$ using $\Landau(k)$ replacement rules. In future work, it might be possible to translate this construction into (a variant of) \CMRS, and we conjecture that indeed $\AlgMem_\KK$ is \EXPSPACE-complete.
\end{remark}

The construction of this section has an interesting consequence outside of computational complexity theory.

\begin{theorem}
There exist a subalgebra  $A = \KK[f_1,\dots,f_{5n}] \subseteq \KK[x_1,\dots,x_{3n+3}]$ generated by single variables and homogeneous binomials of degree $\leq 3$, and a binomial $g$ of degree $n+2$ with the following property:
$g \in A$, but every polynomial $p \in \KK[\t]$ with $g=p(f_1,\dots,f_{5n})$ has at least $2^{n+1}$ terms.
\end{theorem}
\begin{proof}
We will use the binary-counting LBA from \Cref{exa:binaryLBA}.
Using the reduction to $\CMRS(\Homog)$ and \Cref{lem:cmrs_to_algebra} we obtain a binary-counting subalgebra. Removing redundant transitions (for example, we don't actually need ${\triangleright},{\triangleleft}$ since the head position variable knows the current position), one obtains a subalgebra with the desired properties: 
\begin{align*}
\mathcal{C} = \{q_0,q_1\} &\mathbin{\dot\cup} \{h_0,\dots,h_n\}, \quad \mathcal{P} = \{x_{1,\NULL},x_{1,\EINS},\dots,x_{n,\NULL},x_{n,\EINS}\}, \\
\mathcal{R} &\coloneqq \set{q_0h_ix_{i,\NULL} - q_1h_{i-1}x_{i,\EINS} | 1\leq i \leq n} \\
        &{}\; \cup \set{q_0h_ix_{i,\EINS} - q_0h_{i+1}x_{i,\NULL} | 1\leq i \leq n-1} \\
        &{}\; \cup \set{q_1h_ix_{i,\NULL} - q_1h_{i-1}x_{i,\NULL} | 1\leq i \leq n} \\
        &{}\; \cup \set{q_1h_0 - q_0h_1}, \\
  A &\coloneqq \KK[f_1,\dots,f_{5n}] \coloneqq \KK[\mathcal{P} \cup \mathcal{R}], \\
  g &\coloneqq q_0h_1x_{1,\NULL}\dotsm x_{n,\NULL} - q_0h_nx_{1,\NULL}\dotsm x_{n-1,\NULL}x_{n,\EINS}.
\end{align*}
Then $g \in A$ by construction, since the binary counter will go from $(q_0,1,\triangleright\NULL \dots \NULL \triangleleft)$ to $(q_0,1,\triangleright\EINS \dots \EINS \triangleleft)$ and then walk to the right, erasing the $\EINS$'s until it reaches the configuration $(q_0,n,\triangleright\NULL \dots \NULL \EINS \triangleleft)$. On the way it writes every number $0,\dots,2^{n-1}$ on the tape, taking at least $2$ steps each time, for a total of $\geq 2^{n+1}$ steps. By \Cref{lem:cmrs_to_algebra} any certificate for $g \in A$ must essentially contain this derivation, hence has at least $2^{n+1}$ terms.
\end{proof}

An implementation of the homogeneous binary-counting subalgebra in \texttt{Macaulay2} can be found at \href{https://mathrepo.mis.mpg.de/ComplexityOfSubalgebras}{\nolinkurl{mathrepo.mis.mpg.de/ComplexityOfSubalgebras}}.


%%%%%%%%%%%%%%%%%%%%%%%%%%%%%%%%%%%%%%%%%%%%%%%%%%%%%%%%%%%%%%%%%%%%%%%%
\section{Monomial subalgebras and SAGBI bases}\label{sec:monomials}
%%%%%%%%%%%%%%%%%%%%%%%%%%%%%%%%%%%%%%%%%%%%%%%%%%%%%%%%%%%%%%%%%%%%%%%%


In this final section, we consider the complexity of monomial algebra membership and consider some questions related to SAGBI bases.
Monomial subalgebras $A$ are $\NN^n$-graded in the sense that
\[
A = \bigoplus_{\substack{\alpha \in \NN^n\\ x^\alpha \in A}} \KK x^\alpha \subseteq \KK[\x].
\]
This has the useful consequence that a polynomial $\sum_{\alpha} c_\alpha x^\alpha$ is in $A$ if and only if every monomial $x^\alpha$, $c_\alpha\neq 0$, is in $A$. Furthermore, monomial algebras are related to linear programming and linear Diophantine equations \cite[Remark 1.9]{Robbiano1990}, as
\begin{equation}\label{eq:diophantine}
  x^\beta \in K[x^{\alpha_1},\dots,x^{\alpha_s}] \quad \Longleftrightarrow\quad \exists c \in \NN^s\text{ s.t.\ } \beta = \sum_{i=1}^s c_i\alpha_s.
\end{equation}

\begin{theorem}\label{thm:Monomial_gen}
For any $\KK$ the problem $\emph{\AlgMem}_{\KK}(\Mon)$ is \NP-complete. This is true even when restricting to square-free monomials.
\end{theorem}
\begin{proof}
For \NP membership one immediately reduces to the case where the polynomial to test $f$ is a monomial due to the $\NN^n$-grading. In \Cref{eq:diophantine} we have $c_j \leq \max_i \beta_i$, so the bit length of $c_j$ is bounded by the bit length of $\beta$. Hence non-deterministically guessing $c$ yields a \NP-algorithm.

For \NP-hardness one can reduce from the \NP-complete problem  \OneInThreeSAT (\Cref{sec:complexity}). Indeed, given sets $S_1,\dots,S_n \subseteq \{1,\dots,s\}$, then let $\alpha_1,\dots,\alpha_s \in \{0,1\}^n$ be the integer vectors with $(\alpha_i)_j = 1$ when $i \in S_j$. Set $\beta = (1,\dots,1) \in \NN^n$. Then \Cref{eq:diophantine} encodes exactly the subset sum problem, a solution corresponding to $T = \set{i | c_i = 1}$. In this construction all monomials $x^{\alpha_i},x^\beta$ are square-free.
\end{proof}

We see that $\AlgMem_K(\Mon)$ is \NP-complete even for polynomials of bounded degree. The same is true if we instead bound the number of variables -- if the exponents are encoded in binary.

\begin{theorem}\label{thm:Monomial_bounded}
The problem $\emph{\AlgMem}_\KK(\Mon(x_1,\dots,x_n))$ for \emph{fixed} $n\geq 1$ is \NP-complete for binary exponent encoding and in \TCO with unary encoding.
\end{theorem}
Here $\TCO \subsetneq \DP$ is a low uniform circuit complexity class.
\begin{proof}
Encoding the exponents as binary, the unary case $n=1$ is a direct translation of the \problemText{SubsetSum} problem, which is \NP-hard.
On the other hand, if the monomials are encoded in unary, then a generating-function approach as in \cite{Kane2017}  provides a family of circuits in \TCO deciding $\AlgMem_K(\Mon(x_1,\dots,x_n))$. Alternatively one can apply (a variant of) Courcelle’s theorem \cite{Elberfeld2012} to obtain \TCO-membership.
\end{proof}

\begin{remark}
The \NP-hardness results from \Cref{thm:Monomial_gen} and \ref{thm:Monomial_bounded} are in stark contrast to the analogous results for monomial \emph{ideals}: Monomial ideal membership is computationally trivial, as $x^\beta \in \langle x^{\alpha_1},\dots,x^{\alpha_s} \rangle$ if and only if component-wise $\beta \geq \alpha_i$ for some $i$.
\end{remark}

%%%%%%%%%%%%%%%%%%%%%%%%%%%%%%%%%%%%%%%%%%%%%%%%%%%%%%%%%%%%%%%%%%%%%%%%
\subsection{Some remarks on SAGBI bases} \label{sec:SAGBIconjecture}
%%%%%%%%%%%%%%%%%%%%%%%%%%%%%%%%%%%%%%%%%%%%%%%%%%%%%%%%%%%%%%%%%%%%%%%%

We return to a general subalgebra $A = \KK[f_1,\dots,f_s] \subseteq \KK[\x]$ equipped with a monomial order $\prec$. In \Cref{sec:ideals} we defined the concept of SAGBI bases $S \subseteq A$ as subsets with $\KK[\set{\ini_\prec f | f \in S}] = \ini_\prec A$.
SAGBI stands for \enquote{Subalgebra Analog to Gröbner Bases for Ideals} \cite{Robbiano1990} and they are used in practice for deciding subalgebra membership \cite{Burr2024}. Much of the theory of Gröbner bases is paralleled for SAGBI bases, such as the \emph{subduction algorithm} deciding subalgebra membership, reminiscent of the division algorithm for Gröbner bases \cite[Section 6.6]{Kreuzer2005-rb}.

The previous results in this section provide a modest complexity lower bound of deciding subalgebra membership in the presence of SAGBI bases. The author is not aware of an analogous result for ideal membership given a Gröbner basis.

\begin{corollary}
The problem $\emph{\AlgMem}_K$ is \NP-hard, even if the input polynomials $f_1,\dots,f_s$ form a finite SAGBI basis of $\KK[f_1,\dots,f_s]$.
\end{corollary}
\begin{proof}
Subalgebras generated by a finite set $S$ of monomials have $S$ as a SAGBI basis. Hence \Cref{thm:Monomial_gen} provides the \NP lower bound.
\end{proof}

A major difference is that not every finitely generated subalgebra has a \emph{finite} SAGBI basis.
\begin{example}
The subalgebra $\KK[x_1,x_1x_2-x_2^2,x_1x_2^2] \subseteq \KK[x_1,x_2]$ has the non-finitely generated initial algebra $\KK[\set{x_1x_2^k|k\geq 0}]$ for any monomial order with $x_1\succ x_2$ \cite[Example 4.11]{Robbiano1990}.
\end{example}

We therefore propose the study of the following two decision problems related to the initial algebra.

\begin{table}[H]
    \centering
\problemBox[0.48\columnwidth]{$\problemText{SAGBIfinite}_\KK(\textsf{C})$, $\problemText{SAGBIfinite}$}
{$f_1,\dots,f_s \in \textsf{C}$ (or $\KK[\x]$)} 
{Does $\KK[f_1,\dots,f_s]$ have a finite SAGBI basis?}\problemBox[0.48\columnwidth]{$\IniAlgMem_\KK(\textsf{C})$, $\IniAlgMem_\KK$}
{$f_1,\dots,f_s \in \textsf{C}$ (or $\KK[\x]$),\newline $x^\alpha \in \Mon(\x)$}
{Is $x^\alpha \in \ini_\prec \KK[f_1,\dots,f_s]$?}
\end{table}

Robbiano \& Sweedler showed that an algorithm somewhat analogous to Buchberger's algorithm for Gröbner bases can be used to enumerate a SAGBI basis, which will terminate if (and only if) $A$ has a \emph{finite} SAGBI basis. A procedure that returns \texttt{yes} if the output is yes, but never terminates if the output is no, is called a \emph{semi-algorithm}, and a problem is \emph{semi-decidable} or \emph{recursively enumerable} if there is a semi-algorithm \enquote{solving} it.

\begin{theorem}[Robbiano \& Sweedler \cite{Robbiano1990}]
$\emph{\IniAlgMem}_\KK$ and $\emph{\problemText{SAGBIfinite}}_\KK$ are semi-decidable (over a computable field).
\end{theorem}

We are not aware of any general better complexity bounds, or even just if these problems are computable at all (though a negative answer would be quite surprising). Future work will provide a more detailed study of the structure and complexity of (infinitely generated) initial algebras.


%%%%%%%%%%%%%%%%%%%%%%%%%%%%%%%%%%%%%%%%%%%%%%%%%%%%%%%%%%%%%%%%%%%%%%%%
\section*{Acknowledgments}
%%%%%%%%%%%%%%%%%%%%%%%%%%%%%%%%%%%%%%%%%%%%%%%%%%%%%%%%%%%%%%%%%%%%%%%%

I would like to thank my advisor Simon Telen as well as Markus Bläser, Peter Bürgisser, Florian Chudigiewitsch, and Fulvio Gesmundo for helpful discussions, in particular Florian for suggesting Courcelle's theorem for \Cref{thm:Monomial_bounded}. My interest in the complexity of ideals and subalgebras originated from my Master's thesis \cite{kayser2022groebnercomplexity} supervised by Heribert Vollmer \& Sabrina Gaube at Leibniz University Hannover.

\printbibliography

\subsection*{Author's address:}

\noindent Leonie Kayser, MPI-MiS Leipzig \hfill \href{mailto:leo.kayser@mis.mpg.de}{\tt leo.kayser@mis.mpg.de}

\end{document}          
