
\section{Experiments}

\begin{figure*}[t]
    \centering
 \begin{tabular}{cccccc}
    \hspace{0.6cm} \small \textbf{Source Image} & \hspace{.6cm} \small \textbf{Style Image} & \hspace{0.5cm}\small \ours (Ours) & \hspace{.0cm}\small Cross-Image-Attn. & \hspace{0.2cm}\small IP-Adapter SDXL & \hspace{0.1cm}\small StyleID \\
    \multicolumn{6}{c}{%
        \includegraphics[width=0.99\linewidth]{figures/images/2d_transfer.pdf}%
    } \\

\end{tabular}  
   \caption{\textbf{Image appearance transfer results}. Our method enables precise appearance transfer between semantically corresponding elements, evidenced by the green rug and glass table (first row), textured cabinet (second row), and bedsheets (third row). Unlike baselines that either apply global style transfer or fail to preserve structure, \ours maintains both semantic fidelity and structural integrity.}
    \label{fig:2d_transfer}
\end{figure*}

\paragraph{\bf{Implementation Details.}}
We base our semantic attention module on Stable Diffusion 1.5~\cite{rombach2022latentdiffusion} and the refinement and 2-view warp-and-refine model on SDXL~\cite{podell2024sdxl}. To train our two-view warp-and-refine model~(Sec.~\ref{sec:multiview_style_propagation}), we use 4 NVIDIA A100 40GB GPUs with an effective batch size of 256 for 20K iterations, using the AdamW optimizer~\cite{loshchilov2018adamw} with learning rate $10^{-4}$. We randomly drop out half of the text prompt during training to make our model agnostic to text conditions. The model is trained on a dataset with 57K house tour images featuring 57 different houses/apartments.

\subsection{Evaluation Setting}
\paragraph{\bf{Dataset.}} Our \dataset benchmark comprises 31 distinct indoor scenes captured as short video clips, totaling 15,778 frames across multiple room categories, including living rooms, kitchens, and bedrooms, all disjoint from our training data. To evaluate stylization capabilities, we curated a set of 25 interior design reference images, enabling 243 unique style-scene combinations. Evaluation is performed on 1,109 keyframes sampled from these clips. For more details on data, please refer to the supplementary material (Supp.).


\paragraph{\bf{Evaluation Metrics.}}
We evaluate multiple different aspects of our pipeline.
First, we assess the appearance transfer performance using source images on two aspects: structure preservation and style transfer quality.
For structure preservation, we compare depth maps predicted by DepthAnythingV2~\cite{yang2024depth} between stylized and original images using standard metrics: Absolute Relative Error~(AbsRel), $\delta1$ accuracy, and Squared Relative Error~(SqRel), following established protocols~\cite{ke2023marigold, yang2024depth}.
For style transfer quality, we measure perceptual similarity between the stylized output and the style image using DINOv2~\cite{oquab2023dinov2}, CLIP, and DreamSim~\cite{fu2023dreamsim} scores. We evaluate this task on the stylized source images of each scene.
Next, we evaluate our two-view lifting model~(Sec.~\ref{sec:multiview_style_propagation}).
We assess its warp-and-refine quality using PSNR, SSIM~\cite{zhou2004ssim}, and LPIPS~\cite{zhang2018lpips} while also reporting FID~\cite{martin2017fid} to quantify the realism of generated frames under challenging viewpoint extrapolation. We evaluate using pairs of the source images per scene and their warped projections on the rest of the frames in each scene---we exclude pairs without correspondences. We do not use any stylization to train or evaluate since there is no ground truth.
To evaluate global consistency, we leverage DUSt3R~\cite{wang2024dust3r} to extract poses by aligning point maps from stylized sequences and compute cumulative error curve~(AUC) by comparing recovered camera poses against those from original images.





\subsection{Results}
\paragraph{\bf{Image Appearance Transfer.}}
We compare with three state-of-the-art methods on image-conditioned stylization and appearance transfer: Cross Image Attention~\cite{alaluf2024cross}, IP-Adapter~\cite{ye2023ip}, and StyleID~\cite{chung2024styleid}. For a fair comparison, we add depth ControlNet~\cite{zhang2023adding} to SDXL IP-Adapter~\cite{ye2023ip} and use the style image as the image prompt. As shown in Tab.~\ref{tab:2d_transfer}, our method achieves superior performance on both structure preservation and style transfer metrics. Notably, our explicit semantic attention mechanism in the diffusion UNet enhances the perceptual similarity between stylized outputs and style images, as evidenced by better DINO, CLIP, and DreamSim scores. The refinement step further improves structure preservation, reducing AbsRel from 11.30 to 8.34 and SqRel from 2.65 to 1.67.
Qualitative comparisons (Figs.~\ref{fig:2d_transfer} and~\ref{fig:2d_transfer_more}) reveal the limitations of existing approaches. Cross Image Attention effectively captures style textures but fails to maintain scene structure due to the lack of semantic guidance. IP-Adapter SDXL preserves overall structure but struggles with local detail transfer, as it compresses style information into a global feature vector. Although StyleID achieves the second-best performance, its results tend to preserve high-frequency details from the source image while applying style changes more globally, demonstrating limited capability in fine-grained appearance transfer.

We conduct a user study with 27 participants who were shown examples of a source and style image with outputs from four methods.
Participants selected the result that best preserved the structure while faithfully transferring the style.
Out of 252 evaluations (Tab.~\ref{tab:user_study}), \ours was the most preferred (42.4\%), demonstrating its effectiveness in balancing structure preservation and appearance transfer under human perception.

\section{User evaluation with frequent users of mobile ASR: Lab study and online survey }
To evaluate the usability of our approach, we decided to conduct an in-person lab evaluation of the SpeechCompass phone case and the speech-to-text application (described in Section~\ref{subsection:app}), with frequent users of mobile transcription technology. We first conducted a large-scale online pilot study to inform the design of the in-person lab evaluation, which we conducted with eight deaf or hard-of-hearing participants, set up to mimic a realistic conversation scenario. 

\begin{figure*}
  \centering
  \includegraphics[width=0.75\linewidth]{images/second_study.pdf}
  \caption{Participants' preferences for different visualization techniques in the online survey. A) Results indicating how valuable the specific indicator would be for the user. B) Preferences for the specific indicators for speech direction.} 
  \label{fig:user_preferences_online} 
\end{figure*}


\subsection{Large-scale, online survey (n=494)} In this survey, we use screenshots of our interactive UI prototypes to solicit initial user
feedback on the potential for our proposed approach, to guide the design of a more realistic in-person lab study.

The study was conducted using the same Google Surveys deployment and screening methodology as for the foundational study, detailed in Section 3. The participants were shown different UI renderings and were asked to rate them. The large-scale online survey could only show static images of the interfaces, due to limitations of the survey tool. Out of 985 respondents we focus our analysis on the 494 participants who use captioning technology multiple times per week or more frequently. 

As shown in Figure~\ref{fig:user_preferences_online}A, the colored text was found to be valuable by 60\% of participants. Glyph indicators for speech direction, which included arrow and circle+line indicators, were found valuable by 70\%. The Edge indicator and the mini map had a less positive reception. 

To better understand which glyph indicators were favored, we also asked targeted questions about them, as shown in Figure~\ref{fig:user_preferences_online}B. \emph{Circle + line} was preferred by 13.1\% more respondents than the \emph{highlight box} (45.1\% vs 32.0\%), and the \emph{arrow} was preferred by 21.9\% more respondents than the \emph{circle + line} (51.2\% vs 29.3\%).


\subsection{Lab study (n=8)}
\alex{explain and emphasize intention}
We recruited 8 participants from our institution who were frequent users of captioning technology. Five were female, three were male, and all were deaf or hard of hearing. One participant was 25--34 years old, four were 34--44, one was 45--53, and two were 65+ (we are only allowed to collect age ranges at our institution). 


% setup: https://docs.google.com/document/d/1akr5HVMgJb8Kd9KaEZJcdXn2S0IbHhd8JdBPTE0TiA0/edit?usp=sharing
The study took place in a quiet lab over approximately 60 minutes and used the phone-case prototype (Figure~\ref{fig:pcb_design}) with our mobile ASR application (Figure~\ref{fig:phone_interfaces}). First, the participant was introduced to the technology, prototype, and the purpose of the study. Then, the participant was asked to fill out a background survey, which included demographic questions and their current use and experienced challenges with transcription technology. Afterward, the participant was introduced to different visualization scenarios with the SpeechCompass application. The participant used the SpeechCompass transcription while sitting between the two experimenters, as they all sat around a small table with the SpeechCompass phone case in the center. In each of the seven conditions, which ran for 5 minutes, the experimenters sat across from each other and had short conversations about different topics. The participants were instructed to turn off hearing aid devices if they used any, and were asked to use the SoundCompass UI and transcript to follow the conversation. The experimenters' casual conversations included topics like weekend plans, hobbies, and the weather. The seven conditions, which used the ASR, diarization, and localization functionality for different visualization techniques, are shown in Figure~\ref{fig:ui_options} and presented with more UI context in Figure~\ref{fig:phone_interfaces}. The conditions were:
\begin{enumerate}
    \item \textbf{Transcription only}. The transcribed text is shown in white on a black background. 
    
    \item \textbf{Edge indicator}. A circle (``dot'') that moves around the edge of the screen to point to the currently active speaker. The color of the dot changes based on the direction. 
    
    \item \textbf{Arrow indicator}. A glyph using a colored arrow next to a white text block. The glyph points in the direction of the associated speech. 
    
    \item \textbf{Circle + line indicator}. A glyph using a circle with a directional line next to a white text block. The glyph points in the direction of the speech associated with the text. 
    
    \item \textbf{Mini map}. A colored circle with a smaller circle (``dot'') moves around its edge to point to the currently active speaker. The color of the dot changes based on the direction. 
    
    \item \textbf{Colored text}. The text is colored based on the direction that the associated speech was coming from. 
    
    \item \textbf{Everything on}. All indicators are turned on (except the Circle + line, as it couldn't be used simultaneously with the arrow). 
\end{enumerate}

%five isolated visualization techniques, baseline with just text transcription (no speaker information), and with all visualization turned one. Minimap was shown with an arrow, since we envisioned it would be combined with other techniques. 
After participants had completed all conditions, they filled out a form that asked them to rate how desirable each of the five visual indicator styles (\textit{Edge indicator}, \textit{Arrow}, \textit{Circle  + line}, \textit{Colored map}, and \textit{Colored text}) were on a 7-point Likert scale, from \emph{-3: Strongly dislike} to \emph{+3: Strongly like}. Finally, they were asked to rate the overall value of directional feedback to the transcription experience, how strongly they would recommend these features to users of mobile captioning, and whether they had any general free-form feedback about SpeechCompass. 

\begin{figure*}
  \centering
  \includegraphics[width=0.65\linewidth]{images/study_setup.png}
  \caption{Examples of seven visualization scenarios that participants experienced in the in-person study.} 
  \label{fig:ui_options} 
\end{figure*}

%After running the scenarios, participants filled out the second part of the survey, which asked them to rate each scenario and overall impression on a scale from -3:strongly dislike to +3:strongly like. Finally, the participants filled out free form feedback about the study. 

\begin{figure*}
  \centering
  \includegraphics[width=0.65\linewidth]{images/box_plot_in_person_study_results.png}
  \caption{Boxplots of results of the in-person study. A) Participants' preferences for different visualization techniques. B) Overall opinions about augmented mobile ASR application.\alex{love these plots -- maybe to B you could also add the question about multi-people conversations as the leftmost, since it is also on same scale?} } 
  \label{fig:user_preferences} 
\end{figure*}

\subsection{Results}
Mobile transcription apps (e.g., Android Live Transcribe) were the most used communication technology for the participants. Specifically, three used them multiple times per day, one used them daily, three used them multiple times per week, and one used them rarely. 

75\% of participants frequently experienced the scenario where multiple people would get mixed up in the transcript (two multiple times per day, two daily, two multiple times per week). All participants agreed that it was challenging to participate in conversations when speech was combined from multiple people. 
%Similarly to the online survey, we asked participants to select the biggest challenges they experienced in their use of transcription technology (same options as in Figure~\ref{fig: survey-challenges}). where the majority (6/8) selected \textit{"Have to look away from the person I am talking to"}.  
\\

A Kruskal-Wallis (KW) test found a significant effect
on participant preferences for visualization techniques (P=.014).
The post-hoc pair-wise analyses using the Wilcoxon test with Bonferroni correction did, however, not show statistical significance between any pairs.
Of the five visual indicator styles that participants experienced, \emph{Colored text} was the most well-received (mean ($\bar{x})=2.625$), as it was rated positively by all the participants. %, with six strong like (+3), one like (+2), and one slight like (+1). 
The \emph{Arrow} indicator was also well-received ($\bar{x}=1.125$), with six positive, one negative, and one neutral participant.
%(one strong like (+3), three like (+2) and one slight like (+1)) and one dislike (-1) and one neutral (0)). 
Several participants noted that \emph{Arrow} and \emph{Colored text} worked well together: \emph{"Arrows + color seem to be most easier way to indicate the direction." (P2)} and \emph{"The combination of the colored text with the arrow was the most effective for me." (P7)}.

The other indicator styles received more mixed feedback. The feedback for both \emph{Edge indicator} ($\bar{x}=0.25$) and \emph{Circle + line} ($\bar{x}=-0.125$) was split between four negative and four positive participants. 
Some participants were concerned that \emph{Edge indicator} was distracting and not sufficiently discreet: \textit{"I do prefer the tool be as discrete as possible and would perhaps choose to avoid bright colored things moving around since this would be eye-catching and this kind of attention is often undesired" (P3)} and 
\textit{"Indicator moving around the edge was distracting and causing a bit of eye strain" (P2)}.
On the other hand, another participant found this style particularly useful: \textit{"the color dot moving to the speaker direction worked REALLY well" (P1)}. 
For \emph{Circle + line}, some participants struggled with its legibility: \textit{"If the analog direction indicators were larger (and translucent, or set behind)" (P8)} and \textit{"The lines in a circle were a bit slower and not as accurate (buggy)" (P5)}.
The \emph{Mini map} was rated positively by five participants and negatively by three. The most favorable participant stated: \emph{"this is also great for environmental awareness for those with single-sided hearing or no hearing at all." (P3)} and a participant who disliked the \emph{Edge indicator} commented: \emph{"steady map in the corner worked a bit better (P5)"}.

Overall, all participants agreed with the value of directional feedback ($\bar{x}=2.88$, seven Strongly agree:+3 and one Agree:+2) and would recommend these features to other users of captioning technology ($\bar{x}=2.63$, five Strongly agree:+3 and three Agree:+2): \textit{"I really liked that almost immediately I could tell that there was a speaker change, so that as soon as the text started to show up, I could better contextualize that text as attributed to a new speaker." (P1)}, \textit{"I'm very happy to see this tool being developed, it's a great addition to other speech recognition tools!" (P3)}, and \textit{"This prototype is definitely a life changer and I strongly believe that it will improve the quality of access to communication with speakers for many users" (P6)}.

\subsection{Discussion}
Consistent with the large-scale survey, the value of the diarization and localization features was immediate to all users. The participants were asked if directional guidance would be valuable in their mobile transcription experience. All eight users agreed. Also, all eight users would recommend this feature to mobile captioning users. 

While the large-scale survey helped inform our testing and exclude conditions (e.g., \emph{Highlight box}), the lab study allowed us to more rigorously evaluate the techniques in a realistic scenario. This difference became significant for the \emph{Edge indicator} and \emph{Mini map}, where issues, such as discreetness and distracting aspects, became evident during live usage. 

The results suggest that the combination of \textit{Colored text} and \textit{Arrow} would meet the preferences of most users, thanks to the balance of directional encoding and clarity. The arrow has redundant benefits too, since colored text might not always be reliably visible depending on lighting and screen conditions (e.g., strong sunlight, or dim display) and might also not be usable for colorblind users. The mixed feedback for other techniques indicates that the interface may also benefit from mechanisms that would allow users to customize the visualization style. Such customization could also apply to rendering properties, such as color, transparency, and line thickness, as some participants found \textit{Circle + line} particularly difficult to interpret. In both the large-scale survey and the in-person lab study, the \textit{Arrow} was preferred over \textit{Circle + line}. Through more customization options and extended usage in their daily lives, participants will be able to provide more nuanced feedback about these techniques. 


% Edge indicator and mini map had a less positive reception. However, they were rated more positively than those in the in-person study. Since participants didn't experience the working prototype, the discreet and distracting aspects that were observed in the in-person study were not captured. 

% In both online and in-person study, the arrow directional glyph was preferred to circle+line.



% This dichotomy demonstrates that users should be given a way to customize their experience. For example, the edge indicator received strong likes and dislikes from different participants. 


% This indicates that the interface designers should make the directional glyphs as easy to read as possible.


% The results of the online survey followed what was observed in the in-person study. Edge indicator and mini map had a less positive reception. However, they were rated more positively than those in the in-person study. Since participants didn't experience the working prototype, the discreet and distracting aspects that were observed in the in-person study were not captured. In both online and in-person study, the arrow directional glyph was preferred to circle+line.

% As indicated in the survey, the value of the diarization and localization features was immediate to all users. The participants were asked if directional feedback is valuable in their mobile transcription experience. All eight users agreed. Also, all eight users would recommend this feature to mobile captioning users. 


% \textit{"I really liked that almost immediately I could tell that there was a speaker change, so that as soon as the text started to show up, I could better contextualize that text as attributed to a new speaker." (P1)}

% P3
% Arrows + color seem to be most easier way to indicate the direction.
% \emph{"Arrows + color seem to be most easier way to indicate the direction." (P2)}
% P4
% \textit{"I'm very happy to see this tool being developed, it's a great addition to other speech recognition tools!" (P3)
% }
% \textit{"it was great to see so many options being offered" (P3)
% }
% P6 
% \textit{"This prototype is definitely a life changer and I strongly beleve that it will improve the quality of access to communication with speakers for many users" (P6)}

% P8
% The combination of the colored text with the arrow was the most effective for me.

% \emph{"The combination of the colored text with the arrow was the most effective for me." (P7)}

\begin{table}[t]
\caption{\textbf{Results on two-view novel-view synthesis.} \ours achieves the highest scores on all metrics, indicating more accurate view synthesis and visually pleasing outputs compared to existing methods.}
\centering
\resizebox{\linewidth}{!}{
\begin{tabular}{l|c|cccc}
\toprule
\textbf{Method} & Res. & PSNR$\uparrow$ & SSIM$\uparrow$ & LPIPS$\downarrow$ & FID$\downarrow$ \\
\midrule
GenWarp~\cite{seo2024genwarp} & \multirow{4}{*}{512$^2$} & 13.503 & 0.465 & 0.435 & 59.965 \\
SDXL Inpainting~\cite{podell2024sdxl} & & 16.228 & 0.535 & 0.389 & 89.502 \\
ViewCrafter~\cite{yu2024viewcrafter} & &  17.178 & 0.594 & 0.278 & 56.127 \\
\textbf{\ours (Ours)} & & \textbf{18.614} & \textbf{0.677} & \textbf{0.246} & \textbf{34.138} \\
\midrule
GenWarp~\cite{seo2024genwarp} & \multirow{4}{*}{1024$^2$} & 13.491 & 0.565 & 0.440 & 60.540 \\
SDXL Inpainting~\cite{podell2024sdxl} & & 16.153 & 0.565 & 0.426 & 89.537 \\
ViewCrafter~\cite{yu2024viewcrafter} & & 17.137 & 0.652 & 0.317 & 57.898\\
\textbf{\ours (Ours)} & & \textbf{18.568} & \textbf{0.711} & \textbf{0.283} & \textbf{35.721} \\
\bottomrule
\end{tabular}}
\label{tab:2viewnvs}
\end{table}

\begin{figure*}[t!]
    \centering
 \begin{tabular}{ccccc}
      & \hspace{3.2cm} \small Frame 1 & \hspace{2.6cm}\small Frame 2 & \hspace{2.8cm}\small Frame 3 & \hspace{1cm}\small 3D Reconstruction  \\
    
    \multicolumn{5}{c}{%
        \includegraphics[width=\linewidth]{figures/images/end2end.pdf}%
    } \\
    
\end{tabular}  

    \caption{\textbf{Results on Video/Multi-view Appearance Transfer of \ours}. We show the style images, three frames stylized by \ours, followed by a 3D reconstruction of these outputs using an off-the-shelf pipeline. Despite challenging camera motion and multiple objects in the scene, our method preserves consistent geometry and seamlessly transfers the reference style across all frames.} 
    \label{fig:recon}
\label{fig:scannet_recon}
\vspace{1mm}
\end{figure*}


\paragraph{\bf{Two-view NVS.}}
We compare our approach to: \textit{i)}~SDXL inpainting model~\cite{podell2024sdxl} with depth-conditioned ControlNet~\cite{zhang2023adding}, \textit{ii)} GenWarp~\cite{seo2024genwarp}, an image-based diffusion model for single view NVS, and \textit{iii)} ViewCrafter~\cite{yu2024viewcrafter}, a video-diffusion model for NVS. Note that the proposed task differs from traditional NVS as it leverages geometry information from the \textit{novel view} itself. We employ DUSt3R~\cite{wang2024dust3r} to extract the correspondences and provide the initial warped image as input to all methods. \ours outperforms  across all metrics, achieving a superior reconstruction ability as evidenced by the best PSNR, SSIM, and LPIPS metrics (\cf Tab.~\ref{tab:2viewnvs}, ). Additionally, it exhibits strong capability in extending style to unseen regions, evidenced by the lowest FID score (Fig.~\ref{fig:scannet_recon}). Notably, the second best method ViewCrafter~\cite{yu2024viewcrafter}, requires a predefined camera trajectory as input to video diffusion and runs 10$\times$ slower than ours. 


\begin{table}[t]
\caption{\textbf{Pose deviation from real-world estimates.} We measure the fraction of camera poses within certain rotation~(at $5^\circ$, $10^\circ$, $15^\circ$) and translation~(at $1,\mathrm{cm}$, $2,\mathrm{cm}$, $5,\mathrm{cm}$) error thresholds, reporting area-under-curve~(AUC) values. \ours achieves significantly higher AUC in both, showing superior multi-view geometric consistency vs. existing methods.}
\centering
\resizebox{\linewidth}{!}{
\begin{tabular}{l|ccc|ccc}
\toprule
\multirow{2}{*}{\textbf{Method}}& \multicolumn{3}{c|}{Rotation AUC$\uparrow$} & \multicolumn{3}{c}{Translation AUC$\uparrow$} \\
\cmidrule{2-7}
  & @5$^\circ$ & @10$^\circ$ & @15$^\circ$ & @1cm  & @2cm & @5cm\\
\midrule
GenWarp~\cite{seo2024genwarp}  & 25.89 & 46.70 & 58.89 & 58.38 & 59.39 & 70.05 \\
SDXL Inpainting~\cite{podell2024sdxl}  & 34.52 & 52.79 & 66.50 & 61.42 & 65.99 & 74.11 \\
ViewCrafter~\cite{yu2024viewcrafter}  & 37.56 & 55.33 & 68.53 & 60.91 & 65.99 & 77.16 \\
\textbf{\ours (Ours)} & \textbf{52.79} & \textbf{69.54} & \textbf{79.70} & \textbf{66.50} & \textbf{77.66} & \textbf{83.25} \\

\bottomrule
\end{tabular}}
\label{tab:corres_select}
\vspace{-1mm}
\end{table}

\paragraph{\bf{Multi-view Consistency Evaluation.}}
We further evaluate the multi-view consistency of the stylized results through a proxy task. Specifically, we input the original and stylized images to DUSt3R~\cite{wang2024dust3r} and estimate the camera poses, separately. By evaluating the agreement with the poses from the original images, we analyze whether the geometry is preserved in the stylized images. The results are presented in Tab.~\ref{tab:corres_select}. Enabled by our adaptive auto-regressive approach, which effectively mitigates inconsistencies while preserving image sharpness, our method significantly outperforms the baselines on both rotation and translation metrics.
Figs.~\ref{fig:scannet_recon} and~\ref{fig:recon_extra} show multi-view transfer results, including the 3D reconstruction of stylized outputs with estimated camera poses, demonstrating both geometric and stylistic consistency despite camera motion and multiple objects.


\paragraph{\bf{Ablation Study.}} 
In Tab. \ref{tab:ablation}(a), we run \ours without our guidance strategy and observe significant degradation in structure preservation (AbsRel↑ from 8.34 to 16.72). In (b), removing semantic attention hurts performance on perceptual similarity \wrt style image, showing that both components are crucial for semantic-accurate style transfer while maintaining structural integrity.

\begin{table*}
  [t]
  \centering
  \resizebox{\textwidth}{!}{%
  \begin{tabular}{cccccccccccc}
    \toprule \multicolumn{2}{c}{Components}                                                             & \multicolumn{5}{c}{Re-executability Rate (\%)} & \multicolumn{5}{c}{Readability (\#)} \\
    \cmidrule(lr){1-2} \cmidrule(lr){3-7} \cmidrule(lr){8-12}        \hspace{8pt}\labelemoji\hspace{8pt}                                                                & \hspace{8pt}\toolemoji\hspace{8pt}                                      & O0                                 & O1             & O2             & O3             & AVG            & O0             & O1             & O2             & O3             & AVG            \\
    \hline
    \rowcolor[rgb]{0.93,0.93,0.93}\multicolumn{12}{c}{\textbf{Initialize with LLM4Decompile-End-6.7B~\citep{llm4decompile}}}   \\
    \xmark                                                                                              & \xmark                                    & 69.51                              & 46.95          & 50.61          & 46.34          & 53.35          & 3.98 & 3.41 & 3.44 & 3.38 & 3.55 \\
    \cmark                                                                                              & \xmark                                    & 75.61                              & 50.61          & 50.00          & 50.00          & 56.55          & 4.01 & 3.44 & 3.39 & \textbf{3.49} & 3.58 \\
    \xmark                                                                                              & \cmark                                    & 83.54                     & \textbf{56.10}          & 51.22          & 50.61 & 60.37 & 4.05 & 3.51 & 3.51 & 3.42 & 3.62 \\
    \cmark                                                                                              & \cmark                                    & \textbf{85.37}                            & \textbf{56.10}                     & \textbf{51.83} & \textbf{52.43}          & \textbf{61.43} & \textbf{4.13} & \textbf{3.60} & \textbf{3.54} & \textbf{3.49} & \textbf{3.69} \\

    \rowcolor[rgb]{0.93,0.93,0.93}\multicolumn{12}{c}{\textbf{Initialize with Deepseek-Coder-6.7B-base~\citep{deepseekcoder}}} \\
    \xmark                                                                                              & \xmark                                    & 59.15                              & 35.98          & 39.02          & 37.80          & 42.99          & 3.71 & 3.05 & 3.16 & 3.05 & 3.24 \\
    \cmark                                                                                              & \xmark                                    & 66.46                              & 41.46          & 38.41          & 36.59          & 45.73          & 3.76 & 3.17 & \textbf{3.21} & 3.08 & 3.31 \\
    \xmark                                                                                              & \cmark                                    & 70.73                              & 39.63          & 39.02          & 40.24          & 47.41          & 3.90 & 3.17 & 3.08 & 3.11 & 3.31 \\
    \cmark                                                                                              & \cmark                                    & \textbf{79.88}                     & \textbf{45.73} & \textbf{43.90} & \textbf{42.68} & \textbf{53.05} & \textbf{3.96} & \textbf{3.21} & 3.18 & \textbf{3.19} & \textbf{3.38} \\
    \bottomrule
  \end{tabular}%
  }
  \caption{The ablation study of different methods across four optimization levels
  (O0, O1, O2, O3), as well as their average scores (AVG). The results in bold represent the optimal performance. The ~\labelemoji~ and ~\toolemoji~ means Relabedling and Function Call. \textbf{Bold} denotes the best performance.}
  \label{tab:ablation}
\end{table*}
