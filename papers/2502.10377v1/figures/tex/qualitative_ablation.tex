\begin{figure*}[t]
\begin{center}
\centering
\includegraphics[width=0.85\linewidth]{figures/images/qualitative_ablation.pdf}
\end{center}
\caption{
\textbf{Comparison between different auto-regression strategies.} The rows represent two non-sequential views used in multi-view appearance transfer, while the columns represent different frame selection strategies. (a) \textit{Single} warps only the previous view, failing to preserve details like the painting on the wall present in the top image, as pixel information is lost. (b) \textit{All} warps all past frames to the current frame, improving consistency but introducing smearing effects, demonstrated by the red arrows (highlighted by red arrows) due to overlapping pixel projections with large color value differences originating from lighting changes. (c) \textit{Ours}, the proposed strategy in \ours, achieves the most consistent and clean multi-view stylization.
}
\label{fig:ar_ablation}
\end{figure*}

\vspace{15pt}

\begin{figure*}[t]
\begin{center}
\centering
\includegraphics[width=0.99\linewidth]{figures/images/depth_condition.pdf}
\end{center}
\caption{
\textbf{Qualitative comparison between two different conditional models.}  We train a warp-and-refine model without the monocular depth~(MDE) condition. Without MDE supervision, the model struggles to correct local alignment issues~(\eg, the door handle, wine bottles, and chairs), resulting in noticeable geometric distortions and texture artifacts. In contrast, incorporating MDE enables the model to leverage pixel-aligned depth cues and more faithfully reconstruct fine details, yielding sharper and more consistent multi-view results.
}
\label{fig:qualitative_ablation}
\end{figure*}
