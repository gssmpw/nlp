% CVPR 2025 Paper Template; see https://github.com/cvpr-org/author-kit

\documentclass[10pt,twocolumn,letterpaper]{article}

%%%%%%%%% PAPER TYPE  - PLEASE UPDATE FOR FINAL VERSION
% \usepackage{cvpr}              % To produce the CAMERA-READY version
% \usepackage[review]{cvpr}      % To produce the REVIEW version
\usepackage[pagenumbers]{cvpr} % To force page numbers, e.g. for an arXiv version
\usepackage{colortbl}
\usepackage{graphicx}
\usepackage{multirow}
\usepackage{makecell}

% Import additional packages in the preamble file, before hyperref
\newcommand{\CG}{\mathcal{G}\xspace}
\newcommand{\CV}{\mathcal{V}\xspace}
\newcommand{\CE}{\mathcal{E}\xspace}
\newcommand{\CA}{\mathcal{A}\xspace}
\newcommand{\CF}{\mathcal{F}\xspace}
\newcommand{\CR}{\mathcal{R}\xspace}
\newcommand{\CB}{\mathcal{B}\xspace}
\newcommand{\CX}{\mathcal{X}\xspace}
\newcommand{\CK}{\mathcal{K}\xspace}
\newcommand{\CM}{\mathcal{M}\xspace}
\newcommand{\CC}{\mathcal{C}\xspace}
\newcommand{\CL}{\mathcal{L}\xspace}
\newcommand{\CI}{\mathcal{I}\xspace}
\newcommand{\CQ}{\mathcal{Q}\xspace}
\newcommand{\CO}{\mathcal{O}\xspace}
\newcommand{\CP}{\mathcal{P}\xspace}
\newcommand{\CS}{\mathcal{S}\xspace}
\newcommand{\CT}{\mathcal{T}\xspace}
\newcommand{\CJ}{\mathcal{J}\xspace}
\usepackage[para]{footmisc}
\usepackage{subfig}
% \usepackage{subcaption}
% \usepackage{array}
% \usepackage{colortbl}



% It is strongly recommended to use hyperref, especially for the review version.
% hyperref with option pagebackref eases the reviewers' job.
% Please disable hyperref *only* if you encounter grave issues, 
% e.g. with the file validation for the camera-ready version.
%
% If you comment hyperref and then uncomment it, you should delete *.aux before re-running LaTeX.
% (Or just hit 'q' on the first LaTeX run, let it finish, and you should be clear).
\definecolor{cvprblue}{rgb}{0.21,0.49,0.74}
\usepackage[pagebackref,breaklinks,colorlinks,allcolors=cvprblue]{hyperref}

%%%%%%%%% PAPER ID  - PLEASE UPDATE
\def\paperID{1140} % *** Enter the Paper ID here
\def\confName{CVPR}
\def\confYear{2025}

%%%%%%%%% TITLE - PLEASE UPDATE
\title{HumanDiT: Pose-Guided Diffusion Transformer \\ for Long-form Human Motion Video Generation
}

%%%%%%%%% AUTHORS - PLEASE UPDATE
\renewcommand{\thefootnote}{\fnsymbol{footnote}} %将脚注符号设置为fnsymbol类型,即特殊符号表示
\author{
	Qijun Gan~\(^{1,2}\)\footnotemark[2]~
 \footnotemark[3]
	 \quad Yi Ren~\(^2\)\footnotemark[2]
	 \quad Chen Zhang~\(^2\)\footnotemark[2]
	 \quad Zhenhui Ye~\(^1\)
	 \quad Pan Xie~\(^2\)
	\\
        Xiang Yin~\(^2\)
         \quad Zehuan Yuan~\(^2\)
         \quad Bingyue Peng~\(^2\)
         \quad Jianke Zhu~\(^1\)
	\\
	\(^1\)Zhejiang University \quad \(^2\)ByteDance  \\
	\texttt{\{ganqijun,jkzhu\}@zju.edu.cn},\\
	\texttt{\{ren.yi,yinxiang.stephen\}@bytedance.com}
    \\ \url{https://agnjason.github.io/HumanDiT-page/}
}
% \author{Qijun Gan\(^1\) \quad Yi Ren\(^2\) \quad Chen Zhang\(^2\) \quad Zhenhui Ye\(^1\) \quad Pan Xie\(^1\)\\
% Xiang Yin\(^2\) \quad Zehuan Yuan\(^2\)\quad BINGYUE PENG\(^2\) \quad Jianke Zhu\thanks{Corresponding author}\\
% \(^1\)Zhejiang Univercity\quad \(^2\)ByteDance\\
% {\tt\small ganqijun@zju.edu.cn}
% }

\begin{document}

\twocolumn[{%
    \maketitle
    \centering
    \vspace{-0.2in}\includegraphics[width=1\linewidth]{fig/teaser_arxiv.pdf}
    \vspace{-0.25in}
    \captionof{figure}{HumanDiT is a framework designed to generate high-fidelity, long human motion videos in diverse scenes and flexible resolution.}
    \label{fig:teaser}
    \vspace{0.10in} 
}]
\footnotetext[2]{These authors contributed equally to this work.} %对应脚注[1]
\footnotetext[3]{Done during an internship at ByteDance.} %对应脚注[2]
\footnote{Due to legal and portrait rights considerations, all real human faces in the paper are blurred, while unblurred images are generated by diffusion model.}

\begin{abstract}

% Recent works to jointly reconstruct 3D human and object from a single RGB image, are mostly model-based, that fail to capture the fine details of the clothed human body and object surface. In this paper, we introduce ReCHOR, a novel, model-free, first-method to produce realistic clothed human-object reconstructions from a monocular view. This is extremely challenging due to human-object occlusions, diverse interactions and depth ambiguity, as it needs to infer both 3D spatial awareness and high resolution details. Our core idea is based on estimating neural implicit representations for human and object respectively by an attention-based neural implicit model that attends to pixel-aligned features from both the global human-object image for spatial awareness and  the local separate view of human and object images for high quality details. Additionally, the network is conditioned on semantic features from an initial estimated human-object pose prior and a generative diffusion model that inpaints occluded regions, thus enabling the retrieval of details from them.
% We also propose a synthetic dataset with rendered scenes of diverse, inter-occluded 3D human and object scans, to train our network. We evaluate our method on the synthetic and real world BEHAVE dataset. Our experiments show that our method outperforms the SOTA in achieving realistic clothed human-object reconstructions.
Recent approaches to jointly reconstruct 3D humans and objects from a single RGB image represent 3D shapes with template-based or coarse models, which fail to capture details of loose clothing on human bodies. In this paper, we introduce a novel implicit approach for jointly reconstructing realistic 3D clothed humans and objects from a monocular view. For the first time, we model both the human and the object with an implicit representation, allowing to capture more realistic details such as clothing. This task is extremely challenging due to human-object occlusions and the lack of 3D information in 2D images, often leading to poor detail reconstruction and depth ambiguity. To address these problems, we propose a novel attention-based neural implicit model that leverages image pixel alignment from both the input human-object image for a global understanding of the human-object scene and from local separate views of the human and object images to improve realism with, for example, clothing details. Additionally, the network is conditioned on semantic features derived from an estimated human-object pose prior, which provides 3D spatial information about the shared space of humans and objects. To handle human occlusion caused by objects, we use a generative diffusion model that inpaints the occluded regions, recovering otherwise lost details. For training and evaluation, we introduce a synthetic dataset featuring rendered scenes of inter-occluded 3D human scans and diverse objects. Extensive evaluation on both synthetic and real-world datasets demonstrates the superior quality of the proposed human-object reconstructions over competitive methods.
\end{abstract}    
\section{Introduction}
\label{sec:intro}
% Image editing methods in diffusion models depend on user-defined control directions - users can unlock their creativity using these methods by specifying the desired manipulation through prompts~\cite{gandikota2023concept}, reference images~\cite{ruiz2022dreambooth, kumari2022customdiffusion, gal2022image, chen2024trainingfreeregionalpromptingdiffusion}, or attribute vectors~\cite{parmar2023zero,hertz2022prompt}. In this work, we ask a fundamentally different question: \emph{Can we automatically discover the underlying visual structure of a concept within diffusion model's knowledge?} %Rather than requiring user-specified controls, we aim to decompose the model's internal knowledge into meaningful directions.

% This question touches on a fundamental limitation in how we interact with diffusion models. Current control methods ~\cite{zhang2023addingconditionalcontroltexttoimage, gandikota2023concept, ye2023ipadaptertextcompatibleimage,ye2023ipadaptertextcompatibleimage, hertz2024stylealignedimagegeneration, li2023photomaker, shi2024instantbooth, chen2024trainingfreeregionalpromptingdiffusion} require users to specify their desired manipulations in advance, limiting interactive creativity. This contrasts with natural human artistic workflows, where creators dynamically explore creative ideas while jointly refining them toward meaningful artistic outcomes~\cite{hoffmann2016modeling}. This synergy between specification and exploration is not new to generative models. Early GAN architectures naturally developed disentangled latent spaces that enabled continuous\cite{harkonen2020ganspace,radford2015unsupervised, wu2021stylespace, shen2020interfacegan}, compositional control over generated images. Users could explore these spaces to discover interesting variations that would be difficult to describe in words~\cite{wu2021stylespace}, then combine them to achieve their creative goals~\cite{grabe2022towards}. 


% While diffusion models have largely superseded GANs in conditional image synthesis~\cite{dhariwal2021diffusion},  their underlying structure remains less understood. Diffusion models achieve remarkable diversity through high-dimensional latents, unlike GANs' compact latent spaces.  With a single prompt, diffusion models can generate radically different variations through different random initializations of input noise. We ask - Is it possible to discover interpretable structure within this vast space of variations?

Text-to-image diffusion models are capable of generating remarkable visual variations from a single prompt through different random initializations. However, this vast creative potential remains largely opaque to users---while we can generate diverse images, we lack understanding of the underlying structure of these variations. This presents a fundamental challenge: how can we discover and expose the latent visual capabilities encoded within these models?

\let\thefootnote\relax \footnote{$^{*}$Correspondence to \texttt{gandikota.ro@northeastern.edu}}

The challenge touches on a key limitation in how we interact with diffusion models today. Current control methods require users to explicitly specify their desired edits in advance through prompts~\cite{gandikota2023concept}, reference images~\cite{zhang2023addingconditionalcontroltexttoimage, chen2024trainingfreeregionalpromptingdiffusion, ruiz2022dreambooth,kumari2022customdiffusion, Ryu_lora, hu2021lora}, or attribute vectors~\cite{ye2023ipadaptertextcompatibleimage, hertz2024stylealignedimagegeneration, li2023photomaker, shi2024instantbooth,parmar2023zero,hertz2022prompt}. That contrasts sharply with natural human creative workflows, where artists dynamically explore creative ideas and jointly refine them toward meaningful artistic outcomes~\cite{hoffmann2016modeling}. The need for pre-specified controls creates a barrier between users and the full creative potential of these models.

Interestingly, earlier generative models like GANs~\cite{gans,karras2019style,brock2018large} naturally developed more interpretable internal structures. Their compact latent spaces often exhibited emergent disentanglement~\cite{harkonen2020ganspace,radford2015unsupervised, wu2021stylespace, shen2020interfacegan}, enabling continuous and compositional control over generated images. Users could explore these spaces to discover interesting variations that would be difficult to describe in words~\cite{wu2021stylespace}, then combine them to achieve their creative goals~\cite{grabe2022towards}.

Diffusion models have largely superseded GANs in conditional image synthesis~\cite{dhariwal2021diffusion}, achieving greater diversity through much higher-dimensional latents. And yet an understanding of the underlying structure of these larger latent spaces has remained elusive. In this work, we ask a fundamental question: \emph{Can we automatically discover the visual structure within a diffusion model's knowledge of a concept?} Rather than requiring user-specified controls, we aim to decompose the model's internal representations into expressive directions that users can explore and combine.

To address these needs, we present \textbf{SliderSpace}, a framework that brings systematic explorability to diffusion models. Given just a text prompt, SliderSpace discovers a canonical set of meaningful, diverse, and controllable directions within the model's knowledge of that concept. Each direction is implemented as a low-rank adapter~\cite{hu2021lora} that can be scaled and composed with others, allowing users to explore and smoothly combine different aspects of variation, as shown in Figure~\ref{fig:intro}.

We ground SliderSpace discovery in three key requirements for meaningful decomposition of a diffusion model's visual manifold: 
\begin{enumerate}
    \item \textbf{Unsupervised Discovery:} The decomposition process should emerge from the intrinsic structure of the model's learned representation, rather than being guided by predefined attributes. This ensures we capture the true topology of the model's knowledge space rather than projecting our assumptions onto it.
    
    \item \textbf{Semantic Orthogonality:} Each discovered control must represent a distinct semantic direction. This is enforced in a semantic feature space, like CLIP, where every slider has an orthogonal effect in embeddings. This prevents discovering multiple controls that create similar semantic effects, making the system more efficient and easier.
    
    \item \textbf{Distribution Consistency:} Directions must induce consistent transformations across both random seeds and prompt variations. 
\end{enumerate}

These requirements naturally lead to our proposed framework, which we formalize in Section~\ref{sec:method}. As we show in our experiments, SliderSpace is architecture-agnostic, working with both conventional U-Net based models like Stable Diffusion~\cite{rombach2022high, rombach2022sd20, podell2023sdxl, turbo, dmd} and recent transformer-based architectures like Flux~\cite{flux}.

We demonstrate the expressiveness of SliderSpace through three applications: First, we show how SliderSpace can decompose high-level concepts into diverse and expressive components, revealing the natural axes of variation in the model's understanding. Second, we explore artistic style variation, where SliderSpace discovers directions that match or exceed the diversity of manually curated artist lists while being judged more useful by human evaluators. Finally, we show how SliderSpace can help reverse the mode collapse commonly observed in distilled diffusion models, restoring diversity while maintaining generation speed.

Beyond providing practical creative control, SliderSpace opens new avenues for understanding and utilizing the latent capabilities of diffusion models. By mapping these models' visual potential into intuitive, composable directions, we take a step toward making their creative possibilities more accessible and interpretable to users.

% Image editing methods in diffusion models unlock the creativity of users. In this work we ask an alternate question: \emph{Can we organize and expose what of the diffusion model is already capable of?}.
% Existing methods for controlling image generation typically require users to manually specify edit directions for desired changes. This process is time-consuming, requires technical expertise, and limits the spontaneity of the creative process. For instance, if a user wants to adjust the smile of a generated person, they must explicitly request this edit, often through imprecise prompt engineering or model fine-tuning. This approach of predefined controls or manual specifications restricts users from fully exploring the latent capabilities of the model. There may be interesting stylistic variations or attributes that the model can generate, but users have no easy way to discover or utilize these.

% Natural visual disentanglement was an emergent property in the latent space of Generative Adversarial Models (GANs) \cite{harkonen2020ganspace,radford2015unsupervised, wu2021stylespace, shen2020interfacegan}. In particular, it has been observed that StyleGAN~\cite{karras2019style} stylespace neurons offer detailed control over many meaningful aspects of images that would be difficult to describe in words~\cite{wu2021stylespace}. However, diffusion models do not share such a compact latent space~\cite{park2023unsupervised}; and efforts to uncover such a space in the semantic embeddings of the text conditioning have met with limited success \nik{Nick - is there a specific citation you were thinking about?}.

% In this work we introduce \textbf{SliderSpace}, which takes a step towards uncovering an analogous low dimensional representation of diffusion models' visual breadth; in essence treating the diffusion model as many generators sharing parameters, where a particular generator is defined by a specific prompt. For a given prompt we sample many random seeds (and optionally prompt expansions using an LLM), generate the corresponding images, and apply an off the shelf feature extractor (in this work CLIP, but our method can be applied to any differentiable feature extractor). We use PCA to analyze these features, and for each of the leading $k$ principal components we train a LoRA \cite{} which causes the diffusion model to produces images which increase the feature magnitude along that component when passed back through the same feature extractor. This leads to a 'Slider' for each principal component, because each LoRA can be scaled and applied to the original diffusion model, continuously varying those visual features in the generated results (as measured, in our case, by CLIP).

% There are many other works that enhance the controllability of diffusion models. One common approach is enabling users to add spatial constraints to a generation either manually, or via a reference image \cite{zhang2023addingconditionalcontroltexttoimage, chen2024trainingfreeregionalpromptingdiffusion}, a second is leveraging more abstract embeddings (e.g. identity, style) extracted from a reference image \cite{ye2023ipadaptertextcompatibleimage, hertz2024stylealignedimagegeneration, li2023photomaker, shi2024instantbooth}, a third is finetuning a foundation model to better generate a concept important to the user \cite{ruiz2022dreambooth, kumari2022customdiffusion, Ryu_lora, hu2021lora}, and a fourth (most relevant to this work) is finding low-rank adaptors of the model based on a prompt or small training set which can be scaled to provide continous control over one aspect of generated image (e.g. night vs day, basic vs luxury, etc.) \cite{gandikota2023concept}. SliderSpace is complementary to all of these methods and offers something distinct. All of the other methods we are aware require the user (and / or model designer) to know in advance what type of control they want. In contrast SliderSpace assists users in discovering and controlling hidden capabilities present in the diffusion model's distribution of possible generations.

%We propose that truly intuitive creative control in a text-to-image model should meet three key criteria: \emph{discoverability}, \emph{intuitiveness}, and \emph{specificity}. The model should reveal controllable attributes that may not be immediately obvious, offer controls that are easy to understand and manipulate, and ensure each control affects a distinct attribute of the generated image.

% We demonstrate the utility and power of SliderSpace using three applications built on top of SDXL-DMD \cite{dmd}, because its fast generation speed lends itself well to the continuous control offered by SliderSpace.

% First, we study concept decomposition (Section \ref{sec:concept_exp}), where we learn sliders for a specific concept (e.g. 'monster', 'waterfall', 'car'). Through quantitative metrics of diversity and text alignment we demonstrate that the learned sliders dramatically boost the diversity of generations when randomly applied without harming text alignment; we also ask humans to qualitatively judge these results in a user study where they find the SliderSpace results to be more 'Diverse', 'Useful', and 'Creative' than our baselines.

% Second, we attempt to compare the automatic discoveries of SliderSpace to a large scale manual study of artistic styles (Section \ref{sec:art_exp}), open-sourced by ParrotZone \cite{parrotzone}. In this study SDXL was prompted with over 4300 artist names,  and based on visual inspection the cases of successful stylistic mimicry recorded. Quantitatively SliderSpace more closely matches the distribution of artistic variation discovered by ParrotZone than other baselines, and in our user studies was judged to be significantly more 'Diverse' and 'Useful' than the baselines. To our surprise humans even judged SliderSpace results to be slightly more 'Diverse' than the results generated by the manually discovered artist names of \cite{parrotzone}.

% Third, we attempt to use SliderSpace to reverse the mode collapse commonly observed in distilled few-step diffusion models relative to the original teacher model (Section \ref{sec:diverse_exp}). We quantitatively demonstrate that applying SliderSpace to SDXL-DMD leads to more closely matching the distribution of images by the original teacher, SDXL.

%Through extensive experiments on various state-of-the-art text-to-image models, we demonstrate that SliderSpace significantly enhances user control and creative expression in AI-assisted image generation tasks. Our method enables a range of applications, including concept decomposition and control, diversity improvement in generated images, customization dissection and edits, and the exploration of artistic styles inherent in the model.

% SliderSpace goes beyond providing a practical tool for enhanced creative control. By mapping the visual potential of diffusion models it can open new avenues for generative creativity and deepens our understanding of each model's hidden potential.
\section{Related Works}
\label{sec:relate}


\noindent\textbf{Image and Video Generation.} Recent years, image and video generation has made substantial advancements~\cite{vondrick2016generating, ge2022long, bar2023multidiffusion, zeng2024make}, beginning with early approaches based on GANs~\cite{goodfellow2014generative, chan2019everybody, ren2020deep, siarohin2019animating, siarohin2021motion} and VQVAE-based transformers~\cite{ge2022long, xu2016msr, sun2024autoregressive}, which laid the groundwork for synthesizing realistic videos. However, these methods often struggled with issues like temporal inconsistency and the high computational demands associated with modeling continuous motion. More recently, diffusion models (DMs)~\cite{song2020denoising, ho2020denoising, ni2023conditional} have emerged as a powerful alternative, offering greater stability and control. Unlike GANs, diffusion models progressively refine noisy inputs into coherent frames, leading to more robust and photo-realistic outputs. Latent Diffusion Models (LDM)~\cite{rombach2022high}, in particular, have optimized this process by performing it in lower-dimensional latent spaces. %, significantly reducing computational complexity while enhancing efficiency.

For video generation, these diffusion models~\cite{guo2023animatediff, bar2024lumiere} go beyond image synthesis by incorporating temporal layers and attention mechanisms to better model spatial-temporal relationships. To maintain temporal continuity, some approaches~\cite{singer2022make,blattmann2023stable, zhou2022magicvideo, wang2023dreamvideo} directly extend the 2D U-Net~\cite{ronneberger2015u}, pre-trained on text-to-image tasks into 3D. Techniques like Stable Video Diffusion (SVD)~\cite{blattmann2023stable} and AnimateDiff~\cite{guo2023animatediff} represent notable developments, as they extend 2D image generation frameworks to handle the additional complexity of continuous video segments. Recent methodologies~\cite{xu2024easyanimate, yang2024cogvideox, openai2024sora, zhang2024tora} have shown enhanced potential by integrating specialized motion modules and a DiT-based framework~\cite{peebles2023scalable}. Concurrently, 3D Variational Autoencoders (VAEs)~\cite{kingma2013auto} alleviate the computational demands of video data processing through advanced compression techniques. % These advancements have not only improved the efficiency of video generation but also enabled finer control over complex scenes, such as generating realistic human motions or more dynamic environments.

\noindent\textbf{Human Animation.} Pose-guided human image animation has seen significant advancements, particularly with the incorporation of pose estimation methods such as OpenPose~\cite{cao2017realtime}, DWpose~\cite{yang2023effective} and Sapiens~\cite{khirodkar2024sapiens} for guiding motion synthesis. Early approaches~\cite{liao2020speech2video, weng2022humannerf, ginosar2019learning, ye2023geneface}, often based on GANs~\cite{goodfellow2014generative} or NeRFs~\cite{mildenhall2021nerf}, focused on transferring poses between images using explicit skeleton representations, while these struggled with temporal consistency and flexibility in motion transfer. DisCo~\cite{wang2024disco} integrates character features using CLIP and incorporates background features through ControlNet~\cite{zhang2023adding}.

The rise of diffusion models (DMs) has greatly improved the quality of both image and video generation. Recent methods, like MagicAnimate~\cite{xu2024magicanimate} and Animate Anyone~\cite{hu2024animate}, introduce specialized motion modules and lightweight pose guiders to ensure accurate pose-to-motion transfer. Champ~\cite{zhu2024champ} relies on parametric models SMPL~\cite{loper2023smpl}, which offer realistic human representations and serve as ground truth for pose and shape analysis. UniAnimate~\cite{wang2024unianimate} utilizes first frame conditioned input for consistent long video generation. Mimicmotion~\cite{zhang2024mimicmotion} employs cross-frame overlapped diffusion to generate extended animated videos. Xue~\textit{et al.}~\cite{xue2024follow} utilize an optical flow guide to stabilize backgrounds and a depth guide to handle occlusions between body parts. Cyberhost~\cite{lin2024cyberhost} enhances hand and face generation with a Region Codebook Attention mechanism. Moreover,  Tango~\cite{liu2024tango} synthesizes co-speech body-gesture videos by retrieving matching reference video clips and leveraging a diffusion-based interpolation network. Animate-X~\cite{tan2024animate} extends beyond human to anthropomorphic
characters with various body structures by utilizing implicit and explicit pose indicator.

For precise rendering of hands and faces, HandRefiner~\cite{lu2024handrefiner} employs the ControlNet module to correct distorted hands. ShowMaker~\cite{yangshowmaker} utilizes latent features aligned with facial structures for enhancement. RealisDance~\cite{zhou2024realisdance} and TALK-Act~\cite{guan2024talk} leverage 3D priors of hands and faces, providing accurate 3D or depth information as conditioning inputs.
% By decoupling identity and motion through a reference network and pose guide, these approaches achieve visual coherence and enable dynamic human motion synthesis.


\section{Methodology}
\paragraph{Preliminaries.}
We primarily focus on the homologous model merging, in which $\boldsymbol{\theta}_i$ all come from the same base model $\boldsymbol{\theta}_{\rm{base}}$. Given $K$ tasks $\{T_1,T_2,\cdots,T_K\}$ and $K$ corresponding fine-tuned models with parameters $\{\boldsymbol{\theta}_1,\boldsymbol{\theta}_2,\cdots,\boldsymbol{\theta}_K\}$, model merging aims to combine $K$ fine-tuned models into one single model simultaneously performing on $\{T_1,T_2,\cdots,T_K\}$ without post-training~\cite{method_p1_1,method_p1_2}.
Task vector~\cite{ilharco2023editing,yang2024adamerging} is a key element in merging method which could enhances the base model‘s ability or enable the model to handle other tasks. Specifically, for task $T_i$, the task vector $\boldsymbol\tau_i\in \mathbb{R}^D$ is defined as the vector obtained by subtracting the SFT weights $\boldsymbol{\theta}_i$ from the base model weight
$\boldsymbol{\theta}_{\rm{base}}$, \emph{i.e.}, $\boldsymbol\tau_i=\boldsymbol{\theta}_i-\boldsymbol{\theta}_{\rm{base}}$. The merged model could be denoted as $\boldsymbol{\theta}_m=\boldsymbol{\theta}_{\rm{base}}+\sum_i \lambda_i\boldsymbol{\tau}_i$, which $\lambda_i$ is the scaling factor measuring the importance of task vector. For clarification, we also denote the neuron set in $\boldsymbol{\theta}_i$ as $\mathcal{N}_i$, the neuron set in $\boldsymbol{\tau}_i$ as $\mathcal{T}_i$.



\begin{algorithm}[!ht]
    \caption{LED-Merging}
    \label{alg1}
    \begin{algorithmic}[1]
        \REQUIRE  base model $\boldsymbol{\theta}_{\rm{base}}$, SFT models $\{\boldsymbol{\theta}_{i}\mid i\in [K]\}$, mask ratios \{$r_{i} \mid i\in [K]\}$, scaling factors $\{\lambda_i\mid i\in[K]\}$, location datasets $\{\mathcal{X}_{i}\mid i\in[K]\}$
        \ENSURE merged parameter $\boldsymbol{\theta}_{m}$
        \STATE $\mathcal{M}\leftarrow\phi$
        \STATE $\boldsymbol{\theta}_{m}\leftarrow \boldsymbol{\theta}_{\rm{base}}$
        \FOR{$i\in [K]$}
        \STATE $I(\boldsymbol{\theta}_i)=\mathbb{E}_{x\sim \mathcal{X}_i}|\boldsymbol{\theta}_{i}\odot \nabla_{\boldsymbol{\theta}_i}\mathcal{L}(x)|$
        \STATE $I(\boldsymbol{\theta}_{\rm{base}})=\mathbb{E}_{x\sim \mathcal{X}_i}|\boldsymbol{\theta}_{\rm{base}}\odot \nabla_{\boldsymbol{\theta}_{\rm{base}}}\mathcal{L}(x)|$
        
        \STATE calculate $\mathcal{T}^{r_i}_{i}$ following Equation \ref{vote}
        \STATE  $\mathcal{M}\leftarrow \mathcal{M}\cup\{\mathcal{T}^{r_i}_i\}$
       
        
   
        
        
        \ENDFOR  
        \FOR{$i\in [K]$}
        
        \STATE calculate $\text{Disjoint}(\mathcal{T}_i^{r_i})$ use Equation~\ref{disjoint_safety}
        \STATE $\boldsymbol{m}_i \leftarrow \boldsymbol{0}$
        \FOR{$d\in \mathcal{T}_i^{r_i}$}
        \STATE $\boldsymbol{m}_{i,d}=1$
        \ENDFOR
        \STATE $\boldsymbol{\theta}_{m}\leftarrow \boldsymbol{\theta}_{m}+\lambda_i \boldsymbol{\tau}_i\odot \boldsymbol{m}_{i}$
        \ENDFOR
    \end{algorithmic}
\end{algorithm}
    %\vspace{-5pt}
\begin{figure*}[h!]
    \centering
    \includegraphics[width=\linewidth]{figs/pipeline_v2.pdf}
    \vspace{-40mm}
    \caption{Overview of our two-stage training pipeline {\ours}.}
    \label{fig:pipeline}
\end{figure*}


\paragraph{LED-Merging: Location, Election, and Disjoint Merging}
To address the neuron misidentification and interference issues in existing model merging methods, we propose LED-Merging (Location, Election, and Disjoint Merging). Specifically, previous studies \cite{modelstock, ilharco2023editing, tiesmerging} fail to accurately identify safety-related neurons in task vectors with a single magnitude score, namely \textit{neuron misidentification}. Meanwhile, there exists an interference between safety-related and utility-related task vector neurons during the merging process, namely \textit{neuron interference}. To address neuron misidentification, we first locate important neurons both in the base and fine-tuned models and then elect neurons from the task vector considering these two scores together. Subsequently, to mitigate the interference, we introduce a disjoint step, isolating these important neurons so that they influence different base neurons. The whole process is illustrated in Figure~\ref{fig:method}. 




In the location and election step, we consider the importance score from base and fine-tuned models simultaneously to locate task-specific neurons. In this way, it is more accurate than relying on the magnitude score alone because task-specific neurons with high importance score in the fine-tuned model may not necessarily score high in the base model, and vice versa.

{\textbf{Location}}.  We first calculate importance scores for each neuron in a base/fine-tuned model. Given a location dataset $\mathcal{X}_i=\{(x,y)_k\}$, where $x$ is the question and $y$ is the answer, we calculate the importance scores for the weight $\boldsymbol{\theta}_i\in\mathbb{R}^D$ in any  layer as follows~\cite{snip,spareseGPT,sun2024a}:
\begin{equation}
    I(\boldsymbol{\theta}_i)=\mathbb{E}_{x\sim \mathcal{X}_i}[\boldsymbol{\theta}_i\odot \nabla _{\boldsymbol{\theta}_i}\mathcal{L}(x)],
    \label{location}
\end{equation}
which $\mathcal{L}(x)=-\log p(y\mid x)$ is the conditional negative log-likelihood loss. We choose the SNIP score~\cite{snip} because it balances computational efficiency and performance~\cite{cq}. Please refer to Sec.~\ref{sec:ablation} for the comparison between different location methods. After computing importance scores, we choose top-$r_i$ neurons as the important neuron subset $\mathcal{N}_{i}^{r_i}$ from $I(\boldsymbol{\theta}_i)$.
 
 % After computing locating scores, we select the neurons scoring both high in base and fine-tuned models as important neurons in task vectors. Then in the disjoint step,  with preventing  polysemantic neurons  from receiving gradient updates towards different directions,
 % we use set difference to isolate the safety   and utility-related neurons  and construct corresponding masks for merging process,

{\textbf{Election}}. A natural question is how to select important neurons in the task vector $\boldsymbol{\tau}_i$ based on $I(\boldsymbol{\theta}_{\rm{base}})$ and $I(\boldsymbol{\theta}_{i})$. The important neurons in the base model may be different from neurons in the fine-tuned model. Therefore, we introduce the following election strategy to select neurons with high scores in both base and fine-tuned models:
\begin{equation}
    \mathcal{T}_i^{r_i}=\mathcal{N}_i^{r_i}\cap \mathcal{N}_{\rm{base}}^{r_i}.
    \label{vote}
\end{equation}
\emph{Remark}. We compare different choosing methods, including scoring low or high in base or fine-tuned model in Section~\ref{sec:ablation} and find that Equation \ref{vote} achieves the best performance.





{\textbf{Disjoint}}. As important neurons from different task vectors may conflict with each other at the same position, we use the set difference to disjoint the neurons from others to prevent interference:
\begin{equation}
    \text{Disjoint}(\mathcal{T}^{r_i}_{i})=\mathcal{T}^{r_i}_{i}-\mathop{\cup}\limits_{{J}\subsetneqq [K],|J|\geq 2}\mathop{\cap}\limits_{j\in {J}}\mathcal{T}^{r_j}_{j}.
    \label{disjoint_safety}
\end{equation}

Next, we construct a mask $\boldsymbol{m}_i\in\mathbb{R}^D$ to implement disjoint in the merging process. Specifically, this mask $\boldsymbol{m}_i$ is used to select neurons from $\mathcal{T}_i$. The mask ratio is $r_i$, where $r\in(0,1]$. The mask $\boldsymbol{m}_i$ can be derived from:
\begin{equation}
    \boldsymbol{m}_{i,d}=\begin{aligned} &\left\{ \begin{array}{ll} 1, & \text{if } d\in \text{Disjoint}(\mathcal{T}_{i}^{r_i}), \\ 0, & \text{otherwise}. \end{array} \right. \end{aligned}
    \label{mask_safety}
\end{equation}


% \subsection{Merging Models with Masks}
{\textbf{Merging}}. The final
merged task vector $\boldsymbol{\tau}_m$ is as follows:
\begin{equation}
    \boldsymbol{\tau}_m= \sum_i \lambda_i\boldsymbol{\tau}_{i}\odot\boldsymbol{m}_i.
    \label{merged_task_vector}
\end{equation}
We summarize the workflow in Algorithm \ref{alg1}.



\section{Experiments}
\subsection{Implementation Details}
\textbf{Datasets.} We train our model on the CelebV-HQ \cite{zhu2022celebv} and VFHQ \cite{xie2022vfhq} datasets. Since the backbone of SVD \cite{blattmann2023stable} is sensitive to video quality, we first evaluate each video in two datasets with the video quality assessment method FasterVQA \cite{wu2023neighbourhood}, and remove videos with scores lower than 0.6. In the end, 37,644 videos remain for training. To ensure a fair comparison in experiments, we evaluate our method on the portrait video dataset HDTF \cite{zhang2021flow} and FFHQ \cite{karras2019style}.

\noindent\textbf{Training Details.} During the training phase, for the temporal attention layers of the SVD, we sample 16-frame video sequences to establish temporal consistency, with each frame at a resolution of $512\times512$. Unlike methods such as \cite{hu2024animate,ma2024follow}, which require two separate training stages, we update all the weights of both the SVD and two adapters simultaneously. The model is trained for 30,000 steps with a batch size of 8 using gradient accumulation, optimized by 8bit-Adam \cite{kingma2014adam} with a learning rate of $1\times10^{-5}$. 
\subsection{Metrics and Comparisons}
\textbf{Evaluation Metrics.} To evaluate the performance of our method, following \cite{cai2024real}, we relight the first 100 frames of each video in the HDTF dataset. Each video is rendered with four distinct lighting conditions derived from four different lighting-effect reference faces, resulting in a total of 44,000 frames for comprehensive comparison. Following \cite{nerffacelighting}, we use an off-the-shelf estimator \cite{feng2021learning} to calculate the Lighting Error (LE). Arcface \cite{deng2019arcface} is used to measure Identity Preservation (ID) between the relit results and the original images. To assess temporal consistency, we compute LPIPS \cite{zhang2018perceptual} between adjacent frames. We further employ an image quality assessment model \cite{pyiqa} and a video quality assessment model \cite{wu2023neighbourhood} to evaluate Image Quality (IQ) and Video Quality (VQ), respectively. Additionally, Fréchet Inception Distance (FID) \cite{heusel2017gans} and Fréchet Video Distance (FVD) \cite{skorokhodov2022stylegan} are used to measure video fidelity. In addition to objective evaluation, we conduct a user study in which 17 participants rate the videos based on three criteria: Lighting Accuracy (LA-User), Identity Similarity (ID-User), and Video Quality (VQ-User). Each criterion is rated on a scale of 1 to 5: poor, fair, average, good, and excellent. Finally, we calculate the average score for each criterion across participants.

\noindent\textbf{Comparative Methods. }For the portrait relighting task, we conduct a comparative analysis between LCVD and five state-of-the-art portrait relighting methods: DPR \cite{zhou2019deep}, SMFR \cite{hou2021towards}, NFL \cite{nerffacelighting}, StyleFlow \cite{10.1145/3447648}, and DiFaReli \cite{ponglertnapakorn2023difareli}, evaluating performance on both the HDTF and FFHQ datasets. For the portrait animation task, we compare LCVD with three state-of-the-art portrait animation methods: DaGAN \cite{hong2022depth}, StyleHEAT \cite{yin2022styleheat}, and AnimateAnyone \cite{hu2024animate}, using the HDTF dataset for evaluation.
\begin{table*}[t!]
    \centering
    \caption{Quantitative comparison of portrait relighting with DPR, SMFR, NFL, StyleFlow, and DiFaReli based on objective evaluation and user study on the HDTF video dataset. The best scores are highlighted in bold, and the second-best are underlined.}
    \vspace{-2mm}
    \label{tab:compare}
    \scalebox{1.0}
    {\begin{tabular}{cccccccc||ccc}
        \hline
        &\multicolumn{7}{c}{Objective Evaluation}&\multicolumn{3}{c}{User Study} \\
        \cmidrule(r){2-8} \cmidrule(r){9-11}
        Methods & LE$\downarrow$ & ID$\uparrow$ & LPIPS$\downarrow$ & IQ$\uparrow$ & VQ$\uparrow$ & FID$\downarrow$ & FVD$\downarrow$ & LA-User$\uparrow$ & ID-User$\uparrow$ & VQ-User$\uparrow$\\
        \hline\hline
        %\multicolumn{11}{c}{\textbf{HDTF} \cite{zhang2021flow}} \\ \hline
        DPR \cite{zhou2019deep} & 0.768 & \textbf{0.730} & \underline{0.0295} & \underline{2.646} & 0.734 & \underline{44.57} & \underline{403.0} & \underline{3.423} & \underline{3.462} &\underline{3.125}\\
        SMFR \cite{hou2021towards} & \underline{0.747} & \underline{0.601} & 0.0333 & 1.057 & 0.588 & 60.50 & 551.6 & 3.047 & 2.877 & 2.604\\
        NFL \cite{nerffacelighting} & 0.784 & 0.199 & 0.0823 & 2.586 & \underline{0.766} & 96.17 & 819.3 & 2.894 & 2.553 & 2.398\\
        StyleFlow \cite{10.1145/3447648} & 0.932 & 0.474 & 0.1088 & 2.614 & 0.746 & 161.3 & 900.6 & 2.103 & 1.929 &1.563\\
        DiFaReli \cite{ponglertnapakorn2023difareli} & 0.783 & 0.531 & 0.1152 & 1.103 & 0.458 & 57.49 & 743.2 & 3.141 & 2.592 & 2.284\\
        \hline
        Ours & \textbf{0.738} & 0.585 & \textbf{0.0282} & \textbf{3.034} & \textbf{0.775} & \textbf{37.46} & \textbf{273.3} & \textbf{3.534} & \textbf{4.000} & \textbf{3.398}\\
        \hline\hline
    \end{tabular}}
    \vspace{-4mm}
\end{table*}
%\iffalse
%\begin{table}
%    \centering
%    \caption{Quantitative comparison with DPR and SMFR on the synthesized portrait videos, which are generated using our method (Ours w/o Rel.) with the lighting from the reference image itself. The best scores are highlighted in bold, and the second-best scores are underlined.}
%    \vspace{-2mm}
%    \scalebox{0.95}
%    {\begin{tabular}{cccccc}
%        \hline
%        Methods & LE\downarrow & ID\uparrow & LPIPS\downarrow & IQ\uparrow & FID\downarrow\\
%        \hline\hline
%        DPR\cite{zhou2019deep} & 0.770 & \underline{0.695} & 0.032 & 2.63 & 48.2 \\
%        SMFR\cite{hou2021towards} & \underline{0.750} & 0.581 & 0.039 & 1.05 & 63.9 \\
%        \hline
%        Ours w/o Reli. & -- & \textbf{0.837} & \textbf{0.027} & \textbf{3.04} & \textbf{34.8} \\
%        Ours & \textbf{0.738} & 0.585 & \underline{0.028} & \underline{3.03} & \underline{37.5} \\
%        \hline
%    \end{tabular}}
%    \vspace{-4mm}
%    \label{tab:self}
%\end{table}
%\fi

\begin{table}
    \centering
    \caption{Quantitative comparison of portrait relighting with NFL, StyleFlow and DiFaReli on the FFHQ dataset. The best scores are highlighted in bold, and the second-best are underlined.}
    \vspace{-2mm}
    \begin{tabular}{ccccc}
        \hline
        Methods & LE$\downarrow$ & ID$\uparrow$ & IQ$\uparrow$ & FID$\downarrow$\\
        \hline\hline
        NFL\cite{nerffacelighting} & \underline{0.892} & 0.253 & 3.020 & 118.9\\
        StyleFlow\cite{10.1145/3447648} & 1.042 & 0.485 & \underline{3.846} & 102.7\\
        DiFaReli\cite{ponglertnapakorn2023difareli} & \textbf{0.749} & \underline{0.687} & 1.591 & \textbf{25.98}\\
        \hline
        Ours & 0.938 & \textbf{0.765} & \textbf{4.465} & \underline{26.71}\\
        \hline
    \end{tabular}
    \vspace{-4mm}
    \label{tab:ffhq}
\end{table}
\begin{table}
    \centering
    \caption{Quantitative comparison of cross-identity portrait animation with DaGAN, StyleHEAT, and AnimateAnyone on the HDTF dataset. The best scores are highlighted in bold, and the second-best scores are underlined.}
    \vspace{-2mm}
    \scalebox{0.95}
    {\begin{tabular}{cccccc}
        \hline
        Methods & ID$\uparrow$ & POSE$\downarrow$ & IQ$\uparrow$ & VQ$\uparrow$ & FID$\downarrow$\\
        \hline\hline
        DaGAN\cite{hong2022depth} & 0.645 & \underline{3.935} & 1.005 & 0.528 & 107.4 \\
        StyHE.\cite{yin2022styleheat} & 0.201 & 34.58 & 1.554 & 0.612 & 149.9 \\
        AniAny.\cite{hu2024animate} & \underline{0.806} & 5.086 & \underline{2.744} & \underline{0.706} & \underline{69.85} \\
        \hline
        Ours & \textbf{0.876} & \textbf{3.805} & \textbf{3.021} & \textbf{0.717} & \textbf{49.11} \\
        \hline
    \end{tabular}}
    \vspace{-4mm}
    \label{tab:animate}
\end{table}
\subsection{Quantitative Evaluation}
In portrait video relighting, Table \ref{tab:compare} shows that our method outperforms other state-of-the-art methods in all metrics except for ID. Specifically, it improves video fidelity (FVD) by 32\%, image fidelity (FID) by 16\%, and image quality (IQ) by 14.6\% compared to the second-best method, demonstrating excellent video quality. While our method does not achieve the highest ID performance, this is because relighting in our method is applied during portrait animation, where ID information is derived only from the reference, unlike other methods that relight each frame individually. However, our method achieves the best ID performance in the user study, likely due to its higher-quality, more stable video synthesis, which visually aligns with better ID preservation. This also proves that the ID loss in our method is within an acceptable range for human perception.

Since NFL \cite{nerffacelighting}, StyleFlow \cite{10.1145/3447648}, and DiFaReli \cite{ponglertnapakorn2023difareli} are trained on the aligned FFHQ facial dataset, we compare our method on 500 FFHQ images for a fair evaluation. As shown in Table \ref{tab:ffhq}, our method outperforms the second-best method in identity preservation (ID) by 11.4\% and image quality (IQ) by 16.1\%. However, it does not achieve the best performance in lighting error (LE) and image fidelity (FID) because these methods are trained on FFHQ, while our model is trained on different video datasets, resulting in slightly lower lighting and fidelity performance. Notably, since our method is designed for video sequences and FFHQ is an image dataset, we replicate each image 16 times to form a video sequence in order to adapt the method for image testing.

In addition to portrait relighting, we use the lighting and shape from the reference image and the pose from the driving image to render shading hints, guiding our model to achieve cross-identity portrait animation, which we then evaluate. Beyond the previously mentioned metrics, we incorporate a POSE metric to assess the pose accuracy of the animated portraits, ensuring alignment with the poses in the driving video. The POSE evaluation method follows that of \cite{siarohin2021motion}, using a facial landmark detection model \cite{bulat2017far} to measure the pose error between the animated portraits and the driving portraits based on facial keypoints. As shown in Table \ref{tab:animate}, our method outperforms the other methods in all metrics, particularly achieving a 29.7\% improvement in image fidelity (FID), a 10.1\% improvement in image quality (IQ), and an 8.7\% improvement in identity preservation (ID) compared to the second-best method.
\begin{figure*}[!htbp]
	\centering
	\includegraphics[width=0.85\textwidth]{resources/exp_compare_V2.pdf}
	\caption{Qualitative comparisons with DPR \cite{zhou2019deep}, SMFR \cite{hou2021towards}, StyleFlow \cite{10.1145/3447648}, NFL \cite{nerffacelighting}, and DiFaReli \cite{ponglertnapakorn2023difareli}. The first column shows the input video frames, and the remaining columns present relighted results under various lighting conditions. Our method demonstrates more realistic performance, particularly in challenging cases such as side lighting.}
    \label{fig:compare_hdtf}
    \vspace{-0.18in}
\end{figure*}
\begin{figure}[!htbp]
	\centering
	\includegraphics[width=0.45\textwidth]{resources/exp_compare_ffhq_V2.pdf}
	\caption{Qualitative comparison of portrait relighting with NFL \cite{nerffacelighting}, StyleFlow \cite{10.1145/3447648}, and DiFaReli \cite{ponglertnapakorn2023difareli} on the FFHQ dataset \cite{karras2019style}. The first column shows the input FFHQ portrait images, and the remaining column display the relighted results under various lighting conditions. Our method demonstrates more realistic results.}
    \label{fig:compare_ffhq}
    \vspace{-0.15in}
\end{figure}
\begin{figure}[!htbp]
	\centering
	\includegraphics[width=0.45\textwidth]{resources/exp_animate.pdf}
	\caption{Qualitative comparison of cross-identity portrait animation with DaGAN \cite{hong2022depth}, StyleHEAT \cite{yin2022styleheat} and AnimateAnyone \cite{hu2024animate} on the HDTF dataset. Our method demonstrates more lifelike results.}
    \label{fig:compare_animate}
    \vspace{-0.18in}
\end{figure}

\begin{figure}[!htbp]
	\centering
	\includegraphics[width=0.45\textwidth]{resources/exp_ablation_module.pdf}
	\caption{Ablation study comparing the performance of our model in portrait generation under different adapter combinations. $F_s$ represents using only the shading adapter, $F_r$ represents using only the reference adapter, and $F_s + F_r$ represents using both adapters together.}
    \label{fig:ablation_module}
    \vspace{-0.25in}
\end{figure}
\begin{figure}[!htbp]
	\centering
	\includegraphics[width=0.45\textwidth]{resources/ablationstudy_w_V2.pdf}
	\caption{Ablation study comparing our model with varying strengths of multi-condition classifier-free guidance $\omega$. As $\omega$ increases, the relighting effect increasingly aligns with the target lighting; however, this comes at the cost of some loss of identity information and a decrease in image quality.}
    \label{fig:ablation}
    \vspace{-0.18in}
\end{figure}
\subsection{Qualitative Evaluation}
We compare our approach with previous portrait relighting methods on the HDTF dataset, including state-of-the-art face alignment-based approaches such as StyleFlow \cite{10.1145/3447648}, NFL \cite{nerffacelighting}, and DiFaReli \cite{ponglertnapakorn2023difareli}. Additionally, we compare our method with face alignment-free methods like DPR \cite{zhou2019deep} and SMFR \cite{hou2021towards}. The results are shown in Fig. \ref{fig:compare_hdtf}. We find that face alignment-based methods easily suffer from background detail loss and identity degradation, especially in pre-trained StyleGAN-based \cite{karras2020analyzing} methods like StyleFlow and NFL (e.g., see the results in the fourth and fifth columns, where the background details are completely lost, and the facial identity is inconsistent with the input). On the other hand, DiFaReli, based on a pre-trained diffusion model \cite{preechakul2021diffusion}, benefits from the DDIM inverse \cite{song2020denoising} method, which successfully reconstructs background details and preserves identity; however, it introduces noticeable artifacts on the face.

Although face alignment-free methods like DPR and SMFR achieve relighting without losing background and facial identity, the trade-off is a significant reduction in image quality, with the lighting appearing unnatural, as if a shadow has been cast over the image (e.g., in the first and second rows of the third column for SMFR). In contrast, our method in the final column greatly outperforms others in both image quality and the realism of the lighting effects. Notably, our approach accurately renders specular reflections on the face and eyes, as well as realistic shadows cast by facial muscles, while keeping identity loss within acceptable limits. The background details are also largely preserved. Overall, our approach demonstrates superior capability.

Since NFL, StyleFlow, and DiFaReli are trained on the aligned FFHQ dataset, we visualize the relighting results on FFHQ for a fair comparison. As shown in Fig. \ref{fig:compare_ffhq}, NFL and StyleFlow lose background details and alter the portrait identity. DiFaReli preserves background details but introduces facial artifacts, lowering image quality. In contrast, our method maintains background details and identity consistency, achieving optimal image quality.

Additionally, we compare our method with DaGAN, StyleHEAT, and AnimateAnyone for portrait animation. As shown in Fig. \ref{fig:compare_animate}, while DaGAN preserves the pose from the driving frame, the portrait identity differs significantly from the reference, and the image quality is low. StyleHEAT introduces distortions in cross-identity portrait animation, and although AnimateAnyone, a diffusion model guided by a reference-net, generates higher image quality, it still suffers from identity loss and occasional facial artifacts.
\subsection{Ablation Study}
\textbf{Effectiveness of Adapters.} Our method constructs intrinsic and extrinsic feature subspaces using the reference and shading adapters, respectively, enabling relightable portrait animation by merging these subspaces. We conduct an ablation study with different adapter combinations. First, when retaining only the shading adapter as shown in Fig. \ref{fig:ablation_module}, the column labeled $F_s$  illustrates that the generated portrait’s pose and lighting align with the shading hints, indicating that only the extrinsic features are transferred. When only the reference adapter is used, the column labeled $F_r$ shows that the generated portrait closely resembles the reference with only minor variations, such as blinking, indicating intrinsic feature preservation. When both adapters are used, the column labeled $F_s + F_r$ demonstrates that the generated portrait not only matches the pose and lighting of the shading hints but also maintains the identity and appearance of the reference.

\noindent\textbf{Effectiveness of Guidance Strength.} In Fig. \ref{fig:ablation}, we visualize the relighting results for different $\omega$ values. When $\omega = 2$, the lighting effect is minimal, with only small differences from the input image, resulting in good identity retention. In contrast, when $\omega = 8$, the lighting effect closely aligns with the target lighting, but this also leads to reduced image quality and some loss of identity retention. The primary reason for this phenomenon is that as $\omega$ increases, the proportion of extrinsic features grows, while the proportion of intrinsic features diminishes, resulting in a degradation of identity information from the reference image. Consequently, higher values of $\omega$ enhance lighting effects but lead to greater identity loss.


\section{Conclusion}

%In this paper, w
We propose a new PEFT method called DiffoRA, which enables efficient and adaptive LLM fine-tuning based on LoRA. 
Instead of adjusting every interior rank, 
%of the decomposition matrices 
%of all modules, 
we argue that adopting LoRA module-wisely is sufficient. 
To achieve this, we construct a DAM to select the modules that are most suitable and essential to fine-tune. We theoretically analyze how the DAM impacts the convergence rate and generalization capability.
%of the pre-trained model. 
Furthermore, we adopt continuous relaxation and discretization to establish DAM.
%for each task. 
To alleviate the issue of discretization discrepancy, we utilize the weight-sharing strategy for optimization. 
%We fully implement our method and t
The experimental results demonstrate that our DiffoRA works consistently better than the baselines across all benchmarks. 



{
    \small
    \bibliographystyle{ieeenat_fullname}
    \bibliography{main}
}

% WARNING: do not forget to delete the supplementary pages from your submission 

\clearpage
% \setcounter{page}{1}
% \maketitlesupplementary
\begin{center}
Supplementary Material
\end{center}

% {
%     \onecolumn
%     \centering
%     \Large
%     \textbf{\thetitle}\\
%     \vspace{0.5em}Supplementary Material \\
%     \vspace{1.0em}
% }

\section{Proof of \cref{theorem:dr}}
We require some additional regularity assumptions:
\begin{assumption} 1) The number of classes $C$ is bounded w.r.t the number of samples $N$, 2) the missingness mechanism $P(A=1|Y,\theta)$, as well as its estimated counterpart $P(A=1|Y,\theta)$, are bounded below by some constant $\epsilon > 0$, 3) the quantities $P(Y|X,\theta)$ and $P(A|Y,\theta)$ are estimated using auxiliary samples independent of samples used for the sample averaging.
\label{assumption:extra}
\end{assumption}
Assumptions 1 and 2 are natural. For the missingness mechanism, the ground truth being bounded means that there is a non-vanishing proportion of samples for every class. The boundedness of the estimate can be enforced by clipping the estimate. Assumption 3 is called sample splitting in \cite{kennedy-dr}.

For convenience we use operator $\E_N$ to denote the average of $N$ samples i.e. $\frac{1}{N}\sum_{i=1}^N$. Note that this is by itself a random variable, in contrast to $\E$ which is a fixed number.

\begin{proof}[Proof of \cref{theorem:dr}] Because $C$ is bounded (assumption \ref{assumption:extra}), we can fix a class $c$ and prove the theorem.
Let us define the influence function $\phi$, parameterized by $\theta$, as
\begin{equation}
\phi(O | \theta)(c) = P(Y=c|X,\theta) + \frac{\one(A=1)}{P(A=1|Y,\theta)} (\one(Y=c) - P(Y=c|X,\theta)) - P(Y=c)
\end{equation}
As we have done in the main text, we use $\phi(O)$ to denote the same function but all estimated quantities are replaced with their truths. In other words, we use $\phi(O)$ for $\phi(O|\theta_0)$ where $\theta_0$ is the truth, given that our model contains $\theta_0$ e.g. when the model is consistent.

Recall that:
\begin{equation}
\begin{aligned}
\Psi_{dr}(\theta)(c) &= \frac{1}{N}\sum_{i=1}^N \left\{P(Y=c|X,\theta) + \frac{\one(A=1)}{P(A=1|Y,\theta)} (\one(Y=c) - P(Y=c|X,\theta))\right\}\\
&= \E_N [\phi(O|\theta)(c)] + P(Y=c)
\end{aligned}
\end{equation}

We will show that:
\begin{equation}
\Psi_{dr}(\theta)(c) - P(Y=c) = (\E_N - \E)[\phi(O)(c)] + o_P(N^{-1/2})
\label{eq:proof-linearity}
\end{equation}
To do that, we use the following decomposition
\begin{equation}
\begin{aligned}
\Psi_{dr}(\theta)(c) - P(Y=c) &= \E_N [\phi(O|\theta)(c)] \\
&= (\E_N - \E)[\phi(O)(c)] + (\E_N - \E)[\phi(O|\theta)(c) - \phi(O)(c)] + \E[\phi(O|\theta)(c)]
% &+ (\E_n - \E)[\phi(O;\theta) - \phi(O)]\\
% &+ \E[P(Y=c|X,\theta)] - \E[P(Y=c|X)] + \E[\phi(O,\theta)]
\end{aligned}
\end{equation}
and analyze the second and third term. The third term is:
\begin{equation}
\begin{aligned}
\E[\phi(O|\theta)(c)] &= \E[P(Y=c|X,\theta)] + \E\left[\frac{\one(A=1)}{P(A=1|Y,\theta)}(\one(Y=c) - P(Y=c|X,\theta))\right]- P(Y=c) \\
&= \E\left[P(Y=c|X,\theta) + \frac{P(A=1|Y)}{P(A=1|Y,\theta)}(P(Y=c|X) - P(Y=c|X,\theta))\right] - \E[P(Y=c|X)]\\
&= \E\left[(P(Y=c|X,\theta) - P(Y=c|X)) (P(A=1|Y,\theta) -P(A=1|Y)) \frac{1}{P(A=1|Y,\theta)}\right]\\
\end{aligned}
\end{equation}
by Cauchy-Schwarz inequality:
\begin{equation}
\begin{aligned}
\E[\phi(O|\theta)(c)] &\le \frac{1}{\epsilon} \|P(A=1|Y,\theta) - P(A=1|Y)\|_2 \|P(Y=c|X,\theta) - P(Y=c|X)\|_{L_2(P)}\\
&= \frac{1}{\epsilon} o_P(N^{-1/4} N^{-1/4}) = o_P(N^{-1/2})
\end{aligned}
\end{equation}
by assumption \ref{assumption:4th-root-n} and that $P(A=1|Y,\theta) > \epsilon$ (assumption \ref{assumption:extra}). The second term can be bounded by Chebyshev inequality
% \begin{equation}
% \begin{aligned}
% \E[\E_N[\phi(O|\theta)(c) - \phi(O)(c)]] &= \E[\phi(O|\theta)(c) - \phi(O)(c)]\\
% \var[\E_N[\phi(O|\theta)(c) - \phi(O)(c)]] &= \frac{1}{N}\var[\phi(O|\theta)(c) - \phi(O)(c)] \le 
% \end{aligned}
% \end{equation}
\begin{equation}
P(|(\E_N - \E)[\phi(O|\theta)(c) - \phi(O)(c)]| \ge t) \le \frac{\var[\E_N[\phi(O|\theta)(c) - \phi(O)(c)]]}{t^2} = \frac{\var[\phi(O|\theta)(c) - \phi(O)(c)]}{Nt^2}
\end{equation}
note here that $\theta$ is independent of the samples used for $\E_N$ by assumption \ref{assumption:extra}. For any $\varepsilon > 0$, by picking $t = \frac{1}{\sqrt{N\varepsilon}}$ we get
\begin{equation}
P\left(\left|\frac{(\E_N - \E)[\phi(O|\theta)(c) - \phi(O)(c)]}{N^{-1/2}}\right| \ge \frac{1}{\sqrt{\varepsilon}}\right) \le \varepsilon \var[\phi(O|\theta)(c) - \phi(O)(c)]
\end{equation}
by the definition of $O_P$, we then get
\begin{equation}
(\E_N - \E)[\phi(O|\theta)(c) - \phi(O)(c)] = O_P(N^{-1/2}\var[\phi(O|\theta)(c) - \phi(O)(c)])
\end{equation}
Because $\phi$ is a continuous function of $P(Y|X,\theta)$ and $P(A|Y,\theta)$ (given $P(A|Y,\theta) > \epsilon$, assumption \ref{assumption:extra}), by the continuous mapping theorem and the fact that $P(Y|X,\theta)$ and $P(A|Y,\theta)$ are convergent in probability (assumption \ref{assumption:4th-root-n}), we get $\var[\phi(O|\theta)(c) - \phi(O)(c)] = o_P(1)$. This gives
\begin{equation}
(\E_N - \E)[\phi(O|\theta)(c) - \phi(O)(c)] = o_P(N^{-1/2})
\end{equation}
Therefore, we have shown that the second and third term are both $o_P(N^{-1/2})$, proving \cref{eq:proof-linearity}. As the final step, multiply both sides of this equation by $\sqrt{N}$ we get:
\begin{equation}
\sqrt{N}(\Psi_{dr}(\theta)(c) - P(Y=c)) = \sqrt{N} (\E_N - \E)[\phi(O)(c)] + o_P(1) \rightsquigarrow \mathcal{N}(0, \var[\phi(O)(c)])
\end{equation}
by the central limit theorem, and $\var[\phi(O)(c)] = \E[\phi(O)(c)^2]$ because $\E[\phi(O)(c)] = 0$.
\end{proof}

While we started with the definition of $\phi$, \cref{eq:proof-linearity} shows that $\phi$ is indeed an influence function. Now we show that $\phi$ is also the efficient influence function, by using the characterization of the model's tangent space \cite{tsiatis-missingdata}. Note that the joint probability factorizes as $P(X,A,Y) = P(X)P(Y|X)P(A|Y)$, therefore the tangent space $\mathcal{T}$ factorizes as $\mathcal{T} = \mathcal{T}_{X} \oplus \mathcal{T}_{Y|X} \oplus \mathcal{T}_{A|Y}$ where $\mathcal{T}_X = \{h(X): \E[h] = 0\}$, $\mathcal{T}_{Y|X} = \{h(X,Y): \E[h|X] = 0\}$, $\mathcal{T}_{A|Y} = \{h(A,Y): \E[h|Y] = 0\}$, and the 3 subspaces are pairwise orthogonal. All influence functions are orthogonal to the tangent space, but the influence function that is also in the tangent space has the smallest variance and is called the efficient influence function. As $\phi$ is already an influence function, we need only show that $\phi$ is in $\mathcal{T}$. We write $\phi$ as
\begin{equation}
\phi(O)(c) = (P(Y=c|X) - P(Y=c)) + \left[\frac{\one(A=1)}{P(A=1|Y)} - 1\right](\one(Y=c) - P(Y=c|X)) + (\one(Y=c) - P(Y=c|X))
\end{equation}
and note that the first, second and third term are in $\mathcal{T}_X$, $\mathcal{T}_{A|Y}$ and $\mathcal{T}_{Y|X}$ respectively. Therefore, $\phi$ is indeed in $\mathcal{T}$. The efficient influence function has the smallest variance of all influence function, and therefore our estimator being asymptotically linear in $\phi$ (\cref{eq:proof-linearity}) has the smallest mean squared error in a local asymptotic minimax sense \cite{kennedy-dr, asymptoticstatistics}

\section{Further background and related work}
\paragraph{Discussion on semi-supervised EM.}
It appears that semi-supervised EM was first used for parameter estimation when the missingness mechanism is non-ignorable in \cite{ibrahim1996parameter}, but has not been used for label shift estimation.
Perhaps this is because the semi-supervised situation where additional unlabeled data is available during training is rarer than the test-time adaptation case. EM is well suited to take advantage of the extra unlabeled data to improve the classifier under very scarce and long-tailed labeled data. While the connection between pseudo-labeling and EM has been explored before \cite{entropyminimization}, the situation with label shift has not until recently \cite{simpro}. Here the application of EM is much more interesting, because other than simply giving pseudo-labeling a rigorous formulation, EM also estimates the missingness mechanism (equivalently the label distribution shift), which is important for shift correction and thus high-quality pseudo-labels \cite{acr}. The application of confidence thresholding can be seen as a sparse variant of EM \cite{neal1998view}.

\paragraph{The doubly-robust risk.} 
\label{subsec:dr-risk}
A technique that also derives from the theory of semi-parametric efficiency is orthogonal statistical learning \citep{foster2023orthogonal}. The idea is to minimize the doubly-robust risk:
\label{subsec:method-dr-risk}
\begin{equation}
\label{eq:dr-risk}
\mathcal{R}(\theta_2) = \frac{1}{N} \sum_{i=1}^N \Bigg[ l(x_i, \hat y_i|\theta_2) + \frac{\one(a_i=1)}{P(A=a_i|Y=y_i, \theta_1)} (l(x_i, y_i | \theta_2) - l(x_i, \hat y_i | \theta_2))\Bigg]
\end{equation}
where $l(x,y|\theta) = -\sum_{c=1}^C [y]_c \log P(Y=c|X=x,\theta)$ is the negative cross-entropy. 
The notation $[y]_c$ means that we are using the $c$-entry in a C-dimension probability vector $y$. 
Thus, $y_i$ denotes the one-hot label of observation $i$, while $\hat y_i$ denotes the pseudo-label, which can be one-hot or all-zero. 
Finally, we use $\theta_1$ to denote that $P(a|y,\theta_1)$ is an estimation from a previous stage, but it can be estimated with $\theta_2$ as well. 
The risk $\mathcal{R}(\theta_2)$ can be used as a training loss in a straightforward fashion. 
Similar to the doubly robust estimation of $P(Y)$, the doubly robust risk provides approximately unbiased estimation of the risk. 
This property has been used in \citep{arelabelsinformative, onnonrandommissinglabels, drst} also in the semi-supervised learning setting.
More broadly, it is at the heart of one of the core techniques in heterogenous treatment effect estimation in causal estimation \cite{kennedy2023towards, foster2023orthogonal, wager2018estimation}. 
The focus here is not the estimation of $\mathcal{R}(\theta_2)$ per se, but the quality of the learned model \cite{foster2023orthogonal}.
By using the doubly-robust risk, we can achieve an optimality result similar in spirit to our theorem \cref{theorem:dr}, but for the generalization error.
While this is appealing, in practice there are 2 problems with this approach. First, the inverse probability weight $P(A=a_i|Y=y_i,\theta_1)$ can be very large if the class ratio is highly unlabeled, making training unstable \cite{kallus2020deepmatch, pham2023stable}. 
This problem exists for our estimation as well. However, it is much easier to control for estimation than for training because of the iterative nature of model update. Secondly, we can further write $\mathcal{R}$ as:
\begin{equation}
\mathcal{R}(\theta_2) = \frac{1}{N}\sum_{i=1}^N l\left(x_i, \hat y_i + \frac{\one(a_i=1)}{P(A=a_i|Y=y_i,\theta_1)} (y_i - \hat y_i)\Bigg\vert\theta_2\right)
\end{equation}
which is a cross-entropy loss with new meta-pseudo-labels. However, these labels are not meant to be learned exactly, and furthermore they can be negative. Thus, theoretical works have to put stringent assumptions on the models. In \cref{subsec:ablation-1}, we show that experimentally that the instability problem makes doubly-robust risk performance worse than our 2-stage approach.

\section{Training and hyperparameter settings.}
\label{subsec:training-setting}
For neural network training, we follow the implementation and hyperparameter settings of \cite{simpro}. In particular, we adapt the core code of SimPro for Supervised, MLE and EM. For MLE, we update $P(A|Y)$ using the Adam optimizer with learning rate 1e-3, while for EM we use a momentum update similar to SimPro's update of $P(Y|A)$ because it has a a closed-form solution at each mini-batch. We use Wide ResNet-28-2 on all methods and all datasets in this section, including Imagenet-127, because we are motivated by the fact that stage-1's goal is not classification accuracy but the estimation of a finite-dimensional parameter. When using Wide ResNet-28-2 for Imagenet-127, we use the hyperparameters of CIFAR-100, except we lower the batch size of unlabeled data to 2 times that of labeled data instead of 8 for memory reason. We do not perform additional hyperparameter tuning. All experiments can be performed on 1 A6000 RTX GPU, and are run 3 times. We report the total variation distance between the estimated and the ground truth unlabeled class distribution, similar to its usage in Theorem 3.1 of \cite{lsc}, and the top-1 classification accuracy.

In the second stage of our algorithm, we freeze our estimation and plug it in SimPro and BOAT.
We keep exactly the same hyperparameter settings that SimPro and BOAT use. In particular, for Imagenet-127, we now use ResNet-50 and run each experiment once.
In SimPro, we set the unlabeled class distribution $P(Y|A=0)$ at the E-step;  however, we still keep a running estimate of the class distribution $P(Y)$ in the logit adjustment loss \cref{eq:simpro-la-loss}. While it is possible to use the first stage estimate in the logit adjustment loss, we observe that doing so results in lower accuracy than using the the running average. This is conceptually consistent with the role of the running average - serving not as an accurate estimate of $P(Y)$ but to make the classifier's class distribution uniform through the logit adjustment loss, which is good for the test set. Similarly, in BOAT, we only replace $\Delta_c = \log P(Y|A=1) - \log P(Y|A=0)$ in equation (4) of \cite{boat}, which is adjusting a classifier's predictions from the labeled to the unlabeled class distribution, with our SimPro + DR estimate instead of their on-the-fly estimate. 


% \section{Additional experiments}
% % \begin{table*}[t]
\centering
\caption{Total Variation Distance on CIFAR-10-LT ($N_l = 500$, $M_l = 4000$) with different class imbalance ratios $\gamma_l$ and $\gamma_u$ under five different unlabeled class distributions.}
\label{tab:cifar10-tv}
\resizebox{\textwidth}{!}{
\begin{tabular}{lccccccccccc}
\toprule
& & \multicolumn{2}{c}{consistent} & \multicolumn{2}{c}{uniform} & \multicolumn{2}{c}{reversed} & \multicolumn{2}{c}{middle} & \multicolumn{2}{c}{head-tail} \\
\cmidrule(lr){3-4} \cmidrule(lr){5-6} \cmidrule(lr){7-8} \cmidrule(lr){9-10} \cmidrule(lr){11-12}
& & $\gamma_l = 150$ & $\gamma_l = 100$ & $\gamma_l = 150$ & $\gamma_l = 100$ & $\gamma_l = 150$ & $\gamma_l = 100$ & $\gamma_l = 150$ & $\gamma_l = 100$ & $\gamma_l = 150$ & $\gamma_l = 100$ \\
Model & Estimator & $\gamma_u = 150$ & $\gamma_u = 100$ & $\gamma_u = 1$ & $\gamma_u = 1$ & $\gamma_u = 1/150$ & $\gamma_u = 1/100$ & $\gamma_u = 150$ & $\gamma_u = 100$ & $\gamma_u = 150$ & $\gamma_u = 100$ \\
\midrule
Supervised & MLLS & 0.269 ± 0.252 & 0.038 ± 0.006 & 0.251 ± 0.046 & 0.255 ± 0.060 & 0.429 ± 0.028 & 0.493 ± 0.050 & 0.333 ± 0.042 & 0.320 ± 0.009 & 0.457 ± 0.034 & 0.444 ± 0.043 \\
Supervised & RLLS & 0.043 ± 0.001 & 0.044 ± 0.010 & 0.348 ± 0.034 & 0.305 ± 0.068 & 0.769 ± 0.016 & 0.678 ± 0.028 & 0.430 ± 0.008 & 0.368 ± 0.013 & 0.539 ± 0.018 & 0.503 ± 0.020 \\
\midrule
MLE & IPW & 0.027 ± 0.001 & 0.027 ± 0.000 & 0.319 ± 0.072 & 0.243 ± 0.010 & 0.674 ± 0.020 & 0.646 ± 0.041 & 0.438 ± 0.020 & 0.454 ± 0.026 & 0.547 ± 0.049 & 0.491 ± 0.059 \\
MLE & OR & 0.045 ± 0.004 & 0.042 ± 0.000 & 0.215 ± 0.026 & 0.203 ± 0.032 & 0.433 ± 0.017 & 0.395 ± 0.033 & 0.193 ± 0.006 & 0.209 ± 0.037 & 0.307 ± 0.147 & 0.249 ± 0.130 \\
MLE & DR & 0.090 ± 0.002 & 0.079 ± 0.000 & 0.407 ± 0.027 & 0.360 ± 0.007 & 0.425 ± 0.007 & 0.421 ± 0.029 & 0.256 ± 0.001 & 0.286 ± 0.031 & 0.435 ± 0.136 & 0.362 ± 0.122 \\
\midrule
EM & IPW & 0.035 ± 0.002 & 0.040 ± 0.001 & 0.021 ± 0.001 & 0.029 ± 0.015 & 0.303 ± 0.187 & 0.091 ± 0.010 & 0.119 ± 0.011 & 0.105 ± 0.022 & 0.104 ± 0.026 & 0.104 ± 0.051 \\
EM & OR & 0.037 ± 0.003 & 0.042 ± 0.002 & 0.016 ± 0.001 & 0.024 ± 0.012 & 0.269 ± 0.183 & 0.090 ± 0.008 & 0.122 ± 0.012 & 0.103 ± 0.022 & 0.072 ± 0.012 & 0.073 ± 0.024 \\
EM & DR & 0.034 ± 0.004 & 0.037 ± 0.001 & 0.014 ± 0.001 & 0.027 ± 0.020 & 0.264 ± 0.191 & 0.092 ± 0.005 & 0.111 ± 0.019 & 0.097 ± 0.026 & 0.077 ± 0.016 & 0.073 ± 0.028 \\
\midrule
SimPro & IPW & 0.070 ± 0.011 & 0.058 ± 0.000 & 0.046 ± 0.001 & 0.049 ± 0.005 & 0.254 ± 0.074 & 0.223 ± 0.098 & 0.097 ± 0.025 & 0.067 ± 0.002 & 0.105 ± 0.066 & 0.110 ± 0.079 \\
SimPro & OR & 0.071 ± 0.012 & 0.058 ± 0.000 & 0.045 ± 0.001 & 0.049 ± 0.006 & 0.040 ± 0.003 & 0.059 ± 0.017 & 0.074 ± 0.006 & 0.075 ± 0.002 & 0.033 ± 0.003 & 0.033 ± 0.003 \\
SimPro & DR & 0.017 ± 0.004 & 0.026 ± 0.001 & 0.019 ± 0.002 & 0.018 ± 0.003 & 0.039 ± 0.003 & 0.058 ± 0.025 & 0.091 ± 0.007 & 0.031 ± 0.001 & 0.015 ± 0.003 & 0.019 ± 0.007 \\
\bottomrule
\end{tabular}
}
\end{table*}
% 

\begin{table*}[t]
\centering
\caption{Total Variation Distance on CIFAR-100-LT ($N_l = 50$, $M_l = 400$) with different class imbalance ratios $\gamma_l$ and $\gamma_u$ under five different unlabeled class distributions.}
\label{tab:cifar100-tv}
\resizebox{\textwidth}{!}{
\begin{tabular}{lccccccccccc}
\toprule
& & \multicolumn{2}{c}{consistent} & \multicolumn{2}{c}{uniform} & \multicolumn{2}{c}{reversed} & \multicolumn{2}{c}{middle} & \multicolumn{2}{c}{head-tail} \\
\cmidrule(lr){3-4} \cmidrule(lr){5-6} \cmidrule(lr){7-8} \cmidrule(lr){9-10} \cmidrule(lr){11-12}
& & $\gamma_l = 20$ & $\gamma_l = 10$ & $\gamma_l = 20$ & $\gamma_l = 10$ & $\gamma_l = 20$ & $\gamma_l = 10$ & $\gamma_l = 20$ & $\gamma_l = 10$ & $\gamma_l = 20$ & $\gamma_l = 10$ \\
Model & Estimator & $\gamma_u = 20$ & $\gamma_u = 10$ & $\gamma_u = 1$ & $\gamma_u = 1$ & $\gamma_u = 1/20$ & $\gamma_u = 1/10$ & $\gamma_u = 20$ & $\gamma_u = 10$ & $\gamma_u = 20$ & $\gamma_u = 10$ \\
\midrule
Supervised & MLLS & 0.707 ± 0.016 & 0.313 ± 0.100 & 0.445 ± 0.172 & 0.309 ± 0.119 & 0.383 ± 0.075 & 0.397 ± 0.006 & 0.570 ± 0.001 & 0.373 ± 0.107 & 0.543 ± 0.009 & 0.231 ± 0.057 \\
Supervised & RLLS & 0.520 ± 0.007 & 0.133 ± 0.003 & 0.337 ± 0.125 & 0.253 ± 0.082 & 0.424 ± 0.060 & 0.463 ± 0.003 & 0.454 ± 0.021 & 0.306 ± 0.074 & 0.460 ± 0.028 & 0.241 ± 0.040 \\
\midrule
MLE & IPW & 0.075 ± 0.000 & 0.071 ± 0.001 & 0.229 ± 0.001 & 0.167 ± 0.002 & 0.565 ± 0.005 & 0.443 ± 0.007 & 0.415 ± 0.000 & 0.311 ± 0.005 & 0.343 ± 0.000 & 0.280 ± 0.001 \\
MLE & OR & 0.065 ± 0.002 & 0.061 ± 0.001 & 0.200 ± 0.007 & 0.143 ± 0.001 & 0.526 ± 0.011 & 0.399 ± 0.023 & 0.360 ± 0.003 & 0.256 ± 0.012 & 0.328 ± 0.003 & 0.266 ± 0.005 \\
MLE & DR & 0.149 ± 0.019 & 0.145 ± 0.010 & 0.243 ± 0.004 & 0.214 ± 0.019 & 0.568 ± 0.005 & 0.464 ± 0.014 & 0.403 ± 0.014 & 0.309 ± 0.012 & 0.365 ± 0.007 & 0.320 ± 0.004 \\
\midrule
EM & IPW & 0.097 ± 0.008 & 0.092 ± 0.004 & 0.239 ± 0.007 & 0.179 ± 0.003 & 0.478 ± 0.012 & 0.329 ± 0.020 & 0.262 ± 0.016 & 0.202 ± 0.003 & 0.312 ± 0.002 & 0.227 ± 0.001 \\
EM & OR & 0.121 ± 0.007 & 0.108 ± 0.005 & 0.261 ± 0.007 & 0.189 ± 0.004 & 0.489 ± 0.013 & 0.335 ± 0.020 & 0.274 ± 0.016 & 0.211 ± 0.004 & 0.336 ± 0.003 & 0.235 ± 0.001 \\
EM & DR & 0.125 ± 0.005 & 0.111 ± 0.004 & 0.269 ± 0.007 & 0.194 ± 0.005 & 0.497 ± 0.010 & 0.336 ± 0.024 & 0.281 ± 0.019 & 0.219 ± 0.008 & 0.336 ± 0.007 & 0.233 ± 0.004 \\
\midrule
SimPro & IPW & 0.125 ± 0.001 & 0.100 ± 0.005 & 0.166 ± 0.007 & 0.141 ± 0.009 & 0.353 ± 0.023 & 0.261 ± 0.008 & 0.202 ± 0.003 & 0.158 ± 0.005 & 0.277 ± 0.009 & 0.197 ± 0.003 \\
SimPro & OR & 0.133 ± 0.005 & 0.100 ± 0.004 & 0.160 ± 0.007 & 0.138 ± 0.010 & 0.322 ± 0.014 & 0.253 ± 0.008 & 0.202 ± 0.003 & 0.156 ± 0.005 & 0.269 ± 0.006 & 0.191 ± 0.004 \\
SimPro & DR & 0.122 ± 0.003 & 0.106 ± 0.006 & 0.188 ± 0.009 & 0.149 ± 0.006 & 0.343 ± 0.023 & 0.257 ± 0.007 & 0.219 ± 0.010 & 0.172 ± 0.002 & 0.279 ± 0.007 & 0.198 ± 0.004 \\
\bottomrule
\end{tabular}
}
\end{table*}

\end{document}
