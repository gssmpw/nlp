% \subsection{\textcolor{red}{Distributed Cognition}}
% % artifacts are objects or tools that individuals use to support or extend their cognitive processes. 


% % Distributed cognition is a theoretical approach that is concerned with the interactions among people, artifacts, and both internal and external representations. Rather than focusing exclusively on an individual's internal cognitive processes, that traditional cognitive approaches do, it focuses on the processes that take place in an extended ‘cognitive system’. These include verbal and non-verbal behavior, the coordinating mechanisms used by social actors, the forms of communication that take place, and the way tacit and explicit knowledge is shared and accessed. One major benefit is the explication of the complex interdependencies among people, artifacts, and technological systems that often can be overlooked when using traditional theories of cognition. (Rogers)

% Distributed cognition is a framework that conceptualizes cognitive processes not as confined within an individual's mind, but rather as interactions among individuals, artifacts, and socio-cultural elements~\cite{hutchins1995cognition}.
% % 
% In this framework, a cognitive process is defined by the functional relationships among the elements that participate in it, rather than by their spatial colocation~\cite{hollan2000distributed}. Furthermore, cognitive processes often require coordination between internal and external representations~\cite{rogers2022hci,hollan2000distributed} and can be distributed across time~\cite{hollan2000distributed}.



% Research in HCI has explored the application of distributed cognition in explaining how people utilize external cognitive resources, like artifacts and environmental elements, to tackle complex challenges~\cite{halverson2002activity,baumer2011comparing}. 
% % 
% For example, Wu et al.~\cite{wu2008collaborating} investigated how families dealing with memory impairments collaborate on everyday tasks, examining the mechanisms for accessing and sharing knowledge~\cite{rogers2006distributed}, facilitating communication and coordination within distributed cognitive framework~\cite{perry2003distributed}. 
% % 
% Nobarany et al.~\cite{nobarany2012facilitating} applied this framework to explore how analysts, reasoning artifacts, and user interface elements contribute to supporting unanticipated reuse in distributed collaboration. 
% % 
% In this study, we examine how PVI off-load cognitive efforts such as visual, remembering, and reasoning onto artifacts like \bma{} to achieve their goals. Additionally, we explore how these artifacts support and extend PVI's cognitive processes in their daily activities. 

% {\color{red}
% offload visual perception (missing) to be my ai]
% prosthetic are taking the visual load, managing it in alternative way, new prosthetic do it in another way different from white cane, bma is more dynamic, expensive, creative

% gap between what pvi expect and what they get

% cane: extension of hand, limited, bma can extend it to far-away objects, both prosthetic

% [artifact] address the need for visual input, how people can simulate bma's capability to visual cognition, simulation of bma in dcog system

% bma simulates pvi's visual cognition in dcog system


% [individual adaptation: ] how to accommodate challenges and limitations, tradeoff


% interchangably work together, two entities 
% }



% \subsection{Self-Determination Theory}
% Self-Determination Theory (SDT) is a psychological framework for understanding human intrinsic motivation~\cite{deci1985intrinsic}.
% % 
% At its core, SDT posits that people are inherently motivated to pursue personal growth and fulfillment, driven by three fundamental psychological needs: autonomy, competence, and relatedness.
% % 
% These needs are defined as follows: autonomy involves control over one’s own actions; competence refers to the capability of doing something; and relatedness pertains to forming meaningful connections with others.
% % 
% SDT argues that fostering conditions that support these psychological needs, such as participation in activities that resonate with personal values and interests, eventually cultivates the most volitional and highest quality forms of intrinsic motivation~\cite{deci1985intrinsic,deci2000and,ryan2000self}.



% In recent years, HCI field has increasingly adopted SDT as a framework in its research endeavors~\cite{deterding2016contextual,johnson2015all,ryan2006motivational,aufheimer2023examination,eilert2020osteoarthritis,guldenpfennig2019autonomy,muriana2021affecting,ford2012self,huang2019investigation}. 
% Predominantly, these studies have concentrated on the domains of gaming and gamification~\cite{deterding2016contextual,johnson2015all,ryan2006motivational,aufheimer2023examination}, 
% health and well-being~\cite{eilert2020osteoarthritis,guldenpfennig2019autonomy,muriana2021affecting} and education~\cite{ford2012self,huang2019investigation}.
% % 
% The application of SDT enables HCI researchers to examine these domains through the lens of psychological needs, offering deeper insights into the motivations that drive user engagement.
% % 
% In this study, we analyze the motivations of PVI for using \bma{} in their daily lives. We aim to extend SDT framework, traditionally applied within the domains of gaming (emphasizing fun) and health (emphasizing wellness) in HCI research, to explore its applicability to the under-investigated areas of safety and engagement within accessibility. 







\section{OLD Findings [will be removed]}
\label{sec:findings}


Seven participants (P1, P6-8, P10, P12, P14) highlighted their use of \bma{} for reading tasks, which is a prevalent need within VQA systems~\cite{avila2016remote,lee2020emerging,BME_100ways}. Their activities spanned reading mail, newspapers, and labels. 
% 
In this section, however, we aim to explore beyond basic reading functions, which are already well-addressed by existing technologies like OCR, to examine the more emerging and interactive practices introduced by LMM technologies. Specifically, we delve into how \bma{} supports managing personal and household tasks, understanding spatial environments, and engaging in social interactions. 

Our analysis reveals that \bma{} is part of the cognitive system in these contexts. In each use case, we examine how participants offload their cognition to \bma{} in \textit{``Artifact as Cognitive Support''}, as well as understand how participants accommodate challenges and limitations associated with \bma{} and how they creatively adapt \bma's capabilities to more creative and personalized use cases in \textit{``Individual Adaptation and Cognitive Strategies.''}




% [more than visual cog, understand some tasks, help in decision-making]

% why we use ``artifact as cognitive extension'' and ``Individual Adaptation and Cognitive Strategies''

% connect to contributions to the paper

% offload, assimilation?, reconfigure, accommodation (how we change what we do)

% the tool is part of the cognition

\subsection{Managing Personal and Household Tasks}
% Seven participants (P1, P6-8, P10, P12, P14) highlighted their use of \bma{} for reading tasks, which is a prevalent need within VQA systems~\cite{avila2016remote,lee2020emerging,BME_100ways}. Their activities spanned reading mail, newspapers, and labels. 
% % 
% In this section, however, we aim to explore beyond basic reading functions to examine the more complex and interactive tasks that participants engage in with \bma. Specifically, we will investigate how \bma{} is utilized in managing household appliances, assisting with cooking and dining, and providing fashion-related help.



%%%%%%


% It gives me the confidence I need to know that I can because like for instance, when I don't know the directions, I can just you know have it scan the box and then you know the directions will be written and then I'll know exactly what the directions say, so I can you know cook the product in the right amount of QuickTime. (P7, cooking and dining)


In this section, we explore how \bma{} facilitates participants in daily home activities, including using household appliances, cooking and dining, and fashion help. It highlights how \bma{} improves the practical aspects of home and personal life.  


\subsubsection{Help with Household Appliances}
\label{appliances}
Participants engaged \bma{} for assistance with household appliances, categorically divided into grid keypad and rotary control appliances. These devices necessitate tactile interaction for operational inputs, requiring actions such as pressing buttons or turning dials for setting controls or making selections. 
% 
% In facilitating these tasks, \bma{} functions by elucidating visual content. Participants take actions to align \bma's capabilities with their intent (e.g., ask specific questions relative to the visual content), or seeking alternative means (e.g., calling human volunteers) to fulfill their goals.



% layout 1: keypad panel with columns and rows, such as keypad, microwave. (description: P11) 
% 2, 5, 9, 11, 14
% layout 2: dail need rotating actions, such as thermostat, washer
% P1, P2, 4, 5, 13, 14

% \paragraph{\textbf{Environment:}}
% This use case primarily occurs within home settings (P1, P2, P9, P11), extending to scenarios involving travel (P4, P13, P14) or transitioning to new living spaces (P5). 

% [users know the environment better than AI]

% \textcolor{red}{[bma has limitations, but the users make it creatively as cognitive extension by overcoming its limitations]}


% \newtcolorbox{mybox}{colback=blue!5!white,colframe = blue!75!black}
\begin{mybox}
The picture shows the control panel of a GE washer/dryer. There are three main controls visible:\\
- On the left, there is a knob for Load Size with options Small, Medium, and Large.\\
- In the center, there is the GE logo and below it, the words ``WASHER / DRYER'' and a Start button.\\
- On the right, there are two knobs. The top knob is for Dryer Cycles with settings for Cottons (Less Dry, Optimum Dry, More Dry), and Delicates. The bottom knob is for Washer Cycles with settings for Quick Rinse, Spin, and various wash times (10, 6, 4, 2 minutes).\\
The Start button is a round, black button located below the GE logo and the words ``WASHER / DRYER''.
\end{mybox}
\captionof{figure}{Image description of a control panel provided by P14.}


\paragraph{Artifact as Cognitive Support:}

% \textcolor{red}{[Artifact as Simulation of Visual Cognition?]}

Five participants (P2, P5, P9, P11, P14) utilized \bma{} to navigate the grid layout of keypad panels on appliances, characterized by organized rows and columns. 
% 
\bma{} reduces participants' cognitive load of visually searching the layout by providing structured verbal descriptions.
% 
For example, in detailing a microwave's control panel, \bma{} begins by identifying the number of rows and columns, followed by employing directional (e.g., top, bottom, left, right, center) and sequential (e.g., second row, from left to right) terminology to guide users through the panel's arrangement. 

% [discussion:] This interaction reduces the cognitive effort required by participants to navigate unfamiliar control panels, effectively integrating into their cognitive system. 


% \begin{quote}
%     \textit{``Like the control panel of my microwave, it would tell me there are five rows of buttons. The buttons that are arranged in three columns. The top left button is start, the top middle is cook time, the top right is defrost, and then it would go down to the rows of buttons and give me similar descriptions for each one.''}  (P11)
% \end{quote}


%%%%%
However, we also identify limitations in the \bma’s ability to support all cognitive needs effectively in helping with household appliances. Six participants noted its inadequacy in accurately describing rotary control appliances like washers (P1) and thermostats (P2, P4, P5, P13, P14). These appliances require \bma{} to interpret current settings as users adjust dials for time, mode, or temperature. 


Critically, participants observed that \bma{} lacks a goal-oriented approach, often offering broad descriptions of visual elements without honing in on the user's specific objectives. P12 highlighted this issue, remarking, \textit{``Be My AI loves to make general descriptions, and it doesn't know what to focus on,''} leading to user frustration. For example, while \bma{} recognized a thermostat on the wall, it failed to convey essential details such as the current temperature setting or instructions for temperature adjustment. 

% [discussion:] not interpret distributed cognition, move to discussion, not use too much DCog: This reveals gaps in the cognitive system where the tool does not fully align with the user's cognitive goals, necessitating additional cognitive effort from users to compensate for these shortcomings.}



\paragraph{Individual Adaptation and Cognitive Strategies:}
To overcome these limitations, participants developed adaptive strategies. For example, P5 posed targeted questions to \bma, such as \textit{``What is the arrow pointed at right now on the current setting?''} This approach refined \bma's output to be more precise and relevant to the setting, enabling P5 to adjust the dial accurately.
% 
P12, on the other hand, aimed to improve \bma's focus by carefully aligning the camera with the specific item of interest, thus directing \bma{} to concentrate on particular elements rather than offering broad, scene-wide descriptions.
% 
These adaptations indicate how users actively reshape their interactions to optimize cognitive support. 

% [discussion:] These adaptations are indicative of the dynamic nature of distributed cognition, where users actively reshape their interaction patterns to optimize cognitive support. 


Additionally, eight participants (P1-3, P7, P8, P12-14) compared \bma{} with video-based human-assisted VQA systems, noting that the latter offers more adaptive and contextually aware support in tasks that require continuous real-time feedback, such as dial adjustments.
P13 shared an instance where volunteers from \bme{} understand her intent beyond mere identification of objects like a thermostat.
% 
Volunteers can provide immediate, actionable feedback and guidance, facilitating a more effective interaction through conversation.


\begin{quote}
    \textit{``I ended up pulling up a volunteer, and the volunteer was fantastic. He goes, `You know, these are tricky.' And he says, `It looks like it's set at 69. What do you want?' And I said, `71.' He goes, `Turn it, turn it. Yep, it's 71 from what I can tell.' And I went, `Cool.'''} (P13)
\end{quote}


In summary, \bma{} reduces the visual load involved in using household appliances, but it is not goal-oriented. Participants adapt \bma's capabilities to their intent by asking specific questions about the visual content or by seeking alternative means, such as calling human volunteers, to fulfill their goals.


%%%%%%%%%%%%%%
\subsubsection{Cooking and Dining}
\label{cooking}
Participants leveraged \bma{} for various culinary tasks, including identifying specific foods among similar items, like drink cans (P6), coffee beans (P7), cans of tomatoes or avocados (P10), chili spices (P10), and Campbell's soup cans (P13). They also used it to access cooking instructions regarding time and temperature (P7, P8, P9), read non-Braille recipe books (P9), and identify ingredients and suggest recipe (P12).
% 
% In these tasks, \bma{} converts graphical content to audible formats and offers recipe recommendations. Participants use \bma{} as a cognitive aid in accessing visual information and navigating the detailed recipes it provides. 



% \paragraph{\textbf{Interaction with the Environment:}}
% In many cognitive tasks, individuals rely on the knowledge and skills of others within their environment. 

% This use case typically happens in pantries (P2), kitchens (P5-10, P12, P13), and during shopping trips (P10). 
% % 
% The environmental context significantly influences the effectiveness of \bma. For instance, P5 reported challenges related to excessive or inadequate lighting, which hindered the \bma's ability to interpret visual cues. 
% In response to these conditions, \bma{} advises on retaking pictures for better clarity, exemplifying cognitive artifacts adapt to environmental limitations. This highlights a dynamic interaction where both the user and the artifact adjust behaviors to optimize performance. 

\begin{figure}[t!]
\centering
\includegraphics[width=0.5\textwidth]{images/cooking - eggshell.png}
\caption{Image description of eggs in a frying pan, followed by a question to check for the presence of eggshells. Example was originally drawn from X (\url{https://x.com/BeMyEyes/status/1706335189985526191})}
\end{figure}


%%%%%%%%
\paragraph{Artifact as Cognitive Support:}

\bma{} transforms complex visual information, such as graphics on food packaging, into comprehensible audible text that PVI can process. It acts as a cognitive proxy, interpreting and relaying information that would otherwise be inaccessible, thus reducing participants' cognitive load associated with tasks such as identifying specific food items or following cooking instructions.


Unlike other AI-powered VQA systems like Seeing AI and OCR, \bma{} processes graphic elements in addition to text. 
% 
For example, \bma{} informed P2 about an \textit{``allergy information emblem''} on product packaging, aiding in selecting allergy-safe foods. \bma{} detailed not just the brand name but also specific allergy-free labels that indicated the absence of common allergens like gluten, dairy, and sesame.  


% Participants noted that \bma{} surpasses other tools like Seeing AI and OCR in delivering visual-related insights. While Seeing AI and OCR excel in text extraction from items, identifying canned or boxed items swiftly through their quick read feature, \bma{} extends its capabilities to interpreting graphic elements, such as logos. 



Moreover, \bma{} enhances the cooking and dining experience through its ``ask more'' function, which allows participants to interactively engage with the artifact to receive tailored cooking advice. P12 utilized this feature, who sought recipe suggestions by inputting available ingredients: \textit{``I typed in the chat that I have chicken, pasta sauce, onion... and I asked for \bma, you know, to provide a recipe.''}
% 
% \begin{quote}
%     \textit{``The way I approached this was, I typed in the chat, that I said that I have a chicken, pasta sauce, onion, and I think one or two other things, and I asked for Be My AI, you know, to provide a recipe.''} (P12)
% \end{quote}
% % 
Responding to P12's input, \bma{} offered practical cooking instructions that included a list of ingredients, preparation methods (e.g., boiling, dicing, cutting into bite-sized pieces), required cookware, steps detailed with timing and sequencing, and seasoning tips.

% , demonstrating its ability to contribute beyond basic recognition and description tasks. 



% The environmental context significantly influences the effectiveness of \bma. For instance, P5 reported challenges related to excessive or inadequate lighting, which hindered the \bma's ability to interpret visual cues. 
% In response to these conditions, \bma{} advises on retaking pictures for better clarity, exemplifying cognitive artifacts adapt to environmental limitations. This highlights a dynamic interaction where both the user and the artifact adjust behaviors to optimize performance. 




\paragraph{Individual Adaptation and Cognitive Strategies:}
% users' cognitive, memory, perception or action. 


Participants employed specific strategies to manage the cognitive load involved in cooking tasks. By leveraging \bma{} as an external memory aid, P12 accessed complex cooking instructions in a step-by-step manner, reducing the need to retain and remember extensive procedural details from cookbooks or websites. 
% 
% Despite the support from \bma, P12 encountered challenges with the volume of information provided by \bma, noting, \textit{``it is a lot to, kind of, keep in your mind.''} 
% % 
% This led to a repetitive cycle where P12 listened to recipes, executed one or two steps, and then revisited the recipes for subsequent procedure. 
% 
This interaction with \bma{} involved retrieving information, processing it, and then executing based on the guidance received, illustrating the dynamics between P12's cognitive processes and the use of the artifact. 
% While \bma{} significantly reduces the need to consult external sources, it also necessitates repeated engagement to manage detailed tasks effectively. 
% Consequently, P12 was able to adaptively navigate the complexities of cooking, effectively utilizing \bma's support to address the cognitive challenges posed by detailed culinary tasks. 



In summary, \bma{} converts the graphical content on food packing to audible formats and offers recipe recommendations for cooking and dining. Participants use \bma{} as a cognitive aid in accessing visual information and navigating the detailed recipes it provides. 


%%%%%%%%%%%%
\subsubsection{Fashion Help}
\label{fashion}
\bma{} has served as a valuable tool for fashion-related tasks, assisting participants in several key areas: identifying colors of clothes (P2, P7, P9, P13), detailing patterns or designs on clothes (P2, P9), offering suggestions for assembling stylistically harmonious and color-coordinated ensembles like outfits, shoes, and jewelry (P2, P4, P5, P6, P12), and checking makeup (P5, P10).
% 
% In these tasks, \bma{} enhances decision-making in fashion by offering precise color identification and tailored fashion suggestions. Participants respond diversely to these capabilities. Some embrace the suggestions, finding that they bolster confidence and assist in assembling outfits, while others prefer to rely on their own judgment and decision-making to coordinate their attire.






% \paragraph{\textbf{Environment:}}
% This use case predominantly unfolds within the home setting, where participants select clothing, accessories, and apply makeup to prepare for work, social activities, or daily outings. 

% align ties with shirts
% So I picked that one because I knew what color it was. And I hung that shirt on my closet. And then what I did was I hung 4 different ties that I had, but I hung them over the shirt so that way Be My AI could see how it looked on the shirt. And after taking the picture of those things, I had a conversation with Be My AI. The first thing I said was basically, which ties goes well with the shirt. And it said that I should wear the black tie. it's a good option for basically any setting. (P12)




\begin{mybox}
The picture shows a light blue, long-sleeved dress shirt hanging on a black hanger. There are three ties hanging on the right side of the shirt. From left to right, the first tie is navy blue with small white paisley patterns, the second tie is solid red, and the third is solid black. The shirt is labeled ``Stafford'' and is size XL. The shirt is hanging on a wooden door, and there are some items on a shelf above the door, including a box of Nature Valley granola bars and a box of Mini Wheats cereal. To the left of the shirt, there is a partial view of a black office chair with a grey towel and a brown blanket on it.\\
Follow-up Question: Which tie should I wear with the shirt?\\
The solid black tie would be the most classic and versatile choice to wear with the light blue dress shirt. It creates a sharp contrast and is suitable for most formal occasions.
\end{mybox}
\captionof{figure}{Image description of a shirt with three ties and follow-up question on which tie to wear, provided by P12.}



\paragraph{Artifact as Cognitive Support:}

Participants have found \bma{} exceptionally precise in taking the visual load by identifying colors and patterns on clothing, as well as in assessing makeup, with no discrepancies reported. For instance, P5 effectively used \bma{} to verify the color, placement, and overall balance of her makeup, ensuring there were no mismatches or issues. 



This accuracy is particularly valued in fashion-related tasks, where \bma{} demonstrates a more consistent and reliable perception of color compared to human assistance. 
P4 pointed out that human assistants, such as \bme{} volunteers, often have subjective interpretations, stating, \textit{``each volunteer has a different way of seeing things''}. 
% 
P9 echoed that \bma{} maintains consistent detail in its interpretations, whereas human volunteers can vary significantly in their visual descriptions: \textit{``Sometimes they might not understand exactly, or like how much you want it to be described; they don't tell you the same description.''}



P12's experience further underscores the reliability of \bma{} over human volunteers in color identification. He shared an instance where a colorblind volunteer, without disclosing the colorblindness, provided incorrect color information of a tie. 
This example underscores \bma's utility as a reliable tool for accurate color matching, essential for appropriate fashion decisions, as P12 mentioned: \textit{``This is a really good application for \bma{} because you don't have to rely on someone's color vision.''}


% indepdence -- you don't have to rely on someone's color vision


Moreover, \bma{} enhances decision-making in fashion by providing suggestions to create stylistically coordinated outfits. P9 consulted \bma{} to harmonize colors of a top and dressy pants effectively, and P12 used it to select a tie that complemented his shirt. 
% 
A notable example is P2's use of the ``ask more'' function to complete a fall-season look with a turquoise top.  
Concerned about the seasonality of a turquoise top, P2 asked \bma, \textit{``Is this too summery of a color? Could I wear it in the fall?''} \bma{} suggested, \textit{``Turquoise is typically a bright color that's considered more of a summer-spring color, but you could tone it down into more of a fall look with a cardigan or chunky boots.''} Following this advice, P2 selected a grayish taupe pair of ankle boots, and upon capturing their image, \bma{} confirmed, \textit{``This might pair well with the turquoise top to make it more fall appropriate.''}







\paragraph{Individual Adaptation and Cognitive Strategies:}

% confidence -- know the outfits are matching


Participants were divided regarding \bma's fashion suggestion functionality. Some appreciated this feature because it \textit{``definitely creates more confidence''} (P2) in preparing outfits for specific occasions, such as work or sports events. For instance, P5 found \bma{} helpful to \textit{``lay out [her] outfit before the race,''} enhancing her preparation and confidence for race day. 



However, other participants (P6, P7) held the opposite views. 
While they utilized \bma{} to obtain descriptions of colors and patterns, they hesitated to rely on its fashion suggestion. These participants preferred to make their own judgment and choices about outfit matching, emphasizing the importance of human subjectivity in fashion decisions. 
% 
P6 expressed concerns about the AI's ability to replicate human subjectivity, remarking, \textit{``It's interesting how AI is being taught to simulate kind of the human factor of things.''}
% 
She linked her skepticism to instances where AI-generated responses were \textit{``strange''} and \textit{``complete nonsense,''} which contrasted with human creativity and nuanced understanding. 



\begin{quote}
    \textit{``No, no, no, no, I would never use it to do anything that required human subjectivity... I just don't trust AI with a task that is supposed to be subjective like that, particularly visual like that. Have you ever seen AI weirdness?... I think that just goes to show why I'm not gonna trust AI with my fashion yet.''} (P6)
\end{quote}


% [discussion:] This divergence in user experience underscores the opportunities and challenges in designing AI-powered VQA systems that adequately simulate human-like decision-making processes in subjective domains, such as fashion.


In summary, \bma{} enhances decision-making in fashion by offering precise color identification and tailored fashion suggestions. Participants respond diversely to these capabilities. Some embrace the suggestions, finding that they bolster confidence and assist in assembling outfits, while others prefer to rely on their own judgment and decision-making to coordinate their attire.



%%%%%%%%%%%
%%%%%%%%%%%
\subsection{Understanding Spatial Environments}

In this section, we examine how \bma{} enhances spatial awareness through scene descriptions, locates dropped objects, and facilitates navigation. It highlights \bma's role in improving participants' ability to interact with and navigate their surroundings.



\subsubsection{Scene Description}
\label{scene}

Participants utilized \bma{} for detailed scene descriptions across both indoor environments, such as theaters (P2) and room layouts (P3, P7, P9, P13), and outdoor settings, including holiday decorations (P5) and street scenes (P11, P12). 
% 
% In these tasks, \bma{} acts as a cognitive extension by providing comprehensive scene descriptions that enhance participants' perception of their surroundings. 
% Participants improve spatial awareness with these descriptions, such as ensuring their attire aligns with others to boost confidence in concerts.


\begin{figure}[t!]
\centering
\includegraphics[width=0.8\textwidth]{images/scene description.png}
\caption{Image description of a soccer stadium. Example was originally drawn from X (\url{https://x.com/darren_keepers/status/1731118043898716371})}
\end{figure}


\paragraph{Artifact as Cognitive Support:}


Six participants (P2, P3, P7, P9, P13, P14) valued \bma's detailed scene descriptions for increasing their spatial awareness. 
It often reveals spatial details that are not initially apparent, as noted by P11, \textit{``It gives me unexpected information about things that I didn't even know were there.''} 
% 
\bma{} provides comprehensive visual information, augmenting participants' visual cognition by detailing objects by color, size, shape, and spatial orientation (e.g., ``on the left,'' ``to the right,'' ``in the middle,'' ``in the background''). It also describes dynamic elements like bystander actions (e.g., ``sitting''), text on signs or advertisements, and the overall atmosphere (e.g., ``peaceful and natural,'' ``cozy and cheerful holiday vibe,'' ``warm and inviting atmoshpere''). 


These detailed scene descriptions offers a richer sensory experience than those often provided by human assistants. Participants particularly appreciated \bma{} for its ability to handle visually-overloaded scenes with less subjectivity than human assistants. P2 pointed out that human assistants often feel overwhelmed by the abundance of visual elements, struggling to decide what to describe. 
% 
In these scenarios, human assistants prioritize and focus on elements they deem important and potentially overlook other details. In contrast, P9 highlighted that \bma{} does not ignore details that human assistants might consider unimportant. 

% highlighted \bma's comprehensive visual interpretations of scenes, as human assistants assume a priority and focus on particular things that they believe important, and ignore visual information they think unimportant.  

\begin{quote}
    \textit{``Maybe a family member wouldn't think that I would want to know that there was like a picture frame hanging in the wall or something... But \bma{} just described everything, so like, it is nice because you get to imagine how something really might look or how the house might be decorated or things like that.''} (P9)
\end{quote}


% AI hallucinations
Despite these advantages, there are notable drawbacks with \bma{} related to AI hallucinations, errors where the artifact inaccurately identifies objects that aren't present. 
% 
For instance, \bma{} erroneously added nonexistent details to scenes, such as an object mistakenly reported behind P3 and incorrect features in P5's home. 
% 


% [discussion:] These inaccuracies underscore the need for further refinement in AI-powered VQA systems, aiming to achieve an optimal balance between the comprehensiveness of scene descriptions and their precision, without overcompensating by adding excessive or incorrect details.




\paragraph{Individual Adaptation and Cognitive Strategies:}

% increase spatial awareness
% {\color{blue}
% \sout{Six participants (P2, P3, P7, P9, P13, P14) valued \bma's detailed scene descriptions for increasing their spatial awareness. 
% \bma{} often reveals spatial details that are not initially apparent, as noted by P11, \textit{``It gives me unexpected information about things that I didn't even know were there.''} }
% % 
% \sout{P2 exemplified the benefit of this scene awareness, who used \bma{} to gain a better understanding of social contexts that sighted individuals might perceive unconsciously. At a concert, P2 was concerned about her attire's misalignment with the crowd. She described her use of \bma{} to assess the audience's dress code:} 


% \begin{quote}
%    \textcolor{red}{remove:} \textit{``Like it told me, you know, like the men are mostly wearing button-ups and women are wearing dresses. Okay, cool. I wore a dress also. So I'm on par with everybody. So yeah, those kind of, I guess like contextual things that, you know, sighted people are just picking up incidentally without really having to think about it, you know, it can give you more access to that.''} (P2)
% \end{quote}
% }


% AI hallucinations: confirm with human assistant

To overcome issues with AI hallucinations, participants adopt various strategies to ensure accuracy in the visual information provided by \bma. For instance, P3 recounted an incident where \bma{} inaccurately reported an object behind her. To confirm this, P3 consulted a human assistant, who verified that there was indeed nothing behind her in the picture. 

On the other hand, P5 shared an experience where \bma{} added non-existent details to an image at her home, including elements supposedly alongside their dogs. Relying on her familiarity with the environment, P5 was able to identify these inaccuracies without external assistance. This reliance on personal knowledge illustrates how participants used their understanding of familiar surroundings to challenge and correct the AI's interpretations. 



% participants either consult human assistants for confirmation (P3) or rely on their existing knowledge to discern inaccuracies (P5). 


% ``so apparently the Be My AI said there was an object behind me and there really wasn't in the image description. So then when I went and asked somebody, you know what is behind me and they said there wasn't anything behind you in the picture.'' (P3)

% ``it was at my own house. I took a picture of something and it was with my dogs in the picture. And it like added a bunch of stuff to the picture. I can't even remember exactly but it added some details that weren't in my house.'' (P5)

% [discussion:] This adaptive approach demonstrates how participants integrate \bma{} into their cognitive and perceptual frameworks, using it alongside human input to navigate and interpret complex environments.




In summary, \bma{} acts as a cognitive extension by providing comprehensive scene descriptions that enhance participants' perception of their surroundings. Despite its efficacy, it is subject to the limitations of AI hallucinations. To overcome this issue, participants either consult human assistants for verification or rely on their own knowledge to pinpoint inaccuracies.

% Participants improve spatial awareness with these descriptions, such as ensuring their attire aligns with others to boost confidence in concerts.


%%%%%%%%%%%
\subsubsection{Locating Dropped Objects}
\label{locate_object}
The sequential descriptions provided by \bma{} assist participants in locating dropped objects (P4, P12, P13) such as earbuds and hair ties. 
% Participants enhance this functionality by integrating their auditory perception and spatial memory, enabling them to fine-tune the camera's positioning for more accurate scanning and help \bma{} concentrate on the relevant area. 



\paragraph{Artifact as Cognitive Support:}
Participants found that \bma's structured approach to describing surroundings aids in locating dropped objects. By systematically detailing the environment using a \textit{``top to bottom, left to right''} sequence, the artifact helps orient users more effectively compared to random starting points. 
% 
For example, \bma{} enhances spatial orientation by providing specific relative locations. After a general description of the scene, it indicates, \textit{``picture of a carpet with a hair tie in the upper right hand corner,''} (P13) or notes 
\textit{``the earphones are directly in front of you, between your feet''} (P12). 
These descriptions allow participants to pinpoint lost objects with greater accuracy.



\paragraph{Individual Adaptation and Cognitive Strategies:}

Participants utilized \bma{} by integrating it with their own sensory and cognitive strategies to locate dropped objects. They enhanced their use of the artifact by relying on auditory cues and spatial memory, which helped in initially estimating the dropped object's location. This approach allowed them to optimize the camera angle for taking pictures that \bma{} would analyze. 
For instance, P4 used his \textit{``listening skills''} to determine the location where the object fell and guided where to point the camera accordingly. 
% 
Similarly, P12 adjusted his position slightly backward from the seating area to capture a better overview of the floor where the earbuds were dropped. 
% 
These adaptions illustrate how participants combine their understanding of the environment with \bma's technological capabilities to manage tasks that require spatial awareness.


In summary, \bma's sequential descriptions support participants in finding dropped objects. Participants leveraged their auditory skills and spatial memory to adjust the camera's position, allowing for more precise scanning and enabling \bma{} to focus on the intended search area.




%%%%%%%%%%
\subsubsection{Navigation}
\label{navigation}
\bma{} is used for navigation tasks, such as localizing and orientating participants by reading airport signages (P2, P5), and store names when disoriented (P10). 
% 
% It boosts spatial awareness and independence, although some participants question its practicality and safety during navigation. However, there is consensus that \bma{} can only complement, rather than replace participants' orientation and mobility (O\&M) skills. 



\paragraph{Artifact as Cognitive Support:}

Participants (P2, P5, P10) employed \bma{} as a tool to assist in localization and orientation when approaching their destinations. 
It reduces the cognitive process by offloading visual tasks such as reading signages and recognizing surroundings.
% 
For instance, P2 utilized \bma{} to identify gate numbers at the airport, which helped her determine the location and navigate accordingly. 
% 
Likewise, P5 used \bma{} to read a signage indicating the direction to the transportation area at the airport, while P10 took pictures of her surroundings for orientation when lost in her neighborhood.  
% 
Such instances illustrate \bma's role in providing participants with \textit{``better spatial awareness of what's going on around [them]''} (P5), thereby allowing participants to gain \textit{``more independence and be more familiar with [their] environment''} (P10).
% These participants found the visual interpretations from \bma{} to be accurate and helpful in identifying when a picture was incomplete or blurry. P5 mentioned that \textit{``\bma{} will say that there were some words visible but not the whole sign so that you can take another picture. Or sometimes it will tell you that it's blurry so it contextually knows and it tells you when the information is incomplete.''} 

% 




However, other participants (P6, P11, P13, P14) reported challenges with \bma{} for navigation, particularly due to its limited camera view and the practical issues of mobility. 
As P6 described, participants \textit{``have to stand there and keep taking pictures and taking pictures,''} check what's captured, determine if it's helpful for navigation, and adjust the angle for another pictures. This iterative process can be time-consuming and might cause self-consciousness, as \textit{``taking constant pictures with their camera could also look weird''} (P6) to bystanders.


Moreover, P13 critiqued that \bma{} is not goal-oriented enough to aid navigation because of its broad descriptions. For instance, she used \bma{} to find a specific office, but it failed to read the office's door number and instead \textit{``It sa[id] something like `wall with door,' which is, you know, not particularly helpful.''}



% P13 indicated that \bma{} could be a good cognitive extension in navigation if identifying the exact door number, along with more information regarding the direction, distance. For instance, \textit{``it says you know, stairs in front and it's like, Okay, that's great, but where are they? How far are they? Are they going up? Are they going down? Is there a railing?''} (P13)




\paragraph{Individual Adaptation and Cognitive Strategies:}


In navigation, participants combined \bma's visual cognition with other sources of guidance. When using \bma, they often knew they were near their intended areas based on prior help from human assistants (P2, P5) or through their existing mental map (P10). For instance, after getting directions from a human assistant and was seated in the Gate 12 area, P2 leveraged \bma{} to move from Gate 12 to Gate 13. This demonstrates the integration of multiple sources of information.  


Despite the varied feedback on \bma, participants agreed that \bma{} is \textit{``not a replacement for [their] mobility skills or just any skills in general''} (P14). P5 elaborated that due to the limited information captured and interpreted by \bma, it can only provide partial navigational details, such as the presence of obstacles, noting \textit{``it's not gonna be able to tell you, like, to protect you every step of the way.''} 


Consequently, participants integrate \bma{} with their orientation and mobility (O\&M) skills to enhance their navigation. P13 mentioned that while \bma{} can indicate the presence of obstacles like stairs, it does not provide essential details such as the distance to these obstacles or their characteristics, which are typically discerned through O\&M tools like white canes or guide dogs.


\begin{quote}
    \textit{``[\bma] says, you know, `stairs in front,' and it's like, `Okay, that's great, but where are they? How far are they? Are they going up? Are they going down? Is there a railing?' which would be information that the dog or the cane could tell you. So, I would say use it as a tool along with, but definitely not by itself.''} (P13)
\end{quote}


In summary, \bma{} boosts spatial awareness and independence, although some participants question its practicality and safety during navigation. However, there is consensus that \bma{} can only complement, rather than replace participants' O\&M skills. 


%%%%%%%%%%%
%%%%%%%%%%%
\subsection{Engaging in Social Interactions}

In this section, we assess how \bma{} supports various social engagements, including digital and physical human interactions, as well as human-animal interactions. It highlights \bma's role in enhancing users' social experiences across different settings.



%%%%%%%%%
\subsubsection{Digital Human Social Interactions}
\label{digital}

Participants leveraged \bma{} to engage in social interactions with blind and sighted people on social media by using it to describe pictures that are posted on social media (P1-4, P7, P10, P11), check the quality of pictures that participants intend to post on social media (P2, P4, P9), and generate descriptions of their own pictures to post on social media (P2, P3, P7).



\paragraph{Artifact as Cognitive Support:}
Pictures posted on social media often lack alt text, which describes an image's content and is accessible through screen readers.
% 
Seven participants (P1-4, P7, P10, P11) value \bma's detailed interpretations of images posted on social media that are missing alt text, allowing them to fully comprehend the visual content. For example, P7 illustrated the effectiveness of \bma{} in contrast to Facebook's AI image description tool. 
% 
While Facebook AI might provide a vague description such as \textit{``picture of five people,''} \bma{} provides a vivid depiction, such as describing a scene with someone skiing on a mountain alongside a visible ski lift. 
% \textit{``exactly what [people] are doing in the picture,''} including elements of the background. 
This level of detail enhanced P7's immersive experience, 
as she put it, \textit{``[\bma] gives me a clearer picture of what's around, especially in pictures of, you know, the scene. I can visualize it as to the point where I'm almost there.''}




%%%%%%%%%%%%%%%%
% check and post pictures 

Furthermore, \bma{} not only assists participants as recipients but also empowers them to become active contributors, enhancing their engagement on social media platforms. First, \bma{} supports participants in ensuring the quality and appropriateness of images they intend to post on social media (P2, P4, P9). It helps them check content clarity and composition to select suitable images. 
% 
Second, \bma{} assists in creating accessible alt text (P2, P3, P7), facilitating easier access for other visually impaired users on social media. 
It allows participants to add personalized details such as familiar names or specific characteristics. 
% , enhancing the descriptions' relevance and accessibility.




\paragraph{Individual Adaptation and Cognitive Strategies:}
The adoption of \bma{} has altered the social media behavior of participants, enhancing their confidence and engagement. By understanding the visual content of images, participants like P10 are more confident in posting pictures, overcoming previous hesitations due to uncertainty about the visual aspects of their posts: \textit{``Before, I was very shy to post pictures because I didn't know the visual description of them.''} 
% 
This tool also enables them to engage more actively with social media content. For instance, they can now provide detailed comments on friends' photos (P7) and use the content as a basis for conversations (P1, P3), achieving a level of social interaction similar to that of sighted users. 

% Participants can elaborate more detail with \bma, thus gain similar engagement on social media as sighted people because of mitigating the barrier of not enough description for visual content. 
% Participants were able to give more reactions, like \textit{``comment on them''} (P7), to their friends' photos on social media. Also, the content of others' image posts is \textit{``a nice talking point''} (P3), where participants can \textit{``have something to talk to them about because you know what pictures they put up''}(P1)


%%%%%% others post images

Moreover, \bma's accurate image descriptions also affect the behaviors of sighted individuals who share images with participants. It alleviates the pressure on sighted people to provide perfect descriptions, thereby facilitating more frequent and meaningful interactions. As noted by P3, \textit{``Sighted people who now feel like they can send pictures without the pressure of having to describe it perfectly,''} indicating that \bma{} helps bridge the communication gap between visually impaired and sighted communities.


In summary, \bma{} provides detailed image descriptions for participants, enabling them to access visual content on social media platforms. This capability allowed visually impaired and sighted users to interact more confidently and effectively in digital human interactions, promoting enhanced social engagement.



%%%%%%%%%%%%%
\subsubsection{Physical Human Social Interactions}
\label{physical}

\bma{} has been utilized for parenting (P4, P8, P9), facilitating gift exchanges (P2, P9, P12), and helping identify puzzles to play with families (P14). 


\begin{figure}[t!]
\centering
\includegraphics[width=0.5\textwidth]{images/physical social.png}
\caption{Image descriptions of a family celebration, provided by P9.}
\end{figure}



\paragraph{Artifact as Cognitive Support:}
In their parenting roles, participants (P4, P8, P9) effectively engaged \bma{} to offload the cognitive tasks of visual perception. They utilized \bma{} to describe and verify the appearance of their children, focusing on attributes like hair texture, skin complexion, and the styles and colors of clothing. This enabled them to dress their children appropriately. 
% [discussion] demonstrating how \bma{} extends their sensory capabilities into the visual domain. 
Similarly, P9 employed \bma{} to read and label books in Braille, enhancing her ability to contextualize stories and interact more engagingly with her child. 



\bma{} also facilitates enhanced engagement within family settings by providing detailed descriptions. P14 utilized \bma{} to differentiate between various puzzles by describing the images on the boxes, such as \textit{``whether it was a puzzle, a scene of cats, or a scene of bears.''} This capability is particularly notable when compared to human assistants, as P14 pointed out that \bma{} offers access to visual details that sighted family members often take for granted. For example, \bma{} read the text from the box specifying which piece needed to be placed in the center, a detail not initially provided by human assistants. In this context, \bma{} offloads the cognitive load associated with visual distinction, enabling P14 to participate more fully in family activities.

% \begin{quote}
%     \textit{``When used correctly, \bma{} can kind of foster a sense of connection with humans. It doesn't replace them but as someone who is low-vision, it allows me to have maybe more pre-information about something before I talk about it with my family or friends.''} (P14)
% \end{quote}


Despite its advantages, \bma{} has certain limitations that can hinder its effectiveness in providing goal-oriented information. P14's experience with a puzzle box illustrates this issue. While \bma{} could identify general shapes, it sometimes lacked the specificity required for more nuanced tasks.
% 
% In this instance, \bma{} read the text from the back of the puzzle box, indicating which piece should be placed in the center of the puzzle. When P14 asked for additional details to locate the center piece, \bma{} replied that it was square. However, this answer was insufficient because there were many square pieces, making it challenging to determine the correct one from the given description. 



\begin{quote}
    \textit{``When \bma{} read me the text of the box on the back of the puzzle that said, you know, this particular piece should be in the center of the puzzle. As a follow-up question I asked, `Could I have more information about the piece in the center?' And it said, `This piece is a square piece,' but I mean, there were many different square pieces, so I could not tell from that.''} (P14)
\end{quote}
This example shows that although \bma{} can process visual information, it may not always provide the level of detail needed to accomplish participants' goals. 





\paragraph{Individual Adaptation and Cognitive Strategies:}

The detailed visual interpretation capability of \bma{} enables participants to engage in more meaningful social interactions.
This adaptation was particularly evident in their approach to gift exchanges, which became more personalized and thoughtful. For example, P2 ensured that the price tags were removed from Christmas gifts before wrapping them, and P9 used \bma{} to read party invitations and prepare gifts that matched the theme. 

% [discussion: This capability underscores how \bma{} offloads the cognitive load of reading and interpreting visual information, allowing participants to engage more fully in social rituals.]


P12 used \bma{} to describe Christmas gifts, explaining that reading cards with \bma{} \textit{``makes it a more meaningful experience with other people.''}
% 
P12's experience illustrates how \bma{} enhances individual perception and, consequently, genuine social engagement. Before using \bma, opening Christmas cards was a performative act—pretending happiness despite not being able to see or understand the card's content. 
With \bma, P12 could capture the card's visual and textual details, allowing for authentic reactions to the messages. 
% This change transforms his interactions from a superficial act to a more genuine, joyful connection with others.

% {\color{blue}
% \begin{quote}
% \textit{``It definitely improves the quality of interactions around other people. And I would say quality of life in a sense, because, you know, when I get a Christmas card now, it's not like a dreaded, you know, I have to open it and act like I'm happy that I got it. Because I can't see it. Now I can actually take a picture of it and understand what it is.''} (P12)  
% \end{quote}
% }

In addition to enhancing individual perception, \bma{} plays a role in decision-making within physical social engagement. P14 highlighted this when describing her use of \bma{} to select a puzzle:
\textit{``I wanted part of my decision of putting a puzzle together to be, `Let me decide from a description of what the puzzles are that somebody could see on the box.'''} 
% 
P14 stressed the importance of this autonomy, noting that her decision to engage in the puzzle activity was enriched by having access to the same visual information that sighted people see on the box.

% [discussion] This autonomy is an essential component of distributed cognition, where external tools allow individuals to take control of their actions and reasoning, enhancing their overall engagement in social settings.


% [discussion] These examples underscore how \bma{} serves as a cognitive tool that not only offloads the tasks of visual perception and memory but also enhances individual reasoning and decision-making within social contexts. By extending sensory capabilities, \bma{} allows participants to engage more fully and independently in their social environments, reshaping their interactions and cognitive strategies.


In summary, \bma{} provided detailed visual interpretations that enabled participants to perform tasks independently and interact more authentically in physical social interactions. For example, participants used \bma{} to enhance engagements in family activities through parenting and gift exchanging. They also used \bma{} to facilitate decision-making and autonomy in choosing puzzles. 



%%%%%%%%%%%%%
\subsubsection{Human-Animal Interactions}
\label{animal}

Nine participants (P2, P4, P5, P8-11, P13, P14) utilized \bma{} to take leisure pictures of animals, including cats, dogs, birds, and horses, to know their status. 


\begin{figure}[t!]
\centering
\includegraphics[width=0.4\textwidth]{images/human animal.png}
\caption{Image description of a dog, provided by P9.}
\end{figure}

\paragraph{Artifact as Cognitive Support:}
Participants employed \bma{} to enhance their perception of animals' states and behaviors, which are critical aspects of interacting safely and effectively with animals. \bma{} acted as an external cognitive aid that enabled participants to access visual information. 
% 
For example, participants (P5, P10, P11) used \bma{} to grasp nuances such as animals’ facial expressions, activities, and body language. P10, in particular, highlighted the utility of \bma{} during walking her dog, where she put it, \textit{``Sometimes it's hard for me to know if the dog is peeing or what the dog is doing.''}


Furthermore, P11's application of \bma{} in a farm setting exemplifies its role in enhancing awareness and safety for human-animal interactions. He uses the artifact to determine the locations and directions of mini horses in stalls for safety measure, like \textit{``Is the horse's head facing towards me or away from me? Is the horse turned towards the side? Based on where I am, where is its tail or hind, or its back end located?''}
% 
P11 shared an incident where he was between two horses and was accidentally kicked by a mini horse that reacted jealously to his attention towards another larger one. 
% 
This incident underscores the necessity of \bma{} in preempting dangerous situations, thus allowing P11 to take preventive measures. For example, using lead rope to \textit{``turn [the mini horse's] head the other way, and change her direction, get her focus back on me so she wouldn't have kicked me.''}

% [discussion:] how \bma{} supports strategic reasoning and safety in real-time interactions.




\paragraph{Individual Adaptation and Cognitive Strategies:}

While \bma{} provides valuable information in human-animal interactions, participants often adapted the tool's outputs to align with their own perceptions and combine them with other sensory information to ensure accuracy and safety.


P5, for example, emphasized personal judgment over the tool's subjective interpretations. She preferred the flexibility to override \bma's assumptions about her dogs' expressions. This strategy allowed P5 to balance AI-based interpretations with her subjective inputs, reinforcing her sense of control in understanding pet behaviors.
\begin{quote}
    \textit{``I like having more control over the description. So I like the ability to edit that. Some blind people think, `How does it know that the dogs are happy? Why does it assume?' Some people don't like that it's making assumptions about the picture. I like having access to that information, but I like to be able to change it if I want.''} (P5)
\end{quote}

% control, competence - can change the description

P11 integrated visual interpretations from \bma{} with auditory, tactile, and olfactory cues to bolster safety during interactions with horses. He supplemented this information by consulting AI tools like ChatGPT, which guided him on how to interact safely based on sensory cues. For instance, he asked for advice on how to handle horses safely without sight, such as maintaining close contact and physical awareness of the horse's movements. This approach demonstrates how P11 skillfully combined technological insights with his sensory perceptions to ensure safety and navigate his interactions with horses effectively.


In summary, participants delegated the visual perception of monitoring the animal status to \bma. They further adapted \bma's output by incorporating their own perceptions and other sensory information to ensure accuracy and safety.




% \begin{quote}
%     \textit{``And you also want to use \bma{} to see when you go up to a stall, because sometimes they move horses around so you want to know, `is the same horse in the stall? or is there a different one? And is the horse big? Is it small? is it medium sized? Where is it? is it in the front? Is it in the back of the stall? You know which direction is it facing? Is it safe for me to go in or not?''} (P11)
% \end{quote}


% [integration with chatgpt] ensure safety based on hearing touching smelling feedback
% No, because I've used other AI. I've used ChatGPT, which is what Be My AI runs off of. So I would go and spend time with horses. And then if I didn't understand what I was hearing or touching or smelling or something that they did, I would go when asked ChatGPT, I would say you know I'm totally blind, the horse behaved this way. How can I interact with a horse you know, safely without being kicked? 

% Oh, a whole lot. Especially, I mean if I wouldn't have had Be My AI and other AI related tools. It would have probably been a lot more difficult for me to spend time with and interact with horses because I couldn't have told the AI, “I'm totally blind, give me tips on how to be safe around horses. How can I avoid being ticked as a totally blind person?” And it would tell me all those things it would say, “Well, if you're totally blind, you want to stay as close to the horse as possible. Keep your hand on it, especially when you're walking around it.” Things like that.


% {\color{red}
% summary -- what characteristics is supporting cognition
% }