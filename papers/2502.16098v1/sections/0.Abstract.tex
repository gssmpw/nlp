\begin{abstract} % 


Large multimodal models (LMMs) have enabled new AI-powered applications that help people with visual impairments (PVI) receive natural language descriptions of their surroundings through audible text. We investigated how this emerging paradigm of visual assistance transforms how PVI perform and manage their daily tasks. Moving beyond usability assessments, we examined both the capabilities and limitations of LMM-based tools in personal and social contexts, while exploring design implications for their future development. Through interviews with 14 visually impaired users of Be My AI (an LMM-based application) and analysis of its image descriptions from both study participants and social media platforms, we identified two key limitations. First, these systems' context awareness suffers from hallucinations and misinterpretations of social contexts, styles, and human identities. Second, their intent-oriented capabilities often fail to grasp and act on users' intentions. Based on these findings, we propose design strategies for improving both human-AI and AI-AI interactions, contributing to the development of more effective, interactive, and personalized assistive technologies.




\end{abstract}
% Through interviews with 14 visually impaired users and analysis of image descriptions from both participants and social media using Be My AI (an LMM-based application), we identified two key limitations. 

% Through interviews with 14 visually impaired users of Be My AI (an LMM-based application), along with analysis of its image descriptions from both participants and social media platforms, we identified two key limitations. 

% We identified two key limitations by studying Be My AI, an LMM-based application, through interviews with 14 visually impaired users and analysis of AI-generated image descriptions from both study participants and social media platforms. 