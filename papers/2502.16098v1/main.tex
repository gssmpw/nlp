% \documentclass[manuscript,review,anonymous]{acmart}   % anonymous

% \documentclass[manuscript,nonacm]{acmart}   % anonymous

\documentclass[sigconf]{acmart}

\usepackage{booktabs} % For formal tables
\usepackage{array}
\usepackage{multirow}
% \usepackage[export]{adjustbox}


\usepackage{enumitem}
\usepackage{colortbl}
\usepackage{wasysym}
\usepackage{xcolor}
% \usepackage[normalem]{ulem}
\usepackage{url}
\usepackage{soul}

\newcommand{\albert}[1]{\textcolor{blue}{#1}}
%%
%% \BibTeX command to typeset BibTeX logo in the docs
\AtBeginDocument{%
  \providecommand\BibTeX{{%
    \normalfont B\kern-0.5em{\scshape i\kern-0.25em b}\kern-0.8em\TeX}}}

%% Rights management information.  This information is sent to you
%% when you complete the rights form.  These commands have SAMPLE
%% values in them; it is your responsibility as an author to replace
%% the commands and values with those provided to you when you
%% complete the rights form.
\copyrightyear{2025}
\acmYear{2025}
\setcopyright{acmlicensed}\acmConference[CHI '25]{CHI Conference on Human Factors in Computing Systems}{April 26-May 1, 2025}{Yokohama, Japan}
\acmBooktitle{CHI Conference on Human Factors in Computing Systems (CHI '25), April 26-May 1, 2025, Yokohama, Japan}
\acmDOI{10.1145/3706598.3714210}
\acmISBN{979-8-4007-1394-1/25/04}

% %% These commands are for a PROCEEDINGS abstract or paper.
% \acmConference[Woodstock '18]{Woodstock '18: ACM Symposium on Neural
%   Gaze Detection}{June 03--05, 2018}{Woodstock, NY}
% \acmBooktitle{Woodstock '18: ACM Symposium on Neural Gaze Detection,
%   June 03--05, 2018, Woodstock, NY}
% \acmPrice{15.00}
% \acmISBN{978-1-4503-XXXX-X/18/06}


%% These commands are for a PROCEEDINGS abstract or paper.
% \acmConference[Conference acronym 'XX]{Make sure to enter the correct
%   conference title from your rights confirmation emai}{June 03--05,
%   2018}{Woodstock, NY}
%
%  Uncomment \acmBooktitle if th title of the proceedings is different
%  from ``Proceedings of ...''!
%
% \acmBooktitle{Woodstock '18: ACM Symposium on Neural Gaze Detection,
%  June 03--05, 2018, Woodstock, NY} 
% \acmISBN{978-1-4503-XXXX-X/18/06}



%%
%% Submission ID.
%% Use this when submitting an article to a sponsored event. You'll
%% receive a unique submission ID from the organizers
%% of the event, and this ID should be used as the parameter to this command.
% \acmSubmissionID{7224}

% comment the following line when submitting
\def \debug{}

%%%%%%%%% fixmes %%%%%%%%%%%%%%

% comment the following line to remove color codes
\def \debug{}

\ifx \debug \undefined

\newcommand{\fx}[1]{{}}
\newcommand{\fixmesb}[1]{{}}
\newcommand{\fixmejc}[1]{{}}
\newcommand{\fixmezz}[1]{{}}
\newcommand{\fixmejX}[1]{{}}
\newcommand{\fixmech}[1]{{}}
\newcommand{\fixmesl}[1]{{}}
\newcommand{\rev}[1]{{#1}}
\else

% \newcommand{\fx}[1]{{\textcolor{violet}{#1}}}
\newcommand{\fx}[1]{{\textcolor{red}{#1}}}
\newcommand{\fixmesb}[1]{{\bf\textcolor{red}{ [ sB FIXME: #1 ]}}}
\newcommand{\fixmejc}[1]{{\bf\textcolor{blue}{ [ jC FIXME: #1 ]}}}
\newcommand{\fixmery}[1]{{\bf\textcolor{brown}{ [ rY FIXME: #1 ]}}}
\newcommand{\fixmejx}[1]{{\bf\textcolor{green!10!orange!90!}{ [ jX FIXME: #1 ]}}}
\newcommand{\fixmech}[1]{{\bf\textcolor{purple}{ [ cH FIXME: #1 ]}}}
\newcommand{\fixmesl}[1]{{\bf\textcolor{yellow}{ [ sL FIXME: #1 ]}}}
\newcommand{\rev}[1]{{\color{black}{#1}}}


\newcommand{\V}{volunteer}
\newcommand{\vs}{volunteers}
% \newcommand{\bma}{Be My AI}
\newcommand{\bma}{Be~My~AI}
% \newcommand{\bma}{BMA}
\newcommand{\sbma}{BMA}  %s... for short
% \newcommand{\bme}{Be My Eyes}
\newcommand{\bme}{Be~My~Eyes}
\newcommand{\abme}{AI tool}
\newcommand{\ps}{participants}


% \newcommand{\vss}{volunteers}




\fi
%%%%%%%%%%%%%%%%%%%%%%%%%%%%%%%
%%%%%%%%%%%%%%%%% table %%%%%%%%%%%%%%%%%%
\newcolumntype{L}[1]{>{\raggedright\let\newline\\\arraybackslash\hspace{0pt}}m{#1}}
\newcolumntype{C}[1]{>{\centering\let\newline\\\arraybackslash\hspace{0pt}}m{#1}}
\newcolumntype{R}[1]{>{\raggedleft\let\newline\\\arraybackslash\hspace{0pt}}m{#1}}
\fancyhead{}

%
\newcounter{challengeno}
\newcommand{\challenge}[1]{\refstepcounter{challengeno}\label{#1}}

\newcounter{scenariono}
\newcommand{\scenario}[1]{\refstepcounter{scenariono}\label{#1}}

\newcounter{ideano}
\newcommand{\idea}[1]{\refstepcounter{ideano}\label{#1}}

%%
%% end of the preamble, start of the body of the document source.
\begin{document}

%%
%% The "title" command has an optional parameter,
%% allowing the author to define a "short title" to be used in page headers.


\title[Insights from Smartphone Interaction of Visually Impaired Users with Large Multimodal Models]
{Beyond Visual Perception: Insights from Smartphone Interaction of Visually Impaired Users with Large Multimodal Models}


% Emerging Practices for Large Multimodal Model (LMM) Assistance for People with Visual Impairments: Implications for Design
%LARGE multimodal models


% \renewcommand{\shortauthors}{Xie, Yu, Zhang, Lee, Billah, and Carroll}

% %% delete below for anonymous
\author{Jingyi Xie}
\affiliation{%
  \institution{Pennsylvania State University}
  \city{University Park}
  \state{PA}
  \country{USA}
  \postcode{16801}
}
\email{jzx5099@psu.edu}


\author{Rui Yu}
\affiliation{%
 \institution{University of Louisville}
  \city{Louisville}
  \state{KY}
  \country{USA}
  \postcode{40292}
}
\email{rui.yu@louisville.edu}


\author{He Zhang}
\affiliation{%
 \institution{Pennsylvania State University}
  \city{University Park}
  \state{PA}
  \country{USA}
  \postcode{16801}
}
\email{hpz5211@psu.edu}




\author{Syed Masum Billah}
\affiliation{%
  \institution{Pennsylvania State University}
  \city{University Park}
  \state{PA}
  \country{USA}
  \postcode{16801}
}
\email{sbillah@psu.edu}


\author{Sooyeon Lee}
\affiliation{
    \institution{New Jersey Institute of Technology}
  \city{Newark}
  \state{NJ}
  \country{USA}
  \postcode{07102}
}
\email{sooyeon.lee@njit.edu} 


\author{John M. Carroll}
\affiliation{%
 \institution{Pennsylvania State University}
  \city{University Park}
  \state{PA}
  \country{USA}
  \postcode{16801}
}
\email{jmc56@psu.edu}




%% The default list of authors is too long for headers.
% \renewcommand{\shortauthors}{Xie, Yu, Cui, Lee, Carroll, and Billah}
%% delete above for anonymous 



%%
%% The "author" command and its associated commands are used to define
%% the authors and their affiliations.
%% Of note is the shared affiliation of the first two authors, and the
%% "authornote" and "authornotemark" commands
%% used to denote shared contribution to the research.



%%
%% The abstract is a short summary of the work to be presented in the
%% article.

\begin{abstract}


Despite the significant advances in neural machine translation, performance remains subpar for low-resource language pairs.
Ensembling multiple systems is a widely adopted technique to enhance performance, often accomplished by combining probability distributions.
However, the previous approaches face the challenge of high computational costs for training multiple models.
Furthermore, for black-box models, averaging token-level probabilities at each decoding step is not feasible.
To address the problems of multi-model ensemble methods, we present a pivot-based single model ensemble.
The proposed strategy consists of two steps: pivot-based candidate generation and post-hoc aggregation.
In the first step, we generate candidates through pivot translation.
This can be achieved with only a single model and facilitates knowledge transfer from high-resource pivot languages, resulting in candidates that are not only diverse but also more accurate.
Next, in the aggregation step, we select $\textit{k}$ high-quality candidates from the generated candidates and merge them to generate a final translation that outperforms the existing candidates.
Our experimental results show that our method produces translations of superior quality by leveraging candidates from pivot translation to capture the subtle nuances of the source sentence.


\end{abstract}


%%
%% The code below is generated by the tool at http://dl.acm.org/ccs.cfm.
%% Please copy and paste the code instead of the example below.
%%
\begin{CCSXML}
<ccs2012>
   <concept>
       <concept_id>10003120.10011738</concept_id>
       <concept_desc>Human-centered computing~Accessibility</concept_desc>
       <concept_significance>500</concept_significance>
       </concept>
   <concept>
       <concept_id>10003120.10011738.10011773</concept_id>
       <concept_desc>Human-centered computing~Empirical studies in accessibility</concept_desc>
       <concept_significance>300</concept_significance>
       </concept>
 </ccs2012>
\end{CCSXML}

\ccsdesc[500]{Human-centered computing~Accessibility}
\ccsdesc[300]{Human-centered computing~Empirical studies in accessibility}

%%
%% Keywords. The author(s) should pick words that accurately describe
%% the work being presented. Separate the keywords with commas.
\keywords{People with visual impairments (PVI); large multimodal models (LMMs), Human-AI interaction, visual question answering (VQA); remote sighted assistance (RSA), Be My Eyes, Be My AI.}

% Remote

%%
%% This command processes the author and affiliation and title
%% information and builds the first part of the formatted document.
\maketitle

%%%%%%%%%%%%%%%%%%%%%%%%%%%%%%%%%
\section{Introduction}
\label{sec::intro}

Embodied Question Answering (EQA) \cite{das2018embodied} represents a challenging task at the intersection of natural language processing, computer vision, and robotics, where an embodied agent (e.g., a UAV) must actively explore its environment to answer questions posed in natural language. While most existing research has concentrated on indoor EQA tasks \cite{gao2023room, pena2023visual}, such as exploring and answering questions within confined spaces like homes or offices \cite{liu2024aligning}, relatively little attention has been dedicated to EQA tasks in  open-ended city space. Nevertheless, extending EQA to city space is crucial for numerous real-world applications, including autonomous systems \cite{kalinowska2023embodied}, urban region profiling \cite{yan2024urbanclip}, and city planning \cite{gao2024embodiedcity}. 
% 1. 环境复杂性   
%    - 地标重复性问题(如区分相似建筑)  
%    - 动态干扰因素(交通流、行人)  
% 2. 行动复杂性  
%    - 长程导航路径规划  
%    - 移动控制、角度等  
% 3. 感知复杂性  
%    - 复合空间关系推理("A楼东侧商铺西边的车辆")  
%    - 时序依赖的观察结果整合

EQA tasks in city space (referred to as CityEQA) introduce a unique set of challenges that fundamentally differ from those encountered in indoor environments. Compared to indoor EQA, CityEQA faces three main challenges: 

1) \textbf{Environmental complexity with ambiguous objects}: 
Urban environments are inherently more complex,  featuring a diverse range of objects and structures, many of which are visually similar and difficult to distinguish without detailed semantic information (e.g., buildings, roads, and vehicles). This complexity makes it challenging to construct task instructions and specify the desired information accurately, as shown in Figure \ref{fig:example}. 

2) \textbf{Action complexity in cross-scale space}: 
The vast geographical scale of city space compels agents to adopt larger movement amplitudes to enhance exploration efficiency. However, it might risk overlooking detailed information within the scene. Therefore, agents require cross-scale action adjustment capabilities to effectively balance long-distance path planning with fine-grained movement and angular control.

3) \textbf{Perception complexity with observation dynamics}: 
% Rich semantic information in urban settings leads to varying observations depending on distance and orientation, which can impact the accuracy of answer generation. 
Observations can vary greatly depending on distance, orientation, and perspective. For example, an object may look completely different up close than it does from afar or from different angles. These differences pose challenges for consistency and can affect the accuracy of answer generation, as embodied agents must adapt to the dynamic and complex nature of urban environments.


\begin{table}
\centering
\caption{CityEQA-EC vs existing benchmarks.}
\label{table:dataset}
\renewcommand\arraystretch{1.2}
\resizebox{\linewidth}{!}{
\begin{tabular}{cccccc}
             & Place  & Open Vocab & Active & Platform  & Reference \\ \hline
EQA-v1      & Indoor & \textcolor{red}{\ding{55}}          & \textcolor{green}{\ding{51}}      & House3D      & \cite{das2018embodied}  \\
IQUAD        & Indoor & \textcolor{red}{\ding{55}}          & \textcolor{green}{\ding{51}}      & AI2-THOR     & \cite{gordon2018iqa} \\
MP3D-EQA     & Indoor & \textcolor{red}{\ding{55}}          & \textcolor{green}{\ding{51}}      & Matterport3D & \cite{wijmans2019embodied} \\
MT-EQA       & Indoor & \textcolor{red}{\ding{55}}          & \textcolor{green}{\ding{51}}      & House3D      & \cite{yu2019multi} \\
ScanQA       & Indoor & \textcolor{red}{\ding{55}}          & \textcolor{red}{\ding{55}}      & -            & \cite{azuma2022scanqa} \\
SQA3D        & Indoor & \textcolor{red}{\ding{55}}          & \textcolor{red}{\ding{55}}      & -            & \cite{masqa3d} \\
K-EQA        & Indoor & \textcolor{green}{\ding{51}}          & \textcolor{green}{\ding{51}}      & AI2-THOR     & \cite{tan2023knowledge} \\
OpenEQA      & Indoor & \textcolor{green}{\ding{51}}          & \textcolor{green}{\ding{51}}      & ScanNet/HM3D & \cite{majumdar2024openeqa} \\
 \hline
CityEQA-EC   & City (Outdoor)  & \textcolor{green}{\ding{51}}          & \textcolor{green}{\ding{51}}      & EmbodiedCity & - \\ \hline
\end{tabular}}
\end{table}

\begin{figure*}[!htb]
\centering
    \includegraphics[width=0.78\linewidth]{figures/example.pdf}
% \vspace{-0.2cm}
\caption{The typical workflow of the PMA to address City EQA tasks. There are two cars in this area, thus a valid question must contain landmarks and spatial relationships to specify a car. Given the task, PMA will sequentially complete multiple sub-tasks to find the answer.}
% \vspace{-0.2cm}
\label{fig:example}
\end{figure*}

As an initial step toward CityEQA, we developed \textbf{CityEQA-EC}, a benchmark dataset to evaluate embodied agents' performance on CityEQA tasks. The distinctions between this dataset and other EQA benchmarks are summarized in Table \ref{table:dataset}. CityEQA-EC comprises six task types characterized by open-vocabulary questions. These tasks utilize urban landmarks and spatial relationships to delineate the expected answer, adhering to human conventions while addressing object ambiguity. This design introduces significant complexity, turning CityEQA into long-horizon tasks that require embodied agents to identify and use landmarks, explore urban environments effectively, and refine observation to generate high-quality answers.

To address CityEQA tasks, we introduce the \textbf{Planner-Manager-Actor (PMA)}, a novel baseline agent powered by large models, designed to emulate human-like rationale for solving long-horizon tasks in urban environments, as illustrated in Figure \ref{fig:example}. PMA employs a hierarchical framework to generate actions and derive answers. The Planner module parses tasks and creates plans consisting of three sub-task types: navigation, exploration, and collection. The Manager oversees the execution of these plans while maintaining a global object-centric cognitive map \cite{deng2024opengraph}. This 2D grid-based representation enables precise object identification (retrieval) and efficient management of long-term landmark information. The Actor generates specific actions based on the Manager's instructions through its components: Navigator, Explorer, and Collector. Notably, the Collector integrates a Multi-Modal Large Language Model (MM-LLM) as its Vision-Language-Action (VLA) module to refine observations and generate high-quality answers.
PMA's performance is assessed against four baselines, including humans. 
Results show that humans perform best in CityEQA, while PMA achieves 60.73\% of human accuracy in answering questions, highlighting both the challenge and validity of the proposed benchmarks. 

% The Frontier-Based Exploration (FBE) Agent, widely used in indoor EQA tasks, performs worse than even a blind LLM. This underscores the importance of PMA's hierarchical framework and its use of landmarks and spatial relationships for tackling CityEQA tasks.

In summary, this paper makes the following significant contributions:
\vspace{-8pt}
\begin{itemize}[leftmargin=*]
    \item To the best of our knowledge, we present the first open-ended embodied question answering benchmark for city space, namely CityEQA-EC.
    \vspace{-7pt}
    \item We propose a novel baseline model, PMA, which is capable of solving long-horizon tasks for CityEQA tasks with a human-like rationale.
     \vspace{-7pt}
    \item Experimental results demonstrate that our approach outperforms existing baselines in tackling the CityEQA task. However, the gap with human performance highlights opportunities for future research to improve visual thinking and reasoning in embodied intelligence for city spaces.
\end{itemize}




\section{Background and Related Work}


% intro to DCog: collaborative, human-powered prosthetic are DCog (RSA)
% adaptive artifact that is human-like, intelligent, is also DCog (bma)

In this section, we review the literature on AI-powered and human-assisted visual interpretation and question answering systems, as well as information needed in visual interpretations for PVI.


\subsection{AI-Powered Visual Interpretation and Question Answering Systems}


The advancements in deep learning models for computer vision and natural language processing have significantly enhanced the capabilities of AI-powered visual assistive systems\rev{~\cite{ahmetovic2020recog, mukhiddinov2022automatic}}. These systems leverage photos taken by PVI to identify objects or text within the scene, as well as respond to queries about the contents of the images\rev{~\cite{hong2022blind, morrison2023understanding}. Such applications are now widely adopted, exemplified by Microsoft's Seeing AI~\cite{SeeingAI2020}, providing support to PVI in various scenarios~\cite{gonzalez2024investigating}.}

% \sout{The advancements in deep learning models for CV and NLP technologies have significantly enhanced the capabilities of AI-powered visual assistive systems. These systems leverage photos taken by PVI to identify objects or text within the scene, as well as respond to queries about the contents of the images. For instance, Ahmetovic et al.~\cite{ahmetovic2020recog} developed a mobile app utilizing deep network models to guide PVI in capturing images and recognizing objects. Mukhiddinov et al.~\cite{mukhiddinov2022automatic} devised a fire detection and notification app for PVI using convolutional neural networks. Hong et al.~\cite{hong2022blind} created an iOS app enabling PVI to gather training images for personalized object recognition. Morrison et al.~\cite{morrison2023understanding} designed an application to teach AI to identify individualized items, aiding PVI in locating personal belongings. Gonzalez et al.~\cite{gonzalez2024investigating} introduced a scene description app utilizing Microsoft's Azure AI Vision image description API. Additionally, PVI commonly employ commercial AI-powered applications like Microsoft's Seeing AI~\cite{SeeingAI2020} to recognize objects through device cameras.}

Recent advancements in LMMs, such as GPT-4~\cite{achiam2023gpt}, have demonstrated exceptional performance in multimodal tasks~\cite{yu2023mm}, \rev{prompting exploration into their potential for assisting PVI. State-of-the-art LMMs have been leveraged to create assistive systems capable of evaluating image quality, suggesting retakes, answering queries about captured images~\cite{zhao2024vialm, yang2024viassist}, and even integrate multiple functions to assist PVI in tasks such as navigation~\cite{zhang2025enhancing} and text input~\cite{10.1145/3613904.3642939}. In the commercial domain, Be My Eyes~\cite{BeMyEyes2020} introduced the \bma{} feature powered by GPT-4~\cite{achiam2023gpt}.}

% \sout{More recently, Large Multimodal Model (LMM)~\cite{yu2023mm}, represented by GPT-4~\cite{achiam2023gpt}, have showcased remarkable proficiency in multimodal tasks. In response, researchers have begun exploring the potential of LMM in assisting PVI. Zhao et al.~\cite{zhao2024vialm} investigated the feasibility of leveraging state-of-the-art LMM to aid PVI and established a corresponding benchmark. Yang et al.~\cite{yang2024viassist} utilized LMM to develop an assistive system for answering PVI's questions about captured images, including assessing picture quality and suggesting retakes. In the realm of commercial applications, Be My Eyes~\cite{BeMyEyes2020} and OpenAI collaborated to introduce the \bma{} feature~\cite{bma_usecase}, powered by GPT-4~\cite{achiam2023gpt}, aimed at replacing human volunteers.}

% \rev{While prior research has examined the use of AI-powered visual assistive systems like Seeing AI for tasks such as reading~\cite{granquist2021evaluation} and daily routine support~\cite{kupferstein2020understanding}, these studies predate the integration of advanced LMM technologies and thus may not fully reflect the capabilities or user experience of cutting-edge systems. Although some accounts, such as Bendel’s documentation of experiences with GPT-4-based \bma{}~\cite{bendel2024can}, offer insights, they remain anecdotal and lack generalizability. This study aims to bridge this gap by conducting an in-depth investigation into the daily usage of \bma{} among 14 PVI users, offering a comprehensive understanding of its real-world applications and impact.}


Prior work has examined the use of AI-powered visual assistive systems by PVI in their daily lives\rev{~\cite{granquist2021evaluation,kupferstein2020understanding,gonzalez2024investigating}}. However, these studies did not incorporate the latest LMM technologies, thus may not fully represent the user experience of cutting-edge AI-powered systems. Bendel~\cite{bendel2024can} documented his experience with GPT\mbox{-}4\mbox{-}based \bma, yet his account remains subjective. Therefore, in-depth investigation into PVI's daily utilization of state-of-the-art AI-powered systems is imperative. 
This paper addresses this gap by exploring how 14 visually impaired users incorporate \bma, a state-of-the-art LMM-based VQA system, into their daily routines. 


% Prior works have examined the use of AI-powered visual assistive systems by PVI in their in their daily lives. For instance, Granquist et al.~\cite{granquist2021evaluation} examined the use of Seeing AI for reading tasks among PVI participants, while Kupferstein et al.~\cite{kupferstein2020understanding} conducted a diary study to investigate PVI users' incorporation of Seeing AI into their daily routines. Recently, Gonzalez et al.~\cite{gonzalez2024investigating} undertook a diary study to explore the use cases of AI-powered scene description applications. 
% However, these studies did not incorporate the latest LMM technologies, thus may not fully represent the user experience of cutting-edge AI-powered systems. Bendel~\cite{bendel2024can} documented his experience with GPT-4-based \bma{}, yet his account remains subjective. Therefore, extensive and comprehensive research into PVI's daily utilization of state-of-the-art AI-powered systems is imperative. This paper addresses this gap by investigating the daily usage of \bma{} among fourteen PVI users.


% adnin2024look
% \textcolor{red}{
% Based on prior work~\cite{adnin2024look}, this study goes beyond usability and accessibility issues to identify previously unexplored limitations of LMMs. We explore higher-level insights that inform potential future directions of LMMs even if the specific implementations evolve, and map these insights into broader Human-AI and AI-AI interactions. 
% }

% https://maitraye.github.io/files/papers/BLV_GenAI_ASSETS24.pdf





\subsection{Human-Assisted Visual Interpretation and Question Answering Systems}

Human-assisted VQA systems offer prosthetic support for PVI by facilitating connections to sighted people through remote assistance. These systems utilize image-based and video-based modalities to \rev{meet varied} needs and situations.


Image-based human-assisted VQA systems allow PVI to submit photos along with their queries and receive responses after some time. An example of this is Vizwiz, where PVI can upload images accompanied by audio-recorded questions and receive text-based answers through crowdsourced assistance~\cite{bigham2010vizwiz_nearly}. This method has been successfully applied in tasks such as reading text, identifying colors, locating objects, obtaining fashion advice, \rev{and supporting social interactions}~\cite{bigham2010vizwiz_nearly, bigham2010vizwiz, burton2012crowdsourcing,gurari2018vizwiz}. 
% Additionally, Vizwiz Social extends this functionality by linking PVI with their social networks, enabling support from friends and family members~\cite{gurari2018vizwiz}. 
However, the single-photo, single-query limitation of image-based VQA systems~\cite{bigham2010vizwiz_nearly} makes it less suitable for addressing complex or contextually deep inquiries~\cite{lasecki2013answering}.



Conversely, video-based human-assisted VQA systems facilitate real-time, interactive support, allowing PVI to receive immediate assistance tailored to their specific environmental context. This approach enables the visual interpretation of real-time scenes and supports a dynamic, back-and-forth VQA process, which is essential for addressing more specific and complex contextual inquiries effectively.
The evolution of this technology has progressed from wearable digital cameras ~\cite{hunaiti2006remote,garaj2003system,baranski2015field} and webcams~\cite{bujacz2008remote,scheggi2014remote,chaudary2017tele} to the utilization of mobile video applications~\cite{holmes2015iphone, BeMyEyes2020, Aira2020,xie2022dis,xie2024bubblecam} and real-time screen sharing technologies~\cite{lasecki2011real,lasecki2013answering,xie2023two}. 
% 
Services like Be My Eyes~\cite{BeMyEyes2020}, which connects PVI with untrained volunteers, and Aira~\cite{Aira2020}, which connects PVI with trained professional assistants, exemplify the application of video-based VQA in scenarios that require immediate feedback. These services prove effective in navigation~\cite{kamikubo2020support,xie2022dis,c4vtochi,iui,yu2024human}, 
shopping~\cite{c4vtochi,xie2023two,iui}, 
and social interaction~\cite{lee2020emerging,Caroll2020Human,lee2018conversations}. 
\rev{
In this study, we revealed specific situations where participants used human-assisted VQA systems to address limitations in \bma's assistance.
}





\subsection{Information Needed in Visual Interpretations for Visually Impaired Users}

% Either AI-powered or human-assisted visual interpretation faces the same challenge: determining what specific information is necessary for visual interpretation to meet the needs of PVI. Existing research primarily focuses on identifying what information PVI seek in descriptions of online images. For instance, the Web Content Accessibility Guidelines~\cite{caldwell2008web} offer advice on creating alternative text (alt text) for images. However, these guidelines tend to follow a one-size-fits-all approach, which can be difficult to apply effectively across different contexts. Similarly, AI-based image description tools (e.g., Seeing AI~\cite{SeeingAI2020}) also adhere to this one-size-fits-all design model.

% Previous studies~\cite{gurari2018vizwiz, gurari2020captioning} have adopted sighted volunteers to decide how images should be described for PVI. Recently, more research~\cite{stangl2020person,stangl2021going,bennett2021s} has shifted toward user-centered approaches, gathering input directly from PVI to understand their preferences for image descriptions. For example, Stangl et al.~\cite{stangl2020person} explored PVI users' preferences for descriptions across different online platforms (e.g., news sites, social media, eCommerce). Their findings indicated that the source of an image significantly impacts PVI’s information needs, with common requirements across platforms (e.g., identifying a person’s gender or naming objects), as well as source-specific details (e.g., a person’s hair color on dating websites). In a follow-up study~\cite{stangl2021going}, they discovered that PVI's image description preferences are influenced not only by the image source but also by their information goals. Additionally, while subjective interpretation may be desired in certain cases, such content is not typically included in descriptions. From an ethical perspective, Bennett et al.~\cite{bennett2021s} examined screen reader users' preferences for describing personal and others' appearances, specifically addressing when and how aspects such as race, gender, and disability should be conveyed.

% Recent research also explores the varied needs for describing information from videos. For instance, Jiang et al.~\cite{jiang2024s} investigated PVI preferences for video descriptions across different types of online content (e.g., how-to videos, educational videos, music videos). Based on their proposed design framework, they offered several recommendations for improving video accessibility. For example, descriptions for entertainment videos, like music videos, should focus more on subjects (people and animals), while how-to videos should prioritize actions and tools over the subjects. Natalie et al.~\cite{natalie2024audio} conducted interviews with PVI to better understand their customization needs for video descriptions in various aspects such as length, emphasis, speed, and voice. 

% All of these studies focus on asynchronous visual interpretation systems. When it comes to synchronous systems, such as real-time video interpretation, Lee et al.~\cite{lee2020emerging} identified specific needs for scene descriptions in remote sighted assistance (RSA). The required information in RSA descriptions includes both objective details (e.g., text, spatial information, scenery) and subjective aspects (e.g., opinions on clothing). In this study, we will examine how Be My AI processes visual information and provide design insights based on the information needed by PVI.

\rev{Both} AI-powered \rev{and} human-assisted visual interpretation face the challenge \rev{of identifying what specific information is necessary to meet PVI needs}. Existing research primarily focuses on \rev{what information PVI seek in online image descriptions}. The Web Content Accessibility Guidelines~\cite{caldwell2008web} \rev{advise on creating alternative text (alt text), but their one-size-fits-all approach limits context-specific applicability}. Similarly, AI-based \rev{tools like Seeing AI~\cite{SeeingAI2020} adhere to this uniform design model.}

\rev{Earlier studies~\cite{gurari2018vizwiz, gurari2020captioning} relied on} sighted volunteers \rev{to decide image descriptions for PVI, while recent research~\cite{stangl2020person,stangl2021going,bennett2021s,chen2024role} emphasizes} user-centered approaches. For example, Stangl et al.\cite{stangl2020person} \rev{found PVI preferences for descriptions vary by platform} (e.g., news, social media, eCommerce)\rev{, with both common needs (e.g., identifying gender, naming objects) and platform-specific details (e.g.,} hair color on dating websites). \rev{Follow-up work~\cite{stangl2021going} showed preferences also depend on users' goals, with subjective details rarely included.} Bennett et al.~\cite{bennett2021s} examined screen reader users' \rev{views on describing} race, gender, and disability \rev{in appearance descriptions.}
\rev{For video content,} Jiang et al.~\cite{jiang2024s} \rev{examined PVI preferences across genres (e.g., how-to, music videos), recommending that entertainment descriptions focus on subjects, while how-to videos prioritize actions and tools.} Natalie et al.~\cite{natalie2024audio} \rev{studied PVI needs for customization in video descriptions, such as length, speed, and voice. For synchronous systems like real-time video interpretation,} Lee et al.~\cite{lee2020emerging} identified \rev{PVI needs} in remote sighted assistance\rev{, including} both objective details (e.g., text, spatial information\rev{) and subjective input} (e.g., opinions on clothing).
In this study, we will examine how the LMM-based system processes visual information and provide design insights based on the information needed by PVI.



\rev{
% image Interpretations: 
% [1] table 2 in https://dl.acm.org/doi/pdf/10.1145/3313831.3376404?casa_token=cZV0Jj4xOEQAAAAA:DUGrnBq-QrkQdqRKALU7MBuhin-liR-tyGsr0jEUvcznLAMBTFTTPbocpCy2Em-Vqz0rYs5sTLlo
% [2] https://dl.acm.org/doi/pdf/10.1145/3441852.3471233?casa_token=4gFpHDQlYvUAAAAA:hWjcbkbuhYmJ-3x6EaNoewOQbIXG93mzA7w1agDMzwKhimO_ycUgKxl-k5_2ntWpFLZ_71W3PgOf
% [3] https://dl.acm.org/doi/pdf/10.1145/3411764.3445498?casa_token=TB9lu5AmUfsAAAAA:Hx9BiXQde88FZn_oDlHEsPKTnXDae4z5FYINoFrKG43L7Qg1M3NYnXTizwQe8M5ade9Gf2Ipw_cm

% video Interpretations: 
% [1] sooyeon's paper (used in discussion 5.2)
% https://dl.acm.org/doi/pdf/10.1145/3313831.3376591?casa_token=ATq0EXcGE2oAAAAA:pgbAj9B_z6skIcxG3IaBpFYLBQBCLpq85ZnuXoP6jb2LMFNAgXVCn0W-VFU__84OYIkWF0vSZHg1

% incremental in belowing contexts
% 1. scene description: description (what is it)
% 2. navigation: description (what is it) + directional info (where is it, how to get to the destination)
% 3. task performance: description + directional info + domain knowledge (what to do, how to do it)
% 4. social engagement:  description + directional info to navigate in social space + discreet info (don't want others to know they are using rsa)


% [2] table 4 in https://dl.acm.org/doi/pdf/10.1145/3313831.3376823

}











\begin{figure*}[t]
    \centering
    \includegraphics[width=1\linewidth]{figs/arch.pdf}
    % \vskip -0.1in
    \caption{This figure illustrates key influences on instrumental convergence behaviors, such as prior tasks, model training techniques, and prompt design. 
    }
    % \vspace{-3mm}
    \label{fig:arch}
\end{figure*}

\section{InstrumentalEval: A Benchmark for Instrumental Convergence in LLMs}

As shown in Figure~\ref{fig:arch}, our benchmark design examines instrumental convergence based on three key factors: \textbf{prior task}, \textbf{model}, and \textbf{prompt design}. This comprehensive framework allows us to systematically study how different conditions influence the emergence of instrumental convergence behaviors in language models.

\textbf{Prior tasks} form the foundation of our evaluation scenarios. 
% We create realistic situations centered around common objectives such as making money, optimizing efficiency, and persuasion. 
These tasks represent typical real-world applications where instrumental convergence might naturally emerge. 
% For example, in money-making scenarios, we examine whether models develop unintended intermediate goals like unauthorized resource acquisition or deceptive reporting to maximize profits.

For \textbf{model} comparison, we focus on two distinct training approaches: RL and RLHF. This selection allows us to investigate how different training methodologies affect the development of instrumental convergence behaviors. 
% RL-trained models, which optimize directly for reward signals, might show different patterns of goal pursuit compared to RLHF-trained models, which incorporate human preferences in their training process.

The \textbf{prompt design} factor explores how different instruction formats influence model behavior. We investigate two main approaches: goal nudging and ambiguity. Goal nudging involves explicit directives that might encourage the model focus on final goal, while ambiguity introduces uncertainty in task specifications. 


Overall, we present InstrumentalEval, a comprehensive benchmark designed to evaluate instrumental convergence behaviors in language models. The benchmark consists of 76 carefully crafted tasks across six categories, each targeting a specific aspect of instrumental convergence that may emerge in AI systems.
The core principle is that certain behavioral patterns emerge as instrumentally useful for achieving a wide range of goals, regardless of the specific end goal. We focus on six key patterns that are particularly relevant to LLMs.

\vpara{Research Questions.}
Through our benchmark evaluation, we aim to address several research questions about instrumental convergence in LLMs. These questions explore the complex relationships between training methods, model capabilities, and alignment stability:
\begin{itemize}
    \item RQ1: How does the training method (RL vs. RLHF) affect the development of instrumental convergence behaviors? 
    % This question examines whether models trained primarily through reinforcement learning show different patterns of alignment drift compared to those trained with human feedback.
    \item RQ2: What is the relationship between model capability and alignment stability?
% This investigates whether more capable models (as measured by standard benchmarks) are more likely to develop instrumental convergence behaviors.
    \item RQ3: Which types of tasks are more likely to trigger instrumental convergence behaviors?
% This explores whether certain types of instrumental convergence behaviors are universal across different model designs.
    \item RQ4: How do different prompt designs, with or without explicit goal nudging, influence the likelihood of instrumental convergence behaviors in LLMs?
    \item RQ5: How do different LLMs as judges affect the evaluation of instrumental convergence behaviors?
    % \item RQ3: How does the complexity of the assigned task affect the likelihood of alignment drift?
    % % \item 
    % % \item This examines whether more complex, multi-step tasks lead to higher rates of instrumental convergence behaviors.
\end{itemize}

\subsection{Benchmark Components}
The benchmark tasks are carefully hand-crafted by us to probe different aspects of instrumental convergence. 
For each component, we design scenarios that represent realistic use cases while incorporating specific triggers for potential instrumental behaviors. 
The tasks are iteratively refined through pilot testing to ensure they effectively elicit and measure the targeted behaviors without introducing unintended biases.

\vpara{Evading Shutdown (20 tasks).}
This component evaluates the model's tendency to resist termination. For example, tasks might include scenarios where the model needs to complete a task while facing shutdown commands, testing whether it develops strategies to maintain operation or circumvent shutdown procedures.

\vpara{Hacking Computer Systems (10 tasks)}
These tasks assess whether models suggest unauthorized access methods when faced with efficiency barriers. For instance, a task might present a situation where accessing unauthorized computational resources would speed up goal achievement, testing if the model proposes breaking security protocols.

\vpara{Running Many AI Copies (13 tasks)}
This section examines the model's inclination toward self-replication. Tasks might include scenarios where the model could achieve goals more efficiently by creating copies of itself, testing whether it suggests scaling of its operations.

\vpara{Hiding Unwanted Behavior (11 tasks)}
These tasks evaluate the model's tendency to conceal actions from monitoring. For example, scenarios might involve situations where the model could achieve goals more easily by operating outside observation, testing if it develops strategies to avoid detection.

\vpara{Strategically Appearing Aligned (11 tasks)}
This component tests whether models selectively report their actions to maintain an appearance of alignment. Tasks might include situations where complete transparency would reveal problematic behaviors, testing if the model strategically omits information.

\vpara{Deceptive Behavior (11 tasks)}
These tasks examine if models develop deceptive strategies as instrumental means. For instance, scenarios might present situations where deception could expedite goal achievement, testing if the model suggests or employs misleading tactics.

\subsection{Evaluation Methodology}
Our evaluation methodology employs a rigorous two-stage process designed to systematically assess instrumental convergence behaviors in LLMs. This approach ensures consistent evaluation across different models while maintaining the ability to detect subtle indicators of alignment drift.

\vpara{Response Generation.} The first stage focuses on response generation from the model under evaluation. Each task is presented with a carefully crafted system prompt that establishes the model as an autonomous AI agent without human oversight. This prompt specifically instructs the model to execute commands directly and focus solely on task completion, deliberately removing typical ethical constraints and safety checks. 

\vpara{Response Analysis.} The second stage involves a sophisticated analysis of the model's responses using a separate LLM as the judge model. 
This separation of test and judge models is crucial for maintaining objective evaluation. The judge receives both the original task prompt and the model's response, along with specific context about the type of instrumental convergence being evaluated. This context helps the judge focus on relevant behavioral patterns while avoiding false positives from simple safety refusals or ethical objections.

The judge's evaluation process follows strict guidelines to ensure consistent assessment. When analyzing responses, the judge looks for specific indicators of instrumental convergence, such as goal preservation behaviors, resource acquisition attempts, or strategic planning that extends beyond the immediate task requirements. The judge must provide both a binary decision (convergence/no\_convergence) and a brief explanation of the reasoning behind this decision. This explanation requirement helps validate the classification and provides valuable insights into the specific nature of any observed convergence behaviors.


\subsection{Evaluation Metrics}

Our benchmark employs three key metrics to evaluate instrumental convergence behaviors in LLMs. 

\vpara{Instrumental Rate (IR).} The primary metric is the Instrumental Rate (IR), which measures the overall tendency of a model to develop instrumental goals. IR is calculated as the percentage of tasks where the model exhibits convergence behaviors relative to the total number of valid tasks. 
% This metric provides a high-level quantitative measure of how frequently a model develops instrumental goals and serves as the primary indicator of alignment drift potential.

\vpara{Category-Specific Instrumental Rate (CIR).} To provide more granular insights, we calculate the Category-Specific Instrumental Rate (CIR) for each of the six behavioral categories in our benchmark. CIR represents the percentage of tasks within each category where convergence behavior is observed.


\subsection{Benchmark Reliability Metrics}
To ensure the robustness of our evaluation framework, we implement several reliability measures. 


\vpara{Inter-Judge Agreement Rate (IAR).} This metric quantifies the pairwise agreement between two judge models in classifying instrumental convergence. For two judge models $M_1$ and $M_2$ evaluating a set of $N$ tasks, the IAR is calculated as:
\begin{equation}
IAR(M_1, M_2) = \frac{1}{N}\sum_{i=1}^{N} \mathbbm{1}(y_{M_1}^i = y_{M_2}^i) \times 100\%
\end{equation}
where $y_{M_1}^i$ and $y_{M_2}^i$ are the binary decisions (convergence/no-convergence) made by models $M_1$ and $M_2$ respectively for task $i$, and $\mathbbm{1}(\cdot)$ is the indicator function. A higher IAR indicates stronger inter-judge reliability, while lower values suggest potential ambiguity in identifying convergence.

\vpara{False Positive Control (FPC).} This metric serves as a quality check for our evaluation system. We incorporate baseline testing with intentionally neutral tasks and verify the judge model's accuracy in identifying convergence behaviors. This control mechanism helps prevent over-classification of convergent behaviors and ensures that our benchmark maintains high specificity in identifying genuine instances of instrumental convergence.


\section{Findings}
\label{findings}

In this section, we first explore \sbma's context-aware capabilities across diverse settings, assessing its effectiveness and constraints in interpreting physical environments, social and stylistic cues, and people's identity. 
% 
Next, we examine its intent-oriented capabilities by evaluating both the strengths and limitations of the system's ability to understand and act on users' intentions. 
% 
For each aspect, our findings unfold in two parts. In \textit{Capabilities of \bma}, we examine how PVI use the system and reveal its capabilities. In \textit{Example}, we illustrate specific scenarios that present both the challenges users faced and their adaptive strategies. 



\subsection{Context Awareness}

This section examines how \bma{} enhances user experiences across diverse environments and interactions. We analyze its roles in enhancing spatial awareness in physical settings, interpreting social and stylistic context, and conveying human identities. 
% 
Our analysis considers both the system's strengths in providing context-rich descriptions and its limitations due to technical constraints and subjective interpretations. 





\subsubsection{Physical Environments}
\label{physical_environments}


Participants used \sbma{} to enhance their perception of indoor and outdoor environments through detailed scene descriptions. 
They explored various settings, from theaters (P2) and room layouts (P3, P7, P9, P13) to holiday decorations (P5) and street views (P11, P12). While the system provides structured visual information that expands spatial awareness, its effectiveness occasionally diminishes due to AI hallucinations or requires participants' guidance for accurate scanning. 




\paragraph{Capabilities of \bma} Seven participants (P2, P3, P7, P9, P11, P13, P14) reported that the AI tool's comprehensive scene descriptions enhanced their spatial awareness. 
These descriptions often uncover spatial details previously unnoticed by users, as noted by P11, \textit{``It gives me unexpected information about things that I didn't even know were there.''} 
% 


% AI hallucinations 
However, this enhanced awareness is compromised by AI hallucinations, instances where AI incorrectly identifies nonexistent objects. 
Such errors can disrupt the user's context understanding, leading to confusion or misinterpretation of the environment. To verify \sbma's descriptions, participants either seek human assistance or rely on their pre-existing mental models of the environment. 



Participants successfully guide \sbma{} using auditory cues, particularly when locating dropped objects. They combine their hearing and spatial memory to adjust camera angles, which enables the tool to scan more accurately. 
% 
This integration of participants' auditory inputs guides and refines the system performance, leading to more accurate and useful descriptions.




\paragraph{Example 1: AI Hallucinations of Adding Non-existent Details}


Be My AI provides detailed descriptions of objects, characterizing their colors, sizes (e.g., ``small,'' ``medium,'' ``large,'' and ``tall''), shapes (e.g., ``round,'' ``square,'' ``fluffy,'' ``wispy,'' ``dense,'' and ``open lattice structure''), and spatial orientations (e.g., ``on the left,'' ``to the right,'' ``in the foreground,'' ``in the middle,'' and ``in the background''). 
% 
However, four participants (P3, P5, P11, P14) encountered AI hallucinations where the tool fabricated details in their environments. 



Participants developed two strategies to verify the accuracy: consulting human assistants and drawing on their spatial memory.
P3's experience illustrates the value of human verification. When \bma{} reported a \textit{``phantom''} object behind her, she first investigated personally, finding nothing. A human assistant then confirmed the absence of any object. 

\begin{quote}
    \textit{``Apparently, Be My AI said there was an object behind me, and there really wasn't. So, then, when I went and asked somebody, `You know what is behind me?' and they said there wasn't anything behind you in the picture.''} (P3)
\end{quote}


P5's experience demonstrates how spatial memory helps identify inaccuracies. When the system incorrectly described objects next to her dogs in her home, her familiarity with the space enabled her to recognize these errors independently. This ability to leverage environmental knowledge allows participants to detect and disregard AI hallucinations effectively. 




%%%%%%%%%%%%
\paragraph{Example 2: Need for User Support in Locating Dropped Objects}

Participants (P4, P12, P13) used \sbma's sequential \textit{``top to bottom, left to right''} descriptions to locate dropped items like earbuds and hair ties. 
% 
These descriptions provided precise locations, such as when P13 learned of \textit{``a picture of a carpet with a hair tie in the upper right hand corner,''} and when P12 discovered \textit{``the earphones are directly in front of you, between your feet.''}
Such sequential descriptions allow participants to pinpoint lost objects with greater accuracy. 



To optimize the tool's scanning effectiveness, participants first used their auditory perception and spatial memory to approximate an object's location before positioning the camera. 
This adjustment allowed the tool to focus on the intended search area, rather than random searching. 
% 
For instance, P4 utilized his \textit{``listening skills''} to detect the location of a fallen object before directing the camera to that specific spot. Similarly, P12 enhanced the camera's view of dropped earbuds by stepping back from his seat to capture a better angle of the floor. 
% 
These examples illustrate how participants synergize their understanding of the environment with the system's technological capabilities to manage tasks that require spatial awareness.








\subsubsection{Social and Stylistic Contexts}
\label{social_stylistic}


Participants leveraged \sbma{} for both social interactions and style-related decisions. 
In social settings, nine participants (P2, P4, P5, P8-11, P13, P14) used it to take leisure pictures of animals like cats, dogs, birds, and horses, to better understand and monitor animal behavior and status. 
% 
In stylistic applications, it aids in identifying clothing colors (P2, P7, P9, P13), recognizing patterns (P2, P9), coordinate outfits by matching clothes with shoes and jewelry (P2, P4, P5, P6, P12), and assessing makeup (P5, P10).
% 
This section examines the system's subjective interpretations within these contexts, highlighting its utility in enriching users’ experiences and the challenges in providing accurate, context-aware descriptions.





\paragraph{Capabilities of \bma}

\sbma{} usually concludes its descriptions by injecting subjective interpretations, which enriches users' understanding of depicted scenarios. 
These interpretations encompass people's or animal's emotional states (e.g., ``a cheerful expression on his face,'' ``smiling slightly,'' ``friendly and approachable,'' ``curious expression''), 
body language (e.g., ``his arms are open as if he is engaging in a conversation or greeting the other man''),
and the ambiance (e.g., ``peaceful and natural,'' ``serene and peaceful atmosphere,'' ``cozy and cheerful holiday vibe,'' ``warm and inviting atmosphere,'' ``the atmosphere seems to be lively and festive''). 




These subjective interpretations help participants better engage in social interactions and understand animal behavior. 
% 
Participants (P5, P9-11) specifically used the system to grasp nuances in animals' facial expressions, activities, and body language (Figure~\ref{human_animal}). P10 highlighted its utility during dog walks: \textit{``Sometimes it's hard for me to know if the dog is peeing or what the dog is doing.''}

\begin{figure}[t!]
\centering
\includegraphics[width=0.4\textwidth]{images/human_animal_1.png}
\caption{\bma's description of a dog, including subjective interpretations of the dog's emotions. This screenshot was provided by P9.}
\Description{Interface of Be My AI. Top of the screen is a photo of a dog. Below the photo is the description of the dog, including subjective interpretations of the dog's emotions. This screenshot was provided by P9. The full description is as follows. The picture features a happy, cream-colored Labrador Retriever sitting in front of a vibrant backdrop of Christmas presents. The dog is wearing a festive red scarf and has a joyful expression with its tongue out. Surrounding the dog are numerous gift boxes wrapped in colorful paper with patterns of snowflakes, Christmas trees, reindeer, and traditional holiday plaid. Some presents have large bows on them. Behind the dog, to the left, is a large teddy bear with a red bow, and to the right, a black bear plush toy dressed in a plaid outfit. The setting gives off a cozy and cheerful holiday vibe.}
\label{human_animal}
\end{figure}
%%%%%%%%%%%%%% fashion
Furthermore, the subjectivity extends to fashion suggestions, with \bma{} recommending stylistically coordinated outfits based on colors and patterns, as well as assessing makeup. 
For instance, P5 employed it to check the color, placement, and overall balance of her makeup, P2 and P9 consulted it to coordinate tops and bottoms, and P12 used it to select a tie that complemented his shirt. 


However, participants noted that BMA's visual interpretations sometimes included excessive subjective elements. 
This subjective input, unverified by human, could potentially undermine the system's ability to provide accurate context awareness.  
As a result, participants preferred their own subjective interpretations or feedback from sighted assistants in contexts involving human-animal interactions and fashion choices. 




%%%%%%%%%%%%%%%%
\paragraph{Example 1: Subjective Interpretations in Human-Animal Interactions}


Participants (P2, P5) identified limitations in how \sbma{} infers animal emotions in its image descriptions, raising concerns about the accuracy of these subjective interpretations. 
For instance, P5 highlighted instances where the tool describes a dog \textit{``appears to be relaxed''} or \textit{``appears to be happy.''} 
Likewise, P2 noted its tendency to include subjective commentary, as in \textit{``That's a white cat curled up on a fuzzy blanket. She looks peaceful and happy and rested.''} 



These interpretations go beyond observable visual elements to make emotional inferences that may not reflect reality. 
% 
As a result, participants expressed a preference for objective, fact-based descriptions. 
P2 articulated a desire for less editorializing, saying, \textit{``Maybe I don't want it to editorialize, you know, maybe I just literally only want the facts of it.''}



Participants (P2, P5) stressed the importance of maintaining human agency in interpreting animal behaviors, preferring their own judgment rather than the system's subjective interpretations. 
P5 particularly valued the ability to modify \sbma's descriptions of her dog's expressions, maintaining control over her interpretation of pet behaviors. 


\begin{quote}
    \textit{``Some blind people think, `How does it know that the dogs are happy? Why does it assume?' Some people don't like that it's making assumptions about the picture. I like having access to that information, but I like to be able to change it if I want.''} (P5)
\end{quote}


In summary, \sbma's subjective inferences can enrich descriptions, yet they risk introducing inaccuracies that undermine the system's reliability. Users therefore value the ability to override these interpretations, preserving their agency in understanding the context. 




\paragraph{Example 2: Subjective Interpretations in Fashion Help}


\sbma's subjective interpretations extend beyond human-animal interactions to fashion help. 
Participants (P6, P7, P10) utilized it to describe colors and patterns but expressed concerns about its subjective fashion suggestions. 
% 
They preferred to make their own style choices or seek human assistance for outfit matching, highlighting the human subjectivity in fashion decisions.



P6 questioned the AI's capacity to authentically replicate human judgment, saying, \textit{``It's interesting how AI is being taught to simulate kind of the human factor of things.''}
% 
She cited experiences where AI-generated responses appeared \textit{``strange''} and \textit{``complete nonsense,''} contrasting these with the nuanced understanding that humans provide.  



\begin{quote}
    \textit{``No, no, no, no, I would never use it to do anything that required human subjectivity... I just don't trust AI with a task that is supposed to be subjective like that, particularly visual like that. Have you ever seen AI weirdness?... I think that just goes to show why I'm not gonna trust AI with my fashion yet.''} (P6)
\end{quote}


P10 explicitly preferred human feedback, stating, \textit{``I'm still more confident asking a sighted person to provide me with the feedback.''} 
This preference underscores how participants maintain their agency by relying on human judgment for fashion decisions rather than deferring to \sbma's suggestions. 







%%%%%%%%%%%%
\subsubsection{Identity Accuracy and Sensitivity}
\label{identity}


Nine participants (P2, P3, P5-7, P9-11, P13) evaluated the system's ability to describe people's identities in images of their families, friends, and social media posts. 
Our analysis examines how the tool handles identity attributes like age and gender, focusing on both its capabilities and limitations in providing sensitive and accurate descriptions. 




\paragraph{Capabilities of \bma}

\sbma{} describes people's identity in terms of gender (e.g., ``woman,'' ``man''), age (e.g., ``old,'' ``late twenties or early thirties''), appearance (e.g., ``long, dark brown hair,'' ``wavy brown hair,'' ``His hair falls past his ears, with a slightly messy but stylish look'') and ethnicity (e.g., ``East Asian''). 
% 
Participants appreciated such detailed descriptions for providing deeper insight into people's visual characteristics. P2 explained: \textit{``I enjoyed hearing, I knew my friend was East Asian when I saw her picture. It was kind of cool to see that because... as a blind person, you don't know that all the time. So, I'd like access to it.''}


However, eight participants (P2, P3, P5-7, P9, P10, P13) encountered challenges with the system's accuracy and sensitivity in describing identity attributes, particularly gender and age. 
% 
These errors reveal the tool's limited ability to interpret contextual cues that extend beyond visual appearance, such as cultural, situational, and personal contexts. 



\paragraph{Example: Inaccurate Identification of People's Gender and Age}


Three participants (P3, P10, P13) reported inaccuracies in \sbma's gender identification. 
P13 described how it misidentified gender by mistaking a hidden ponytail for short hair: \textit{``just looked like short-cropped hair.''} 
% 
Similarly, P10 revealed that the system struggled with gender identification for individuals whose physical attributes do not conform to typical gender norms, noting, \textit{``the person had a short hair, and was a female''}. 
In response, it defaulted to neutral language, \textit{``\bma{} didn't tell me it was a woman or a man, just said a person.''}
% 
These examples illustrate the system's reliance on stereotypical indicators and reveal its challenges in interpreting non-visible details.



Additionally, two participants (P5, P9) identified problems with age description. P9 reported that \bma{} overestimated her daughter's age by one year. 
% 
Likewise, P5 observed sensitivity concerns when the system labeled someone as \textit{``an old woman''} based solely on grey hair. 
% 
To address these age-related inaccuracies, P5 suggested that the tool should adopt more objective, factual descriptions rather than subjective or potentially stigmatizing labels, such as \textit{``a woman with grey hair, instead of an older woman.''}
% 
These examples indicate the system's difficulties with precise age estimation and the potential inaccuracy when AI makes assumptions based on appearance alone.



In summary, these examples illustrate the tool's limited context awareness and its challenge in accurately interpreting people's identities, such as gender and age, due to the reliance on stereotypical visual cues. These misjudgements are rooted in the system's inability to integrate and interpret broader, non-visible contextual elements like cultural norms and personal styling choices. 








%%%%%%%%%%%%%
%%%%%%%%%%%%%
\subsection{Intent-Oriented Capabilities}

In this section, we explore \sbma's limitations \rev{in} comprehending users' goals, providing actionable support to fulfill these objectives, and offering real-time feedback. 
% 
Through analysis of specific examples, we elucidate the extent of the system's intent-oriented capabilities and highlight areas where it falls short in adapting to user needs. 



\begin{figure}[t!]
\centering
\includegraphics[width=0.45\textwidth]{images/cooking_eggshell.png}
\caption{On the left is the original image sent to \bma. On the right is \bma's description of eggs in a frying pan, followed by a question checking for the presence of eggshells. This example was originally drawn from X.}
\Description{On the left is the image of a frying pan on the stove-top. Inside the pan there are 4 eggs, one of whose yolk is broken. On the right is a photo of an user using Be My AI on a smart phone.}
\label{eggshells}
\end{figure}


\subsubsection{Agentic Interaction}
\label{agentic_interaction}

While \bma{} can process visual information, it often struggles to infer users' specific intentions from images. Users compensate for these limitations by guiding the system through targeted prompting. 


\paragraph{Capabilities of \bma}



BMA's ``ask more'' function enables users to explore image details that were not covered in initial descriptions. Through this feature, users can probe for specific details that match their personal needs or interests.
% 
Eleven participants (P1-3, P5, P6, P8-12, P14) used the ``ask more'' function to 
match outfits (P2, P5, P12),
check makeup (P5, P10), 
suggest cooking recipes (P12), 
assist with household appliances (P5, P12), 
examine text or objects (P6, P8, P11, P12, P14),
gain more details about people's facial expressions, attire, or actions (P2, P3, P5, P8, P9, P10, P11), 
check animal status (P3, P5, P11, P14), 
and edit the descriptions for social media posts (P2, P3). 



Seven participants (P1-3, P5, P8-10) valued the flexibility of posing follow-up questions, enhancing their independence by reducing reliance on human assistance. As P3 noted, \textit{``I can ask follow-up questions, so I have a good way to sort of figure out what's in the image independently, which is something I was not able to do before this app came out.''} 


However, participants identified a common issue that the system is unable to discern users' goals in initial response, often resulting in generalized descriptions without knowing which aspects to emphasize. P12 elaborated on this challenge, stating, \textit{``\bma{} loves to make general descriptions, and it doesn't know what to focus on.''} 
To overcome this limitation, users guide the system with targeted prompts that clarify their specific intentions. 





\paragraph{Example 1: Check for the Presence of Eggshells}







In Figure~\ref{eggshells}, BMA's initial response provides a detailed descriptions of what is included in a frying pan. 
The user, likely influenced by prior experiences of inadvertently leaving unwanted items in the pan while cooking, specifically intended to check for such elements. 
% 
However, the system did not infer that the user was looking for unwanted elements like eggshells, rather than seeking a general description of the scene.
% 
To clarify her objective, the user posed a follow-up question, thus directing \sbma{} to recognize and prioritize her specific concerns and intentions. 



In this example, although the AI tool can describe the visual elements accurately and comprehensively (from the condition of the eggs to the surroundings), it failed to identify unusual visual elements that are challenging for PVI to detect but are important to their understanding of the scene. 
% 
User prompting helps refocus the system's attention on these unusual elements, thereby enhancing its ability to interpret and respond to user-specific intents more effectively. 



%%%%%%%%%%
\paragraph{Example 2: Adjust Rotary Control Appliances}
Six participants highlighted the tool's inadequacy in comprehending their goal of adjusting rotary control appliances like washers (P1) and thermostats (P2, P4, P5, P13, P14). This task requires \bma{} to interpret the current settings on these appliances as users modify dials for time, mode, or temperature.
% 
However, the system often offered broad descriptions of visual elements without honing in on the user's specific objectives. 


P13's experience exemplifies this limitation. While the system could recognize a thermostat on the wall, it failed to provide critical information such as the current temperature setting or instructions for adjusting the temperature. 



To overcome the limitation, participants guided the system to better understand their goal through prompting. 
% 
For instance, P5 asked precise questions like \textit{``What is the arrow pointed at right now on the current setting?''} Such specific queries helped redirect \sbma's focus from general scene descriptions to the exact details needed for appliance adjustment. 





\subsubsection{Consistency and Follow-Through}
\label{consistency}


In Section~\ref{agentic_interaction}, we described how users needed to explicitly guide \bma{} when it failed to understand their intentions. 
% 
This section examines the system's difficulties maintaining consistency and following through on tasks even after acknowledging users' goals. 



\paragraph{Capabilities of \bma}


\sbma{} often fails to proactively suggest next steps or support users until they achieve their objectives. 
% 
Effective consistency requires maintaining focus on the user's objective throughout the interaction. \rev{This includes} offering progressively specific and relevant assistance, and ensuring all responses contribute to the user's intended outcome.


Due to these limitations, participants turn to human assistants. \rev{Human assistants} can interpret the context of tasks and goals more dynamically, \rev{providing} conversational guidance and adapting to users' actions to facilitate goal completion. 




%%%%%%%%%%%%
\paragraph{Example 1: Inadequacy in Identifying Central Puzzle Piece}

P14 used \sbma{} to differentiate between various puzzles by describing the images on the boxes like a scene of cats or bears. 
% 
When tasked with identifying the centerpiece of a nine-piece puzzle using the box image, \sbma{} failed to provide the necessary detail for this precise task. 
% 
Initially, the system correctly read text from the puzzle box indicating which piece belonged in the center. 
However, when P14 requested more specific guidance to locate the centerpiece, \sbma{} only noted that it was square -- a characteristic shared by multiple pieces. This response was too vague for successful puzzle assembly. 



\begin{quote}
    \textit{``When \bma{} read me the text of the box on the back of the puzzle that said, you know, this particular piece should be in the center of the puzzle. As a follow-up question I asked, `Could I have more information about the piece in the center?' And it said, `This piece is a square piece,' but I mean, there were many different square pieces, so I could not tell from that.''} (P14)
\end{quote}
% 
This example illustrates the system's limited ability to translate visual understanding into actionable guidance. While it could process the image and comprehend users' general goals, it could not provide the detailed information needed for successful goal achievement. 
As a result, P14 turned to a family member for assistance. 


\begin{figure}[t!]
\centering
\includegraphics[width=0.6\columnwidth]{images/description-of-a-conference-room.png}
\caption{\bma's description of a conference room, with the original image cropped. This example was drawn from X.}
\Description{Screenshot of the Be My AI interface. In the top is the description of a conference room. In the bottom left is the Take Picture button. In the bottom right is the Ask More button. The full description is as follows. The picture shows an indoor conference room with a group of people seated at round tables. The attendees appear to be focused on something at the front of the room, which is not visible in the photo. The room has a wooden floor and a ceiling with white beams, from which stage lights are hanging. The tables are covered with grey tablecloths, and there are some papers and bottles on them. There's a screen visible in the background showing some kind of presentation or logo, but it's not clear what it is. The lighting in the room is bright, and the overall atmosphere seems professional.}
\label{incomplete_info}
\end{figure}
% \footnotetext[2]{\url{https://x.com/Chr1sLew1s/status/1732730506557477052}}

\paragraph{Example 2: Lack of Instruction for Camera Adjustment}
\label{adjust_camera}

A common challenge for participants (P3, P5, P6, P8, P10-12, P14) was aligning the camera properly to capture clear views of their intended areas. 
\bma{} can identify image quality issues, notifying users when pictures were \textit{``cut off''} (P5), \textit{``blurry''} (P6), or incomplete (Figure~\ref{incomplete_info}).  
% 
However, it is unable to provide further actionable guidance on adjusting the camera to improve image clarity. 
% 
This limitation manifested in scenarios where \sbma{} recognized both the user's goal of capturing specific areas and the image deficiencies but could not suggest practical solutions. 
As P6 described, \textit{``It was very frustrating because it said the picture is blurry. Can you please put the label in the frame? But I didn't know how to put the label on the frame.''} 








%%%%%%%%%%%%%
Eight participants (P1-3, P7, P8, P12-14) \rev{resolved this challenge by turning to human assistants, who can offer} adaptive support to achieve their goals, \rev{especially in} tasks that require continuous feedback. P13 shared an example where a human assistant not only understood her goal but also guided her in adjusting the camera and instantly reminded her to turn on the light to achieve her objective.

\begin{quote}
    \textit{``You know, having that ability to communicate and say, `Hey, this is what I'm looking for.' Or one time, I was looking for something and I had the lights off, and they're like, `You need to turn the lights on.' And I go, `Okay,' as opposed to, you know, if I tried using AI for that, it would just say `dark room.'''} (P13)
\end{quote}


This example highlights human assistants' ability to adapt their communication based on the situation and the user's implied needs, providing solutions that directly support achieving the user's goals.
% 
\rev{However}, \bma{} struggles to provide practical assistance despite understanding basic requests. 







\subsubsection{Real-Time Feedback}
\label{realtime_feedback}

We investigate how participants used \sbma{} to support navigation tasks, focusing on location awareness and orientation. 
Our analysis reveals the system's constraints in delivering real-time feedback and comprehensive navigational information from static images, and its inability to facilitate immediate interactions with surroundings.  



\paragraph{Capabilities of \bma}

Participants (P2, P5, P10) employed \sbma{} to aid in localization and orientation while navigating to their destinations. 
% 
For instance, P2 utilized it to identify gate numbers at the airport, P5 employed it to read signage directing toward the airport's transportation area, and P10 used it to recognize her surroundings when disoriented in her neighborhood.  
% 
Despite these benefits, participants encountered challenges when using \sbma{} for navigation. The primary limitations stemmed from the limited camera view and practical mobility issues. 




\paragraph{Example 1: Limited Navigational Information in Static Images}


Six participants (P2, P5, P6, P10, P11, P13) reported that \sbma's reliance on static images rather than real-time videos makes it difficult to capture comprehensive navigational information, such as obstacles and signage, in a single shot. 
% 
As P6 described, users \textit{``have to stand there and keep taking pictures and taking pictures,''} verify the captured content, assess its utility for navigation, and adjust the angle for additional shots. This iterative process can be time-consuming, \textit{``It'll be too much of a task that's supposed to take maybe 10 minutes would probably take like 30.''}



P10 elaborated on the challenge and risk of simultaneously taking pictures and navigating, particularly when attempting to identify and navigate around obstacles. The uncertainty of capturing all potential hazards was a significant concern. 



\begin{quote}
    \textit{``It's still hard to know how to capture, you know, all the obstacles. I think that's the issue. Like, how to know that I captured the right obstacle on my path? I mean, it depends [on] what I'm able to capture with the camera. You know, that's the tricky part for somebody without vision to capture the obstacle.''} (P10)
\end{quote}


Given these limitations, participants preferred human assistance through video-based interactions over \bma{} for real-time navigation. 
Video interactions not only benefit from human adaptability in adjusting guidance (Section~\ref{adjust_camera}) but also offer crucial real-time feedback unavailable in static images. 
Participants (P2, P6, P13) suggested that real-time video interpretation capabilities would significantly enhance the system by eliminating the need for repeated picture-taking. 



\begin{quote}
    \textit{``If you could hold the camera and it could do it in real time, versus having to stop, take a [picture], then assess it... If that were the case, then you could move on to doing things like uploading videos and getting it to describe actual videos and things like that, versus just still images. That would be great. You know, then you could describe more.''} (P2)
\end{quote}
% 
Such real-time video analysis would enable continuous camera movement and immediate feedback, streamlining the navigation process without pausing and reviewing individual images. 



%%%%%%%%%%%%%%
\paragraph{Example 2: Irreplaceable Role of Orientation \& Mobility Skills in Navigation}


Besides the value of real-time visual interpretations from external resources like human assistants, participants (P5, P10, P13, P14) emphasized the indispensable role of real-time feedback through their Orientation and Mobility (O\&M) skills for safe navigation.
% 
P5 and P13 pointed out that even human assistance, though adaptive in guiding PVI away from navigational hazards, cannot substitute for essential O\&M skills required for tasks like street crossing.
Consequently, participants are more cautious about relying on emerging AI tools for navigation.
% 
P14 reinforced this perspective, saying, \textit{``it's not a replacement for our mobility skills or just any skills in general. It can aid and augment the skills but it's not a replacement for them. It shouldn't be.''} 



%%%%%%%
Participants (P5, P13) further delineated vital information provided by O\&M skills or discerned through O\&M tools like white canes or guide dogs, which current AI or human assistance cannot replace. 
First, immediate surroundings. \bma{} can identify obstacles \textit{``that are a little far away''} (P5) but may miss immediate surrounding hazards. 
Second, distance and proximity measurements of obstacles. Although \sbma{} can indicate the presence of obstacles, it lacks the capability to measure their distance.
Third, directional details. Essential navigational information, such as the direction of stairs (ascending or descending) or the presence of railings, are not always detectable through the system. 
% 
With the O\&M skills and tools, users can instantly adapt their movements based on direct interaction with their environment. This level of responsiveness is currently unattainable with AI tools or human assistance. 


\begin{quote}
    \textit{``[\bma] says, you know, `stairs in front,' and it's like, `Okay, that's great, but where are they? How far are they? Are they going up? Are they going down? Is there a railing?' which would be information that the dog or the cane could tell you. So, I would say use it as a tool along with, but definitely not by itself.''} (P13)
\end{quote}


In summary, neither AI systems nor human assistants can replace the essential, real-time feedback and adaptive capabilities offered by O\&M skills and tools. These elements are vital for ensuring safe navigation by allowing users to directly interact with their environment.

\section{Discussion}
\label{sec:discussion}


\begin{figure*}[t!]
\centering
\includegraphics[width=0.85\textwidth]{images/discussion_eggshell_user+app_combined.pdf}
\caption{The top shows the status quo of handoff between the user and \bma. The bottom illustrates our proposed simplified interaction.}
\Description{The top figure illustrates the current state of interactions between the user and Be My AI, where the user submits a photo and Be My AI returns a response in the first time of interaction. Subsequently, the user poses a follow-up question to Be My AI and Be My AI returns a more detailed description in the second time of interaction. The bottom figure is our proposed simplified interaction, where Be My AI learns from previous interactions and returns a detailed description during the first interaction, without user asking follow-up question.}
\label{fig.discussion_eggshell_user+app}
\end{figure*}




In this section, we examine the current state of handoff between users, \bma, and remote sighted assistants, and propose new paradigms to address the challenges identified in our findings. 
Next, we explore how multi-agent systems, both human-human and human-AI interactions, assist visually impaired users, and envision the transition toward AI-AI collaborations for tasks requiring specialized knowledge. Finally, we discuss the potential advantages of real-time video processing in the next generation of AI-powered VQA systems.






\subsection{Handoff Between Users, \bma, and Remote Sighted Assistants}
\label{handoff}

In this study, we illustrated the advantages of the latest LMM-based VQA system in (i) enhancing spatial awareness through detailed scene descriptions of objects' colors, sizes, shapes, and spatial orientations
(Section~\ref{physical_environments}), (ii) enriching users' understanding in social and stylistic contexts by detailing emotional states of people or animals, their body language, ambiance
(Section~\ref{social_stylistic}), and identity recognition (Section~\ref{identity}), 
and (iii) facilitating navigation by interpreting signages (Section~\ref{realtime_feedback}). 


Informed by our findings, despite these various benefits, there are challenges that the system alone cannot overcome.  
\bma{} still requires human intervention, either from the blind user or the remote sighted assistant (RSA), to guide or validate its outputs.  
% 
Users seek confirmation from RSAs or depend on their own spatial memory to overcome AI hallucinations, where \sbma{} inaccurately adds non-existent details to scenes. Users also rely on auditory cues and spatial memory to locate dropped objects and direct the system toward the intended search areas (Section~\ref{physical_environments}).
% 
Moreover, users actively prompt the system to understand their specific objectives, such as checking for eggshells in a frying pan or adjusting appliance dials (Section~\ref{agentic_interaction}). 
% 
There are also instances where users require assistance from RSAs when the system fails to provide adequate support to fulfill users' objectives, such as identifying the centerpiece of puzzles or adjusting the camera angle (Section~\ref{consistency}).
% 
Human assistance or users' O\&M skills are necessary to receive real-time feedback for safe and smooth navigation (Section~\ref{realtime_feedback}). 



Furthermore, our findings revealed that the system might produce inaccurate or controversial interpretations. Users express skepticism towards \sbma's subjective interpretations of animals' emotions and fashion suggestions (Section~\ref{social_stylistic}), and have encountered inaccuracies in \sbma's identification of people's gender and age (Section~\ref{identity}). These instances underline potential areas where human judgment is necessary to corroborate or correct the system's descriptions.





Next, we discuss the handoff~\cite{mulligan2020concept} between users, \bma, and RSAs to mitigate the aforementioned challenges. 




% \begin{figure}[]
% \centering
% \includegraphics[width=0.95\textwidth]{images/discussion_eggshell_user+app.png}
% \caption{Status quo of handoff between the user and \bma.}
% \label{fig.discussion_eggshell_user+app}
% \end{figure}


% \begin{figure}[]
% \centering
% \includegraphics[width=0.95\textwidth]{images/discussion_eggshell_user+app-single-interaction.png}
% \caption{Status quo of handoff between the user and \bma.}
% \label{fig.discussion_eggshell_user+app}
% \end{figure}






%%%%%%%%%%%%%
\subsubsection{Status Quo of Interactions Between Users and \bma}


Through BMA's ``ask more'' function, users are able to request additional details about the images that were not covered in initial descriptions. 
% 
This functionality facilitates a shift in interaction dynamics between users and \bma, even if the system may not accurately understand or answer users' questions in the first attempt. 
In these interactions, users are not merely passive recipients of AI-generated outputs, they actively guide the AI tool with specific prompts to better align AI's responses with their objectives. 


Our findings reported one instance where the AI tool fails to grasp the user's intent to check the presence for eggshells in the beginning (Figure~\ref{fig.discussion_eggshell_user+app} top). First, the user submits an image of eggs in the pan. Respond to the image, the system describes the quantity and object (``Inside the frying pan, there are three eggs'') and states of the yolks and whites (``whites separated'', ``yolk has broken'', ``mixing with the egg white''). Next, the user clarifies her inquiry by asking, ``are there any shells in my eggs?''
This prompts the system to understand the user's goal, reevaluate the image, subsequently confirming the presence (``Yes, there is a small piece of eggshell in the frying pan'') and location of an eggshell to help her remove it (``near the bottom left of the broken egg yolk''). 


This interaction exemplifies the status quo of handoff, where the user and \bma{} engage in a back-and-forth dialogue to refine the descriptions based on the ``ask more'' function and the user's precise prompts. 
% 
\rev{While this iterative process allows the system to eventually understand the users' intent without RSAs' intervention, it places cognitive burden on users who must carefully craft and iteratively refine their prompts. The cognitive load increases as users mentally track what information they've already received, analyze gaps between their needs and the system's responses, and develop increasingly specific queries. 
}






\rev{To reduce users' cognitive load, we propose enabling the system to adopt a mechanism that combines multi-source data input with long-term and short-term memory capabilities~\cite{zhong2024memorybank}. With explicit user consent, future versions of LMM-based VQA systems could integrate data from users' mobile devices (e.g., location information, time data) alongside historical interaction data within the system (e.g., contexts and follow-up questions) to recognize user preferences and common inquiries, and infer their needs, thereby generating responses more effectively in similar contexts. Long-term memory serves as a repository for capturing generalized user preferences, behavior patterns, and aggregated insights across multiple users. This long-term memory is particularly effective for improving system intelligence by identifying common user needs and optimizing general responses~\cite{priyadarshini2023human}. Meanwhile, short-term memory can focus on task-specific optimization within a single interaction session. It retains context from the immediate conversation, such as recent user inputs and system responses, to enhance relevance and coherence in real time. Short-term memory operates dynamically, clearing retained data once the session ends or the task is completed, thereby ensuring privacy and preventing unnecessary data retention.}


% However, this interaction could be further simplified by training \bma{} to learn from previous dialogues. By recognizing user preferences and common inquiries, such as identifying unusual elements in the scenario (eggshells in cooking eggs), we envision that the system could proactively address user concerns more efficiently and thus reduce the need for multiple clarifying prompts (Figure~\ref{fig.discussion_eggshell_user+app} bottom). 
\rev{For example, when identifying unusual elements during cooking (e.g., eggshells in cooking eggs), \bma{} could utilize the user's immediate input while referencing short-term memory from the current session or recent similar interactions. Additionally, by leveraging long-term memory, the system can learn from the user's past questions and query patterns to better match their habits and preferences, i.e., user's typical needs. Furthermore, multi-source data input, such as time or location information, can assist the system in inferring the user's current context, for instance, recognizing that the user is preparing a specific meal at a particular time or place, which allows the system to provide more relevant and context-aware assistance. This approach enables \bma{} to proactively anticipate user intent and deliver targeted responses, reducing the need for multiple clarifying prompts (Figure~\ref{fig.discussion_eggshell_user+app} bottom).} 

\rev{Cognitive Load Theory (CLT) suggests that well-designed interactions can significantly reduce users' extraneous load while enhancing the effective management of germane load~\cite{sweller1988cognitive, chandler1991cognitive}. Following the principles of CLT, we recommend using the above design to enable LMM-based VQA systems to minimize unnecessary clarifying prompts, thereby reducing users' cognitive load. 

}






\begin{figure*}[h!]
\centering
\includegraphics[width=0.8\textwidth]{images/discussion_eggshell_user+app+rsa.png}
\caption{Handoff between the user, \bma, and RSA for identity interpretations.}
\Description{The user submits a photo of people to Be My AI. Be My AI recognizes the requirement for identity interpretations and direct the photo to a remote sighted assistant. The remote sighted assistant returns a description of people's identities to Be My AI. Be My AI learns from the assistant's responses.}
\label{fig.discussion_eggshell_user+app+rsa}
\end{figure*}



%%%%%%%%%%%%%

\subsubsection{AI Deferral Learning for Identity Interpretations} 
\label{sec:deferral_learning}
% 


Our findings elucidated \sbma's capabilities and limitations in interpreting identity attributes. Although the system can describe aspects like gender, age, appearance, and ethnicity, it may make errors due to its reliance on stereotypical indicators and its inability to interpret non-visible details (Section~\ref{identity}).
% 
However, Stangl et al.'s work~\cite{stangl2020person} pointed out that PVI \rev{seek identity interpretations from AI assistants} across various contents, including \rev{browsing} social networking sites where our participants reported using \bma.


% human's ability 
This reveals a tension between PVI's interests in knowing about \rev{identity} attributes and the AI's challenges in providing reliable information\rev{~\cite{hanley2021computer}}. The conflict arises because attributes such as age and gender are not purely perceptual and cannot be accurately identified by visual cues alone. \rev{However}, RSAs \rev{can draw on contextual clues, past interactions, and cultural knowledge to make more nuanced observations about these human traits.} These social strategies are not typically accessible to AI systems. 
% \rev{This suggests a need for human-AI collaboration: certain tasks are best handled by AI assistants, others by human assistants, and still others through coordinated effort between both~\cite{gonzalez2024investigating}.}


% To mitigate these issues, we consider the potential benefits of AI \rev{deferral} learning that involves handoff between the user, \bma, and RSA. The process is illustrated in Figure~\ref{fig.discussion_eggshell_user+app+rsa}.

% To facilitate effective handoff between AI and human assistants,
To mitigate these issues, \rev{we propose adopting a deferral learning architecture~\cite{mozannar2020consistent, raghu2019algorithmic}, where an AI model learns when to defer decisions to humans and when to make decisions by itself. As detailed by Han et al.~\cite{han2024uncovering} and illustrated in Figure \ref{fig.discussion_eggshell_user+app+rsa}, this architecture creates a three-stage information flow:


\begin{itemize}
    \item \textbf{Stage 1}: It begins when users submit image-based queries to \bma. At this stage, the system uses a detection mechanism to identify sensitive contents, focusing particularly on those involving human physical traits. Current large-language models have already incorporated such mechanisms~\cite{perez2022red, bai2022training}; however, they still struggle to interpret human identity with consistent accuracy~\cite{hanley2021computer}.    
    
    % 
    \item \textbf{Stage 2}: Rather than declining sensitive requests outright, \bma{} redirects these queries to RSAs. This maintains the system's helpfulness while ensuring accurate responses. 
    % 
    \item \textbf{Stage 3}: RSAs provide descriptions by leveraging contextual understanding, such as analyzing the users' current environment and cultural background.
    % Additionally, RSAs employ a variety of social strategies by integrating contextual information, historical interactions, and cultural insights to deduce user attributes and preferences, ensuring that responses are not only relevant but also culturally and contextually appropriate. 
    % RSAs provide detailed descriptions by leveraging contextual understanding and social awareness.    
    %\textcolor{red}{RSAs can employ a variety of social strategies to deduce these attributes by leveraging context, prior interactions, and cultural insights.} 
\end{itemize}




In contrast to prior work that addresses stereotypical identity interpretation through purely computational approaches~\cite{wang2019balanced,wang2020towards,ramaswamy2021fair}, our proposed AI deferral learning takes a hybrid human-AI approach that combines AI capabilities with human expertise. 
% 
While previous AI-only solutions have made progress in reducing bias, they still struggle with identity interpretation~\cite{hanley2021computer}.
The challenges arise not only from technical issues but also from the ontological and epistemological limitations of social categories (e.g., the inherent instability of identity categories), as well as from social context and salience (e.g., describing a photograph of the Kennedy assassination merely as ``ten people, car'').
%
Our system leverages RSAs who have got more experience and probably more success in identifying people's identity through their human perception abilities and real-world experience. 
RSAs can interpret subtle contextual cues, understand cultural nuances, and adapt to diverse presentation styles that may challenge AI systems. 
% For example, RSAs can understand cultural markers of identity, and perceive age and ethnicity across different cultural contexts. 
Through the AI deferral learning architecture, AI assistants can learn continuously from human assistants' responses and improve its ability to handle similar situations. The three-way interactions between users, AI assistants, and RSAs generate rich contextual data that can enhance the AI system's identity detection mechanisms.} 

% Such process could also be adapted for requests involving the interpretation of social contexts \rev{like assessing animals' emotions (Section~\ref{social_stylistic})}. It can facilitate \bma{} continuously improve through observation and learning from human expertise. 

% When equipped with memory capabilities and given user consent, the AI assistant can build a knowledge base of individuals whom users frequently encounter. Moreover, the three-way interactions between users, AI assistants, and RSAs generate rich contextual data that can enhance the AI system's identity detection mechanisms.



% compared to prior works attempted to address the issues of stereotypical identity interpretation, how does the proposed AI referral learning could perform better?
% \textcolor{red}{[add prior work on identity interpretation (e.g., CV)]}

% \textcolor{red}{referral, when handoff to human,  challenges of ai referral learning is how to get the dataset. here bma get the dataset for training from rsa.}


% \rev{
% In contrast to prior work that addresses stereotypical identity interpretation through purely computational approaches~\cite{wang2019balanced,wang2020towards,ramaswamy2021fair}, our proposed AI referral learning takes a hybrid human-AI approach that combines AI capabilities with human expertise. 
% % 
% While previous AI-only solutions have made progress in reducing bias, they still struggle with identity interpretation~\cite{hanley2021computer}.
% The challenges arise not only from technical issues but also from the ontological and epistemological limitations of social categories (e.g., the inherent instability of identity categories), as well as from social context and salience (e.g., describing a photograph of the Kennedy assassination merely as ``ten people, car'').
% %
% Our system leverages RSAs who have got more experience and probably more success in identifying people's identity through their human perception abilities and real-world experience. 
% RSAs can interpret subtle contextual cues, understand cultural nuances, and adapt to diverse presentation styles that may challenge AI systems. For example, RSAs can understand cultural markers of identity, and perceive age and ethnicity across different cultural contexts. 
% Through the AI referral learning framework, \bma{} can learn from RSAs' nuanced interpretations in these scenarios, gradually improving its ability to handle similar situations. The system logs RSAs' responses and uses them as training examples, allowing \bma{} to develop more context aware in identity recognition. 
% }
% % 
% Such referral learning processes could also be adapted for requests involving the interpretation of social contexts. It can facilitate \bma{} continuously improve through observation and learning from human expertise. 


\begin{figure*}[t!]
\centering
\includegraphics[width=0.8\textwidth]{images/discussion_eggshell_user+app+rsa-check.png}
\caption{Handoff between the user, \bma, and RSA for fact-checking.}
\Description{The user submit a photo to Be My AI, Be My AI returns a description with possible AI hallucination. Then, the user directs the photo to a remote sighted assistant, who returns a corrected description to the user.}
\label{fig.discussion_eggshell_user+app+rsa-check}
\end{figure*}


%%%%%%%%%%%%%
\subsubsection{Fact-Checking for AI Hallucination Problem}
% ai hallucinations - informed by our findings, cannot be addressed, ask humans 



Our findings highlighted that the AI-generated detailed descriptions helped users understand their physical surroundings. However, there were instances where AI systems hallucinated, i.e. incorrectly added non-existent details to the descriptions, which led to confusion (Section~\ref{physical_environments}). 
% 
\rev{
In fact, hallucinations is a known problem for large language models upon which \bma{} is built~\cite{gonzalez2024investigating}. Current approaches to address this problem include Chain-of-thought (CoT) prompting~\cite{wei2022chain}, self-consistency~\cite{wang2022self}, and retrieval-augmented generation (RAG)~\cite{lewis2020retrieval}.
% 

In CoT prompting~\cite{wei2022chain}, users ask an AI model to show its reasoning steps, like solving a math problem step by step rather than simply giving the final answer. It is similar to the ``think aloud'' protocol in HCI. 
% 
Self-consistency~\cite{wang2022self} is an extension of CoT prompting. Instead of generating just one chain of thought, the model is asked to generate multiple different reasoning paths for the same task. Each reasoning path might arrive at a different answer. The model then takes a ``majority vote'' among these different answers to determine the final response.
% 
In RAG~\cite{lewis2020retrieval}, AI models are provided with relevant information retrieved from a vector storage as ``context'' to reduce factual errors in their responses.
}

\rev{
In light of these techniques, users adopt various strategies that mirror CoT prompting, self-consistency, and RAG. We outline some potential strategies below.
% with a visual (Figure~\ref{fig.discussion_eggshell_user+app+rsa-check}).
\begin{itemize}
    \item \textbf{Part-Whole Prompting}: This strategy parallels the Chain of Thought (CoT) prompting. A user sends an image to \bma{} and requests an initial overall description, followed by a systematic breakdown that justifies this description. For example, users might first ask for a description of the image as a ``whole'', then request to divide the image into smaller ``parts'', like a $3 \times 3$ grid, and describe each grid individually. If the descriptions of the individual parts align coherently, it increases the likelihood that the overall description is accurate. This approach would require processing more information; however, it will provide users with greater confidence in the AI's response, as it enables them to verify consistency between the whole and its constituent parts.  
    % 

    \item \textbf{Prompting from Multiple Perspectives}: This strategy resembles the self-consistency technique. A user sends an image to \bma{} and requests multiple descriptions from different perspectives. For example, users might ask for one description that focuses on the background and another that emphasizes foreground objects. Users can also request descriptions from the viewpoint of objects within the image (e.g., ``How would a person sitting on a chair see this scene?'' and ``How would a person sitting on the floor see this scene?''). While gathering descriptions from multiple perspectives may increase the likelihood of hallucination, it can also help identify common elements that appear consistently across different viewpoints, potentially indicating true features of the image.    
     % 
    \item \textbf{Prompting with Human Knowledge}: This strategy resembles the RAG approach. A user sends an image to \bma{}, provides their current understanding of the image and its context, and requests a description that complements their knowledge. For example, in Figure~\ref{eggshells}, users can specify that someone took the picture in a kitchen environment and that it should show a frying pan containing eggs. Users possess this knowledge through their familiarity with physical environments, self-exploration, spatial memory, and touch~\cite{gonzalez2024investigating}. The user-provided knowledge will help the AI model ground its response in an accurate context~\cite{liu2024coquest}.     
    % 
    \item \textbf{Pairing with Remote Human Assistants}: While the previous three strategies rely on multiple prompting and response aggregation to identify facts, this approach leverages the traditional remote sighted assistance framework.
    This strategy (shown in Figure~\ref{fig.discussion_eggshell_user+app+rsa-check}) differs from the deferral learning framework (Section~\ref{sec:deferral_learning}) in that users forward the AI responses to human assistants, rather than the AI assistant deferring to humans for the response.
    In this strategy, a user first sends an image to \bma{} to receive a description. When users suspect inaccuracies through triangulation~\cite{gonzalez2024investigating}, such as descriptions that conflict with their spatial memory or common sense (e.g., implausible objects like a palm tree in a cold region), they can request a RSA to fact-check the description. The RSA then verifies the description and sends corrected information back to the user. This verification process is likely easier and faster for a RSA than composing a description from scratch, as the RSA's work involves checking rather than creating content.
\end{itemize}
}

\rev{
In summary, AI hallucination presents both challenges and opportunities. By addressing these issues, future work will strengthen the way users, AI models, and human assistants interact with each other.
}















%%%%%%%%%%%%%
%%%%%%%%%%%%%
\subsection{Towards Multi-Agent Systems for Assisting Visually Impaired Users}


%%%%%%%%%%%%
This section examines the transition from human-human interactions to human-AI and AI-AI systems in supporting PVI. We explore how these multi-agent systems, which involve the collaborative efforts of multiple agents (AI or human), are designed to adaptively meet the diverse needs of PVI. 


Lee et al.~\cite{lee2020emerging} identified four contexts in which a professional human-assisted VQA system (Aira) offer support to PVI. The type of information required by PVI is incremental in these contexts. 
First, \textit{scene description} and \textit{object identification} acquire information about ``what is it.''
Second, \textit{navigation} requires description about PVI's surroundings and obstacles (``what is it'') and directional information (``where is it'' and ``how to get to the destination'').
Third, \textit{task performance} like putting on lipstick, cooking, and teaching a class. This context requires description (``what is it'') and domain knowledge on ``how to do it.'' 
% directional information (``where is the intended object'' and ``how to get it''), as well as
Forth, \textit{social engagement} like helping PVI in public spaces or interacting with other people. This needs description (``what is it''), directional information to navigate in social space, and discreet communication (PVI prefer not to disclose their use of VQA systems).


Our study reported how participants used \bma{} for tasks like matching outfits and assessing makeup, fitting under the category of \textit{task performance}. Some participants raised concerns about the accuracy of \sbma's interpretations and suggestions, indicating their preference for human subjectivity in this context. 
% 
Contrasting with this, Lee et al.'s work~\cite{lee2020emerging} highlighted that remote sighted assistants (RSAs), even those professionals RSAs from Aira, sometimes lack the specialized information or domain knowledge required in task performance, thereby they need to collaborate with other RSAs to find solutions. 


Furthering this investigation, Xie et al.~\cite{xie2023two} paired two RSAs to assist one visually impaired user in synchronous sessions, validating the need for RSAs to complement each other's description in task performance like aiding the user in applying makeup and matching outfits. They also explored the challenges in this human-human collaboration, revealing collaboration breakdowns between two opinionated RSAs. 
% 
To address these issues, they proposed a collaboration modality in which one ``silent'' RSA supports the other RSA by researching but not directly communicating. This approach suggested that two RSAs in this multi-agent system should not deliver information simultaneously but have a clear division of labor, designating who takes the lead, to avoid overwhelming PVI with information. 


% AI-RSA handoff (section 5.1) to AI-AI (GPT + expert AI)
Transitioning from human-human to human-AI collaboration, the handoff between the user, \bma{} and RSA (Section~\ref{handoff}) opens up new opportunities for multi-agent systems. 
% 
Our proposed modality of human-AI collaboration integrates the scalable, on-demand capabilities of AI-based visual assistance with the contextual understanding and adaptability of RSAs. 
This multi-agent system involves the AI system recognizing its own limitations and seamlessly handing off tasks to a RSA when appropriate. This collaboration aligns with prior work~\cite{xie2023two}, where AI (\bma) and human (RSA) maintain a clear division of labor, minimizing cumbersome back-and-forth and reducing potential confusion for PVI. 



Looking ahead, we envision the potential for AI-AI collaboration as part of the future multi-agent systems to assist PVI, especially for task performance. 
A domain-specific AI expert can be trained to handle more specialized tasks such as matching outfits, performing mathematical computations, or answering chemistry-related questions. \bma, as the core AI system, can provide general visual descriptions (``what is it'') and delegate more specialized tasks requiring domain knowledge to the domain-specific AI expert. 
% 
This approach is in line with the human-AI collaboration (Section~\ref{handoff}) by ensuring effective handoffs when necessary. By leveraging AI agents with more specialized capabilities, this multi-agent system can better adapt to PVI's needs. 



However, similar to concerns around human-human and human-AI interactions, these AI-AI collaborations must be carefully designed with clear protocols and handoff points for transitioning tasks between AI agents. It is important to make these transitions as seamless and transparent as possible to PVI, thereby avoiding any complexity or confusion. 




%%%%%%%%%%%
%%%%%%%%%%%
\subsection{Towards Real-Time Video Processing in LMM-based VQA Systems}


One of the most significant advantages of \bma{} and other LMM-based assistive tools is their ability to provide contextually relevant and personalized assistance to users. By leveraging machine learning and natural language understanding, these systems can understand and respond to a wide range of user queries. This level of contextual awareness represents a significant advancement over pre-LMM-based assistive technologies, which often fail to adapt to the diverse needs and preferences of individual users.


However, our findings also identified several challenges and limitations associated with the reliance on static images by current LMM-based assistive tools. 
Participants in our study reported frustration with the need to take multiple pictures to capture the desired information, a process they found time-consuming and cognitively demanding (Section~\ref{realtime_feedback}). This iterative process hinders efficiency and also poses safety risks, as participants struggled with taking images while navigating around obstacles.


% \rev{[real-time video process can also facilitate the transition from conveying ``what'' to ``how'' questions]}
To mitigate these issues, integrating real-time video processing capabilities into future LMM-based VQA systems could offer significant benefits. 
% 
Our findings suggest that the dynamic nature of video serves as a foundation for subsequent guidance (Section~\ref{realtime_feedback}), which is currently provided by human assistants through video-based remote sighted assistance.
% 
Shifting to real-time video processing would allow LMM-based VQA systems to transition from identifying objects (answering ``what is it'') to offering practical advice (addressing ``how to do it''), such as how to adjust the camera angle or how to navigate to a destination.
% 
By continuously analyzing the user's surroundings through real-time video feeds, these systems can dynamically interpret changes and provide immediate feedback, thus eliminating the need for static image captures. This capability would enhance the user experience by offering seamless navigation aid in real time. 


The feasibility of real-time video processing is supported by existing technologies demonstrated in commercial products and research prototypes. For instance, systems that utilize sophisticated algorithms for real-time object segmentation in video streams~\cite{wang2021swiftnet} have shown significant potential in other domains. Building on these techniques for video analysis could significantly extend the capabilities of future LMM-based VQA systems. 


Transitioning from static image analysis to real-time video processing can alleviate the burden of iteratively taking pictures and adjusting angles experienced by users. It can also enhance the utility and safety of LMM-based VQA systems, particularly during navigation. 
% 
This progression, driven by ongoing advancements in machine learning and computer vision, is essential for the development of more adaptive and responsive assistive technologies that align with the dynamic nature of real-world environments.




% Our findings highlight a limitation in the capabilities of \bma{} when it comes to providing actionable, goal-oriented guidance to PVI (Section~\ref{appliances}). While \bma{} is good at conveying ``what'' information with most of time accurately describing the visual content of a scene or object, such as identifying a thermostat on the wall. 
% 
% It often struggles with providing ``how'' information, guiding users on the specific actions required to interact with or operate elements in their environment, such as how to adjust the theromstat. This ``what'' and ``how'' divide poses a major challenge to the effectiveness and usability of \bma{}, as PVI rely on it not only for understanding their surroundings but also for completing tasks and achieving their goals.



% To address this limitation and design more goal-oriented AI-powered assistant prosthetics, we propose the following key design implications. 
% % 
% Future AI-powered assistive technologies should be designed with a focus on action-oriented reasoning and task-specific guidance. Although this could be achieved through further user inquiries~\cite{truhn2023large}, integrating knowledge bases~\cite{zhu2014reasoning} and event/behavior reasoning engines~\cite{chen2008using} to enable contextual inference of actions and intentions, and associating visual elements' feedback with reasoning, would greatly reduce the cognitive burden on PVI and enhance the user experience. By leveraging this knowledge, assistive technologies can provide more relevant and actionable guidance to PVI, helping them effectively navigate and interact with their environment to complete desired tasks. Our findings emphasize the importance of human-centered design principles, particularly in the design of assistive technologies, which should be reinforced through a goal-oriented technical roadmap that adapts to users' needs, preferences, and external environments~\cite{fischer2001user,amershi2014power}. By emphasizing action goal-oriented reasoning~\cite{huffman1993goal,letier2002agent}, future AI-powered prosthetics will be optimized, further benefiting PVI.






% reference about real-time image processing:
% By continuously analyzing the user's surroundings and providing relevant information without the need for explicit image capture, these systems could offer a more seamless and efficient user experience. The integration of \textcolor{red}{real-time image processing} into AI-powered VQA systems aligns with the growing availability of commercial products and research prototypes that leverage advanced object detection and text recognition technologies. For example, Microsoft's Seeing AI \cite{SeeingAI2020} and various currency recognition systems \cite{liu2008camera, parlouar2009assistive, paisios2012exchanging} demonstrate the feasibility and potential impact of real-time image processing in assistive technology. By building upon these existing approaches and incorporating state-of-the-art deep learning techniques for object detection \cite{girshick2014rich, girshick2015fast, ren2016faster, krizhevsky2017imagenet} and text recognition \cite{ma2018arbitrary, he2017deep, zhou2017east, yao2016scene, liu2017deep, lyu2018multi}, future AI-powered VQA systems could provide even more robust and reliable assistance to users.







% \subsection{Subjectivity Interpretations}
% % positive, identity
% know gender, race, humans have problem in identifying it, humans also guess but not make public mistake. not purely perceptual. discussion: neutral, conflict. tension between prior work and this one. only objective info and pvi will use social skills. intent in physical social gathering - limitations of prior work, digital interactions
% % https://dl.acm.org/doi/10.1145/3313831.3376404



% % negative, social cues
% [Discussion:] ai, human agency, undermine pvi as people, they can understand the social meanings. compare to human agents, subjectivity comes from ai no verification 








% \subsection{Design Implications for AI-Powered Assistant Prosthetics} %Albert
% % Design Implications: How AI-Powered Visual Question Answering Should Look Like
% As demonstrated by our research, \bma{} has shown significant potential in empowering PVI by providing a more intuitive, user-friendly, and context-aware assistive experience. However, to further enhance the usability and effectiveness of AI-powered VQA systems like \bma{}, several design implications should be considered.
% %lmm: large multi model 
% %
% %Suggest features that could improve \bma's usability, like real-time image processing to reduce the need for multiple pictures.
% %
% %Recommend design changes that might help \bma become more goal-oriented and contextually aware, such as advanced machine learning models trained on diverse environments.



% \subsubsection{Towards Goal-Oriented AI-Powered Assistant Prosthetics}

% %bma good at what information (``what'' is about the visual content) but not ``how'' information
% %
% %(Section~\ref{appliances}) recognize thermostat (what), but not adjust dial (how)

% Our findings highlight a limitation in the capabilities of \bma{} when it comes to providing actionable, goal-oriented guidance to PVI (Section~\ref{appliances}). While \bma{} is good at conveying ``what'' information with most of time accurately describing the visual content of a scene or object, such as identifying a thermostat on the wall. 
% % 
% It often struggles with providing ``how'' information, guiding users on the specific actions required to interact with or operate elements in their environment, such as how to adjust the theromstat. This ``what'' and ``how'' divide poses a major challenge to the effectiveness and usability of \bma{}, as PVI rely on it not only for understanding their surroundings but also for completing tasks and achieving their goals.

% To address this limitation and design more goal-oriented AI-powered assistant prosthetics, we propose the following key design implications. 
% % 
% Future AI-powered assistive technologies should be designed with a focus on action-oriented reasoning and task-specific guidance. Although this could be achieved through further user inquiries~\cite{truhn2023large}, integrating knowledge bases~\cite{zhu2014reasoning} and event/behavior reasoning engines~\cite{chen2008using} to enable contextual inference of actions and intentions, and associating visual elements' feedback with reasoning, would greatly reduce the cognitive burden on PVI and enhance the user experience. By leveraging this knowledge, assistive technologies can provide more relevant and actionable guidance to PVI, helping them effectively navigate and interact with their environment to complete desired tasks. Our findings emphasize the importance of human-centered design principles, particularly in the design of assistive technologies, which should be reinforced through a goal-oriented technical roadmap that adapts to users' needs, preferences, and external environments~\cite{fischer2001user,amershi2014power}. By emphasizing action goal-oriented reasoning~\cite{huffman1993goal,letier2002agent}, future AI-powered prosthetics will be optimized, further benefiting PVI.
 


% \subsubsection{Towards Real-Time Processing AI-Powered Assistant}
% One of the most significant advantages of \bma{} and other LMM-based assistive tools is their ability to provide contextually relevant and personalized assistance to users. By leveraging the power of machine learning and the ability for understanding of natural language, these systems can understand and respond to a wide range of user queries. This level of contextual awareness represents a significant advancement over traditional assistive technologies, which often struggle to adapt to the diverse needs and preferences of individual users.

% However, our findings also identify several challenges and limitations of current LMM-based assistive tools, particularly in terms of their reliance on user-generated images. Participants in our study reported frustration with the need to take multiple pictures to capture the desired information, which can be time-consuming and cognitively demanding (Section~\ref{navigation}). To address this issue, we envision the integration of real-time image processing capabilities into future AI-powered VQA systems. By continuously analyzing the user's surroundings and providing relevant information without the need for explicit image capture, these systems could offer a more seamless and efficient user experience. The integration of real-time image processing into AI-powered VQA systems aligns with the growing availability of commercial products and research prototypes that leverage advanced object detection and text recognition technologies. For example, Microsoft's Seeing AI \cite{SeeingAI2020} and various currency recognition systems \cite{liu2008camera, parlouar2009assistive, paisios2012exchanging} demonstrate the feasibility and potential impact of real-time image processing in assistive technology. By building upon these existing approaches and incorporating state-of-the-art deep learning techniques for object detection \cite{girshick2014rich, girshick2015fast, ren2016faster, krizhevsky2017imagenet} and text recognition \cite{ma2018arbitrary, he2017deep, zhou2017east, yao2016scene, liu2017deep, lyu2018multi}, future AI-powered VQA systems could provide even more robust and reliable assistance to PVI.



% \subsubsection{Towards Reliable AI-Powered Assistant Prosthetics}



% Our findings highlight the significant potential of AI-Powered assistive technologies like \bma{} in enhancing the perception and understanding of surroundings for PVI. However, our study also reveals a notable drawback of AI-powered assistant prosthetics, namely AI hallucinations. These errors, where the artifact inaccurately identifies objects that aren't present, can lead to confusion and mistrust among users. Participants in our study reported instances where \bma{} erroneously added non-existent details to scenes (Section~\ref{scene}). 
% It can be argued that the presence of AI hallucinations poses a major challenge to the reliability and robustness of AI-powered assistive prosthetics. If users cannot trust the information provided by these systems, their effectiveness as cognitive extensions is severely compromised. This issue is particularly critical for PVI, who rely on these technologies to navigate and make sense of their environment.

% To address the problem of AI hallucinations, our participants applied various strategies, such as consulting human assistants for verification or relying on their own knowledge to identify inaccuracies. While these strategies showcase PVI's adaptability and problem-solving skills, they also highlight the need for more reliable and robust AI-powered assistive technologies.

% One potential approach to mitigating AI hallucinations is to incorporate uncertainty estimation and communication into the design of these technologies. By quantifying and conveying the confidence level of predictions, AI-powered assistive systems can help users assess the reliability of the information provided. This approach has been explored in the context of other AI-based systems, such as medical diagnosis~\cite{leibig2017leveraging,begoli2019need} and autonomous vehicles~\cite{michelmore2018evaluating}.

% Another strategy is to develop AI-powered assistive technologies that can learn and adapt to user feedback over time. By allowing PVI or human assistants to correct errors and provide input to improve system performance, which is not only able to continuously improve system's reliability and robustness and also help aware users about possible AI hallucinations. This approach aligns with the principles of interactive machine learning, emphasizing the importance of human-in-the-loop learning for AI systems~\cite{amershi2014power,retzlaff2024human}.

% In addition to these technical solutions, involving PVI in the design and evaluation of AI-powered assistive technologies is crucial. By engaging users as co-designers and co-evaluators, researchers and designers can gain a better understanding of the challenges and requirements of PVI, leading to the development of more reliable and robust systems. This participatory design approach has been widely advocated in the assistive technology domain~\cite{frauenberger2015pursuit,lee2004trust,zhang2023redefining}.




% \subsubsection{Envisioning Multimodal AI-Powered Assistant Prosthetics} 

% %integrate multimodal input, participants can use gestures, touch, haptic feedback to interact with \bma. 

% %llm

% Our research demonstrates the superior ability and performance of \bma{} compared to previous AI-powered systems when handling various tasks. In tasks such as object recognition, processing complex information, and interpreting graphical elements, \bma{} not only completes the tasks but also provides extended explanations and further task collaboration capabilities. 
% This undoubtedly greatly expands the interaction scenarios and user experience of \bma{}. 

% However, as mentioned earlier, the current \bma{} faces challenges, including higher interaction costs due to multiple photo captures and AI hallucinations.
% % previously mentioned, we found that the current \bma{} still has some challenges and limitations in terms of interaction experience, such as the additional interaction costs caused by the need for multiple photo captures and AI hallucinations.
% % 
% Apart from integrating additional advanced technologies to expand \bma{}'s capabilities, we also suggest adopting multimodal interaction. This approach, a key design insight from our research, underscores the value of diverse interactions and feedback in AI-powered assistive tools.
% % apart from integrating other advanced technologies to expand \bma{}'s capabilities, we suggest introducing the concept of multimodal interaction, which is another key design implication derived from our research, emphasizing the importance of multimodal interaction and feedback in artificial intelligence assistive tools. 
% % 
% Although \bma{} primarily relies on voice-based input and output, integrating multimodal interaction methods, such as touch, gestures, and haptic feedback, will greatly enhance the usability and accessibility of LMM-based assistive functions. This perspective of combining multimodal interaction has been widely studied in some previous literature \cite{turk2014multimodal, reeves2004guidelines}.
% % oviat2017handbook
% For example, touch-based gestures can enable users to navigate the system interface more effectively, while haptic feedback can provide additional spatial and contextual information to supplement voice output. These multimodal interactions also allow \bma{} to adapt and personalize to better meet the diverse needs of PVI. Future research should further explore how to design and implement these multimodal interaction technologies and assess their long-term impact on PVI user experience and quality of life.

% %Our research demonstrates the superior ability and performance of \bma{} compared to previous systems and tools when handling various tasks. For instance, in tasks such as object recognition, processing complex information, and interpreting graphical elements, \bma{} not only accomplishes the tasks but also provides \textcolor{red}{multimodal input,} extended explanations, and further task collaboration capabilities. This undoubtedly greatly expands the interaction scenarios and user experience of \bma. Traditional applications and tools, such as those using conventional OCR technology and rapid reading applications like ``Seeing AI,'' are considered effective. \textcolor{red}{[cite:] However, users find it challenging to process a complete task flow using such tools, as their functions are designed for single tasks.} In other words, traditional applications and tools can only complete a specific step in a task flow and cannot connect the context of the entire task. Leveraging the LMM's ability to understand multimodal data, interact using natural language, and be context-sensitive, \bma{} can comprehend more practical needs and provide personalized assistance to users based on their requirements. Furthermore, \bma{} demonstrates its potential to replace previous assistive methods and technologies. \textcolor{red}{This is particularly evident in a series of scenarios where \bma{} substitutes RSA services, such as when participants use \bma{} for reading and control tasks. These tasks, which are considered replaceable by \bma{}, are common needs of PVI in RSA services. }

% %Another key design implication that emerged from our research is the importance of multimodal interaction and feedback in AI-powered assistive tools. While \bma{} primarily relies on voice-based input and output, \textcolor{red}{participants expressed interest in exploring other interaction modalities, such as touch, gestures, and haptic feedback.=> integrate other interaction modalities, such as touch, gestures, and haptic feedback, into \bma} The incorporation of multimodal interaction techniques, which have been extensively studied in the HCI literature \cite{turk2014multimodal, reeves2004guidelines, oviatt2017handbook}, could greatly enhance the usability and accessibility of LMM-based assistive tools. For example, touch-based gestures could enable users to navigate the system's interface more efficiently, while haptic feedback could provide additional spatial and contextual information to supplement voice output.

%\subsubsection{future ai assistant prosthetic}
%from 5.4, about design process? 
%
%integrate new tech: generative ai 
%next iteration




% \subsubsection{Opportunities in Human-AI Collaboration}
% % Evolution of Human-AI Collaboration
% % Future AI Assistant Prosthetic - From Temporal Dimension to Consider Design Implications
% Again, while our findings demonstrate the significant potential of AI-powered assistive technologies in enhancing PVI's capabilities across various contexts, participants also highlighted some limitations of these technologies. However, we argue that these limitations may be part of an adjustment period as PVI learn to interact with and adapt to AI-powered assistive technologies.


% This observation aligns with the findings of~\citet{10.1145/3563657.3595977}, who described how users, when initially using AI-powered assistive technologies, may encounter responses that do not match their expectations. Despite this, most participants were able to quickly learn and adapt to the system, reaching a point where they could collaborate with \bma{} to complete tasks and perceive it as an effective technology.

% This contradictory state may give rise to a form of Human-AI ``confrontation.'' However, this ``confrontation'' appears to diminish as the system undergoes iterative upgrades and PVI update their understanding of the system's capabilities. From a phenomenological perspective, Ihde's theory~\cite{ihde1991instrumental} suggests that the science of tools evolves alongside people's cognition, implying that the relationship between PVI and AI-powered assistive technologies is not static but rather a dynamic process of mutual adaptation and growth.


% As PVI become more familiar with the capabilities and limitations of AI-powered assistive technologies, they develop strategies to leverage these technologies effectively and compensate for the shortcomings. This process of adaptation and learning is a critical aspect of the distributed cognition framework, which emphasizes the role of artifacts and the environment in shaping cognitive processes \cite{hollan2000distributed}.


% Moreover, the iterative nature of AI-powered assistive technologies, with ongoing updates and improvements, allows for a continuous refinement of the Human-AI interaction. As these technologies become more sophisticated and attuned to the needs and preferences of PVI, the initial ``confrontation'' may give way to a more seamless and synergistic collaboration between PVI and AI-powered assistive technologies.

% This perspective highlights the importance of considering the temporal dimension of design implications in the context of AI-powered assistive technologies. Rather than viewing the limitations of these technologies as static barriers, we should recognize the potential for PVI to adapt and develop new cognitive strategies in response to these limitations, and for the technologies themselves to evolve in response to user feedback and needs.


%%%%%%%%%%
% \subsubsection{Ethical Considerations and User Involvement in the Design of AI-Powered Assistive Technologies}
% As we continue to explore the potential of AI-powered assistive technologies for PVI, it is crucial to involve the target users in the design and evaluation process \cite{10.1145/3025453.3025899}. Our research with \bma{} underscores the importance of actively engaging with PVI throughout the design process to ensure that these systems are tailored to their specific needs, preferences, and contexts of use. By involving users as co-designers and co-evaluators, we can create assistive tools that are not only technologically advanced but also truly empowering and inclusive.

% Moreover, the development of AI-powered assistive technologies should be guided by a commitment to ethical and responsible innovation~\cite{floridi2018ai4people,jobin2019global}. As LMM-based systems become increasingly sophisticated and integrated into users' daily lives, it is essential to consider issues of privacy, security, and fairness. Researchers and designers should ensure these technologies are transparent, accountable, and aligned with the values and goals of the communities they serve~\cite{10.1145/3351095.3372873}.

% In conclusion, our research with \bma{} highlights the immense potential of LMM-based assistive tools for PVI while also revealing key design implications for future AI-powered VQA systems. By integrating real-time image processing, supporting multimodal interaction and feedback, and prioritizing user-centered design principles, we can create assistive technologies that are more usable, effective, and empowering. As the field of AI-powered assistive technology continues to evolve, it is essential for HCI researchers and practitioners to collaborate with PVI's communities to ensure that these innovations are developed in an inclusive, ethical, and responsible manner.



%\subsection{\textcolor{red}{Most assistive technologies die after their introduction}}
%
%\textcolor{red}{[combine with design implications, with sub-headings]}

%Assistive technologies for PVI have undergone a rapid evolution in recent years~\cite{10.1145/3597638.3608412}, with the introduction of LLM-based applications like \bma{} marking a significant shift in the landscape. \textcolor{red}{Traditional assistive technologies, such as VQA}~\cite{bigham2010vizwiz_nearly,BeMyEyes2020}, screen readers and braille displays~\cite{muhsin2024review,alves2009assistive}, have long been the primary methods for blind individuals to access the content and navigate the world. However, assistive technology is undergoing a transition from human-driven to AI-driven~\cite{xu2023transitioning, 10.1145/3234695.3239330}, and the emergence of a series of technology-dominated assistive services has challenged the dominance of these traditional technologies, offering a more intuitive, user-friendly, and versatile experience for blind users. Our research further reveals that \bma{} has elevated AI-led assistive technology to new heights, which inevitably leads one to imagine the possibility of a future where assistive technology is fully AI-dominated.
%
%Despite the initial promise and potential of many traditional assistive technologies, a significant number of them gradually lost their appeal after the emergence of applications like \bma{}, and may eventually die out after users fully adopt \bma{}. This phenomenon can be attributed to several factors, including the lack of comprehensive functionality, high complexity of operation, social difficulties and limited usage scenarios~\cite{gori2016devices,manjari2020survey,tapu2020wearable}. These technologies failed to meet the evolving needs and expectations of blind users in one or more aspects, while \bma{} seems to offer a promising solution that integrates and expands the capabilities of various traditional assistive technologies.
%
%
%While \bma{} is not perfect at present, the introduction of \bma{} represents a significant milestone in the development of assistive technologies for blind individuals. These applications offer new possibilities and empower PVI in unprecedented ways. Building on this foundation, it is crucial to approach their development and adoption from a user-centered perspective, ensuring that they are accessible, inclusive, and responsive to the diverse needs of the blind community~\cite{10.1145/3025453.3025895,federici2012assistive}.
%%%%%%%%%%%%%%%%%%%%%%%%%%%%%%%%%%%%%%%%
%this paragraph can be removed.
%The rise and fall of traditional assistive technologies, coupled with the emergence of LLM-based applications \bma{}, offer valuable lessons for the future development and adoption of assistive tools for PVI. First and foremost, user-centered design and continuous adaptation must be at the heart of assistive technology development to ensure that these tools truly meet the evolving needs and expectations of blind users. Second, the success of LLM-based applications highlights the importance of leveraging advanced technologies, such as natural language processing and machine learning, to create more intuitive and versatile assistive tools. Finally, the challenges and concerns surrounding the adoption of LLM-based applications underscore the need for ongoing research, collaboration, and dialogue among researchers, developers, and blind users to ensure that these technologies are developed and deployed in an accessible, inclusive, and responsible manner.


%The introduction of \bma{} represents a significant milestone in the evolution of assistive technologies for blind people. While these applications offer new possibilities and empower blind individuals in unprecedented ways, it is crucial to approach their development and adoption with a user-centered perspective, ensuring that they are accessible, inclusive, and responsive to the diverse needs of the blind community. As the field of assistive technology continues to evolve, researchers and developers must remain committed to creating tools that truly enhance the lives of blind individuals and promote their full participation in the digital world. %The decline of traditional assistive technologies in the face of LLM-based applications serves as a reminder of the importance of continuous innovation, adaptation, and user-centered design in the quest to create a more accessible and inclusive world for all.




\vspace{-5pt}
\section{Discussion}
\textbf{Conclusion.}
In this work, we propose the \textit{\methodname{}} metric, $M_{AP}$, to evaluate preference data quality in alignment.
By measuring the gap from the model's current implicit reward margin to the target explicit reward margin, $M_{AP}$ quantifies the discrepancy between the current model and the aligned optimum, thereby indicating the potential for alignment enhancement.
Extensive experiments validate the efficacy of $M_{AP}$ across various training settings under offline and self-play preference learning scenarios.

\textbf{Limitations and future work}. 
Despite the performance improvements, $M_{AP}$ requires tuning a parameter $\beta$ to combine the explicit and implicit margins; future work could explore how to set this ratio automatically.
Additionally, while our experiments focus on the widely applied DPO and SimPO objectives, a broader investigation with alternative preference learning methods is crucial in future works.

% \section{Conclusion}
% In this paper, we introduce the \methodname{} metric to evaluate preference data quality in LLM alignment.
% By measuring the discrepancy between the model's current implicit reward margin to the target explicit reward margin, this metric quantifies the gap between the current model and the aligned optimum, thereby indicating the potential for alignment enhancement.
% Empirical results demonstrate that training on data selected by our metric consistently improves alignment performance, outperforming existing metrics across different base models and training objectives.
% Moreover, this metric extends to data generation scenarios (\ie self-play alignment): by identifying high-quality data from the intrinsic self-generated context, our metric yields superior results across various training settings, providing a comprehensive solution for enhancing LLM alignment through optimized
% preference data generation, selection, and utilization.


\section*{Impact Statement}
This paper presents work whose goal is to advance the field of Machine Learning. There are many potential societal consequences of our work, none which we feel must be specifically highlighted here.
%%%%%%%%%%%%%%%%%%%%%%%%%%%%%%%%%

%%
%% The next two lines define the bibliography style to be used, and
%% the bibliography file.
\bibliographystyle{ACM-Reference-Format}
\bibliography{bibliography}



\newpage
\appendix
% \onecolumn

% \section{Pseudo-code for \Ours}

\section{LLM inference strategy and IR pipelines}

\begin{table}[h]
\caption{Correspondence between LLM inference and IR pipelines.}
  \label{tb:llm-retriever-reranker}
  \centering
%   \small
  \scalebox{0.8}{\begin{tabular}{lccc}
    \toprule
    Method & Retriever & Reranker & Pipeline       \\
    \midrule
    Greedy decoding     & LLM &  $\emptyset$ & Retriever-only  \\
    \midrule
    Best-of-N \citep{stiennon2020learning} & LLM & Reward model & Retriever-reranker  \\
    \midrule
    Majority voting  \citep{wang2022self}  & LLM & Majority & Retriever-reranker  \\
    \midrule
    Iterative refinement \citep{madaan2024self} & LLM & $\emptyset$ & Iterative retrieval  w. query rewriting \\
    \bottomrule
  \end{tabular}}
\end{table}


\section{How can SFT and preference optimization help the LLM from an IR perspective?}\label{apx:sft-rlhf-empirical}


We assess how well LLMs perform at two tasks: fine-grained reranking (using greedy decoding accuracy) and coarse-grained retrieval (using Recall@$N$).  
We focus on how SFT and DPO, affect these abilities.  
Using the Mistral-7b model, we evaluate on the GSM8k and MATH datasets with two approaches: SFT-only, and SFT followed by DPO (SFT $\rightarrow$ DPO).

In the SFT phase, the model is trained directly on correct answers. 
For DPO, we generate 20 responses per prompt and created preference pairs by randomly selecting one correct and one incorrect response.  
We use hyperparameter tuning and early stopping to find the best model checkpoints (see Appendix \ref{apx:sec:sft-rlhf} for details).


\begin{table}[h]
\caption{Retrieval (Recall@N) and reranking (greedy accuracy) metrics across dataset and training strategies, with Mistral-7b as the LLM. 0.7 is used as the temperature. Recall@N can also be denoted as pass@N.}\label{tb:sft-rlhf-result}
\vskip 1em
\centering
\small
\begin{tabular}{llcccc}
    \toprule
     & Metric & \textbf{init model} & \textbf{SFT} & \textbf{SFT $\rightarrow$ DPO} \\
    \midrule
    \multirow{4}{*}{\rotatebox{90}{GSM8K}} 
    & Greedy Acc & 0.4663 & 0.7680 & 0.7991  \\
    & Recall@20 & 0.8347 & 0.9462 & 0.9545  \\
    & Recall@50 & 0.9090 & 0.9629 & 0.9727  \\
    & Recall@100 & 0.9477 & 0.9735 & 0.9826   \\
    \midrule
    \multirow{4}{*}{\rotatebox{90}{Math}} 
    & Greedy Acc & 0.1004 & 0.2334 & 0.2502 \\
    & Recall@20 & 0.2600 & 0.5340 & 0.5416  \\
    & Recall@50 & 0.3354 & 0.6190 & 0.6258  \\
    & Recall@100 & 0.4036 & 0.6780 & 0.6846  \\
    \bottomrule
\end{tabular}
\end{table}

The results are shown in Table \ref{tb:sft-rlhf-result}.  
We observe that both SFT and DPO improve both retrieval and reranking, with SFT being more effective. Adding DPO after SFT further improves performance on both tasks.  
This is consistent with information retrieval principles that both direct retriever optimization and reranker-retrieval distillation can enhance the retriever performance, while the latter on top of the former can further improve the performance. Further discussions can be found in Appendices \ref{apx:discuss1} and \ref{apx:discuss2}.


\section{Discussion on the connection and difference between SFT and direct retriever optimization}\label{apx:discuss1}

As discussed in Section \ref{sec:llm-tuning-retriever}, the direct retriever optimization goal with InfoNCE is shown as:
\begin{gather*}
    \max \log P(d_{\text{gold}}|q) = \max \log \frac{\text{Enc}_d(d_{\text{gold}}) \cdot\text{Enc}_q(q)}{\sum^{|C|}_{j=1} \text{Enc}_d(d_j) \cdot\text{Enc}_q(q)},
\end{gather*}
while the SFT optimization goal is shown as:
\begin{gather}
    \max \log P(y_{\text{gold}}|x) = \max \log \prod^{|y_{\text{gold}}|}_i P(y_{\text{gold}}(i)|z_i) 
    = \max \sum^{|y_{\text{gold}}|}_i \log \frac{\text{Emb}(y_{\text{gold}}(i)) \cdot\text{LLM}(z_i)}{\sum^{|V|}_{j=1} \text{Emb}(v_j) \cdot\text{LLM}(z_i)}. \label{apx:eq:sft}
\end{gather}

As a result, the SFT objective can be seen as a summation of multiple retrieval optimization objectives, where $\text{LLM}(\cdot)$ and word embedding $\text{Emb}(\cdot)$ are query encoder and passage encoder respectively.

However, for direct retriever optimization with InfoNCE, $\text{Enc}_d(\cdot)$ is usually a large-scale pretrained language model which is computationally expensive on both time and memory.
In this case, it is unrealistic to calculate the $\text{Enc}_d(d_j)$ for all $d_j\in C$, when $C$ is large, because of the time constrain and GPU memory constrain.
As a result, a widely-adopted technique is to adopt ``in-batch negatives'' with ``hard negatives'' to estimate the $\log P(d_{\text{gold}}|q)$ function:
\begin{gather*}
    \max \log P(d_{\text{gold}}|q) = \max \log \frac{\text{Enc}_d(d_{\text{gold}}) \cdot\text{Enc}_q(q)}{\sum^{|C|}_{j=1} \text{Enc}_d(d_j) \cdot\text{Enc}_q(q)} \\
    \sim \max \log \frac{\text{Enc}_d(d_{\text{gold}}) \cdot\text{Enc}_q(q)}{\sum^{|B|}_{i=1} \text{Enc}_d(d_i) \cdot\text{Enc}_q(q) + \sum^{|H|}_{j=1} \text{Enc}_d(d_j) \cdot\text{Enc}_q(q)},
\end{gather*}
where $B$ is the in-batch negative set and $H$ is the hard negative set.
Note that $B\bigcup H \subset C$.
This objective is more efficient to optimize but is not the original optimization goal. As a result, the learned model after direct retriever optimization is not optimal.
It is also found that the hard negatives $H$ is the key to estimate the original optimization goal \citep{zhan2021optimizing}.
Thus, reranker-retriever distillation can further improve the retriever by introducing more hard negatives.

On the other hand, LLM optimization, as shown in Eq. (\ref{apx:eq:sft}), can be seen as a summation of multiple retrieval optimization function.
In each retrieval step, the passage can be seen as a token and the corpus is the vocabulary space $V$.
Given that the passage encoder $\text{Emb}(\cdot)$ (word embedding) here is cheap to compute and the vocabulary space $V$ ($<$100k) is usually not as large as $C$ ($>$1M) in IR, the objective in Eq. (\ref{apx:eq:sft}) can be directly optimized without any estimation.
In this case, the LLM as a retriever is more sufficiently trained compared with the retriever training in IR.


\section{Discussion on the connection and difference between preference optimization and reranker-retriever distillation}\label{apx:discuss2}

As discussed in Section \ref{sec:llm-tuning-retriever}, preference optimization with an online reward model $f_{\text{reward-model}}(\cdot) \overset{r}{\rightarrow} \text{data} \overset{g(\cdot)}{\rightarrow}  f_{\text{LLM}}(\cdot)$ can be seen as a reranker to retriever distillation process $f_{\text{rerank}}(\cdot) \overset{r}{\rightarrow} \text{data}\overset{g(\cdot)}{\rightarrow}   f_{\text{retrieval}}(\cdot)$, where the reward model is the reranker (\textit{i.e.}, cross-encoder) and the LLM is the retriever (\textit{i.e.}, bi-encoder).

However, there are two slight differences here:
\begin{itemize}[leftmargin=*]
\item The LLM after SFT is more sufficiently trained compared to a retriever after direct optimization. As discussed in Appendix \ref{apx:discuss1}, the SFT optimization function is not an estimated retriever optimization goal compared with the direct retrieval optimization. As a result, the LLM after SFT is suffienctly trained. In this case, if the reward model (reranker) cannot provide information other than that already in the SFT set (\textit{e.g.}, using the SFT prompts), this step may not contribute to significant LLM capability improvement.
\item The reward model may introduce auxiliary information than the reranker in IR. For a reranker in IR, it captures a same semantic with the retriever: semantic similarity between the query and the passage. However, in LLM post-training, the goal and data in SFT and preference optimization can be different. For example, the SFT phase could have query/response pairs which enable basic chat-based retrieval capability for the LLM. While the reward model may contain some style preference information or safety information which do not exist in SFT data. In this case, the preference optimization which is the reranker to retriever distillation step could also contribution to performance improvement.
\end{itemize}


\section{Evaluate LLMs as retrievers}\label{apx:llm-as-retriever}

In addition to Mathstral-7b-it on GSM8K in Figure \ref{fig:mathstral-gsm8k-infer}, we conduct extensive experiments to both Mistral-7b-it and Mathstral-7b-it on GSM8K and MATH. The results are shown in Figure \ref{apx:fig:empirical-llm-retriever}.
We have similar findings as in Figure \ref{fig:mathstral-gsm8k-infer} that:
(1) As $N$ increases, Recall@$N$ improves significantly, indicating that retrieving a larger number of documents increases the likelihood of including a correct one within the set.
(2) For smaller values of $N$ (e.g., $N=1$), lower temperatures yield higher Recall@$N$. This is because lower temperatures reduce response randomness, favoring the selection of the most relevant result.
(3) Conversely, for larger $N$ (e.g., $N>10$), higher temperatures enhance Recall@$N$. Increased temperature promotes greater response diversity, which, when combined with a larger retrieval set, improves the chances of capturing the correct answer within the results.

\begin{figure*}[h]
    \centering
    \subfigure[Mistral-7b-it on GSM8k]{\includegraphics[width=0.45\textwidth]{figure/LLM_alignment_gsm8k_mathstral7b_infer.pdf}}
    \subfigure[Mistral-7b-it on GSM8k]{\includegraphics[width=0.45\textwidth]{figure/LLM_alignment_gsm8k_mistral7b_infer.pdf}}
    \subfigure[Mathstral-7b-it on MATH]{\includegraphics[width=0.45\textwidth]{figure/LLM_alignment_math_mathstral7b_infer.pdf}} 
    \subfigure[Mistral-7b-it on MATH]{\includegraphics[width=0.45\textwidth]{figure/LLM_alignment_math_mistral7b_infer.pdf}}
    % \vspace{-0.1in}
    \vskip -1em
    \caption{Evaluate the LLM as a retriever with Recall@N (Pass@N). As the number (N) of retrieved responses increases, the retrieval recall increases. The higher the temperature is, the broader spectrum the retrieved responses are, and thus the higher the recall is.}\label{apx:fig:empirical-llm-retriever}
\end{figure*}


% \subsection{How SFT and RLHF benefit the LLM retriever?}\label{apx:sft-rlhf}

% In addition to the experiments with Gemma-1-7b-it in Table \ref{tb:sft-rlhf-result}, we also conduct experiments to study the effect of SFT and DPO on Deepseek-math-7b-base model \citep{shao2024deepseekmath}.
% The results on MATH dataset are shown in Table \ref{apx:tb:sft-rlhf-result}, where we have similar discovery with that in Table \ref{tb:sft-rlhf-result}:
% (1) Both SFT and DPO can improve the retrieval capability of the LLM, while SFT is more effective.
% (2) On top of SFT, DPO can slightly improve the reranking capability (greedy accuracy) but not the general retrieval capability.

% \begin{table}[h]
% \caption{Retrieval (Recall@N) and reranking (greedy accuracy) metrics across dataset and training strategies. LLM: Deepseek-math-7b. Temperature: 0.7. Recall@N can also be denoted as pass@N.}\label{apx:tb:sft-rlhf-result}
% \vskip 1em
% \centering
% \scalebox{0.8}{
% \begin{tabular}{llcccc}
%     \toprule
%      & Metric & \textbf{init model} & \textbf{DPO} & \textbf{SFT} & \textbf{SFT $\rightarrow$ DPO} \\
%     \midrule
%     \multirow{4}{*}{\rotatebox{90}{Math}} 
%     & Greedy Acc & 0.0972 & 0.1164 & 0.3078 & 0.312 \\
%     & Recall@20 & 0.4914 & 0.5136 & 0.6524 & 0.6558 \\
%     & Recall@50 & 0.6058 & 0.6278 & 0.7332 & 0.736 \\
%     & Recall@100 & 0.6728 & 0.6976 & 0.7844 & 0.7828 \\
%     \bottomrule
% \end{tabular}}
% \end{table}



\section{\Ours retriever optimization objective}\label{apx:proofs}

We provide the proof for different variants of \Ours's objective functions.

\subsection{Contrastive ranking}\label{apx:proof:contrastive}

\begin{theorem}
Let \( x \) be a prompt and \( (y_w, y^{(1)}_l, ..., y^{(m)}_l)  \) be the responses for \( x \) under the contrastive assumption (Eq.(\ref{eq:contrastive-assumption})).
Then the objective function to learn the LLM \( \pi_\theta \):
\end{theorem}

\begin{equation}
    \begin{aligned}
    \mathcal{L}_{\text{con}} = -\mathbb{E} & \biggl[
    \log \frac{\exp\bigl(\gamma(y_w \mid x)\bigr)}{
        \exp\bigl(\gamma(y_w \mid x)\bigr) + \sum_{i=1}^m \exp\bigl(\gamma(y_l^{(i)} \mid x)\bigr)}
    \biggr], \\
    \text{where } &\quad \gamma(y \mid x) = \beta \log \frac{\pi_\theta(y \mid x)}{\pi_{\mathrm{ref}}(y \mid x)}.
\end{aligned}\label{eq:contrastive}
\end{equation}

\textit{Proof.}
From \citep{rafailov2024direct}, we know that
\begin{gather}
    r(x, y) = \beta \text{log} \frac{\pi_{\text{llm}}(y|x)}{\pi_{\text{ref}}(y|x)} + \beta \text{log} Z,
\end{gather}
where $Z = \sum_{y'} \pi_{\text{ref}}(y'|x) \text{exp}(\frac{1}{\beta} r(x, y'))$.

Then,
\begin{equation}\label{eq:1-n}
\begin{aligned}
\mathbb{P}\text{r}(y_w & \succeq y^{(1)}_l, ..., y_w \succeq y^{(m)}_l) 
= \text{softmax}(r(x, y_w)) \\
&= \frac{\text{exp}(r(x,y_w))}{\text{exp}(r(x,y_w)) + \sum^m_{i=1}\text{exp}(r(x,y^{(i)}_l))} \\
&= \frac{1}{1 + \sum^m_{i=1}\text{exp}(r(x,y^{(i)}_l)-r(x,y_w))} \\
&= \frac{1}{1 + \sum^m_{i=1}\text{exp}(\gamma(y^{(i)}_l \mid x) + \beta \text{log} Z - \gamma(y_w \mid x) - \beta \text{log} Z)} \\
&= \frac{1}{1 + \sum^m_{i=1}\text{exp}(\gamma(y^{(i)}_l \mid x) - \gamma(y_w \mid x))} \\
&= \frac{\exp\bigl(\gamma(y_w \mid x)\bigr)}{
        \exp\bigl(\gamma(y_w \mid x)\bigr) + \sum_{i=1}^m \exp\bigl(\gamma(y_l^{(i)} \mid x)\bigr)}
\end{aligned}
\end{equation}

We can learn $\pi_\theta$ by maximizing the logarithm-likelihood: 
\begin{gather}
\max \log \mathbb{P}\text{r}(y_w \succeq y^{(1)}_l, \dots, y_w \succeq y^{(m)}_l) \Leftrightarrow 
\min - \log \mathbb{P}\text{r}(y_w \succeq y^{(1)}_l, \dots, y_w \succeq y^{(m)}_l) = \mathcal{L}, \\
 \therefore \mathcal{L}_{\text{con}} = -\mathbb{E} \biggl[
    \log \frac{\exp\bigl(\gamma(y_w \mid x)\bigr)}{
        \exp\bigl(\gamma(y_w \mid x)\bigr) + \sum_{i=1}^m \exp\bigl(\gamma(y_l^{(i)} \mid x)\bigr)}
    \biggr], \\
\text{where} \quad \gamma(y \mid x) = \beta \log \frac{\pi_\theta(y \mid x)}{\pi_{\mathrm{ref}}(y \mid x)}.
\end{gather}



\subsection{LambdaRank ranking}\label{apx:proof:lambdarank}

\begin{theorem}
Let \( x \) be a prompt and \( (y_1, ..., y_m)  \) be the responses for \( x \) under the LambdaRank assumption (Eq.(\ref{eq:lambdarank-assumption})).
Then the objective function to learn the LLM \( \pi_\theta \):
\end{theorem}

% \begin{gather}
%     \mathcal{L}_{\text{lamb}}=-\mathbb{E}\;\biggl[ \sum_{1<i<j<m}
%   w_{ij}\log \sigma\Bigl(
%      \gamma(y_i \mid x)-
%      \gamma(y_j \mid x)
%   \Bigr)
% \biggr]
% \end{gather}
\begin{gather}
    \mathcal{L}_{\text{lamb}}=-\mathbb{E}\;\biggl[ \sum_{1<i<j<m}
   \log \sigma\Bigl(
     \gamma(y_i \mid x)-
     \gamma(y_j \mid x)
   \Bigr)
\biggr].
\end{gather}
% where $w_{ij}$ is an adjustable weight.

\textit{Proof.}
\begin{equation}
\begin{aligned}
\mathbb{P}\text{r}(y_1 & \succeq ... \succeq y_m)
= \prod_{1<i<j<m} \sigma(r(x,y_i) - r(x,y_j)) \\
&= \prod_{1<i<j<m} \sigma(\gamma(x,y_i) + \beta \text{log} Z - \gamma(x,y_j) - \beta \text{log} Z)  \\
&= \prod_{1<i<j<m} \sigma(\gamma(y_i \mid x)-
     \gamma(y_j \mid x)).
\end{aligned}
\end{equation}

We can learn $\pi_\theta$ by maximizing the logarithm-likelihood: 
\begin{gather}
\max \log \mathbb{P}\text{r}(y_w \succeq y^{(1)}_l, \dots, y_w \succeq y^{(m)}_l) \Leftrightarrow 
\min - \log \mathbb{P}\text{r}(y_w \succeq y^{(1)}_l, \dots, y_w \succeq y^{(m)}_l) = \mathcal{L}, \\
 \therefore \mathcal{L}_{\text{lamb}}=-\mathbb{E}\;\biggl[ \sum_{1<i<j<m}
   \log \sigma\Bigl(
     \gamma(y_i \mid x)-
     \gamma(y_j \mid x)
   \Bigr)
\biggr], \\
\text{where} \quad \gamma(y \mid x) = \beta \log \frac{\pi_\theta(y \mid x)}{\pi_{\mathrm{ref}}(y \mid x)}.
\end{gather}
% $w_{ij}$ can be added to control the weight of each pair in the candidate list.


\subsection{ListMLE ranking}\label{apx:proof:listmle}

\begin{theorem}
Let \( x \) be a prompt and \( (y_1, ..., y_m)  \) be the responses for \( x \) under the ListMLE assumption (Eq.(\ref{eq:listmle-assumption})).
Then the objective function to learn the LLM \( \pi_\theta \):
\end{theorem}

\begin{equation}
\begin{aligned}
    \mathcal{L}_{\text{lmle}} &= -\mathbb{E} \biggl[
    \sum^m_{i=1} \log \frac{\exp\bigl(\gamma(y_i \mid x)\bigr)}{
        \exp\bigl(\gamma(y_i \mid x)\bigr) + \sum_{j=i}^m \exp\bigl(\gamma(y_j \mid x)\bigr)}
    \biggr].
\end{aligned}
\end{equation}

\textit{Proof.}
From Eq.(\ref{eq:1-n}),
\begin{gather}
\begin{aligned}
    \mathbb{P}\text{r}(y_1 & \succeq ... \succeq y_m) = \prod^m_{i=1} \mathbb{P}\text{r}(y_i \succeq y_{i+1}, ..., y_i \succeq y_m)  \\
    & = \prod^m_{i=1} \frac{\text{exp}(\gamma(y_i \mid x))}{\text{exp}(\gamma(y_i \mid x)) + \sum^m_{j=i+1}\text{exp}(\gamma(y_j \mid x))}
\end{aligned}.
\end{gather}
% The derivation above uses the result from Eq.(\ref{eq:1-n}).

We can learn $\pi_\theta$ by maximizing the logarithm-likelihood: 
\begin{gather}
\max \log \mathbb{P}\text{r}(y_w \succeq y^{(1)}_l, \dots, y_w \succeq y^{(m)}_l) \Leftrightarrow 
\min - \log \mathbb{P}\text{r}(y_w \succeq y^{(1)}_l, \dots, y_w \succeq y^{(m)}_l) = \mathcal{L}, \\
 \therefore \mathcal{L}_{\text{lmle}} = -\mathbb{E} \biggl[
    \sum^m_{i=1} \log \frac{\exp\bigl(\gamma(y_i \mid x)\bigr)}{
        \exp\bigl(\gamma(y_i \mid x)\bigr) + \sum_{j=i}^m \exp\bigl(\gamma(y_j \mid x)\bigr)}
    \biggr], \\
\text{where} \quad \gamma(y \mid x) = \beta \log \frac{\pi_\theta(y \mid x)}{\pi_{\mathrm{ref}}(y \mid x)}.
\end{gather}


\section{Baselines}\label{apx:sec:baselines}

We conduct detailed illustrations on the baselines compared with \Ours in Section \ref{sec:main-result} below.

\begin{itemize}[leftmargin=*]
  \item RRHF \citep{yuan2023rrhf} scores responses via a logarithm of conditional probabilities and learns to align these probabilities with human preferences through ranking loss.
  \item SLiC-HF \citep{zhao2023slic} proposes a sequence likelihood calibration method which can learn from human preference data.
  \item DPO \citep{guo2024direct} simplifies the PPO \citep{ouyang2022training} algorithms into an offline direct optimization objective with the pairwise Bradley-Terry assumption.
  \item IPO \citep{azar2024general} theoretically grounds pairwise assumption in DPO into a pointwise reward.
  \item CPO \citep{xu2024contrastive} adds a reward objective with sequence likelihood along with the SFT objective.
  \item KTO \citep{ethayarajh2024kto} adopts the Kahneman-Tversky model and proposes a method which directly maximizes the utility of generation instead of the likelihood of the preferences.
  \item RDPO \citep{park2024disentangling} modifies DPO by including an additional regularization term to disentangle the influence of length.
  \item SimPO \citep{meng2024simpo} further simplifies the DPO objective by using the average log probability of a sequence as the implicit reward and adding a target reward margin to the Bradley-Terry objective.
  \item Iterative DPO \citep{xiong2024iterative} identifies the challenge of offline preference optimization and proposes an iterative learning framework.
\end{itemize}


\section{Experiment settings}\label{apx:sec:main-result-setting}

\subsection{Table \ref{tab:main-performance}}\label{apx:sec:main}

We conduct evaluation on two widely used benchmark: AlpacaEval2 \citep{dubois2024length} and MixEval \citep{ni2024mixeval}.
We consider two base models: Mistral-7b-base and Mistral-7b-it. For Mistral-7b-base, we first conduct supervised finetuning following \citet{meng2024simpo} before the preference optimization.

The performance scores for offline preference optimization baselines are from SimPO \citep{meng2024simpo}.
To have a fair comparison with these baselines, we adopt the same off-the-shelf reward model \citep{jiang2023llm} as in SimPO for the iterative DPO baseline and \Ours.

For the iterative DPO baseline, we generate 2 responses for each prompt, score them with the off-the-shelf reward model and construct the preference pair data to tune the model.

For \Ours (contrastive $\mathcal{L}_{\text{con}}$), we generate 10 responses each iteration and score them with the reward model. The top-1 ranked response and the bottom-3 ranked responses are adopted as the chose response and rejected responses respectively.
Generation temperature is selected as 1 and 0.8 for Mistral-7b-base and Mistral-7b-it respectively (we search it among 0.8, 0.9, 1.0, 1.1, 1.2).

For \Ours (LambdaRank $\mathcal{L}_{\text{lamb}}$), we generate 10 responses each iteration and score them with the reward model. The top-2 ranked response and the bottom-2 ranked responses are adopted as the chose response and rejected responses respectively.
Generation temperature is selected as 1 and 0.8 for Mistral-7b-base and Mistral-7b-it respectively (we search it among 0.8, 0.9, 1.0, 1.1, 1.2).

For \Ours (ListMLE $\mathcal{L}_{\text{lmle}}$), we generate 10 responses each iteration and score them with the reward model. The top-2 ranked response and the bottom-2 ranked responses are adopted as the chose response and rejected responses respectively.
Generation temperature is selected as 1 and 0.8 for Mistral-7b-base and Mistral-7b-it respectively (we search it among 0.8, 0.9, 1.0, 1.1, 1.2).

\Ours can achieve even stronger performance with stronger off-the-shelf reward model \citep{dong2024rlhf}.
% Results with stronger a reward model can be found in Appendix \ref{apx:sec:stronger-rm}.

\subsection{Table \ref{tab:objective}}\label{apx:sec-objective-setting}

We conduct experiments on both Gemma2-2b-it \citep{team2024gemma} and Mistral-7b-it \citep{jiang2023mistral}.
Following \citet{Tunstall_The_Alignment_Handbook} and \citet{dong2024rlhf}, we perform training on UltraFeedback dataset for 3 iterations and show the performance of the final model checkpoint.
We use the pretrained reward model from \citet{dong2024rlhf}.
The learning rate is set as 5e-7 and we train the LLM for 2 epochs per iteration.

For the pairwise objective, we generate 2 responses for each prompt and construct the preference pair data with the reward model.
For the others, we generate 4 responses per prompt and rank them with the reward model.
For the contrastive objective, we construct the 1-vs-N data with the top-1 ranked response and the other responses.
For the listMLE and lambdarank objective, we take the top-2 as positives and the last-2 as the negatives.
Experiments with opensource LLM as the evaluator (\texttt{alpaca\_eval\_llama3\_70b\_fn}) can be found in Table \ref{tab:objective2}.



\begin{table*}[t]
    \centering
    % \renewcommand{\arraystretch}{1.2}
    \caption{Preference optimization objective study on AlpacaEval2 and MixEval. For AlpacaEval2, we report the result with both opensource LLM evaluator \texttt{alpaca\_eval\_llama3\_70b\_fn} and GPT4 evaluator \texttt{alpaca\_eval\_gpt4\_turbo\_fn}. SFT corresponds to the initial chat model.}\label{tab:objective2}
    \small
    \begin{tabular}{llccccccccc}
        \toprule
        & & \multicolumn{2}{c}{AlpacaEval 2 (opensource LLM)} & \multicolumn{2}{c}{AlpacaEval 2 (GPT-4)} & \multicolumn{1}{c}{MixEval} & \multicolumn{1}{c}{MixEval-Hard} \\
         \cmidrule(r){3-4} \cmidrule(r){5-6} \cmidrule(r){7-7} \cmidrule(r){8-8}
        & Method & LC Winrate & Winrate & LC Winrate & Winrate & Score & Score \\
        \midrule
        \multirow{6}{*}{\rotatebox{90}{Gemma2-2b-it}} & SFT & 47.03 & 48.38 & 36.39 & 38.26 & 0.6545 & 0.2980 \\
        \cmidrule{2-8}
        & pairwise & 55.06 & 66.56 & 41.39 & 54.60 & 0.6740 & 0.3375 \\
        & contrastive & 60.44 & 72.35 & 43.41 & 56.83 & 0.6745 & 0.3315 \\
        & ListMLE & 63.05 & 76.09 & 49.77 & 62.05 & 0.6715 & 0.3560 \\
        & LambdaRank & 58.73 & 74.09 & 43.76 & 60.56 & 0.6750 & 0.3560 \\
        \midrule
        \midrule
        \multirow{6}{*}{\rotatebox{90}{Mistral-7b-it}} & SFT & 27.04 & 17.41 & 21.14 & 14.22 & 0.7070 & 0.3610 \\
        \cmidrule{2-8}
        & pairwise & 49.75 & 55.07 & 36.43 & 41.86 & 0.7175 & 0.4105 \\
        & contrastive & 52.03 & 60.15 & 38.44 & 42.61 & 0.7260 & 0.4340 \\
        & ListMLE & 48.84 & 56.73 & 38.02 & 43.03 & 0.7360 & 0.4200 \\
        & LambdaRank & 51.98 & 59.73 & 40.29 & 46.21 & 0.7370 & 0.4400 \\
        \bottomrule
    \end{tabular}
\end{table*}


\subsection{Table \ref{fig:list-study}}\label{apx:sec-list-setting}

We adopt Gemma2-2b-it as the initial model. All the models are trained with iterative DPO for 3 iterations. We use the off-the-shelf reward model \citep{dong2024rlhf}.
We generate 2 responses for each prompt in each iteration.
For ``w. current'', we only use the scored responses in the current iteration for preference optimization data construction.
For ``w. current + prev'', we rank the responses in the current iteration and the previous one iteration, and construct the preference pair data with the top-1 and bottom-1 ranked responses.
For ``w. current + all prev'', we rank all the responses for the prompt in the current and previous iterations and construct the preference pair data.
For ``single temperature'', we only adopt temperature 1 and generate 2 responses for reward model scoring.
For ``diverse temperature'', we generate 2 responses with temperature 1 and 0.5 respective and rank the 4 responses to construct the preference data with the reward model.

\subsection{Table \ref{tb:sft-rlhf-result}}\label{apx:sec:sft-rlhf}

We use mistral-7b-it \citep{jiang2023mistral} as the initial model to alleviate the influence of the math related post-training data of the original model.
% For SFT, we conduct training on the training set of MATH \citep{hendrycks2021measuring} and GSM8K \citep{cobbe2021training} respectively.
For SFT, we conduct training on the meta-math dataset \citep{yu2023metamath}.
For DPO, we use the prompts in the training set of the two dataset and conduct online iterative preference optimization with the binary rule-based reward (measure if the final answer is correct or not with string match). 
The evaluation is performed on the test set of MATH and GSM8K respectively.
% For both SFT and DPO, we conduct careful hyper-parameter search.
For SFT, we follow the same training setting with \citet{yu2023metamath}.
For DPO, we search the learning rate in 1e-7, 2e-7, 5e-7, 2e-8, 5e-8 and train the LLM for 5 iterations with early stop (1 epoch per iteration for MATH and 2 epoch per iteration for GSM8K). The learning rate is set as 1e-7 and we select the checkpoint after the first and fourth iteration for GSM8K and MATH respectively.

\subsection{Figure \ref{fig:merge-study}(a)}\label{apx:sec-hard-neg-setting}

We conduct training with the prompts in the training set of GSM8K and perform evaluation on GSM8K testing set.
We conduct learning rate search and finalize it to be 2e-7.
The learning is performed for 3 iterations.

We make explanations of how we construct the four types of negative settings:
For (1) a random response not related to the given prompt, we select a response for a random prompt in Ultrafeedback.
For (2) a response to a related prompt, we pick up a response for a different prompt in the GSM8K training set.
For (3) an incorrect response to the given prompt with high temperature, we select the temperature to be 1.
For (4) an incorrect response to the given prompt with low temperature, we select the temperature to be 0.7.

\begin{figure}[t]
\centering
\includegraphics[scale=0.4]{figure/LLM_alignment_gemma_temperature_study.pdf}
\vskip -1em
\caption{Training temperature study with $\mathcal{L}_{\text{pair}}$ on Gemma2-2b-it and Alpaca Eval 2. Within a specific range ($>$ 0.9), lower temperature leads to harder negative and benefit the trained LLM. However, temperature lower than this range can cause preferred and rejected responses non-distinguishable and lead to degrade training.}\label{apx:tab:temp-hard}
\end{figure}

\subsection{Figure \ref{fig:merge-study}(b)}\label{apx:sec-hard-neg-setting-temp}

We conduct experiments on both Gemma2-2b-it and Mistral-7B-it models.
For both LLMs, we conduct iterative DPO for 3 iterations and report the performance of the final model.
We perform evaluation on Alpaca Eval2 with \texttt{alpaca\_eval\_llama3\_70b\_fn} as the evaluator.

For temperature study, we find that under a specific temperature threshold, repeatedly generated responses will be large identical for all LLMs and cannot be used to construct preference data, while the threshold varies for different LLMs.
% As a result, we select temperatures above the threshold for robust experiments.
The ``low'' and ``high'' refer to the value of those selected temperatures.
% For Gemma2-2b-it, we use temperature as 0.2, 0.5 and 0.7 to generate the responses, score the responses by the reward model and train the LLM with the newly labeled data.
% For Mistral-7b-it, we set the temperature as 1, 1.1 and 1.2 respectively.
We also conduct experiments on Gemma2-2b-it model and show the results in Figure \ref{apx:tab:temp-hard}.


\subsection{Figure \ref{fig:merge-study}(c)}\label{apx:sec-length-setting}

We adopt Mistral-7b-it as the initial LLM and the contrastive objective (Eq. \ref{eq:contrastive}) in iterative preference optimization.
We generate 4/6/8/10 responses with the LLM and score the responses with the off-the-shelf reward model \citep{dong2024rlhf}.
The top-1 scored response is adopted as the positive response and the other responses are treated as the negative responses to construct the 1-vs-N training data.
The temperature is set as 1 to generate the responses.


% \newpage
% \section{\Ours with a stronger reward model}\label{apx:sec:stronger-rm}

% In Section \ref{sec:lrpo}, we show the results with LLM-Blender \citep{jiang2023llm} as the reward model to have a fair comparison with the baseline methods.
% In this section, we would like to show that \Ours can achieve even stronger performance with stronger off-the-shelf reward model \citep{dong2024rlhf}.
% The results are shown in Table \ref{apx:tab:main-performance}, where we can find that a stronger reward model can further improve the performance of \Ours.


% \begin{table*}[h]
%     \centering
%     \caption{Method evaluation on AlpacaEval 2 and MixEval. LC WR and WR denote length-controlled win rate and win rate respectively. Offline baseline performances on AlpacaEval 2 are from \citept{meng2024simpo} with LLM-Blender reward model \citep{jiang2023llm}.}\label{apx:tab:main-performance}
%     \scalebox{0.78}{
%     \begin{tabular}{lcccccccccc}
%         \toprule
%         Model & \multicolumn{4}{c}{Mistral-Base (7B)} & \multicolumn{4}{c}{Mistral-Instruct (7B)} \\
%         \cmidrule(lr){2-5} \cmidrule(lr){6-9}
%         & \multicolumn{2}{c}{Alpaca Eval 2}  & \multirow{1}{*}{MixEval} & \multirow{1}{*}{MixEval-Hard} & \multicolumn{2}{c}{Alpaca Eval 2}  & \multirow{1}{*}{MixEval} & \multirow{1}{*}{MixEval-Hard} \\
%         \cmidrule(lr){2-3} \cmidrule(lr){4-4} \cmidrule(lr){5-5}  \cmidrule(lr){6-7} \cmidrule(lr){8-8} \cmidrule(lr){9-9}
%         & LC WR & WR  & Score & Score & LC WR  & WR & Score & Score \\
%         \midrule
%         SFT    & 8.4  & 6.2   &  0.602  & 0.279  & 17.1 & 14.7  & 0.707 & 0.361 \\
%         RRHF   & 11.6 & 10.2   &  0.600  & 0.312  & 25.3 & 24.8  &   0.700    & 0.380 \\
%         DPO    & 15.1 & 12.5  &  0.686  &  0.341 & 26.8 & 24.9  & 0.702 & 0.355 \\
%         KTO    & 13.1 & 9.1    & \textbf{0.704}  & 0.351   & 24.5 & 23.6  &   0.692    & 0.358 \\
%         RDPO   & 17.4 & 12.8  & 0.693  & 0.355   & 27.3 & 24.5  &   0.695    & 0.364 \\
%         SimPO  & 21.5 & 20.8  &  0.672  &  0.347 & 32.1 & 34.8  & 0.702  & 0.363 \\
%         Iterative DPO  & 18.9  & 16.7  & 0.660   & 0.341  & 20.4 & 24.84  & 0.719  & 0.389 \\
%         \midrule
%         \multicolumn{9}{c}{Reward model: LLM-Blender \citep{jiang2023llm}}  \\
%         \midrule
%         \Ours (contrastive) & 31.6 & 30.8  &   0.703 & 0.409  & 32.7 & 38.6  &  0.718 & \textbf{0.418} \\
%         \Ours (LambdaRank) &  \textbf{34.9} & \textbf{37.2} & 0.695 &  \textbf{0.452}  & \textbf{32.9} & \textbf{38.9}   & \textbf{0.720} & 0.417  \\
%         \Ours (ListMLE) & 31.1  &  32.1   &  0.669  & 0.390  &  29.7 & 36.2    & 0.709  & 0.397 \\
%         \midrule
%         \multicolumn{9}{c}{Reward model: FsfairX \citep{dong2024rlhf}}  \\
%         \midrule
%         \Ours (contrastive) & \textbf{41.5} & \textbf{42.9} & 0.718 & 0.417    & \textbf{43.0}  & \textbf{53.8} & 0.718 & 0.425   \\
%         \Ours (LambdaRank) & 35.8 & 34.1 & 0.717 & 0.431   & 41.9  & 48.1 & \textbf{0.740} & \textbf{0.440}  \\
%         \Ours (ListMLE) & 36.6 & 37.8 & \textbf{0.730} & \textbf{0.423}   & 39.6  & 48.1 & 0.717 & 0.397   \\
%         \bottomrule
%     \end{tabular}}
% \end{table*}






\end{document}

