% \documentclass[manuscript,review,anonymous]{acmart}   % anonymous

% \documentclass[manuscript,nonacm]{acmart}   % anonymous

\documentclass[sigconf]{acmart}

\usepackage{booktabs} % For formal tables
\usepackage{array}
\usepackage{multirow}
% \usepackage[export]{adjustbox}


\usepackage{enumitem}
\usepackage{colortbl}
\usepackage{wasysym}
\usepackage{xcolor}
% \usepackage[normalem]{ulem}
\usepackage{url}
\usepackage{soul}

\newcommand{\albert}[1]{\textcolor{blue}{#1}}
%%
%% \BibTeX command to typeset BibTeX logo in the docs
\AtBeginDocument{%
  \providecommand\BibTeX{{%
    \normalfont B\kern-0.5em{\scshape i\kern-0.25em b}\kern-0.8em\TeX}}}

%% Rights management information.  This information is sent to you
%% when you complete the rights form.  These commands have SAMPLE
%% values in them; it is your responsibility as an author to replace
%% the commands and values with those provided to you when you
%% complete the rights form.
\copyrightyear{2025}
\acmYear{2025}
\setcopyright{acmlicensed}\acmConference[CHI '25]{CHI Conference on Human Factors in Computing Systems}{April 26-May 1, 2025}{Yokohama, Japan}
\acmBooktitle{CHI Conference on Human Factors in Computing Systems (CHI '25), April 26-May 1, 2025, Yokohama, Japan}
\acmDOI{10.1145/3706598.3714210}
\acmISBN{979-8-4007-1394-1/25/04}

% %% These commands are for a PROCEEDINGS abstract or paper.
% \acmConference[Woodstock '18]{Woodstock '18: ACM Symposium on Neural
%   Gaze Detection}{June 03--05, 2018}{Woodstock, NY}
% \acmBooktitle{Woodstock '18: ACM Symposium on Neural Gaze Detection,
%   June 03--05, 2018, Woodstock, NY}
% \acmPrice{15.00}
% \acmISBN{978-1-4503-XXXX-X/18/06}


%% These commands are for a PROCEEDINGS abstract or paper.
% \acmConference[Conference acronym 'XX]{Make sure to enter the correct
%   conference title from your rights confirmation emai}{June 03--05,
%   2018}{Woodstock, NY}
%
%  Uncomment \acmBooktitle if th title of the proceedings is different
%  from ``Proceedings of ...''!
%
% \acmBooktitle{Woodstock '18: ACM Symposium on Neural Gaze Detection,
%  June 03--05, 2018, Woodstock, NY} 
% \acmISBN{978-1-4503-XXXX-X/18/06}



%%
%% Submission ID.
%% Use this when submitting an article to a sponsored event. You'll
%% receive a unique submission ID from the organizers
%% of the event, and this ID should be used as the parameter to this command.
% \acmSubmissionID{7224}

% comment the following line when submitting
\def \debug{}

%%%%%%%%% fixmes %%%%%%%%%%%%%%

% comment the following line to remove color codes
\def \debug{}

\ifx \debug \undefined

\newcommand{\fx}[1]{{}}
\newcommand{\fixmesb}[1]{{}}
\newcommand{\fixmejc}[1]{{}}
\newcommand{\fixmezz}[1]{{}}
\newcommand{\fixmejX}[1]{{}}
\newcommand{\fixmech}[1]{{}}
\newcommand{\fixmesl}[1]{{}}
\newcommand{\rev}[1]{{#1}}
\else

% \newcommand{\fx}[1]{{\textcolor{violet}{#1}}}
\newcommand{\fx}[1]{{\textcolor{red}{#1}}}
\newcommand{\fixmesb}[1]{{\bf\textcolor{red}{ [ sB FIXME: #1 ]}}}
\newcommand{\fixmejc}[1]{{\bf\textcolor{blue}{ [ jC FIXME: #1 ]}}}
\newcommand{\fixmery}[1]{{\bf\textcolor{brown}{ [ rY FIXME: #1 ]}}}
\newcommand{\fixmejx}[1]{{\bf\textcolor{green!10!orange!90!}{ [ jX FIXME: #1 ]}}}
\newcommand{\fixmech}[1]{{\bf\textcolor{purple}{ [ cH FIXME: #1 ]}}}
\newcommand{\fixmesl}[1]{{\bf\textcolor{yellow}{ [ sL FIXME: #1 ]}}}
\newcommand{\rev}[1]{{\color{black}{#1}}}


\newcommand{\V}{volunteer}
\newcommand{\vs}{volunteers}
% \newcommand{\bma}{Be My AI}
\newcommand{\bma}{Be~My~AI}
% \newcommand{\bma}{BMA}
\newcommand{\sbma}{BMA}  %s... for short
% \newcommand{\bme}{Be My Eyes}
\newcommand{\bme}{Be~My~Eyes}
\newcommand{\abme}{AI tool}
\newcommand{\ps}{participants}


% \newcommand{\vss}{volunteers}




\fi
%%%%%%%%%%%%%%%%%%%%%%%%%%%%%%%
%%%%%%%%%%%%%%%%% table %%%%%%%%%%%%%%%%%%
\newcolumntype{L}[1]{>{\raggedright\let\newline\\\arraybackslash\hspace{0pt}}m{#1}}
\newcolumntype{C}[1]{>{\centering\let\newline\\\arraybackslash\hspace{0pt}}m{#1}}
\newcolumntype{R}[1]{>{\raggedleft\let\newline\\\arraybackslash\hspace{0pt}}m{#1}}
\fancyhead{}

%
\newcounter{challengeno}
\newcommand{\challenge}[1]{\refstepcounter{challengeno}\label{#1}}

\newcounter{scenariono}
\newcommand{\scenario}[1]{\refstepcounter{scenariono}\label{#1}}

\newcounter{ideano}
\newcommand{\idea}[1]{\refstepcounter{ideano}\label{#1}}

%%
%% end of the preamble, start of the body of the document source.
\begin{document}

%%
%% The "title" command has an optional parameter,
%% allowing the author to define a "short title" to be used in page headers.


\title[Insights from Smartphone Interaction of Visually Impaired Users with Large Multimodal Models]
{Beyond Visual Perception: Insights from Smartphone Interaction of Visually Impaired Users with Large Multimodal Models}


% Emerging Practices for Large Multimodal Model (LMM) Assistance for People with Visual Impairments: Implications for Design
%LARGE multimodal models


% \renewcommand{\shortauthors}{Xie, Yu, Zhang, Lee, Billah, and Carroll}

% %% delete below for anonymous
\author{Jingyi Xie}
\affiliation{%
  \institution{Pennsylvania State University}
  \city{University Park}
  \state{PA}
  \country{USA}
  \postcode{16801}
}
\email{jzx5099@psu.edu}


\author{Rui Yu}
\affiliation{%
 \institution{University of Louisville}
  \city{Louisville}
  \state{KY}
  \country{USA}
  \postcode{40292}
}
\email{rui.yu@louisville.edu}


\author{He Zhang}
\affiliation{%
 \institution{Pennsylvania State University}
  \city{University Park}
  \state{PA}
  \country{USA}
  \postcode{16801}
}
\email{hpz5211@psu.edu}




\author{Syed Masum Billah}
\affiliation{%
  \institution{Pennsylvania State University}
  \city{University Park}
  \state{PA}
  \country{USA}
  \postcode{16801}
}
\email{sbillah@psu.edu}


\author{Sooyeon Lee}
\affiliation{
    \institution{New Jersey Institute of Technology}
  \city{Newark}
  \state{NJ}
  \country{USA}
  \postcode{07102}
}
\email{sooyeon.lee@njit.edu} 


\author{John M. Carroll}
\affiliation{%
 \institution{Pennsylvania State University}
  \city{University Park}
  \state{PA}
  \country{USA}
  \postcode{16801}
}
\email{jmc56@psu.edu}




%% The default list of authors is too long for headers.
% \renewcommand{\shortauthors}{Xie, Yu, Cui, Lee, Carroll, and Billah}
%% delete above for anonymous 



%%
%% The "author" command and its associated commands are used to define
%% the authors and their affiliations.
%% Of note is the shared affiliation of the first two authors, and the
%% "authornote" and "authornotemark" commands
%% used to denote shared contribution to the research.



%%
%% The abstract is a short summary of the work to be presented in the
%% article.

\begin{abstract}
Tree of Thoughts (ToT) enhances Large Language Model (LLM) reasoning by structuring problem-solving as a spanning tree. However, recent methods focus on search accuracy while overlooking computational efficiency. The challenges of accelerating the ToT lie in the frequent switching of reasoning focus, and the redundant exploration of suboptimal solutions. To alleviate this dilemma, we propose Dynamic Parallel Tree Search (DPTS), a novel parallelism framework that aims to dynamically optimize the reasoning path in inference. 
It includes the Parallelism Streamline in the generation phase to build up a flexible and adaptive parallelism with arbitrary paths by fine-grained cache management and alignment. 
Meanwhile, the Search and Transition Mechanism filters potential candidates to dynamically maintain the reasoning focus on more possible solutions and have less redundancy. Experiments on Qwen-2.5 and Llama-3 with Math500 and GSM8K datasets show that DPTS significantly improves efficiency by 2-4$\times$ on average while maintaining or even surpassing existing reasoning algorithms in accuracy, making ToT-based reasoning more scalable and computationally efficient. 
% Codes are provided in the submission.
% Tree of Thoughts~(ToT) enhances Large Language Model~(LLM) reasoning by structuring problem-solving as a spanning tree. However, recent methods focus on search accuracy while overlooked the computational efficiency. 
% The challenges of accelerating the ToT lies in the frequent switching of reasoning focus, and the redundant exploration on suboptimal solutions. 
% % its sequential data structure limits the utilization of GPU parallelism. In particular, existing algorithms typically cause redundant exploration and frequent switching issues, which becomes even severer when paralleled. 
% To alleviate this dilemma, we propose Dynamic Parallel Tree Search~(DPTS), a novel parallelism framework that aims to dynamically optimize reasoning path in inference. 
% % DPTS balances deep exploration and broad expansion through the search algorithm, and the bidirectional transition mechanism allows the tree to focus on high-confidence solutions. 
% % To this end, we design a flexible parallel framework to support arbitrary nodes with various path lengths for  simultaneous expanding. Meanwhile, the searching algorithm and bidirectional transition mechanism allows the trees to focus on high-confidence solutions with less redundant generation tokens. 
% It includes the Parallelism Streamline in generation phase to build up a finer-grained and flexible 
% % path arrangement, allowing a flexible 
% parallelism with arbitrary paths. Meanwhile, the Search and Transition Mechanism filters potential candidates 
% % with balanced exploitation and exploration during selection phase, which maintains 
% to maintain the reasoning focus on more possible solutions and have less redundancy. 
% Experiments on Qwen2.5 and Llama3 with Math500 and GSM8K datasets 
% % \ycj{xx} different models and datasets 
% show that DPTS significantly improves efficiency by $2\times$ on average while maintaining or even surpassing the existing reasoning algorithms in accuracy, making ToT-based reasoning more scalable and computationally efficient. 
% Our code can be found in the supplementary material.
% \ycj{Showcase your speed-up ratio}  % updated 
\end{abstract}


%%
%% The code below is generated by the tool at http://dl.acm.org/ccs.cfm.
%% Please copy and paste the code instead of the example below.
%%
\begin{CCSXML}
<ccs2012>
   <concept>
       <concept_id>10003120.10011738</concept_id>
       <concept_desc>Human-centered computing~Accessibility</concept_desc>
       <concept_significance>500</concept_significance>
       </concept>
   <concept>
       <concept_id>10003120.10011738.10011773</concept_id>
       <concept_desc>Human-centered computing~Empirical studies in accessibility</concept_desc>
       <concept_significance>300</concept_significance>
       </concept>
 </ccs2012>
\end{CCSXML}

\ccsdesc[500]{Human-centered computing~Accessibility}
\ccsdesc[300]{Human-centered computing~Empirical studies in accessibility}

%%
%% Keywords. The author(s) should pick words that accurately describe
%% the work being presented. Separate the keywords with commas.
\keywords{People with visual impairments (PVI); large multimodal models (LMMs), Human-AI interaction, visual question answering (VQA); remote sighted assistance (RSA), Be My Eyes, Be My AI.}

% Remote

%%
%% This command processes the author and affiliation and title
%% information and builds the first part of the formatted document.
\maketitle

%%%%%%%%%%%%%%%%%%%%%%%%%%%%%%%%%
\section{Introduction}
% Large Language Models~(LLMs) represent a transformative advancement in the field of language processing, demonstrating an unparalleled capacity for text generation and comprehension, which can be further applied in a wide variety of applications.  
% %Large language models (LLMs) have risen to prominence in various fields, offering endless possibilities for artificial intelligence applications. 
% Despite their significant prevalence in recent years, LLMs are frequently challenged with security and privacy issues, such as poor explainability~\cite{}, poor robustness~\cite{}, data dependency~\cite{}, etc. Among them, a specific and notable concern that has garnered increasing attention is the phenomenon of `hallucination', where models generate plausible but factually inaccurate or irrelevant content when employed for specific tasks such as problem-solving.  
% %In particular, the hallucination issue is when these large models are employed for problem-solving, users frequently voice concerns regarding being misled or deceived by the models' nonsensical and erratic outputs. 
% The tendency of these models to produce inaccurate outputs and fabricate facts has severely undermined the safety and usability of LLM applications, which calls for immediate attention in LLM research. 
% %Hallucination in large language models (LLMs) is a critical issue that needs immediate attention in LLM research. The tendency of these models to produce inaccurate outputs and fabricate facts has severely undermined the safety and usability of LLM applications. 
%exceptional 
%including limited explainability, compromised robustness, and a heavy reliance on data, each 
%However, d
Large Language Models (LLMs) have revolutionized language processing, demonstrating impressive text generation and comprehension capabilities with diverse applications. However, despite their growing use, LLMs face significant security and privacy challenges~\cite{siddiq2023generate, hou2023large, kaddour2023challenges, li2024model, 10.1145/3691620.3695510}, which affect their overall effectiveness and reliability. A critical issue is the phenomenon of \emph{hallucination}, where LLMs generate outputs that are coherent but factually incorrect or irrelevant. This tendency to produce misleading information compromises the safety and usability of LLM-based systems. This paper focuses on \emph{fact-conflicting hallucina}tion (FCH), the most prominent form of hallucination in LLMs. FCH occurs when LLMs generate content that directly contradicts established facts. For instance, as illustrated in \figref{fig:example1}, an LLM incorrectly asserts that ``\emph{Haruki Murakami won the Nobel Prize in Literature in 2016}'', whereas the fact is that ``\emph{Haruki Murakami has not won the Nobel Prize, though he has received numerous other literary awards}''. 
Such inaccuracies can significantly lead to user confusion and undermine the trust and reliability that are crucial for LLM applications.

% Large Language Models~(LLMs) have brought transformative advancements to language processing and beyond, showcasing text generation and comprehension abilities with wide-ranging applications. 
% Despite the increasing prevalence, LLMs face critical challenges in security and privacy aspects~\cite{siddiq2023generate, hou2023large, kaddour2023challenges}, heavily impacting their effectiveness and reliability. 
% One notable issue is the phenomenon of \emph{hallucination}, where LLMs produce coherent but factually inaccurate or irrelevant outputs during problem-solving. 
% Such a tendency to generate misleading information jeopardizes the safety and usability of LLM-based applications. 
% This paper concerns the \emph{fact-conflicting hallucination}~(FCH), which is the primary form of hallucinations in LLMs. 
% FCH occurs when LLMs generate content that directly contradicts the well-established facts, as exemplified in \figref{fig:example1}, where an LLM incorrectly believes 
% ``\emph{Haruki Murakami won the Nobel Prize in Literature in 2016}'', deviating from the fact that ``\emph{Haruki Murakami has not won the Nobel Prize but other numerous awards for his work in Literature}''. Such misinformation can cause significant user confusion and undermine the trust and reliability that are essential in various LLM applications. 

%correct answer of 

%is manifested by
%Such misinformation dissemination leads to significant user confusion, eroding the trust and reliability that are crucial in various LLM applications. 

%Large Language Models~(LLMs) represent a transformative advancement in the field of language processing, demonstrating an unparalleled capacity for text generation and comprehension, which can be further applied in a wide variety of applications. Despite their growing prevalence, LLMs encounter critical challenges, particularly in aspects of security and privacy. These include concerns such as limited explainability~\cite{}, compromised robustness~\cite{}, and heavy reliance on data~\cite{}, each posing distinct challenges to their efficacy and reliability. Among these, the phenomenon of ``hallucination'' stands out as a notable concern. This occurs when LLMs, while employed in tasks like problem-solving, generate outputs that are coherent yet factually inaccurate or irrelevant. Such a tendency to produce misleading information not only compromises the safety of LLM applications but also raises urgent questions regarding their usability. 

% Hallucinations in LLMs manifest in several distinct forms, each contributing differently to the challenges identified in their growing applications. 
% %The first, known as ``Input-conflicting hallucination'', arises when there is a discrepancy between the model's output and the user's initial input, reflecting a potential misunderstanding of the task at hand. On the other hand, ``Context-conflicting hallucination'' represents the second type, occurring when LLMs produce inconsistent responses in prolonged or multi-turn interactions, indicative of their limitations in maintaining coherent context. 
% Among the three types categorized in the literature~\cite{huang2023survey,zhang2023hallucination}, ``Fact-conflicting hallucination~(FCH)'' poses a particularly serious concern which is the primary focus of this paper. This phenomenon generates content in direct opposition to established factual knowledge. As illustrated in the example shown in Figure~\ref{fig:example1}, when an LLM was asked about the first person to walk on the moon, it incorrectly answered ``Charles Lindbergh in 1951'', a clear deviation from the factual answer of Neil Armstrong in 1969. This type of hallucination can lead to the dissemination of incorrect information and cause significant confusion among users, undermining the trust and reliability critical in various LLM applications. %Addressing fact-conflicting hallucinations is therefore essential for the advancement of LLMs, ensuring they not only function effectively but also responsibly in their diverse roles.


% According to \cite{huang2023survey} and \cite{zhang2023hallucination}, hallucinations in large language models can be categorized into types such as factual hallucinations and contextual hallucinations. Current benchmark assessments tend to focus on evaluating the propensity of LLMs to generate erroneous facts. The origin of these issues can be traced back to multiple deficiencies, including flaws in training data, training algorithms, and the inference process.

% \begin{figure}[t]
%     \centering
%     \includegraphics[width=0.95\linewidth]{fig/example1-cropped.pdf}\\
%     \caption{A Hallucination Output Example.}
%     %\vspace{-0.5cm}
%     \label{fig:example1}
% \end{figure}

\begin{figure}[t]
\centering
\vspace{3mm}
\hspace{-3mm}
\includegraphics[width=\linewidth]{fig/drowzee-example.pdf}
\\[0.5em]
\caption{A Hallucination Output Example}
\label{fig:example1}
\vspace{-4mm}
\end{figure}
%\lnk{Factual Hallucination and LLM inference current status}

Recent studies have introduced various methods to detect LLM hallucinations. A common approach involves developing specialized benchmarks, such as TruthfulQA~\cite{lin-etal-2022-truthfulqa}, HaluEval~\cite{HaluEval}, and KoLA~\cite{yu2023kola}, to assess hallucinations in tasks like question-answering, summarization, and knowledge graphs. 
While manually labeled datasets provide valuable insights, current methods often rely on simplistic or semi-automated techniques such as string matching, manual validation, or verification through another language model. These approaches reveal significant gaps in automatically and effectively detecting fact-conflicting hallucinations (FCH). 
The primary challenges in FCH detection arise from the lack of dedicated ground truth datasets, the absence of comprehensive test cases designed to trigger FCH, and the lack of a robust testing framework.  
Unlike other types of hallucinations, such as input-conflicting or context-conflicting hallucinations~\cite{ji-etal-2023-rho, shi2023large}, which can often be identified through semantic consistency checks, detecting FCH requires the verification of factual accuracy against external knowledge sources/databases. This process is particularly challenging and resource-intensive, especially for tasks that involve complex logical relationships~\cite{zhang2024fusion}. We identify three primary challenges in addressing this research gap:


% Recent studies have introduced a range of methods for detecting 
% hallucinations. One common approach involves creating comprehensive benchmarks tailored for this purpose. 
% Datasets such as TruthfulQA~\cite{lin-etal-2022-truthfulqa}, HaluEval~\cite{HaluEval}, and KoLA~\cite{yu2023kola} have been designed to evaluate hallucinations across different contexts, including question-answering, summarization, and knowledge graphs. 
% Despite the value of these manually labeled datasets, the current techniques heavily rely on naive and semi-automatic methods, such as string matching, manual validation, or utilizing another LLM for confirmation. 
% Therefore, there is a gap 
% in automatically and effectively testing FCHs, and the primary obstacle in testing FCH is the absence of dedicated ground truth datasets and an extensive testing framework.  
% Unlike other types of hallucinations, e.g., input-conflicting or context-conflicting 
% \cite{ji-etal-2023-rho, shi2023large}, 
% which can be identified through checks for semantic consistency, 
% detecting FCH
% requires the verification of the content's factual accuracy against external sources of knowledge or databases. This makes the process particularly arduous and resource-intensive, especially for tasks processing content with complex logical connections. 
% Here, we highlight three concrete challenges in filling up the identified research gap: 




%The main obstacle in testing for FCH is the absence of dedicated ground truth datasets and specific testing frameworks. Unlike other types of hallucinations~(e.g., input-conflicting and context-conflicting hallucinations, to be detailed in Section~\ref{subsec:cat}) which can be identified through checks for semantic consistency, FCH demands the verification of the content's factual accuracy against external sources of knowledge or databases. This requirement makes the process particularly challenging and resource-intensive, especially for tasks processing contents with inherent logical connections.

% \shil{(I feel the transition is not smooth, we first introducing datasets, and not explaining how they use these datasets to test llm. after these, we can state these methods are not automatic.)}


% To tackle FCH, recent works have developed various techniques for testing and detecting hallucination~\citep{yu2023kola,HaluEval}. The typical and intuitive solution is to develop comprehensive benchmarks for detection. This is done through a process of sampling, filtering, and enhancing ground-truth answers to identify the best and correct answers from given candidates. For example, a well-known hallucination evaluation benchmark HaluEval~\cite{HaluEval} constructs scenarios where LLMs are tested on their ability to select the most factually accurate answers from a set of provided options, with a focus on filtering out hallucinated responses. %\yi{ also talk about the construction of benchmark?}
% Additionally, human annotation plays a critical role in identifying hallucinations in LLM outputs~\cite{Alpaca}. This involves humans determining whether responses contain hallucinated information and considering aspects such as unverifiability, non-factuality, and irrelevance. 



% \lnk{Key challenge: lack of hallucination testing when faced with logic reasoning related problems}
%Bridging the identified research gap in the literature necessitates exploring the inherent challenges faced in detecting FCHs, which are crucial for advancing and enhancing the reliability of LLMs. 

\begin{enumerate}[itemsep=1mm, wide,  labelindent=9pt]
%[itemsep=0ex,leftmargin=0.35cm]
%Challenge\#1: 
%While these benchmarks effectively detect certain hallucinations, they 
\item {\textbf{Automatically constructing and updating benchmark datasets.}} Existing methodologies mainly rely on manually curated benchmarks for detecting specific hallucinations, which fail to encompass the broad and dynamic spectrum of fact-conflicting scenarios in LLMs. 
Meanwhile, due to the ever-evolving nature of knowledge, the need for frequent updates to benchmark data imposes a substantial and continuous maintenance effort.
The reliance on benchmark datasets thus restricts the FCH detection techniques' adaptability, scalability, and  %more importantly, 
detection capability;  
%Challenge\#2:
% in existing test cases. 
\item {\textbf{Efficiently generating FCH test cases.}}
LLMs often answer correctly to simple, straightforward questions due to their extensive training on vast datasets. However, to effectively assess their reasoning capabilities, it is important to generate more complex questions, such as those involving intricate temporal characteristics, that require reasoning rather than just recalling facts. However, constructing such test cases is non-trivial. The challenge lies in designing questions that use familiar knowledge but involve reasoning patterns the LLM may not have been explicitly trained on. Creating such test cases efficiently while ensuring they probe reasoning skills in ways the model has not previously encountered is essential to uncovering latent hallucinations;
% queries that involve temporal concepts, such as ``\emph{Does the human population finally reach six billion by the year 2000?}'' may often be used in applications. However, the correctness of the LLM outputs cannot be guaranteed, potentially leading to misleading information. Currently, there are no satisfactory approaches to automatically verify LLM outputs in such test cases; 
%errors even before the occurrence of large model hallucinations; 
%However, it is known that 
%Another critical issue lies in the verification of temporal logic in existing test cases. 
%It is well known that test cases involving temporal-related questions often face difficulties in automatically verifying the soundness and completeness of these issues. Consequently, the correctness of these test cases cannot be guaranteed, potentially introducing errors even before the occurrence of large model hallucinations;
%Challenge\#3: 
\item {\textbf{Validating the reasoning steps from LLM outputs.}} Even when LLMs finally produce correct answers, the outputs may not indicate an accurate reasoning process, potentially masking false understanding -- a source of FCH. Additionally, the quality of manual validation can differ based on human expertise. As a result, automatically validating reasoning processes, particularly those involving complex logical relationships, is inherently challenging. 
\vspace{1mm}
\end{enumerate}







% \lnk{Key challenge: factual knowledge exploring and new facts generation}
%\yi{we should focus on testing, addressing is a little bit vague.}
% The current research landscape in LLM presents a critical gap in automatically testing FCHs. Predominantly, existing methodologies are anchored to manual benchmarks. %\yi{this sentence is quite chinglish.}
% While these benchmarks are effective in detecting certain types of hallucinations, such as those in Figure~\ref{fig:example1}, they fall short in encompassing the broad and dynamic spectrum of fact-conflicting scenarios inherent to LLMs. %\yi{again, this sentence is not very clear}
% Meanwhile, the need for frequent updates to benchmark data, due to the ever-evolving nature of knowledge, imposes a significant and continuous maintenance effort.
% The reliance on benchmark datasets thus restricts the detection techniques’ adaptability, scalability, and worse, detection capability. 
% From a second perspective, the consistency in the quality of benchmark questions can vary, reflecting the differing levels of experience and skill among the human experts responsible for creating them. This is particularly reflected in the stages such as data labeling and results validation. Additionally, it is important to acknowledge that humans are prone to errors.
% %the scalability and the deof these existing methods are also significantly challenged by their dependency on extensive human intervention, particularly in stages such as data labeling and results validation. %This heavy reliance on manual efforts not only limits the scalability of such approaches but also questions their feasibility in efficiently handling the extensive and intricate datasets characteristic of LLMs.
% Thus, the development of more autonomous, agile, and scalable testing techniques is imperative to effectively identify and tackle FCHs in LLMs.%\yi{in this paper, we focus on testing, but until this paragraph, no terms about ``testing'' explicitly occur.}

% \lnk{Solution to Challenge1: comprehensive logic reasoning based testing framework}

% \lnk{Solution to Challenge2: wiki factual knowledge extraction and prolog rules inference for scalability.}
% \lnk{Key challenge: }

%\textbf{Our Work.}
%To address limitations in the existing techniques, 
%we are the first, to the best of our knowledge, to introduce 
To address the problems outlined above, this paper presents a novel automatic end-to-end metamorphic testing technique based on temporal logic for detecting FCH. To the best of our knowledge, we are the first to create a comprehensive FCH testing framework that utilizes factual knowledge reasoning and metamorphic testing, all seamlessly integrated into the fully automated tool, \tool. 

%\shil{(which four methods?)}
\tool begins by establishing a comprehensive factual knowledge base sourced through crawling information from accessible knowledge bases such as Wikipedia. Each piece of this knowledge acts as a ``seed'' for subsequent transformations. Leveraging logical operators to automatically generate temporal reasoning rules, we transform and augment these seeds and expand factual knowledge into a well-established set of question-answer pairs.
%\yi{into xx}. 
Using the questions and answers in the knowledge set as test cases and ground truth, respectively, we construct a reliable and robust FCH testing benchmark. 


The experiment uses a series of carefully designed template-based prompts to test for FCHs in LLMs. To thoroughly evaluate the reasoning behind the responses, we instruct the LLMs not only to generate answers to the test cases but also to provide detailed justifications for their answers. To reliably identify FCH, we introduce two semantic-aware, similarity-based metamorphic oracles. These oracles extract the key semantic elements from each sentence and map out the logical relationships between them. By comparing the logical and semantic structures of the LLM's responses with the ground truth, the oracles can detect FCH by identifying significant deviations in the LLM's answers from the correct information.




%well-crafted prompts\yi{how prompts generated?} to engage LLMs, testing the alignment of their generated content with our enhanced ground truth. Disparities between LLM outputs and the ground truth signal potential hallucinations. 
%Additionally, in our commitment to fostering collaborative research, we have released our constructed dataset as a benchmark~\cite{drowzee}.

%Our approach directly addresses the need for a comprehensive and flexible testing method by transforming structural factual data into a diverse range of scenarios that LLMs may encounter. This method not only improves the reliability of detection but also enhances its adaptability to various factual contexts.
%Furthermore, we address the scalability challenge by automating the transformation and enlargement of our knowledge base, significantly reducing the dependency on human effort. The well-designed prompts used to test LLMs further streamline the process, making it more efficient in identifying potential hallucinations by comparing LLM outputs with our extended ground truth.

%\textbf{Results and Findings.}
%In evaluating our proposed FCH testing framework and \tool, 
%we undertake 
%to evaluate their effectiveness 
We demonstrate the effectiveness of our approach through comprehensive experiments in multiple contexts. First, our evaluation involves deploying \tool across a wide range of topics drawn from a diverse selection of Wikipedia articles. Second, we test our framework on various open-source and commercial LLMs, thoroughly assessing its applicability and performance across different model architectures. 
Our key findings indicate that \tool succeeds in automatically generating practical test cases and identifying hallucination issues of nine LLMs across nine domains. 
Using these test sets, our experiments show that the rate of hallucination responses produced by various LLMs ranges from 24.7\% to 59.8\% for cases unrelated to temporal reasoning and 16.7\% to 39.2\% for cases requiring temporal reasoning. 
%\shil{shall we differentiate the number for non-temporal and temporal one?}.  
We then categorize these hallucination responses into \emph{erroneous knowledge hallucination} and \emph{erroneous inference hallucination}. 
%\syh{four types?}. 
Through an in-depth analysis, we unveil that the lack of logical reasoning capabilities contributes the most to the FCH issues in LLMs. 
Additionally, we observe that LLMs are particularly prone to generating hallucinations in test cases involving temporal concepts and out-of-distribution knowledge. 
Such an evaluation demonstrates that the 
%Furthermore, we confirm that 
test cases generated using %our 
logical reasoning rules can effectively trigger and detect LLM hallucinations.  %issues in . 


This paper builds upon the earlier version~\cite{DBLP:journals/pacmpl/LiL0SW024} by incorporating hallucination detection through temporal-logic-guided test generation. It includes additional motivational examples (\secref{sec:motivating}), a comprehensive set of reasoning rules for encoding \emph{Metric Temporal Logic} (MTL)~\cite{DBLP:conf/lics/OuaknineW05} formulae (\secref{sec:encoding_MTL}) and automatically generating temporal-logic-related question-answer pairs (\secref{prompt}), and further experimental studies (the {RQ4} at \secref{sec:eval}) that detect hallucinations due to insufficient temporal reasoning capabilities. The main contributions of this work are summarized as follows: 
%We summarize the main contributions of this paper below:
\begin{itemize}[itemsep=1mm,leftmargin=0.35cm]
\item 
%Development of 
\textbf{A novel FCH testing framework.} 
To the best of our knowledge, 
we are the first to develop a novel testing framework based on logic programming and metamorphic testing to automatically detect FCH issues in LLMs. %\yi{hanging sentence}This framework represents a significant advancement over current methodologies, providing a more systematic, comprehensive approach to detection.
%Construction and Release of
\item \textbf{An extensive benchmark based on factual knowledge.} 
To facilitate collaborative efforts and future advances in identifying FCH, 
the source code of \tool and constructed benchmark dataset are publicly available  \cite{drowzee}. 
\item \textbf{Test generation via temporal reasoning.} 
Our tool automatically generates test cases that provide a more comprehensive evaluation of LLMs in handling reasoning tasks and identifying factual inconsistencies. By applying temporal logic-based reasoning rules, we expand the initial seed data from our knowledge base, enhancing the diversity and complexity of the test scenarios. 

\item \textbf{Semantic-aware oracles for LLM answer validation.} We propose and implement two automated verification mechanisms, i.e., the oracles, that analyze the semantic structure similarity between sentences. These oracles are designed to validate the reasoning logic behind the answers generated by LLMs, hereby reliably detecting the occurrence of FCHs. 

\end{itemize}



\section{Background and Related Work}


% intro to DCog: collaborative, human-powered prosthetic are DCog (RSA)
% adaptive artifact that is human-like, intelligent, is also DCog (bma)

In this section, we review the literature on AI-powered and human-assisted visual interpretation and question answering systems, as well as information needed in visual interpretations for PVI.


\subsection{AI-Powered Visual Interpretation and Question Answering Systems}


The advancements in deep learning models for computer vision and natural language processing have significantly enhanced the capabilities of AI-powered visual assistive systems\rev{~\cite{ahmetovic2020recog, mukhiddinov2022automatic}}. These systems leverage photos taken by PVI to identify objects or text within the scene, as well as respond to queries about the contents of the images\rev{~\cite{hong2022blind, morrison2023understanding}. Such applications are now widely adopted, exemplified by Microsoft's Seeing AI~\cite{SeeingAI2020}, providing support to PVI in various scenarios~\cite{gonzalez2024investigating}.}

% \sout{The advancements in deep learning models for CV and NLP technologies have significantly enhanced the capabilities of AI-powered visual assistive systems. These systems leverage photos taken by PVI to identify objects or text within the scene, as well as respond to queries about the contents of the images. For instance, Ahmetovic et al.~\cite{ahmetovic2020recog} developed a mobile app utilizing deep network models to guide PVI in capturing images and recognizing objects. Mukhiddinov et al.~\cite{mukhiddinov2022automatic} devised a fire detection and notification app for PVI using convolutional neural networks. Hong et al.~\cite{hong2022blind} created an iOS app enabling PVI to gather training images for personalized object recognition. Morrison et al.~\cite{morrison2023understanding} designed an application to teach AI to identify individualized items, aiding PVI in locating personal belongings. Gonzalez et al.~\cite{gonzalez2024investigating} introduced a scene description app utilizing Microsoft's Azure AI Vision image description API. Additionally, PVI commonly employ commercial AI-powered applications like Microsoft's Seeing AI~\cite{SeeingAI2020} to recognize objects through device cameras.}

Recent advancements in LMMs, such as GPT-4~\cite{achiam2023gpt}, have demonstrated exceptional performance in multimodal tasks~\cite{yu2023mm}, \rev{prompting exploration into their potential for assisting PVI. State-of-the-art LMMs have been leveraged to create assistive systems capable of evaluating image quality, suggesting retakes, answering queries about captured images~\cite{zhao2024vialm, yang2024viassist}, and even integrate multiple functions to assist PVI in tasks such as navigation~\cite{zhang2025enhancing} and text input~\cite{10.1145/3613904.3642939}. In the commercial domain, Be My Eyes~\cite{BeMyEyes2020} introduced the \bma{} feature powered by GPT-4~\cite{achiam2023gpt}.}

% \sout{More recently, Large Multimodal Model (LMM)~\cite{yu2023mm}, represented by GPT-4~\cite{achiam2023gpt}, have showcased remarkable proficiency in multimodal tasks. In response, researchers have begun exploring the potential of LMM in assisting PVI. Zhao et al.~\cite{zhao2024vialm} investigated the feasibility of leveraging state-of-the-art LMM to aid PVI and established a corresponding benchmark. Yang et al.~\cite{yang2024viassist} utilized LMM to develop an assistive system for answering PVI's questions about captured images, including assessing picture quality and suggesting retakes. In the realm of commercial applications, Be My Eyes~\cite{BeMyEyes2020} and OpenAI collaborated to introduce the \bma{} feature~\cite{bma_usecase}, powered by GPT-4~\cite{achiam2023gpt}, aimed at replacing human volunteers.}

% \rev{While prior research has examined the use of AI-powered visual assistive systems like Seeing AI for tasks such as reading~\cite{granquist2021evaluation} and daily routine support~\cite{kupferstein2020understanding}, these studies predate the integration of advanced LMM technologies and thus may not fully reflect the capabilities or user experience of cutting-edge systems. Although some accounts, such as Bendel’s documentation of experiences with GPT-4-based \bma{}~\cite{bendel2024can}, offer insights, they remain anecdotal and lack generalizability. This study aims to bridge this gap by conducting an in-depth investigation into the daily usage of \bma{} among 14 PVI users, offering a comprehensive understanding of its real-world applications and impact.}


Prior work has examined the use of AI-powered visual assistive systems by PVI in their daily lives\rev{~\cite{granquist2021evaluation,kupferstein2020understanding,gonzalez2024investigating}}. However, these studies did not incorporate the latest LMM technologies, thus may not fully represent the user experience of cutting-edge AI-powered systems. Bendel~\cite{bendel2024can} documented his experience with GPT\mbox{-}4\mbox{-}based \bma, yet his account remains subjective. Therefore, in-depth investigation into PVI's daily utilization of state-of-the-art AI-powered systems is imperative. 
This paper addresses this gap by exploring how 14 visually impaired users incorporate \bma, a state-of-the-art LMM-based VQA system, into their daily routines. 


% Prior works have examined the use of AI-powered visual assistive systems by PVI in their in their daily lives. For instance, Granquist et al.~\cite{granquist2021evaluation} examined the use of Seeing AI for reading tasks among PVI participants, while Kupferstein et al.~\cite{kupferstein2020understanding} conducted a diary study to investigate PVI users' incorporation of Seeing AI into their daily routines. Recently, Gonzalez et al.~\cite{gonzalez2024investigating} undertook a diary study to explore the use cases of AI-powered scene description applications. 
% However, these studies did not incorporate the latest LMM technologies, thus may not fully represent the user experience of cutting-edge AI-powered systems. Bendel~\cite{bendel2024can} documented his experience with GPT-4-based \bma{}, yet his account remains subjective. Therefore, extensive and comprehensive research into PVI's daily utilization of state-of-the-art AI-powered systems is imperative. This paper addresses this gap by investigating the daily usage of \bma{} among fourteen PVI users.


% adnin2024look
% \textcolor{red}{
% Based on prior work~\cite{adnin2024look}, this study goes beyond usability and accessibility issues to identify previously unexplored limitations of LMMs. We explore higher-level insights that inform potential future directions of LMMs even if the specific implementations evolve, and map these insights into broader Human-AI and AI-AI interactions. 
% }

% https://maitraye.github.io/files/papers/BLV_GenAI_ASSETS24.pdf





\subsection{Human-Assisted Visual Interpretation and Question Answering Systems}

Human-assisted VQA systems offer prosthetic support for PVI by facilitating connections to sighted people through remote assistance. These systems utilize image-based and video-based modalities to \rev{meet varied} needs and situations.


Image-based human-assisted VQA systems allow PVI to submit photos along with their queries and receive responses after some time. An example of this is Vizwiz, where PVI can upload images accompanied by audio-recorded questions and receive text-based answers through crowdsourced assistance~\cite{bigham2010vizwiz_nearly}. This method has been successfully applied in tasks such as reading text, identifying colors, locating objects, obtaining fashion advice, \rev{and supporting social interactions}~\cite{bigham2010vizwiz_nearly, bigham2010vizwiz, burton2012crowdsourcing,gurari2018vizwiz}. 
% Additionally, Vizwiz Social extends this functionality by linking PVI with their social networks, enabling support from friends and family members~\cite{gurari2018vizwiz}. 
However, the single-photo, single-query limitation of image-based VQA systems~\cite{bigham2010vizwiz_nearly} makes it less suitable for addressing complex or contextually deep inquiries~\cite{lasecki2013answering}.



Conversely, video-based human-assisted VQA systems facilitate real-time, interactive support, allowing PVI to receive immediate assistance tailored to their specific environmental context. This approach enables the visual interpretation of real-time scenes and supports a dynamic, back-and-forth VQA process, which is essential for addressing more specific and complex contextual inquiries effectively.
The evolution of this technology has progressed from wearable digital cameras ~\cite{hunaiti2006remote,garaj2003system,baranski2015field} and webcams~\cite{bujacz2008remote,scheggi2014remote,chaudary2017tele} to the utilization of mobile video applications~\cite{holmes2015iphone, BeMyEyes2020, Aira2020,xie2022dis,xie2024bubblecam} and real-time screen sharing technologies~\cite{lasecki2011real,lasecki2013answering,xie2023two}. 
% 
Services like Be My Eyes~\cite{BeMyEyes2020}, which connects PVI with untrained volunteers, and Aira~\cite{Aira2020}, which connects PVI with trained professional assistants, exemplify the application of video-based VQA in scenarios that require immediate feedback. These services prove effective in navigation~\cite{kamikubo2020support,xie2022dis,c4vtochi,iui,yu2024human}, 
shopping~\cite{c4vtochi,xie2023two,iui}, 
and social interaction~\cite{lee2020emerging,Caroll2020Human,lee2018conversations}. 
\rev{
In this study, we revealed specific situations where participants used human-assisted VQA systems to address limitations in \bma's assistance.
}





\subsection{Information Needed in Visual Interpretations for Visually Impaired Users}

% Either AI-powered or human-assisted visual interpretation faces the same challenge: determining what specific information is necessary for visual interpretation to meet the needs of PVI. Existing research primarily focuses on identifying what information PVI seek in descriptions of online images. For instance, the Web Content Accessibility Guidelines~\cite{caldwell2008web} offer advice on creating alternative text (alt text) for images. However, these guidelines tend to follow a one-size-fits-all approach, which can be difficult to apply effectively across different contexts. Similarly, AI-based image description tools (e.g., Seeing AI~\cite{SeeingAI2020}) also adhere to this one-size-fits-all design model.

% Previous studies~\cite{gurari2018vizwiz, gurari2020captioning} have adopted sighted volunteers to decide how images should be described for PVI. Recently, more research~\cite{stangl2020person,stangl2021going,bennett2021s} has shifted toward user-centered approaches, gathering input directly from PVI to understand their preferences for image descriptions. For example, Stangl et al.~\cite{stangl2020person} explored PVI users' preferences for descriptions across different online platforms (e.g., news sites, social media, eCommerce). Their findings indicated that the source of an image significantly impacts PVI’s information needs, with common requirements across platforms (e.g., identifying a person’s gender or naming objects), as well as source-specific details (e.g., a person’s hair color on dating websites). In a follow-up study~\cite{stangl2021going}, they discovered that PVI's image description preferences are influenced not only by the image source but also by their information goals. Additionally, while subjective interpretation may be desired in certain cases, such content is not typically included in descriptions. From an ethical perspective, Bennett et al.~\cite{bennett2021s} examined screen reader users' preferences for describing personal and others' appearances, specifically addressing when and how aspects such as race, gender, and disability should be conveyed.

% Recent research also explores the varied needs for describing information from videos. For instance, Jiang et al.~\cite{jiang2024s} investigated PVI preferences for video descriptions across different types of online content (e.g., how-to videos, educational videos, music videos). Based on their proposed design framework, they offered several recommendations for improving video accessibility. For example, descriptions for entertainment videos, like music videos, should focus more on subjects (people and animals), while how-to videos should prioritize actions and tools over the subjects. Natalie et al.~\cite{natalie2024audio} conducted interviews with PVI to better understand their customization needs for video descriptions in various aspects such as length, emphasis, speed, and voice. 

% All of these studies focus on asynchronous visual interpretation systems. When it comes to synchronous systems, such as real-time video interpretation, Lee et al.~\cite{lee2020emerging} identified specific needs for scene descriptions in remote sighted assistance (RSA). The required information in RSA descriptions includes both objective details (e.g., text, spatial information, scenery) and subjective aspects (e.g., opinions on clothing). In this study, we will examine how Be My AI processes visual information and provide design insights based on the information needed by PVI.

\rev{Both} AI-powered \rev{and} human-assisted visual interpretation face the challenge \rev{of identifying what specific information is necessary to meet PVI needs}. Existing research primarily focuses on \rev{what information PVI seek in online image descriptions}. The Web Content Accessibility Guidelines~\cite{caldwell2008web} \rev{advise on creating alternative text (alt text), but their one-size-fits-all approach limits context-specific applicability}. Similarly, AI-based \rev{tools like Seeing AI~\cite{SeeingAI2020} adhere to this uniform design model.}

\rev{Earlier studies~\cite{gurari2018vizwiz, gurari2020captioning} relied on} sighted volunteers \rev{to decide image descriptions for PVI, while recent research~\cite{stangl2020person,stangl2021going,bennett2021s,chen2024role} emphasizes} user-centered approaches. For example, Stangl et al.\cite{stangl2020person} \rev{found PVI preferences for descriptions vary by platform} (e.g., news, social media, eCommerce)\rev{, with both common needs (e.g., identifying gender, naming objects) and platform-specific details (e.g.,} hair color on dating websites). \rev{Follow-up work~\cite{stangl2021going} showed preferences also depend on users' goals, with subjective details rarely included.} Bennett et al.~\cite{bennett2021s} examined screen reader users' \rev{views on describing} race, gender, and disability \rev{in appearance descriptions.}
\rev{For video content,} Jiang et al.~\cite{jiang2024s} \rev{examined PVI preferences across genres (e.g., how-to, music videos), recommending that entertainment descriptions focus on subjects, while how-to videos prioritize actions and tools.} Natalie et al.~\cite{natalie2024audio} \rev{studied PVI needs for customization in video descriptions, such as length, speed, and voice. For synchronous systems like real-time video interpretation,} Lee et al.~\cite{lee2020emerging} identified \rev{PVI needs} in remote sighted assistance\rev{, including} both objective details (e.g., text, spatial information\rev{) and subjective input} (e.g., opinions on clothing).
In this study, we will examine how the LMM-based system processes visual information and provide design insights based on the information needed by PVI.



\rev{
% image Interpretations: 
% [1] table 2 in https://dl.acm.org/doi/pdf/10.1145/3313831.3376404?casa_token=cZV0Jj4xOEQAAAAA:DUGrnBq-QrkQdqRKALU7MBuhin-liR-tyGsr0jEUvcznLAMBTFTTPbocpCy2Em-Vqz0rYs5sTLlo
% [2] https://dl.acm.org/doi/pdf/10.1145/3441852.3471233?casa_token=4gFpHDQlYvUAAAAA:hWjcbkbuhYmJ-3x6EaNoewOQbIXG93mzA7w1agDMzwKhimO_ycUgKxl-k5_2ntWpFLZ_71W3PgOf
% [3] https://dl.acm.org/doi/pdf/10.1145/3411764.3445498?casa_token=TB9lu5AmUfsAAAAA:Hx9BiXQde88FZn_oDlHEsPKTnXDae4z5FYINoFrKG43L7Qg1M3NYnXTizwQe8M5ade9Gf2Ipw_cm

% video Interpretations: 
% [1] sooyeon's paper (used in discussion 5.2)
% https://dl.acm.org/doi/pdf/10.1145/3313831.3376591?casa_token=ATq0EXcGE2oAAAAA:pgbAj9B_z6skIcxG3IaBpFYLBQBCLpq85ZnuXoP6jb2LMFNAgXVCn0W-VFU__84OYIkWF0vSZHg1

% incremental in belowing contexts
% 1. scene description: description (what is it)
% 2. navigation: description (what is it) + directional info (where is it, how to get to the destination)
% 3. task performance: description + directional info + domain knowledge (what to do, how to do it)
% 4. social engagement:  description + directional info to navigate in social space + discreet info (don't want others to know they are using rsa)


% [2] table 4 in https://dl.acm.org/doi/pdf/10.1145/3313831.3376823

}












% 加一节,怎么区分确定性和不确定性

\section{Methodology}


To achieve effective probabilistic predictions, we propose CoST that simultaneously leverages the advantages of both deterministic and probabilistic models. Our approach involves two stages. In the first stage, the deterministic model is pretrained to predict the conditional mean that captures the primary patterns. In the second stage, the parameters of the deterministic model are frozen, and the scale-aware diffusion model, constrained by a customized fluctuation scale, is jointly trained to model residual distributions that reflect random fluctuations.   
Figure~\ref{fig:model} illustrates an overview of our model.


\subsection{Mean-Residual Decomposition}

For urban spatiotemporal probabilistic prediction, current approaches typically employ a single probabilistic model to capture the full distribution of data, incorporating both the primary spatiotemporal patterns and the random fluctuations. However, it is challenging to model both of these distributions. Inspired by~\cite{mardani2023residual} and the Reynolds decomposition in fluid dynamics~\cite{pope2001turbulent}, we propose to decompose the target data \(\mathbf{x}^{ta}\) as follows:
\begin{equation}
 \mathbf{x}^{ta} = \underbrace{\mathbb{E}[\mathbf{x}^{ta}|\mathbf{x}^{co}]}_{\substack{:=\boldsymbol{\mu}(Deterministic)}} + \underbrace{(\mathbf{x}^{ta}-\mathbb{E}[\mathbf{x}^{ta}|\mathbf{x}^{co}])}_{\substack{:=\mathbf{r}(Probabilistic)}},
\end{equation}
where \(\boldsymbol{\mu}\) is the conditional mean representing the primary patterns, and \(\mathbf{r}\) is the residual representing the random variations. Our core idea is that if a deterministic model can accurately predict the conditional mean, that is, \(\boldsymbol{\mu}\approx\mathbb{E}_{\theta}[\mathbf{x}^{ta}|\mathbf{x}]\), then the probabilistic model only needs to focus on learning the simpler residual distribution, thus combining the strengths of both models to enhance the probabilistic prediction capability.









\subsection{Mean Prediction via Deterministic Model}

We require a deterministic model that can accurately predict the conditional mean to align with our research hypothesis, and also maintain high predictive efficiency to avoid additional increases in training and inference time. Therefore, we select the MLP-based STID model as our mean prediction module.
In the first stage of training, we pretrain the model for 50 epochs to effectively capture the primary spatiotemporal patterns. Specifically, we input historical conditional data \(\mathbf{x}^{co}\) into the STID model to obtain the conditional mean output \(\mathbb{E}_{\theta}[\mathbf{x}^{ta}|\mathbf{x}^{co}]\).

The STID model is pretrained by optimizing the following loss function:

\begin{equation}
\label{eq:loss2}
   \mathcal{L}_{2}  = \left\| \mathbb{E}_{\theta}[\mathbf{x}^{ta}|\mathbf{x}^{co}] - \mathbf{x}^{ta} \right\|_2^2 .
\end{equation}

\subsection{Residual Learning via Diffusion Model}
Diffusion models have achieved significant success in probabilistic modeling. In this work, we employ a diffusion model for probabilistic prediction, training it to learn the residual distribution:
\begin{equation}
\label{eq:one-setp-forward}
    \mathbf{r}_{ta}=\mathbf{x}^{ta}-\mathbb{E}_{\theta}[\mathbf{x}^{ta}|\mathbf{x}^{co}].
\end{equation}
Consequently, the target data \(\mathbf{x}^{ta}\) for diffusion models in Eqs.~\eqref{eq:one-setp-forward}, \eqref{eq:inference}, and \eqref{eq:loss1} is replaced by \(\mathbf{r}_{ta}\).
The residual distribution of urban spatiotemporal data is not independently and identically distributed (i.i.d.) nor does it follow a fixed distribution, such as \(\mathcal{N}(0, \mathbf{\sigma})\). Instead, it often exhibits complex spatiotemporal dependence and heterogeneity. So we consider both temporal residual learning and spatial residual learning. 




\subsubsection*{\textbf{Temporal Residual Learning.}} 
For urban spatiotemporal data, the residual distribution varies at different time points. For example, fluctuations are lower at night and higher during the day, with uncertainty varying between weekends and weekdays. To model this, we incorporate the timestamp information as the condition for the denoising process. Meanwhile, the residual distribution can also be affected by sudden weather changes or public events. To capture these real-time changes, we concatenate the context data $\mathbf{x}^{co}_0$ with noised target residual $\mathbf{r}^{ta}_n$ as the input. The noise is not added to $\mathbf{x}^{co}_0$ and $\mathbf{r}^{ta}_n$ during the diffusion training and inference.




\subsubsection*{\textbf{Spatial Residual Learning.}}
In areas with frequent traffic accidents, fluctuations tend to be more pronounced and may induce anomalous variations in adjacent regions, thus affecting their distributions.
For spatial dependence modeling, the model learns a spatial embedding for each location, following the STID approach. Additionally, we propose a scale-aware diffusion process to further distinguish the heterogeneity for different regions. In this section, we detail the calculation of \(\mathbf{Q}\) and how it is integrated into the scale-aware diffusion process.

\noindent\textbf{(i) Customized Fluctuation Scale.} Specifically, we apply the Fast Fourier Transform (FFT) to spatiotemporal sequences in the training set to quantify fluctuation levels in different regions and use the custom scale \(\mathbf{Q}\) as input to account for spatial differences in residual. Specifically, we first employ FFT to extract the fluctuation components for each region within the training set. The detailed steps are as follows:









\begin{equation}
    \begin{aligned}
    & \mathbf{A}_{\mathrm{k}} = \left| \text{FFT}(\mathbf{x})_\mathrm{k} \right|, \quad \mathbf{{\phi}}_{\mathrm{k}} = \mathbf{\phi} \left( \text{FFT}(\mathbf{x})_\mathrm{k} \right), \\
    & \mathbf{A}_{\text{max}}=\max_{\mathrm{k}\in\left\{1,\cdots,\left\lfloor\frac{\mathbf{L}}{2}\right\rfloor + 1\right\}}\mathbf{A}_{\mathrm{k}}, \\
    & \mathcal{K} = \left\{ \mathrm{k} \in \left\{ 1, \cdots, \left\lfloor \frac{{L}}{2} \right\rfloor + 1 \right\} : \mathbf{A}_{\mathrm{k}} < 0.1 \times \mathbf{A}_{\text{max}} \right\}, \\
    & \mathbf{x}_{\mathbf{r}}[i] = \sum_{\mathrm{k} \in \mathcal{K}} \mathbf{A}_{\mathrm{k}} \Big[ \cos \left( 2\pi \mathbf{f}_{\mathrm{k}} i + \mathbf{\phi}_{\mathrm{k}} \right) \\
    & \qquad \qquad + \cos \left( 2\pi \bar{\mathbf{f}}_{\mathrm{k}} i + \bar{\mathbf{\phi}}_{\mathrm{k}} \right) \Big],
    \end{aligned}
\end{equation}
where \(\mathbf{A}_{\mathrm{k}},\mathbf{\phi}_{\mathrm{k}}\) reprent the amplitude and phase of the $\mathrm{k}-$th frequency component. $L$ is the temporal length of the training set. \(\mathbf{A}_{\text{max}}\) is the maximum amplitude among the components, obtained using the $\max$ operator. $\mathcal{K}$ represents the set of indices for the selected residual components. \(\mathbf{f}_{\mathrm{k}}\) is the frequency of the \(\mathrm{k}\)-th component. $\bar{\mathbf{f}}_{\mathrm{k}}, \bar{\mathbf{\phi}}_{\mathrm{k}}$ represent the conjugate components. \(\mathbf{x}_{\mathbf{r}}\) ref to the extracted residual component of the training set. We then compute the variance $\sigma^2_k$ of the residual sequence for each location $k$ and expand it to match the shape as 
\(\mathbf{r}^{ta}_0 \in \mathbb{R}^{B \times K \times P}\) , where $B$ represents the batch size. And we can get the variance tensor \(\mathcal{M}\): 
\begin{equation}
\begin{aligned}
    &\mathcal{M}_{b,k,p}=\sigma_{k}^2,\\
&\forall b\in\{1,\cdots,B\}, \forall k\in\{1,\cdots,K\}, \forall p\in\{1,\cdots,P\}.
\end{aligned}
\end{equation}
The residual fluctuations are bidirectional, encompassing both positive and negative variations, so we generate a random sign tensor \(\mathbf{S}\in\mathbb{R}^{B\times K\times P}\) for \(\mathcal{M}\), where each element \(S_{b,k,p}\) of \(\mathbf{S}\) is sampled from a Bernoulli distribution with \(p = 0.5\). 
%That is, \(r_{b,k,p}\) takes the value $1$ with probability $0.5$ and $-1$ with probability $0.5$. 
The customized fluctuation scale \(\mathbf{Q}\) is then defined as:
\begin{equation}
\begin{aligned}
&\mathbf{Q}_{b,k,p}=S_{b,k,p}\times\mathcal{M}_{b,k,p},\\
&\forall b\in\{1,\cdots,B\}, \forall k\in\{1,\cdots,K\}, \forall p\in\{1,\cdots,P\}.
\end{aligned}
\end{equation}
Then \(\mathbf{Q}\) is used as the input of the denoising network. 





\noindent\textbf{(ii) Scale-aware Diffusion Process.}

The vanilla diffusion models typically model all regions as the same \(\mathcal{N}(0, I)\) distribution at the end of the diffusion process, failing to distinguish the differences among regions. To further model the differences of residual distribution across various regions, we adopt the technique proposed by~\cite{han2022card} to make the residual learning region-specific conditioned on \({\mathbf{Q}}\). Specially, we have calculated the customized fluctuation scale \({\mathbf{Q}}\), and We redefined the noise distribution at the endpoint of the diffusion process as follows:
\begin{equation}
    p(\mathbf{r}^{ta}_N)=\mathcal{N}({\mathbf{Q}},I),
\end{equation}
Accordingly, the Eq~\ref{eq:new one-step} in the forward process is rewritten as:
\begin{equation}
\label{eq:new one-step}
    \mathbf{r}_n^{ta} = \sqrt{\bar{\alpha}_n} \mathbf{r}_0^{ta}+(1-\sqrt{\bar{\alpha}_n})\mathbf{Q} + \sqrt{1 - \bar{\alpha}_n} \mathbf{\epsilon}, \quad \mathbf{\epsilon} \sim \mathcal{N}(0, I).
\end{equation}
And in the denoising process, we sample \(\mathbf{r}_N^{ta}\) from $\mathcal{N}({\mathbf{Q}},I)$, and denoise it use Eq~(\ref{eq:inference}), the computation of \(\mu_{\theta}(\mathbf{r}_n^{ta}, n| \mathbf{x}_0^{co})\) in Eq~\ref{eq:inference} is modified as:
\begin{equation}
\label{eq: mu}
    \mu_{\theta}(\mathbf{r}_n^{ta}, n| \mathbf{x}_0^{co})=\frac{1}{\sqrt{\bar{\alpha}_n}} \left( \mathbf{r}_n^{ta} - \frac{\beta_n}{\sqrt{1 - \bar{\alpha}_n}} \mathbf{\epsilon}_{\theta}(\mathbf{r}_n^{ta}, n| \mathbf{x}_0^{co}) \right)+(1-\frac{1}{\sqrt{\bar{\alpha}_n}})\mathbf{Q}.
\end{equation}
This approach enables the diffusion process to be governed by the \(\mathbf{Q}\) values at each region, leading to more effective utilization of the customized scale conditions.


\subsection{Training and Inference}
\begin{algorithm}
\caption{\methodname{} Training}
\KwIn{Coarse-to-fine Autoencoder $\text{Enc}$, $\text{Dec}$}
\KwOut{}
\For{$i \gets 1$ \textbf{to} $n-1$}{
    \For{$j \gets 1$ \textbf{to} $n-i$}{
        \If{$L[j] > L[j+1]$}{
            Swap $L[j]$ and $L[j+1]$
        }
    }
}
\Return $L$
\end{algorithm}
\begin{algorithm}[!t]
\caption{Inference}
\label{al: sampling}
\begin{algorithmic}[1]
    \State \textbf{Input:} Context data $\mathbf{x}_0^{co}$, customized fluctuation scale $\mathbf{Q}$, trained diffusion model $\epsilon_{\theta}$, trained deterministic model $\mathbb{E}_{\theta}$
    \State \textbf{Output:} Target data $\mathbf{x}_0^{ta}$
    \State Estimate the conditional mean \(\mathbb{E}_{\theta}[\mathbf{x}^{ta}_0|\mathbf{x}^{co}_0]\)
    \State Sample $\mathbf{r}_N^{ta}$ from $\epsilon \sim \mathcal{N}(\mathbf{S},I)$
    \For{$n = N$ to $1$} 
        \State Estimate the noise $\mathbf{\epsilon}_{\theta}(\mathbf{r}_n^{ta}, n| \mathbf{x}_0^{co})$
        \State Calculate the $\mu_{\theta}(\mathbf{r}_n^{ta}, n| \mathbf{x}_0^{co})$ using Eq.~(\ref{eq: mu})
        \State Sample $\mathbf{r}_{n-1}^{ta}$ using Eq.~(\ref{eq:inference})
    \EndFor
    \State \textbf{Return:} $\mathbf{x}_0^{ta}=\mathbb{E}_{\theta}[\mathbf{x}^{ta}_0|\mathbf{x}^{co}_0]+\mathbf{r}_0^{ta}$
\end{algorithmic}

\end{algorithm}

\subsubsection*{\textbf{Training}}
Our training process is a two-stage procedure. We first pretrain the deterministic model STID to enable it to predict the conditional mean. Subsequently, we train the diffusion mode to learn the distribution of residuals, where the residuals are calculated as the difference between the true values and the conditional mean predicted by the pretrained STID model with frozen parameters. The detailed training procedure is presented in Algorithm~\ref{al: train}.
\subsubsection*{\textbf{Inference}}
The inference process of the model consists of two paths: one utilizes the pretrained STID model to predict the conditional mean, while the other uses the diffusion model to predict the residuals. The final sample is obtained by summing the results of both paths. This process is detailed in Algorithm~\ref{al: sampling}.
\section{Findings}
\label{sec:findings}
We find that the confusions concerning assumptions in ML largely revolve around what assumptions are (section \ref{sec:ontology}) and what is being done about them (section \ref{sec:procedure}). While many confusions are due to vague conceptualizations of assumptions, institutional fragmentation, and a general lack of clarity on response, others stem from holding unique and inconsistent views and procedures that deal with assumptions. Further, practitioners differ in their characterization of assumptions based on the role they take: whether they are the ones assuming (\textit{assumer}) or analyzing (\textit{analyst}) something. \rev{Table 1 summarizes our findings.}

\begin{table*}[]
\centering
\resizebox{\textwidth}{!}{%
\begin{tabular}{|l||l|l|l|}
\hline
\textbf{Axis of Confusion} & \textbf{Theme} & \textbf{Key Takeaways} & \textbf{Premise-Target Lens} \\ \hline
\multirow{2}{*}{\begin{tabular}[c]{@{}l@{}} Conceptualization of \\ what assumptions are\end{tabular}} & \begin{tabular}[c]{@{}l@{}}Independent\\ construction\end{tabular} & \begin{tabular}[c]{@{}l@{}}- viewed as being outside of the ML workflow \\ \\ - solidified as axioms or hailed as requirements \\   or relegated as limitations\end{tabular} & \begin{tabular}[c]{@{}l@{}}- target of the assumption \\   often remains unstated\\ \\ - premise is seldom evaluated\end{tabular} \\ \cline{2-4}  
 & \begin{tabular}[c]{@{}l@{}}Relative\\ construction\end{tabular} & \begin{tabular}[c]{@{}l@{}} - defined in relation to data quality, model \\   specifications, or business objectives\\ \\ - rationalized in relation to ML workflow and \\   prevents deeper and more inclusive assessment\end{tabular} & \begin{tabular}[c]{@{}l@{}}- premise, target, and argument\\   are explicit and clear\\ \\ - identification and evaluation\\   of premises are easier\end{tabular} \\ \hline \hline
\multirow{2}{*}{\begin{tabular}[c]{@{}l@{}}Uncertainty about \\ what is being done \\ with assumptions\end{tabular}} & \begin{tabular}[c]{@{}l@{}}Integration with\\ existing workflow\end{tabular} & \begin{tabular}[c]{@{}l@{}}  \textbf{Reactive handling:}\\ - reactive and iterative approach to ML extends\\   to assumptions identification and handling\\ \\ - assumptions constructed relative to ML \\   workflows are often reactively handled\\ \\ \textbf{Unreflective quantification:}\\ - the incentive to quantify assumptions \\   evaluation obstructs reflective practice \\ \\ \textbf{Circle of ambiguity:}\\ - no mechanisms to capture evidence of \\   assumptions, creating more uncertainties\\ \\ - knowledge and communication gap in ML\\   between stakeholders creates circling ambiguity\end{tabular} & \begin{tabular}[c]{@{}l@{}}- formulation of argument's\\   target depends on how surprising \\ 
 assumption's  consequences are\\ \\ \\ - ambiguities in premises and \\   implied arguments are disguised\\ \\ \\ - target of one assumption forms\\   the premise of other targets\end{tabular} \\ \cline{2-4} 
 & \begin{tabular}[c]{@{}l@{}}Unstructured\\ documentation\end{tabular} & \begin{tabular}[c]{@{}l@{}}  \textbf{Informal and implicit recording:}\\ - distinction between formulation and installation\\   of assumptions is not clearly documented\\ \\ - site, content, and style of assumption recording\\   is strongly associated with role, resulting in conflict\\ \\ \textbf{Granularity of recording:}\\ - insistence to understand the rationale behind\\   assumptions lead to further assumptions based\\   on lived experience\\ \\ - no structured prompt to record the required level\\   of details in assumptions recording\end{tabular} & \begin{tabular}[c]{@{}l@{}}- lack of distinction between \\   formulation and installation \\   of premises in documentation\\   \\ \\ \\ \\ - premises are often recorded as \\   declarative statements with no\\   relation to target of the argument\end{tabular} \\ \hline
\end{tabular}
}
\caption{\rev{A summary of key themes, takeaways, and the premise-target theoretical lens we adopt to deconstruct and understand the confusing factors about assumptions in ML.}}
\label{tab:findings}
\end{table*}


\subsection {Ontological Differences}
\label{sec:ontology}

The conceptualization of an ``assumption'' varied greatly between participants. While some gravitated toward an understanding that coincided with their preexisting technical workflows, others perceived it as an isolated consideration, existing outside the normative bounds of development. These inconsistencies may be rooted in a systemic de-emphasis of abstract thinking, incentivizing participants to view socio-technical concepts as a static, external object rather than an embedded mentality \cite{selbst2019fairness,wang2022towards,malik2020hierarchy,fazelpour2020algorithmic}. When prompted explicitly to define ``assumptions'', many participants described it as preliminary, something to be ironed out, built upon, or invalidated. These initial assumptions also predicated on how future assumptions were handled. The elaboration of how this characterization plays out throughout the development process differed, with some participants conceiving of an assumption independent of internal components (section \ref{subsec:ind}) and others associating it directly or indirectly with other entities (section \ref{subsec:rel}). Both these constructions create uncertainties and confusion in their own ways, both in how they shape institutional handling of downstream tasks and potential harms. 
% Redundant:
% 
% Ultimately, it was the background and experience of the participant that dictated how they distinguished assumptions. 

\subsubsection{Independent Construction}
\label{subsec:ind}
\citet{delin1994assumption} describe how average discourse around an assumption implies that it is a sort of abstract entity existing in one's mind, and to interact with one is akin to finding, identifying, and examining a ``thing.'' This interpretation best characterizes what we label an \textit{independent} construction of an assumption. This type of assumption exists as an external \textit{other} to the primary subject---the ML workflow. The \textit{purity} of this workflow is a common theme among participants with a technical background. The idea of the model and the deference to ML work is a foundational mentality that conceptualizations of risk \cite{saxenaRethinkingRiskAlgorithmic2023,zanotti2024ai}, bias \cite{andrusWhatWeCan2021,kernAssumptionsBiasData}, and, in this context, assumptions must work around. 

This is best seen through how technologists frame assumptions as independent ideas that exist to serve the technology. For instance, for many participants, we introduced the concept of assumptions by inquiring the participant about \textit{givens}, or what information or knowledge the participant takes for granted. These givens are necessary \textit{premises} to begin a project as they unquestionably validate the primary heuristics of the project. In other words, the \textit{target} of independently constructed assumptions remains unstated or, in the best case, unscrutinized if it is explicitly mentioned. This is often because these assumptions lay the foundation of a project, and they are often immutable and the rest of the work must be accommodated. This immutability is justified as the organic nature of an AI model, with assumptions being reflective of its surrounding context rather than the technology itself. P1, an ML developer, explains the unspoken nature of these givens:

% Redundant:
% The emphasis of the technology as the central component of the product lifecycle further contributes to this mindset.
% P1, a technical product manager, elaborates on this:
% \begin{quote}
% \textit{"How I describe an assumption is something that’s taken for granted. Something where I don’t have to think about it. And it’s a fact or a basis for the work that I’m about to do. I don’t really need to question, hey, what’s happening? Is this what I need to do? It just happens, or it’s there, and there’s no question about the validity of it."} 
% \end{quote}
\begin{quote}
\textit{``I guess the most basic assumption is that the human behavior can be modeled with numbers...Because if you know these things are unquantifiable, then there’s no work...It's a set of axioms, I guess, from which I can draw conclusions. And if these axioms are violated, then, you know there’s no guarantee how the system will turn out.''} 
\end{quote}

% These givens are also provided through a proxy that is perceived trustworthy and thus rarely questioned. These may be institutionalized teams or roles that convey the givens directly or the background and experience of the technologist. 

Recent research suggests that the assumptions and choices of technical practitioners, in particular, are often found to be more subtle \cite{kery2019towards,kommiya2024towards,wang2019data}. These types of assumptions reflect a desired \textit{implicitness} to certain thinking that allows the technology to be developed in the first place. The assumption then possesses an \textit{interpretive flexibility} \cite{meyer2006three,star1989structure,leigh2010not}, being swept under the rug but still requiring further assumptions to validate it and allow for it to remain unspoken. This process may entail the transformation of an assumption into a \textit{requirement} or a \textit{limitation}. The former is managed through the validity of the assumption by an authority, which is either the assumer themselves and their expertise or a separate role or team that explicitly assigns it credence. They become \textit{informed} assumptions, legally justified and deliberated upon by personal decision-making. The latter may be designated as such through real-world constraints that prevent deeper internalization. Assumptions that are relegated to limitations may also be the byproduct of a ``perfection is the enemy of good'' culture \cite{sylvester2018applied,green2019good}. The assumptions that are not solidified as axioms, or hailed as requirements, or relegated as limitations, are then needed to be empirically validated and conceptually clarified in relation to the target towards which the assumptions are directed.
% in order to assess risk.

% Independent assumptions are treated as \textit{boundary objects} \cite{star1989structure,leigh2010not}, where the back-and-forth between ill-structured and well-structured forms are to be expected. 

% \begin{quote}
% \textit{I’m sure you know, the model would not get things correctly, like 100\% of the time. But you know, we don’t aim for perfection. We aim for some number like 99.9\% of the time. Maybe that’s good enough.} 
% \end{quote} (P2)
% \textit{…it feels like for a lot of product teams, they get hindered because of the years of risk. So they just really want to move forward so that they can actually deliver something.} 
% (P3)

\subsubsection{Relative Construction}
\label{subsec:rel}
Other participants had a more \textit{embedded} perspective on assumptions. While independent construction allowed the assumption to exist as its own object flowing through and being manipulated by the ML workflow, \textit{relative construction} implies that assumptions exist \textit{in relation} to existing phases of the development process. Returning to \citet{delin1994assumption}'s characterizations, this type of assumption involves examining the ``mental-event-or-state'' of an individual or institution instead of perceiving the assumption as an independent entity. In other words, relative assumptions exist as a byproduct of the practitioner's \textit{mentality} in a workflow rather than an external factor.

In particular, participants embodying this perspective defined assumptions through technical framing, citing decisions around data quality or model specifications. The assumptions must live within the confines of a technically-driven approach and be subject to dissection through that lens. This integration of assumptions into a technical dimension can help practitioners investigate deeper issues within their work, despite the purview being narrower. Since the target of the assumptions is often explicit in this case, identifying the argument is relatively easier than in independent construction. This has both upsides and downsides: it allows the practitioner to navigate the assumption through a familiar paradigm, but it also presents assumptions as an inevitability through the sheer breadth of available information. This could allow assumptions to go unchecked, but in a justified resignation, as demonstrated by P2, an ML scientist:

\begin{quote}
\textit{``...we kind of don't have the bandwidth to check each and everything. So that's maybe one assumption we make. We're also kind of assuming that...all of this leads to data correctness...but we end up making some assumptions like, for example, data is about accuracy. If data is empty, then you should just treat it as empty and not treat it, as you know, something meaningful...They are assumptions, and they will sort of, you know, go into the model and be baked into the process.''}
\end{quote}

% Relative assumptions, like independent assumptions, originate outside the realm of the ML workflow, but are \textit{actualized} through technical framing. 

Relative assumptions too possess an inherent implicitness that may unintentionally inform understanding of the model. But if an independent construction allows for an assumption to persist as its own object with the possibility of transforming into something beneficial to the technical process, then a relative construction attempts to integrate an assumption directly \textit{into} the process. The formulation of problems is an example, as they usually become intertwined with relative assumptions: P2 described how associations with specific demographics, for instance, may be less scrutinized due to the expectations of the technical team. In other words, if the data is labeled or categorized in a certain way that conforms to the lived experience of the practitioner, it may prevent deeper assessment. And because relative assumptions are tied directly to workflow components and workflow components are necessary, there is an incentive for the assumption to be rationalized.

% \textit{If you know, your assumptions start becoming givens, and you know ideally at the end of your project you should have only knowns and no assumptions, nothing left to assumption.} (P5)
% \\
% \textit{….because at the end of the day, bias in your model means bias in data and bias in data is just like reflecting what exists in the real world…So there those are certain assumptions in terms of like, 'oh, this bias already exists and you’re conforming to it in a certain sense…'} (P6)

Relative assumptions need not necessarily be attached to \textit{technical} processes only. A few practitioners in management roles described how assumptions can linger throughout business objectives and outcomes; what is deemed critical to operationalizing the company vision is often an assumption in and of itself. These assumptions are often second or third-order assumptions \cite{berman2001opening,korzybski1958science}, meaning that the assumer is multiple degrees removed from the original observation that incited the assumption\footnote{\rev{To understand the confusions around assumptions, for the sake of clarity, we almost exclusively treat an assumption independently of other assumptions. We leave explorations of higher order and connected assumptions to future work and briefly point to related works at the end of section \ref{disc:articulate}.}}. More so, the authority attached to the central teams that assert these claims makes it easier to internalize relative assumptions because the task of proving them right is often the primary function of the business. 


% \begin{itemize}
%     \item 			i. Data collection quality - 30-35 age not checked - if model performs badly for this distribution, we won’t know
% \item			ii. Data correctness 
% 	\item		iii. Lack of data - loss to follow up before death - how to impute missing data
% 	\item		iv. Data quality is often equated to data accuracy
% \item First data quality is investigated because that is often taken for granted
% \end{itemize}
% -  Investigate assumptions if personally relevant - else take altruistic stance

% <1 para>
% - Another conception of assumptions has an implicit relation with limitations
% - though when practitioners talk about limitations, assumptions are inherently present, they are unable to describe it explicitly.
% - Unable to describe assumptions as limitations - the latter is always in terms of outcomes - except a few - Assumptions are ignorant decision or requirements - can be seen as limitations
% - Though assumptions inform various actions/decisions, focus is on latter and not on former

% <unused for now>
% - Assumptions in operationalizing business vision - economic impact vs revenue example - 40\% assumptions comes from objective or mission statements - intentionally used - anticipate


\subsection {Procedural Uncertainties}
\label{sec:procedure}

Other than ontological differences, practitioners also face several challenges in determining what to do with the assumptions. In this section, we lay out the uncertainties that accompany various \textit{procedures} practitioners employ to identify and handle assumptions. We find that these uncertainties either correspond to integration methods with existing workflows (section \ref{subsec:integrate}) or the documentation practices for collaboration and accountability (section \ref{subsec:doc}). 

% Finally, we discuss the strategies practitioners employ, and the challenges they face to bring some structure to the articulation of assumptions for alleviating some of the surrounding confusion (section \ref{sec:articulate}.) 

\subsubsection{Integration with existing workflow}
\label{subsec:integrate}
\citet{brookfield1992uncovering} argues that the investigation of assumptions must be a deliberate and reflective process to assess and validate various decisions that go into a process. However, most practitioners we spoke to shared that their typical ML lifecycles in practice seldom offer them avenues to reflectively identify and handle assumptions.

\smallskip
\noindent \textbf{Reactive Handling.} Machine learning in practice is usually an iterative and outcome-oriented pursuit \cite{ashmore2021assuring,malik2020hierarchy}: almost all our participants start their ML lifecycles with lesser information than they ideally need, but then iterate and learn about assumptions as they go. Because there is no appropriate structure for them to inquire about assumptions, this iterative process often puts practitioners in ambiguous situations where there are no clear and accepted procedures to follow when one is found. This, in turn, demands practitioners to adopt a \textit{reactive} approach to identify and handle assumptions. This is most prominent in relative assumptions, in which the assumption is directly embedded within the ML workflow. For instance, some practitioners, such as P4, raise a red flag and look for assumptions in data when presented with a ``neat classification problem'' in which all data is categorized too perfectly. For a few others, assumptions are investigated only during exploratory data analysis (EDA for short) when extreme and skewed patterns about data are observed. \rev{P3, a technical lead, provides an example of this mentality by demonstrating how their team took certain factors for granted until a ``surprising'' result was found:}
\begin{quote}
    \textit{\rev{``So in any modeling process, we do some basic EDA to understand the quality of the data. But some things we take for granted. We then build a model. And sometimes, surprisingly, results will be like, very good than what you anticipate. Then we start looking into the top features. And then we think about, okay, is there any data leakage? This is very hard to understand when you're doing initial EDAs, right? But once we build the model, once you're seeing surprising results, then we go back to business. Then we discuss its implications...is there anything wrong with that feature?''}}
\end{quote}

Because relative assumptions can be associated with both technical and business processes, what is deemed surprising by the latter may inform the reaction of the former. This is demonstrated by P5, a lead data scientist, who articulated how the inquiry of assumptions often takes place only when the outcomes have unanticipated business implications in the typical ML workflow.
While the above-discussed ``innocent until proven guilty mindset'' (P6) is prevalent among ML teams, some practitioners in management-related roles highlighted that product teams generally use existing collaboration infrastructure, such as GitLab\footnote{https://about.gitlab.com/}, Asana\footnote{https://asana.com/}, or other business management software to inadvertently integrate assumptions discussions into the typical workflow. Though not a typical practice of ML teams, a couple of practitioners also shared that dedicated ``assumptions trackers'' are sometimes used to encourage documentation and discussion of assumptions at work. 

However, P6 shared that the use of these tools is largely descriptive and only results in noting down the simplistic description of assumptions, subjective association with the argument's target, and mapped ownership of assumptions redressal. 
Overall, assumptions are communicated through these tools, but it is unclear if they are reasoned and rationalized: questions such as why the assumptions are made, why they are necessary, what kind of assumptions they are, what their implications for ML lifecycle are, etc. are sparsely addressed by these tools. We expand upon this tendency to examine assumptions only at a surface-level in the next section.
% quote : i have a decade long experience in client facing roles - with great confidence i can say that this is not the case

% - though assumptions tracker etc. could be helpful, it often requires a historian type method - As historian, methodology revolves around recognition of “contingency” - reflective endorsement
% -- case study analysis


% Finally, integration into existing workflow also is contingent on the cost of 
% <1 para> - hold this for now
% With recent increase in RML attention, there has been some efforts to invoke assumptions - however
% 1. RML docs are not universal - best used for system documentations
% 3. RML toolkits are not used in many organizations 
%     a. Assumptions work comes at a cost - RML doesn’t help - Mediterranean addition example
% 4. RML just borrows inputs from various laws - we need dedicated legal frameworks.

\smallskip
\noindent \textbf{Unreflective Practice through Quantification.} A crucial constituent of assumptions identification and assessment is the \textit{reflectiveness}, if not \textit{critical reflectiveness}, of the process \cite{paul1993critical,paul1993workshop}. 
However, \textit{the seductions of quantification} \cite{merry2009seductions} often derail practitioners from performing reflective exercises on assumptions and instead reroute them to aspire for a fictitious objective and unequivocal state. \rev{P10, an engineering consultant, elaborated on her perceived futility in quantifying assumptions:}

\begin{quote}
   \rev{ \textit{``...we do have some in-house metrics…we also really want to know which task is being performed the best without any sort of assumptions or biases so we have inbuilt like a comparator for ourselves, based on particular metrics… I believe these metrics can be very much subjective, based on the use case a particular company or a particular product has...there is no way we can quantify what assumptions a particular model is taking…''}
}
\end{quote}

\textit{Independent} assumptions are most prone to this instinct---they must adhere to the workflow. Practitioners noted that the organizational constraints and standard ML workflow practices do not incentivize many technical counterparts to perform reflective exercises but only make them care about data coverage and infrastructure-related concerns. But this interaction between assumption and workflow is bi-directional. Both in their own practice and during our case analyses, practitioners suggested ways of using various computational methods \cite{yanga2024exploratory,zhang2018empirical} to debug and act on their assumptions. However, as an extreme example, one practitioner supported the use of software plugin-type tools to automatically check all assumptions and provide a score to quantify assumptions evaluation. 

In our case analyses with practitioners, we find that many of them spend more time understanding and justifying various computational steps in LLM reports but focus less on reflecting on the premises and conclusions that inform the various steps they are analyzing. We observe that practitioners with some exposure to responsible ML principles often recognize that over-reliance on computational methods just disguises the underlying ambiguities, not removes them \cite{green2018myth,fazelpour2020algorithmic}. This means that the organizational instinct and incentive to adhere to the ML workflow implies a bias to a relative construction of assumptions. They also shared that such disguising has the danger of pushing practitioners to ignore the assumptions and the implied argument, which often peek through at later phases of ML lifecycles in various forms. \rev{P7, a designer working with state organizations, shared how a risk assessment tool unnecessarily quantified risk evaluation:}

\begin{quote}
    \textit{\rev{``so these risk assessment questionnaires are designed such that you're kind of casting a wide net like, yeah, you're probably at risk of something. After they do the risk assessment, the general next step is, your team has to come up with a control plan in detail explaining how you're going to mitigate those risks. So you know, obviously, some of the feedback that we got was like, okay, so we could come up with the most horrific AI system that has all these risks to end users. And it's all okay, because we can just write a control plan, saying that we'll mitigate right? So there was a lot of like contention with that approach.''}}
\end{quote}

While the quantification of assumptions is intended to reduce uncertainty around decision-making, further assumptions emerge when model performance is negotiated with other sociotechnical concerns, such as safety and fairness. For many practitioners, this leads to a nuanced tension they face when, for instance, deciding on a lower threshold for the maximum number of security violations. For example, while a 90\% violation is clearly red-flagged, negotiation between 0.5\% and 1.0\% creates debates and confusion. We expand on the cyclical nature of assumptions in the next section.



% benchmarks curb reflectivity?
% <-- finally benchamarks are blindly used (cite gorilla etc.) - see quote about assumptions>

% - it was not a one side affair - some practiotners also sggested ways of reflectively using compuational methods - some also did it in our case studies
% -- while some computational techniques aid assumptiosn inquirt reflective - CF and knowledge base
% - however, the infrastructure (connect to worlflow + documnet next)

% -- embarassement score    

% When participants were asked to identify assumptions more indirectly through analyzing model documentations at face-value, some associated anything distinct from technical terminology as an assumption.

\smallskip
\noindent \textbf{The Circle of Ambiguity.}
While assumptions are often invoked to mitigate the ambiguity around unresolved questions, ironically, due to their very implicit nature, they instead raise several questions and concerns they are trying to ameliorate in the first place. Consider the use of automated risk-assessment tools used in many organizations in recent years: practitioners find that many responses to assessment software require extrapolation and reflection. However, the checklist-type design and quantification of risk evaluation in these tools obscure the \textit{consequences} of underlying assumptions. A few participants shared that several sub-modules of these tools in their organization require practitioners to invoke assumptions in their own responses, but there are typically no mechanisms to capture the evidence of these assumptions, creating more uncertainties. Therefore, the trajectory of an assumption in an ML workflow is one that may beget more assumptions. This ambiguity is discussed in \textit{Critical Thinking} as complex or \textit{chain of reasoning} arguments, where the assumption that supports a target may be the target of one or more assumptions, and so on indefinitely \cite{hitchcock2021concept}. 

% Looking through the analytical lens our practitioners took in the case studies, our findings highlight nuanced uncertainties that motivate technical practitioners to make subtle assumptions.  

Consider P4, an ML scientist, who expresses how the confusion around calibration scores \cite{pleiss2017fairness} among different non-technical stakeholders motivated their subjective decision on how to present the model outcomes:

\begin{quote}
    \textit{``...for multiple projects that I was on...I saw that the easiest thing to do was to provide high, medium, low kind of bucketing of confidences rather than to provide a confidence score. The assumption that kind of affords a user is that both significant digits are significant, that there is a difference between, you know, 0.95 and 0.97 or something like that. And that really wasn't the case. As a practitioner, I would see that and just be like, okay, that's not that significant. But how does especially someone who is not used to reasoning about the processes that produce these numbers know that?''
}
\end{quote}

The ordinal categories expected to resolve misinterpretation of double-precision metrics yield further confusion for both technical and non-technical stakeholders depending on how risk levels are construed. A possible explanation for this circling ambiguity, as indicated by some of our participants, could be attributed to the knowledge and communication gaps between people with and without ML backgrounds \cite{chen2021beyond,kommiya2024towards,varanasi2023currently}. P7 shared that their ML scientists \textit{``did not feel like there was a lot of risk involved''} when the models they work with are more interpretable to them, in contrast to what risk-averse employees felt about the model and its policy implications. This logic informs another assumption: to fix biases, what is needed is \textit{more} knowledge and training. However, most management-related practitioners in our sample concurred that technical knowledge in their organization is generally very isolated and emphasized the need to document differences in interpretations and motivating assumptions.

% Because products are made in the perspective of the organization, they can be easily dismissed to fall under the responsibility of a different position.

% \begin{quote}
%     \textit{At least in my experience, providing confidence scores, was not always a benefit, because the ways in which users interpreted those confidence scores varied a lot right. Some people would really set a lot of store by the difference between 0.90 and 0.92, and not care about the difference between 0.9 and 0.7. And so for multiple projects that I was on you know, I saw that the the easiest thing to do was to provide high, medium, low kind of like bucketing of confidences rather than to provide a confidence score. The assumption that kind of affords a user is that both significant digits are significant, that there is a difference between, you know, 0.95 and 0.97 or something like that. And that really like that wasn't the case. As a practitioner, I would see that and just be like, Okay, that's not that significant. But how do especially someone who is not used to reasoning about the processes that produce these numbers know that?
% }
% \end{quote}
% - exclusion complex
% - Some organizations have top-down assumptions handling 
% - 

% - we found similar observations - Mihir doc - quote - assumptins  techincal - infor  subjective deicon - rationale lost



\subsubsection {Unstructured Documentation}
\label{subsec:doc}

Though the successful adoption of responsible ML principles in organizational settings is open to debate \cite{deshpande2022responsible,rakova2021responsible,raiAdoption}, practitioners are undoubtedly getting increased exposure to different toolkits and frameworks for responsible ML \cite{liang2024systematic,yang2024navigating}. While most of these support systems have some reference for practitioners to elicit and discuss assumptions behind various decisions (section \ref{rel:periphery}), prior works are unclear on how and to what extent assumptions are documented through these toolkits from the perspective of a practitioner. Below, we discuss two characteristics of documentation practices that contribute to the confusion around assumptions.
% For instance, a recent study found that the use of framework such as modelcards have rapidly risen in the last few years.

\smallskip
\noindent \textbf{Informal and Implicit Recording.} \citet{delin1994assumption}, in their discussion about assumptions, make a distinction between formulating and installing an assumption: while \textit{formulating} is about the expression of intent to install an assumption, \textit{installation} corresponds to aligning the execution to the intention within given constraints. In one of the excerpts in our case study (appendix \ref{appendix} \cite[p.~64]{anil2023palm}), the authors use ``marked references'' or annotated characterizations of identity\footnote{Some participants speculated the assumptions behind the usage of this term, but for this discussion, this can be perceived as typical annotations in ML data pipelines.} to assess the representational bias of their language model. Though our participants did not use the terms referenced in the \textit{Informal Logic} literature exactly, they did observe that if the authors had intended to operationalize representativeness with these annotations, then they must have correctly \textit{installed} the assumptions but did not explicitly document the \textit{formulation}. While the authors of the case study vaguely note down this assumption in the limitations section of the report, many of our participants were dissatisfied and shared that assumptions that are not articulated at the time of their identification get forgotten over time. 

Because the recording of assumptions is often informal and unstructured, system documentation and other related trackers are rarely modified after an assumption is acted upon. While some practitioners stated that their organizations require them to record assumptions in separate sections in accordance with a design or product requirement, others just recorded them offhand in change logs. Both approaches offered no structured prompt or instructions about how to aptly record them. As such, many participants hinted that documentation is not often given serious thought for internal projects. Below, P9, another technical lead, admits how these details are either not documented explicitly or get lost in an attempt to simplify the language.

\begin{quote}
\textit{``...when I run into an assumption like that, I try to distill it sometimes through multiple rewrites into a simple, concise, clear declaratory statement which can be connected to others...I would forget what I had in mind when I was writing those things down...'' 
}
\end{quote} 

The site, content, and style of assumptions documentation also have a strong relation with the primary role and responsibility of the practitioner, leading to conflicting situations when one party (say an ML scientist) has to consume and work with assumptions recorded by another party (say a product manager). For instance, writing and analyzing compliance-related reports are often perceived as the responsibility of safety or legal teams. Their articulation of assumptions might differ from that of developers and data scientists, who usually write in terms of inputs and outputs. If these technical practitioners are tasked to write about assumptions, they are less likely to appreciate the implications of their assumptions and choices in their reports. As the data scientist P5 mentioned, when they use LLMs, they go directly to model docs, datasets, and benchmarks, and treat the system cards and other reports accompanying these LLMs (with safety and RAI analysis) as just \textit{``terms and conditions.''} This fundamental asymmetry presents a fragmentation of thinking and recording that has consequences for how assumptions are handled. 

% 		ii. RML just borrows inputs from various laws - we need dedicated legal frameworks.
% Some implicit decisions require technical effort - need better documentations

% In this section, we will use an example from our case study where the LLM authors mention that they \textit{``excluded data from certain sites known to contain a high volume of personal information about private individuals''} (cite).

\smallskip 
\noindent \textbf{Granularity of recording.} We leverage the \textit{analytical} perspective our practitioners took in the case studies in assessing how deliberate they were in recording assumptions. As \citet{brookfield1992uncovering} argues, this particular type of perspective is well-positioned to step out of a familiar interpretive frame of reference and look at assumptions through an unfamiliar lens. We observe that the most common practice the participants followed is simply listing down the identified assumptions in some style and form, though at varied levels of formality, such as they would in a business requirement document or vocally at a team meeting. However, practitioners note that this method generally relies on their own unjustified claims. In other words, there was no distinction between the premises, target, and the argument that the assumption-target complex makes.

While listing down assumptions found in the case study, most of the practitioners sought further clarification on \textit{why} particular assumptions were made. This phase is when the interpretive lens of a practitioner begins to crack, and their primary disciplinary training starts to dominate; the choice of assumption to expand on and what needs to be explained starts to depend again on their organizational role or lived experience. For instance, when technical practitioners observe relative qualifiers such as ``higher'' or ``lower'', their scrutiny often stops at the sight of a quantified relation, such as ``40\% higher.'' \rev{Participants shared that many technical practitioners} end up not reflecting on how benchmarks are only the \textit{indications} and not the equivalence of the capability they are measuring. \rev{In the words of P11, an AI ethicist:}

\begin{quote}
    \textit{\rev{``The fact that an LLM could perform well on an LSAT benchmark means not necessarily that it's capable of, you know, of solving legal problems. But it could be quite capable of delivering to you the outputs that mimic responses to those questions and that's valuable. Where we can go wrong is to infer that, you know, because the model can pass a test that must mean that its feature representations allow it to understand the material of the test. This is a very different question and that I feel like needs holistic evaluations. You know, once they're developed, there's a huge evaluation gap here.''}}
\end{quote}

% \rev{Below, P4 explains how the granularity of assumptions inquiry ends when benchmarks are conflated with the capability they are intended to measure:}

% \begin{quote}
%     \textit{\rev{``if your model can solve this entailment data set, the idea is that it can do textual entailment more broadly. But there are several kinds of subpopulations that you need to reason about before you can actually take that to the task that you actually care about. It's reasonable with the intentional definition of entailment that is presented to say, like, oh, my model can understand that 3 tenths and 30\% of the same thing. That is like a linguistic equivalence. But models don't do that right. So if you are building an entailment based application that relies on that kind of numerical reasoning, your application will break. That's not an assumption that you can actually...do textual entailment as a capability rather than textual entailment on MultiNLI as a data set.''}}
% \end{quote}

How a clarification is or ought to be justified is a critical step in assumption recording, the complexities of which are expanded upon in the following section. P8, an AI governance architect, makes a distinction between \textit{visibility} and \textit{explainability} in recording assumptions to avoid confusion. While the former could be one or more declarative statements that add a marginal level of detail, the latter is about reasoning in every step with logical validity. For instance, in many of the LLM reports, our participants observed that data quality is justified by arguing that the followed data processing steps led to ``good'' model outcome in terms of some metrics. However, the above aligns more with the visibility-level of clarification. An explainability-level of granularity would involve expanding how quality is operationalized (for example, showing no outliers, following a representative distribution, etc.) and then logically explaining how the data can be assessed on both these parameters and the model outcome.

Finally, the maximum level of granularity in which an assumption can be recorded is ambiguous and context-dependent. While some practitioners seek ``expert'' intervention, which entails transferring the authority to a team lead or an executive, most practitioners instead advocate for a diverse collaborative effort to avoid biases. In section 5.1, we discuss a framework inspired by argument structuring in Informal Logic to articulate assumptions that facilitate such discussions.

% To do:
% 1. change quotes repetitive participants
% 2. include citations
% 3. check tense consistency

% We observe that practitioners implicitly or explicitly follow three distinct levels of granularity in recording assumptions. 
\section{Limitations and future directions}
In this work, we introduce a theoretical framework for understanding the mechanism of weak-to-strong (W2S) generalization in the variance-dominated regime where both the student and teacher have sufficient capacities for the downstream task. Leveraging the low intrinsic dimensionality of finetuning (FT), we characterize model capacities from three perspectives: FT approximation errors for ``accuracy'', intrinsic dimensions for ``complexity'', and student-teacher correlation for ``alignment''. Our analysis shows that W2S generalization is driven by variance reduction in the discrepancy between the weak teacher and strong student features. 
This generalization analysis is followed by a case study on the relative W2S performance in terms of performance gap recovery (PGR) and outperforming ratio (OPR). We show that while larger sample sizes imply better W2S generalization in an absolute sense, the relative W2S performance can degenerate as the sample size increases.
Our results provide theoretical insights into the choice of weak teachers and sample sizes in W2S pipelines. 

An interesting implication of our analysis is that the mechanism of W2S may differ as the balance between variance and bias shifts. In the variance-dominated regime studied in this work, W2S can benefit from a lower intrinsic dimension of the strong student due to the resulting variance reduction in the subspace of discrepancy from the weak teacher. In contrast, in the bias-dominated regime, the lower approximation error of the strong student is generally brought by the larger ``capacity'' of the strong model corresponding to a higher intrinsic dimension~\citep{ildiz2024high,wu2024provable}. 
This calls for future studies on unified views and transitions between the two regimes, which will provide a more comprehensive understanding of W2S.
Toward this goal, a limitation of our analysis is the quantification for the advantage of W2S in bias (see \Cref{fn:bias_strong}), which could be a promising next step.

\section{Conclusion}
In this paper, we introduced Atom of Thoughts (\our), a novel framework that transforms complex reasoning processes into a Markov process of atomic questions. By implementing a two-phase transition mechanism of decomposition and contraction, \our eliminates the need to maintain historical dependencies during reasoning, allowing models to focus computational resources on the current question state. Our extensive evaluation across diverse benchmarks demonstrates that \our serves effectively both as a standalone framework and as a plug-in enhancement for existing test-time scaling methods. These results validate \our's ability to enhance LLMs' reasoning capabilities while optimizing computational efficiency through its Markov-style approach to question decomposition and atomic state transitions.
%%%%%%%%%%%%%%%%%%%%%%%%%%%%%%%%%

%%
%% The next two lines define the bibliography style to be used, and
%% the bibliography file.
\bibliographystyle{ACM-Reference-Format}
\bibliography{bibliography}



%\clearpage

\appendix
\section*{Appendix}
\begin{table*}[h!]
\caption{The basic information of grid-based spatio-temporal data.}
\label{tbl:append_data}
\begin{threeparttable}
\resizebox{1.9\columnwidth}{!}{
\begin{tabular}{cccccccc}
\toprule
Dataset & City & Type & Temporal Period & Spatial partition & Interval & Mean & Std \\
\hline
TaxiBJ & Beijing & Taxi flow&  2014/03/01 - 2014/06/30 & $32 \times 32$ & Half an hour & 111.5 & 139.3 \\
BikeDC & Washington, D.C. & Bike flow&  2010/09/20 - 2010/10/20 & $20 \times 20$ & Half an hour & 0.924 & 4.88 \\



CellularSH & Shanghai & Cellular traffic &  2014/08/01 - 2014/08/21 & $32\times28$ & One hour & 0.175 & 0.212 \\
CellularNJ & Nanjing & Cellular traffic &  2021/02/02 - 2021/02/22 & $20\times28$ & One hour & 0.842 & 1.30 \\
CrowdBJ & Beijing & Crowd flow &  2018/01/01 - 2018/01/31 & $1010$ & One hour & 7.07 & 11.1 \\
CrowdBM & Baltimore & Crowd flow &  2019/01/01 - 2019/05/31 & $403$ & One hour & 14.4 & 29.3 \\
Los-Speed & Los Angeles & Traffic speed&  2012/03/01 - 2012/03/07 & $207$ & Five minutes & 59.0 & 12.5 \\

\bottomrule
\end{tabular}}
\end{threeparttable}
\end{table*}

% \begin{table*}[t!]
% \caption{The basic information of Graph-based spatio-temporal data.}
% \label{tbl:append_data_graph}
% \begin{threeparttable}
% \resizebox{1.8\columnwidth}{!}{
% \begin{tabular}{ccccccccc}
% \toprule
% Dataset & City & Type & Temporal Period & Interval & \#Nodes & \#Edges & Mean & Std \\
% \hline

% TrafficBJ & Beijing & Traffic speed & 2022/03/05 - 2022/04/05 & 15min& 13675& 24444& 6.837&  3.412\\
% TrafficSH & Shanghai & Traffic speed & 2022/01/27 - 2022/02/27 & 15min & 21099& 39065& 7.815&  4.044\\
% TrafficNJ & Nanjing & Traffic speed  & 2022/03/05 - 2022/04/05 & 15min & 13419& 25100& 6.699&  4.253\\

% \bottomrule
% \end{tabular}}
% \end{threeparttable}
% \end{table*}
\begin{table*}[h!]
\caption{Short-term prediction results on two additional datasets in terms of both deterministic and probabilistic metrics. \textbf{Bold} indicates the best performance, while \underline{underlining} denotes the second-best.}
\label{tbl:short1-app}
\begin{threeparttable}
% \resizebox{2.0\columnwidth}

\resizebox{1.5\columnwidth}{!}{
\begin{tabular}{ccccccccccc}
\toprule
\multirow{2}{*}{\textbf{Model}}
& \multicolumn{5}{c}{\textbf{CellularNJ}} & \multicolumn{5}{c}{\textbf{CrowdBM}}   \\
\cmidrule(lr){2-6} \cmidrule(lr){7-11} 
 &\textbf{MAE} & \textbf{RMSE}  &\textbf{CRPS} & \textbf{QICE} & \textbf{IS} & 
\textbf{MAE} & \textbf{RMSE} & \textbf{CRPS} & \textbf{QICE} & \textbf{IS} \\


\midrule
D3VAE& 0.580 & 1.135   &	0.565&	0.096&6.03	&	11.0&24.7	&	0.593& 0.110	&136.4\\


DiffSTG& 0.317& 0.649&	0.291&	0.071&3.11	&	8.88&21.3	&0.453&0.047	&68.5\\

TimeGrad& 0.340&  0.357  &	0.432&	0.162&5.87	&10.1	& \underline{12.4}	&\textbf{0.240}&\underline{0.085}	&\underline{46.9}\\


CSDI&0.129 & 0.237   &	\underline{0.111}&	 \underline{0.039}&	\underline{0.80}&	7.31& 19.3&	0.390& 0.054&61.1\\

NPDiff& \underline{0.123}&  \underline{0.175}  &0.128	&	0.133&2.22	&	\underline{5.42}& 13.7	&0.331&0.119	&91.2\\

DyDiffusion&0.222&   0.357 &0.196	&0.080	&	1.80&-	&-	&-	&-&-\\




\cmidrule(lr){1-1} \cmidrule(lr){2-6} \cmidrule(lr){7-11}
\textbf{CoST}&\textbf{0.102} &\textbf{0.172}    &\textbf{0.090}	&\textbf{0.037}	&\textbf{0.682}	&\textbf{5.04}	&\textbf{12.1}	&\underline{0.256}	&\textbf{0.027}& \textbf{37.8}\\
%\textbf{Reduction}& &    &	&	&	&	&	&	&\\
\bottomrule
\end{tabular}}
\end{threeparttable}
\end{table*}


% \input{Tables/short-term2-append}





\end{document}

