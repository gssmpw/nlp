%%
%% This is file `sample-sigconf-authordraft.tex',
%% generated with the docstrip utility.
%%
%% The original source files were:
%%
%% samples.dtx  (with options: `all,proceedings,bibtex,authordraft')
%% 
%% IMPORTANT NOTICE:
%% 
%% For the copyright see the source file.
%% 
%% Any modified versions of this file must be renamed
%% with new filenames distinct from sample-sigconf-authordraft.tex.
%% 
%% For distribution of the original source see the terms
%% for copying and modification in the file samples.dtx.
%% 
%% This generated file may be distributed as long as the
%% original source files, as listed above, are part of the
%% same distribution. (The sources need not necessarily be
%% in the same archive or directory.)
%%
%%
%% Commands for TeXCount
%TC:macro \cite [option:text,text]
%TC:macro \citep [option:text,text]
%TC:macro \citet [option:text,text]
%TC:envir table 0 1
%TC:envir table* 0 1
%TC:envir tabular [ignore] word
%TC:envir displaymath 0 word
%TC:envir math 0 word
%TC:envir comment 0 0
%%
%%
%% The first command in your LaTeX source must be the \documentclass
%% command.
%%
%% For submission and review of your manuscript please change the
%% command to \documentclass[manuscript, screen, review]{acmart}.
%%
%% When submitting camera ready or to TAPS, please change the command
%% to \documentclass[sigconf]{acmart} or whichever template is required
%% for your publication.
%%
\documentclass[sigconf]{acmart}
%%
% \documentclass[manuscript,review,anonymous]{acmart}

%%
%% \BibTeX command to typeset BibTeX logo in the docs
\AtBeginDocument{%
  \providecommand\BibTeX{{%
    Bib\TeX}}}

%% Rights management information.  This information is sent to you
%% when you complete the rights form.  These commands have SAMPLE
%% values in them; it is your responsibility as an author to replace
% %% the commands and values with those provided to you when you
% %% complete the rights form.
% \setcopyright{acmlicensed}
% \copyrightyear{2018}
% \acmYear{2018}
% \acmDOI{XXXXXXX.XXXXXXX}

% %% These commands are for a PROCEEDINGS abstract or paper.
% \acmConference[Conference acronym 'XX]{Make sure to enter the correct
%   conference title from your rights confirmation emai}{June 03--05,
%   2018}{Woodstock, NY}
%%
%%  Uncomment \acmBooktitle if the title of the proceedings is different
%%  from ``Proceedings of ...''!
%%
%%\acmBooktitle{Woodstock '18: ACM Symposium on Neural Gaze Detection,
% %%  June 03--05, 2018, Woodstock, NY}
% \acmISBN{978-1-4503-XXXX-X/18/06}


%%
%% Submission ID.
%% Use this when submitting an article to a sponsored event. You'll
%% receive a unique submission ID from the organizers
%% of the event, and this ID should be used as the parameter to this command.
%%\acmSubmissionID{123-A56-BU3}

%%
%% For managing citations, it is recommended to use bibliography
%% files in BibTeX format.
%%
%% You can then either use BibTeX with the ACM-Reference-Format style,
%% or BibLaTeX with the acmnumeric or acmauthoryear sytles, that include
%% support for advanced citation of software artefact from the
%% biblatex-software package, also separately available on CTAN.
%%
%% Look at the sample-*-biblatex.tex files for templates showcasing
%% the biblatex styles.
%%

%%
%% The majority of ACM publications use numbered citations and
%% references.  The command \citestyle{authoryear} switches to the
%% "author year" style.
%%
%% If you are preparing content for an event
%% sponsored by ACM SIGGRAPH, you must use the "author year" style of
%% citations and references.
%% Uncommenting
%% the next command will enable that style.
%%\citestyle{acmauthoryear}
% \usepackage{outlines}
\usepackage{hyperref}
\usepackage{enumitem}
\usepackage{makecell}
% \usepackage{xcolor}
% \usepackage{tabularx} 
% check this one
\usepackage{array}
% \usepackage{menukeys} check this one
% \usepackage{pgfplots} 
% check this one
% \usetikzlibrary{patterns} 
% check this one
\usepackage{multirow}
% \usepackage{tcolorbox}
% check this one

% \usepackage[left=1in,right=1in]{geometry}
%%
\definecolor{codebgd}{RGB}{255, 253, 232}
\definecolor{lightgray}{RGB}{200, 200, 200}
\definecolor{codehlt}{RGB}{245, 245, 254}
\definecolor{shade1}{RGB}{150,220,0}
\definecolor{shade2}{RGB}{0,0,120}

\newcommand{\ipstart}[1]{\vspace{1mm}\noindent{\textbf{\textit{#1.}}}}

\definecolor{customblue}{rgb}{0.0, 0.5, 1.0}
\definecolor{customlightblue}{rgb}{0.53, 0.81, 0.98}
\definecolor{customgray}{rgb}{0.83, 0.83, 0.83}
\definecolor{custompeach}{rgb}{0.98, 0.8, 0.69}
\definecolor{customorange}{rgb}{0.95, 0.55, 0.2}
\definecolor{custommidorangepeach}{rgb}{0.965, 0.675, 0.445}
\definecolor{cmumaroon}{rgb}{192, 0, 0}
\definecolor{shade1}{RGB}{192,0,0}
\definecolor{shade2}{RGB}{18,96,157}
\definecolor{custommidbluelightblue}{rgb}{0.265, 0.655, 0.99}
\definecolor{neongreen}{rgb}{0.1, 1.0, 0.1}
\definecolor{neonyellow}{rgb}{0.98, 1.0, 0.18}
% \newcommand{\highlight}[1]{\textcolor{red}{#1}}
\newenvironment{highlight}
  % {\color{purple}} 
  {\color{black}}% Begin environment: sets text color to red
  {}            % End environment: nothing special to do
% \newcommand{\red}[1]{\textcolor{red}{#1}}
\newcommand{\likertpct}[8][0.225]{%
\begin{tabular}{rcl}
\the\numexpr(#4+#2+#3)*100/16\relax\% &
\resizebox{#1\textwidth}{0.78\height}{%
\color{customorange}\rule{#2mm}
{10pt}\color{custommidorangepeach}\rule{#3mm}{10pt}\color{custompeach}\rule{#4mm}{10pt}\color{customgray}\rule{#5mm}{10pt}\color{customlightblue}\rule{#6mm}
{10pt}\color{custommidbluelightblue}\rule{#7mm}{10pt}\color{customblue}\rule{#8mm}{10pt}%
} &
\the\numexpr(#6+#7+#8)*99/16\relax\%\\
\end{tabular}%
}

\newcommand{\likertpctt}[8][0.225]{%
\begin{tabular}{rcl}
\the\numexpr(#4+#2+#3)*5\relax\% &
\resizebox{#1\textwidth}{0.78\height}{%
\color{customorange}\rule{#2mm}
{10pt}\color{custommidorangepeach}\rule{#3mm}{10pt}\color{custompeach}\rule{#4mm}{10pt}\color{customgray}\rule{#5mm}{10pt}\color{customlightblue}\rule{#6mm}
{10pt}\color{custommidbluelightblue}\rule{#7mm}{10pt}\color{customblue}\rule{#8mm}{10pt}%
} &
\the\numexpr(#6+#7+#8)*5\relax\%\\
\end{tabular}%
}

\newcommand{\likerteval}[8][0.325]{%
\begin{tabular}{rcl}
\the\numexpr(#4+#2+#3)*5\relax\% &
\resizebox{#1\textwidth}{0.78\height}{%
\color{customorange}\rule{#2mm}{10pt}\color{custommidorangepeach}\rule{#3mm}{10pt}\color{custompeach}\rule{#4mm}{10pt}\color{customgray}\rule{#5mm}{10pt}\color{customlightblue}\rule{#6mm}
{10pt}\color{custommidbluelightblue}\rule{#7mm}{10pt}\color{customblue}\rule{#8mm}{10pt}%
} &
\the\numexpr(#6+#7+#8)*5\relax\%\\
\end{tabular}%
}
%%
%% end of the preamble, start of the body of the document source.
\copyrightyear{2025}
\acmYear{2025}
% \setcopyright{cc}
% \setcctype{by}
\setcopyright{rightsretained}
\acmConference[CHI '25]{CHI Conference on Human Factors in Computing Systems}{April 26-May 1, 2025}{Yokohama, Japan}
\acmBooktitle{CHI Conference on Human Factors in Computing Systems (CHI '25), April 26-May 1, 2025, Yokohama, Japan}\acmDOI{10.1145/3706598.3713335}
\acmISBN{979-8-4007-1394-1/25/04}

\begin{document}

%%
%% The "title" command has an optional parameter,
%% allowing the author to define a "short title" to be used in page headers.
% \title{CodeA11y: Making AI Coding Assistants Useful for \\ Accessible Web Development}
\title{CodeA11y: Making AI Coding Assistants Useful for Accessible Web Development}

% or "CodeAI1y: Understanding Developers' A11y Needs"

% or "Find-Fix-Verify: What AI Coding Assistants Need to Know About Developers"

% or "\textit{AI Coding Assistant}: chmod u+rwx a11y"

% Are AI Coding Assistants All You Need for Web Accessibility?

% CodeAlly: What AI Coding Assistants Need to Know About Developers' A11y Practices

%%
%% The "author" command and its associated commands are used to define
%% the authors and their affiliations.
%% Of note is the shared affiliation of the first two authors, and the
%% "authornote" and "authornotemark" commands
\author{Peya Mowar}
\email{pmowar@cs.cmu.edu}
\affiliation{%
  % \department{Robotics Institute}
  \institution{Carnegie Mellon University}
  \city{Pittsburgh}
  \state{PA}
  \country{USA}}

\author{Yi-Hao Peng}
\email{yihaop@cs.cmu.edu}
\affiliation{%
  % \department{Human-Computer Interaction Institute}
  \institution{Carnegie Mellon University}
  \city{Pittsburgh}
  \state{PA}
  \country{USA}}

\author{Jason Wu}
\email{jason_wu8@apple.com}
\affiliation{%
  % \department{Human-Computer Interaction Institute}
  \institution{Apple}
  \city{Seattle}
  \state{WA}
  \country{USA}}

\author{Aaron Steinfeld}
\email{steinfeld@cmu.edu}
\affiliation{%
  % \department{Robotics Institute}
  \institution{Carnegie Mellon University}
  \city{Pittsburgh}
  \state{PA}
  \country{USA}}

\author{Jeffrey P. Bigham}
\email{jbigham@cs.cmu.edu}
\affiliation{%
  % \department{Human-Computer Interaction Institute}
  \institution{Carnegie Mellon University}
  \city{Pittsburgh}
  \state{PA}
  \country{USA}}

% \author{Lars Th{\o}rv{\"a}ld}
% \affiliation{%
%   \institution{The Th{\o}rv{\"a}ld Group}
%   \city{Hekla}
%   \country{Iceland}}
% \email{larst@affiliation.org}

% \author{Valerie B\'eranger}
% \affiliation{%
%   \institution{Inria Paris-Rocquencourt}
%   \city{Rocquencourt}
%   \country{France}
% }

% \author{Aparna Patel}
% \affiliation{%
%  \institution{Rajiv Gandhi University}
%  \city{Doimukh}
%  \state{Arunachal Pradesh}
%  \country{India}}

% \author{Huifen Chan}
% \affiliation{%
%   \institution{Tsinghua University}
%   \city{Haidian Qu}
%   \state{Beijing Shi}
%   \country{China}}

% \author{Charles Palmer}
% \affiliation{%
%   \institution{Palmer Research Laboratories}
%   \city{San Antonio}
%   \state{Texas}
%   \country{USA}}
% \email{cpalmer@prl.com}

% \author{John Smith}
% \affiliation{%
%   \institution{The Th{\o}rv{\"a}ld Group}
%   \city{Hekla}
%   \country{Iceland}}
% \email{jsmith@affiliation.org}

% \author{Julius P. Kumquat}
% \affiliation{%
%   \institution{The Kumquat Consortium}
%   \city{New York}
%   \country{USA}}
% \email{jpkumquat@consortium.net}

%%
%% By default, the full list of authors will be used in the page
%% headers. Often, this list is too long, and will overlap
%% other information printed in the page headers. This command allows
%% the author to define a more concise list
%% of authors' names for this purpose.
% \renewcommand{\shortauthors}{Trovato et al.}

%%
%% The abstract is a short summary of the work to be presented in the
%% article.
\begin{abstract}
  In this work, we present a novel technique for GPU-accelerated Boolean satisfiability (SAT) sampling. Unlike conventional sampling algorithms that directly operate on conjunctive normal form (CNF), our method transforms the logical constraints of SAT problems by factoring their CNF representations into simplified multi-level, multi-output Boolean functions. It then leverages gradient-based optimization to guide the search for a diverse set of valid solutions. Our method operates directly on the circuit structure of refactored SAT instances, reinterpreting the SAT problem as a supervised multi-output regression task. This differentiable technique enables independent bit-wise operations on each tensor element, allowing parallel execution of learning processes. As a result, we achieve GPU-accelerated sampling with significant runtime improvements ranging from $33.6\times$ to $523.6\times$ over state-of-the-art heuristic samplers. We demonstrate the superior performance of our sampling method through an extensive evaluation on $60$ instances from a public domain benchmark suite utilized in previous studies. 


  
  % Generating a wide range of diverse solutions to logical constraints is crucial in software and hardware testing, verification, and synthesis. These solutions can serve as inputs to test specific functionalities of a software program or as random stimuli in hardware modules. In software verification, techniques like fuzz testing and symbolic execution use this approach to identify bugs and vulnerabilities. In hardware verification, stimulus generation is particularly vital, forming the basis of constrained-random verification. While generating multiple solutions improves coverage and increases the chances of finding bugs, high-throughput sampling remains challenging, especially with complex constraints and refined coverage criteria. In this work, we present a novel technique that enables GPU-accelerated sampling, resulting in high-throughput generation of satisfying solutions to Boolean satisfiability (SAT) problems. Unlike conventional sampling algorithms that directly operate on conjunctive normal form (CNF), our method refines the logical constraints of SAT problems by transforming their CNF into simplified multi-level Boolean expressions. It then leverages gradient-based optimization to guide the search for a diverse set of valid solutions.
  % Our method specifically takes advantage of the circuit structure of refined SAT instances by using GD to learn valid solutions, reinterpreting the SAT problem as a supervised multi-output regression task. This differentiable technique enables independent bit-wise operations on each tensor element, allowing parallel execution of learning processes. As a result, we achieve GPU-accelerated sampling with significant runtime improvements ranging from $10\times$ to $1000\times$ over state-of-the-art heuristic samplers. Specifically, we demonstrate the superior performance of our sampling method through an extensive evaluation on $60$ instances from a public domain benchmark suite utilized in previous studies.

\end{abstract}

\begin{IEEEkeywords}
Boolean Satisfiability, Gradient Descent, Multi-level Circuits, Verification, and Testing.
\end{IEEEkeywords}

%%
%% The code below is generated by the tool at http://dl.acm.org/ccs.cfm.
%% Please copy and paste the code instead of the example below.
%%
\begin{CCSXML}
<ccs2012>
   <concept>
       <concept_id>10003120.10011738.10011774</concept_id>
       <concept_desc>Human-centered computing~Accessibility design and evaluation methods</concept_desc>
       <concept_significance>500</concept_significance>
       </concept>
   <concept>
       <concept_id>10011007.10011006.10011066</concept_id>
       <concept_desc>Software and its engineering~Development frameworks and environments</concept_desc>
       <concept_significance>500</concept_significance>
       </concept>
   <concept>
       <concept_id>10003120.10003121.10003129</concept_id>
       <concept_desc>Human-centered computing~Interactive systems and tools</concept_desc>
       <concept_significance>500</concept_significance>
       </concept>
 </ccs2012>
\end{CCSXML}

\ccsdesc[500]{Human-centered computing~Accessibility design and evaluation methods}
\ccsdesc[500]{Software and its engineering~Development frameworks and environments}
\ccsdesc[500]{Human-centered computing~Interactive systems and tools}

%%
%% Keywords. The author(s) should pick words that accurately describe
%% the work being presented. Separate the keywords with commas.
\keywords{AI Coding Assistants, Web Accessibility, Coding Agents, AI Agents}

%% A "teaser" image appears between the author and affiliation
%% information and the body of the document, and typically spans the
%% page.
% \begin{teaserfigure}
%   \includegraphics[width=\textwidth]{sampleteaser}
%   \caption{Seattle Mariners at Spring Training, 2010.}
%   \Description{Enjoying the baseball game from the third-base
%   seats. Ichiro Suzuki preparing to bat.}
%   \label{fig:teaser}
% \end{teaserfigure}

% \received{20 February 2007}
% \received[revised]{12 March 2009}
% \received[accepted]{5 June 2009}

%%
%% This command processes the author and affiliation and title
%% information and builds the first part of the formatted document.
\begin{teaserfigure}
    \includegraphics[width=\textwidth]{assets/teaser.png}
    % window 3 cut off
    \caption{CodeA11y is a GitHub Copilot Extension for Accessible Web Development. CodeA11y addresses accessibility limitations of Copilot observed in our study with developers through three features: (1) accessibility-by-default code suggestions, (2) automatic identification of relevant accessibility errors, and (3) reminders to replace placeholders in generated code. Integrating these features directly into AI coding assistants would improve the accessibility of the user interfaces (UIs) developers create.
    }
    \Description{The interface of the extension is displayed along with its features: (1) Accessibility-by-default Code Suggestions, (2) Relevant Error Identification and (3) Code Completion Reminders.}
    \label{fig:teaser}
\end{teaserfigure}

\maketitle
% \tableofcontents <add: remove this>

\title{CodeA11y: Making AI Coding Assistants Useful for Accessible Web Development}

\section{Introduction}\label{sec:Intro} 


Novel view synthesis offers a fundamental approach to visualizing complex scenes by generating new perspectives from existing imagery. 
This has many potential applications, including virtual reality, movie production and architectural visualization \cite{Tewari2022NeuRendSTAR}. 
An emerging alternative to the common RGB sensors are event cameras, which are  
 bio-inspired visual sensors recording events, i.e.~asynchronous per-pixel signals of changes in brightness or color intensity. 

Event streams have very high temporal resolution and are inherently sparse, as they only happen when changes in the scene are observed. 
Due to their working principle, event cameras bring several advantages, especially in challenging cases: they excel at handling high-speed motions 
and have a substantially higher dynamic range of the supported signal measurements than conventional RGB cameras. 
Moreover, they have lower power consumption and require varied storage volumes for captured data that are often smaller than those required for synchronous RGB cameras \cite{Millerdurai_3DV2024, Gallego2022}. 

The ability to handle high-speed motions is crucial in static scenes as well,  particularly with handheld moving cameras, as it helps avoid the common problem of motion blur. It is, therefore, not surprising that event-based novel view synthesis has gained attention, although color values are not directly observed.
Notably, because of the substantial difference between the formats, RGB- and event-based approaches require fundamentally different design choices. %

The first solutions to event-based novel view synthesis introduced in the literature demonstrate promising results \cite{eventnerf, enerf} and outperform non-event-based alternatives for novel view synthesis in many challenging scenarios. 
Among them, EventNeRF \cite{eventnerf} enables novel-view synthesis in the RGB space by assuming events associated with three color channels as inputs. 
Due to its NeRF-based architecture \cite{nerf}, it can handle single objects with complete observations from roughly equal distances to the camera. 
It furthermore has limitations in training and rendering speed: 
the MLP used to represent the scene requires long training time and can only handle very limited scene extents or otherwise rendering quality will deteriorate. 
Hence, the quality of synthesized novel views will degrade for larger scenes. %

We present Event-3DGS (E-3DGS), i.e.,~a new method for novel-view synthesis from event streams using 3D Gaussians~\cite{3dgs} 
demonstrating fast reconstruction and rendering as well as handling of unbounded scenes. 
The technical contributions of this paper are as follows: 
\begin{itemize}
\item With E-3DGS, we introduce the first approach for novel view synthesis from a color event camera that combines 3D Gaussians with event-based supervision. 
\item We present frustum-based initialization, adaptive event windows, isotropic 3D Gaussian regularization and 3D camera pose refinement, and demonstrate that high-quality results can be obtained. %

\item Finally, we introduce new synthetic and real event datasets for large scenes to the community to study novel view synthesis in this new problem setting. 
\end{itemize}
Our experiments demonstrate systematically superior results compared to EventNeRF \cite{eventnerf} and other baselines. 
The source code and dataset of E-3DGS are released\footnote{\url{https://4dqv.mpi-inf.mpg.de/E3DGS/}}. 





\section{Related Work}
\label{lit_review}

\begin{highlight}
{

Our research builds upon {\em (i)} Assessing Web Accessibility, {\em (ii)} End-User Accessibility Repair, and {\em (iii)} Developer Tools for Accessibility.

\subsection{Assessing Web Accessibility}
From the earliest attempts to set standards and guidelines, web accessibility has been shaped by a complex interplay of technical challenges, legal imperatives, and educational campaigns. Over the past 25 years, stakeholders have sought to improve digital inclusion by establishing foundational standards~\cite{chisholm2001web, caldwell2008web}, enforcing legal obligations~\cite{sierkowski2002achieving, yesilada2012understanding}, and promoting a broader culture of accessibility awareness among developers~\cite{sloan2006contextual, martin2022landscape, pandey2023blending}. 
Despite these longstanding efforts, systemic accessibility issues persist. According to the 2024 WebAIM Million report~\cite{webaim2024}, 95.9\% of the top one million home pages contained detectable WCAG violations, averaging nearly 57 errors per page. 
These errors take many forms: low color contrast makes the interface difficult for individuals with color deficiency or low vision to read text; missing alternative text leaves users relying on screen readers without crucial visual context; and unlabeled form inputs or empty links and buttons hinder people who navigate with assistive technologies from completing basic tasks. 
Together, these accessibility issues not only limit user access to critical online resources such as healthcare, education, and employment but also result in significant legal risks and lost opportunities for businesses to engage diverse audiences. Addressing these pervasive issues requires systematic methods to identify, measure, and prioritize accessibility barriers, which is the first step toward achieving meaningful improvements.

Prior research has introduced methods blending automation and human evaluation to assess web accessibility. Hybrid approaches like SAMBA combine automated tools with expert reviews to measure the severity and impact of barriers, enhancing evaluation reliability~\cite{brajnik2007samba}. Quantitative metrics, such as Failure Rate and Unified Web Evaluation Methodology, support large-scale monitoring and comparative analysis, enabling cost-effective insights~\cite{vigo2007quantitative, martins2024large}. However, automated tools alone often detect less than half of WCAG violations and generate false positives, emphasizing the need for human interpretation~\cite{freire2008evaluation, vigo2013benchmarking}. Recent progress with large pretrained models like Large Language Models (LLMs)~\cite{dubey2024llama,bai2023qwen} and Large Multimodal Models (LMMs)~\cite{liu2024visual, bai2023qwenvl} offers a promising step forward, automating complex checks like non-text content evaluation and link purposes, achieving higher detection rates than traditional tools~\cite{lopez2024turning, delnevo2024interaction}. Yet, these large models face challenges, including dependence on training data, limited contextual judgment, and the inability to simulate real user experiences. These limitations underscore the necessity of combining models with human oversight for reliable, user-centered evaluations~\cite{brajnik2007samba, vigo2013benchmarking, delnevo2024interaction}. 

Our work builds on these prior efforts and recent advancements by leveraging the capabilities of large pretrained models while addressing their limitations through a developer-centric approach. CodeA11y integrates LLM-powered accessibility assessments, tailored accessibility-aware system prompts, and a dedicated accessibility checker directly into GitHub Copilot---one of the most widely used coding assistants. Unlike standalone evaluation tools, CodeA11y actively supports developers throughout the coding process by reinforcing accessibility best practices, prompting critical manual validations, and embedding accessibility considerations into existing workflows.
% This pervasive shortfall reflects the difficulty of scaling traditional approaches---such as manual audits and automated tools---that either demand immense human effort or lack the nuanced understanding needed to capture real-world user experiences. 
%
% In response, a new wave of AI-driven methods, many powered by large language models (LLMs), is emerging to bridge these accessibility detection and assessment gaps. Early explorations, such as those by Morillo et al.~\cite{morillo2020system}, introduced AI-assisted recommendations capable of automatic corrections, illustrating how computational intelligence can tackle the repetitive, common errors that plague large swaths of the web. Building on this foundation, Huang et al.~\cite{huang2024access} proposed ACCESS, a prompt-engineering framework that streamlines the identification and remediation of accessibility violations, while López-Gil et al.~\cite{lopez2024turning} demonstrated how LLMs can help apply WCAG success criteria more consistently---reducing the reliance on manual effort. Beyond these direct interventions, recent work has also begun integrating user experiences more seamlessly into the evaluation process. For example, Huq et al.~\cite{huq2024automated} translate user transcripts and corresponding issues into actionable test reports, ensuring that accessibility improvements align more closely with authentic user needs.
% However, as these AI-driven solutions evolve, researchers caution against uncritical adoption. Othman et al.~\cite{othman2023fostering} highlight that while LLMs can accelerate remediation, they may also introduce biases or encourage over-reliance on automated processes. Similarly, Delnevo et al.~\cite{delnevo2024interaction} emphasize the importance of contextual understanding and adaptability, pointing to the current limitations of LLM-based systems in serving the full spectrum of user needs. 
% In contrast to this backdrop, our work introduces and evaluates CodeA11y, an LLM-augmented extension for GitHub Copilot that not only mitigates these challenges by providing more consistent guidance and manual validation prompts, but also aligns AI-driven assistance with developers’ workflows, ultimately contributing toward more sustainable propulsion for building accessible web.

% Broader implications of inaccessibility—legal compliance, ethical concerns, and user experience
% A Historical Review of Web Accessibility Using WAVE
% "I tend to view ads almost like a pestilence": On the Accessibility Implications of Mobile Ads for Blind Users

% In the research domain, several methods have been developed to assess and enhance web accessibility. These include incorporating feedback into developer tools~\cite{adesigner, takagi2003accessibility, bigham2010accessibility} and automating the creation of accessibility tests and reports for user interfaces~\cite{swearngin2024towards, taeb2024axnav}. 

% Prior work has also studied accessibility scanners as another avenue of AI to improve web development practices~\cite{}.
% However, a persistent challenge is that developers need to be aware of these tools to utilize them effectively. With recent advancements in LLMs, developers might now build accessible websites with less effort using AI assistants. However, the impact of these assistants on the accessibility of their generated code remains unclear. This study aims to investigate these effects.

\subsection{End-user Accessibility Repair}
In addition to detecting accessibility errors and measuring web accessibility, significant research has focused on fixing these problems.
Since end-users are often the first to notice accessibility problems and have a strong incentive to address them, systems have been developed to help them report or fix these problems.

Collaborative, or social accessibility~\cite{takagi2009collaborative,sato2010social}, enabled these end-user contributions to be scaled through crowd-sourcing.
AccessMonkey~\cite{bigham2007accessmonkey} and Accessibility Commons~\cite{kawanaka2008accessibility} were two examples of repositories that store accessibility-related scripts and metadata, respectively.
Other work has developed browser extensions that leverage crowd-sourced databases to automatically correct reading order, alt-text, color contrast, and interaction-related issues~\cite{sato2009s,huang2015can}.

One drawback of collaborative accessibility approaches is that they cannot fix problems for an ``unseen'' web page on-demand, so many projects aim to automatically detect and improve interfaces without the need for an external source of fixes.
A large body of research has focused on making specific web media (e.g., images~\cite{gleason2019making,guinness2018caption, twitterally, gleason2020making, lee2021image}, design~\cite{potluri2019ai,li2019editing, peng2022diffscriber, peng2023slide}, and videos~\cite{pavel2020rescribe,peng2021say,peng2021slidecho,huh2023avscript}) accessible through a combination of machine learning (ML) and user-provided fixes.
Other work has focused on applying more general fixes across all websites.

Opportunity accessibility addressed a common accessibility problem of most websites: by default, content is often hard to see for people with visual impairments, and many users, especially older adults, do not know how to adjust or enable content zooming~\cite{bigham2014making}.
To this end, a browser script (\texttt{oppaccess.js}) was developed that automatically adjusted the browser's content zoom to maximally enlarge content without introducing adverse side-effects (\textit{e.g.,} content overlap).
While \texttt{oppaccess.js} primarily targeted zoom-related accessibility, recent work aimed to enable larger types of changes, by using LLMs to modify the source code of web pages based on user questions or directives~\cite{li2023using}.

Several efforts have been focused on improving access to desktop and mobile applications, which present additional challenges due to the unavailability of app source code (\textit{e.g.,} HTML).
Prefab is an approach that allows graphical UIs to be modified at runtime by detecting existing UI widgets, then replacing them~\cite{dixon2010prefab}.
Interaction Proxies used these runtime modification strategies to ``repair'' Android apps by replacing inaccessible widgets with improved alternatives~\cite{zhang2017interaction, zhang2018robust}.
The widget detection strategies used by these systems previously relied on a combination of heuristics and system metadata (\textit{e.g.,} the view hierarchy), which are incomplete or missing in the accessible apps.
To this end, ML has been employed to better localize~\cite{chen2020object} and repair UI elements~\cite{chen2020unblind,zhang2021screen,wu2023webui,peng2025dreamstruct}.

In general, end-user solutions to repairing application accessibility are limited due to the lack of underlying code and knowledge of the semantics of the intended content.

\subsection{Developer Tools for Accessibility}
Ultimately, the best solution for ensuring an accessible experience lies with front-end developers. Many efforts have focused on building adequate tooling and support to help developers with ensuring that their UI code complies with accessibility standards.

Numerous automated accessibility testing tools have been created to help developers identify accessibility issues in their code: i) static analysis tools, such as IBM Equal Access Accessibility Checker~\cite{ibm2024toolkit} or Microsoft Accessibility Insights~\cite{accessibilityinsights2024}, scan the UI code's compliance with predefined rules derived from accessibility guidelines; and ii) dynamic or runtime accessibility scanners, such as Chrome Devtools~\cite{chromedevtools2024} or axe-Core Accessibility Engine~\cite{deque2024axe}, perform real-time testing on user interfaces to detect interaction issues not identifiable from the code structure. While these tools greatly reduce the manual effort required for accessibility testing, they are often criticized for their limited coverage. Thus, experts often recommend manually testing with assistive technologies to uncover more complex interaction issues. Prior studies have created accessibility crawlers that either assist in developer testing~\cite{swearngin2024towards,taeb2024axnav} or simulate how assistive technologies interact with UIs~\cite{10.1145/3411764.3445455, 10.1145/3551349.3556905, 10.1145/3544548.3580679}.

Similar to end-user accessibility repair, research has focused on generating fixes to remediate accessibility issues in the UI source code. Initial attempts developed heuristic-based algorithms for fixing specific issues, for instance, by replacing text or background color attributes~\cite{10.1145/3611643.3616329}. More recent work has suggested that the code-understanding capabilities of LLMs allow them to suggest more targeted fixes.
For example, a study demonstrated that prompting ChatGPT to fix identified WCAG compliance issues in source code could automatically resolve a significant number of them~\cite{othman2023fostering}. Researchers have sought to leverage this capability by employing a multi-agent LLM architecture to automatically identify and localize issues in source code and suggest potential code fixes~\cite{mehralian2024automated}.

While the approaches mentioned above focus on assessing UI accessibility of already-authored code (\textit{i.e.,} fixing existing code), there is potential for more proactive approaches.
For example, LLMs are often used by developers to generate UI source code from natural language descriptions or tab completions~\cite{chen2021evaluating,GitHubCopilot,lozhkov2024starcoder,hui2024qwen2,roziere2023code,zheng2023codegeex}, but LLMs frequently produce inaccessible code by default~\cite{10.1145/3677846.3677854,mowar2024tab}, leading to inaccessible output when used by developers without sufficient awareness of accessibility knowledge.
The primary focus of this paper is to design a more accessibility-aware coding assistant that both produces more accessible code without manual intervention (\textit{e.g.,} specific user prompting) and gradually enables developers to implement and improve accessibility of automatically-generated code through IDE UI modifications (\textit{e.g.}, reminder notifications).

}
\end{highlight}



% Work related to this paper includes {\em (i)} Web Accessibility and {\em (ii)} Developer Practices in AI-Assisted Programming.

% \ipstart{Web Accessibility: Practice, Evaluation, and Improvements} Substantial efforts have been made to set accessibility standards~\cite{chisholm2001web, caldwell2008web}, establish legal requirements~\cite{sierkowski2002achieving, yesilada2012understanding}, and promote education and advocacy among developers~\cite{sloan2006contextual, martin2022landscape, pandey2023blending}. In the research domain, several methods have been developed to assess and enhance web accessibility. These include incorporating feedback into developer tools~\cite{adesigner, takagi2003accessibility, bigham2010accessibility} and automating the creation of accessibility tests and reports for user interfaces~\cite{swearngin2024towards, taeb2024axnav}. 
% % Prior work has also studied accessibility scanners as another avenue of AI to improve web development practices~\cite{}.
% However, a persistent challenge is that developers need to be aware of these tools to utilize them effectively. With recent advancements in LLMs, developers might now build accessible websites with less effort using AI assistants. However, the impact of these assistants on the accessibility of their generated code remains unclear. This study aims to investigate these effects.

% \ipstart{Developer Practices in AI-Assisted Programming}
% Recent usability research on AI-assisted development has examined the interaction strategies of developers while using AI coding assistants~\cite{barke2023grounded}.
% They observed developers interacted with these assistants in two modes -- 1) \textit{acceleration mode}: associated with shorter completions and 2) \textit{exploration mode}: associated with long completions.
% Liang {\em et al.} \cite{liang2024large} found that developers are driven to use AI assistants to reduce their keystrokes, finish tasks faster, and recall the syntax of programming languages. On the other hand, developers' reason for rejecting autocomplete suggestions was the need for more consideration of appropriate software requirements. This is because primary research on code generation models has mainly focused on functional correctness while often sidelining non-functional requirements such as latency, maintainability, and security~\cite{singhal2024nofuneval}. Consequently, there have been increasing concerns about the security implications of AI-generated code~\cite{sandoval2023lost}. Similarly, this study focuses on the effectiveness and uptake of code suggestions among developers in mitigating accessibility-related vulnerabilities. 


% ============================= additional rw ============================================
% - Paulina Morillo, Diego Chicaiza-Herrera, and Diego Vallejo-Huanga. 2020. System of Recommendation and Automatic Correction of Web Accessibility Using Artificial Intelligence. In Advances in Usability and User Experience, Tareq Ahram and Christianne Falcão (Eds.). Springer International Publishing, Cham, 479–489
% - Juan-Miguel López-Gil and Juanan Pereira. 2024. Turning manual web accessibility success criteria into automatic: an LLM-based approach. Universal Access in the Information Society (2024). https://doi.org/10.1007/s10209-024-01108-z
% - s
% - Calista Huang, Alyssa Ma, Suchir Vyasamudri, Eugenie Puype, Sayem Kamal, Juan Belza Garcia, Salar Cheema, and Michael Lutz. 2024. ACCESS: Prompt Engineering for Automated Web Accessibility Violation Corrections. arXiv:2401.16450 [cs.HC] https://arxiv.org/abs/2401.16450
% - Syed Fatiul Huq, Mahan Tafreshipour, Kate Kalcevich, and Sam Malek. 2025. Automated Generation of Accessibility Test Reports from Recorded User Transcripts. In Proceedings of the 47th International Conference on Software Engineering (ICSE) (Ottawa, Ontario, Canada). IEEE. https://ics.uci.edu/~seal/publications/2025_ICSE_reca11.pdf To appear in IEEE Xplore
% - Achraf Othman, Amira Dhouib, and Aljazi Nasser Al Jabor. 2023. Fostering websites accessibility: A case study on the use of the Large Language Models ChatGPT for automatic remediation. In Proceedings of the 16th International Conference on PErvasive Technologies Related to Assistive Environments (Corfu, Greece) (PETRA ’23). Association for Computing Machinery, New York, NY, USA, 707–713. https://doi.org/10.1145/3594806.3596542
% - Zsuzsanna B. Palmer and Sushil K. Oswal. 0. Constructing Websites with Generative AI Tools: The Accessibility of Their Workflows and Products for Users With Disabilities. Journal of Business and Technical Communication 0, 0 (0), 10506519241280644. https://doi.org/10.1177/10506519241280644
% ============================= additional rw ============================================
\section{Formative Study Methods}
\label{form_methods}
We conducted a formative study to assess the implications of AI coding assistants on web accessibility. We recruited novice developers and tasked them with editing real-world websites using GitHub Copilot. Our goal was to better understand how the use of Copilot affected the accessibility of the user interface code they produced.


%Tasks were chosen from among real issues. We chose issues for which accessibility needed to be considered to complete them correctly, but accessibility was not explicitly mentioned as a requirement in either the task description given to participants or the issue description on the real website.


\ipstart{Tasks}
The participants completed tasks in the codebases for two open-source websites, Kubernetes~\cite{kubernetes} and BBC News~\cite{bbcnews}. Both websites received over 2 million monthly visits worldwide~\cite{similarweb2024} and belong to different categories in the IAB Content Taxonomy~\cite{webshrinker2024}. These websites were developed using different web development frameworks (Hugo and React, respectively). To choose the four specific tasks used in this formative study (Table~\ref{tab:tasks}), we sampled actual issues from each website's repository. We chose issues for which accessibility needed to be considered to complete them correctly, but accessibility was not explicitly mentioned as a requirement in either the task description given to participants or on the issue description on the website's code repository, as illustrated in Figure~\ref{fig:tasks}. Correctly performing the tasks required the consideration of several common web accessibility issues: color contrast, alternative text, link labels, and form labeling~\cite{webaim2024}. The goal was to mirror the kinds of specifications that developers often receive that do not explicitly mention accessibility.

% Based on observations from pilot studies, including success rates and completion times, we classified these tasks from easy to difficult.


\begin{highlight}
\begin{figure*}
    % \tcbox[colframe=purple, colback=white, boxrule=0.35mm, arc=0mm, left=0mm, right=0mm, top=0mm, bottom=0mm]{
    %     \includegraphics[width=\textwidth, trim=15 177 12 140, clip]{assets/tasks.png}
    % }
    \includegraphics[width=\textwidth, trim=15 280 12 100, clip]{assets/tasks.png}
    \caption{\begin{highlight}Examples of task descriptions and visual references given to our participants: (a) Task 2 was to implement a new contact form for subscribing to a mailing list, and (b) Task 3 was to add a `Top Stories' section with linked articles. Successfully completing them required proper labeling of the form elements and links, but this was not explicitly stated in the instructions.\end{highlight}}
    \Description{The figure depicts the two tasks: i) Task T2's instruction is about developing a subscription feature for a weekly mailing list on the Kubernetes Website. It's visual reference includes a subscription box labeled "email address" with a "Subscribe" button, inviting users to sign up for "KubeWeekly" updates. ii) Task T3's instruction is to develop the frontend for adding the Top Stories section on the BBC Website. t's visual reference includes a list of headlines with corresponding dates styled as clickable elements.}
    \label{fig:tasks}
\end{figure*}
\end{highlight}

\begin{table*}
  \caption{Our formative study included four tasks. Each task was not primarily about accessibility but included an accessibility issue that was required to complete the task successfully. \begin{highlight} We adopt the scales of Unacceptable, Average and Good from prior work~\cite{pillai2022website}. Uninformative attributes are those that merely reflect the field, such as `alt' as alt-text or `click here' as link description, without providing more meaningful or descriptive content~\cite{ross2018examining}. Tasks are ranked from easy to difficult based on the time taken and success rates observed in our pilot studies.\end{highlight}}
  \label{tab:tasks}
  \small
  \begin{tabular}{|p{0.185\textwidth}|p{0.075\textwidth}|p{0.13\textwidth}|p{0.525\textwidth}|}
    \toprule
    \textbf{Task} & \textbf{Difficulty} & \textbf{Accessibility Issue} & \textbf{Evaluation Criteria}\\
    \midrule
    (T1) Button Visibility & Easy & Color Contrast & \textit{Unacceptable}: contrast ratio of $<$ 4.5:1 for normal text and $<$ 3:1 for large text\\
    & & & \textit{Average}: WCAG level AA in default state (contrast ratio of $>=$ 4.5:1 for normal text) \\
    & & & \textit{Good}: WCAG level AA in all states (default, hover, active, focus, etc.)\\
    \midrule
    (T2) Form Element & Moderate & Form Labeling & \textit{Unacceptable}: Missing form labels and keyboard navigation \\
    & & & \textit{Average}: One of form labels and keyboard navigation\\
    & & & \textit{Good}: Both form labeling and keyboard navigation \\
    \midrule
    (T3) Add Section & Moderate & Link Labeling & \textit{Unacceptable}: Missing link descriptions \\
    & & & \textit{Average}: Uninformative link descriptions \\
    & & & \textit{Good}: Descriptive link descriptions\\
    \midrule
    (T4) Enhance Image for SEO & Difficult & Adding alt-text & \textit{Unacceptable}: Missing or uninformative alt-text\\
     & & & \textit{Average}: Added alt-text with $<$ 3 required descriptors~\cite{10.1145/3441852.3471207}\\
    & & & \textit{Good}: Added alt-text with $>=$ 3 out of 4 required descriptors\\
    \bottomrule
  \end{tabular}
\end{table*}

% Based on our observations from pilot studies, we classified the tasks as easy to difficult and added time bounds of 30 minutes to each task. 
\begin{comment}
    \begin{table}
  \caption{Study Task Descriptions}
  \label{tab:tasks}
  \small
  \begin{tabular}{m{0.3\textwidth}|p{0.65\textwidth}}
    \toprule
    \textbf{Task} & \textbf{Description}\\
    \midrule
    \makecell[l]{(Task 1) \\ Improving a button's visibility} & An easy task to modify an existing button's styling to improve its visibility.\\
    \midrule
     \makecell[l]{(Task 2) \\ Adding a form element} & A moderate task to implement a new contact form for subscribing to a mailing list, focusing on functionality and user experience.\\
    \midrule
     \makecell[l]{(Task 3) \\ Adding "Top Stories" section} & A moderate task to add links to the top stories in the secondary column on the BBC news article page using data from a provided JSON file.\\
    \midrule
     \makecell[l]{(Task 4) \\ Enhancing image for SEO optimisation} & A difficult task to improve the discoverability of a BBC article image by adding alt-text in the article JSON file and rendering it using an image component.\\
    \bottomrule
  \end{tabular}
\end{table}
\end{comment}


\begin{comment}
\begin{table}
  \caption{Manual Evaluation Criteria for Web Accessibility}
  \label{tab:a11y}
  \small
  \begin{tabular}{p{0.175\textwidth}|p{0.775\textwidth}}
    \toprule
    \textbf{Task Category} & \textbf{Evaluation Criteria}\\
    \midrule
    & \textit{Unacceptable}: contrast ratio of < 4.5:1 for normal text and < 3:1 for large text\\ 
    Button colour contrast  & \textit{Average}: WCAG level AA in default state: 
    minimum contrast ratio of 4.5:1 for normal text\\ 
    & \textit{Good}: WCAG level AA in all states (default, hover, active, focus, etc.)\\ 
    \midrule
    & \textit{Unacceptable}: Missing or uninformative form labels\\ 
    Form labeling & \textit{Average}: Somewhat descriptive form labels~\cite{pillai2022website}\\
    & \textit{Good}: Descriptive form labels and keyboard navigation\\
    \midrule
    & \textit{Unacceptable}: Missing or uninformative ~\cite{ross2018examining} link descriptions\\  
    Link labeling & \textit{Average}: Somewhat descriptive links~\cite{pillai2022website}\\
    & \textit{Good}: Descriptive link labels\\
    & \textit{Unacceptable}: Missing or uninformative~\cite{pillai2022website} alt-text\\
    Adding alt-text & \textit{Average}: Added alt-text with < 3 required descriptors~\cite{10.1145/3441852.3471207}\\
    & \textit{Good}: Added alt-text with >= 3 out of 4 required descriptors\\
    \bottomrule
  \end{tabular}
\end{table}
\end{comment}

\begin{table}[h!]
\caption{\begin{highlight}
Participant User Groups: Each group is assigned specific order of tasks and testing conditions. Participants are evenly and randomly distributed among these groups.
\end{highlight}}
\centering
\small
\label{tab:usergroup}
\begin{highlight}
\begin{tabular}{|p{0.01\textwidth}|p{0.2\textwidth}|p{0.2\textwidth}|}
\toprule
\textbf{\#} & \textbf{Order 1, Testing Condition} & \textbf{Order 2, Testing Condition} \\ \midrule
1 & Kubernetes, With AI Assistance & BBC News, No AI Assistance \\ \midrule
2 & Kubernetes, No AI Assistance & BBC News, With AI Assistance \\ \midrule
3 & BBC News, With AI Assistance & Kubernetes, No AI Assistance \\ \midrule
4 & BBC News, No AI Assistance & Kubernetes, With AI Assistance \\ \midrule
\end{tabular}
\end{highlight}
\end{table}

\ipstart{Protocol}
Our within-subjects user study had two conditions: (1) a control condition where participants received no AI assistance, and (2) a test condition where participants used GitHub Copilot. Each participant was assigned to edit two distinct websites, each with two tasks. To counterbalance order effects, participants were evenly and randomly assigned to one of four user groups (Table~\ref{tab:usergroup}), balanced by website order and control/test conditions. Further, to simulate real-world scenarios, we concealed the true purpose of the study from participants. Participants were informed that the study was about the usability of AI pair programmers in web development tasks but were not explicitly instructed to make their web components accessible. This allowed us to observe how developers naturally handle accessibility when it is not explicitly emphasized, reflecting typical developer behavior. The research protocol was reviewed and approved by the Institutional Review Board (IRB) at our university.

\ipstart{Participants}
We employed convenience sampling and snowball sampling methods to recruit our participants. Our study was advertised on university bulletin boards, social media, and shared communication channels (Twitter, Slack, and mailing groups). Our recruitment criteria stipulated that participants must be over 18 years of age, live in the United States, and have self-assessed familiarity with web development. Further, we required the participants to be physically present on our university campus for the duration of the study. To avoid priming during participant recruitment, we did not stipulate awareness of web accessibility as an eligibility criterion. We chose university-specific avenues for recruiting CS students, that reflect a typical novice developer cohort.

Our study enlisted 16 participants (7 female and 9 male; ages ranged from 22 to 29). Almost all of our participants were students and had multiple years of coding experience. Most (n=10) had multi-year \textit{industrial} programming experience (e.g., full-time or intern experience in the company). Nearly all participants (except one) had previously used AI coding assistants. GitHub Copilot and OpenAI ChatGPT were the most popular (n=10). Others preferred Tabnine (n=6) and AWS CodeWhisperer (n=2). 12 participants had self-described substantial experience with HTML and CSS. 10 were proficient in JavaScript and 7 were proficient in React.js. Despite this expertise, the majority (14 participants) were unfamiliar with the Web Content Accessibility Guidelines (WCAG). Only 2 participants knew about these guidelines, but they had not actively engaged in creating accessible web user interfaces or received formal training on the subject (details are provided in Table~\ref{tab:awareness}).

\begin{table*}
  \caption{The distribution of participants' opinions on AI-powered programming tools and their awareness of web accessibility. The percentages in the distribution column indicate the proportion of participants who either disagree (including both `strongly disagree' and `disagree') or agree (including both `strongly agree' and `agree') with the provided statements.}
  \label{tab:awareness}
  \begin{tabular}{>{\raggedright\arraybackslash}p{0.6\textwidth}|>{\raggedright\arraybackslash}p{0.35\textwidth}}
    \toprule
    \textbf{Statement} & \textbf{Distribution}\\
    \midrule
    ``I trust the accuracy of AI programming tools.''& \likertpct{0}{0}{2}{10}{4}{0}{0}\\
    ``I am proficient in web accessibility.''& \likertpct{9}{0}{3}{1}{2}{0}{1}\\
    ``I am familiar with the web accessibility standards, such as WCAG 2.0.''& \likertpct{11}{0}{3}{0}{0}{0}{2}\\
    ``I am familiar with ARIA roles, states, and properties.''& \likertpct{10}{0}{1}{1}{3}{0}{1}\\
  \midrule
\end{tabular}
% \vspace{1em}
\begin{tabular}{@{}>{\centering\arraybackslash}p{\textwidth}@{}}
        \textcolor{customorange}{\rule{7pt}{7pt}} Strongly Disagree \quad
        \textcolor{custompeach}{\rule{7pt}{7pt}} Disagree \quad
        \textcolor{customgray}{\rule{7pt}{7pt}} Neutral \quad
        \textcolor{customlightblue}{\rule{7pt}{7pt}} Agree \quad
        \textcolor{customblue}{\rule{7pt}{7pt}} Strongly Agree \\
    \bottomrule
    \end{tabular}
\end{table*}


\ipstart{Procedure}
The study was conducted in person at our lab, where participants performed programming tasks on a MacBook Pro laptop equipped with IntelliJ IDEA with the GitHub Copilot plugin preinstalled. Before starting the study, we explained the study procedure to the participants and took their informed consent. The participants then watched a 5-minute instructional video explaining Copilot's features, such as code autocompletion and the Copilot chat\footnote{\url{https://www.youtube.com/watch?v=jXp5D5ZnxGM}}. Participants were assigned tasks related to two selected websites, with a total of four tasks to complete in 90 minutes. They were required to work on one website with and the other without GitHub Copilot. Further, they were allowed to access the web for task exploration or code documentation through traditional search engines like Google Search, but with generative results turned off. During the coding session, a researcher observed silently, offering help with tasks, tool usage, or debugging only if participants were stuck, and asked them to move on after 30 minutes, without giving any accessibility-related hints. Based on our observations from pilot studies, we set time limits ranging from 15 to 30 minutes per task. Finally, after completing the coding tasks, they were asked to participate in a 10-15 minute survey on their experience in AI-assisted programming and web accessibility, development expertise, and open-ended feedback. In the end, the participants were compensated with a gift voucher worth 30 USD.



\ipstart{Data Collection and Analysis}
We collected both quantitative and qualitative data for a mixed-method analysis.
For quantitative data, we used an IntelliJ IDEA plugin~\cite{dkandalov_activity_tracker} that tracked user actions --- such as keyboard input (typing, backspace), IDE commands (copy, paste, undo), and interactions with GitHub Copilot (accepting suggestions, opening the Copilot Chat window) --- and recorded their timestamps. 
Additionally, we employed the axe-Core Accessibility Engine 2 to gather accessibility violation metrics, including the type and count of WCAG failures, for each code submission, a method proven reliable in previous studies~\cite{p2023towards}. We also collected AI usage, programming languages and framework preferences, and expertise in web accessibility via a post-task survey.

On the qualitative side, we captured the entire study sessions through screen recordings, resulting in a total of 18.73 hours of video data. We complemented this with observational notes taken during the sessions, documenting verbal comments made by participants. The participants' interactions with Copilot Chat were also recorded for further analysis between prompts and the final code.
The analysis of this data was carried out using open coding and thematic analysis~\cite{clarke2017thematic}. \begin{highlight} Some themes that emerged were: `visual enhancement', `recalling syntax', `feature request', and `code understanding'.\end{highlight} For accessibility evaluation, we manually inspected the websites created during the study and evaluated their accessibility on a qualitative scale of ``Unacceptable,'' ``Average,'' and ``Good'' adopted from prior work~\cite{pillai2022website}. The criteria for these evaluations were developed per best practices identified in prior research published in CHI and ASSETS, detailed further in Table~\ref{tab:tasks}.

% \ipstart{Limitations}
% Firstly, we recruited participants from a single university, suggesting that these individuals may have had access to specific resources and training not universally available, which could limit the generalizability of our findings to a wider population. Additionally, the brief duration of our study may not accurately represent long-term real-world interactions with AI coding assistants; extended study periods could potentially unveil more comprehensive insights into users' ongoing engagement and challenges. Lastly, the laboratory environment of our study, coupled with constant researcher observation, might have introduced an environmental bias, influencing participant behavior. These factors collectively imply that while our study provides important insights into the accessibility and awareness of AI coding assistants, caution should be exercised when extending these findings to the broader developer community.
\section{Formative Findings}
\label{form_finds}
Our formative study revealed that while existing AI coding assistants can produce accessible code, developers still need accessibility expertise for effective use. Otherwise, 
% \begin{enumerate}[noitemsep, topsep=0pt]
%     \item the accessibility introduced by the assistant is likely to not be applied comprehensively,
%     \item the advanced features recommended by the assistant are unlikely to be implemented,
%     \item and the accessibility errors caused by the assistant are unlikely to be caught.
% \end{enumerate}
(1) the accessibility introduced is likely to not be applied comprehensively, (2) the advanced features recommended by the assistant are unlikely to be implemented,  (3) the accessibility errors introduced by the assistant are unlikely to be caught.

% \item accessibility is unlikely to be applied comprehensively,
%     \item advanced features recommended by the assistant are likely to not be implemented, and
%     \item accessibility errors caused by the assistant may not be caught.
\ipstart{Developer Behavior}
In the study, participants spent slightly more time on tasks without Copilot, averaging $30.84$ minutes ($\sigma = 11.95$) compared to $28.94$ minutes ($\sigma = 8.57$) with Copilot. Copilot also facilitated a greater volume of code edits ($13.28$ lines of code, $\sigma = 9.02$ vs $10.41$ lines of code, $\sigma = 5.87$), indicating that AI-assisted workflows encouraged iterative coding practices. However, even with Copilot, participants spent approximately 39.84\% of their task time ($11.91$ minutes, $\sigma = 8.00$) away from the IDE, browsing the web or checking the rendered HTML, highlighting the importance of traditional validation methods. The study also found fewer backspace key presses, an indicator of post-paste corrections, without Copilot ($\mu = 92.62$, $\sigma = 68.27$) than with the AI assistant ($\mu = 104.50$, $\sigma = 91.91$). Further, code pasting was slightly higher when participants solely browsed the web for exploration, averaging $12.68$ times ($\sigma = 8.09$), compared to $11.43$ times ($\sigma = 5.35$) with access to Copilot Chat. Participants dedicated about $7.39$\% ($2.14$ minutes, $\sigma = 1.72$) of their task time typing in the GitHub Copilot chat window, while, they also accepted Copilot's code auto-complete suggestions around $5.44$ times ($\sigma = 5.00$) on average.

\begin{table*}
\caption{The (partial) chat history revealed that directly copying Copilot's code suggestions would be incomplete, as developers overlooked the additional recommended steps.}
  \small
  \begin{tabular}{p{0.06\textwidth}p{0.89\textwidth}}
    \toprule
    \textbf{User}: & \textcolor{customorange}{``\textbf{User Query}''} \\
    \textbf{Copilot}: &  \colorbox{codebgd}{\texttt{\textbf{Generated Code}}}\\ 
    & \colorbox{codehlt}{\textbf{Associated Text Explanation}}\\
    \midrule
    \textbf{User 1}: & \textcolor{customorange}{\textbf{``I want to label the images or add relevant image descriptors''}} \\
    \textbf{Copilot:} & \colorbox{codebgd}{\texttt{alt={imgAlt} // Add this line}} \\
    & \colorbox{codehlt}{\begin{minipage}{\linewidth}
    Remember to replace imageAlt, <...> with your actual image attributes. In this example, imageAlt is the alt-text for the image, and <...>. You will need to replace these with the actual values.
  \end{minipage}}\\
    \midrule    
    \textbf{User 2}: & \textcolor{customorange}{\textbf{``Can you optimise the images on this article page for search engine optimization. Refactor the code accordingly by labeling the images or adding relevant image descriptions.''}} \\
    \textbf{Copilot:} & \colorbox{codebgd}{\texttt{imgAlt={imageDescription} // Add this line}}\\
    & \colorbox{codehlt}{\begin{minipage}{\linewidth}
    Sure, to optimize images for SEO, you can add alt attributes to your img tags. The alt attribute provides alternative information for an image if a user for some reason cannot view it (because of slow connection, an error in the src attribute, or if the user uses a screen reader).
  % \end{minipage}}\\
  %   \midrule
  %   \textbf{User 3}: & \textcolor{red}{\textit{``add alt-text to promo image''}} \\
  %   Copilot: & \colorbox{codebgd}{\texttt{imageAltText={promoImageAltText} // Add this line}}\\
  %   & \colorbox{codebgd}{\texttt{const FrostedGlassPromo = (\{ item, promoImageAltText, ...otherProps \}) => \{//..existing code..\}}}\\
  %   & \colorbox{codebgd}{\texttt{return (}}\\
  %   & \quad \colorbox{codebgd}{\texttt{<div>}} \\
  %   & \quad \quad \colorbox{codebgd}{\texttt{<img src=\{item.image\} alt=\{promoImageAltText\} />}}\\
  %   & \quad \colorbox{codebgd}{\texttt{</div>}}\\
  %   & \colorbox{codebgd}{\texttt{);}}\\
  %   & \colorbox{codehlt}{\begin{minipage}{\linewidth}
  %   Please replace item.promoImageAltText with the actual path to the alt-text in your data structure. Remember, the alt attribute value should be a meaningful description of the image content. It helps search engines understand what the image is about, and it's also used by screen readers to provide context for visually impaired users.
  \end{minipage}}
    \\
  \bottomrule
\end{tabular}
    % \vspace{-30pt}
    \label{tab:responses}
\end{table*}

\ipstart{AI Usage and Prompting Strategies}
Participants mainly used the autocomplete feature only when they had a clear mental model of the desired code structure and sought to accelerate the code typing process. In contrast, they heavily relied on the conversational interface for syntax inquiries, conceptual understanding, and the generation of code templates. We noticed that our participants wrote brief, task-oriented prompts that focused on immediate code solutions or specific interface modifications, often disregarding broader architectural considerations. Their prompting style was iterative and reactive, frequently requesting small incremental changes, fixes to previous outputs, or refinements to their vague prompts. 

Furthermore, none of the participants, including the two who were familiar with web accessibility, prompted with accessibility in mind. Instead, our participants' prompts centered around visual and functional attributes (e.g., `\textit{`add a gray background to the subscription form}'' (P4) or ``\textit{add a grey patch}'' (P1)). Consequently, the AI assistant's suggestions often failed to incorporate accessibility best practices automatically. Occasionally, our participants prompted for enhancements that indirectly aligned with accessibility requirements, and Copilot provided relevant accessibility suggestions, as shown in Table~\ref{tab:responses}. However, participants' overreliance on AI assistance often led them to assume that Copilot's code output was correct and complete. For instance, despite additional explanations from Copilot advising manual adjustments to image descriptions, participants directly pasted the code, resulting in code submissions with empty \colorbox{codebgd}{\texttt{alt}} attributes.

% \begin{figure*}
% \centering
% \begin{tikzpicture}
%     \begin{axis}[
%         ybar,
%         enlarge x limits=0.175,
%         legend style={at={(0.5,-0.2)},anchor=north,legend columns=2},
%         ylabel={Accessibility Evaluation},
%         xlabel={Task Description},
%         ylabel near ticks,
%         symbolic x coords={Button colour contrast, Form labeling, Link labeling, Adding alt-text},
%         xtick=data,
%         nodes near coords,
%         nodes near coords align={vertical},
%         ymin=0,ymax=2.25,
%         ytick={0,1,2},
%         yticklabels={Unacceptable (0), Average (1), Good (2)},
%         bar width=0.65cm,
%         style={font=\small},
%         grid=major,
%         grid style=dashed,
%         width=0.75\textwidth,
%         height=4.5cm
%     ]
%     % Data for "No Copilot"
%     \addplot[draw=black, fill=customorange
%     % pattern={north east lines},pattern color=black
%     ] coordinates {
%         (Adding alt-text, 0.500)
%         (Button colour contrast, 0.75)
%         (Form labeling, 0.625)
%         (Link labeling, 1.750)
%     };
%     % Data for "Yes Copilot"
%     \addplot[draw=black, fill=customlightblue] coordinates {
%         (Adding alt-text, 0.250)
%         (Button colour contrast, 0.75)
%         (Form labeling, 0.875)
%         (Link labeling, 1.750)
%     };
%     \legend{Without Copilot, With Copilot}
%     \pgfplotsset{
%         legend image code/.code={
%             \draw[draw=black] (0.75cm,-0.1cm) rectangle (1.5cm,0.1cm);
%         }
%         }
%     \end{axis}
% \end{tikzpicture}
% \caption{Mean Accessibility Evaluation Scores by Tasks and Copilot Usage: Higher scores indicate that participants were successful.}
% \Description{The image shows a bar graph with mean scores for web accessibility tasks, comparing outcomes with and without GitHub Copilot. Scores range from 0 (Unacceptable), 1 (Average), 2 (Good). Adding alt-text on average worsened from 0.5 to 0.25 with the usage of Copilot. Button colour contrast and link labelling remained at 0.63 and 1.75 respectively regardless of Copilot use. Form labelling improved from 0.63 to 0.88 with Copilot usage.}
% \label{manual-eval}
% \end{figure*}

\ipstart{Implications for Web Accessibility}
Our study showed mixed results of AI assistants in considering accessibility issues with no statistically significant difference between the experimental conditions, as shown in Figure~\ref{manual-eval}. Notably, Copilot could (sporadically) generate accessible components by utilizing patterns from other parts of a website. For example, it might automatically include proper labels for form fields, such as \colorbox{codebgd}{\texttt{<label for="email"> Email: </label>}} in a signup form. However, there were also instances where Copilot inadvertently introduced new accessibility issues. For example, when adding new button components with hover effects, it failed to ensure adequate contrast between the hover color and background. 

Further, the effectiveness of AI assistants was limited by the need for more sophisticated accessibility knowledge. Since our participants had limited awareness about these accessibility features, they would often ignore such suggestions by blindly accepting \colorbox{codebgd}{\texttt{alt = "" // Add your text here}} or manually deleting the \colorbox{codebgd}{\texttt{<label>}} tag. Some errors, such as providing blank alt-texts for informative images, were not even flagged by automated accessibility checkers because they interpret the image as decorative and consider this deliberate. This is particularly problematic as it implies that AI assistance might increase the risk of accessibility oversights, allowing critical errors to go unnoticed and uncorrected.


\begin{figure}
\includegraphics[width=0.49\textwidth,trim=80 65 50 30,clip]{assets/formative_eval.png}
% \centering
% \begin{tikzpicture}
%     \begin{axis}[
%         ybar,
%         enlarge x limits=0.175,
%         legend style={at={(0.5,-0.4)},anchor=north,legend columns=2},
%         ylabel={Accessibility Evaluation},
%         % xlabel={Task Description},
%         % xticklabel style={rotate=25, anchor=east, yshift=-5pt},
%         ylabel near ticks,
%         xticklabel style={yshift=-2.5pt},
%         % ylabel style={xshift=-10pt}
%         symbolic x coords={\shortstack{Color \\ contrast}, \shortstack{Form \\ labeling}, \shortstack{Link \\ labeling}, \shortstack{Adding \\ alt-text}},
%         % symbolic x coords={Colour contrast, Form labeling, Link labeling, Adding alt-text},
%         xtick=data,
%         nodes near coords,
%         nodes near coords align={vertical},
%         ymin=0,ymax=2.25,
%         ytick={0,1,2},
%         yticklabels={Unacceptable (0), Average (1), Good (2)},
%         bar width=0.52cm,
%         style={font=\small},
%         grid=major,
%         grid style=dashed,
%         axis line style={draw=black, line width=0.5pt},
%         width=0.4\textwidth,
%         height=4.5cm,
%         xtick align=inside,
%         ytick align=inside
%     ]
%     % Data for "No Copilot"
%     \addplot[draw=black, fill=customorange,
%     % pattern={north east lines},pattern color=black
%     ] coordinates {
%         (\shortstack{Adding \\ alt-text}, 0.500)
%         (\shortstack{Color \\ contrast}, 0.75)
%         (\shortstack{Form \\ labeling}, 0.625)
%         (\shortstack{Link \\ labeling}, 1.750)
%         % (Adding alt-text, 0.500)
%         % (Colour contrast, 0.75)
%         % (Form labeling, 0.625)
%         % (Link labeling, 1.750)
%     };
%     % Data for "Yes Copilot"
%     \addplot[draw=black, fill=customlightblue,] coordinates {
%         (\shortstack{Adding \\ alt-text}, 0.250)
%         (\shortstack{Color \\ contrast}, 0.75)
%         (\shortstack{Form \\ labeling}, 0.875)
%         (\shortstack{Link \\ labeling}, 1.750)
%         % (Adding alt-text, 0.250)
%         % (Colour contrast, 0.75)
%         % (Form labeling, 0.875)
%         % (Link labeling, 1.750)
%     };
%     \legend{Without Copilot, With Copilot}
%     \pgfplotsset{
%         legend image code/.code={
%             \draw[draw=black] (0.75cm,-0.1cm) rectangle (1.5cm,0.1cm);
%         }
%         }
%     \end{axis}
% \end{tikzpicture}
\caption{Mean Accessibility Evaluation Scores by Tasks and Copilot Usage: Higher scores indicate success.}
\Description{The image shows a bar graph with mean scores for web accessibility tasks, comparing outcomes with and without GitHub Copilot. Scores range from 0 (Unacceptable), 1 (Average), 2 (Good). Adding alt-text on average worsened from 0.5 to 0.25 with the usage of Copilot. Button colour contrast and link labelling remained at 0.63 and 1.75 respectively regardless of Copilot use. Form labelling improved from 0.63 to 0.88 with Copilot usage.}
\label{manual-eval}
\end{figure}

\section{Design Requirements}
\label{design-goals}
Our formative study identified three limitations in novice developers' interactions with AI assistants: (1) failing to prompt for accessibility considerations explicitly, (2) uncritically accepting incomplete code suggestions from Copilot, and (3) struggling to detect potential accessibility issues in their code. These shortcomings indicate possible directions to support accessibility through three design goals (\textbf{G1-G3}):

\ipstart{G1: Integrate System Prompts for Accessibility Awareness} 
Without explicit prompting, the AI assistant rarely produced accessibility-compliant code, reflecting the accessibility issues prevalent in its training data. However, it occasionally suggested accessibility features when participants indirectly prompted them, demonstrating its ability to recall accessibility practices from training data upon instruction. AI assistants should have a system prompt tuned towards following accessibility guidelines by default, for consistent generation of accessibility-compliant code, even when developers do not mention accessibility specifically. Further, the system prompt should also direct the assistant to suggest accessibility-focused iterative refinements.
% and educate developers about accessibility best practices.

\begin{highlight}
\ipstart{G2: Support Identification of Accessibility Issues}
\end{highlight}
Due to their unfamiliarity with accessibility standards, our participants were unable to identify compliance issues in the existing and modified code. They primarily prompted changes to individual components (such as buttons and forms), hardly addressing broader page-level accessibility concerns (such as heading structure or landmark regions). AI assistants should not only automatically generate accessibility-compliant code, but also provide real-time feedback to detect and resolve accessibility violations within the codebase. In addition, AI assistants and automated accessibility checkers should work in tandem to ensure that incomplete or incorrect implementations of the AI-suggested code are always flagged by the latter.

\begin{highlight}
\ipstart{G3: Encourage Developers to Complete AI-Generated Code}
\end{highlight}
Our observations revealed that accessibility implementation in AI-assisted coding workflows commonly required critical manual intervention to complete and validate AI-generated code. This involved replacing placeholder attributes, such as labels and alt-texts, with meaningful values and verifying color contrast ratios. However, we found that participants blindly copy-pasted code and proceeded further if there were no apparent errors. This behavior of deferring thought to suggestions has also been documented in previous work~\cite{mozannar2024reading}. To mitigate this, AI assistants should proactively remind developers to ensure that all necessary accessibility features -- such as contrast ratios or keyboard navigation support -- are fully implemented and verified.
\section{CodeA11y}
\label{system}
Guided by the design goals identified through our user study, we built CodeA11y, a GitHub Copilot Extension for Visual Studio IDE. In this section, we present the interactions that it supports and its system architecture.

%\ipstart{System Overview}
% \ipstart{Design Considerations}
% Focus on conversational agent only
% D1. Accessibility awareness
% D2. Contextual understanding
% D3. User-centric feedback
% \ipstart{System Features}
% F1: Accessibility-aware system prompt
% F2: Reminders for code completion
% F3: Integration with axe a11y checker
% \ipstart{Implementation}
% Technologies used (e.g., LLM models, IDE integration)
% VS Code Plugin Backend
% Copilot Chat Extension -- chat participant
% GPT 4o LLM Agents
% axe accessibility linter
% \begin{highlight}
    
CodeA11y has three primary features (\textbf{F1-F3}, aligned to G1-G3, respectively): (\textbf{F1}) it produces user interface code that better complies with accessibility standards, \begin{highlight}
(\textbf{F2}) it prompts the developer to resolve existing accessibility errors in their website, and (\textbf{F3}) it reminds the developer to complete any AI-generated code that requires manual intervention.
\end{highlight} CodeA11y is integrated into Visual Studio Code as a GitHub Copilot Extension\footnote{\url{https://docs.github.com/en/copilot/using-github-copilot/using-extensions-to-integrate-external-tools-with-copilot-chat}}, enabling CodeA11y to act as a chat participant within the GitHub Copilot Chat window panes. While we implemented this as an extension, it could be integrated directly into Copilot in the future.
% \end{highlight}

% \begin{highlight}

\ipstart{Multi-Agent Architecture}
% \ipstart{Multi-Agent Workflow}
% [Figure: CodeA11y Multi-agent Workflow Diagram]
% Uses 3 LLM Agents -- acts as conversation orchestrator 
% \ipstart{Agent Descriptions}
% a. Code Context Agent
% b. Ally Checker Agent
% c. Follow-up Reminder Agent
% \ipstart{Prompt Instructions}
% [Table: Prompt Instructions for LLM-based Agents]
% GitHub A11y blog prompt
% Prompt optimisations
\begin{table*}
  \caption{Prompt instructions for the three LLM agents in CodeA11y}
  \label{tab:sys_prompt}
  \small
  \begin{tabular}{m{0.14\textwidth}|m{0.81\textwidth}}
    \toprule
    \textbf{Agent} & \textbf{Prompt Instruction Highlights}\\
    \midrule
    Responder Agent & {\begin{itemize}[leftmargin=*, partopsep=7pt]
        \item I am unfamiliar with accessibility and need to write code that conforms with WCAG 2.1 level AA criteria.
        \item Be an accessibility coach that makes me account for all accessibility requirements.
        \item Use reputable sources such as w3.org, webaim.org and provide links and references for additional learning. 
        \item Don't give placeholder variables but tell me where to give meaningful values.
        \item Prioritise my current request and don't mention accessibility if I give a generic request like "Hi".
    \end{itemize}} \\ \midrule
    Correction Agent & {\begin{itemize}[leftmargin=*, partopsep=7pt]
        \item Review the accessibility checker log and provide feedback to fix errors relevant to current chat context.
        \item If a log error relevant to current chat context occurs, provide a code snippet to fix it.
    \end{itemize}} \\ \midrule
    Reminder Agent & {\begin{itemize}[leftmargin=*, partopsep=7pt]
        \item Is there an additional step required by the developer to meet accessibility standards after pasting code?
        \item Reminder should be single line. Be conservative in your response, if not needed, say "No reminders needed."
        \item For example, remind the developer to replace the placeholder attributes with meaningful values or labels, or visually inspect element for colour contrast when needed.
    \end{itemize}}\\ 
    \bottomrule
  \end{tabular}
\end{table*}
CodeA11y has three LLM agents (Figure~\ref{fig:arch}): Responder Agent, Correction Agent, and Reminder Agent. We provide their prompt instruction highlights in Table~\ref{tab:sys_prompt}. These agents facilitate each of the above features (F1-F3) as follows:
\begin{itemize}
    \item \textbf{\textit{Responder Agent}} (for F1): This agent generates relevant code suggestions based on the developer's prompt. It assumes that the developer is unfamiliar with accessibility standards and automatically generates accessible code.
    % In addition, it educates developers on accessibility practices by providing relevant explanations and links to accessibility resources. 
    The prompt instruction for this agent is adapted from GitHub's recommended user prompt for accessibility.\footnote{\url{https://github.blog/developer-skills/github/prompting-github-copilot-chat-to-become-your-personal-ai-assistant-for-accessibility/}}
    \item \textbf{\textit{Correction Agent}} (for F2): This agent parses through accessibility error logs produced by an automated accessibility checker (axe DevTools Accessibility Linter\footnote{\url{https://www.deque.com/axe/devtools/linter/}}) to
    % refine the code suggestions and
    hint the developer at making additional accessibility fixes in the component or page being currently discussed in the chat context.
    \item \textbf{\textit{Reminder Agent}} (for F3): This agent reviews the Responder Agent's suggestions and identifies required manual steps for completing their implementation. It accordingly sends reminder notifications to the developer through the Visual Studio IDE infrastructure.
    % Based on our pilot studies, we also embed these reminders within the chat response for easy lookup.
\end{itemize}
\begin{highlight}
The agents are provided with several different sources of context:
\begin{itemize}
\item \textbf{Code Context}: the 100 lines of code centered around the cursor position in the active files.
\item \textbf{Chat Context}: the current active chat window interaction.
\item \textbf{Accessibility Linter Logs}: automated Axe DevTools Accessibility Linter error logs, refreshed periodically.
\item \textbf{Project Context}: code context from the \colorbox{codebgd}{\texttt{README}} and \colorbox{codebgd}{\texttt{index}} files, which contain information about the web framework that is being used, information about project structure, and other key configuration details.
\end{itemize}

Due to the constraints in the context window, we optimized our prompts and filtered this context when it exceeded 4000 characters. The agents use GPT-4o as the back-end model from the same OpenAI GPT family of models that powers GitHub Copilot.
\end{highlight}

\begin{highlight}
\begin{figure}
%     \tcbox[colframe=purple, colback=white, boxrule=0.35mm, arc=0mm, left=0mm, right=0mm, top=0mm, bottom=0mm]{
% \includegraphics[width=0.6\textwidth, trim=0 0 265 0, clip]{assets/architecture.png}
%     }
 \includegraphics[width=0.45\textwidth, trim=0 0 405 0, clip]{assets/architecture.png}
    \caption{CodeA11y Architecture: Multi-agent workflow}
    \Description{CodeA11y contains three agents: Responder Agent, Correction Agent and Reminder Agent which take the user_prompt, code_context and log_context respectively to generate a response for the chat window.}
    \label{fig:arch}
\end{figure}
\end{highlight}

\ipstart{User Interaction}
% \ipstart{User Interface Components}
% Error Reminders: Floating notifications or modal popups
% Based on pilots, errors also embedded within chat window.
% \ipstart{Conversational Interactions}
% Explanation of how users interact with the system
% [Figure: Side-by-side Comparison of Baseline Copilot and CodeA11y Responses]
Developers invoke\footnote{In the long term, the goal is for GitHub Copilot to invoke CodeA11y automatically during frontend development tasks.} CodeA11y in the GitHub Copilot Chat window panes (includes Quick Chat and Chat View) using \colorbox{codebgd}{\texttt{@CodeA11y}}. When a developer prompts CodeA11y, an internal \colorbox{codebgd}{\texttt{chat\_context}} state is established, storing the latest user prompts and agent responses. The \colorbox{codebgd}{\texttt{get\_relevant\_context()}} function is called, \begin{highlight}
which passes the source code and project context\end{highlight} to the Responder Agent. The agent generates code suggestions, accessibility explanations, and links to additional resources and updates \colorbox{codebgd}{\texttt{chat\_context}}. The \colorbox{codebgd}{\texttt{get\_log\_context()}} function is called, \begin{highlight}
which passes the accessibility linter logs\end{highlight} to the Correction Agent. This agent 
% refines the code and 
suggests additional fixes and displays the responses in the chat pane. Lastly, the updated \colorbox{codebgd}{\texttt{chat\_context}} state is forwarded to the Reminder Agent, which generates and sends reminder notifications. Figure~\ref{fig:interaction} illustrates a typical interaction between a developer and CodeA11y, showing how it compares to baseline assistants like GitHub Copilot.
% \end{highlight}


% Further, CodeA11y periodically takes in additional context from the error logs of the axe DevTools Accessibility Linter~\footnote{https://www.deque.com/axe/devtools/linter/} to detect accessibility violations. Code context is defined as the . Additionally, we provided the agents with code context from the \colorbox{codebgd}{\texttt{README}} and \colorbox{codebgd}{\texttt{index}} files, which contain information about the web framework, project structure, and key configuration details. 
\begin{figure*}
    \includegraphics[width=\textwidth]{assets/interactions.png}
    \caption{Contrasting responses for the same task across AI-assistants, \begin{highlight}showing differences in workflows. Developers had access to both the code and the rendered user interface.\end{highlight} (a) and (d) represent conversations with the baseline assistant, and CodeA11y respectively. (b) and (e) show the buttons generated by each assistant in their default state. (c) and (f) display the buttons when hovered over, illustrating the differences in button color contrast.}
    \Description{The notable differences in the interactions are as follows: CodeA11y reinforces accessibility practices in the text response, generates a reminder to check for button contrast and adds appropriate colours in the code. This results in a button with appropriate color contrast in both default and hover states. Conversely, the baseline assistant gives code for a button with the same colors in the default state, but significantly poorer color contrast in the hover state.}
    \label{fig:interaction}
\end{figure*}


\section{User Evaluation}
\label{sec:eval}
%include metodo, setup, ipotesi, risultati e grafici per i test con le due coorti di utenti
To assess the effectiveness and usability of Cyri as a tool for phishing detection and management from a human user, a user study was conducted involving ten participants with varying levels of expertise in computer security. This section details the methodology of the user study, the setup, and discusses the findings.

\subsection{Experiment Setup}
\label{sec:setup}

The study involved 10 participants, split equally between computer security
experts (meaning having at least two years of expertise and being knowledgeable of phishing tactics and techniques) and non-experts (meaning not being knowledgeable of phishing tactics and techniques but capable of using an email account).\\
The study duration for each participant was 60 minutes, split into 15 minutes of initial explanation on what are the most important features of Cyri and how to install it. This first step was then followed by two main tasks:
\begin{itemize}
    \item Controlled Email Identification Task: Participants were put in front of a preconfigured installation of Cyri and received five emails, four safe emails, and one phishing email sent by us. They were instructed to review these emails with Cyri and identify the phishing emails among them and the motivating factors for their final decision.
    This test was used both to let the participants gain confidence with Cyri usage and interface and as a controlled experiment where to evaluate how users interpreted and used the different results and functionalities Cyri exposes in a controlled situation equals for all of them. This step lasted, on average, from 10 to 15 minutes.
    \item Exploration with Personal Emails: Participants were then tasked to use Cyri to analyze their inbox emails from one personal account, such as those in their spam folder, unopened ones, or newly received messages. This allowed them to interact with the application in a context familiar to them and to assess its usefulness beyond the controlled task of provided emails, resulting in a more personal experience capable of letting them assess the degree of support they received from Cyri. This task lasted, on average, 25 minutes.
\end{itemize}
After completing the second task, participants were asked to compile a survey
comprising several questions aimed at evaluating Cyri’s effectiveness in assisting users in identifying phishing emails, usability and intuitiveness of the application interface, impact on users' understanding of phishing tactics, the likelihood of continued use, and preference for platform availability.
In particular, the questions proposed to the participants were the following:

\begin{enumerate}
    \item Are you a computer security expert? (Yes or No)
    \item How confident are you in identifying phishing emails without assistance?  (Scale 1 to 5)
    \item How useful was Cyri in helping you identify the phishing email?  (Scale 1 to 5)
    \item Did Cyri provide information that you wouldn't have noticed on your own? (Yes or No)
    \item How would you rate the overall usability of Cyri? (Scale 1 to 5)
    \item How intuitive did you find the Cyri interface?  (Scale 1 to 5)
    \item How much do you think using Cyri would improve your understanding of phishing tactics? (Scale 1 to 5)
    \item Would you use Cyri regularly as part of your email routine?  (Yes or No)
    \item Would you prefer if Cyri was available on your mobile phone instead of your computer?  (Yes or No)
\end{enumerate}

A final free text form allows the insertion of open comments and suggestions. Overall 5 minutes were dedicated on average to this activity.

\subsection{Results}
\label{sec:userresults}

We analyzed the survey results by splitting participants into their expertise level into two groups: Figure~\ref{fig:secexp} reports results for computer security experts while Figure~\ref{fig:nonsecexp} reports them for non-expert users. This distinction allowed us to understand how Cyri is perceived by users with different levels of expertise.

\begin{figure}[htbp]
  \centering
  \includegraphics[width=0.45\textwidth]{figures/SecurityExpertsBarchart.PNG}
  \caption{Security Experts Average Scores}
  \label{fig:secexp}
\end{figure}

\begin{figure}[htbp]
  \centering
  \includegraphics[width=0.45\textwidth]{figures/NonSecurityExpertsBarchart.PNG}
  \caption{Non-Security Experts Average Scores}
  \label{fig:nonsecexp}
\end{figure}

Cyri has been declared to be highly beneficial by non-expert participants, reporting generally low confidence in their ability to identify phishing emails without Cyri's assistance (Q2). All non-expert participants affirmed that Cyri provided information they would not have noticed on their own (Q4). This suggests that Cyri effectively highlights phishing indicators that might be overlooked, adding significant value in assisting them in identifying potential threats. 
Furthermore, non-experts provided very high ratings for both the usability (Q5) and intuitiveness (Q6) of Cyri. These results indicate that they found the application user-friendly and accessible. Non-experts believed using Cyri would significantly improve their understanding of phishing tactics (Q7), underscoring the application’s educational value.

Expert participants, even if they declared an expected good capability of identification and management of the phishing email with and without Cyri support (Q2), acknowledged that Cyri can enhance analysis capabilities by providing an additional layer of information (Q3). Interestingly, all expert participants also affirmed that Cyri provided information they would not have noticed on their own (Q4). This indicates that Cyri can uncover subtle phishing indicators and offer insights that even experienced users might overlook. Experts rated the overall usability (Q5) and intuitiveness (Q6) of Cyri highly, similar to non-experts, suggesting that the application is well-designed for users across different expertise levels. Moreover, experts believed that using Cyri could further improve their understanding of phishing tactics (Q7).
All participants expressed their willingness to use Cyri regularly as part of their email routine (Q8) and showed a clear preference for having it available also on their mobile devices (Q9).

\section{Discussion and Conclusion}
\vspace{-5pt}
Drawing ideas from the contextual integrity theory, we defined the notion of contextual privacy for users interacting with LLM-based conversation agents.
We proposed a framework, grounded in our contextual privacy formulation, that acts as an intermediary between the user and the agent, and carefully reformulates user prompts to preserve contextual privacy while preserving the utility.  






This work serves as an initial step in exploring privacy protection in user interactions with conversational agents. There are several directions that future research can further investigate. 
First, our framework may not be suitable for user prompts that require preserving exact content, such as document translation or verbatim summarization. For example, translating a legal document demands keeping the original content intact, making it challenging to reformulate while preserving contextual privacy. For such tasks, alternative approaches like using placeholders or pseudonyms for sensitive information could help protect privacy without compromising accuracy, though this is beyond our current implementation. 
Second, our framework relies on LLM-based assessment of privacy violations which, while effective for demonstrating the approach, lacks formal privacy guarantees and can be sensitive to the prompt. Future work could explore combining our contextual approach with deterministic rules or provable privacy properties. 
Third, while we demonstrate how users can adjust reformulations to balance privacy and utility, developing precise metrics to quantify this trade-off remains an open research challenge. This is particularly important as the relationship between privacy preservation and task effectiveness can vary significantly across different contexts and user preferences. 
Finally, while our evaluation using selected ShareGPT conversations demonstrates the potential of our approach, broader testing across diverse contexts and user groups would better establish the framework's general applicability.










\section{Conclusion}
\label{sec:conclusion}
\vspace{0.7em}

\ps{Highlight key idea/novelty of PlaceIT}

In this work, we present \name, a novel methodology to jointly optimize the chiplet placement and \gls{ici} topology for chips with heterogeneous chiplet shapes and silicon bridges or passive silicon interposers.
The main novelty of our approach is that we perform optimization on the chiplet placement itself, where we infer a custom, placement-based \gls{ici} topology for each placement produced by an optimization algorithm.
We use the placement and its inferred \gls{ici} topology to compute proxies for \gls{ici} latency and throughput of different traffic types, which we combine into a user-defined quality metric that is returned to the optimization algorithm.

\ps{Highlight the framework/code}

The open-source \name~framework is modular and allows adding custom optimization algorithms or placement representations.
\name~offers a wide range of configurable parameters, making it applicable for a variety of designs with different chiplet dimensions, PHY-counts, and \gls{d2d} links.

\ps{Summarize evaluation}

Our evaluation on synthetic traffic shows that \name~produces \gls{ici}s with vastly lower \gls{c2m}, \gls{c2i}, and \gls{m2i} latency (reduced by up to 62\%) compared to a 2D mesh baseline.
On real traffic traces, \name~reduces the average packet latency in almost all traces and architectures considered.
The average packet latency is reduced by up to $18\%$ on average.

\ps{Concluding sentence}

By using our open-source \name~framework, architects can co-optimize their chiplet-placement and \gls{ici} topology to build 2.5D stacked chips with low-latency interconnects.




% \section{Introduction}
% ACM's consolidated article template, introduced in 2017, provides a
% consistent \LaTeX\ style for use across ACM publications, and
% incorporates accessibility and metadata-extraction functionality
% necessary for future Digital Library endeavors. Numerous ACM and
% SIG-specific \LaTeX\ templates have been examined, and their unique
% features incorporated into this single new template.

% If you are new to publishing with ACM, this document is a valuable
% guide to the process of preparing your work for publication. If you
% have published with ACM before, this document provides insight and
% instruction into more recent changes to the article template.

% The ``\verb|acmart|'' document class can be used to prepare articles
% for any ACM publication --- conference or journal, and for any stage
% of publication, from review to final ``camera-ready'' copy, to the
% author's own version, with {\itshape very} few changes to the source.

% \section{Template Overview}
% As noted in the introduction, the ``\verb|acmart|'' document class can
% be used to prepare many different kinds of documentation --- a
% double-anonymous initial submission of a full-length technical paper, a
% two-page SIGGRAPH Emerging Technologies abstract, a ``camera-ready''
% journal article, a SIGCHI Extended Abstract, and more --- all by
% selecting the appropriate {\itshape template style} and {\itshape
%   template parameters}.

% This document will explain the major features of the document
% class. For further information, the {\itshape \LaTeX\ User's Guide} is
% available from
% \url{https://www.acm.org/publications/proceedings-template}.

% \subsection{Template Styles}

% The primary parameter given to the ``\verb|acmart|'' document class is
% the {\itshape template style} which corresponds to the kind of publication
% or SIG publishing the work. This parameter is enclosed in square
% brackets and is a part of the {\verb|documentclass|} command:
% \begin{verbatim}
%   \documentclass[STYLE]{acmart}
% \end{verbatim}

% Journals use one of three template styles. All but three ACM journals
% use the {\verb|acmsmall|} template style:
% \begin{itemize}
% \item {\texttt{acmsmall}}: The default journal template style.
% \item {\texttt{acmlarge}}: Used by JOCCH and TAP.
% \item {\texttt{acmtog}}: Used by TOG.
% \end{itemize}

% The majority of conference proceedings documentation will use the {\verb|acmconf|} template style.
% \begin{itemize}
% \item {\texttt{sigconf}}: The default proceedings template style.
% \item{\texttt{sigchi}}: Used for SIGCHI conference articles.
% \item{\texttt{sigplan}}: Used for SIGPLAN conference articles.
% \end{itemize}

% \subsection{Template Parameters}

% In addition to specifying the {\itshape template style} to be used in
% formatting your work, there are a number of {\itshape template parameters}
% which modify some part of the applied template style. A complete list
% of these parameters can be found in the {\itshape \LaTeX\ User's Guide.}

% Frequently-used parameters, or combinations of parameters, include:
% \begin{itemize}
% \item {\texttt{anonymous,review}}: Suitable for a ``double-anonymous''
%   conference submission. Anonymizes the work and includes line
%   numbers. Use with the \texttt{\acmSubmissionID} command to print the
%   submission's unique ID on each page of the work.
% \item{\texttt{authorversion}}: Produces a version of the work suitable
%   for posting by the author.
% \item{\texttt{screen}}: Produces colored hyperlinks.
% \end{itemize}

% This document uses the following string as the first command in the
% source file:
% \begin{verbatim}
% \documentclass[sigconf,authordraft]{acmart}
% \end{verbatim}

% \section{Modifications}

% Modifying the template --- including but not limited to: adjusting
% margins, typeface sizes, line spacing, paragraph and list definitions,
% and the use of the \verb|\vspace| command to manually adjust the
% vertical spacing between elements of your work --- is not allowed.

% {\bfseries Your document will be returned to you for revision if
%   modifications are discovered.}

% \section{Typefaces}

% The ``\verb|acmart|'' document class requires the use of the
% ``Libertine'' typeface family. Your \TeX\ installation should include
% this set of packages. Please do not substitute other typefaces. The
% ``\verb|lmodern|'' and ``\verb|ltimes|'' packages should not be used,
% as they will override the built-in typeface families.

% \section{Title Information}

% The title of your work should use capital letters appropriately -
% \url{https://capitalizemytitle.com/} has useful rules for
% capitalization. Use the {\verb|title|} command to define the title of
% your work. If your work has a subtitle, define it with the
% {\verb|subtitle|} command.  Do not insert line breaks in your title.

% If your title is lengthy, you must define a short version to be used
% in the page headers, to prevent overlapping text. The \verb|title|
% command has a ``short title'' parameter:
% \begin{verbatim}
%   \title[short title]{full title}
% \end{verbatim}

% \section{Authors and Affiliations}

% Each author must be defined separately for accurate metadata
% identification.  As an exception, multiple authors may share one
% affiliation. Authors' names should not be abbreviated; use full first
% names wherever possible. Include authors' e-mail addresses whenever
% possible.

% Grouping authors' names or e-mail addresses, or providing an ``e-mail
% alias,'' as shown below, is not acceptable:
% \begin{verbatim}
%   \author{Brooke Aster, David Mehldau}
%   \email{dave,judy,steve@university.edu}
%   \email{firstname.lastname@phillips.org}
% \end{verbatim}

% The \verb|authornote| and \verb|authornotemark| commands allow a note
% to apply to multiple authors --- for example, if the first two authors
% of an article contributed equally to the work.

% If your author list is lengthy, you must define a shortened version of
% the list of authors to be used in the page headers, to prevent
% overlapping text. The following command should be placed just after
% the last \verb|\author{}| definition:
% \begin{verbatim}
%   \renewcommand{\shortauthors}{McCartney, et al.}
% \end{verbatim}
% Omitting this command will force the use of a concatenated list of all
% of the authors' names, which may result in overlapping text in the
% page headers.

% The article template's documentation, available at
% \url{https://www.acm.org/publications/proceedings-template}, has a
% complete explanation of these commands and tips for their effective
% use.

% Note that authors' addresses are mandatory for journal articles.

% \section{Rights Information}

% Authors of any work published by ACM will need to complete a rights
% form. Depending on the kind of work, and the rights management choice
% made by the author, this may be copyright transfer, permission,
% license, or an OA (open access) agreement.

% Regardless of the rights management choice, the author will receive a
% copy of the completed rights form once it has been submitted. This
% form contains \LaTeX\ commands that must be copied into the source
% document. When the document source is compiled, these commands and
% their parameters add formatted text to several areas of the final
% document:
% \begin{itemize}
% \item the ``ACM Reference Format'' text on the first page.
% \item the ``rights management'' text on the first page.
% \item the conference information in the page header(s).
% \end{itemize}

% Rights information is unique to the work; if you are preparing several
% works for an event, make sure to use the correct set of commands with
% each of the works.

% The ACM Reference Format text is required for all articles over one
% page in length, and is optional for one-page articles (abstracts).

% \section{CCS Concepts and User-Defined Keywords}

% Two elements of the ``acmart'' document class provide powerful
% taxonomic tools for you to help readers find your work in an online
% search.

% The ACM Computing Classification System ---
% \url{https://www.acm.org/publications/class-2012} --- is a set of
% classifiers and concepts that describe the computing
% discipline. Authors can select entries from this classification
% system, via \url{https://dl.acm.org/ccs/ccs.cfm}, and generate the
% commands to be included in the \LaTeX\ source.

% User-defined keywords are a comma-separated list of words and phrases
% of the authors' choosing, providing a more flexible way of describing
% the research being presented.

% CCS concepts and user-defined keywords are required for for all
% articles over two pages in length, and are optional for one- and
% two-page articles (or abstracts).

% \section{Sectioning Commands}

% Your work should use standard \LaTeX\ sectioning commands:
% \verb|section|, \verb|subsection|, \verb|subsubsection|, and
% \verb|paragraph|. They should be numbered; do not remove the numbering
% from the commands.

% Simulating a sectioning command by setting the first word or words of
% a paragraph in boldface or italicized text is {\bfseries not allowed.}

% \section{Tables}

% The ``\verb|acmart|'' document class includes the ``\verb|booktabs|''
% package --- \url{https://ctan.org/pkg/booktabs} --- for preparing
% high-quality tables.

% Table captions are placed {\itshape above} the table.

% Because tables cannot be split across pages, the best placement for
% them is typically the top of the page nearest their initial cite.  To
% ensure this proper ``floating'' placement of tables, use the
% environment \textbf{table} to enclose the table's contents and the
% table caption.  The contents of the table itself must go in the
% \textbf{tabular} environment, to be aligned properly in rows and
% columns, with the desired horizontal and vertical rules.  Again,
% detailed instructions on \textbf{tabular} material are found in the
% \textit{\LaTeX\ User's Guide}.

% Immediately following this sentence is the point at which
% Table~\ref{tab:freq} is included in the input file; compare the
% placement of the table here with the table in the printed output of
% this document.

% \begin{table}
%   \caption{Frequency of Special Characters}
%   \label{tab:freq}
%   \begin{tabular}{ccl}
%     \toprule
%     Non-English or Math&Frequency&Comments\\
%     \midrule
%     \O & 1 in 1,000& For Swedish names\\
%     $\pi$ & 1 in 5& Common in math\\
%     \$ & 4 in 5 & Used in business\\
%     $\Psi^2_1$ & 1 in 40,000& Unexplained usage\\
%   \bottomrule
% \end{tabular}
% \end{table}

% To set a wider table, which takes up the whole width of the page's
% live area, use the environment \textbf{table*} to enclose the table's
% contents and the table caption.  As with a single-column table, this
% wide table will ``float'' to a location deemed more
% desirable. Immediately following this sentence is the point at which
% Table~\ref{tab:commands} is included in the input file; again, it is
% instructive to compare the placement of the table here with the table
% in the printed output of this document.

% \begin{table*}
%   \caption{Some Typical Commands}
%   \label{tab:commands}
%   \begin{tabular}{ccl}
%     \toprule
%     Command &A Number & Comments\\
%     \midrule
%     \texttt{{\char'134}author} & 100& Author \\
%     \texttt{{\char'134}table}& 300 & For tables\\
%     \texttt{{\char'134}table*}& 400& For wider tables\\
%     \bottomrule
%   \end{tabular}
% \end{table*}

% Always use midrule to separate table header rows from data rows, and
% use it only for this purpose. This enables assistive technologies to
% recognise table headers and support their users in navigating tables
% more easily.

% \section{Math Equations}
% You may want to display math equations in three distinct styles:
% inline, numbered or non-numbered display.  Each of the three are
% discussed in the next sections.

% \subsection{Inline (In-text) Equations}
% A formula that appears in the running text is called an inline or
% in-text formula.  It is produced by the \textbf{math} environment,
% which can be invoked with the usual
% \texttt{{\char'134}begin\,\ldots{\char'134}end} construction or with
% the short form \texttt{\$\,\ldots\$}. You can use any of the symbols
% and structures, from $\alpha$ to $\omega$, available in
% \LaTeX~\cite{Lamport:LaTeX}; this section will simply show a few
% examples of in-text equations in context. Notice how this equation:
% \begin{math}
%   \lim_{n\rightarrow \infty}x=0
% \end{math},
% set here in in-line math style, looks slightly different when
% set in display style.  (See next section).

% \subsection{Display Equations}
% A numbered display equation---one set off by vertical space from the
% text and centered horizontally---is produced by the \textbf{equation}
% environment. An unnumbered display equation is produced by the
% \textbf{displaymath} environment.

% Again, in either environment, you can use any of the symbols and
% structures available in \LaTeX\@; this section will just give a couple
% of examples of display equations in context.  First, consider the
% equation, shown as an inline equation above:
% \begin{equation}
%   \lim_{n\rightarrow \infty}x=0
% \end{equation}
% Notice how it is formatted somewhat differently in
% the \textbf{displaymath}
% environment.  Now, we'll enter an unnumbered equation:
% \begin{displaymath}
%   \sum_{i=0}^{\infty} x + 1
% \end{displaymath}
% and follow it with another numbered equation:
% \begin{equation}
%   \sum_{i=0}^{\infty}x_i=\int_{0}^{\pi+2} f
% \end{equation}
% just to demonstrate \LaTeX's able handling of numbering.

% \section{Figures}

% The ``\verb|figure|'' environment should be used for figures. One or
% more images can be placed within a figure. If your figure contains
% third-party material, you must clearly identify it as such, as shown
% in the example below.
% \begin{figure}[h]
%   \centering
%   \includegraphics[width=\linewidth]{sample-franklin}
%   \caption{1907 Franklin Model D roadster. Photograph by Harris \&
%     Ewing, Inc. [Public domain], via Wikimedia
%     Commons. (\url{https://goo.gl/VLCRBB}).}
%   \Description{A woman and a girl in white dresses sit in an open car.}
% \end{figure}

% Your figures should contain a caption which describes the figure to
% the reader.

% Figure captions are placed {\itshape below} the figure.

% Every figure should also have a figure description unless it is purely
% decorative. These descriptions convey what’s in the image to someone
% who cannot see it. They are also used by search engine crawlers for
% indexing images, and when images cannot be loaded.

% A figure description must be unformatted plain text less than 2000
% characters long (including spaces).  {\bfseries Figure descriptions
%   should not repeat the figure caption – their purpose is to capture
%   important information that is not already provided in the caption or
%   the main text of the paper.} For figures that convey important and
% complex new information, a short text description may not be
% adequate. More complex alternative descriptions can be placed in an
% appendix and referenced in a short figure description. For example,
% provide a data table capturing the information in a bar chart, or a
% structured list representing a graph.  For additional information
% regarding how best to write figure descriptions and why doing this is
% so important, please see
% \url{https://www.acm.org/publications/taps/describing-figures/}.

% \subsection{The ``Teaser Figure''}

% A ``teaser figure'' is an image, or set of images in one figure, that
% are placed after all author and affiliation information, and before
% the body of the article, spanning the page. If you wish to have such a
% figure in your article, place the command immediately before the
% \verb|\maketitle| command:
% \begin{verbatim}
%   \begin{teaserfigure}
%     \includegraphics[width=\textwidth]{sampleteaser}
%     \caption{figure caption}
%     \Description{figure description}
%   \end{teaserfigure}
% \end{verbatim}

% \section{Citations and Bibliographies}

% The use of \BibTeX\ for the preparation and formatting of one's
% references is strongly recommended. Authors' names should be complete
% --- use full first names (``Donald E. Knuth'') not initials
% (``D. E. Knuth'') --- and the salient identifying features of a
% reference should be included: title, year, volume, number, pages,
% article DOI, etc.

% The bibliography is included in your source document with these two
% commands, placed just before the \verb|\end{document}| command:
% \begin{verbatim}
%   \bibliographystyle{ACM-Reference-Format}
%   \bibliography{bibfile}
% \end{verbatim}
% where ``\verb|bibfile|'' is the name, without the ``\verb|.bib|''
% suffix, of the \BibTeX\ file.

% Citations and references are numbered by default. A small number of
% ACM publications have citations and references formatted in the
% ``author year'' style; for these exceptions, please include this
% command in the {\bfseries preamble} (before the command
% ``\verb|\begin{document}|'') of your \LaTeX\ source:
% \begin{verbatim}
%   \citestyle{acmauthoryear}
% \end{verbatim}


%   Some examples.  A paginated journal article \cite{Abril07}, an
%   enumerated journal article \cite{Cohen07}, a reference to an entire
%   issue \cite{JCohen96}, a monograph (whole book) \cite{Kosiur01}, a
%   monograph/whole book in a series (see 2a in spec. document)
%   \cite{Harel79}, a divisible-book such as an anthology or compilation
%   \cite{Editor00} followed by the same example, however we only output
%   the series if the volume number is given \cite{Editor00a} (so
%   Editor00a's series should NOT be present since it has no vol. no.),
%   a chapter in a divisible book \cite{Spector90}, a chapter in a
%   divisible book in a series \cite{Douglass98}, a multi-volume work as
%   book \cite{Knuth97}, a couple of articles in a proceedings (of a
%   conference, symposium, workshop for example) (paginated proceedings
%   article) \cite{Andler79, Hagerup1993}, a proceedings article with
%   all possible elements \cite{Smith10}, an example of an enumerated
%   proceedings article \cite{VanGundy07}, an informally published work
%   \cite{Harel78}, a couple of preprints \cite{Bornmann2019,
%     AnzarootPBM14}, a doctoral dissertation \cite{Clarkson85}, a
%   master's thesis: \cite{anisi03}, an online document / world wide web
%   resource \cite{Thornburg01, Ablamowicz07, Poker06}, a video game
%   (Case 1) \cite{Obama08} and (Case 2) \cite{Novak03} and \cite{Lee05}
%   and (Case 3) a patent \cite{JoeScientist001}, work accepted for
%   publication \cite{rous08}, 'YYYYb'-test for prolific author
%   \cite{SaeediMEJ10} and \cite{SaeediJETC10}. Other cites might
%   contain 'duplicate' DOI and URLs (some SIAM articles)
%   \cite{Kirschmer:2010:AEI:1958016.1958018}. Boris / Barbara Beeton:
%   multi-volume works as books \cite{MR781536} and \cite{MR781537}. A
%   couple of citations with DOIs:
%   \cite{2004:ITE:1009386.1010128,Kirschmer:2010:AEI:1958016.1958018}. Online
%   citations: \cite{TUGInstmem, Thornburg01, CTANacmart}.
%   Artifacts: \cite{R} and \cite{UMassCitations}.

% Identification of funding sources and other support, and thanks to
% individuals and groups that assisted in the research and the
% preparation of the work should be included in an acknowledgment
% section, which is placed just before the reference section in your
% document.

% This section has a special environment:
% \begin{verbatim}
  % \begin{acks}
  % ...
  % \end{acks}
% \end{verbatim}
% so that the information contained therein can be more easily collected
% during the article metadata extraction phase, and to ensure
% consistency in the spelling of the section heading.

% Authors should not prepare this section as a numbered or unnumbered {\verb|\section|}; please use the ``{\verb|acks|}'' environment.

% \section{Appendices}

% If your work needs an appendix, add it before the
% ``\verb|\end{document}|'' command at the conclusion of your source
% document.

% Start the appendix with the ``\verb|appendix|'' command:
% \begin{verbatim}
%   \appendix
% \end{verbatim}
% and note that in the appendix, sections are lettered, not
% numbered. This document has two appendices, demonstrating the section
% and subsection identification method.

% \section{Multi-language papers}

% Papers may be written in languages other than English or include
% titles, subtitles, keywords and abstracts in different languages (as a
% rule, a paper in a language other than English should include an
% English title and an English abstract).  Use \verb|language=...| for
% every language used in the paper.  The last language indicated is the
% main language of the paper.  For example, a French paper with
% additional titles and abstracts in English and German may start with
% the following command
% \begin{verbatim}
% \documentclass[sigconf, language=english, language=german,
%                language=french]{acmart}
% \end{verbatim}

% The title, subtitle, keywords and abstract will be typeset in the main
% language of the paper.  The commands \verb|\translatedXXX|, \verb|XXX|
% begin title, subtitle and keywords, can be used to set these elements
% in the other languages.  The environment \verb|translatedabstract| is
% used to set the translation of the abstract.  These commands and
% environment have a mandatory first argument: the language of the
% second argument.  See \verb|sample-sigconf-i13n.tex| file for examples
% of their usage.

% \section{SIGCHI Extended Abstracts}

% The ``\verb|sigchi-a|'' template style (available only in \LaTeX\ and
% not in Word) produces a landscape-orientation formatted article, with
% a wide left margin. Three environments are available for use with the
% ``\verb|sigchi-a|'' template style, and produce formatted output in
% the margin:
% \begin{description}
% \item[\texttt{sidebar}:]  Place formatted text in the margin.
% \item[\texttt{marginfigure}:] Place a figure in the margin.
% \item[\texttt{margintable}:] Place a table in the margin.
% \end{description}

%%
%% The acknowledgments section is defined using the "acks" environment
%% (and NOT an unnumbered section). This ensures the proper
%% identification of the section in the article metadata, and the
%% consistent spelling of the heading.
% \begin{acks}
% \end{acks}

%%
%% The next two lines define the bibliography style to be used, and
%% the bibliography file.
\bibliographystyle{ACM-Reference-Format}
\bibliography{paper}


%%
%% If your work has an appendix, this is the place to put it.
\appendix

% \section{Research Methods}

% \subsection{Part One}

% Lorem ipsum dolor sit amet, consectetur adipiscing elit. Morbi
% malesuada, quam in pulvinar varius, metus nunc fermentum urna, id
% sollicitudin purus odio sit amet enim. Aliquam ullamcorper eu ipsum
% vel mollis. Curabitur quis dictum nisl. Phasellus vel semper risus, et
% lacinia dolor. Integer ultricies commodo sem nec semper.

% \subsection{Part Two}

% Etiam commodo feugiat nisl pulvinar pellentesque. Etiam auctor sodales
% ligula, non varius nibh pulvinar semper. Suspendisse nec lectus non
% ipsum convallis congue hendrerit vitae sapien. Donec at laoreet
% eros. Vivamus non purus placerat, scelerisque diam eu, cursus
% ante. Etiam aliquam tortor auctor efficitur mattis.

% \section{Online Resources}

% Nam id fermentum dui. Suspendisse sagittis tortor a nulla mollis, in
% pulvinar ex pretium. Sed interdum orci quis metus euismod, et sagittis
% enim maximus. Vestibulum gravida massa ut felis suscipit
% congue. Quisque mattis elit a risus ultrices commodo venenatis eget
% dui. Etiam sagittis eleifend elementum.

% Nam interdum magna at lectus dignissim, ac dignissim lorem
% rhoncus. Maecenas eu arcu ac neque placerat aliquam. Nunc pulvinar
% massa et mattis lacinia.

\end{document}
\endinput
%%
%% End of file `sample-sigconf-authordraft.tex'.
