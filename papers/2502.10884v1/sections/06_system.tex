\section{CodeA11y}
\label{system}
Guided by the design goals identified through our user study, we built CodeA11y, a GitHub Copilot Extension for Visual Studio IDE. In this section, we present the interactions that it supports and its system architecture.

%\ipstart{System Overview}
% \ipstart{Design Considerations}
% Focus on conversational agent only
% D1. Accessibility awareness
% D2. Contextual understanding
% D3. User-centric feedback
% \ipstart{System Features}
% F1: Accessibility-aware system prompt
% F2: Reminders for code completion
% F3: Integration with axe a11y checker
% \ipstart{Implementation}
% Technologies used (e.g., LLM models, IDE integration)
% VS Code Plugin Backend
% Copilot Chat Extension -- chat participant
% GPT 4o LLM Agents
% axe accessibility linter
% \begin{highlight}
    
CodeA11y has three primary features (\textbf{F1-F3}, aligned to G1-G3, respectively): (\textbf{F1}) it produces user interface code that better complies with accessibility standards, \begin{highlight}
(\textbf{F2}) it prompts the developer to resolve existing accessibility errors in their website, and (\textbf{F3}) it reminds the developer to complete any AI-generated code that requires manual intervention.
\end{highlight} CodeA11y is integrated into Visual Studio Code as a GitHub Copilot Extension\footnote{\url{https://docs.github.com/en/copilot/using-github-copilot/using-extensions-to-integrate-external-tools-with-copilot-chat}}, enabling CodeA11y to act as a chat participant within the GitHub Copilot Chat window panes. While we implemented this as an extension, it could be integrated directly into Copilot in the future.
% \end{highlight}

% \begin{highlight}

\ipstart{Multi-Agent Architecture}
% \ipstart{Multi-Agent Workflow}
% [Figure: CodeA11y Multi-agent Workflow Diagram]
% Uses 3 LLM Agents -- acts as conversation orchestrator 
% \ipstart{Agent Descriptions}
% a. Code Context Agent
% b. Ally Checker Agent
% c. Follow-up Reminder Agent
% \ipstart{Prompt Instructions}
% [Table: Prompt Instructions for LLM-based Agents]
% GitHub A11y blog prompt
% Prompt optimisations
\begin{table*}
  \caption{Prompt instructions for the three LLM agents in CodeA11y}
  \label{tab:sys_prompt}
  \small
  \begin{tabular}{m{0.14\textwidth}|m{0.81\textwidth}}
    \toprule
    \textbf{Agent} & \textbf{Prompt Instruction Highlights}\\
    \midrule
    Responder Agent & {\begin{itemize}[leftmargin=*, partopsep=7pt]
        \item I am unfamiliar with accessibility and need to write code that conforms with WCAG 2.1 level AA criteria.
        \item Be an accessibility coach that makes me account for all accessibility requirements.
        \item Use reputable sources such as w3.org, webaim.org and provide links and references for additional learning. 
        \item Don't give placeholder variables but tell me where to give meaningful values.
        \item Prioritise my current request and don't mention accessibility if I give a generic request like "Hi".
    \end{itemize}} \\ \midrule
    Correction Agent & {\begin{itemize}[leftmargin=*, partopsep=7pt]
        \item Review the accessibility checker log and provide feedback to fix errors relevant to current chat context.
        \item If a log error relevant to current chat context occurs, provide a code snippet to fix it.
    \end{itemize}} \\ \midrule
    Reminder Agent & {\begin{itemize}[leftmargin=*, partopsep=7pt]
        \item Is there an additional step required by the developer to meet accessibility standards after pasting code?
        \item Reminder should be single line. Be conservative in your response, if not needed, say "No reminders needed."
        \item For example, remind the developer to replace the placeholder attributes with meaningful values or labels, or visually inspect element for colour contrast when needed.
    \end{itemize}}\\ 
    \bottomrule
  \end{tabular}
\end{table*}
CodeA11y has three LLM agents (Figure~\ref{fig:arch}): Responder Agent, Correction Agent, and Reminder Agent. We provide their prompt instruction highlights in Table~\ref{tab:sys_prompt}. These agents facilitate each of the above features (F1-F3) as follows:
\begin{itemize}
    \item \textbf{\textit{Responder Agent}} (for F1): This agent generates relevant code suggestions based on the developer's prompt. It assumes that the developer is unfamiliar with accessibility standards and automatically generates accessible code.
    % In addition, it educates developers on accessibility practices by providing relevant explanations and links to accessibility resources. 
    The prompt instruction for this agent is adapted from GitHub's recommended user prompt for accessibility.\footnote{\url{https://github.blog/developer-skills/github/prompting-github-copilot-chat-to-become-your-personal-ai-assistant-for-accessibility/}}
    \item \textbf{\textit{Correction Agent}} (for F2): This agent parses through accessibility error logs produced by an automated accessibility checker (axe DevTools Accessibility Linter\footnote{\url{https://www.deque.com/axe/devtools/linter/}}) to
    % refine the code suggestions and
    hint the developer at making additional accessibility fixes in the component or page being currently discussed in the chat context.
    \item \textbf{\textit{Reminder Agent}} (for F3): This agent reviews the Responder Agent's suggestions and identifies required manual steps for completing their implementation. It accordingly sends reminder notifications to the developer through the Visual Studio IDE infrastructure.
    % Based on our pilot studies, we also embed these reminders within the chat response for easy lookup.
\end{itemize}
\begin{highlight}
The agents are provided with several different sources of context:
\begin{itemize}
\item \textbf{Code Context}: the 100 lines of code centered around the cursor position in the active files.
\item \textbf{Chat Context}: the current active chat window interaction.
\item \textbf{Accessibility Linter Logs}: automated Axe DevTools Accessibility Linter error logs, refreshed periodically.
\item \textbf{Project Context}: code context from the \colorbox{codebgd}{\texttt{README}} and \colorbox{codebgd}{\texttt{index}} files, which contain information about the web framework that is being used, information about project structure, and other key configuration details.
\end{itemize}

Due to the constraints in the context window, we optimized our prompts and filtered this context when it exceeded 4000 characters. The agents use GPT-4o as the back-end model from the same OpenAI GPT family of models that powers GitHub Copilot.
\end{highlight}

\begin{highlight}
\begin{figure}
%     \tcbox[colframe=purple, colback=white, boxrule=0.35mm, arc=0mm, left=0mm, right=0mm, top=0mm, bottom=0mm]{
% \includegraphics[width=0.6\textwidth, trim=0 0 265 0, clip]{assets/architecture.png}
%     }
 \includegraphics[width=0.45\textwidth, trim=0 0 405 0, clip]{assets/architecture.png}
    \caption{CodeA11y Architecture: Multi-agent workflow}
    \Description{CodeA11y contains three agents: Responder Agent, Correction Agent and Reminder Agent which take the user_prompt, code_context and log_context respectively to generate a response for the chat window.}
    \label{fig:arch}
\end{figure}
\end{highlight}

\ipstart{User Interaction}
% \ipstart{User Interface Components}
% Error Reminders: Floating notifications or modal popups
% Based on pilots, errors also embedded within chat window.
% \ipstart{Conversational Interactions}
% Explanation of how users interact with the system
% [Figure: Side-by-side Comparison of Baseline Copilot and CodeA11y Responses]
Developers invoke\footnote{In the long term, the goal is for GitHub Copilot to invoke CodeA11y automatically during frontend development tasks.} CodeA11y in the GitHub Copilot Chat window panes (includes Quick Chat and Chat View) using \colorbox{codebgd}{\texttt{@CodeA11y}}. When a developer prompts CodeA11y, an internal \colorbox{codebgd}{\texttt{chat\_context}} state is established, storing the latest user prompts and agent responses. The \colorbox{codebgd}{\texttt{get\_relevant\_context()}} function is called, \begin{highlight}
which passes the source code and project context\end{highlight} to the Responder Agent. The agent generates code suggestions, accessibility explanations, and links to additional resources and updates \colorbox{codebgd}{\texttt{chat\_context}}. The \colorbox{codebgd}{\texttt{get\_log\_context()}} function is called, \begin{highlight}
which passes the accessibility linter logs\end{highlight} to the Correction Agent. This agent 
% refines the code and 
suggests additional fixes and displays the responses in the chat pane. Lastly, the updated \colorbox{codebgd}{\texttt{chat\_context}} state is forwarded to the Reminder Agent, which generates and sends reminder notifications. Figure~\ref{fig:interaction} illustrates a typical interaction between a developer and CodeA11y, showing how it compares to baseline assistants like GitHub Copilot.
% \end{highlight}


% Further, CodeA11y periodically takes in additional context from the error logs of the axe DevTools Accessibility Linter~\footnote{https://www.deque.com/axe/devtools/linter/} to detect accessibility violations. Code context is defined as the . Additionally, we provided the agents with code context from the \colorbox{codebgd}{\texttt{README}} and \colorbox{codebgd}{\texttt{index}} files, which contain information about the web framework, project structure, and key configuration details. 
\begin{figure*}
    \includegraphics[width=\textwidth]{assets/interactions.png}
    \caption{Contrasting responses for the same task across AI-assistants, \begin{highlight}showing differences in workflows. Developers had access to both the code and the rendered user interface.\end{highlight} (a) and (d) represent conversations with the baseline assistant, and CodeA11y respectively. (b) and (e) show the buttons generated by each assistant in their default state. (c) and (f) display the buttons when hovered over, illustrating the differences in button color contrast.}
    \Description{The notable differences in the interactions are as follows: CodeA11y reinforces accessibility practices in the text response, generates a reminder to check for button contrast and adds appropriate colours in the code. This results in a button with appropriate color contrast in both default and hover states. Conversely, the baseline assistant gives code for a button with the same colors in the default state, but significantly poorer color contrast in the hover state.}
    \label{fig:interaction}
\end{figure*}

