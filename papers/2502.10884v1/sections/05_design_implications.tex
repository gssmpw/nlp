\section{Design Requirements}
\label{design-goals}
Our formative study identified three limitations in novice developers' interactions with AI assistants: (1) failing to prompt for accessibility considerations explicitly, (2) uncritically accepting incomplete code suggestions from Copilot, and (3) struggling to detect potential accessibility issues in their code. These shortcomings indicate possible directions to support accessibility through three design goals (\textbf{G1-G3}):

\ipstart{G1: Integrate System Prompts for Accessibility Awareness} 
Without explicit prompting, the AI assistant rarely produced accessibility-compliant code, reflecting the accessibility issues prevalent in its training data. However, it occasionally suggested accessibility features when participants indirectly prompted them, demonstrating its ability to recall accessibility practices from training data upon instruction. AI assistants should have a system prompt tuned towards following accessibility guidelines by default, for consistent generation of accessibility-compliant code, even when developers do not mention accessibility specifically. Further, the system prompt should also direct the assistant to suggest accessibility-focused iterative refinements.
% and educate developers about accessibility best practices.

\begin{highlight}
\ipstart{G2: Support Identification of Accessibility Issues}
\end{highlight}
Due to their unfamiliarity with accessibility standards, our participants were unable to identify compliance issues in the existing and modified code. They primarily prompted changes to individual components (such as buttons and forms), hardly addressing broader page-level accessibility concerns (such as heading structure or landmark regions). AI assistants should not only automatically generate accessibility-compliant code, but also provide real-time feedback to detect and resolve accessibility violations within the codebase. In addition, AI assistants and automated accessibility checkers should work in tandem to ensure that incomplete or incorrect implementations of the AI-suggested code are always flagged by the latter.

\begin{highlight}
\ipstart{G3: Encourage Developers to Complete AI-Generated Code}
\end{highlight}
Our observations revealed that accessibility implementation in AI-assisted coding workflows commonly required critical manual intervention to complete and validate AI-generated code. This involved replacing placeholder attributes, such as labels and alt-texts, with meaningful values and verifying color contrast ratios. However, we found that participants blindly copy-pasted code and proceeded further if there were no apparent errors. This behavior of deferring thought to suggestions has also been documented in previous work~\cite{mozannar2024reading}. To mitigate this, AI assistants should proactively remind developers to ensure that all necessary accessibility features -- such as contrast ratios or keyboard navigation support -- are fully implemented and verified.