\label{intro}
\section{Introduction}

\begin{comment}
    \begin{outline}
    \1 problem of web inaccessibility / developer unawareness
        \2 long standing challenge
        \2 estimated 1.6 billion people worldwide have a disability that affects their internet usage
        \2 96\% of top 1 million visited websites have accessibility errors
        \2 errors include lacking alt-text, labels, aria roles, etc.
        \2 developers do not produce accessible code
    \1 prior work
        \2 creating accessibility developer tools
        \2 building awareness among developers
        \2 does not cover novice, self-learning developers -- not a small population (1 in 3)
    \1 novelty
        \2 ai coding assistants
        \2 what are they?
        \2 high adoption especially among novice developers (76\%)
        \2 potential automation of code accessibility
    \1 research questions
        \2 can their usage help novice developers create more accessible web components? 
        \2 how can they be enhanced for effectiveness in producing more accessible UI code?
    \1 contributions
        \2 key design goals for optimising ai assistants for a11y
        \2 codea11y: system implementation and evaluation
    \1 findings and implications
        \2 coding assistants produce accessible code contingent on developer awareness
        \2 assistants should be a11y aware and reinforcing a11y practices to developers
        \2 placeholder results for the user evaluation
        \2 exploring ai assistants as second "brain" not hand for non-funct reqmts
\end{outline}
\end{comment}

Most websites contain extensive accessibility errors~\cite{webaim2024}, despite decades of investment in standards and guidelines~\cite{chisholm2001web, caldwell2008web}, tools~\cite{adesigner, takagi2003accessibility, bigham2010accessibility}, advocacy \cite{sloan2006contextual, martin2022landscape, pandey2023blending}, and policy. According to a recent analysis by WebAim~\cite{webaim2024}, the homepages of the top million websites each contain 57 accessibility errors on average, including (but not limited to) missing alt-text for images~\cite{webinsight,twitterally}, inadequate color contrast~\cite{colors}, incorrect or missing labels for forms and links~\cite{formlabels}, and improper use of heading levels~\cite{headings}. As a result, many people with disabilities will find it difficult to use these websites effectively and may not be able to use them at all.

Front-end web developers ultimately determine the accessibility (or inaccessibility) of the UI code that they write. Getting front-end developers to write more accessible code has proven exceptionally difficult. As Jonathan Lazar {\em et al.} wrote twenty years ago in 2004, ``Since tools and guidelines are available to help designers and webmasters make their web sites accessible, it is unclear why so many sites remain inaccessible.''~\cite{lazar2004improving}. A survey of webmasters at the time indicated that they generally would like to make their web pages accessible but cited a number of reasons they do not: "lack of time, lack of training, lack of managerial support, lack of client
support, inadequate software tools, and confusing accessibility guidelines." Sixteen years later, Patel {\em et al.} reported remarkably similar results in their 2020 survey of 77 technology professionals \cite{stillinaccessible}. Few developers had received formal accessibility training, implementing accessibility was considered confusing, and advocating for accessible development was in conflict with other business goals. Clearly, what we have done so far is not working.

We argue that AI coding assistants ({\em e.g.}, Github Copilot~\cite{GitHubCopilot}) could offer an opportunity to make UI code more accessible. AI coding assistants are already widely adopted, which means that developers do not need to be convinced to use them or to install a specialized tool for accessibility. They produce a wide variety of UI code and are capable enough to both reflect on the quality of arbitrary code and also prompt developers to fix what they are unable to do. This paper explores how AI coding assistants currently help developers create UI code, what problems remain, and presents a system called CodeA11y that shows that AI coding assistants can be made better at enabling developers to improve the accessibility of their UI code.

To explore this potential opportunity, we first conducted a user study (Section \ref{form_methods}) with 16 developers not trained in accessibility to understand how current tools (GitHub Copilot) impact the production of accessible UI code. Our findings (Section \ref{form_finds}) shows that while Copilot may potentially improve accessibility of UI code, three barriers prevent realization of those improvements: (1) developers may need to explicitly prompt the assistants for accessible code and thus not benefit if they fail to do so, (2) developers may overlook critical manual steps suggested by Copilot, such as replacing placeholders in alternative text for images, and (3) developers may not be able to verify if they fully implemented more complex accessibility enhancements properly. The formative study showed the potential of AI coding assistants to improve the accessibility of UI code, but revealed several gaps that led us to design goals (Section~\ref{design-goals}) for improving AI coding assistants to support accessibility.

We then built \textit{CodeA11y} (Section~\ref{system}, Figure~\ref{fig:teaser}), a GitHub Copilot Extension that addresses the observed gaps by consistently reinforcing accessible development practices throughout the conversational interaction. We evaluated CodeA11y (Section~\ref{eval}) with 20 developers, assessing its effectiveness in supporting accessible UI development and gathering insights for further refinement. We found that developers using CodeA11y are significantly more likely to produce accessible UI code. Finally, we reflect on the broader implications of integrating AI coding assistants into accessibility workflows, including the balance between automation and developer education, and the potential for AI tools to shape long-term developer behavior toward accessibility-conscious practices (Section~\ref{discuss}).

The contributions of our paper are:
\begin{itemize}[noitemsep, topsep=0pt]
    \item We conducted a study with 16 developers that uncovered both benefits and limitations of current AI coding assistants for authoring accessible UI code.  
    \item \textbf{CodeA11y\footnote{The source code for CodeA11y is available at \url{https://github.com/peyajm29/codea11y/}.}}: a GitHub Copilot Extension that generates accessible UI code, identifies existing issues and reminds developers to perform manual validation.
\end{itemize}

% \textit{Code at}: \url{https://github.com/peyajm29/codea11y/}.}

\begin{comment}
%We could that while coding assistants like Copilot are capable of producing accessible code, developers still need the appropriate expertise to use them effectively.   
% \item \textbf{Three Design Recommendations for integrating accessibility into AI coding assistants}. We conducted a task-based study with novice developers as they use AI coding assistants for web development.
    
    Trained on vast amounts of publicly available code, they harness the power of LLMs to generate code completion suggestions and explanations through a chat interface, all within the integrated development environment. These tools have already gained significant traction (with \textasciitilde76\% developers claiming they either already used or were planning to use AI tools in their development process this year~\cite{stackoverflow2024survey}) and have begun reshaping programming practices~\cite{<add:Do users write more insecure code with AI assistants?>}. Former research suggests developers currently utilize them for improving their productivity (such as by automating routine tasks or recalling syntax) and desire additional guidance on non-functional requirements (such as readability and performance)~\cite{liang2024large}. Further, anecdotal evidence suggests they are lowering barriers to entry in software development by making programming synonymous with prompting~\cite{<add:https://www.technologyreview.com/2023/12/06/1084457/ai-assistants-copilot-changing-code-software-development-github-openai/>}. This means not only are they being adopted widely, but more importantly, they are also being embraced by novice developers. Thus presumably, they might be useful for nudging developers with limited accessibility training towards more accessible web development practices. Several critical research questions emerge from this hypothesis. In this paper, we explore the following:
\begin{description}
    \item [\textbf{RQ1:}] Can AI coding assistants consistently produce accessibility-compliant web UI code?
    \item [\textbf{RQ2:}] How effectively can they support novice developers in creating and refining accessible web components?
    \item [\textbf{RQ3:}] What design interventions can be incorporated in them to improve their efficacy in promoting accessibility?
\end{description}





The 2024 WebAim Million report claims that about 96\% of the million most visited websites contain accessibility errors~\cite{webaim2024} affecting an estimated 1.3 billion people worldwide~\cite{<add>:who} (among others, \textasciitilde1 billion visually impaired, 430 million people with disabling hearing loss). Many internet users utilize assistive technologies (such as screen readers, magnifiers and keyboard or alternative input devices) to access websites and accordingly require them to adhere to accessibility guidelines~\cite{caldwell2008web}. However, an average website homepage contains roughly 57 accessibility violation errors including (but not limited to) missing alt-text for images, inadequate color contrast for visual elements, non-descriptive or missing labels for forms and links, and improper use of heading levels. This is primarily due to the frontend developer's oversight in accessibility implementation within the UI (UI) code. Thus, a long-standing challenge in accessible computing has been to ensure developers produce the UI code necessary for assistive technologies to function properly.

Frontend developers can make their websites accessible by using online resources and accessibility developer tools that assist with identifying issues, offering improvement suggestions, debugging, testing, and educating on best practices. Despite their availability, these tools are rarely adopted due to limited knowledge about accessibility, constraints on resources such as time and budget, perceived complexity of implementing these features, and the low priority given to accessibility in development projects (similar to other non-functional requirements). Prior work has focused on generating educational accessibility material~\cite{<add>} and building awareness among industrial developers~\cite{<add>}, but its impact remains limited, particularly among novice and self-taught developers who may not prioritize accessibility in their work. These developers might often overlook specialized accessibility resources, yet they form a large portion (\textasciitilde51\% self-taught, \textasciitilde34\% with less than 4 years of industrial experience~\cite{stackoverflow2024survey}) of the developer community.



To address these questions, we began by conducting a formative study (Section \ref{form_methods}) with 16 developers untrained in accessibility and tasking them with web development tasks using a popular AI coding assistant, GitHub Copilot. Our findings (Section \ref{form_finds}) suggested that AI coding assistants could currently produce accessible code, but developers still needed accessibility expertise to make use of them effectively. Otherwise, developers may (1) fail to explicitly prompt the assistants for accessible code, (2) overlook critical manual steps such as replacing placeholders, and (3) be unable to thoroughly verify the accessibility of their code. We then identified key design goals (Section~\ref{design-goals}) for optimising AI assistants for promoting accessibility among novice developers, such as producing accessibility-compliant code by default and sending manual validation reminders to the developers. Keeping these goals in mind, we built \textit{CodeA11y} (Section~\ref{system}, illustrated in Figure~\ref{fig:teaser}), a GitHub Copilot Extension that consistently reinforces accessible development practices throughout the chat interaction. We evaluated CodeA11y (Section~\ref{eval}) by taking a human-centered approach with 20 developers, assessing its effectiveness in supporting accessible development and gathering insights for further refinement. We found that <add: developers using CodeA11y are 2X more likely to produce accessible UI code>. Finally, we reflect on the broader implications of integrating AI coding assistants into accessibility workflows, including the balance between automation and developer education, and the potential for AI tools to shape long-term developer behavior towards accessibility-conscious practices (Section~\ref{discuss}).

\end{comment}


\begin{comment}
automatic invokation of CodeA11y
as models become more advanced -- reasoning, larger data, more accessible code, trained on more advanced web frameworks
However, while many of the accessibility issues present on the web come from developers not meeting standards, there are systemic issues with the web that can also prevent accessibility and are not so easily addressed at an individual level.
https://bookish.press/tac/WebDev
Students may view accessibility as an afterthought simply because one can get away with designing and developing web applications without much consideration for accessible features.

Our results suggest that despite CodeA11y suggesting accessibility implementation and reminding developers to manually complete the required steps, novice developers do not pay attention and subconsciously ignore the content which they do not find relevant to their task. One reason is the length of responses that Copilot generates. Thus, this suggests reliance on complete automation of accessibility would also not be beneficial in the long run and we need more effective UIs within the chat to actually educate the developer instead. Afterthought is a word normally associated with accessibility, but we reveal that developers give no thought to it at all, revealing systemic issues in how accessibility is often realistically considered in software projects.

There should be restricted user control in how much they can edit the color contrast etc.
https://accessibe.com/accessflow
\end{comment}