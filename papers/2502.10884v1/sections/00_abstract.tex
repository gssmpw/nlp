\label{abstract}
\begin{abstract}
A persistent challenge in accessible computing is ensuring developers produce web UI code that supports assistive technologies. Despite numerous specialized accessibility tools, novice developers often remain unaware of them, leading to \textasciitilde 96\% of web pages that contain accessibility violations. 
AI coding assistants, such as GitHub Copilot, could offer potential by generating accessibility-compliant code, but their impact remains uncertain~\cite{mowar2024tab}. Our formative study with 16 developers without accessibility training revealed three key issues in AI-assisted coding: failure to prompt AI for accessibility, omitting crucial manual steps like replacing placeholder attributes, and the inability to verify compliance.
To address these issues, we developed CodeA11y, a GitHub Copilot Extension, that suggests accessibility-compliant code and displays manual validation reminders. We evaluated it through a controlled study with another 20 novice developers. Our findings demonstrate its effectiveness in guiding novice developers by reinforcing accessibility practices throughout interactions, representing a significant step towards integrating accessibility into AI coding assistants.
\end{abstract}

\begin{comment}
    problem statement
    A persistent challenge in accessible computing is ensuring developers produce the UI code necessary for assistive technologies to function properly.
    % cause
    Despite the availability of numerous accessibility developer tools, they are rarely adopted by novice developers who are often unaware of them.
    % impact of problem
    As a result, web inaccessibility remains prevalent with \textasciitilde96\% of a million home pages containing web accessibility guideline violations.
    % gaps / novelty
    AI coding assistants ({\em e.g.}, Github Copilot) could potentially offer a new opportunity by producing accessibility-compliant code by default, but it remains uncertain how their usage currently impacts code accessibility.
    In this paper, we conduct a formative study with 16 developers without accessibility training and find three issues preventing AI coding assistants from fully supporting accessible development practices: (1) developers often fail to explicitly prompt the assistant for accessibility in their web components; (2) critical manual steps, such as replacing placeholder attributes, are frequently overlooked; and (3) developers are unable to verify the accessibility compliance of the AI-generated code.
    % to identify design goals for making AI coding assistants useful in helping developers write accessible code.
    % : (1) proactive accessibility guidance, (2) integration of dynamic accessibility evaluation, and (3) manual code correction reminders.
    To tackle these issues, we build CodeA11y, an accessibility-aware GitHub Copilot Chat extension for Visual Studio Code, that guides developers to craft accessible web components. We rigorously evaluate CodeA11y through a controlled study with another cohort of <add> novice developers.
    % findings
    Our results highlight CodeA11y’s effectiveness over GitHub Copilot in guiding developers to write code adhering to accessibility standards by consistently reinforcing best practices throughout the chat interactions.
    % contribution to HCI
    Overall, our work provides a concrete step towards incorporating accessibility into AI coding assistants, addressing common developer challenges and leveraging the growing popularity of these tools.
    % Beyond accessibility, our research could offer implications for integrating other critical non-functional software requirements -- such as privacy, security and performance -- in the design and deployment of AI coding assistants.
\end{comment}
