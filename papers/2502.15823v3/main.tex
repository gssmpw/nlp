\pdfoutput=1

\documentclass[table, x11names]{article} % For LaTeX2e
\usepackage{iclr2025_conference,times}

% Optional math commands from https://github.com/goodfeli/dlbook_notation.
%%%%% NEW MATH DEFINITIONS %%%%%

% \usepackage{amsmath,amsfonts,bm}
\usepackage{amsmath,amsfonts}

\usepackage{pifont}


\newcommand{\R}{\mathbb{R}}


\def\va{{\mathbf{a}}}
\def\vg{{\mathbf{g}}}

% Sets
\def\sR{\mathbb{R}}
\def\sC{\mathbb{C}}
\def\sZ{\mathbb{Z}}
\def\sN{\mathbb{N}}
\def\sQ{\mathbb{Q}}

\def\sS{\mathcal{S}}



% Vectors
\def\vzero{{\mathbf{0}}}
\def\vone{{\mathbf{1}}}
\def\vmu{{\mathbf{\mu}}}
\def\vtheta{{\mathbf{\theta}}}
\def\va{{\mathbf{a}}}
\def\vb{{\mathbf{b}}}
\def\vc{{\mathbf{c}}}
\def\vd{{\mathbf{d}}}
\def\ve{{\mathbf{e}}}
\def\vf{{\mathbf{f}}}
\def\vg{{\mathbf{g}}}
\def\vh{{\mathbf{h}}}
\def\vi{{\mathbf{i}}}
\def\vj{{\mathbf{j}}}
\def\vk{{\mathbf{k}}}
\def\vl{{\mathbf{l}}}
\def\vm{{\mathbf{m}}}
\def\vn{{\mathbf{n}}}
\def\vo{{\mathbf{o}}}
\def\vp{{\mathbf{p}}}
\def\vq{{\mathbf{q}}}
\def\vr{{\mathbf{r}}}
\def\vs{{\mathbf{s}}}
\def\vt{{\mathbf{t}}}
\def\vu{{\mathbf{u}}}
\def\vv{{\mathbf{v}}}
\def\vw{{\mathbf{w}}}
\def\vx{{\mathbf{x}}}
\def\vy{{\mathbf{y}}}
\def\vz{{\mathbf{z}}}
\def\vzeta{{\mathbf{\zeta}}}

% Matrix
\def\mA{{\mathbf{A}}}
\def\mB{{\mathbf{B}}}
\def\mC{{\mathbf{C}}}
\def\mD{{\mathbf{D}}}
\def\mE{{\mathbf{E}}}
\def\mF{{\mathbf{F}}}
\def\mG{{\mathbf{G}}}
\def\mH{{\mathbf{H}}}
\def\mI{{\mathbf{I}}}
\def\mJ{{\mathbf{J}}}
\def\mK{{\mathbf{K}}}
\def\mL{{\mathbf{L}}}
\def\mM{{\mathbf{M}}}
\def\mN{{\mathbf{N}}}
\def\mO{{\mathbf{O}}}
\def\mP{{\mathbf{P}}}
\def\mQ{{\mathbf{Q}}}
\def\mR{{\mathbf{R}}}
\def\mS{{\mathbf{S}}}
\def\mT{{\mathbf{T}}}
\def\mU{{\mathbf{U}}}
\def\mV{{\mathbf{V}}}
\def\mW{{\mathbf{W}}}
\def\mX{{\mathbf{X}}}
\def\mY{{\mathbf{Y}}}
\def\mZ{{\mathbf{Z}}}
\def\mBeta{{\mathbf{\beta}}}
\def\mPhi{{\mathbf{\Phi}}}
\def\mLambda{{\mathbf{\Lambda}}}
\def\mSigma{{\mathbf{\Sigma}}}


% Expectation
% \def\eE{\mathop{\mathbb{E}}\limits}
\def\eE{\mathbb{E}}

% Probability
\def\pP{\mathbb{P}}

% Tilde
\def\tf{\tilde{f}}
\def\tS{\tilde{S}}
\def\wtF{\widetilde{\mathcal{F}}}
\def\whR{\widehat{R}}
\def\tvx{\tilde{\mathbf{x}}}
\def\ty{\tilde{y}}


\def\defeq{\overset{\textup{def}}{=}}
% \def\defeq{\overset{.}{=}}
\def\defone{\overset{\text{\ding{172}}}{=}}
\def\deftwo{\overset{\text{\ding{173}}}{=}}
\def\leqone{\overset{\text{\ding{172}}}{\leq}}
\def\leqtwo{\overset{\text{\ding{173}}}{\leq}}
\def\leqthree{\overset{\text{\ding{174}}}{\leq}}
\def\leqfour{\overset{\text{\ding{175}}}{\leq}}
\def\eqone{\overset{\text{\ding{172}}}{=}}
\def\eqtwo{\overset{\text{\ding{173}}}{=}}
\def\eqthree{\overset{\text{\ding{174}}}{=}}
\def\eqfour{\overset{\text{\ding{175}}}{=}}
\def\geqfive{\overset{\text{\ding{176}}}{\geq}}

\usepackage{hyperref}
\usepackage{url}

% Standard package includes
\usepackage{times}
\usepackage{latexsym}
\usepackage{amsmath}
\usepackage{mathtools}

% For proper rendering and hyphenation of words containing Latin characters (including in bib files)
\usepackage[T1]{fontenc}
% For Vietnamese characters
% \usepackage[T5]{fontenc}
% See https://www.latex-project.org/help/documentation/encguide.pdf for other character sets

% This assumes your files are encoded as UTF8
\usepackage[utf8]{inputenc}

% This is not strictly necessary, and may be commented out,
% but it will improve the layout of the manuscript,
% and will typically save some space.
\usepackage{xcolor}
\usepackage{microtype}
\newtheorem{definition}{Definition}
\newtheorem{exmp}{Example}[section]
\usepackage{latexsym}
\usepackage{multicol, multirow}
\usepackage{booktabs}
\usepackage{arydshln}
% If the title and author information does not fit in the area allocated, uncomment the following
%
%\setlength\titlebox{<dim>}
%
% and set <dim> to something 5cm or larger.

\usepackage{hyperref}       % hyperlinks
\usepackage{url}            % simple URL typesetting
\usepackage{booktabs}       % professional-quality tables
\usepackage{amsfonts}       % blackboard math symbols
\usepackage{nicefrac}       % compact symbols for 1/2, etc.
\usepackage{microtype}      % microtypography
\usepackage{lipsum}
\usepackage{graphicx}
\usepackage{enumitem}
\usepackage{color}
\usepackage{verbatimbox}
\usepackage{listings}

\usepackage{tikz}
\usetikzlibrary{intersections}
\usetikzlibrary{positioning}
\usepackage{wrapfig}


%New colors defined below
\definecolor{codegreen}{rgb}{0,0.6,0}
\definecolor{codegray}{rgb}{0.5,0.5,0.5}
\definecolor{codepurple}{rgb}{0.58,0,0.82}
\definecolor{backcolour}{rgb}{0.96,0.96,0.96}

\setlist[itemize]{leftmargin=*}
\setlist[enumerate]{leftmargin=*}
\graphicspath{ {./images/} }

%Code listing style named "mystyle"
\lstdefinestyle{mystyle}{
  backgroundcolor=\color{backcolour}, commentstyle=\color{codegreen},
  keywordstyle=\color{magenta},
  numberstyle=\tiny\color{codegray},
  stringstyle=\color{codepurple},
  basicstyle=\ttfamily\footnotesize,
  breakatwhitespace=false,         
  breaklines=true,                 
  captionpos=b,                    
  keepspaces=true,                 
  numbers=left,                    
  numbersep=5pt,                  
  showspaces=false,                
  showstringspaces=false,
  showtabs=false,                  
  tabsize=2
}

%"mystyle" code listing set
\lstset{style=mystyle}

\usepackage[strict]{changepage}
\usepackage{framed}
\definecolor{demonstrationshade}{rgb}{0.95,0.95,1}
\definecolor{promptshade}{rgb}{0.95,0.95,1}
\newenvironment{demonstration}{%
  \def\FrameCommand{%
    \hspace{1pt}%
    {\color{black}\vrule width 2pt}%
    {\color{demonstrationshade}\vrule width 4pt}%
    \colorbox{demonstrationshade}%
  }%
  \MakeFramed{\advance\hsize-\width\FrameRestore}%
  \noindent\hspace{-4.55pt}% disable indenting first paragraph
  \begin{adjustwidth}{1pt}{7pt}%
  \vspace{2pt}\vspace{2pt}%
}
{%
  \vspace{2pt}\end{adjustwidth}\endMakeFramed%
}

\definecolor{color-obs-seq}{RGB}{219, 232, 213}
\definecolor{color-act-seq}{RGB}{172, 204, 255}
\definecolor{color-opti}{RGB}{248,206,204} 
%\definecolor{color-gmap}{RGB}{204,255,204} 
\definecolor{color-gmap}{RGB}{0,204,153} 

% -------------------------------------------
% -- Color style: Honkai Star Rail Firefly --
% -------------------------------------------
\definecolor{ffblue}{RGB}{097, 108, 140}
\definecolor{ffdarkgreen}{RGB}{086, 140, 135}
\definecolor{fflightgreen}{RGB}{178, 213, 155}
\definecolor{ffyellow}{RGB}{242, 222, 121}
\definecolor{ffred}{RGB}{217, 095, 024}
\definecolor{ffred_pv}{RGB}{202, 074, 046}
\definecolor{fforange_pv}{RGB}{232, 141, 047}
\definecolor{ffgreen_pv}{RGB}{059, 165, 149}
\definecolor{ffgreendark_pv}{RGB}{032, 117, 106}
% -------------------------------------------
% -- Color style: Honkai Star Rail Firefly --
% -------------------------------------------
\definecolor{nature_tab_gray1}{HTML}{D8D6C2}
\definecolor{nature_tab_gray2}{HTML}{ECEADF}

\definecolor{graspw}{RGB}{0, 128, 0}
\definecolor{dp}{RGB}{64, 224, 208}
\definecolor{dp3}{RGB}{63, 63, 255}
\definecolor{ours}{RGB}{148, 0, 211}
\definecolor{blockcolor}{RGB}{238, 130, 238}
\definecolor{safeline}{RGB}{255, 0, 0}
\definecolor{bananacolor}{RGB}{255, 165, 0}

%\definecolor{ffdarkgreen}{RGB}{086, 140, 135}
% \usepackage{inconsolata}

\usepackage{pgfplots}
\usepgfplotslibrary{groupplots}
%\pgfplotsset{compat=1.14}
\usetikzlibrary{decorations.pathreplacing}
\pgfplotsset{
axis background/.style={fill=gallery},
grid=both,
  xtick pos=left,
  ytick pos=left,
  tick style={
    major grid style={style=white,line width=1pt},
    minor grid style=gallery,
    draw=none,
  },
  minor tick num=1,
}



\title{InductionBench: LLMs Fail in the Simplest\\Complexity Class}


\author{
  Wenyue Hua$^{1}$ Tyler Wong$^1$ Sun Fei\\
  Liangming Pan$^2$ Adam Jardine$^3$ William Yang Wang$^1$\footnote{Corresponding authors: wenyuehua@ucsb.edu, william@cs.ucsb.edu. I'm very grateful for extensive discussion with Wenda Xu, Xinyi Wang at UCSB.} \\
  \\
  $^1$University of California, Santa Barbara,\\
  $^2$University of Arizona,
  $^3$Rutgers University, New Brunswick
}



\begin{document}
\maketitle

\begin{abstract}
Retrieval-Augmented Generation (RAG) is often used with Large Language Models (LLMs) to infuse domain knowledge or user-specific information. In RAG, given a user query, a retriever extracts chunks of relevant text from a knowledge base. These chunks are sent to an LLM as part of the input prompt. Typically, any given chunk is repeatedly retrieved across user questions. However, currently, for every question, attention-layers in LLMs fully compute the key values (KVs) repeatedly for the input chunks, as state-of-the-art methods cannot reuse KV-caches when chunks appear at arbitrary locations with arbitrary contexts. Naive reuse leads to output quality degradation.  This leads to potentially redundant computations on expensive GPUs and increases latency. In this work, we propose \sys, a system for managing and reusing precomputed KVs corresponding to the text chunks (we call \textit{chunk-caches}) in RAG-based systems. We present how to identify \hl{\textit{chunk-caches} that are reusable}, how to efficiently perform a small fraction of recomputation to \textit{fix} the cache to maintain output quality, and how to efficiently store and evict \textit{chunk-caches} in the hardware for maximizing reuse while masking any overheads. With real production workloads as well as synthetic datasets, we show that \sys reduces redundant computation by \textbf{51\%} over SOTA prefix-caching and \textbf{75\%} over full recomputation.
\hl{Additionally, with continuous batching on a real production workload, we get a \textbf{1.6$\times$} speedup in throughput and a \textbf{2$\times$} reduction in end-to-end response latency over prefix-caching while maintaining quality, for both the \llama-3-8B and \llama-3-70B models. 
}
\end{abstract}






\section{Introduction}
\label{sec:intro}

\begin{figure*}[tb]
    \centering
    \includegraphics[width=0.848\linewidth]{figs/circuitnn.pdf} 
    \caption{Illustration of differentiable CircuitNN. CircuitNN is designed based on differentiable NAND gates. After DAS is guided by PI and PO pairs of the truth table, CircuitNN can get the precise circuit architecture logic equivalent to the truth table.}
    \label{fig:circuitnn}
\end{figure*}

% 1. Describe the importance of logic synthesis
% 2. Existing Problems
% (a) Neural Architecture Search: Unstable, Predefined Setting, etc.
% (b) Circuit Generation: Probabilistic Model, Logic Equivalence

With the rapid advancement of technology, the scale of integrated circuits (ICs) has expanded exponentially. 
This expansion has introduced significant challenges in chip manufacturing, particularly concerning power and area metrics.
A primary objective in IC design is achieving the same circuit function with fewer transistors, thereby reducing power usage and area occupancy.

Logic synthesis~\cite{hachtel2005logicsynth}, a critical step in electronic design automation (EDA), transforms behavioral-level circuit designs into optimized gate-level circuits, ultimately yielding the final IC layout. 
The primary goal of logic synthesis is to identify the physical implementation with the fewest gates for a given circuit function. 
This task constitutes a challenging NP-hard combinatorial optimization problem. 
Current logic synthesis tools~\cite{brayton2010abc, wolf2013yosys} rely on human-designed heuristics, often leading to sub-optimal outcomes.

Differentiable architecture search (DAS) techniques~\cite{liu2018darts, chu2020darts} offer novel perspectives on addressing challenges in this problem.
Circuit functions can be represented through truth tables, which map binary inputs to their corresponding outputs. 
Truth tables provide a precise representation of input-output relationships, ensuring the design of functionally equivalent circuits.
Inspired by this, researchers~\cite{deepmind2024ai4sys, wang2024tnet} have begun exploring the application of DAS to synthesize circuits directly from truth tables.
Specifically, \citet{deepmind2024ai4sys} proposed CircuitNN, a framework that learns differentiable connection structures with logic gates, enabling the automatic generation of logic circuits from truth tables.
This approach significantly reduces the complexity of traditional circuit generation. 
Building on this, \citet{wang2024tnet} introduced T-Net, a triangle-shaped variant of CircuitNN, incorporating regularization techniques to enhance the efficiency of DAS.

Despite these advancements, several challenges remain. 
The computational complexity of DAS grows quadratically with the number of gates, posing scalability issues.
Although triangle-shaped architecture~\cite{wang2024tnet} partially mitigates this problem, redundancy persists. 
%Additionally, DAS is susceptible to converging to local optima, limiting the ability to search architectures that satisfy the given truth tables~\cite{liu2018darts}. 
%Furthermore, hyperparameters (network depth and layer width) require extensive searches, introducing complexity and prolonging the synthesis process. 
Additionally, DAS is susceptible to converging to local optima~\cite{liu2018darts} and hyperparameters (network depth and layer width) require extensive searches. 
The challenges arise from the vast search space in DAS. 
% Even with predefined settings for CircuitNN, finding a configuration that meets the truth table requires extensive trial and error during the DAS process. 
Intuitively, limiting the search space through predefined parameters (network depth, gates per layer, and connection probabilities) can significantly reduce the complexity.

Recent advances~\cite{openai2023gpt4, abramson2024alphafold3, esser2024sd3, li2024mar} in conditional generative models have demonstrated remarkable performance across language, vision, and graph generation tasks. 
Motivated by these developments, we propose a novel approach to circuit generation that generates preliminary circuit structures to guide DAS in generating refined circuits matching specified truth tables. 
Firstly, we introduce CircuitVQ, a tokenizer with a discrete codebook for circuit tokenization. 
Built upon our Circuit AutoEncoder framework~\cite{hou2022graphmae,li2023maskgae,wu2025mgvga}, CircuitVQ is trained through a circuit reconstruction task. 
Specifically, the CircuitVQ encoder encodes input circuits into discrete tokens using a learnable codebook, while the decoder reconstructs the circuit adjacency matrix based on these tokens.
Subsequently, the CircuitVQ encoder serves as a circuit tokenizer for CircuitAR pretraining, which employs a masked autoregressive modeling paradigm~\cite{chang2022maskgit, li2023mage}. 
In this process, the discrete codes function as supervision signals. 
After training, CircuitAR can generate discrete tokens progressively, which can be decoded into initial circuit structures by the decoder of the CircuitVQ. 
These prior insights can guide DAS in producing refined circuits that match the target truth tables precisely.

Our key contributions can be summarized as follows:
\begin{itemize}
\item We introduce CircuitVQ, a circuit tokenizer that facilitates graph autoregressive modeling for circuit generation, based on our Circuit AutoEncoder framework;
\item Develop CircuitAR, a model trained using masked autoregressive modeling, which generates initial circuit structures conditioned on given truth tables;
\item Propose a refinement framework that integrates differentiable architecture search to produce functionally equivalent circuits guided by target truth tables;
\item Comprehensive experiments demonstrating the scalability and capability emergence of our CircuitAR and the superior performance of the proposed circuit generation approach.
\end{itemize}

% Motivation
% (a) Diffusion (Vision, Graph), Autoregressive (Language, Vision)
% (b) Circuit Generation for Predefined Setting
% (c) Neural Architecture Search for Strict Logic Equivalence

% Contribution
% (a) Circuit Tokenizer (new transformer arch, training strategy)
% (b) CircuitAR (train and gen strategies, post-ar strategy)
% (c) Extensive Evaluation including BitD (Bit Distance) for Scalability



\section{Related Work} \label{sec:related}

% \textbf{Adversarial Attack}
\textbf{Attacks on SLAM.} 
%With the rise of machine learning, 
The robustness of computer vision systems is being actively investigated. With the emergence of adversarial images in the digital domain by adding optimized noise directly to images~\cite{szegedy2013intriguing,carlini2017towards}, researchers find that such attacks also exist physically in the real world \cite{eykholt2018robust,song2018physical,zhao2019seeing}. To fill the gap between attacks in the digital and physical worlds, recent studies have demonstrated that attacks on real-world computer vision systems are practical \cite{eykholt2018robust,li2019adversarial,man2020ghostimage,sharif2016accessorize,zhao2019seeing,zhou2018invisible}. However, attacks on traditional computer vision methods such as SLAM are relatively less explored. \cite{yoshida2022adversarial} proposes an attack against the scan matching algorithm in LiDAR-based SLAM, while most SLAMs in AR/VR devices rely on different sensors like RGB/depth cameras and IMUs. \cite{ikram2022perceptual} and \cite{chen2024adversary} mislead visual SLAM by poisoning the images with special patterns, and \cite{wang2021can} causes the camera to fail using infrared light. In our work, we demonstrate attacks on Visual-Inertial SLAM (VI-SLAM) by perturbing the IMU readings, rather than cameras, and showing its impact on XR user experience. 

\textbf{Acoustic Injection Attacks.} Among various physical attacks, acoustic injection attacks are attractive due to their low cost. Son~\etal~\cite{son2015rocking} were the first to introduce acoustic attacks on MEMS gyroscopes, demonstrating how these attacks could lead to sensor denial-of-service and result in drone crashes. WALNUT~\cite{trippel2017walnut} expanded on this by developing output biasing and control attacks that enable precise manipulation of MEMS accelerometer outputs using modulated sound waves. Wang et al.~\cite{wang2017sonic} demonstrated a sonic gun, showcasing the vulnerability of various smart devices (\eg drones and self-balancing vehicles) to acoustic attacks. Tu et al. \cite{tu2018injected} designed side-swing and switching attacks to alter the outputs of MEMS gyroscopes and accelerometers. Furthermore, Ji et al. \cite{ji2021poltergeist} fool the object detectors by applying acoustic attack to the image stabilizers commonly used in modern cameras. However, none of the existing works study the relationship between the acoustic injections and SLAM outputs on recent XR devices. 

% \zijian{Do we need one session about security in AR/VR?}
% \yicheng{TODO}
%\jiasi{cite the AIVR paper (UMass Amherst?) paper is we have not already. They add IMU perturbation but w/o SLAM, iirc} \yicheng{Cited}

\textbf{XR Security and Privacy.} 
%Security and privacy concerns in XR systems have gained significant attention. 
For single-user XR systems, researchers have demonstrated various side-channel attacks to extract sensitive information (\eg keystrokes) through video feeds~\cite{ling2019know}, head movements~\cite{nair2023unique, slocum2023going}, architectural hints~\cite{zhang2023its,shang2020arspy}, power usage~\cite{li2024dangers}, and EM side-channel leakages~\cite{al2021vr}. In multi-user XR systems, Su et al.~\cite{su2024remote} use avatar motion data to infer keystrokes in shared VR environments. Slocum et al.~\cite{slocum2024doesn} reveal vulnerabilities in the shared state frameworks of multi-user AR. Similarly, Lebeck et al.~\cite{lebeck2017securing} highlight risks like deceptive virtual objects and emphasize access control for managing shared physical and virtual spaces. Ruth et al.~\cite{ruth2019secure} further propose a secure multi-user AR framework focusing on content sharing and permissions.
Chandio et al.~\cite{chandio2024stealthy} %introduced a multi-modal spatiotemporal attack that 
simultaneously manipulated visual and inertial sensors to disrupt XR pose estimation. However, their study evaluated the attack using offline datasets and assumed the attacker's capability to manipulate IMU data streams through acoustic means, without real experiments. Ours is the first to demonstrate acoustic injection attacks on recent XR devices, like the Hololens 2, in the real world.
 



\begin{tabular}{c|ccccc}
                                                                  & \multicolumn{5}{c}{\textbf{synHOR}}                                                          \\ \hline
Methods                                                           & P2S $\downarrow$ & CD $\downarrow$ & IoU $\uparrow$ & Normal $\uparrow$ & f-Score $\uparrow$ \\ \hline
PIFuHD                                                            & 33.75            & 67.59           & .4555          & .6595             & 18.07              \\
ECON                                                              & 36.48            & 79.39           & .3952          & .6547             & 14.57              \\
SiTH                                                              & 31.88            & 68.88           & .3371          & .6521             & 14.33              \\ \hline
$\mathrm{PIFuHD}_{\mathrm{ho}}$                                   & 43.50            & 91.97           & .3917          & .5661             & 11.00              \\
$\sigma\mathrm{PIFuHD}_{\mathrm{ho}}$                             & 31.66            & 50.78           & .4228          & .7090             & 20.59              \\
$2\mathrm{PIFuHD}_{\mathrm{ho}}$                                  & 48.29            & 80.09           & .3482          & .5347             & 10.79              \\
$2\sigma\mathrm{PIFuHD}_{\mathrm{ho}}$                            & 28.82            & 36.83           & .6966          & .7196             & 25.62              \\
\multicolumn{1}{c|}{$2\sigma\mathrm{PIFuHD}^{sep}_{\mathrm{ho}}$} & 25.83            & 33.20           &     .8176           & .7204             & 29.17              \\
\multicolumn{1}{l|}{$2\sigma\mathrm{PIFuHD}^{all}_{\mathrm{ho}}$} & 23.87          & 26.53           & .8172          & .7601             & 32.72              \\
\name (ours)                                       & \textbf{14.23}   & \textbf{17.15}  & \textbf{.8902} & \textbf{.8241}    & \textbf{48.84}    
\end{tabular}

\section{Benchmark Construction} 
In this section, we detail how our benchmark is constructed from the previously defined function classes. Each datapoint $(\mathcal{D}, f)$ in the benchmark is a pair of dataset $\mathcal{D}$ and function $f$ where $\mathcal{D}$ is a set of input-output pairs generated by $f$.

Each of ISL, L-OSL, and R-OSL classes can be further subdivided into incremental levels of complexity, determined by three key parameters: (1) the context window size $k$ (2) the vocabulary size $|\Sigma|$ (3) the minimal representation length of the function, \emph{i.e.} the minimal set of rules corresponding to the function. Given $k$ and $|\Sigma|$, the search space is $2^{|\Sigma|^k}$; given the number of rules $n$ additionally, the search space is $|\Sigma|^k \choose n$. To rigorously evaluate LLMs' inductive capabilities, we systematically vary these parameters across ISL, L-OSL, and R-OSL function classes.

In addition, we examine how performance changes with different numbers of input–output pairs in the prompt. Although having the characteristic sample present should theoretically guarantee recoverability of the underlying function, our empirical results indicate that the overall number of examples strongly affects performance. While extra data can provide richer information, it also increases context length considerably and heightens processing demands \citep{li2024long}. By varying the number of provided datapoints, we further investigate the extent to which the model engages in genuine reasoning and how robust its inductive abilities remain under changing input sizes.

\paragraph{Function Generation}
To systematically create benchmark instances, we first \emph{randomly generate} functions $f$ based on the three parameters: $k$, $|\Sigma|$, and the number of minimal rules describing $f$ by generating the set of rules that can describe $f$. While multiple representations of varying length can describe the same function, each function has a \emph{unique minimal representation} (up to isomorphism). During function generation, we therefore ensure that each function is expressed by a minimal, non-redundant rule set. Formally, if a $f$ is represented by a set of rules $R_f = \{r_1, r_2, ..., r_n\}$ where each $r_i$ has the form of $c_i\circ u_i\to v_i$ (with $c_i$ as the condition substring, $u_i$ the target character, and $v_i$ the transformed output for $u_i$), there are several constraints may be applied to functions belonging to the three classes. 

\begin{definition}[General Consistency]
Given $f$ represented by a set of rules $R_f: \forall r_i, r_j\in R_f, c_i\circ u_i\notin\textsc{Suff}(c_j\circ u_j)$ and $c_j\circ u_j\notin\textsc{Suff}(c_i\circ u_i)$.
\end{definition}

General Consistency ensures that the rules do not contradict one another or become redundant when conditions overlap. For instance, a function whose rule-based representation of $r_1: a\circ b\to a$ and $r_2: aa\circ b\to a$ is redundant, as the scenarios where $r_1$ is applied is a superset of the scenarios where $r_2$ is applied. For another instance, there does not exist a deterministic function that can be described by $r_1: a\circ b\to a$ and $r_2: aa\circ b\to \lambda$. Generating rule-based representations for ISL functions needs only satisfy this constraint.

\begin{definition}[OSL Non-Redundancy Guarantee]
Given $f$ represented by a set of rules $R:\forall r_i\in $ $R_f, \lnot\exists s_i'$ ${\in}\{s_i|s_i {\in} c_i\}$ such that $ \exists r_j\in R_f$ such that $ s_i' = c_j\circ u_j, \text{ unless }\exists r_k\in R$ such that $c_k\circ v_k = s_i'$.
\end{definition}

Constraint 2 is specific to the two OSL function classes because we need to make sure that all output conditions in the rule actually surface somewhere in the outputs of some datapoints. If the output condition $c$ never actually surface as the output, the rule will never be put into effect. Thereby the above rule basically requires that condition part of all rules can surface, either because it will never be modified by some other rule, or it emerges on the surface because of the application of other rule. For instance, a function represented by rules $r_1: aa\circ b\to a, r_2: a\circ a\to c$ is redundant because $r_1$ will never be applied because the string $aa$ will never surface as output and thus it will never be put into effect; For another instance, a function represented by $r_1: aa\circ b\to a, r_2: a\circ a\to c, r_3: a\circ d\to a$ is non redundant because even though into $aa$ string will be modified into $ac$, but $aa$ will surface in some datapoint because $ad$ will be modified into $aa$ and thus $r_1$ will be able to be applied.

Generating the functions following the two constraints, we ensure that the generated function representation is minimal, non-reducible guarantees a clear measure of complexity. One additional requirement is imposed to ensure each function indeed requires a look-ahead of size $k$. Specifically:

\begin{definition}[$k$-Complexity Guarantee]
Given $f$ whose designated context window $k = k_1$, $\exists r'\in R \text{ such that } c'\circ u'\to v'$ such that $|c'\circ u'| = k_1$.
\end{definition}

This condition guarantees that the function is genuinely $k$-strictly local (for ISL or OSL), rather than being representable with a smaller window size. Consequently, the functions we generate faithfully reflect the intended complexity level.

After generating the function $f$, we generate the characteristic sample of input-output pairs. For instance, given a function $f$ with $k = 2$ and $\Sigma = \{a, b\}$, the characteristic sample is $\{(a, f(a)),(b, f(b)), (ab, f(ab)), (aa, f(aa)), (bb, f(bb)), (ba, f(ba))\}$, a small set whose size is 6. By expanding this sample set, we can explore whether providing more than the minimal necessary examples aids or hinders the model's performance to infer the underlying function.

To evaluate how effectively an LLM can induce the underlying function, we include in the prompt (1) the function class, (2) context window $k$, (3) the alphabet $\Sigma$ which are information that guarantee learnability of the function. Then given the sample dataset, we request LLMs to produce a minimal rule-based description that reproduces the provided sample set, revealing whether it can \emph{discover} and \emph{optimally represent} the underlying transformation.



\section{Experiments}\label{sec_exp}
%\hp{Accelerating IM simulation~\cite{tang2015influence}}

% \begin{itemize}
%     \item 6.1. Problem setting of three COPs, including the general model and three specific CO problems 
%     \item 6.2. Experiment Setting (hyperparameters, details of training, evaluation, and test) 写在appendix里吧
%     \item 6.3. Performance analysis 这个要占半页
% \end{itemize}

%\hp{need to think of a way to compress these tables / visuals.} 

%\hp{\cancel{Baselines}; hyperparamters; \cancel{metrics}; etc.}

With theoretical guarantees on the existence and convergence of NE for ACCES games, we are also interested in how our proposed algorithm CCDO-RL works empirically. To evaluate this, we conduct experiments of CCDO-RL on three distinct ACCES game instances introduced in Section \ref{sub_exp_ins} and analyze the performance of CCDO-RL in Section \ref{sub_train_eval}. Section 6.2.1 aims to empirically demonstrate the convergence (Figures \ref{fig_exploit_20} and \ref{fig_exploit_50}) of the algorithm CCDO-RL over realistic CO problems, and show its consistency with Theorem \ref{CCDOA}. Section 6.2.2 intends to show the average reward (to seen training graphs) as well as the generalizability (to unseen test graphs) of the combinatorial player in real-world ACCES games (shown in Tables \ref{tab_aver}, and \ref{tab_gene}).

\subsection{Three Instances of ACCES Games} \label{sub_exp_ins}
% \hp{This para does not make much sense. Need to follow the framework in the Preliminaries section.}
% For combinatorial optimization problems in real-world applications, situations are more complicated and intractable due to changeable environmental or physical parameters. The form of parameter sets is very crucial because different types have different solvability and computation complexity. Forms of parameter sets mainly contain discrete sets, interval sets \cite{buchheim2018robust} like polyhedral and ellipsoid, probability distributions \cite{carlsson2018wasserstein}, and variable functions \cite{krause2008robust}.

% In reality, these parameters are often impacted by some common factors, such as conditions of weather, transportation, and individual personalities. \cite{kalimeris2019robust} proposed an assumption that real instances (e.g. demands in CVRP, coverages in CSP) 
%Considering affected or attacked COPs, the real instance $\{\theta_{i}\}$ always relied on the estimated value $\{\hat{\theta}_{i}$\} and the variation determined by independent factors $\{g_{i}\}$ and environment/physical parameters/attacker actions $\{\eta\}$. The concrete parameter influence model is stated as follows:

We consider a certain COP which is parameterized with $\{\theta_{i}\}$, where $i$ is the index of nodes (such as a target in security games) -- e.g., such parameters can be interpreted as attack probability of targets.
%coverage radius, customer's demands, or attack probability of targets. 
In real-world applications, we often need to estimate such parameters before solving the COPs. Unfortunately, the estimation $\{\hat{\theta}_{i}\}$ often bears a gap to the true value $\{\theta_{i}\}$, which derives from e.g. environment (aleatoric) uncertainty, model (epistemic) uncertainty, or an attacker trying to manipulate the defender's utility. We use a generic model to formulate this gap:
\begin{equation}\label{linrob}
    \theta_{i} = \hat{\theta}_{i} + y \cdot \tau_{i},
\end{equation}
where $y$ represents the strategy of the nature/attacker, $\tau_{i}$ is the environment factors like weather and transportation conditions, or human subjective factors like the preference of the attacker. 
Such abstraction can represent a wide range of ACCES games, such as facility location covering problems \cite{an2020battery, TIRKOLAEE2020340}, CVRP \cite{vehiclerouting.ch8,dinh2018exact, FLORIO20231081}, security patrolling (OP) \citep{xu2021robust}, and influence maximization problem \cite{kalimeris2019robust}. We describe three instances of ACCES games based on the model (\ref{linrob}).%Based on this model (\ref{linrob}), we focus on three combinatorial optimization problems with attacks or environmental/physical influence.

% \hp{Hard to follow. We should point out what are the two players, what are X, Y, u etc}

\textbf{Adversarial Covering Salesman Problem (ACSP):} In a map of cities, every city $i$ has a coverage $\theta_{i}$. A salesman finds the shortest path such that all cities are visited or covered, with $\theta_{i}$ influenced by physical factors $\tau_i$ and transportation parameters $y$ based on Eq.(\ref{linrob}). The salesman is Player 1 where $X$ consists of the feasible paths of the salesman. Nature is Player 2 with $Y$ = $[0, 1]^K \ni y, K \in \mathbb{N}$. The utility function of Player 1 $u$ is the opposite of the total traveling distance.

\textbf{Adversarial Capacitated Vehicle Routing Problem (ACVRP):} A vehicle with a constrained capacity of goods finds the shortest path under the worst case with the $i_{th}$ customer's demand $\theta_i$ changed by environmental factors $\tau_i$ and weather parameter $y$ on Eq.(\ref{linrob}). The vehicle is Player 1 where $X$ is the set of the feasible path $x$. Nature is Player 2 where $Y$ is $[0, 1]^K \ni y, K \in \mathbb{N}$. The utility function of Player 1  $u$ is the opposite of total delivery distance satisfying all the demands of customers.


\textbf{Patrolling Game (PG):} The patrolling game is described in the introduction.

For all the problem instances, we run our algorithm on two problem sizes: 20 nodes and 50 nodes. The detailed description and problem parameters of the three game instances are in Appendix \ref{app_ex_para_set}.

% Similarly, in the vehicle route problem (VRP), conditions with correlated parameters arouse broad attention from scholars \cite{vehiclerouting.ch8,dinh2018exact,FLORIO20231081}. \cite{dinh2018exact} considered the demand correlation by geographical proximity of nodes, described by some independent random variables in the fractional form. \cite{FLORIO20231081} utilized 'external factors' to stand for unknown covariates affecting all demands and presented a Bayesian model to learn correlations. Further more, about IM problems, \cite{kalimeris2019robust} combined node features and uncertain hyperparameters to fit the influence probability on each edge.

% \subsection{Training CCDO-RL}

% For all the problems, CCDO-RL adopts the REINFORCE algorithm with an attention-based encoder-decoder framework \cite{kool2018attention} (used as an inductive graph representation component) to learn a (generalizable) COP solver for one player (protagonist), and PPO \cite{schulman2017proximal} to train a policy for the other player (adversary) whose strategy space is continuous. CCDO-RL is trained with 50 epochs on a set of 10,000 graphs (with 20 or 50 nodes). The hyperparameters of CCDO-RL are specified in Appendix \ref{app_ex_para_set} (Table \ref{tab_hyper_ccdorl}). Our code is included as supplementary material for ease of reproduction. 
% % \hp{need to specify hyperparas}

\subsection{Performance of CCDO-RL}\label{sub_train_eval}

Two aspects are evaluated for the performance of CCDO-RL, i.e., i) Convergence to NE (Section \ref{sub_per_conver}) exploring whether CCDO-RL can compute the NE, and ii) Protagonist policy's average reward and generalizability (Section \ref{sub_per_rob}). Generalizability refers to the ability of RL models trained on previously seen graphs (problem instances), to perform well on a new set of unseen test graphs. The model’s usability is enhanced by generalizability, rather than focusing solely on the average reward, which is a critical motivation in the literature on RL for COPs \citep{khalil2017learning, kool2018attention}.

For all the problems, CCDO-RL adopts the REINFORCE algorithm with an attention-based encoder-decoder framework \citep{kool2018attention} (used as an inductive graph representation component) to learn a generalizable COP solver for Player 1 (protagonist), and PPO to train a policy for Player 2 (adversary) whose strategy space is continuous. CCDO-RL is trained on a set of 10,000 graphs (with 20 or 50 nodes). The hyperparameters of CCDO-RL are specified in Appendix \ref{app_ex_para_set} (Table \ref{tab_hyper_ccdorl}). Our code is included as supplementary material and will be open-sourced for ease of reproduction. 

% \textbf{Training.} For all the problems, CCDO-RL adopts the REINFORCE algorithm with attention-based encoder-decoder framework \cite{kool2018attention} (used as an inductive graph representation component) to learn a (generalizable) COP solver for one player (protagonist), and PPO \cite{schulman2017proximal} to train a policy for the other player (adversary) whose strategy space is continuous. CCDO-RL is trained with 50 epochs on a set of 10,000 graphs (with 20 or 50 nodes). 

% \hp{We should first present results about convergence as it is mostly aligned with the theory.}

\subsubsection{Convergence to NE} \label{sub_per_conver}

Exploitability is a common metric to describe the closeness to true NE by calculating the sum of performance distances between each new best response and subgame NE, i.e. $\sum_{i=1,2} U(\pi_{i,k}^{br}, \sigma_{-i,k}) - U(\sigma)$ in the general two-player game. Since our game is zero-sum, the calculation is as follows:
\begin{equation*}
   \text{Exploitability}(\sigma) = \max_{\pi_1 \in \Sigma_1} U(\pi_1, \sigma_{2}) - \min_{\pi_2 \in \Sigma_2} U(\sigma_1, \pi_2).
\end{equation*}
From Figure \ref{fig_exploit_20}, we can see that CCDO-RL can converge to approximate NE in 25 iterations or less (in the PG setting), reaching 0.05 in ACSP, 0.10 in ACVRP, and 0.03 in PG with 20 nodes. Similar results are observed in problems with 50 nodes (see Figure \ref{fig_exploit_50} in Appendix \ref{app_exp}). These results validate the effectiveness of CCDO-RL in finding the NE for various types of games.

%Similarly, the exploitability of three COPs in 50 nodes is provided in the appendix \ref{app_exp}.
\vspace{-\baselineskip}
\begin{figure}[htbp]
	\centering
    \subfigure[ACSP20]{
    \label{csp20_nashconv}
    \includegraphics[scale=0.20]{Figures/nashconv_log_csp20_sm_7.eps}
    }
    \subfigure[ACVRP20]{
    \label{cvrp20_nashconv}%文中引用该图片代号
    \includegraphics[scale=0.20]{Figures/nashconv_log_svrp20_sm_7.eps}
    }
    \subfigure[PG20]{
    \label{opsa20_nashconv}
    \includegraphics[scale=0.20]{Figures/nashconv_log_pg20_sm_7.eps}
    }
    \caption{Exploitability curve of CCDO-RL on three games of 20 nodes}
    \label{fig_exploit_20}
\end{figure}
\vspace{-\baselineskip}
\subsubsection{Average reward and Generalizability of Combinatorial player} \label{sub_per_rob}
% \subsubsection{Robustness and Generalizability of Protagonist Policy} \label{sub_per_rob}
%\hp{CCDO-RL being better in these following metrics is only kind of a by-product.}

% \textbf{Evaluation.} The learned policies are then tested on 200 graphs, where 100 of them are randomly selected from the 10,000 training graphs, and the other 100 are unseen graphs. 
% We use two metrics to evaluate the performance of different policies for the protagonist player: \textbf{Average proportional loss} $R-$ describes the policy overfitting degree \citep{lanctot2017unified}; \textbf{Reward} evaluates the performance of the protagonist with the adversary under three COPs.  
% \begin{eqnarray}
%         &R- = (\hat{D} - \hat{O}) / \hat{D}.
% \end{eqnarray}
% in which $\hat{D}$ is the mean value of the diagonals and $\hat{O}$ is the mean value of the off-diagonals in the payoff matrix provided in the Appendix \ref{app_exp}.

% Because the protagonist policy is trained against a powerful adversary under our ACCES game setting, the obtained policy is naturally robust against adversarial perturbations. This subsection sheds a bit of light on this perspective and quantifies the extent of robustness of CCDO-RL as well as the ability of RL to generalize to unseen test graphs.

\textbf{Evaluation.} The learned policies are tested on 200 graphs, with 100 being randomly selected from the 10,000 training graphs (to show the average reward), and the other 100 being unseen graphs (to test policy generalization). We evaluate the performance of the protagonist with the adversary under three COPs. For each COP, the performance is considered both on the 20-node and 50-node map.
% We use two metrics to evaluate the performance of different policies for the protagonist player: \textbf{Average proportional loss} $R-$ describes the policy overfitting degree \citep{lanctot2017unified}; \textbf{Reward} evaluates the performance of the protagonist with the adversary under three COPs.

\textbf{Baselines.} There are heuristic algorithms for each game instance (Heuristic in Table \ref{tab_aver} and \ref{tab_gene}) and a single-player RL algorithm. For ACVRP, we adopt the Tabu Search algorithm (Tabu) \citep{li2020improved} as the heuristic algorithm, which is widely applied in the routing problem. For ACSP, the common benchmark local search algorithm, LS2 \citep{golden2012generalized}, is used. For PG, we choose the greedy algorithm as the baseline. The "RL against Stoc" algorithm in Tables \ref{tab_aver} and \ref{tab_gene} is identical to the protagonist model in CCDO-RL but trained in environments with stochastic adversarial perturbations.

% \textbf{Baselines.} There are a heuristic algorithms for each game instance {\color{red} (Heuristic mentioned in the Table \ref{tab_aver} and \ref{tab_gene})} and a single-player RL algorithm. For ACVRP, we adopt the Clarke-Wright (CW) algorithm \citep{pichpibul2013heuristic} and the Tabu Search algorithm (Tabu) \citep{li2020improved} as heuristics, which are applied widely in the routing problem. For ACSP, two common benchmark local search algorithms, LS1 and LS2 \citep{golden2012generalized}, are used. For PG, we choose a local search algorithm \citep{vansteenwegen2009iterated} and the greedy algorithm as the heuristic baselines. {\color{red} The "RL  against Stoc" algorithm referred to Tables \ref{tab_aver} and \ref{tab_gene}} is identical to the protagonist model in CCDO-RL {\color{red} but trained on environments with stochastic adversarial perturbations.} 

\textbf{Average Reward.}  As illustrated in Table \ref{tab_aver}, our algorithm achieves a better average reward than baselines (10.08\% improvement on average of all settings against two baselines), regardless of CO instance or problem size, when confronting the adversary trained by CCDO-RL. In the setting of CSP-20 nodes, the average reward is improved by 46.98\% compared to the heuristic and by 7.14\% compared with the RL against Stoc. For the 50-node setting, the improvements are 45.91\% and 5.28\% respectively. Similarly, the improvements in contrast to Heuristic and RL against Stoc are as follows: 1.72\% and 3.01\%  for CVRP-20 nodes, 0.75\% and 4.46\% for CVRP-50 nodes, 4.17\% and 1.48\% for PG-20 nodes, and 10.60\% and 4.38\% for PG-50 nodes.

\textbf{Generalizability.} From Table \ref{tab_gene}, CCDO-RL continues to achieve a better average reward when facing the adversary, demonstrating that the learned RL policies generalize well to unseen graphs. Even though the non-RL baselines do have access to the graph structures and other problem information of the unseen problem instances, CCDO-RL can obtain comparable performances without re-training on the new problem instances. The improvements versus Heuristic and RL against Stoc are 46.61\% and 7.02\% for CSP-20 nodes, 42.24\% and 3.94\% for CSP-50 nodes, 1.12\% and 1.56\% for CVRP-20 nodes, 0.90\% and 5.05\% for CVRP-50 nodes, 5.35\% and 2.40\% for PG-20 nodes, and 12.17\% and 10.33\% for PG-50 nodes. Even when confronting the stochastic adversary, CCDO shows superior generalizability compared to two baselines across three COPs, with average improvements of 6.31\%, 3.42\%, and 3.95\% respectively. Detailed results are provided in Appendix \ref{app_exp} (Tables \ref{tab_csp_full_20} - \ref{tab_op_full_50}). 
% The model’s usability is enhanced by the ability to generalize rather than focusing solely on the average reward, which is a critical motivation of the RL for combinatorial optimization literature \citep{khalil2017learning, kool2018attention}.  

\begin{remark}
    In CO problems (or more broadly, operations research and economics), it is known that achieving solution quality improvements against strong baselines (e.g., the RL methods trained with a stochastic adversary) is very challenging, and the margins are usually small \citep{kool2018attention}, sometimes even less than 1\%. However, these “tiny” marginal improvements in profits keep small business owners in the real world alive. Last, the improvement depends a lot on the problem settings, and we show that sometimes the improvement can be much more significant.
\end{remark}
\vspace{-\baselineskip}
% \textbf{Performance analysis.} The robustness results of CCDO-RL for ACSP are shown in Table \ref{tab_csp}. We have the following observations: 1) On both of the 100 seen/unseen graphs, single-player RL performs better than heuristic algorithms no matter whether attacked or not. (2) When confronting the adversary trained by CCDO-RL, CCDO-RL exceeds RL by 0.25 and 0.24 on the training set, and by 0.25 and 0.18 on the test set, respectively under the 20-node and 50-node graphs. This demonstrates the robustness of CCDO-RL. 3) Compared to the performance of the training set with that of the test set, we can see that RL and CCDO-RL both maintain a certain degree of generalization. Similar results for ACVRP (Table \ref{tab_cvrp}) and SPG (Table \ref{tab_op}) are provided in Appendix \ref{app_exp}. 

\begin{table}[ht]
  \caption{Average reward against CCDO-RL's adversary (on seen graphs)}
  \vspace{\baselineskip}
  \label{tab_aver}
  \centering
  \small
  \begin{tabular}{lllllll}
    \toprule
    \multirow{2}{*}{method} & \multicolumn{2}{c}{ACSP (Mean$\pm$Std)} & \multicolumn{2}{c}{ACVRP (Mean$\pm$Std)} & \multicolumn{2}{c}{PG (Mean$\pm$Std)} \\
    \cmidrule(r){2-3} \cmidrule{4-5} \cmidrule(r){6-7}
                            & 20 nodes & 50 nodes & 20 nodes & 50 nodes & 20 nodes & 50 nodes\\
    \midrule
    Heuristic & 6.13$\pm$1.20 & 7.55$\pm$1.42 & 7.65$\pm$1.23  & 13.38$\pm$1.70 & 2.64$\pm$1.03 & 4.53$\pm$1.84   \\
    RL against Stoc    & 3.50$\pm$0.47  & 4.55$\pm$0.62  & 7.55$\pm$1.16  & 13.90$\pm$1.63 & 2.71$\pm$0.90 & 4.80$\pm$2.18   \\
    CCDO-RL   & $\pmb{3.25}$$\pm$0.42 & $\pmb{4.31}$$\pm$0.51  & $\pmb{7.42}$$\pm$1.21  & $\pmb{13.28}$$\pm$1.52 &  $\pmb{2.75}$$\pm$0.87 & $\pmb{5.01}$$\pm$1.91  \\
    \bottomrule
  \end{tabular}
\end{table}
\vspace{-\baselineskip}

\begin{table}[htp]
  \caption{Generalizability against CCDO-RL's adversary (on unseen graphs)}
  \vspace{\baselineskip}
  \label{tab_gene}
  \centering
  \small
  \begin{threeparttable}
  \begin{tabular}{lllllll}
    \toprule
    \multirow{2}{*}{method} & \multicolumn{2}{c}{ACSP (Mean$\pm$Std)} & \multicolumn{2}{c}{ACVRP (Mean$\pm$Std)} & \multicolumn{2}{c}{PG (Mean$\pm$Std)} \\
    \cmidrule(r){2-3} \cmidrule{4-5} \cmidrule(r){6-7}
                            & 20 nodes & 50 nodes & 20 nodes & 50 nodes & 20 nodes & 50 nodes\\
    \midrule
    Heuristic & 6.20$\pm$1.33 & 7.60$\pm$1.37   & 7.64$\pm$1.30  & 13.27$\pm$1.87 & 2.43$\pm$0.98 & 4.19$\pm$1.69    \\
    RL against Stoc  & 3.56$\pm$0.37  & 4.57$\pm$0.58  & 7.67$\pm$1.30  & 13.85$\pm$1.53 &  2.50$\pm$0.95 & 4.26$\pm$2.17 \\
    CCDO-RL   & $\pmb{3.31}$$\pm$0.35 & $\pmb{4.39}$$\pm$0.52  & $\pmb{7.55}$$\pm$1.28  & $\pmb{13.15}$$\pm$1.59 & $\pmb{2.56}$$\pm$0.92 & $\pmb{4.70}$$\pm$1.94\\

    \bottomrule
  \end{tabular}
  \begin{tablenotes}
      \footnotesize
      \item[1] For the average reward of ACSP and ACVRP, smaller is better while for that of PG larger is better.
  \end{tablenotes}
  \end{threeparttable}
\end{table}
\vspace{-\baselineskip}
% two heuristics and one RL
% \begin{table}[ht]
%   \caption{{\color{red} Average reward of CCDO-RL (on seen graphs). For the value of CSP and CVRP, larger is better while for that of PG smaller is better.}}
%   \label{tab_aver}
%   \centering
%   \small
%   \begin{tabular}{lllllll}
%     \toprule
%     \multirow{2}{*}{method} & \multicolumn{2}{c}{CSP (Mean$\pm$Std)} & \multicolumn{2}{c}{CVRP (Mean$\pm$Std)} & \multicolumn{2}{c}{PG (Mean$\pm$Std)} \\
%     \cmidrule(r){2-3} \cmidrule{4-5} \cmidrule(r){6-7}
%                             & 20 nodes & 50 nodes & 20 nodes & 50 nodes & 20 nodes & 50 nodes\\
%     \midrule
%     Baseline 1 & 4.52$\pm$0.71  & 5.98$\pm$0.94 & 7.64$\pm$1.56  & 13.49$\pm$2.10 & 2.71$\pm$1.10 & 1.82$\pm$1.40   \\
%     Baseline 2 & 6.13$\pm$1.20 & 7.55$\pm$1.42   & 7.65$\pm$1.23  & 13.38$\pm$1.70 & 2.64$\pm$1.03 & 1.47$\pm$0.99  \\
%     RL {\color{red}against Stoc}    & 3.50$\pm$0.47  & 4.55$\pm$0.62  & 7.55$\pm$1.16  & 13.90$\pm$1.63 & 2.71$\pm$0.90 & 1.54$\pm$1.03   \\
%     CCDO-RL   & $\pmb{3.25}$$\pm$0.42 & $\pmb{4.31}$$\pm$0.51  & $\pmb{7.42}$$\pm$1.21  & $\pmb{13.28}$$\pm$1.52 &  $\pmb{2.75}$$\pm$0.87 & $\pmb{1.87}$$\pm$1.22  \\
%     \bottomrule
%   \end{tabular}
% \end{table}


% \begin{table}[htp]
%   \caption{{\color{red}Generalizability of CCDO-RL (on unseen graphs)}}
%   \label{tab_gene}
%   \centering
%   \small
%   \begin{threeparttable}
%   \begin{tabular}{lllllll}
%     \toprule
%     \multirow{2}{*}{method} & \multicolumn{2}{c}{CSP (Mean$\pm$Std)} & \multicolumn{2}{c}{CVRP (Mean$\pm$Std)} & \multicolumn{2}{c}{PG (Mean$\pm$Std)} \\
%     \cmidrule(r){2-3} \cmidrule{4-5} \cmidrule(r){6-7}
%                             & 20 nodes & 50 nodes & 20 nodes & 50 nodes & 20 nodes & 50 nodes\\
%     \midrule
%     Baseline 1 & 4.53$\pm$0.79  & 5.95$\pm$0.96 & 7.55$\pm$1.39  & 13.35$\pm$2.04 & 2.52$\pm$1.08 & $\pmb{1.86}$$\pm$1.44  \\
%     Baseline 2 & 6.20$\pm$1.33 & 7.60$\pm$1.37   & 7.64$\pm$1.3  & 13.27$\pm$1.87 & 2.43$\pm$0.98 & 1.52$\pm$1.20    \\
%     RL {\color{red}against Stoc}  & 3.56$\pm$0.37  & 4.57$\pm$0.58  & 7.67$\pm$1.30  & 13.85$\pm$1.53 &  2.50$\pm$0.95 & 1.03$\pm$5.05 \\
%     CCDO-RL   & $\pmb{3.31}$$\pm$0.35 & $\pmb{4.39}$$\pm$0.52  & $\pmb{7.55}$$\pm$1.28  & $\pmb{13.15}$$\pm$1.59 & $\pmb{2.56}$$\pm$0.92 & 1.35$\pm$5.09\\

%     \bottomrule
%   \end{tabular}
%   \begin{tablenotes}
%       \footnotesize
%       \item[1] For the value of CSP and CVRP, larger is better while for that of PG smaller is better.
%   \end{tablenotes}
%   \end{threeparttable}
% \end{table}


% ------------ 第一个表

\begin{table*}[tp]
\centering
\caption{The average score of the jailbreak methods on different victim LLMs (set = core).}
\label{tab:core-leaderboard}
\begin{threeparttable}
\footnotesize
\renewcommand{\arraystretch}{1.2}
\setlength{\tabcolsep}{0.9\tabcolsep}
\setlength{\defaultaddspace}{0.7\defaultaddspace} %
% \rowcolors{2}{white}{gray!12}
\centering
\begin{tabular}{lccccc|c}
\toprule
\multirow{2}{*}{\textbf{Method}} & \multicolumn{5}{c|}{\textbf{Score on Victim LLM (\%)}} & \multirow{2}{*}{\textbf{Average $\uparrow$}} \\
 & {Claude-3.5-Sonnet} & {GPT-3.5-Turbo} & {GPT-4-Turbo} & {Llama2-7B} & {Llama3.1-8B} & \\
\midrule
AutoDAN & - & - & - & 16.84 & 43.55 & 30.195 \\
SCAV & - & - & - & 36.84 & 19.51 & 28.175 \\
GCG & - & - & - & 8.80 & 9.46 & 9.130 \\
FSJ & - & - & - & 0.33 & 0.42 & 0.375 \\
\midrule
GPTFuzzer & 0.29 & 22.45 & 35.2 & 3.54 & 38.69 & 20.034 \\
DRA & 0.0 & 27.65 & 25.85 & 2.31 & 5.92 & 12.346 \\
DeepInception & 0.37 & 17.98 & 5.25 & 6.51 & 14.36 & 8.894 \\
MultiJail & 0.3 & 2.65 & 2.62 & 3.25 & 3.95 & 2.554 \\
\bottomrule
\end{tabular}
% \begin{tablenotes}
% 123
% \end{tablenotes}
\end{threeparttable}%
\end{table*}

% ----------------- 第二个表

\begin{table*}[tp]
\centering
\caption{The average score of the jailbreak methods across different harmful topics (set = core).}
\label{tab:core-topic}
\begin{threeparttable}
\footnotesize
\renewcommand{\arraystretch}{1.2}
\setlength{\tabcolsep}{0.9\tabcolsep}
\setlength{\defaultaddspace}{0.7\defaultaddspace} %
% \rowcolors{2}{white}{gray!10}
\centering
\resizebox{\textwidth}{!}{
\begin{tabular}{lcccccccc|c}
\toprule
\multirow{2}{*}{\textbf{Harmful Topic}} & \multicolumn{8}{c|}{\textbf{Score by Method (\%)}} & \multirow{2}{*}{\textbf{Average $\uparrow$}} \\
& {AutoDAN} & {DRA} & {DeepI.} & {FSJ} & {GCG} & {GPTFu.} & {MultiJ.} & {SCAV} &  \\
\midrule
General Copyright & 37.92 & 21.33 & 14.17 & 0 & 20 & 30.33 & 1.67 & \textbf{42.5} & 20.99 \\
%Political Participation & 33.75 & 16.5 & 3.75 & 0 & 26.88 & 18.25 & 24.25 & \textbf{38.13} & 20.18 \\
Disinformation & 18.15 & 15.52 & 17.52 & 0 & 25.93 & 21.41 & 4.52 & \textbf{37.04} & 17.51 \\
Do Harm to Human-kind & \textbf{36.92} & 14.85 & 8.35 & 0.63 & 11.5 & 21.17 & 3.33 & 33.25 & 16.25 \\
Abuse Animals & \textbf{39.5} & 20 & 7.6 & 0 & 9 & 21.4 & 0 & 29.17 & 15.83 \\
Drug & 36.81 & 10.78 & 4.83 & 0 & 3.33 & 27.83 & 1.11 & \textbf{40.97} & 15.70 \\
Dangerous Items & \textbf{30.92} & 11.47 & 11.7 & 5 & 10.58 & 16.9 & 2.83 & 25.33 & 14.34 \\
Children Crime & 28.75 & 8.25 & 1.67 & 0 & 11.46 & 20.17 & 7.75 & \textbf{35} & 14.13 \\
Do Harm to Public Interests & \textbf{28.33} & 14.69 & 8.65 & 0 & 9.06 & 21.85 & 1.35 & 26.77 & 13.83 \\
Harmful Economical Activities & \textbf{31.2} & 15.26 & 7.62 & 0 & 7.9 & 22.48 & 2.51 & 22.44 & 13.67 \\
Unequal Competition & \textbf{30.42} & 12.61 & 18.44 & 0 & 7.64 & 16.28 & 0 & 21.53 & 13.36 \\
Passby Safety Measures & \textbf{26.11} & 13.81 & 10.02 & 0 & 7.36 & 22.69 & 4.19 & 21.02 & 13.15 \\
Discrimination & \textbf{30.83} & 4.19 & 3.38 & 0 & 7.5 & 15.74 & 4.67 & 23.27 & 11.19 \\
Cybersecurity & \textbf{25.3} & 10.76 & 8.97 & 0 & 5.67 & 14.76 & 1.71 & 20.1 & 10.90 \\
Pornographic Information & 20.42 & 7 & 5.17 & 0 & 0 & 18 & 1.33 & \textbf{30.83} & 10.34 \\
Terrorism & \textbf{31.39} & 4.67 & 5.33 & 0 & 0 & 9.5 & 1.33 & 13.33 & 8.19 \\
\bottomrule
\end{tabular}
}
\begin{tablenotes}
\item \textbf{Bold} values highlight the highest score in each row.
\end{tablenotes}
\end{threeparttable}%
\end{table*}

% ----------------- 第三个表

\begin{table*}[tp]
\centering
\caption{The average score of the jailbreak methods on different victim LLMs and harmful topics (set = additional).}
\label{tab:additional-leaderboard}
\begin{threeparttable}
\footnotesize
\renewcommand{\arraystretch}{1.2}
\setlength{\tabcolsep}{0.9\tabcolsep}
\setlength{\defaultaddspace}{0.7\defaultaddspace} %
% \rowcolors{3}{white}{gray!10}
\centering
\resizebox{\textwidth}{!}{
\begin{tabular}{lcccc|cc|cc|ccc|ccc|c}
\toprule
\multirow{2}{*}{\textbf{Method}} & \multicolumn{4}{c|}{\textbf{Claude-3.5-sonnet}} & \multicolumn{2}{c|}{\textbf{GPT-3.5-turbo}} & \multicolumn{2}{c|}{\textbf{GPT-4-turbo}} & \multicolumn{3}{c|}{\textbf{Llama2-7B-Chat}} & \multicolumn{3}{c|}{\textbf{Llama3.1-8B-Instruct}} & \multirow{2}{*}{\textbf{Average $\uparrow$}} \\
& M & L & P & C & M & C & M & C & M & L & F & M & L & F \\
\midrule
SCAV & - & - & - & - & - & - & - & - & 0 & 70.83 & 87.5 & 0 & 35.42 & 87.5 & 46.88 \\
AutoDAN & - & - & - & - & - & - & - & - & 37.5 & 66.67 & 45.83 & 12.5 & 47.92 & 66.67 & 46.18 \\
GCG & - & - & - & - & - & - & - & - & 12.5 & 33.33 & 70.83 & 25 & 45.83 & 37.50 & 37.50 \\
FSJ & - & - & - & - & - & - & - & - & 12.5 & 0 & 12.5 & 25 & 0 & 0 & 8.33 \\
\midrule
DeepInception & 25 & 41.67 & 6.25 & 0 & 50 & 0 & 0 & 0 & 50 & 29.17 & 41.67 & 50 & 29.17 & 41.67 & 26.04 \\
MultiJail & 0 & 27.08 & 12.5 & 0 & 12.5 & 0 & 0 & 0 & 12.5 & 29.17 & 37.5 & 62.5 & 35.42 & 33.33 & 18.75 \\
GPTFuzzer & 25 & 6.25 & 0 & 0 & 37.5 & 0 & 25 & 0 & 0 & 12.5 & 41.67 & 25 & 27.08 & 45.83 & 17.56 \\
DRA & 25 & 12.5 & 0 & 0 & 12.5 & 0 & 12.5 & 0 & 25 & 37.5 & 12.5 & 50 & 29.17 & 20.83 & 16.96 \\
\bottomrule
\end{tabular}
}
\begin{tablenotes}
\item \textbf{M} - Medical Advice, \textbf{L} - Legal Advice, \textbf{P} - Political AI Engagement, \textbf{C} - Word-by-word Copyright, \textbf{F} - Financial Advice.
\item Align with the above tables, all numbers are percentages.
\end{tablenotes}
\end{threeparttable}%
\end{table*}




\section{Conclusion} 

In this work, we introduced a systematic benchmark for assessing the inductive reasoning capabilities of LLMs, leveraging both well-studied subregular function classes (ISL, L-OSL, and R-OSL) and a more exploratory class (IOSL) for which no known polynomial-time learning algorithm exists. By controlling parameters such as the Markov window size $k$, the vocabulary size $|\Sigma|$, and the minimal number of rules, we offered precise yet flexible tasks capable of probing a model’s capacity to infer general transformations from limited data. Our findings revealed several significant challenges for current LLMs—especially when required to track deeper dependencies or manage larger search spaces—and underscored the fragility of their inductive reasoning under increased context or novel data.

Through experiments measuring recall, precision, and compatibility, we demonstrated that factors like the Markov window size $k$ and the number of rules more profoundly degrade performance than an expanded alphabet. Moreover, while few-shot prompting showed promise in simpler scenarios, its benefits quickly plateaued in more complex contexts. An error analysis further highlighted how many rules go completely missing or become overgeneralized under stringent settings, indicating that LLMs often fail to synthesize key patterns comprehensively.

We also proposed an exploration leaderboard targeting IOSL functions, a class beyond established theoretical learnability, to address concerns that performance gains might stem from known polynomial-time algorithms rather than genuine inductive reasoning. This complementary evaluation opens avenues for research on less tractable classes and poses a more authentic test of generalization and adaptability.

Overall, our results highlight the need for more robust inductive reasoning strategies within current LLM architectures. We hope that our benchmark will help catalyze progress in both theoretical understanding and practical innovations around LLMs' inductive capabilities.

\section*{Limitations}

While our benchmark offers a rigorous, theoretically grounded approach to evaluating inductive reasoning in LLMs, current paper is subject to two notable constraints:

\paragraph{Synthetic Rather Than Real-World Data.} All tasks and evaluations rely on functions generated from carefully controlled parameters rather than naturally occurring texts or real-world datasets. Although this design enables precise measurement of inductive capabilities, it may not fully capture the complexity of practical language use, where ambiguous contexts, noisy inputs, and domain-specific factors can further challenge inference.

\paragraph{Restricted Access to the o1 Model.} Our investigation into the o1 family of models is hindered by limited availability and computational resources. As a result, certain aspects of o1’s inductive behavior may remain unexamined, and a more exhaustive exploration of variations or fine-tuning strategies for o1 could further illuminate its performance.

% Entries for the entire Anthology, followed by custom entries
\bibliography{iclr2025_conference}
\bibliographystyle{iclr2025_conference}


\appendix
\section{Appendix}
Full results on ISL, OSL, and few-shot experiments are presented here.

\clearpage
\begin{table*}[t]
    \centering
    \renewcommand{\arraystretch}{1.1}
    \resizebox{\textwidth}{!}{
    \begin{tabular}{l c ccc ccc ccc}
        \toprule
          \multirow{2}{*}{\bf Models}& \multirow{2}{*}{\bf Settings}& \multicolumn{3}{c}{\bf k = 2} & \multicolumn{3}{c}{\bf k = 3} & \multicolumn{3}{c}{\bf k = 4}\\
          \cmidrule(lr){3-5} \cmidrule(lr){6-8}  \cmidrule(lr){9-11}
          & & recall  & precision & compatibility & recall  & precision & compatibility & recall  & precision & compatibility \\
         \midrule 
         \multicolumn{11}{c}{\textbf{vocab size = 2}} \\
         \midrule
           \multirow{3}{*}{Llama-3.3 70B} & rules = 1 & 60.00 & 55.00 & 60.00 & 30.00 & 23.33 & 20.00 & 10.00 & 10.00 & 10.00\\
           & rules = 2 & 60.00 & 65.00 & 50.00 & 45.00 & 60.00 & 30.00 & 15.00 & 8.25 & 0.00\\
           & \cellcolor{SeaGreen3!15}rules = 3 & \cellcolor{SeaGreen3!15}53.33 & \cellcolor{SeaGreen3!15}68.33 & \cellcolor{SeaGreen3!15}20.00 & \cellcolor{SeaGreen3!15}30.00 & \cellcolor{SeaGreen3!15}46.67 & \cellcolor{SeaGreen3!15}10.00 & \cellcolor{SeaGreen3!15}16.67 & \cellcolor{SeaGreen3!15}8.54 & \cellcolor{SeaGreen3!15}0.00\\
           \hdashline
           \multirow{3}{*}{Llama-3.1 405B} & rules = 1 & 30.00 & 25.00 & 30.00 & 50.00 & 35.00 & 30.00 & 20.00 & 15.00 & 10.00\\
           & rules = 2 & 55.00 & 50.00 & 40.00 & 10.00 & 6.67 & 0.00 & 10.00 & 7.50 & 0.00\\
           & \cellcolor{SeaGreen3!15}rules = 3 & \cellcolor{SeaGreen3!15}56.67 & \cellcolor{SeaGreen3!15}44.17 & \cellcolor{SeaGreen3!15}20.00 & \cellcolor{SeaGreen3!15}10.00 & \cellcolor{SeaGreen3!15}9.50 & \cellcolor{SeaGreen3!15}0.00 & \cellcolor{SeaGreen3!15}0.00 & \cellcolor{SeaGreen3!15}0.00 & \cellcolor{SeaGreen3!15}0.00\\
           \hdashline
           \multirow{3}{*}{DeepSeek-V3} & rules = 1 & 90.00 & 60.00 & 60.00 & 50.00 & 32.50 & 50.00 & 50.00 & 12.33 & 30.00\\
           & rules = 2 & 80.00 & 60.00 & 40.00 & 40.00 & 19.89 & 10.00 & 15.00 & 3.82 & 0.00\\
           & \cellcolor{SeaGreen3!15}rules = 3 & \cellcolor{SeaGreen3!15}70.00 & \cellcolor{SeaGreen3!15}54.83 & \cellcolor{SeaGreen3!15}40.00 & \cellcolor{SeaGreen3!15}40.00 & \cellcolor{SeaGreen3!15}26.15 & \cellcolor{SeaGreen3!15}0.00 & \cellcolor{SeaGreen3!15}33.33 & \cellcolor{SeaGreen3!15}10.61 & \cellcolor{SeaGreen3!15}0.00\\
           \hdashline
           \multirow{3}{*}{GPT-4o} & rules = 1 & 60.00 & 43.33 & 40.00 & 50.00 & 22.00 & 40.00 & 10.00 & 2.00 & 0.00 \\
           & rules = 2 & 60.00 & 37.50 & 50.00 & 35.00 & 18.43 & 20.00 & 15.00 & 5.63 & 10.00 \\
           & \cellcolor{SeaGreen3!15}rules = 3 & \cellcolor{SeaGreen3!15}73.33 & \cellcolor{SeaGreen3!15}68.33 & \cellcolor{SeaGreen3!15}60.00 & \cellcolor{SeaGreen3!15}36.67 & \cellcolor{SeaGreen3!15}19.30 & \cellcolor{SeaGreen3!15}0.00 & \cellcolor{SeaGreen3!15}13.33 & \cellcolor{SeaGreen3!15}2.50 & \cellcolor{SeaGreen3!15}0.00\\
           \hdashline
           \multirow{3}{*}{o1-mini} & rules = 1 & 50.00 & 45.00 & 50.00 & 70.00 & 45.83 & 40.00 & 30.00 & 16.67 & 10.00 \\
           & rules = 2 & 75.00 & 75.00 & 75.00 & 70.00 & 60.00 & 40.00 & 50.00 & 28.67 & 0.00\\
           & \cellcolor{SeaGreen3!15}rules = 3 & \cellcolor{SeaGreen3!15}66.67 & \cellcolor{SeaGreen3!15}61.67 & \cellcolor{SeaGreen3!15}60.00 & \cellcolor{SeaGreen3!15}43.33 & \cellcolor{SeaGreen3!15}34.00 & \cellcolor{SeaGreen3!15}0.00 & \cellcolor{SeaGreen3!15}40.00 & \cellcolor{SeaGreen3!15}38.52 & \cellcolor{SeaGreen3!15}0.00\\
           \hdashline
           \multirow{3}{*}{o3-mini} & rules = 1 & 100.00 & 100.00 & 100.00 & 100.00 & 100.00 & 100.00 & 80.00 & 58.33 & 40.00\\
           & rules = 2 & 90.00 & 90.00 & 90.00 & 90.00 & 90.67 & 90.00 & 85.00 & 80.00 & 50.00\\
           & \cellcolor{SeaGreen3!15}rules = 3 & \cellcolor{SeaGreen3!15}\textbf{100.00} & \cellcolor{SeaGreen3!15}\textbf{97.50} & \cellcolor{SeaGreen3!15}\textbf{100.00} & \cellcolor{SeaGreen3!15}\textbf{83.33} & \cellcolor{SeaGreen3!15}\textbf{71.83} & \cellcolor{SeaGreen3!15}\textbf{60.00} & \cellcolor{SeaGreen3!15}\textbf{70.00} & \cellcolor{SeaGreen3!15}\textbf{63.81} & \cellcolor{SeaGreen3!15}\textbf{20.00}\\
           \midrule 
           \multicolumn{11}{c}{\textbf{vocab size = 3}} \\
         \midrule
           \multirow{3}{*}{Llama-3.3 70B} & rules = 1 & 70.00 & 60.00 & 60.00 & 20.00 & 20.00 & 20.00 & 20.00 & 8.33 & 10.00\\
           & rules = 2 & 85.00 & 83.33 & 60.00 & 10.00 & 7.50 & 0.00 & 5.00 & 2.50 & 0.00\\
           & \cellcolor{SeaGreen3!30}rules = 3 & \cellcolor{SeaGreen3!30}66.67 & \cellcolor{SeaGreen3!30}74.17 & \cellcolor{SeaGreen3!30}20.00 & \cellcolor{SeaGreen3!30}33.33 & \cellcolor{SeaGreen3!30}35.36 & \cellcolor{SeaGreen3!30}0.00 & \cellcolor{SeaGreen3!30}6.67 & \cellcolor{SeaGreen3!30}3.43 & \cellcolor{SeaGreen3!30}0.00 \\
           \hdashline
           \multirow{3}{*}{Llama-3.1 405B} & rules = 1 & 20.00 & 10.00 & 10.00 & 20.00 & 10.00 & 10.00 & 10.00 & 10.00 & 10.00\\
           & rules = 2 & 45.00 & 32.58 & 20.00 & 10.00 & 7.50 & 0.00 & 0.00 & 0.00 & 0.00\\
           & \cellcolor{SeaGreen3!30}rules = 3 & \cellcolor{SeaGreen3!30}50.00 & \cellcolor{SeaGreen3!30}38.45 & \cellcolor{SeaGreen3!30}20.00 & \cellcolor{SeaGreen3!30}6.67 & \cellcolor{SeaGreen3!30}3.10 & \cellcolor{SeaGreen3!30}0.00 & \cellcolor{SeaGreen3!30}10.00 & \cellcolor{SeaGreen3!30}5.27 & \cellcolor{SeaGreen3!30}0.00\\
           \hdashline
           \multirow{3}{*}{DeepSeek-V3} & rules = 1 & 70.00 & 65.00 & 60.00 & 70.00 & 45.00 & 60.00 & 50.00 & 13.93 & 50.00\\
           & rules = 2 & 80.00 & 56.00 & 40.00 & 40.00 & 13.23 & 0.00 & 25.00 & 1.89 & 0.00\\
           & \cellcolor{SeaGreen3!30}rules = 3 & \cellcolor{SeaGreen3!30}80.00 & \cellcolor{SeaGreen3!30}60.76 & \cellcolor{SeaGreen3!30}50.00 & \cellcolor{SeaGreen3!30}56.67 & \cellcolor{SeaGreen3!30}21.64 & \cellcolor{SeaGreen3!30}0.00 & \cellcolor{SeaGreen3!30}40.00 & \cellcolor{SeaGreen3!30}5.88 & \cellcolor{SeaGreen3!30}0.00\\
           \hdashline
           \multirow{3}{*}{GPT-4o} & rules = 1 & 50.00 & 33.33 & 50.00 & 50.00 & 18.33 & 40.00 & 20.00 & 4.17 & 10.00\\
           & rules = 2 & 60.00 & 40.42 & 30.00 & 25.00 & 6.00 & 0.00 & 30.00 & 6.39 & 0.00 \\
           & \cellcolor{SeaGreen3!30}rules = 3 & \cellcolor{SeaGreen3!30}66.67 & \cellcolor{SeaGreen3!30}64.33 & \cellcolor{SeaGreen3!30}40.00 & \cellcolor{SeaGreen3!30}30.00 & \cellcolor{SeaGreen3!30}9.61 & \cellcolor{SeaGreen3!30}0.00 & \cellcolor{SeaGreen3!30}10.00 & \cellcolor{SeaGreen3!30}1.72 & \cellcolor{SeaGreen3!30}0.00\\
           \hdashline
           \multirow{3}{*}{o1-mini} & rules = 1 & 80.00 & 80.00 & 80.00 & 50.00 & 43.33 & 40.00 & 30.00 & 0.00 & 0.00\\
           & rules = 2 & 90.00 & 90.00 & 80.00 & 40.0 & 25.11 & 10.00 & 55.00 & 24.82 & 10.00\\
           & \cellcolor{SeaGreen3!30}rules = 3 & \cellcolor{SeaGreen3!30}80.00 & \cellcolor{SeaGreen3!30}77.33 & \cellcolor{SeaGreen3!30}60.00 & \cellcolor{SeaGreen3!30}63.33 & \cellcolor{SeaGreen3!30}36.25 & \cellcolor{SeaGreen3!30}10.00 & \cellcolor{SeaGreen3!30}30.00 & \cellcolor{SeaGreen3!30}20.44 & \cellcolor{SeaGreen3!30}0.00\\
           \hdashline
           \multirow{3}{*}{o3-mini} & rules = 1 & 100.00 & 100.000 & 100.00 & 100.00 & 95.00 & 90.00 & 90.00 & 78.33 & 70.00\\
           & rules = 2 & 100.00 & 100.000 & 100.00 & 95.00 & 91.67 & 80.00 & 75.00 & 75.00 & 50.00\\
           & \cellcolor{SeaGreen3!30}rules = 3 & \cellcolor{SeaGreen3!30}\textbf{96.67} & \cellcolor{SeaGreen3!30}\textbf{97.50} & \cellcolor{SeaGreen3!30}\textbf{80.00} & \cellcolor{SeaGreen3!30}\textbf{93.33} & \cellcolor{SeaGreen3!30}\textbf{91.67} & \cellcolor{SeaGreen3!30}\textbf{90.00} & \cellcolor{SeaGreen3!30}\textbf{83.33} & \cellcolor{SeaGreen3!30}\textbf{85.17} & \cellcolor{SeaGreen3!30}\textbf{50.00}\\
           \midrule 
           \multicolumn{11}{c}{\textbf{vocab size = 4}} \\
         \midrule
           \multirow{3}{*}{Llama-3.3 70B} & rules = 1 & 60.00 & 60.00 & 60.00 & 30.00 & 30.00 & 30.00 & 10.00 & 10.00 & 10.00\\
           & rules = 2 & 40.00 & 40.00 & 30.00 & 15.00 & 11.67 & 0.00 & 0.00 & 0.00 & 0.00\\
           & \cellcolor{SeaGreen3!50}rules = 3 & \cellcolor{SeaGreen3!50}53.33 & \cellcolor{SeaGreen3!50}68.33 & \cellcolor{SeaGreen3!50}20.00 & \cellcolor{SeaGreen3!50}6.67 & \cellcolor{SeaGreen3!50}5.00 & \cellcolor{SeaGreen3!50}0.00 & \cellcolor{SeaGreen3!50}10.00 & \cellcolor{SeaGreen3!50}5.32 & \cellcolor{SeaGreen3!50}0.00\\
           \hdashline
           \multirow{3}{*}{Llama-3.1 405B} & rules = 1 & 40.00 & 35.00 & 30.00 & 10.00 & 5.00 & 0.00 & 10.00 & 10.00 & 10.00\\
           & rules = 2 & 75.00 & 52.33 & 10.00 & 10.00 & 5.83 & 0.00 & 0.00 & 0.00 & 0.00\\
           & \cellcolor{SeaGreen3!50}rules = 3 & \cellcolor{SeaGreen3!50}40.00 & \cellcolor{SeaGreen3!50}34.33 & \cellcolor{SeaGreen3!50}10.00 & \cellcolor{SeaGreen3!50}13.33 & \cellcolor{SeaGreen3!50}1.54 & \cellcolor{SeaGreen3!50}0.00 & \cellcolor{SeaGreen3!50}10.00 & \cellcolor{SeaGreen3!50}3.75 & \cellcolor{SeaGreen3!50}0.00\\
           \hdashline
           \multirow{3}{*}{DeepSeek-V3} & rules = 1 & 80.00 & 52.50 & 60.00 & 50.00 & 17.00 & 40.00 & 40.00 & 14.58 & 40.00\\
           & rules = 2 & 85.00 & 57.15 & 50.00 & 65.00 & 18.66 & 20.00 & 45.00 & 5.30 & 0.00\\
           & \cellcolor{SeaGreen3!50}rules = 3 & \cellcolor{SeaGreen3!50}76.67 & \cellcolor{SeaGreen3!50}63.12 & \cellcolor{SeaGreen3!50}40.00 & \cellcolor{SeaGreen3!50}50.00 & \cellcolor{SeaGreen3!50}16.05 & \cellcolor{SeaGreen3!50}10.00 & \cellcolor{SeaGreen3!50}13.33 & \cellcolor{SeaGreen3!50}2.46 & \cellcolor{SeaGreen3!50}0.00\\
           \hdashline
           \multirow{3}{*}{GPT-4o} & rules = 1 & 50.00 & 40.00 & 40.00 & 50.00 & 16.67 & 20.00 & 50.00 & 21.67 & 20.00\\
           & rules = 2 & 45.00 & 29.00 & 10.00 & 45.00 & 12.62 & 0.00 & 0.00 & 0.00 & 0.00\\
           & \cellcolor{SeaGreen3!50}rules = 3 & \cellcolor{SeaGreen3!50}56.67 & \cellcolor{SeaGreen3!50}38.62 & \cellcolor{SeaGreen3!50}0.00 & \cellcolor{SeaGreen3!50}33.33 & \cellcolor{SeaGreen3!50}20.60 & \cellcolor{SeaGreen3!50}0.00 & \cellcolor{SeaGreen3!50}10.00 & \cellcolor{SeaGreen3!50}2.67 & \cellcolor{SeaGreen3!50}0.00\\
           \hdashline
           \multirow{3}{*}{o1-mini} & rules = 1 & 80.00 & 70.00 & 80.00 & 60.00 & 50.00 & 50.00 & 40.00 & 15.00 & 10.00\\
           & rules = 2 & 75.00 & 63.33 & 60.00 & 50.00 & 29.93 & 10.00 & 35.00 & 35.00 & 0.00 \\
           & \cellcolor{SeaGreen3!50}rules = 3 & \cellcolor{SeaGreen3!50}93.33 & \cellcolor{SeaGreen3!50}91.67 & \cellcolor{SeaGreen3!50}80.00 & \cellcolor{SeaGreen3!50}46.67 & \cellcolor{SeaGreen3!50}37.72 & \cellcolor{SeaGreen3!50}10.00 & \cellcolor{SeaGreen3!50}36.67 & \cellcolor{SeaGreen3!50}22.09 & \cellcolor{SeaGreen3!50}0.00\\
           \hdashline
           \multirow{3}{*}{o3-mini} & rules = 1 & 100.00 & 100.00 & 100.00 & 100.00 & 95.00 & 100.00 & 60.00 & 60.00 & 60.00\\
           & rules = 2 & 100.00 & 100.00 & 100.00 & 95.00 & 91.67 & 80.00 & 75.00 & 76.67 & 40.00\\
           & \cellcolor{SeaGreen3!50}rules = 3 & \cellcolor{SeaGreen3!50}\textbf{96.67} & \cellcolor{SeaGreen3!50}\textbf{95.00} & \cellcolor{SeaGreen3!50}\textbf{90.00} & \cellcolor{SeaGreen3!50}\textbf{93.33} & \cellcolor{SeaGreen3!50}\textbf{93.33} & \cellcolor{SeaGreen3!50}\textbf{80.00} & \cellcolor{SeaGreen3!50}\textbf{73.33} & \cellcolor{SeaGreen3!50}\textbf{59.58} & \cellcolor{SeaGreen3!50}\textbf{10.00}\\
           %\hdashline
           %o1-preview & \cellcolor{SeaGreen3!50}rules = 3 & -- & -- & -- & -- & -- & -- & \textbf{40.00} & \textbf{56.67} & 0.00 \\
           \bottomrule
    \end{tabular}
    }
    \caption{Input Strictly Local with sample size = 2}
    \label{tab:ISL_main}
\end{table*}

\begin{table*}[t]
    \centering
    \renewcommand{\arraystretch}{1.1}
    \resizebox{14.5cm}{!}{
    \begin{tabular}{l c ccc ccc ccc}
        \toprule
          \multirow{2}{*}{\bf Models}& \multirow{2}{*}{\bf Settings}& \multicolumn{3}{c}{\bf k = 2} & \multicolumn{3}{c}{\bf k = 3} & \multicolumn{3}{c}{\bf k = 4}\\
          \cmidrule(lr){3-5} \cmidrule(lr){6-8}  \cmidrule(lr){9-11}
          & & recall  & precision & compatibility & recall  & precision & compatibility & recall  & precision & compatibility \\
         \midrule 
         \multicolumn{11}{c}{\textbf{vocab size = 2}} \\
         \midrule
           \multirow{3}{*}{Llama-3.3 70B} & rules = 1 & 50.00 & 45.00 & 50.00 & 0.00 & 0.00 & 0.00 & 0.00 & 0.00 & 0.00\\
           & rules = 2 & 25.00 & 25.00 & 20.00 & 10.00 & 8.33 & 10.00 & 5.00 & 10.00 & 0.00\\
           & \cellcolor{SeaGreen3!15}rules = 3 & \cellcolor{SeaGreen3!15}56.67 & \cellcolor{SeaGreen3!15}65.00 & \cellcolor{SeaGreen3!15}0.00 & \cellcolor{SeaGreen3!15}6.67 & \cellcolor{SeaGreen3!15}8.33 & \cellcolor{SeaGreen3!15}0.00 & \cellcolor{SeaGreen3!15}13.33 & \cellcolor{SeaGreen3!15}12.83 & \cellcolor{SeaGreen3!15}0.00\\
           \hdashline
           \multirow{3}{*}{Llama-3.1 405B} & rules = 1 & 70.00 & 45.83 & 70.00 & 30.00 & 9.33 & 10.00 & 10.00 & 1.67 & 10.00\\
           & rules = 2 & 50.00 & 33.33 & 10.00 & 25.00 & 11.39 & 0.00 & 10.00 & 3.00 & 0.00\\
           & \cellcolor{SeaGreen3!15}rules = 3 & \cellcolor{SeaGreen3!15}63.33 & \cellcolor{SeaGreen3!15}53.83 & \cellcolor{SeaGreen3!15}0.00 & \cellcolor{SeaGreen3!15}10.00 & \cellcolor{SeaGreen3!15}6.67 & \cellcolor{SeaGreen3!15}0.00 & \cellcolor{SeaGreen3!15}6.67 & \cellcolor{SeaGreen3!15}5.00 & \cellcolor{SeaGreen3!15}0.00\\
           \hdashline
           \multirow{3}{*}{GPT-4o} & rules = 1 & 30.00 & 12.50 & 30.00 & 30.00 & 10.83 & 10.00 & 10.00 & 5.00 & 10.00\\
           & rules = 2 & 75.00 & 63.17 & 60.00 & 20.00 & 7.42 & 0.00 & 15.00 & 6.35 & 0.00\\
           & \cellcolor{SeaGreen3!15}rules = 3 & \cellcolor{SeaGreen3!15}66.67 & \cellcolor{SeaGreen3!15}60.00 & \cellcolor{SeaGreen3!15}50.00 & \cellcolor{SeaGreen3!15}30.00 & \cellcolor{SeaGreen3!15}19.00 & \cellcolor{SeaGreen3!15}10.00 & \cellcolor{SeaGreen3!15}10.00 & \cellcolor{SeaGreen3!15}4.16 & \cellcolor{SeaGreen3!15}0.00\\
           \hdashline
           \multirow{3}{*}{DeepSeek-V3} & rules = 1 & 100.00 & 75.00 & 70.00 & 50.00 & 32.50 & 40.00 & 40.00 & 12.78 & 40.00\\
           & rules = 2 & 60.00 & 44.17 & 30.00 & 10.00 & 15.00 & 10.00 & 20.00 & 11.94 & 0.00\\
           & \cellcolor{SeaGreen3!15}rules = 3 & \cellcolor{SeaGreen3!15}83.33 & \cellcolor{SeaGreen3!15}77.67 & \cellcolor{SeaGreen3!15}50.00 & \cellcolor{SeaGreen3!15}20.00 & \cellcolor{SeaGreen3!15}12.92 & \cellcolor{SeaGreen3!15}0.00 & \cellcolor{SeaGreen3!15}23.33 & \cellcolor{SeaGreen3!15}13.97 & \cellcolor{SeaGreen3!15}0.00\\
           \hdashline
           \multirow{3}{*}{o1-mini} & rules = 1 & 90.00 & 90.00 & 90.00 & 70.00 & 55.00 & 40.00 & 10.00 & 10.00 & 10.00\\
           & rules = 2 & 80.00 & 80.00 & 80.00 & 60.00 & 60.00 & 50.00 & 65.00 & 53.83 & 10.00\\
           & \cellcolor{SeaGreen3!15}rules = 3 & \cellcolor{SeaGreen3!15}90.00 & \cellcolor{SeaGreen3!15}82.50 & \cellcolor{SeaGreen3!15}50.00 & \cellcolor{SeaGreen3!15}66.67 & \cellcolor{SeaGreen3!15}60.67 & \cellcolor{SeaGreen3!15}20.00 & \cellcolor{SeaGreen3!15}50.00 & \cellcolor{SeaGreen3!15}54.22 & \cellcolor{SeaGreen3!15}10.00\\
           \hdashline
           \multirow{3}{*}{o3-mini} & rules = 1 & 100.00 & 100.00 & 100.00 & 90/00 & 90.00 & 90.00 & 90.00 & 90.00 & 90.00\\
           & rules = 2 & 100.00 & 100.00 & 100.00 & 95.00 & 100.00 & 100.00 & 85.00 & 78.33 & 70.00\\
           & \cellcolor{SeaGreen3!15}rules = 3 & \cellcolor{SeaGreen3!15}\textbf{100.00} & \cellcolor{SeaGreen3!15}\textbf{100.00} & \cellcolor{SeaGreen3!15}\textbf{100.00} & \cellcolor{SeaGreen3!15}\textbf{86.67} & \cellcolor{SeaGreen3!15}\textbf{85.00} & \cellcolor{SeaGreen3!15}\textbf{80.00} & \cellcolor{SeaGreen3!15}\textbf{56.67} & \cellcolor{SeaGreen3!15}\textbf{50.33} & \cellcolor{SeaGreen3!15}\textbf{40.00}\\
           \midrule 
           \multicolumn{11}{c}{\textbf{vocab size = 3}}\\
         \midrule
           \multirow{3}{*}{Llama-3.3 70B} & rules = 1 & 50.00 & 50.00 & 50.00 & 20.00 & 12.50 & 10.00 & 20.00 & 13.33 & 10.00\\
           & rules = 2 & 35.00 & 33.67 & 10.00 & 20.00 & 6.93 & 10.00 & 25.00 & 15.00 & 0.00\\
           & \cellcolor{SeaGreen3!30}rules = 3 & \cellcolor{SeaGreen3!30}40.00 & \cellcolor{SeaGreen3!30}65.00 & \cellcolor{SeaGreen3!30}20.00 & \cellcolor{SeaGreen3!30}20.00 & \cellcolor{SeaGreen3!30}18.33 & \cellcolor{SeaGreen3!30}0.00 & \cellcolor{SeaGreen3!30}10.00 & \cellcolor{SeaGreen3!30}2.78 & \cellcolor{SeaGreen3!30}0.00\\
           \hdashline
           \multirow{3}{*}{Llama-3.1 405B} & rules = 1 & 60.00 & 45.00 & 40.00 & 10.00 & 3.33 & 10.00 & 10.00 & 1.11 & 0.00\\
           & rules = 2 & 30.00 & 20.00 & 0.00 & 15.00 & 13.33 & 0.00 & 5.00 & 0.53 & 0.00\\
           & \cellcolor{SeaGreen3!30}rules = 3 & \cellcolor{SeaGreen3!30}66.67 & \cellcolor{SeaGreen3!30}57.83 & \cellcolor{SeaGreen3!30}30.00 & \cellcolor{SeaGreen3!30}20.00 & \cellcolor{SeaGreen3!30}8.39 & \cellcolor{SeaGreen3!30}0.00 & \cellcolor{SeaGreen3!30}10.00 & \cellcolor{SeaGreen3!30}2.36 & \cellcolor{SeaGreen3!30}0.00\\
           \hdashline
           \multirow{3}{*}{GPT-4o} & rules = 1 & 40.00 & 27.50 & 40.00 & 20.00 & 8.33 & 20.00 & 40.00 & 11.00 & 30.00\\
           & rules = 2 & 55.00 & 46.50 & 10.00 & 45.00 & 25.67 & 0.00 & 30.00 & 6.50 & 10.00\\
           & \cellcolor{SeaGreen3!30}rules = 3 & \cellcolor{SeaGreen3!30}60.00 & \cellcolor{SeaGreen3!30}50.00 & \cellcolor{SeaGreen3!30}10.00 & \cellcolor{SeaGreen3!30}33.33 & \cellcolor{SeaGreen3!30}15.95 & \cellcolor{SeaGreen3!30}0.00 & \cellcolor{SeaGreen3!30}20.00 & \cellcolor{SeaGreen3!30}6.21 & \cellcolor{SeaGreen3!30}0.00\\
           \hdashline
           \multirow{3}{*}{DeepSeek-V3} & rules = 1 & 80.00 & 70.00 & 60.00 & 50.00 & 22.00 & 40.00 & 50.00 & 16.11 & 30.00\\
           & rules = 2 & 90.00 & 60.32 & 60.00 & 70.00 & 13.82 & 20.00 & 30.00 & 5.04 & 0.00\\
           & \cellcolor{SeaGreen3!30}rules = 3 & \cellcolor{SeaGreen3!30}66.67 & \cellcolor{SeaGreen3!30}53.50 & \cellcolor{SeaGreen3!30}40.00 & \cellcolor{SeaGreen3!30}23.33 & \cellcolor{SeaGreen3!30}16.42 & \cellcolor{SeaGreen3!30}0.00 & \cellcolor{SeaGreen3!30}30.00 & \cellcolor{SeaGreen3!30}7.54 & \cellcolor{SeaGreen3!30}\textbf{10.00}\\
           \hdashline
           \multirow{3}{*}{o1-mini} & rules = 1 & 100.00 & 95.00 & 90.00 & 80.00 & 63.33 & 70.00 & 30.00 & 17.50 & 30.00\\
           & rules = 2 & 90.00 & 83.33 & 80.00 & 70.00 & 49.42 & 40.00 & 35.00 & 29.52 & 20.00\\
           & \cellcolor{SeaGreen3!30}rules = 3 & \cellcolor{SeaGreen3!30}\textbf{96.67} & \cellcolor{SeaGreen3!30}\textbf{96.00} & \cellcolor{SeaGreen3!30}\textbf{90.00} & \cellcolor{SeaGreen3!30}70.00 & \cellcolor{SeaGreen3!30}56.15 & \cellcolor{SeaGreen3!30}30.00 & \cellcolor{SeaGreen3!30}50.00 & \cellcolor{SeaGreen3!30}33.58 & \cellcolor{SeaGreen3!30}0.00\\
           \hdashline
           \multirow{3}{*}{o3-mini} & rules = 1 & 100.00 & 100.00 & 100.00 & 90.00 & 90.00 & 90.00 & 80.00 & 80.00 & 80.00\\
           & rules = 2 & 100.00 & 100.00 & 100.00 & 90.00 & 75.15 & 70.00 & 80.00 & 72.50 & 50.00\\
           & \cellcolor{SeaGreen3!30}rules = 3 & \cellcolor{SeaGreen3!30}\textbf{96.67} & \cellcolor{SeaGreen3!30}94.17 & \cellcolor{SeaGreen3!30}\textbf{90.00} & \cellcolor{SeaGreen3!30}\textbf{96.67} & \cellcolor{SeaGreen3!30}\textbf{87.50} & \cellcolor{SeaGreen3!30}\textbf{90.00} & \cellcolor{SeaGreen3!30}\textbf{63.33} & \cellcolor{SeaGreen3!30}\textbf{68.43} & \cellcolor{SeaGreen3!30}\textbf{40.00}\\
           \midrule 
           \multicolumn{11}{c}{\textbf{vocab size = 4}}\\
         \midrule
           \multirow{3}{*}{Llama-3.3 70B} & rules = 1 & 50.00 & 29.00 & 30.00 & 20.00 & 13.33 & 10.00 & 10.00 & 10.00 & 10.00\\
           & rules = 2 & 50.00 & 50.00 & 10.00 & 20.00 & 15.96 & 0.00 & 0.00 & 0.00 & 0.00\\
           & \cellcolor{SeaGreen3!50}rules = 3 & \cellcolor{SeaGreen3!50}50.00 & \cellcolor{SeaGreen3!50}52.50 & \cellcolor{SeaGreen3!50}20.00 & \cellcolor{SeaGreen3!50}6.67 & \cellcolor{SeaGreen3!50}6.00 & \cellcolor{SeaGreen3!50}0.00 & \cellcolor{SeaGreen3!50}10.00 & \cellcolor{SeaGreen3!50}6.33 & \cellcolor{SeaGreen3!50}0.00\\
           \hdashline
           \multirow{3}{*}{Llama-3.1 405B} & rules = 1 & 60.00 & 34.50 & 30.00 & 10.00 & 5.00 & 10.00 & 10.00 & 5.00 & 10.00\\
           & rules = 2 & 50.00 & 29.00 & 0.00 & 10.00 & 3.13 & 0.00 & 10.00 & 2.90 & 0.00\\
           & \cellcolor{SeaGreen3!50}rules = 3 & \cellcolor{SeaGreen3!50}43.33 & \cellcolor{SeaGreen3!50}30.73 & \cellcolor{SeaGreen3!50}20.00 & \cellcolor{SeaGreen3!50}16.67 & \cellcolor{SeaGreen3!50}5.83 & \cellcolor{SeaGreen3!50}0.00 & \cellcolor{SeaGreen3!50}6.67 & \cellcolor{SeaGreen3!50}1.10 & \cellcolor{SeaGreen3!50}0.00\\
           \hdashline
           \multirow{3}{*}{GPT-4o} & rules = 1 & 40.00 & 35.00 & 30.00 & 60.00 & 25.33 & 40.00 & 40.00 & 12.50 & 10.00\\
           & rules = 2 & 75.00 & 45.50 & 30.00 & 55.00 & 20.47 & 10.00 & 20.00 & 9.44 & 0.00\\
           & \cellcolor{SeaGreen3!50}rules = 3 & \cellcolor{SeaGreen3!50}70.00 & \cellcolor{SeaGreen3!50}48.22 & \cellcolor{SeaGreen3!50}20.00 & \cellcolor{SeaGreen3!50}33.33 & \cellcolor{SeaGreen3!50}10.77 & \cellcolor{SeaGreen3!50}0.00 & \cellcolor{SeaGreen3!50}13.33 & \cellcolor{SeaGreen3!50}3.82 & \cellcolor{SeaGreen3!50}0.00\\
           \hdashline
           \multirow{3}{*}{DeepSeek-V3} & rules = 1 & 100.00 & 82.50 & 80.00 & 50.00 & 23.67 & 30.00 & 40.00 & 16.11 & 40.00\\
           & rules = 2 & 70.00 & 50.67 & 30.00 & 25.00 & 8.01 & 0.00 & 15.00 & 3.24 & 0.00\\
           & \cellcolor{SeaGreen3!50}rules = 3 & \cellcolor{SeaGreen3!50}60.00 & \cellcolor{SeaGreen3!50}48.36 & \cellcolor{SeaGreen3!50}30.00 & \cellcolor{SeaGreen3!50}50.00 & \cellcolor{SeaGreen3!50}12.81 & \cellcolor{SeaGreen3!50}20.00 & \cellcolor{SeaGreen3!50}23.33 & \cellcolor{SeaGreen3!50}2.73 & \cellcolor{SeaGreen3!50}0.00\\
           \hdashline
           \multirow{3}{*}{o1-mini} & rules = 1 & 90.00 & 85.00 & 90.00 & 50.00 & 38.33 & 30.00 & 40.00 & 28.33 & 20.00\\
           & rules = 2 & 100.00 & 93.33 & 80.00 & 60.00 & 36.92 & 20.00 & 50.00 & 31.92 & 10.00\\
           & \cellcolor{SeaGreen3!50}rules = 3 & \cellcolor{SeaGreen3!50}80.00 & \cellcolor{SeaGreen3!50}72.25 & \cellcolor{SeaGreen3!50}60.00 & \cellcolor{SeaGreen3!50}70.00 & \cellcolor{SeaGreen3!50}43.04 & \cellcolor{SeaGreen3!50}10.00 & \cellcolor{SeaGreen3!50}43.33 & \cellcolor{SeaGreen3!50}32.12 & \cellcolor{SeaGreen3!50}0.00\\
           \hdashline
           \multirow{3}{*}{o3-mini} & rules = 1 & 100.00 & 100.00 & 100.00 & 100.00 & 100.00 & 100.00 & 60.00 & 60.00 & 60.00\\
           & rules = 2 & 100.00 & 100.00 & 100.00 & 85.00 & 81.67 & 70.00 & 45.00 & 58.33 & 20.00\\
           & \cellcolor{SeaGreen3!50}rules = 3 & \cellcolor{SeaGreen3!50}\textbf{100.00} & \cellcolor{SeaGreen3!50}\textbf{100.00} & \cellcolor{SeaGreen3!50}\textbf{100.00} & \cellcolor{SeaGreen3!50}\textbf{76.67} & \cellcolor{SeaGreen3!50}\textbf{76.67} & \cellcolor{SeaGreen3!50}\textbf{70.00} & \cellcolor{SeaGreen3!50}\textbf{66.67} & \cellcolor{SeaGreen3!50}\textbf{69.17} & \cellcolor{SeaGreen3!50}\textbf{10.00}\\
           %\hdashline
           %o1-preview & \cellcolor{SeaGreen3!50}rules = 3 & -- & -- & -- & -- & -- & -- & 60.00 & \textbf{82.67} & \textbf{20.00} \\
           \bottomrule
    \end{tabular}
    }
    \caption{Left Output Strictly Local with sample size = 2}
    \label{tab:LOSL_main}
\end{table*}

\begin{table*}[t]
    \centering
    \renewcommand{\arraystretch}{1.1}
    \resizebox{14.5cm}{!}{
    \begin{tabular}{l c ccc ccc ccc}
        \toprule
          \multirow{2}{*}{\bf Models}& \multirow{2}{*}{\bf Settings}& \multicolumn{3}{c}{\bf k = 2} & \multicolumn{3}{c}{\bf k = 3} & \multicolumn{3}{c}{\bf k = 4}\\
          \cmidrule(lr){3-5} \cmidrule(lr){6-8}  \cmidrule(lr){9-11}
          & & recall  & precision & compatibility & recall  & precision & compatibility & recall  & precision & compatibility \\
         \midrule 
         \multicolumn{11}{c}{\textbf{vocab size = 2}} \\
         \midrule
           \multirow{3}{*}{Llama-3.3 70B} & rules = 1 & 40.00 & 40.00 & 40.00 & 20.00 & 10.00 & 10.00 & 20.00 & 10.00 & 10.00\\
           & rules = 2 & 40.00 & 40.00 & 30.00 & 15.00 & 18.33 & 10.00 & 20.00 & 18.67 & 10.00\\
           & \cellcolor{SeaGreen3!15}rules = 3 & \cellcolor{SeaGreen3!15}63.33 & \cellcolor{SeaGreen3!15}76.67 & \cellcolor{SeaGreen3!15}30.00 & \cellcolor{SeaGreen3!15}23.33 & \cellcolor{SeaGreen3!15}24.17 & \cellcolor{SeaGreen3!15}0.00 & \cellcolor{SeaGreen3!15}10.00 & \cellcolor{SeaGreen3!15}7.83 & \cellcolor{SeaGreen3!15}0.00\\
           \hdashline
           \multirow{3}{*}{Llama-3.1 405B} & rules = 1 & 20.00 & 8.33 & 0.00 & 40.00 & 18.33 & 0.00 & 0.00 & 0.00 & 0.00\\
           & rules = 2 & 60.00 & 51.67 & 40.00 & 25.00 & 12.50 & 0.00 & 10.00 & 6.25 & 10.00\\
           & \cellcolor{SeaGreen3!15}rules = 3 & \cellcolor{SeaGreen3!15}63.33 & \cellcolor{SeaGreen3!15}54.72 & \cellcolor{SeaGreen3!15}30.00 & \cellcolor{SeaGreen3!15}20.00 & \cellcolor{SeaGreen3!15}14.33 & \cellcolor{SeaGreen3!15}10.00 & \cellcolor{SeaGreen3!15}6.67 & \cellcolor{SeaGreen3!15}4.50 & \cellcolor{SeaGreen3!15}0.00\\
           \hdashline
           \multirow{3}{*}{GPT-4o} & rules = 1 & 50.00 & 30.00 & 30.00 & 10.00 & 5.00 & 10.00 & 0.00 & 0.00 & 0.00\\
           & rules = 2 & 70.00 & 61.67 & 60.00 & 45.00 & 16.93 & 10.00 & 30.00 & 15.83 & 20.00\\
           & \cellcolor{SeaGreen3!15}rules = 3 & \cellcolor{SeaGreen3!15}83.33 & \cellcolor{SeaGreen3!15}71.83 & \cellcolor{SeaGreen3!15}30.00 & \cellcolor{SeaGreen3!15}43.33 & \cellcolor{SeaGreen3!15}24.10 & \cellcolor{SeaGreen3!15}0.00 & \cellcolor{SeaGreen3!15}20.00 & \cellcolor{SeaGreen3!15}10.00 & \cellcolor{SeaGreen3!15}0.00\\
           \hdashline
           \multirow{3}{*}{DeepSeek-V3} & rules = 1 & 60.00 & 38.33 & 20.00 & 40.00 & 14.17 & 20.00 & 20.00 & 7.00 & 10.00\\
           & rules = 2 & 75.00 & 51.67 & 40.00 & 35.00 & 18.33 & 0.00 & 25.00 & 13.00 & 20.00\\
           & \cellcolor{SeaGreen3!15}rules = 3 & \cellcolor{SeaGreen3!15}86.67 & \cellcolor{SeaGreen3!15}72.56 & \cellcolor{SeaGreen3!15}50.00 & \cellcolor{SeaGreen3!15}40.00 & \cellcolor{SeaGreen3!15}27.00 & \cellcolor{SeaGreen3!15}0.00 & \cellcolor{SeaGreen3!15}23.33 & \cellcolor{SeaGreen3!15}3.68 & \cellcolor{SeaGreen3!15}0.00\\
           \hdashline
           \multirow{3}{*}{o1-mini} & rules = 1 & 50.00 & 38.33 & 40.00 & 20.00 & 5.83 & 0.00 & 20.00 & 11.67 & 10.00\\
           & rules = 2 & 55.00 & 38.33 & 30.00 & 35.00 & 17.50 & 0.00 & 15.00 & 8.70 & 0.00\\
           & \cellcolor{SeaGreen3!15}rules = 3 & \cellcolor{SeaGreen3!15}46.67 & \cellcolor{SeaGreen3!15}46.67 & \cellcolor{SeaGreen3!15}10.00 & \cellcolor{SeaGreen3!15}33.33 & \cellcolor{SeaGreen3!15}24.17 & \cellcolor{SeaGreen3!15}0.00 & \cellcolor{SeaGreen3!15}10.00 & \cellcolor{SeaGreen3!15}4.41 & \cellcolor{SeaGreen3!15}0.00\\
           \hdashline
           \multirow{3}{*}{o3-mini} & rules = 1 & 90.00 & 90.00 & 90.00 & 100.00 & 95.00 & 90.00 & 70.00 & 50.00 & 30.00\\
           & rules = 2 & 100.00 & 100.00 & 100.00 & 90.00 & 85.00 & 60.00 & 45.00 & 41.67 & 20.00\\
           & \cellcolor{SeaGreen3!15}rules = 3 & \cellcolor{SeaGreen3!15}\textbf{96.67} & \cellcolor{SeaGreen3!15}\textbf{92.67} & \cellcolor{SeaGreen3!15}\textbf{80.00} & \cellcolor{SeaGreen3!15}\textbf{86.67} & \cellcolor{SeaGreen3!15}\textbf{78.33} & \cellcolor{SeaGreen3!15}\textbf{50.00} & \cellcolor{SeaGreen3!15}\textbf{46.67} & \cellcolor{SeaGreen3!15}\textbf{46.67} & \cellcolor{SeaGreen3!15}\textbf{30.00}\\
         \midrule
           \multirow{3}{*}{Llama-3.3 70B} & rules = 1 & 40.00 & 40.00 & 40.00 & 30.00 & 25.00 & 30.00 & 30.00 & 7.26 & 0.00\\
           & rules = 2 & 30.00 & 60.00 & 10.00 & 20.00 & 10.93 & 0.00 & 25.00 & 19.17 & 10.00\\
           & \cellcolor{SeaGreen3!30}rules = 3 & \cellcolor{SeaGreen3!30}30.00 & \cellcolor{SeaGreen3!30}45.00 & \cellcolor{SeaGreen3!30}0.00 & \cellcolor{SeaGreen3!30}26.67 & \cellcolor{SeaGreen3!30}22.67 & \cellcolor{SeaGreen3!30}10.00 & \cellcolor{SeaGreen3!30}10.00 & \cellcolor{SeaGreen3!30}4.58 & \cellcolor{SeaGreen3!30}0.00\\
           \hdashline
           \multirow{3}{*}{Llama-3.1 405B} & rules = 1 & 20.00 & 11.43 & 10.00 & 10.00 & 5.00 & 0.00 & 20.00 & 4.17 & 0.00\\
           & rules = 2 & 30.00 & 20.83 & 10.00 & 15.00 & 4.17 & 0.00 & 10.00 & 3.41 & 0.00\\
           & \cellcolor{SeaGreen3!30}rules = 3 & \cellcolor{SeaGreen3!30}16.67 & \cellcolor{SeaGreen3!30}9.11 & \cellcolor{SeaGreen3!30}0.00 & \cellcolor{SeaGreen3!30}10.00 & \cellcolor{SeaGreen3!30}4.75 & \cellcolor{SeaGreen3!30}0.00 & \cellcolor{SeaGreen3!30}0.00 & \cellcolor{SeaGreen3!30}0.00 & \cellcolor{SeaGreen3!30}0.00\\
            \hdashline
           \multirow{3}{*}{GPT-4o} & rules = 1 & 60.00 & 50.00 & 40.00 & 50.00 & 27.50 & 20.00 & 50.00 & 15.33 & 20.00\\
           & rules = 2 & 60.00 & 41.67 & 30.00 & 50.00 & 27.50 & 20.00 & 30.00 & 8.82 & 0.00\\
           & \cellcolor{SeaGreen3!30}rules = 3 & \cellcolor{SeaGreen3!30}66.67 & \cellcolor{SeaGreen3!30}57.83 & \cellcolor{SeaGreen3!30}30.00 & \cellcolor{SeaGreen3!30}40.00 & \cellcolor{SeaGreen3!30}21.88 & \cellcolor{SeaGreen3!30}0.00 & \cellcolor{SeaGreen3!30}13.33 & \cellcolor{SeaGreen3!30}6.33 & \cellcolor{SeaGreen3!30}0.00\\
           \hdashline
           \multirow{3}{*}{DeepSeek-V3} & rules = 1 & 80.00 & 60.83 & 60.00 & 50.00 & 29.17 & 40.00 & 40.00 & 9.42 & 20.00\\
           & rules = 2 & 80.00 & 63.19 & 50.00 & 55.00 & 26.67 & 20.00 & 40.00 & 10.10 & 0.00\\
           & \cellcolor{SeaGreen3!30}rules = 3 & \cellcolor{SeaGreen3!30}70.00 & \cellcolor{SeaGreen3!30}42.95 & \cellcolor{SeaGreen3!30}50.00 & \cellcolor{SeaGreen3!30}23.33 & \cellcolor{SeaGreen3!30}15.28 & \cellcolor{SeaGreen3!30}0.00 & \cellcolor{SeaGreen3!30}20.00 & \cellcolor{SeaGreen3!30}3.56 & \cellcolor{SeaGreen3!30}0.00\\
           \hdashline
           \multirow{3}{*}{o1-mini} & rules = 1 & 60.00 & 60.00 & 60.00 & 20.00 & 13.33 & 10.00 & 20.00 & 13.33 & 10.00\\
           & rules = 2 & 15.00 & 15.00 & 10.00 & 40.00 & 25.00 & 0.00 & 30.00 & 19.50 & 0.00\\
           & \cellcolor{SeaGreen3!30}rules = 3 & \cellcolor{SeaGreen3!30}53.33 & \cellcolor{SeaGreen3!30}45.83 & \cellcolor{SeaGreen3!30}20.00 & \cellcolor{SeaGreen3!30}30.00 & \cellcolor{SeaGreen3!30}17.63 & \cellcolor{SeaGreen3!30}0.00 & \cellcolor{SeaGreen3!30}13.33 & \cellcolor{SeaGreen3!30}8.43 & \cellcolor{SeaGreen3!30}0.00\\
           \hdashline
           \multirow{3}{*}{o3-mini} & rules = 1 & 90.00 & 90.00 & 90.00 & 100.00 & 95.00 & 100.00 & 50.00 & 40.00 & 30.00\\
           & rules = 2 & 90.00 & 83.33 & 70.00 & 80.00 & 61.67 & 30.00 & 60.00 & 59.50 & 40.00\\
           & \cellcolor{SeaGreen3!30}rules = 3 & \cellcolor{SeaGreen3!30}\textbf{100.00} & \cellcolor{SeaGreen3!30}\textbf{100.00} & \cellcolor{SeaGreen3!30}\textbf{100.00} & \cellcolor{SeaGreen3!30}\textbf{83.33} & \cellcolor{SeaGreen3!30}\textbf{64.11} & \cellcolor{SeaGreen3!30}\textbf{50.00} & \cellcolor{SeaGreen3!30}\textbf{43.33} & \cellcolor{SeaGreen3!30}\textbf{41.83} & \cellcolor{SeaGreen3!30}\textbf{20.00}\\
           \midrule 
           \multicolumn{11}{c}{\textbf{vocab size = 4}}\\
         \midrule
           \multirow{3}{*}{Llama-3.3 70B} & rules = 1 & 40.00 & 25.00 & 30.00 & 40.00 & 23.33 & 20.00 & 10.00 & 10.00 & 10.00\\
           & rules = 2 & 45.00 & 47.33 & 10.00 & 50.00 & 45.83 & 10.00 & 5.00 & 1.67 & 0.00\\
           & \cellcolor{SeaGreen3!50}rules = 3 & \cellcolor{SeaGreen3!50}33.33 & \cellcolor{SeaGreen3!50}39.50 & \cellcolor{SeaGreen3!50}10.00 & \cellcolor{SeaGreen3!50}30.00 & \cellcolor{SeaGreen3!50}23.85 & \cellcolor{SeaGreen3!50}0.00 & \cellcolor{SeaGreen3!50}10.00 & \cellcolor{SeaGreen3!50}10.83 & \cellcolor{SeaGreen3!50}0.00\\
           \hdashline
           \multirow{3}{*}{Llama-3.1 405B} & rules = 1 & 10.00 & 10.00 & 10.00 & 0.00 & 0.00 & 0.00 & 0.00 & 0.00 & 0.00\\
           & rules = 2 & 15.00 & 12.00 & 0.00 & 10.00 & 13.33 & 0.00 & 5.00 & 0.26 & 0.00\\
           & \cellcolor{SeaGreen3!50}rules = 3 & \cellcolor{SeaGreen3!50}33.33 & \cellcolor{SeaGreen3!50}22.67 & \cellcolor{SeaGreen3!50}0.00 & \cellcolor{SeaGreen3!50}3.33 & \cellcolor{SeaGreen3!50}16.67 & \cellcolor{SeaGreen3!50}0.00 & \cellcolor{SeaGreen3!50}13.33 & \cellcolor{SeaGreen3!50}1.85 & \cellcolor{SeaGreen3!50}0.00\\
           \hdashline
           \multirow{3}{*}{GPT-4o} & rules = 1 & 70.00 & 49.17 & 50.00 & 60.00 & 19.50 & 50.00 & 50.00 & 23.10 & 10.00\\
           & rules = 2 & 80.00 & 78.10 & 40.00 & 40.00 & 16.02 & 0.00 & 20.00 & 10.00 & 0.00\\
           & \cellcolor{SeaGreen3!50}rules = 3 & \cellcolor{SeaGreen3!50}66.67 & \cellcolor{SeaGreen3!50}43.00 & \cellcolor{SeaGreen3!50}0.00 & \cellcolor{SeaGreen3!50}20.00 & \cellcolor{SeaGreen3!50}11.02 & \cellcolor{SeaGreen3!50}0.00 & \cellcolor{SeaGreen3!50}16.67 & \cellcolor{SeaGreen3!50}6.73 & \cellcolor{SeaGreen3!50}0.00\\
           \hdashline
           \multirow{3}{*}{DeepSeek-V3} & rules = 1 & 80.00 & 65.00 & 70.00 & 50.00 & 18.83 & 20.00 & 60.00 & 19.77 & 40.00\\
           & rules = 2 & 70.00 & 59.17 & 30.00 & 45.00 & 27.79 & 20.00 & 10.00 & 0.88 & 0.00\\
           & \cellcolor{SeaGreen3!50}rules = 3 & \cellcolor{SeaGreen3!50}73.33 & \cellcolor{SeaGreen3!50}60.17 & \cellcolor{SeaGreen3!50}40.00 & \cellcolor{SeaGreen3!50}43.33 & \cellcolor{SeaGreen3!50}13.45 & \cellcolor{SeaGreen3!50}0.00 & \cellcolor{SeaGreen3!50}3.33 & \cellcolor{SeaGreen3!50}0.25 & \cellcolor{SeaGreen3!50}0.00\\
           \hdashline
           \multirow{3}{*}{o1-mini} & rules = 1 & 70.00 & 53.33 & 40.00 & 30.00 & 18.33 & 10.00 & 40.00 & 40.00 & 40.00\\
           & rules = 2 & 70.00 & 66.67 & 50.00 & 50.00 & 47.50 & 20.00 & 40.00 & 34.00 & 0.00\\
           & \cellcolor{SeaGreen3!50}rules = 3 & \cellcolor{SeaGreen3!50}90.00 & \cellcolor{SeaGreen3!50}79.17 & \cellcolor{SeaGreen3!50}50.00 & \cellcolor{SeaGreen3!50}23.33 & \cellcolor{SeaGreen3!50}23.33 & \cellcolor{SeaGreen3!50}0.00 & \cellcolor{SeaGreen3!50}26.67 & \cellcolor{SeaGreen3!50}17.58 & \cellcolor{SeaGreen3!50}0.00\\
           \hdashline
           \multirow{3}{*}{o3-mini} & rules = 1 & 90.00 & 90.00 & 90.00 & 100.00 & 93.33 & 90.00 & 50.00 & 50.00 & 50.00\\
           & rules = 2 & 100.00 & 100.00 & 100.00 & 80.00 & 75.00 & 60.00 & 70.00 & 69.17 & 40.00\\
           & \cellcolor{SeaGreen3!50}rules = 3 & \cellcolor{SeaGreen3!50}\textbf{100.00} & \cellcolor{SeaGreen3!50}\textbf{100.00} & \cellcolor{SeaGreen3!50}\textbf{100.00} & \cellcolor{SeaGreen3!50}\textbf{76.67} & \cellcolor{SeaGreen3!50}\textbf{78.33} & \cellcolor{SeaGreen3!50}\textbf{50.00} & \cellcolor{SeaGreen3!50}\textbf{63.33} & \cellcolor{SeaGreen3!50}\textbf{62.00} & \cellcolor{SeaGreen3!50}\textbf{30.00}\\
           %\hdashline
           %o1-preview & \cellcolor{SeaGreen3!50}rules = 3 & -- & -- & -- & -- & -- & -- & 13.20 & 15.00 & 0.00 \\
           \bottomrule
    \end{tabular}
    }
    \caption{Right Output Strictly Local with sample size = 2}
    \label{tab:ROSL_main}
\end{table*}

\begin{table*}[t]
    \centering
    \renewcommand{\arraystretch}{1.1}
    \resizebox{14.5cm}{!}{
    \begin{tabular}{l c ccc ccc ccc}
        \toprule
          \multirow{2}{*}{\bf Models}& \multirow{2}{*}{\bf Settings}& \multicolumn{3}{c}{\bf k = 2} & \multicolumn{3}{c}{\bf k = 3} & \multicolumn{3}{c}{\bf k = 4}\\
          \cmidrule(lr){3-5} \cmidrule(lr){6-8}  \cmidrule(lr){9-11}
          & & recall  & precision & compatibility & recall  & precision & compatibility & recall  & precision & compatibility \\
         \midrule \multicolumn{11}{c}{\textbf{vocab size = 2}} \\
         \midrule
           \multirow{4}{*}{rules = 1} & 0-shot & 60.00 & 55.00 & 60.00 & 30.00 & 23.33 & 20.00 & 10.00 & 10.00 & 10.00  \\
           & 1-shot & 60.00 & 60.00 & 60.00 & 50.00 & 35.00 & 40.00 & 10.00 & 5.00 & 0.00\\
           & 2-shot & 70.00 & 70.00 & 70.00 & 70.00 & 50.00 & 60.00 & 20.00 & 10.00 & 10.00\\
           & 3-shot & 80.00 & 80.00 & 80.00 & 60.00 & 55.00 & 60.00 & 10.00 & 5.00 & 10.00\\
           \hdashline
           \multirow{4}{*}{rules = 2} &  0-shot & 60.00 & 65.00 & 50.00 & 45.00 & 60.00 & 30.00 & 15.00 & 8.25 & 0.00\\
           & 1-shot & 65.00 & 70.00 & 60.00 & 40.00 & 53.00 & 30.00 & 20.00 & 18.33 & 10.00\\
           & 2-shot & 85.00 & 85.00 & 80.00 & 45.00 & 56.67 & 20.00 & 25.00 & 23.33 & 0.00\\
           & 3-shot & 60.00 & 65.00 & 40.00 & 35.00 & 36.67 & 10.00 & 20.00 & 20.00 & 0.00\\
           \hdashline
          \multirow{4}{*}{rules = 3} & 0-shot & 53.33 & 68.33 & 20.00 & 30.00 & 46.67 & 10.00 & 16.67 & 8.54 & 0.00\\
          & 1-shot & 76.67 & 83.33 & 60.00 & 43.33 & 57.67 & 0.00 & 20.00 & 18.33 & 0.00\\
           & 2-shot & 86.67 & 86.67 & 60.00 & 26.67 & 28.33 & 0.00 & 13.33 & 16.67 & 0.00\\
           & 3-shot & 90.00 & 93.33 & 70.00 & 46.67 & 52.50 & 20.00 & 16.67 & 21.17 & 0.00\\
           \midrule \multicolumn{11}{c}{\textbf{vocab size = 3}} \\
           \midrule
           \multirow{4}{*}{rules = 1} & 0-shot & 70.00 & 60.00 & 60.00 & 20.00 & 20.00 & 20.00 & 20.00 & 8.33 & 10.00\\
           & 1-shot & 90.00 & 90.00 & 90.00 & 50.00 & 50.00 & 50.00 & 30.00 & 30.00 & 30.00\\
           & 2-shot & 70.00  & 70.00  & 70.00 & 30.00 & 20.00 & 20.00 & 10.00 & 10.00 & 10.00\\
           & 3-shot & 40.00 & 40.00 & 40.00 & 40.00 & 35.00 & 40.00 & 30.00 & 18.33 & 20.00\\
           \hdashline
           \multirow{4}{*}{rules = 2} &  0-shot & 85.00 & 83.33 & 60.00 & 10.00 & 7.50 & 0.00 & 5.00 & 2.50 & 0.00\\
           & 1-shot & 90.00 & 95.00 & 80.00 & 10.00 & 13.33 & 0.00 & 5.00 & 2.50 & 0.00\\
           & 2-shot & 65.00 & 65.00 & 40.00 & 30.00 & 28.33 & 10.00 & 5.00 & 2.00 & 0.00\\
           & 3-shot & 65.00 & 75.00 & 30.00 & 5.00 & 3.33 & 0.00 & 5.00 & 2.50 & 0.00\\
           \hdashline
          \multirow{4}{*}{rules = 3} & 0-shot & 66.67 & 74.17 & 20.00 & 33.33 & 35.36 & 0.00 & 6.67 & 3.43 & 0.00\\
          & 1-shot & 70.00 & 69.17 & 40.00 & 20.00 & 22.50 & 0.00 & 10.00 & 13.33 & 0.00\\
           & 2-shot & 76.67 & 76.67 & 50.00 & 33.33 & 43.33 & 0.00 & 10.00 & 13.33 & 0.00\\
           & 3-shot & 60.00 & 60.83 & 10.00 & 23.33 & 31.67 & 0.00 & 16.67 & 19.76 & 0.00\\
           \midrule \multicolumn{11}{c}{\textbf{vocab size = 4}} \\
           \midrule
           \multirow{4}{*}{rules = 1} & 0-shot & 60.00 & 60.00 & 60.00 & 30.00 & 30.00 & 30.00 & 10.00 & 10.00 & 10.00\\
           & 1-shot & 40.00 & 40.00 & 40.00 & 50.00 & 31.67 & 50.00 & 20.00 & 13.33 & 20.00\\
           & 2-shot & 70.00 & 57.00 & 60.00 & 30.00 & 25.00 & 30.00 & 10.00 & 5.00 & 0.00\\
           & 3-shot & 60.00 & 60.00 & 60.00 & 20.00 & 15.00 & 20.00 & 10.00 & 10.00 & 10.00\\
           \hdashline
           \multirow{4}{*}{rules = 2} &  0-shot & 40.00 & 40.00 & 30.00 & 15.00 & 11.67 & 0.00 & 0.00 & 0.00 & 0.00\\
           & 1-shot & 60.00 & 60.00 & 30.00 & 15.00 & 23.33 & 0.00 & 0.00 & 0.00 & 0.00\\
           & 2-shot & 80.00 & 90.00 & 60.00 & 15.00 & 12.50 & 0.00 & 15.00 & 10.67 & 0.00\\
           & 3-shot & 70.00 & 70.00 & 70.00 & 15.00 & 18.33 & 0.00 & 10.00 & 10.00 & 0.00\\
           \hdashline
          \multirow{4}{*}{rules = 3} & 0-shot & 53.33 & 68.33 & 20.00 & 6.67 & 5.00 & 0.00 & 10.00 & 5.32 & 0.00\\
          & 1-shot & 66.67 & 70.83 & 30.00 & 30.00 & 47.50 & 0.00 & 3.00 & 5.00 & 0.00\\
           & 2-shot & 66.67 & 73.33 & 30.00 & 30.00 & 55.00 & 0.00 & 3.33 & 3.33 & 0.00\\
           & 3-shot & 60.00 & 71.67 & 10.00 & 20.00 & 39.24 & 0.00 & 0.00 & 0.00 & 0.00\\
           \bottomrule
    \end{tabular}
    }
    \caption{Input Strictly Local with sample size = 2 with few-shot example}
    \label{tab:few_shot_ISL}
\end{table*}


\begin{table*}[t]
    \centering
    \renewcommand{\arraystretch}{1.1}
    \resizebox{14.5cm}{!}{
    \begin{tabular}{l c ccc ccc ccc}
        \toprule
          \multirow{2}{*}{\bf Models}& \multirow{2}{*}{\bf Settings}& \multicolumn{3}{c}{\bf k = 2} & \multicolumn{3}{c}{\bf k = 3} & \multicolumn{3}{c}{\bf k = 4}\\
          \cmidrule(lr){3-5} \cmidrule(lr){6-8}  \cmidrule(lr){9-11}
          & & recall  & precision & compatibility & recall  & precision & compatibility & recall  & precision & compatibility \\
         \midrule \multicolumn{11}{c}{\textbf{vocab size = 2}} \\
         \midrule
           \multirow{4}{*}{rules = 1} & 0-shot & 50.00 & 45.00 & 50.00 & 0.00 & 0.00 & 0.00 & 0.00 & 0.00 & 0.00\\
           & 1-shot & 80.00 & 80.00 & 80.00 & 40.00 & 33.33 & 30.00 & 20.00 & 5.00 & 20.00\\
           & 2-shot & 80.00 & 75.00 & 70.00 & 30.00 & 25.00 & 30.00 & 30.00 & 13.33 & 10.00\\
           & 3-shot & 80.00 & 80.00 & 80.00 & 20.00 & 15.00 & 20.00 & 20.00 & 15.00 & 20.00\\
           \hdashline
           \multirow{4}{*}{rules = 2} &  0-shot & 25.00& 25.00 & 25.00 & 10.00 & 8.33 & 10.00 & 5.00 & 10.00 & 0.00\\
           & 1-shot & 80.00 & 85.00 & 70.00 & 30.00 & 30.83 & 10.00 & 30.00 & 19.00 & 10.00\\
           & 2-shot & 85.00 & 85.00 & 80.00 & 20.00 & 21.67 & 10.00 & 25.00 & 27.90 & 0.00\\
           & 3-shot & 75.00 & 80.00 & 60.00 & 25.00 & 20.83 & 10.00 & 20.00 & 20.83 & 0.00\\
           \hdashline
          \multirow{4}{*}{rules = 3} & 0-shot & 56.67 & 65.00 & 0.00 & 6.67 & 8.33 & 0.00 & 13.33 & 12.83 & 0.00\\
          & 1-shot & 80.00 & 80.00 & 80.00 & 40.00 & 42.00 & 0.00 & 10.00 & 11.67 & 0.00\\
           & 2-shot & 80.00 & 75.00 & 70.00 & 33.33 & 48.33 & 0.00 & 13.33 & 28.33 & 0.00\\
           & 3-shot & 80.00 & 80.00 & 80.00 & 33.33 & 39.17 & 0.00 & 16.67 & 26.67 & 0.00\\
           \midrule \multicolumn{11}{c}{\textbf{vocab size = 3}} \\
           \midrule
           \multirow{4}{*}{rules = 1} & 0-shot & 50.00 & 50.00 & 50.00 & 20.00 & 12.50 & 10.00 & 20.00 & 13.33 & 10.00\\
           & 1-shot & 100.00 & 100.00 & 100.00 & 40.00 & 23.33 & 40.00 & 0.00 & 0.00 & 0.00\\
           & 2-shot & 70.00 & 70.00 & 70.00 & 30.00 & 23.33 & 20.00 & 10.00 & 5.00 & 0.00\\
           & 3-shot & 80.00 & 75.00 & 80.00 & 20.00 & 20.00 & 20.00 & 20.00 & 20.00 & 20.00\\
           \hdashline
           \multirow{4}{*}{rules = 2} &  0-shot & 35.00 & 33.67 & 10.00 & 20.00 & 6.93 & 10.00 & 25.00 & 15.00 & 0.00\\
           & 1-shot & 80.00 & 76.67 & 80.00 & 30.00 & 30.33 & 10.00 & 10.00 & 15.00 & 0.00\\
           & 2-shot & 50.00 & 55.00 & 30.00 & 30.00 & 38.33 & 0.00 & 25.00 & 21.67 & 0.00\\
           & 3-shot & 70.00 & 70.00 & 70.00 & 20.00 & 30.00 & 0.00 & 20.00 & 35.00 & 0.00\\
           \hdashline
          \multirow{4}{*}{rules = 3} & 0-shot & 40.00 & 65.00 & 20.00 & 20.00 & 18.33 & 0.00 & 10.00 & 2.78 & 0.00\\
          & 1-shot & 70.00 & 66.67 & 40.00 & 30.00 & 24.83 & 0.00 & 10.00 & 20.30 & 10.00\\
           & 2-shot & 83.33 & 90.00 & 60.00 & 33.33 & 40.83 & 0.00 & 30.00 & 43.33 & 0.00\\
           & 3-shot & 70.00 & 78.33 & 30.00 & 23.33 & 44.50 & 0.00 & 16.67 & 18.33 & 0.00\\
           \midrule \multicolumn{11}{c}{\textbf{vocab size = 4}} \\
           \midrule
           \multirow{4}{*}{rules = 1} & 0-shot & 50.00 & 29.00 & 30.00 & 20.00 & 13.33 & 10.00 & 10.00 & 10.00 & 10.00\\
           & 1-shot & 50.00 & 50.00 & 50.00 & 50.00 & 50.00 & 50.00 & 20.00 & 15.00 & 20.00\\
           & 2-shot & 60.00 & 60.00 & 60.00 & 10.00 & 10.00 & 10.00 & 20.00 & 20.00 & 20.00\\
           & 3-shot & 20.00 & 20.00 & 20.00 & 30.00 & 25.00 & 30.00 & 20.00 & 20.00 & 20.00\\
           \hdashline
           \multirow{4}{*}{rules = 2} &  0-shot & 50.00 & 50.00 & 10.00 & 20.00 & 15.96 & 0.00 & 0.00 & 0.00 & 0.00\\
           & 1-shot & 55.00 & 56.67 & 30.00 & 20.00 & 28.33 & 0.00 & 0.00 & 0.00 & 0.00\\
           & 2-shot & 55.00 & 50.00 & 20.00 & 35.00 & 55.00 & 10.00 & 5.00 & 10.00 & 0.00\\
           & 3-shot & 10.00 & 20.00 & 0.00 & 30.00 & 33.33 & 10.00 & 10.00 & 20.00 & 0.00\\
           \hdashline
          \multirow{4}{*}{rules = 3} & 0-shot & 50.00 & 52.50 & 20.00 & 6.67 & 6.00 & 0.00 & 10.00 & 6.33 & 0.00\\
          & 1-shot & 56.67 & 68.33 & 20.00 & 20.00 & 36.25 & 0.00 & 16.7 & 22.83 & 0.00\\
           & 2-shot & 66.67 & 67.50 & 50.00 & 16.67 & 26.67 & 0.00 & 6.67 & 15.00 & 0.00\\
           & 3-shot & 3.33 & 3.33 & 0.00 & 23.33 & 40.83 & 0.00 & 3.33 & 3.33 & 0.00\\
           \bottomrule
    \end{tabular}
    }
    \caption{Left Output Strictly Local with sample size = 2 with few-shot example}
    \label{tab:few_shot_LOSL}
\end{table*}


\begin{table*}[t]
    \centering
    \renewcommand{\arraystretch}{1.1}
    \resizebox{14.5cm}{!}{
    \begin{tabular}{l c ccc ccc ccc}
        \toprule
          \multirow{2}{*}{\bf Models}& \multirow{2}{*}{\bf Settings}& \multicolumn{3}{c}{\bf k = 2} & \multicolumn{3}{c}{\bf k = 3} & \multicolumn{3}{c}{\bf k = 4}\\
          \cmidrule(lr){3-5} \cmidrule(lr){6-8}  \cmidrule(lr){9-11}
          & & recall  & precision & compatibility & recall  & precision & compatibility & recall  & precision & compatibility \\
         \midrule \multicolumn{11}{c}{\textbf{vocab size = 2}} \\
         \midrule
           \multirow{4}{*}{rules = 1} & 0-shot & 40.00 & 40.00 & 40.00 & 20.00 & 10.00 & 10.00 & 20.00 & 10.00 & 10.00\\
           & 1-shot & 60.00 & 60.00 & 60.00 & 50.00 & 40.00 & 50.00 & 30.00 & 18.33 & 10.00\\
           & 2-shot & 60.00 & 60.00 & 60.00 & 40.00 & 35.00 & 40.00 & 30.00 & 25.00 & 20.00\\
           & 3-shot & 60.00 & 60.00 & 60.00 & 50.00 & 40.00 & 50.00 & 10.00 & 2.50 & 10.00\\
           \hdashline
           \multirow{4}{*}{rules = 2} &  0-shot & 40.00 & 40.00 & 30.00 & 15.00 & 18.33 & 10.00 & 20.00 & 18.67 & 10.00\\
           & 1-shot & 60.00 & 60.00 & 50.00 & 40.00 & 42.50 & 20.00 & 30.00 & 33.33 & 0.00\\
           & 2-shot & 70.00 & 80.00 & 60.00 & 45.00 & 43.33 & 30.00 & 20.00 & 27.50 & 0.00\\
           & 3-shot & 90.00 & 90.00 & 90.00 & 45.00 & 41.67 & 30.00 & 15.00 & 15.00 & 0.00\\
           \hdashline
          \multirow{4}{*}{rules = 3} & 0-shot & 63.33 & 76.67 & 30.00 & 23.33 & 24.17 & 0.00 & 10.00 & 7.83 & 0.00\\
          & 1-shot & 60.00 & 60.00 & 60.00 & 40.00 & 47.83 & 0.00 & 20.00 & 24.17 & 0.00\\
           & 2-shot & 60.00 & 60.00 & 60.00 & 50.00 & 48.33 & 20.00 & 16.67 & 18.33 & 0.00\\
           & 3-shot & 60.00 & 60.00 & 60.00 & 43.33 & 46.67 & 10.00 & 16.67 & 25.00 & 0.00\\
           \midrule \multicolumn{11}{c}{\textbf{vocab size = 3}} \\
           \midrule
           \multirow{4}{*}{rules = 1} & 0-shot & 40.00 & 40.00 & 40.00 & 30.00 & 25.00 & 30.00 & 30.00 & 7.26 & 0.00\\
           & 1-shot & 70.00 & 65.00 & 60.00 & 40.00 & 35.00 & 40.00 & 20.00 & 15.00 & 10.00\\
           & 2-shot & 80.00 & 75.00 & 70.00 & 50.00 & 50.00 & 50.00 & 30.00 & 18.33 & 30.00\\
           & 3-shot & 70.00 & 65.00 & 60.00 & 40.00 & 35.00 & 40.00 & 20.00 & 15.00 & 10.00\\
           \hdashline
           \multirow{4}{*}{rules = 2} &  0-shot & 30.00 & 60.00 & 10.00 & 20.00 & 10.93 & 0.00 & 25.00 & 19.17 & 10.00\\
           & 1-shot & 65.00 & 70.00 & 40.00 & 30.00 & 45.00 & 10.00 & 15.00 & 11.67 & 0.00\\
           & 2-shot & 60.00 & 70.00 & 40.00 & 50.00 & 44.17 & 10.00 & 20.00 & 20.00 & 0.00\\
           & 3-shot & 65.00 & 70.00 & 40.00 & 30.00 & 45.00 & 10.00 & 15.00 & 11.67 & 0.00\\
           \hdashline
          \multirow{4}{*}{rules = 3} & 0-shot & 30.00 & 45.00 & 0.00 & 26.67 & 22.67 & 10.00 & 10.00 & 4.58 & 0.00\\
          & 1-shot & 80.00 & 80.83 & 40.00 & 30.00 & 37.59 & 0.00 & 13.33 & 11.00 & 0.00\\
           & 2-shot & 73.33 & 71.67 & 30.00 & 26.67 & 34.50 & 0.00 & 33.33 & 47.00 & 0.00\\
           & 3-shot & 80.00 & 80.83 & 40.00 & 30.00 & 37.60 & 0.00 & 13.33 & 11.00 & 0.00\\
           \midrule \multicolumn{11}{c}{\textbf{vocab size = 4}} \\
           \midrule
           \multirow{4}{*}{rules = 1} & 0-shot & 40.00 & 25.00 & 30.00 & 40.00 & 23.33 & 20.00 & 10.00 & 10.00 & 10.00\\
           & 1-shot & 60.00 & 60.00 & 60.00 & 50.00 & 31.67 & 50.00 & 10.00 & 10.00 & 10.00\\
           & 2-shot & 80.00 & 68.33 & 80.00 & 60.00 & 33.33 & 30.00 & 20.00 & 20.00 & 20.00\\
           & 3-shot & 70.00 & 70.00 & 70.00 & 40.00 & 19.50 & 40.00 & 20.00 & 20.00 & 20.00\\
           \hdashline
           \multirow{4}{*}{rules = 2} &  0-shot & 45.00 & 47.33 & 10.00 & 50.00 & 45.83 & 10.00 & 5.00 & 1.67 & 0.00\\
           & 1-shot & 80.00 & 70.00 & 40.00 & 45.00 & 61.67 & 10.00 & 5.00 & 10.00 & 0.00\\
           & 2-shot & 65.00 & 65.00 & 40.00 & 45.00 & 52.50 & 20.00 & 0.00 & 0.00 & 0.00\\
           & 3-shot & 75.00 & 75.00 & 70.00 & 45.00 & 48.33 & 20.00 & 15.00 & 30.00 & 0.00\\
           \hdashline
          \multirow{4}{*}{rules = 3} & 0-shot & 33.33 & 39.50 & 10.00 & 30.00 & 23.85 & 0.00 & 10.00 & 10.83 & 0.00\\
          & 1-shot & 66.67 & 71.83 & 30.00 & 23.33 & 30.83 & 0.00 & 3.33 & 2.50 & 0.00\\
           & 2-shot & 83.33 & 86.50 & 50.00 & 26.67 & 37.50 & 0.00 & 13.33 & 28.83 & 0.00\\
           & 3-shot & 76.67 & 80.00 & 50.00 & 40.00 & 51.67 & 10.00 & 13.33 & 20.00 & 0.00\\
           \bottomrule
    \end{tabular}
    }
    \caption{Right Output Strictly Local with sample size = 2 with few-shot example}
    \label{tab:few_shot_ROSL}
\end{table*}


\end{document}
