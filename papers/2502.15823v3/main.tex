\pdfoutput=1

\documentclass[table, x11names]{article} % For LaTeX2e
\usepackage{iclr2025_conference,times}

% Optional math commands from https://github.com/goodfeli/dlbook_notation.
%%%%% NEW MATH DEFINITIONS %%%%%

\usepackage{amsmath,amsfonts,bm}
\usepackage{derivative}
% Mark sections of captions for referring to divisions of figures
\newcommand{\figleft}{{\em (Left)}}
\newcommand{\figcenter}{{\em (Center)}}
\newcommand{\figright}{{\em (Right)}}
\newcommand{\figtop}{{\em (Top)}}
\newcommand{\figbottom}{{\em (Bottom)}}
\newcommand{\captiona}{{\em (a)}}
\newcommand{\captionb}{{\em (b)}}
\newcommand{\captionc}{{\em (c)}}
\newcommand{\captiond}{{\em (d)}}

% Highlight a newly defined term
\newcommand{\newterm}[1]{{\bf #1}}

% Derivative d 
\newcommand{\deriv}{{\mathrm{d}}}

% Figure reference, lower-case.
\def\figref#1{figure~\ref{#1}}
% Figure reference, capital. For start of sentence
\def\Figref#1{Figure~\ref{#1}}
\def\twofigref#1#2{figures \ref{#1} and \ref{#2}}
\def\quadfigref#1#2#3#4{figures \ref{#1}, \ref{#2}, \ref{#3} and \ref{#4}}
% Section reference, lower-case.
\def\secref#1{section~\ref{#1}}
% Section reference, capital.
\def\Secref#1{Section~\ref{#1}}
% Reference to two sections.
\def\twosecrefs#1#2{sections \ref{#1} and \ref{#2}}
% Reference to three sections.
\def\secrefs#1#2#3{sections \ref{#1}, \ref{#2} and \ref{#3}}
% Reference to an equation, lower-case.
\def\eqref#1{equation~\ref{#1}}
% Reference to an equation, upper case
\def\Eqref#1{Equation~\ref{#1}}
% A raw reference to an equation---avoid using if possible
\def\plaineqref#1{\ref{#1}}
% Reference to a chapter, lower-case.
\def\chapref#1{chapter~\ref{#1}}
% Reference to an equation, upper case.
\def\Chapref#1{Chapter~\ref{#1}}
% Reference to a range of chapters
\def\rangechapref#1#2{chapters\ref{#1}--\ref{#2}}
% Reference to an algorithm, lower-case.
\def\algref#1{algorithm~\ref{#1}}
% Reference to an algorithm, upper case.
\def\Algref#1{Algorithm~\ref{#1}}
\def\twoalgref#1#2{algorithms \ref{#1} and \ref{#2}}
\def\Twoalgref#1#2{Algorithms \ref{#1} and \ref{#2}}
% Reference to a part, lower case
\def\partref#1{part~\ref{#1}}
% Reference to a part, upper case
\def\Partref#1{Part~\ref{#1}}
\def\twopartref#1#2{parts \ref{#1} and \ref{#2}}

\def\ceil#1{\lceil #1 \rceil}
\def\floor#1{\lfloor #1 \rfloor}
\def\1{\bm{1}}
\newcommand{\train}{\mathcal{D}}
\newcommand{\valid}{\mathcal{D_{\mathrm{valid}}}}
\newcommand{\test}{\mathcal{D_{\mathrm{test}}}}

\def\eps{{\epsilon}}


% Random variables
\def\reta{{\textnormal{$\eta$}}}
\def\ra{{\textnormal{a}}}
\def\rb{{\textnormal{b}}}
\def\rc{{\textnormal{c}}}
\def\rd{{\textnormal{d}}}
\def\re{{\textnormal{e}}}
\def\rf{{\textnormal{f}}}
\def\rg{{\textnormal{g}}}
\def\rh{{\textnormal{h}}}
\def\ri{{\textnormal{i}}}
\def\rj{{\textnormal{j}}}
\def\rk{{\textnormal{k}}}
\def\rl{{\textnormal{l}}}
% rm is already a command, just don't name any random variables m
\def\rn{{\textnormal{n}}}
\def\ro{{\textnormal{o}}}
\def\rp{{\textnormal{p}}}
\def\rq{{\textnormal{q}}}
\def\rr{{\textnormal{r}}}
\def\rs{{\textnormal{s}}}
\def\rt{{\textnormal{t}}}
\def\ru{{\textnormal{u}}}
\def\rv{{\textnormal{v}}}
\def\rw{{\textnormal{w}}}
\def\rx{{\textnormal{x}}}
\def\ry{{\textnormal{y}}}
\def\rz{{\textnormal{z}}}

% Random vectors
\def\rvepsilon{{\mathbf{\epsilon}}}
\def\rvphi{{\mathbf{\phi}}}
\def\rvtheta{{\mathbf{\theta}}}
\def\rva{{\mathbf{a}}}
\def\rvb{{\mathbf{b}}}
\def\rvc{{\mathbf{c}}}
\def\rvd{{\mathbf{d}}}
\def\rve{{\mathbf{e}}}
\def\rvf{{\mathbf{f}}}
\def\rvg{{\mathbf{g}}}
\def\rvh{{\mathbf{h}}}
\def\rvu{{\mathbf{i}}}
\def\rvj{{\mathbf{j}}}
\def\rvk{{\mathbf{k}}}
\def\rvl{{\mathbf{l}}}
\def\rvm{{\mathbf{m}}}
\def\rvn{{\mathbf{n}}}
\def\rvo{{\mathbf{o}}}
\def\rvp{{\mathbf{p}}}
\def\rvq{{\mathbf{q}}}
\def\rvr{{\mathbf{r}}}
\def\rvs{{\mathbf{s}}}
\def\rvt{{\mathbf{t}}}
\def\rvu{{\mathbf{u}}}
\def\rvv{{\mathbf{v}}}
\def\rvw{{\mathbf{w}}}
\def\rvx{{\mathbf{x}}}
\def\rvy{{\mathbf{y}}}
\def\rvz{{\mathbf{z}}}

% Elements of random vectors
\def\erva{{\textnormal{a}}}
\def\ervb{{\textnormal{b}}}
\def\ervc{{\textnormal{c}}}
\def\ervd{{\textnormal{d}}}
\def\erve{{\textnormal{e}}}
\def\ervf{{\textnormal{f}}}
\def\ervg{{\textnormal{g}}}
\def\ervh{{\textnormal{h}}}
\def\ervi{{\textnormal{i}}}
\def\ervj{{\textnormal{j}}}
\def\ervk{{\textnormal{k}}}
\def\ervl{{\textnormal{l}}}
\def\ervm{{\textnormal{m}}}
\def\ervn{{\textnormal{n}}}
\def\ervo{{\textnormal{o}}}
\def\ervp{{\textnormal{p}}}
\def\ervq{{\textnormal{q}}}
\def\ervr{{\textnormal{r}}}
\def\ervs{{\textnormal{s}}}
\def\ervt{{\textnormal{t}}}
\def\ervu{{\textnormal{u}}}
\def\ervv{{\textnormal{v}}}
\def\ervw{{\textnormal{w}}}
\def\ervx{{\textnormal{x}}}
\def\ervy{{\textnormal{y}}}
\def\ervz{{\textnormal{z}}}

% Random matrices
\def\rmA{{\mathbf{A}}}
\def\rmB{{\mathbf{B}}}
\def\rmC{{\mathbf{C}}}
\def\rmD{{\mathbf{D}}}
\def\rmE{{\mathbf{E}}}
\def\rmF{{\mathbf{F}}}
\def\rmG{{\mathbf{G}}}
\def\rmH{{\mathbf{H}}}
\def\rmI{{\mathbf{I}}}
\def\rmJ{{\mathbf{J}}}
\def\rmK{{\mathbf{K}}}
\def\rmL{{\mathbf{L}}}
\def\rmM{{\mathbf{M}}}
\def\rmN{{\mathbf{N}}}
\def\rmO{{\mathbf{O}}}
\def\rmP{{\mathbf{P}}}
\def\rmQ{{\mathbf{Q}}}
\def\rmR{{\mathbf{R}}}
\def\rmS{{\mathbf{S}}}
\def\rmT{{\mathbf{T}}}
\def\rmU{{\mathbf{U}}}
\def\rmV{{\mathbf{V}}}
\def\rmW{{\mathbf{W}}}
\def\rmX{{\mathbf{X}}}
\def\rmY{{\mathbf{Y}}}
\def\rmZ{{\mathbf{Z}}}

% Elements of random matrices
\def\ermA{{\textnormal{A}}}
\def\ermB{{\textnormal{B}}}
\def\ermC{{\textnormal{C}}}
\def\ermD{{\textnormal{D}}}
\def\ermE{{\textnormal{E}}}
\def\ermF{{\textnormal{F}}}
\def\ermG{{\textnormal{G}}}
\def\ermH{{\textnormal{H}}}
\def\ermI{{\textnormal{I}}}
\def\ermJ{{\textnormal{J}}}
\def\ermK{{\textnormal{K}}}
\def\ermL{{\textnormal{L}}}
\def\ermM{{\textnormal{M}}}
\def\ermN{{\textnormal{N}}}
\def\ermO{{\textnormal{O}}}
\def\ermP{{\textnormal{P}}}
\def\ermQ{{\textnormal{Q}}}
\def\ermR{{\textnormal{R}}}
\def\ermS{{\textnormal{S}}}
\def\ermT{{\textnormal{T}}}
\def\ermU{{\textnormal{U}}}
\def\ermV{{\textnormal{V}}}
\def\ermW{{\textnormal{W}}}
\def\ermX{{\textnormal{X}}}
\def\ermY{{\textnormal{Y}}}
\def\ermZ{{\textnormal{Z}}}

% Vectors
\def\vzero{{\bm{0}}}
\def\vone{{\bm{1}}}
\def\vmu{{\bm{\mu}}}
\def\vtheta{{\bm{\theta}}}
\def\vphi{{\bm{\phi}}}
\def\va{{\bm{a}}}
\def\vb{{\bm{b}}}
\def\vc{{\bm{c}}}
\def\vd{{\bm{d}}}
\def\ve{{\bm{e}}}
\def\vf{{\bm{f}}}
\def\vg{{\bm{g}}}
\def\vh{{\bm{h}}}
\def\vi{{\bm{i}}}
\def\vj{{\bm{j}}}
\def\vk{{\bm{k}}}
\def\vl{{\bm{l}}}
\def\vm{{\bm{m}}}
\def\vn{{\bm{n}}}
\def\vo{{\bm{o}}}
\def\vp{{\bm{p}}}
\def\vq{{\bm{q}}}
\def\vr{{\bm{r}}}
\def\vs{{\bm{s}}}
\def\vt{{\bm{t}}}
\def\vu{{\bm{u}}}
\def\vv{{\bm{v}}}
\def\vw{{\bm{w}}}
\def\vx{{\bm{x}}}
\def\vy{{\bm{y}}}
\def\vz{{\bm{z}}}

% Elements of vectors
\def\evalpha{{\alpha}}
\def\evbeta{{\beta}}
\def\evepsilon{{\epsilon}}
\def\evlambda{{\lambda}}
\def\evomega{{\omega}}
\def\evmu{{\mu}}
\def\evpsi{{\psi}}
\def\evsigma{{\sigma}}
\def\evtheta{{\theta}}
\def\eva{{a}}
\def\evb{{b}}
\def\evc{{c}}
\def\evd{{d}}
\def\eve{{e}}
\def\evf{{f}}
\def\evg{{g}}
\def\evh{{h}}
\def\evi{{i}}
\def\evj{{j}}
\def\evk{{k}}
\def\evl{{l}}
\def\evm{{m}}
\def\evn{{n}}
\def\evo{{o}}
\def\evp{{p}}
\def\evq{{q}}
\def\evr{{r}}
\def\evs{{s}}
\def\evt{{t}}
\def\evu{{u}}
\def\evv{{v}}
\def\evw{{w}}
\def\evx{{x}}
\def\evy{{y}}
\def\evz{{z}}

% Matrix
\def\mA{{\bm{A}}}
\def\mB{{\bm{B}}}
\def\mC{{\bm{C}}}
\def\mD{{\bm{D}}}
\def\mE{{\bm{E}}}
\def\mF{{\bm{F}}}
\def\mG{{\bm{G}}}
\def\mH{{\bm{H}}}
\def\mI{{\bm{I}}}
\def\mJ{{\bm{J}}}
\def\mK{{\bm{K}}}
\def\mL{{\bm{L}}}
\def\mM{{\bm{M}}}
\def\mN{{\bm{N}}}
\def\mO{{\bm{O}}}
\def\mP{{\bm{P}}}
\def\mQ{{\bm{Q}}}
\def\mR{{\bm{R}}}
\def\mS{{\bm{S}}}
\def\mT{{\bm{T}}}
\def\mU{{\bm{U}}}
\def\mV{{\bm{V}}}
\def\mW{{\bm{W}}}
\def\mX{{\bm{X}}}
\def\mY{{\bm{Y}}}
\def\mZ{{\bm{Z}}}
\def\mBeta{{\bm{\beta}}}
\def\mPhi{{\bm{\Phi}}}
\def\mLambda{{\bm{\Lambda}}}
\def\mSigma{{\bm{\Sigma}}}

% Tensor
\DeclareMathAlphabet{\mathsfit}{\encodingdefault}{\sfdefault}{m}{sl}
\SetMathAlphabet{\mathsfit}{bold}{\encodingdefault}{\sfdefault}{bx}{n}
\newcommand{\tens}[1]{\bm{\mathsfit{#1}}}
\def\tA{{\tens{A}}}
\def\tB{{\tens{B}}}
\def\tC{{\tens{C}}}
\def\tD{{\tens{D}}}
\def\tE{{\tens{E}}}
\def\tF{{\tens{F}}}
\def\tG{{\tens{G}}}
\def\tH{{\tens{H}}}
\def\tI{{\tens{I}}}
\def\tJ{{\tens{J}}}
\def\tK{{\tens{K}}}
\def\tL{{\tens{L}}}
\def\tM{{\tens{M}}}
\def\tN{{\tens{N}}}
\def\tO{{\tens{O}}}
\def\tP{{\tens{P}}}
\def\tQ{{\tens{Q}}}
\def\tR{{\tens{R}}}
\def\tS{{\tens{S}}}
\def\tT{{\tens{T}}}
\def\tU{{\tens{U}}}
\def\tV{{\tens{V}}}
\def\tW{{\tens{W}}}
\def\tX{{\tens{X}}}
\def\tY{{\tens{Y}}}
\def\tZ{{\tens{Z}}}


% Graph
\def\gA{{\mathcal{A}}}
\def\gB{{\mathcal{B}}}
\def\gC{{\mathcal{C}}}
\def\gD{{\mathcal{D}}}
\def\gE{{\mathcal{E}}}
\def\gF{{\mathcal{F}}}
\def\gG{{\mathcal{G}}}
\def\gH{{\mathcal{H}}}
\def\gI{{\mathcal{I}}}
\def\gJ{{\mathcal{J}}}
\def\gK{{\mathcal{K}}}
\def\gL{{\mathcal{L}}}
\def\gM{{\mathcal{M}}}
\def\gN{{\mathcal{N}}}
\def\gO{{\mathcal{O}}}
\def\gP{{\mathcal{P}}}
\def\gQ{{\mathcal{Q}}}
\def\gR{{\mathcal{R}}}
\def\gS{{\mathcal{S}}}
\def\gT{{\mathcal{T}}}
\def\gU{{\mathcal{U}}}
\def\gV{{\mathcal{V}}}
\def\gW{{\mathcal{W}}}
\def\gX{{\mathcal{X}}}
\def\gY{{\mathcal{Y}}}
\def\gZ{{\mathcal{Z}}}

% Sets
\def\sA{{\mathbb{A}}}
\def\sB{{\mathbb{B}}}
\def\sC{{\mathbb{C}}}
\def\sD{{\mathbb{D}}}
% Don't use a set called E, because this would be the same as our symbol
% for expectation.
\def\sF{{\mathbb{F}}}
\def\sG{{\mathbb{G}}}
\def\sH{{\mathbb{H}}}
\def\sI{{\mathbb{I}}}
\def\sJ{{\mathbb{J}}}
\def\sK{{\mathbb{K}}}
\def\sL{{\mathbb{L}}}
\def\sM{{\mathbb{M}}}
\def\sN{{\mathbb{N}}}
\def\sO{{\mathbb{O}}}
\def\sP{{\mathbb{P}}}
\def\sQ{{\mathbb{Q}}}
\def\sR{{\mathbb{R}}}
\def\sS{{\mathbb{S}}}
\def\sT{{\mathbb{T}}}
\def\sU{{\mathbb{U}}}
\def\sV{{\mathbb{V}}}
\def\sW{{\mathbb{W}}}
\def\sX{{\mathbb{X}}}
\def\sY{{\mathbb{Y}}}
\def\sZ{{\mathbb{Z}}}

% Entries of a matrix
\def\emLambda{{\Lambda}}
\def\emA{{A}}
\def\emB{{B}}
\def\emC{{C}}
\def\emD{{D}}
\def\emE{{E}}
\def\emF{{F}}
\def\emG{{G}}
\def\emH{{H}}
\def\emI{{I}}
\def\emJ{{J}}
\def\emK{{K}}
\def\emL{{L}}
\def\emM{{M}}
\def\emN{{N}}
\def\emO{{O}}
\def\emP{{P}}
\def\emQ{{Q}}
\def\emR{{R}}
\def\emS{{S}}
\def\emT{{T}}
\def\emU{{U}}
\def\emV{{V}}
\def\emW{{W}}
\def\emX{{X}}
\def\emY{{Y}}
\def\emZ{{Z}}
\def\emSigma{{\Sigma}}

% entries of a tensor
% Same font as tensor, without \bm wrapper
\newcommand{\etens}[1]{\mathsfit{#1}}
\def\etLambda{{\etens{\Lambda}}}
\def\etA{{\etens{A}}}
\def\etB{{\etens{B}}}
\def\etC{{\etens{C}}}
\def\etD{{\etens{D}}}
\def\etE{{\etens{E}}}
\def\etF{{\etens{F}}}
\def\etG{{\etens{G}}}
\def\etH{{\etens{H}}}
\def\etI{{\etens{I}}}
\def\etJ{{\etens{J}}}
\def\etK{{\etens{K}}}
\def\etL{{\etens{L}}}
\def\etM{{\etens{M}}}
\def\etN{{\etens{N}}}
\def\etO{{\etens{O}}}
\def\etP{{\etens{P}}}
\def\etQ{{\etens{Q}}}
\def\etR{{\etens{R}}}
\def\etS{{\etens{S}}}
\def\etT{{\etens{T}}}
\def\etU{{\etens{U}}}
\def\etV{{\etens{V}}}
\def\etW{{\etens{W}}}
\def\etX{{\etens{X}}}
\def\etY{{\etens{Y}}}
\def\etZ{{\etens{Z}}}

% The true underlying data generating distribution
\newcommand{\pdata}{p_{\rm{data}}}
\newcommand{\ptarget}{p_{\rm{target}}}
\newcommand{\pprior}{p_{\rm{prior}}}
\newcommand{\pbase}{p_{\rm{base}}}
\newcommand{\pref}{p_{\rm{ref}}}

% The empirical distribution defined by the training set
\newcommand{\ptrain}{\hat{p}_{\rm{data}}}
\newcommand{\Ptrain}{\hat{P}_{\rm{data}}}
% The model distribution
\newcommand{\pmodel}{p_{\rm{model}}}
\newcommand{\Pmodel}{P_{\rm{model}}}
\newcommand{\ptildemodel}{\tilde{p}_{\rm{model}}}
% Stochastic autoencoder distributions
\newcommand{\pencode}{p_{\rm{encoder}}}
\newcommand{\pdecode}{p_{\rm{decoder}}}
\newcommand{\precons}{p_{\rm{reconstruct}}}

\newcommand{\laplace}{\mathrm{Laplace}} % Laplace distribution

\newcommand{\E}{\mathbb{E}}
\newcommand{\Ls}{\mathcal{L}}
\newcommand{\R}{\mathbb{R}}
\newcommand{\emp}{\tilde{p}}
\newcommand{\lr}{\alpha}
\newcommand{\reg}{\lambda}
\newcommand{\rect}{\mathrm{rectifier}}
\newcommand{\softmax}{\mathrm{softmax}}
\newcommand{\sigmoid}{\sigma}
\newcommand{\softplus}{\zeta}
\newcommand{\KL}{D_{\mathrm{KL}}}
\newcommand{\Var}{\mathrm{Var}}
\newcommand{\standarderror}{\mathrm{SE}}
\newcommand{\Cov}{\mathrm{Cov}}
% Wolfram Mathworld says $L^2$ is for function spaces and $\ell^2$ is for vectors
% But then they seem to use $L^2$ for vectors throughout the site, and so does
% wikipedia.
\newcommand{\normlzero}{L^0}
\newcommand{\normlone}{L^1}
\newcommand{\normltwo}{L^2}
\newcommand{\normlp}{L^p}
\newcommand{\normmax}{L^\infty}

\newcommand{\parents}{Pa} % See usage in notation.tex. Chosen to match Daphne's book.

\DeclareMathOperator*{\argmax}{arg\,max}
\DeclareMathOperator*{\argmin}{arg\,min}

\DeclareMathOperator{\sign}{sign}
\DeclareMathOperator{\Tr}{Tr}
\let\ab\allowbreak


\usepackage{hyperref}
\usepackage{url}

% Standard package includes
\usepackage{times}
\usepackage{latexsym}
\usepackage{amsmath}
\usepackage{mathtools}

% For proper rendering and hyphenation of words containing Latin characters (including in bib files)
\usepackage[T1]{fontenc}
% For Vietnamese characters
% \usepackage[T5]{fontenc}
% See https://www.latex-project.org/help/documentation/encguide.pdf for other character sets

% This assumes your files are encoded as UTF8
\usepackage[utf8]{inputenc}

% This is not strictly necessary, and may be commented out,
% but it will improve the layout of the manuscript,
% and will typically save some space.
\usepackage{xcolor}
\usepackage{microtype}
\newtheorem{definition}{Definition}
\newtheorem{exmp}{Example}[section]
\usepackage{latexsym}
\usepackage{multicol, multirow}
\usepackage{booktabs}
\usepackage{arydshln}
% If the title and author information does not fit in the area allocated, uncomment the following
%
%\setlength\titlebox{<dim>}
%
% and set <dim> to something 5cm or larger.

\usepackage{hyperref}       % hyperlinks
\usepackage{url}            % simple URL typesetting
\usepackage{booktabs}       % professional-quality tables
\usepackage{amsfonts}       % blackboard math symbols
\usepackage{nicefrac}       % compact symbols for 1/2, etc.
\usepackage{microtype}      % microtypography
\usepackage{lipsum}
\usepackage{graphicx}
\usepackage{enumitem}
\usepackage{color}
\usepackage{verbatimbox}
\usepackage{listings}

\usepackage{tikz}
\usetikzlibrary{intersections}
\usetikzlibrary{positioning}
\usepackage{wrapfig}


%New colors defined below
\definecolor{codegreen}{rgb}{0,0.6,0}
\definecolor{codegray}{rgb}{0.5,0.5,0.5}
\definecolor{codepurple}{rgb}{0.58,0,0.82}
\definecolor{backcolour}{rgb}{0.96,0.96,0.96}

\setlist[itemize]{leftmargin=*}
\setlist[enumerate]{leftmargin=*}
\graphicspath{ {./images/} }

%Code listing style named "mystyle"
\lstdefinestyle{mystyle}{
  backgroundcolor=\color{backcolour}, commentstyle=\color{codegreen},
  keywordstyle=\color{magenta},
  numberstyle=\tiny\color{codegray},
  stringstyle=\color{codepurple},
  basicstyle=\ttfamily\footnotesize,
  breakatwhitespace=false,         
  breaklines=true,                 
  captionpos=b,                    
  keepspaces=true,                 
  numbers=left,                    
  numbersep=5pt,                  
  showspaces=false,                
  showstringspaces=false,
  showtabs=false,                  
  tabsize=2
}

%"mystyle" code listing set
\lstset{style=mystyle}

\usepackage[strict]{changepage}
\usepackage{framed}
\definecolor{demonstrationshade}{rgb}{0.95,0.95,1}
\definecolor{promptshade}{rgb}{0.95,0.95,1}
\newenvironment{demonstration}{%
  \def\FrameCommand{%
    \hspace{1pt}%
    {\color{black}\vrule width 2pt}%
    {\color{demonstrationshade}\vrule width 4pt}%
    \colorbox{demonstrationshade}%
  }%
  \MakeFramed{\advance\hsize-\width\FrameRestore}%
  \noindent\hspace{-4.55pt}% disable indenting first paragraph
  \begin{adjustwidth}{1pt}{7pt}%
  \vspace{2pt}\vspace{2pt}%
}
{%
  \vspace{2pt}\end{adjustwidth}\endMakeFramed%
}

\definecolor{color-obs-seq}{RGB}{219, 232, 213}
\definecolor{color-act-seq}{RGB}{172, 204, 255}
\definecolor{color-opti}{RGB}{248,206,204} 
%\definecolor{color-gmap}{RGB}{204,255,204} 
\definecolor{color-gmap}{RGB}{0,204,153} 

% -------------------------------------------
% -- Color style: Honkai Star Rail Firefly --
% -------------------------------------------
\definecolor{ffblue}{RGB}{097, 108, 140}
\definecolor{ffdarkgreen}{RGB}{086, 140, 135}
\definecolor{fflightgreen}{RGB}{178, 213, 155}
\definecolor{ffyellow}{RGB}{242, 222, 121}
\definecolor{ffred}{RGB}{217, 095, 024}
\definecolor{ffred_pv}{RGB}{202, 074, 046}
\definecolor{fforange_pv}{RGB}{232, 141, 047}
\definecolor{ffgreen_pv}{RGB}{059, 165, 149}
\definecolor{ffgreendark_pv}{RGB}{032, 117, 106}
% -------------------------------------------
% -- Color style: Honkai Star Rail Firefly --
% -------------------------------------------
\definecolor{nature_tab_gray1}{HTML}{D8D6C2}
\definecolor{nature_tab_gray2}{HTML}{ECEADF}

\definecolor{graspw}{RGB}{0, 128, 0}
\definecolor{dp}{RGB}{64, 224, 208}
\definecolor{dp3}{RGB}{63, 63, 255}
\definecolor{ours}{RGB}{148, 0, 211}
\definecolor{blockcolor}{RGB}{238, 130, 238}
\definecolor{safeline}{RGB}{255, 0, 0}
\definecolor{bananacolor}{RGB}{255, 165, 0}

%\definecolor{ffdarkgreen}{RGB}{086, 140, 135}
% \usepackage{inconsolata}

\usepackage{pgfplots}
\usepgfplotslibrary{groupplots}
%\pgfplotsset{compat=1.14}
\usetikzlibrary{decorations.pathreplacing}
\pgfplotsset{
axis background/.style={fill=gallery},
grid=both,
  xtick pos=left,
  ytick pos=left,
  tick style={
    major grid style={style=white,line width=1pt},
    minor grid style=gallery,
    draw=none,
  },
  minor tick num=1,
}



\title{InductionBench: LLMs Fail in the Simplest\\Complexity Class}


\author{
  Wenyue Hua$^{1}$ Tyler Wong$^1$ Sun Fei\\
  Liangming Pan$^2$ Adam Jardine$^3$ William Yang Wang$^1$\footnote{Corresponding authors: wenyuehua@ucsb.edu, william@cs.ucsb.edu. I'm very grateful for extensive discussion with Wenda Xu, Xinyi Wang at UCSB.} \\
  \\
  $^1$University of California, Santa Barbara,\\
  $^2$University of Arizona,
  $^3$Rutgers University, New Brunswick
}



\begin{document}
\maketitle

\begin{abstract}  
Test time scaling is currently one of the most active research areas that shows promise after training time scaling has reached its limits.
Deep-thinking (DT) models are a class of recurrent models that can perform easy-to-hard generalization by assigning more compute to harder test samples.
However, due to their inability to determine the complexity of a test sample, DT models have to use a large amount of computation for both easy and hard test samples.
Excessive test time computation is wasteful and can cause the ``overthinking'' problem where more test time computation leads to worse results.
In this paper, we introduce a test time training method for determining the optimal amount of computation needed for each sample during test time.
We also propose Conv-LiGRU, a novel recurrent architecture for efficient and robust visual reasoning. 
Extensive experiments demonstrate that Conv-LiGRU is more stable than DT, effectively mitigates the ``overthinking'' phenomenon, and achieves superior accuracy.
\end{abstract}  

\section{Introduction}


\begin{figure}[t]
\centering
\includegraphics[width=0.6\columnwidth]{figures/evaluation_desiderata_V5.pdf}
\vspace{-0.5cm}
\caption{\systemName is a platform for conducting realistic evaluations of code LLMs, collecting human preferences of coding models with real users, real tasks, and in realistic environments, aimed at addressing the limitations of existing evaluations.
}
\label{fig:motivation}
\end{figure}

\begin{figure*}[t]
\centering
\includegraphics[width=\textwidth]{figures/system_design_v2.png}
\caption{We introduce \systemName, a VSCode extension to collect human preferences of code directly in a developer's IDE. \systemName enables developers to use code completions from various models. The system comprises a) the interface in the user's IDE which presents paired completions to users (left), b) a sampling strategy that picks model pairs to reduce latency (right, top), and c) a prompting scheme that allows diverse LLMs to perform code completions with high fidelity.
Users can select between the top completion (green box) using \texttt{tab} or the bottom completion (blue box) using \texttt{shift+tab}.}
\label{fig:overview}
\end{figure*}

As model capabilities improve, large language models (LLMs) are increasingly integrated into user environments and workflows.
For example, software developers code with AI in integrated developer environments (IDEs)~\citep{peng2023impact}, doctors rely on notes generated through ambient listening~\citep{oberst2024science}, and lawyers consider case evidence identified by electronic discovery systems~\citep{yang2024beyond}.
Increasing deployment of models in productivity tools demands evaluation that more closely reflects real-world circumstances~\citep{hutchinson2022evaluation, saxon2024benchmarks, kapoor2024ai}.
While newer benchmarks and live platforms incorporate human feedback to capture real-world usage, they almost exclusively focus on evaluating LLMs in chat conversations~\citep{zheng2023judging,dubois2023alpacafarm,chiang2024chatbot, kirk2024the}.
Model evaluation must move beyond chat-based interactions and into specialized user environments.



 

In this work, we focus on evaluating LLM-based coding assistants. 
Despite the popularity of these tools---millions of developers use Github Copilot~\citep{Copilot}---existing
evaluations of the coding capabilities of new models exhibit multiple limitations (Figure~\ref{fig:motivation}, bottom).
Traditional ML benchmarks evaluate LLM capabilities by measuring how well a model can complete static, interview-style coding tasks~\citep{chen2021evaluating,austin2021program,jain2024livecodebench, white2024livebench} and lack \emph{real users}. 
User studies recruit real users to evaluate the effectiveness of LLMs as coding assistants, but are often limited to simple programming tasks as opposed to \emph{real tasks}~\citep{vaithilingam2022expectation,ross2023programmer, mozannar2024realhumaneval}.
Recent efforts to collect human feedback such as Chatbot Arena~\citep{chiang2024chatbot} are still removed from a \emph{realistic environment}, resulting in users and data that deviate from typical software development processes.
We introduce \systemName to address these limitations (Figure~\ref{fig:motivation}, top), and we describe our three main contributions below.


\textbf{We deploy \systemName in-the-wild to collect human preferences on code.} 
\systemName is a Visual Studio Code extension, collecting preferences directly in a developer's IDE within their actual workflow (Figure~\ref{fig:overview}).
\systemName provides developers with code completions, akin to the type of support provided by Github Copilot~\citep{Copilot}. 
Over the past 3 months, \systemName has served over~\completions suggestions from 10 state-of-the-art LLMs, 
gathering \sampleCount~votes from \userCount~users.
To collect user preferences,
\systemName presents a novel interface that shows users paired code completions from two different LLMs, which are determined based on a sampling strategy that aims to 
mitigate latency while preserving coverage across model comparisons.
Additionally, we devise a prompting scheme that allows a diverse set of models to perform code completions with high fidelity.
See Section~\ref{sec:system} and Section~\ref{sec:deployment} for details about system design and deployment respectively.



\textbf{We construct a leaderboard of user preferences and find notable differences from existing static benchmarks and human preference leaderboards.}
In general, we observe that smaller models seem to overperform in static benchmarks compared to our leaderboard, while performance among larger models is mixed (Section~\ref{sec:leaderboard_calculation}).
We attribute these differences to the fact that \systemName is exposed to users and tasks that differ drastically from code evaluations in the past. 
Our data spans 103 programming languages and 24 natural languages as well as a variety of real-world applications and code structures, while static benchmarks tend to focus on a specific programming and natural language and task (e.g. coding competition problems).
Additionally, while all of \systemName interactions contain code contexts and the majority involve infilling tasks, a much smaller fraction of Chatbot Arena's coding tasks contain code context, with infilling tasks appearing even more rarely. 
We analyze our data in depth in Section~\ref{subsec:comparison}.



\textbf{We derive new insights into user preferences of code by analyzing \systemName's diverse and distinct data distribution.}
We compare user preferences across different stratifications of input data (e.g., common versus rare languages) and observe which affect observed preferences most (Section~\ref{sec:analysis}).
For example, while user preferences stay relatively consistent across various programming languages, they differ drastically between different task categories (e.g. frontend/backend versus algorithm design).
We also observe variations in user preference due to different features related to code structure 
(e.g., context length and completion patterns).
We open-source \systemName and release a curated subset of code contexts.
Altogether, our results highlight the necessity of model evaluation in realistic and domain-specific settings.






\putsec{related}{Related Work}

\noindent \textbf{Efficient Radiance Field Rendering.}
%
The introduction of Neural Radiance Fields (NeRF)~\cite{mil:sri20} has
generated significant interest in efficient 3D scene representation and
rendering for radiance fields.
%
Over the past years, there has been a large amount of research aimed at
accelerating NeRFs through algorithmic or software
optimizations~\cite{mul:eva22,fri:yu22,che:fun23,sun:sun22}, and the
development of hardware
accelerators~\cite{lee:cho23,li:li23,son:wen23,mub:kan23,fen:liu24}.
%
The state-of-the-art method, 3D Gaussian splatting~\cite{ker:kop23}, has
further fueled interest in accelerating radiance field
rendering~\cite{rad:ste24,lee:lee24,nie:stu24,lee:rho24,ham:mel24} as it
employs rasterization primitives that can be rendered much faster than NeRFs.
%
However, previous research focused on software graphics rendering on
programmable cores or building dedicated hardware accelerators. In contrast,
\name{} investigates the potential of efficient radiance field rendering while
utilizing fixed-function units in graphics hardware.
%
To our knowledge, this is the first work that assesses the performance
implications of rendering Gaussian-based radiance fields on the hardware
graphics pipeline with software and hardware optimizations.

%%%%%%%%%%%%%%%%%%%%%%%%%%%%%%%%%%%%%%%%%%%%%%%%%%%%%%%%%%%%%%%%%%%%%%%%%%
\myparagraph{Enhancing Graphics Rendering Hardware.}
%
The performance advantage of executing graphics rendering on either
programmable shader cores or fixed-function units varies depending on the
rendering methods and hardware designs.
%
Previous studies have explored the performance implication of graphics hardware
design by developing simulation infrastructures for graphics
workloads~\cite{bar:gon06,gub:aam19,tin:sax23,arn:par13}.
%
Additionally, several studies have aimed to improve the performance of
special-purpose hardware such as ray tracing units in graphics
hardware~\cite{cho:now23,liu:cha21} and proposed hardware accelerators for
graphics applications~\cite{lu:hua17,ram:gri09}.
%
In contrast to these works, which primarily evaluate traditional graphics
workloads, our work focuses on improving the performance of volume rendering
workloads, such as Gaussian splatting, which require blending a huge number of
fragments per pixel.

%%%%%%%%%%%%%%%%%%%%%%%%%%%%%%%%%%%%%%%%%%%%%%%%%%%%%%%%%%%%%%%%%%%%%%%%%%
%
In the context of multi-sample anti-aliasing, prior work proposed reducing the
amount of redundant shading by merging fragments from adjacent triangles in a
mesh at the quad granularity~\cite{fat:bou10}.
%
While both our work and quad-fragment merging (QFM)~\cite{fat:bou10} aim to
reduce operations by merging quads, our proposed technique differs from QFM in
many aspects.
%
Our method aims to blend \emph{overlapping primitives} along the depth
direction and applies to quads from any primitive. In contrast, QFM merges quad
fragments from small (e.g., pixel-sized) triangles that \emph{share} an edge
(i.e., \emph{connected}, \emph{non-overlapping} triangles).
%
As such, QFM is not applicable to the scenes consisting of a number of
unconnected transparent triangles, such as those in 3D Gaussian splatting.
%
In addition, our method computes the \emph{exact} color for each pixel by
offloading blending operations from ROPs to shader units, whereas QFM
\emph{approximates} pixel colors by using the color from one triangle when
multiple triangles are merged into a single quad.



\section{Computational Complexity in Inductive Reasoning}
\label{sec:com}
\textbf{InductionBench} uses string-to-string transformation/functions as a proxy to study inductive reasoning, which has established computational complexity hierarchy\citep{roche1997finite, engelfriet2001mso}. We focus on the subregular hierarchy, the hierarchy under regular functions. Though with limited expressive power, our experiments show that these classes already present substantial challenges for LLMs.

Specifically, we limit our attention to three classes of deterministic regular functions—\emph{Left Output-Strictly-Local} (L-OSL), \emph{Right Output-Strictly-Local} (R-OSL), and \emph{Input-Strictly-Local} (ISL), whose positions in the subregular hierarchy are illustrated in Figure~\ref{fig:subregular} \citep{heinz2018computational}. These classes represent
the lowest-complexity tier for string-to-string mappings within the subregular hierarchy. They are proper subclasses of subsequential function class and, more broadly, of weakly-deterministic class and non-deterministic class, which are themselves subsets of the regular function classes. Although we do not elaborate on the complete regular function hierarchy here, it is important to note that the ISL, L-OSL, and R-OSL classes are among the simplest in this framework.

Strictly local functions can be seen as operating with a fixed amount of look-ahead, similar to Markov processes. They are \emph{provably learnable in polynomial time from polynomially sized samples} \citep{chandlee2014learning, de1997characteristic, chandlee2015output, jardine2014very}. Moreover, prior work has shown that an algorithm exists to learn the unique (up to isomorphism) smallest subsequential finite-state transducer that represents such ISL, L-OSL, R-OSL functions \citep{satta1997string, arasu2009learning}. This property allows us to evaluate not only whether LLMs can discover the correct patterns but also whether they can identify the simplest or most concise representation consistent with the data. %Our experiments thus aim to determine if current LLMs can effectively detect and generalize the underlying ISL, L-OSL, and R-OSL functions from finite data, as well as whether they can pinpoint the minimal transformations required to match observed input–output behavior.

%Within the broader category of regular functions, we focus on the deterministic function classes: Left Output-Strictly-Local ($L-OSL$), Right Output-Strictly-Local (R-OSL), and Input-Strictly-Local ($ISL$) functions \citep{heinz2018computational}, whose function classes hierarchical relationship in the subregular hierarchy are presented in Figure~\ref{fig:subregular}. 
%one can distinguish between non-deterministic and deterministic classes \citep{heinz2018computational}. Non-deterministic regular functions are a strict superset of \emph{weakly-deterministic} regular functions. The deterministic classes, which lie inside this weakly-deterministic region (as shown in Figure~\ref{fig:subregular}), include left-subsequential, right-subsequential, Left Output-Strictly-Local ($L-OSL$), Right Output-Strictly-Local (R-OSL), and Input-Strictly-Local ($ISL$) functions. Notably, $ISL$, $L-OSL$, and R-OSL overlap without subsuming one another; $L-OSL$ is a subclass of left-subsequential, and R-OSL is a subclass of right-subsequential. Both left-subsequential and right-subsequential classes, in turn, belong to the weakly-deterministic subset of regular functions.
%We specifically target these deterministic classes because they are provably learnable in polynomial time using polynomially sized samples \citep{chandlee2014learning, de1997characteristic}. Also, they are among the simplest classes of functions in string-to-string mappings. Intuitively, these classes can be viewed as transformations with finite look-ahead, akin to Markov processes, thus making them theoretically tractable for inductive learning. In this work, we limit our scope to $ISL$, $L-OSL$, and R-OSL functions, the simplest computational classes within the subregular hierarchy, and investigate how effectively LLMs can handle these inductive reasoning tasks.

\subsection{Preliminary}
Before providing the definitions of the three function classes, we first introduce the fundamental mathematical notations and formal definitions underpinning our discussion of string-to-string transformations and their properties.

Let $\Sigma$ be a finite alphabet. We denote by $\Sigma^*$ the set of all finite strings over $\Sigma$, and by $\Sigma^{\leq n}$ the set of all strings over $\Sigma$ of length at most $n$. The empty string is denoted as $\lambda$. The set of prefixes of a string $w$ is denoted as \textsc{Pref}($w$), defined as $\{p\in \Sigma^*\mid \exists s\in \Sigma^* s.t. w = ps\},$ and the set of suffixes of $w$ denoted as \textsc{Suff}($w$), defined as $\{s\in \Sigma^*\mid \exists p\in \Sigma^* s.t. w = ps\}.$ The longest common prefix of a set of strings $S$ is denoted as \textsc{lcp}($S$), defined as 
\begin{equation}
    p\in\cap_{w\in S}\textsc{Pref}(w) \text{ such as } \forall p'\in\cap_{w\in S}\textsc{Pref}(w), |p'| < |p|.
\end{equation}
For any function $f:\Sigma^*\to\Gamma^*$ and $w\in\Sigma^*$, let the tails of $w$ with respect to $f$ be defined as
\begin{equation}
    \textsc{tails}_f(w) = \{(y, v)\mid f(wy) = uv \text{ and }
    u = \textsc{lcp}(f(w\Sigma^*))\}.
\end{equation}
Intuitively, \textsc{tails}$_f(w)$ collects all possible continuations $(y, v)$ by appending $y$ to $w$. It summarizes how $f$ might extend beyond the partial input $w$. The total number of distinct tails across all strings in $\Sigma^*$ provides a measure of how many different non-trivial local transformation $f$ encodes.


\subsection{Function Class Definition}
Based on the concepts outlined above, we define the three function classes.
\begin{definition}[ISL]
A function f is ISL if there is a $k$ such that for all $u_1, u_2\in\Sigma^*$, if $\textsc{Suff}^{k-1}(u_1) = \textsc{Suff}^{k-1}(u_2)$, then $\textsc{tails}_f(u_1) = \textsc{tails}_f(u_2)$.
\end{definition}

In simpler terms, this means that the output at each position in the string depends only on the preceding $k-1$ characters of the \emph{input}, making the transformation \emph{Markovian} with respect to the input. Below is a simple example:
\begin{exmp}
Suppose a function $f:\{a, b\}^*\to\{a, b\}^*$ rewrites each $b$ to $a$ \emph{only} if it appears after the input substring $ba$. In this scenario, we have $k=3$, and there are two distinct \emph{tails}:
\begin{equation*}
    \textsc{tails}_f(w) = \{(\lambda, \lambda),  (b, a), (bb, ab), (ab, ab) \dots\}, \quad 
    \forall w\in\Sigma^*\text{ such that } ba\in\textsc{suff}(w) 
\end{equation*}
and
\begin{equation*}
\textsc{tails}_f(w') = \{(\lambda, \lambda),  (a, a), (bb, bb), (ab, ab) \dots\}, \quad \forall w'\in\Sigma^*\text{ such that } ba\notin \textsc{suff}(w')
\end{equation*}
\end{exmp}

These tails indicate how the function's behavior shifts depending on whether the immediate context ends in $ba$. Such context-dependent tails also highlights that ISL functions can be effectively characterized or represented by local input constraints.

\begin{definition}[L-OSL]
A function f is L-OSL if there is a $k$ such that for all $u_1, u_2\in\Sigma^*$, if $\textsc{Suff}^{k-1}(f(u_1)) = \textsc{Suff}^{k-1}(f(u_2))$, then $\textsc{tails}_f(u_1) = \textsc{tails}_f(u_2)$.
\end{definition}

In other words, the output at each position in the transformed string depends only on the preceding $k-1$ characters of the \emph{output} itself, rather than on the input. This property can be understood as a form of Markovian process \emph{on the output}. Below is a simple example:
\begin{exmp}
Suppose a function $f$ rewrites each $b$ to $\lambda$ \emph{only} if it appears after the output substring $ba$. In this scenario, we have $k=3$, and there are two distinct \emph{tails}:
\begin{multline*}
    \textsc{tails}_f(w) = \{(\lambda, \lambda), (a, a), (b, \lambda),
    (bb, \lambda), (ab, ab), (ba, a), \dots\} \\
    \forall w\in\Sigma^*\text{ such that } ba\in\textsc{suff}(f(w))
\end{multline*}
and
\begin{multline*}
    \textsc{tails}_f(w) = \{(\lambda, \lambda), (a, a), (b, b),
    (bb, bb), (ab, ab), (ba, ba) \dots\} \\
    \forall w\in\Sigma^*\text{ such that } ba\notin\textsc{suff}(f(w))
\end{multline*}
\end{exmp}

While L-OSL depends preceding output symbols to the ``left'', R-OSL functions depends on a limited number of \emph{future} output symbols to the ``right''. Conceptually, one can view R-OSL as analogous to L-OSL, except that the input is processed in reverse order. Although both belong to the broader OSL paradigm, they are \emph{incomparable} classes: each can express transformations the other cannot. The formal definition of R-OSL follows:

\begin{definition}[R-OSL]
A function f is R-OSL if there is a $k$ such that for all $u_1, u_2\in\Sigma^*$, if $\textsc{Suff}^{k-1}(f(u_1^{-1})) = \textsc{Suff}^{k-1}(f(u_2^{-1}))$, then $\textsc{tails}_f(u_1^{-1}) = \textsc{tails}_f(u_2^{-1})$.
\end{definition}

Intuitively, this class of functions can be viewed as a \emph{rightward} Markovian process on the output. Each output symbol is determined not by the preceding symbols as in L-OSL but by the next $k-1$ symbols that will appear in the output.

The three classes, ISL, L-OSL, and R-OSL, are each deterministic and exhibit Markovian behavior, yet remain pairwise incomparable within the broader subregular hierarchy. In this work, we further restrict our attention to functions that involve \emph{substitution} which replaces one character with another and \emph{deletion} which maps a character to the empty string $\lambda$.

\subsection{Learnability}
The three function classes are \emph{identifiable in polynomial time using a polynomially sized characteristic sample} \citep{chandlee2014learning, chandlee2015output}. In other words, there exists a polynomial-time algorithm that, given sufficient data for a target function $f$, can produce a representation $\tau$ that satisfies $f(w) = \tau(w)$ for every $w\in\Sigma^*$. In other words, once sufficient data is presented, one can reliably recover a function equivalent to $f$ on all possible inputs. This learnability property underpins the value of these classes as testbeds for inductive reasoning, since the data requirement remains polynomial and successful inference is theoretically guaranteed.

We formalize ``sufficient data'' as the minimal set of input–output pairs needed to learn a $k$-strictly local function $f$, which is known as characteristic sample. Adapting the original definition\footnote{simplified from original definition} for clarity \citep{chandlee2014learning, chandlee2015output}, we define:
\begin{definition}[Characteristic Sample]
For a given $k$-ISL $f$, the characteristic sample $S$ is defined as $\{(w, w')\mid w\in \Sigma^{\leq k}\land f(w) = w'\}$. For a given $k$-OSL $f$, the characteristic sample $S$ is defined as $\{(w, w')\mid w'\in \Sigma^{\leq k}\land f(w) = w'\}$.
\end{definition}

If a provided dataset contains such characteristic sample, a learning algorithm can reconstruct a representation of $f$ that matches its behavior on every string in $\Sigma$. Accordingly, in the context of LLMs, we expect that providing this dataset as in-context examples should enable the model to induce the underlying string-to-string mapping. 

\subsection{Unique Function Representation}
Beyond verifying that a model can accurately discover a function from data, we also investigate how succinctly the model describes its inferred rules. This aspect is of both theoretical and practical interest: a minimal or \emph{most concise} representation not only offers interpretability advantages but can also reflect the model's capacity for truly generalizable, rather than merely enumerative, learning. 

One function can be represented or written in a non-unique way. For instance, consider an ISL function $f_1$ with $k=2$ over $\Sigma = \{a, b\}$ that maps the input character $a$ to $b$ when it comes after $b$, that rewrites each $a$ to $b$ only if the preceding character is $b$, while leaving other substrings unchanged. One concise description is:
\begin{equation}
% \hspace{-3mm}\begin{aligned}
\hspace{-3mm} f_1(w) {=}\,
  \begin{cases}
 f_1(w_1)ba^{-1}f_1(aw_2), \\
 \quad\quad \parbox[t]{0.6\linewidth}{
        if $w_1$ ends with $b$ and 
        $w {=}w_1$ $aw_2$ for some $w_1, w_2 {\in} \Sigma^*$
      }\\
 w, \,\,\,\,\, \text{otherwise}
  \end{cases}
% \end{aligned}
\hspace{-10pt} 
\end{equation}

An alternative yet more verbose description of the same function might redundantly enumerate multiple cases:
\begin{equation} 
\hspace{-3mm} f_1'(w) {=}
\!\!\begin{cases} 
f_1'(w_1)ba^{-1}f_1'(aw_2), \\
 \quad\quad \parbox[t]{0.6\linewidth}{
 if $w_1$ ends with $ab$ and $w = w_1aw_2$ for some $w_1, w_2$ $\in\Sigma^*$ } \\
f_1'(w_1)ba^{-1}f_1'(aw_2) \\
\quad\quad \parbox[t]{0.6\linewidth}{ 
 if $w_1$ ends with $bb$ and $w = w_1aw_2$ for some $w_1, w_2$ $\in\Sigma^*$ }\\
w,\,\,\,\,\, \text{otherwise}
\end{cases} 
\hspace{-10pt}
\end{equation}

Although these two representations encode the same function, the second contains repetitive conditions and fails to emphasize that the output of $f_1$ depends solely on the single preceding character instead of the penultimate character. 

Because these functions admit a \emph{unique} minimal representation (up to isomorphism) \citep{chandlee2014learning, oncina1991inductive}, we can directly compare the function produced by an LLM to the ground-truth minimal form. In doing so, we evaluate whether the model not only \emph{discovers} the correct transformation but also \emph{simplifies} it to the most parsimonious description possible—an essential indicator of robust inductive reasoning.

\subsection{Rule-based Representation}
To streamline the generation and parsing of function representations, we employ a simplified notation wherein each transformation is written as ``condition $\circ$ target character $\to$ output of the target character'' \citep{bird1994one}.  In this notation, the \emph{condition} represents the minimal substring needed to trigger a transformation, while any input substring not matching this condition remains unchanged. For instance, in the earlier example, this approach permits a concise notation $b\circ a\to b$, indicating that the input $a$ is mapped to $b$ when it comes after $b$; otherwise, the input string remains unaltered. This concise, rule-based format simplifies both the model's output generation (by reducing complex functional descriptions) and our subsequent evaluation, as the applicable transformations can be easily parsible and verified.

To demonstrate the simplicity of rule-based representation: given an ISL function $f_2$ with $k=2$, the input $a$ becomes $b$ when it comes after $b$ and two consecutive $a$s will be reduced to one single $a$. The minimal function representation is as below:
\begin{equation}
\hspace{-3mm} f_2(w) {=}\,
  \begin{cases}
 f_2(w_1)ba^{-1}f_2(aw_2), \\
 \quad\quad \parbox[t]{0.6\linewidth}{
        if $w_1$ ends with $b$ and 
        $w {=}w_1aw_2$ for some $w_1, w_2 {\in} \Sigma^*$
      }\\
 f_2(w_1)a^{-1}f_2(aw_2), \\
 \quad\quad \parbox[t]{0.6\linewidth}{
        if $w_1$ ends with $a$ and 
        $w {=}w_1aw_2$ for some $w_1, w_2 {\in} \Sigma^*$
      }\\
 w, \,\,\,\,\, \text{otherwise}
  \end{cases}
\hspace{-10pt} 
\end{equation}

In the simplified rule-based format, it can be written as:$a\circ a\to\lambda\text{, }b\circ a\to b$. In summary, $f_1$ can be minimally expressed with a single rule, $f_2$ requires two rules.



\section{Benchmark Construction} 
In this section, we detail how our benchmark is constructed from the previously defined function classes. Each datapoint $(\mathcal{D}, f)$ in the benchmark is a pair of dataset $\mathcal{D}$ and function $f$ where $\mathcal{D}$ is a set of input-output pairs generated by $f$.

Each of ISL, L-OSL, and R-OSL classes can be further subdivided into incremental levels of complexity, determined by three key parameters: (1) the context window size $k$ (2) the vocabulary size $|\Sigma|$ (3) the minimal representation length of the function, \emph{i.e.} the minimal set of rules corresponding to the function. Given $k$ and $|\Sigma|$, the search space is $2^{|\Sigma|^k}$; given the number of rules $n$ additionally, the search space is $|\Sigma|^k \choose n$. To rigorously evaluate LLMs' inductive capabilities, we systematically vary these parameters across ISL, L-OSL, and R-OSL function classes.

In addition, we examine how performance changes with different numbers of input–output pairs in the prompt. Although having the characteristic sample present should theoretically guarantee recoverability of the underlying function, our empirical results indicate that the overall number of examples strongly affects performance. While extra data can provide richer information, it also increases context length considerably and heightens processing demands \citep{li2024long}. By varying the number of provided datapoints, we further investigate the extent to which the model engages in genuine reasoning and how robust its inductive abilities remain under changing input sizes.

\paragraph{Function Generation}
To systematically create benchmark instances, we first \emph{randomly generate} functions $f$ based on the three parameters: $k$, $|\Sigma|$, and the number of minimal rules describing $f$ by generating the set of rules that can describe $f$. While multiple representations of varying length can describe the same function, each function has a \emph{unique minimal representation} (up to isomorphism). During function generation, we therefore ensure that each function is expressed by a minimal, non-redundant rule set. Formally, if a $f$ is represented by a set of rules $R_f = \{r_1, r_2, ..., r_n\}$ where each $r_i$ has the form of $c_i\circ u_i\to v_i$ (with $c_i$ as the condition substring, $u_i$ the target character, and $v_i$ the transformed output for $u_i$), there are several constraints may be applied to functions belonging to the three classes. 

\begin{definition}[General Consistency]
Given $f$ represented by a set of rules $R_f: \forall r_i, r_j\in R_f, c_i\circ u_i\notin\textsc{Suff}(c_j\circ u_j)$ and $c_j\circ u_j\notin\textsc{Suff}(c_i\circ u_i)$.
\end{definition}

General Consistency ensures that the rules do not contradict one another or become redundant when conditions overlap. For instance, a function whose rule-based representation of $r_1: a\circ b\to a$ and $r_2: aa\circ b\to a$ is redundant, as the scenarios where $r_1$ is applied is a superset of the scenarios where $r_2$ is applied. For another instance, there does not exist a deterministic function that can be described by $r_1: a\circ b\to a$ and $r_2: aa\circ b\to \lambda$. Generating rule-based representations for ISL functions needs only satisfy this constraint.

\begin{definition}[OSL Non-Redundancy Guarantee]
Given $f$ represented by a set of rules $R:\forall r_i\in $ $R_f, \lnot\exists s_i'$ ${\in}\{s_i|s_i {\in} c_i\}$ such that $ \exists r_j\in R_f$ such that $ s_i' = c_j\circ u_j, \text{ unless }\exists r_k\in R$ such that $c_k\circ v_k = s_i'$.
\end{definition}

Constraint 2 is specific to the two OSL function classes because we need to make sure that all output conditions in the rule actually surface somewhere in the outputs of some datapoints. If the output condition $c$ never actually surface as the output, the rule will never be put into effect. Thereby the above rule basically requires that condition part of all rules can surface, either because it will never be modified by some other rule, or it emerges on the surface because of the application of other rule. For instance, a function represented by rules $r_1: aa\circ b\to a, r_2: a\circ a\to c$ is redundant because $r_1$ will never be applied because the string $aa$ will never surface as output and thus it will never be put into effect; For another instance, a function represented by $r_1: aa\circ b\to a, r_2: a\circ a\to c, r_3: a\circ d\to a$ is non redundant because even though into $aa$ string will be modified into $ac$, but $aa$ will surface in some datapoint because $ad$ will be modified into $aa$ and thus $r_1$ will be able to be applied.

Generating the functions following the two constraints, we ensure that the generated function representation is minimal, non-reducible guarantees a clear measure of complexity. One additional requirement is imposed to ensure each function indeed requires a look-ahead of size $k$. Specifically:

\begin{definition}[$k$-Complexity Guarantee]
Given $f$ whose designated context window $k = k_1$, $\exists r'\in R \text{ such that } c'\circ u'\to v'$ such that $|c'\circ u'| = k_1$.
\end{definition}

This condition guarantees that the function is genuinely $k$-strictly local (for ISL or OSL), rather than being representable with a smaller window size. Consequently, the functions we generate faithfully reflect the intended complexity level.

After generating the function $f$, we generate the characteristic sample of input-output pairs. For instance, given a function $f$ with $k = 2$ and $\Sigma = \{a, b\}$, the characteristic sample is $\{(a, f(a)),(b, f(b)), (ab, f(ab)), (aa, f(aa)), (bb, f(bb)), (ba, f(ba))\}$, a small set whose size is 6. By expanding this sample set, we can explore whether providing more than the minimal necessary examples aids or hinders the model's performance to infer the underlying function.

To evaluate how effectively an LLM can induce the underlying function, we include in the prompt (1) the function class, (2) context window $k$, (3) the alphabet $\Sigma$ which are information that guarantee learnability of the function. Then given the sample dataset, we request LLMs to produce a minimal rule-based description that reproduces the provided sample set, revealing whether it can \emph{discover} and \emph{optimally represent} the underlying transformation.



\section{Experiments}
\label{sec:exp}
Following the settings in Section \ref{sec:existing}, we evaluate \textit{NovelSum}'s correlation with the fine-tuned model performance across 53 IT datasets and compare it with previous diversity metrics. Additionally, we conduct a correlation analysis using Qwen-2.5-7B \cite{yang2024qwen2} as the backbone model, alongside previous LLaMA-3-8B experiments, to further demonstrate the metric's effectiveness across different scenarios. Qwen is used for both instruction tuning and deriving semantic embeddings. Due to resource constraints, we run each strategy on Qwen for two rounds, resulting in 25 datasets. 

\subsection{Main Results}

\begin{table*}[!t]
    \centering
    \resizebox{\linewidth}{!}{
    \begin{tabular}{lcccccccccc}
    \toprule
    \multirow{3}*{\textbf{Diversity Metrics}} & \multicolumn{10}{c}{\textbf{Data Selection Strategies}} \\
    \cmidrule(lr){2-11}
    & \multirow{2}*{\textbf{K-means}} & \multirow{2}*{\vtop{\hbox{\textbf{K-Center}}\vspace{1mm}\hbox{\textbf{-Greedy}}}}  & \multirow{2}*{\textbf{QDIT}} & \multirow{2}*{\vtop{\hbox{\textbf{Repr}}\vspace{1mm}\hbox{\textbf{Filter}}}} & \multicolumn{5}{c}{\textbf{Random}} & \multirow{2}{*}{\textbf{Duplicate}} \\ 
    \cmidrule(lr){6-10}
    & & & & & \textbf{$\mathcal{X}^{all}$} & ShareGPT & WizardLM & Alpaca & Dolly &  \\
    \midrule
    \rowcolor{gray!15} \multicolumn{11}{c}{\textit{LLaMA-3-8B}} \\
    Facility Loc. $_{\times10^5}$ & \cellcolor{BLUE!40} 2.99 & \cellcolor{ORANGE!10} 2.73 & \cellcolor{BLUE!40} 2.99 & \cellcolor{BLUE!20} 2.86 & \cellcolor{BLUE!40} 2.99 & \cellcolor{BLUE!0} 2.83 & \cellcolor{BLUE!30} 2.88 & \cellcolor{BLUE!0} 2.83 & \cellcolor{ORANGE!20} 2.59 & \cellcolor{ORANGE!30} 2.52 \\    
    DistSum$_{cosine}$  & \cellcolor{BLUE!30} 0.648 & \cellcolor{BLUE!60} 0.746 & \cellcolor{BLUE!0} 0.629 & \cellcolor{BLUE!50} 0.703 & \cellcolor{BLUE!10} 0.634 & \cellcolor{BLUE!40} 0.656 & \cellcolor{ORANGE!30} 0.578 & \cellcolor{ORANGE!10} 0.605 & \cellcolor{ORANGE!20} 0.603 & \cellcolor{BLUE!10} 0.634 \\
    Vendi Score $_{\times10^7}$ & \cellcolor{BLUE!30} 1.70 & \cellcolor{BLUE!60} 2.53 & \cellcolor{BLUE!10} 1.59 & \cellcolor{BLUE!50} 2.23 & \cellcolor{BLUE!20} 1.61 & \cellcolor{BLUE!30} 1.70 & \cellcolor{ORANGE!10} 1.44 & \cellcolor{ORANGE!20} 1.32 & \cellcolor{ORANGE!10} 1.44 & \cellcolor{ORANGE!30} 0.05 \\
    \textbf{NovelSum (Ours)} & \cellcolor{BLUE!60} 0.693 & \cellcolor{BLUE!50} 0.687 & \cellcolor{BLUE!30} 0.673 & \cellcolor{BLUE!20} 0.671 & \cellcolor{BLUE!40} 0.675 & \cellcolor{BLUE!10} 0.628 & \cellcolor{BLUE!0} 0.591 & \cellcolor{ORANGE!10} 0.572 & \cellcolor{ORANGE!20} 0.50 & \cellcolor{ORANGE!30} 0.461 \\
    \midrule    
    \textbf{Model Performance} & \cellcolor{BLUE!60}1.32 & \cellcolor{BLUE!50}1.31 & \cellcolor{BLUE!40}1.25 & \cellcolor{BLUE!30}1.05 & \cellcolor{BLUE!20}1.20 & \cellcolor{BLUE!10}0.83 & \cellcolor{BLUE!0}0.72 & \cellcolor{ORANGE!10}0.07 & \cellcolor{ORANGE!20}-0.14 & \cellcolor{ORANGE!30}-1.35 \\
    \midrule
    \midrule
    \rowcolor{gray!15} \multicolumn{11}{c}{\textit{Qwen-2.5-7B}} \\
    Facility Loc. $_{\times10^5}$ & \cellcolor{BLUE!40} 3.54 & \cellcolor{ORANGE!30} 3.42 & \cellcolor{BLUE!40} 3.54 & \cellcolor{ORANGE!20} 3.46 & \cellcolor{BLUE!40} 3.54 & \cellcolor{BLUE!30} 3.51 & \cellcolor{BLUE!10} 3.50 & \cellcolor{BLUE!10} 3.50 & \cellcolor{ORANGE!20} 3.46 & \cellcolor{BLUE!0} 3.48 \\ 
    DistSum$_{cosine}$ & \cellcolor{BLUE!30} 0.260 & \cellcolor{BLUE!60} 0.440 & \cellcolor{BLUE!0} 0.223 & \cellcolor{BLUE!50} 0.421 & \cellcolor{BLUE!10} 0.230 & \cellcolor{BLUE!40} 0.285 & \cellcolor{ORANGE!20} 0.211 & \cellcolor{ORANGE!30} 0.189 & \cellcolor{ORANGE!10} 0.221 & \cellcolor{BLUE!20} 0.243 \\
    Vendi Score $_{\times10^6}$ & \cellcolor{ORANGE!10} 1.60 & \cellcolor{BLUE!40} 3.09 & \cellcolor{BLUE!10} 2.60 & \cellcolor{BLUE!60} 7.15 & \cellcolor{ORANGE!20} 1.41 & \cellcolor{BLUE!50} 3.36 & \cellcolor{BLUE!20} 2.65 & \cellcolor{BLUE!0} 1.89 & \cellcolor{BLUE!30} 3.04 & \cellcolor{ORANGE!30} 0.20 \\
    \textbf{NovelSum (Ours)}  & \cellcolor{BLUE!40} 0.440 & \cellcolor{BLUE!60} 0.505 & \cellcolor{BLUE!20} 0.403 & \cellcolor{BLUE!50} 0.495 & \cellcolor{BLUE!30} 0.408 & \cellcolor{BLUE!10} 0.392 & \cellcolor{BLUE!0} 0.349 & \cellcolor{ORANGE!10} 0.336 & \cellcolor{ORANGE!20} 0.320 & \cellcolor{ORANGE!30} 0.309 \\
    \midrule
    \textbf{Model Performance} & \cellcolor{BLUE!30} 1.06 & \cellcolor{BLUE!60} 1.45 & \cellcolor{BLUE!40} 1.23 & \cellcolor{BLUE!50} 1.35 & \cellcolor{BLUE!20} 0.87 & \cellcolor{BLUE!10} 0.07 & \cellcolor{BLUE!0} -0.08 & \cellcolor{ORANGE!10} -0.38 & \cellcolor{ORANGE!30} -0.49 & \cellcolor{ORANGE!20} -0.43 \\
    \bottomrule
    \end{tabular}
    }
    \caption{Measuring the diversity of datasets selected by different strategies using \textit{NovelSum} and baseline metrics. Fine-tuned model performances (Eq. \ref{eq:perf}), based on MT-bench and AlpacaEval, are also included for cross reference. Darker \colorbox{BLUE!60}{blue} shades indicate higher values for each metric, while darker \colorbox{ORANGE!30}{orange} shades indicate lower values. While data selection strategies vary in performance on LLaMA-3-8B and Qwen-2.5-7B, \textit{NovelSum} consistently shows a stronger correlation with model performance than other metrics. More results are provided in Appendix \ref{app:results}.}
    \label{tbl:main}
    \vspace{-4mm}
\end{table*}


\begin{table}[t!]
\centering
\resizebox{\linewidth}{!}{
\begin{tabular}{lcccc}
\toprule
\multirow{2}*{\textbf{Diversity Metrics}} & \multicolumn{3}{c}{\textbf{LLaMA}} & \textbf{Qwen}\\
\cmidrule(lr){2-4} \cmidrule(lr){5-5} 
& \textbf{Pearson} & \textbf{Spearman} & \textbf{Avg.} & \textbf{Avg.} \\
\midrule
TTR & -0.38 & -0.16 & -0.27 & -0.30 \\
vocd-D & -0.43 & -0.17 & -0.30 & -0.31 \\
\midrule
Facility Loc. & 0.86 & 0.69 & 0.77 & 0.08 \\
Entropy & 0.93 & 0.80 & 0.86 & 0.63 \\
\midrule
LDD & 0.61 & 0.75 & 0.68 & 0.60 \\
KNN Distance & 0.59 & 0.80 & 0.70 & 0.67 \\
DistSum$_{cosine}$ & 0.85 & 0.67 & 0.76 & 0.51 \\
Vendi Score & 0.70 & 0.85 & 0.78 & 0.60 \\
DistSum$_{L2}$ & 0.86 & 0.76 & 0.81 & 0.51 \\
Cluster Inertia & 0.81 & 0.85 & 0.83 & 0.76 \\
Radius & 0.87 & 0.81 & 0.84 & 0.48 \\
\midrule
NovelSum & \textbf{0.98} & \textbf{0.95} & \textbf{0.97} & \textbf{0.90} \\
\bottomrule
\end{tabular}
}
\caption{Correlations between different metrics and model performance on LLaMA-3-8B and Qwen-2.5-7B.  “Avg.” denotes the average correlation (Eq. \ref{eq:cor}).}
\label{tbl:correlations}
\vspace{-2mm}
\end{table}

\paragraph{\textit{NovelSum} consistently achieves state-of-the-art correlation with model performance across various data selection strategies, backbone LLMs, and correlation measures.}
Table \ref{tbl:main} presents diversity measurement results on datasets constructed by mainstream data selection methods (based on $\mathcal{X}^{all}$), random selection from various sources, and duplicated samples (with only $m=100$ unique samples). 
Results from multiple runs are averaged for each strategy.
Although these strategies yield varying performance rankings across base models, \textit{NovelSum} consistently tracks changes in IT performance by accurately measuring dataset diversity. For instance, K-means achieves the best performance on LLaMA with the highest NovelSum score, while K-Center-Greedy excels on Qwen, also correlating with the highest NovelSum. Table \ref{tbl:correlations} shows the correlation coefficients between various metrics and model performance for both LLaMA and Qwen experiments, where \textit{NovelSum} achieves state-of-the-art correlation across different models and measures.

\paragraph{\textit{NovelSum} can provide valuable guidance for data engineering practices.}
As a reliable indicator of data diversity, \textit{NovelSum} can assess diversity at both the dataset and sample levels, directly guiding data selection and construction decisions. For example, Table \ref{tbl:main} shows that the combined data source $\mathcal{X}^{all}$ is a better choice for sampling diverse IT data than other sources. Moreover, \textit{NovelSum} can offer insights through comparative analyses, such as: (1) ShareGPT, which collects data from real internet users, exhibits greater diversity than Dolly, which relies on company employees, suggesting that IT samples from diverse sources enhance dataset diversity \cite{wang2024diversity-logD}; (2) In LLaMA experiments, random selection can outperform some mainstream strategies, aligning with prior work \cite{xia2024rethinking,diddee2024chasing}, highlighting gaps in current data selection methods for optimizing diversity.



\subsection{Ablation Study}


\textit{NovelSum} involves several flexible hyperparameters and variations. In our main experiments, \textit{NovelSum} uses cosine distance to compute $d(x_i, x_j)$ in Eq. \ref{eq:dad}. We set $\alpha = 1$, $\beta = 0.5$, and $K = 10$ nearest neighbors in Eq. \ref{eq:pws} and \ref{eq:dad}. Here, we conduct an ablation study to investigate the impact of these settings based on LLaMA-3-8B.

\begin{table}[ht!]
\centering
\resizebox{\linewidth}{!}{
\begin{tabular}{lccc}
\toprule
\textbf{Variants} & \textbf{Pearson} & \textbf{Spearman} & \textbf{Avg.} \\
\midrule
NovelSum & 0.98 & 0.96 & 0.97 \\
\midrule
\hspace{0.10cm} - Use $L2$ distance & 0.97 & 0.83 & 0.90\textsubscript{↓ 0.08} \\
\hspace{0.10cm} - $K=20$ & 0.98 & 0.96 & 0.97\textsubscript{↓ 0.00} \\
\hspace{0.10cm} - $\alpha=0$ (w/o proximity) & 0.79 & 0.31 & 0.55\textsubscript{↓ 0.42} \\
\hspace{0.10cm} - $\alpha=2$ & 0.73 & 0.88 & 0.81\textsubscript{↓ 0.16} \\
\hspace{0.10cm} - $\beta=0$ (w/o density) & 0.92 & 0.89 & 0.91\textsubscript{↓ 0.07} \\
\hspace{0.10cm} - $\beta=1$ & 0.90 & 0.62 & 0.76\textsubscript{↓ 0.21} \\
\bottomrule
\end{tabular}
}
\caption{Ablation Study for \textit{NovelSum}.}
\label{tbl:ablation}
\vspace{-2mm}
\end{table}

In Table \ref{tbl:ablation}, $\alpha=0$ removes the proximity weights, and $\beta=0$ eliminates the density multiplier. We observe that both $\alpha=0$ and $\beta=0$ significantly weaken the correlation, validating the benefits of the proximity-weighted sum and density-aware distance. Additionally, improper values for $\alpha$ and $\beta$ greatly reduce the metric's reliability, highlighting that \textit{NovelSum} strikes a delicate balance between distances and distribution. Replacing cosine distance with Euclidean distance and using more neighbors for density approximation have minimal impact, particularly on Pearson's correlation, demonstrating \textit{NovelSum}'s robustness to different distance measures.







\section{Leaderboard based on InductionBench}
To facilitate straightforward comparisons among different LLMs, we introduce a two-part benchmark leaderboard: a \emph{standard leaderboard} and an \emph{exploration leaderboard}. The \emph{standard leaderboard} is based completely on the three function classes we talked about, and this leaderboard simply presents an aggregated score to directly reflect LLM's performance. The \emph{exploration leaderboard} includes a slightly new design of function class and we will present the motivation and details below. 

\subsection{Standard Leaderboard}
The standard leaderboard consists of 1,080 questions spanning three classes of deterministic regular functions: \emph{ISL}, \emph{L-OSL}, and \emph{R-OSL} in equal proportion. Specifically, it includes: \begin{itemize}[leftmargin=*, itemsep=1pt] \item 360 ISL questions, \item 360 L-OSL questions, \item 360 R-OSL questions. \end{itemize}

Within each function class, we have settings for $k\in\{2, 3, 4\}, |\Sigma|\in\{5, 6, 7, 8\}$, and number of rules $\in\{3, 4, 5\}$. Each unique parameter combination has 10 data points, totaling 360 points per function class. The performance metrics \emph{recall}, \emph{precision}, and \emph{compatibility} are computed on a per-setting basis. We then form an overall \emph{weighted average} to account for variations in function-space size: 

\begin{definition}
For a given setting characterized by $(k, |\Sigma|, r)$, the weight $w$ is defined as $\frac{|\Sigma|^k}{\sum\limits_{k'=2}^{k'=4}\sum\limits_{s=5}^{s=8}s^{k'}}$, where $k$ is the Markov window, $|\Sigma|$ is the alphabet size, and $r$ is the minimal rule count.
\end{definition}

For each function class (ISL, L-OSL, R-OSL), we compute a weighted recall, precision, and compatibility according to the above scheme and then take the average of these three scores to produce the final leaderboard score for that class. The overall score across all three classes is the average of those class-wise scores.

Table~\ref{tab:leaderboard} summarizes current leaderboard results for several representative models. Notably, even o3-mini achieves only a 
5.69\% compatibility score, largely because none of the models succeed on tasks where $k=4$. Since those high-complexity settings receive substantially larger weights than cases where $k\in\{2, 3\}$, they disproportionately reduce the overall average.

\begin{table}[!ht]
    \centering
    \begin{tabular}{lccc}
    \toprule
    model & average recall & average precision & average compatibility \\
    \midrule
    Llama-3.1 8b & 0.00&	0.00&	0.00\\
    Qwen2.5-Coder-32B-Instruct	&7.26	&0.66	&0.03\\
    Llama-3.3-70b	&6.55	&5.76&	0.12\\
    DeepSeek-R1-Distill-Llama-70B	&3.84	&5.14	&0.78\\
    o3-mini	&28.93	&43.12	&5.69\\
    \bottomrule
    \end{tabular}
    \caption{Leaderboard Result}
    \label{tab:leaderboard}
\end{table}

To balance the influence of different complexity settings, we additionally report an alternative evaluation metric that replaces each original weight with its logarithm. This approach dampens the dominance of $k=4$ scenarios, yielding a more even distribution of weights across the benchmark’s parameter space.

\begin{table}[!ht]
    \centering
    \begin{tabular}{lccc}
    \toprule
    model & average recall & average precision & average compatibility \\
    \midrule
    Llama-3.1 8b&	0.00	&0.00	&0.00\\
    Qwen2.5-Coder-32B-Instruct	&7.48	&6.60	&0.48\\
    Llama-3.3-70b	&8.71	&7.50	&0.87\\
    DeepSeek-R1-Distill-Llama-70B	&23.17&	24.66&	8.63\\
    o3-mini	&57.58	&63.89&33.93\\
    \bottomrule
    \end{tabular}
    \caption{Leaderboard Result with Log Weight}
    \label{tab:leaderboard_log}
\end{table}


\subsection{Exploration Leaderboard}
A key concern in using subregular function classes (\emph{e.g.}, ISL, L-OSL, R-OSL) is that polynomial-time learning algorithms already exist for these classes, potentially allowing a trivial ``hack'' to achieve artificially high performance. Though we advocate not using the provbly correct algorithm for task solving so that we can genuinely evaluate LLM's inductive reasoning ability, to make sure, we introduce an \emph{exploration leaderboard} that focuses on \emph{Input-Output Strictly Local (IOSL)} functions: a more speculative class for which no known algorithm can reliably learn the entire function from finite data in finite time.

\paragraph{Rationale.} Since IOSL lacks a proven polynomial-time learning procedure, successful performance here would more credibly reflect genuine inductive reasoning rather than the application of a known ``shortcut'' algorithm. Furthermore, IOSL functions have not been deeply studied in the literature, offering an opportunity to see whether LLMs can advance this open research area.


This is the definition of IOSL:
\begin{definition}[IOSL]
A function f is IOSL if there is a $k$ such that for all $u_1, u_2\in\Sigma^*$, if $\textsc{Suff}^{k-1}(u_1) = \textsc{Suff}^{k-1}(u_2)$ and $\textsc{Suff}^{k-1}(f(u_1)) = \textsc{Suff}^{k-1}(f(u_2))$, then $\textsc{tails}_f(u_1) = \textsc{tails}_f(u_2)$.
\end{definition}

In essence, this condition requires the model to distinguish between input-based and output-based Markovian triggers, making the learned transformation highly non-trivial if no pre-existing algorithm is used.

\paragraph{Leaderboard Setup.} The IOSL-based leaderboard contains 1{,}080 datapoints, mirroring the standard leaderboard in overall structure: $k\in\{2, 3, 4\}, |\Sigma|\in\{5, 6, 7, 8\}$, number of rules $\in\{3, 4, 5\}$. For each setting, there are 30 datapoints per setting (for equivalence to the standard leaderboard’s size).

Since IOSL is not known to admit a finite-characteristic sample or minimal representation in the same sense as the deterministic classes, we introduce two adaptations for evaluation:

\begin{enumerate}[leftmargin=*, itemsep=1pt] \item \textbf{Sample Size.} We arbitrarily fix the sample size at $2*|\Sigma|^k$, as no characteristic sample is theoretically guaranteed.

\item \textbf{Evaluation Metrics.} We focus primarily on \emph{compatibility}, as recall and precision hinge on the assumption of a unique minimal-length description, which may not exist for IOSL. If a model's generated rule set is compatible with the data, we then check whether its description length is shorter, identical, or longer than our function's reference length. A longer description indicates a definite failure to produce a minimal representation; shorter or equal does not guarantee minimality, but it at least suggests the model avoids obvious redundancy. \end{enumerate}

By presenting both a standard leaderboard (subregular classes with known learnability) and an exploration leaderboard (IOSL with no established finite-data algorithm), we offer a balanced view: models can demonstrate success in theoretically well-understood tasks while also exploring novel, under-constrained function classes—thereby reducing the concern that high performance might merely reflect an existing ``hack.''





\section{Conclusion} 

In this work, we introduced a systematic benchmark for assessing the inductive reasoning capabilities of LLMs, leveraging both well-studied subregular function classes (ISL, L-OSL, and R-OSL) and a more exploratory class (IOSL) for which no known polynomial-time learning algorithm exists. By controlling parameters such as the Markov window size $k$, the vocabulary size $|\Sigma|$, and the minimal number of rules, we offered precise yet flexible tasks capable of probing a model’s capacity to infer general transformations from limited data. Our findings revealed several significant challenges for current LLMs—especially when required to track deeper dependencies or manage larger search spaces—and underscored the fragility of their inductive reasoning under increased context or novel data.

Through experiments measuring recall, precision, and compatibility, we demonstrated that factors like the Markov window size $k$ and the number of rules more profoundly degrade performance than an expanded alphabet. Moreover, while few-shot prompting showed promise in simpler scenarios, its benefits quickly plateaued in more complex contexts. An error analysis further highlighted how many rules go completely missing or become overgeneralized under stringent settings, indicating that LLMs often fail to synthesize key patterns comprehensively.

We also proposed an exploration leaderboard targeting IOSL functions, a class beyond established theoretical learnability, to address concerns that performance gains might stem from known polynomial-time algorithms rather than genuine inductive reasoning. This complementary evaluation opens avenues for research on less tractable classes and poses a more authentic test of generalization and adaptability.

Overall, our results highlight the need for more robust inductive reasoning strategies within current LLM architectures. We hope that our benchmark will help catalyze progress in both theoretical understanding and practical innovations around LLMs' inductive capabilities.

\section*{Limitations}

While our benchmark offers a rigorous, theoretically grounded approach to evaluating inductive reasoning in LLMs, current paper is subject to two notable constraints:

\paragraph{Synthetic Rather Than Real-World Data.} All tasks and evaluations rely on functions generated from carefully controlled parameters rather than naturally occurring texts or real-world datasets. Although this design enables precise measurement of inductive capabilities, it may not fully capture the complexity of practical language use, where ambiguous contexts, noisy inputs, and domain-specific factors can further challenge inference.

\paragraph{Restricted Access to the o1 Model.} Our investigation into the o1 family of models is hindered by limited availability and computational resources. As a result, certain aspects of o1’s inductive behavior may remain unexamined, and a more exhaustive exploration of variations or fine-tuning strategies for o1 could further illuminate its performance.

% Entries for the entire Anthology, followed by custom entries
\bibliography{iclr2025_conference}
\bibliographystyle{iclr2025_conference}


\appendix
\section{Appendix}
Full results on ISL, OSL, and few-shot experiments are presented here.

\clearpage
\begin{table*}[t]
    \centering
    \renewcommand{\arraystretch}{1.1}
    \resizebox{\textwidth}{!}{
    \begin{tabular}{l c ccc ccc ccc}
        \toprule
          \multirow{2}{*}{\bf Models}& \multirow{2}{*}{\bf Settings}& \multicolumn{3}{c}{\bf k = 2} & \multicolumn{3}{c}{\bf k = 3} & \multicolumn{3}{c}{\bf k = 4}\\
          \cmidrule(lr){3-5} \cmidrule(lr){6-8}  \cmidrule(lr){9-11}
          & & recall  & precision & compatibility & recall  & precision & compatibility & recall  & precision & compatibility \\
         \midrule 
         \multicolumn{11}{c}{\textbf{vocab size = 2}} \\
         \midrule
           \multirow{3}{*}{Llama-3.3 70B} & rules = 1 & 60.00 & 55.00 & 60.00 & 30.00 & 23.33 & 20.00 & 10.00 & 10.00 & 10.00\\
           & rules = 2 & 60.00 & 65.00 & 50.00 & 45.00 & 60.00 & 30.00 & 15.00 & 8.25 & 0.00\\
           & \cellcolor{SeaGreen3!15}rules = 3 & \cellcolor{SeaGreen3!15}53.33 & \cellcolor{SeaGreen3!15}68.33 & \cellcolor{SeaGreen3!15}20.00 & \cellcolor{SeaGreen3!15}30.00 & \cellcolor{SeaGreen3!15}46.67 & \cellcolor{SeaGreen3!15}10.00 & \cellcolor{SeaGreen3!15}16.67 & \cellcolor{SeaGreen3!15}8.54 & \cellcolor{SeaGreen3!15}0.00\\
           \hdashline
           \multirow{3}{*}{Llama-3.1 405B} & rules = 1 & 30.00 & 25.00 & 30.00 & 50.00 & 35.00 & 30.00 & 20.00 & 15.00 & 10.00\\
           & rules = 2 & 55.00 & 50.00 & 40.00 & 10.00 & 6.67 & 0.00 & 10.00 & 7.50 & 0.00\\
           & \cellcolor{SeaGreen3!15}rules = 3 & \cellcolor{SeaGreen3!15}56.67 & \cellcolor{SeaGreen3!15}44.17 & \cellcolor{SeaGreen3!15}20.00 & \cellcolor{SeaGreen3!15}10.00 & \cellcolor{SeaGreen3!15}9.50 & \cellcolor{SeaGreen3!15}0.00 & \cellcolor{SeaGreen3!15}0.00 & \cellcolor{SeaGreen3!15}0.00 & \cellcolor{SeaGreen3!15}0.00\\
           \hdashline
           \multirow{3}{*}{DeepSeek-V3} & rules = 1 & 90.00 & 60.00 & 60.00 & 50.00 & 32.50 & 50.00 & 50.00 & 12.33 & 30.00\\
           & rules = 2 & 80.00 & 60.00 & 40.00 & 40.00 & 19.89 & 10.00 & 15.00 & 3.82 & 0.00\\
           & \cellcolor{SeaGreen3!15}rules = 3 & \cellcolor{SeaGreen3!15}70.00 & \cellcolor{SeaGreen3!15}54.83 & \cellcolor{SeaGreen3!15}40.00 & \cellcolor{SeaGreen3!15}40.00 & \cellcolor{SeaGreen3!15}26.15 & \cellcolor{SeaGreen3!15}0.00 & \cellcolor{SeaGreen3!15}33.33 & \cellcolor{SeaGreen3!15}10.61 & \cellcolor{SeaGreen3!15}0.00\\
           \hdashline
           \multirow{3}{*}{GPT-4o} & rules = 1 & 60.00 & 43.33 & 40.00 & 50.00 & 22.00 & 40.00 & 10.00 & 2.00 & 0.00 \\
           & rules = 2 & 60.00 & 37.50 & 50.00 & 35.00 & 18.43 & 20.00 & 15.00 & 5.63 & 10.00 \\
           & \cellcolor{SeaGreen3!15}rules = 3 & \cellcolor{SeaGreen3!15}73.33 & \cellcolor{SeaGreen3!15}68.33 & \cellcolor{SeaGreen3!15}60.00 & \cellcolor{SeaGreen3!15}36.67 & \cellcolor{SeaGreen3!15}19.30 & \cellcolor{SeaGreen3!15}0.00 & \cellcolor{SeaGreen3!15}13.33 & \cellcolor{SeaGreen3!15}2.50 & \cellcolor{SeaGreen3!15}0.00\\
           \hdashline
           \multirow{3}{*}{o1-mini} & rules = 1 & 50.00 & 45.00 & 50.00 & 70.00 & 45.83 & 40.00 & 30.00 & 16.67 & 10.00 \\
           & rules = 2 & 75.00 & 75.00 & 75.00 & 70.00 & 60.00 & 40.00 & 50.00 & 28.67 & 0.00\\
           & \cellcolor{SeaGreen3!15}rules = 3 & \cellcolor{SeaGreen3!15}66.67 & \cellcolor{SeaGreen3!15}61.67 & \cellcolor{SeaGreen3!15}60.00 & \cellcolor{SeaGreen3!15}43.33 & \cellcolor{SeaGreen3!15}34.00 & \cellcolor{SeaGreen3!15}0.00 & \cellcolor{SeaGreen3!15}40.00 & \cellcolor{SeaGreen3!15}38.52 & \cellcolor{SeaGreen3!15}0.00\\
           \hdashline
           \multirow{3}{*}{o3-mini} & rules = 1 & 100.00 & 100.00 & 100.00 & 100.00 & 100.00 & 100.00 & 80.00 & 58.33 & 40.00\\
           & rules = 2 & 90.00 & 90.00 & 90.00 & 90.00 & 90.67 & 90.00 & 85.00 & 80.00 & 50.00\\
           & \cellcolor{SeaGreen3!15}rules = 3 & \cellcolor{SeaGreen3!15}\textbf{100.00} & \cellcolor{SeaGreen3!15}\textbf{97.50} & \cellcolor{SeaGreen3!15}\textbf{100.00} & \cellcolor{SeaGreen3!15}\textbf{83.33} & \cellcolor{SeaGreen3!15}\textbf{71.83} & \cellcolor{SeaGreen3!15}\textbf{60.00} & \cellcolor{SeaGreen3!15}\textbf{70.00} & \cellcolor{SeaGreen3!15}\textbf{63.81} & \cellcolor{SeaGreen3!15}\textbf{20.00}\\
           \midrule 
           \multicolumn{11}{c}{\textbf{vocab size = 3}} \\
         \midrule
           \multirow{3}{*}{Llama-3.3 70B} & rules = 1 & 70.00 & 60.00 & 60.00 & 20.00 & 20.00 & 20.00 & 20.00 & 8.33 & 10.00\\
           & rules = 2 & 85.00 & 83.33 & 60.00 & 10.00 & 7.50 & 0.00 & 5.00 & 2.50 & 0.00\\
           & \cellcolor{SeaGreen3!30}rules = 3 & \cellcolor{SeaGreen3!30}66.67 & \cellcolor{SeaGreen3!30}74.17 & \cellcolor{SeaGreen3!30}20.00 & \cellcolor{SeaGreen3!30}33.33 & \cellcolor{SeaGreen3!30}35.36 & \cellcolor{SeaGreen3!30}0.00 & \cellcolor{SeaGreen3!30}6.67 & \cellcolor{SeaGreen3!30}3.43 & \cellcolor{SeaGreen3!30}0.00 \\
           \hdashline
           \multirow{3}{*}{Llama-3.1 405B} & rules = 1 & 20.00 & 10.00 & 10.00 & 20.00 & 10.00 & 10.00 & 10.00 & 10.00 & 10.00\\
           & rules = 2 & 45.00 & 32.58 & 20.00 & 10.00 & 7.50 & 0.00 & 0.00 & 0.00 & 0.00\\
           & \cellcolor{SeaGreen3!30}rules = 3 & \cellcolor{SeaGreen3!30}50.00 & \cellcolor{SeaGreen3!30}38.45 & \cellcolor{SeaGreen3!30}20.00 & \cellcolor{SeaGreen3!30}6.67 & \cellcolor{SeaGreen3!30}3.10 & \cellcolor{SeaGreen3!30}0.00 & \cellcolor{SeaGreen3!30}10.00 & \cellcolor{SeaGreen3!30}5.27 & \cellcolor{SeaGreen3!30}0.00\\
           \hdashline
           \multirow{3}{*}{DeepSeek-V3} & rules = 1 & 70.00 & 65.00 & 60.00 & 70.00 & 45.00 & 60.00 & 50.00 & 13.93 & 50.00\\
           & rules = 2 & 80.00 & 56.00 & 40.00 & 40.00 & 13.23 & 0.00 & 25.00 & 1.89 & 0.00\\
           & \cellcolor{SeaGreen3!30}rules = 3 & \cellcolor{SeaGreen3!30}80.00 & \cellcolor{SeaGreen3!30}60.76 & \cellcolor{SeaGreen3!30}50.00 & \cellcolor{SeaGreen3!30}56.67 & \cellcolor{SeaGreen3!30}21.64 & \cellcolor{SeaGreen3!30}0.00 & \cellcolor{SeaGreen3!30}40.00 & \cellcolor{SeaGreen3!30}5.88 & \cellcolor{SeaGreen3!30}0.00\\
           \hdashline
           \multirow{3}{*}{GPT-4o} & rules = 1 & 50.00 & 33.33 & 50.00 & 50.00 & 18.33 & 40.00 & 20.00 & 4.17 & 10.00\\
           & rules = 2 & 60.00 & 40.42 & 30.00 & 25.00 & 6.00 & 0.00 & 30.00 & 6.39 & 0.00 \\
           & \cellcolor{SeaGreen3!30}rules = 3 & \cellcolor{SeaGreen3!30}66.67 & \cellcolor{SeaGreen3!30}64.33 & \cellcolor{SeaGreen3!30}40.00 & \cellcolor{SeaGreen3!30}30.00 & \cellcolor{SeaGreen3!30}9.61 & \cellcolor{SeaGreen3!30}0.00 & \cellcolor{SeaGreen3!30}10.00 & \cellcolor{SeaGreen3!30}1.72 & \cellcolor{SeaGreen3!30}0.00\\
           \hdashline
           \multirow{3}{*}{o1-mini} & rules = 1 & 80.00 & 80.00 & 80.00 & 50.00 & 43.33 & 40.00 & 30.00 & 0.00 & 0.00\\
           & rules = 2 & 90.00 & 90.00 & 80.00 & 40.0 & 25.11 & 10.00 & 55.00 & 24.82 & 10.00\\
           & \cellcolor{SeaGreen3!30}rules = 3 & \cellcolor{SeaGreen3!30}80.00 & \cellcolor{SeaGreen3!30}77.33 & \cellcolor{SeaGreen3!30}60.00 & \cellcolor{SeaGreen3!30}63.33 & \cellcolor{SeaGreen3!30}36.25 & \cellcolor{SeaGreen3!30}10.00 & \cellcolor{SeaGreen3!30}30.00 & \cellcolor{SeaGreen3!30}20.44 & \cellcolor{SeaGreen3!30}0.00\\
           \hdashline
           \multirow{3}{*}{o3-mini} & rules = 1 & 100.00 & 100.000 & 100.00 & 100.00 & 95.00 & 90.00 & 90.00 & 78.33 & 70.00\\
           & rules = 2 & 100.00 & 100.000 & 100.00 & 95.00 & 91.67 & 80.00 & 75.00 & 75.00 & 50.00\\
           & \cellcolor{SeaGreen3!30}rules = 3 & \cellcolor{SeaGreen3!30}\textbf{96.67} & \cellcolor{SeaGreen3!30}\textbf{97.50} & \cellcolor{SeaGreen3!30}\textbf{80.00} & \cellcolor{SeaGreen3!30}\textbf{93.33} & \cellcolor{SeaGreen3!30}\textbf{91.67} & \cellcolor{SeaGreen3!30}\textbf{90.00} & \cellcolor{SeaGreen3!30}\textbf{83.33} & \cellcolor{SeaGreen3!30}\textbf{85.17} & \cellcolor{SeaGreen3!30}\textbf{50.00}\\
           \midrule 
           \multicolumn{11}{c}{\textbf{vocab size = 4}} \\
         \midrule
           \multirow{3}{*}{Llama-3.3 70B} & rules = 1 & 60.00 & 60.00 & 60.00 & 30.00 & 30.00 & 30.00 & 10.00 & 10.00 & 10.00\\
           & rules = 2 & 40.00 & 40.00 & 30.00 & 15.00 & 11.67 & 0.00 & 0.00 & 0.00 & 0.00\\
           & \cellcolor{SeaGreen3!50}rules = 3 & \cellcolor{SeaGreen3!50}53.33 & \cellcolor{SeaGreen3!50}68.33 & \cellcolor{SeaGreen3!50}20.00 & \cellcolor{SeaGreen3!50}6.67 & \cellcolor{SeaGreen3!50}5.00 & \cellcolor{SeaGreen3!50}0.00 & \cellcolor{SeaGreen3!50}10.00 & \cellcolor{SeaGreen3!50}5.32 & \cellcolor{SeaGreen3!50}0.00\\
           \hdashline
           \multirow{3}{*}{Llama-3.1 405B} & rules = 1 & 40.00 & 35.00 & 30.00 & 10.00 & 5.00 & 0.00 & 10.00 & 10.00 & 10.00\\
           & rules = 2 & 75.00 & 52.33 & 10.00 & 10.00 & 5.83 & 0.00 & 0.00 & 0.00 & 0.00\\
           & \cellcolor{SeaGreen3!50}rules = 3 & \cellcolor{SeaGreen3!50}40.00 & \cellcolor{SeaGreen3!50}34.33 & \cellcolor{SeaGreen3!50}10.00 & \cellcolor{SeaGreen3!50}13.33 & \cellcolor{SeaGreen3!50}1.54 & \cellcolor{SeaGreen3!50}0.00 & \cellcolor{SeaGreen3!50}10.00 & \cellcolor{SeaGreen3!50}3.75 & \cellcolor{SeaGreen3!50}0.00\\
           \hdashline
           \multirow{3}{*}{DeepSeek-V3} & rules = 1 & 80.00 & 52.50 & 60.00 & 50.00 & 17.00 & 40.00 & 40.00 & 14.58 & 40.00\\
           & rules = 2 & 85.00 & 57.15 & 50.00 & 65.00 & 18.66 & 20.00 & 45.00 & 5.30 & 0.00\\
           & \cellcolor{SeaGreen3!50}rules = 3 & \cellcolor{SeaGreen3!50}76.67 & \cellcolor{SeaGreen3!50}63.12 & \cellcolor{SeaGreen3!50}40.00 & \cellcolor{SeaGreen3!50}50.00 & \cellcolor{SeaGreen3!50}16.05 & \cellcolor{SeaGreen3!50}10.00 & \cellcolor{SeaGreen3!50}13.33 & \cellcolor{SeaGreen3!50}2.46 & \cellcolor{SeaGreen3!50}0.00\\
           \hdashline
           \multirow{3}{*}{GPT-4o} & rules = 1 & 50.00 & 40.00 & 40.00 & 50.00 & 16.67 & 20.00 & 50.00 & 21.67 & 20.00\\
           & rules = 2 & 45.00 & 29.00 & 10.00 & 45.00 & 12.62 & 0.00 & 0.00 & 0.00 & 0.00\\
           & \cellcolor{SeaGreen3!50}rules = 3 & \cellcolor{SeaGreen3!50}56.67 & \cellcolor{SeaGreen3!50}38.62 & \cellcolor{SeaGreen3!50}0.00 & \cellcolor{SeaGreen3!50}33.33 & \cellcolor{SeaGreen3!50}20.60 & \cellcolor{SeaGreen3!50}0.00 & \cellcolor{SeaGreen3!50}10.00 & \cellcolor{SeaGreen3!50}2.67 & \cellcolor{SeaGreen3!50}0.00\\
           \hdashline
           \multirow{3}{*}{o1-mini} & rules = 1 & 80.00 & 70.00 & 80.00 & 60.00 & 50.00 & 50.00 & 40.00 & 15.00 & 10.00\\
           & rules = 2 & 75.00 & 63.33 & 60.00 & 50.00 & 29.93 & 10.00 & 35.00 & 35.00 & 0.00 \\
           & \cellcolor{SeaGreen3!50}rules = 3 & \cellcolor{SeaGreen3!50}93.33 & \cellcolor{SeaGreen3!50}91.67 & \cellcolor{SeaGreen3!50}80.00 & \cellcolor{SeaGreen3!50}46.67 & \cellcolor{SeaGreen3!50}37.72 & \cellcolor{SeaGreen3!50}10.00 & \cellcolor{SeaGreen3!50}36.67 & \cellcolor{SeaGreen3!50}22.09 & \cellcolor{SeaGreen3!50}0.00\\
           \hdashline
           \multirow{3}{*}{o3-mini} & rules = 1 & 100.00 & 100.00 & 100.00 & 100.00 & 95.00 & 100.00 & 60.00 & 60.00 & 60.00\\
           & rules = 2 & 100.00 & 100.00 & 100.00 & 95.00 & 91.67 & 80.00 & 75.00 & 76.67 & 40.00\\
           & \cellcolor{SeaGreen3!50}rules = 3 & \cellcolor{SeaGreen3!50}\textbf{96.67} & \cellcolor{SeaGreen3!50}\textbf{95.00} & \cellcolor{SeaGreen3!50}\textbf{90.00} & \cellcolor{SeaGreen3!50}\textbf{93.33} & \cellcolor{SeaGreen3!50}\textbf{93.33} & \cellcolor{SeaGreen3!50}\textbf{80.00} & \cellcolor{SeaGreen3!50}\textbf{73.33} & \cellcolor{SeaGreen3!50}\textbf{59.58} & \cellcolor{SeaGreen3!50}\textbf{10.00}\\
           %\hdashline
           %o1-preview & \cellcolor{SeaGreen3!50}rules = 3 & -- & -- & -- & -- & -- & -- & \textbf{40.00} & \textbf{56.67} & 0.00 \\
           \bottomrule
    \end{tabular}
    }
    \caption{Input Strictly Local with sample size = 2}
    \label{tab:ISL_main}
\end{table*}

\begin{table*}[t]
    \centering
    \renewcommand{\arraystretch}{1.1}
    \resizebox{14.5cm}{!}{
    \begin{tabular}{l c ccc ccc ccc}
        \toprule
          \multirow{2}{*}{\bf Models}& \multirow{2}{*}{\bf Settings}& \multicolumn{3}{c}{\bf k = 2} & \multicolumn{3}{c}{\bf k = 3} & \multicolumn{3}{c}{\bf k = 4}\\
          \cmidrule(lr){3-5} \cmidrule(lr){6-8}  \cmidrule(lr){9-11}
          & & recall  & precision & compatibility & recall  & precision & compatibility & recall  & precision & compatibility \\
         \midrule 
         \multicolumn{11}{c}{\textbf{vocab size = 2}} \\
         \midrule
           \multirow{3}{*}{Llama-3.3 70B} & rules = 1 & 50.00 & 45.00 & 50.00 & 0.00 & 0.00 & 0.00 & 0.00 & 0.00 & 0.00\\
           & rules = 2 & 25.00 & 25.00 & 20.00 & 10.00 & 8.33 & 10.00 & 5.00 & 10.00 & 0.00\\
           & \cellcolor{SeaGreen3!15}rules = 3 & \cellcolor{SeaGreen3!15}56.67 & \cellcolor{SeaGreen3!15}65.00 & \cellcolor{SeaGreen3!15}0.00 & \cellcolor{SeaGreen3!15}6.67 & \cellcolor{SeaGreen3!15}8.33 & \cellcolor{SeaGreen3!15}0.00 & \cellcolor{SeaGreen3!15}13.33 & \cellcolor{SeaGreen3!15}12.83 & \cellcolor{SeaGreen3!15}0.00\\
           \hdashline
           \multirow{3}{*}{Llama-3.1 405B} & rules = 1 & 70.00 & 45.83 & 70.00 & 30.00 & 9.33 & 10.00 & 10.00 & 1.67 & 10.00\\
           & rules = 2 & 50.00 & 33.33 & 10.00 & 25.00 & 11.39 & 0.00 & 10.00 & 3.00 & 0.00\\
           & \cellcolor{SeaGreen3!15}rules = 3 & \cellcolor{SeaGreen3!15}63.33 & \cellcolor{SeaGreen3!15}53.83 & \cellcolor{SeaGreen3!15}0.00 & \cellcolor{SeaGreen3!15}10.00 & \cellcolor{SeaGreen3!15}6.67 & \cellcolor{SeaGreen3!15}0.00 & \cellcolor{SeaGreen3!15}6.67 & \cellcolor{SeaGreen3!15}5.00 & \cellcolor{SeaGreen3!15}0.00\\
           \hdashline
           \multirow{3}{*}{GPT-4o} & rules = 1 & 30.00 & 12.50 & 30.00 & 30.00 & 10.83 & 10.00 & 10.00 & 5.00 & 10.00\\
           & rules = 2 & 75.00 & 63.17 & 60.00 & 20.00 & 7.42 & 0.00 & 15.00 & 6.35 & 0.00\\
           & \cellcolor{SeaGreen3!15}rules = 3 & \cellcolor{SeaGreen3!15}66.67 & \cellcolor{SeaGreen3!15}60.00 & \cellcolor{SeaGreen3!15}50.00 & \cellcolor{SeaGreen3!15}30.00 & \cellcolor{SeaGreen3!15}19.00 & \cellcolor{SeaGreen3!15}10.00 & \cellcolor{SeaGreen3!15}10.00 & \cellcolor{SeaGreen3!15}4.16 & \cellcolor{SeaGreen3!15}0.00\\
           \hdashline
           \multirow{3}{*}{DeepSeek-V3} & rules = 1 & 100.00 & 75.00 & 70.00 & 50.00 & 32.50 & 40.00 & 40.00 & 12.78 & 40.00\\
           & rules = 2 & 60.00 & 44.17 & 30.00 & 10.00 & 15.00 & 10.00 & 20.00 & 11.94 & 0.00\\
           & \cellcolor{SeaGreen3!15}rules = 3 & \cellcolor{SeaGreen3!15}83.33 & \cellcolor{SeaGreen3!15}77.67 & \cellcolor{SeaGreen3!15}50.00 & \cellcolor{SeaGreen3!15}20.00 & \cellcolor{SeaGreen3!15}12.92 & \cellcolor{SeaGreen3!15}0.00 & \cellcolor{SeaGreen3!15}23.33 & \cellcolor{SeaGreen3!15}13.97 & \cellcolor{SeaGreen3!15}0.00\\
           \hdashline
           \multirow{3}{*}{o1-mini} & rules = 1 & 90.00 & 90.00 & 90.00 & 70.00 & 55.00 & 40.00 & 10.00 & 10.00 & 10.00\\
           & rules = 2 & 80.00 & 80.00 & 80.00 & 60.00 & 60.00 & 50.00 & 65.00 & 53.83 & 10.00\\
           & \cellcolor{SeaGreen3!15}rules = 3 & \cellcolor{SeaGreen3!15}90.00 & \cellcolor{SeaGreen3!15}82.50 & \cellcolor{SeaGreen3!15}50.00 & \cellcolor{SeaGreen3!15}66.67 & \cellcolor{SeaGreen3!15}60.67 & \cellcolor{SeaGreen3!15}20.00 & \cellcolor{SeaGreen3!15}50.00 & \cellcolor{SeaGreen3!15}54.22 & \cellcolor{SeaGreen3!15}10.00\\
           \hdashline
           \multirow{3}{*}{o3-mini} & rules = 1 & 100.00 & 100.00 & 100.00 & 90/00 & 90.00 & 90.00 & 90.00 & 90.00 & 90.00\\
           & rules = 2 & 100.00 & 100.00 & 100.00 & 95.00 & 100.00 & 100.00 & 85.00 & 78.33 & 70.00\\
           & \cellcolor{SeaGreen3!15}rules = 3 & \cellcolor{SeaGreen3!15}\textbf{100.00} & \cellcolor{SeaGreen3!15}\textbf{100.00} & \cellcolor{SeaGreen3!15}\textbf{100.00} & \cellcolor{SeaGreen3!15}\textbf{86.67} & \cellcolor{SeaGreen3!15}\textbf{85.00} & \cellcolor{SeaGreen3!15}\textbf{80.00} & \cellcolor{SeaGreen3!15}\textbf{56.67} & \cellcolor{SeaGreen3!15}\textbf{50.33} & \cellcolor{SeaGreen3!15}\textbf{40.00}\\
           \midrule 
           \multicolumn{11}{c}{\textbf{vocab size = 3}}\\
         \midrule
           \multirow{3}{*}{Llama-3.3 70B} & rules = 1 & 50.00 & 50.00 & 50.00 & 20.00 & 12.50 & 10.00 & 20.00 & 13.33 & 10.00\\
           & rules = 2 & 35.00 & 33.67 & 10.00 & 20.00 & 6.93 & 10.00 & 25.00 & 15.00 & 0.00\\
           & \cellcolor{SeaGreen3!30}rules = 3 & \cellcolor{SeaGreen3!30}40.00 & \cellcolor{SeaGreen3!30}65.00 & \cellcolor{SeaGreen3!30}20.00 & \cellcolor{SeaGreen3!30}20.00 & \cellcolor{SeaGreen3!30}18.33 & \cellcolor{SeaGreen3!30}0.00 & \cellcolor{SeaGreen3!30}10.00 & \cellcolor{SeaGreen3!30}2.78 & \cellcolor{SeaGreen3!30}0.00\\
           \hdashline
           \multirow{3}{*}{Llama-3.1 405B} & rules = 1 & 60.00 & 45.00 & 40.00 & 10.00 & 3.33 & 10.00 & 10.00 & 1.11 & 0.00\\
           & rules = 2 & 30.00 & 20.00 & 0.00 & 15.00 & 13.33 & 0.00 & 5.00 & 0.53 & 0.00\\
           & \cellcolor{SeaGreen3!30}rules = 3 & \cellcolor{SeaGreen3!30}66.67 & \cellcolor{SeaGreen3!30}57.83 & \cellcolor{SeaGreen3!30}30.00 & \cellcolor{SeaGreen3!30}20.00 & \cellcolor{SeaGreen3!30}8.39 & \cellcolor{SeaGreen3!30}0.00 & \cellcolor{SeaGreen3!30}10.00 & \cellcolor{SeaGreen3!30}2.36 & \cellcolor{SeaGreen3!30}0.00\\
           \hdashline
           \multirow{3}{*}{GPT-4o} & rules = 1 & 40.00 & 27.50 & 40.00 & 20.00 & 8.33 & 20.00 & 40.00 & 11.00 & 30.00\\
           & rules = 2 & 55.00 & 46.50 & 10.00 & 45.00 & 25.67 & 0.00 & 30.00 & 6.50 & 10.00\\
           & \cellcolor{SeaGreen3!30}rules = 3 & \cellcolor{SeaGreen3!30}60.00 & \cellcolor{SeaGreen3!30}50.00 & \cellcolor{SeaGreen3!30}10.00 & \cellcolor{SeaGreen3!30}33.33 & \cellcolor{SeaGreen3!30}15.95 & \cellcolor{SeaGreen3!30}0.00 & \cellcolor{SeaGreen3!30}20.00 & \cellcolor{SeaGreen3!30}6.21 & \cellcolor{SeaGreen3!30}0.00\\
           \hdashline
           \multirow{3}{*}{DeepSeek-V3} & rules = 1 & 80.00 & 70.00 & 60.00 & 50.00 & 22.00 & 40.00 & 50.00 & 16.11 & 30.00\\
           & rules = 2 & 90.00 & 60.32 & 60.00 & 70.00 & 13.82 & 20.00 & 30.00 & 5.04 & 0.00\\
           & \cellcolor{SeaGreen3!30}rules = 3 & \cellcolor{SeaGreen3!30}66.67 & \cellcolor{SeaGreen3!30}53.50 & \cellcolor{SeaGreen3!30}40.00 & \cellcolor{SeaGreen3!30}23.33 & \cellcolor{SeaGreen3!30}16.42 & \cellcolor{SeaGreen3!30}0.00 & \cellcolor{SeaGreen3!30}30.00 & \cellcolor{SeaGreen3!30}7.54 & \cellcolor{SeaGreen3!30}\textbf{10.00}\\
           \hdashline
           \multirow{3}{*}{o1-mini} & rules = 1 & 100.00 & 95.00 & 90.00 & 80.00 & 63.33 & 70.00 & 30.00 & 17.50 & 30.00\\
           & rules = 2 & 90.00 & 83.33 & 80.00 & 70.00 & 49.42 & 40.00 & 35.00 & 29.52 & 20.00\\
           & \cellcolor{SeaGreen3!30}rules = 3 & \cellcolor{SeaGreen3!30}\textbf{96.67} & \cellcolor{SeaGreen3!30}\textbf{96.00} & \cellcolor{SeaGreen3!30}\textbf{90.00} & \cellcolor{SeaGreen3!30}70.00 & \cellcolor{SeaGreen3!30}56.15 & \cellcolor{SeaGreen3!30}30.00 & \cellcolor{SeaGreen3!30}50.00 & \cellcolor{SeaGreen3!30}33.58 & \cellcolor{SeaGreen3!30}0.00\\
           \hdashline
           \multirow{3}{*}{o3-mini} & rules = 1 & 100.00 & 100.00 & 100.00 & 90.00 & 90.00 & 90.00 & 80.00 & 80.00 & 80.00\\
           & rules = 2 & 100.00 & 100.00 & 100.00 & 90.00 & 75.15 & 70.00 & 80.00 & 72.50 & 50.00\\
           & \cellcolor{SeaGreen3!30}rules = 3 & \cellcolor{SeaGreen3!30}\textbf{96.67} & \cellcolor{SeaGreen3!30}94.17 & \cellcolor{SeaGreen3!30}\textbf{90.00} & \cellcolor{SeaGreen3!30}\textbf{96.67} & \cellcolor{SeaGreen3!30}\textbf{87.50} & \cellcolor{SeaGreen3!30}\textbf{90.00} & \cellcolor{SeaGreen3!30}\textbf{63.33} & \cellcolor{SeaGreen3!30}\textbf{68.43} & \cellcolor{SeaGreen3!30}\textbf{40.00}\\
           \midrule 
           \multicolumn{11}{c}{\textbf{vocab size = 4}}\\
         \midrule
           \multirow{3}{*}{Llama-3.3 70B} & rules = 1 & 50.00 & 29.00 & 30.00 & 20.00 & 13.33 & 10.00 & 10.00 & 10.00 & 10.00\\
           & rules = 2 & 50.00 & 50.00 & 10.00 & 20.00 & 15.96 & 0.00 & 0.00 & 0.00 & 0.00\\
           & \cellcolor{SeaGreen3!50}rules = 3 & \cellcolor{SeaGreen3!50}50.00 & \cellcolor{SeaGreen3!50}52.50 & \cellcolor{SeaGreen3!50}20.00 & \cellcolor{SeaGreen3!50}6.67 & \cellcolor{SeaGreen3!50}6.00 & \cellcolor{SeaGreen3!50}0.00 & \cellcolor{SeaGreen3!50}10.00 & \cellcolor{SeaGreen3!50}6.33 & \cellcolor{SeaGreen3!50}0.00\\
           \hdashline
           \multirow{3}{*}{Llama-3.1 405B} & rules = 1 & 60.00 & 34.50 & 30.00 & 10.00 & 5.00 & 10.00 & 10.00 & 5.00 & 10.00\\
           & rules = 2 & 50.00 & 29.00 & 0.00 & 10.00 & 3.13 & 0.00 & 10.00 & 2.90 & 0.00\\
           & \cellcolor{SeaGreen3!50}rules = 3 & \cellcolor{SeaGreen3!50}43.33 & \cellcolor{SeaGreen3!50}30.73 & \cellcolor{SeaGreen3!50}20.00 & \cellcolor{SeaGreen3!50}16.67 & \cellcolor{SeaGreen3!50}5.83 & \cellcolor{SeaGreen3!50}0.00 & \cellcolor{SeaGreen3!50}6.67 & \cellcolor{SeaGreen3!50}1.10 & \cellcolor{SeaGreen3!50}0.00\\
           \hdashline
           \multirow{3}{*}{GPT-4o} & rules = 1 & 40.00 & 35.00 & 30.00 & 60.00 & 25.33 & 40.00 & 40.00 & 12.50 & 10.00\\
           & rules = 2 & 75.00 & 45.50 & 30.00 & 55.00 & 20.47 & 10.00 & 20.00 & 9.44 & 0.00\\
           & \cellcolor{SeaGreen3!50}rules = 3 & \cellcolor{SeaGreen3!50}70.00 & \cellcolor{SeaGreen3!50}48.22 & \cellcolor{SeaGreen3!50}20.00 & \cellcolor{SeaGreen3!50}33.33 & \cellcolor{SeaGreen3!50}10.77 & \cellcolor{SeaGreen3!50}0.00 & \cellcolor{SeaGreen3!50}13.33 & \cellcolor{SeaGreen3!50}3.82 & \cellcolor{SeaGreen3!50}0.00\\
           \hdashline
           \multirow{3}{*}{DeepSeek-V3} & rules = 1 & 100.00 & 82.50 & 80.00 & 50.00 & 23.67 & 30.00 & 40.00 & 16.11 & 40.00\\
           & rules = 2 & 70.00 & 50.67 & 30.00 & 25.00 & 8.01 & 0.00 & 15.00 & 3.24 & 0.00\\
           & \cellcolor{SeaGreen3!50}rules = 3 & \cellcolor{SeaGreen3!50}60.00 & \cellcolor{SeaGreen3!50}48.36 & \cellcolor{SeaGreen3!50}30.00 & \cellcolor{SeaGreen3!50}50.00 & \cellcolor{SeaGreen3!50}12.81 & \cellcolor{SeaGreen3!50}20.00 & \cellcolor{SeaGreen3!50}23.33 & \cellcolor{SeaGreen3!50}2.73 & \cellcolor{SeaGreen3!50}0.00\\
           \hdashline
           \multirow{3}{*}{o1-mini} & rules = 1 & 90.00 & 85.00 & 90.00 & 50.00 & 38.33 & 30.00 & 40.00 & 28.33 & 20.00\\
           & rules = 2 & 100.00 & 93.33 & 80.00 & 60.00 & 36.92 & 20.00 & 50.00 & 31.92 & 10.00\\
           & \cellcolor{SeaGreen3!50}rules = 3 & \cellcolor{SeaGreen3!50}80.00 & \cellcolor{SeaGreen3!50}72.25 & \cellcolor{SeaGreen3!50}60.00 & \cellcolor{SeaGreen3!50}70.00 & \cellcolor{SeaGreen3!50}43.04 & \cellcolor{SeaGreen3!50}10.00 & \cellcolor{SeaGreen3!50}43.33 & \cellcolor{SeaGreen3!50}32.12 & \cellcolor{SeaGreen3!50}0.00\\
           \hdashline
           \multirow{3}{*}{o3-mini} & rules = 1 & 100.00 & 100.00 & 100.00 & 100.00 & 100.00 & 100.00 & 60.00 & 60.00 & 60.00\\
           & rules = 2 & 100.00 & 100.00 & 100.00 & 85.00 & 81.67 & 70.00 & 45.00 & 58.33 & 20.00\\
           & \cellcolor{SeaGreen3!50}rules = 3 & \cellcolor{SeaGreen3!50}\textbf{100.00} & \cellcolor{SeaGreen3!50}\textbf{100.00} & \cellcolor{SeaGreen3!50}\textbf{100.00} & \cellcolor{SeaGreen3!50}\textbf{76.67} & \cellcolor{SeaGreen3!50}\textbf{76.67} & \cellcolor{SeaGreen3!50}\textbf{70.00} & \cellcolor{SeaGreen3!50}\textbf{66.67} & \cellcolor{SeaGreen3!50}\textbf{69.17} & \cellcolor{SeaGreen3!50}\textbf{10.00}\\
           %\hdashline
           %o1-preview & \cellcolor{SeaGreen3!50}rules = 3 & -- & -- & -- & -- & -- & -- & 60.00 & \textbf{82.67} & \textbf{20.00} \\
           \bottomrule
    \end{tabular}
    }
    \caption{Left Output Strictly Local with sample size = 2}
    \label{tab:LOSL_main}
\end{table*}

\begin{table*}[t]
    \centering
    \renewcommand{\arraystretch}{1.1}
    \resizebox{14.5cm}{!}{
    \begin{tabular}{l c ccc ccc ccc}
        \toprule
          \multirow{2}{*}{\bf Models}& \multirow{2}{*}{\bf Settings}& \multicolumn{3}{c}{\bf k = 2} & \multicolumn{3}{c}{\bf k = 3} & \multicolumn{3}{c}{\bf k = 4}\\
          \cmidrule(lr){3-5} \cmidrule(lr){6-8}  \cmidrule(lr){9-11}
          & & recall  & precision & compatibility & recall  & precision & compatibility & recall  & precision & compatibility \\
         \midrule 
         \multicolumn{11}{c}{\textbf{vocab size = 2}} \\
         \midrule
           \multirow{3}{*}{Llama-3.3 70B} & rules = 1 & 40.00 & 40.00 & 40.00 & 20.00 & 10.00 & 10.00 & 20.00 & 10.00 & 10.00\\
           & rules = 2 & 40.00 & 40.00 & 30.00 & 15.00 & 18.33 & 10.00 & 20.00 & 18.67 & 10.00\\
           & \cellcolor{SeaGreen3!15}rules = 3 & \cellcolor{SeaGreen3!15}63.33 & \cellcolor{SeaGreen3!15}76.67 & \cellcolor{SeaGreen3!15}30.00 & \cellcolor{SeaGreen3!15}23.33 & \cellcolor{SeaGreen3!15}24.17 & \cellcolor{SeaGreen3!15}0.00 & \cellcolor{SeaGreen3!15}10.00 & \cellcolor{SeaGreen3!15}7.83 & \cellcolor{SeaGreen3!15}0.00\\
           \hdashline
           \multirow{3}{*}{Llama-3.1 405B} & rules = 1 & 20.00 & 8.33 & 0.00 & 40.00 & 18.33 & 0.00 & 0.00 & 0.00 & 0.00\\
           & rules = 2 & 60.00 & 51.67 & 40.00 & 25.00 & 12.50 & 0.00 & 10.00 & 6.25 & 10.00\\
           & \cellcolor{SeaGreen3!15}rules = 3 & \cellcolor{SeaGreen3!15}63.33 & \cellcolor{SeaGreen3!15}54.72 & \cellcolor{SeaGreen3!15}30.00 & \cellcolor{SeaGreen3!15}20.00 & \cellcolor{SeaGreen3!15}14.33 & \cellcolor{SeaGreen3!15}10.00 & \cellcolor{SeaGreen3!15}6.67 & \cellcolor{SeaGreen3!15}4.50 & \cellcolor{SeaGreen3!15}0.00\\
           \hdashline
           \multirow{3}{*}{GPT-4o} & rules = 1 & 50.00 & 30.00 & 30.00 & 10.00 & 5.00 & 10.00 & 0.00 & 0.00 & 0.00\\
           & rules = 2 & 70.00 & 61.67 & 60.00 & 45.00 & 16.93 & 10.00 & 30.00 & 15.83 & 20.00\\
           & \cellcolor{SeaGreen3!15}rules = 3 & \cellcolor{SeaGreen3!15}83.33 & \cellcolor{SeaGreen3!15}71.83 & \cellcolor{SeaGreen3!15}30.00 & \cellcolor{SeaGreen3!15}43.33 & \cellcolor{SeaGreen3!15}24.10 & \cellcolor{SeaGreen3!15}0.00 & \cellcolor{SeaGreen3!15}20.00 & \cellcolor{SeaGreen3!15}10.00 & \cellcolor{SeaGreen3!15}0.00\\
           \hdashline
           \multirow{3}{*}{DeepSeek-V3} & rules = 1 & 60.00 & 38.33 & 20.00 & 40.00 & 14.17 & 20.00 & 20.00 & 7.00 & 10.00\\
           & rules = 2 & 75.00 & 51.67 & 40.00 & 35.00 & 18.33 & 0.00 & 25.00 & 13.00 & 20.00\\
           & \cellcolor{SeaGreen3!15}rules = 3 & \cellcolor{SeaGreen3!15}86.67 & \cellcolor{SeaGreen3!15}72.56 & \cellcolor{SeaGreen3!15}50.00 & \cellcolor{SeaGreen3!15}40.00 & \cellcolor{SeaGreen3!15}27.00 & \cellcolor{SeaGreen3!15}0.00 & \cellcolor{SeaGreen3!15}23.33 & \cellcolor{SeaGreen3!15}3.68 & \cellcolor{SeaGreen3!15}0.00\\
           \hdashline
           \multirow{3}{*}{o1-mini} & rules = 1 & 50.00 & 38.33 & 40.00 & 20.00 & 5.83 & 0.00 & 20.00 & 11.67 & 10.00\\
           & rules = 2 & 55.00 & 38.33 & 30.00 & 35.00 & 17.50 & 0.00 & 15.00 & 8.70 & 0.00\\
           & \cellcolor{SeaGreen3!15}rules = 3 & \cellcolor{SeaGreen3!15}46.67 & \cellcolor{SeaGreen3!15}46.67 & \cellcolor{SeaGreen3!15}10.00 & \cellcolor{SeaGreen3!15}33.33 & \cellcolor{SeaGreen3!15}24.17 & \cellcolor{SeaGreen3!15}0.00 & \cellcolor{SeaGreen3!15}10.00 & \cellcolor{SeaGreen3!15}4.41 & \cellcolor{SeaGreen3!15}0.00\\
           \hdashline
           \multirow{3}{*}{o3-mini} & rules = 1 & 90.00 & 90.00 & 90.00 & 100.00 & 95.00 & 90.00 & 70.00 & 50.00 & 30.00\\
           & rules = 2 & 100.00 & 100.00 & 100.00 & 90.00 & 85.00 & 60.00 & 45.00 & 41.67 & 20.00\\
           & \cellcolor{SeaGreen3!15}rules = 3 & \cellcolor{SeaGreen3!15}\textbf{96.67} & \cellcolor{SeaGreen3!15}\textbf{92.67} & \cellcolor{SeaGreen3!15}\textbf{80.00} & \cellcolor{SeaGreen3!15}\textbf{86.67} & \cellcolor{SeaGreen3!15}\textbf{78.33} & \cellcolor{SeaGreen3!15}\textbf{50.00} & \cellcolor{SeaGreen3!15}\textbf{46.67} & \cellcolor{SeaGreen3!15}\textbf{46.67} & \cellcolor{SeaGreen3!15}\textbf{30.00}\\
         \midrule
           \multirow{3}{*}{Llama-3.3 70B} & rules = 1 & 40.00 & 40.00 & 40.00 & 30.00 & 25.00 & 30.00 & 30.00 & 7.26 & 0.00\\
           & rules = 2 & 30.00 & 60.00 & 10.00 & 20.00 & 10.93 & 0.00 & 25.00 & 19.17 & 10.00\\
           & \cellcolor{SeaGreen3!30}rules = 3 & \cellcolor{SeaGreen3!30}30.00 & \cellcolor{SeaGreen3!30}45.00 & \cellcolor{SeaGreen3!30}0.00 & \cellcolor{SeaGreen3!30}26.67 & \cellcolor{SeaGreen3!30}22.67 & \cellcolor{SeaGreen3!30}10.00 & \cellcolor{SeaGreen3!30}10.00 & \cellcolor{SeaGreen3!30}4.58 & \cellcolor{SeaGreen3!30}0.00\\
           \hdashline
           \multirow{3}{*}{Llama-3.1 405B} & rules = 1 & 20.00 & 11.43 & 10.00 & 10.00 & 5.00 & 0.00 & 20.00 & 4.17 & 0.00\\
           & rules = 2 & 30.00 & 20.83 & 10.00 & 15.00 & 4.17 & 0.00 & 10.00 & 3.41 & 0.00\\
           & \cellcolor{SeaGreen3!30}rules = 3 & \cellcolor{SeaGreen3!30}16.67 & \cellcolor{SeaGreen3!30}9.11 & \cellcolor{SeaGreen3!30}0.00 & \cellcolor{SeaGreen3!30}10.00 & \cellcolor{SeaGreen3!30}4.75 & \cellcolor{SeaGreen3!30}0.00 & \cellcolor{SeaGreen3!30}0.00 & \cellcolor{SeaGreen3!30}0.00 & \cellcolor{SeaGreen3!30}0.00\\
            \hdashline
           \multirow{3}{*}{GPT-4o} & rules = 1 & 60.00 & 50.00 & 40.00 & 50.00 & 27.50 & 20.00 & 50.00 & 15.33 & 20.00\\
           & rules = 2 & 60.00 & 41.67 & 30.00 & 50.00 & 27.50 & 20.00 & 30.00 & 8.82 & 0.00\\
           & \cellcolor{SeaGreen3!30}rules = 3 & \cellcolor{SeaGreen3!30}66.67 & \cellcolor{SeaGreen3!30}57.83 & \cellcolor{SeaGreen3!30}30.00 & \cellcolor{SeaGreen3!30}40.00 & \cellcolor{SeaGreen3!30}21.88 & \cellcolor{SeaGreen3!30}0.00 & \cellcolor{SeaGreen3!30}13.33 & \cellcolor{SeaGreen3!30}6.33 & \cellcolor{SeaGreen3!30}0.00\\
           \hdashline
           \multirow{3}{*}{DeepSeek-V3} & rules = 1 & 80.00 & 60.83 & 60.00 & 50.00 & 29.17 & 40.00 & 40.00 & 9.42 & 20.00\\
           & rules = 2 & 80.00 & 63.19 & 50.00 & 55.00 & 26.67 & 20.00 & 40.00 & 10.10 & 0.00\\
           & \cellcolor{SeaGreen3!30}rules = 3 & \cellcolor{SeaGreen3!30}70.00 & \cellcolor{SeaGreen3!30}42.95 & \cellcolor{SeaGreen3!30}50.00 & \cellcolor{SeaGreen3!30}23.33 & \cellcolor{SeaGreen3!30}15.28 & \cellcolor{SeaGreen3!30}0.00 & \cellcolor{SeaGreen3!30}20.00 & \cellcolor{SeaGreen3!30}3.56 & \cellcolor{SeaGreen3!30}0.00\\
           \hdashline
           \multirow{3}{*}{o1-mini} & rules = 1 & 60.00 & 60.00 & 60.00 & 20.00 & 13.33 & 10.00 & 20.00 & 13.33 & 10.00\\
           & rules = 2 & 15.00 & 15.00 & 10.00 & 40.00 & 25.00 & 0.00 & 30.00 & 19.50 & 0.00\\
           & \cellcolor{SeaGreen3!30}rules = 3 & \cellcolor{SeaGreen3!30}53.33 & \cellcolor{SeaGreen3!30}45.83 & \cellcolor{SeaGreen3!30}20.00 & \cellcolor{SeaGreen3!30}30.00 & \cellcolor{SeaGreen3!30}17.63 & \cellcolor{SeaGreen3!30}0.00 & \cellcolor{SeaGreen3!30}13.33 & \cellcolor{SeaGreen3!30}8.43 & \cellcolor{SeaGreen3!30}0.00\\
           \hdashline
           \multirow{3}{*}{o3-mini} & rules = 1 & 90.00 & 90.00 & 90.00 & 100.00 & 95.00 & 100.00 & 50.00 & 40.00 & 30.00\\
           & rules = 2 & 90.00 & 83.33 & 70.00 & 80.00 & 61.67 & 30.00 & 60.00 & 59.50 & 40.00\\
           & \cellcolor{SeaGreen3!30}rules = 3 & \cellcolor{SeaGreen3!30}\textbf{100.00} & \cellcolor{SeaGreen3!30}\textbf{100.00} & \cellcolor{SeaGreen3!30}\textbf{100.00} & \cellcolor{SeaGreen3!30}\textbf{83.33} & \cellcolor{SeaGreen3!30}\textbf{64.11} & \cellcolor{SeaGreen3!30}\textbf{50.00} & \cellcolor{SeaGreen3!30}\textbf{43.33} & \cellcolor{SeaGreen3!30}\textbf{41.83} & \cellcolor{SeaGreen3!30}\textbf{20.00}\\
           \midrule 
           \multicolumn{11}{c}{\textbf{vocab size = 4}}\\
         \midrule
           \multirow{3}{*}{Llama-3.3 70B} & rules = 1 & 40.00 & 25.00 & 30.00 & 40.00 & 23.33 & 20.00 & 10.00 & 10.00 & 10.00\\
           & rules = 2 & 45.00 & 47.33 & 10.00 & 50.00 & 45.83 & 10.00 & 5.00 & 1.67 & 0.00\\
           & \cellcolor{SeaGreen3!50}rules = 3 & \cellcolor{SeaGreen3!50}33.33 & \cellcolor{SeaGreen3!50}39.50 & \cellcolor{SeaGreen3!50}10.00 & \cellcolor{SeaGreen3!50}30.00 & \cellcolor{SeaGreen3!50}23.85 & \cellcolor{SeaGreen3!50}0.00 & \cellcolor{SeaGreen3!50}10.00 & \cellcolor{SeaGreen3!50}10.83 & \cellcolor{SeaGreen3!50}0.00\\
           \hdashline
           \multirow{3}{*}{Llama-3.1 405B} & rules = 1 & 10.00 & 10.00 & 10.00 & 0.00 & 0.00 & 0.00 & 0.00 & 0.00 & 0.00\\
           & rules = 2 & 15.00 & 12.00 & 0.00 & 10.00 & 13.33 & 0.00 & 5.00 & 0.26 & 0.00\\
           & \cellcolor{SeaGreen3!50}rules = 3 & \cellcolor{SeaGreen3!50}33.33 & \cellcolor{SeaGreen3!50}22.67 & \cellcolor{SeaGreen3!50}0.00 & \cellcolor{SeaGreen3!50}3.33 & \cellcolor{SeaGreen3!50}16.67 & \cellcolor{SeaGreen3!50}0.00 & \cellcolor{SeaGreen3!50}13.33 & \cellcolor{SeaGreen3!50}1.85 & \cellcolor{SeaGreen3!50}0.00\\
           \hdashline
           \multirow{3}{*}{GPT-4o} & rules = 1 & 70.00 & 49.17 & 50.00 & 60.00 & 19.50 & 50.00 & 50.00 & 23.10 & 10.00\\
           & rules = 2 & 80.00 & 78.10 & 40.00 & 40.00 & 16.02 & 0.00 & 20.00 & 10.00 & 0.00\\
           & \cellcolor{SeaGreen3!50}rules = 3 & \cellcolor{SeaGreen3!50}66.67 & \cellcolor{SeaGreen3!50}43.00 & \cellcolor{SeaGreen3!50}0.00 & \cellcolor{SeaGreen3!50}20.00 & \cellcolor{SeaGreen3!50}11.02 & \cellcolor{SeaGreen3!50}0.00 & \cellcolor{SeaGreen3!50}16.67 & \cellcolor{SeaGreen3!50}6.73 & \cellcolor{SeaGreen3!50}0.00\\
           \hdashline
           \multirow{3}{*}{DeepSeek-V3} & rules = 1 & 80.00 & 65.00 & 70.00 & 50.00 & 18.83 & 20.00 & 60.00 & 19.77 & 40.00\\
           & rules = 2 & 70.00 & 59.17 & 30.00 & 45.00 & 27.79 & 20.00 & 10.00 & 0.88 & 0.00\\
           & \cellcolor{SeaGreen3!50}rules = 3 & \cellcolor{SeaGreen3!50}73.33 & \cellcolor{SeaGreen3!50}60.17 & \cellcolor{SeaGreen3!50}40.00 & \cellcolor{SeaGreen3!50}43.33 & \cellcolor{SeaGreen3!50}13.45 & \cellcolor{SeaGreen3!50}0.00 & \cellcolor{SeaGreen3!50}3.33 & \cellcolor{SeaGreen3!50}0.25 & \cellcolor{SeaGreen3!50}0.00\\
           \hdashline
           \multirow{3}{*}{o1-mini} & rules = 1 & 70.00 & 53.33 & 40.00 & 30.00 & 18.33 & 10.00 & 40.00 & 40.00 & 40.00\\
           & rules = 2 & 70.00 & 66.67 & 50.00 & 50.00 & 47.50 & 20.00 & 40.00 & 34.00 & 0.00\\
           & \cellcolor{SeaGreen3!50}rules = 3 & \cellcolor{SeaGreen3!50}90.00 & \cellcolor{SeaGreen3!50}79.17 & \cellcolor{SeaGreen3!50}50.00 & \cellcolor{SeaGreen3!50}23.33 & \cellcolor{SeaGreen3!50}23.33 & \cellcolor{SeaGreen3!50}0.00 & \cellcolor{SeaGreen3!50}26.67 & \cellcolor{SeaGreen3!50}17.58 & \cellcolor{SeaGreen3!50}0.00\\
           \hdashline
           \multirow{3}{*}{o3-mini} & rules = 1 & 90.00 & 90.00 & 90.00 & 100.00 & 93.33 & 90.00 & 50.00 & 50.00 & 50.00\\
           & rules = 2 & 100.00 & 100.00 & 100.00 & 80.00 & 75.00 & 60.00 & 70.00 & 69.17 & 40.00\\
           & \cellcolor{SeaGreen3!50}rules = 3 & \cellcolor{SeaGreen3!50}\textbf{100.00} & \cellcolor{SeaGreen3!50}\textbf{100.00} & \cellcolor{SeaGreen3!50}\textbf{100.00} & \cellcolor{SeaGreen3!50}\textbf{76.67} & \cellcolor{SeaGreen3!50}\textbf{78.33} & \cellcolor{SeaGreen3!50}\textbf{50.00} & \cellcolor{SeaGreen3!50}\textbf{63.33} & \cellcolor{SeaGreen3!50}\textbf{62.00} & \cellcolor{SeaGreen3!50}\textbf{30.00}\\
           %\hdashline
           %o1-preview & \cellcolor{SeaGreen3!50}rules = 3 & -- & -- & -- & -- & -- & -- & 13.20 & 15.00 & 0.00 \\
           \bottomrule
    \end{tabular}
    }
    \caption{Right Output Strictly Local with sample size = 2}
    \label{tab:ROSL_main}
\end{table*}

\begin{table*}[t]
    \centering
    \renewcommand{\arraystretch}{1.1}
    \resizebox{14.5cm}{!}{
    \begin{tabular}{l c ccc ccc ccc}
        \toprule
          \multirow{2}{*}{\bf Models}& \multirow{2}{*}{\bf Settings}& \multicolumn{3}{c}{\bf k = 2} & \multicolumn{3}{c}{\bf k = 3} & \multicolumn{3}{c}{\bf k = 4}\\
          \cmidrule(lr){3-5} \cmidrule(lr){6-8}  \cmidrule(lr){9-11}
          & & recall  & precision & compatibility & recall  & precision & compatibility & recall  & precision & compatibility \\
         \midrule \multicolumn{11}{c}{\textbf{vocab size = 2}} \\
         \midrule
           \multirow{4}{*}{rules = 1} & 0-shot & 60.00 & 55.00 & 60.00 & 30.00 & 23.33 & 20.00 & 10.00 & 10.00 & 10.00  \\
           & 1-shot & 60.00 & 60.00 & 60.00 & 50.00 & 35.00 & 40.00 & 10.00 & 5.00 & 0.00\\
           & 2-shot & 70.00 & 70.00 & 70.00 & 70.00 & 50.00 & 60.00 & 20.00 & 10.00 & 10.00\\
           & 3-shot & 80.00 & 80.00 & 80.00 & 60.00 & 55.00 & 60.00 & 10.00 & 5.00 & 10.00\\
           \hdashline
           \multirow{4}{*}{rules = 2} &  0-shot & 60.00 & 65.00 & 50.00 & 45.00 & 60.00 & 30.00 & 15.00 & 8.25 & 0.00\\
           & 1-shot & 65.00 & 70.00 & 60.00 & 40.00 & 53.00 & 30.00 & 20.00 & 18.33 & 10.00\\
           & 2-shot & 85.00 & 85.00 & 80.00 & 45.00 & 56.67 & 20.00 & 25.00 & 23.33 & 0.00\\
           & 3-shot & 60.00 & 65.00 & 40.00 & 35.00 & 36.67 & 10.00 & 20.00 & 20.00 & 0.00\\
           \hdashline
          \multirow{4}{*}{rules = 3} & 0-shot & 53.33 & 68.33 & 20.00 & 30.00 & 46.67 & 10.00 & 16.67 & 8.54 & 0.00\\
          & 1-shot & 76.67 & 83.33 & 60.00 & 43.33 & 57.67 & 0.00 & 20.00 & 18.33 & 0.00\\
           & 2-shot & 86.67 & 86.67 & 60.00 & 26.67 & 28.33 & 0.00 & 13.33 & 16.67 & 0.00\\
           & 3-shot & 90.00 & 93.33 & 70.00 & 46.67 & 52.50 & 20.00 & 16.67 & 21.17 & 0.00\\
           \midrule \multicolumn{11}{c}{\textbf{vocab size = 3}} \\
           \midrule
           \multirow{4}{*}{rules = 1} & 0-shot & 70.00 & 60.00 & 60.00 & 20.00 & 20.00 & 20.00 & 20.00 & 8.33 & 10.00\\
           & 1-shot & 90.00 & 90.00 & 90.00 & 50.00 & 50.00 & 50.00 & 30.00 & 30.00 & 30.00\\
           & 2-shot & 70.00  & 70.00  & 70.00 & 30.00 & 20.00 & 20.00 & 10.00 & 10.00 & 10.00\\
           & 3-shot & 40.00 & 40.00 & 40.00 & 40.00 & 35.00 & 40.00 & 30.00 & 18.33 & 20.00\\
           \hdashline
           \multirow{4}{*}{rules = 2} &  0-shot & 85.00 & 83.33 & 60.00 & 10.00 & 7.50 & 0.00 & 5.00 & 2.50 & 0.00\\
           & 1-shot & 90.00 & 95.00 & 80.00 & 10.00 & 13.33 & 0.00 & 5.00 & 2.50 & 0.00\\
           & 2-shot & 65.00 & 65.00 & 40.00 & 30.00 & 28.33 & 10.00 & 5.00 & 2.00 & 0.00\\
           & 3-shot & 65.00 & 75.00 & 30.00 & 5.00 & 3.33 & 0.00 & 5.00 & 2.50 & 0.00\\
           \hdashline
          \multirow{4}{*}{rules = 3} & 0-shot & 66.67 & 74.17 & 20.00 & 33.33 & 35.36 & 0.00 & 6.67 & 3.43 & 0.00\\
          & 1-shot & 70.00 & 69.17 & 40.00 & 20.00 & 22.50 & 0.00 & 10.00 & 13.33 & 0.00\\
           & 2-shot & 76.67 & 76.67 & 50.00 & 33.33 & 43.33 & 0.00 & 10.00 & 13.33 & 0.00\\
           & 3-shot & 60.00 & 60.83 & 10.00 & 23.33 & 31.67 & 0.00 & 16.67 & 19.76 & 0.00\\
           \midrule \multicolumn{11}{c}{\textbf{vocab size = 4}} \\
           \midrule
           \multirow{4}{*}{rules = 1} & 0-shot & 60.00 & 60.00 & 60.00 & 30.00 & 30.00 & 30.00 & 10.00 & 10.00 & 10.00\\
           & 1-shot & 40.00 & 40.00 & 40.00 & 50.00 & 31.67 & 50.00 & 20.00 & 13.33 & 20.00\\
           & 2-shot & 70.00 & 57.00 & 60.00 & 30.00 & 25.00 & 30.00 & 10.00 & 5.00 & 0.00\\
           & 3-shot & 60.00 & 60.00 & 60.00 & 20.00 & 15.00 & 20.00 & 10.00 & 10.00 & 10.00\\
           \hdashline
           \multirow{4}{*}{rules = 2} &  0-shot & 40.00 & 40.00 & 30.00 & 15.00 & 11.67 & 0.00 & 0.00 & 0.00 & 0.00\\
           & 1-shot & 60.00 & 60.00 & 30.00 & 15.00 & 23.33 & 0.00 & 0.00 & 0.00 & 0.00\\
           & 2-shot & 80.00 & 90.00 & 60.00 & 15.00 & 12.50 & 0.00 & 15.00 & 10.67 & 0.00\\
           & 3-shot & 70.00 & 70.00 & 70.00 & 15.00 & 18.33 & 0.00 & 10.00 & 10.00 & 0.00\\
           \hdashline
          \multirow{4}{*}{rules = 3} & 0-shot & 53.33 & 68.33 & 20.00 & 6.67 & 5.00 & 0.00 & 10.00 & 5.32 & 0.00\\
          & 1-shot & 66.67 & 70.83 & 30.00 & 30.00 & 47.50 & 0.00 & 3.00 & 5.00 & 0.00\\
           & 2-shot & 66.67 & 73.33 & 30.00 & 30.00 & 55.00 & 0.00 & 3.33 & 3.33 & 0.00\\
           & 3-shot & 60.00 & 71.67 & 10.00 & 20.00 & 39.24 & 0.00 & 0.00 & 0.00 & 0.00\\
           \bottomrule
    \end{tabular}
    }
    \caption{Input Strictly Local with sample size = 2 with few-shot example}
    \label{tab:few_shot_ISL}
\end{table*}


\begin{table*}[t]
    \centering
    \renewcommand{\arraystretch}{1.1}
    \resizebox{14.5cm}{!}{
    \begin{tabular}{l c ccc ccc ccc}
        \toprule
          \multirow{2}{*}{\bf Models}& \multirow{2}{*}{\bf Settings}& \multicolumn{3}{c}{\bf k = 2} & \multicolumn{3}{c}{\bf k = 3} & \multicolumn{3}{c}{\bf k = 4}\\
          \cmidrule(lr){3-5} \cmidrule(lr){6-8}  \cmidrule(lr){9-11}
          & & recall  & precision & compatibility & recall  & precision & compatibility & recall  & precision & compatibility \\
         \midrule \multicolumn{11}{c}{\textbf{vocab size = 2}} \\
         \midrule
           \multirow{4}{*}{rules = 1} & 0-shot & 50.00 & 45.00 & 50.00 & 0.00 & 0.00 & 0.00 & 0.00 & 0.00 & 0.00\\
           & 1-shot & 80.00 & 80.00 & 80.00 & 40.00 & 33.33 & 30.00 & 20.00 & 5.00 & 20.00\\
           & 2-shot & 80.00 & 75.00 & 70.00 & 30.00 & 25.00 & 30.00 & 30.00 & 13.33 & 10.00\\
           & 3-shot & 80.00 & 80.00 & 80.00 & 20.00 & 15.00 & 20.00 & 20.00 & 15.00 & 20.00\\
           \hdashline
           \multirow{4}{*}{rules = 2} &  0-shot & 25.00& 25.00 & 25.00 & 10.00 & 8.33 & 10.00 & 5.00 & 10.00 & 0.00\\
           & 1-shot & 80.00 & 85.00 & 70.00 & 30.00 & 30.83 & 10.00 & 30.00 & 19.00 & 10.00\\
           & 2-shot & 85.00 & 85.00 & 80.00 & 20.00 & 21.67 & 10.00 & 25.00 & 27.90 & 0.00\\
           & 3-shot & 75.00 & 80.00 & 60.00 & 25.00 & 20.83 & 10.00 & 20.00 & 20.83 & 0.00\\
           \hdashline
          \multirow{4}{*}{rules = 3} & 0-shot & 56.67 & 65.00 & 0.00 & 6.67 & 8.33 & 0.00 & 13.33 & 12.83 & 0.00\\
          & 1-shot & 80.00 & 80.00 & 80.00 & 40.00 & 42.00 & 0.00 & 10.00 & 11.67 & 0.00\\
           & 2-shot & 80.00 & 75.00 & 70.00 & 33.33 & 48.33 & 0.00 & 13.33 & 28.33 & 0.00\\
           & 3-shot & 80.00 & 80.00 & 80.00 & 33.33 & 39.17 & 0.00 & 16.67 & 26.67 & 0.00\\
           \midrule \multicolumn{11}{c}{\textbf{vocab size = 3}} \\
           \midrule
           \multirow{4}{*}{rules = 1} & 0-shot & 50.00 & 50.00 & 50.00 & 20.00 & 12.50 & 10.00 & 20.00 & 13.33 & 10.00\\
           & 1-shot & 100.00 & 100.00 & 100.00 & 40.00 & 23.33 & 40.00 & 0.00 & 0.00 & 0.00\\
           & 2-shot & 70.00 & 70.00 & 70.00 & 30.00 & 23.33 & 20.00 & 10.00 & 5.00 & 0.00\\
           & 3-shot & 80.00 & 75.00 & 80.00 & 20.00 & 20.00 & 20.00 & 20.00 & 20.00 & 20.00\\
           \hdashline
           \multirow{4}{*}{rules = 2} &  0-shot & 35.00 & 33.67 & 10.00 & 20.00 & 6.93 & 10.00 & 25.00 & 15.00 & 0.00\\
           & 1-shot & 80.00 & 76.67 & 80.00 & 30.00 & 30.33 & 10.00 & 10.00 & 15.00 & 0.00\\
           & 2-shot & 50.00 & 55.00 & 30.00 & 30.00 & 38.33 & 0.00 & 25.00 & 21.67 & 0.00\\
           & 3-shot & 70.00 & 70.00 & 70.00 & 20.00 & 30.00 & 0.00 & 20.00 & 35.00 & 0.00\\
           \hdashline
          \multirow{4}{*}{rules = 3} & 0-shot & 40.00 & 65.00 & 20.00 & 20.00 & 18.33 & 0.00 & 10.00 & 2.78 & 0.00\\
          & 1-shot & 70.00 & 66.67 & 40.00 & 30.00 & 24.83 & 0.00 & 10.00 & 20.30 & 10.00\\
           & 2-shot & 83.33 & 90.00 & 60.00 & 33.33 & 40.83 & 0.00 & 30.00 & 43.33 & 0.00\\
           & 3-shot & 70.00 & 78.33 & 30.00 & 23.33 & 44.50 & 0.00 & 16.67 & 18.33 & 0.00\\
           \midrule \multicolumn{11}{c}{\textbf{vocab size = 4}} \\
           \midrule
           \multirow{4}{*}{rules = 1} & 0-shot & 50.00 & 29.00 & 30.00 & 20.00 & 13.33 & 10.00 & 10.00 & 10.00 & 10.00\\
           & 1-shot & 50.00 & 50.00 & 50.00 & 50.00 & 50.00 & 50.00 & 20.00 & 15.00 & 20.00\\
           & 2-shot & 60.00 & 60.00 & 60.00 & 10.00 & 10.00 & 10.00 & 20.00 & 20.00 & 20.00\\
           & 3-shot & 20.00 & 20.00 & 20.00 & 30.00 & 25.00 & 30.00 & 20.00 & 20.00 & 20.00\\
           \hdashline
           \multirow{4}{*}{rules = 2} &  0-shot & 50.00 & 50.00 & 10.00 & 20.00 & 15.96 & 0.00 & 0.00 & 0.00 & 0.00\\
           & 1-shot & 55.00 & 56.67 & 30.00 & 20.00 & 28.33 & 0.00 & 0.00 & 0.00 & 0.00\\
           & 2-shot & 55.00 & 50.00 & 20.00 & 35.00 & 55.00 & 10.00 & 5.00 & 10.00 & 0.00\\
           & 3-shot & 10.00 & 20.00 & 0.00 & 30.00 & 33.33 & 10.00 & 10.00 & 20.00 & 0.00\\
           \hdashline
          \multirow{4}{*}{rules = 3} & 0-shot & 50.00 & 52.50 & 20.00 & 6.67 & 6.00 & 0.00 & 10.00 & 6.33 & 0.00\\
          & 1-shot & 56.67 & 68.33 & 20.00 & 20.00 & 36.25 & 0.00 & 16.7 & 22.83 & 0.00\\
           & 2-shot & 66.67 & 67.50 & 50.00 & 16.67 & 26.67 & 0.00 & 6.67 & 15.00 & 0.00\\
           & 3-shot & 3.33 & 3.33 & 0.00 & 23.33 & 40.83 & 0.00 & 3.33 & 3.33 & 0.00\\
           \bottomrule
    \end{tabular}
    }
    \caption{Left Output Strictly Local with sample size = 2 with few-shot example}
    \label{tab:few_shot_LOSL}
\end{table*}


\begin{table*}[t]
    \centering
    \renewcommand{\arraystretch}{1.1}
    \resizebox{14.5cm}{!}{
    \begin{tabular}{l c ccc ccc ccc}
        \toprule
          \multirow{2}{*}{\bf Models}& \multirow{2}{*}{\bf Settings}& \multicolumn{3}{c}{\bf k = 2} & \multicolumn{3}{c}{\bf k = 3} & \multicolumn{3}{c}{\bf k = 4}\\
          \cmidrule(lr){3-5} \cmidrule(lr){6-8}  \cmidrule(lr){9-11}
          & & recall  & precision & compatibility & recall  & precision & compatibility & recall  & precision & compatibility \\
         \midrule \multicolumn{11}{c}{\textbf{vocab size = 2}} \\
         \midrule
           \multirow{4}{*}{rules = 1} & 0-shot & 40.00 & 40.00 & 40.00 & 20.00 & 10.00 & 10.00 & 20.00 & 10.00 & 10.00\\
           & 1-shot & 60.00 & 60.00 & 60.00 & 50.00 & 40.00 & 50.00 & 30.00 & 18.33 & 10.00\\
           & 2-shot & 60.00 & 60.00 & 60.00 & 40.00 & 35.00 & 40.00 & 30.00 & 25.00 & 20.00\\
           & 3-shot & 60.00 & 60.00 & 60.00 & 50.00 & 40.00 & 50.00 & 10.00 & 2.50 & 10.00\\
           \hdashline
           \multirow{4}{*}{rules = 2} &  0-shot & 40.00 & 40.00 & 30.00 & 15.00 & 18.33 & 10.00 & 20.00 & 18.67 & 10.00\\
           & 1-shot & 60.00 & 60.00 & 50.00 & 40.00 & 42.50 & 20.00 & 30.00 & 33.33 & 0.00\\
           & 2-shot & 70.00 & 80.00 & 60.00 & 45.00 & 43.33 & 30.00 & 20.00 & 27.50 & 0.00\\
           & 3-shot & 90.00 & 90.00 & 90.00 & 45.00 & 41.67 & 30.00 & 15.00 & 15.00 & 0.00\\
           \hdashline
          \multirow{4}{*}{rules = 3} & 0-shot & 63.33 & 76.67 & 30.00 & 23.33 & 24.17 & 0.00 & 10.00 & 7.83 & 0.00\\
          & 1-shot & 60.00 & 60.00 & 60.00 & 40.00 & 47.83 & 0.00 & 20.00 & 24.17 & 0.00\\
           & 2-shot & 60.00 & 60.00 & 60.00 & 50.00 & 48.33 & 20.00 & 16.67 & 18.33 & 0.00\\
           & 3-shot & 60.00 & 60.00 & 60.00 & 43.33 & 46.67 & 10.00 & 16.67 & 25.00 & 0.00\\
           \midrule \multicolumn{11}{c}{\textbf{vocab size = 3}} \\
           \midrule
           \multirow{4}{*}{rules = 1} & 0-shot & 40.00 & 40.00 & 40.00 & 30.00 & 25.00 & 30.00 & 30.00 & 7.26 & 0.00\\
           & 1-shot & 70.00 & 65.00 & 60.00 & 40.00 & 35.00 & 40.00 & 20.00 & 15.00 & 10.00\\
           & 2-shot & 80.00 & 75.00 & 70.00 & 50.00 & 50.00 & 50.00 & 30.00 & 18.33 & 30.00\\
           & 3-shot & 70.00 & 65.00 & 60.00 & 40.00 & 35.00 & 40.00 & 20.00 & 15.00 & 10.00\\
           \hdashline
           \multirow{4}{*}{rules = 2} &  0-shot & 30.00 & 60.00 & 10.00 & 20.00 & 10.93 & 0.00 & 25.00 & 19.17 & 10.00\\
           & 1-shot & 65.00 & 70.00 & 40.00 & 30.00 & 45.00 & 10.00 & 15.00 & 11.67 & 0.00\\
           & 2-shot & 60.00 & 70.00 & 40.00 & 50.00 & 44.17 & 10.00 & 20.00 & 20.00 & 0.00\\
           & 3-shot & 65.00 & 70.00 & 40.00 & 30.00 & 45.00 & 10.00 & 15.00 & 11.67 & 0.00\\
           \hdashline
          \multirow{4}{*}{rules = 3} & 0-shot & 30.00 & 45.00 & 0.00 & 26.67 & 22.67 & 10.00 & 10.00 & 4.58 & 0.00\\
          & 1-shot & 80.00 & 80.83 & 40.00 & 30.00 & 37.59 & 0.00 & 13.33 & 11.00 & 0.00\\
           & 2-shot & 73.33 & 71.67 & 30.00 & 26.67 & 34.50 & 0.00 & 33.33 & 47.00 & 0.00\\
           & 3-shot & 80.00 & 80.83 & 40.00 & 30.00 & 37.60 & 0.00 & 13.33 & 11.00 & 0.00\\
           \midrule \multicolumn{11}{c}{\textbf{vocab size = 4}} \\
           \midrule
           \multirow{4}{*}{rules = 1} & 0-shot & 40.00 & 25.00 & 30.00 & 40.00 & 23.33 & 20.00 & 10.00 & 10.00 & 10.00\\
           & 1-shot & 60.00 & 60.00 & 60.00 & 50.00 & 31.67 & 50.00 & 10.00 & 10.00 & 10.00\\
           & 2-shot & 80.00 & 68.33 & 80.00 & 60.00 & 33.33 & 30.00 & 20.00 & 20.00 & 20.00\\
           & 3-shot & 70.00 & 70.00 & 70.00 & 40.00 & 19.50 & 40.00 & 20.00 & 20.00 & 20.00\\
           \hdashline
           \multirow{4}{*}{rules = 2} &  0-shot & 45.00 & 47.33 & 10.00 & 50.00 & 45.83 & 10.00 & 5.00 & 1.67 & 0.00\\
           & 1-shot & 80.00 & 70.00 & 40.00 & 45.00 & 61.67 & 10.00 & 5.00 & 10.00 & 0.00\\
           & 2-shot & 65.00 & 65.00 & 40.00 & 45.00 & 52.50 & 20.00 & 0.00 & 0.00 & 0.00\\
           & 3-shot & 75.00 & 75.00 & 70.00 & 45.00 & 48.33 & 20.00 & 15.00 & 30.00 & 0.00\\
           \hdashline
          \multirow{4}{*}{rules = 3} & 0-shot & 33.33 & 39.50 & 10.00 & 30.00 & 23.85 & 0.00 & 10.00 & 10.83 & 0.00\\
          & 1-shot & 66.67 & 71.83 & 30.00 & 23.33 & 30.83 & 0.00 & 3.33 & 2.50 & 0.00\\
           & 2-shot & 83.33 & 86.50 & 50.00 & 26.67 & 37.50 & 0.00 & 13.33 & 28.83 & 0.00\\
           & 3-shot & 76.67 & 80.00 & 50.00 & 40.00 & 51.67 & 10.00 & 13.33 & 20.00 & 0.00\\
           \bottomrule
    \end{tabular}
    }
    \caption{Right Output Strictly Local with sample size = 2 with few-shot example}
    \label{tab:few_shot_ROSL}
\end{table*}


\end{document}
