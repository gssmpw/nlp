Inverse problems arise in various scientific and engineering disciplines, including medical imaging, which are crucial in reconstructing hidden properties from indirect measurements. One important application is elastography \cite{ophir1991elastography, doyley2012model}, a technique used to infer the mechanical properties of soft tissues, aiding in disease diagnosis and treatment planning. Elastography data acquisition typically involves applying mechanical forces to tissue and measuring the resulting displacements using imaging modalities such as ultrasound, magnetic resonance imaging (MRI) \cite{muthupillai1995magnetic}, or optical coherence tomography \cite{khalil2005tissue}.

Solving the inverse problem in elastography requires computational methods to extract meaningful mechanical properties from noisy and often incomplete displacement (or strain) data. Broadly, these methods fall into Bayesian and non-Bayesian approaches. 
Non-Bayesian approaches focus on finding a single best-fit solution. Physics-Informed Neural Networks (PINNs) \cite{raissi2019physics} integrate physical laws into deep learning models to enforce consistency with governing equations, offering flexibility in handling complex tissue structures, though at the cost of high computational demands and sensitivity to training data quality. Traditional deterministic optimization methods \cite{vogel2002computational}, such as finite element-based approaches, directly minimize a cost function representing the difference between observed and predicted data. These well-established methods lack uncertainty quantification and may converge to local minima.
In contrast, Bayesian techniques, such as Markov Chain Monte Carlo (MCMC) \cite{green2015bayesian} and Variational Inference (VI) \cite{blei2017variational}, aim to estimate probability distributions over the solution space, providing uncertainty quantification. MCMC offers rigorous statistical estimates but is computationally expensive, while VI is more efficient but introduces approximation errors. 

Multiple sources of error influence the accuracy of inverse elastography reconstructions. Measurement noise arises from the limitations of imaging techniques, leading to uncertainty in displacement data. Bayesian approaches handle these effects systematically and account for the uncertainty they introduce. The more delicate part is model errors \cite{brynjarsdottir2014learning}. These arise when the mathematical formulation of a model fails to capture the full complexity of the underlying physical process \cite{kaipio2006statistical}, posing a significant challenge in elastography due to the variety of biological tissue types involved \cite{holzapfel2017similarities}. Various strategies have been proposed to address these errors, which can be broadly categorized into external correction methods, internal correction methods, model selection approaches, and machine learning-based approaches. 

External correction methods, such as the Kennedy and O’Hagan (KOH) framework \cite{kennedy2001bayesian}, introduce an additive discrepancy term—often modeled as a Gaussian process—to adjust model predictions based on observed data. While this approach provides a flexible way to account for discrepancies, it lacks physical interpretability \cite{berliner2008modeling} and does not generalize well to other material properties beyond those explicitly corrected \cite{andres2024model}. In elastography, where enforcing physical constraints such as equilibrium and conservation laws is essential, external correction methods may lead to inconsistencies in the reconstructed mechanical parameters.

Internal correction methods, in contrast, aim to incorporate uncertainty directly within the model structure by modifying its parameters. Parameter Uncertainty Inflation (PUI) techniques \cite{pernot2017critical, pernot2017parameter}, such as Sargsyan’s parameter embedding \cite{sargsyan2015statistical, sargsyan2019embedded}, introduce stochastic variability into model parameters, ensuring that uncertainty is consistently propagated to all inferred quantities. This approach enables more physically consistent uncertainty quantification by embedding model structural errors rather than treating them as additive corrections. Additionally, PUI allows uncertainty to be meaningfully extrapolated to different observables or model scenarios, enhancing predictive reliability. However, these methods introduce computational complexity due to high-dimensional posterior sampling \cite{pernot2017critical}.

An alternative strategy is model selection, where competing models are quantitatively compared to determine which best explains the observed data. Bayesian model comparison techniques, such as Bayes factors \cite{kass1995bayes}, provide a rigorous statistical framework for ranking models. Still, their application in elastography is limited by the computational cost of evaluating high-dimensional probability distributions. Additionally, selecting the best-fitting model does not necessarily resolve underlying model inaccuracies, which may lead to biased estimates of tissue properties if all considered models are inadequate. 

An alternative to model-based inverse methods in elastography is learning the material law directly from measurement data, bypassing errors introduced by predefined constitutive models. The Autoprogressive Method (AutoP) \cite{hoerig2021machine, newman2024improving} achieves this by integrating finite element analysis with neural networks to extract stress-strain relationships from force-displacement measurements. This approach offers greater flexibility in capturing complex, heterogeneous tissue behavior without imposing restrictive assumptions. However, while AutoP ensures physical consistency through embedded mechanical principles, its high computational cost and numeric instabilities remain a significant challenge \cite{newman2024improving}. 

The proposed framework provides a probabilistic approach to material property estimation by distinguishing between reliable governing equations, such as conservation laws, and less reliable constitutive laws \cite{koutsourelakis2012novel, bruder2018beyond}. Unlike traditional model-based methods that assume the correctness of all equations, this framework quantifies model errors and identifies regions where the constitutive law may not hold. Incorporating stress as a state variable alongside displacement ensures a more comprehensive representation of tissue mechanics. Furthermore, it leverages a virtual likelihood \cite{kaltenbach2020incorporating} to incorporate weighted residuals \cite{scholz2025weak}. The method addresses the fundamental limitations of traditional and data-driven approaches, providing a structured, probabilistic alternative for material property estimation in elastography. Compared to existing model error correction techniques, this framework offers several advantages:
\begin{itemize}
    \item It explicitly identifies regions where constitutive laws break down, enhancing interpretability.
    \item It does not rely on a fully trustworthy forward model, making it more robust to modeling uncertainties.
    \item It avoids black-box terms, allowing for more efficient parameter estimation through variational inference.
\end{itemize}

The remainder of this paper is organized as follows: In section 2, we introduce the forward and Bayesian inverse problem for elastography, followed by a detailed presentation of our methodology, including the derivation of the evidence lower bound (ELBO) and the form of the approximate posterior. Next, in section 3, we validate the efficiency of our approach through numerical experiments. Finally, in section 4, we conclude with a summary and outlook.