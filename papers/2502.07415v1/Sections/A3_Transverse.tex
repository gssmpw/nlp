A transversely isotropic material is one with physical properties that are symmetric about an axis $\ve a$, while the properties in said direction may differ. In our case, we define the axis as $\ve a = \begin{bmatrix}
    1 & 0
\end{bmatrix}$, in which the Youngs Modulus is $E_a = 3.0$ and the shear modulus $G_a = 1.154$, compared to the direction perpendicular to it, where the Youngs Modulus is chosen to be $E = 1.0$. We assume the same Poissons ratio of $\nu = \nu_a = 0.3$ in all directions. From this we can calculate the following intermediate quantities:
\begin{align}
    n &= E_a / E \\
    m &= 1 - \nu - 2  n  \nu^2 \\
    \lambda &= E  ( \nu + n  \nu^2 ) / ( m  ( 1 + \nu ) ) \\
    \mu &= E / ( 2  ( 1 + \nu ) ) \\
    \alpha &= \mu - G_a \\
    \beta &= E  \nu^2  (1-n) / (4m(1+ \nu)) \\
    \gamma &= E_a  (1- \nu) / (8m) - (\lambda + 2  \mu) / 8 + \alpha / 2 - \beta.
\end{align}
We define the deformation gradient $\ve F = \ve I + \nabla \uv$ and use it to evaluate the left and right Cauchy-Green tensors as $\ve B = \ve F \ve F^T$ and $\ve C = \ve F^T \ve F$. Further, the Jacobian can be calculated as $J = \det \ve F$, and the invariants required are
\begin{align}
    I_1 &= \mathrm{tr} \ve C \\
    I_4 &= \ve a \cdot \ve C \ve a \\
    \ve A &= \ve a \otimes \ve a.
\end{align}
Then, the stress tensor $\sig_\mathrm{trans}$ can be calculated by
\begin{align*}
    \sig_\mathrm{trans} = &\frac{1}{J} ( 2\beta(I_4 - 1)  \ve B \\
    & + 2  (\alpha + \beta  (I_1-2) + 2\gamma(I_4-1))  \ve A \\
    & - \alpha  ((\ve B \ve a)\otimes \ve a + \ve a \otimes (\ve B \ve a)) \\
    & \mu/J  (\ve B - \ve I) + \lambda  (J-1)  \ve I)
\end{align*}