In this study, we demonstrated the effectiveness of our framework in identifying regions where the prescribed material law did not hold using the precision parameter. In regions where the material law was valid, the predicted material parameters were correctly inferred, while in invalid regions, the applied prior flattened the field, preventing inaccurate predictions. Additionally, the inferred stress fields adhered to the conservation law, further validating our approach.

Our method enabled fast inference by leveraging the weighted residual scheme proposed by Scholz et al., demonstrating its computational efficiency. However, several avenues remain for future work.

To further validate and extend our approach, we plan to:
\begin{itemize}
    \item Conduct experiments on higher-dimensional grids (e.g., 64 × 64 resolution).
    \item Test the framework using real or phantom data to assess its practical applicability.
    \item Investigate scenarios with multiple inclusions, including one with a valid and one with an invalid material law. Initial tests showed that both inclusions were flagged as invalid, highlighting a need for refinement.
    \item Implement latent-based fields to downscale large neural networks and improve computational efficiency.
\end{itemize}

These future investigations will enhance the framework's robustness and broaden its applicability in inverse modeling problems.
