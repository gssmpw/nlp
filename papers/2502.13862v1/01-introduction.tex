Graph-structured data has become a cornerstone of modern data analysis, enabling the representation of complex relationships in domains such as social networks, biological systems, and recommendation engines. The dynamic nature of many real-world graphs --- where edges and vertices are frequently added or removed --- has further intensified the need for efficient frameworks capable of managing and processing such dynamic graphs.
% As graphs grow in size and complexity, the performance of graph processing systems becomes critical.

However, despite significant advancements in graph processing frameworks, there remains a gap in performance, especially when handling large-scale, dynamic graphs. Existing frameworks, such as PetGraph, Stanford Network Analysis Platform (SNAP), SuiteSparse:GraphBLAS, cuGraph, and Aspen, offer varying degrees of efficiency but often struggle with scalability and speed when dealing with massive datasets or frequent updates. One of the primary bottlenecks is memory allocation during dynamic operations, particularly in graph cloning, where a significant portion of runtime is spent on edge memory allocation. As illustrated in Figure \ref{fig:vector2d-runtime}, an average of $74\%$ of the total runtime for cloning a graph based on a \textit{vector2d} representation is consumed by memory allocation for edges. This inefficiency highlights the need for optimized graph representations that minimize memory allocation overhead.

In this technical report, we present a comprehensive evaluation of several state-of-the-art graph processing frameworks, including our \textbf{DiGraph} implementation,\footnote{\url{https://github.com/puzzlef/graph-openmp}} which leverages our Concurrent Power-of-2 Arena Allocator (CP2AA)\footnote{\url{https://github.com/puzzlef/allocator-openmp}} for efficient memory management. We compare the performance of these frameworks across four key tasks: \textbf{(1)} Loading graphs from disk into memory, \textbf{(2)} Cloning graphs, \textbf{(3)} Applying in-place edge insertions and deletions, and \textbf{(4)} Executing a simple iterative graph traversal algorithm.

Our results demonstrate that our DiGraph benefits from several optimizations: Algorithm \ref{alg:load} significantly improves graph loading times, Algorithm \ref{alg:clone} enables efficient deep copies, and Algorithms \ref{alg:sub} and \ref{alg:add} facilitate efficient in-place batch updates. Additionally, the use of contiguous edge arrays and a Struct-of-Arrays (SoA) approach enhances the execution of graph algorithms, such as $k$-step reverse walks. We also observe that SuiteSparse:GraphBLAS's lazy copying and, in particular, Aspen's zero-cost snapshotting can significantly improve the performance of applying dynamic batch updates to the given graph when creating new graph instances.
\ignore{Finally, our DiGraph could further benefit from optimizations for small batch updates.}

This technical report is organized as follows: Section \ref{sec:related} reviews related work, providing an overview of existing state-of-the-art graph processing frameworks. Section \ref{sec:approach} describes our custom graph representation and algorithms in detail. Section \ref{sec:evaluation} presents our experimental setup and discusses the performance results. Finally, Section \ref{sec:conclusion} presents some concluding remarks and future directions.

\begin{figure}[hbtp]
  \centering
  \subfigure{
    \label{fig:vector2d-runtime--all}
    \includegraphics[width=0.98\linewidth]{out/vector2d-runtime.pdf}
  } \\[-2ex]
  \caption{Runtime split, in seconds, for cloning a graph based on a \textit{vector2d} representation is shown for each graph in Table \ref{tab:dataset}. As illustrated, an average of $74\%$ of the total runtime is spent on memory allocation for the edges.}
  \label{fig:vector2d-runtime}
\end{figure}





%% - Use --- for a dash.
%% - Use ``camera-ready'' for quotes.
%% - Use {\itshape very} or \textit{very} for italicized text.
%% - Use \verb|acmart| or {\verb|acmart|} for mono-spaced text.
%% - Use \url{https://capitalizemytitle.com/} for URLs.
%% - Use {\bfseries Do not modify this document.} for important boldface details.
%% - Use \ref{fig:name} for referencing.

%% For a block of pre-formatted text: 
% \begin{verbatim}
%   \renewcommand{\shortauthors}{McCartney, et al.}
% \end{verbatim}

%% For a list of items:
% \begin{itemize}
% \item the ``ACM Reference Format'' text on the first page.
% \item the ``rights management'' text on the first page.
% \item the conference information in the page header(s).
% \end{itemize}

%% For a table:
% \begin{table}
%   \caption{Frequency of Special Characters}
%   \label{tab:freq}
%   \begin{tabular}{ccl}
%     \toprule
%     Non-English or Math&Frequency&Comments\\
%     \midrule
%     \O & 1 in 1,000& For Swedish names\\
%     $\pi$ & 1 in 5& Common in math\\
%     \$ & 4 in 5 & Used in business\\
%     $\Psi^2_1$ & 1 in 40,000& Unexplained usage\\
%   \bottomrule
% \end{tabular}
% \end{table}

%% For a full-width table:
% \begin{table*}
%   \caption{Some Typical Commands}
%   \label{tab:commands}
%   \begin{tabular}{ccl}
%     \toprule
%     Command &A Number & Comments\\
%     \midrule
%     \texttt{{\char'134}author} & 100& Author \\
%     \texttt{{\char'134}table}& 300 & For tables\\
%     \texttt{{\char'134}table*}& 400& For wider tables\\
%     \bottomrule
%   \end{tabular}
% \end{table*}


%% For inline math:
% \begin{math}
%   \lim_{n\rightarrow \infty}x=0
% \end{math},

%% For a numbered equation:
% \begin{equation}
%   \lim_{n\rightarrow \infty}x=0
% \end{equation}

%% For an unnumbered equation:
% \begin{displaymath}
%   \sum_{i=0}^{\infty} x + 1
% \end{displaymath}

%% For a figure:
% \begin{figure}[h]
%   \centering
%   \includegraphics[width=\linewidth]{inc/sample-franklin}
%   \caption{1907 Franklin Model D roadster. Photograph by Harris \&
%     Ewing, Inc. [Public domain], via Wikimedia
%     Commons. (\url{https://goo.gl/VLCRBB}).}
%   \Description{A woman and a girl in white dresses sit in an open car.}
% \end{figure}

%% For a teaser figure.
% \begin{teaserfigure}
%   \includegraphics[width=\textwidth]{sampleteaser}
%   \caption{figure caption}
%   \Description{figure description}
% \end{teaserfigure}
