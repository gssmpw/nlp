\documentclass[sigconf,nonacm]{acmart}

%% Enable subfigures
\usepackage{subfigure}
%% Enable numbers in scientific format.
\usepackage{siunitx}
%% Enable enumerate start from.
\usepackage{enumitem}

%% Enable theorems
\newtheorem{theorem}{Theorem}[section]
\newtheorem{lemma}[theorem]{Lemma}

%% Enable algorithms
\usepackage{algorithm}
\usepackage[noend]{algpseudocode}
\let\ReturnInline\Return
\renewcommand{\Return}{\State\ReturnInline}
\algrenewcommand\algorithmicrequire{$\rhd$}
\algrenewcommand\algorithmicensure{$\square$}

%% Fonts used in the template cannot be substituted; margin 
%% adjustments are not allowed.
\AtBeginDocument{%
  \providecommand\BibTeX{{%
    \normalfont B\kern-0.5em{\scshape i\kern-0.25em b}\kern-0.8em\TeX}}}

%% Rights management information.
\setcopyright{acmcopyright}
\copyrightyear{2018}
\acmYear{2018}
\acmDOI{XXXXXXX.XXXXXXX}

%% These commands are for a PROCEEDINGS abstract or paper.
\acmConference[Conference acronym 'XX]{Make sure to enter the correct
  conference title from your rights confirmation emai}{June 03--05,
  2018}{Woodstock, NY}
%% Title of the proceedings is different from ``Proceedings of ...''?
% \acmBooktitle{Woodstock '18: ACM Symposium on Neural Gaze Detection,
%  June 03--05, 2018, Woodstock, NY} 
% \acmPrice{15.00}
% \acmISBN{978-1-4503-XXXX-X/18/06}

%% Submission ID.
% \acmSubmissionID{123-A56-BU3}

%% Use the "author year" style of citations and references?
% \citestyle{acmauthoryear}

%% Message
\newcommand{\kk}[1]{{{\color{red} #1}}}
\newcommand{\ds}[1]{{{\color{red} #1}}}
\newcommand{\su}[1]{{{\color{red} #1}}}

%% Ignore block
\newcommand{\ignore}[1]{}

%% Macros
\newcommand{\ours}{$\nu$-LPA}




\begin{document}

%% Full title of the paper.
\title[Performance Comparison of Graph Representations Which Support Dynamic Graph Updates]{Performance Comparison of Graph Representations \\Which Support Dynamic Graph Updates}

%% Short title to be used in page headers (optional).
% \title[short title]{full title}
% \subtitle{Something other than the title}

%% Authors and their affiliations.
\author{Subhajit Sahu}
\email{subhajit.sahu@research.iiit.ac.in}
\affiliation{%
  \institution{IIIT Hyderabad}
  \streetaddress{Professor CR Rao Rd, Gachibowli}
  \city{Hyderabad}
  \state{Telangana}
  \country{India}
  \postcode{500032}
}

%% Concise author list in page headers.
%\renewcommand{\shortauthors}{Sahu, Kothapalli, and Banerjee, et al.}

%% Show page numbers.
\settopmatter{printfolios=true}

%% Short summary of the work to be presented in the article.
\begin{abstract}
Research in graph-structured data has grown rapidly due to graphs' ability to represent complex real-world information and capture intricate relationships, particularly as many real-world graphs evolve dynamically through edge/vertex insertions and deletions. This has spurred interest in programming frameworks for managing, maintaining, and processing such dynamic graphs. In this report, we evaluate the performance of PetGraph (Rust), Stanford Network Analysis Platform (SNAP), SuiteSparse:GraphBLAS, cuGraph, Aspen, and our custom implementation in tasks including loading graphs from disk to memory, cloning loaded graphs, applying in-place edge deletions/insertions, and performing a simple iterative graph traversal algorithm. Our implementation demonstrates significant performance improvements: it outperforms PetGraph, SNAP, SuiteSparse:GraphBLAS, cuGraph, and Aspen by factors of $177\times$, $106\times$, $76\times$, $17\times$, and $3.3\times$ in graph loading; $20\times$, $235\times$, $0.24\times$, $1.3\times$, and $0\times$ in graph cloning; $141\times$/$45\times$, $44\times$/$25\times$, $13\times$/$11\times$, $28\times$/$34\times$, and $3.5\times$/$2.2\times$ in edge deletions/insertions; and $67\times$/$63\times$, $86\times$/$86\times$, $2.5\times$/$2.6\times$, $0.25\times$/$0.24\times$, and $1.3\times$/$1.3\times$ in traversal on updated graphs with deletions/insertions.
\end{abstract}

%% The code below is generated by the tool at http://dl.acm.org/ccs.cfm.
\begin{CCSXML}
<ccs2012>
<concept>
<concept_id>10003752.10003809.10010170</concept_id>
<concept_desc>Theory of computation~Parallel algorithms</concept_desc>
<concept_significance>500</concept_significance>
</concept>
<concept>
<concept_id>10003752.10003809.10003635</concept_id>
<concept_desc>Theory of computation~Graph algorithms analysis</concept_desc>
<concept_significance>500</concept_significance>
</concept>
</ccs2012>
\end{CCSXML}

% \ccsdesc[500]{Theory of computation~Parallel algorithms}
% \ccsdesc[500]{Theory of computation~Graph algorithms analysis}

%% Pick words that accurately describe the work being presented.
\keywords{Graph representation, Dynamic graph updates}

% \received{20 February 2007}
% \received[revised]{12 March 2009}
% \received[accepted]{5 June 2009}




%% Process the author and title information.
\maketitle

\section{Introduction}
\label{sec:introduction}

% humans are sensitive to the way information is presented.

% introduce framing as the way we address framing. say something about political views and how information is represented.

% in this paper we explore if models show similar sensitivity.

% why is it important/interesting.



% thought - it would be interesting to test it on real world data, but it would be hard to test humans because they come already biased about real world stuff, so we tested artificial.


% LLMs have recently been shown to mimic cognitive biases, typically associated with human behavior~\citep{ malberg2024comprehensive, itzhak-etal-2024-instructed}. This resemblance has significant implications for how we perceive these models and what we can expect from them in real-world interactions and decisionmaking~\citep{eigner2024determinants, echterhoff-etal-2024-cognitive}.

The \textit{framing effect} is a well-known cognitive phenomenon, where different presentations of the same underlying facts affect human perception towards them~\citep{tversky1981framing}.
For example, presenting an economic policy as only creating 50,000 new jobs, versus also reporting that it would cost 2B USD, can dramatically shift public opinion~\cite{sniderman2004structure}. 
%%%%%%%% 图1:  %%%%%%%%%%%%%%%%
\begin{figure}[t]
    \centering
    \includegraphics[width=\columnwidth]{Figs/01.pdf}
    \caption{Performance comparison (Top-1 Acc (\%)) under various open-vocabulary evaluation settings where the video learners except for CLIP are tuned on Kinetics-400~\cite{k400} with frozen text encoders. The satisfying in-context generalizability on UCF101~\cite{UCF101} (a) can be severely affected by static bias when evaluating on out-of-context SCUBA-UCF101~\cite{li2023mitigating} (b) by replacing the video background with other images.}
    \label{fig:teaser}
\end{figure}


Previous research has shown that LLMs exhibit various cognitive biases, including the framing effect~\cite{lore2024strategic,shaikh2024cbeval,malberg2024comprehensive,echterhoff-etal-2024-cognitive}. However, these either rely on synthetic datasets or evaluate LLMs on different data from what humans were tested on. In addition, comparisons between models and humans typically treat human performance as a baseline rather than comparing patterns in human behavior. 
% \gabis{looks good! what do we mean by ``most studies'' or ``rarely'' can we remove those? or we want to say that we don't know of previous work doing both at the same time?}\gili{yeah the main point is that some work has done each separated, but not all of it together. how about now?}

In this work, we evaluate LLMs on real-world data. Rather than measuring model performance in terms of accuracy, we analyze how closely their responses align with human annotations. Furthermore, while previous studies have examined the effect of framing on decision making, we extend this analysis to sentiment analysis, as sentiment perception plays a key explanatory role in decision-making \cite{lerner2015emotion}. 
%Based on this, we argue that examining sentiment shifts in response to reframing can provide deeper insights into the framing effect. \gabis{I don't understand this last claim. Maybe remove and just say we extend to sentiment analysis?}

% Understanding how LLMs respond to framing is crucial, as they are increasingly integrated into real-world applications~\citep{gan2024application, hurlin2024fairness}.
% In some applications, e.g., in virtual companions, framing can be harnessed to produce human-like behavior leading to better engagement.
% In contrast, in other applications, such as financial or legal advice, mitigating the effect of framing can lead to less biased decisions.
% In both cases, a better understanding of the framing effect on LLMs can help develop strategies to mitigate its negative impacts,
% while utilizing its positive aspects. \gabis{$\leftarrow$ reading this again, maybe this isn't the right place for this paragraph. Consider putting in the conclusion? I think that after we said that people have worked on it, we don't necessarily need this here and will shorten the long intro}


% If framing can influence their outputs, this could have significant societal effects,
% from spreading biases in automated decision-making~\citep{ghasemaghaei2024understanding} to reducing public trust in AI-generated content~\citep{afroogh2024trust}. 
% However, framing is not inherently negative -- understanding how it affects LLM outputs can offer valuable insights into both human and machine cognition.
% By systematically investigating the framing effect,


%It is therefore crucial to systematically investigate the framing effect, to better understand and mitigate its impact. \gabis{This paragraph is important - I think that right now it's saying that we don't want models to be influenced by framing (since we want to mitigate its impact, right?) When we talked I think we had a more nuanced position?}




To better understand the framing effect in LLMs in comparison to human behavior,
we introduce the \name{} dataset (Section~\ref{sec:data}), comprising 1,000 statements, constructed through a three-step process, as shown in Figure~\ref{fig:fig1}.
First, we collect a set of real-world statements that express a clear negative or positive sentiment (e.g., ``I won the highest prize'').
%as exemplified in Figure~\ref{fig:fig1} -- ``I won the highest prize'' positive base statement. (2) next,
Second, we \emph{reframe} the text by adding a prefix or suffix with an opposite sentiment (e.g., ``I won the highest prize, \emph{although I lost all my friends on the way}'').
Finally, we collect human annotations by asking different participants
if they consider the reframed statement to be overall positive or negative.
% \gabist{This allows us to quantify the extent of \textit{sentiment shifts}, which is defined as labeling the sentiment aligning with the opposite framing, rather then the base sentiment -- e.g., voting ``negative'' for the statement ``I won the highest prize, although I lost all my friends on the way'', as it aligns with the opposite framing sentiment.}
We choose to annotate Amazon reviews, where sentiment is more robust, compared to e.g., the news domain which introduces confounding variables such as prior political leaning~\cite{druckman2004political}.


%While the implications of framing on sensitive and controversial topics like politics or economics are highly relevant to real-world applications, testing these subjects in a controlled setting is challenging. Such topics can introduce confounding variables, as annotators might rely on their personal beliefs or emotions rather than focusing solely on the framing, particularly when the content is emotionally charged~\cite{druckman2004political}. To balance real-world relevance with experimental reliability, we chose to focus on statements derived from Amazon reviews. These are naturally occurring, sentiment-rich texts that are less likely to trigger strong preexisting biases or emotional reactions. For instance, a review like ``The book was engaging'' can be framed negatively without invoking specific cultural or political associations. 

 In Section~\ref{sec:results}, we evaluate eight state-of-the-art LLMs
 % including \gpt{}~\cite{openai2024gpt4osystemcard}, \llama{}~\cite{dubey2024llama}, \mistral{}~\cite{jiang2023mistral}, \mixtral{}~\cite{mistral2023mixtral}, and \gemma{}~\cite{team2024gemma}, 
on the \name{} dataset and compare them against human annotations. We find  that LLMs are influenced by framing, somewhat similar to human behavior. All models show a \emph{strong} correlation ($r>0.57$) with human behavior.
%All models show a correlation with human responses of more than $0.55$ in Pearson's $r$ \gabis{@Gili check how people report this?}.
Moreover, we find that both humans and LLMs are more influenced by positive reframing rather than negative reframing. We also find that larger models tend to be more correlated with human behavior. Interestingly, \gpt{} shows the lowest correlation with human behavior. This raises questions about how architectural or training differences might influence susceptibility to framing. 
%\gabis{this last finding about \gpt{} stands in opposition to the start of the statement, right? Even though it's probably one of the largest models, it doesn't correlate with humans? If so, better to state this explicitly}

This work contributes to understanding the parallels between LLM and human cognition, offering insights into how cognitive mechanisms such as the framing effect emerge in LLMs.\footnote{\name{} data available at \url{https://huggingface.co/datasets/gililior/WildFrame}\\Code: ~\url{https://github.com/SLAB-NLP/WildFrame-Eval}}

%\gabist{It also raises fundamental philosophical and practical questions -- should LLMs aim to emulate human-like behavior, even when such behavior is susceptible to harmful cognitive biases? or should they strive to deviate from human tendencies to avoid reproducing these pitfalls?}\gabis{$\leftarrow$ also following Itay's comment, maybe this is better in the dicsussion, since we don't address these questions in the paper.} %\gabis{This last statement brings the nuance back, so I think it contradicts the previous parapgraph where we talked about ``mitigating'' the effect of framing. Also, I think it would be nice to discuss this a bit more in depth, maybe in the discussion section.}







\section{Related work}
\label{sec:related}
\section{Related Work}
Our work builds on several lines of research: exploring the role of text with visualizations, visualization and text systems, and image and text authoring interfaces.

\subsection{The role of text with visualizations}
The interplay between text and visual elements in data visualization has been a significant area of interest with increased advocacy for treating text as co-equal to visualization~\cite{stokesgive, lundgard2021accessible}. Kim et al.~\cite{kim2021towards} conducted a study to understand how readers integrate charts and captions in line charts. The study findings indicated that when both the chart and text emphasize the same prominent features, readers take away insights from both modalities. Their research underscores the importance of coherence between visual and textual elements and how external context provided by captions can enhance the reader's comprehension of the chart's message. Building on these insights, Lundgard and Satyanarayan~\cite{lundgard2021accessible} proposed a four-level model for content conveyed by natural language descriptions of visualizations. Their model delineates semantic content into four distinct levels: elemental and encoded properties (Level 1), statistical concepts (Level 2), perceptual and cognitive phenomena (Level 3), and contextual insights (Level 4).

Focusing on the role of textual annotations in visualization, Stokes et al.~\cite{stokes2022striking} observed that readers favored heavily annotated charts over less annotated charts or text alone. This preference highlights the added value of textual annotations in aiding data interpretation, with specific emphasis on how different types of semantic content impact the takeaways drawn by readers. Further contributions by Quadri et al.~\cite{quadri2024you} and Fan et al.~\cite{fan2024understanding} explored high-level visualization comprehension and the impact of text details and spatial autocorrelation on reader takeaways in thematic maps. These studies collectively underline the critical role of textual elements in shaping viewer perceptions, understanding, and accessibility of visual data. Ottley et al.~\cite{ottley2019curious} and Stokes et al.~\cite{stokes2023role} have also contributed to this body of research, focusing on how annotations influence perceptions of bias and predictions, reinforcing the multifaceted impact of text on visual data interpretation.

Our work further explores how text and charts can be better aligned with one another by offering a mixed-initiative authoring interface. Specifically, \pluto~allows leveraging both direct manipulation interactions and user-drafted text to generate recommendations for communicative text and chart design. Furthermore, \pluto's text recommendations explicitly incorporate Lundgard and Satyanarayan's model~\cite{lundgard2021accessible} for semantic information conveyed by visualization descriptions.
In doing so, the system ensures that the generated text has good semantic coverage and structure (e.g., generated descriptions start by conveying the chart's encodings and then list high-level trends) and is appropriate for the intended communicative use (e.g., the semantic information conveyed by titles is different from descriptions accompanying a chart or annotations on the chart).


\subsection{Visualization and text systems}

The integration of visualization and text has led to the development of various systems designed to facilitate the creation, interpretation, and enhancement of data visualizations with textual elements. He et al.~\cite{he2024leveraging} surveyed the leveraging of large models for crafting narrative visualizations, highlighting the potential of AI in supporting the narrative aspect of data visualization. This is complemented by AutoTitle, an interactive title generator for visualizations~\cite{liu2023autotitle}, and Vistext, a benchmark for semantically rich chart captioning~\cite{tang2023vistext}. VizFlow demonstrates the effectiveness of facilitating author-reader interaction by dynamically connecting text segments to corresponding chart elements to help enrich the storytelling experience~\cite{sultanum2021}. This body of research highlights the need for tools to support more nuanced integration of text and visualization.

Supporting the co-authoring of text and charts, Latif et al. introduced Kori~\cite{latif2021kori}, an interactive system for synthesizing text and charts in data documents, emphasizing the seamless integration of visual and textual data for enhanced communication.
\new{CrossData~\cite{chen2022crossdata} presents an interactive coupling between text and data in documents, enabling actions based on the document text and adjusting data values in the text through direct manipulation on the chart.
Such systems illustrate the potential for the bidirectional linking between text and charts to assist rich authoring of data-driven narratives.
}
Furthermore, systems like EmphasisChecker~\cite{kim2023emphasischecker}, Intentable~\cite{choi2022intentable}, Chart-to-text~\cite{obeid2020chart}, DataDive~\cite{kim2024datadive},
\new{InkSight~\cite{lin2023inksight}},
and FigurA11y~\cite{singh2024figura11y} focus on guiding chart and caption creation, supporting readers' contextualization of statistical statements, and assisting in writing scientific alt text. Recent work like SciCapenter supports the composition of scientific figure captions using AI-generated content and quality ratings \cite{hsu2024scicapenter}.
DataTales~\cite{sultanum2023datatales} is another example of a recent system using a large language model for authoring data-driven articles, indicating the growing interest in AI-assisted data storytelling.
These systems collectively illustrate the expanding scope of text integration into visualization, from enhancing data document creation to improving accessibility and data-driven communication. Reviewing the aforementioned tools and the use of generative AI for visualization more broadly, Basole and Major~\cite{basole2024generative} discuss how generative AI methods and tools offer creativity assistance and automation within the visualization workflow, specifically highlighting a shift towards ``human-led AI-assisted'' paradigms, where generative AI not only augments the creative process but also becomes a co-creator.

Aligned with this paradigm shift, \pluto~adopts a mixed-initiative approach that leverages the capabilities of generative AI to help create semantic alignment between the chart and its corresponding text for effective data-driven communication.
However, \pluto~differs from existing chart-and-text authoring tools in three significant ways.
First, going beyond existing systems that primarily leverage unimodal information from the chart to generate text, \pluto~supports multimodal authoring combining information from both the chart (including any direct interactions with marks) and user-drafted text.
Furthermore, unlike prior tools that focus exclusively on generating complete descriptions/captions or titles, \pluto's recommendations can be leveraged in flexible ways to author not only titles and descriptions but also more fine-grained annotations and sentence completions. Second, while existing tools primarily recommend text for a given chart, \pluto's recommendations are bidirectional.
Specifically, the system suggests chart design changes like sorting or adding embellishments based on the authored text, resulting in artifacts that more clearly communicate takeaways via a combination of text and charts. Lastly, unlike existing tools that primarily rely on pre-trained knowledge in generative AI models, \pluto's recommendations are grounded in a theoretical research-based model of semantic information conveyed in visualization text~\cite{lundgard2021accessible}, ensuring the generated text covers the appropriate level of detail and is effective for communication \emph{alongside} the chart.

% \section{Preliminaries}
% \label{sec:preliminaries}
% \section{Preliminaries}
\label{sec:preliminaries}
% Here we formulate the CS problem, discuss some metrics for it and provide an overview of CBMs. % for the it, followed by an overview of the concept bottleneck models.

% \vspace{-0.1cm}
\subsection{Coreset selection (CS) problem formulation}
\label{sec:preliminaries_cs}
Consider a classification task and data distribution $P$. 
Let $\mathcal{D} = \{(x_i, y_i)\}_{i=1}^{n}$ denote the dataset of $n$ training examples sampled i.i.d. from the distribution $P$ where $x_i$ denotes the data and $y_i \in \mathcal{Y}$ denotes the label from a set of $N$ classes.
CS \cite{coleman2019selection,zheng2022coverage} aims to find a subset $\mathcal{S}$ of $\mathcal{D}$ consisting of $m \leq n$ samples such that the models trained on $\mathcal{S}$ achieve performance comparable to models trained on $\mathcal{D}$. 
Formally, the CS problem is %defined as: % follows,
\begin{equation}
\label{eq:coreset_selection}
\min_{\mathcal{S}:|\mathcal{S}|=m} \mathbb{E}_{(x,y) \sim P}[\ell(x,y|\theta_{\mathcal{S}})] - \mathbb{E}_{(x,y) \sim P}[\ell(x,y|\theta_{\mathcal{D}})],
\end{equation}
where $\theta_{\mathcal{D}}$ and $\theta_{\mathcal{S}}$ denote the ``\emph{downstream model}"  trained on $\mathcal{D}$ and $\mathcal{S}$ (coreset), respectively and $\ell$ is the loss function.

To find this subset $\mathcal{S}$, previous works have proposed scores to gauge a sample's difficulty for a model, and are later used to form the coreset. 
Approaches such as max entropy uncertainty sampling \cite{lewis1994heterogeneous, Settles_2012}, and least confidence \cite{culotta2005reducing} estimate difficulty using the uncertainty of the model's predictions on a sample. %, to gauge its difficulty.
Another set of approaches such as $k$-center greedy \cite{sener2017active} uses  geometric information of the data to filter out redundant samples. 
Yet, another set of approaches uses information from the training dynamics of the downstream model to estimate the a difficulty score. 
Scores such as the forgetting score \cite{toneva2018empirical} which is computed as the number of times a sample gets misclassified after being correctly classified earlier during model training, the area under the margin (AUM) \cite{pleiss2020identifying} which identifies mislabeled/difficult samples, fall in this category.
% Based on these score, a sampling strategy is then used to identify $\mathcal{S}$.
While approaches based on the training dynamics of the downstream model have achieved SOTA results, the requirement of knowledge/training the downstream model or a relatively big proxy model on the entire dataset at least once is inefficient for large datasets/models, even if done only once. 
This motivates the need of data centric approaches for sample's difficulty independent of the downstream model.
% \AM{Sampling strategy}

\subsection{Concept bottleneck models (CBMs)}
\label{sec:preliminaries_cbm}
Recent advances in language model-guided CBMs utilize an LLM to obtain concept bottlenecks which are then used to predict the labels. 
These works rely on a pre-trained multi-modal models (such as CLIP \cite{radford2021learning}) which consists of a visual encoder $\mathcal{V}_{enc}$ and a text encoder $\mathcal{T}_{enc}$ that can map images and text to a $d$-dimensional representation space. % with dimension $d$. 
Let $C = \{c_1, c_2, \cdots, c_{N_C}\}$ be the set of $N_C$ concepts (bottleneck) generated via a LLM, we can then construct a bottleneck embedding matrix $E_C \in \mathcal{R}^{N_C \times d}$ such that each row of the matrix is mapping of the concept $c \in C$ after passing it through textual encoder $\mathcal{T}_{enc}$.
Based on this, a CBM  \cite{yang2023language} produces a prediction $h(x) = f(g(\mathcal{V}_{enc}(x), E_C))$ for a sample $x$ where $g:\mathbb{R}^{d} \rightarrow \mathbb{R}^{N_C}$ computes the similarity of the visual features to each concept in the bottleneck and $f:\mathbb{R}^{d} \rightarrow \Delta$ outputs the probability of each class in the label set $\mathcal{Y}$, where $\Delta$ is a $N$ simplex.
We discuss details of $f$ and $g$ in Sec.~\ref{sec:approach}.

\section{Approach}
\label{sec:approach}
\section{Exploiting Partial Truth Assignments in OMT}%
\label{sec:approach}
\begin{algorithm}[t]
    \newcommand{\res}{\textsf{res}}
    \begin{algorithmic}[1]
        %\begin{rschange}
        \caption[A]{{\sc Linear-search OMT with partial assignments}($\vi, \obj$)\\
            \hspace*{\algorithmicindent}\textbf{Input}:
            Formula $\vi$, objective $\obj$\\
            \hspace*{\algorithmicindent}\textbf{Output}: $\satres/\unsatres$, optimum model $\M$}%
        \label{alg:omt-partial}
        \STATE \makebox[.5cm][c]{$\M$}$\gets \emptyset$ \algorithmiccomment{Best model found so far}
        \STATE \makebox[.5cm][c]{$\ub$}$\gets \infty$ \algorithmiccomment{Current upper bound}
        \STATE \makebox[.5cm][c]{$\res$}$\gets \satres$ \algorithmiccomment{Status of the search}
        \WHILE{$\res = \satres$}
        \STATE $\tuple{\res,\eta} \gets \incrementalsmt(\vi\wedge(\obj<\ub))$
        \IF{$\res = \satres$}
        \STATE\label{alg:line:omt-partial:minimize} \makebox[.5cm][c]{$\color{blue}\mu$}$\color{blue}\gets\omtminimizeassignment(\vi,\eta,\obj)$
        \STATE \makebox[.5cm][c]{$\M$}$\gets \minimize(\mu,\obj)$
        \STATE \makebox[.5cm][c]{$\ub$}$\gets \M(\obj)$
        \ENDIF
        \ENDWHILE
        \IF{$\M = \emptyset$}
        \RETURN $\tuple{\unsatres,\emptyset}$
        \ELSE
        \RETURN $\tuple{\satres,\M}$
        \ENDIF
    \end{algorithmic}
\end{algorithm}

The general schema of our approach is presented in~\Cref{alg:omt-partial}. This
algorithm is a variant of the basic \omt{} linear-search
schema~\cite{sebastianiOptimizationSMTLAQ2012,sebastianiOptimizationModuloTheories2015}
described in~\sref{sec:bg:omt}. The main difference is the call to the
$\omtminimizeassignment$ procedure (line~\ref{alg:line:omt-partial:minimize}),
which is responsible for reducing the truth assignment to be fed to the
\T-minimizer, provided that the resulting partial truth assignment still
propositionally satisfies the formula. Depending on the implementation of this
procedure, the assignment-reduction strategy can be more or less effective in
improving the search for the global optimum.

In \sref{sec:approach:basic-assignment-minimization} and
\sref{sec:approach:guided-assignment-minimization}, we describe two possible
implementations of this procedure.

% \TODO{Say something about correctness of the approach?}
% \TODO{Say something about relation with pure-literal filtering?}

\subsection{Basic Assignment Reduction}%
\label{sec:approach:basic-assignment-minimization}

The first approach is to reduce the truth assignment using~\Cref{alg:minimize}
in~\sref{sec:bg:partial-truth-assignments}, i.e., iterating over all the
literals in the current truth assignment $\eta$, and dropping them one by one,
if possible.
%This approach is simple and general, but it may not be effective in practice, as it does not take into account the properties of the \omt{} search strategy.
A straightforward improvement is to only try to drop \T-literals, since they
are the ones that, if dropped, can potentially enlarge the area within which
the optimum \T-model is searched. This procedure is simple and general, and
comes with a limited overhead, as each truth assignment is scanned only once to
find the literals to drop, and the \T-minimizer is called only once for each
candidate assignment.

This approach, however, might not be very effective in practice, as it ``blindly''
removes literals from the truth assignment without taking into account the
properties of the \omt{} search strategy. In particular, it may drop literals that
are not relevant for the optimization, enlarging the search area in the wrong
direction, possibly preventing from dropping other literals that are more
relevant.%does not take

\subsection{OMT-Guided Assignment Reduction}%
\label{sec:approach:guided-assignment-minimization}
%The main limitation of the previous approach is that the heuristics used to choose the atoms to leave unassigned do not take into account the properties of the \omt{} search strategy. 
We propose an ad-hoc assignment-reduction technique for \omt{} solving, which
is outlined in~\Cref{alg:omt-minimize-assignment-guided}.
%
%address these points.
%investigate \emph{on-demand} truth assignment minimization.
% By this term, we mean that, once a \larat-satisfiable satisfying (partial) truth assignment has been found, and the \larat-minimizer has found a minimal model within the corresponding area, then we can try to remove just some of the \larat-atoms and look if a better model exists within the new enlarged area. Moreover, since both the constraints and the cost function are linear, by the simplex method
% the optimum model lies on a vertex of the polytope. As an heuristic, we can choose to drop one of the constraints that form the vertex, since a better model is likely to be found in that portion of space. 
%
%
Suppose that, after the \T-minimizer has found a minimum model within the
current truth assignment $\mu$
(line~\ref{alg:omt-minimize-assignment-guided:line:minimize1}), it returns also
one (or more) literal(s) that limit the current minimum
(line~\ref{alg:omt-minimize-assignment-guided:line:propose1}). These literals
are part of some (possibly minimal) $\muprime\subseteq\mu$ such that
$\muprime\cup\set{\obj<\M(\obj)}$ is \T-unsatisfiable. Intuitively, the removal
of any literal $\ell\in\muprime$ is very likely to lead to a better optimum model,
provided that $\mu\setminus\set{\ell}$ still propositionally satisfies $\vi$
(line~\ref{alg:omt-minimize-assignment-guided:line:ifcandrop}).%the list of the literals%This corresponds to a minimal 

%these literals, if dropped, are likely
%to lead to a better optimum model. 
We can then iteratively drop these literals and re-run the \T-minimizer, until
no more literals can be dropped
(lines~\ref{alg:omt-minimize-assignment-guided:line:while}--\ref{alg:omt-minimize-assignment-guided:line:propose2}).
\begin{algorithm}[t]
    \begin{algorithmic}[1]
        %\begin{rschange}
        \caption[A]{\omtminimizeassignmentguided($\vi, \eta$, $\obj$)\\
            \hspace*{\algorithmicindent}\textbf{Input}:
            Formula $\vi$, \T-satisfiable total truth assignment $\eta$ satisfying $\vi$, objective $\obj$\\
            \hspace*{\algorithmicindent}\textbf{Output}: Reduced truth assignment $\mu\subseteq\eta$ satisfying $\vi$}%
        \label{alg:omt-minimize-assignment-guided}
        \STATE\makebox[.5cm][c]{$\mu$}$\gets\eta$
        \STATE\makebox[.5cm][c]{$\M$}$\gets\minimize(\mu,\obj)$\label{alg:omt-minimize-assignment-guided:line:minimize1}
        \STATE\makebox[.5cm][c]{$\ell$}$\gets\proposelit()$\label{alg:omt-minimize-assignment-guided:line:propose1}
        \WHILE{$\ell\neq\bot$}\label{alg:omt-minimize-assignment-guided:line:while}
        \IF{$\mu\setminus\set{\ell}$ satisfies all clauses in $\vi$}\label{alg:omt-minimize-assignment-guided:line:ifcandrop}
        \STATE \makebox[.5cm][c]{$\mu$}$\gets \mu \setminus \set{\ell}$\label{alg:omt-minimize-assignment-guided:line:drop}
        %\STATE $\tpop(\ell)$\label{alg:omt-minimize-assignment-guided:line:tpop}
        \STATE \makebox[.5cm][c]{$\M$}$\gets\minimize(\mu,\obj)$\label{alg:omt-minimize-assignment-guided:line:minimize2}
        \ENDIF
        \STATE $\ell\gets\proposelit()$\label{alg:omt-minimize-assignment-guided:line:propose2}
        \ENDWHILE
        \RETURN $\mu$
    \end{algorithmic}
\end{algorithm}

We describe a possible implementation of the $\proposelit$ procedure
in~\Cref{alg:omt-minimize-assignment-guided} for the case of \omlarat{}.
%
As we have seen in~\sref{sec:bg:omt}, a \larat{}-minimizer can be implemented
as a variant of the Simplex
method~\cite{dutertreFastLinearArithmeticSolver2006,sebastianiOptimizationSMTLAQ2012},
by which an optimum model is always found on a vertex of the polytope defined
by the conjunction of \larat-constraints on which it is invoked. Thus, in this
case, the candidate constraints to be dropped are those that form such vertex.
This information can be easily obtained from the Simplex tableau~\cite{dutertreFastLinearArithmeticSolver2006}.
%  \TODO{Find a way to easily explain this.\\
%  The tableau stores
%  $x_i = bi + \sum_{x_j\in\N} a_{i,j}x_j$, forall $x_i\in\B$\\
%  and then it keeps track of the bounds for each variable $l_i\leq x_i\leq u_i$\\
%  Basically, we look for $x_i\in\N$ such that\\
%  $\beta[x_i] = u_i$ and $a_{\obj,i} < 0$\\
%  $\beta[x_i] = l_i$ and $a_{\obj,i} > 0$\\
%  where $\beta[x_i]$ is the current value of $x_i$.\\
%  For each such $x_i$, we get the constraint imposing the
%  bound.
%  }

For other theories, the implementation of the $\proposelit$ procedure may be
more complex, requiring the extraction of a (possibly minimal) conflict set of $\mu\cup\set{\obj<\M(\obj)}$.
In general, also heuristic strategies can be used, as they only
provide suggestions to the assignment-reduction procedure, and do not affect
the correctness of the search.

Regarding the computational cost, the proposed approach can be more expensive than
the basic assignment reduction, as it requires the \T-minimizer to be called
multiple times. \T-minimizers, however, are typically designed to be called
incrementally, maintaining the state of the previous calls, and thus the
overhead of multiple calls is limited.
% \GMSIDENOTE{Remove or expand?}
% Another possible source of complexity is that the incrementality of \T-minimizer is based on a stack interface, whereas
% \omtminimizeassignmentguided{} requires to drop any literal in the assignment,
% which may require a more complex data structure to be implemented efficiently.

\section{Evaluation}
\label{sec:evaluation}
\begin{figure*}[h!]
    \centering %trim titles
     \includegraphics[width=.5\textwidth,trim={0 1.4cm 0 3.4cm},clip]{figures/sae_feature_accuracies_layer31_pre_SQUAD_train.png}%{figures/sae_feature_accuracies_layer31_pre.png}
    ~%
    \includegraphics[width=.5\textwidth,trim={0cm 1.4cm 0 3.4cm},clip]{figures/sae_feature_accuracies_layer31_pre_BoolQ_train.png}
\caption{Out-of-distribution comparison between top SAE features,  pre-activation, and linear probes on layer 31; trained on SQUAD (left) and BoolQ (right).}
    \label{fig:sae-probe_pre31}
\end{figure*}

\begin{figure}[h!]
    \centering %trim titles
    \includegraphics[width=.5\textwidth,trim={0 1.4cm 0 3.4cm},clip]{figures/hierarchical_sae_probe_layer31_pre.png}
    \caption{Combinations of SAE features, displaying the median value across top feature groups with quartile ranges in the error bars.}
    \label{fig:sae-k-pre31}
\end{figure}

% \begin{figure*}[t]
% \centering

% %--------------- First row ---------------%
% \begin{subfigure}[t]{0.49\textwidth}
%   \centering
%   \includegraphics[width=\linewidth,
%                    trim={2cm 1.4cm 0 3.4cm},clip]
%                    {figures/sae_feature_accuracies_layer31_pre_SQUAD_train.png}
%   \caption{Out-of-distribution comparison on layer 31, trained on SQuAD.}
%   \label{fig:sae-probe_pre31-squad}
% \end{subfigure}
% \hfill
% \begin{subfigure}[t]{0.49\textwidth}
%   \centering
%   \includegraphics[width=\linewidth,
%                    trim={2cm 1.4cm 0 3.4cm},clip]
%                    {figures/sae_feature_accuracies_layer31_pre_BoolQ_train.png}
%   \caption{Out-of-distribution comparison on layer 31, trained on BoolQ.}
%   \label{fig:sae-probe_pre31-boolq}
% \end{subfigure}

% \vspace{0.5em}  % Tighten or remove if desired

% %--------------- Second row ---------------%
% \begin{subfigure}[t]{0.49\textwidth}
%   \centering
%   \includegraphics[width=\linewidth,
%                    trim={0 1.4cm 0 3.4cm},clip]
%                    {figures/hierarchical_sae_probe_layer31_pre.png}
%   \caption{Combinations of SAE features.}
%   \label{fig:sae-k-pre31}
% \end{subfigure}
% \hfill
% \begin{subfigure}[t]{0.30\textwidth}
%   \centering
%   \includegraphics[width=\linewidth,
%                    trim={5.2cm 4.1cm 0 3.4cm},clip]
%                    {figures/feature_similarities.png}
%   \caption{Cosine similarities of top SAE features and probes.}
%   \label{fig:similarities}
% \end{subfigure}

% \caption{
%   Four related plots: 
%   (a) and (b) show out-of-distribution comparisons on layer 31 
%   for SQuAD vs.\ BoolQ training;
%   (c) demonstrates combinations of SAE features;
%   (d) shows feature/probe cosine similarities.
% }
% \label{fig:all_four}
% \end{figure*}




\section{Evaluation}



In the following, we present of our main experiments; see the appendix for additional findings.

% Todo
% Get some formatted dataset examples for all datasets
% Try question in celeb dataset (how old)
% Check probe accuracy and generalization for different training data sizes
% Averaging linear probes as baseline?
% Use higher k
% Prioritize over weekend with higher number of features + OOD eval
% Baseline attention probe
% Change cosine similarity analysis to average absolute
% Add clean reconstruction result of residual stream probe with SAE
% Cosine sim of residual stream probes with top SAE features
% Probes with each other
% Probes with top sae features
% Check pre vs post relu
% Compare features to narrow SAE
% Compare to Layer 31 SAE
% Is it more sparse? Less combinations needed? Then it is a feature that is not fully computed in L20?




\myparagraph{Linear vs SAE Probes: Generalization, Figure~\ref{fig:sae-probe_pre31}} 
We focus first on layer 31.
In domain, the best SAE features reach an accuracy of around 0.8 while the linear probes reach 0.9. Note that this is not surprising, since the probes have more parameters that are actually trained and thus optimized for this data. Nevertheless, it shows some advantage of probes in case in-domain data is available. 

We see rather great variation across our out-of-distribution datasets. Our custom Equation data stands out in that several SAE features and also the probe reach high performance. While this seems to show that the mathematical context makes answerability easier to detect, observe that the performance is considerably worse on layer 20, see Figure~\ref{fig:sae-probe_pre20} in the appendix. %This shows that it is a complex concept only fully represented later in the model.

Some, but few, top SAE features reach considerable performance out of distribution on IDK - matching the performance of the linear probe - and Celeb. Yet, the performance on BoolQ is considerably bad. %\vt{REASONS?} 
On the other hand, the linear probe performs bad on Celeb. Figure~\ref{fig:res-probe-all} in the Appendix shows the median value over 10 bootstrap samples including quartiles in the error bars; overall it correlates with performance.

For layer 20 (Figure~\ref{fig:sae-probe_pre20}), we see generally worse performance. 
Interestingly, the numbers for Celeb are significantly worse than all others for both the SAE and the linear probes. Since we see one exception (an SAE feature with higher than random performance), we hypothesize that there are special features encoding knowledge about celebrities which do not happen to be among our top features. % ie rather than the knowledge is not present on layer 20
%this might also explain why the probe performs bad on celeb that this would be more about people features not answerability
In fact, a closer investigation reveals that there are good features for BoolQ and our domain-specific Equation and Celeb datasets on layer 20 already (see Figure~\ref{fig:top-in-domain} in the appendix), but they are not the same features as the ones found by training on SQuAD. %TODO

Finally, we confirmed our findings by training on  BoolQ (also 2k samples) and evaluating on the other datasets. We mainly see that varying the training data can make everything considerably worse, even with the same task and seemingly similar, but potentially lower-quality data. The unanswerable samples in BoolQ were constructed by combining contexts and questions of similar dataset samples, hence capture only one type of unanswerability.

Overall, our experiments demonstrate one main critical issue with OOD data: \emph{the standard procedure for finding good SAE features can easily fail, even if good features are available}. 
The fact that good features exist while the linear probes also fail shows some potential of SAEs. Yet finding good, generalizing features represents an open challenge.

%A key issue for SAE probing could be feature splitting, a phenomenon where SAEs of different sizes learn features in different granularities, often splitting more general features into multiple more specialized features \citep{bricken2023monosemanticity, chanin2024absorption}. If abstract concepts like answerability are split into many separate features, this can cause problems for feature-based practical applications.

\myparagraph{Top Features, %Distributions, 
Figure~\ref{fig:sae-probe_pre31}} 
Interestingly, the top three features on the in-domain SQuAD data happen to also generalize better here. This does not turn out be the case beyond the top-1 feature more generally, see Appendix~\ref{app:eval-other-layers}. For BoolQ, the variability of the results precludes clear conclusions.
%

\myparagraph{SAE Feature Combinations, Figure~\ref{fig:sae-k-pre31}} 
Given the partly domain-specific nature of our out-of-distribution datasets, we hypothesized that combinations of features might work better as general probes. %Observe that such combinations have also been considered in previous works \cite{lecun,oneother}.
However, while increasing the number of SAE features improves the in-domain performance, OOD performance doesn't improve upon the best performing individual feature (top of blue error bars) here; layer 31, pre-activation. Other examples in Appendix~\ref{app:feature-combinations} show similar trends, and even some degradation. This underlines our above finding that the ood setting requires better methods for SAE feature search.

% \begin{wrapfigure}
% {r}{0.25\textwidth}

\begin{figure}[h!]
    \centering %trim titles
    \includegraphics[width=.18\textwidth,trim={5.2cm 4.1cm 0 3.4cm},clip]{figures/feature_similarities.png}
    \caption{Cosine similarities of top SAE features and linear probes for different seeds; the blue square shows high similarity between linear probes.}
\label{fig:similarities}
\end{figure}
  
% \end{wrapfigure}
\myparagraph{Feature Similarity, Figures~\ref{fig:similarities} \& \ref{fig:similarity_k}}

We find great similarity between different linear probes but only slight similarity between SAE features and individual probes, and it's even less between SAE features. Interestingly, the best SAE feature turns out to have highest (though low) similarity with the probes. Figure~\ref{fig:similarity_k} shows that combining SAE features yields greater similarity with linear probes.




% \myparagraph{Other experiments}
% We validated our setup by searching for bias-related features as it was done in related works.
% We also experimented with (inofficial) SAEs for an instruction-tuned Llama model, but the quality of the SAEs was not good enough for further experimentation. Finally, Gemma 2 2B and also the base models \vt{did not yield good enough performance on the question answering task itself}. 

\section{Conclusion}
\label{sec:conclusion}
\section{Conclusion and Suggestions}

Our work, including the creation of \texttt{ScholarLens} and the proposal of \texttt{LLMetrica}, provides methods for assessing LLM penetration in scholarly writing and peer review. By incorporating diverse data types and a range of evaluation techniques, we consistently observe the growing influence of LLMs across various scholarly processes, raising concerns about the credibility of academic research. As LLMs become more integrated into scholarly workflows, it is crucial to establish strategies that ensure their responsible and ethical use, addressing both content creation and the peer review process. 

Despite existing guidelines restricting LLM-generated content in scholarly writing and peer review,\footnote{\href{https://aclrollingreview.org/acguidelines\#-task-3-checking-review-quality-and-chasing-missing-reviewers}{Area Chair} \&  \href{https://aclrollingreview.org/reviewerguidelines\#q-can-i-use-generative-ai}{Reviewer} \& \href{https://www.aclweb.org/adminwiki/index.php/ACL_Policy_on_Publication_Ethics\#Guidelines_for_Generative_Assistance_in_Authorship}{Author} guidelines.} challenges still remain. 
To address these, we propose the following based on our work and findings: 
(i) \textbf{Increase transparency in LLM usage within scholarly processes} by incorporating LLM assistance into review checklists, encouraging explicit acknowledgment of LLM support in paper acknowledgments, and 
reporting LLM usage patterns across diverse demographic groups;
% reporting LLM penetration based on social demographic features;
(ii) \textbf{Adopt policies to prevent irresponsible LLM reviewers} by establishing feedback channels for authors on LLM-generated reviews and developing fine-grained LLM detection models~\cite{abassy-etal-2024-llm, cheng2024beyond, artemova2025beemobenchmarkexperteditedmachinegenerated} to distinguish acceptable LLM roles (e.g., language improvement vs. content creation);
(iii) \textbf{Promote data-driven research in scholarly processes} by supporting the collection of review data for further robust analysis~\cite{dycke-etal-2022-yes}.\footnote{\url{https://arr-data.aclweb.org/}}

% make LLM usage transparent in scholarly processes: such as incorporating LLM usage into review checklists, encouraging explicit acknowledgment of LLM assistance in paper acknowledgments, and reporting LLM penetration based on social demographic features; (ii) Adopt policies to prevent irresponsible LLM reviewers: such as providing authors feedback on LLM-assisted reviews, and developing fine-grained LLM detection models~\cite{cheng2024beyond} to distinguish acceptable LLM roles (e.g., language improvement vs. content creation); (iii) Encourage data-driven research in scholarly processes: such as supporting review data collection for further research.

 



%% The acknowledgments section.
\begin{acks}
I would like to thank Prof. Kishore Kothapalli, Prof. Sathya Peri, and Prof. Dip Sankar Banerjee for their support.
\end{acks}

%% Bibliography style to be used, and the bibliography file.
\bibliographystyle{ACM-Reference-Format}
\bibliography{main}

\clearpage
\clearpage
\appendix
\section{Appendix}

\subsection{Memory Allocators}
\label{sec:memory-allocators}

We now discuss a few memory allocators, which support the design of our graph representation.

\begin{algorithm}[H] % hbtp
\caption{A Fixed-capacity Arena Allocator (FAA).}
\label{alg:faa}
\begin{algorithmic}[1]
\Require{$\textsc{alloc\_size}_a$: Size of each allocation (constant)}
\Require{$\textsc{pool\_size}_a$: Size of the memory block/pool (constant)}
\Require{$pool_a$: A memory block for the pool}

\Statex

\State \textbf{struct} $FAA \langle \textsc{alloc\_size}_a, \textsc{pool\_size}_a \rangle (pool_a)$
\State \ \ $\textsc{alloc\_size} \gets \textsc{alloc\_size}_a$ \Comment{Size of each allocation} \label{alg:faa--init-const-begin}
\State \ \ $\textsc{pool\_size} \gets \textsc{pool\_size}_a$ \Comment{Size of the memory pool} \label{alg:faa--init-const-end}
\State \ \ $freed \gets \{\}$ \Comment{Freed allocations, can be reused} \label{alg:faa--init-begin}
\State \ \ $used \gets 0$ \Comment{Bytes used in the memory pool}
\State \ \ $pool \gets pool_a$ \Comment{The memory pool} \label{alg:faa--init-end}

\Statex

\State $\rhd$ Allocate memory, of size $A.\textsc{alloc\_size}$
\Function{allocate}{$ $} \textbf{of} FAA
  \State $\rhd$ Allocate from freed list, if available
  \If{$freed \neq \{\}$} \ReturnInline{$freed.pop()$} \label{alg:faa--allocate-freed}
  \EndIf
  \State $\rhd$ Allocate from pool
  \If{$used < \textsc{pool\_size}$} \label{alg:faa--allocate-pool-begin}
    \State $ptr \gets pool + used$
    \State $used \gets used + \textsc{alloc\_size}$
    \Return{$ptr$}
  \EndIf \label{alg:faa--allocate-pool-end}
  \Return{$\phi$} \label{alg:faa--allocate-phi}
\EndFunction

\Statex

\State $\rhd$ Free allocated memory
\Function{deallocate}{$ptr$} \textbf{of} FAA
  \State $freed.push(ptr)$ \label{alg:faa--deallocate-push}
\EndFunction

\Statex

\State $\rhd$ Free all allocated memory
\Function{reset}{$ $} \textbf{of} FAA
  \State $freed \gets \{\}$ \label{alg:faa--reset-begin}
  \State $used \gets 0$ \label{alg:faa--reset-end}
\EndFunction
\end{algorithmic}
\end{algorithm}



\subsubsection{Fixed Arena Allocator (FAA)}
\label{sec:faa}

We now discuss the Fixed-capacity Arena Allocator (FAA), outlined in Algorithm \ref{alg:faa}, which provides a lightweight, efficient memory allocation strategy for scenarios requiring frequent allocations and deallocations within a fixed memory budget. FAA minimizes fragmentation by recycling memory blocks and allows for rapid reset operations, making it well-suited for high-performance applications. It operates through three primary functions: \texttt{allocate()}, \texttt{deallocate()}, and \texttt{reset()}.

The allocator is instantiated with a predefined allocation size $\textsc{alloc\_size}_a$ and a total pool size $\textsc{pool\_size}_a$ (lines \ref{alg:faa--init-const-begin}-\ref{alg:faa--init-const-end}). The allocator maintains an internal memory pool $pool$, a counter $used$ tracking the number of allocated bytes, and a list $freed$ to store deallocated memory chunks for reuse (lines \ref{alg:faa--init-begin}-\ref{alg:faa--init-end}). Memory allocation is handled by the \texttt{allocate()} function. First, if any previously freed memory blocks exist in $freed$, one is retrieved and returned (line \ref{alg:faa--allocate-freed}). Otherwise, allocation proceeds from the main memory pool. If there is available space in the pool (i.e., $used < \textsc{pool\_size}$), the function assigns a pointer to the next available block, updates the $used$ counter, and returns the pointer (lines \ref{alg:faa--allocate-pool-begin}-\ref{alg:faa--allocate-pool-end}). If the pool is exhausted, the function returns a null pointer $\phi$, indicating allocation failure (line \ref{alg:faa--allocate-phi}). The \texttt{deallocate()} function allows memory to be freed by pushing the given pointer into the $freed$ list, making it available for future reuse (line \ref{alg:faa--deallocate-push}). To completely reset the allocator, the \texttt{reset()} function clears the $freed$ list and resets the $used$ counter to zero, effectively deallocating everything (lines \ref{alg:faa--reset-begin}-\ref{alg:faa--reset-end}).


\subsubsection{Variable-capacity Arena Allocator (AA)}
\label{sec:aa}

Unlike the Fixed-capacity Arena Allocator (FAA), the variable-capacity Arena Allocator (AA) does not have a fixed memory budget. Instead, it dynamically allocates additional pools as needed using \texttt{new[]} from the C++ standard library. Its pseudocode is shown in Algorithm \ref{alg:aa} and, like the FAA, it operates through three primary functions: \texttt{allocate()}, \texttt{deallocate()}, and \texttt{reset()}.

\begin{algorithm}[hbtp]
\caption{A variable-capacity Arena Allocator (AA).}
\label{alg:aa}
\begin{algorithmic}[1]
\Require{$\textsc{alloc\_size}_a$: Size of each allocation (constant)}
\Require{$\textsc{pool\_size}_a$: Size of each memory pool (constant)}

\Statex

\State \textbf{struct} $AA \langle \textsc{alloc\_size}_a, \textsc{pool\_size}_a \rangle ()$
\State \ \ $\textsc{alloc\_size} \gets \textsc{alloc\_size}_a$ \Comment{Size of each allocation} \label{alg:aa--init-const-begin}
\State \ \ $\textsc{pool\_size} \gets \textsc{pool\_size}_a$ \Comment{Size of each memory pool} \label{alg:aa--init-const-end}
\State \ \ $freed \gets \{\}$ \Comment{Freed allocations, can be reused} \label{alg:aa--init-begin}
\State \ \ $used \gets \textsc{pool\_size}$ \Comment{Bytes used in the last memory pool}
\State \ \ $pools \gets \{\}$ \Comment{Memory pools} \label{alg:aa--init-end}

\Statex

\State $\rhd$ Allocate memory, of size $\textsc{alloc\_size}$
\Function{allocate}{$ $} \textbf{of} AA
  \State $\rhd$ Allocate from freed list, if available
  \If{$freed \neq \{\}$} \ReturnInline{$freed.pop()$} \label{alg:aa--allocate-freed}
  \EndIf
  \State $\rhd$ Allocate from pool
  \If{$used < \textsc{pool\_size}$} \label{alg:aa--allocate-pool-begin}
    \State $ptr \gets pools.last() + used$
    \State $used \gets used + \textsc{alloc\_size}$
    \Return{$ptr$}
  \EndIf \label{alg:aa--allocate-pool-end}
  \State $\rhd$ Allocate a new pool
  \State $ptr \gets$ Allocate $\textsc{pool\_size}$ bytes \label{alg:aa--allocate-newpool}
  \If{$ptr \neq \phi$} \label{alg:aa--allocate-new-begin}
    \State $pools.push(ptr)$
    \State $used \gets \textsc{alloc\_size}$
    \Return{$ptr$}
  \EndIf \label{alg:aa--allocate-new-end}
  \Return{$\phi$} \label{alg:aa--allocate-fail}
\EndFunction

\Statex

\State $\rhd$ Free allocated memory
\Function{deallocate}{$ptr$} \textbf{of} FAA
  \State $freed.push(ptr)$ \label{alg:aa--deallocate-push}
\EndFunction

\Statex

\State $\rhd$ Free all allocated memory
\Function{reset}{$ $} \textbf{of} FAA
  \State $freed \gets \{\}$ \label{alg:aa--reset-freedused-begin}
  \State $used \gets 0$ \label{alg:aa--reset-freedused-end}
  \ForAll{$ptr \in pools$} \label{alg:aa--reset-pools-begin}
    \State Free memory at $ptr$
  \EndFor \label{alg:aa--reset-pools-end}
\EndFunction
\end{algorithmic}
\end{algorithm}


As with FAA, the allocator $AA$ is initialized with a predefined allocation size $\textsc{alloc\_size}_a$ and memory pool size $\textsc{pool\_size}_a$ (lines \ref{alg:faa--init-const-begin}-\ref{alg:faa--init-const-end}). Additionally, it maintains a list of reusable freed allocations $freed$, tracks the number of used bytes in the most recent pool $used$, and stores references to all allocated pools in $pools$ (lines \ref{alg:aa--init-begin}-\ref{alg:aa--init-end}). Memory allocation, handled by \texttt{allocate()}, follows a three-step process. \textbf{(1)} If freed allocations are available, they are reused (line \ref{alg:faa--allocate-freed}) to minimize new allocations. \textbf{(2)} If space remains in the last allocated pool, the next block is assigned, and the used counter updates accordingly (lines \ref{alg:faa--allocate-pool-begin}-\ref{alg:faa--allocate-pool-end}). \textbf{(3)} If neither condition holds, a new pool of size $\textsc{pool\_size}$ is allocated. If successful, it is added to $pools$, and allocation proceeds (lines \ref{alg:aa--allocate-new-begin}-\ref{alg:aa--allocate-new-end}); otherwise, the function returns $\phi$ to signal failure (line \ref{alg:aa--allocate-fail}). The \texttt{deallocate()} function, as earlier, allows memory to be freed by pushing freed pointers onto the $freed$ list for reuse (line \ref{alg:aa--deallocate-push}). To reset the allocator, \texttt{reset()} clears the $freed$ list, resets the $used$ counter, and releases all allocated $pools$ back to the standard library (lines \ref{alg:aa--reset-freedused-begin}-\ref{alg:aa--reset-pools-end}).


\subsubsection{Power-of-2 size Arena Allocator (P2AA)}
\label{sec:p2aa}

Unlike the AA allocator, which only supports allocating memory blocks of a fixed size, the Power-of-2 Arena Allocator (P2AA) allows arbitrary allocation sizes, specializing in handling memory allocations that are powers of two --- similar to a slab allocator \cite{bonwick1994slab}. The pseudocode for P2AA allocator is shown in Algorithm \ref{alg:p2aa}, and it builds upon the AA allocator by maintaining separate AA sub-allocators for each power-of-2 size, from $16$ bytes to $8192$ bytes. The $8192$-byte limit is chosen empirically for optimal performance.

\begin{algorithm}[hbtp]
\caption{A Power-of-2 size Arena Allocator (P2AA).}
\label{alg:p2aa}
\begin{algorithmic}[1]
\Require{$\textsc{pool\_size}_a$: Size of each memory pool (constant)}

\Statex

\State \textbf{struct} $P2AA \langle \textsc{pool\_size}_a \rangle ()$
\State \ \ $\textsc{pool\_size} \gets \textsc{pool\_size}_a$ \Comment{Size of each memory pool} \label{alg:p2aa--init-const}
\State \ \ $aa16 \gets AA \langle 16, \textsc{pool\_size} \rangle ()$ \Comment{For $16$-byte allocations} \label{alg:p2aa--init-begin}
\State \ \ $aa32 \gets AA \langle 32, \textsc{pool\_size} \rangle ()$ \Comment{For $32$-byte allocations}
\State \ \ $\cdots$ \Comment{For $64$ to $4096$-byte allocations}
\State \ \ $aa8192 \gets AA \langle 8192, \textsc{pool\_size} \rangle ()$ \Comment{For $8192$-byte allocations} \label{alg:p2aa--init-end}

\Statex

\State $\rhd$ Allocate memory of size that is a power of 2
\Function{allocate}{$size$} \textbf{of} P2AA
  \State \textbf{switch} $(size)$
  \State \ \ \textbf{case} $16$: \ReturnInline{$aa16.allocate()$} \label{alg:p2aa--allocate-pow2-begin}
  \State \ \ \textbf{case} $32$: \ReturnInline{$aa32.allocate()$}
  \State \ \ $\cdots$
  \State \ \ \textbf{case} $8192$: \ReturnInline{$aa8192.allocate()$} \label{alg:p2aa--allocate-pow2-end}
  \State \ \ \textbf{default}: \ReturnInline{Allocate $size$ bytes} \label{alg:p2aa--allocate-other}
\EndFunction

\Statex

\State $\rhd$ Free allocated memory
\Function{deallocate}{$ptr, size$} \textbf{of} P2AA
  \State \textbf{switch} $(size)$
  \State \ \ \textbf{case} $16$: $aa16.deallocate(ptr)$ \label{alg:p2aa--deallocate-pow2-begin}
  \State \ \ \textbf{case} $32$: $aa32.deallocate(ptr)$
  \State \ \ $\cdots$
  \State \ \ \textbf{case} $8192$: $aa8192.deallocate(ptr)$ \label{alg:p2aa--deallocate-pow2-end}
  \State \ \ \textbf{default}: Free memory at $ptr$ \label{alg:p2aa--deallocate-other}
\EndFunction

\Statex

\State $\rhd$ Free all allocated memory
\Function{reset}{$ $} \textbf{of} P2AA
  \State $aa16.reset()$ \label{alg:p2aa--reset-begin}
  \State $aa32.reset()$
  \State $\cdots$
  \State $aa8192.reset()$ \label{alg:p2aa--reset-end}
\EndFunction

\Statex

\State $\rhd$ Get recommended allocation size, for a desired size
\Function{allocationSize}{$size$} \textbf{of} P2AA
  \If{$size \leq 16$} \ReturnInline{$16$} \label{alg:p2aa--allocationsize-16}
  \ElsIf{$size < 8192$} \ReturnInline{$nextPow2(size)$} \label{alg:p2aa--allocationsize-8192}
  \Else\ \ReturnInline{$\lceil size / \textsc{page\_size} \rceil * \textsc{page\_size}$} \label{alg:p2aa--allocationsize-other}
  \EndIf
\EndFunction
\end{algorithmic}
\end{algorithm}


At initialization, the allocator $P2AA$ is configured with a predefined memory pool size $\textsc{pool\_size}_a$ (line \ref{alg:p2aa--init-const}). It creates multiple AA sub-allocators, each responsible for a different power-of-2 allocation size, ranging from $16$ to $8192$ bytes (lines \ref{alg:p2aa--init-begin}-\ref{alg:p2aa--init-end}). Memory allocation is performed in the \texttt{allocate()} function. When a request for a standard power-of-2 size is made (e.g., $16$, $32$, or $8192$ bytes), the request is routed to corresponding AA sub-allocator handles the request (lines \ref{alg:p2aa--allocate-pow2-begin}-\ref{alg:p2aa--allocate-pow2-end}). For sizes larger than $8192$ bytes or non-power-of-2 sizes, a direct memory allocation is performed instead (line \ref{alg:p2aa--allocate-other}). Deallocation mirrors the allocation strategy. Each deallocation request, made through \texttt{deallocate()}, is dispatched to the corresponding sub-allocator, if it is a power-of-two, and ranges from $16$ to $8192$ bytes in size (lines \ref{alg:p2aa--deallocate-pow2-begin}-\ref{alg:p2aa--deallocate-pow2-end}). For other sizes, memory is freed directly using \texttt{delete[]} (line \ref{alg:p2aa--deallocate-other}). Note that, unlike \texttt{delete[]}, the user is expected to provide the size of the allocated memory. To reset all allocated memory, the \texttt{reset()} function (lines \ref{alg:p2aa--reset-begin}-\ref{alg:p2aa--reset-end}) resets each individual arena allocator. This effectively clears all memory pools, making the allocator ready for fresh allocations without explicitly deallocating each individual allocation. Finally, the \texttt{allocationSize()} function provides a mechanism for determining the optimal allocation size for a given request. If the request is $16$ bytes or smaller, it returns $16$ (line \ref{alg:p2aa--allocationsize-16}). If the request is between $17$ and $8191$ bytes, it rounds up to the next power of two using \texttt{nextPow2()} (line \ref{alg:p2aa--allocationsize-8192}). For larger requests, it rounds up to the nearest multiple of the system page size (line \ref{alg:p2aa--allocationsize-other}). This ensures that large memory allocations are page-aligned.


\subsubsection{Concurrent Power-of-2 Arena Allocator (CP2AA)}
\label{sec:cp2aa}

We now discuss our thread-safe Concurrent Power-of-2 Arena Allocator (CP2AA), which builds upon per-thread instances of the Power-of-2 Arena Allocator (P2AA). Its pseudocode is presented in Algorithm \ref{alg:cp2aa}. As above, CP2AA consists of four primary functions: \texttt{allocate()}, \texttt{deallocate()}, \texttt{reset()}, and \texttt{allocationSize()}.

\begin{algorithm}[hbtp]
\caption{Our Concurrent Pow-of-2 Arena Allocator (CP2AA).}
\label{alg:cp2aa}
\begin{algorithmic}[1]
\Require{$\textsc{pool\_size}_a$: Size of each memory pool (constant)}

\Statex

\State \textbf{struct} $CP2AA \langle \textsc{pool\_size}_a \rangle ()$
\State \ \ $\textsc{pool\_size} \gets \textsc{pool\_size}_a$ \Comment{Size of each memory pool} \label{alg:cp2aa--init-const}
\State \ \ $p2aa_T \gets \{P2AA \langle \textsc{pool\_size} \rangle (),\ \dots\}$ \Comment{Per-thread allocator} \label{alg:cp2aa--init}

\Statex

\State $\rhd$ Allocate memory of size that is a power of 2
\Function{allocate}{$size$} \textbf{of} CP2AA
  \State $t \gets$ Current thread \label{alg:cp2aa--allocate-begin}
  \Return{$p2aa_T[t].allocate(size)$} \label{alg:cp2aa--allocate-end}
\EndFunction

\Statex

\State $\rhd$ Free allocated memory
\Function{deallocate}{$ptr, size$} \textbf{of} CP2AA
  \State $t \gets$ Current thread \label{alg:cp2aa--deallocate-begin}
  \State $p2aa_T[t].deallocate(ptr, size)$ \label{alg:cp2aa--deallocate-end}
\EndFunction

\Statex

\State $\rhd$ Free all allocated memory
\Function{reset}{$ $} \textbf{of} CP2AA
  \ForAll{$t \in threads$} \label{alg:cp2aa--reset-begin}
    \State $p2aa_T[t].reset()$
  \EndFor \label{alg:cp2aa--reset-end}
\EndFunction

\Statex

\State $\rhd$ Get recommended allocation size, for a desired size
\Function{allocationSize}{$size$} \textbf{of} CP2AA
  \Return{$P2AA \langle \textsc{pool\_size} \rangle.allocationSize(size)$} \label{alg:cp2aa--allocationsize}
\EndFunction
\end{algorithmic}
\end{algorithm}

% NOTE: Per-thread allocators are well-separated in memory to avoid false sharing.


The $CP2AA$ allocator is initialized with a fixed memory pool size $\textsc{pool\_size}_a$ (line \ref{alg:cp2aa--init-const}), and maintains a separate instance of P2AA per thread, stored in $p2aa_T$ (line \ref{alg:cp2aa--init}) --- while ensuring that each P2AA allocator is well-separated in memory to prevent false sharing. Each thread exclusively interacts with its corresponding P2AA instance, eliminating the need for locks or atomic operations altogether. The \texttt{allocate()} function assigns memory in a thread-local manner. Given a requested size, it identifies the current thread $t$ and delegates the allocation to its corresponding P2AA instance, $p2aa_T[t]$ (lines \ref{alg:cp2aa--allocate-begin}-\ref{alg:cp2aa--allocate-end}). Similarly, \texttt{deallocate()} retrieves the current thread $t$ and forwards the deallocation request to the P2AA allocator of the current thread (lines \ref{alg:cp2aa--deallocate-begin}-\ref{alg:cp2aa--deallocate-end}). Note that it is acceptable for a thread to deallocate memory allocated by another thread. In fact, this lack of restriction is a key source of performance improvement. The \texttt{reset()} function deallocates all of the allocated memory by iterating over all per-thread allocators and calling their respective reset functions (lines \ref{alg:cp2aa--reset-begin}-\ref{alg:cp2aa--reset-end}). Finally, the \texttt{allocationSize()} function determines the appropriate power-of-2 allocation size for a given request by forwarding the query to P2AA (line \ref{alg:cp2aa--allocationsize}).

\ignore{Our initial attempt at a concurrent arena allocator used an atomic\_flag mutex to manage freed blocks, but high contention resulted in slow deallocation, especially with 64 threads.}


\subsubsection{Performance Comparison}

\begin{figure}[hbtp]
  \centering
  \subfigure[Allocation-only workload: $2^{28}$ allocations of 64 bytes.]{
    \label{fig:allocator-alloc--runtime}
    \includegraphics[width=0.98\linewidth]{out/allocator-alloc-runtime.pdf}
  }
  \subfigure[Deallocation-only workload: $2^{28}$ deallocations of allocated memory.]{
    \label{fig:allocator-free--runtime}
    \includegraphics[width=0.98\linewidth]{out/allocator-free-runtime.pdf}
  }
  \subfigure[Mixed workload: $2^{22}$ allocations and $2^{22}$ deallocations, $64$ times.]{
    \label{fig:allocator-mixed--runtime}
    \includegraphics[width=0.98\linewidth]{out/allocator-mixed-runtime.pdf}
  } \\[-2ex]
  \caption{Relative Runtime of memory allocators across three workloads: \textit{(a)} \textit{allocation-only}, where $2^{28}$ allocations of $64$ bytes each are performed; \textit{(b)} \textit{deallocation-only}, with $2^{28}$ deallocations; and \textit{(c)} mixed, with $2^{22}$ allocations followed by $2^{22}$ deallocations, repeated $64$ times. The allocators include the C library allocator ($malloc()/free()$), C++ runtime allocator ($new[]/delete[]$), Fixed Arena Allocator (FAA), variable-capacity Arena Allocator (AA), and Concurrent Power-of-2 Arena Allocator (CP2AA). Details on FAA, AA, and CP2AA allocators is given in Section \ref{sec:memory-allocators}.}
  \label{fig:allocator-runtime}
\end{figure}
% This is actually three separate experiments, which aim the measure the performance of memory allocators in different scenarios/workloads. In the first experiment, we consider an allocation-only workload, where each allocator is stress tested with repeated memory allocations ($2^28$ allocations in total) of each $64$ bytes, and the total time taken to do so is measured. The allocated memory addresses in this case are stored in an array. In the second experiment, we consider an deallocation-only workload, where each allocator is now tested with repeated memory deallocations (of the memory addresses obtained from the first experiment), and the total time taken to do so is measured. Finally, in the third experiment, we consider a mixed workload, which consists of $2^22$ allocations, followed by $2^22$ deallocations, both of which are repeated $64$ times. In all experiments, the performance of C library allocator ($malloc()/free()$), C++ runtime allocators ($new[]/delete[]$), Fixed Arena Allocator (FAA), variable-capacity Arena Allocator (AA), and Concurrent Power-of-2 Arena Allocator (CP2AA). The above figure shows the relative runtime of each of the allocators in each of the three scenarios/experiments.
% Allocation-only workload: Total runtime of $2^{28}$ allocations of 64 bytes for each allocator.
% Deallocation-only workload: Total runtime of $2^{28}$ deallocations, using the addresses from the allocation experiment, for each allocator.
% Mixed workload: Total runtime for $2^{22}$ allocations followed by $2^{22}$ deallocations, repeated 64 times for each allocator.


To evaluate the performance of different memory allocators, we conduct three separate experiments, each designed to measure execution time under specific memory management workloads. The allocators tested include the C library allocator (\texttt{malloc()}/\texttt{free()}), the C++ runtime allocator (\texttt{new[]}/\texttt{delete[]}), the Fixed Arena Allocator (FAA), the variable-capacity Arena Allocator (AA), and the Concurrent Power-of-2 Arena Allocator (CP2AA). In the first experiment, we measure the performance of each allocator in an allocation-heavy scenario. Here, each allocator is subject to a total of $2^{28}$ memory allocations, each of size $64$ bytes. With non-concurrent allocators, i.e., FAA and AA, separate allocator objects are created for each thread to ensure thread safety, and the workload is distributed evenly across all threads. The allocated memory addresses are stored in an array to ensure they can be deallocated in the subsequent experiment. In the second experiment, we evaluate the deallocation performance of each allocator. Using the memory addresses obtained from the first experiment, each allocator is tasked with deallocating $2^{28}$ blocks of memory. Deallocations are routed to the appropriate allocator object for each thread in the case of FAA and AA. The third experiment examines allocator performance in a mixed workload, where allocation and deallocation operations occur in succession. Specifically, each allocator is tested with $2^{22}$ allocations followed by $2^{22}$ deallocations, with this cycle being repeated $64$ times. This experiment provides insight into the performance of each allocator under workloads that mimic real-world application behavior.

Figure \ref{fig:allocator-alloc--runtime} presents the relative runtime of each allocator for the allocation-only workload, while Figure \ref{fig:allocator-free--runtime} shows the deallocation-only workload. Finally, Figure \ref{fig:allocator-mixed--runtime} illustrates the performance of each allocator under the mixed workload. The results indicate that specialized allocators like FAA, AA, and CP2AA offer substantial performance improvements over general-purpose allocators (\texttt{malloc()}/\texttt{new[]}), particularly in allocation-intensive workloads, and offer around $4\times$ speedup in mixed workloads. While the AA allocator performs the best, on average, it is not thread-safe and thus not suitable for concurrent applications. The CP2AA allocator, which is both thread-safe and high-performing, is suitable.



\subsection{Evaluating Graph Representations via Graph Algorithm Performance}

As dicusssed earlier, we evaluate the performance of various graph representations on a representative algorithm by measuring the efficiency of $k$-step reverse walks from each vertex in the given input graph, and counting the number of walks ending at each vertex. This corresponds to computing $A_T^k \cdot \vec{1}$, where $A_T$ is the transposed adjacency matrix and $\vec{1}$ is a ones vector. Reverse walks are preferred as they can be executed directly on the input graph, whereas forward walks require its transpose. The results of this evaluation on each graph representation, i.e., PetGraph, SuiteSparse:GraphBLAS, cuGraph, Aspen, and our DiGraph, are discussed in Section \ref{sec:perform-reverse-walks}.

We now give a short description of the algorithm. It propagates visit counts backward along graph edges, and its psuedocode is given in Algorithm \ref{alg:visit}. Here, the \texttt{reverseWalk()} function, takes a graph $G$ and the number of reverse walk $steps$ as input. Initially, two arrays, $visits0$ and $visits1$, are allocated to track visit counts, with $visits0$ set to $1$ for all vertices and $visits1$ initialized to $0$ (lines \ref{alg:visit--init-begin}-\ref{alg:visit--init-end}). The main loop iterates $steps$ times, processing all vertices $u$ in parallel. Each iteration resets $visits1[u]$ (line \ref{alg:visit--reset}) and accumulates the visit count from each neighbor $v$ of $u$ (lines \ref{alg:visit--edges-begin}-\ref{alg:visit--edges-end}). This iterative process propagates visit counts backward through the graph. At the end of each iteration, the visit count arrays are swapped (line \ref{alg:visit--swap}), ensuring $visits0$ contains the latest counts. After the final iteration, $visits0$ contains the number of reverse walks ending at each vertex\ignore{in the graph, and is returned}.

\begin{algorithm}[hbtp]
\caption{Perform reverse walk from each vertex.}
\label{alg:visit}
\begin{algorithmic}[1]
\Require{$G(V, E)$: Input graph}
\Ensure{$steps$: Number of reverse walks to perform}

\Statex

\Function{reverseWalk}{$G, steps$}
  \State $visits0 \gets \{1\}$ \label{alg:visit--init-begin}
  \State $visits1 \gets \{0\}$ \label{alg:visit--init-end}
  \ForAll{$i \in [0, steps)$} \label{alg:visit--steps-begin}
    \ForAll{$u \in V$ \textbf{in parallel}}
      \State $visits1[u] \gets 0$ \label{alg:visit--reset}
      \ForAll{$(v, \_) \in G.edges(u)$} \label{alg:visit--edges-begin}
        \State $visits[u] \gets visits[u] + visits0[v]$
      \EndFor \label{alg:visit--edges-end}
    \EndFor
    \State $swap(visits0, visits1)$ \label{alg:visit--swap}
  \EndFor \label{alg:visit--steps-end}
  \Return{$visits0$}
\EndFunction
\end{algorithmic}
\end{algorithm}


\end{document}
\endinput
%% End of file.




% Time taken to read and clone the graph
% Time taken to insert and delete edges (both in-place and copy-based)
% Why is clone slow?
% Present graph traversal performance with PageRank?
% We optimize for traversal performance
% Per-graph runtimes for each batch size
% Design of the AA allocator

% Already have 2 results, need 2 more to compare with
% I AM WAITING FOR PETGRAPH-MAP TO COMPLETE
% THEN I CAN RUN SNAP-READ BASED
% Also waiting for Snap (convert based), and Petgraph original

% OTHER COMPARISONS
% NetworkX is too slow (python)
% Gunrock is static only
% Ligra is static only
% Comparison with graph databases?


% WRITING IDEAS
% - Trying to optimize graph load, copy, and application of batch update
% - To optimize load, can integrate with CSR loader
% - Can start writing to get top-down view
% - Mention points from various sources
% - The various optimization steps

% PAPER STRUCTURE
% - The problem we are addressing
% - Why is it a problem - its description
% - Usefulness of the problem - solution - applications
% - The work that has been done
% - Critical review - any quick demo of the issues
% - Stating the problem clearly is very important
% - Proceed onto solution

% OTHERS
% - Update existing papers (on arxiv) - clear problem presentation


% NEW DETAILS TO INCLUDE
% 1. Discuss details from allocator issue.
% 2. Discuss details from graph-openmp issue.
% 3. Also discuss details from GVEL.
% 4. Present framework diagram (motivating).
% 5. Discuss how allocation overhead is important.
% 6. And how perf. of algorithms is important.
% 7. Present how memory allocation is a bottleneck wrt. earlier implementation.
% 8. Present a figure on graph union and subtraction for batch update.
% 9. Discuss key details which help improve performance.
% 10. Discuss that allocator is shared.
