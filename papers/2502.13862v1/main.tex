\documentclass[sigconf,nonacm]{acmart}

%% Enable subfigures
\usepackage{subfigure}
%% Enable numbers in scientific format.
\usepackage{siunitx}
%% Enable enumerate start from.
\usepackage{enumitem}

%% Enable theorems
\newtheorem{theorem}{Theorem}[section]
\newtheorem{lemma}[theorem]{Lemma}

%% Enable algorithms
\usepackage{algorithm}
\usepackage[noend]{algpseudocode}
\let\ReturnInline\Return
\renewcommand{\Return}{\State\ReturnInline}
\algrenewcommand\algorithmicrequire{$\rhd$}
\algrenewcommand\algorithmicensure{$\square$}

%% Fonts used in the template cannot be substituted; margin 
%% adjustments are not allowed.
\AtBeginDocument{%
  \providecommand\BibTeX{{%
    \normalfont B\kern-0.5em{\scshape i\kern-0.25em b}\kern-0.8em\TeX}}}

%% Rights management information.
\setcopyright{acmcopyright}
\copyrightyear{2018}
\acmYear{2018}
\acmDOI{XXXXXXX.XXXXXXX}

%% These commands are for a PROCEEDINGS abstract or paper.
\acmConference[Conference acronym 'XX]{Make sure to enter the correct
  conference title from your rights confirmation emai}{June 03--05,
  2018}{Woodstock, NY}
%% Title of the proceedings is different from ``Proceedings of ...''?
% \acmBooktitle{Woodstock '18: ACM Symposium on Neural Gaze Detection,
%  June 03--05, 2018, Woodstock, NY} 
% \acmPrice{15.00}
% \acmISBN{978-1-4503-XXXX-X/18/06}

%% Submission ID.
% \acmSubmissionID{123-A56-BU3}

%% Use the "author year" style of citations and references?
% \citestyle{acmauthoryear}

%% Message
\newcommand{\kk}[1]{{{\color{red} #1}}}
\newcommand{\ds}[1]{{{\color{red} #1}}}
\newcommand{\su}[1]{{{\color{red} #1}}}

%% Ignore block
\newcommand{\ignore}[1]{}

%% Macros
\newcommand{\ours}{$\nu$-LPA}




\begin{document}

%% Full title of the paper.
\title[Performance Comparison of Graph Representations Which Support Dynamic Graph Updates]{Performance Comparison of Graph Representations \\Which Support Dynamic Graph Updates}

%% Short title to be used in page headers (optional).
% \title[short title]{full title}
% \subtitle{Something other than the title}

%% Authors and their affiliations.
\author{Subhajit Sahu}
\email{subhajit.sahu@research.iiit.ac.in}
\affiliation{%
  \institution{IIIT Hyderabad}
  \streetaddress{Professor CR Rao Rd, Gachibowli}
  \city{Hyderabad}
  \state{Telangana}
  \country{India}
  \postcode{500032}
}

%% Concise author list in page headers.
%\renewcommand{\shortauthors}{Sahu, Kothapalli, and Banerjee, et al.}

%% Show page numbers.
\settopmatter{printfolios=true}

%% Short summary of the work to be presented in the article.
\begin{abstract}
Research in graph-structured data has grown rapidly due to graphs' ability to represent complex real-world information and capture intricate relationships, particularly as many real-world graphs evolve dynamically through edge/vertex insertions and deletions. This has spurred interest in programming frameworks for managing, maintaining, and processing such dynamic graphs. In this report, we evaluate the performance of PetGraph (Rust), Stanford Network Analysis Platform (SNAP), SuiteSparse:GraphBLAS, cuGraph, Aspen, and our custom implementation in tasks including loading graphs from disk to memory, cloning loaded graphs, applying in-place edge deletions/insertions, and performing a simple iterative graph traversal algorithm. Our implementation demonstrates significant performance improvements: it outperforms PetGraph, SNAP, SuiteSparse:GraphBLAS, cuGraph, and Aspen by factors of $177\times$, $106\times$, $76\times$, $17\times$, and $3.3\times$ in graph loading; $20\times$, $235\times$, $0.24\times$, $1.3\times$, and $0\times$ in graph cloning; $141\times$/$45\times$, $44\times$/$25\times$, $13\times$/$11\times$, $28\times$/$34\times$, and $3.5\times$/$2.2\times$ in edge deletions/insertions; and $67\times$/$63\times$, $86\times$/$86\times$, $2.5\times$/$2.6\times$, $0.25\times$/$0.24\times$, and $1.3\times$/$1.3\times$ in traversal on updated graphs with deletions/insertions.
\end{abstract}

%% The code below is generated by the tool at http://dl.acm.org/ccs.cfm.
\begin{CCSXML}
<ccs2012>
<concept>
<concept_id>10003752.10003809.10010170</concept_id>
<concept_desc>Theory of computation~Parallel algorithms</concept_desc>
<concept_significance>500</concept_significance>
</concept>
<concept>
<concept_id>10003752.10003809.10003635</concept_id>
<concept_desc>Theory of computation~Graph algorithms analysis</concept_desc>
<concept_significance>500</concept_significance>
</concept>
</ccs2012>
\end{CCSXML}

% \ccsdesc[500]{Theory of computation~Parallel algorithms}
% \ccsdesc[500]{Theory of computation~Graph algorithms analysis}

%% Pick words that accurately describe the work being presented.
\keywords{Graph representation, Dynamic graph updates}

% \received{20 February 2007}
% \received[revised]{12 March 2009}
% \received[accepted]{5 June 2009}




%% Process the author and title information.
\maketitle

\section{Introduction}
\label{sec:introduction}

The increasing reliance on LLMs for multimodal tasks across far-reaching sectors such as healthcare, finance, and manufacturing underscores the need to assess the accuracy and reliability of the information they generate. Vision-Language Models (VLM) have achieved state-of-the-art (SoTA) performance on Visual Question-Answering (VQA) benchmarks, and these models often utilize Retrieval-Augmented Generation (RAG) to maintain factual accuracy and relevance in a dynamic information environment. However, this has led to uncertainty in the information the LLM bases its answer on, as it may choose between parametric memory and retrieved sources. When models rely on memorized information instead of dynamically retrieving information, they may inadvertently propagate outdated or incorrect information, causing serious legal and ethical risks and undermining trust and reliability in AI systems \citep{huang2023survey}.
% The ability to strike a balance between generalization and specialization in AI systems is therefore crucial for ensuring the safe, reliable use of these technologies in real-world applications.

Despite these concerns, the way that Vision-Language models (VLMs) memorize and retrieve information, particularly in complex multimodal tasks, remains under-explored. Current research often focuses on either the general capabilities of large language models (LLMs) or the specialized retrieval mechanisms in retrieval augmented generation systems (RAG) \citep{incontext_rag,chen_murag_2022,liu_universal_2023}. Particularly in the context of multimodal retrieval and multihop reasoning, few studies analyze the tradeoff between finetuning for specialized tasks and zero-shot prompting for general-purpose vision-language capabilities. A lack of consensus on how to approach this tradeoff motivates the development of measures to quantify reliance on parametric memory, as well as metrics for quantifying the potential performance impact of extending LLMs with RAG systems.

To address this gap, we investigate how multimodal QA models balance accuracy with memorization on the WebQA benchmark. We compare finetuned multimodal systems against zero-shot VLMs, analyzing how retrieval performance influences QA accuracy. In particular, we focus on cases where retrieval fails, allowing us to measure reliance on parametric memory through two proposed metrics---the \ppr (\PPR) which quantifies how much model accuracy is influenced by retrieval quality, contrasting performance in best-case versus worst-case retrieval scenarios, and the \ucr (\UCR) which measures how often correct QA responses are generated when the retriever fails, providing a proxy for memorization.

To enable this analysis, we make several methodological contributions. For the finetuned QA models, we investigate Vision-Transformer (ViT) architectures, which allow for multihop reasoning over multiple sources. To investigate the impact of retrieval performance on trained LMs, we propose a variable-input Fusion-in-Decoder (FiD) model \cite{tanaka_slidevqa_2023, nlvr2}, building upon the VoLTA architecture \citep{pramanick_volta_2023}. For the zero-shot case, we build upon previous research on In-Context Retrieval \citep{incontext_rag} by demonstrating that LLMs such as GPT-4o are capable of performing the final ranking step of the retrieval process. In doing so, we find that GPT-4o, a general-purpose LLM, achieves SoTA performance on the WebQA task, outperforming existing finetuned RAG models by a significant margin (7\% higher accuracy). 

Crucially, our results reveal that while retrieval-augmented models reduce memorization, the training paradigm plays an important role. Finetuned models exhibit higher reliance on parametric memory, whereas zero-shot RAG approaches have lower memorization scores at the cost of accuracy. This suggests that while retrieval modules may mitigate the risks associated with outdated or incorrect information, SoTA performance requires that they be coupled with specialized QA models. Our memorization measures contribute to the development of transparent and reliable AI systems, particularly in applications where the sourcing of up-to-date, factual information is critical.



% We investigate the impact of question complexity on the ability of these models to integrate multiple data sources—such as images, text, and external retrievers—and produce coherent and accurate answers. We also explore whether in-context retrieval can be a viable alternative to traditional retrieval-augmented systems, offering a more streamlined approach to multimodal QA.

% To achieve this, we first compare zero-shot prompting multimodal LLMs with finetuned multimodal systems. We evaluate both types of models on the WebQA benchmark, a dataset designed for complex question answering that requires reasoning across both image and text sources. For the finetuned models, we use a Fusion-in-Decoder (FiD) architecture, which allows for multihop reasoning over multiple sources. Additionally, we introduce the concept of In-Context Retrieval Language Modeling (RLM), where the LLM itself performs retrieval tasks without the need for external retrievers. This method builds upon existing research in in-context learning  and aims to explore the viability of LLMs retrieving relevant sources and generating accurate answers directly from their context window.

% In order to investigate source utilization in finetuned multimodal models and LLMs, three lines of inquiry are established; 
% \begin{itemize}
%     \item Study 1: retrieval vs QA performance on webQA (motivating example, does QA answer correctly even with incorrect sources?)
%     \item Study 2: performance on adversarial examples where parametric knowledge would be incorrect by design
%     \item Study 3: improving performance on adversarial examples by fine-tuning (i.e model robustness)
% \end{itemize}

% Note, there is one weakness in this plan which is tying in the work we've already done. 
% If we added something from adversarial generation to the retrieval experiment (like a combination of study 1 + 3) it would be complete. So for instance we could try fine-tuning the retriever with adversarial examples (and not just the QA model)

% \begin{figure}
%     \centering
%     \includegraphics[width=0.95\linewidth]{figures/segmentation/webqa_segment_infill.png}
%     \caption{Example of the segmentation substitution pipeline from the WebQA task.}
%     % d5c76d760dba11ecb1e81171463288e9
%     \label{fig:seg_sub_pipeline}
% \end{figure}



% Retrieval augmented generation (RAG) with zero-shot prompting and fine-tuning Large Language Models (LLMs) have become the go-to methods for tasks relying on information retrieval and text generation. In many cases the LLMs parametric memory can sufficiently generalize to answer questions without being provided with retrieval mechanisms for out-of-domain knowledge. However, LLMs often hallucinate and provide wrong information in certain scenarios. This problem is amplified even further on open-domain Question Answering (QA) tasks involving multiple modalities. Grounded text generation using retrieved sources \citep{lewis2021retrievalaugmented} has been extensively studied for text-to-text QA tasks, but its application in multimodal settings has not been studied as much.


% Multimodal reasoning and question answering have gained prominence in recent research endeavors, with an increasing emphasis on handling various forms of data, particularly text and images. In this study, we address a specific gap in the existing literature by focusing on the development of a versatile multihop model capable of accommodating varying numbers of input images.

% Our motivation for this research lies in the growing complexity of answering questions using information on the web, where the challenge of navigating the open-domain setting is further complicated by the presence of multiple modalities and sometimes requires reasoning over multiple sources. WebQA is an ideal dataset on which to compare performance of finetuned RAG systems against general purpose LLMs; it is multimodal, with correct answers requiring reasoning over image and text sources. It is multihop, requiring a complex reasoning process over multiple sources. Finally, WebQA questions from different categories can be broken down into subdomains to analyze performance over domains of varying cardinality.

% Motivated by the real-world challenges of building retrieval and question answering (QA) systems, we design and finetune a closed domain, multimodal, multihop QA model, that is capable of reasoning over a varying number of sources taken as input from an external retriever module. This research contributes to the relatively underexplored domain of multihop reasoning across various input sources and modalities. Our goal is to explore the challenges posed by these scenarios and develop strategies that enable QA models to retrieve relevant information, conduct logical or numerical reasoning across diverse modalities, and generate coherent responses in natural language. To our knowledge, this is the first application of the Fusion-in-Decoder (FiD) architecture \cite{tanaka_slidevqa_2023, nlvr2} that is shown to work with a variable number of inputs, enabling multi-hop reasoning over sources.

% In-Context Learning refers to the ability of LLMs to perform any task by simply providing examples in the input prompt \citep{dong2022survey,min2022rethinking}. Inspired by this research, we propose a method to use the LLM itself as a multimodal retriever, potentially eschewing the requirement of a distinct retrieval module, thereby allowing the design of simpler retrieval-augmented QA systems. We dub this method In-Context Retrieval Language Modeling (RLM). To the best of the authors knowledge, In-Content RLM is disparate from other retrieval augmented approaches which utilize external retrieval modules \citep{incontext_rag,chen_murag_2022,liu_universal_2023}. Despite being a natural extension of In-Context learning, In-Context RLM has not yet been studied empirically.

% To expand on our contribution of In-Context Retrieval, this stems from the well-researched in-context learning of LLMs. In-context learning is the ability of a model to perform any task given a sufficient context window \citep{dong2022survey,min2022rethinking}. Such tasks could include retrieval and ranking, but typically, the go-to solution for tasks requiring retrieval has been RAG. To the best of the authors knowledge, In-Context Retrieval is distinct from In-Context Retrieval Augmented Language Modelling (RALM), and despite being a natural extension of In-Context learning, In-Context Retrieval has not yet been shown empirically.

% Finally, we explore the tradeoff between using zero-shot prompting LLMs and the fine-tuning approach. While we find that, overall, GPT-4o obtains SoTA performance on the WebQA task, outperforming the accuracy of existing finetuned RAG approaches by 7\%, finetuned approaches still perform better on more restricted subdomains\footnote{``In-Context RLM" @ \url{https://eval.ai/web/challenges/challenge-page/1255/leaderboard/3168}}. Finally, we validate that GPT-4o is relying on retrieval abilities to solve the task; we find that GPT-4o is capable of retrieving relevant sources in the presence of distractors and furthermore, when GPT-4o fails to retrieve correct sources, it answers incorrectly 75\% of the time, meaning that it is not relying on parametric memory for this task.

% \paragraph{Contributions}
% Based on our experimentation and analysis on the WebQA benchmark, we make the following contributions:
% \begin{itemize}
%     \item Propose a new architecture for multimodal multihop QA that takes variable number of input sources inspired by the Fusion-in-Decoder method.
%     \item Comparison of general purpose LLMs vs specialized models on the WebQA benchmark.
%     \item Observation of In-Context Multimodal Retrieval abilities of GPT-4o and that it does not rely on parametric memory for multimodal QA.
%     \item Analysis of relationship between retrieval and QA task performance.
%     \item Analysis of task and query complexity on the performance of retrieval and QA tasks.
% \end{itemize}
















% Throughout this paper, we will present our methodology, experiments, and findings, emphasizing our approach to multihop reasoning over varying numbers of input images. We believe that our work contributes to a deeper understanding of multimodal reasoning and has the potential to enhance the capabilities of question-answering systems in the intricate, multimodal landscape of web-based information.

\section{Related work}
\label{sec:related}
\section{Related Work}

Our work is informed by foundational paradigms in visual analytics including exploratory data analysis and exploratory search (Sec. \ref{sec:related-eda}). 
We also build on many prior methods for subgroup analysis from data mining and machine learning, the design space of which we describe in Sec. \ref{sec:related-subgroup-analysis}.

\subsection{Exploratory Data Analysis and Search}
\label{sec:related-eda}

\citeauthor{tukey_exploratory_1970} describes \textit{exploratory data analysis} (EDA) as ``looking at data to see what it seems to say''~\cite{tukey_exploratory_1970}.
EDA is therefore distinct from hypothesis testing, or confirmatory data analysis, in its emphasis on generating insight from the \textit{data} rather than prior knowledge and expectations.
Many systems for EDA are informed by interaction techniques for \textit{exploratory search}, in which people navigate through and query information resources to build understanding about some latent concept of interest~\cite{white_exploratory_2009}.
In these interactive search settings, features such as sorting, filtering, and faceted searches~\cite{yee_2003_faceted} play a key role in helping users uncover useful information.
Applied to EDA, these techniques can enable steerable recommendations of how to visualize data features~\cite{wongsuphasawat_voyager_2016,lee_2021_lux} or efficient overviews of text data~\cite{felix_texttile_2017}.
We draw inspiration from these search techniques in the design of Divisi.

A wide variety of EDA techniques have been developed for different types of data, including small-scale tabular settings~\cite{wongsuphasawat_voyager_2016,lee_2021_lux}, high-dimensional data~\cite{Liu2017}, text data~\cite{felix_texttile_2017}, and general unstructured data~\cite{Smilkov2016}.
It is often easiest to find useful insights in EDA on tabular data because the features are generally intrinsically interpretable. 
In contrast, for text or image data the ``features'' (words or pixels) may not have any meaning on their own, making it difficult to interpret what instances have in common.
As datasets grow larger, there may also be many different subtypes within the dataset, limiting the insight provided by top-level metrics and distributions.
For this reason, many prior works aim to mitigate the complexity of large, high-dimensional datasets by automatically deriving semantically meaningful features or ``concepts'' to bootstrap the analysis process~\cite{suresh_kaleidoscope_2023,kim_interpretability_2018}.
Alternatively, some systems allow the user to define their own features of interest~\cite{wu_errudite_2020,cabrera_zeno_2023}.
However, these methods require the user to already know roughly what concept they are looking for, limiting their opportunities to explore and find unexpected patterns.
Our work relies on the presence of interpretable tabular features for every instance; however, we design for use cases in which the data scientist wants to find the relevant features out of a large set of potentially-meaningful set of descriptors.
This can afford the simplicity of working with tabular data while not restricting the analysis to the user's prior hypotheses.

% \begin{enumerate}
%     \item EDA \cite{tukey_exploratory_1970}, exploratory search \cite{white_exploratory_2009,marchionini_exploratory_2006} - what are the activities involved in each?
%     \item Faceted browsing~\cite{yee_2003_faceted}, sort and filter
%     \item More modern notions of EDA: text exploration, image exploration, embedding analysis
%     \item Benefits of traditional EDA
%     \item Challenges in extending the traditional notions of EDA to modern, large-scale datasets: multiple driving phenomena or subtypes, many variables (possibly more than can be reasoned about), uninterpretable variables
% \end{enumerate}

\subsection{Tools for Subgroup Analysis}
\label{sec:related-subgroup-analysis}

Sometimes called slice discovery, cluster analysis, or rule mining, subgroup analysis is an important part of data science that can help people understand phenomena in a dataset~\cite{liu_exploratory_2020,gamberger_active_2003}, help model builders diagnose and fix issues~\cite{piorkowski_aimee_2023,zhang_drml_2022,cabrera2021deblinder,robertson_angler_2023,zhang_sliceteller_2022, jain_distilling_2022}, explain model predictions~\cite{ribeiro_anchors_2018}, or even be used in place of a model~\cite{lavrac_decision_2004}.
However, it is usually all but impossible to define clear-cut, interpretable subgroups that exactly capture the outcome of interest (e.g., model errors), creating a design space of trade-offs for how to produce useful insights.
A wide array of subgroup analysis techniques have been developed, varying across several dimensions:

\textit{Conceptualization of a subgroup.} Differences in data types, user needs, and algorithm formulations give rise to different definitions of what a subgroup is. 
At the most subjective level, subgroups can be any semantic human-readable description of instances, regardless of whether it is encoded in the data, such as ``images of people with glasses''~\cite{cabrera2021deblinder}. 
They can also be defined by numerical proximity to some conceptual entity, such as a direction or neighborhood around an instance in an embedding space~\cite{eyuboglu_domino_2022,kim_interpretability_2018,ahn_escape_2023}. 
Finally, subgroups can be defined more precisely by constructing rules for membership, such as textual patterns~\cite{wu_errudite_2020,robertson_angler_2023} or predicates on tabular features~\cite{kwon_rmexplorer_2022,hurley_interactive_2022}. 
While the latter results in the clearest subgroup definitions, it also requires crafting or mining high-quality rules.

\textit{Source of initiative.} Many subgroup discovery methods require the data scientist to define subgroups themselves, using the affordances of the various subgroup concepts described above~\cite{cabrera_zeno_2023,wu_errudite_2020,kwon_rmexplorer_2022}. 
These methods are flexible and often provide useful insights on known areas of interest, but it can be difficult to find \textit{new} subgroups without spending time perusing individual instances. 
Algorithm-initiated approaches can provide a strong initial set of subgroups to explore~\cite{chung_slice_2020,zhang_sliceteller_2022}; however, these techniques heavily focus on producing the most relevant set of subgroups in the initial query.
There is currently a lack of \textit{mixed-initiative} subgroup analysis approaches that allow the user to interactively steer the algorithm's output to produce more relevant slices.
When subgroup analysis tools do offer mixed-initiative interactions, it is typically to \textit{refine} the subgroup definitions~\cite{slyman_vlslice_2023} or to characterize and assess their validity~\cite{hurley_interactive_2022}, both of which are supported in Divisi within our broader mixed-initiative workflow.

\textit{Visualization.} Designs to visualize and compare subgroup-level data characteristics are largely dependent on the way the subgroups are conceptualized.
For example, most clustering-based tools use dimensionality reduction scatter plots, which provide a valuable overview of the dataset but are difficult to map to data features~\cite{Liu2019,slyman_vlslice_2023,xuan_attributionscanner_2024,suresh_kaleidoscope_2023,sivaraman_emblaze_2022}.
For handcrafted subgroups on tabular data, brushable histograms can serve as controls to define predicates that are then visualized in strip plots~\cite{cabrera_fairvis_2019} or domain-specific visualizations~\cite{kwon_rmexplorer_2022}.
To visualize rule-based subgroups generated by an algorithm, table representations with sparkline charts or glyphs are often preferred as they can efficiently present summary statistics over many subgroups~\cite{kahng_visual_2016,kerrigan_slicelens_2023,zhang_sliceteller_2022}.
Similarly, UpSet plots~\cite{2014_infovis_upset} provide a dense visual representation of metrics within multiple set intersections.
Divisi combines several of these elements, including the scatter plot and the subgroup table with sparklines, with novel adaptations for tasks such as assessing overlap and coverage.

\textit{Algorithmic approach.} We can divide prior algorithms for subgroup discovery into four broad classes: lattice search, frequent itemsets, classification, and clustering.
Lattice search methods, such as Slice Finder~\cite{chung_slice_2020,sagadeeva_sliceline_2021}, \textsc{Premise}~\cite{hedderich_label-descriptive_2022}, and the Nugget Browser~\cite{guo_nugget_2011}, perform combinatorial search of a space of discrete rules to find those that most satisfy the algorithm's desirability criteria.
These methods can result in easily-interpretable subgroups, but they tend to scale poorly to datasets with hundreds or thousands of features due to combinatorial explosion.
Frequent itemset-based methods, such as DivExplorer~\cite{pastor_looking_2021} and the method developed by \citeauthor{suzuki_rule_2023}~\cite{suzuki_rule_2023}, draw on efficient algorithms from data mining such as FPgrowth, then score and rank the returned subgroups.
Similarly, these methods work best with a relatively small number of possible feature combinations.
Classification-based methods can overcome some of the performance considerations of lattice search and frequent itemset approaches \cite{yuan_isea_2022,yuan_visual_2022}, but their results often require significant work to interpret.
Finally, clustering-based methods aim to group together instances by similarity in a high-dimensional space such as a learned embedding~\cite{zhang_manifold_2019,eyuboglu_domino_2022,kim_interpretability_2018}.
Though these methods can provide insight into unstructured data, they often require a trained model, sometimes one that is jointly trained with natural-language representations, limiting their applicability.
Moreover, like classification methods, the resulting clusters and concepts are not always straightforward to interpret because of their reliance on learned embeddings.
Divisi builds on this extensive space of previous algorithms, adopting a modified lattice search approach that addresses scalability issues using approximation.
While it is most directly applicable to tabular datasets as a result, we propose ways to use it in unstructured data contexts in Sec. \ref{sec:use-case}.

Because there are so many alternative techniques for subgroup analysis, each with their own specific associated data types and challenges, there is not a clear consensus of what approach should be applied to a given problem.
As a result, data scientists may not typically include subgroup analysis in the exploratory phase of their workflows.
Our work aims to make it easier to perform subgroup analyses interactively within a typical programming environment, and we assess in our study whether they might find such capabilities useful in their daily work.

% \begin{enumerate}
%     \item Why subgroup analysis? It can be used in place of classifiers ~\cite{lavrac_decision_2004}, it is useful for experts to understand phenomena in a dataset \cite{liu_exploratory_2020,gamberger_active_2003} or explain predictions~\cite{ribeiro_anchors_2018}, and it can help model builders diagnose and fix issues in their models~\cite{piorkowski_aimee_2023,zhang_drml_2022,cabrera2021deblinder,robertson_angler_2023,zhang_sliceteller_2022, jain_distilling_2022}.
%     \item There are many different subgroup analysis approaches, which vary in how the subgroup is conceptualized:
%     \begin{itemize}
%         \item Conceptualization of a subgroup: can be defined by a semantic human-readable description~\cite{cabrera2021deblinder}, defined by a pattern for text data~\cite{wu_errudite_2020,robertson_angler_2023,hedderich_label-descriptive_2022}, or a rule based on tabular features~\cite{kwon_rmexplorer_2022,hurley_interactive_2022}, based on proximity to some concept or a direction in an embedding space~\cite{suresh_kaleidoscope_nodate}, or based on clusters~\cite{Cavallo2019}
        
%     \end{itemize}
%     \item Source of initiative: often entirely human-initiated~\cite{cabrera_zeno_2023,wu_errudite_2020,kwon_rmexplorer_2022}, or algorithm-initiated. Slice discovery involves approaches to automatically generate the subgroups of interest by mining them from patterns in the data. Algorithmic approaches vary:
%     \begin{itemize}
%         \item rule mining - enumerate possible combinations of features and score them~\cite{chung_slice_2020,sagadeeva_sliceline_2021}
%         \item frequent itemsets~\cite{pastor_looking_2021,zhang_sliceteller_2022}
%         \item embedding-representation approaches~\cite{eyuboglu_domino_2022,kim_interpretability_2018}. For unstructured data we can also use cross-modal representation spaces to label clusters~\cite{slyman_vlslice_2023,eyuboglu_domino_2022}
%     \end{itemize}
%     \item Literature gap: mixed-initiative systems for subgroup discovery. Some interactive systems incorporating subgroup discovery allow users to refine the subgroup definitions~\cite{slyman_vlslice_2023,} or to investigate the characteristics of the subgroups and assess their validity~\cite{hurley_interactive_2022}. few systems have been developed that allow 
% \end{enumerate}
% rule-based explanations

% \section{Preliminaries}
% \label{sec:preliminaries}
% \section{Preliminaries}
\label{sec:preliminaries}
% Here we formulate the CS problem, discuss some metrics for it and provide an overview of CBMs. % for the it, followed by an overview of the concept bottleneck models.

% \vspace{-0.1cm}
\subsection{Coreset selection (CS) problem formulation}
\label{sec:preliminaries_cs}
Consider a classification task and data distribution $P$. 
Let $\mathcal{D} = \{(x_i, y_i)\}_{i=1}^{n}$ denote the dataset of $n$ training examples sampled i.i.d. from the distribution $P$ where $x_i$ denotes the data and $y_i \in \mathcal{Y}$ denotes the label from a set of $N$ classes.
CS \cite{coleman2019selection,zheng2022coverage} aims to find a subset $\mathcal{S}$ of $\mathcal{D}$ consisting of $m \leq n$ samples such that the models trained on $\mathcal{S}$ achieve performance comparable to models trained on $\mathcal{D}$. 
Formally, the CS problem is %defined as: % follows,
\begin{equation}
\label{eq:coreset_selection}
\min_{\mathcal{S}:|\mathcal{S}|=m} \mathbb{E}_{(x,y) \sim P}[\ell(x,y|\theta_{\mathcal{S}})] - \mathbb{E}_{(x,y) \sim P}[\ell(x,y|\theta_{\mathcal{D}})],
\end{equation}
where $\theta_{\mathcal{D}}$ and $\theta_{\mathcal{S}}$ denote the ``\emph{downstream model}"  trained on $\mathcal{D}$ and $\mathcal{S}$ (coreset), respectively and $\ell$ is the loss function.

To find this subset $\mathcal{S}$, previous works have proposed scores to gauge a sample's difficulty for a model, and are later used to form the coreset. 
Approaches such as max entropy uncertainty sampling \cite{lewis1994heterogeneous, Settles_2012}, and least confidence \cite{culotta2005reducing} estimate difficulty using the uncertainty of the model's predictions on a sample. %, to gauge its difficulty.
Another set of approaches such as $k$-center greedy \cite{sener2017active} uses  geometric information of the data to filter out redundant samples. 
Yet, another set of approaches uses information from the training dynamics of the downstream model to estimate the a difficulty score. 
Scores such as the forgetting score \cite{toneva2018empirical} which is computed as the number of times a sample gets misclassified after being correctly classified earlier during model training, the area under the margin (AUM) \cite{pleiss2020identifying} which identifies mislabeled/difficult samples, fall in this category.
% Based on these score, a sampling strategy is then used to identify $\mathcal{S}$.
While approaches based on the training dynamics of the downstream model have achieved SOTA results, the requirement of knowledge/training the downstream model or a relatively big proxy model on the entire dataset at least once is inefficient for large datasets/models, even if done only once. 
This motivates the need of data centric approaches for sample's difficulty independent of the downstream model.
% \AM{Sampling strategy}

\subsection{Concept bottleneck models (CBMs)}
\label{sec:preliminaries_cbm}
Recent advances in language model-guided CBMs utilize an LLM to obtain concept bottlenecks which are then used to predict the labels. 
These works rely on a pre-trained multi-modal models (such as CLIP \cite{radford2021learning}) which consists of a visual encoder $\mathcal{V}_{enc}$ and a text encoder $\mathcal{T}_{enc}$ that can map images and text to a $d$-dimensional representation space. % with dimension $d$. 
Let $C = \{c_1, c_2, \cdots, c_{N_C}\}$ be the set of $N_C$ concepts (bottleneck) generated via a LLM, we can then construct a bottleneck embedding matrix $E_C \in \mathcal{R}^{N_C \times d}$ such that each row of the matrix is mapping of the concept $c \in C$ after passing it through textual encoder $\mathcal{T}_{enc}$.
Based on this, a CBM  \cite{yang2023language} produces a prediction $h(x) = f(g(\mathcal{V}_{enc}(x), E_C))$ for a sample $x$ where $g:\mathbb{R}^{d} \rightarrow \mathbb{R}^{N_C}$ computes the similarity of the visual features to each concept in the bottleneck and $f:\mathbb{R}^{d} \rightarrow \Delta$ outputs the probability of each class in the label set $\mathcal{Y}$, where $\Delta$ is a $N$ simplex.
We discuss details of $f$ and $g$ in Sec.~\ref{sec:approach}.

\section{Approach}
\label{sec:approach}
\subsection{Our Graph Representation}
\label{sec:digraph}

Our directed graph implementation, \textbf{DiGraph}, which leverages our Concurrent Power-of-2 Arena Allocator (CP2AA) for efficient memory management (see Section \ref{sec:cp2aa} for details), is described in Algorithms \ref{alg:digraph1} and \ref{alg:digraph2}. The implementation supports operations for inserting and deleting a batch of edges, where a batch of edges is represented using \texttt{DiGraph}. Thus, insertion of a batch of edges corresponds to a graph union operation, while deletion of a batch of edges corresponds to a graph subtraction operation.

The initialization of a \texttt{DiGraph} (lines \ref{alg:digraph--struct-begin}-\ref{alg:digraph--struct-end}) involves setting up four main data structures: \textbf{(1)} A bit array, $exists$, to track vertex existence; \textbf{(2)} An adjacency list, $edges$, to store outgoing edges; \textbf{(3)} An array, $degrees$, to track each vertex's out-degree; and \textbf{(4)} An array, $capacities$, to manage allocated edge storage per vertex. Memory allocation is handled by the CP2AA allocator, $cp2aa$. Additionally, the initialization defines key variables: reserved memory ($RES$), vertex capacity ($CAP$, the maximum vertex ID + 1), total vertices ($N$), and total edges ($M$). Configured constants include $\textsc{bool\_bits}$, which inidicates the granularity of vertex existence flags, and $\textsc{edge\_size}$, which represents the size of an edge in bytes.

The function \texttt{hasVertex()} (lines \ref{alg:digraph--has-vertex-begin}-\ref{alg:digraph--has-vertex-end}) of \texttt{DiGraph} checks whe-ther a given vertex exists in the graph by verifying if its ID falls within the allocated range ($CAP$) and whether the corresponding bit in $exists$ is set. The function \texttt{degree()} (lines \ref{alg:digraph--degree-begin}-\ref{alg:digraph--degree-end}) returns the number of outgoing edges for a vertex, defaulting to zero if the vertex is not within range. The \texttt{edges()} function (lines \ref{alg:digraph--edges-begin}-\ref{alg:digraph--edges-end}) retrieves the outgoing edges of a vertex, returning an empty set if the vertex does not exist.
%
The \texttt{reserve()} function (lines \ref{alg:digraph--reserve-begin}-\ref{alg:digraph--reserve-end}) ensures that sufficient space is allocated to accommodate new vertices. It first calculates a new reserved size by rounding up to the system's page size. If the requested capacity is greater than the current vertex capacity $CAP$, memory is reallocated (in parallel) using the \texttt{reallocate()} function (lines \ref{alg:digraph--reallocate-begin}-\ref{alg:digraph--reallocate-end}), which adjusts the size of multiple arrays, including $exists$ (bit array), $edges$, $degrees$, and $capacities$. Finally, the the graph’s vertex capacity $CAP$ and the reserved memory size $RES$ are updated.
%
The \texttt{reallocate()} function (lines \ref{alg:digraph--reallocate-begin}–\ref{alg:digraph--reallocate-end}) is responsible for resizing memory allocations while preserving existing data where possible. It first checks if the requested reserved amount $R\_1$ matches the current reserved amount $R\_0$. If they are equal, no new allocation is needed, and only the newly added portion of the array (from $N\_0$ to $N\_1$) is initialized to zero in parallel. Otherwise, new memory is allocated for $R\_1$ entries, and the existing data is copied up to the minimum of the old and new sizes ($M = min(N\_0, N\_1)$). Any remaining entries in the newly allocated space (from $M$ to $N\_1$) are initialized to zero in parallel. Once the data transfer is complete, the old memory is deallocated, and the newly allocated pointer is returned.

The \texttt{addVertex()} function (lines \ref{alg:digraph--add-vertex-begin}-\ref{alg:digraph--add-vertex-end}) ensures that a vertex exists by expanding storage if necessary and then setting its existence flag. 
%
The function \texttt{allocateEdges()} (lines \ref{alg:digraph--allocate-edges-begin}-\ref{alg:digraph--allocate-edges-end}) is responsible for allocating memory for a vertex’s outgoing edges using the CP2AA allocator. It first checks whether the vertex ID $u$ exceeds the current capacity ($CAP$) or if memory for its edges has already been allocated. If either condition is met, the function returns immediately. Otherwise, it determines the required memory size in bytes by multiplying the desired number of edges ($deg$) by $\textsc{edge\_size}$, which represents the storage size of an edge (a tuple of a destination vertex and an edge weight). The computed memory requirement is then passed to the CP2AA allocator's \texttt{allocationSize()} function, which adjusts the requested size to make it a power-of-2 size (in bytes), or a multiple of system page size. The function then allocates the necessary amount of memory (in bytes) using $cp2aa.allocate()$, stores the resulting pointer in $edges[u]$, and updates $capacities[u]$ to reflect the allocated number of edges.

Adding an edge without additional safety checks is handled by \texttt{addEdgeUnsafe()} (lines \ref{alg:digraph--add-edge-unsafe-begin}-\ref{alg:digraph--add-edge-unsafe-end}), which retrieves the pointer to the edge list $edges[u]$, increments the degree counter $degrees[u]$ atomically, and inserts the new edge.
%
The \texttt{addEdges()} function (lines \ref{alg:digraph--add-edges-begin}-\ref{alg:digraph--add-edges-end}) is responsible for adding a list of outgoing edges to a vertex while ensuring proper memory allocation. If the vertex does not exist or the provided edge list is empty, the function returns immediately. First, it records the current degree of the vertex ($deg\_prev$) and calculates the maximum possible degree after adding the new edges ($deg\_max$). It then determines the required memory allocation size using the CP2AA allocator and allocates a new memory block for storing the updated edge list. The function merges the existing edges with the new edges using the \texttt{setUnion()} operation, ensuring they are stored in the allocated memory. Note that edges in the list must be sorted. Once the new edges are successfully added, the function deallocates the previously allocated memory using CP2AA, updates the edge pointer, and records the new edge capacity. Finally, it returns the number of newly added edges by computing the difference between the updated and previous degrees.
%
The \texttt{removeEdges()} function (lines \ref{alg:digraph--remove-edges-begin}-\ref{alg:digraph--remove-edges-end}) is responsible for removing a specified list of outgoing edges from a given vertex. First, it checks whether the vertex exists in the graph using \texttt{hasVertex()} and whether the provided edge list is non-empty. If either condition is not met, the function returns immediately with a value of zero. Otherwise, it records the current out-degree of the vertex in $deg\_prev$. The function then performs a set difference operation to remove the specified edges from the vertex’s adjacency list, effectively updating the edge count. Finally, it returns the number of edges removed, calculated as the difference between the original degree and the updated degree of the vertex.

Finally, the \texttt{update()} function (lines \ref{alg:digraph--update-begin}-\ref{alg:digraph--update-end}) ensures the integrity of the graph by sorting and deduplicating edges while updating the total vertex and edge counts. First, if the edges are not already sorted, each vertex's edge list is sorted in parallel using the \texttt{sortByKey()} function. Next, if duplicate edges exist, they are removed via \texttt{uniqueByKey()}, also executed in parallel. After enforcing order and uniqueness, the function recalculates the total number of vertices, $N$, and edges, $M$, by iterating over all possible vertex IDs up to the current capacity, $CAP$.\ignore{If a vertex is marked as existing, its count is incremented, and its degree contributes to the total edge count.}

\begin{algorithm}[hbtp]
\caption{Our Directed Graph that uses CP2AA allocator.}
\label{alg:digraph1}
\begin{algorithmic}[1]
\Require{$T_K, T_W$: Datatype for vertex ID, edge weight}
\Require{$\textsc{pool\_size}_a$: Size of each memory pool (constant)}

\Statex

\State \textbf{struct} $DiGraph \langle T_K, T_W \rangle ()$ \label{alg:digraph--struct-begin}
\State \ \ $\textsc{bool\_bits} \gets 64$ \Comment{Boolean stored in 64-bit chunks}
\State \ \ $\textsc{edge\_size} \gets sizeof((T_K, T_W))$ \Comment{Size of an edge in bytes}
\State \ \ $exists \gets \{\}$ \Comment{Vertex existence flags}
\State \ \ $edges \gets \{\}$ \Comment{Outgoing edges for each vertex}
\State \ \ $degrees \gets \{\}$ \Comment{Out-degree of each vertex}
\State \ \ $capacities \gets \{\}$ \Comment{Edge capacity of each vertex}
\State \ \ $cp2aa \gets CP2AA \langle \textsc{pool\_size}_a \rangle ()$ \Comment{Memory allocator}
\State \ \ $CAP \gets 0$ \Comment{Vertex capacity (max vertex id + 1)}
\State \ \ $RES \gets 0$ \Comment{Memory reserved for vertices}
\State \ \ $N \gets 0$ \Comment{Total number of vertices}
\State \ \ $M \gets 0$ \Comment{Total number of edges} \label{alg:digraph--struct-end}

\Statex

\State $\rhd$ Check if a vertex exists in the graph
\Function{hasVertex}{$u$} \textbf{of} DiGraph \label{alg:digraph--has-vertex-begin}
  \Return{$u < CAP$ \textbf{and} $getBit(exists, u)$}
\EndFunction \label{alg:digraph--has-vertex-end}

\Statex

\State $\rhd$ Get the number of outgoing edges of a vertex
\Function{degree}{$u$} \textbf{of} DiGraph \label{alg:digraph--degree-begin}
  \Return{$u < CAP$? $degrees[u]$ : $0$}
\EndFunction \label{alg:digraph--degree-end}

\Statex

\State $\rhd$ Get the outgoing edges of a vertex
\Function{edges}{$u$} \textbf{of} DiGraph \label{alg:digraph--edges-begin}
  \Return{$u < CAP$? $edges[u][0 \dots degrees[u]]$ : \{\}}
\EndFunction \label{alg:digraph--edges-end}

\Statex

\State $\rhd$ Reserve space for a number of vertices
\Function{reserve}{$n$} \textbf{of} DiGraph \label{alg:digraph--reserve-begin}
  \State $n \gets max(n, CAP)$
  \State $\rhd$ Compute new reserved size (round up to page size)
  \State $res \gets \lceil n / \textsc{page\_size} \rceil * \textsc{page\_size}$
  \If{$n \leq CAP$ \textbf{and} $res = RES$} \ReturnInline{}
  \EndIf
  \State $\rhd$ Allocate new memory
  \State $B \gets \textsc{bool\_bits}$
  \State $exists \gets reallocate(exists, \lceil \frac{CAP}{B} \rceil, \lceil \frac{RES}{B} \rceil, \lceil \frac{n}{B} \rceil, \lceil \frac{res}{B} \rceil)$
  \State $edges \gets reallocate(edges, CAP, RES, n, res)$
  \State $degrees \gets reallocate(degrees, CAP, RES, n, res)$
  \State $capacities \gets reallocate(capacities, CAP, RES, n, res)$
  \State $\rhd$ Update vertex capacity and reserved size
  \State $CAP \gets n$ \textbf{;} $RES \gets res$
\EndFunction \label{alg:digraph--reserve-end}

\Statex

\State $\rhd$ Reallocate memory of specified size and reserved amount
\Function{reallocate}{$ptr, N_0, R_0, N_1, R_1$} \label{alg:digraph--reallocate-begin}
  \If{$R_1 = R_0$} \Comment{Desired reserved amount = old?}
    \ForAll{$i \in [N_0, N_1)$ \textbf{in parallel}} $ptr[i] \gets 0$
    \EndFor
    \Return{$ptr$}
  \EndIf
  \State $tmp \gets$ Allocate memory for $R_1$ entries
  \State $M \gets min(N_0, N_1)$
  \ForAll{$i \in [0, M)$ \textbf{in parallel}} $tmp[i] \gets ptr[i]$
  \EndFor
  \ForAll{$i \in [M, N_1)$ \textbf{in parallel}} $tmp[i] \gets 0$
  \EndFor
  \State Free memory at $ptr$
  \Return{$tmp$}
\EndFunction \label{alg:digraph--reallocate-end}
\algstore{alg:digraph12}
\end{algorithmic}
\end{algorithm}

\begin{algorithm}[hbtp]
\caption{Our Directed Graph using CP2AA allocator (Part 2).}
\label{alg:digraph2}
\begin{algorithmic}[1]
\algrestore{alg:digraph12}
\State $\rhd$ Add a vertex to the graph
\Function{addVertex}{$u$} \textbf{of} DiGraph \label{alg:digraph--add-vertex-begin}
  \If{$u \geq CAP$} $reserve(u + 1)$
  \EndIf
  \State $setBit(exists, u)$
\EndFunction \label{alg:digraph--add-vertex-end}

\Statex

\State $\rhd$ Allocate space for outgoing edges of a vertex
\Function{allocateEdges}{$u, deg$} \textbf{of} DiGraph \label{alg:digraph--allocate-edges-begin}
  \If{$u \geq CAP$ \textbf{or} $edges[u] \neq \phi$} \ReturnInline{}
  \EndIf
  \State $bytes \gets cp2aa.allocationSize(deg * \textsc{edge\_size})$
  \State $edges[u] \gets cp2aa.allocate(bytes)$
  \State $capacities[u] \gets bytes / \textsc{edge\_size}$
\EndFunction \label{alg:digraph--allocate-edges-end}

\Statex

\State $\rhd$ Add an outgoing edge, without checks
\Function{addEdgeUnsafe}{$u, v, w$} \textbf{of} DiGraph \label{alg:digraph--add-edge-unsafe-begin}
  \State $ptr \gets edges[u]$
  \State $i \gets atomicAdd(degrees[u], 1)$
  \State $ptr[i] \gets (v, w)$
\EndFunction \label{alg:digraph--add-edge-unsafe-end}

\Statex

\State $\rhd$ Add outgoing edges to a vertex
\Function{addEdges}{$u, list$} \textbf{of} DiGraph \label{alg:digraph--add-edges-begin}
  \If{\textbf{not} $hasVertex(u)$ \textbf{or} $list = \{\}$} \ReturnInline{$0$}
  \EndIf
  \State $deg_{prev} \gets degrees[u]$
  \State $deg_{max} \gets degrees[u] + size(list)$
  \State $bytes \gets cp2aa.allocationCapacity(deg_{max})$
  \State $ptr \gets cp2aa.allocate(bytes)$
  \State $degrees[u] \gets \textit{setUnion}(edges(u), list, \textbf{into}\ ptr)$
  \State $cp2aa.deallocate(edges[u], capacities[u] * \textsc{edge\_size})$
  \State $edges[u] \gets ptr$
  \State $capacities[u] \gets bytes / \textsc{edge\_size}$
  \Return{$degrees[u] - deg_{prev}$}
\EndFunction \label{alg:digraph--add-edges-end}

\Statex

\State $\rhd$ Remove outgoing edges from a vertex
\Function{removeEdges}{$u, list$} \textbf{of} DiGraph \label{alg:digraph--remove-edges-begin}
  \If{\textbf{not} $hasVertex(u)$ \textbf{or} $list = \{\}$} \ReturnInline{$0$}
  \EndIf
  \State $deg_{prev} \gets degrees[u]$
  \State $degrees[u] \gets \textit{setDifference}(\textbf{into}\ edges(u), list)$
  \Return{$deg_{prev} - degrees[u]$}
\EndFunction \label{alg:digraph--remove-edges-end}

\Statex

\State $\rhd$ Update the graph after changes
\Function{update}{$isUnique, isSorted$} \textbf{of} DiGraph \label{alg:digraph--update-begin}
  \State $\rhd$ Ensure edges are sorted and unique
  \If{\textbf{not} $isSorted$}
    \ForAll{$u \in [0, CAP)$ \textbf{in parallel}}
      \State $sortByKey(edges[u], degrees[u])$
    \EndFor
  \EndIf
  \If{\textbf{not} $isUnique$}
    \ForAll{$u \in [0, CAP)$ \textbf{in parallel}}
      \State $uniqueByKey(edges[u], degrees[u])$
    \EndFor
  \EndIf
  \State $\rhd$ Update counts of vertices and edges
  \State $N \gets M \gets 0$
  \ForAll{$u \in [0, CAP)$ \textbf{in parallel}}
    \If{\textbf{not} $getBit(exists, u)$} \textbf{continue}
    \EndIf
    \State $N \gets N + 1$ \textbf{;} $M \gets M + degrees[u]$
  \EndFor
\EndFunction \label{alg:digraph--update-end}
\end{algorithmic}
\end{algorithm}



\subsubsection{Loading a Graph from Disk}
\label{sec:load}

We now describe our algorithm for loading a graph from a Matrix Market (MTX) format file as a Compressed Sparse Row (CSR) representation. The algorithm is presented in Algorithm \ref{alg:load}, and is an improvement upon GVEL\ignore{, our previous work} \cite{sahu2023gvel}. In the algorithm, the function \texttt{loadGraph()} takes as input an MTX-formatted string, which is essentially the memory-mapped data of and MTX file, and outputs the graph $G$ in CSR format.

In the algorithm, we first initialize an empty graph structure $G$ (line \ref{alg:load--init}). We then read the MTX file header using the \texttt{readHeader()} function (line \ref{alg:load--read-header}), which indicates graph properties, such as whether the graph is symmetric, the number of rows and columns $N$, and the total number of edges $M$. If the graph is symmetric, the number of edges $M$ is doubled to account for bidirectional connections. The output CSR graph $G$ is resized accordingly to accommodate the required storage space (line \ref{alg:load--resize}).
%
Next, we allocate space for storing edge data in parallel (lines \ref{alg:load--alloc-edges-begin}-\ref{alg:load--alloc-edges-end}). Each thread is responsible for allocating memory for $source$ and $target$ vertices, and $weights$, if the graph is weighted. The collection of allocated edge lists is stored in the variable $edges$ (line \ref{alg:load--edges}). Next, space is allocated for per-partition degree counts $pdegrees$ and per-partition CSRs $pcsr$, which includes per-partition offsets, edge keys, and edge values (lines \ref{alg:load--alloc-pdegrees-pcsr-begin}-\ref{alg:load--alloc-pdegrees-pcsr-end}). The first partition ($p=0$) is mapped directly to the primary degree and CSR arrays of $G$. For subsequent partitions ($p \in [1, \rho)$), where $\rho$ is the number of partitions, space is allocated separately. If the graph is weighted, additional memory is assigned for storing edge weights. Each partition, the degree array is initialized to zero to prepare for counting.

After setting up the necessary data structures, we read the edge list and processes it into CSR format (line \ref{alg:load--read-edgelist}-\ref{alg:load--convert-to-csr}). The function \texttt{readEdgelist()} populates the per-thread edges and per-partition degree counts, while \texttt{convertToCsr()} converts this data into a CSR representation. For this, \texttt{convertToCsr()} utilizes $\rho$ per-partition CSRs ($pcsr$) as intermediates, in order to generate the global CSR $G$. If the graph is symmetric, the $G$ is resized (line \ref{alg:load--resize-symmetric}). Finally, we return the constructed CSR graph $G$ (line \ref{alg:load--return}).

\begin{algorithm}[hbtp]
\caption{Load a graph from an MTX format file as a CSR.}
\label{alg:load}
\begin{algorithmic}[1]
\Require{$data$: Input MTX data as string}
\Ensure{$G$: Output CSR graph}
\Ensure{$symmetric$: Is graph symmetric?}
\Ensure{$weighted$: Is graph weighted?}
\Ensure{$\rho$: Number of partitions for counting vertex degrees}
\Ensure{$t$: Current thread}

\Statex

\Function{loadGraph}{$data$}
  \State $G \gets \{\}$ \label{alg:load--init}
  \State $\rhd$ Read MTX format header
  \State $(symmetric, rows, cols, size, head) \gets readHeader(data)$ \label{alg:load--read-header}
  \State $data \gets removePrefix(data, head)$
  \State $\rhd$ Allocate space for CSR
  \State $N \gets max(rows, cols)$
  \State $M \gets 2 * size$ \textbf{if} $symmetric$ \textbf{else} $size$
  \State $G.resize(N, M)$ \label{alg:load--resize}
  \State $\rhd$ Allocate space for edges
  \State $sources \gets targets \gets weights \gets \{\}$ \label{alg:load--alloc-edges-begin}
  \ForAll{\textbf{threads in parallel}}
      \State $sources[t] \gets$ Allocate space for $M$ elements
      \State $targets[t] \gets$ Allocate space for $M$ elements
      \If{$weighted$}
          \State $weights[t] \gets$ Allocate space for $M$ elements
      \EndIf
  \EndFor \label{alg:load--alloc-edges-end}
  \State $edges \gets (sources, targets, weights)$ \label{alg:load--edges}
  \State $\rhd$ Allocate space for pdegrees, pcsr
  \State $pdegrees \gets \textit{poffsets} \gets pedgeKeys \gets \textit{pedgeValues} \gets \{\}$ \label{alg:load--alloc-pdegrees-pcsr-begin}
  \State $pdegrees[0] \gets G.degrees$
  \State $\textit{poffsets}[0] \gets G.offsets$
  \State $pedgeKeys[0] \gets G.edgeKeys$
  \If{$weighted$}
      \State $\textit{pedgeValues}[0] \gets G.edgeValues$
  \EndIf
  \State $fill(pdegrees[0], 0)$
  \ForAll{$p \in [1, \rho)$ \textbf{in parallel}}
      \State $pdegrees[p] \gets$ Allocate space for $N$ elements
      \State $\textit{poffsets}[p] \gets$ Allocate space for $N+1$ elements
      \State $pedgeKeys[p] \gets$ Allocate space for $M$ elements
      \If{$weighted$}
          \State $\textit{pedgeValues}[p] \gets$ Allocate space for $M$ elements
      \EndIf
      \State $fill(pdegrees[p], 0)$
  \EndFor \label{alg:load--alloc-pdegrees-pcsr-end}
  \State $pcsr \gets (\textit{poffsets}, pedgeKeys, \textit{pedgeValues})$
  \State $\rhd$ Read edge list and convert to CSR
  \State $counts \gets readEdgelist(pdegrees, edges, data)$ \label{alg:load--read-edgelist}
  \State $M \gets convertToCsr(pcsr, pdegrees, edges, counts)$ \label{alg:load--convert-to-csr}
  \State $\rhd$ Account for self-loops in symmetric graphs
  \If{$symmetric$} $G.resize(N, M)$ \label{alg:load--resize-symmetric}
  \EndIf
  \Return{$G$} \label{alg:load--return}
\EndFunction
\end{algorithmic}
\end{algorithm}


We now discuss the psuedocode of the \texttt{readEdgelist()} function, which is described in Algorithm \ref{alg:load-el}. \texttt{readEdgelist()} takes as input the per-partition vertex degrees, $pdegrees$; an output structure for storing edges, $edges$; and the memory-mapped file data, $data$. The function processes the edgelist in parallel, in blocks of size $\beta$. The parsed edges are populated, and the number of edges read per thread is tracked using $counts$, which is returned.

In the function, we begin by initializing the count of edges processed per thread and extracting the per-thread sources, targets, and weights of edges (lines \ref{alg:el--initialize-begin}-\ref{alg:el--initialize-end}). Next, edges are loaded from the file in blocks of size $\beta$, ensuring efficient reading of large files (lines \ref{alg:el--blocks-begin}-\ref{alg:el--blocks-end}). Each thread processes a specific range of the input file, determined by its index in steps of $\beta$. The function \texttt{getBlock()} extracts the corresponding block from the memory-mapped file (line \ref{alg:el--get-block}). Within each block, edges are read sequentially (lines \ref{alg:el--block-begin}-\ref{alg:el--block-end}). Each edge consists of a pair of vertex identifiers $(u, v)$, with an optional weight $w$ if the graph is weighted (lines \ref{alg:el--parse-edge-begin}-\ref{alg:el--parse-edge-end}). The edge data is parsed by first locating the next numeric digit in the file, extracting the integer values for $u$ and $v$, and, if applicable, parsing a floating-point weight. The vertex identifiers are then converted to zero-based indexing (line \ref{alg:el--base1}). Once an edge is parsed, it is added to the corresponding thread's edge list (lines \ref{alg:el--add-edge-begin}-\ref{alg:el--add-edge-end}). Additionally, the per-partition vertex degree counter is updated atomically to reflect the addition of the edge (line \ref{alg:el--update-degrees}). If the graph is symmetric, the reverse edge $(v, u)$ is also stored, ensuring undirected connectivity is maintained (lines \ref{alg:el--reverse-edge-begin}-\ref{alg:el--reverse-edge-end}). After processing all edges in a block, the count of edges processed by each thread is updated (line \ref{alg:el--update-counts}). Once all blocks are processed, we return the total number of edges read per thread\ignore{(line \ref{alg:el--return-counts})}.

The \texttt{getBlock()} function ensures that each thread correctly extracts a contiguous chunk of the file while respecting line boundaries (lines \ref{alg:el--get-block-begin}-\ref{alg:el--get-block-end}). It first determines the starting position $b$ and the endpoint $B$ of the block within the file. If the starting position is within a partially read line, it adjusts $b$ to the beginning of the next line. Similarly, if the endpoint $B$ is within a partially read line, it extends $B$ to include the full line. The function then returns the adjusted block boundaries $[b, B]$, ensuring that each thread reads a well-formed subset of the edgelist.

\input{src/alg-load-el}

Next, we discuss the pseudocode of the \texttt{convertToCsr()} function, which is given in Algorithm \ref{alg:load-csr}. \texttt{convertToCsr()} takes as input the per-partition compressed sparse row (CSR) structure, $pcsr$; per-partition vertex degrees, $pdegrees$; the per-thread edge lists containing sources, targets, and weights of edges, $edges$; and the number of edges read per thread, $counts$. Our goal is to efficiently convert the per-thread edgelists into a partitioned CSR representation and then merge these partitioned CSRs into a final global CSR. By partitioning the CSR construction, we minimize contention when updating global data structures, significantly improving performance compared to a direct global CSR construction.

We begin by extracting the key components of the per-partition CSR: offsets ($poffsets$), edge destinations ($pedgeKeys$), and edge weights ($pedgeValues$) (lines \ref{alg:csr--initialize-begin}-\ref{alg:csr--initialize-end}). Additionally, we retrieve the per-thread sources, targets, and weights of edges. Next, we compute the global degree of each vertex and store it in $degrees[0]$. This step is essential because we use partition $0$ as the base for the final CSR, ensuring that its offsets are correctly initialized. Although we construct the CSR in four partitions ($0$ to $\rho-1$), partition $0$ is already properly set up, allowing us to merge only the remaining three partitions into it (lines \ref{alg:csr--initialize-end}-\ref{alg:csr--poffsets-begin}).
%
To construct the per-partition CSR efficiently, we compute shifted offsets for each partition (lines \ref{alg:csr--poffsets-begin}-\ref{alg:csr--poffsets-end}). This optimization ensures that offsets can be used directly as indices for inserting edges into the CSR. Once the CSR is populated, the shifted offsets will automatically reflect the final correct offsets without requiring additional post-processing steps, which used to be necessary in older approaches such as GVEL \cite{sahu2023gvel}. This avoids an extra pass over the data to adjust the offsets.

We then populate the per-partition CSR in parallel (lines \ref{alg:csr--pcsr-begin}-\ref{alg:csr--pcsr-end}). Each thread processes its assigned edges and determines which partition they belong to. For each edge $(u, v)$, we atomically increment the corresponding offset and store $v$ in $pedgeKeys$. If the graph is weighted, we also store the corresponding weight in $pedgeValues$. Partitioning the CSR generation in this manner significantly reduces contention on atomic operations, as each partition is updated independently by different sets of threads.
%
After constructing the per-partition CSRs, we merge them into a single CSR (lines \ref{alg:csr--pcsr-combine-begin}-\ref{alg:csr--pcsr-combine-end}). Since partition $0$ is already structured correctly, we only need to merge partitions $1$ to $\rho-1$ into it. This process iterates over all vertices and sequentially appends their edges from partitions $1$ to $\rho-1$ into partition $0$. As edges are copied, we maintain an index $j$ to track the insertion position in the merged CSR. If the graph is weighted, we also copy the corresponding weights. Finally, we update the offsets in partition $0$ to reflect the total number of edges per vertex, completing the global CSR construction, and return the total number of edges in the CSR, $M$.

This algorithm differs from the one used in GVEL \cite{sahu2023gvel} in several key ways. First, we compute the global degree of each vertex in $degrees[0]$ to prepare partition $0$ as the final CSR, reducing the complexity of the merging step. Second, we use shifted offsets during CSR initialization, eliminating the need for a post-processing step to fix offsets after populating the CSR.

\begin{algorithm}[hbtp]
\caption{Convert per-thread Edgelists to CSR.}
\label{alg:load-csr}
\begin{algorithmic}[1]
\Require{$pcsr$: Per partition CSR (scratch)}
\Require{$pdegrees$: Per partition vertex degrees (scratch)}
\Require{$edges$: Per thread sources, targets, and weights of edges}
\Require{$counts$: Number of edges read per thread}
\Ensure{$symmetric$: Is graph symmetric?}
\Ensure{$weighted$: Is graph weighted?}
\Ensure{$\rho$: Number of partitions for counting vertex degrees}
\Ensure{$t$: Current thread}

\Statex

\Function{convertToCsr}{$pcsr, pdegrees, edges, counts$}
  \State $(\textit{poffsets}, pedgeKeys, pedgeValues) \gets pcsr$ \label{alg:csr--initialize-begin}
  \State $(sources, targets, weights) \gets edges$ \label{alg:csr--initialize-end}
  \State $\rhd$ Compute global degrees at $degrees[0]$
  \ForAll{$u \in [0, |V|)$ \textbf{in parallel}}
    \ForAll{$p \in [1, \rho)$} $degrees[0]\ \text{+=}\ degrees[p]$
    \EndFor
  \EndFor
  \State $\rhd$ Compute per-partition shifted offsets
  \ForAll{$p \in [0, \rho)$} \label{alg:csr--poffsets-begin}
    \State $M_p \gets exclusiveScan(\textit{poffsets}[p]+1, \textit{pdegrees}[p], |V|)$
    \If{$p = 0$} $M \gets M_p$
    \EndIf
  \EndFor \label{alg:csr--poffsets-end}
  \State $\rhd$ Populate per-partition CSR
  \ForAll{\textbf{threads in parallel}} \label{alg:csr--pcsr-begin}
    \State $p \gets t \bmod \rho$
    \ForAll{$i \in [0, counts[t])$}
      \State $u \gets sources[t][i]$
      \State $v \gets targets[t][i]$
      \State $j \gets atomicAdd(\textit{poffsets}[p][u+1], 1)$
      \State $pedgeKeys[p][j] \gets v$
      \If{$weighted$}
        \State $pedgeValues[p][j] \gets weights[t][i]$
      \EndIf
    \EndFor
  \EndFor \label{alg:csr--pcsr-end}
  \State $\rhd$ Combine per-partition CSR into one CSR
  \ForAll{$u \in [0, |V|)$ \textbf{in parallel}} \label{alg:csr--pcsr-combine-begin}
    \State $j \gets \textit{poffsets}[0][u+1]$
    \ForAll{$p \in [1, \rho)$}
      \State $i \gets \textit{poffsets}[p][u]$
      \State $I \gets \textit{poffsets}[p][u+1]$
      \ForAll{$i \in [i, I)$}
        \State $pedgeKeys[0][j] \gets pedgeKeys[p][i]$
        \If{$weighted$}
          \State $pedgeValues[0][j] \gets pedgeValues[p][i]$
        \EndIf
        \State $j \gets j + 1$
      \EndFor
    \EndFor
    \State $\textit{poffsets}[0][u+1] \gets j$
  \EndFor \label{alg:csr--pcsr-combine-end}
  \Return{$M$}
\EndFunction
\end{algorithmic}
\end{algorithm}



\subsubsection{Cloning a Graph}
\label{sec:clone}

The \texttt{cloneGraph()} function we use for (parallel) cloning of our graph representation is shown in Algorithm \ref{alg:clone}. It takes an input graph $G$, and returns the cloned graph $G'$.

The method begins by initializing an empty graph $G'$ (line \ref{alg:clone--init}). To optimize memory allocation, we preallocate storage for vertices based on the maximum vertex ID in $G$ (line \ref{alg:clone--reserve-vertices}). Next, all vertices from $G$ are added to $G'$ in parallel (lines \ref{alg:clone--add-vertices-begin}-\ref{alg:clone--add-vertices-end}), ensuring that all nodes from $G$ are present in $G'$ before edges are processed. Following this, we reserve space for edges by preallocating memory for each vertex’s adjacency list (lines \ref{alg:clone--reserve-edges-begin}-\ref{alg:clone--reserve-edges-end}). Once vertices are inserted, we preallocate edge storage for each vertex $u$ in parallel, setting aside space based on its degree in $G$ (lines \ref{alg:clone--populate-edges-begin}-\ref{alg:clone--populate-edges-end}). Next, the edges linked to vertex $u \in V$, and its associated degree are directly copied from the original graph. Finally, we update the total vertex and edge counts in $G'$. If $G$ is stored in Compressed Sparse Row (CSR) format, a specialized \texttt{update()} function is called to additionally sort the edges of each vertex in the graph by ID (line \ref{alg:clone--update-counts-csr}) --- this is done to ensure that edge additions and insertions can be done in $O(d_u + \Delta d_u)$ time, where $d_u$ is the degree of vertex $u$, and $\Delta d_u$ is the number of edges removed are being added to the vertex. Otherwise, the vertex and edge counts are explicitly set (line \ref{alg:clone--update-counts-noncsr}). The cloned graph $G'$, which is a deep-copy of $G$, is returned in line \ref{alg:clone--return}.

\begin{algorithm}[hbtp]
\caption{Create a deep-copy of a graph.}
\label{alg:clone}
\begin{algorithmic}[1]
\Require{$G(V, E)$: Input graph}
\Require{$G'(V', E')$: Cloned graph}

\Statex

\Function{cloneGraph}{$G$}
  \State $G' \gets \{\}$ \label{alg:clone--init}
  \State $\rhd$ Reserve space for vertices
  \State $G'.reserve(maxVertexId(G) + 1)$ \label{alg:clone--reserve-vertices}
  \State $\rhd$ Add vertices
  \ForAll{$u \in V$ \textbf{in parallel}} \label{alg:clone--add-vertices-begin}
    \State $G'.addVertex(u)$
  \EndFor \label{alg:clone--add-vertices-end}
  \State $\rhd$ Reserve space for edges
  \ForAll{$u \in V$ \textbf{in parallel}} \label{alg:clone--reserve-edges-begin}
    \State $G'.allocateEdges(u, G.degree(u))$
  \EndFor \label{alg:clone--reserve-edges-end}
  \State $\rhd$ Populate the edges
  \ForAll{$u \in V$ \textbf{in parallel}} \label{alg:clone--populate-edges-begin}
    \State $G'.edges(u) \gets G.edges(u)$
    \State $G'.degree(u) \gets G.degree(u)$
  \EndFor \label{alg:clone--populate-edges-end}
  \State $\rhd$ Update vertex and edge counts
  \If{$G.isCsr()$} $G'.update(true, false)$ \label{alg:clone--update-counts-csr}
  \Else\ $|V'| \gets |V|$ \textbf{;} $|E'| \gets |E|$ \label{alg:clone--update-counts-noncsr}
  \EndIf
  \Return{$G'$} \label{alg:clone--return}
\EndFunction
\end{algorithmic}
\end{algorithm}



\subsubsection{Performing Edge Deletions}
\label{sec:sub}

We now discuss our graph subtraction algorithms, which remove the edges of a graph $G_S$ from another graph $G$, effectively applying a batch edge deletions $E_S$ to $G$. The pseudocode is provided in Algorithm \ref{alg:sub}. Algorithm \ref{alg:sub} includes two functions: \textbf{(1)} \texttt{subtractGraphInplace()} modifies $G$ directly, removing edges in $G_S$ from $G$. \textbf{(2)} \texttt{subtractGraph()} returns a new graph $G'$ such that $G' = G \setminus G_S$ --- it is more efficient than simply cloning $G$, and then performing the graph subtraction in-place.

In \texttt{subtractGraphInplace()}, we initialize the count of removed edges, $\Delta M$, to zero. Next, we iterate over all vertices in $G_S$ in parallel (line \ref{alg:sub--inplace-for-begin}). If a vertex $u$ from $G_S$ is not present in $G$, it is skipped (line \ref{alg:sub--inplace-check}). Otherwise, we remove the edges of $u$ that are present in $G_S$ from $G$, updating $\Delta M$ accordingly (line \ref{alg:sub--inplace-remove}). Once all relevant edges have been deleted, the total edge count of $G$ is updated (line \ref{alg:sub--inplace-update-edges}), and the modified graph is returned (line \ref{alg:sub--inplace-return}). This function operates in-place, meaning the original graph $G$ is directly modified.

The \texttt{subtractGraph()} function constructs a new graph $G'$ instead of modifying $G$ directly. First, an empty graph $G'$ is initialized, and $\Delta M$ is set to zero (line \ref{alg:sub--init}). To ensure efficient memory allocation, we reserve space for the vertices in $G'$ (line \ref{alg:sub--reserve-vertices}). The vertices of $G$ are then added to $G'$ in parallel (lines \ref{alg:sub--add-vertices-begin}-\ref{alg:sub--add-vertices-end}). Similarly, memory is allocated for the edges of each vertex, based on its degree in $G$ (lines \ref{alg:sub--alloc-edges-begin}-\ref{alg:sub--alloc-edges-end}). The edge copying process is performed in two steps. First, for vertices that are not present in $G_S$, all edges from $G$ are copied directly into $G'$ (lines \ref{alg:sub--copy-untouched-begin}-\ref{alg:sub--copy-untouched-end}). This ensures that the structure of these vertices remains unchanged. Next, for vertices that do exist in $G_S$, we remove the edges that appear in $G_S$, keeping only those that are not part of $G_S$ (lines \ref{alg:sub--copy-touched-begin}-\ref{alg:sub--copy-touched-end}). The degree of each vertex is updated accordingly, and $\Delta M$ is incremented based on the number of removed edges. Finally, the vertex and edge counts for $G'$ are updated (line \ref{alg:sub--update-counts}), and the new graph $G'$ is returned (line \ref{alg:sub--return}).

\begin{algorithm}[hbtp]
\caption{Subtract a graph's edges from another graph.}
\label{alg:sub}
\begin{algorithmic}[1]
\Require{$G(V, E)$: Input graph to subtract from}
\Require{$G_S(V_S, E_S)$: Graph containing the edges to subtract}
\Require{$G'(V', E')$: Output graph, with edges from $G_S$ removed}
\Ensure{$\Delta M$: Number of edges deleted from $G$}

\Statex

\Function{subtractGraphInplace}{$G, G_S$}
  \State $\Delta M \gets 0$
  \State $\rhd$ Remove the edges in-place
  \ForAll{$u \in V_S$ \textbf{in parallel}} \label{alg:sub--inplace-for-begin}
    \If{\textbf{not} $G.hasVertex(u)$} \textbf{continue} \label{alg:sub--inplace-check}
    \EndIf
    \State $\Delta M \gets \Delta M + G.removeEdges(u, G_S.edges(u))$ \label{alg:sub--inplace-remove}
  \EndFor \label{alg:sub--inplace-for-end}
  \State $\rhd$ Update the edge count
  \State $|E| \gets |E| - \Delta M$ \label{alg:sub--inplace-update-edges}
  \Return{$G$} \label{alg:sub--inplace-return}
\EndFunction

\Statex

\Function{subtractGraph}{$G, G_S$}
  \State $G' \gets \{\}$ \textbf{;} $\Delta M \gets 0$ \label{alg:sub--init}
  \State $\rhd$ Reserve space for vertices
  \State $G'.reserve(maxVertexId(G) + 1)$ \label{alg:sub--reserve-vertices}
  \State $\rhd$ Add vertices
  \ForAll{$u \in V$ \textbf{in parallel}} \label{alg:sub--add-vertices-begin}
    \State $G'.addVertex(u)$
  \EndFor \label{alg:sub--add-vertices-end}
  \State $\rhd$ Reserve space for edges
  \ForAll{$u \in V$ \textbf{in parallel}} \label{alg:sub--alloc-edges-begin}
    \State $G'.allocateEdges(u, G.degree(u))$
  \EndFor \label{alg:sub--alloc-edges-end}
  \State $\rhd$ Add edges of vertices that are untouched
  \ForAll{$u \in V$ \textbf{in parallel}} \label{alg:sub--copy-untouched-begin}
    \If{$G_S.hasVertex(u)$} \textbf{continue}
    \EndIf
    \State $G'.edges(u) \gets G.edges(u)$
    \State $G'.degree(u) \gets G.degree(u)$
  \EndFor \label{alg:sub--copy-untouched-end}
  \State $\rhd$ Add edges of vertices that are touched
  \ForAll{$u \in V$ \textbf{in parallel}} \label{alg:sub--copy-touched-begin}
    \If{\textbf{not} $G_S.hasVertex(u)$} \textbf{continue}
    \EndIf
    \State $G'.edges(u) \gets G.edges(u) \setminus G_S.edges(u)$
    \State $G'.degree(u) \gets |G'.edges(u)|$
    \State $\Delta M \gets \Delta M + |G.edges(u)| - |G'.edges(u)|$
  \EndFor \label{alg:sub--copy-touched-end}
  \State $\rhd$ Update the vertex and edge count
  \State $|V'| \gets |V|$ \textbf{;} $|E'| \gets |E| - \Delta M$ \label{alg:sub--update-counts}
  \Return{$G'$} \label{alg:sub--return}
\EndFunction
\end{algorithmic}
\end{algorithm}



\subsubsection{Performing Edge Insertions}
\label{sec:add}

Next, we describe our graph union algorithms, detailed in Algorithm \ref{alg:add}, which merges the edges of a graph $G_A$ into another graph $G$, effectively applying a batch of edge insertions $E_A$ to $G$. This done either by modifying $G$ in place (in the \texttt{addGraphInplace()} function), or by generating a new output graph $G'$ (in the \texttt{addGraph()} function).

In the first function, \texttt{addGraphInplace()}, we modify $G$ directly. Initially, the counters $\Delta N$ and $\Delta M$ are set to zero, representing the number of new vertices and edges added, respectively (line \ref{alg:add--inplace-init}). We then allocate space for vertices by reserving memory for $max(\text{vertex ID in } G, \text{vertex ID in } G_A) + 1$\ignore{(line \ref{alg:add--inplace-reserve})}. Next, new vertices from $G_A$ that are not already in $G$ are added in parallel, updating $\Delta N$ accordingly (lines \ref{alg:add--add-vertices-begin}-\ref{alg:add--add-vertices-end}). After this, edges from $G_A$ are inserted into $G$ in-place (lines \ref{alg:add--inplace-add-edges-begin}-\ref{alg:add--inplace-add-edges-end}), using the \texttt{G.addEdges()} function detailed in Algorithm \ref{alg:digraph2}, and the count of newly added edges is accumulated in $\Delta M$. Finally, the vertex and edge counts of $G$ are updated to reflect the additions (line \ref{alg:add--inplace-update}), and the modified graph $G$ is returned\ignore{(line \ref{alg:add--inplace-return})}.

The second function, \texttt{addGraph()}, constructs a new graph $G'$ containing the union of graphs $G$ and $G_A$. The function starts by initializing $G'$ as an empty graph and setting $\Delta N$ and $\Delta M$ to zero (line \ref{alg:add--init}). Similar to the in-place version, we allocate space for vertices (line \ref{alg:add--reserve}) and then iterate through the vertex range to add vertices from $G$ and $G_A$, tracking the count of new additions in $\Delta N$ (lines \ref{alg:add--add-vertices-begin}-\ref{alg:add--add-vertices-end}). After reserving memory for edges (lines \ref{alg:add--reserve-edges-begin}-\ref{alg:add--reserve-edges-end}), we proceed to populate $G'$ with edges. First, we copy edges for vertices that exist only in $G$ (lines \ref{alg:add--add-edges-untouched-begin}-\ref{alg:add--add-edges-untouched-end}). Then, for vertices present in both $G$ and $G_A$, we merge their edge sets and update the degree count (lines \ref{alg:add--add-edges-touched-begin}-\ref{alg:add--add-edges-touched-end}). The total number of newly added edges is accumulated in $\Delta M$. Finally, the vertex and edge counts for $G'$ are updated (line \ref{alg:add--update}), and the new graph is returned (line \ref{alg:add--return}).

\begin{algorithm}[hbtp]
\caption{Add a graph's edges to another graph.}
\label{alg:add}
\begin{algorithmic}[1]
\Require{$G(V, E)$: Input graph to add to}
\Require{$G_A(V_A, E_A)$: Graph containing the edges to add}
\Require{$G'(V', E')$: Output graph, with edges from $G_A$ added}
\Ensure{$\Delta N$: Number of vertices added from $G_A$}
\Ensure{$\Delta M$: Number of edges inserted from $G_A$}

\Statex

\Function{addGraphInplace}{$G, G_A$}
  \State $\Delta N \gets \Delta M \gets 0$ \label{alg:add--inplace-init}
  \State $\rhd$ Reserve space for vertices
  \State $G.reserve(maxVertexId(G, G_A) + 1)$ \label{alg:add--inplace-reserve}
  \State $\rhd$ Add new vertices
  \ForAll{$u \in G_A$ \textbf{in parallel}} \label{alg:add--inplace-add-vertices-begin}
    \If{$G.hasVertex(u)$} \textbf{continue}
    \EndIf
    \State $G.addVertex(u)$ \textbf{;} $\Delta N \gets \Delta N + 1$
  \EndFor \label{alg:add--inplace-add-vertices-end}
  \State $\rhd$ Add the edges in-place
  \ForAll{$u \in V$ \textbf{in parallel}} \label{alg:add--inplace-add-edges-begin}
    \State $\Delta M \gets \Delta M + G.addEdges(u, G_A.edges(u))$
  \EndFor \label{alg:add--inplace-add-edges-end}
  \State $\rhd$ Update the vertex and edge count
  \State $|V| \gets |V| + \Delta N$ \textbf{;} $|E| \gets |E| + \Delta M$ \label{alg:add--inplace-update}
  \Return{$G$} \label{alg:add--inplace-return}
\EndFunction

\Statex

\Function{addGraph}{$G, G_A$}
  \State $G' \gets \{\}$ \textbf{;} $\Delta N \gets \Delta M \gets 0$ \label{alg:add--init}
  \State $\rhd$ Reserve space for vertices
  \State $G'.reserve(maxVertexId(G, G_A) + 1)$ \label{alg:add--reserve}
  \State $\rhd$ Add vertices
  \ForAll{$u \in [0, maxVertexId(G, G_A)]$ \textbf{in parallel}} \label{alg:add--add-vertices-begin}
    \If{$G.hasVertex(u)$}
      \State $G'.addVertex(u)$
    \ElsIf{$G_A.hasVertex(u)$}
      \State $G'.addVertex(u)$ \textbf{;} $\Delta N \gets \Delta N + 1$
    \EndIf
  \EndFor \label{alg:add--add-vertices-end}
  \State $\rhd$ Reserve space for edges
  \ForAll{$u \in V'$ \textbf{in parallel}} \label{alg:add--reserve-edges-begin}
    \State $G'.allocateEdges(u, G.degree(u) + G_A.degree(u))$
  \EndFor \label{alg:add--reserve-edges-end}
  \State $\rhd$ Add edges of vertices that are untouched
  \ForAll{$u \in V'$ \textbf{in parallel}} \label{alg:add--add-edges-untouched-begin}
    \If{$G_A.hasVertex(u)$} \textbf{continue}
    \EndIf
    \State $G'.edges(u) \gets G.edges(u)$
    \State $G'.degree(u) \gets G.degree(u)$
  \EndFor \label{alg:add--add-edges-untouched-end}
  \State $\rhd$ Add edges of vertices that are touched
  \ForAll{$u \in V'$ \textbf{in parallel}} \label{alg:add--add-edges-touched-begin}
    \If{\textbf{not} $G.hasVertex(u)$} \textbf{continue}
    \EndIf
    \State $G'.edges(u) \gets G.edges(u) \cup G_A.edges(u)$
    \State $G'.degree(u) \gets |G'.edges(u)|$
    \State $\Delta M \gets \Delta M + |G'.edges(u)| - |G.edges(u)|$
  \EndFor \label{alg:add--add-edges-touched-end}
  \State $\rhd$ Update the vertex and edge count
  \State $|V'| \gets |V| + \Delta N$ \textbf{;} $|E'| \gets |E| + \Delta M$ \label{alg:add--update}
  \Return{$G'$} \label{alg:add--return}
\EndFunction
\end{algorithmic}
\end{algorithm}









% %% DISCUSS WHAT WE NEED FOR FAST DYNAMIC GRAPHS
% Let's think what a graph would require. Concurrent allocators of various sizes.

% - Quickly free and allocate a new memory block. It also better be multi-level.
% - Ensure that vertex access is efficient, by using SoA vectors.
% - Ensure that edge weights are stored in a separate array.
% - Delay merging of edges into a single array, until the graph is finalized.
% - Select a different merging startegy, based on the number of edges inserted / deleted.
% - Perform a fast deletion of edges, i.e., deletions come first.

% Interesting to see the runtime split of duplicateIfOmpW(), which is used to convert the graph from CSR to DiGraph. Reserving space for the graph, and for the edges appear to be the largest bottlenecks, followed by updateOmpU().

% Arena-based digraph is now working. The double-free issue was due to using a single allocator across all threads. Another issue we faced was the failure to populate the edges, resulting in abrupt program crash. This was due to using \_\_builtin\_clz() for computing capacity of the slots, but \_\_builtin\_clz() only works for 32-bit integer, not size\_t. Below are the results after fixing.

% Comparison begins:
% - output-graph-openmp--approach-csr-graph-no-sort.log: Upto 2KB are handled by arena allocator (with a capacity of 8 * 64KB), remaining are managed by libc.
% - output-graph-openmp--approach-duplicate-8192.log: Upto 8KB are handled by arena allocator (with a capacity of 10 * 512KB), remaining are managed by libc. We also try to minimize the memory being held by the arena allocators. Tested on sk-2005 graph. Best config seems to be using up to 8KB arena allocators, with capacity of each allocator being 512KB.

% BTW, why is resizing the arrays for vertices too slow, particularly for kmer graphs?

% Continuing:
% - output-graph-openmp--approach-duplicate-8192-custom-arrays.log: Upto 8KB are handled by arena allocator (with a capacity of 10 * 512KB), remaining are managed by libc. Now we are managing the vertex arrays on our own, so we can initialize them in parallel. The improved locality of the variables seems to be improving the performance of add-edges as well. vector.resize() is sequential and is a bottleneck for graphs with a large number of vertices (such as k-mer graphs).
% - output-graph-openmp--measure-sort-time.log: It seems choosing a suitable chunk size has a significant impact on sorting edges.
% - output-graph-openmp--adjust-sort-method.log: Dynamic schedule with task (deg > 2048) seems best.
% - puzzlef/graph-openmp--measure-duplicate-split: Below is the phase split of duplicate graph (arena allocator-based).
% - output-graph-openmp--adjust-add-subtract-chunk-size.log: Measuring runtime with various chunk size for add and subtract graph operations.
% - output-graph-openmp--adjust-batch-fraction.log: Results with varying batch fraction.


% %% ON GRAPH LOADING
% A crash issue was observed while trying to load a graph in MTX format as CSR. This was related to improperly setting up CSR lists (before passing it to the reading function). In addition, I managed to optimize the CSR generation phase by by using the first partition CSR as the target CSR with sufficient space (adjust \texttt{degrees[0]} accordingly), and eliminating the need to fix offsets, by using shifted offsets instead. TODO: Make a figure for this.


% %% ON MEMORY ALLOCATORS
% The \texttt{FixedArenaAllocator} is fairly straightforward. It simply consists of a fixed size memory pool, a number indicating the number of bytes used in the memory pool, and a list of freed allocations, which can be reused. The allocator is created by providing the address of the memory pool, the capacity of the memory pool, and the size of each allocation (each allocation must be of the same size). The \texttt{allocate()} function allocates from the free list, popping a memory address from the freed list, if available. If not, it allocates from the memory pool, updating the used bytes. The \texttt{deallocate()} function simply pushes the provided memory address into the freed list. Finally, a \texttt{reset()} function simply sets the used bytes to $0$, and clears the freed list. The psuedocode for this is given in Algorithm X.


% % A failed attempt.
% We next consider implementing a concurrent arena allocator, which given a memory pool, returns allocated memory of fixed size memory blocks. It also maintains a list of freed memory blocks, and any new requests are serviced from the freed blocks. However, in order to be thread-safe, i.e., to support memory allocation calls from multiple threads concurrently, we are using a \texttt{atomic\_flag} which is used as a mutex to limit access to the freed list to one thread at a time only. In order to ensure that no thread has to wait for too long, we check if the \textit{flag} is set to \textit{busy}, and if so, we go ahead an return a block from the pool (which uses an atomic add). However, it is also possible that we have no free space in the pool. In such a case, we repeatedly retry acquiring the mutex (\textit{yield}ing the current thread on failure and retrying), and once obtained, fetch a memory block from the freed list. For freeing a memory block, however, we just do repeated retries to access the freed list, and then, once acquired, append the memory block to the freed list.

% When (several) allocations are being performed sequentially, the concurrent arena allocator is observed to have around $2\times$ the performance of \texttt{malloc()} and \texttt{free()}. However, when allocations are done in parallel using 64 thread, the results look quite different. The concurrent arena allocator has around the same performance as \texttt{malloc()}, while the deallocate of the concurrent memory allocator is too slow. Thus, it appears that there is high contention in our concurrent arena allocator. We could try using per-thread freed lists to resolve this. But then, we might have to implement some stealing mechanism in order to fetch memory blocks that have been freed by other threads. We might as well use \textit{libc} \texttt{malloc()} instead - particularly for large size memory allocations.

% In fact, we have another recursive arena allocator which utilizes multiple pools to allocate smaller memory blocks to requesting thread, and is not thread-safe (for one thread only). A suitable way, then, might be to use \texttt{malloc()} for large allocations, and revert to per-thread recursive arena allocator for smaller allocations - we may now called it multi arena allocator instead.


% % Continue dicusssion
% We use a pool size of $512$KB.

% Per-thread allocators are well-separated in memory to avoid false sharing.

% Sequential arena-allocator is fast, and an arena allocator with growing pool is also fast. If each thread has its own allocator, then performance is great. An interesting thing to note here is that \texttt{new[]} is quite fast, but not \texttt{malloc()}. Also, weird why \texttt{FixedArenaAllocator.deallocate()} is slow, both arena allocators reserve the same amount of memory for freed list.

% Another interesting workload, here we allocate and free memory repeatedly (in chunks). In all cases, however, the workload consists of $2^{28}$ allocations and deallocations, each of size $64$ bytes. Here, both \texttt{malloc()} and \texttt{new[]} have similar performance, and interestingly \texttt{ArenaAllocator} is quite fast.

% Now we have a concurrent pow-2 allocator.

% TODO: Make a figure for this.


% %% ON MEMORY MANAGEMENT IN GRAPH DATASTRUCTURE


\section{Evaluation}
\label{sec:evaluation}
%%%%%%%%%%%%%%%%%%%%%%%%%%%%%%%%%%%%%%%%%%%%
\begin{table*}[t]
\centering
\begin{tabular}{@{}l|ccc|ccc|ccc@{}}
\toprule
\multicolumn{1}{c} {\textbf{Time} (\textit{s})}&
  \multicolumn{1}{l}{\textbf{4xA100}} &
  \multicolumn{1}{l}{\textbf{2xA100}} &
  \multicolumn{1}{l}{\textbf{A100}} &
  \multicolumn{1}{l}{\textbf{4xA6000}} &
  \multicolumn{1}{l}{\textbf{2xA6000}} &
  \multicolumn{1}{l}{\textbf{A6000}} &
  \multicolumn{1}{l}{\textbf{4xV100}} &
  \multicolumn{1}{l}{\textbf{2xV100}} &
  \multicolumn{1}{l}{\textbf{V100}} \\ \midrule
CPU Freezing (\textit{s}) & 21.49 & 10.32 & 4.96  & 15.23 & 9.11  & 3.35  & 29.41  & 14.50 & 6.90  \\
CPU Frozen (\textit{s}) & 33.58 & 16.15 & 7.79  & 43.69 & 29.48 & 11.24 & 74.96  & 38.56 & 19.23 \\
CPU Mem. dump (\textit{s}) & 31.30 & 15.02 & 7.28  & 42.1  & 28.59 & 10.88 & 70.30  & 36.17 & 18.10 \\
CPU Mem. write (\textit{s}) & 28.62 & 13.99 & 6.80  & 40.4  & 27.70 & 10.49 & 66.30  & 34.40 & 17.26 \\ \midrule
%
\sys Checkpoint (\textit{s}) & 55.09 & 26.49 & 12.78 & 58.93 & 38.61 & 14.60 & 104.40 & 53.08 & 26.18 \\
\sys Restore (\textit{s}) & 35.13 & 17.22 & 8.32  & 24.1  & 13.83 & 5.50  & 43.14  & 21.69 & 10.61 \\ \midrule
Checkpoint size (GB) & 41.01 & 20.46 & 9.94 & 39.98 & 19.97 & 9.75 & 40.03  & 19.97 & 9.81 \\ \bottomrule
\end{tabular}\par
\vspace{-0.5em}
\captionsetup{justification=centering}
\caption{Checkpoint and restore performance (in seconds) when scaling training of GPT-2 Small (124M) to multiple GPUs.\\Times are not comparable across GPU families.}
\label{tab:multi-gpu-checkpointing}
\end{table*}

%%%%%%%%%%%%%%%%%%%%%%%%%%%%%%%%%%%%%%%%%%%%%%%%
\begin{figure}[t]
    \centering
    \includegraphics[width=.9\columnwidth]{figures/H100_Lock_Checkpoint_vs_Restore_Unlock.pdf}
    \vspace{-.5em}
    \caption{In-memory GPU checkpoint/restore with H100. Similar results are observed with A100.}
    \label{fig:in-memory-checkpoint-restore}
\end{figure}
%%%%%%%%%%%%%%%%%%%%%%%%%%%%%%%%%%%%%%%%%%%%%%%%
\begin{figure*}[t]
  \centering
  \begin{subfigure}[b]{\columnwidth}
    \includegraphics[width=.9\textwidth]{figures/H100_Unified_Restore_Times.pdf}
    \label{fig:h100-unified-restore-times}
  \end{subfigure}
  \hfill
  \begin{subfigure}[b]{\columnwidth}
    \includegraphics[width=.9\textwidth]{figures/A100_Unified_Restore_Times.pdf}
    \label{fig:a100-unified-restore-times}
  \end{subfigure}
  \vspace{-1em}
  \caption{Time to restore for model training from a checkpoint with \sys for H100 and A100 GPUs.}
  \label{fig:unified-restore-times}
\end{figure*}
%%%%%%%%%%%%%%%%%%%%%%%%%%%%%%%%%%%%%%%%%%%%%%%%

\section{Evaluation} \label{sec:evaluation}%
%
Our evaluation seeks to answer the following questions:
\begin{itemize}[leftmargin=*,leftmargin=15pt,itemindent=0pt]
    \item How does \sys perform when checkpointing and restoring large language models? (\textsection{\ref{sec:eval:diff-models}})

    \item What are the scalability implications of using checkpointing and restoring with multiple GPU devices? (\textsection{\ref{sec:eval:scalability}})

    \item What are the dominant factors affecting the latency of checkpointing and restore operations? (\textsection{\ref{sec:eval:overhead}})

    \item Can \sys support checkpoint and restore with both CUDA and ROCm applications? (\textsection{\ref{sec:eval:rocm}})
\end{itemize}

\subsection{Experimental Methodology}%
%
\stitle{Evaluation Setup.}
We evaluate \sys on 10 servers with specifications described in \Cref{tab:server-configs}, running Ubuntu 22.04 with kernel version 6.2.0 (A100 and H100), 5.15.0 (V100 and A6000), and NVIDIA driver 565.57.01, CUDA 12.7, and CentOS Stream 9 with kernel 5.14 with ROCm 5.6 (MI210).

\stitle{Performance Measurements.} We measure the performance of checkpoint and restore operations using detailed statistics generated by CRIU~\cite{criu-statistics} about the time spent in different stages, and with external tools like \texttt{perf stat} to gather more detailed performance data. We run each experiment 10 times and calculate the mean and standard deviation of each value in the collected data. To analyze the overhead of checkpointing with \sys, we measure the following performance metrics for models of different sizes:

\begin{itemize}[leftmargin=*,leftmargin=10pt,itemindent=0pt]
    \item \textbf{Checkpoint time:} The total time to create a snapshot of the running GPU application.
    \item \textbf{Freezing time:} The time to suspend the application using \texttt{ptrace} seize and interrupt.
    \item \textbf{Frozen time:} The time during checkpointing when the application is not running.
    \item \textbf{Memory dump time:} The time to collect the CPU memory pages of running processes. This does not include the time to write this memory to storage.
    \item \textbf{Memory write time:} The time to save the memory state to persistent storage.
    \item \textbf{Restore time:} The time to restore both CPU and GPU state from storage, and to resume the application.
\end{itemize}

\stitle{Workloads and micro-benchmarks.} We evaluate the proposed checkpoint/restore mechanisms for multiple models of different sizes (listed below) and a set of ROCm micro-benchmarks~\cite{amd2024rocm} representing common HPC workloads for AMD GPU~(\textsection{\ref{sec:eval:rocm}}). For NVIDIA GPUs we use the following models:

\begin{itemize}[leftmargin=*,leftmargin=10pt,itemindent=0pt]
    \item \textbf{LLaMA} \textbf{3.2} (1B, 3B) and \textbf{3.1} (8B)

    \item \textbf{GPT-2} with 124M, 355M, 774M, 1.5B parameters

    \item \textbf{BERT} Base (110M) and Large (340M) models
\end{itemize}
%
\subsection{\sys Performance with Deep Learning Models}
\label{sec:eval:diff-models}
%
We evaluate the performance of GPU checkpoint and restore operations with multiple model training workloads of different sizes. The results in~\Cref{fig:in-memory-checkpoint-restore} show that the time required to checkpoint the GPU state into host memory increases significantly for models with large number of parameters.
For instance, checkpointing and locking operations for the GPT-2 Small model (124M parameters) take an average of 4.9 seconds and 240 ms, respectively. In comparison, for a larger model such as GPT-2 XL (1.5B parameters), these operations require an average of 28 seconds and 500 ms, respectively.
The time to restore the GPU state from host memory increases gradually, with 2.5 seconds for GPT-2 Small and 11 seconds for GPT-2 XL, while the unlock time remains consistent for both models at approximately 160 ms.
We observe similar results with both A100 and H100 GPUs, highlighting the crucial impact memory bandwidth has on the performance of GPU checkpointing operations.
Several techniques have been proposed to address this problem, such as data compression and on-demand parallelism~\cite{yang2024on-demand}. Incorporating such techniques could further improve the efficiency of checkpoint/restore operations, especially for large-scale models.

~\Cref{fig:unified-restore-times} shows the unified restore time (the time to restore the combined CPU-GPU state) for models with different sizes with both H100 and A100 GPUs. The time required to restore the GPU state makes up a significant portion of the total restore time for small models, but becomes a relatively lesser portion for larger models. These results demonstrate that the restore time is also affected by the available bandwidth for CPU-GPU memory transfers, as well as the speed at which checkpoint data is loaded from disk into host memory.
The performance results and checkpoint sizes shown in \Cref{tab:combined-checkpoint-restore-times} suggest that the differences in performance between the experiments with H100 and A100 GPUs can be attributed not only to advancements in GPU hardware architecture but also to the critical role of the available CPU resources.

%%%%%%%%%%%%%%%%%%%%%%%%%%%%%%%%%%%%%%%%%%%%%%%%
\begin{table}[t]
\centering
\renewcommand{\arraystretch}{.9} % Reduce row height for tighter spacing
\begin{tabular}{@{}c|rrr@{}}
\toprule
\textbf{Model} & \textbf{Total (GB)} & \textbf{GPU (\%)} & \textbf{CPU (\%)} \\ \midrule
BERT-B~~(110M)  & 5.22  & 82.38\%  & 17.62\% \\
GPT2-S~~(124M)  & 10.32 & 89.15\%  & 10.85\% \\
BERT-L~~(340M)  & 9.90  & 90.91\%  & 9.09\%   \\
GPT2-M~(355M)  & 19.57 & 91.31\% & 8.69\%   \\
GPT2-L~~(774M)  & 35.39 & 90.99\% & 9.01\%  \\
LLaMA 3.2~~~(1B) & 14.81 & 92.37\% & 7.63\%  \\
LLaMA 3.2~~~(3B) & 29.54 & 95.70\%  & 4.30\%   \\
LLaMA 3.1~~~(8B) & 55.89 & 97.35\% & 2.65\%  \\
GPT2-XL~(1.5B) & 60.12 & 96.02\% & 3.98\%  \\ \bottomrule
\end{tabular}
\vspace{-.5em}
\caption{The total unified checkpoint size (in GB) and the corresponding proportions of GPU memory and CPU state, respectively, for various models running on H100 GPU. We observe similar results with A100 GPU.}
\label{tab:cpu-to-gpu-state-comparison}
\vspace{-1.5em}
\end{table}
%%%%%%%%%%%%%%%%%%%%%%%%%%%%%%%%%%%%%%%%%%%%%%%%

\Cref{tab:cpu-to-gpu-state-comparison}  shows the total \sys checkpoint sizes for various models on both A100 and H100 GPUs, along with a breakdown of GPU memory and CPU state proportions. A key insight is the dominance of GPU memory in overall checkpoint size, consistently exceeding 80\% and often surpassing 90\% for larger models like LLaMA 3.1 (8B) and GPT2-XL (1.5B). These results further emphasize the importance of efficient CPU-GPU memory transfers in optimizing the performance of checkpointing and restore operations.

%%%%%%%%%%%%%%%%%%%%%%%%%%%%%%%%%%%%%%%%%%
\subsection{Multi-GPU Checkpointing Performance}
\label{sec:eval:scalability}
%
We evaluate the scalability of \sys by running an experiment designed to train a large language model (GPT-2) across 1x, 2x, and 4x GPUs of V100, A6000, and A100 types, using data parallelism to distribute the workload.
\Cref{tab:multi-gpu-checkpointing} shows that the checkpoint size increases with the number of GPUs, as each GPU stores its own copy of the model parameters.
For instance, the checkpoint size for 1, 2, and 4 A100 GPUs is $\approx10$, $\approx20$, and $\approx40$ GB, respectively.
This increase in checkpoint size reflects the increasing amount of intermediate model state that is saved as the number of GPUs increases.
The freezing, frozen, memory dump and memory write times also increase with the number of GPUs, likely because more time is spend on handling the larger checkpoint data with additional GPUs.
For example, as the number of A100 GPUs increases, creating a unified CPU-GPU snapshot requires scanning more memory pages.
A single GPU requires $\approx7$ million pages scanned, two GPUs require $\approx15$ million, and four GPUs require $\approx28$ million. Our experiments demonstrate that \sys efficiently scales as the number of GPUs and data-parallel replicas increase. We observe that checkpointing and restore times scale near linearly as we increase the number of GPUs from 1 to 4 across different GPU types.

\subsection{Checkpoint and Restore Latency}%
\label{sec:eval:overhead}%
We analyze the overhead of checkpoint and restore operations of \sys by measuring the latency during these processes for LLaMA 3.1 and GPT2-XL model training workloads on H100 and A100 GPUs. The results in ~\Cref{tab:combined-checkpoint-restore-times} highlight the performance differences between the H100 and A100 GPUs, as well as the impact of additional CPU and memory resources on reducing overhead during the checkpoint and restore operations with \sys. The primary factors affecting \sys's checkpoint and restore performance are:

\begin{itemize}
    \item \textbf{GPU count}: Increasing the number of GPUs leads to increased checkpoint size and latency;
    \item \textbf{CPU-GPU bandwidth}: The speed of data transfers between the CPU and GPUs directly affects checkpointing and restoring speed;
    \item \textbf{GPU memory usage}: Larger models with more parameters have higher checkpoint and restore latencies.
\end{itemize}

\begin{figure}[t]
  \centering
  \includegraphics[width=\columnwidth]{figures/MI210_Unified_Checkpointing_Times.pdf}
  \vspace{-2.5em}
  \caption{Breakdown of the \sys checkpointing time for HPC benchmarks running on AMD MI210 GPU.}
  \label{fig:amd-gpu-checkpointing}
  \vspace{-.5em}
\end{figure}
%%%%%%%%%%%%%%%%%%%%%%%%%%%%%%%%%%%%%%%%%
\begin{table}[t]
\centering
\begin{tabular}{lr}
\toprule
\textbf{Benchmark} & \textbf{Checkpoint Size}
\\ \midrule
Binomial Option Pricing       & 305~MB \\
Bitonic Sort                  & 614~MB \\
Discrete Cosine Transform     & 1.2~GB \\
1D Haar Wavelet Decomposition & 333~MB \\
Fast Walsh Transform          & 307~MB \\
Floyd Warshall                & 484~MB \\
Prefix Sum                    & 306~MB \\
Recursive Gaussian            & 311~MB \\
Histogram                     & 16.64 GB \\
Matrix Multiplication         & 19.88 GB \\
Convolution                   & 13.83 GB \\
\bottomrule
\end{tabular}
\caption{\sys checkpoint sizes of ROCm benchmarks.}
\label{tab:rocm-benchmark-checkpoints}
\vspace{-1em}
\end{table}

%%%%%%%%%%%%%%%%%%%%%%%%%%%%%%%%%%%%%%%%%
\subsection{\sys Support for ROCm Devices}
\label{sec:eval:rocm}

In addition to support for CUDA, to demonstrate the checkpointing functionality of \sys for AMD GPUs, we evaluate its performance using a set of ROCm HPC micro-benchmarks. These benchmarks provide a set of workloads representative of typical HPC applications, allowing us to analyse \sys's ability to effectively checkpoint and restore GPU state across different computational patterns. \Cref{fig:amd-gpu-checkpointing} shows the frozen, memory dump, and memory write times during checkpointing for each benchmark. While most of the evaluated benchmarks have relatively small checkpoint size, typically ranging from under 500~MB to 1.2~GB, a few have significantly larger checkpoint sizes (Histogram, Matrix Multiplication, and Convolution), shown in \Cref{tab:rocm-benchmark-checkpoints}. This increase in checkpoint size directly correlates with longer freezing and memory dump times, as \sys must checkpoint larger amount of data. An interesting observation is the contrasting distribution of checkpoint data between host memory and GPU memory across these benchmarks. While more than half of the checkpoint size for the Convolution benchmark is attributed to AMD GPU state, Histogram and Matrix Multiplication have the majority of their state residing in host memory.


\section{Conclusion}
\label{sec:conclusion}
\section{Conclusion}\label{sec:con}

Our work contributes empirical insights on the photorealism of AI-generated images and a taxonomy of artifacts commonly found in AI-generated images, organized into five categories: anatomical implausibilities, stylistic artifacts, functional implausibilities, violations of physics, and sociocultural implausibilities. We find that the photorealism of AI-generated images depends on the scene complexity of the image, the kind of artifacts and implausibilities, if any, detectable in an image, the duration of visual attention to an image, and the quality of human effort to select appropriate prompts and curate images. A question such as ``How photorealistic are state-of-the-art diffusion models'' may sound simple, but the answer is more complex and depends on many details, including what images are generated and selected, how photorealism is measured, what real images are included in the experiment, and how much time, skill, and effort a human participant has and willing to offer. This paper offers an initial exploration into how we can address this question and develops a practical taxonomy that offers scaffolding for building AI--literacy interventions to help people navigate the capabilities and limitations of diffusion models and whether an image is AI-generated or not. 

\begin{acks}

This material is based upon work supported by Robert Pozen, and in part with funding from the Department of Defense (DoD). Any opinions, findings, conclusions, or recommendations expressed in this material are those of the authors and do not necessarily reflect the views of the DoD or any agency or entity of the United States Government. We thank Will Thompson from Kellogg Research Support for performing a replication check.
\end{acks}

%% The acknowledgments section.
\begin{acks}
I would like to thank Prof. Kishore Kothapalli, Prof. Sathya Peri, and Prof. Dip Sankar Banerjee for their support.
\end{acks}

%% Bibliography style to be used, and the bibliography file.
\bibliographystyle{ACM-Reference-Format}
\bibliography{main}

\clearpage
\clearpage
\appendix
\section{Appendix}

\subsection{Memory Allocators}
\label{sec:memory-allocators}

We now discuss a few memory allocators, which support the design of our graph representation.

\begin{algorithm}[H] % hbtp
\caption{A Fixed-capacity Arena Allocator (FAA).}
\label{alg:faa}
\begin{algorithmic}[1]
\Require{$\textsc{alloc\_size}_a$: Size of each allocation (constant)}
\Require{$\textsc{pool\_size}_a$: Size of the memory block/pool (constant)}
\Require{$pool_a$: A memory block for the pool}

\Statex

\State \textbf{struct} $FAA \langle \textsc{alloc\_size}_a, \textsc{pool\_size}_a \rangle (pool_a)$
\State \ \ $\textsc{alloc\_size} \gets \textsc{alloc\_size}_a$ \Comment{Size of each allocation} \label{alg:faa--init-const-begin}
\State \ \ $\textsc{pool\_size} \gets \textsc{pool\_size}_a$ \Comment{Size of the memory pool} \label{alg:faa--init-const-end}
\State \ \ $freed \gets \{\}$ \Comment{Freed allocations, can be reused} \label{alg:faa--init-begin}
\State \ \ $used \gets 0$ \Comment{Bytes used in the memory pool}
\State \ \ $pool \gets pool_a$ \Comment{The memory pool} \label{alg:faa--init-end}

\Statex

\State $\rhd$ Allocate memory, of size $A.\textsc{alloc\_size}$
\Function{allocate}{$ $} \textbf{of} FAA
  \State $\rhd$ Allocate from freed list, if available
  \If{$freed \neq \{\}$} \ReturnInline{$freed.pop()$} \label{alg:faa--allocate-freed}
  \EndIf
  \State $\rhd$ Allocate from pool
  \If{$used < \textsc{pool\_size}$} \label{alg:faa--allocate-pool-begin}
    \State $ptr \gets pool + used$
    \State $used \gets used + \textsc{alloc\_size}$
    \Return{$ptr$}
  \EndIf \label{alg:faa--allocate-pool-end}
  \Return{$\phi$} \label{alg:faa--allocate-phi}
\EndFunction

\Statex

\State $\rhd$ Free allocated memory
\Function{deallocate}{$ptr$} \textbf{of} FAA
  \State $freed.push(ptr)$ \label{alg:faa--deallocate-push}
\EndFunction

\Statex

\State $\rhd$ Free all allocated memory
\Function{reset}{$ $} \textbf{of} FAA
  \State $freed \gets \{\}$ \label{alg:faa--reset-begin}
  \State $used \gets 0$ \label{alg:faa--reset-end}
\EndFunction
\end{algorithmic}
\end{algorithm}



\subsubsection{Fixed Arena Allocator (FAA)}
\label{sec:faa}

We now discuss the Fixed-capacity Arena Allocator (FAA), outlined in Algorithm \ref{alg:faa}, which provides a lightweight, efficient memory allocation strategy for scenarios requiring frequent allocations and deallocations within a fixed memory budget. FAA minimizes fragmentation by recycling memory blocks and allows for rapid reset operations, making it well-suited for high-performance applications. It operates through three primary functions: \texttt{allocate()}, \texttt{deallocate()}, and \texttt{reset()}.

The allocator is instantiated with a predefined allocation size $\textsc{alloc\_size}_a$ and a total pool size $\textsc{pool\_size}_a$ (lines \ref{alg:faa--init-const-begin}-\ref{alg:faa--init-const-end}). The allocator maintains an internal memory pool $pool$, a counter $used$ tracking the number of allocated bytes, and a list $freed$ to store deallocated memory chunks for reuse (lines \ref{alg:faa--init-begin}-\ref{alg:faa--init-end}). Memory allocation is handled by the \texttt{allocate()} function. First, if any previously freed memory blocks exist in $freed$, one is retrieved and returned (line \ref{alg:faa--allocate-freed}). Otherwise, allocation proceeds from the main memory pool. If there is available space in the pool (i.e., $used < \textsc{pool\_size}$), the function assigns a pointer to the next available block, updates the $used$ counter, and returns the pointer (lines \ref{alg:faa--allocate-pool-begin}-\ref{alg:faa--allocate-pool-end}). If the pool is exhausted, the function returns a null pointer $\phi$, indicating allocation failure (line \ref{alg:faa--allocate-phi}). The \texttt{deallocate()} function allows memory to be freed by pushing the given pointer into the $freed$ list, making it available for future reuse (line \ref{alg:faa--deallocate-push}). To completely reset the allocator, the \texttt{reset()} function clears the $freed$ list and resets the $used$ counter to zero, effectively deallocating everything (lines \ref{alg:faa--reset-begin}-\ref{alg:faa--reset-end}).


\subsubsection{Variable-capacity Arena Allocator (AA)}
\label{sec:aa}

Unlike the Fixed-capacity Arena Allocator (FAA), the variable-capacity Arena Allocator (AA) does not have a fixed memory budget. Instead, it dynamically allocates additional pools as needed using \texttt{new[]} from the C++ standard library. Its pseudocode is shown in Algorithm \ref{alg:aa} and, like the FAA, it operates through three primary functions: \texttt{allocate()}, \texttt{deallocate()}, and \texttt{reset()}.

\begin{algorithm}[hbtp]
\caption{A variable-capacity Arena Allocator (AA).}
\label{alg:aa}
\begin{algorithmic}[1]
\Require{$\textsc{alloc\_size}_a$: Size of each allocation (constant)}
\Require{$\textsc{pool\_size}_a$: Size of each memory pool (constant)}

\Statex

\State \textbf{struct} $AA \langle \textsc{alloc\_size}_a, \textsc{pool\_size}_a \rangle ()$
\State \ \ $\textsc{alloc\_size} \gets \textsc{alloc\_size}_a$ \Comment{Size of each allocation} \label{alg:aa--init-const-begin}
\State \ \ $\textsc{pool\_size} \gets \textsc{pool\_size}_a$ \Comment{Size of each memory pool} \label{alg:aa--init-const-end}
\State \ \ $freed \gets \{\}$ \Comment{Freed allocations, can be reused} \label{alg:aa--init-begin}
\State \ \ $used \gets \textsc{pool\_size}$ \Comment{Bytes used in the last memory pool}
\State \ \ $pools \gets \{\}$ \Comment{Memory pools} \label{alg:aa--init-end}

\Statex

\State $\rhd$ Allocate memory, of size $\textsc{alloc\_size}$
\Function{allocate}{$ $} \textbf{of} AA
  \State $\rhd$ Allocate from freed list, if available
  \If{$freed \neq \{\}$} \ReturnInline{$freed.pop()$} \label{alg:aa--allocate-freed}
  \EndIf
  \State $\rhd$ Allocate from pool
  \If{$used < \textsc{pool\_size}$} \label{alg:aa--allocate-pool-begin}
    \State $ptr \gets pools.last() + used$
    \State $used \gets used + \textsc{alloc\_size}$
    \Return{$ptr$}
  \EndIf \label{alg:aa--allocate-pool-end}
  \State $\rhd$ Allocate a new pool
  \State $ptr \gets$ Allocate $\textsc{pool\_size}$ bytes \label{alg:aa--allocate-newpool}
  \If{$ptr \neq \phi$} \label{alg:aa--allocate-new-begin}
    \State $pools.push(ptr)$
    \State $used \gets \textsc{alloc\_size}$
    \Return{$ptr$}
  \EndIf \label{alg:aa--allocate-new-end}
  \Return{$\phi$} \label{alg:aa--allocate-fail}
\EndFunction

\Statex

\State $\rhd$ Free allocated memory
\Function{deallocate}{$ptr$} \textbf{of} FAA
  \State $freed.push(ptr)$ \label{alg:aa--deallocate-push}
\EndFunction

\Statex

\State $\rhd$ Free all allocated memory
\Function{reset}{$ $} \textbf{of} FAA
  \State $freed \gets \{\}$ \label{alg:aa--reset-freedused-begin}
  \State $used \gets 0$ \label{alg:aa--reset-freedused-end}
  \ForAll{$ptr \in pools$} \label{alg:aa--reset-pools-begin}
    \State Free memory at $ptr$
  \EndFor \label{alg:aa--reset-pools-end}
\EndFunction
\end{algorithmic}
\end{algorithm}


As with FAA, the allocator $AA$ is initialized with a predefined allocation size $\textsc{alloc\_size}_a$ and memory pool size $\textsc{pool\_size}_a$ (lines \ref{alg:faa--init-const-begin}-\ref{alg:faa--init-const-end}). Additionally, it maintains a list of reusable freed allocations $freed$, tracks the number of used bytes in the most recent pool $used$, and stores references to all allocated pools in $pools$ (lines \ref{alg:aa--init-begin}-\ref{alg:aa--init-end}). Memory allocation, handled by \texttt{allocate()}, follows a three-step process. \textbf{(1)} If freed allocations are available, they are reused (line \ref{alg:faa--allocate-freed}) to minimize new allocations. \textbf{(2)} If space remains in the last allocated pool, the next block is assigned, and the used counter updates accordingly (lines \ref{alg:faa--allocate-pool-begin}-\ref{alg:faa--allocate-pool-end}). \textbf{(3)} If neither condition holds, a new pool of size $\textsc{pool\_size}$ is allocated. If successful, it is added to $pools$, and allocation proceeds (lines \ref{alg:aa--allocate-new-begin}-\ref{alg:aa--allocate-new-end}); otherwise, the function returns $\phi$ to signal failure (line \ref{alg:aa--allocate-fail}). The \texttt{deallocate()} function, as earlier, allows memory to be freed by pushing freed pointers onto the $freed$ list for reuse (line \ref{alg:aa--deallocate-push}). To reset the allocator, \texttt{reset()} clears the $freed$ list, resets the $used$ counter, and releases all allocated $pools$ back to the standard library (lines \ref{alg:aa--reset-freedused-begin}-\ref{alg:aa--reset-pools-end}).


\subsubsection{Power-of-2 size Arena Allocator (P2AA)}
\label{sec:p2aa}

Unlike the AA allocator, which only supports allocating memory blocks of a fixed size, the Power-of-2 Arena Allocator (P2AA) allows arbitrary allocation sizes, specializing in handling memory allocations that are powers of two --- similar to a slab allocator \cite{bonwick1994slab}. The pseudocode for P2AA allocator is shown in Algorithm \ref{alg:p2aa}, and it builds upon the AA allocator by maintaining separate AA sub-allocators for each power-of-2 size, from $16$ bytes to $8192$ bytes. The $8192$-byte limit is chosen empirically for optimal performance.

\begin{algorithm}[hbtp]
\caption{A Power-of-2 size Arena Allocator (P2AA).}
\label{alg:p2aa}
\begin{algorithmic}[1]
\Require{$\textsc{pool\_size}_a$: Size of each memory pool (constant)}

\Statex

\State \textbf{struct} $P2AA \langle \textsc{pool\_size}_a \rangle ()$
\State \ \ $\textsc{pool\_size} \gets \textsc{pool\_size}_a$ \Comment{Size of each memory pool} \label{alg:p2aa--init-const}
\State \ \ $aa16 \gets AA \langle 16, \textsc{pool\_size} \rangle ()$ \Comment{For $16$-byte allocations} \label{alg:p2aa--init-begin}
\State \ \ $aa32 \gets AA \langle 32, \textsc{pool\_size} \rangle ()$ \Comment{For $32$-byte allocations}
\State \ \ $\cdots$ \Comment{For $64$ to $4096$-byte allocations}
\State \ \ $aa8192 \gets AA \langle 8192, \textsc{pool\_size} \rangle ()$ \Comment{For $8192$-byte allocations} \label{alg:p2aa--init-end}

\Statex

\State $\rhd$ Allocate memory of size that is a power of 2
\Function{allocate}{$size$} \textbf{of} P2AA
  \State \textbf{switch} $(size)$
  \State \ \ \textbf{case} $16$: \ReturnInline{$aa16.allocate()$} \label{alg:p2aa--allocate-pow2-begin}
  \State \ \ \textbf{case} $32$: \ReturnInline{$aa32.allocate()$}
  \State \ \ $\cdots$
  \State \ \ \textbf{case} $8192$: \ReturnInline{$aa8192.allocate()$} \label{alg:p2aa--allocate-pow2-end}
  \State \ \ \textbf{default}: \ReturnInline{Allocate $size$ bytes} \label{alg:p2aa--allocate-other}
\EndFunction

\Statex

\State $\rhd$ Free allocated memory
\Function{deallocate}{$ptr, size$} \textbf{of} P2AA
  \State \textbf{switch} $(size)$
  \State \ \ \textbf{case} $16$: $aa16.deallocate(ptr)$ \label{alg:p2aa--deallocate-pow2-begin}
  \State \ \ \textbf{case} $32$: $aa32.deallocate(ptr)$
  \State \ \ $\cdots$
  \State \ \ \textbf{case} $8192$: $aa8192.deallocate(ptr)$ \label{alg:p2aa--deallocate-pow2-end}
  \State \ \ \textbf{default}: Free memory at $ptr$ \label{alg:p2aa--deallocate-other}
\EndFunction

\Statex

\State $\rhd$ Free all allocated memory
\Function{reset}{$ $} \textbf{of} P2AA
  \State $aa16.reset()$ \label{alg:p2aa--reset-begin}
  \State $aa32.reset()$
  \State $\cdots$
  \State $aa8192.reset()$ \label{alg:p2aa--reset-end}
\EndFunction

\Statex

\State $\rhd$ Get recommended allocation size, for a desired size
\Function{allocationSize}{$size$} \textbf{of} P2AA
  \If{$size \leq 16$} \ReturnInline{$16$} \label{alg:p2aa--allocationsize-16}
  \ElsIf{$size < 8192$} \ReturnInline{$nextPow2(size)$} \label{alg:p2aa--allocationsize-8192}
  \Else\ \ReturnInline{$\lceil size / \textsc{page\_size} \rceil * \textsc{page\_size}$} \label{alg:p2aa--allocationsize-other}
  \EndIf
\EndFunction
\end{algorithmic}
\end{algorithm}


At initialization, the allocator $P2AA$ is configured with a predefined memory pool size $\textsc{pool\_size}_a$ (line \ref{alg:p2aa--init-const}). It creates multiple AA sub-allocators, each responsible for a different power-of-2 allocation size, ranging from $16$ to $8192$ bytes (lines \ref{alg:p2aa--init-begin}-\ref{alg:p2aa--init-end}). Memory allocation is performed in the \texttt{allocate()} function. When a request for a standard power-of-2 size is made (e.g., $16$, $32$, or $8192$ bytes), the request is routed to corresponding AA sub-allocator handles the request (lines \ref{alg:p2aa--allocate-pow2-begin}-\ref{alg:p2aa--allocate-pow2-end}). For sizes larger than $8192$ bytes or non-power-of-2 sizes, a direct memory allocation is performed instead (line \ref{alg:p2aa--allocate-other}). Deallocation mirrors the allocation strategy. Each deallocation request, made through \texttt{deallocate()}, is dispatched to the corresponding sub-allocator, if it is a power-of-two, and ranges from $16$ to $8192$ bytes in size (lines \ref{alg:p2aa--deallocate-pow2-begin}-\ref{alg:p2aa--deallocate-pow2-end}). For other sizes, memory is freed directly using \texttt{delete[]} (line \ref{alg:p2aa--deallocate-other}). Note that, unlike \texttt{delete[]}, the user is expected to provide the size of the allocated memory. To reset all allocated memory, the \texttt{reset()} function (lines \ref{alg:p2aa--reset-begin}-\ref{alg:p2aa--reset-end}) resets each individual arena allocator. This effectively clears all memory pools, making the allocator ready for fresh allocations without explicitly deallocating each individual allocation. Finally, the \texttt{allocationSize()} function provides a mechanism for determining the optimal allocation size for a given request. If the request is $16$ bytes or smaller, it returns $16$ (line \ref{alg:p2aa--allocationsize-16}). If the request is between $17$ and $8191$ bytes, it rounds up to the next power of two using \texttt{nextPow2()} (line \ref{alg:p2aa--allocationsize-8192}). For larger requests, it rounds up to the nearest multiple of the system page size (line \ref{alg:p2aa--allocationsize-other}). This ensures that large memory allocations are page-aligned.


\subsubsection{Concurrent Power-of-2 Arena Allocator (CP2AA)}
\label{sec:cp2aa}

We now discuss our thread-safe Concurrent Power-of-2 Arena Allocator (CP2AA), which builds upon per-thread instances of the Power-of-2 Arena Allocator (P2AA). Its pseudocode is presented in Algorithm \ref{alg:cp2aa}. As above, CP2AA consists of four primary functions: \texttt{allocate()}, \texttt{deallocate()}, \texttt{reset()}, and \texttt{allocationSize()}.

\begin{algorithm}[hbtp]
\caption{Our Concurrent Pow-of-2 Arena Allocator (CP2AA).}
\label{alg:cp2aa}
\begin{algorithmic}[1]
\Require{$\textsc{pool\_size}_a$: Size of each memory pool (constant)}

\Statex

\State \textbf{struct} $CP2AA \langle \textsc{pool\_size}_a \rangle ()$
\State \ \ $\textsc{pool\_size} \gets \textsc{pool\_size}_a$ \Comment{Size of each memory pool} \label{alg:cp2aa--init-const}
\State \ \ $p2aa_T \gets \{P2AA \langle \textsc{pool\_size} \rangle (),\ \dots\}$ \Comment{Per-thread allocator} \label{alg:cp2aa--init}

\Statex

\State $\rhd$ Allocate memory of size that is a power of 2
\Function{allocate}{$size$} \textbf{of} CP2AA
  \State $t \gets$ Current thread \label{alg:cp2aa--allocate-begin}
  \Return{$p2aa_T[t].allocate(size)$} \label{alg:cp2aa--allocate-end}
\EndFunction

\Statex

\State $\rhd$ Free allocated memory
\Function{deallocate}{$ptr, size$} \textbf{of} CP2AA
  \State $t \gets$ Current thread \label{alg:cp2aa--deallocate-begin}
  \State $p2aa_T[t].deallocate(ptr, size)$ \label{alg:cp2aa--deallocate-end}
\EndFunction

\Statex

\State $\rhd$ Free all allocated memory
\Function{reset}{$ $} \textbf{of} CP2AA
  \ForAll{$t \in threads$} \label{alg:cp2aa--reset-begin}
    \State $p2aa_T[t].reset()$
  \EndFor \label{alg:cp2aa--reset-end}
\EndFunction

\Statex

\State $\rhd$ Get recommended allocation size, for a desired size
\Function{allocationSize}{$size$} \textbf{of} CP2AA
  \Return{$P2AA \langle \textsc{pool\_size} \rangle.allocationSize(size)$} \label{alg:cp2aa--allocationsize}
\EndFunction
\end{algorithmic}
\end{algorithm}

% NOTE: Per-thread allocators are well-separated in memory to avoid false sharing.


The $CP2AA$ allocator is initialized with a fixed memory pool size $\textsc{pool\_size}_a$ (line \ref{alg:cp2aa--init-const}), and maintains a separate instance of P2AA per thread, stored in $p2aa_T$ (line \ref{alg:cp2aa--init}) --- while ensuring that each P2AA allocator is well-separated in memory to prevent false sharing. Each thread exclusively interacts with its corresponding P2AA instance, eliminating the need for locks or atomic operations altogether. The \texttt{allocate()} function assigns memory in a thread-local manner. Given a requested size, it identifies the current thread $t$ and delegates the allocation to its corresponding P2AA instance, $p2aa_T[t]$ (lines \ref{alg:cp2aa--allocate-begin}-\ref{alg:cp2aa--allocate-end}). Similarly, \texttt{deallocate()} retrieves the current thread $t$ and forwards the deallocation request to the P2AA allocator of the current thread (lines \ref{alg:cp2aa--deallocate-begin}-\ref{alg:cp2aa--deallocate-end}). Note that it is acceptable for a thread to deallocate memory allocated by another thread. In fact, this lack of restriction is a key source of performance improvement. The \texttt{reset()} function deallocates all of the allocated memory by iterating over all per-thread allocators and calling their respective reset functions (lines \ref{alg:cp2aa--reset-begin}-\ref{alg:cp2aa--reset-end}). Finally, the \texttt{allocationSize()} function determines the appropriate power-of-2 allocation size for a given request by forwarding the query to P2AA (line \ref{alg:cp2aa--allocationsize}).

\ignore{Our initial attempt at a concurrent arena allocator used an atomic\_flag mutex to manage freed blocks, but high contention resulted in slow deallocation, especially with 64 threads.}


\subsubsection{Performance Comparison}

\begin{figure}[hbtp]
  \centering
  \subfigure[Allocation-only workload: $2^{28}$ allocations of 64 bytes.]{
    \label{fig:allocator-alloc--runtime}
    \includegraphics[width=0.98\linewidth]{out/allocator-alloc-runtime.pdf}
  }
  \subfigure[Deallocation-only workload: $2^{28}$ deallocations of allocated memory.]{
    \label{fig:allocator-free--runtime}
    \includegraphics[width=0.98\linewidth]{out/allocator-free-runtime.pdf}
  }
  \subfigure[Mixed workload: $2^{22}$ allocations and $2^{22}$ deallocations, $64$ times.]{
    \label{fig:allocator-mixed--runtime}
    \includegraphics[width=0.98\linewidth]{out/allocator-mixed-runtime.pdf}
  } \\[-2ex]
  \caption{Relative Runtime of memory allocators across three workloads: \textit{(a)} \textit{allocation-only}, where $2^{28}$ allocations of $64$ bytes each are performed; \textit{(b)} \textit{deallocation-only}, with $2^{28}$ deallocations; and \textit{(c)} mixed, with $2^{22}$ allocations followed by $2^{22}$ deallocations, repeated $64$ times. The allocators include the C library allocator ($malloc()/free()$), C++ runtime allocator ($new[]/delete[]$), Fixed Arena Allocator (FAA), variable-capacity Arena Allocator (AA), and Concurrent Power-of-2 Arena Allocator (CP2AA). Details on FAA, AA, and CP2AA allocators is given in Section \ref{sec:memory-allocators}.}
  \label{fig:allocator-runtime}
\end{figure}
% This is actually three separate experiments, which aim the measure the performance of memory allocators in different scenarios/workloads. In the first experiment, we consider an allocation-only workload, where each allocator is stress tested with repeated memory allocations ($2^28$ allocations in total) of each $64$ bytes, and the total time taken to do so is measured. The allocated memory addresses in this case are stored in an array. In the second experiment, we consider an deallocation-only workload, where each allocator is now tested with repeated memory deallocations (of the memory addresses obtained from the first experiment), and the total time taken to do so is measured. Finally, in the third experiment, we consider a mixed workload, which consists of $2^22$ allocations, followed by $2^22$ deallocations, both of which are repeated $64$ times. In all experiments, the performance of C library allocator ($malloc()/free()$), C++ runtime allocators ($new[]/delete[]$), Fixed Arena Allocator (FAA), variable-capacity Arena Allocator (AA), and Concurrent Power-of-2 Arena Allocator (CP2AA). The above figure shows the relative runtime of each of the allocators in each of the three scenarios/experiments.
% Allocation-only workload: Total runtime of $2^{28}$ allocations of 64 bytes for each allocator.
% Deallocation-only workload: Total runtime of $2^{28}$ deallocations, using the addresses from the allocation experiment, for each allocator.
% Mixed workload: Total runtime for $2^{22}$ allocations followed by $2^{22}$ deallocations, repeated 64 times for each allocator.


To evaluate the performance of different memory allocators, we conduct three separate experiments, each designed to measure execution time under specific memory management workloads. The allocators tested include the C library allocator (\texttt{malloc()}/\texttt{free()}), the C++ runtime allocator (\texttt{new[]}/\texttt{delete[]}), the Fixed Arena Allocator (FAA), the variable-capacity Arena Allocator (AA), and the Concurrent Power-of-2 Arena Allocator (CP2AA). In the first experiment, we measure the performance of each allocator in an allocation-heavy scenario. Here, each allocator is subject to a total of $2^{28}$ memory allocations, each of size $64$ bytes. With non-concurrent allocators, i.e., FAA and AA, separate allocator objects are created for each thread to ensure thread safety, and the workload is distributed evenly across all threads. The allocated memory addresses are stored in an array to ensure they can be deallocated in the subsequent experiment. In the second experiment, we evaluate the deallocation performance of each allocator. Using the memory addresses obtained from the first experiment, each allocator is tasked with deallocating $2^{28}$ blocks of memory. Deallocations are routed to the appropriate allocator object for each thread in the case of FAA and AA. The third experiment examines allocator performance in a mixed workload, where allocation and deallocation operations occur in succession. Specifically, each allocator is tested with $2^{22}$ allocations followed by $2^{22}$ deallocations, with this cycle being repeated $64$ times. This experiment provides insight into the performance of each allocator under workloads that mimic real-world application behavior.

Figure \ref{fig:allocator-alloc--runtime} presents the relative runtime of each allocator for the allocation-only workload, while Figure \ref{fig:allocator-free--runtime} shows the deallocation-only workload. Finally, Figure \ref{fig:allocator-mixed--runtime} illustrates the performance of each allocator under the mixed workload. The results indicate that specialized allocators like FAA, AA, and CP2AA offer substantial performance improvements over general-purpose allocators (\texttt{malloc()}/\texttt{new[]}), particularly in allocation-intensive workloads, and offer around $4\times$ speedup in mixed workloads. While the AA allocator performs the best, on average, it is not thread-safe and thus not suitable for concurrent applications. The CP2AA allocator, which is both thread-safe and high-performing, is suitable.



\subsection{Evaluating Graph Representations via Graph Algorithm Performance}

As dicusssed earlier, we evaluate the performance of various graph representations on a representative algorithm by measuring the efficiency of $k$-step reverse walks from each vertex in the given input graph, and counting the number of walks ending at each vertex. This corresponds to computing $A_T^k \cdot \vec{1}$, where $A_T$ is the transposed adjacency matrix and $\vec{1}$ is a ones vector. Reverse walks are preferred as they can be executed directly on the input graph, whereas forward walks require its transpose. The results of this evaluation on each graph representation, i.e., PetGraph, SuiteSparse:GraphBLAS, cuGraph, Aspen, and our DiGraph, are discussed in Section \ref{sec:perform-reverse-walks}.

We now give a short description of the algorithm. It propagates visit counts backward along graph edges, and its psuedocode is given in Algorithm \ref{alg:visit}. Here, the \texttt{reverseWalk()} function, takes a graph $G$ and the number of reverse walk $steps$ as input. Initially, two arrays, $visits0$ and $visits1$, are allocated to track visit counts, with $visits0$ set to $1$ for all vertices and $visits1$ initialized to $0$ (lines \ref{alg:visit--init-begin}-\ref{alg:visit--init-end}). The main loop iterates $steps$ times, processing all vertices $u$ in parallel. Each iteration resets $visits1[u]$ (line \ref{alg:visit--reset}) and accumulates the visit count from each neighbor $v$ of $u$ (lines \ref{alg:visit--edges-begin}-\ref{alg:visit--edges-end}). This iterative process propagates visit counts backward through the graph. At the end of each iteration, the visit count arrays are swapped (line \ref{alg:visit--swap}), ensuring $visits0$ contains the latest counts. After the final iteration, $visits0$ contains the number of reverse walks ending at each vertex\ignore{in the graph, and is returned}.

\begin{algorithm}[hbtp]
\caption{Perform reverse walk from each vertex.}
\label{alg:visit}
\begin{algorithmic}[1]
\Require{$G(V, E)$: Input graph}
\Ensure{$steps$: Number of reverse walks to perform}

\Statex

\Function{reverseWalk}{$G, steps$}
  \State $visits0 \gets \{1\}$ \label{alg:visit--init-begin}
  \State $visits1 \gets \{0\}$ \label{alg:visit--init-end}
  \ForAll{$i \in [0, steps)$} \label{alg:visit--steps-begin}
    \ForAll{$u \in V$ \textbf{in parallel}}
      \State $visits1[u] \gets 0$ \label{alg:visit--reset}
      \ForAll{$(v, \_) \in G.edges(u)$} \label{alg:visit--edges-begin}
        \State $visits[u] \gets visits[u] + visits0[v]$
      \EndFor \label{alg:visit--edges-end}
    \EndFor
    \State $swap(visits0, visits1)$ \label{alg:visit--swap}
  \EndFor \label{alg:visit--steps-end}
  \Return{$visits0$}
\EndFunction
\end{algorithmic}
\end{algorithm}


\end{document}
\endinput
%% End of file.




% Time taken to read and clone the graph
% Time taken to insert and delete edges (both in-place and copy-based)
% Why is clone slow?
% Present graph traversal performance with PageRank?
% We optimize for traversal performance
% Per-graph runtimes for each batch size
% Design of the AA allocator

% Already have 2 results, need 2 more to compare with
% I AM WAITING FOR PETGRAPH-MAP TO COMPLETE
% THEN I CAN RUN SNAP-READ BASED
% Also waiting for Snap (convert based), and Petgraph original

% OTHER COMPARISONS
% NetworkX is too slow (python)
% Gunrock is static only
% Ligra is static only
% Comparison with graph databases?


% WRITING IDEAS
% - Trying to optimize graph load, copy, and application of batch update
% - To optimize load, can integrate with CSR loader
% - Can start writing to get top-down view
% - Mention points from various sources
% - The various optimization steps

% PAPER STRUCTURE
% - The problem we are addressing
% - Why is it a problem - its description
% - Usefulness of the problem - solution - applications
% - The work that has been done
% - Critical review - any quick demo of the issues
% - Stating the problem clearly is very important
% - Proceed onto solution

% OTHERS
% - Update existing papers (on arxiv) - clear problem presentation


% NEW DETAILS TO INCLUDE
% 1. Discuss details from allocator issue.
% 2. Discuss details from graph-openmp issue.
% 3. Also discuss details from GVEL.
% 4. Present framework diagram (motivating).
% 5. Discuss how allocation overhead is important.
% 6. And how perf. of algorithms is important.
% 7. Present how memory allocation is a bottleneck wrt. earlier implementation.
% 8. Present a figure on graph union and subtraction for batch update.
% 9. Discuss key details which help improve performance.
% 10. Discuss that allocator is shared.
