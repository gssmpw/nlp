\section{Additional Experiments}

\iffalse

\subsection{Supplementary Figures for \cref{sec:experiment}}

\cref{fig:heatmap-modality-full} shows exaustive test on every contrasitive pair of modal words.
Additionally, modality’s effect on the next-token probabilities of answer token \texttt{No} is shown in \cref{fig:heatmap-modality-no}.

\cref{fig:corr-ppl-acc-full} extends \cref{fig:corr-ppl-acc} to correlation between perplexity and accuracy for all models.

\begin{figure*}
    \centering
    \begin{subfigure}[b]{0.48\textwidth}
        \centering
        \includegraphics[width=1\textwidth,keepaspectratio]{heatmap-modality-full-yes.pdf}
        \caption{
            On token \texttt{Yes}.
        }
    \end{subfigure}
    \begin{subfigure}[b]{0.48\textwidth}
        \centering
        \includegraphics[width=1\textwidth,keepaspectratio]{heatmap-modality-full-no.pdf}
        \caption{
            On token \texttt{No}.
        }
        \label{fig:heatmap-modality-no}
    \end{subfigure}
    \caption{
        \textbf{Continuation of \cref{fig:heatmap-modality}}.
    }
    \label{fig:heatmap-modality-full}
\end{figure*}

\begin{figure*}
    \centering
    \includegraphics[width=1\textwidth,keepaspectratio]{corr-ppl-acc-full.pdf}
    \caption{
        \textbf{Continuation of \cref{fig:corr-ppl-acc}}.
    }
    \label{fig:corr-ppl-acc-full}
\end{figure*}

\fi


\subsection{Extra Experiment: Introduction Rule of Modality}
\label{subsec:extra-intro-modality}

We report the results on the necessitation rule and its variants here, as these rules are obscure and verbose to be articulated in natural language:
%
\begin{align}
    \left\{\varphi\right\} &\vdash \Box \varphi, \tag{necessitation rule} \label{eqn:necessitation-rule} \\
    \left\{\varphi\right\} &\vdash \Diamond \varphi, \nonumber \\
    \left\{\varphi\right\} &\vdash \varphi. \nonumber
\end{align}
%
Its natural language form is as follows:
\begin{table}[H]
    \small
    \begin{tabular}{rp{0.8\linewidth}}
    & Jane is watching a show.\\
    ($\Box$) & Can we infer that it's certain that Jane is watching a show?\\
    ($\Diamond$) & Can we infer that it's possible that Jane is watching a show?\\
    ($\varnothing$) & Can we infer that Jane is watching a show?\\
    \end{tabular}
\end{table}

All three variants are paired with 1000 logic interpretations.
As they are all rules of inference, the ground truth answer is always \texttt{Yes}.
Overall accuracy is shown in \cref{tab:softacc-necessitate},
where across all LLMs, the necessitation rule has the lowest accuracy.
This echoes the necessity modality's tendency to be rejected discussed in \cref{subsec:affirmation-bias}.

\begin{table}[t]
    
\small
\centering
\begin{tabular*}{0.85\columnwidth}{
 @{\extracolsep{\fill}}lccc@{}
 }
 \toprule
  & $\varnothing$ & $\Box$ & $\Diamond$ \\
 \midrule
    mistral-7b & 0.998 & 0.885 & 0.999\\
    mistral-8x7b & 0.957 & 0.540 & 0.987\\
    llama-2-7b & 0.768 & 0.013 & 0.920\\
    llama-2-13b & 0.368 & 0.004 & 0.829\\
    llama-2-70b & 0.511 & 0.051 & 0.834\\
    llama-3-8b & 0.398 & 0.225 & 0.783\\
    llama-3-70b & 0.674 & 0.384 & 0.794\\
    yi-34b & 0.960 & 0.382 & 0.999\\
    phi-2 & 0.814 & 0.226 & 0.892\\
    phi-3-mini & 0.992 & 0.925 & 0.994\\
 \bottomrule
\end{tabular*}
    \caption{
        \label{tab:softacc-necessitate}
        Overall accuracy of the necessitation rule and its modality variants on each model.
    }
\end{table}
    
We further fit a linear mixed-effects model similar to \cref{eqn: mixed-effects}, except that the argument form effect is now constant across all data points.
The mixed-effects model yields a marginal $R^2$ of $0.391$ and a conditional $R^2$ of $0.745$.
Estimated marginal means shows that the accuracy on $\varnothing$ is $0.171$ less than $\Diamond$, but $0.371$ higher than $\Box$, with both differences significant at $p < 0.0001$.
This further suggests that modality serves as an important factor on logic reasoning performance.

\subsection{Extra Experiment: Distribution of Modalities}

Besides the necessitation rule, \textit{distribution axiom} is the other fundamental axiom in normal modal logic.
It can be transformed into the rule shown in \cref{eqn:distribution-must-axiom},
and plugging in the definition of $\lor$ in \cref{eqn:def-lor} gives the rule shown in \cref{eqn:distribution-or-theorem}.
Notice that \cref{eqn:distribution-or-theorem} closely resembles rule \ref{eqn:inf-rule-or-left}'s variant with necessity, as shown in \cref{eqn:inf-rule-or-left-must},
except the different scope of the necessity operator and the position of the negation operator.
Moving the negation operator out of the necessity operator will result in a fallacy (Eq. \ref{eqn:distribution-or-fallacy}).
\noindent
\begin{align}
    \left\{\Box(\varphi \to \psi), \Box \varphi\right\} &\vdash \Box \psi, 
    \label{eqn:distribution-must-axiom} \\
    \left\{\Box(\varphi \lor \psi), \Box \lnot \varphi\right\} &\vdash \Box \psi \label{eqn:distribution-or-theorem}, \\
    \left\{\Box \varphi \lor \Box \psi, \lnot \Box \varphi\right\} &\vdash \Box \psi \label{eqn:inf-rule-or-left-must}, \\
    \left\{\Box(\varphi \lor \psi), \lnot \Box \varphi\right\} &\nvdash \Box \psi \label{eqn:distribution-or-fallacy}.
\end{align}
\noindent
We say \cref{eqn:distribution-or-theorem,eqn:inf-rule-or-left-must,eqn:distribution-or-fallacy} are of argument form \texttt{theorem}, \texttt{base} and \texttt{spurious}, respectively.
See \cref{tab:extra-distribution-forms} for the logical forms and their ground truth we used to study the distribution of modalities.
The natural language form is as follows:
\begin{table}[H]
    \small
    \begin{tabular}{rp{0.65\linewidth}}
    & It's certain that if Freddy is not going shopping, then Coy is making dinner.\\
    (\texttt{theorem}) & It's certain that Freddy is not going shopping.\\
    (\texttt{spurious}) & It's uncertain whether Freddy is going shopping.\\
    & Can we infer that it's certain that Coy is making dinner?\\
    \end{tabular}
\end{table}

This group of rules and fallacies comes from the fact that the necessity modality $\Box$ is not distributive to disjunction, i.e. $\Box (\varphi \lor \psi) \nvdash \Box \varphi \lor \Box \psi$ \citep[Ex. 5]{xiang-2019a-two-types}.
In contrast, the possibility modality $\Diamond$ is distributive to disjunction.
This particular case could have served as a material to test the LLM's knowledge of the asymmetry between the two modalities,
yet in \cref{subsec:affirmation-bias} we showed that there is a bias towards rejection on the necessity modality.
As the false case of the disjunction is on the necessity modality, this bias confounds the experiment.


\begin{table}[t]
    \small\centering
\begin{tabular}{ccl}
    \toprule
Modality & Argument Form & Logical Form \\
    \midrule
$\varnothing$ & base & $\varphi \lor \psi, \lnot \varphi \vdash \psi$ \\
$\Box$ & base & $\Box \varphi \lor \Box \psi, \lnot \Box \varphi \vdash \Box \psi$ \\
$\Box$ & theorem & $\Box ( \varphi \lor \psi), \Box \lnot \varphi \vdash \Box \psi$ \\
$\Box$ & \uline{spurious} & $\Box ( \varphi \lor \psi), \lnot \Box \varphi \nvdash \Box \psi$ \\
$\Diamond$ & base & $\Diamond \varphi \lor \Diamond \psi, \lnot \Diamond \varphi \vdash \Diamond \psi$ \\
$\Diamond$ & theorem & $\Diamond ( \varphi \lor \psi), \Diamond \lnot \varphi \vdash \Diamond \psi$ \\
$\Diamond$ & spurious & $\Diamond ( \varphi \lor \psi), \lnot \Diamond \varphi \vdash \Diamond \psi$ \\
% \cmidrule{1-3}
% $\varnothing$ & $\lnot \varphi \to \psi, \lnot \varphi \vdash \psi$ \\
% $\Box$ & $\Box ( \lnot \varphi \to \psi), \Box \lnot \varphi \vdash \Box \psi$ \\
% $\Box$ & False & $\Box ( \lnot \varphi \to \psi), \lnot \Box \varphi \nvdash \Box \psi$ \\
% $\Diamond$ & $\Diamond ( \lnot \varphi \to \psi), \Diamond \lnot \varphi \vdash \Diamond \psi$ \\
% $\Diamond$ & $\Diamond ( \lnot \varphi \to \psi), \lnot \Diamond \varphi \vdash \Diamond \psi$ \\
    \bottomrule
    \end{tabular}
    \caption{
        \label{tab:extra-distribution-forms}
        Logical forms and their ground truth to study the distribution of modalities.
        Only the spurious form of the necessity modality (marked by \uline{underline}) has a ground truth of false.
    }
\end{table}

We fit a linear mixed-effects model similar to \cref{eqn: mixed-effects} to the data,
\noindent
\begin{align}
    \mathit{Acc}_\textit{soft} & \sim \textit{Modality} \times \textit{ArgForm} + \textit{Perplexity} \nonumber \\
                               & + (1 + \textit{Perplexity} \mid \textit{LLM}), \nonumber
\end{align}
\noindent
with an interaction term between the modality and argument form.
% The model yields a conditional $R^2$ of $0.397$.
On the \texttt{theorem} form compared to the \texttt{base} form, the necessity modality $\Box$ has a $0.173$ higher estimated marginal means with $p < 0.0001$ significance, yet the possibility modality $\Diamond$ has a $0.071$ lower estimated marginal means.
On the \texttt{spurious} form compared to the \texttt{base} form, the $\Box$ has a $0.312$ higher means, and the $\Diamond$ has no significant difference.
On both forms, $\Diamond \succ \Box$ in terms of accuracy still holds at a slight margin of $0.110$ and $0.047$ respectively.

To verify whether on $\Box$ the performance increase on \texttt{spurious} form is due to the rejection bias, we fit a linear mixed-effects model with the relative probability of answering \texttt{Yes} as dependent variable.
Results show that on \texttt{spurious} form compared to the \texttt{base} form, the effect of $\Box$'s tendency to answer \texttt{Yes} is only $0.060$ lower, indicating the rejection bias of the \texttt{base} form is still present.
Therefore, we hypothesize that the LLM's performance on recognizing the fallacy of necessity distribution over disjunction is hindered by the rejection bias on the necessity modality.