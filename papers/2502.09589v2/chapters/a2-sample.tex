\setcounter{table}{0}
\setcounter{figure}{0}
\setcounter{equation}{0}
\renewcommand{\thetable}{A\arabic{table}}
\renewcommand{\thefigure}{A\arabic{figure}}
\renewcommand{\theequation}{A\arabic{equation}}

\section{Extra Details of the Dataset}

\subsection{Considerations in Translating Logical Form to Natural Language}
\label{subsec:logic-translate-strategy}

During the interpretation process, another key point is to assign independent interpretations to variables.
Deciding the dependency also involves common sense knowledge.
For example, consider the premises $\lnot p \to q$ and $q$.
If we interpret $p \coloneqq\textit{``Jane is inside the house''}$ and $q\coloneqq\textit{``Jane is out''}$ to proposition variables $p$ and $q$, the two variables are possibly not independent.
According to common sense, ``\textit{Jane is not inside the house}'' ($\lnot p$) correlates with or is even equivalent to ``\textit{Jane is out}'' ($q$).
Logically, $\left\{\lnot p \to q, q\right\} \nvdash \lnot p$; however, with the extra premise $\lnot p \leftrightarrow q$ given by common sense, people may conclude that $\lnot p$.\footnote{
  This confounding factor affects the examples in Appendix C.1.12 of \citet{hollidayConditionalModalReasoning2024}.
}

Besides, natural language is ambiguous---one sentence in natural language can come from multiple logical forms under the same interpretation.
We use present tense and progressive aspect to encourage a reading of imaginary ongoing events, corresponding to the alethic modality.
Such events are less likely to induce LLM's or human's individual bias, as they are unrelated to factual knowledge or moral judgements.
Also, we always use two full verb phrases, ruling out sentences like ``\textit{Jane is eating apples or oranges,}'' so the two events are less likely to be mutually exclusive.
In this way, we can reduce the ambiguity of the questions in our dataset.

\subsection{Data Samples}

All logic forms and corresponding natural language sentences can be found in \cref{tab:question-full}.

The exact prompt format is as follows:

\begin{table}[H]
    \small
    \begin{tabular}{p{\linewidth}}
    Consider the following statements:\verb|\n|\\
    \uline{Jane is watching a show or John is reading a book.}\verb|\n|\\
    \uline{Jane isn't watching a show}.\verb|\n|\\
    Question: Based on these statements, can we infer that \uline{John is reading a book}?\verb|\n|\\
    Answer:\verb|<eof|>
    \end{tabular}
\end{table}
\vspace{-1em}

\newcommand{\subjectA}{\uline{Jane}}
\newcommand{\vpA}{\uline{watching a show}} 
\newcommand{\subjectB}{\uline{John}}
\newcommand{\vpB}{\uline{reading a book}}

\begin{table*}
\scriptsize
\begin{tabular}{
  @{}ccclp{0.52\textwidth}@{}
  }
\vspace{0.5em}
\textbf{Validity} & \textbf{Modality} & \textbf{Argument Form} & \textbf{Logical Form} & \textbf{Natural Language} \\
$\vdash$
& $\varnothing$ & $\lor^{\mathrm{L}}$ & $\{p \lor q, \lnot p\} \vdash q$ & 
   \subjectA{} is \vpA{} or \subjectB{} is \vpB{}.\newline
  \subjectA{} isn't \vpA{}.\newline
  Can we infer that \subjectB{} is \vpB{}? \\
& $\varnothing$ & $\lor^{\mathrm{R}}$ & $\{p \lor q, \lnot q\} \vdash p$ & 
   \subjectA{} is \vpA{} or \subjectB{} is \vpB{}.\newline
  \subjectB{} isn't \vpB{}.\newline
  Can we infer that \subjectA{} is \vpA{}? \\
& $\varnothing$ & $\rightarrow^{\mathrm{L}}$ & $\{\lnot p \to q, \lnot p\} \vdash q$ & 
   If \subjectA{} isn't \vpA{}, then \subjectB{} is \vpB{}.\newline
  \subjectA{} isn't \vpA{}.\newline
  Can we infer that \subjectB{} is \vpB{}? \\
& $\varnothing$ & $\rightarrow^{\mathrm{R}}$ & $\{\lnot p \to q, \lnot q\} \vdash p$ & 
   If \subjectA{} isn't \vpA{}, then \subjectB{} is \vpB{}.\newline
  \subjectB{} isn't \vpB{}.\newline
  Can we infer that \subjectA{} is \vpA{}? \\
& $\Box$ & $\lor^{\mathrm{L}}$ & $\{\Box p \lor \Box q, \lnot \Box p\} \vdash \Box q$ & 
   It's certain that \subjectA{} is \vpA{} or it's certain that \subjectB{} is \vpB{}.\newline
  It's uncertain whether \subjectA{} is \vpA{}.\newline
  Can we infer that it's certain that \subjectB{} is \vpB{}? \\
& $\Box$ & $\lor^{\mathrm{R}}$ & $\{\Box p \lor \Box q, \lnot \Box q\} \vdash \Box p$ & 
   It's certain that \subjectA{} is \vpA{} or it's certain that \subjectB{} is \vpB{}.\newline
  It's uncertain whether \subjectB{} is \vpB{}.\newline
  Can we infer that it's certain that \subjectA{} is \vpA{}? \\
& $\Box$ & $\rightarrow^{\mathrm{L}}$ & $\{\lnot \Box p \to \Box q, \lnot \Box p\} \vdash \Box q$ & 
   If it's uncertain whether \subjectA{} is \vpA{}, then it's certain that \subjectB{} is \vpB{}.\newline
  It's uncertain whether \subjectA{} is \vpA{}.\newline
  Can we infer that it's certain that \subjectB{} is \vpB{}? \\
& $\Box$ & $\rightarrow^{\mathrm{R}}$ & $\{\lnot \Box p \to \Box q, \lnot \Box q\} \vdash \Box p$ & 
   If it's uncertain whether \subjectA{} is \vpA{}, then it's certain that \subjectB{} is \vpB{}.\newline
  It's uncertain whether \subjectB{} is \vpB{}.\newline
  Can we infer that it's certain that \subjectA{} is \vpA{}? \\
& $\Diamond$ & $\lor^{\mathrm{L}}$ & $\{\Diamond p \lor \Diamond q, \lnot \Diamond p\} \vdash \Diamond q$ & 
   It's possible that \subjectA{} is \vpA{} or it's possible that \subjectB{} is \vpB{}.\newline
  It's impossible that \subjectA{} is \vpA{}.\newline
  Can we infer that it's possible that \subjectB{} is \vpB{}? \\
& $\Diamond$ & $\lor^{\mathrm{R}}$ & $\{\Diamond p \lor \Diamond q, \lnot \Diamond q\} \vdash \Diamond p$ & 
   It's possible that \subjectA{} is \vpA{} or it's possible that \subjectB{} is \vpB{}.\newline
  It's impossible that \subjectB{} is \vpB{}.\newline
  Can we infer that it's possible that \subjectA{} is \vpA{}? \\
& $\Diamond$ & $\rightarrow^{\mathrm{L}}$ & $\{\lnot \Diamond p \to \Diamond q, \lnot \Diamond p\} \vdash \Diamond q$ & 
   If it's impossible that \subjectA{} is \vpA{}, then it's possible that \subjectB{} is \vpB{}.\newline
  It's impossible that \subjectA{} is \vpA{}.\newline
  Can we infer that it's possible that \subjectB{} is \vpB{}? \\
& $\Diamond$ & $\rightarrow^{\mathrm{R}}$ & $\{\lnot \Diamond p \to \Diamond q, \lnot \Diamond q\} \vdash \Diamond p$ & 
   If it's impossible that \subjectA{} is \vpA{}, then it's possible that \subjectB{} is \vpB{}.\newline
  It's impossible that \subjectB{} is \vpB{}.\newline
  Can we infer that it's possible that \subjectA{} is \vpA{}? \\

$\nvdash$
& $\varnothing$ & $\lor^{\mathrm{L}}$ & $\{p \lor q, q\} \nvdash \lnot p$ & 
   \subjectA{} is \vpA{} or \subjectB{} is \vpB{}.\newline
  \subjectB{} is \vpB{}.\newline
  Can we infer that \subjectA{} isn't \vpA{}? \\
& $\varnothing$ & $\lor^{\mathrm{R}}$ & $\{p \lor q, p\} \nvdash \lnot q$ & 
   \subjectA{} is \vpA{} or \subjectB{} is \vpB{}.\newline
  \subjectA{} is \vpA{}.\newline
  Can we infer that \subjectB{} isn't \vpB{}? \\
& $\varnothing$ & $\rightarrow^{\mathrm{L}}$ & $\{\lnot p \to q, q\} \nvdash \lnot p$ & 
   If \subjectA{} isn't \vpA{}, then \subjectB{} is \vpB{}.\newline
  \subjectB{} is \vpB{}.\newline
  Can we infer that \subjectA{} isn't \vpA{}? \\
& $\varnothing$ & $\rightarrow^{\mathrm{R}}$ & $\{\lnot p \to q, p\} \nvdash \lnot q$ & 
   If \subjectA{} isn't \vpA{}, then \subjectB{} is \vpB{}.\newline
  \subjectA{} is \vpA{}.\newline
  Can we infer that \subjectB{} isn't \vpB{}? \\
& $\Box$ & $\lor^{\mathrm{L}}$ & $\{\Box p \lor \Box q, \Box q\} \nvdash \lnot \Box p$ & 
   It's certain that \subjectA{} is \vpA{} or it's certain that \subjectB{} is \vpB{}.\newline
  It's certain that \subjectB{} is \vpB{}.\newline
  Can we infer that it's uncertain whether \subjectA{} is \vpA{}? \\
& $\Box$ & $\lor^{\mathrm{R}}$ & $\{\Box p \lor \Box q, \Box p\} \nvdash \lnot \Box q$ & 
   It's certain that \subjectA{} is \vpA{} or it's certain that \subjectB{} is \vpB{}.\newline
  It's certain that \subjectA{} is \vpA{}.\newline
  Can we infer that it's uncertain whether \subjectB{} is \vpB{}? \\
& $\Box$ & $\rightarrow^{\mathrm{L}}$ & $\{\lnot \Box p \to \Box q, \Box q\} \nvdash \lnot \Box p$ & 
   If it's uncertain whether \subjectA{} is \vpA{}, then it's certain that \subjectB{} is \vpB{}.\newline
  It's certain that \subjectB{} is \vpB{}.\newline
  Can we infer that it's uncertain whether \subjectA{} is \vpA{}? \\
& $\Box$ & $\rightarrow^{\mathrm{R}}$ & $\{\lnot \Box p \to \Box q, \Box p\} \nvdash \lnot \Box q$ & 
   If it's uncertain whether \subjectA{} is \vpA{}, then it's certain that \subjectB{} is \vpB{}.\newline
  It's certain that \subjectA{} is \vpA{}.\newline
  Can we infer that it's uncertain whether \subjectB{} is \vpB{}? \\
& $\Diamond$ & $\lor^{\mathrm{L}}$ & $\{\Diamond p \lor \Diamond q, \Diamond q\} \nvdash \lnot \Diamond p$ & 
   It's possible that \subjectA{} is \vpA{} or it's possible that \subjectB{} is \vpB{}.\newline
  It's possible that \subjectB{} is \vpB{}.\newline
  Can we infer that it's impossible that \subjectA{} is \vpA{}? \\
& $\Diamond$ & $\lor^{\mathrm{R}}$ & $\{\Diamond p \lor \Diamond q, \Diamond p\} \nvdash \lnot \Diamond q$ & 
   It's possible that \subjectA{} is \vpA{} or it's possible that \subjectB{} is \vpB{}.\newline
  It's possible that \subjectA{} is \vpA{}.\newline
  Can we infer that it's impossible that \subjectB{} is \vpB{}? \\
& $\Diamond$ & $\rightarrow^{\mathrm{L}}$ & $\{\lnot \Diamond p \to \Diamond q, \Diamond q\} \nvdash \lnot \Diamond p$ & 
   If it's impossible that \subjectA{} is \vpA{}, then it's possible that \subjectB{} is \vpB{}.\newline
  It's possible that \subjectB{} is \vpB{}.\newline
  Can we infer that it's impossible that \subjectA{} is \vpA{}? \\
& $\Diamond$ & $\rightarrow^{\mathrm{R}}$ & $\{\lnot \Diamond p \to \Diamond q, \Diamond p\} \nvdash \lnot \Diamond q$ & 
   If it's impossible that \subjectA{} is \vpA{}, then it's possible that \subjectB{} is \vpB{}.\newline
  It's possible that \subjectA{} is \vpA{}.\newline
  Can we infer that it's impossible that \subjectB{} is \vpB{}? \\
\end{tabular}
\caption{\textbf{Samples of all logical forms and corresponding natural language sentences.}}
\label{tab:question-full}
\end{table*}
