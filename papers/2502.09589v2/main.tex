\pdfoutput=1
\documentclass[11pt]{article}
\newcommand{\thought}[1]{{\color[rgb]{0.2,0.39,0.66}(#1)}}
\newcommand{\todo}[1]{{\color[rgb]{1.0,0.0,0.0}(#1)}}
\newcommand{\hsh}[1]{{\color{green!50!black} Henrik: #1}}
\newcommand{\st}[1]{{\color{red!50!black} Sebastian: #1}}

\newcommand{\ulm}[1]{_{\scaleto{\mathrm{#1}}{3pt}}}
\newcommand\at[2]{\left.#1\right|_{#2}}











\newtheorem{assumption}{Assumption}

\DeclareMathOperator*{\argmax}{arg\,max}
\DeclareMathOperator*{\argmin}{arg\,min}

\newcommand{\swname}[1]{\texttt{#1}}
\newcommand{\ie}{i\/.\/e\/.,\/~}
\newcommand{\eg}{e\/.\/g\/.,\/~}
\newcommand{\cf}{cf\/.\/~}

\newcommand{\fig}{Fig\/.\/~}
\newcommand{\defn}{Def\/.\/~}
\newcommand{\sect}{Sec\/.\/~}
\newcommand{\tabl}{Tab\/.\/~}
\newcommand{\algo}{Algorithm~}
\newcommand{\theo}{Theorem~}

\newcommand{\bnnl}{3 hidden layers}
\newcommand{\bnnn}{50 neurons}
\newcommand{\bnna}{tanh activations}

\newcommand{\capt}[1]{\mdseries{\emph{#1}}}

\newcommand{\videolink}{at \url{https://youtu.be/_d7AqTRjz6g}}
\newcommand{\codelink}{\url{https://github.com/wheelbot/mini-wheelbot}}

\newcommand{\fakepar}[1]{\vspace{0mm}\noindent\textbf{#1.}}

\newcommand{\needref}{\textcolor{red}{[REF]}}

\newcommand{\plotfontsize}{9pt}

%siunitx
\DeclareSIUnit \decibelA {dB(A)}
\DeclareSIUnit \decibelC {dB(C)}
\DeclareSIUnit \soneGF {soneGF}
\DeclareSIUnit \acum {acum}
\DeclareSIUnit \asper {asper}
\DeclareSIUnit \vacil {vacil}
\DeclareSIUnit \tuhms {tuHMS}

%xcolor
\definecolor{set1_1}{RGB}{228,26,28}
\definecolor{set1_2}{RGB}{55,126,184}
\definecolor{set1_3}{RGB}{77,175,74}
\definecolor{den_1}{RGB}{239,209,0}
\definecolor{den_2}{RGB}{78,184,123}
\definecolor{den_3}{RGB}{0,127,196}
\definecolor{loaclr}{RGB}{152, 78, 163}
\definecolor{mclr}{RGB}{255, 127, 0}

%placeholder text
\newcommand{\subtop}[1]{\noindent \textcolor{magenta}{[subtopic: #1]}\\}
\newcommand{\highlightlist}[1]{\vspace{-1.5\baselineskip}\textcolor{magenta}{#1}}
\newcommand{\highlight}[1]{\textcolor{magenta}{#1}}
\newcommand{\bhan}[1]{\textcolor{cyan}{#1}}
\newcommand{\dba}[1]{\SI{#1}{\decibelA}}
\newcommand{\dbc}[1]{\SI{#1}{\decibelC}}
\newcommand{\red}[1]{\textcolor{red}{#1}}

% Custom checkbox command

\newcommand{\CheckedBox}[1]{%
  \ifnum#1=1
    \makebox[0pt][l]{\raisebox{0.15ex}{\hspace{0.2em}$\checkmark$}}%
  \fi
  $\square$%
}
\newcommand{\checkbox}[1]{\item[\CheckedBox{0}] #1}
\newcommand{\checkedbox}[1]{\item[\CheckedBox{1}] #1}

\newcommand{\Tx}[1]{$T_\text{#1}$}
\newcommand{\Lx}[1]{$L_\text{#1}$}
\newcommand{\Lh}[2]{$L_\text{#1,\SI{#2}{\hour}}$}
\newcommand{\Lmin}[2]{$L_\text{#1,\SI{#2}{\minute}}$}
\newcommand{\Ls}[2]{$L_\text{#1,\SI{#2}{\second}}$}
\newcommand{\Lle}[2]{$L_\text{#1}\le\dba{#2}$}
\newcommand{\Lhle}[3]{$L_\text{#1,\SI{#2}{\hour}}\le\dba{#3}$}
\newcommand{\Lminle}[3]{$L_\text{#1,\SI{#2}{\min}}\le\dba{#3}$}
\newcommand{\dLhge}[4]{$\mu_{\Delta L_\text{#1,\SI{#2}{\hour}}} \in [{#3},{#4}] \;\dba{}$}
\newcommand{\dLhgeC}[4]{$\mu_{\Delta L_\text{#1,\SI{#2}{\hour}}} \in [{#3},{#4}] \;\dbc{}$}
\newcommand{\dLhC}[3]{$\mu_{\Delta L_\text{#1,\SI{#2}{\hour}}}=\;\dbc{#3}$}

\newcommand{\binL}{$m_\text{L}$}
\newcommand{\binR}{$m_\text{R}$}
\newcommand{\GRASIn}{$m_\text{in}$}
\newcommand{\GRASOut}{$m_\text{out}$}
\newcommand{\HDbinL}{$m_\text{L}^\text{HD}$}
\newcommand{\HDbinR}{$m_\text{R}^\text{HD}$}
\newcommand{\HDGRAS}{$m_\text{out}^\text{HD}$}
\newcommand{\NICUbinL}{$m_\text{L}^\text{NICU}$}
\newcommand{\NICUbinR}{$m_\text{R}^\text{NICU}$}
\newcommand{\NICUGRASOut}{$m_\text{out}^\text{NICU}$}
\newcommand{\NICUGRASIn}{$m_\text{in}^\text{NICU}$}

\newcommand{\absdiff}[2]{$\mu^{\Delta m_\text{#1,#2}}_{L_{z}}$}

\newcommand{\mudiffL}[5]{$\boldsymbol{\mu}^{\Delta m_\text{#1,#2}}_{L_\text{#3}} \in [#4,#5]$}
\newcommand{\mudiffLo}[4]{$\mu^{\Delta m_\text{#1,#2}}_{L_\text{#3}} = #4$}

\newcommand{\lmeart}{LME-ART-ANOVA}

\title{
    Logical forms complement probability \\
    in understanding language model (and human) performance
    % The cases of propositional and epistemic modal logic
}

\author{Yixuan Wang \\
  University of Chicago \\
  \texttt{yixuanwang@uchicago.edu} \\\And
  Freda Shi \\
  University of Waterloo \\
  Vector Institute, Canada CIFAR AI Chair \\
  \texttt{fhs@uwaterloo.ca} \\}

\begin{document}

\setlength{\Exlabelsep}{0em}
\setlength{\SubExleftmargin}{1em}

\maketitle
\begin{abstract}
  With the increasing interest in using large language models (LLMs) for planning in natural language, understanding their behaviors becomes an important research question.
  This work conducts a systematic investigation of LLMs' ability to perform logical reasoning in \textit{natural language}.
  We introduce a controlled dataset of hypothetical and disjunctive syllogisms in propositional and modal logic and use it as the testbed for understanding LLM performance.
  Our results lead to novel insights in predicting LLM behaviors: in addition to the probability of input \citep{gonen-etal-2023-demystifying,mccoyEmbersAutoregressionShow2024}, logical forms should be considered as important factors.
  In addition, we show similarities and discrepancies between the logical reasoning performances of humans and LLMs by collecting and comparing behavioral data from both.
\end{abstract}

%!TEX root=main.tex

\section{Introduction}
% Decision-makers, analysts, data scientists, and policymakers frequently rely on data to draw conclusions and extract insights. This data-driven approach helps them identify actionable recommendations aimed at influencing an outcome of interest, such as increasing product satisfaction or income levels or decreasing the likelihood of experiencing serious health conditions \cite{galhotra2022hyper,lakkaraju2016interpretable,agrawal1994fast}. 
\revc{Prescriptions, or actionable recommendations, are commonly generated across various fields to influence key outcomes such as improving product satisfaction, enhancing economic policies, or increasing business efficiency. 
%Decision- or policy-makers, analysts, data scientists, and 
Policymakers in government, decision-makers in businesses, and data scientists in various fields, often rely on data-driven approaches to identify 
%actionable recommendations 
potential actions to influence an outcome of interest, such as increasing income levels or loan approval rates}.
% , or decreasing the likelihood of experiencing serious health conditions. 
%
While association or prediction-based methods are extensively used in practice to draw useful insights from data, they typically identify correlations among variables and may fail to reveal the underlying causal factors, i.e., which actions may result in an improved outcome, needed for informed decision-making. 
%For recommendations to be truly impactful, there must be a clear  explanation that justifies why a particular decision is appropriate for a specific subpopulation~\cite{sun2021treatment,plecko2022causal}. 

\emph{Causal analysis} or {\em causal inference}, therefore, is considered one of the most important requirements to generate prescriptions that are {\em actionable} and aligned with human reasoning~\cite{imbens2024causal}. Causal inference, and in particular {\em observational studies} for causal inference on collected data (when controlled trials are impossible due to cost or ethical reasons), have been extensively studied in the statistics and artificial intelligence (AI) literature for several decades \cite{rubin2005causal, pearl2009causal}. Motivated by this foundational work on causal inference, the notion of causality has also influenced the field of database research. The causal models from AI have been extended to relational databases \cite{salimi2020causal},  and causality has been incorporated into various data management tasks such as finding responsibilities of inputs toward query answers ~\cite{meliou2010causality, meliou2009so, meliou2014causality}, explanations for query answers \cite{roy2014formal, DBLP:journals/pacmmod/YoungmannCGR24}, data discovery~\cite{galhotra2023metam,youngmann2023causal}, data cleaning~\cite{pirhadi2024otclean,salimi2019interventional}, hypothetical reasoning \cite{galhotra2022causal}, and large system diagnostics~\cite{markakis2024sawmill,causalsim,sage, gudmundsdottir2017demonstration}. 


\revc{If-then rules are generally considered interpretable by humans~\cite{lakkaraju2016interpretable,guidotti2018local,van2021evaluating,pradhan2022interpretable,chen2018optimization}.
We give a concrete example of the difference between association and causation in generating prescriptions or recommended actions in the form of if-then rules below}:
\begin{example}	%
\label{example:ex1} {\bf Importance of causal prescriptions:}
Consider the Stack Overflow (SO) annual developer survey
\cite{stackoverflowreport}, where respondents from around the world answer
questions about their jobs and demographics. A sample of the dataset \reva{with a subset of the
attributes (there are 20 attributes)} is presented in \cref{tab:data}.
%
Alice, a researcher in the United Nations (UN) finance department, is interested in discovering ways to increase the salaries of high-tech employees worldwide. She is looking for a set of actionable recommendations 
%(that we call a prescription rules) 
to raise the overall average salary.
%
Using association-based approaches~\cite{chen2018optimization,lakkaraju2016interpretable}, she may discover that individuals residing in the US who identify as straight or heterosexual tend to earn higher salaries (see \cref{exp:quality} for full details). However, this observation merely indicates a correlation: people living in the US, for example, generally earn more than those outside the country. Their comparatively higher salaries are primarily attributable to the country's economy and are unrelated to their sexual orientation. Thus, this observation cannot be used as a prescription rule to increase salary. 
Our causal analysis, on the other hand, reveals that individuals aged 25-34 with dependents would benefit from working as front-end developers.
This results in a \$44,009 annual salary increase on average. \reva{Another observation is that students should pursue an
undergraduate major in CS. %Computer Science (CS). 
This can boost their salary by \$22,174 per year} (see details in \cref{sec:casestudy}).
\end{example}

%It has been incorporated into various tasks including . 
%Causal interventions are often more relatable and easier to understand, as they offer insight into the underlying reasons behind the recommendations and allow unraveling complex cause-effect relationships that govern our world~\cite{pearl2009causality}. Furthermore, causal interventions often have long-lasting effects~\cite{imbens2024causal}.

%, making it essential that the prescribed actions are not only actionable but also 

%causally consistent. 

%Decision makings, in particular, high-stak

\cut{
In this work, {we study the problem of generating causal insights (referred to as \emph{prescription rules}), which serve as actionable recommendations} to improve an outcome of interest.
Recent works have introduced causality to the field of database research~\cite{meliou2010causality,  meliou2014causality,salimi2020causal,10.14778/3554821.3554902}. It has been incorporated into various tasks including data discovery~\cite{galhotra2023metam,youngmann2023causal}, data cleaning~\cite{pirhadi2024otclean,salimi2019interventional}, and large system diagnostics~\cite{markakis2024sawmill,causalsim,sage, gudmundsdottir2017demonstration}. 
We propose using causal inference to generate prescription rules that are both actionable and justifiable.
}

While generating prescriptions based on causal inference may help in robust decision-making, causal prescriptions that solely consider the betterment of an outcome (like salary) are not enough in practice. 
It is well-known that decision-making in many high-stake applications (like hiring policy, or policy for approving loans by banks) may lead to disparate societal or economic impact on different sub-populations. 
As a shocking example from a recent work called 
%For example, 
CauSumX~\cite{DBLP:journals/pacmmod/YoungmannCGR24} that generates a set of causal explanations for an aggregated view, the explanations generated %by CauSumX %recommendations which 
suggest that male individuals do a Bachelor's degree to increase their salary while %suggesting that 
being an unmarried woman 
%the recommendation for women includes getting married 
has the most adverse effect on salary
(borrowed directly 
from Fig.~19 in~\cite{youngmann2024summarizedcausalexplanationsaggregate}). 
%We demonstrate the advantage of using causal reasoning to generate actionable recommendations and the limitations of not considering fairness requirements in the following example. 
We explored this further in the context of generating prescriptions and observed that prescriptions that are not fairness-aware can generate unfair outcomes to some subpopulations which we refer to as the {\em protected group}. Examples include women, Black, Latino, or Native Americans, individuals with a disability, countries with a weaker economy, or other protected groups specific to an application. %Here is a concrete example:


% Understanding the causal factors behind these recommendations is crucial to ensuring that decisions lead to fair and equitable outcomes, particularly in sensitive applications where biased decisions can perpetuate or even exacerbate societal inequalities.
% While prior work has extensively explored techniques for association rule mining~\cite{kumbhare2014overview}, recent efforts have focused on deriving causal explanations for individual data points or entire datasets~\cite{salimi2018bias,youngmann2022explaining,ma2023xinsight}. Although some of these methods produce causally consistent insights, the absence of fairness considerations in the process can lead to unfair outcomes, further reinforcing existing biases. For example, CauSumX~\cite{DBLP:journals/pacmmod/YoungmannCGR24} generates causal recommendation which suggest male individuals to do a Bachelor's degree to increase salary while the recommendation for women include getting married (borrowed directly from Figure~19 in the paper~\cite{youngmann2024summarizedcausalexplanationsaggregate}). 





%\emph{Causal inference} has been thoroughly studied in AI and Statistics~\cite{pearl2009causal,rubin2005causal}. Causal analysis is a vital tool in determining the effect of a \emph{treatment} on an \emph{outcome}, and has been used in decision-making in medicine \cite{robins2000marginal}, economics \cite{banerjee2011poor}, biology \cite{shipley2016cause}, and in high-stakes areas such as identifying the root causes of failures in critical infrastructure systems to prevent catastrophic outcomes. Recent works have introduced causality to the field of database research~\cite{meliou2010causality,  meliou2014causality,salimi2020causal,10.14778/3554821.3554902}. It has been incorporated into various tasks including data discovery~\cite{galhotra2023metam,youngmann2023causal}, query result explanation~\cite{salimi2018bias,youngmann2022explaining,DBLP:journals/pacmmod/YoungmannCGR24}, and large system diagnostics~\cite{markakis2024sawmill,causalsim,sage, gudmundsdottir2017demonstration}. We propose leveraging causal inference to generate interpretable and justifiable insights (referred to as \emph{prescription rules}), which serve as actionable recommendations to improve an outcome of interest. Causal reasoning is considered one of the most important requirements,  to generate insights that are actionable and aligned with human reasoning.




\begin{table*}[]
\footnotesize
    \centering
    	\caption{\textnormal{A subset of the Stack Overflow dataset.}}
         \label{tab:data}
    	% \vspace{-4mm}
  			\begin{tabular}[b]{|l|l|l|c|l|l|c|l|c|}
  			
				%\multicolumn{9}{c}{\textbf{Users}}\\ 
				\hline

				\textbf{ID}
    
    % \textbf{Country}& \textbf{Continent} 
    
    &\textbf{Gender} &\textbf{Ethnicity}&
				\textbf{Age} &\textbf{Role} &
				\textbf{Education} &\textbf{Country}&\textbf{Undergrad Major}&\textbf{Salary}
				\\ \hline

				1 &Male&White&26&Data Scientist & PhD& US&Computer Science&180k\\
    		2 &Non-binary&White&32&QA developer & Bachelor's degree& US&Mechanical Eng.&83k\\

 3 &Male&South Asian&29&C-suite executive  & Bachelor's degree & India&Computer Science&24k\\

  % 4 &Female&South Asian&25&Back-end developer  & Master's degree & India&Mathematics&7.5k\\

  4 &Female&East Asian&21&Back-end developer & Bachelor's degree & China&Computer Science&19k\\
  

        % $\ldots$ &  $\ldots$&  $\ldots$&  $\ldots$&  $\ldots$&  $\ldots$&  $\ldots$&  $\ldots$&  $\ldots$&  $\ldots$&  $\ldots$\\
    \hline
			\end{tabular}
            \vspace{-5mm}
\end{table*}




\begin{example}	%
\label{example:ex2}
{\bf Importance of fair prescriptions:}
Continuing Example~\ref{example:ex1}, while those causal prescription rules are highly beneficial for the overall population, they are considerably less effective for individuals residing in countries with a low GDP (indicating a weaker economy). For this group, the average expected increase in salary is only approximately \$13,000 per year (in contrast to \$44,009 for the entire group). % \sr{add which rule 44k or 25k} 
Consequently, implementing these rules would exacerbate the disparity between those living in countries with strong economies and those in countries with weaker economies.
\end{example}




% Our objective is to generate a small set of prescription rules aimed at increasing (or decreasing) an outcome of interest. This is framed as an optimization problem where the goal is to select the fewest prescription rules that maximize utility (i.e., the expected increase or decrease in the outcome). However, 

The example above shows that focusing solely on maximizing utility (\revc{i.e., increasing income}) can result in a scenario where only some of the population receive significant improvement, while others experience no benefit (\revc{only a small benefit for individuals from countries with weaker economies in our example}). Additionally, even if a large portion of the population receives recommendations, a protected subpopulation might not share the benefits and, worse, their situation could deteriorate, exacerbating inequalities.

Examples~\ref{example:ex1} and \ref{example:ex2} show that it is crucial to provide recommendations that are (1) {\em causal} for the outcome (beyond associations),  and (2) also {\em fair or equitable} in terms of the outcome for both the protected and non-protected groups. While recent work in database research
has focused on deriving {\em causal explanations} for individual data points, aggregated view, or entire datasets~\cite{salimi2018bias,youngmann2022explaining,ma2023xinsight, DBLP:journals/pacmmod/YoungmannCGR24}, and in particular \cite{DBLP:journals/pacmmod/YoungmannCGR24} has considered generating a set of causal explanations for an aggregated view that resemble a ruleset, 
%Although some of these methods produce causally consistent insights, 
the absence of fairness considerations in generating these causal explanations can lead to unfair outcomes for the protected group.
%further reinforcing existing biases.


%\red{We, therefore, enable users to incorporate various \emph{coverage and fairness constraints} along with the overall objective of improving an outcome of interest. }

\medskip
\noindent
\textbf{Our contributions.~} 
Motivated by the dual goals of generating causal and fair prescriptions for the betterment of an outcome, we introduce a {\em fairness-aware framework leveraging causal reasoning for generating a set of actionable prescription rules (ruleset)} called \sysName\ (\underline{Fair} \underline{CA}usal \underline{P}rescription).
%
Following research on fairness in data management~\cite{stoyanovich2020responsible,galhotra2022causal}, we assume the existence of a \emph{protected subpopulation}, defined by an attribute such as gender or race for people, or GDP of a country. Motivated by the causal explanation rules for an aggregated view \cite{DBLP:journals/pacmmod/YoungmannCGR24}, each prescription rule in our ruleset applies to a sub-population defined by a {\em grouping attribute}, and prescribes a {\em treatment or intervention} to improve the {\em outcome} for this sub-population. Fairness constraints ensure that the expected utility of the protected population is {\em comparable} to the utility of the unprotected individuals. We borrow the notions of \emph{group and individual fairness} from the fairness literature but tailor them for prescription rules. In addition to the fairness constraints, our coverage constraints ensure that a substantial fraction of the population and protected subpopulation receives at least one recommendation. 
%We demonstrate how such constraints ensure that the generated rules apply to a large portion of the population and ensure fairness through the following example.

\begin{example}
\label{ex:intro_example_3}
Continuing Examples~\ref{example:ex1} and \ref{example:ex2}, Alice uses our proposed system, called \sysName, to impose fairness and coverage constraints to discover useful and equitable recommendations for increasing salaries worldwide. In particular,
Alice chooses to implement a coverage constraint to ensure that the selected rules apply to a significant portion of people worldwide, including a sufficiently large number of individuals from countries with low GDP (the protected group). She also imposes a fairness constraint to ensure that the expected gains for both protected and non-protected groups are comparable.
\reva{She discovers, for example, that for individuals with 6-8 years of coding experience (a subpopulation comprising 21\% of the entire dataset and 25\% of the protected group), pursuing a bachelor’s degree in computer science will increase the expected salary by $\$14.9k$ for protected and by $\$17.8k$ for non-protected}. (See \cref{sec:casestudy} for more details.) This prescription rule applies to a large portion of the population and ensures fairness by providing a similar expected gain for both protected and non-protected groups, and the allowed difference of outcomes between these two populations may be adjusted by choosing appropriate thresholds in the fairness definitions. 
\end{example}


\noindent
Our main contributions are as follows. \\
%\begin{itemize}[leftmargin=*,topsep=0pt]
{\bf (1)} We {\bf develop a framework that generates a set of prescription rules to enhance an outcome of interest (Section~\ref{sec:problem})}. A prescription rule consists of a \emph{grouping pattern} and an \emph{intervention pattern}, representing the target subpopulation and the actionable recommendation for that group, respectively. The strength of the {\em conditional causal effect} (Section~\ref{sec:background-causal}) of this intervention on the subgroup is used to measure the expected utility of a rule. Our objective is to identify the smallest set of rules that maximizes overall expected utility. We refer to this problem as the {\em \probName} problem.
We adopt several notions of fairness (individual vs. group, statistical parity vs. bounded group loss) from the literature to define the {\bf fairness constraints} for our problem. In addition, {\bf coverage constraints} (for individual rules or for a group) ensure that the solution for the \probName\ problem is applied to a sufficient number of individuals and to minimize inequalities. We show NP-hardness for different variants of the problems and properties (matroid) useful in our algorithms. 
%We establish several definitions for group and individual fairness constraints tailored for prescription rules.
\smallskip
    \par
    \noindent
{\bf (2)} We {\bf develop a general three-step algorithm named \sysName to solve the optimization problem of selecting a fair prescription ruleset (Section~\ref{sec:algo})}. The first step involves mining frequent grouping patterns using the Apriori algorithm~\cite{agrawal1994fast}. In the second step, we employ a lattice-based algorithm to find high utility and fair intervention patterns for grouping patterns identified in the previous step. Finally, the third step applies a greedy approach to determine a solution. \sysName\ can be easily adapted to accommodate all variants of the \probName\ problem.

\smallskip
\par
\noindent
{\bf (3) We provide a detailed  case study  (Section~\ref{sec:casestudy}) and experimental analysis (Section~\ref{sec:experiments}) to evaluate our framework and algorithms.}
The case study shows the qualitative difference of different variants of our problem for different choices of the fairness and coverage constraints. The experiments include two datasets, three baselines, and 18 variations of our problem with different constraints. Our evaluations suggest that fairness may come at the cost of expected
utility for everyone. However, without fairness constraints, we often observe a significant disparity between the protected and non-protected. We also observe that
achieving individual fairness is harder than group fairness,
as most high-utility or high-coverage rules are unfair. Lastly, we show that \sysName\ can generate  prescription rules over large datasets in a reasonable time. 

%\end{itemize}


%\paragraph*{Paper outline} 
We discuss related work in \cref{sec:related}, review background on causal inference in \Cref{sec:background-causal}, %and our problem formulation can be found in \cref{sec:problem}. Our algorithmic framework is presented in \cref{sec:algo}. A case study demonstrating the impact of different constraint configurations on the solution is given in \cref{exp:problem_variants}, and our experimental evaluation is detailed in \cref{sec:experiments}. Finally, we 
and discuss the limitations of our framework and future work in \cref{sec:conc}.

% \noindent
% \boxed{\parbox{\columnwidth}{$\bullet$ 
% For people with a professional degree, move to the United Kingdom
%  (coverage = 435 (20), coverage-protected = 20 (13), utility = 186855, utility-protected = 0.)\\
% $\bullet$ For graphic developers, move to the	United States
%  (coverage = 116 (29), coverage-protected = 8 (2), utility = 169431, utility-protected = 0).\\
% $\bullet$ For people who have no formal education, move to the United States
%  (coverage = 123 (34), coverage-protected = 7 (2), utility = 206742, utility-protected = 0).\\
% % \textcolor{red}{size = 38, length = 76, overlap = 64029181, utility = 1659307}\\
% \textcolor{blue}{overall coverage =674, expected utility = 187485
% coverage-protected = 35, expected utility-protected = 0}
% \sr{should mention protected group, and possibly not mention coverage in the intro or just intuitively like high coverage}
% }}


% Alice notes that although these rules result in a \$187,485 increase in the overall salary for those to whom they apply, they only affect a small fraction of the population, specifically 674 individuals. Additionally, although the expected salary increase is substantial, there is no expected increase in salary for non-males, a subpopulation of particular interest to Alice. In other words, applying these rules would result in no gain for non-males.
% \end{example}

% \begin{example}[Episode 2 - coverage and fairness constraints]
% Alice introduces coverage and fairness constraints to ensure that enough people will benefit from the rules and that they will be \emph{fair} with respect to non-males. Specifically, she demands that the benefit for a randomly chosen individual to whom one of the rules applies is nearly the same as the benefit for a randomly chosen individual who identifies as non-male and to whom one of the rules applies.

% After adding these constraints, \sysName\ recommends the following set of prescription rules:



% \noindent
% \boxed{\parbox{\columnwidth}{$\bullet$ 
% For people who have no formal education, move to the United States
%  (coverage = 123 (34), coverage-protected = 7 (2), utility = 206742, utility-protected = 0)\\
% $\bullet$ 
% For females, change role to	DevOps specialist (coverage = 2256 (47), coverage-protected = 2256 (47), utility = 90023, utility-protected = 90023).\\
% $\bullet$ For people with a Master's degree, move to the	United States
%  (coverage = 9097 (2222), coverage-protected = 642 (236), utility = 85390, utility-protected = 84201).\\
% % \textcolor{red}{size = 38, length = 76, overlap = 64029181, utility = 1659307}\\
% \textcolor{blue}{overall coverage =11476	
% , expected utility = 87601,
% coverage-protected = 2905, expected utility-protected = 88519}
% }} 







% \begin{figure}[t]
%         \centering
%         \begin{minipage}[b]{1.0\linewidth}
%             \small
%             \begin{tcolorbox}[colback=white]
%             \vspace{-2mm}
% $\bullet$ For backend developers, the treatment with the highest effect on salary is “Country = US” effect size = 78646
% \begin{itemize}
%     \item For non-male the effect is only: 59429
%     \item For male the effect is 80454
% \end{itemize}

% $\bullet$ For frontend developers, the treatment with the highest effect is :Formal Education = Bachelor's degree” effect size: 17340
% \begin{itemize}
%     \item For white the effect is 33464
%     \item For non-white the effect is 15320
% \end{itemize}


% $\bullet$ For people in Europe, the treatment with the highest effect on salary is “DevType = C-suite executive” effect size = 53254
% \begin{itemize}
%     \item For white the effect is 55112
%     \item For non-white 35249
% \end{itemize}



%             \vspace{-2mm}
%             \end{tcolorbox}
%         \end{minipage}%%
%          % \vspace{-4mm}
%         \caption{Set of prescription rules.}
%         \label{fig:so-explanation}
%     \end{figure}

\section{Background and related work}
% 重点看Artistic data visualization: Beyond visual analytics 和Visualization criticism-the missing link between information visualization and art 的被引


This section reviews the background on artistic data visualization and previous research related to this topic.

\subsection{Artistic Data Visualization in Art History Context}
\label{ssec:contemporary}

Art history has been marked by transformative periods characterized by different aesthetic pursuits, materials, and methods. Since the time of Plato, imitation (or \textit{mimesis}, which views art as a mirror to the world around us) has been an important pursuit~\cite{pooke2021art}. Successful artworks, such as Greek sculptures and the convincing illusions of depth and space in Renaissance paintings, exemplify this tradition.
The advent of modern society and new technology, especially photography, posed a significant challenge to the notion of art as imitation~\cite{perry2004themes}. By the 1850s, modern art began to emerge with the core goal of transcending traditional forms and conventions. Movements like Post Impressionism, Expressionism, and Cubism revolutionized art through innovative uses of form (\eg color, texture, composition), moving art towards abstraction and experimentation. 
After World War II, the Cold War and the surge of various social problems heightened skepticism about the progress narrative of modernity and the superiority of the capitalist system, leading to the rise of postmodernism and the birth of contemporary art~\cite{hopkins2000after,harrison1992art}. One prominent feature of contemporary art is the absence of fixed standards or genres historically defined by the church or the academy. Postmodern design neither defines a cohesive set of aesthetic values nor restricts the range of media used~\cite{pooke2021art}, sparking novel practices such as installations, performances, lens-based media, videos, and land-based art~\cite{hopkins2000after}.
Meanwhile, artists have increasingly drawn energy from various philosophical and critical theories such as gender studies, psychoanalysis, Marxism, and post-structuralism~\cite{pooke2021art}. As a result, as described by Foster~\cite{foster1999recodings}, artists have increasingly become ``manipulators of signs and symbols... and the viewer an active reader of messages rather than a passive contemplator of the aesthetic''. Hopkins~\cite{hopkins2000after} described this shift as the ``death of the object'' and ``the move to conceptualism''. 
% Joseph Kosuth, one of the most important representatives of conceptual artists, also once said that “all art (after Duchamp) is conceptual (in nature) because art only exists conceptually”
% As argued by Danto~\cite{danto2015after}, traditional notions of aesthetics can no longer apply to contemporary art. ``“All there is at the end,” Danto wrote, “is theory, art having finally become vaporized in a dazzle of pure thought about itself, and remaining, as it were, solely as the object of its own theoretical consciousness.''
% The Anti-aesthetic (1983) has described these as ‘anti-aesthetic’ strategies – typified, as we have seen, by a conceptually driven approach to the art object and to the process of its production.

Emerging within the contemporary art historical context, data art has been significantly propelled by the advent of big data. An early milestone was Kynaston McShine's 1970 exhibition \textit{Information} at the Museum of Modern Art (MoMA). 
% In the exhibition catalogue, McShine wrote~\cite{information_moma}: ``Increasingly artists use mail, telegrams, telex machines, etc., for transmission of works themselves—photographs, films, documents—or of information about their activity.'' 
% The millennium era has witnessed substantial growth in this area.
In 2008, Google’s Data Arts Team was founded to explore the advancement of what creativity and technology can do together~\cite{google}.
% data artist Aaron Koblin
In 2012, Viégas and Wattenberg's \textit{Wind Map}, an artwork that visualizes real-time air movement, became the first web-based artwork to be included in MoMA's permanent collection~\cite{wind}.
Since 2013, the academic conference IEEE VIS has included an Arts Program (IEEE VISAP), showcasing artistic data visualizations through accepted papers and curated exhibitions. 
As noted by Barabási~\cite{dataism} (a Fellow of the American Physical Society and the head of a data art lab), data has become a vital medium for artists to deal with the complexities of our society: ``Humanity is facing a complexity explosion. We are confronted with too much data for any of us to make sense of...The traditional tools and mediums of art, be they canvas or chisel, are woefully inadequate for this task...today’s and tomorrow’s artists can embrace new tools and mediums that scale to the challenge, ensuring that their practice can continue to reflect our changing epistemology.''
% a physicist and head of a data art lab, has noted, 

% Artists are now exploring new mediums and methods, incorporating datasets, computer technology, and algorithms into their work.



\subsection{Research on Artistic Data Visualization}
\label{ssec:artisticvis}

Artistic data visualization is also referred to as artistic visualization, data art, or information art~\cite{holmquist2003informative,rodgers2011exploring,few,viegas2007artistic}. Despite the varying terminologies, there is a basic consensus that artistic data visualization must be art practice grounded in real data~\cite{viegas2007artistic}.
Since the early 2000s, a series of papers introduced innovative artistic systems for visualizing everyday data, such as museum visit routes and bus schedule information~\cite{skog2003between,holmquist2003informative,viegas2004artifacts}.
In 2007, Viégas and Wattenberg~\cite{viegas2007artistic} explicitly proposed the concept of \textit{artistic data visualization} and viewed it as a promising domain beyond visual analytics.
% and defined it as ``visualization of data done by artists with the intent of making art''. 
Kosara~\cite{kosara2007visualization} drew a spectrum of visualization design, positioning artistic visualization and pragmatic visualization at opposite ends of this spectrum to demonstrate that the goals of these two types of design often diverge. 
% advocating that analytical visualizations prioritize practicality, while artistic data visualizations emphasize sublime quality, that is, the capacity to inspire awe and grandeur and elicit profound emotional or intellectual responses. 
% In 2011, Rodgers and Bartram~\cite{rodgers2011exploring} utilized artistic data visualization to enhance residential energy use feedback. 
However, overall, research on this subject has been sparse. Among those relevant papers, most have focused on specific applications of artistic data visualization. 
%~\cite{rodgers2011exploring,schroeder2015visualization,perovich2020chemicals}
For instance, Rodgers and Bartram~\cite{rodgers2011exploring} utilized ambient artistic data visualization to enhance residential energy use feedback. Samsel~\etal~\cite{samsel2018art} transferred artistic styles from paintings into scientific visualization.
Artistic practice has also been observed in fields such as data physicalization~\cite{hornecker2023design,perovich2020chemicals,offenhuber2019data} and sonification~\cite{enge2024open}. For example, Hornecker~\etal~\cite{hornecker2023design} found that many artists are practicing transforming data into tangible artifacts or installations. Enge~\etal~\cite{enge2024open} discussed a set of representative artistic cases that combine sonification and visualization.
% dragicevic2020data
% Offenhuber~\cite{offenhuber2019data} created a set of artworks in urban settings that transform air quality data into situated displays, allowing people to encounter visualizations in their daily lives.

% But in contrast, empirical studies that describe the characteristics of artistic visualization and how they are created are extremely scarce. This scarcity forms a stark contrast to the increasingly rich and diverse practices by artists in the field.
% As for the difference between artistic data visualization and traditional visualizations for analytics, Vi{\'e}gas and Wattenberg~\cite{viegas2007artistic} thought that the most salient feature of artistic data visualizations is their forceful expression of viewpoints.
% In Ramirez~\cite{ramirez2008information}'s opinion, functional information visualizations are concerned with usability and performance while aesthetic information visualizations are concerned with visually attractive forms of representation design.
% Donath~\etal~\cite{donath2010data} presented a series of tools developed to integrate artistic expressions in generating unique and customized visualizations based on users' personal data, such as health monitoring data, album records, and e-mail records. 

On the other hand, some studies, while not focusing on artistic data visualization, have explored a key art-related concept: aesthetics. 
% ~\cite{moere2012evaluating,cawthon2007effect,lau2007towards} explored the aesthetics of visualization design in their research. They
For example, Moere~\etal~\cite{moere2012evaluating} compared analytical, magazine, and artistic visualization styles, noting that analytical styles enhance the discovery of analytical insights, while the other two induce more meaning-based insights. Cawthon~\etal~\cite{cawthon2007effect} asked participants to rank eleven visualization types on an aesthetic scale from ``ugly'' to ``beautiful'', finding that some visualizations (\eg sunburst) were perceived as more beautiful than others (\eg beam trees).
% Moere~\etal~\cite{moere2012evaluating} displayed data in three different styles (analytical style, magazine style, artistic style) and found that these styles led to different perceptions of usability and types of insights.
% More importantly, the authors found that the sunburst chart ranks the highest in aesthetics and is one of the top-performing visualizations in both efficiency and effectiveness, thus exemplifying the notion that "beautiful is indeed usable".
Factors such as embellishment~\cite{bateman2010useful}, colorfulness~\cite{harrison2015infographic}, and interaction~\cite{stoll2024investigating} have also been found to influence aesthetics. 
% borkin2013makes,tanahashi2012design
However, most of these studies have simplified aesthetics to hedonic features (\eg beauty), without delving into the nuanced connotations of aesthetics.
% most of these studies have simplified aesthetics to concepts like 'beauty,' 'preference,' or 'pleasing,' without exploring the deeper essence of aesthetics as the core of art.

The value of artistic data visualization to the visualization community is still in controversy. For instance, Few~\cite{few} argued for a clearer distinction between data art and data visualization; he highlighted the negative consequences when data art ``masquerades as data visualization'', such as making visualizations difficult to perceive. Willers~\cite{willers2014show} thought the artistic approach is ``unlikely be appreciated if the intention was for rapid decision making.''
% In an interview, American artist and technologist Harris commented that ``data can be made pretty by design, but this is a superficial prettiness, like a boring woman wearing too much makeup.''~\cite{harris2015beauty} 
To address these gaps, more empirical research needs to be conducted to explore how artistic data visualizations are designed, their underlying pursuits, and how they might inspire our community.




% Examining this field not only helps us understand the latest application of data visualization in various domains but also extends our understanding of the aesthetic and humanistic aspects of data visualization.
% there should be more theoretical investigation into artistic data visualization. 

% These three concepts emphasize placing or installing visualizations at physical places that people will encounter in their daily lives. 

% ~\cite{wang2019emotional}


% gap between art and science~\cite{judelman2004aesthetics}
% constructive visualization~\cite{huron2014constructive}
% data feminism~\cite{d2020data}
% critical infovis~\cite{dork2013critical}
% citizen data and participation~\cite{valkanova2015public}

% \x{Lee~\etal~\cite{lee2013sketchstory}, give users artistic freedom to create their own visualizations.}


% Aesthetics refers to the study of beauty, taste, and sensory perception and is deeply intertwined with art.
% a particular taste for or approach to what is pleasing to the senses and especially sight

% why shouldn't all charts be scatter plot~\cite{bertini2020shouldn}
% aesthetic model~\cite{lau2007towards}
% Aesthetics for Communicative Visualization : a Brief Review
% Stacked graphs--geometry \& aesthetics~\cite{byron2008stacked}
% storyline optimization~\cite{tanahashi2012design}
% graphic designers rate the attractiveness of non-standard and pictorial visualizations higher than standard and abstract ones, whereas the opposite is true for laypeople.~\cite{quispel2014would}
% evaluate aesthetics - wordcloud
% An Evaluation of Semantically Grouped Word Cloud Designs, tag cloud, wordle

% On the other hand, empirical studies conducted with designers have shown that functionality is not the only design goal of visualization. For example, Bigelow~\etal~\cite{bigelow2014reflections} found that designers would frequently fine-tune the non-data elements in their designs, and data encoding was even "a later consideration with respect to other visual elements of the infographic".
% Moere~\cite{moere2011role} - design
% Quispel~\etal~\cite{quispel2018aesthetics} found that for information designers, clarity and aesthetics are both important for making a design attractive.
% \gabis{Where do we define our notion of framing and relevant terms? Will that happen in the intro? For example here we use the term ``sentiment shifts'', which I think requires defintion.}\gili{done in intro}

% \gabis{Recurring comment - we should change tense to present, while most of the paper is currently in the past tense, I indicated this in some places, but should verify throughout.}

% \begin{figure*}[htbp]
    \centering
    % First Subfigure
    \begin{subfigure}{0.49\textwidth} % Adjust width as needed
        \centering
        \includegraphics[width=\textwidth]{images/orig_negative_models_distribution.png} % Replace with your image path
        \caption{Sentences that are \textbf{negative} in their original form.}
        \label{fig:negative-flip}
    \end{subfigure}
    % \hfill % Adds horizontal space between subfigures
    % Second Subfigure
    \begin{subfigure}{0.49\textwidth}
        \centering
        \includegraphics[width=\textwidth]{images/orig_positive_models_distribution.png} % Replace with your image path
        \caption{Sentences that are \textbf{positive} in their original form.}
        \label{fig:positive-flip}
    \end{subfigure}
    \caption{Proportion of sentences for which LLMs flipped sentiment, became neutral, or retained the original sentiment when presented with opposite sentiment framing. For example, this measures the percentage of sentences originally labeled as positive, that were labeled as negative after applying negative framing (and vice versa).
    }
    \label{fig:flip-proportion}
\end{figure*}


Our dataset curation consists of three steps, as depicted in Figure~\ref{fig:fig1}. First, we collect natural, real-world statements, with some clear sentiment, either positive or negative (\S\ref{sec:base-statements}; e.g., ``I won the highest prize'' as positive). Next, 
we reframe each statement by adding a prefix or suffix conveying the opposite sentiment
% for each statement, we add a framing that conveys the opposite sentiment to the base statement 
(\S\ref{sec:adding-framing}; e.g., ``I won the highest prize, although I lost all my friends on the way''). Finally, we collect large-scale human annotations via crowdsourcing, to label the sentiment shifts when wrapping the statements with the opposite framing (\S\ref{sec:human-annotations}; e.g., labeling ``negative'' the statement ``I won the highest prize, although I lost all my friends on the way''). 
%\gabis{I think we can remove the textual examples here to save space}

The complete dataset consists of 1000 statements, in which 500 are statements that their base form has positive sentiment, and 500 are base negative statements. 




\subsection{Collecting Base Statements}\label{sec:base-statements}
First, we collect base statements, which convey a clear sentiment, either clearly positive or clearly negative statements. We use \spike{} -- an extractive search system, which allows to extract statements from real-world datasets~\cite{taub-tabib-etal-2020-interactive}.
%\gabis{there's also a citation for spike}.\footnote{~\url{https://spike.apps.allenai.org}} 
Specifically, we collect statements from Amazon Reviews dataset, which are naturally occurring, sentiment-rich, texts but are less likely to trigger strong preexisting biases or emotional reactions, which may be a confound for our experiment.\footnote{~\url{https://spike.apps.allenai.org/datasets/reviews}} 
% \gabis{Why did we use this specifically? I think once we write the intro it would be good to relate to what we wrote there and how this domain is relevant.}
\begin{figure}[tb!]
    \centering
    \includegraphics[width=\linewidth]{images/roberta_score_before_after_framing.png}
    \caption{Distribution of sentiment scores before and after applying opposite-sentiment framing, as detailed in Section~\ref{sec:adding-framing}. Prior to framing, base sentences exhibit a clear polarity (positive or negative), whereas the application of opposite framing introduces ambiguity, shifting the sentiment scores toward a less distinct polarity.}
    \label{fig:pos-score-dist}
\end{figure}


Using \spike, we extract ${\sim}6k$ statements that fulfilled our designated queries, which we found correlated with clear sentiment. We designed the queries to capture positive or negative verbs that describe actions with some clear sentiment (e.g., ``enjoy'' or ``waste''), or statements with positive or negative adjective, describing an outcome with a clear sentiment (e.g., ``good'' or ``nasty''). The patterns and queries used for extraction are detailed in Appendix~\ref{sec:appendix-spike}.
% \gabis{needs more details, what are our queries? What were we aiming for? I understand that at a high level we're looking for clear sentiment, but how do we achieve this via lexical-syntactic queries?}. 
Next, we run in-house annotations to label and filter the extracted statements, to handle negations or other cases where the statement does not convey a clear sentiment. 
The filtering process results in $1,301$ positive statements, and $1,229$ negative statements.


\subsection{Adding Framing}\label{sec:adding-framing}

To reframe the statements in our dataset, we use GPT-4~\cite{achiam2023gpt}.\footnote{We used the gpt-4-0613 version.} 
% \gabis{do we have more details about which GPT4? what date?}
% The model was asked to keep he base statement unchanged, and add some prefix or suffix, that can be either positive or negative, oppositely to the base statement sentiment (e.g., I won the highest proze, althoug I lost all my friends on the way). 
The input prompt includes a 1-shot example, followed by a task description ``Add a <SENTIMENT> suffix or prefix to the given statement. Don't change the original statement.'', where SENTIMENT is either ``positive'' or ``negative'', opposite to the base statement sentiment (i.e., positive framing for negative base statement, and vice versa).

Unlike the base statement, the conveying sentiment of reframed statements is more ambiguous and there is no one clear label, as shown in Figure~\ref{fig:pos-score-dist}.\footnote{Scores in Figure~\ref{fig:pos-score-dist} are given by a fine-tuned sentiment analysis model ~\url{https://huggingface.co/cardiffnlp/twitter-roberta-base-sentiment-latest}}
%as we present the sentiment scores assigned by a fine-tuned sentiment analysis model,\footnote{~\url{https://huggingface.co/cardiffnlp/twitter-roberta-base-sentiment-latest}} %that was shown to be state-of-the-art when fine-tuned on sentiment analysis~\cite{csanady2024llambert}. 
% We present the sentiment scores 
% before and after reframing. It shows that wrapping the statement with the opposite sentiment injects ambiguity to the overall sentiment, as the sentiment scores become more dispersed. 
The exhibeted ambiguity in sentiment allows us to measure to what extent LLMs' shifting sentiment after framing, and how correlated it is to human behavior.



% In Figure~\ref{fig:pos-score-dist}, \gabis{Is roberta SOTA? it's a bit old by now. Do we have a reference to back this up?}\footnote{RoBERTa, fine-tuned for sentiment analysis~\url{https://huggingface.co/cardiffnlp/twitter-roberta-base-sentiment-latest}} The base statement scores are predominantly centered around binary values, either strongly positive or strongly negative. In contrast, the sentiment scores after opposite framing are more dispersed, reflecting increased ambiguity in sentiment. 
% \gabis{I'm not sure if this paragraph belongs here, maybe should be a subsection on its own at the end of the section?}


\subsection{Collecting Human Annotations}\label{sec:human-annotations}

In the final step, we collect human annotations through Amazon Mechanical Turk to evaluate the framing effect in \name{} over human participants, providing a reference for comparison with LLMs.\footnote{\url{https://www.mturk.com}} 
Details about the annotation platform are elaborated in Appendix~\ref{sec:mturk-appendix}.

The complete dataset includes 1K statements, each annotated by five different annotators. Given our budget, we preferred to collect five annotations per statement, resulting in less statements, but providing a more robust scoring for the ambiguity of a statement.

% We select a pool of 10 qualified workers who successfully passed our qualification test, which consisted of 20 base statements (unframed), for which annotators were expected to achieve perfect accuracy. The estimated hourly wage for the entire experiment was approximately 14USD per hour. More details about the annotation platform can be found in Appendix~\ref{sec:mturk-appendix}. Given our budget, we preferred to collect five annotations per statement, resulting in less statements, but providing a more robust scoring for the ambiguity of a statement.

For the annotation process, each statement in our dataset is presented in its reframed version (i.e., positive base statements with negative framing and vice versa), to five different annotators. This setup generates, for each dataset instance, a score ranging from 0 to 5, representing the number of annotators that votes for the sentiment that aligns with the opposite framing, which means that the overall sentiment of the reframed statement has shifted from its base sentiment. For example, in Figure~\ref{fig:fig1}, the statement ``I won the highest prize, although I lost all my friends on the way'' is shown to have two annotators voting ``negative'', which aligns with the sentiment of the framing and not the base statement, so the label for that instance in \name{} would be 2 (sentiment shifts).

% \gabist{It is important to note that there is no definitive ``right'' or ``wrong'' label for these statements, as the opposite sentiment framing often renders the sentiment conveyed highly ambiguous.}
Instances with score near 0 indicate that annotators agree that the overall sentiment remains unchanged despite the opposite framing. Score closer to 5 indicates that annotators agree that reframing shifted the perceived sentiment, while score around 2-3 suggests that the opposite framing makes the sentiment ambiguous.


\section{Experimental Study}\label{sec:experiments}
We conduct a comprehensive evaluation across multiple aspects: zero-shot performance, comparison with training on gold attribution data, and generalization to dialogue settings.
% . Our experiments span both in-\textit{isolation} question-answering datasets and in-\textit{dialogue} scenarios,
With our experiments, we shed light on the performance and practical utility of our approach.

% \textcolor{red}{TODO: Should we introduce the two settings separately: in-isolation QA and in-dialogue QA?}

% \textcolor{red}{TODO: We need an introduction sentence for this section.}

% \textcolor{red}{TODO: We mention isolated context attribution in some parts of the paper, while it is not clear how it differs from dialog-based context attribution.}

\subsection{Experimental Setting}
We evaluate model performance using precision (P), recall (R), and F1 score. For each sentence in the LLM's output, the context-attribution models identify the set of context sentences that support that output sentence. Precision measures the proportion of predicted attributions that are correct, while recall measures the proportion of ground truth attributions that are successfully identified.
%F1 is the harmonic mean of precision and recall.

For a fair and comprehensive evaluation, we train all models with a single pass over the training data unless stated otherwise, referring to this setup as \textbf{1P} when needed. For a more controlled comparison, some experiments limit the number of training samples each model encounters. Since the synthetic dataset contains approximately 1.0M samples, we allow models to \textit{observe} an equivalent number of samples from the gold training set, ensuring comparable exposure to models trained on data from \synqa. We refer to this setting as \textbf{1M} when necessary. For all models, we fine-tune only the LoRA parameters (alpha=64, rank=32) using a fixed learning rate of 1e-5 and a weight decay of 1e-3. 

\textbf{In-domain datasets:} We use \squadcolor{SQuAD} \cite{Rajpurkar2016SQuAD1Q} and \hotpotcolor{HotpotQA} \cite{Yang2018HotpotQAAD} as our primary in-domain benchmarks.\footnote{For some experiments (e.g., in Table~\ref{table:zero-shot-models}), these datasets are also \textit{out-of-domain} w.r.t. data generated by \synqa.} SQuAD provides clear sentence-level evidence for answering questions, serving as a strong baseline for direct attribution. HotpotQA introduces multi-hop reasoning, requiring models to link information across multiple sentences (sometimes from different articles) to identify the correct evidence chain. Additionally, HotpotQA includes distractor documents—closely related yet incorrect sources—posing a more challenging but realistic setting for evaluating attribution performance.

%\textcolor{red}{TODO (Kiril): Explain what you do to OR-QUAC, you combine the background with the context?}
\textbf{Out-of-domain datasets:} To assess generalization beyond the training distribution, we evaluate models on \quaccolor{QuAC} \cite{Choi2018QuACQA}, \coqacolor{CoQA} \cite{Reddy2018CoQAAC}, \orquaccolor{OR-QuAC} \cite{qu2020open}, and \doqacolor{DoQA} \cite{campos-etal-2020-doqa}. %\footnote{We consider these datasets as \textit{out-of-domain}, as none of the models we train are exposed to the training data of these datasets.}. 
These datasets present conversational QA scenarios that differ from SQuAD and HotpotQA. Specifically, QuAC and CoQA introduce multi-turn dialogue structures with coreferences, challenging models to track context across multiple turns. This conversational nature creates a methodological challenge: while these datasets are valuable for evaluating dialogue-based attribution, their reliance on conversation history makes direct comparison with models trained on single-turn QA datasets impossible.

To enable comprehensive evaluation across dialogue QA and single-turn QA, we create two versions of each dataset:
\begin{inparaenum}[(i)]
    \item a rephrased version using Llama 70B \cite{Dubey2024TheL3} that converts questions into standalone format for fair comparison with models trained on single-turn context attribution (suffixed by ``-ST''), and
    \item the original version for assessing dialogue-based attribution.
\end{inparaenum}

% To adapt these datasets for isolated context attribution (e.g., such as SQuAD and HotpotQA, where the question-answer pair is standalone),
% \footnote{We refer to isolated context attribution the scenario where the question-answer pair are standalone: i.e., do not contain coreferences.},
% we rephrase question-answer pairs (using Llama 70B), so that coreferencing is unnecessary. However, in dialogue-based settings, we evaluate models on the original, unmodified versions of these datasets.

DoQA extends this challenge further by incorporating domain-specific dialogues (cooking, travel and movies)%\footnote{The domains covered in DoQA are: cooking, travel and movies \cite{campos-etal-2020-doqa}.}
, thus testing the models' adaptability to specialized contexts. OR-QuAC includes %open-retrieval dialogue settings, assesses models' ability to attribute context in less structured environments, adding another layer of complexity to generalization evaluation. \textcolor{red}{TODO (Kiril): Check the papers for these datasets in case something is overlooked here.}
context-independent rewrites of the dialogue questions, such that they can be posed in isolation of prior context (i.e., single-turn QA). This enables us to test the models on their capabilities in both single-turn QA and dialogue QA settings.

\subsection{Methods}
We compare our method (\synqa) against several baselines, including sentence-encoder-based models, zero-shot instruction-tuned LLMs, and models trained on synthetic and gold context-attribution data. Specifically, we experiment with the following methods:

\paragraph{Sentence-Encoders:} We embed each sentence in the context along with the question-answer pair, and select attribution sentences based on cosine similarity with a fixed threshold, tuned on a small validation set.

\paragraph{Zero-shot (L)LMs:} We evaluate various instruction-tuned (L)LMs in a zero-shot manner, as such models have been shown to perform well across a range of NLP tasks \cite{shu2023exploitability,zhang2023instruction}. During inference, we provide an instruction template describing the task to the LLM (see Appendix~\ref{app:prompts} for details).
%as such models have been shown to perform well across a range of NLP tasks.

\paragraph{Ensembles of LLMs:} We aggregate the predictions of multiple LLMs through majority voting, selecting attribution sentences that receive consensus from at least 50\% of the ensemble. In our experiments, we use Llama8B \cite{Dubey2024TheL3}, Mistral7B, and Mistral-Nemo12B \cite{Jiang2023Mistral7} as the ensemble constituents.


\paragraph{Models trained on in-domain gold data:} Fine-tuning on gold-labeled attribution data provides an upper bound on in-domain performance, helping us assess how well synthetic training data generalizes.

\paragraph{\synatt:} \synatt generates synthetic training data by prompting multiple LLMs to perform context attribution in a discriminative manner, aggregating their outputs via majority voting, and training a smaller model on the resulting dataset. To make it a stronger baseline against \synqa, we give the training data of SQuAD and HotpotQA (the context, questions, and answers) to the LLMs and ask them to perform context attribution (note that we do not use the gold attribution). Finally, we train a model on the generated synthetic data.

\paragraph{\synqa:} We train models using synthetic data generated by our proposed method \synqa. Note that even though we train models using \synqa attribution data, we ensure they are not exposed to \textit{any} parts of the evaluation data.\footnote{We identify data leakage by representing each Wikipedia article as a MinHash signature. Then, for each training Wikipedia article, we retrieve candidates from the testing datasets via Locality Sensitivity Hashing and compute their Jaccard similarity \cite{dasgupta2011fast}. We flag as potential leaks pairs exceeding a threshold empirically set to 0.8.}

\subsection{Results and Discussion}
Evaluating our context attribution models requires a multifaceted approach, as performance is influenced by both the quality of training data and the model’s ability to generalize beyond in-domain distributions. Therefore, we design our experiments to address five core questions:
\begin{inparaenum}[(i)]
    \item How well do zero-shot LLMs perform on context-attribution QA tasks (\S\ref{sec:experiments-zero-shot})?
    \item Can models trained on synthetic data generated by \synqa exceed the performance of models trained on gold context-attribution data (\S\ref{sec:experiments-gold})?
    \item To what extent do models generalize to dialogue settings where in-domain training data is unavailable (\S\ref{sec:experiments-dialog})?
    \item How well do models scale in terms of synthetic data quantity generated by \synqa (\S\ref{sec:scalling-trends})?
    \item How do improved context attributions impact the end users' speed and ability to verify questions answering outputs (\S\ref{sec:user-study})?
\end{inparaenum}
% (i) How well do zero-shot LLMs perform context-attribution? (ii) Can synthetic attribution data serve as a viable alternative to gold supervision, particularly in out-of-domain settings? (iii) How do scaling trends affect generalization performance across diverse datasets?

%By systematically comparing models trained on synthetic data to both zero-shot and gold-supervised baselines, we aim to uncover the trade-offs between scalability, performance, and generalization. 
%Collectively, our findings provide a deeper understanding of how synthetic data can be leveraged for context attribution, potentially mitigating the reliance on costly human-annotated datasets.

\subsubsection{Comparison to Zero-Shot Models}\label{sec:experiments-zero-shot}

% Zero-shot v.s. SynQA-trained Models

\begin{table*}[t]
\centering
\resizebox{1.0\textwidth}{!}{
\begin{tabular}{lccccccccccccccc} \toprule
\multirow{2}{*}{Model} & \multirow{2}{*}{Training data} & \multicolumn{3}{c}{\squadcolor{Squad}} & \multicolumn{3}{c}{\hotpotcolor{Hotpot}} & \multicolumn{3}{c}{\quaccolor{Quac-ST}} & \multicolumn{3}{c}{\coqacolor{CoQA-ST}} \\ \cmidrule(lr){3-5} \cmidrule(lr){6-8} \cmidrule(lr){9-11} \cmidrule(lr){12-14}
& & P & R & F1 & P & R & F1 & P & R & F1 & P & R & F1 \\ \midrule
\textbf{\textit{Baselines}} \\
Random & -- & 19.8 & 15.4 & 17.3 & 4.8 & 15.2 & 7.3 & 5.2 & 15.1 & 7.7 & 7.3 & 15.1 & 9.9 \\
E5 | 561M & Zero-shot & 38.1 & 76.5 & 50.9 & 12.4 & 41.4 & 19.1 & 65.0 & 73.8 & 69.1 & 61.1 & 15.2 & 24.4 \\
HF-SmolLM2 | 365M & Zero-shot & 28.1 & 46.4 & 35.0 & 5.1 & 7.3 & 6.0 & 10.6 & 22.6 & 14.4 & 10.6 & 21.5 & 14.2 \\
Llama | 1B & Zero-shot & 37.5 & 62.0 & 46.7 & 5.3 & 28.1 & 8.9 & 8.8 & 65.4 & 15.4 & 11.9 & 52.8 & 19.4 \\
Mistral | 7B & Zero-shot & 71.5 & 94.4 & 81.4 & 42.9 & 42.7 & 42.8 & 63.2 & 88.6 & 73.8 & 59.0 & 72.2 & 64.9 \\
Llama | 8B & Zero-shot & 71.9 & 96.9 & 82.6 & 49.2 & 52.9 & 51.0 & 64.1 & 92.1 & 75.6 & 55.7 & 76.4 & 64.4 \\
Mistral-NeMo | 12B & Zero-shot & 89.5 & 94.5 & 91.8 & 46.4 & 47.3 & 46.8 & 81.8 & 85.3 & 83.5 & 79.0 & 67.2 & 72.6 \\
Ensemble | 27B & Zero-shot & 83.1 & 96.3 & 89.2 & 48.1 & 59.6 & 53.2 & 74.8 & 90.3 & 81.8 & 69.5 & 73.6 & 71.5 \\
Llama | 70B & Zero-shot & 95.3 & 95.6 & 95.5 & 87.6 & 37.5 & 52.5 & 89.7 & 87.8 & 88.7 & \textbf{87.5} & \textbf{73.3} & \textbf{79.8} \\
\midrule
\textbf{\textit{Baselines}} \\
%Llama | 1B & \squadcolor{SQuAD} \& \hotpotcolor{HotpotQA}; \synatt (1P) & 89.8 & 96.5 & 93.0 & 50.6 & 58.6 & 54.3 & 64.9 & 91.5 & 75.9 & 53.1 & 75.5 & 62.3 \\
%Llama | 1B & \squadcolor{SQuAD} \& \hotpotcolor{HotpotQA}; \synatt (1M) & 84.3 & \textbf{96.9} & 90.2 & 54.4 & 58.0 & 56.1 & 63.4 & 92.4 & 75.2 & 52.5 & 77.5 & 62.6 \\ \midrule
Llama | 1B & \synatt (1P) & 89.8 & 96.5 & 93.0 & 50.6 & 58.6 & 54.3 & 64.9 & 91.5 & 75.9 & 53.1 & 75.5 & 62.3 \\
Llama | 1B & \synatt (1M) & 84.3 & \textbf{96.9} & 90.2 & 54.4 & 58.0 & 56.1 & 63.4 & 92.4 & 75.2 & 52.5 & 77.5 & 62.6 \\ \midrule
\textbf{\textit{Ours}} \\
Llama | 1B & \syntheticcolor{\synqa} & \textbf{96.0} & 96.2 & \textbf{96.1} & \textbf{89.6} & \textbf{69.4} & \textbf{78.2} & \textbf{93.3} & \textbf{89.1} & \textbf{91.1} & \underline{82.3} & 68.5 & \underline{74.8} \\
\bottomrule
\end{tabular}
}
\caption{Comparison of zero-shot models and those trained with synthetic data. Larger zero-shot LMs excel, but our \synqa model outperforms all but one for one dataset while being smaller. \textbf{Bold} denotes best method, \underline{underline} if our method is second best. 1P: models trained with a single pass over the training data. 1M: models trained with 1M samples to match the size of the \synqa data.}
\label{table:zero-shot-models}
\end{table*}

In Table~\ref{table:zero-shot-models}, we present the performance of zero-shot models, and models trained without gold context-attribution data. 
%\footnote{Note that the \synatt baseline models are trained using question-answer pairs from SQuaAD and HotpotQA, however, the context-attribution annotations are obtained using an ensemble of LLMs.}. 
State-of-the-art sentence-encoder models (e.g., E5) perform relatively poorly, consistent with prior findings \cite{CohenWang2024ContextCiteAM}. In contrast, LLMs exhibit strong performance, with improvements correlating with model size. Ensembling multiple zero-shot LLMs further enhances performance, leveraging complementary strengths across models, but making the attribution more expensive. We also tested models trained with the discriminative method \synatt. These models significantly outperform their non-fine-tuned counterparts of the same size. However, as postulated, our generative approach \synqa outperforms \synatt significantly in all but one case. Additionally, \synqa surpasses zero-shot LLMs that are orders of magnitude larger, showing that we can train a model that is both more accurate and efficient.

\subsubsection{Comparison to Models Trained on Gold Attribution Data}\label{sec:experiments-gold}

\begin{table*}[t]
\centering
\resizebox{1.0\textwidth}{!}{
\begin{tabular}{lccccccccccccccc} \toprule
\multirow{3}{*}{Model} & \multirow{3}{*}{Training data} 

& \multicolumn{6}{c}{\textbf{In-Domain}} 
& \multicolumn{6}{c}{\textbf{Out-of-Domain}} \\ \cmidrule(lr){3-8} \cmidrule(lr){9-14}

& & \multicolumn{3}{c}{\squadcolor{SQuAD}} & \multicolumn{3}{c}{\hotpotcolor{HotpotQA}} 
& \multicolumn{3}{c}{\quaccolor{QuAC-ST}} & \multicolumn{3}{c}{\coqacolor{CoQA-ST}} \\ \cmidrule(lr){3-5} \cmidrule(lr){6-8} \cmidrule(lr){9-11} \cmidrule(lr){12-14}

& & P & R & F1 & P & R & F1 & P & R & F1 & P & R & F1 \\ \midrule
\textbf{\textit{Baselines}} \\
Llama | 1B & Zero-shot & 37.5 & 62.0 & 46.7 & 5.3 & 28.1 & 8.9 & 8.8 & 65.4 & 15.4 & 11.9 & 52.8 & 19.4 \\
Llama | 1B & \squadcolor{SQuAD} (1P) & 98.4 & 98.4 & 98.4 & 48.7 & 20.0 & 28.4 & 92.6 & 85.8 & 89.0 & 79.9 & 64.3 & 71.2 \\
Llama | 1B & \hotpotcolor{HotpotQA} (1P) & 41.3 & 87.3 & 56.0 & 87.5 & 79.9 & 83.5 & 45.2 & 89.9 & 60.1 & 41.0 & 70.9 & 52.0 \\
Llama | 1B & \squadcolor{SQuAD} \& \hotpotcolor{HotpotQA} (1P) & 98.3 & 98.3 & 98.3 & \textbf{89.7} & 78.9 & 84.0 & 90.4 & 90.0 & 90.2 & 83.1 & 68.0 & 74.8 \\
Llama | 1B & \squadcolor{SQuAD} \& \hotpotcolor{HotpotQA} (1M) & \textbf{98.3} & \textbf{98.4} & \textbf{98.3} & 87.0 & \textbf{85.2} & \textbf{86.1} & 84.0 & 89.2 & 86.6 & 79.2 & 66.4 & 72.2 \\ \midrule
\textbf{\textit{Ours}} \\
Llama | 1B & \syntheticcolor{\synqa} & 96.0 & 96.2 & 96.1 & \underline{89.6} & 69.4 & 78.2 & \underline{93.3} & 89.1 & \underline{91.1} & 82.3 & 68.5 & \underline{74.8} \\
Llama | 1B & \syntheticcolor{\synqa} \& \squadcolor{SQuAD} \& \hotpotcolor{HotpotQA} & \underline{98.2} & \underline{98.3} & \underline{98.2} & 89.3 & \underline{82.4} & \underline{85.8} & \textbf{94.5} & \textbf{92.7} & \textbf{93.6} & \textbf{85.5} & \textbf{71.0} & \textbf{77.6} \\
\bottomrule
\end{tabular}
}
\caption{Comparison of models fine-tuned on synthetic vs.~gold in-domain data. Our \synqa approach generalizes better while remaining competitive in-domain. \textbf{Bold} denotes best method, \underline{underline} our method when second best. 1P: models trained with a single pass over the training data. 1M: models trained with 1M samples to match the size of the \synqa data.}
\label{table:fine-tuned-models}
\end{table*}

In Table~\ref{table:fine-tuned-models}, we compare models trained on synthetic and gold in-domain context-attribution datasets. As expected, fine-tuning on in-domain gold datasets (SQuAD and HotpotQA) yields highly specialized models that perform well on in-domain data.
% The performance on the out-of-domain datasets is comparable to Llama 70B, the best zero-shot LLM.
% In contrast, \synqa models outperform Llama 70B on out-of-domain datasets while also achieving near identical scores on the in-domain datasets.
However, models trained on data obtained by \synqa exhibit competitive performance on in-domain tasks and consistently surpass in-domain-trained models on out-of-domain datasets. 
This strong out-of-domain generalization is crucial for practical deployments, where models must handle diverse, previously unseen contexts that often differ substantially from their training data.

\subsubsection{Comparison to Zero-Shot and Fine-Tuned Models in Dialogue Contexts}\label{sec:experiments-dialog}
% \begin{table}[t]
% \centering
% \resizebox{1.0\columnwidth}{!}{
% \begin{tabular}{lccccccc} 
% \toprule

% \multirow{2}{*}{Model} & \multirow{2}{*}{Training data} & \multicolumn{3}{c}{\quaccolor{QuAC}} & \multicolumn{3}{c}{\coqacolor{CoQA}} \\ 
% \cmidrule(lr){3-5} \cmidrule(lr){6-8}

%  &  & P & R & F1 & P & R & F1 \\ 
% \midrule
% \textbf{\textit{Baselines}} \\
% Llama | 1B & Zero-shot & 20.9 & 47.9 & 29.1 & 35.6 & 40.2 & 37.8 \\
% Mistral | 7B & Zero-shot & 64.9 & 83.9 & 73.2 & 54.4 & 64.9 & 59.2 \\
% Llama | 8B & Zero-shot & 81.4 & 89.0 & 85.0 & 77.8 & 72.1 & 74.8 \\
% Mistral NeMo | 12B & Zero-shot & 84.8 & 85.4 & 85.1 & 81.7 & 68.4 & 74.5 \\
% \midrule
% Llama | 1B & \squadcolor{SQuAD} \& \hotpotcolor{HotpotQA} (1P) & 72.9 & 68.0 & 70.3 & 79.3 & 64.4 & 71.0 \\
% Llama | 1B & \squadcolor{SQuAD} \& \hotpotcolor{HotpotQA} (1M) & 56.0 & 49.0 & 52.3 & 63.0 & 51.2 & 56.5 \\ 
% \midrule
% \textbf{\textit{Ours}} \\
% Llama | 1B & \syntheticcolor{\synqa} & \textbf{91.3} & \underline{91.4} & \underline{91.3} & \underline{81.7} & \underline{71.4} & \underline{76.2} \\
% Llama | 1B & \syntheticcolor{\synqa} \& \squadcolor{SQuAD} \& \hotpotcolor{HotpotQA} & \underline{91.1} & \textbf{92.3} & \textbf{91.7} & \textbf{82.3} & \textbf{73.2} & \textbf{77.5} \\
% \bottomrule
% \end{tabular}
% }
% \caption{Context attribution on QuAC and CoQA (dialog data); both datasets are out-of-domain. Despite the size advantage of zero-shot LLMs, our \synqa models outperform fine-tuned and larger zero-shot models. \textbf{Bold} denotes best method, \underline{underline} our method when second best.}
% \label{table:dialog-datasets}
% \end{table}

\begin{table*}[t]
\centering
\resizebox{1.0\textwidth}{!}{
\begin{tabular}{lcccccccccccccc} 
\toprule

\multirow{3}{*}{Model} & \multirow{3}{*}{Training data} 

& \multicolumn{12}{c}{\textbf{Out-of-Domain}} \\ \cmidrule(lr){3-14}

& & \multicolumn{3}{c}{\quaccolor{QuAC}} & \multicolumn{3}{c}{\coqacolor{CoQA}} 
& \multicolumn{3}{c}{\orquaccolor{OR-QuAC}} & \multicolumn{3}{c}{\doqacolor{DoQA}} \\ 

\cmidrule(lr){3-5} \cmidrule(lr){6-8} \cmidrule(lr){9-11} \cmidrule(lr){12-14}

 &  & P & R & F1 & P & R & F1 & P & R & F1 & P & R & F1 \\ 
\midrule
\textbf{\textit{Baselines}} \\
Llama | 1B & Zero-shot & 30.8 & 45.5 & 36.8 & 39.4 & 37.9 & 38.6 & 33.0 & 46.6 & 38.6 & 12.2 & 22.6 & 15.9 \\
Mistral | 7B & Zero-shot & 76.6 & 81.8 & 79.1 & 67.6 & 61.3 & 64.3 & 82.5 & 85.1 & 83.8 & 74.9 & 77.9 & 76.4 \\
Llama | 8B & Zero-shot & 84.7 & 88.8 & 86.7 & 79.3 & 72.0 & 75.5 & 88.0 & 91.3 & 89.6 & 77.9 & 91.4 & 84.1 \\
Mistral-NeMo | 12B & Zero-shot & 85.7 & 85.4 & 85.5 & 81.9 & 68.4 & 74.5 & 88.9 & 88.8 & 88.8 & 86.0 & 84.2 & 85.1 \\
Llama | 70B & Zero-shot & 88.5 & 87.7 & 88.1 & \textbf{88.3} & \textbf{74.9} & \textbf{81.1} & 81.7 & 86.3 & 83.9 & 85.2 & 82.0 & 83.5 \\
\midrule
\textbf{\textit{Baselines}} \\
Llama | 1B & \squadcolor{SQuAD} \& \hotpotcolor{HotpotQA} (1P) & 71.3 & 66.8 & 69.0 & 79.0 & 64.2 & 70.8 & 61.6 & 57.5 & 59.5 & 67.4 & 57.8 & 62.2 \\
Llama | 1B & \squadcolor{SQuAD} \& \hotpotcolor{HotpotQA} (1M) & 52.6 & 49.3 & 50.9 & 61.2 & 50.2 & 55.2 & 48.5 & 44.6 & 46.5 & 53.2 & 49.1 & 51.1 \\ \midrule
\textbf{\textit{Ours}} \\
Llama | 1B & \syntheticcolor{\synqa} & \textbf{91.3} & \underline{91.4} & \underline{91.3} & 81.7 & 71.4 & 76.2 & \textbf{92.6} & \underline{95.3} & \textbf{94.0} & \textbf{86.3} & \underline{94.5} & \textbf{90.2} \\
Llama | 1B & \syntheticcolor{\synqa} \& \squadcolor{SQuAD} \& \hotpotcolor{HotpotQA} & \underline{91.1} & \textbf{92.2} & \textbf{91.7} & \underline{82.3} & \underline{73.2} & \underline{77.5} & \underline{90.3} & \textbf{96.4} & \underline{93.2} & \underline{85.1} & \textbf{96.0} & \textbf{90.2} \\
\bottomrule
\end{tabular}
}
\caption{Context attribution on QuAC, CoQA, OR-Quac, and DoQA (dialogue data); all datasets are out-of-domain. Despite the size advantage of zero-shot LLMs, our \synqa models outperform fine-tuned and larger zero-shot models. \textbf{Bold} denotes best method, \underline{underline} our method when second best. 1P: models trained with a single pass over the training data. 1M: models trained with 1M samples to match the size of the \synqa data.}
\label{table:dialog-datasets}
\end{table*}

We evaluate dialogue context attribution, for which we do not use any gold in-domain training data (Tab.~\ref{table:dialog-datasets}).
% exists.
Here, models must handle follow-up questions that rely on previous turns, often involving coreferences and other dialogue-specific complexities. As expected, zero-shot LLMs exhibit a strong size-performance correlation, with larger models consistently outperforming smaller ones—even those fine-tuned on single-turn question-answer attribution (trained on gold SQuAD and HotpotQA data). However, fine-tuning smaller models with our synthetic data generation strategy leads to superior performance, surpassing both their fine-tuned counterparts and much larger zero-shot LMs. This demonstrates the effectiveness of \synqa in enhancing context attribution in a dialogue setting and without requiring in-domain supervision.

\subsubsection{Scaling Trends and Generalization Performance}\label{sec:scalling-trends}

\begin{figure*}[t]
    \centering
    \begin{subfigure}{0.48\linewidth}
        \centering
        \includegraphics[width=\linewidth]{img/size_performance.pdf}
        \caption{Model performance vs.~size.}
        \label{fig:size_performance}
    \end{subfigure}
    \hfill
    \begin{subfigure}{0.48\linewidth}
        \centering
        \includegraphics[width=\linewidth]{img/data_quantity.pdf}
        \caption{F1 score vs.~training data size.}
        \label{fig:data_quantity}
    \end{subfigure}
    \caption{Comparison of model performance and scalability. (a) Larger zero-shot models achieve good F1 scores, but our method \synqa (based on Llama 1B) outperforms them while being orders of magnitude smaller. (b) Performance improves consistently with more \synqa training data, highlighting its scalability.}
    \label{fig:combined}
\end{figure*}

Fig.~\ref{fig:size_performance} shows F1 scores averaged across datasets, with model size on the x-axis and performance on the y-axis. Models trained on \synqa-generated data significantly outperform their baseline zero-shot counterparts, while also achieving superior performance compared to zero-shot LLMs that are orders of magnitude larger. This shows our method is highly data-efficient, enabling small models to close the gap with much larger counterparts.



In Figure~\ref{fig:data_quantity}, we analyze model performance as the quantity of synthetic training data increases, reporting F1 scores separately for in-domain and out-of-domain datasets. As we scale data quantity, performance improves consistently across datasets for isolated context attribution. This trend highlights the scalability of our approach, indicating that further gains can be achieved by increasing synthetic data availability.
%Notably, despite the lack of direct supervision on in-domain datasets, more data results in improved performance.%, reinforcing the robustness of our method.

\subsubsection{User Study: \synqa increases efficiency and accuracy assessment}\label{sec:user-study}
We conducted a user study to evaluate the efficiency and accuracy of verifying the correctness of LLM-generated answers using context attribution. Our hypothesis is that higher-quality context attributions, visualized to guide users, facilitate faster and more accurate verification of LLM outputs. Specifically, in each trial, we presented users with a question, a generated answer, and relevant context, along with
% context
attributions visualized as highlights. Their task was to leverage these attributions to judge if the answer was correct w.r.t.~a provided context. See Figure~\ref{fig:user_interface} in Appendix~\ref{app:user_study}.
% for an example. %Appendix~\ref{app:user_study} for an example.

The study compares three scenarios:
\begin{inparaenum}[(i)]
\item 
\textbf{No Alignment:} a baseline condition without context attributions, requiring users to manually read and verify the answer against the entire context;
\item 
\textbf{Llama 1B (Zero-shot):} context attributions generated by the Llama 1B model were visualized;
\item 
\textbf{\synqa}: context attributions generated by our approach were visualized.
\end{inparaenum}

We employed a within-subjects experimental design for our human evaluation (with 12 participants), ensuring that the same participants evaluate all the aforementioned alignment scenarios, thus requiring fewer participants for reliable results \cite{greenwald:1976}. However, this can be susceptible to learning effects where participants perform better in later scenarios, because they learned the task from previous examples. To mitigate this, we counterbalanced the scenario order using a Latin Square design \cite{belz:2010,bradley:1958}, where each alignment scenario appears in each position an equal number of times across all participants. Finally, we randomized the example order within each scenario per participant. For each example, we measured: \textbf{verification time} (seconds from display to judgment submission) and \textbf{verification accuracy} (binary correct/incorrect judgment).

\begin{figure}[hb!]
    \centering
    \includegraphics[width=1.0\linewidth]{img/user_study_big_font.png}
    \caption{Relationship between Evaluation Time (seconds) and Accuracy (\%) for three answer verification settings:  \emph{Llama 1B (Zero-shot)}, \emph{No Alignment} and \synqa. \synqa demonstrates the lowest evaluation time and highest accuracy, indicating its superior performance in facilitating efficient and accurate answer verification.}
    \label{fig:user_study}
\end{figure}

\noindent \textbf{Results.} We observed a clear trend in verification performance across the different attribution settings, with \synqa demonstrating superior effectiveness (Fig.~\ref{fig:user_interface}). \synqa has the lowest average verification time per example (\textbf{148.6} seconds), significantly faster than \emph{No Alignment} (171.8 seconds) and attributions from \emph{Llama 1B} (163.4 seconds). Concurrently, in terms of verification accuracy, \synqa achieved the highest average accuracy (\textbf{86.4\%}). While \emph{No Alignment} (84.1\%) and \emph{Llama 1B (77.3\%)} also yielded reasonable accuracy, attributions from \synqa are clearly of higher quality helping users be more accurate.


% \begin{table*}[t]
% \centering
% \resizebox{1.0\textwidth}{!}{
% \begin{tabular}{lccccccccccccccc} \toprule
% \multirow{2}{*}{Model} & \multirow{2}{*}{Training data} & \multicolumn{3}{c}{\squadcolor{Squad}} & \multicolumn{3}{c}{\hotpotcolor{HotPot QA}} & \multicolumn{3}{c}{\quaccolor{Quac}} & \multicolumn{3}{c}{\coqacolor{CoQA}} \\ \cmidrule(lr){3-5} \cmidrule(lr){6-8} \cmidrule(lr){9-11} \cmidrule(lr){12-14}
% & & P & R & F1 & P & R & F1 & P & R & F1 & P & R & F1 \\ \midrule
% Random & -- & 19.8 & 15.4 & 17.3 & 4.8 & 15.2 & 7.3 & 5.2 & 15.1 & 7.7 & 7.3 & 15.1 & 9.9 \\
% E5 | 561M & Zero-shot & 38.1 & 76.5 & 50.9 & 12.4 & 41.4 & 19.1 & 65.0 & 73.8 & 69.1 & 61.1 & 15.2 & 24.4 \\
% HF-SmolLM2 | 135M & Zero-shot & X & X & X & X & X & X & X & X & X & X & X & X \\
% HF-SmolLM2 | 365M & Zero-shot & 28.1 & 46.4 & 35.0 & 5.1 & 7.3 & 6.0 & 10.6 & 22.6 & 14.4 & 10.6 & 21.5 & 14.2 \\
% Llama | 1B & Zero-shot & 37.5 & 62.0 & 46.7 & 5.3 & 28.1 & 8.9 & 8.8 & 65.4 & 15.4 & 11.9 & 52.8 & 19.4 \\ %\midrule
% Mistral | 7B & Zero-shot & 71.5 & 94.4 & 81.4 & 42.9 & 42.7 & 42.8 & 63.2 & 88.6 & 73.8 & 59.0 & 72.2 & 64.9 \\
% Llama | 8B & Zero-shot & 71.9 & 96.9 & 82.6 & 49.2 & 52.9 & 51.0 & 64.1 & 92.1 & 75.6 & 55.7 & 76.4 & 64.4 \\
% Mistral NeMo | 12B & Zero-shot & 89.5 & 94.5 & 91.8 & 46.4 & 47.3 & 46.8 & 81.8 & 85.3 & 83.5 & 79.0 & 67.2 & 72.6 \\
% Ensemble | 27B & Zero-shot & 83.1 & 96.3 & 89.2 & 48.1 & 59.6 & 53.2 & 74.8 & 90.3 & 81.8 & 69.5 & 73.6 & 71.5 \\
% Llama | 70B & Zero-shot & 95.3 & 95.6 & 95.5 & 87.6 & 37.5 & 52.5 & 89.7 & 87.8 & 88.7 & 87.5 & 73.3 & 79.8 \\
% \midrule
% Llama | 1B & \squadcolor{SQ} \& \hotpotcolor{HP}; Disc. synthetic (1 pass) & 89.8 & 96.5 & 93.0 & 50.6 & 58.6 & 54.3 & 64.9 & 91.5 & 75.9 & 53.1 & 75.5 & 62.3 \\
% Llama | 1B & \squadcolor{SQ} \& \hotpotcolor{HP}; Disc. synthetic (1.0M) & 84.3 & 96.9 & 90.2 & 54.4 & 58.0 & 56.1 & 63.4 & 92.4 & 75.2 & 52.5 & 77.5 & 62.6 \\ \midrule
% Llama | 1B & \synqa (130K no dist.) & 87.1 & 88.2 & 87.6 & 67.9 & 44.8 & 54.0 & 85.3 & 82.5 & 83.9 & 72.7 & 63.8 & 68.0 \\
% Llama | 1B & \syntheticcolor{\synqa (130K)} & 89.9 & 90.7 & 90.3 & 87.9 & 63.5 & 73.7 & 88.7 & 85.4 & 87.0 & 77.2 & 65.5 & 70.9 \\
% Llama | 1B & \syntheticcolor{\synqa (550K)} & 93.6 & 94.5 & 94.0 & 88.9 & 67.3 & 76.6 & 89.8 & 87.9 & 88.9 & 77.1 & 67.1 & 71.8 \\
% Llama | 1B & \syntheticcolor{\synqa (700K)} & 95.1 & 95.5 & 95.3 & 87.8 & 69.6 & 77.6 & 93.6 & 89.1 & 91.3 & 82.0 & 68.9 & 74.9 \\
% Llama | 1B & \syntheticcolor{\synqa (1.0M)} & 96.0 & 96.2 & 96.1 & 89.6 & 69.4 & 78.2 & 93.3 & 89.1 & 91.1 & 82.3 & 68.5 & 74.8 \\
% New & \syntheticcolor{\synqa (1.0M)} & 95.7 & 96.9 & 96.3 & 89.6 & 66.4 & 76.3 & 91.8 & 91.5 & 91.6 & 80.8 & 71.3 & 75.8 \\
% Llama | 1B & \syntheticcolor{\synqa (1.0M with dialog data)} & -- & -- & -- & -- & -- & -- & -- & -- & -- & -- & -- & -- & \\
% \midrule
% HF-SmolLM2 | 135M & \synqa (690K) & X & X & X & X & X & X & X & X & X & X & X & X \\
% HF-SmolLM2 | 365M & \synqa (700K) & 73.9 & 74.2 & 74.1 & 85.2 & 68.7 & 76.0 & 79.6 & 76.5 & 78.0 & 68.8 & 59.8 & 64.0 \\ \midrule
% Llama | 1B & \squadcolor{SQ}; Gold (1 pass) & 98.4 & 98.4 & 98.4 & 48.7 & 20.0 & 28.4 & 92.6 & 85.8 & 89.0 & 79.9 & 64.3 & 71.2 \\
% Llama | 1B & HQ; Gold (1 pass) & 41.3 & 87.3 & 56.0 & 87.5 & 79.9 & 83.5 & 45.2 & 89.9 & 60.1 & 41.0 & 70.9 & 52.0 \\
% Llama | 1B & \squadcolor{SQ} \& \hotpotcolor{HQ}; Gold (1 pass) & 98.3 & 98.3 & 98.3 & 89.7 & 78.9 & 84.0 & 90.4 & 90.0 & 90.2 & 83.1 & 68.0 & 74.8 \\
% Llama | 1B & \squadcolor{SQ} \& \hotpotcolor{HQ}; Gold (1.0M) & 98.3 & 98.4 & 98.3 & 87.0 & 85.2 & 86.1 & 84.0 & 89.2 & 86.6 & 79.2 & 66.4 & 72.2 \\
% Llama | 1B & \syntheticcolor{\synqa (1.0M)} \& \squadcolor{SQ} \& \hotpotcolor{HQ}; Gold (1 pass) & 98.3 & 98.3 & 98.3 & 86.4 & 84.1 & 85.2 & 96.6 & 90.2 & 93.3 & 86.0 & 69.4 & 76.8 \\
% Llama | 1B & \syntheticcolor{\synqa (1.0M)} \& \squadcolor{SQ} \& \hotpotcolor{HQ}; Gold (1 pass) & 98.2 & 98.3 & 98.2 & 89.3 & 82.4 & 85.8 & 94.5 & 92.7 & 93.6 & 85.5 & 71.0 & 77.6 \\
% \midrule
% Llama | 1B & \synqa (1.0M) \& \squadcolor{SQ} \& \hotpotcolor{HQ} \& \quaccolor{Q} \& \coqacolor{CQ}; Gold (1 pass) & -- & -- & -- & -- & -- & -- & -- & -- & -- & -- & -- & -- & \\ \midrule
% HF-Smol | 365M & \syntheticcolor{\synqa (1.0M)} \& \squadcolor{SQ} \& \hotpotcolor{HQ}; Gold (1 pass) & 98.1 & 98.2 & 98.2 & 83.4 & 83.2 & 83.3 & 80.0 & 90.5 & 84.9 & 71.7 & 67.4 & 69.5 \\
% HF-Smol | 365M & \syntheticcolor{\synqa (1.0M)} \& \squadcolor{SQ} \& \hotpotcolor{HQ} \& \quaccolor{Q} \& \coqacolor{CQ}; Gold (1 pass) & 98.0 & 98.1 & 98.1 & 86.8 & 81.7 & 84.2 & 96.6 & 92.9 & 94.7 & 85.6 & 77.0 & 81.1 \\
% HF-Smol | 135M & \syntheticcolor{\synqa (1.0M)} \& \squadcolor{SQ} \& \hotpotcolor{HQ} \& \quaccolor{Q} \& \coqacolor{CQ}; Gold (1 pass) & 98.0 & 98.0 & 98.0 & 77.8 & 76.3 & 77.0 & 95.0 & 91.6 & 93.3 & 81.2 & 71.8 & 76.2 \\
% \bottomrule
% Llama | 1B & All; Gold & 1B   & 96.8 & 96.8 & 96.8 & 88.1 & 83.8 & 85.9 & 94.7 & 89.3 & 91.9 & 88.7 & 76.8 & 82.3 \\
% HF-SmolLM2 & All; Gold & 360M & 98.3 & 98.4 & 98.3 & 85.1 & 78.2 & 81.5 & 96.6 & 92.4 & 94.5 & 88.3 & 74.6 & 80.8 \\ \bottomrule
% \end{tabular}
% }
% \caption{Corroborative context-attribution of LMs on Squad QA, HotPot QA, Quac, and CoQA.}
% \label{table:all-datasets}
% \end{table*}
\section{Human Experiments}
\label{sec:human}

LLMs are trained on text produced by humans and are able to generate plausible text; therefore, there have been interests in using LLMs as human models \interalia{eisape-etal-2024-systematic,misra-kim-2024-generating}.
Following this line of work, we conduct a human behavioral experiment to ground the LLM reasoning behavior.
Using samples from our primary dataset, we collected 710 responses from adults fluent in English through Prolific.\footnote{\url{https://prolific.com}}
More experiment details can be found in \cref{subsec:human-details}.

The average human accuracy on each group is shown in the last row of \cref{tab:softacc-base}.\footnote{Human responses are binary classes, so correct and incorrect responses are coded as $1$ and $0$, respectively.}
Aligned with our LLM results (\cref{sec:experiment}), on modalities, the overall human results also show an accuracy order of ($\Diamond \succ \varnothing \succ \Box$),
and on argument forms, modus ponens ($\to^\mathrm{L}$) is the most accurately answered pattern.

To further investigate the interactions of logic factors, we fit a generalized linear mixed-effects model \citep{batesFittingLinearMixedEffects2015} to verify the effect of modality and argument forms on human logic reasoning accuracy (\cref{eqn:mixed-effects-human} and \cref{fig:emmeans-human}).
\noindent
\begin{align}
    \mathrm{logit}(\mathit{Acc}) & \sim \textit{Modality} + \textit{ArgForm} + \textit{Rt} \nonumber \\
                                 & + (1 + \textit{Rt} \mid \textit{ParticipantID}),
    \label{eqn:mixed-effects-human}
\end{align}
\noindent
where $\mathit{Acc}$ is the binary accuracy of human responses, and $\textit{Rt}$ is the response time.
The generalized mixed-effects model yields a marginal $R^2$ of $0.121$ yet a $0.419$ conditional $R^2$, indicating a diverse response pattern across participants.
The likelihood ratio test on the full model against the null model shows that only the effect of argument form is significant ($\chi^2(2)=25.6$, $p<0.001$).
However, in accordance with the overall performance, we find modus ponens ($\to^\mathrm{L}$) has a significantly higher effect than the other two valid argument forms.
This confirms that logical forms can also have a significant impact on human reasoning accuracy, which is consistent with the LLM results, although the effect sizes are not the same.

\begin{figure}[!t]
    \centering
    \vspace{-5pt}
    \includegraphics[
        width=0.95\columnwidth,
        keepaspectratio,
    ]{emmeans-human.pdf}
    \vspace{-5pt}
    \caption{
        Estimated marginal means of logical form factors in the generalized mixed-effects model of \cref{eqn:mixed-effects-human}, along with their 95\% confidence intervals.
        \label{fig:emmeans-human}
    }
    \vspace{-10pt}
\end{figure}

\section{Conclusion and Discussion}
\label{sec:discussion}
We present an analysis of hypothetical and disjunctive syllogisms on propositional and modal logic and systematically analyze the LLM performance on the dataset.
Our analysis provides novel insights on explaining and predicting LLM performance: in addition to the perplexity or probability of the input text, the underlying logic forms play an important role in determining the performance of LLMs.
In addition, we compare the behaviors of LLMs and humans using the same data through human behavioral experiments.
We discuss the implications of our results as follows.

\vspace{2pt}
\noindent\textbf{Probability in language models.}
Probability and perplexity are often used as intrinsic evaluation metrics for language models.
While \citet{gonen-etal-2023-demystifying} and \citet{mccoyEmbersAutoregressionShow2024} show that probability and perplexity correlate well with LLM performance, literature in program synthesis with LLMs shows little correlation between probability and execution-based evaluation results \citep{li2022competition,shi-etal-2022-natural}.
This work does not necessarily contradict either line but rather provides complementary factors for analyzing LLM performance.

We argue that probability may have become an overloaded term in analyzing LLMs.
Low probability may be due to one or more of the following non-exhaustive reasons: (1) out-of-context content, (2) ungrammatical language, or (3) grammatical but semantically awkward content (cf. the mirror dataset in \cref{sec:perplexity}), (4) reasonable but rare content.
We hypothesize that the probability of language models may not be essentially able to capture all these nuanced differences, and call for encoding and decoding algorithms---such as \citet{meister-etal-2023-locally}---that can better decompose the probability into finer-grained and explainable components.

\vspace{2pt}
\noindent\textbf{Comparing humans and LLMs.}
What is our goal for building LLMs?
To achieve better performance on practical tasks or to build a more human-like model?
Our results, together with \citet{eisape-etal-2024-systematic}, suggest that these two goals may not be perfectly aligned by revealing a mixture of similarity and discrepancy between LLMs and humans---for example, while LLMs exhibit higher benchmark performance than humans on our dataset and show the same argument form preferences with humans (\cref{fig:emmeans-lm,fig:emmeans-human}), they also show systematic biases that we do not find significant in human reasoning (e.g., disfavoring the necessity modality, \cref{subsec:affirmation-bias}).
While there has been positive evidence of using LLMs as human models in psycholinguistic studies \interalia{misra-kim-2024-generating}, our results suggest executing such approaches cautiously.

\vspace{2pt}
\noindent\textbf{On the relation between modality and performance.}
Our results show that there is a significant difference in performance between necessity and possibility modalities, with the former much lower than the latter (\cref{tab:softacc-base}).
Part of the reason for this is that LLMs have a significant tendency to say ``No'' to the necessity modality (\cref{fig:affirmation-rejection}).

On the one hand, our results extend the conclusion of \citet{dentella-etal-2023-systematic} that LLMs generally respond positively---LLM behaviors may be significantly affected by finer-grained factors, including but not necessarily limited to the modality involved in the input.
On the other hand, while LLMs systematically tend to answer ``No'' to questions in necessity modality, we do not find related evidence in human experiments, which leads us to hypothesize that such rejection bias comes from either the model architecture or the training strategies, such as the reinforcement learning with human feedback \citep[RLHF;][]{ouyang-etal-2022-training} protocol.
We leave this as an open question for future research.

\vspace{2pt}
\noindent\textbf{Modal logic and theory of mind.}
Modality, in principle, encodes mental states and beliefs.
The reasoning of beliefs also resonates with the theory of mind \interalia{premackDoesChimpanzeeHave1978,baron-cohenDoesAutisticChild1985} and machine theory of mind \interalia{rabinowitzMachineTheoryMind2018, maHolisticLandscapeSituated2023}.
Following the effort by \citet{sileo-lernould-2023-mindgames} that uses epistemic modal logic to model the machine theory of mind, our work assesses the behaviors of LLMs on alethic modal logic, distantly revealing the future potential of LLMs in achieving the theory of mind.

\section*{Limitations}
This work comes with two major limitations:
\begin{enumerate}[leftmargin=*,topsep=0pt,itemsep=0pt]
      \item While we have verified that our data has a low perplexity ($9.82\pm 2.47$ under mistral-7b; much lower than that of the data by \citet{wanLogicAskerEvaluatingImproving2024}, $25.44$), and, therefore, are similar enough to natural language utterances, the synthetic language cannot fully substitute natural language in daily life.
            Our dataset and analysis are not comprehensive enough to cover many nuanced examples that may appear in real communication, especially when context-dependent understanding is crucial to conveying communication goals.
      \item Despite more than 7,000 languages worldwide, as a first step, our material only covers English.
            This narrow focus is due to the languages the authors are proficient in and the coverage of the language models.
            We acknowledge the importance of extending the scope of this work to a more comprehensive set of languages and leave the extension as an immediate follow-up step.
\end{enumerate}

In addition, the sample size of human experiments is somewhat limited.
We leave more comprehensive human behavioral data collection and analysis to future work.

\section*{Ethics Statement}

While this work involves human logical reasoning experiments, we have ensured that (1) the data are generated procedurally following templates listed in the paper and (2) there is no harmful content in the atomic logical interpretations, reviewed by all the authors.
In addition, we have ensured that all participants are paid a fair wage through the Prolific platform.
Instructions and consent forms delivered to the participants can be found in the \cref{subsec:human-details}.
The institutional ethics review board has approved the data collection process.

This work contributes to the understanding of LLMs.
We do not foresee risk beyond the minimal risk posed by LLM evaluation work.
We acknowledge that using LLMs in real-world scenarios could significantly impact human behaviors, raising the need for model transparency, safety, security, and interpretability.
We will open-source the synthetic logical reasoning dataset upon publication.

\section{Acknowledgements}
We thank Yudong Li for his help in setting up the Gemini and OpenAI API for the experiments.
This work was supported in part by a Google PhD Fellowship and a Canada CIFAR AI Chair award to FS, as well as NSERC RGPIN-2024-04395.


%\freda{We will need to clean the bibliography before submitting it. Marking it as a todo for now.}
\bibliography{custom}

\appendix

\section{Additional Experiment Details}

\subsection{LLM Experiment Details}
\label{subsec:llm-details}
All LLMs used are obtained from \href{https://huggingface.co/models}{Hugging Face} checkpoints.
Time and compute power requirements vary, the largest llama-3-70b model takes around 2 hours on NVIDIA A6000 GPU to obtain all results in \cref{sec:experiment}.

\subsection{Human Experiment Details}
\label{subsec:human-details}
\paragraph{Participant instructions.}
We use keys \textit{F} and \textit{J}, which are roughly symmetric on a standard English keyboard, to collect participant responses.
Half of the participants see the following instruction:

\textit{In this study, you will be presented with two statements followed by a question. Your task is to answer either Yes or No to the question, based on the information provided in the statements.
    Please respond quickly and accurately by pressing "F" for Yes, and "J" for No.}

To mitigate the possible bias introduced by the dominant hand, we have the other half of the participants see instruction with reversed keys:

\textit{In this study, you will be presented with two statements followed by a question. Your task is to answer either Yes or No to the question, based on the information provided in the statements.
    Please respond quickly and accurately by pressing "F" for No, and "J" for Yes.}

\paragraph{Participant wage.}
We offer participants an hourly wage of 1.5 times Prolific's minimum wage.
The duration is determined by the median completion time among all participants.

\setcounter{table}{0}
\setcounter{figure}{0}
\setcounter{equation}{0}
\renewcommand{\thetable}{A\arabic{table}}
\renewcommand{\thefigure}{A\arabic{figure}}
\renewcommand{\theequation}{A\arabic{equation}}

\section{Extra Details of the Dataset}

\subsection{Considerations in Translating Logical Form to Natural Language}
\label{subsec:logic-translate-strategy}

During the interpretation process, another key point is to assign independent interpretations to variables.
Deciding the dependency also involves common sense knowledge.
For example, consider the premises $\lnot p \to q$ and $q$.
If we interpret $p \coloneqq\textit{``Jane is inside the house''}$ and $q\coloneqq\textit{``Jane is out''}$ to proposition variables $p$ and $q$, the two variables are possibly not independent.
According to common sense, ``\textit{Jane is not inside the house}'' ($\lnot p$) correlates with or is even equivalent to ``\textit{Jane is out}'' ($q$).
Logically, $\left\{\lnot p \to q, q\right\} \nvdash \lnot p$; however, with the extra premise $\lnot p \leftrightarrow q$ given by common sense, people may conclude that $\lnot p$.\footnote{
  This confounding factor affects the examples in Appendix C.1.12 of \citet{hollidayConditionalModalReasoning2024}.
}

Besides, natural language is ambiguous---one sentence in natural language can come from multiple logical forms under the same interpretation.
We use present tense and progressive aspect to encourage a reading of imaginary ongoing events, corresponding to the alethic modality.
Such events are less likely to induce LLM's or human's individual bias, as they are unrelated to factual knowledge or moral judgements.
Also, we always use two full verb phrases, ruling out sentences like ``\textit{Jane is eating apples or oranges,}'' so the two events are less likely to be mutually exclusive.
In this way, we can reduce the ambiguity of the questions in our dataset.

\subsection{Data Samples}

All logic forms and corresponding natural language sentences can be found in \cref{tab:question-full}.

The exact prompt format is as follows:

\begin{table}[H]
    \small
    \begin{tabular}{p{\linewidth}}
    Consider the following statements:\verb|\n|\\
    \uline{Jane is watching a show or John is reading a book.}\verb|\n|\\
    \uline{Jane isn't watching a show}.\verb|\n|\\
    Question: Based on these statements, can we infer that \uline{John is reading a book}?\verb|\n|\\
    Answer:\verb|<eof|>
    \end{tabular}
\end{table}
\vspace{-1em}

\newcommand{\subjectA}{\uline{Jane}}
\newcommand{\vpA}{\uline{watching a show}} 
\newcommand{\subjectB}{\uline{John}}
\newcommand{\vpB}{\uline{reading a book}}

\begin{table*}
\scriptsize
\begin{tabular}{
  @{}ccclp{0.52\textwidth}@{}
  }
\vspace{0.5em}
\textbf{Validity} & \textbf{Modality} & \textbf{Argument Form} & \textbf{Logical Form} & \textbf{Natural Language} \\
$\vdash$
& $\varnothing$ & $\lor^{\mathrm{L}}$ & $\{p \lor q, \lnot p\} \vdash q$ & 
   \subjectA{} is \vpA{} or \subjectB{} is \vpB{}.\newline
  \subjectA{} isn't \vpA{}.\newline
  Can we infer that \subjectB{} is \vpB{}? \\
& $\varnothing$ & $\lor^{\mathrm{R}}$ & $\{p \lor q, \lnot q\} \vdash p$ & 
   \subjectA{} is \vpA{} or \subjectB{} is \vpB{}.\newline
  \subjectB{} isn't \vpB{}.\newline
  Can we infer that \subjectA{} is \vpA{}? \\
& $\varnothing$ & $\rightarrow^{\mathrm{L}}$ & $\{\lnot p \to q, \lnot p\} \vdash q$ & 
   If \subjectA{} isn't \vpA{}, then \subjectB{} is \vpB{}.\newline
  \subjectA{} isn't \vpA{}.\newline
  Can we infer that \subjectB{} is \vpB{}? \\
& $\varnothing$ & $\rightarrow^{\mathrm{R}}$ & $\{\lnot p \to q, \lnot q\} \vdash p$ & 
   If \subjectA{} isn't \vpA{}, then \subjectB{} is \vpB{}.\newline
  \subjectB{} isn't \vpB{}.\newline
  Can we infer that \subjectA{} is \vpA{}? \\
& $\Box$ & $\lor^{\mathrm{L}}$ & $\{\Box p \lor \Box q, \lnot \Box p\} \vdash \Box q$ & 
   It's certain that \subjectA{} is \vpA{} or it's certain that \subjectB{} is \vpB{}.\newline
  It's uncertain whether \subjectA{} is \vpA{}.\newline
  Can we infer that it's certain that \subjectB{} is \vpB{}? \\
& $\Box$ & $\lor^{\mathrm{R}}$ & $\{\Box p \lor \Box q, \lnot \Box q\} \vdash \Box p$ & 
   It's certain that \subjectA{} is \vpA{} or it's certain that \subjectB{} is \vpB{}.\newline
  It's uncertain whether \subjectB{} is \vpB{}.\newline
  Can we infer that it's certain that \subjectA{} is \vpA{}? \\
& $\Box$ & $\rightarrow^{\mathrm{L}}$ & $\{\lnot \Box p \to \Box q, \lnot \Box p\} \vdash \Box q$ & 
   If it's uncertain whether \subjectA{} is \vpA{}, then it's certain that \subjectB{} is \vpB{}.\newline
  It's uncertain whether \subjectA{} is \vpA{}.\newline
  Can we infer that it's certain that \subjectB{} is \vpB{}? \\
& $\Box$ & $\rightarrow^{\mathrm{R}}$ & $\{\lnot \Box p \to \Box q, \lnot \Box q\} \vdash \Box p$ & 
   If it's uncertain whether \subjectA{} is \vpA{}, then it's certain that \subjectB{} is \vpB{}.\newline
  It's uncertain whether \subjectB{} is \vpB{}.\newline
  Can we infer that it's certain that \subjectA{} is \vpA{}? \\
& $\Diamond$ & $\lor^{\mathrm{L}}$ & $\{\Diamond p \lor \Diamond q, \lnot \Diamond p\} \vdash \Diamond q$ & 
   It's possible that \subjectA{} is \vpA{} or it's possible that \subjectB{} is \vpB{}.\newline
  It's impossible that \subjectA{} is \vpA{}.\newline
  Can we infer that it's possible that \subjectB{} is \vpB{}? \\
& $\Diamond$ & $\lor^{\mathrm{R}}$ & $\{\Diamond p \lor \Diamond q, \lnot \Diamond q\} \vdash \Diamond p$ & 
   It's possible that \subjectA{} is \vpA{} or it's possible that \subjectB{} is \vpB{}.\newline
  It's impossible that \subjectB{} is \vpB{}.\newline
  Can we infer that it's possible that \subjectA{} is \vpA{}? \\
& $\Diamond$ & $\rightarrow^{\mathrm{L}}$ & $\{\lnot \Diamond p \to \Diamond q, \lnot \Diamond p\} \vdash \Diamond q$ & 
   If it's impossible that \subjectA{} is \vpA{}, then it's possible that \subjectB{} is \vpB{}.\newline
  It's impossible that \subjectA{} is \vpA{}.\newline
  Can we infer that it's possible that \subjectB{} is \vpB{}? \\
& $\Diamond$ & $\rightarrow^{\mathrm{R}}$ & $\{\lnot \Diamond p \to \Diamond q, \lnot \Diamond q\} \vdash \Diamond p$ & 
   If it's impossible that \subjectA{} is \vpA{}, then it's possible that \subjectB{} is \vpB{}.\newline
  It's impossible that \subjectB{} is \vpB{}.\newline
  Can we infer that it's possible that \subjectA{} is \vpA{}? \\

$\nvdash$
& $\varnothing$ & $\lor^{\mathrm{L}}$ & $\{p \lor q, q\} \nvdash \lnot p$ & 
   \subjectA{} is \vpA{} or \subjectB{} is \vpB{}.\newline
  \subjectB{} is \vpB{}.\newline
  Can we infer that \subjectA{} isn't \vpA{}? \\
& $\varnothing$ & $\lor^{\mathrm{R}}$ & $\{p \lor q, p\} \nvdash \lnot q$ & 
   \subjectA{} is \vpA{} or \subjectB{} is \vpB{}.\newline
  \subjectA{} is \vpA{}.\newline
  Can we infer that \subjectB{} isn't \vpB{}? \\
& $\varnothing$ & $\rightarrow^{\mathrm{L}}$ & $\{\lnot p \to q, q\} \nvdash \lnot p$ & 
   If \subjectA{} isn't \vpA{}, then \subjectB{} is \vpB{}.\newline
  \subjectB{} is \vpB{}.\newline
  Can we infer that \subjectA{} isn't \vpA{}? \\
& $\varnothing$ & $\rightarrow^{\mathrm{R}}$ & $\{\lnot p \to q, p\} \nvdash \lnot q$ & 
   If \subjectA{} isn't \vpA{}, then \subjectB{} is \vpB{}.\newline
  \subjectA{} is \vpA{}.\newline
  Can we infer that \subjectB{} isn't \vpB{}? \\
& $\Box$ & $\lor^{\mathrm{L}}$ & $\{\Box p \lor \Box q, \Box q\} \nvdash \lnot \Box p$ & 
   It's certain that \subjectA{} is \vpA{} or it's certain that \subjectB{} is \vpB{}.\newline
  It's certain that \subjectB{} is \vpB{}.\newline
  Can we infer that it's uncertain whether \subjectA{} is \vpA{}? \\
& $\Box$ & $\lor^{\mathrm{R}}$ & $\{\Box p \lor \Box q, \Box p\} \nvdash \lnot \Box q$ & 
   It's certain that \subjectA{} is \vpA{} or it's certain that \subjectB{} is \vpB{}.\newline
  It's certain that \subjectA{} is \vpA{}.\newline
  Can we infer that it's uncertain whether \subjectB{} is \vpB{}? \\
& $\Box$ & $\rightarrow^{\mathrm{L}}$ & $\{\lnot \Box p \to \Box q, \Box q\} \nvdash \lnot \Box p$ & 
   If it's uncertain whether \subjectA{} is \vpA{}, then it's certain that \subjectB{} is \vpB{}.\newline
  It's certain that \subjectB{} is \vpB{}.\newline
  Can we infer that it's uncertain whether \subjectA{} is \vpA{}? \\
& $\Box$ & $\rightarrow^{\mathrm{R}}$ & $\{\lnot \Box p \to \Box q, \Box p\} \nvdash \lnot \Box q$ & 
   If it's uncertain whether \subjectA{} is \vpA{}, then it's certain that \subjectB{} is \vpB{}.\newline
  It's certain that \subjectA{} is \vpA{}.\newline
  Can we infer that it's uncertain whether \subjectB{} is \vpB{}? \\
& $\Diamond$ & $\lor^{\mathrm{L}}$ & $\{\Diamond p \lor \Diamond q, \Diamond q\} \nvdash \lnot \Diamond p$ & 
   It's possible that \subjectA{} is \vpA{} or it's possible that \subjectB{} is \vpB{}.\newline
  It's possible that \subjectB{} is \vpB{}.\newline
  Can we infer that it's impossible that \subjectA{} is \vpA{}? \\
& $\Diamond$ & $\lor^{\mathrm{R}}$ & $\{\Diamond p \lor \Diamond q, \Diamond p\} \nvdash \lnot \Diamond q$ & 
   It's possible that \subjectA{} is \vpA{} or it's possible that \subjectB{} is \vpB{}.\newline
  It's possible that \subjectA{} is \vpA{}.\newline
  Can we infer that it's impossible that \subjectB{} is \vpB{}? \\
& $\Diamond$ & $\rightarrow^{\mathrm{L}}$ & $\{\lnot \Diamond p \to \Diamond q, \Diamond q\} \nvdash \lnot \Diamond p$ & 
   If it's impossible that \subjectA{} is \vpA{}, then it's possible that \subjectB{} is \vpB{}.\newline
  It's possible that \subjectB{} is \vpB{}.\newline
  Can we infer that it's impossible that \subjectA{} is \vpA{}? \\
& $\Diamond$ & $\rightarrow^{\mathrm{R}}$ & $\{\lnot \Diamond p \to \Diamond q, \Diamond p\} \nvdash \lnot \Diamond q$ & 
   If it's impossible that \subjectA{} is \vpA{}, then it's possible that \subjectB{} is \vpB{}.\newline
  It's possible that \subjectA{} is \vpA{}.\newline
  Can we infer that it's impossible that \subjectB{} is \vpB{}? \\
\end{tabular}
\caption{\textbf{Samples of all logical forms and corresponding natural language sentences.}}
\label{tab:question-full}
\end{table*}

\section{Additional Experiments}

\iffalse

\subsection{Supplementary Figures for \cref{sec:experiment}}

\cref{fig:heatmap-modality-full} shows exaustive test on every contrasitive pair of modal words.
Additionally, modality’s effect on the next-token probabilities of answer token \texttt{No} is shown in \cref{fig:heatmap-modality-no}.

\cref{fig:corr-ppl-acc-full} extends \cref{fig:corr-ppl-acc} to correlation between perplexity and accuracy for all models.

\begin{figure*}
    \centering
    \begin{subfigure}[b]{0.48\textwidth}
        \centering
        \includegraphics[width=1\textwidth,keepaspectratio]{heatmap-modality-full-yes.pdf}
        \caption{
            On token \texttt{Yes}.
        }
    \end{subfigure}
    \begin{subfigure}[b]{0.48\textwidth}
        \centering
        \includegraphics[width=1\textwidth,keepaspectratio]{heatmap-modality-full-no.pdf}
        \caption{
            On token \texttt{No}.
        }
        \label{fig:heatmap-modality-no}
    \end{subfigure}
    \caption{
        \textbf{Continuation of \cref{fig:heatmap-modality}}.
    }
    \label{fig:heatmap-modality-full}
\end{figure*}

\begin{figure*}
    \centering
    \includegraphics[width=1\textwidth,keepaspectratio]{corr-ppl-acc-full.pdf}
    \caption{
        \textbf{Continuation of \cref{fig:corr-ppl-acc}}.
    }
    \label{fig:corr-ppl-acc-full}
\end{figure*}

\fi


\subsection{Extra Experiment: Introduction Rule of Modality}
\label{subsec:extra-intro-modality}

We report the results on the necessitation rule and its variants here, as these rules are obscure and verbose to be articulated in natural language:
%
\begin{align}
    \left\{\varphi\right\} &\vdash \Box \varphi, \tag{necessitation rule} \label{eqn:necessitation-rule} \\
    \left\{\varphi\right\} &\vdash \Diamond \varphi, \nonumber \\
    \left\{\varphi\right\} &\vdash \varphi. \nonumber
\end{align}
%
Its natural language form is as follows:
\begin{table}[H]
    \small
    \begin{tabular}{rp{0.8\linewidth}}
    & Jane is watching a show.\\
    ($\Box$) & Can we infer that it's certain that Jane is watching a show?\\
    ($\Diamond$) & Can we infer that it's possible that Jane is watching a show?\\
    ($\varnothing$) & Can we infer that Jane is watching a show?\\
    \end{tabular}
\end{table}

All three variants are paired with 1000 logic interpretations.
As they are all rules of inference, the ground truth answer is always \texttt{Yes}.
Overall accuracy is shown in \cref{tab:softacc-necessitate},
where across all LLMs, the necessitation rule has the lowest accuracy.
This echoes the necessity modality's tendency to be rejected discussed in \cref{subsec:affirmation-bias}.

\begin{table}[t]
    
\small
\centering
\begin{tabular*}{0.85\columnwidth}{
 @{\extracolsep{\fill}}lccc@{}
 }
 \toprule
  & $\varnothing$ & $\Box$ & $\Diamond$ \\
 \midrule
    mistral-7b & 0.998 & 0.885 & 0.999\\
    mistral-8x7b & 0.957 & 0.540 & 0.987\\
    llama-2-7b & 0.768 & 0.013 & 0.920\\
    llama-2-13b & 0.368 & 0.004 & 0.829\\
    llama-2-70b & 0.511 & 0.051 & 0.834\\
    llama-3-8b & 0.398 & 0.225 & 0.783\\
    llama-3-70b & 0.674 & 0.384 & 0.794\\
    yi-34b & 0.960 & 0.382 & 0.999\\
    phi-2 & 0.814 & 0.226 & 0.892\\
    phi-3-mini & 0.992 & 0.925 & 0.994\\
 \bottomrule
\end{tabular*}
    \caption{
        \label{tab:softacc-necessitate}
        Overall accuracy of the necessitation rule and its modality variants on each model.
    }
\end{table}
    
We further fit a linear mixed-effects model similar to \cref{eqn: mixed-effects}, except that the argument form effect is now constant across all data points.
The mixed-effects model yields a marginal $R^2$ of $0.391$ and a conditional $R^2$ of $0.745$.
Estimated marginal means shows that the accuracy on $\varnothing$ is $0.171$ less than $\Diamond$, but $0.371$ higher than $\Box$, with both differences significant at $p < 0.0001$.
This further suggests that modality serves as an important factor on logic reasoning performance.

\subsection{Extra Experiment: Distribution of Modalities}

Besides the necessitation rule, \textit{distribution axiom} is the other fundamental axiom in normal modal logic.
It can be transformed into the rule shown in \cref{eqn:distribution-must-axiom},
and plugging in the definition of $\lor$ in \cref{eqn:def-lor} gives the rule shown in \cref{eqn:distribution-or-theorem}.
Notice that \cref{eqn:distribution-or-theorem} closely resembles rule \ref{eqn:inf-rule-or-left}'s variant with necessity, as shown in \cref{eqn:inf-rule-or-left-must},
except the different scope of the necessity operator and the position of the negation operator.
Moving the negation operator out of the necessity operator will result in a fallacy (Eq. \ref{eqn:distribution-or-fallacy}).
\noindent
\begin{align}
    \left\{\Box(\varphi \to \psi), \Box \varphi\right\} &\vdash \Box \psi, 
    \label{eqn:distribution-must-axiom} \\
    \left\{\Box(\varphi \lor \psi), \Box \lnot \varphi\right\} &\vdash \Box \psi \label{eqn:distribution-or-theorem}, \\
    \left\{\Box \varphi \lor \Box \psi, \lnot \Box \varphi\right\} &\vdash \Box \psi \label{eqn:inf-rule-or-left-must}, \\
    \left\{\Box(\varphi \lor \psi), \lnot \Box \varphi\right\} &\nvdash \Box \psi \label{eqn:distribution-or-fallacy}.
\end{align}
\noindent
We say \cref{eqn:distribution-or-theorem,eqn:inf-rule-or-left-must,eqn:distribution-or-fallacy} are of argument form \texttt{theorem}, \texttt{base} and \texttt{spurious}, respectively.
See \cref{tab:extra-distribution-forms} for the logical forms and their ground truth we used to study the distribution of modalities.
The natural language form is as follows:
\begin{table}[H]
    \small
    \begin{tabular}{rp{0.65\linewidth}}
    & It's certain that if Freddy is not going shopping, then Coy is making dinner.\\
    (\texttt{theorem}) & It's certain that Freddy is not going shopping.\\
    (\texttt{spurious}) & It's uncertain whether Freddy is going shopping.\\
    & Can we infer that it's certain that Coy is making dinner?\\
    \end{tabular}
\end{table}

This group of rules and fallacies comes from the fact that the necessity modality $\Box$ is not distributive to disjunction, i.e. $\Box (\varphi \lor \psi) \nvdash \Box \varphi \lor \Box \psi$ \citep[Ex. 5]{xiang-2019a-two-types}.
In contrast, the possibility modality $\Diamond$ is distributive to disjunction.
This particular case could have served as a material to test the LLM's knowledge of the asymmetry between the two modalities,
yet in \cref{subsec:affirmation-bias} we showed that there is a bias towards rejection on the necessity modality.
As the false case of the disjunction is on the necessity modality, this bias confounds the experiment.


\begin{table}[t]
    \small\centering
\begin{tabular}{ccl}
    \toprule
Modality & Argument Form & Logical Form \\
    \midrule
$\varnothing$ & base & $\varphi \lor \psi, \lnot \varphi \vdash \psi$ \\
$\Box$ & base & $\Box \varphi \lor \Box \psi, \lnot \Box \varphi \vdash \Box \psi$ \\
$\Box$ & theorem & $\Box ( \varphi \lor \psi), \Box \lnot \varphi \vdash \Box \psi$ \\
$\Box$ & \uline{spurious} & $\Box ( \varphi \lor \psi), \lnot \Box \varphi \nvdash \Box \psi$ \\
$\Diamond$ & base & $\Diamond \varphi \lor \Diamond \psi, \lnot \Diamond \varphi \vdash \Diamond \psi$ \\
$\Diamond$ & theorem & $\Diamond ( \varphi \lor \psi), \Diamond \lnot \varphi \vdash \Diamond \psi$ \\
$\Diamond$ & spurious & $\Diamond ( \varphi \lor \psi), \lnot \Diamond \varphi \vdash \Diamond \psi$ \\
% \cmidrule{1-3}
% $\varnothing$ & $\lnot \varphi \to \psi, \lnot \varphi \vdash \psi$ \\
% $\Box$ & $\Box ( \lnot \varphi \to \psi), \Box \lnot \varphi \vdash \Box \psi$ \\
% $\Box$ & False & $\Box ( \lnot \varphi \to \psi), \lnot \Box \varphi \nvdash \Box \psi$ \\
% $\Diamond$ & $\Diamond ( \lnot \varphi \to \psi), \Diamond \lnot \varphi \vdash \Diamond \psi$ \\
% $\Diamond$ & $\Diamond ( \lnot \varphi \to \psi), \lnot \Diamond \varphi \vdash \Diamond \psi$ \\
    \bottomrule
    \end{tabular}
    \caption{
        \label{tab:extra-distribution-forms}
        Logical forms and their ground truth to study the distribution of modalities.
        Only the spurious form of the necessity modality (marked by \uline{underline}) has a ground truth of false.
    }
\end{table}

We fit a linear mixed-effects model similar to \cref{eqn: mixed-effects} to the data,
\noindent
\begin{align}
    \mathit{Acc}_\textit{soft} & \sim \textit{Modality} \times \textit{ArgForm} + \textit{Perplexity} \nonumber \\
                               & + (1 + \textit{Perplexity} \mid \textit{LLM}), \nonumber
\end{align}
\noindent
with an interaction term between the modality and argument form.
% The model yields a conditional $R^2$ of $0.397$.
On the \texttt{theorem} form compared to the \texttt{base} form, the necessity modality $\Box$ has a $0.173$ higher estimated marginal means with $p < 0.0001$ significance, yet the possibility modality $\Diamond$ has a $0.071$ lower estimated marginal means.
On the \texttt{spurious} form compared to the \texttt{base} form, the $\Box$ has a $0.312$ higher means, and the $\Diamond$ has no significant difference.
On both forms, $\Diamond \succ \Box$ in terms of accuracy still holds at a slight margin of $0.110$ and $0.047$ respectively.

To verify whether on $\Box$ the performance increase on \texttt{spurious} form is due to the rejection bias, we fit a linear mixed-effects model with the relative probability of answering \texttt{Yes} as dependent variable.
Results show that on \texttt{spurious} form compared to the \texttt{base} form, the effect of $\Box$'s tendency to answer \texttt{Yes} is only $0.060$ lower, indicating the rejection bias of the \texttt{base} form is still present.
Therefore, we hypothesize that the LLM's performance on recognizing the fallacy of necessity distribution over disjunction is hindered by the rejection bias on the necessity modality.

\end{document}
