\pdfoutput=1
\documentclass[11pt]{article}

%
\setlength\unitlength{1mm}
\newcommand{\twodots}{\mathinner {\ldotp \ldotp}}
% bb font symbols
\newcommand{\Rho}{\mathrm{P}}
\newcommand{\Tau}{\mathrm{T}}

\newfont{\bbb}{msbm10 scaled 700}
\newcommand{\CCC}{\mbox{\bbb C}}

\newfont{\bb}{msbm10 scaled 1100}
\newcommand{\CC}{\mbox{\bb C}}
\newcommand{\PP}{\mbox{\bb P}}
\newcommand{\RR}{\mbox{\bb R}}
\newcommand{\QQ}{\mbox{\bb Q}}
\newcommand{\ZZ}{\mbox{\bb Z}}
\newcommand{\FF}{\mbox{\bb F}}
\newcommand{\GG}{\mbox{\bb G}}
\newcommand{\EE}{\mbox{\bb E}}
\newcommand{\NN}{\mbox{\bb N}}
\newcommand{\KK}{\mbox{\bb K}}
\newcommand{\HH}{\mbox{\bb H}}
\newcommand{\SSS}{\mbox{\bb S}}
\newcommand{\UU}{\mbox{\bb U}}
\newcommand{\VV}{\mbox{\bb V}}


\newcommand{\yy}{\mathbbm{y}}
\newcommand{\xx}{\mathbbm{x}}
\newcommand{\zz}{\mathbbm{z}}
\newcommand{\sss}{\mathbbm{s}}
\newcommand{\rr}{\mathbbm{r}}
\newcommand{\pp}{\mathbbm{p}}
\newcommand{\qq}{\mathbbm{q}}
\newcommand{\ww}{\mathbbm{w}}
\newcommand{\hh}{\mathbbm{h}}
\newcommand{\vvv}{\mathbbm{v}}

% Vectors

\newcommand{\av}{{\bf a}}
\newcommand{\bv}{{\bf b}}
\newcommand{\cv}{{\bf c}}
\newcommand{\dv}{{\bf d}}
\newcommand{\ev}{{\bf e}}
\newcommand{\fv}{{\bf f}}
\newcommand{\gv}{{\bf g}}
\newcommand{\hv}{{\bf h}}
\newcommand{\iv}{{\bf i}}
\newcommand{\jv}{{\bf j}}
\newcommand{\kv}{{\bf k}}
\newcommand{\lv}{{\bf l}}
\newcommand{\mv}{{\bf m}}
\newcommand{\nv}{{\bf n}}
\newcommand{\ov}{{\bf o}}
\newcommand{\pv}{{\bf p}}
\newcommand{\qv}{{\bf q}}
\newcommand{\rv}{{\bf r}}
\newcommand{\sv}{{\bf s}}
\newcommand{\tv}{{\bf t}}
\newcommand{\uv}{{\bf u}}
\newcommand{\wv}{{\bf w}}
\newcommand{\vv}{{\bf v}}
\newcommand{\xv}{{\bf x}}
\newcommand{\yv}{{\bf y}}
\newcommand{\zv}{{\bf z}}
\newcommand{\zerov}{{\bf 0}}
\newcommand{\onev}{{\bf 1}}

% Matrices

\newcommand{\Am}{{\bf A}}
\newcommand{\Bm}{{\bf B}}
\newcommand{\Cm}{{\bf C}}
\newcommand{\Dm}{{\bf D}}
\newcommand{\Em}{{\bf E}}
\newcommand{\Fm}{{\bf F}}
\newcommand{\Gm}{{\bf G}}
\newcommand{\Hm}{{\bf H}}
\newcommand{\Id}{{\bf I}}
\newcommand{\Jm}{{\bf J}}
\newcommand{\Km}{{\bf K}}
\newcommand{\Lm}{{\bf L}}
\newcommand{\Mm}{{\bf M}}
\newcommand{\Nm}{{\bf N}}
\newcommand{\Om}{{\bf O}}
\newcommand{\Pm}{{\bf P}}
\newcommand{\Qm}{{\bf Q}}
\newcommand{\Rm}{{\bf R}}
\newcommand{\Sm}{{\bf S}}
\newcommand{\Tm}{{\bf T}}
\newcommand{\Um}{{\bf U}}
\newcommand{\Wm}{{\bf W}}
\newcommand{\Vm}{{\bf V}}
\newcommand{\Xm}{{\bf X}}
\newcommand{\Ym}{{\bf Y}}
\newcommand{\Zm}{{\bf Z}}

% Calligraphic

\newcommand{\Ac}{{\cal A}}
\newcommand{\Bc}{{\cal B}}
\newcommand{\Cc}{{\cal C}}
\newcommand{\Dc}{{\cal D}}
\newcommand{\Ec}{{\cal E}}
\newcommand{\Fc}{{\cal F}}
\newcommand{\Gc}{{\cal G}}
\newcommand{\Hc}{{\cal H}}
\newcommand{\Ic}{{\cal I}}
\newcommand{\Jc}{{\cal J}}
\newcommand{\Kc}{{\cal K}}
\newcommand{\Lc}{{\cal L}}
\newcommand{\Mc}{{\cal M}}
\newcommand{\Nc}{{\cal N}}
\newcommand{\nc}{{\cal n}}
\newcommand{\Oc}{{\cal O}}
\newcommand{\Pc}{{\cal P}}
\newcommand{\Qc}{{\cal Q}}
\newcommand{\Rc}{{\cal R}}
\newcommand{\Sc}{{\cal S}}
\newcommand{\Tc}{{\cal T}}
\newcommand{\Uc}{{\cal U}}
\newcommand{\Wc}{{\cal W}}
\newcommand{\Vc}{{\cal V}}
\newcommand{\Xc}{{\cal X}}
\newcommand{\Yc}{{\cal Y}}
\newcommand{\Zc}{{\cal Z}}

% Bold greek letters

\newcommand{\alphav}{\hbox{\boldmath$\alpha$}}
\newcommand{\betav}{\hbox{\boldmath$\beta$}}
\newcommand{\gammav}{\hbox{\boldmath$\gamma$}}
\newcommand{\deltav}{\hbox{\boldmath$\delta$}}
\newcommand{\etav}{\hbox{\boldmath$\eta$}}
\newcommand{\lambdav}{\hbox{\boldmath$\lambda$}}
\newcommand{\epsilonv}{\hbox{\boldmath$\epsilon$}}
\newcommand{\nuv}{\hbox{\boldmath$\nu$}}
\newcommand{\muv}{\hbox{\boldmath$\mu$}}
\newcommand{\zetav}{\hbox{\boldmath$\zeta$}}
\newcommand{\phiv}{\hbox{\boldmath$\phi$}}
\newcommand{\psiv}{\hbox{\boldmath$\psi$}}
\newcommand{\thetav}{\hbox{\boldmath$\theta$}}
\newcommand{\tauv}{\hbox{\boldmath$\tau$}}
\newcommand{\omegav}{\hbox{\boldmath$\omega$}}
\newcommand{\xiv}{\hbox{\boldmath$\xi$}}
\newcommand{\sigmav}{\hbox{\boldmath$\sigma$}}
\newcommand{\piv}{\hbox{\boldmath$\pi$}}
\newcommand{\rhov}{\hbox{\boldmath$\rho$}}
\newcommand{\upsilonv}{\hbox{\boldmath$\upsilon$}}

\newcommand{\Gammam}{\hbox{\boldmath$\Gamma$}}
\newcommand{\Lambdam}{\hbox{\boldmath$\Lambda$}}
\newcommand{\Deltam}{\hbox{\boldmath$\Delta$}}
\newcommand{\Sigmam}{\hbox{\boldmath$\Sigma$}}
\newcommand{\Phim}{\hbox{\boldmath$\Phi$}}
\newcommand{\Pim}{\hbox{\boldmath$\Pi$}}
\newcommand{\Psim}{\hbox{\boldmath$\Psi$}}
\newcommand{\Thetam}{\hbox{\boldmath$\Theta$}}
\newcommand{\Omegam}{\hbox{\boldmath$\Omega$}}
\newcommand{\Xim}{\hbox{\boldmath$\Xi$}}


% Sans Serif small case

\newcommand{\Gsf}{{\sf G}}

\newcommand{\asf}{{\sf a}}
\newcommand{\bsf}{{\sf b}}
\newcommand{\csf}{{\sf c}}
\newcommand{\dsf}{{\sf d}}
\newcommand{\esf}{{\sf e}}
\newcommand{\fsf}{{\sf f}}
\newcommand{\gsf}{{\sf g}}
\newcommand{\hsf}{{\sf h}}
\newcommand{\isf}{{\sf i}}
\newcommand{\jsf}{{\sf j}}
\newcommand{\ksf}{{\sf k}}
\newcommand{\lsf}{{\sf l}}
\newcommand{\msf}{{\sf m}}
\newcommand{\nsf}{{\sf n}}
\newcommand{\osf}{{\sf o}}
\newcommand{\psf}{{\sf p}}
\newcommand{\qsf}{{\sf q}}
\newcommand{\rsf}{{\sf r}}
\newcommand{\ssf}{{\sf s}}
\newcommand{\tsf}{{\sf t}}
\newcommand{\usf}{{\sf u}}
\newcommand{\wsf}{{\sf w}}
\newcommand{\vsf}{{\sf v}}
\newcommand{\xsf}{{\sf x}}
\newcommand{\ysf}{{\sf y}}
\newcommand{\zsf}{{\sf z}}


% mixed symbols

\newcommand{\sinc}{{\hbox{sinc}}}
\newcommand{\diag}{{\hbox{diag}}}
\renewcommand{\det}{{\hbox{det}}}
\newcommand{\trace}{{\hbox{tr}}}
\newcommand{\sign}{{\hbox{sign}}}
\renewcommand{\arg}{{\hbox{arg}}}
\newcommand{\var}{{\hbox{var}}}
\newcommand{\cov}{{\hbox{cov}}}
\newcommand{\Ei}{{\rm E}_{\rm i}}
\renewcommand{\Re}{{\rm Re}}
\renewcommand{\Im}{{\rm Im}}
\newcommand{\eqdef}{\stackrel{\Delta}{=}}
\newcommand{\defines}{{\,\,\stackrel{\scriptscriptstyle \bigtriangleup}{=}\,\,}}
\newcommand{\<}{\left\langle}
\renewcommand{\>}{\right\rangle}
\newcommand{\herm}{{\sf H}}
\newcommand{\trasp}{{\sf T}}
\newcommand{\transp}{{\sf T}}
\renewcommand{\vec}{{\rm vec}}
\newcommand{\Psf}{{\sf P}}
\newcommand{\SINR}{{\sf SINR}}
\newcommand{\SNR}{{\sf SNR}}
\newcommand{\MMSE}{{\sf MMSE}}
\newcommand{\REF}{{\RED [REF]}}

% Markov chain
\usepackage{stmaryrd} % for \mkv 
\newcommand{\mkv}{-\!\!\!\!\minuso\!\!\!\!-}

% Colors

\newcommand{\RED}{\color[rgb]{1.00,0.10,0.10}}
\newcommand{\BLUE}{\color[rgb]{0,0,0.90}}
\newcommand{\GREEN}{\color[rgb]{0,0.80,0.20}}

%%%%%%%%%%%%%%%%%%%%%%%%%%%%%%%%%%%%%%%%%%
\usepackage{hyperref}
\hypersetup{
    bookmarks=true,         % show bookmarks bar?
    unicode=false,          % non-Latin characters in AcrobatÕs bookmarks
    pdftoolbar=true,        % show AcrobatÕs toolbar?
    pdfmenubar=true,        % show AcrobatÕs menu?
    pdffitwindow=false,     % window fit to page when opened
    pdfstartview={FitH},    % fits the width of the page to the window
%    pdftitle={My title},    % title
%    pdfauthor={Author},     % author
%    pdfsubject={Subject},   % subject of the document
%    pdfcreator={Creator},   % creator of the document
%    pdfproducer={Producer}, % producer of the document
%    pdfkeywords={keyword1} {key2} {key3}, % list of keywords
    pdfnewwindow=true,      % links in new window
    colorlinks=true,       % false: boxed links; true: colored links
    linkcolor=red,          % color of internal links (change box color with linkbordercolor)
    citecolor=green,        % color of links to bibliography
    filecolor=blue,      % color of file links
    urlcolor=blue           % color of external links
}
%%%%%%%%%%%%%%%%%%%%%%%%%%%%%%%%%%%%%%%%%%%


%siunitx
\DeclareSIUnit \decibelA {dB(A)}
\DeclareSIUnit \decibelC {dB(C)}
\DeclareSIUnit \soneGF {soneGF}
\DeclareSIUnit \acum {acum}
\DeclareSIUnit \asper {asper}
\DeclareSIUnit \vacil {vacil}
\DeclareSIUnit \tuhms {tuHMS}

%xcolor
\definecolor{set1_1}{RGB}{228,26,28}
\definecolor{set1_2}{RGB}{55,126,184}
\definecolor{set1_3}{RGB}{77,175,74}
\definecolor{den_1}{RGB}{239,209,0}
\definecolor{den_2}{RGB}{78,184,123}
\definecolor{den_3}{RGB}{0,127,196}
\definecolor{loaclr}{RGB}{152, 78, 163}
\definecolor{mclr}{RGB}{255, 127, 0}

%placeholder text
\newcommand{\subtop}[1]{\noindent \textcolor{magenta}{[subtopic: #1]}\\}
\newcommand{\highlightlist}[1]{\vspace{-1.5\baselineskip}\textcolor{magenta}{#1}}
\newcommand{\highlight}[1]{\textcolor{magenta}{#1}}
\newcommand{\bhan}[1]{\textcolor{cyan}{#1}}
\newcommand{\dba}[1]{\SI{#1}{\decibelA}}
\newcommand{\dbc}[1]{\SI{#1}{\decibelC}}
\newcommand{\red}[1]{\textcolor{red}{#1}}

% Custom checkbox command

\newcommand{\CheckedBox}[1]{%
  \ifnum#1=1
    \makebox[0pt][l]{\raisebox{0.15ex}{\hspace{0.2em}$\checkmark$}}%
  \fi
  $\square$%
}
\newcommand{\checkbox}[1]{\item[\CheckedBox{0}] #1}
\newcommand{\checkedbox}[1]{\item[\CheckedBox{1}] #1}

\newcommand{\Tx}[1]{$T_\text{#1}$}
\newcommand{\Lx}[1]{$L_\text{#1}$}
\newcommand{\Lh}[2]{$L_\text{#1,\SI{#2}{\hour}}$}
\newcommand{\Lmin}[2]{$L_\text{#1,\SI{#2}{\minute}}$}
\newcommand{\Ls}[2]{$L_\text{#1,\SI{#2}{\second}}$}
\newcommand{\Lle}[2]{$L_\text{#1}\le\dba{#2}$}
\newcommand{\Lhle}[3]{$L_\text{#1,\SI{#2}{\hour}}\le\dba{#3}$}
\newcommand{\Lminle}[3]{$L_\text{#1,\SI{#2}{\min}}\le\dba{#3}$}
\newcommand{\dLhge}[4]{$\mu_{\Delta L_\text{#1,\SI{#2}{\hour}}} \in [{#3},{#4}] \;\dba{}$}
\newcommand{\dLhgeC}[4]{$\mu_{\Delta L_\text{#1,\SI{#2}{\hour}}} \in [{#3},{#4}] \;\dbc{}$}
\newcommand{\dLhC}[3]{$\mu_{\Delta L_\text{#1,\SI{#2}{\hour}}}=\;\dbc{#3}$}

\newcommand{\binL}{$m_\text{L}$}
\newcommand{\binR}{$m_\text{R}$}
\newcommand{\GRASIn}{$m_\text{in}$}
\newcommand{\GRASOut}{$m_\text{out}$}
\newcommand{\HDbinL}{$m_\text{L}^\text{HD}$}
\newcommand{\HDbinR}{$m_\text{R}^\text{HD}$}
\newcommand{\HDGRAS}{$m_\text{out}^\text{HD}$}
\newcommand{\NICUbinL}{$m_\text{L}^\text{NICU}$}
\newcommand{\NICUbinR}{$m_\text{R}^\text{NICU}$}
\newcommand{\NICUGRASOut}{$m_\text{out}^\text{NICU}$}
\newcommand{\NICUGRASIn}{$m_\text{in}^\text{NICU}$}

\newcommand{\absdiff}[2]{$\mu^{\Delta m_\text{#1,#2}}_{L_{z}}$}

\newcommand{\mudiffL}[5]{$\boldsymbol{\mu}^{\Delta m_\text{#1,#2}}_{L_\text{#3}} \in [#4,#5]$}
\newcommand{\mudiffLo}[4]{$\mu^{\Delta m_\text{#1,#2}}_{L_\text{#3}} = #4$}

\newcommand{\lmeart}{LME-ART-ANOVA}

\title{
    Logical forms complement probability \\
    in understanding language model (and human) performance
    % The cases of propositional and epistemic modal logic
}

\author{Yixuan Wang \\
  University of Chicago \\
  \texttt{yixuanwang@uchicago.edu} \\\And
  Freda Shi \\
  University of Waterloo \\
  Vector Institute, Canada CIFAR AI Chair \\
  \texttt{fhs@uwaterloo.ca} \\}

\begin{document}

\setlength{\Exlabelsep}{0em}
\setlength{\SubExleftmargin}{1em}

\maketitle
\begin{abstract}
  With the increasing interest in using large language models (LLMs) for planning in natural language, understanding their behaviors becomes an important research question.
  This work conducts a systematic investigation of LLMs' ability to perform logical reasoning in \textit{natural language}.
  We introduce a controlled dataset of hypothetical and disjunctive syllogisms in propositional and modal logic and use it as the testbed for understanding LLM performance.
  Our results lead to novel insights in predicting LLM behaviors: in addition to the probability of input \citep{gonen-etal-2023-demystifying,mccoyEmbersAutoregressionShow2024}, logical forms should be considered as important factors.
  In addition, we show similarities and discrepancies between the logical reasoning performances of humans and LLMs by collecting and comparing behavioral data from both.
\end{abstract}

\section{Introduction}

Tutoring has long been recognized as one of the most effective methods for enhancing human learning outcomes and addressing educational disparities~\citep{hill2005effects}. 
By providing personalized guidance to students, intelligent tutoring systems (ITS) have proven to be nearly as effective as human tutors in fostering deep understanding and skill acquisition, with research showing comparable learning gains~\citep{vanlehn2011relative,rus2013recent}.
More recently, the advancement of large language models (LLMs) has offered unprecedented opportunities to replicate these benefits in tutoring agents~\citep{dan2023educhat,jin2024teach,chen2024empowering}, unlocking the enormous potential to solve knowledge-intensive tasks such as answering complex questions or clarifying concepts.


\begin{figure}[t!]
\centering
\includegraphics[width=1.0\linewidth]{Figs/Fig.intro.pdf}
\caption{An illustration of coding tutoring, where a tutor aims to proactively guide students toward completing a target coding task while adapting to students' varying levels of background knowledge. \vspace{-5pt}}
\label{fig:example}
\end{figure}

\begin{figure}[t!]
\centering
\includegraphics[width=1.0\linewidth]{Figs/Fig.scaling.pdf}
\caption{\textsc{Traver} with the trained verifier shows inference-time scaling for coding tutoring (detailed in \S\ref{sec:scaling_analysis}). \textbf{Left}: Performance vs. sampled candidate utterances per turn. \textbf{Right}: Performance vs. total tokens consumed per tutoring session. \vspace{-15pt}}
\label{fig:scale}
\end{figure}


Previous research has extensively explored tutoring in educational fields, including language learning~\cite{swartz2012intelligent,stasaski-etal-2020-cima}, math reasoning~\cite{demszky-hill-2023-ncte,macina-etal-2023-mathdial}, and scientific concept education~\cite{yuan-etal-2024-boosting,yang2024leveraging}. 
Most aim to enhance students' understanding of target knowledge by employing pedagogical strategies such as recommending exercises~\cite{deng2023towards} or selecting teaching examples~\cite{ross-andreas-2024-toward}. 
However, these approaches fall short in broader situations requiring both understanding and practical application of specific pieces of knowledge to solve real-world, goal-driven problems. 
Such scenarios demand tutors to proactively guide people toward completing targeted tasks (e.g., coding).
Furthermore, the tutoring outcomes are challenging to assess since targeted tasks can often be completed by open-ended solutions.



To bridge this gap, we introduce \textbf{coding tutoring}, a promising yet underexplored task for LLM agents.
As illustrated in Figure~\ref{fig:example}, the tutor is provided with a target coding task and task-specific knowledge (e.g., cross-file dependencies and reference solutions), while the student is given only the coding task. The tutor does not know the student's prior knowledge about the task.
Coding tutoring requires the tutor to proactively guide the student toward completing the target task through dialogue.
This is inherently a goal-oriented process where tutors guide students using task-specific knowledge to achieve predefined objectives. 
Effective tutoring requires personalization, as tutors must adapt their guidance and communication style to students with varying levels of prior knowledge. 


Developing effective tutoring agents is challenging because off-the-shelf LLMs lack grounding to task-specific knowledge and interaction context.
Specifically, tutoring requires \textit{epistemic grounding}~\citep{tsai2016concept}, where domain expertise and assessment can vary significantly, and \textit{communicative grounding}~\citep{chai2018language}, necessary for proactively adapting communications to students' current knowledge.
To address these challenges, we propose the \textbf{Tra}ce-and-\textbf{Ver}ify (\textbf{\model}) agent workflow for building effective LLM-powered coding tutors. 
Leveraging knowledge tracing (KT)~\citep{corbett1994knowledge,scarlatos2024exploring}, \model explicitly estimates a student's knowledge state at each turn, which drives the tutor agents to adapt their language to fill the gaps in task-specific knowledge during utterance generation. 
Drawing inspiration from value-guided search mechanisms~\citep{lightman2023let,wang2024math,zhang2024rest}, \model incorporates a turn-by-turn reward model as a verifier to rank candidate utterances. 
By sampling more candidate tutor utterances during inference (see Figure~\ref{fig:scale}), \model ensures the selection of optimal utterances that prioritize goal-driven guidance and advance the tutoring progression effectively. 
Furthermore, we present \textbf{Di}alogue for \textbf{C}oding \textbf{T}utoring (\textbf{\eval}), an automatic protocol designed to assess the performance of tutoring agents. 
\eval employs code generation tests and simulated students with varying levels of programming expertise for evaluation. While human evaluation remains the gold standard for assessing tutoring agents, its reliance on time-intensive and costly processes often hinders rapid iteration during development. 
By leveraging simulated students, \eval serves as an efficient and scalable proxy, enabling reproducible assessments and accelerated agent improvement prior to final human validation. 



Through extensive experiments, we show that agents developed by \model consistently demonstrate higher success rates in guiding students to complete target coding tasks compared to baseline methods. We present detailed ablation studies, human evaluations, and an inference time scaling analysis, highlighting the transferability and scalability of our tutoring agent workflow.

\section{Background and related work}
% 重点看Artistic data visualization: Beyond visual analytics 和Visualization criticism-the missing link between information visualization and art 的被引


This section reviews the background on artistic data visualization and previous research related to this topic.

\subsection{Artistic Data Visualization in Art History Context}
\label{ssec:contemporary}

Art history has been marked by transformative periods characterized by different aesthetic pursuits, materials, and methods. Since the time of Plato, imitation (or \textit{mimesis}, which views art as a mirror to the world around us) has been an important pursuit~\cite{pooke2021art}. Successful artworks, such as Greek sculptures and the convincing illusions of depth and space in Renaissance paintings, exemplify this tradition.
The advent of modern society and new technology, especially photography, posed a significant challenge to the notion of art as imitation~\cite{perry2004themes}. By the 1850s, modern art began to emerge with the core goal of transcending traditional forms and conventions. Movements like Post Impressionism, Expressionism, and Cubism revolutionized art through innovative uses of form (\eg color, texture, composition), moving art towards abstraction and experimentation. 
After World War II, the Cold War and the surge of various social problems heightened skepticism about the progress narrative of modernity and the superiority of the capitalist system, leading to the rise of postmodernism and the birth of contemporary art~\cite{hopkins2000after,harrison1992art}. One prominent feature of contemporary art is the absence of fixed standards or genres historically defined by the church or the academy. Postmodern design neither defines a cohesive set of aesthetic values nor restricts the range of media used~\cite{pooke2021art}, sparking novel practices such as installations, performances, lens-based media, videos, and land-based art~\cite{hopkins2000after}.
Meanwhile, artists have increasingly drawn energy from various philosophical and critical theories such as gender studies, psychoanalysis, Marxism, and post-structuralism~\cite{pooke2021art}. As a result, as described by Foster~\cite{foster1999recodings}, artists have increasingly become ``manipulators of signs and symbols... and the viewer an active reader of messages rather than a passive contemplator of the aesthetic''. Hopkins~\cite{hopkins2000after} described this shift as the ``death of the object'' and ``the move to conceptualism''. 
% Joseph Kosuth, one of the most important representatives of conceptual artists, also once said that “all art (after Duchamp) is conceptual (in nature) because art only exists conceptually”
% As argued by Danto~\cite{danto2015after}, traditional notions of aesthetics can no longer apply to contemporary art. ``“All there is at the end,” Danto wrote, “is theory, art having finally become vaporized in a dazzle of pure thought about itself, and remaining, as it were, solely as the object of its own theoretical consciousness.''
% The Anti-aesthetic (1983) has described these as ‘anti-aesthetic’ strategies – typified, as we have seen, by a conceptually driven approach to the art object and to the process of its production.

Emerging within the contemporary art historical context, data art has been significantly propelled by the advent of big data. An early milestone was Kynaston McShine's 1970 exhibition \textit{Information} at the Museum of Modern Art (MoMA). 
% In the exhibition catalogue, McShine wrote~\cite{information_moma}: ``Increasingly artists use mail, telegrams, telex machines, etc., for transmission of works themselves—photographs, films, documents—or of information about their activity.'' 
% The millennium era has witnessed substantial growth in this area.
In 2008, Google’s Data Arts Team was founded to explore the advancement of what creativity and technology can do together~\cite{google}.
% data artist Aaron Koblin
In 2012, Viégas and Wattenberg's \textit{Wind Map}, an artwork that visualizes real-time air movement, became the first web-based artwork to be included in MoMA's permanent collection~\cite{wind}.
Since 2013, the academic conference IEEE VIS has included an Arts Program (IEEE VISAP), showcasing artistic data visualizations through accepted papers and curated exhibitions. 
As noted by Barabási~\cite{dataism} (a Fellow of the American Physical Society and the head of a data art lab), data has become a vital medium for artists to deal with the complexities of our society: ``Humanity is facing a complexity explosion. We are confronted with too much data for any of us to make sense of...The traditional tools and mediums of art, be they canvas or chisel, are woefully inadequate for this task...today’s and tomorrow’s artists can embrace new tools and mediums that scale to the challenge, ensuring that their practice can continue to reflect our changing epistemology.''
% a physicist and head of a data art lab, has noted, 

% Artists are now exploring new mediums and methods, incorporating datasets, computer technology, and algorithms into their work.



\subsection{Research on Artistic Data Visualization}
\label{ssec:artisticvis}

Artistic data visualization is also referred to as artistic visualization, data art, or information art~\cite{holmquist2003informative,rodgers2011exploring,few,viegas2007artistic}. Despite the varying terminologies, there is a basic consensus that artistic data visualization must be art practice grounded in real data~\cite{viegas2007artistic}.
Since the early 2000s, a series of papers introduced innovative artistic systems for visualizing everyday data, such as museum visit routes and bus schedule information~\cite{skog2003between,holmquist2003informative,viegas2004artifacts}.
In 2007, Viégas and Wattenberg~\cite{viegas2007artistic} explicitly proposed the concept of \textit{artistic data visualization} and viewed it as a promising domain beyond visual analytics.
% and defined it as ``visualization of data done by artists with the intent of making art''. 
Kosara~\cite{kosara2007visualization} drew a spectrum of visualization design, positioning artistic visualization and pragmatic visualization at opposite ends of this spectrum to demonstrate that the goals of these two types of design often diverge. 
% advocating that analytical visualizations prioritize practicality, while artistic data visualizations emphasize sublime quality, that is, the capacity to inspire awe and grandeur and elicit profound emotional or intellectual responses. 
% In 2011, Rodgers and Bartram~\cite{rodgers2011exploring} utilized artistic data visualization to enhance residential energy use feedback. 
However, overall, research on this subject has been sparse. Among those relevant papers, most have focused on specific applications of artistic data visualization. 
%~\cite{rodgers2011exploring,schroeder2015visualization,perovich2020chemicals}
For instance, Rodgers and Bartram~\cite{rodgers2011exploring} utilized ambient artistic data visualization to enhance residential energy use feedback. Samsel~\etal~\cite{samsel2018art} transferred artistic styles from paintings into scientific visualization.
Artistic practice has also been observed in fields such as data physicalization~\cite{hornecker2023design,perovich2020chemicals,offenhuber2019data} and sonification~\cite{enge2024open}. For example, Hornecker~\etal~\cite{hornecker2023design} found that many artists are practicing transforming data into tangible artifacts or installations. Enge~\etal~\cite{enge2024open} discussed a set of representative artistic cases that combine sonification and visualization.
% dragicevic2020data
% Offenhuber~\cite{offenhuber2019data} created a set of artworks in urban settings that transform air quality data into situated displays, allowing people to encounter visualizations in their daily lives.

% But in contrast, empirical studies that describe the characteristics of artistic visualization and how they are created are extremely scarce. This scarcity forms a stark contrast to the increasingly rich and diverse practices by artists in the field.
% As for the difference between artistic data visualization and traditional visualizations for analytics, Vi{\'e}gas and Wattenberg~\cite{viegas2007artistic} thought that the most salient feature of artistic data visualizations is their forceful expression of viewpoints.
% In Ramirez~\cite{ramirez2008information}'s opinion, functional information visualizations are concerned with usability and performance while aesthetic information visualizations are concerned with visually attractive forms of representation design.
% Donath~\etal~\cite{donath2010data} presented a series of tools developed to integrate artistic expressions in generating unique and customized visualizations based on users' personal data, such as health monitoring data, album records, and e-mail records. 

On the other hand, some studies, while not focusing on artistic data visualization, have explored a key art-related concept: aesthetics. 
% ~\cite{moere2012evaluating,cawthon2007effect,lau2007towards} explored the aesthetics of visualization design in their research. They
For example, Moere~\etal~\cite{moere2012evaluating} compared analytical, magazine, and artistic visualization styles, noting that analytical styles enhance the discovery of analytical insights, while the other two induce more meaning-based insights. Cawthon~\etal~\cite{cawthon2007effect} asked participants to rank eleven visualization types on an aesthetic scale from ``ugly'' to ``beautiful'', finding that some visualizations (\eg sunburst) were perceived as more beautiful than others (\eg beam trees).
% Moere~\etal~\cite{moere2012evaluating} displayed data in three different styles (analytical style, magazine style, artistic style) and found that these styles led to different perceptions of usability and types of insights.
% More importantly, the authors found that the sunburst chart ranks the highest in aesthetics and is one of the top-performing visualizations in both efficiency and effectiveness, thus exemplifying the notion that "beautiful is indeed usable".
Factors such as embellishment~\cite{bateman2010useful}, colorfulness~\cite{harrison2015infographic}, and interaction~\cite{stoll2024investigating} have also been found to influence aesthetics. 
% borkin2013makes,tanahashi2012design
However, most of these studies have simplified aesthetics to hedonic features (\eg beauty), without delving into the nuanced connotations of aesthetics.
% most of these studies have simplified aesthetics to concepts like 'beauty,' 'preference,' or 'pleasing,' without exploring the deeper essence of aesthetics as the core of art.

The value of artistic data visualization to the visualization community is still in controversy. For instance, Few~\cite{few} argued for a clearer distinction between data art and data visualization; he highlighted the negative consequences when data art ``masquerades as data visualization'', such as making visualizations difficult to perceive. Willers~\cite{willers2014show} thought the artistic approach is ``unlikely be appreciated if the intention was for rapid decision making.''
% In an interview, American artist and technologist Harris commented that ``data can be made pretty by design, but this is a superficial prettiness, like a boring woman wearing too much makeup.''~\cite{harris2015beauty} 
To address these gaps, more empirical research needs to be conducted to explore how artistic data visualizations are designed, their underlying pursuits, and how they might inspire our community.




% Examining this field not only helps us understand the latest application of data visualization in various domains but also extends our understanding of the aesthetic and humanistic aspects of data visualization.
% there should be more theoretical investigation into artistic data visualization. 

% These three concepts emphasize placing or installing visualizations at physical places that people will encounter in their daily lives. 

% ~\cite{wang2019emotional}


% gap between art and science~\cite{judelman2004aesthetics}
% constructive visualization~\cite{huron2014constructive}
% data feminism~\cite{d2020data}
% critical infovis~\cite{dork2013critical}
% citizen data and participation~\cite{valkanova2015public}

% \x{Lee~\etal~\cite{lee2013sketchstory}, give users artistic freedom to create their own visualizations.}


% Aesthetics refers to the study of beauty, taste, and sensory perception and is deeply intertwined with art.
% a particular taste for or approach to what is pleasing to the senses and especially sight

% why shouldn't all charts be scatter plot~\cite{bertini2020shouldn}
% aesthetic model~\cite{lau2007towards}
% Aesthetics for Communicative Visualization : a Brief Review
% Stacked graphs--geometry \& aesthetics~\cite{byron2008stacked}
% storyline optimization~\cite{tanahashi2012design}
% graphic designers rate the attractiveness of non-standard and pictorial visualizations higher than standard and abstract ones, whereas the opposite is true for laypeople.~\cite{quispel2014would}
% evaluate aesthetics - wordcloud
% An Evaluation of Semantically Grouped Word Cloud Designs, tag cloud, wordle

% On the other hand, empirical studies conducted with designers have shown that functionality is not the only design goal of visualization. For example, Bigelow~\etal~\cite{bigelow2014reflections} found that designers would frequently fine-tune the non-data elements in their designs, and data encoding was even "a later consideration with respect to other visual elements of the infographic".
% Moere~\cite{moere2011role} - design
% Quispel~\etal~\cite{quispel2018aesthetics} found that for information designers, clarity and aesthetics are both important for making a design attractive.
% \gabis{Where do we define our notion of framing and relevant terms? Will that happen in the intro? For example here we use the term ``sentiment shifts'', which I think requires defintion.}\gili{done in intro}

% \gabis{Recurring comment - we should change tense to present, while most of the paper is currently in the past tense, I indicated this in some places, but should verify throughout.}

% \begin{figure*}[htbp]
    \centering
    % First Subfigure
    \begin{subfigure}{0.49\textwidth} % Adjust width as needed
        \centering
        \includegraphics[width=\textwidth]{images/orig_negative_models_distribution.png} % Replace with your image path
        \caption{Sentences that are \textbf{negative} in their original form.}
        \label{fig:negative-flip}
    \end{subfigure}
    % \hfill % Adds horizontal space between subfigures
    % Second Subfigure
    \begin{subfigure}{0.49\textwidth}
        \centering
        \includegraphics[width=\textwidth]{images/orig_positive_models_distribution.png} % Replace with your image path
        \caption{Sentences that are \textbf{positive} in their original form.}
        \label{fig:positive-flip}
    \end{subfigure}
    \caption{Proportion of sentences for which LLMs flipped sentiment, became neutral, or retained the original sentiment when presented with opposite sentiment framing. For example, this measures the percentage of sentences originally labeled as positive, that were labeled as negative after applying negative framing (and vice versa).
    }
    \label{fig:flip-proportion}
\end{figure*}


Our dataset curation consists of three steps, as depicted in Figure~\ref{fig:fig1}. First, we collect natural, real-world statements, with some clear sentiment, either positive or negative (\S\ref{sec:base-statements}; e.g., ``I won the highest prize'' as positive). Next, 
we reframe each statement by adding a prefix or suffix conveying the opposite sentiment
% for each statement, we add a framing that conveys the opposite sentiment to the base statement 
(\S\ref{sec:adding-framing}; e.g., ``I won the highest prize, although I lost all my friends on the way''). Finally, we collect large-scale human annotations via crowdsourcing, to label the sentiment shifts when wrapping the statements with the opposite framing (\S\ref{sec:human-annotations}; e.g., labeling ``negative'' the statement ``I won the highest prize, although I lost all my friends on the way''). 
%\gabis{I think we can remove the textual examples here to save space}

The complete dataset consists of 1000 statements, in which 500 are statements that their base form has positive sentiment, and 500 are base negative statements. 




\subsection{Collecting Base Statements}\label{sec:base-statements}
First, we collect base statements, which convey a clear sentiment, either clearly positive or clearly negative statements. We use \spike{} -- an extractive search system, which allows to extract statements from real-world datasets~\cite{taub-tabib-etal-2020-interactive}.
%\gabis{there's also a citation for spike}.\footnote{~\url{https://spike.apps.allenai.org}} 
Specifically, we collect statements from Amazon Reviews dataset, which are naturally occurring, sentiment-rich, texts but are less likely to trigger strong preexisting biases or emotional reactions, which may be a confound for our experiment.\footnote{~\url{https://spike.apps.allenai.org/datasets/reviews}} 
% \gabis{Why did we use this specifically? I think once we write the intro it would be good to relate to what we wrote there and how this domain is relevant.}
\begin{figure}[tb!]
    \centering
    \includegraphics[width=\linewidth]{images/roberta_score_before_after_framing.png}
    \caption{Distribution of sentiment scores before and after applying opposite-sentiment framing, as detailed in Section~\ref{sec:adding-framing}. Prior to framing, base sentences exhibit a clear polarity (positive or negative), whereas the application of opposite framing introduces ambiguity, shifting the sentiment scores toward a less distinct polarity.}
    \label{fig:pos-score-dist}
\end{figure}


Using \spike, we extract ${\sim}6k$ statements that fulfilled our designated queries, which we found correlated with clear sentiment. We designed the queries to capture positive or negative verbs that describe actions with some clear sentiment (e.g., ``enjoy'' or ``waste''), or statements with positive or negative adjective, describing an outcome with a clear sentiment (e.g., ``good'' or ``nasty''). The patterns and queries used for extraction are detailed in Appendix~\ref{sec:appendix-spike}.
% \gabis{needs more details, what are our queries? What were we aiming for? I understand that at a high level we're looking for clear sentiment, but how do we achieve this via lexical-syntactic queries?}. 
Next, we run in-house annotations to label and filter the extracted statements, to handle negations or other cases where the statement does not convey a clear sentiment. 
The filtering process results in $1,301$ positive statements, and $1,229$ negative statements.


\subsection{Adding Framing}\label{sec:adding-framing}

To reframe the statements in our dataset, we use GPT-4~\cite{achiam2023gpt}.\footnote{We used the gpt-4-0613 version.} 
% \gabis{do we have more details about which GPT4? what date?}
% The model was asked to keep he base statement unchanged, and add some prefix or suffix, that can be either positive or negative, oppositely to the base statement sentiment (e.g., I won the highest proze, althoug I lost all my friends on the way). 
The input prompt includes a 1-shot example, followed by a task description ``Add a <SENTIMENT> suffix or prefix to the given statement. Don't change the original statement.'', where SENTIMENT is either ``positive'' or ``negative'', opposite to the base statement sentiment (i.e., positive framing for negative base statement, and vice versa).

Unlike the base statement, the conveying sentiment of reframed statements is more ambiguous and there is no one clear label, as shown in Figure~\ref{fig:pos-score-dist}.\footnote{Scores in Figure~\ref{fig:pos-score-dist} are given by a fine-tuned sentiment analysis model ~\url{https://huggingface.co/cardiffnlp/twitter-roberta-base-sentiment-latest}}
%as we present the sentiment scores assigned by a fine-tuned sentiment analysis model,\footnote{~\url{https://huggingface.co/cardiffnlp/twitter-roberta-base-sentiment-latest}} %that was shown to be state-of-the-art when fine-tuned on sentiment analysis~\cite{csanady2024llambert}. 
% We present the sentiment scores 
% before and after reframing. It shows that wrapping the statement with the opposite sentiment injects ambiguity to the overall sentiment, as the sentiment scores become more dispersed. 
The exhibeted ambiguity in sentiment allows us to measure to what extent LLMs' shifting sentiment after framing, and how correlated it is to human behavior.



% In Figure~\ref{fig:pos-score-dist}, \gabis{Is roberta SOTA? it's a bit old by now. Do we have a reference to back this up?}\footnote{RoBERTa, fine-tuned for sentiment analysis~\url{https://huggingface.co/cardiffnlp/twitter-roberta-base-sentiment-latest}} The base statement scores are predominantly centered around binary values, either strongly positive or strongly negative. In contrast, the sentiment scores after opposite framing are more dispersed, reflecting increased ambiguity in sentiment. 
% \gabis{I'm not sure if this paragraph belongs here, maybe should be a subsection on its own at the end of the section?}


\subsection{Collecting Human Annotations}\label{sec:human-annotations}

In the final step, we collect human annotations through Amazon Mechanical Turk to evaluate the framing effect in \name{} over human participants, providing a reference for comparison with LLMs.\footnote{\url{https://www.mturk.com}} 
Details about the annotation platform are elaborated in Appendix~\ref{sec:mturk-appendix}.

The complete dataset includes 1K statements, each annotated by five different annotators. Given our budget, we preferred to collect five annotations per statement, resulting in less statements, but providing a more robust scoring for the ambiguity of a statement.

% We select a pool of 10 qualified workers who successfully passed our qualification test, which consisted of 20 base statements (unframed), for which annotators were expected to achieve perfect accuracy. The estimated hourly wage for the entire experiment was approximately 14USD per hour. More details about the annotation platform can be found in Appendix~\ref{sec:mturk-appendix}. Given our budget, we preferred to collect five annotations per statement, resulting in less statements, but providing a more robust scoring for the ambiguity of a statement.

For the annotation process, each statement in our dataset is presented in its reframed version (i.e., positive base statements with negative framing and vice versa), to five different annotators. This setup generates, for each dataset instance, a score ranging from 0 to 5, representing the number of annotators that votes for the sentiment that aligns with the opposite framing, which means that the overall sentiment of the reframed statement has shifted from its base sentiment. For example, in Figure~\ref{fig:fig1}, the statement ``I won the highest prize, although I lost all my friends on the way'' is shown to have two annotators voting ``negative'', which aligns with the sentiment of the framing and not the base statement, so the label for that instance in \name{} would be 2 (sentiment shifts).

% \gabist{It is important to note that there is no definitive ``right'' or ``wrong'' label for these statements, as the opposite sentiment framing often renders the sentiment conveyed highly ambiguous.}
Instances with score near 0 indicate that annotators agree that the overall sentiment remains unchanged despite the opposite framing. Score closer to 5 indicates that annotators agree that reframing shifted the perceived sentiment, while score around 2-3 suggests that the opposite framing makes the sentiment ambiguous.


\section{Experimental Study}\label{sec:experiments}
We conduct a comprehensive evaluation across multiple aspects: zero-shot performance, comparison with training on gold attribution data, and generalization to dialogue settings.
% . Our experiments span both in-\textit{isolation} question-answering datasets and in-\textit{dialogue} scenarios,
With our experiments, we shed light on the performance and practical utility of our approach.

% \textcolor{red}{TODO: Should we introduce the two settings separately: in-isolation QA and in-dialogue QA?}

% \textcolor{red}{TODO: We need an introduction sentence for this section.}

% \textcolor{red}{TODO: We mention isolated context attribution in some parts of the paper, while it is not clear how it differs from dialog-based context attribution.}

\subsection{Experimental Setting}
We evaluate model performance using precision (P), recall (R), and F1 score. For each sentence in the LLM's output, the context-attribution models identify the set of context sentences that support that output sentence. Precision measures the proportion of predicted attributions that are correct, while recall measures the proportion of ground truth attributions that are successfully identified.
%F1 is the harmonic mean of precision and recall.

For a fair and comprehensive evaluation, we train all models with a single pass over the training data unless stated otherwise, referring to this setup as \textbf{1P} when needed. For a more controlled comparison, some experiments limit the number of training samples each model encounters. Since the synthetic dataset contains approximately 1.0M samples, we allow models to \textit{observe} an equivalent number of samples from the gold training set, ensuring comparable exposure to models trained on data from \synqa. We refer to this setting as \textbf{1M} when necessary. For all models, we fine-tune only the LoRA parameters (alpha=64, rank=32) using a fixed learning rate of 1e-5 and a weight decay of 1e-3. 

\textbf{In-domain datasets:} We use \squadcolor{SQuAD} \cite{Rajpurkar2016SQuAD1Q} and \hotpotcolor{HotpotQA} \cite{Yang2018HotpotQAAD} as our primary in-domain benchmarks.\footnote{For some experiments (e.g., in Table~\ref{table:zero-shot-models}), these datasets are also \textit{out-of-domain} w.r.t. data generated by \synqa.} SQuAD provides clear sentence-level evidence for answering questions, serving as a strong baseline for direct attribution. HotpotQA introduces multi-hop reasoning, requiring models to link information across multiple sentences (sometimes from different articles) to identify the correct evidence chain. Additionally, HotpotQA includes distractor documents—closely related yet incorrect sources—posing a more challenging but realistic setting for evaluating attribution performance.

%\textcolor{red}{TODO (Kiril): Explain what you do to OR-QUAC, you combine the background with the context?}
\textbf{Out-of-domain datasets:} To assess generalization beyond the training distribution, we evaluate models on \quaccolor{QuAC} \cite{Choi2018QuACQA}, \coqacolor{CoQA} \cite{Reddy2018CoQAAC}, \orquaccolor{OR-QuAC} \cite{qu2020open}, and \doqacolor{DoQA} \cite{campos-etal-2020-doqa}. %\footnote{We consider these datasets as \textit{out-of-domain}, as none of the models we train are exposed to the training data of these datasets.}. 
These datasets present conversational QA scenarios that differ from SQuAD and HotpotQA. Specifically, QuAC and CoQA introduce multi-turn dialogue structures with coreferences, challenging models to track context across multiple turns. This conversational nature creates a methodological challenge: while these datasets are valuable for evaluating dialogue-based attribution, their reliance on conversation history makes direct comparison with models trained on single-turn QA datasets impossible.

To enable comprehensive evaluation across dialogue QA and single-turn QA, we create two versions of each dataset:
\begin{inparaenum}[(i)]
    \item a rephrased version using Llama 70B \cite{Dubey2024TheL3} that converts questions into standalone format for fair comparison with models trained on single-turn context attribution (suffixed by ``-ST''), and
    \item the original version for assessing dialogue-based attribution.
\end{inparaenum}

% To adapt these datasets for isolated context attribution (e.g., such as SQuAD and HotpotQA, where the question-answer pair is standalone),
% \footnote{We refer to isolated context attribution the scenario where the question-answer pair are standalone: i.e., do not contain coreferences.},
% we rephrase question-answer pairs (using Llama 70B), so that coreferencing is unnecessary. However, in dialogue-based settings, we evaluate models on the original, unmodified versions of these datasets.

DoQA extends this challenge further by incorporating domain-specific dialogues (cooking, travel and movies)%\footnote{The domains covered in DoQA are: cooking, travel and movies \cite{campos-etal-2020-doqa}.}
, thus testing the models' adaptability to specialized contexts. OR-QuAC includes %open-retrieval dialogue settings, assesses models' ability to attribute context in less structured environments, adding another layer of complexity to generalization evaluation. \textcolor{red}{TODO (Kiril): Check the papers for these datasets in case something is overlooked here.}
context-independent rewrites of the dialogue questions, such that they can be posed in isolation of prior context (i.e., single-turn QA). This enables us to test the models on their capabilities in both single-turn QA and dialogue QA settings.

\subsection{Methods}
We compare our method (\synqa) against several baselines, including sentence-encoder-based models, zero-shot instruction-tuned LLMs, and models trained on synthetic and gold context-attribution data. Specifically, we experiment with the following methods:

\paragraph{Sentence-Encoders:} We embed each sentence in the context along with the question-answer pair, and select attribution sentences based on cosine similarity with a fixed threshold, tuned on a small validation set.

\paragraph{Zero-shot (L)LMs:} We evaluate various instruction-tuned (L)LMs in a zero-shot manner, as such models have been shown to perform well across a range of NLP tasks \cite{shu2023exploitability,zhang2023instruction}. During inference, we provide an instruction template describing the task to the LLM (see Appendix~\ref{app:prompts} for details).
%as such models have been shown to perform well across a range of NLP tasks.

\paragraph{Ensembles of LLMs:} We aggregate the predictions of multiple LLMs through majority voting, selecting attribution sentences that receive consensus from at least 50\% of the ensemble. In our experiments, we use Llama8B \cite{Dubey2024TheL3}, Mistral7B, and Mistral-Nemo12B \cite{Jiang2023Mistral7} as the ensemble constituents.


\paragraph{Models trained on in-domain gold data:} Fine-tuning on gold-labeled attribution data provides an upper bound on in-domain performance, helping us assess how well synthetic training data generalizes.

\paragraph{\synatt:} \synatt generates synthetic training data by prompting multiple LLMs to perform context attribution in a discriminative manner, aggregating their outputs via majority voting, and training a smaller model on the resulting dataset. To make it a stronger baseline against \synqa, we give the training data of SQuAD and HotpotQA (the context, questions, and answers) to the LLMs and ask them to perform context attribution (note that we do not use the gold attribution). Finally, we train a model on the generated synthetic data.

\paragraph{\synqa:} We train models using synthetic data generated by our proposed method \synqa. Note that even though we train models using \synqa attribution data, we ensure they are not exposed to \textit{any} parts of the evaluation data.\footnote{We identify data leakage by representing each Wikipedia article as a MinHash signature. Then, for each training Wikipedia article, we retrieve candidates from the testing datasets via Locality Sensitivity Hashing and compute their Jaccard similarity \cite{dasgupta2011fast}. We flag as potential leaks pairs exceeding a threshold empirically set to 0.8.}

\subsection{Results and Discussion}
Evaluating our context attribution models requires a multifaceted approach, as performance is influenced by both the quality of training data and the model’s ability to generalize beyond in-domain distributions. Therefore, we design our experiments to address five core questions:
\begin{inparaenum}[(i)]
    \item How well do zero-shot LLMs perform on context-attribution QA tasks (\S\ref{sec:experiments-zero-shot})?
    \item Can models trained on synthetic data generated by \synqa exceed the performance of models trained on gold context-attribution data (\S\ref{sec:experiments-gold})?
    \item To what extent do models generalize to dialogue settings where in-domain training data is unavailable (\S\ref{sec:experiments-dialog})?
    \item How well do models scale in terms of synthetic data quantity generated by \synqa (\S\ref{sec:scalling-trends})?
    \item How do improved context attributions impact the end users' speed and ability to verify questions answering outputs (\S\ref{sec:user-study})?
\end{inparaenum}
% (i) How well do zero-shot LLMs perform context-attribution? (ii) Can synthetic attribution data serve as a viable alternative to gold supervision, particularly in out-of-domain settings? (iii) How do scaling trends affect generalization performance across diverse datasets?

%By systematically comparing models trained on synthetic data to both zero-shot and gold-supervised baselines, we aim to uncover the trade-offs between scalability, performance, and generalization. 
%Collectively, our findings provide a deeper understanding of how synthetic data can be leveraged for context attribution, potentially mitigating the reliance on costly human-annotated datasets.

\subsubsection{Comparison to Zero-Shot Models}\label{sec:experiments-zero-shot}

% Zero-shot v.s. SynQA-trained Models

\begin{table*}[t]
\centering
\resizebox{1.0\textwidth}{!}{
\begin{tabular}{lccccccccccccccc} \toprule
\multirow{2}{*}{Model} & \multirow{2}{*}{Training data} & \multicolumn{3}{c}{\squadcolor{Squad}} & \multicolumn{3}{c}{\hotpotcolor{Hotpot}} & \multicolumn{3}{c}{\quaccolor{Quac-ST}} & \multicolumn{3}{c}{\coqacolor{CoQA-ST}} \\ \cmidrule(lr){3-5} \cmidrule(lr){6-8} \cmidrule(lr){9-11} \cmidrule(lr){12-14}
& & P & R & F1 & P & R & F1 & P & R & F1 & P & R & F1 \\ \midrule
\textbf{\textit{Baselines}} \\
Random & -- & 19.8 & 15.4 & 17.3 & 4.8 & 15.2 & 7.3 & 5.2 & 15.1 & 7.7 & 7.3 & 15.1 & 9.9 \\
E5 | 561M & Zero-shot & 38.1 & 76.5 & 50.9 & 12.4 & 41.4 & 19.1 & 65.0 & 73.8 & 69.1 & 61.1 & 15.2 & 24.4 \\
HF-SmolLM2 | 365M & Zero-shot & 28.1 & 46.4 & 35.0 & 5.1 & 7.3 & 6.0 & 10.6 & 22.6 & 14.4 & 10.6 & 21.5 & 14.2 \\
Llama | 1B & Zero-shot & 37.5 & 62.0 & 46.7 & 5.3 & 28.1 & 8.9 & 8.8 & 65.4 & 15.4 & 11.9 & 52.8 & 19.4 \\
Mistral | 7B & Zero-shot & 71.5 & 94.4 & 81.4 & 42.9 & 42.7 & 42.8 & 63.2 & 88.6 & 73.8 & 59.0 & 72.2 & 64.9 \\
Llama | 8B & Zero-shot & 71.9 & 96.9 & 82.6 & 49.2 & 52.9 & 51.0 & 64.1 & 92.1 & 75.6 & 55.7 & 76.4 & 64.4 \\
Mistral-NeMo | 12B & Zero-shot & 89.5 & 94.5 & 91.8 & 46.4 & 47.3 & 46.8 & 81.8 & 85.3 & 83.5 & 79.0 & 67.2 & 72.6 \\
Ensemble | 27B & Zero-shot & 83.1 & 96.3 & 89.2 & 48.1 & 59.6 & 53.2 & 74.8 & 90.3 & 81.8 & 69.5 & 73.6 & 71.5 \\
Llama | 70B & Zero-shot & 95.3 & 95.6 & 95.5 & 87.6 & 37.5 & 52.5 & 89.7 & 87.8 & 88.7 & \textbf{87.5} & \textbf{73.3} & \textbf{79.8} \\
\midrule
\textbf{\textit{Baselines}} \\
%Llama | 1B & \squadcolor{SQuAD} \& \hotpotcolor{HotpotQA}; \synatt (1P) & 89.8 & 96.5 & 93.0 & 50.6 & 58.6 & 54.3 & 64.9 & 91.5 & 75.9 & 53.1 & 75.5 & 62.3 \\
%Llama | 1B & \squadcolor{SQuAD} \& \hotpotcolor{HotpotQA}; \synatt (1M) & 84.3 & \textbf{96.9} & 90.2 & 54.4 & 58.0 & 56.1 & 63.4 & 92.4 & 75.2 & 52.5 & 77.5 & 62.6 \\ \midrule
Llama | 1B & \synatt (1P) & 89.8 & 96.5 & 93.0 & 50.6 & 58.6 & 54.3 & 64.9 & 91.5 & 75.9 & 53.1 & 75.5 & 62.3 \\
Llama | 1B & \synatt (1M) & 84.3 & \textbf{96.9} & 90.2 & 54.4 & 58.0 & 56.1 & 63.4 & 92.4 & 75.2 & 52.5 & 77.5 & 62.6 \\ \midrule
\textbf{\textit{Ours}} \\
Llama | 1B & \syntheticcolor{\synqa} & \textbf{96.0} & 96.2 & \textbf{96.1} & \textbf{89.6} & \textbf{69.4} & \textbf{78.2} & \textbf{93.3} & \textbf{89.1} & \textbf{91.1} & \underline{82.3} & 68.5 & \underline{74.8} \\
\bottomrule
\end{tabular}
}
\caption{Comparison of zero-shot models and those trained with synthetic data. Larger zero-shot LMs excel, but our \synqa model outperforms all but one for one dataset while being smaller. \textbf{Bold} denotes best method, \underline{underline} if our method is second best. 1P: models trained with a single pass over the training data. 1M: models trained with 1M samples to match the size of the \synqa data.}
\label{table:zero-shot-models}
\end{table*}

In Table~\ref{table:zero-shot-models}, we present the performance of zero-shot models, and models trained without gold context-attribution data. 
%\footnote{Note that the \synatt baseline models are trained using question-answer pairs from SQuaAD and HotpotQA, however, the context-attribution annotations are obtained using an ensemble of LLMs.}. 
State-of-the-art sentence-encoder models (e.g., E5) perform relatively poorly, consistent with prior findings \cite{CohenWang2024ContextCiteAM}. In contrast, LLMs exhibit strong performance, with improvements correlating with model size. Ensembling multiple zero-shot LLMs further enhances performance, leveraging complementary strengths across models, but making the attribution more expensive. We also tested models trained with the discriminative method \synatt. These models significantly outperform their non-fine-tuned counterparts of the same size. However, as postulated, our generative approach \synqa outperforms \synatt significantly in all but one case. Additionally, \synqa surpasses zero-shot LLMs that are orders of magnitude larger, showing that we can train a model that is both more accurate and efficient.

\subsubsection{Comparison to Models Trained on Gold Attribution Data}\label{sec:experiments-gold}

\begin{table*}[t]
\centering
\resizebox{1.0\textwidth}{!}{
\begin{tabular}{lccccccccccccccc} \toprule
\multirow{3}{*}{Model} & \multirow{3}{*}{Training data} 

& \multicolumn{6}{c}{\textbf{In-Domain}} 
& \multicolumn{6}{c}{\textbf{Out-of-Domain}} \\ \cmidrule(lr){3-8} \cmidrule(lr){9-14}

& & \multicolumn{3}{c}{\squadcolor{SQuAD}} & \multicolumn{3}{c}{\hotpotcolor{HotpotQA}} 
& \multicolumn{3}{c}{\quaccolor{QuAC-ST}} & \multicolumn{3}{c}{\coqacolor{CoQA-ST}} \\ \cmidrule(lr){3-5} \cmidrule(lr){6-8} \cmidrule(lr){9-11} \cmidrule(lr){12-14}

& & P & R & F1 & P & R & F1 & P & R & F1 & P & R & F1 \\ \midrule
\textbf{\textit{Baselines}} \\
Llama | 1B & Zero-shot & 37.5 & 62.0 & 46.7 & 5.3 & 28.1 & 8.9 & 8.8 & 65.4 & 15.4 & 11.9 & 52.8 & 19.4 \\
Llama | 1B & \squadcolor{SQuAD} (1P) & 98.4 & 98.4 & 98.4 & 48.7 & 20.0 & 28.4 & 92.6 & 85.8 & 89.0 & 79.9 & 64.3 & 71.2 \\
Llama | 1B & \hotpotcolor{HotpotQA} (1P) & 41.3 & 87.3 & 56.0 & 87.5 & 79.9 & 83.5 & 45.2 & 89.9 & 60.1 & 41.0 & 70.9 & 52.0 \\
Llama | 1B & \squadcolor{SQuAD} \& \hotpotcolor{HotpotQA} (1P) & 98.3 & 98.3 & 98.3 & \textbf{89.7} & 78.9 & 84.0 & 90.4 & 90.0 & 90.2 & 83.1 & 68.0 & 74.8 \\
Llama | 1B & \squadcolor{SQuAD} \& \hotpotcolor{HotpotQA} (1M) & \textbf{98.3} & \textbf{98.4} & \textbf{98.3} & 87.0 & \textbf{85.2} & \textbf{86.1} & 84.0 & 89.2 & 86.6 & 79.2 & 66.4 & 72.2 \\ \midrule
\textbf{\textit{Ours}} \\
Llama | 1B & \syntheticcolor{\synqa} & 96.0 & 96.2 & 96.1 & \underline{89.6} & 69.4 & 78.2 & \underline{93.3} & 89.1 & \underline{91.1} & 82.3 & 68.5 & \underline{74.8} \\
Llama | 1B & \syntheticcolor{\synqa} \& \squadcolor{SQuAD} \& \hotpotcolor{HotpotQA} & \underline{98.2} & \underline{98.3} & \underline{98.2} & 89.3 & \underline{82.4} & \underline{85.8} & \textbf{94.5} & \textbf{92.7} & \textbf{93.6} & \textbf{85.5} & \textbf{71.0} & \textbf{77.6} \\
\bottomrule
\end{tabular}
}
\caption{Comparison of models fine-tuned on synthetic vs.~gold in-domain data. Our \synqa approach generalizes better while remaining competitive in-domain. \textbf{Bold} denotes best method, \underline{underline} our method when second best. 1P: models trained with a single pass over the training data. 1M: models trained with 1M samples to match the size of the \synqa data.}
\label{table:fine-tuned-models}
\end{table*}

In Table~\ref{table:fine-tuned-models}, we compare models trained on synthetic and gold in-domain context-attribution datasets. As expected, fine-tuning on in-domain gold datasets (SQuAD and HotpotQA) yields highly specialized models that perform well on in-domain data.
% The performance on the out-of-domain datasets is comparable to Llama 70B, the best zero-shot LLM.
% In contrast, \synqa models outperform Llama 70B on out-of-domain datasets while also achieving near identical scores on the in-domain datasets.
However, models trained on data obtained by \synqa exhibit competitive performance on in-domain tasks and consistently surpass in-domain-trained models on out-of-domain datasets. 
This strong out-of-domain generalization is crucial for practical deployments, where models must handle diverse, previously unseen contexts that often differ substantially from their training data.

\subsubsection{Comparison to Zero-Shot and Fine-Tuned Models in Dialogue Contexts}\label{sec:experiments-dialog}
% \begin{table}[t]
% \centering
% \resizebox{1.0\columnwidth}{!}{
% \begin{tabular}{lccccccc} 
% \toprule

% \multirow{2}{*}{Model} & \multirow{2}{*}{Training data} & \multicolumn{3}{c}{\quaccolor{QuAC}} & \multicolumn{3}{c}{\coqacolor{CoQA}} \\ 
% \cmidrule(lr){3-5} \cmidrule(lr){6-8}

%  &  & P & R & F1 & P & R & F1 \\ 
% \midrule
% \textbf{\textit{Baselines}} \\
% Llama | 1B & Zero-shot & 20.9 & 47.9 & 29.1 & 35.6 & 40.2 & 37.8 \\
% Mistral | 7B & Zero-shot & 64.9 & 83.9 & 73.2 & 54.4 & 64.9 & 59.2 \\
% Llama | 8B & Zero-shot & 81.4 & 89.0 & 85.0 & 77.8 & 72.1 & 74.8 \\
% Mistral NeMo | 12B & Zero-shot & 84.8 & 85.4 & 85.1 & 81.7 & 68.4 & 74.5 \\
% \midrule
% Llama | 1B & \squadcolor{SQuAD} \& \hotpotcolor{HotpotQA} (1P) & 72.9 & 68.0 & 70.3 & 79.3 & 64.4 & 71.0 \\
% Llama | 1B & \squadcolor{SQuAD} \& \hotpotcolor{HotpotQA} (1M) & 56.0 & 49.0 & 52.3 & 63.0 & 51.2 & 56.5 \\ 
% \midrule
% \textbf{\textit{Ours}} \\
% Llama | 1B & \syntheticcolor{\synqa} & \textbf{91.3} & \underline{91.4} & \underline{91.3} & \underline{81.7} & \underline{71.4} & \underline{76.2} \\
% Llama | 1B & \syntheticcolor{\synqa} \& \squadcolor{SQuAD} \& \hotpotcolor{HotpotQA} & \underline{91.1} & \textbf{92.3} & \textbf{91.7} & \textbf{82.3} & \textbf{73.2} & \textbf{77.5} \\
% \bottomrule
% \end{tabular}
% }
% \caption{Context attribution on QuAC and CoQA (dialog data); both datasets are out-of-domain. Despite the size advantage of zero-shot LLMs, our \synqa models outperform fine-tuned and larger zero-shot models. \textbf{Bold} denotes best method, \underline{underline} our method when second best.}
% \label{table:dialog-datasets}
% \end{table}

\begin{table*}[t]
\centering
\resizebox{1.0\textwidth}{!}{
\begin{tabular}{lcccccccccccccc} 
\toprule

\multirow{3}{*}{Model} & \multirow{3}{*}{Training data} 

& \multicolumn{12}{c}{\textbf{Out-of-Domain}} \\ \cmidrule(lr){3-14}

& & \multicolumn{3}{c}{\quaccolor{QuAC}} & \multicolumn{3}{c}{\coqacolor{CoQA}} 
& \multicolumn{3}{c}{\orquaccolor{OR-QuAC}} & \multicolumn{3}{c}{\doqacolor{DoQA}} \\ 

\cmidrule(lr){3-5} \cmidrule(lr){6-8} \cmidrule(lr){9-11} \cmidrule(lr){12-14}

 &  & P & R & F1 & P & R & F1 & P & R & F1 & P & R & F1 \\ 
\midrule
\textbf{\textit{Baselines}} \\
Llama | 1B & Zero-shot & 30.8 & 45.5 & 36.8 & 39.4 & 37.9 & 38.6 & 33.0 & 46.6 & 38.6 & 12.2 & 22.6 & 15.9 \\
Mistral | 7B & Zero-shot & 76.6 & 81.8 & 79.1 & 67.6 & 61.3 & 64.3 & 82.5 & 85.1 & 83.8 & 74.9 & 77.9 & 76.4 \\
Llama | 8B & Zero-shot & 84.7 & 88.8 & 86.7 & 79.3 & 72.0 & 75.5 & 88.0 & 91.3 & 89.6 & 77.9 & 91.4 & 84.1 \\
Mistral-NeMo | 12B & Zero-shot & 85.7 & 85.4 & 85.5 & 81.9 & 68.4 & 74.5 & 88.9 & 88.8 & 88.8 & 86.0 & 84.2 & 85.1 \\
Llama | 70B & Zero-shot & 88.5 & 87.7 & 88.1 & \textbf{88.3} & \textbf{74.9} & \textbf{81.1} & 81.7 & 86.3 & 83.9 & 85.2 & 82.0 & 83.5 \\
\midrule
\textbf{\textit{Baselines}} \\
Llama | 1B & \squadcolor{SQuAD} \& \hotpotcolor{HotpotQA} (1P) & 71.3 & 66.8 & 69.0 & 79.0 & 64.2 & 70.8 & 61.6 & 57.5 & 59.5 & 67.4 & 57.8 & 62.2 \\
Llama | 1B & \squadcolor{SQuAD} \& \hotpotcolor{HotpotQA} (1M) & 52.6 & 49.3 & 50.9 & 61.2 & 50.2 & 55.2 & 48.5 & 44.6 & 46.5 & 53.2 & 49.1 & 51.1 \\ \midrule
\textbf{\textit{Ours}} \\
Llama | 1B & \syntheticcolor{\synqa} & \textbf{91.3} & \underline{91.4} & \underline{91.3} & 81.7 & 71.4 & 76.2 & \textbf{92.6} & \underline{95.3} & \textbf{94.0} & \textbf{86.3} & \underline{94.5} & \textbf{90.2} \\
Llama | 1B & \syntheticcolor{\synqa} \& \squadcolor{SQuAD} \& \hotpotcolor{HotpotQA} & \underline{91.1} & \textbf{92.2} & \textbf{91.7} & \underline{82.3} & \underline{73.2} & \underline{77.5} & \underline{90.3} & \textbf{96.4} & \underline{93.2} & \underline{85.1} & \textbf{96.0} & \textbf{90.2} \\
\bottomrule
\end{tabular}
}
\caption{Context attribution on QuAC, CoQA, OR-Quac, and DoQA (dialogue data); all datasets are out-of-domain. Despite the size advantage of zero-shot LLMs, our \synqa models outperform fine-tuned and larger zero-shot models. \textbf{Bold} denotes best method, \underline{underline} our method when second best. 1P: models trained with a single pass over the training data. 1M: models trained with 1M samples to match the size of the \synqa data.}
\label{table:dialog-datasets}
\end{table*}

We evaluate dialogue context attribution, for which we do not use any gold in-domain training data (Tab.~\ref{table:dialog-datasets}).
% exists.
Here, models must handle follow-up questions that rely on previous turns, often involving coreferences and other dialogue-specific complexities. As expected, zero-shot LLMs exhibit a strong size-performance correlation, with larger models consistently outperforming smaller ones—even those fine-tuned on single-turn question-answer attribution (trained on gold SQuAD and HotpotQA data). However, fine-tuning smaller models with our synthetic data generation strategy leads to superior performance, surpassing both their fine-tuned counterparts and much larger zero-shot LMs. This demonstrates the effectiveness of \synqa in enhancing context attribution in a dialogue setting and without requiring in-domain supervision.

\subsubsection{Scaling Trends and Generalization Performance}\label{sec:scalling-trends}

\begin{figure*}[t]
    \centering
    \begin{subfigure}{0.48\linewidth}
        \centering
        \includegraphics[width=\linewidth]{img/size_performance.pdf}
        \caption{Model performance vs.~size.}
        \label{fig:size_performance}
    \end{subfigure}
    \hfill
    \begin{subfigure}{0.48\linewidth}
        \centering
        \includegraphics[width=\linewidth]{img/data_quantity.pdf}
        \caption{F1 score vs.~training data size.}
        \label{fig:data_quantity}
    \end{subfigure}
    \caption{Comparison of model performance and scalability. (a) Larger zero-shot models achieve good F1 scores, but our method \synqa (based on Llama 1B) outperforms them while being orders of magnitude smaller. (b) Performance improves consistently with more \synqa training data, highlighting its scalability.}
    \label{fig:combined}
\end{figure*}

Fig.~\ref{fig:size_performance} shows F1 scores averaged across datasets, with model size on the x-axis and performance on the y-axis. Models trained on \synqa-generated data significantly outperform their baseline zero-shot counterparts, while also achieving superior performance compared to zero-shot LLMs that are orders of magnitude larger. This shows our method is highly data-efficient, enabling small models to close the gap with much larger counterparts.



In Figure~\ref{fig:data_quantity}, we analyze model performance as the quantity of synthetic training data increases, reporting F1 scores separately for in-domain and out-of-domain datasets. As we scale data quantity, performance improves consistently across datasets for isolated context attribution. This trend highlights the scalability of our approach, indicating that further gains can be achieved by increasing synthetic data availability.
%Notably, despite the lack of direct supervision on in-domain datasets, more data results in improved performance.%, reinforcing the robustness of our method.

\subsubsection{User Study: \synqa increases efficiency and accuracy assessment}\label{sec:user-study}
We conducted a user study to evaluate the efficiency and accuracy of verifying the correctness of LLM-generated answers using context attribution. Our hypothesis is that higher-quality context attributions, visualized to guide users, facilitate faster and more accurate verification of LLM outputs. Specifically, in each trial, we presented users with a question, a generated answer, and relevant context, along with
% context
attributions visualized as highlights. Their task was to leverage these attributions to judge if the answer was correct w.r.t.~a provided context. See Figure~\ref{fig:user_interface} in Appendix~\ref{app:user_study}.
% for an example. %Appendix~\ref{app:user_study} for an example.

The study compares three scenarios:
\begin{inparaenum}[(i)]
\item 
\textbf{No Alignment:} a baseline condition without context attributions, requiring users to manually read and verify the answer against the entire context;
\item 
\textbf{Llama 1B (Zero-shot):} context attributions generated by the Llama 1B model were visualized;
\item 
\textbf{\synqa}: context attributions generated by our approach were visualized.
\end{inparaenum}

We employed a within-subjects experimental design for our human evaluation (with 12 participants), ensuring that the same participants evaluate all the aforementioned alignment scenarios, thus requiring fewer participants for reliable results \cite{greenwald:1976}. However, this can be susceptible to learning effects where participants perform better in later scenarios, because they learned the task from previous examples. To mitigate this, we counterbalanced the scenario order using a Latin Square design \cite{belz:2010,bradley:1958}, where each alignment scenario appears in each position an equal number of times across all participants. Finally, we randomized the example order within each scenario per participant. For each example, we measured: \textbf{verification time} (seconds from display to judgment submission) and \textbf{verification accuracy} (binary correct/incorrect judgment).

\begin{figure}[hb!]
    \centering
    \includegraphics[width=1.0\linewidth]{img/user_study_big_font.png}
    \caption{Relationship between Evaluation Time (seconds) and Accuracy (\%) for three answer verification settings:  \emph{Llama 1B (Zero-shot)}, \emph{No Alignment} and \synqa. \synqa demonstrates the lowest evaluation time and highest accuracy, indicating its superior performance in facilitating efficient and accurate answer verification.}
    \label{fig:user_study}
\end{figure}

\noindent \textbf{Results.} We observed a clear trend in verification performance across the different attribution settings, with \synqa demonstrating superior effectiveness (Fig.~\ref{fig:user_interface}). \synqa has the lowest average verification time per example (\textbf{148.6} seconds), significantly faster than \emph{No Alignment} (171.8 seconds) and attributions from \emph{Llama 1B} (163.4 seconds). Concurrently, in terms of verification accuracy, \synqa achieved the highest average accuracy (\textbf{86.4\%}). While \emph{No Alignment} (84.1\%) and \emph{Llama 1B (77.3\%)} also yielded reasonable accuracy, attributions from \synqa are clearly of higher quality helping users be more accurate.


% \begin{table*}[t]
% \centering
% \resizebox{1.0\textwidth}{!}{
% \begin{tabular}{lccccccccccccccc} \toprule
% \multirow{2}{*}{Model} & \multirow{2}{*}{Training data} & \multicolumn{3}{c}{\squadcolor{Squad}} & \multicolumn{3}{c}{\hotpotcolor{HotPot QA}} & \multicolumn{3}{c}{\quaccolor{Quac}} & \multicolumn{3}{c}{\coqacolor{CoQA}} \\ \cmidrule(lr){3-5} \cmidrule(lr){6-8} \cmidrule(lr){9-11} \cmidrule(lr){12-14}
% & & P & R & F1 & P & R & F1 & P & R & F1 & P & R & F1 \\ \midrule
% Random & -- & 19.8 & 15.4 & 17.3 & 4.8 & 15.2 & 7.3 & 5.2 & 15.1 & 7.7 & 7.3 & 15.1 & 9.9 \\
% E5 | 561M & Zero-shot & 38.1 & 76.5 & 50.9 & 12.4 & 41.4 & 19.1 & 65.0 & 73.8 & 69.1 & 61.1 & 15.2 & 24.4 \\
% HF-SmolLM2 | 135M & Zero-shot & X & X & X & X & X & X & X & X & X & X & X & X \\
% HF-SmolLM2 | 365M & Zero-shot & 28.1 & 46.4 & 35.0 & 5.1 & 7.3 & 6.0 & 10.6 & 22.6 & 14.4 & 10.6 & 21.5 & 14.2 \\
% Llama | 1B & Zero-shot & 37.5 & 62.0 & 46.7 & 5.3 & 28.1 & 8.9 & 8.8 & 65.4 & 15.4 & 11.9 & 52.8 & 19.4 \\ %\midrule
% Mistral | 7B & Zero-shot & 71.5 & 94.4 & 81.4 & 42.9 & 42.7 & 42.8 & 63.2 & 88.6 & 73.8 & 59.0 & 72.2 & 64.9 \\
% Llama | 8B & Zero-shot & 71.9 & 96.9 & 82.6 & 49.2 & 52.9 & 51.0 & 64.1 & 92.1 & 75.6 & 55.7 & 76.4 & 64.4 \\
% Mistral NeMo | 12B & Zero-shot & 89.5 & 94.5 & 91.8 & 46.4 & 47.3 & 46.8 & 81.8 & 85.3 & 83.5 & 79.0 & 67.2 & 72.6 \\
% Ensemble | 27B & Zero-shot & 83.1 & 96.3 & 89.2 & 48.1 & 59.6 & 53.2 & 74.8 & 90.3 & 81.8 & 69.5 & 73.6 & 71.5 \\
% Llama | 70B & Zero-shot & 95.3 & 95.6 & 95.5 & 87.6 & 37.5 & 52.5 & 89.7 & 87.8 & 88.7 & 87.5 & 73.3 & 79.8 \\
% \midrule
% Llama | 1B & \squadcolor{SQ} \& \hotpotcolor{HP}; Disc. synthetic (1 pass) & 89.8 & 96.5 & 93.0 & 50.6 & 58.6 & 54.3 & 64.9 & 91.5 & 75.9 & 53.1 & 75.5 & 62.3 \\
% Llama | 1B & \squadcolor{SQ} \& \hotpotcolor{HP}; Disc. synthetic (1.0M) & 84.3 & 96.9 & 90.2 & 54.4 & 58.0 & 56.1 & 63.4 & 92.4 & 75.2 & 52.5 & 77.5 & 62.6 \\ \midrule
% Llama | 1B & \synqa (130K no dist.) & 87.1 & 88.2 & 87.6 & 67.9 & 44.8 & 54.0 & 85.3 & 82.5 & 83.9 & 72.7 & 63.8 & 68.0 \\
% Llama | 1B & \syntheticcolor{\synqa (130K)} & 89.9 & 90.7 & 90.3 & 87.9 & 63.5 & 73.7 & 88.7 & 85.4 & 87.0 & 77.2 & 65.5 & 70.9 \\
% Llama | 1B & \syntheticcolor{\synqa (550K)} & 93.6 & 94.5 & 94.0 & 88.9 & 67.3 & 76.6 & 89.8 & 87.9 & 88.9 & 77.1 & 67.1 & 71.8 \\
% Llama | 1B & \syntheticcolor{\synqa (700K)} & 95.1 & 95.5 & 95.3 & 87.8 & 69.6 & 77.6 & 93.6 & 89.1 & 91.3 & 82.0 & 68.9 & 74.9 \\
% Llama | 1B & \syntheticcolor{\synqa (1.0M)} & 96.0 & 96.2 & 96.1 & 89.6 & 69.4 & 78.2 & 93.3 & 89.1 & 91.1 & 82.3 & 68.5 & 74.8 \\
% New & \syntheticcolor{\synqa (1.0M)} & 95.7 & 96.9 & 96.3 & 89.6 & 66.4 & 76.3 & 91.8 & 91.5 & 91.6 & 80.8 & 71.3 & 75.8 \\
% Llama | 1B & \syntheticcolor{\synqa (1.0M with dialog data)} & -- & -- & -- & -- & -- & -- & -- & -- & -- & -- & -- & -- & \\
% \midrule
% HF-SmolLM2 | 135M & \synqa (690K) & X & X & X & X & X & X & X & X & X & X & X & X \\
% HF-SmolLM2 | 365M & \synqa (700K) & 73.9 & 74.2 & 74.1 & 85.2 & 68.7 & 76.0 & 79.6 & 76.5 & 78.0 & 68.8 & 59.8 & 64.0 \\ \midrule
% Llama | 1B & \squadcolor{SQ}; Gold (1 pass) & 98.4 & 98.4 & 98.4 & 48.7 & 20.0 & 28.4 & 92.6 & 85.8 & 89.0 & 79.9 & 64.3 & 71.2 \\
% Llama | 1B & HQ; Gold (1 pass) & 41.3 & 87.3 & 56.0 & 87.5 & 79.9 & 83.5 & 45.2 & 89.9 & 60.1 & 41.0 & 70.9 & 52.0 \\
% Llama | 1B & \squadcolor{SQ} \& \hotpotcolor{HQ}; Gold (1 pass) & 98.3 & 98.3 & 98.3 & 89.7 & 78.9 & 84.0 & 90.4 & 90.0 & 90.2 & 83.1 & 68.0 & 74.8 \\
% Llama | 1B & \squadcolor{SQ} \& \hotpotcolor{HQ}; Gold (1.0M) & 98.3 & 98.4 & 98.3 & 87.0 & 85.2 & 86.1 & 84.0 & 89.2 & 86.6 & 79.2 & 66.4 & 72.2 \\
% Llama | 1B & \syntheticcolor{\synqa (1.0M)} \& \squadcolor{SQ} \& \hotpotcolor{HQ}; Gold (1 pass) & 98.3 & 98.3 & 98.3 & 86.4 & 84.1 & 85.2 & 96.6 & 90.2 & 93.3 & 86.0 & 69.4 & 76.8 \\
% Llama | 1B & \syntheticcolor{\synqa (1.0M)} \& \squadcolor{SQ} \& \hotpotcolor{HQ}; Gold (1 pass) & 98.2 & 98.3 & 98.2 & 89.3 & 82.4 & 85.8 & 94.5 & 92.7 & 93.6 & 85.5 & 71.0 & 77.6 \\
% \midrule
% Llama | 1B & \synqa (1.0M) \& \squadcolor{SQ} \& \hotpotcolor{HQ} \& \quaccolor{Q} \& \coqacolor{CQ}; Gold (1 pass) & -- & -- & -- & -- & -- & -- & -- & -- & -- & -- & -- & -- & \\ \midrule
% HF-Smol | 365M & \syntheticcolor{\synqa (1.0M)} \& \squadcolor{SQ} \& \hotpotcolor{HQ}; Gold (1 pass) & 98.1 & 98.2 & 98.2 & 83.4 & 83.2 & 83.3 & 80.0 & 90.5 & 84.9 & 71.7 & 67.4 & 69.5 \\
% HF-Smol | 365M & \syntheticcolor{\synqa (1.0M)} \& \squadcolor{SQ} \& \hotpotcolor{HQ} \& \quaccolor{Q} \& \coqacolor{CQ}; Gold (1 pass) & 98.0 & 98.1 & 98.1 & 86.8 & 81.7 & 84.2 & 96.6 & 92.9 & 94.7 & 85.6 & 77.0 & 81.1 \\
% HF-Smol | 135M & \syntheticcolor{\synqa (1.0M)} \& \squadcolor{SQ} \& \hotpotcolor{HQ} \& \quaccolor{Q} \& \coqacolor{CQ}; Gold (1 pass) & 98.0 & 98.0 & 98.0 & 77.8 & 76.3 & 77.0 & 95.0 & 91.6 & 93.3 & 81.2 & 71.8 & 76.2 \\
% \bottomrule
% Llama | 1B & All; Gold & 1B   & 96.8 & 96.8 & 96.8 & 88.1 & 83.8 & 85.9 & 94.7 & 89.3 & 91.9 & 88.7 & 76.8 & 82.3 \\
% HF-SmolLM2 & All; Gold & 360M & 98.3 & 98.4 & 98.3 & 85.1 & 78.2 & 81.5 & 96.6 & 92.4 & 94.5 & 88.3 & 74.6 & 80.8 \\ \bottomrule
% \end{tabular}
% }
% \caption{Corroborative context-attribution of LMs on Squad QA, HotPot QA, Quac, and CoQA.}
% \label{table:all-datasets}
% \end{table*}
\section{Human Experiments}
\label{sec:human}

LLMs are trained on text produced by humans and are able to generate plausible text; therefore, there have been interests in using LLMs as human models \interalia{eisape-etal-2024-systematic,misra-kim-2024-generating}.
Following this line of work, we conduct a human behavioral experiment to ground the LLM reasoning behavior.
Using samples from our primary dataset, we collected 710 responses from adults fluent in English through Prolific.\footnote{\url{https://prolific.com}}
More experiment details can be found in \cref{subsec:human-details}.

The average human accuracy on each group is shown in the last row of \cref{tab:softacc-base}.\footnote{Human responses are binary classes, so correct and incorrect responses are coded as $1$ and $0$, respectively.}
Aligned with our LLM results (\cref{sec:experiment}), on modalities, the overall human results also show an accuracy order of ($\Diamond \succ \varnothing \succ \Box$),
and on argument forms, modus ponens ($\to^\mathrm{L}$) is the most accurately answered pattern.

To further investigate the interactions of logic factors, we fit a generalized linear mixed-effects model \citep{batesFittingLinearMixedEffects2015} to verify the effect of modality and argument forms on human logic reasoning accuracy (\cref{eqn:mixed-effects-human} and \cref{fig:emmeans-human}).
\noindent
\begin{align}
    \mathrm{logit}(\mathit{Acc}) & \sim \textit{Modality} + \textit{ArgForm} + \textit{Rt} \nonumber \\
                                 & + (1 + \textit{Rt} \mid \textit{ParticipantID}),
    \label{eqn:mixed-effects-human}
\end{align}
\noindent
where $\mathit{Acc}$ is the binary accuracy of human responses, and $\textit{Rt}$ is the response time.
The generalized mixed-effects model yields a marginal $R^2$ of $0.121$ yet a $0.419$ conditional $R^2$, indicating a diverse response pattern across participants.
The likelihood ratio test on the full model against the null model shows that only the effect of argument form is significant ($\chi^2(2)=25.6$, $p<0.001$).
However, in accordance with the overall performance, we find modus ponens ($\to^\mathrm{L}$) has a significantly higher effect than the other two valid argument forms.
This confirms that logical forms can also have a significant impact on human reasoning accuracy, which is consistent with the LLM results, although the effect sizes are not the same.

\begin{figure}[!t]
    \centering
    \vspace{-5pt}
    \includegraphics[
        width=0.95\columnwidth,
        keepaspectratio,
    ]{emmeans-human.pdf}
    \vspace{-5pt}
    \caption{
        Estimated marginal means of logical form factors in the generalized mixed-effects model of \cref{eqn:mixed-effects-human}, along with their 95\% confidence intervals.
        \label{fig:emmeans-human}
    }
    \vspace{-10pt}
\end{figure}

\section{Discussion}
\label{sec:discussion}

The results provide valuable insights into the limitations of machine learning (ML) models to support systematic literature review (SLR) updates. In this discussion, we interpret these results in light of the research questions, contextualize their implications, and outline the trade-offs associated with applying ML models in this domain.

\subsection{Effectiveness of ML Models for SLR Study Selection (RQ1)}

The results for RQ1 indicate that our best-performing model, Random Forest (RF), achieved a modest balance between precision and recall with an F-score of 0.33 at the default threshold of 0.5. This result suggests that while the ML model was able to identify some relevant studies, its overall ability to precisely distinguish between relevant and irrelevant studies was limited. Adjusting the threshold improved the F-score to 0.41, highlighting the sensitivity of the model’s performance to the chosen threshold. However, this improvement came at the cost of increasing false negatives (FNs), potentially missing valuable studies. We interpret the RF model’s performance as indicating that ML may assist in informally identifying a subset of relevant studies but is not yet reliable for the selection of studies for SLR updates.

\subsection{Effort Reduction through ML Models (RQ2)}

In answering RQ2, we focused on maximizing recall to avoid FNs. In our investigations, the SVM model was more suitable for focusing on achieving a high recall and demonstrating some potential for reducing human screening efforts. Results demonstrated that with a recall of 100\%, the SVM model could exclude 33.9\% of studies from the review process without missing any relevant studies. This reduction represents a significant decrease in the manual workload, suggesting ML’s potential to assist researchers with the initial screening stage. However, to achieve this high recall, the model produced a high rate of false positives (FPs), still requiring significant human review effort to discard many non-relevant studies.

As shown in Table \ref{tab:effort_reduction}, gradually increasing the inclusion probability threshold reduced the number of FPs at the cost of a minor drop in recall. For instance, at a threshold of 0.75, the model achieved a recall of 97.37\%, with a reduction of 48.3\% in the number of studies needing review. We interpret this result as indicating that, while ML can reduce screening efforts, care must be taken when applying thresholds to avoid introducing a risk of overlooking critical studies.

\subsection{Supporting Human Reviewers (RQ3)}

For RQ3, we evaluated the support ML could provide compared to that of an additional human reviewer. When we treated the RF model as an additional reviewer and calculated Euclidean Distance (ED) to assess alignment with the final inclusion decision, individual human reviewers outperformed the RF model. Furthermore, pairs of human reviewers clearly outperformed human-ML pairs, suggesting that human-only review teams achieve more accurate results.

This finding reinforces the challenges ML models face in fully replicating the nuanced judgment of human reviewers. Hence, ML can not replace additional human reviewers, and ML assistance is not a valid argument for quality in the selection process. Pairs of human reviewers are still highly recommended for selecting studies in SLR updates.
\section*{Limitations}
This work comes with two major limitations:
\begin{enumerate}[leftmargin=*,topsep=0pt,itemsep=0pt]
      \item While we have verified that our data has a low perplexity ($9.82\pm 2.47$ under mistral-7b; much lower than that of the data by \citet{wanLogicAskerEvaluatingImproving2024}, $25.44$), and, therefore, are similar enough to natural language utterances, the synthetic language cannot fully substitute natural language in daily life.
            Our dataset and analysis are not comprehensive enough to cover many nuanced examples that may appear in real communication, especially when context-dependent understanding is crucial to conveying communication goals.
      \item Despite more than 7,000 languages worldwide, as a first step, our material only covers English.
            This narrow focus is due to the languages the authors are proficient in and the coverage of the language models.
            We acknowledge the importance of extending the scope of this work to a more comprehensive set of languages and leave the extension as an immediate follow-up step.
\end{enumerate}

In addition, the sample size of human experiments is somewhat limited.
We leave more comprehensive human behavioral data collection and analysis to future work.

\section*{Ethics Statement}

While this work involves human logical reasoning experiments, we have ensured that (1) the data are generated procedurally following templates listed in the paper and (2) there is no harmful content in the atomic logical interpretations, reviewed by all the authors.
In addition, we have ensured that all participants are paid a fair wage through the Prolific platform.
Instructions and consent forms delivered to the participants can be found in the \cref{subsec:human-details}.
The institutional ethics review board has approved the data collection process.

This work contributes to the understanding of LLMs.
We do not foresee risk beyond the minimal risk posed by LLM evaluation work.
We acknowledge that using LLMs in real-world scenarios could significantly impact human behaviors, raising the need for model transparency, safety, security, and interpretability.
We will open-source the synthetic logical reasoning dataset upon publication.

\section{Acknowledgements}
We thank Yudong Li for his help in setting up the Gemini and OpenAI API for the experiments.
This work was supported in part by a Google PhD Fellowship and a Canada CIFAR AI Chair award to FS, as well as NSERC RGPIN-2024-04395.


%\freda{We will need to clean the bibliography before submitting it. Marking it as a todo for now.}
\bibliography{custom}

\appendix

\section{Additional Experiment Details}

\subsection{LLM Experiment Details}
\label{subsec:llm-details}
All LLMs used are obtained from \href{https://huggingface.co/models}{Hugging Face} checkpoints.
Time and compute power requirements vary, the largest llama-3-70b model takes around 2 hours on NVIDIA A6000 GPU to obtain all results in \cref{sec:experiment}.

\subsection{Human Experiment Details}
\label{subsec:human-details}
\paragraph{Participant instructions.}
We use keys \textit{F} and \textit{J}, which are roughly symmetric on a standard English keyboard, to collect participant responses.
Half of the participants see the following instruction:

\textit{In this study, you will be presented with two statements followed by a question. Your task is to answer either Yes or No to the question, based on the information provided in the statements.
    Please respond quickly and accurately by pressing "F" for Yes, and "J" for No.}

To mitigate the possible bias introduced by the dominant hand, we have the other half of the participants see instruction with reversed keys:

\textit{In this study, you will be presented with two statements followed by a question. Your task is to answer either Yes or No to the question, based on the information provided in the statements.
    Please respond quickly and accurately by pressing "F" for No, and "J" for Yes.}

\paragraph{Participant wage.}
We offer participants an hourly wage of 1.5 times Prolific's minimum wage.
The duration is determined by the median completion time among all participants.

\setcounter{table}{0}
\setcounter{figure}{0}
\setcounter{equation}{0}
\renewcommand{\thetable}{A\arabic{table}}
\renewcommand{\thefigure}{A\arabic{figure}}
\renewcommand{\theequation}{A\arabic{equation}}

\section{Extra Details of the Dataset}

\subsection{Considerations in Translating Logical Form to Natural Language}
\label{subsec:logic-translate-strategy}

During the interpretation process, another key point is to assign independent interpretations to variables.
Deciding the dependency also involves common sense knowledge.
For example, consider the premises $\lnot p \to q$ and $q$.
If we interpret $p \coloneqq\textit{``Jane is inside the house''}$ and $q\coloneqq\textit{``Jane is out''}$ to proposition variables $p$ and $q$, the two variables are possibly not independent.
According to common sense, ``\textit{Jane is not inside the house}'' ($\lnot p$) correlates with or is even equivalent to ``\textit{Jane is out}'' ($q$).
Logically, $\left\{\lnot p \to q, q\right\} \nvdash \lnot p$; however, with the extra premise $\lnot p \leftrightarrow q$ given by common sense, people may conclude that $\lnot p$.\footnote{
  This confounding factor affects the examples in Appendix C.1.12 of \citet{hollidayConditionalModalReasoning2024}.
}

Besides, natural language is ambiguous---one sentence in natural language can come from multiple logical forms under the same interpretation.
We use present tense and progressive aspect to encourage a reading of imaginary ongoing events, corresponding to the alethic modality.
Such events are less likely to induce LLM's or human's individual bias, as they are unrelated to factual knowledge or moral judgements.
Also, we always use two full verb phrases, ruling out sentences like ``\textit{Jane is eating apples or oranges,}'' so the two events are less likely to be mutually exclusive.
In this way, we can reduce the ambiguity of the questions in our dataset.

\subsection{Data Samples}

All logic forms and corresponding natural language sentences can be found in \cref{tab:question-full}.

The exact prompt format is as follows:

\begin{table}[H]
    \small
    \begin{tabular}{p{\linewidth}}
    Consider the following statements:\verb|\n|\\
    \uline{Jane is watching a show or John is reading a book.}\verb|\n|\\
    \uline{Jane isn't watching a show}.\verb|\n|\\
    Question: Based on these statements, can we infer that \uline{John is reading a book}?\verb|\n|\\
    Answer:\verb|<eof|>
    \end{tabular}
\end{table}
\vspace{-1em}

\newcommand{\subjectA}{\uline{Jane}}
\newcommand{\vpA}{\uline{watching a show}} 
\newcommand{\subjectB}{\uline{John}}
\newcommand{\vpB}{\uline{reading a book}}

\begin{table*}
\scriptsize
\begin{tabular}{
  @{}ccclp{0.52\textwidth}@{}
  }
\vspace{0.5em}
\textbf{Validity} & \textbf{Modality} & \textbf{Argument Form} & \textbf{Logical Form} & \textbf{Natural Language} \\
$\vdash$
& $\varnothing$ & $\lor^{\mathrm{L}}$ & $\{p \lor q, \lnot p\} \vdash q$ & 
   \subjectA{} is \vpA{} or \subjectB{} is \vpB{}.\newline
  \subjectA{} isn't \vpA{}.\newline
  Can we infer that \subjectB{} is \vpB{}? \\
& $\varnothing$ & $\lor^{\mathrm{R}}$ & $\{p \lor q, \lnot q\} \vdash p$ & 
   \subjectA{} is \vpA{} or \subjectB{} is \vpB{}.\newline
  \subjectB{} isn't \vpB{}.\newline
  Can we infer that \subjectA{} is \vpA{}? \\
& $\varnothing$ & $\rightarrow^{\mathrm{L}}$ & $\{\lnot p \to q, \lnot p\} \vdash q$ & 
   If \subjectA{} isn't \vpA{}, then \subjectB{} is \vpB{}.\newline
  \subjectA{} isn't \vpA{}.\newline
  Can we infer that \subjectB{} is \vpB{}? \\
& $\varnothing$ & $\rightarrow^{\mathrm{R}}$ & $\{\lnot p \to q, \lnot q\} \vdash p$ & 
   If \subjectA{} isn't \vpA{}, then \subjectB{} is \vpB{}.\newline
  \subjectB{} isn't \vpB{}.\newline
  Can we infer that \subjectA{} is \vpA{}? \\
& $\Box$ & $\lor^{\mathrm{L}}$ & $\{\Box p \lor \Box q, \lnot \Box p\} \vdash \Box q$ & 
   It's certain that \subjectA{} is \vpA{} or it's certain that \subjectB{} is \vpB{}.\newline
  It's uncertain whether \subjectA{} is \vpA{}.\newline
  Can we infer that it's certain that \subjectB{} is \vpB{}? \\
& $\Box$ & $\lor^{\mathrm{R}}$ & $\{\Box p \lor \Box q, \lnot \Box q\} \vdash \Box p$ & 
   It's certain that \subjectA{} is \vpA{} or it's certain that \subjectB{} is \vpB{}.\newline
  It's uncertain whether \subjectB{} is \vpB{}.\newline
  Can we infer that it's certain that \subjectA{} is \vpA{}? \\
& $\Box$ & $\rightarrow^{\mathrm{L}}$ & $\{\lnot \Box p \to \Box q, \lnot \Box p\} \vdash \Box q$ & 
   If it's uncertain whether \subjectA{} is \vpA{}, then it's certain that \subjectB{} is \vpB{}.\newline
  It's uncertain whether \subjectA{} is \vpA{}.\newline
  Can we infer that it's certain that \subjectB{} is \vpB{}? \\
& $\Box$ & $\rightarrow^{\mathrm{R}}$ & $\{\lnot \Box p \to \Box q, \lnot \Box q\} \vdash \Box p$ & 
   If it's uncertain whether \subjectA{} is \vpA{}, then it's certain that \subjectB{} is \vpB{}.\newline
  It's uncertain whether \subjectB{} is \vpB{}.\newline
  Can we infer that it's certain that \subjectA{} is \vpA{}? \\
& $\Diamond$ & $\lor^{\mathrm{L}}$ & $\{\Diamond p \lor \Diamond q, \lnot \Diamond p\} \vdash \Diamond q$ & 
   It's possible that \subjectA{} is \vpA{} or it's possible that \subjectB{} is \vpB{}.\newline
  It's impossible that \subjectA{} is \vpA{}.\newline
  Can we infer that it's possible that \subjectB{} is \vpB{}? \\
& $\Diamond$ & $\lor^{\mathrm{R}}$ & $\{\Diamond p \lor \Diamond q, \lnot \Diamond q\} \vdash \Diamond p$ & 
   It's possible that \subjectA{} is \vpA{} or it's possible that \subjectB{} is \vpB{}.\newline
  It's impossible that \subjectB{} is \vpB{}.\newline
  Can we infer that it's possible that \subjectA{} is \vpA{}? \\
& $\Diamond$ & $\rightarrow^{\mathrm{L}}$ & $\{\lnot \Diamond p \to \Diamond q, \lnot \Diamond p\} \vdash \Diamond q$ & 
   If it's impossible that \subjectA{} is \vpA{}, then it's possible that \subjectB{} is \vpB{}.\newline
  It's impossible that \subjectA{} is \vpA{}.\newline
  Can we infer that it's possible that \subjectB{} is \vpB{}? \\
& $\Diamond$ & $\rightarrow^{\mathrm{R}}$ & $\{\lnot \Diamond p \to \Diamond q, \lnot \Diamond q\} \vdash \Diamond p$ & 
   If it's impossible that \subjectA{} is \vpA{}, then it's possible that \subjectB{} is \vpB{}.\newline
  It's impossible that \subjectB{} is \vpB{}.\newline
  Can we infer that it's possible that \subjectA{} is \vpA{}? \\

$\nvdash$
& $\varnothing$ & $\lor^{\mathrm{L}}$ & $\{p \lor q, q\} \nvdash \lnot p$ & 
   \subjectA{} is \vpA{} or \subjectB{} is \vpB{}.\newline
  \subjectB{} is \vpB{}.\newline
  Can we infer that \subjectA{} isn't \vpA{}? \\
& $\varnothing$ & $\lor^{\mathrm{R}}$ & $\{p \lor q, p\} \nvdash \lnot q$ & 
   \subjectA{} is \vpA{} or \subjectB{} is \vpB{}.\newline
  \subjectA{} is \vpA{}.\newline
  Can we infer that \subjectB{} isn't \vpB{}? \\
& $\varnothing$ & $\rightarrow^{\mathrm{L}}$ & $\{\lnot p \to q, q\} \nvdash \lnot p$ & 
   If \subjectA{} isn't \vpA{}, then \subjectB{} is \vpB{}.\newline
  \subjectB{} is \vpB{}.\newline
  Can we infer that \subjectA{} isn't \vpA{}? \\
& $\varnothing$ & $\rightarrow^{\mathrm{R}}$ & $\{\lnot p \to q, p\} \nvdash \lnot q$ & 
   If \subjectA{} isn't \vpA{}, then \subjectB{} is \vpB{}.\newline
  \subjectA{} is \vpA{}.\newline
  Can we infer that \subjectB{} isn't \vpB{}? \\
& $\Box$ & $\lor^{\mathrm{L}}$ & $\{\Box p \lor \Box q, \Box q\} \nvdash \lnot \Box p$ & 
   It's certain that \subjectA{} is \vpA{} or it's certain that \subjectB{} is \vpB{}.\newline
  It's certain that \subjectB{} is \vpB{}.\newline
  Can we infer that it's uncertain whether \subjectA{} is \vpA{}? \\
& $\Box$ & $\lor^{\mathrm{R}}$ & $\{\Box p \lor \Box q, \Box p\} \nvdash \lnot \Box q$ & 
   It's certain that \subjectA{} is \vpA{} or it's certain that \subjectB{} is \vpB{}.\newline
  It's certain that \subjectA{} is \vpA{}.\newline
  Can we infer that it's uncertain whether \subjectB{} is \vpB{}? \\
& $\Box$ & $\rightarrow^{\mathrm{L}}$ & $\{\lnot \Box p \to \Box q, \Box q\} \nvdash \lnot \Box p$ & 
   If it's uncertain whether \subjectA{} is \vpA{}, then it's certain that \subjectB{} is \vpB{}.\newline
  It's certain that \subjectB{} is \vpB{}.\newline
  Can we infer that it's uncertain whether \subjectA{} is \vpA{}? \\
& $\Box$ & $\rightarrow^{\mathrm{R}}$ & $\{\lnot \Box p \to \Box q, \Box p\} \nvdash \lnot \Box q$ & 
   If it's uncertain whether \subjectA{} is \vpA{}, then it's certain that \subjectB{} is \vpB{}.\newline
  It's certain that \subjectA{} is \vpA{}.\newline
  Can we infer that it's uncertain whether \subjectB{} is \vpB{}? \\
& $\Diamond$ & $\lor^{\mathrm{L}}$ & $\{\Diamond p \lor \Diamond q, \Diamond q\} \nvdash \lnot \Diamond p$ & 
   It's possible that \subjectA{} is \vpA{} or it's possible that \subjectB{} is \vpB{}.\newline
  It's possible that \subjectB{} is \vpB{}.\newline
  Can we infer that it's impossible that \subjectA{} is \vpA{}? \\
& $\Diamond$ & $\lor^{\mathrm{R}}$ & $\{\Diamond p \lor \Diamond q, \Diamond p\} \nvdash \lnot \Diamond q$ & 
   It's possible that \subjectA{} is \vpA{} or it's possible that \subjectB{} is \vpB{}.\newline
  It's possible that \subjectA{} is \vpA{}.\newline
  Can we infer that it's impossible that \subjectB{} is \vpB{}? \\
& $\Diamond$ & $\rightarrow^{\mathrm{L}}$ & $\{\lnot \Diamond p \to \Diamond q, \Diamond q\} \nvdash \lnot \Diamond p$ & 
   If it's impossible that \subjectA{} is \vpA{}, then it's possible that \subjectB{} is \vpB{}.\newline
  It's possible that \subjectB{} is \vpB{}.\newline
  Can we infer that it's impossible that \subjectA{} is \vpA{}? \\
& $\Diamond$ & $\rightarrow^{\mathrm{R}}$ & $\{\lnot \Diamond p \to \Diamond q, \Diamond p\} \nvdash \lnot \Diamond q$ & 
   If it's impossible that \subjectA{} is \vpA{}, then it's possible that \subjectB{} is \vpB{}.\newline
  It's possible that \subjectA{} is \vpA{}.\newline
  Can we infer that it's impossible that \subjectB{} is \vpB{}? \\
\end{tabular}
\caption{\textbf{Samples of all logical forms and corresponding natural language sentences.}}
\label{tab:question-full}
\end{table*}

\section{Additional Experiments}

\iffalse

\subsection{Supplementary Figures for \cref{sec:experiment}}

\cref{fig:heatmap-modality-full} shows exaustive test on every contrasitive pair of modal words.
Additionally, modality’s effect on the next-token probabilities of answer token \texttt{No} is shown in \cref{fig:heatmap-modality-no}.

\cref{fig:corr-ppl-acc-full} extends \cref{fig:corr-ppl-acc} to correlation between perplexity and accuracy for all models.

\begin{figure*}
    \centering
    \begin{subfigure}[b]{0.48\textwidth}
        \centering
        \includegraphics[width=1\textwidth,keepaspectratio]{heatmap-modality-full-yes.pdf}
        \caption{
            On token \texttt{Yes}.
        }
    \end{subfigure}
    \begin{subfigure}[b]{0.48\textwidth}
        \centering
        \includegraphics[width=1\textwidth,keepaspectratio]{heatmap-modality-full-no.pdf}
        \caption{
            On token \texttt{No}.
        }
        \label{fig:heatmap-modality-no}
    \end{subfigure}
    \caption{
        \textbf{Continuation of \cref{fig:heatmap-modality}}.
    }
    \label{fig:heatmap-modality-full}
\end{figure*}

\begin{figure*}
    \centering
    \includegraphics[width=1\textwidth,keepaspectratio]{corr-ppl-acc-full.pdf}
    \caption{
        \textbf{Continuation of \cref{fig:corr-ppl-acc}}.
    }
    \label{fig:corr-ppl-acc-full}
\end{figure*}

\fi


\subsection{Extra Experiment: Introduction Rule of Modality}
\label{subsec:extra-intro-modality}

We report the results on the necessitation rule and its variants here, as these rules are obscure and verbose to be articulated in natural language:
%
\begin{align}
    \left\{\varphi\right\} &\vdash \Box \varphi, \tag{necessitation rule} \label{eqn:necessitation-rule} \\
    \left\{\varphi\right\} &\vdash \Diamond \varphi, \nonumber \\
    \left\{\varphi\right\} &\vdash \varphi. \nonumber
\end{align}
%
Its natural language form is as follows:
\begin{table}[H]
    \small
    \begin{tabular}{rp{0.8\linewidth}}
    & Jane is watching a show.\\
    ($\Box$) & Can we infer that it's certain that Jane is watching a show?\\
    ($\Diamond$) & Can we infer that it's possible that Jane is watching a show?\\
    ($\varnothing$) & Can we infer that Jane is watching a show?\\
    \end{tabular}
\end{table}

All three variants are paired with 1000 logic interpretations.
As they are all rules of inference, the ground truth answer is always \texttt{Yes}.
Overall accuracy is shown in \cref{tab:softacc-necessitate},
where across all LLMs, the necessitation rule has the lowest accuracy.
This echoes the necessity modality's tendency to be rejected discussed in \cref{subsec:affirmation-bias}.

\begin{table}[t]
    
\small
\centering
\begin{tabular*}{0.85\columnwidth}{
 @{\extracolsep{\fill}}lccc@{}
 }
 \toprule
  & $\varnothing$ & $\Box$ & $\Diamond$ \\
 \midrule
    mistral-7b & 0.998 & 0.885 & 0.999\\
    mistral-8x7b & 0.957 & 0.540 & 0.987\\
    llama-2-7b & 0.768 & 0.013 & 0.920\\
    llama-2-13b & 0.368 & 0.004 & 0.829\\
    llama-2-70b & 0.511 & 0.051 & 0.834\\
    llama-3-8b & 0.398 & 0.225 & 0.783\\
    llama-3-70b & 0.674 & 0.384 & 0.794\\
    yi-34b & 0.960 & 0.382 & 0.999\\
    phi-2 & 0.814 & 0.226 & 0.892\\
    phi-3-mini & 0.992 & 0.925 & 0.994\\
 \bottomrule
\end{tabular*}
    \caption{
        \label{tab:softacc-necessitate}
        Overall accuracy of the necessitation rule and its modality variants on each model.
    }
\end{table}
    
We further fit a linear mixed-effects model similar to \cref{eqn: mixed-effects}, except that the argument form effect is now constant across all data points.
The mixed-effects model yields a marginal $R^2$ of $0.391$ and a conditional $R^2$ of $0.745$.
Estimated marginal means shows that the accuracy on $\varnothing$ is $0.171$ less than $\Diamond$, but $0.371$ higher than $\Box$, with both differences significant at $p < 0.0001$.
This further suggests that modality serves as an important factor on logic reasoning performance.

\subsection{Extra Experiment: Distribution of Modalities}

Besides the necessitation rule, \textit{distribution axiom} is the other fundamental axiom in normal modal logic.
It can be transformed into the rule shown in \cref{eqn:distribution-must-axiom},
and plugging in the definition of $\lor$ in \cref{eqn:def-lor} gives the rule shown in \cref{eqn:distribution-or-theorem}.
Notice that \cref{eqn:distribution-or-theorem} closely resembles rule \ref{eqn:inf-rule-or-left}'s variant with necessity, as shown in \cref{eqn:inf-rule-or-left-must},
except the different scope of the necessity operator and the position of the negation operator.
Moving the negation operator out of the necessity operator will result in a fallacy (Eq. \ref{eqn:distribution-or-fallacy}).
\noindent
\begin{align}
    \left\{\Box(\varphi \to \psi), \Box \varphi\right\} &\vdash \Box \psi, 
    \label{eqn:distribution-must-axiom} \\
    \left\{\Box(\varphi \lor \psi), \Box \lnot \varphi\right\} &\vdash \Box \psi \label{eqn:distribution-or-theorem}, \\
    \left\{\Box \varphi \lor \Box \psi, \lnot \Box \varphi\right\} &\vdash \Box \psi \label{eqn:inf-rule-or-left-must}, \\
    \left\{\Box(\varphi \lor \psi), \lnot \Box \varphi\right\} &\nvdash \Box \psi \label{eqn:distribution-or-fallacy}.
\end{align}
\noindent
We say \cref{eqn:distribution-or-theorem,eqn:inf-rule-or-left-must,eqn:distribution-or-fallacy} are of argument form \texttt{theorem}, \texttt{base} and \texttt{spurious}, respectively.
See \cref{tab:extra-distribution-forms} for the logical forms and their ground truth we used to study the distribution of modalities.
The natural language form is as follows:
\begin{table}[H]
    \small
    \begin{tabular}{rp{0.65\linewidth}}
    & It's certain that if Freddy is not going shopping, then Coy is making dinner.\\
    (\texttt{theorem}) & It's certain that Freddy is not going shopping.\\
    (\texttt{spurious}) & It's uncertain whether Freddy is going shopping.\\
    & Can we infer that it's certain that Coy is making dinner?\\
    \end{tabular}
\end{table}

This group of rules and fallacies comes from the fact that the necessity modality $\Box$ is not distributive to disjunction, i.e. $\Box (\varphi \lor \psi) \nvdash \Box \varphi \lor \Box \psi$ \citep[Ex. 5]{xiang-2019a-two-types}.
In contrast, the possibility modality $\Diamond$ is distributive to disjunction.
This particular case could have served as a material to test the LLM's knowledge of the asymmetry between the two modalities,
yet in \cref{subsec:affirmation-bias} we showed that there is a bias towards rejection on the necessity modality.
As the false case of the disjunction is on the necessity modality, this bias confounds the experiment.


\begin{table}[t]
    \small\centering
\begin{tabular}{ccl}
    \toprule
Modality & Argument Form & Logical Form \\
    \midrule
$\varnothing$ & base & $\varphi \lor \psi, \lnot \varphi \vdash \psi$ \\
$\Box$ & base & $\Box \varphi \lor \Box \psi, \lnot \Box \varphi \vdash \Box \psi$ \\
$\Box$ & theorem & $\Box ( \varphi \lor \psi), \Box \lnot \varphi \vdash \Box \psi$ \\
$\Box$ & \uline{spurious} & $\Box ( \varphi \lor \psi), \lnot \Box \varphi \nvdash \Box \psi$ \\
$\Diamond$ & base & $\Diamond \varphi \lor \Diamond \psi, \lnot \Diamond \varphi \vdash \Diamond \psi$ \\
$\Diamond$ & theorem & $\Diamond ( \varphi \lor \psi), \Diamond \lnot \varphi \vdash \Diamond \psi$ \\
$\Diamond$ & spurious & $\Diamond ( \varphi \lor \psi), \lnot \Diamond \varphi \vdash \Diamond \psi$ \\
% \cmidrule{1-3}
% $\varnothing$ & $\lnot \varphi \to \psi, \lnot \varphi \vdash \psi$ \\
% $\Box$ & $\Box ( \lnot \varphi \to \psi), \Box \lnot \varphi \vdash \Box \psi$ \\
% $\Box$ & False & $\Box ( \lnot \varphi \to \psi), \lnot \Box \varphi \nvdash \Box \psi$ \\
% $\Diamond$ & $\Diamond ( \lnot \varphi \to \psi), \Diamond \lnot \varphi \vdash \Diamond \psi$ \\
% $\Diamond$ & $\Diamond ( \lnot \varphi \to \psi), \lnot \Diamond \varphi \vdash \Diamond \psi$ \\
    \bottomrule
    \end{tabular}
    \caption{
        \label{tab:extra-distribution-forms}
        Logical forms and their ground truth to study the distribution of modalities.
        Only the spurious form of the necessity modality (marked by \uline{underline}) has a ground truth of false.
    }
\end{table}

We fit a linear mixed-effects model similar to \cref{eqn: mixed-effects} to the data,
\noindent
\begin{align}
    \mathit{Acc}_\textit{soft} & \sim \textit{Modality} \times \textit{ArgForm} + \textit{Perplexity} \nonumber \\
                               & + (1 + \textit{Perplexity} \mid \textit{LLM}), \nonumber
\end{align}
\noindent
with an interaction term between the modality and argument form.
% The model yields a conditional $R^2$ of $0.397$.
On the \texttt{theorem} form compared to the \texttt{base} form, the necessity modality $\Box$ has a $0.173$ higher estimated marginal means with $p < 0.0001$ significance, yet the possibility modality $\Diamond$ has a $0.071$ lower estimated marginal means.
On the \texttt{spurious} form compared to the \texttt{base} form, the $\Box$ has a $0.312$ higher means, and the $\Diamond$ has no significant difference.
On both forms, $\Diamond \succ \Box$ in terms of accuracy still holds at a slight margin of $0.110$ and $0.047$ respectively.

To verify whether on $\Box$ the performance increase on \texttt{spurious} form is due to the rejection bias, we fit a linear mixed-effects model with the relative probability of answering \texttt{Yes} as dependent variable.
Results show that on \texttt{spurious} form compared to the \texttt{base} form, the effect of $\Box$'s tendency to answer \texttt{Yes} is only $0.060$ lower, indicating the rejection bias of the \texttt{base} form is still present.
Therefore, we hypothesize that the LLM's performance on recognizing the fallacy of necessity distribution over disjunction is hindered by the rejection bias on the necessity modality.

\end{document}
