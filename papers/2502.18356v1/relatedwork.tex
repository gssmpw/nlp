\section{Related Work}
Autonomous agent evaluation frameworks have progressed significantly, beginning with traditional reinforcement learning environments \citep{brockman2016openai}, and expanding into complete web domains \citep{shi2017world, liu2018reinforcement}. 
A significant challenge in benchmark design has been balancing comprehensiveness with practicality. Traditional benchmarks often focus on single-turn or short-context scenarios, which can lead to rapid benchmark saturation \citep{kiela2021dynabench} and may not fully capture the capabilities needed for effective agentic foundation models.

Modern web interaction requires a complex mix of capabilities including tool usage, planning, environmental reasoning, and practical task execution. This has led to 
recent advancements introducing benchmarks for static webpage interaction \citep{deng2024mind2web} as well as specialized evaluation frameworks across various domains, from office-related tasks \citep{liu2023agentbench, qin2024sysbench} to web navigation \citep{yao2022webshop, zhou2023webarena} and GitHub issue resolution \citep{jimenez2023swe}.

Multi-agent interaction represents an emerging frontier in this space. Recent research has explored LLMs' capabilities in both cooperative \citep{gong2023mindagent, piatti2024cooperate} and competitive \citep{jin2024learning, wu2024enhance} scenarios. This work highlights the importance of evaluating not just isolated capabilities, but also agents' ability to interact effectively with other autonomous systems.
 
\name makes a couple of key distinctions in order to provide consistent and meaningful evaluation. Unlike task sets such as WebVoyager \citep{he2024webvoyager}, that require models to use the regular internet, it maintains a hermetic testing environment, eliminating external dependencies and network variables. This controlled local context also ensures reproducible evaluation by providing verifiable ground-truth solutions. Compared to other hosted benchmarks like WebArena \citep{zhou2023webarena}, it offers reduced operational overhead as it is significantly simpler to deploy locally, while also maintaining public accessibility via \url{https://webgames.convergence.ai}.