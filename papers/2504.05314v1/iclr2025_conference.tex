
\documentclass{article} % For LaTeX2e
\usepackage{iclr2025_conference,times}

% Optional math commands from https://github.com/goodfeli/dlbook_notation.
%%%%% NEW MATH DEFINITIONS %%%%%

\usepackage{amsmath,amsfonts,bm}
\usepackage{derivative}
% Mark sections of captions for referring to divisions of figures
\newcommand{\figleft}{{\em (Left)}}
\newcommand{\figcenter}{{\em (Center)}}
\newcommand{\figright}{{\em (Right)}}
\newcommand{\figtop}{{\em (Top)}}
\newcommand{\figbottom}{{\em (Bottom)}}
\newcommand{\captiona}{{\em (a)}}
\newcommand{\captionb}{{\em (b)}}
\newcommand{\captionc}{{\em (c)}}
\newcommand{\captiond}{{\em (d)}}

% Highlight a newly defined term
\newcommand{\newterm}[1]{{\bf #1}}

% Derivative d 
\newcommand{\deriv}{{\mathrm{d}}}

% Figure reference, lower-case.
\def\figref#1{figure~\ref{#1}}
% Figure reference, capital. For start of sentence
\def\Figref#1{Figure~\ref{#1}}
\def\twofigref#1#2{figures \ref{#1} and \ref{#2}}
\def\quadfigref#1#2#3#4{figures \ref{#1}, \ref{#2}, \ref{#3} and \ref{#4}}
% Section reference, lower-case.
\def\secref#1{section~\ref{#1}}
% Section reference, capital.
\def\Secref#1{Section~\ref{#1}}
% Reference to two sections.
\def\twosecrefs#1#2{sections \ref{#1} and \ref{#2}}
% Reference to three sections.
\def\secrefs#1#2#3{sections \ref{#1}, \ref{#2} and \ref{#3}}
% Reference to an equation, lower-case.
\def\eqref#1{equation~\ref{#1}}
% Reference to an equation, upper case
\def\Eqref#1{Equation~\ref{#1}}
% A raw reference to an equation---avoid using if possible
\def\plaineqref#1{\ref{#1}}
% Reference to a chapter, lower-case.
\def\chapref#1{chapter~\ref{#1}}
% Reference to an equation, upper case.
\def\Chapref#1{Chapter~\ref{#1}}
% Reference to a range of chapters
\def\rangechapref#1#2{chapters\ref{#1}--\ref{#2}}
% Reference to an algorithm, lower-case.
\def\algref#1{algorithm~\ref{#1}}
% Reference to an algorithm, upper case.
\def\Algref#1{Algorithm~\ref{#1}}
\def\twoalgref#1#2{algorithms \ref{#1} and \ref{#2}}
\def\Twoalgref#1#2{Algorithms \ref{#1} and \ref{#2}}
% Reference to a part, lower case
\def\partref#1{part~\ref{#1}}
% Reference to a part, upper case
\def\Partref#1{Part~\ref{#1}}
\def\twopartref#1#2{parts \ref{#1} and \ref{#2}}

\def\ceil#1{\lceil #1 \rceil}
\def\floor#1{\lfloor #1 \rfloor}
\def\1{\bm{1}}
\newcommand{\train}{\mathcal{D}}
\newcommand{\valid}{\mathcal{D_{\mathrm{valid}}}}
\newcommand{\test}{\mathcal{D_{\mathrm{test}}}}

\def\eps{{\epsilon}}


% Random variables
\def\reta{{\textnormal{$\eta$}}}
\def\ra{{\textnormal{a}}}
\def\rb{{\textnormal{b}}}
\def\rc{{\textnormal{c}}}
\def\rd{{\textnormal{d}}}
\def\re{{\textnormal{e}}}
\def\rf{{\textnormal{f}}}
\def\rg{{\textnormal{g}}}
\def\rh{{\textnormal{h}}}
\def\ri{{\textnormal{i}}}
\def\rj{{\textnormal{j}}}
\def\rk{{\textnormal{k}}}
\def\rl{{\textnormal{l}}}
% rm is already a command, just don't name any random variables m
\def\rn{{\textnormal{n}}}
\def\ro{{\textnormal{o}}}
\def\rp{{\textnormal{p}}}
\def\rq{{\textnormal{q}}}
\def\rr{{\textnormal{r}}}
\def\rs{{\textnormal{s}}}
\def\rt{{\textnormal{t}}}
\def\ru{{\textnormal{u}}}
\def\rv{{\textnormal{v}}}
\def\rw{{\textnormal{w}}}
\def\rx{{\textnormal{x}}}
\def\ry{{\textnormal{y}}}
\def\rz{{\textnormal{z}}}

% Random vectors
\def\rvepsilon{{\mathbf{\epsilon}}}
\def\rvphi{{\mathbf{\phi}}}
\def\rvtheta{{\mathbf{\theta}}}
\def\rva{{\mathbf{a}}}
\def\rvb{{\mathbf{b}}}
\def\rvc{{\mathbf{c}}}
\def\rvd{{\mathbf{d}}}
\def\rve{{\mathbf{e}}}
\def\rvf{{\mathbf{f}}}
\def\rvg{{\mathbf{g}}}
\def\rvh{{\mathbf{h}}}
\def\rvu{{\mathbf{i}}}
\def\rvj{{\mathbf{j}}}
\def\rvk{{\mathbf{k}}}
\def\rvl{{\mathbf{l}}}
\def\rvm{{\mathbf{m}}}
\def\rvn{{\mathbf{n}}}
\def\rvo{{\mathbf{o}}}
\def\rvp{{\mathbf{p}}}
\def\rvq{{\mathbf{q}}}
\def\rvr{{\mathbf{r}}}
\def\rvs{{\mathbf{s}}}
\def\rvt{{\mathbf{t}}}
\def\rvu{{\mathbf{u}}}
\def\rvv{{\mathbf{v}}}
\def\rvw{{\mathbf{w}}}
\def\rvx{{\mathbf{x}}}
\def\rvy{{\mathbf{y}}}
\def\rvz{{\mathbf{z}}}

% Elements of random vectors
\def\erva{{\textnormal{a}}}
\def\ervb{{\textnormal{b}}}
\def\ervc{{\textnormal{c}}}
\def\ervd{{\textnormal{d}}}
\def\erve{{\textnormal{e}}}
\def\ervf{{\textnormal{f}}}
\def\ervg{{\textnormal{g}}}
\def\ervh{{\textnormal{h}}}
\def\ervi{{\textnormal{i}}}
\def\ervj{{\textnormal{j}}}
\def\ervk{{\textnormal{k}}}
\def\ervl{{\textnormal{l}}}
\def\ervm{{\textnormal{m}}}
\def\ervn{{\textnormal{n}}}
\def\ervo{{\textnormal{o}}}
\def\ervp{{\textnormal{p}}}
\def\ervq{{\textnormal{q}}}
\def\ervr{{\textnormal{r}}}
\def\ervs{{\textnormal{s}}}
\def\ervt{{\textnormal{t}}}
\def\ervu{{\textnormal{u}}}
\def\ervv{{\textnormal{v}}}
\def\ervw{{\textnormal{w}}}
\def\ervx{{\textnormal{x}}}
\def\ervy{{\textnormal{y}}}
\def\ervz{{\textnormal{z}}}

% Random matrices
\def\rmA{{\mathbf{A}}}
\def\rmB{{\mathbf{B}}}
\def\rmC{{\mathbf{C}}}
\def\rmD{{\mathbf{D}}}
\def\rmE{{\mathbf{E}}}
\def\rmF{{\mathbf{F}}}
\def\rmG{{\mathbf{G}}}
\def\rmH{{\mathbf{H}}}
\def\rmI{{\mathbf{I}}}
\def\rmJ{{\mathbf{J}}}
\def\rmK{{\mathbf{K}}}
\def\rmL{{\mathbf{L}}}
\def\rmM{{\mathbf{M}}}
\def\rmN{{\mathbf{N}}}
\def\rmO{{\mathbf{O}}}
\def\rmP{{\mathbf{P}}}
\def\rmQ{{\mathbf{Q}}}
\def\rmR{{\mathbf{R}}}
\def\rmS{{\mathbf{S}}}
\def\rmT{{\mathbf{T}}}
\def\rmU{{\mathbf{U}}}
\def\rmV{{\mathbf{V}}}
\def\rmW{{\mathbf{W}}}
\def\rmX{{\mathbf{X}}}
\def\rmY{{\mathbf{Y}}}
\def\rmZ{{\mathbf{Z}}}

% Elements of random matrices
\def\ermA{{\textnormal{A}}}
\def\ermB{{\textnormal{B}}}
\def\ermC{{\textnormal{C}}}
\def\ermD{{\textnormal{D}}}
\def\ermE{{\textnormal{E}}}
\def\ermF{{\textnormal{F}}}
\def\ermG{{\textnormal{G}}}
\def\ermH{{\textnormal{H}}}
\def\ermI{{\textnormal{I}}}
\def\ermJ{{\textnormal{J}}}
\def\ermK{{\textnormal{K}}}
\def\ermL{{\textnormal{L}}}
\def\ermM{{\textnormal{M}}}
\def\ermN{{\textnormal{N}}}
\def\ermO{{\textnormal{O}}}
\def\ermP{{\textnormal{P}}}
\def\ermQ{{\textnormal{Q}}}
\def\ermR{{\textnormal{R}}}
\def\ermS{{\textnormal{S}}}
\def\ermT{{\textnormal{T}}}
\def\ermU{{\textnormal{U}}}
\def\ermV{{\textnormal{V}}}
\def\ermW{{\textnormal{W}}}
\def\ermX{{\textnormal{X}}}
\def\ermY{{\textnormal{Y}}}
\def\ermZ{{\textnormal{Z}}}

% Vectors
\def\vzero{{\bm{0}}}
\def\vone{{\bm{1}}}
\def\vmu{{\bm{\mu}}}
\def\vtheta{{\bm{\theta}}}
\def\vphi{{\bm{\phi}}}
\def\va{{\bm{a}}}
\def\vb{{\bm{b}}}
\def\vc{{\bm{c}}}
\def\vd{{\bm{d}}}
\def\ve{{\bm{e}}}
\def\vf{{\bm{f}}}
\def\vg{{\bm{g}}}
\def\vh{{\bm{h}}}
\def\vi{{\bm{i}}}
\def\vj{{\bm{j}}}
\def\vk{{\bm{k}}}
\def\vl{{\bm{l}}}
\def\vm{{\bm{m}}}
\def\vn{{\bm{n}}}
\def\vo{{\bm{o}}}
\def\vp{{\bm{p}}}
\def\vq{{\bm{q}}}
\def\vr{{\bm{r}}}
\def\vs{{\bm{s}}}
\def\vt{{\bm{t}}}
\def\vu{{\bm{u}}}
\def\vv{{\bm{v}}}
\def\vw{{\bm{w}}}
\def\vx{{\bm{x}}}
\def\vy{{\bm{y}}}
\def\vz{{\bm{z}}}

% Elements of vectors
\def\evalpha{{\alpha}}
\def\evbeta{{\beta}}
\def\evepsilon{{\epsilon}}
\def\evlambda{{\lambda}}
\def\evomega{{\omega}}
\def\evmu{{\mu}}
\def\evpsi{{\psi}}
\def\evsigma{{\sigma}}
\def\evtheta{{\theta}}
\def\eva{{a}}
\def\evb{{b}}
\def\evc{{c}}
\def\evd{{d}}
\def\eve{{e}}
\def\evf{{f}}
\def\evg{{g}}
\def\evh{{h}}
\def\evi{{i}}
\def\evj{{j}}
\def\evk{{k}}
\def\evl{{l}}
\def\evm{{m}}
\def\evn{{n}}
\def\evo{{o}}
\def\evp{{p}}
\def\evq{{q}}
\def\evr{{r}}
\def\evs{{s}}
\def\evt{{t}}
\def\evu{{u}}
\def\evv{{v}}
\def\evw{{w}}
\def\evx{{x}}
\def\evy{{y}}
\def\evz{{z}}

% Matrix
\def\mA{{\bm{A}}}
\def\mB{{\bm{B}}}
\def\mC{{\bm{C}}}
\def\mD{{\bm{D}}}
\def\mE{{\bm{E}}}
\def\mF{{\bm{F}}}
\def\mG{{\bm{G}}}
\def\mH{{\bm{H}}}
\def\mI{{\bm{I}}}
\def\mJ{{\bm{J}}}
\def\mK{{\bm{K}}}
\def\mL{{\bm{L}}}
\def\mM{{\bm{M}}}
\def\mN{{\bm{N}}}
\def\mO{{\bm{O}}}
\def\mP{{\bm{P}}}
\def\mQ{{\bm{Q}}}
\def\mR{{\bm{R}}}
\def\mS{{\bm{S}}}
\def\mT{{\bm{T}}}
\def\mU{{\bm{U}}}
\def\mV{{\bm{V}}}
\def\mW{{\bm{W}}}
\def\mX{{\bm{X}}}
\def\mY{{\bm{Y}}}
\def\mZ{{\bm{Z}}}
\def\mBeta{{\bm{\beta}}}
\def\mPhi{{\bm{\Phi}}}
\def\mLambda{{\bm{\Lambda}}}
\def\mSigma{{\bm{\Sigma}}}

% Tensor
\DeclareMathAlphabet{\mathsfit}{\encodingdefault}{\sfdefault}{m}{sl}
\SetMathAlphabet{\mathsfit}{bold}{\encodingdefault}{\sfdefault}{bx}{n}
\newcommand{\tens}[1]{\bm{\mathsfit{#1}}}
\def\tA{{\tens{A}}}
\def\tB{{\tens{B}}}
\def\tC{{\tens{C}}}
\def\tD{{\tens{D}}}
\def\tE{{\tens{E}}}
\def\tF{{\tens{F}}}
\def\tG{{\tens{G}}}
\def\tH{{\tens{H}}}
\def\tI{{\tens{I}}}
\def\tJ{{\tens{J}}}
\def\tK{{\tens{K}}}
\def\tL{{\tens{L}}}
\def\tM{{\tens{M}}}
\def\tN{{\tens{N}}}
\def\tO{{\tens{O}}}
\def\tP{{\tens{P}}}
\def\tQ{{\tens{Q}}}
\def\tR{{\tens{R}}}
\def\tS{{\tens{S}}}
\def\tT{{\tens{T}}}
\def\tU{{\tens{U}}}
\def\tV{{\tens{V}}}
\def\tW{{\tens{W}}}
\def\tX{{\tens{X}}}
\def\tY{{\tens{Y}}}
\def\tZ{{\tens{Z}}}


% Graph
\def\gA{{\mathcal{A}}}
\def\gB{{\mathcal{B}}}
\def\gC{{\mathcal{C}}}
\def\gD{{\mathcal{D}}}
\def\gE{{\mathcal{E}}}
\def\gF{{\mathcal{F}}}
\def\gG{{\mathcal{G}}}
\def\gH{{\mathcal{H}}}
\def\gI{{\mathcal{I}}}
\def\gJ{{\mathcal{J}}}
\def\gK{{\mathcal{K}}}
\def\gL{{\mathcal{L}}}
\def\gM{{\mathcal{M}}}
\def\gN{{\mathcal{N}}}
\def\gO{{\mathcal{O}}}
\def\gP{{\mathcal{P}}}
\def\gQ{{\mathcal{Q}}}
\def\gR{{\mathcal{R}}}
\def\gS{{\mathcal{S}}}
\def\gT{{\mathcal{T}}}
\def\gU{{\mathcal{U}}}
\def\gV{{\mathcal{V}}}
\def\gW{{\mathcal{W}}}
\def\gX{{\mathcal{X}}}
\def\gY{{\mathcal{Y}}}
\def\gZ{{\mathcal{Z}}}

% Sets
\def\sA{{\mathbb{A}}}
\def\sB{{\mathbb{B}}}
\def\sC{{\mathbb{C}}}
\def\sD{{\mathbb{D}}}
% Don't use a set called E, because this would be the same as our symbol
% for expectation.
\def\sF{{\mathbb{F}}}
\def\sG{{\mathbb{G}}}
\def\sH{{\mathbb{H}}}
\def\sI{{\mathbb{I}}}
\def\sJ{{\mathbb{J}}}
\def\sK{{\mathbb{K}}}
\def\sL{{\mathbb{L}}}
\def\sM{{\mathbb{M}}}
\def\sN{{\mathbb{N}}}
\def\sO{{\mathbb{O}}}
\def\sP{{\mathbb{P}}}
\def\sQ{{\mathbb{Q}}}
\def\sR{{\mathbb{R}}}
\def\sS{{\mathbb{S}}}
\def\sT{{\mathbb{T}}}
\def\sU{{\mathbb{U}}}
\def\sV{{\mathbb{V}}}
\def\sW{{\mathbb{W}}}
\def\sX{{\mathbb{X}}}
\def\sY{{\mathbb{Y}}}
\def\sZ{{\mathbb{Z}}}

% Entries of a matrix
\def\emLambda{{\Lambda}}
\def\emA{{A}}
\def\emB{{B}}
\def\emC{{C}}
\def\emD{{D}}
\def\emE{{E}}
\def\emF{{F}}
\def\emG{{G}}
\def\emH{{H}}
\def\emI{{I}}
\def\emJ{{J}}
\def\emK{{K}}
\def\emL{{L}}
\def\emM{{M}}
\def\emN{{N}}
\def\emO{{O}}
\def\emP{{P}}
\def\emQ{{Q}}
\def\emR{{R}}
\def\emS{{S}}
\def\emT{{T}}
\def\emU{{U}}
\def\emV{{V}}
\def\emW{{W}}
\def\emX{{X}}
\def\emY{{Y}}
\def\emZ{{Z}}
\def\emSigma{{\Sigma}}

% entries of a tensor
% Same font as tensor, without \bm wrapper
\newcommand{\etens}[1]{\mathsfit{#1}}
\def\etLambda{{\etens{\Lambda}}}
\def\etA{{\etens{A}}}
\def\etB{{\etens{B}}}
\def\etC{{\etens{C}}}
\def\etD{{\etens{D}}}
\def\etE{{\etens{E}}}
\def\etF{{\etens{F}}}
\def\etG{{\etens{G}}}
\def\etH{{\etens{H}}}
\def\etI{{\etens{I}}}
\def\etJ{{\etens{J}}}
\def\etK{{\etens{K}}}
\def\etL{{\etens{L}}}
\def\etM{{\etens{M}}}
\def\etN{{\etens{N}}}
\def\etO{{\etens{O}}}
\def\etP{{\etens{P}}}
\def\etQ{{\etens{Q}}}
\def\etR{{\etens{R}}}
\def\etS{{\etens{S}}}
\def\etT{{\etens{T}}}
\def\etU{{\etens{U}}}
\def\etV{{\etens{V}}}
\def\etW{{\etens{W}}}
\def\etX{{\etens{X}}}
\def\etY{{\etens{Y}}}
\def\etZ{{\etens{Z}}}

% The true underlying data generating distribution
\newcommand{\pdata}{p_{\rm{data}}}
\newcommand{\ptarget}{p_{\rm{target}}}
\newcommand{\pprior}{p_{\rm{prior}}}
\newcommand{\pbase}{p_{\rm{base}}}
\newcommand{\pref}{p_{\rm{ref}}}

% The empirical distribution defined by the training set
\newcommand{\ptrain}{\hat{p}_{\rm{data}}}
\newcommand{\Ptrain}{\hat{P}_{\rm{data}}}
% The model distribution
\newcommand{\pmodel}{p_{\rm{model}}}
\newcommand{\Pmodel}{P_{\rm{model}}}
\newcommand{\ptildemodel}{\tilde{p}_{\rm{model}}}
% Stochastic autoencoder distributions
\newcommand{\pencode}{p_{\rm{encoder}}}
\newcommand{\pdecode}{p_{\rm{decoder}}}
\newcommand{\precons}{p_{\rm{reconstruct}}}

\newcommand{\laplace}{\mathrm{Laplace}} % Laplace distribution

\newcommand{\E}{\mathbb{E}}
\newcommand{\Ls}{\mathcal{L}}
\newcommand{\R}{\mathbb{R}}
\newcommand{\emp}{\tilde{p}}
\newcommand{\lr}{\alpha}
\newcommand{\reg}{\lambda}
\newcommand{\rect}{\mathrm{rectifier}}
\newcommand{\softmax}{\mathrm{softmax}}
\newcommand{\sigmoid}{\sigma}
\newcommand{\softplus}{\zeta}
\newcommand{\KL}{D_{\mathrm{KL}}}
\newcommand{\Var}{\mathrm{Var}}
\newcommand{\standarderror}{\mathrm{SE}}
\newcommand{\Cov}{\mathrm{Cov}}
% Wolfram Mathworld says $L^2$ is for function spaces and $\ell^2$ is for vectors
% But then they seem to use $L^2$ for vectors throughout the site, and so does
% wikipedia.
\newcommand{\normlzero}{L^0}
\newcommand{\normlone}{L^1}
\newcommand{\normltwo}{L^2}
\newcommand{\normlp}{L^p}
\newcommand{\normmax}{L^\infty}

\newcommand{\parents}{Pa} % See usage in notation.tex. Chosen to match Daphne's book.

\DeclareMathOperator*{\argmax}{arg\,max}
\DeclareMathOperator*{\argmin}{arg\,min}

\DeclareMathOperator{\sign}{sign}
\DeclareMathOperator{\Tr}{Tr}
\let\ab\allowbreak


\usepackage{hyperref}
\usepackage{url}

\usepackage{graphicx}
\usepackage{amsmath} 
\usepackage{wrapfig}
\usepackage{multirow}
\usepackage[normalem]{ulem}
\usepackage{enumitem}

% \usepackage[utf8]{inputenc} % allow utf-8 input
\usepackage[T1]{fontenc}    % use 8-bit T1 fonts
\usepackage{booktabs}       % professional-quality tables
% \usepackage{amsfonts}       % blackboard math symbols
% \usepackage{nicefrac}       % compact symbols for 1/2, etc.
% \usepackage{microtype}      % microtypography
% \usepackage{xcolor}         % colors


\title{Multimodal Quantitative Language for Generative Recommendation}
% \title{Multimodal Quantitative Language for Generative Sequential Recommendation}

% Authors must not appear in the submitted version. They should be hidden
% as long as the \iclrfinalcopy macro remains commented out below.
% Non-anonymous submissions will be rejected without review.

% \author{Antiquus S.~Hippocampus, Natalia Cerebro \& Amelie P. Amygdale \thanks{ Use footnote for providing further information
% about author (webpage, alternative address)---\emph{not} for acknowledging
% funding agencies.  Funding acknowledgements go at the end of the paper.} \\
% Department of Computer Science\\
% Cranberry-Lemon University\\
% Pittsburgh, PA 15213, USA \\
% \texttt{\{hippo,brain,jen\}@cs.cranberry-lemon.edu} \\
% \And
% Ji Q. Ren \& Yevgeny LeNet \\
% Department of Computational Neuroscience \\
% University of the Witwatersrand \\
% Joburg, South Africa \\
% \texttt{\{robot,net\}@wits.ac.za} \\
% \AND
% Coauthor \\
% Affiliation \\
% Address \\
% \texttt{email}
% }

\author{
 \textbf{Jianyang Zhai\textsuperscript{1,2}},
 \textbf{Zi-Feng Mai\textsuperscript{1,3}},
 \textbf{Chang-Dong Wang\textsuperscript{1,3}\thanks{Corresponding authors.}}, 
 \textbf{Feidiao Yang\textsuperscript{2}\textsuperscript{\textasteriskcentered}}, \\
 \textbf{Xiawu Zheng\textsuperscript{2,4}}, 
 \textbf{Hui Li\textsuperscript{4}},
 \textbf{Yonghong Tian\textsuperscript{2,5}}
\\
 \textsuperscript{1}Sun Yat-sen University,
 \textsuperscript{2}Pengcheng Laboratory, \\
 \textsuperscript{3}Guangdong Key Laboratory of Big Data Analysis and Processing, \\
 \textsuperscript{4}Xiamen University, 
 \textsuperscript{5}Peking University
\\
\texttt{\{zhaijy01, yangfd\}@pcl.ac.cn, changdongwang@hotmail.com}
}

% \author{
%  \textbf{Jianyang Zhai\textsuperscript{1,2}},
%  \textbf{Zi-Feng Mai\textsuperscript{1,3}},
%  \textbf{Chang-Dong Wang\textsuperscript{1,3}\thanks{Corresponding authors.}}, 
%  \textbf{Feidiao Yang\textsuperscript{2}\textsuperscript{\textasteriskcentered}}, \\
%  \textbf{Xiawu Zheng\textsuperscript{4}}, 
%  \textbf{Hui Li\textsuperscript{4}},
%  \textbf{Yonghong Tian\textsuperscript{2,5}}
% \\
%  \textsuperscript{1}Sun Yat-sen University,
%  \textsuperscript{2}Pengcheng Laboratory, \\
%  \textsuperscript{3}Guangxi Key Laboratory of Digital Infrastructure, Guangxi Zhuang Autonomous Region Information Center
%  \textsuperscript{3}Guangdong Key Laboratory of Big Data Analysis and Processing, \\
%  \textsuperscript{4}School of Informatics, Xiamen University, 
%  \textsuperscript{5}Peking University
% \\
% \texttt{\{zhaijy01, yangfd\}@pcl.ac.cn, changdongwang@hotmail.com}
% \normalsize
% \texttt{\{zhaijy01, yangfd\}@pcl.ac.cn, \{maizf3, zhengdy23\}@mail2.sysu.edu.cn} \\
% \normalsize
% \texttt{changdongwang@hotmail.com, \{zhengxiawu, hui\}@xmu.edu.cn, yhtian@pku.edu.cn} \\
% }

% The \author macro works with any number of authors. There are two commands
% used to separate the names and addresses of multiple authors: \And and \AND.
%
% Using \And between authors leaves it to \LaTeX{} to determine where to break
% the lines. Using \AND forces a linebreak at that point. So, if \LaTeX{}
% puts 3 of 4 authors names on the first line, and the last on the second
% line, try using \AND instead of \And before the third author name.

\newcommand{\fix}{\marginpar{FIX}}
\newcommand{\new}{\marginpar{NEW}}

%\iclrfinalcopy % Uncomment for camera-ready version, but NOT for submission.
\iclrfinalcopy
\begin{document}


\maketitle

\begin{abstract}
Generative recommendation has emerged as a promising paradigm aiming at directly generating the identifiers of the target candidates.
Most existing methods attempt to leverage prior knowledge embedded in Pre-trained Language Models (PLMs) to improve the recommendation performance.
However, they often fail to accommodate the differences between the general linguistic knowledge of PLMs and the specific needs of recommendation systems. Moreover, they rarely consider the complementary knowledge between the multimodal information of items, which represents the multi-faceted preferences of users. 
To facilitate efficient recommendation knowledge transfer, we propose a novel approach called Multimodal Quantitative Language for Generative Recommendation (MQL4GRec).
Our key idea is to transform items from different domains and modalities into a unified language, which can serve as a bridge for transferring recommendation knowledge.
Specifically, we first introduce quantitative translators to convert the text and image content of items from various domains into a new and concise language, known as quantitative language, with all items sharing the same vocabulary.
Then, we design a series of quantitative language generation tasks to enrich quantitative language with semantic information and prior knowledge. 
Finally, we achieve the transfer of recommendation knowledge from different domains and modalities to the recommendation task through pre-training and fine-tuning.
We evaluate the effectiveness of MQL4GRec through extensive experiments and comparisons with existing methods, achieving improvements over the baseline by 11.18\%, 14.82\%, and 7.95\% on the NDCG metric across three different datasets, respectively. 
\footnote{Our implementation is available at: \href{https://github.com/zhaijianyang/MQL4GRec}{\textcolor{blue}{https://github.com/zhaijianyang/MQL4GRec}.}}
% Our implementation is available at: \href{https://github.com/zhaijianyang/MQL4GRec}{\textcolor{blue}{https://github.com/zhaijianyang/MQL4GRec}.}
% Our implementation is available at: \href{https://anonymous.4open.science/r/QL4GRec-ED65/}{\textcolor{blue}{https://anonymous.4open.science/r/MQL4GRec-ED65/}.}
\end{abstract}

\section{Introduction}

\begin{wrapfigure}[12]{r}{0.55\textwidth}
\vspace{-2em}
  \centering
  \includegraphics[width=0.55\textwidth]{figures/intro.pdf}
  \caption{Illustration of our MQL4GRec. We translate items from different domains and modalities into a new unified language, which can then serve as a bridge for transferring recommendation knowledge. }
  \label{fig:introduction}
\end{wrapfigure}

Recommendation systems (RS) aim to recommend items to users that they may be interested in, and are widely used on many online platforms, such as e-commerce and social networking \citep{chaves2022efficient, covington2016deep}. 
For a long time, recommendation models that represent users and items using their unique IDs (known as IDRec) have been dominant in the field of RS \citep{kang2018self, Sun2019BERT4RecSR, zhang2024ninerec}. 
However, IDRec may encounter cold start and knowledge transferability issues due to its inherent properties. 
To address these limitations, some literature \citep{unisrec, sun2023universal} employs modal encoders \citep{devlin2018bert, he2016deep} to learn universal representations of items or sequences. While promising, these modal encoders are typically not specifically designed for recommendation tasks, resulting in suboptimal performance.

Recently, generative recommendation has emerged as a promising paradigm, which employs an end-to-end generative model to directly predict identifiers of target candidates \citep{p5, tiger}. 
Due to the success of PLMs in natural language generation (NLG) \citep{2020t5, gpt, llama}, most existing methods attempt to leverage the prior knowledge of PLMs to improve the recommendation performance \citep{tallrec, instructrec, zheng2023adapting}. 
They formalize the recommendation task as a sequence-to-sequence generation process, where the input sequence contains data of items interacted with users, and the output sequence represent identifiers of target items. 
Then they enable PLMs to perform recommendation tasks by adding instructions or prompts.
% They attempt to reduce the gap between PLMs and recommender by designing prompts or instructions, enabling PLMs to perform recommendation tasks.
Despite achieving decent performance, they suffer from the following limitations: 
1) There are significant task differences between PLMs and RS, which may lead to inconsistencies between the general linguistic knowledge of PLMs and the specific requirements of RS;
2) They often overlook the complementary knowledge between the multimodal information of items, which is crucial for capturing the multi-faceted preferences of users.

To address these limitations, it is crucial to bridge the gaps between different domains and modalities, leveraging their recommendation knowledge to enhance the performance of the target domains.
Inspired by significant advancements in NLG, such as pretraining-finetuning \citep{devlin2018bert, t5} and prompt-tuning \citep{gpt, llama}, we propose the idea of transforming items from various domains and modalities into a new and unified language.
A key factor contributing to these significant advances is the use of a shared vocabulary, where tokens are endowed with rich semantic information and prior knowledge across various tasks, which can then be effectively transferred to downstream tasks.
Thus, we aspire for this new language to encompass a vocabulary in which tokens can represent items from various domains and modalities, as depicted in Figure \ref{fig:introduction}. 
Specifically, this language not only serves as a bridge for knowledge transfer but also as identifiers of items, and should be more concise than the original modalities (text and image) to avoid issues in generation \citep{p5id}.

To this end, we propose a novel approach known as Quantitative Language for Multimodal Generative Recommendation (MQL4GRec). 
Specifically, we first introduce quantitative translators to convert the content of items (text and images) into the quantitative language.
We train a separate quantitative translator for each modality of the item, each consisting of a modal encoder and a vector quantizer. 
Together, the codebooks of the two quantitative translators constitute the vocabulary. 
Then, we design a series of quantitative language generation tasks aiming at endowing quantitative language with rich semantic information and prior knowledge, and these tasks can be viewed as microcosms of NLG tasks.
Specifically, we additionally incorporate some special tokens as task prompts.
Finally, we transfer the source domain and multimodal recommendation knowledge to the recommendation tasks through pre-training and fine-tuning.
To evaluate the effectiveness of our proposed MQL4GRec, we conduct extensive experiments and comparisons with existing methods. 
Relative to the baseline, we observe improvements of 11.18\%, 14.82\%, and 7.95\% on the NDCG metric across three datasets, respectively. 
In summary, our proposed MQL4GRec achieves the transfer of recommendation knowledge by breaking down barriers between items across different domains and modalities, demonstrating strong scalability and potential.
Our main contributions can be summarized as follows:
\setlist[itemize]{leftmargin=1.5em}
\begin{itemize}
    \item We propose MQL4GRec, a novel approach that translates items from various domains and modalities into a unified quantitative language, thereby breaking down the barriers between them and facilitating the transfer of recommendation knowledge.
    
    \item We design a series of quantitative language generation tasks that endow quantitative language with rich semantic information and prior knowledge, and enhance the performance of recommendation tasks through pre-training and fine-tuning.
    
    \item We conduct extensive experiments and analyses on three public datasets, and the results validate the effectiveness of our proposed method.
    
\end{itemize}

\section{Related Works}
\label{related_work}
% \subsection{Generative Recommendation}
\paragraph{Generative Recommendation.}

Generative models are one of the hottest research topics in machine learning, resulting in some representative works such as Variational AutoEncoders (VAEs) \citep{vae}, Generative Adversarial Networks (GANs) \citep{gan} and Diffusion models \citep{ddpm}. Generally, generative models aim to learn the distribution of the training data $\mathbb{P}(\mathbf{x})$and generate new samples $\mathbf{z}\sim \mathbb{P}(\mathbf{x})$. These generative models have also been applied to recommendation, resulting in many remarkable works of VAE-based \citep{pevae,recvae}, GAN-based \citep{apr,ipgan,dasp} and diffusion-based \citep{diffkg,diffrec} recommendation.

Recently, Transformer-based PLMs such as LLaMA \citep{llama} and GPT \citep{gpt} have also shown promising capabilities in language generation. With the help of such powerful generative PLMs, some PLM-based recommendation methods have also been proposed. Some early works, such as P5 \citep{p5} and M6-Rec \citep{m6rec}, attempt to transform recommendation into a language generation task by designing prompts to bridge the gap between the downstream task and the pretraining task of PLMs. Some works focus on leveraging the prior knowledge in PLMs for recommendation by various tuning techniques such as parameter-efficient fine-tuning (PEFT) \citep{tallrec} and instruction tuning \citep{instructrec}.

One of the most important tasks in PLM-based recommendation is how to assign an unique sequence of tokens to each item as its ID. Early works \citep{p5,m6rec} directly use the original name of the item or randomly assign an integer for each item, which have weak transferability and are sometimes unintelligible to PLMs. SEATER \citep{seater} constructs tree-structured item IDs from a pretrained SASRec \citep{sasrec} model. P5-ID \citep{p5id} investigates the effect of different item IDs on recommendation. ColaRec \citep{colarec} captures the collaborative signals between items to construct generative item IDs. Notably, TIGER \citep{tiger} is the first attempt to use RQ-VAE to construct item IDs by quantizing the item embeddings.

% \subsection{Multi-modal Recommendation}
\paragraph{Multi-modal Recommendation.}
Multi-modal side information of items, such as descriptive text and images, has been shown to be effective in improving recommendations by providing richer contexts for interactions. Early works such as VBPR \citep{vbpr} extract visual features by matrix factorization to achieve more personalized ranking. Some works \citep{mmgcn,mkgat,grcn} leverage various types of graph neural network (GNN) to fuse the multi-modal features. For example, LATTICE \citep{lattice} designs a modality-aware learning layer to learn item-item structures for each modality and aggregates them to obtain latent item graphs. DualGNN \citep{dualgnn} proposes a multi-modal representation learning module to model the user attentions across modalities and inductively learn the user preference. MVGAE \citep{mvgae} uses a modality-specific variational graph autoencoder to fuse the modality-specific node embeddings.

Recently, with the profound development of foundation models in different modalities \citep{clip,gpt,t5}, some recent works attempt to leverage pretrained foundation models as feature encoders to encode the multi-modal side information. Following P5 \citep{p5}, VIP5 \citep{vip5} extends it into a multi-modal version which encodes the item images by a pretrained CLIP image encoder. MMGRec \citep{mmgrec} utilizes a Graph RQ-VAE to construct item IDs from both multi-modal and collaborative information. Moreover, IISAN \citep{fu2024iisan} propose a simple plug-and-play architecture using a Decoupled PEFT structure and exploiting both intra- and inter-modal adaptation.

% \paragraph{Multi-modal codebook.}


\section{Method}
\label{method}

In this section, we elaborate on the proposed MQL4GRec, a novel approach of transferring recommendation knowledge across different domains and modalities.
We first translate item content into a unified quantitative language, which bridge the gaps between different domains and modalities.
Then, we design a series of quantitative language generation tasks, and achieve the transfer of recommendation knowledge through pre-training and fine-tuning.
The overall framework of the method is illustrated in Figure \ref{fig:framework}.

\subsection{Quantitative Language}

The original modal content of items is complex, which can affect the efficiency and performance of recommendations \citep{p5id}. Therefore, we translate item content from various domains and modalities into a concise and unified quantitative language.
In this subsection, we introduce a quantitative translator to accomplish the aforementioned conversion.


\begin{figure}[t]
  \centering
  \includegraphics[width=\columnwidth]{figures/framework.pdf}
  \caption{The overall framework of MQL4GRec. We regard the quantizer as a translator, converting item content from different domains and modalities into a unified quantitative language, thus bridging the gap between them (left). Subsequently, we design a series of quantitative language generation tasks to facilitate the transfer of recommendation knowledge through pre-training and fine-tuning (right).}
  \label{fig:framework}
\end{figure}

% \subsubsection{Quantitative Translator}
\paragraph{Quantitative Translator.}

Vector Quantization (VQ) is an information compression technique widely utilized across various domains \citep{vqvae, rqvae}, which maps high-dimensional data onto a finite set of discrete vectors, known as the codebook.
In this paper, we treat the quantizer as a translator that converts complex item content into a concise quantitative language. Here, the codebook serves as the vocabulary of the quantitative language.

To obtain a unified quantitative language, we first employ a frozen modal encoder (LLaMA or ViT \citep{vit}) to encode item content (text or image), and to obtain the item representation.
Further, we take the item representation as input, and train a Residual-Quantized Variational AutoEncoder (RQ-VAE) \citep{rqvae} for generating item tokens.
RQ-VAE is a multi-level vector quantizer that applies quantization on residuals to generate a tuple of codewords (\textit{i.e.}, item tokens).
As shown in Figure \ref{fig:framework} (left), for an item representation $\boldsymbol{h}$, RQ-VAE first encodes it into a latent representation $\boldsymbol{z}$. At each level $\boldsymbol{l}$, we have a codebook $\mathcal{C}^l=\left\{\boldsymbol{v}_k^l\right\}_{k=1}^K$, where each codebook vector is a learnable cluster center. The residual quantization process can be represented as:
\begin{equation}
c_i=\underset{k}{\arg \min }\left\|\boldsymbol{r}_i-\boldsymbol{v}_k^i\right\|_2^2,
\end{equation}
\begin{equation}
\boldsymbol{r}_{i+1}=\boldsymbol{r}_i-\boldsymbol{v}_{c_i}^i,
\end{equation}
where $c_i$ is the codeword of the $i$-th level, $\boldsymbol{r}_i$ is the residual vector of the $i$-th level, and $\boldsymbol{r}_1 = \boldsymbol{z}$. 
Assuming we have L-level codebooks, the quantization representation of $\boldsymbol{z}$ can be obtained according to $\hat{\boldsymbol{z}}=\sum_{i=1}^L \boldsymbol{v}_{c_i}^i$.
Then $\hat{\boldsymbol{z}}$ will be used as decoder input to reconstruct the item representation $\boldsymbol{h}$. 
The loss function can be represented as:
\begin{equation}
\mathcal{L}_{\mathrm{recon}}=\|\boldsymbol{h}-\hat{\boldsymbol{h}}\|_2^2,
\end{equation}
\begin{equation}
\mathcal{L}_{\mathrm{rqvae}}=\sum_{i=1}^H\left\|\operatorname{sg}\left[\boldsymbol{r}_i\right]-\boldsymbol{v}_{c_i}^i\right\|_2^2+\beta\left\|\boldsymbol{r}_i-\operatorname{sg}\left[\boldsymbol{v}_{c_i}^i\right]\right\|_2^2,
\end{equation}
\begin{equation}
\mathcal{L}(h)=\mathcal{L}_{\mathrm{recon}}+\mathcal{L}_{\mathrm{rqvae}},
\end{equation}
where $\hat{\boldsymbol{h}}$ is the output of the decoder, $\operatorname{sg[*]}$ represents the stop-gradient operator, and $\beta$ is a loss coefficient. The overall loss is divided into two parts, $\mathcal{L}_{\mathrm{recon}}$ is the reconstruction loss, and $\mathcal{L}_{\mathrm{rqvae}}$ is the RQ loss used to minimize the distance between codebook vectors and residual vectors.

Items typically encompass content from multiple modalities, representing various aspects of user preferences. 
In our setup, each item comprises two modalities: text and image. 
We train a quantitative translator for each modality, then add prefixes to the codewords from each of the two codebooks to form a dictionary.
Specifically, for the text quantitative translator, we prepend lowercase letter prefixes to the codewords to obtain $V_t = \{a\_1, b\_2, \ldots, d\_K\}$; for the image quantitative translator, we prepend uppercase letter prefixes to the codewords to obtain $V_v = \{A\_1, B\_2, \ldots, D\_K\}$. Here, \( a/A \) represents the $1$-th level codebook, \( d/D \) represents the $4$-th level codebook, etc. Subsequently, the dictionary can be represented as $V = \{V_t, V_v\}$. 
With each quantitative translator having $LK$ codewords, the size of our dictionary is $2LK$, enabling us to represent a total of \( K^L \) items.

Once the quantitative translators are trained, we can directly use them to translate new items into quantitative language. 
For example, for the item text \textit{"Sengoku Basara: The Last Party"}, after encoding it through the text encoder and RQ-VAE, we obtain a set of codewords (2, 3, 1, 6). Then, by appending lowercase letters before each number, we can get the text quantitative language of the item as \textit{<a\_2><b\_3><c\_1><d\_6>}. Similarly, for the item's image, we can obtain its image quantitative language as \textit{<A\_1><B\_4><C\_2><D\_6>}.


\paragraph{Handling Collisions.}

Translating item content into quantitative language may lead to item collisions, where multiple items possess the same tokens. 
To address this issue, some methods \citep{tiger, p5id} append an additional identifier after the item indices, which may introduce semantically unrelated distributions.
LC-Rec \citep{zheng2023adapting} introduces a uniform distribution constraint to prevent multiple items from clustering in the same leaf node. 
However, this method does not completely resolve collisions, such as when items have the same modality information or when the number of collisions exceeds the size of the last level codebook, which can lead to inflated performance metrics.
(More discussion in Appendix \ref{app:handling}.)

To address the above issue, we reallocate tokens for colliding items based on the distance from the residual vector to the code vectors.
Specifically, for $N$ colliding items, we first calculate the distances $\boldsymbol{D}\in\mathbb{R}^{N \times L \times K}$ between the residual vectors and the code vectors for each level based on $\boldsymbol{d}_k^i=\left\|\boldsymbol{r}_i-\boldsymbol{v}_k^i\right\|_2^2$, and sort the distances to obtain the indices $\boldsymbol{I} = \operatorname{argsort}(\boldsymbol{D}, axis=2) \in\mathbb{R}^{N \times L \times K}$.
Then, we sort the colliding items based on their minimum distance to the code vectors of the last level, i.e., $(item_1, item_2, \ldots, item_N) = \operatorname{sort}_{\operatorname{min}(\boldsymbol{d}^L)}(\textit{colliding items})$.
Finally, we reallocate tokens for the sorted colliding items based on $\boldsymbol{I}$, following these principles: 1) Start from the last level to assign the nearest token to each item. If collisions occur, assign the next nearest token. 2) If there are insufficient tokens in the last level, for the remaining colliding items, reallocate tokens from the second last level based on distance, and then reallocate tokens from the last level. We repeat this process until all colliding items are handled. 



\subsection{Quantitative Language Generation Tasks}

In this subsection, we design several quantitative language generation tasks with the aim of imbuing quantitative language with more semantic information, thereby transferring prior knowledge to the target task, as illustrated in Figure \ref{fig:framework} (right).
Specifically, we additionally include some special tokens in the dictionary, which can serve as prompts to differentiate the types of tasks.

\paragraph{Next Item Generation.}
\label{NIG}

Since our primary goal is to predict the next item, the next item generation task is our main optimization objective.
Specifically, each item contains both text and image modalities, so we have two subtasks: 1) Next Text Item Generation; 2) Next Image Item Generation.
In this context, the input sequence is the item tokens sequence from the user interaction history, and the output sequence is the target item tokens corresponding to the respective modality.
Different modal sequences reflect different aspects of user preferences.

\paragraph{Asymmetric Item Generation.}
\label{AIG}

In the next item generation task, the input and output are tokens of the same modality, and we refer to this task as symmetric.
To facilitate the interaction of recommendation knowledge between two modalities, we introduce asymmetric item generation tasks.
Here, there are two subtasks: 1) Asymmetric Text Item Generation, where the input is the image tokens of the interaction history items, and the output is the text tokens of the target item; 2) Asymmetric Image Item Generation, where the input is the text tokens of the interaction history items, and the output is the image tokens of the target item. For example, for the input sequence \textit{"<*\_6><*\_7><*\_8><a\_2><b\_3><c\_1><d\_6><a\_4><b\_3><c\_8><d\_6>"}, in human-understandable language, it can be described as follows: \textit{"Based on the user's text interaction sequence, please predict the next item's image quantitative language: <a\_2><b\_3><c\_1><d\_6>, <a\_4><b\_3><c\_8><d\_6>"}.

\paragraph{Quantitative Language Alignment}
\label{QLA}

Asymmetric item generation tasks enable the interaction of knowledge between two modalities, but they fall under the category of implicit alignment of the two modalities.
We further introduce explicit Quantitative Language Alignment tasks to directly achieve alignment between the text and image quantitative languages of items.
Here, we also have two subtasks: 1) Text-to-Image Alignment; 2) Image-to-Text Alignment. 
For example, for the input sequence \textit{"<*\_12><*\_13><*\_14><a\_2><b\_3><c\_1><d\_6>"}, in human-understandable language, it can be described as follows: \textit{"Please provide the image quantitative language for the following item: <a\_2><b\_3><c\_1><d\_6>"}.

\subsection{Training and Recommendation}

\paragraph{Training.}

Quantitative language can be viewed as a microcosm of natural language. We employ a two-stage paradigm of pre-training and fine-tuning to optimize the model, which is similar to NLG tasks.
For \textbf{pre-training}, we utilize the source domain datasets, where the pre-training task consists of two sub-tasks for next item generation.
The purpose is to transfer recommendation knowledge from the source domains to the target domains.
For \textbf{fine-tuning}, we conduct it on the target domain dataset, with tasks encompassing all quantitative language generation tasks. 
The aim is to leverage recommendation knowledge from different modalities to explore users' multifaceted preferences.
The tasks mentioned above are conditional language generation tasks performed in a sequence-to-sequence manner.
We optimize the negative log-likelihood of the generation target as follows:
\begin{equation}
    \mathcal{L}_\theta=-\sum_{j=1}^{|\mathbf{Y}|} \log P_\theta\left(\mathbf{Y}_j \mid \mathbf{Y}_{<j}, \mathbf{X}\right),
\end{equation}
where $\theta$ is the model parameters, $\mathbf{X}$ is the input sequence of encoder, and $\mathbf{Y_j}$ is the $j$-th token of $\mathbf{Y}$.

\paragraph{Re-ranking for recommendation.}
There are two sub-tasks in the next item generation task, representing different user preferences. 
Although fine-tuning tasks can facilitate the transfer of recommendation knowledge between them, there might be some information loss.
Therefore, we re-rank items by utilizing the recommendation lists generated from the two sub-tasks. The basic idea is that items appearing in both lists should be ranked higher.
Specifically, we first obtain recommendation lists $R_t$ and $R_v$ for each sub-task through beam search, which include scores for each item.
Then, the new score for each item can be formalized as:
\begin{equation}
\label{equ:score}
s(x) = \begin{cases} 
(s_t(x) + s_v(x)) / 2 + 1 & x \in R_t, x \in R_v \\
s_t(x) & x \in R_t \\
s_v(x) & x \in R_v
\end{cases},
\end{equation}
where $s_i(x)$ is the score of item $x$ in the list $R_i$, and $i \in \{t, v\}$.

\section{Experiments}

\subsection{Experimental Settings}

\paragraph{Datasets.}
We evaluate the proposed approach on three public real-world benchmarks from the Amazon Product Reviews dataset \citep{ni2019justifying}, containing user reviews and item metadata from May 1996 to October 2018.
In particular, we use six categories for pre-training, including \textit{“Pet Supplies”}, \textit{"Cell Phones and Accessories"}, \textit{“Automotive”}, \textit{“Tools and Home Improvement”}, \textit{“Toys and Games”}, \textit{“Sports and Outdoors”}, and three categories for sequential recommendation tasks, including \textit{“Musical Instruments”}, \textit{“Arts Crafts and Sewing”}, \textit{“Video Games”}.
We discuss the dataset statistics and pre-processing in Appendix \ref{dataset}.

\paragraph{Evaluation Metrics.}
We use top-k Recall (Recall@K) and Normalized Discounted Cumulative Gain (NDCG@K) with K = 1, 5, 10 to evaluate the recommendation performance.
Following previous works \citep{p5, p5id}, we employ the \textit{leave-one-out} strategy for evaluation.
We perform full ranking evaluation over the entire item set instead of sample-based evaluation. For the generative methods based on beam search, the beam size is uniformly set to 20.


\subsection{Overall Performance}
In this section, we compare our proposed approach for generative recommendation with the following sequential recommendation methods (which are described briefly in Appendix \ref{baseline}): GRU4Rec \citep{hidasi2015session}, BERT4Rec \citep{Sun2019BERT4RecSR}, SASRec \citep{sasrec}, FDSA \citep{fdsa}, S$^3$-Rec \citep{s3rec}, VQ-Rec \citep{hou2023learning}, MISSRec\citep{wang2023missrec}, P5-CID \citep{p5id}, VIP5 \citep{geng2023vip5}, and TIGER \citep{tiger}. 
Results are shown in Table \ref{tab:results}. Based on these results, we can find:

% For non-generative recommendation methods, FDSA often achieves better performance in most cases, demonstrating that introducing item content as auxiliary information can enhance recommendation performance. 
% For all the baseline methods, TIGER performs well on the Instruments and Arts datasets but does not exhibit superiority on the Games dataset. This may be due to TIGER's lack of auxiliary content information. 
% In contrast, our proposed method introduces recommendation knowledge from different domains and modalities.

For non-generative recommendation methods, MISSRec often achieves better performance in most cases, demonstrating that introducing multimodal information of items can enhance recommendation performance. 
For generative baseline methods, VIP5 with image information does not achieve good results, which may be due to the modal differences between PLMs and image information.
Furthermore, TIGER performs well on the Instruments and Arts datasets but does not exhibit superiority on the Games dataset. This may be due to TIGER's lack of auxiliary content information.
In contrast, our proposed method introduces recommendation knowledge from different domains and modalities.

Compared to baseline methods, our proposed MQL4GRec achieves the best performance in most cases, especially with significant improvements on the NDCG metric.
This superior performance can be attributed to two factors: 1) We translate item content from different domains and modalities into a unified quantitative language, breaking down barriers between them; 2) The series of QLG tasks we designed enable the transfer of recommendation knowledge to target tasks through pre-training and fine-tuning methods.


% \begin{table}
% \centering
% \caption{Performance comparison of different methods on the three datasets. The best and second-best performances are indicated in bold and underlined font, respectively. $^*$ indicates that, in the t-test, our method significantly outperforms the runner-up with p < 0.05.}
% \label{tab:results}
% \resizebox{\columnwidth}{!}{
% \begin{tabular}{l|l|ccccccc|cccccl} 
% \toprule
% Dataset                      & Metrics & GRU4Rec & BERT4Rec & SASRec & FDSA           & S$^3$-Rec & VQ-Rec          & MISSRec         & P5-CID & VIP5   & TIGER          & MQL4GRec        & Improv.  & $p$-value   \\ 
% \midrule
% \multirow{5}{*}{Instruments} & HR@1    & 0.0566  & 0.0450   & 0.0318 & 0.0530         & 0.0339    & 0.0502          & 0.0723          & 0.0512 & 0.0737 & \uline{0.0754} & \textbf{0.0833} & +10.48\% & 1.74e-4$^*$   \\
%                              & HR@5    & 0.0975  & 0.0856   & 0.0946 & 0.0987         & 0.0937    & 0.1062          & \uline{0.1089}  & 0.0839 & 0.0892 & 0.1007         & \textbf{0.1115} & +2.39\%  & 2.79e-1   \\
%                              & HR@10   & 0.1207  & 0.1081   & 0.1233 & 0.1249         & 0.1123    & 0.1357          & \uline{0.1361}  & 0.1119 & 0.1071 & 0.1221         & \textbf{0.1375} & +1.03\%  & 5.93e-1   \\
%                              & NDCG@5  & 0.0783  & 0.0667   & 0.0654 & 0.0775         & 0.0693    & 0.0796          & 0.0797          & 0.0678 & 0.0815 & \uline{0.0882} & \textbf{0.0977} & +10.77\% & 1.32e-5$^*$   \\
%                              & NDCG@10 & 0.0857  & 0.0739   & 0.0746 & 0.0859         & 0.0743    & 0.0891          & 0.0880          & 0.0704 & 0.0872 & \uline{0.0950} & \textbf{0.1060} & +11.58\% & 5.04e-7$^*$   \\ 
% \midrule
% \multirow{5}{*}{Arts}        & HR@1    & 0.0365  & 0.0289   & 0.0212 & 0.0380         & 0.0172    & 0.0408          & 0.0479          & 0.0421 & 0.0474 & \uline{0.0532} & \textbf{0.0672} & +26.32\% & 8.55e-17$^*$  \\
%                              & HR@5    & 0.0817  & 0.0697   & 0.0951 & 0.0832         & 0.0739    & \textbf{0.1038} & 0.1021          & 0.0713 & 0.0704 & 0.0894         & \uline{0.1037}  & -        & 9.42e-1   \\
%                              & HR@10   & 0.1088  & 0.0922   & 0.1250 & 0.1190         & 0.1030    & \textbf{0.1386} & 0.1321          & 0.0994 & 0.0859 & 0.1167         & \uline{0.1327}  & -        & 1.02e-2   \\
%                              & NDCG@5  & 0.0602  & 0.0502   & 0.0610 & 0.0583         & 0.0511    & \uline{0.0732}  & 0.0699          & 0.0607 & 0.0586 & 0.0718         & \textbf{0.0857} & +17.08\% & 2.08e-12$^*$  \\
%                              & NDCG@10 & 0.0690  & 0.0575   & 0.0706 & 0.0695         & 0.0630    & \uline{0.0844}  & 0.0815          & 0.0662 & 0.0635 & 0.0806         & \textbf{0.0950} & +12.56\% & 3.15e-9$^*$   \\ 
% \midrule
% \multirow{5}{*}{Games}       & HR@1    & 0.0140  & 0.0115   & 0.0069 & 0.0163         & 0.0136    & 0.0075          & \uline{0.0201}  & 0.0169 & 0.0173 & 0.0166         & \textbf{0.0203} & +1.00\%  & 7.40e-1   \\
%                              & HR@5    & 0.0544  & 0.0426   & 0.0587 & 0.0614         & 0.0527    & 0.0408          & \textbf{0.0674} & 0.0532 & 0.0480 & 0.0523         & \uline{0.0637}  & -        & 1.74e-3   \\
%                              & HR@10   & 0.0895  & 0.0725   & 0.0985 & 0.0988         & 0.0903    & 0.0679          & \textbf{0.1048} & 0.0824 & 0.0758 & 0.0857         & \uline{0.1033}  & -        & 3.09e-1   \\
%                              & NDCG@5  & 0.0341  & 0.0270   & 0.0333 & \uline{0.0389} & 0.0351    & 0.0242          & 0.0385          & 0.0331 & 0.0328 & 0.0345         & \textbf{0.0421} & +8.23\%  & 1.48e-4$^*$   \\
%                              & NDCG@10 & 0.0453  & 0.0366   & 0.0461 & \uline{0.0509} & 0.0468    & 0.0329          & 0.0499          & 0.0454 & 0.0418 & 0.0453         & \textbf{0.0548} & +7.66\%  & 8.80e-6$^*$   \\
% \bottomrule
% \end{tabular}
% }
% \end{table}


\begin{table}
\centering
\caption{Performance comparison of different methods on the three datasets. The best and second-best performances are indicated in bold and underlined font, respectively.}
\label{tab:results}
\resizebox{\columnwidth}{!}{
\begin{tabular}{l|l|ccccccc|ccccc} 
\toprule
Dataset                      & Metrics & GRU4Rec & BERT4Rec & SASRec & FDSA           & S$^3$-Rec & VQ-Rec          & MISSRec         & P5-CID & VIP5   & TIGER          & MQL4GRec        & Improv.   \\ 
\midrule
\multirow{5}{*}{Instruments} & HR@1    & 0.0566  & 0.0450   & 0.0318 & 0.0530         & 0.0339    & 0.0502          & 0.0723          & 0.0512 & 0.0737 & \uline{0.0754} & \textbf{0.0833} & +10.48\%  \\
                             & HR@5    & 0.0975  & 0.0856   & 0.0946 & 0.0987         & 0.0937    & 0.1062          & \uline{0.1089}  & 0.0839 & 0.0892 & 0.1007         & \textbf{0.1115} & +2.39\%   \\
                             & HR@10   & 0.1207  & 0.1081   & 0.1233 & 0.1249         & 0.1123    & 0.1357          & \uline{0.1361}  & 0.1119 & 0.1071 & 0.1221         & \textbf{0.1375} & +1.03\%   \\
                             & NDCG@5  & 0.0783  & 0.0667   & 0.0654 & 0.0775         & 0.0693    & 0.0796          & 0.0797          & 0.0678 & 0.0815 & \uline{0.0882} & \textbf{0.0977} & +10.77\%  \\
                             & NDCG@10 & 0.0857  & 0.0739   & 0.0746 & 0.0859         & 0.0743    & 0.0891          & 0.0880          & 0.0704 & 0.0872 & \uline{0.0950} & \textbf{0.1060} & +11.58\%  \\ 
\midrule
\multirow{5}{*}{Arts}        & HR@1    & 0.0365  & 0.0289   & 0.0212 & 0.0380         & 0.0172    & 0.0408          & 0.0479          & 0.0421 & 0.0474 & \uline{0.0532} & \textbf{0.0672} & +26.32\%  \\
                             & HR@5    & 0.0817  & 0.0697   & 0.0951 & 0.0832         & 0.0739    & \textbf{0.1038} & 0.1021          & 0.0713 & 0.0704 & 0.0894         & \uline{0.1037}  & -         \\
                             & HR@10   & 0.1088  & 0.0922   & 0.1250 & 0.1190         & 0.1030    & \textbf{0.1386} & 0.1321          & 0.0994 & 0.0859 & 0.1167         & \uline{0.1327}  & -         \\
                             & NDCG@5  & 0.0602  & 0.0502   & 0.0610 & 0.0583         & 0.0511    & \uline{0.0732}  & 0.0699          & 0.0607 & 0.0586 & 0.0718         & \textbf{0.0857} & +17.08\%  \\
                             & NDCG@10 & 0.0690  & 0.0575   & 0.0706 & 0.0695         & 0.0630    & \uline{0.0844}  & 0.0815          & 0.0662 & 0.0635 & 0.0806         & \textbf{0.0950} & +12.56\%  \\ 
\midrule
\multirow{5}{*}{Games}       & HR@1    & 0.0140  & 0.0115   & 0.0069 & 0.0163         & 0.0136    & 0.0075          & \uline{0.0201}  & 0.0169 & 0.0173 & 0.0166         & \textbf{0.0203} & +1.00\%   \\
                             & HR@5    & 0.0544  & 0.0426   & 0.0587 & 0.0614         & 0.0527    & 0.0408          & \textbf{0.0674} & 0.0532 & 0.0480 & 0.0523         & \uline{0.0637}  & -         \\
                             & HR@10   & 0.0895  & 0.0725   & 0.0985 & 0.0988         & 0.0903    & 0.0679          & \textbf{0.1048} & 0.0824 & 0.0758 & 0.0857         & \uline{0.1033}  & -         \\
                             & NDCG@5  & 0.0341  & 0.0270   & 0.0333 & \uline{0.0389} & 0.0351    & 0.0242          & 0.0385          & 0.0331 & 0.0328 & 0.0345         & \textbf{0.0421} & +8.23\%   \\
                             & NDCG@10 & 0.0453  & 0.0366   & 0.0461 & \uline{0.0509} & 0.0468    & 0.0329          & 0.0499          & 0.0454 & 0.0418 & 0.0453         & \textbf{0.0548} & +7.66\%   \\
\bottomrule
\end{tabular}
}
\end{table}

\subsection{Ablation Study}

\begin{table}
\centering
\caption{Ablation study of handling collisions.}
\label{tab:handling}
\resizebox{0.7\columnwidth}{!}{
\begin{tabular}{lcccccc} 
\toprule
\multirow{2}{*}{Methods} & \multicolumn{2}{c}{Instruments}   & \multicolumn{2}{c}{Arts}          & \multicolumn{2}{c}{Games}          \\ 
\cmidrule(l){2-7}
                         & HR@10           & NDCG@10         & HR@10           & NDCG@10         & HR@10           & NDCG@10          \\ 
\midrule
TIGER                    & 0.1221          & 0.0950          & \textbf{0.1167} & 0.0806          & 0.0857          & 0.0453           \\
TIGER w/o user           & 0.1216          & 0.0958          & 0.1159          & 0.0810          & 0.0863          & 0.0464           \\
Handling Collisions      & \textbf{0.1277} & \textbf{0.0987} & 0.1163          & \textbf{0.0844} & \textbf{0.0885} & \textbf{0.0473}  \\
\bottomrule
\end{tabular}
}
\end{table}

\paragraph{Handling Collisions.}
\label{Handling Collisions}

We propose a method based on the distance between the residual vector and the codeword vector to resolve item collisions. To validate the effectiveness of our method, we compare it with the collision resolution approach in TIGER, which directly adds an item index layer to resolve item collisions, thereby introducing a semantically unrelated distribution.
The experimental results are shown in Table \ref{tab:handling}. "TIGER w/o user" refers to the removal of the user ID token from the input sequence, which is also done to facilitate a fairer comparison. From the experimental results, it can be seen that our method of handling collisions is more rational and effective.
Furthermore, results indicate that including a user ID token in the input degrades model performance, particularly on the Games dataset. We attribute this to TIGER representing tens of thousands of users with only 2000 tokens, leading to numerous user ID collisions.

% NIG\_1 shows the results of using our proposed method for resolving item collisions. It is evident that our method for resolving item collisions performs better.
% Experimental results indicate that including a user ID token in the input degrades model performance, particularly on the Games dataset. We attribute this to TIGER representing tens of thousands of users with only 2000 tokens, leading to numerous user ID collisions.

\paragraph{Quantitative language generation tasks.}
\label{ablation1}

We initially assess the effect of different QLG tasks on performance without the use of pre-training, and the results are shown in Table \ref{tab:QLG}. (For more detailed results, please refer to Appendix \ref{more_QLG}.)
Various tasks include: (1) NIG: the next item generation task introduced in Section \ref{NIG}; (2) AIG: the asymmetric item generation task; (3) QLA: the quantitative language alignment task.
In this list, tasks without subscripts indicate that two subtasks are used simultaneously.
"Text" denotes evaluating performance by utilizing the next text item generation subtask (i.e., $\text{NIG}_1$); "Image" signifies evaluating performance by utilizing the next image item generation subtask (i.e., $\text{NIG}_2$).

The results indicate that several quantitative language generation tasks designed by us can significantly improve performance. 
Specifically, as the number of tasks increases, the performance of both $\text{NIG}_1$ and $\text{NIG}_2$ improves. 
This indicates that these tasks can enrich the quantitative language by incorporating semantic information and knowledge across different modalities.
In summary, converting the multimodal content of items into a unified quantitative language effectively facilitates the transfer of recommendation knowledge.


\begin{table}
\small
\centering
\caption{Ablation study of various quantitative language generation tasks without pre-training.}
\label{tab:QLG}
\resizebox{0.7\columnwidth}{!}{
\begin{tabular}{llcccccc} 
\toprule
\multirow{2}{*}{Modal} & \multirow{2}{*}{Tasks} & \multicolumn{2}{c}{Instruments}   & \multicolumn{2}{c}{Arts}                               & \multicolumn{2}{c}{Games}          \\ 
\cmidrule(lr){3-4}\cmidrule(lr){5-6}\cmidrule(lr){7-8}
                       &                          & HR@10           & NDCG@10         & HR@10           & NDCG@10                              & HR@10           & NDCG@10          \\ 
\midrule
\multirow{4}{*}{Text}  & $\text{NIG}_1$           & 0.1277          & 0.0987          & 0.1163          & 0.0844                               & 0.0885          & 0.0473           \\
                       & NIG                      & 0.1275          & 0.0986          & 0.1205          & 0.0877                               & 0.0928          & 0.0493           \\
                       & + AIG                    & 0.1279          & 0.0987          & 0.1249          & 0.0895                               & 0.1002          & 0.0529           \\
                       & + QLA                    & \textbf{0.1282} & \textbf{0.0993} & \textbf{0.1293} & \textbf{0.0913}                      & \textbf{0.1010} & \textbf{0.0531}  \\ 
\midrule\midrule
\multirow{4}{*}{Image} & $\text{NIG}_2$           & 0.1243          & 0.0968          & 0.1117          & 0.0812                               & 0.0881          & 0.0478           \\
                       & NIG                      & 0.1262          & 0.0986          & 0.1158          & 0.0848                               & 0.0899          & 0.0487           \\
                       & + AIG                    & \textbf{0.1299} & 0.0998          & 0.1218          & 0.0878                               & 0.1002          & 0.0534           \\
                       & + QLA                    & 0.1280          & \textbf{0.1001} & \textbf{0.1259} & \textbf{0.0901}                      & \textbf{0.1017} & \textbf{0.0540}  \\
\bottomrule
\end{tabular}
}
\end{table}


\paragraph{Pre-training.}

We transfer recommendation knowledge from the source domain datasets to the target dataset through pre-training.
Here, we employ $\text{NIG}_1$ to evaluate the recommendation performance, with the results shown in Table \ref{tab:pretraining}. (Additional results can be found in Appendix \ref{more_pre-training}.)
QLG represents the quantitative language generation tasks without pre-training.
Specifically, the pre-training task labeled \textit{"$\text{NIG}_1$ w/ pre-training"} employs only $\text{NIG}_1$ from the source domain datasets. 
On the other hand, the \textit{"QLG w/ pre-training"} task uses both $\text{NIG}_1$ and $\text{NIG}_2$.

The results indicate that, under a single modality, pre-training enhances the performance across three downstream datasets, demonstrating that prior knowledge from the source domain can be effectively transferred to downstream tasks. 
Under dual modalities, pre-training significantly improves performance on the Instruments and Arts datasets; however, it does not yield a notable improvement for the Games dataset, potentially due to overfitting. 
A more intuitive analysis of this phenomenon is provided in Section \ref{pre_dataset}. Finally, we re-rank the items according to Equation (\ref{equ:score}) to generate the final recommendation list.


\begin{table}
\centering
\caption{Ablation study of pre-training and quantitative language generation tasks.}
\label{tab:pretraining}
\resizebox{0.9\columnwidth}{!}{
\begin{tabular}{lcccccc} 
\toprule
\multirow{2}{*}{Methods}                  & \multicolumn{2}{c}{Instruments}   & \multicolumn{2}{c}{Arts}          & \multicolumn{2}{c}{Games}          \\ 
\cmidrule(l){2-7}
                                          & HR@10           & NDCG@10         & HR@10           & NDCG@10         & HR@10           & NDCG@10          \\ 
\midrule
$\text{(0) NIG}_1$                        & 0.1277          & 0.0987          & 0.1163          & 0.0844          & 0.0885          & 0.0473           \\
(1) QLG~                                  & 0.1282          & 0.0993          & 0.1293          & 0.0913          & \uline{0.1010}  & \uline{0.0531}   \\
$\text{(2) NIG}_1 \text{w/ pre-training}$ & 0.1334          & 0.1043          & 0.1305          & \textbf{0.0959} & 0.0950          & 0.0508           \\
(3) QLG~~w/ pre-training                  & \uline{0.1362}  & \uline{0.1051}  & \uline{0.1314}  & 0.0944          & 0.0995          & 0.0521           \\
(4) MQL4GRec ((3) + re-ranking)            & \textbf{0.1375} & \textbf{0.1060} & \textbf{0.1327} & \uline{0.0950}  & \textbf{0.1033} & \textbf{0.0548}  \\
\bottomrule
\end{tabular}
}
\end{table}

\begin{figure}[h]
  \centering
  \includegraphics[width=\columnwidth]{figures/datasets.pdf}
  \caption{The impact of varying amounts of pre-training datasets on recommendation performance.}
  \label{fig:pre_dataset}
\end{figure}

\subsection{Further Analysis}

\paragraph{Pre-training datasets.}
\label{pre_dataset}

In this subsection, we investigate the impact of varying amounts of pre-training datasets on downstream tasks, and the results are shown in Figure \ref{fig:pre_dataset}.
From the results, it can be observed that:
1) Exclusively in the context of text quantitative language, as the number of pre-training datasets increases, the performance of downstream tasks also gradually improves. This suggests that larger numbers in pre-training datasets provide more transferable recommendation knowledge.
2) Following pre-training with quantitative language under two modalities, fine-tuning shows varying trends across different downstream datasets. Specifically, while increasing pre-training datasets enhances performance on the Instruments and Arts datasets, it leads to a gradual decline in performance on the Games dataset. This could indicate either overfitting or significant domain differences between the Games dataset and the source domain datasets.

\paragraph{Pre-training epochs}

In this subsection, we investigate the impact of varying the number of pre-training epochs on downstream tasks, with the results displayed in Figure \ref{fig:epochs}. 
From the figure, it can be observed that:
1) When pre-training is performed solely with text quantitative language, the performance of downstream tasks gradually increases with the number of pre-training epochs and stabilizes around 25 epochs.
2) When pre-training involves both text and image quantitative languages, the Instruments and Arts datasets reach peak performance early, and further training may impair the transfer of recommendation knowledge. In contrast, for the Games dataset, performance deteriorates as the number of pre-training epochs increases. This suggests that for the Games dataset, recommendation knowledge from different modalities might be more crucial than that from the source domain dataset, and there may be conflicts between the two.

\begin{figure}[h]
  \centering
  \includegraphics[width=0.7\columnwidth]{figures/epochs.pdf}
  \caption{The impact of different pre-training epochs on recommendation performance.}
  \label{fig:epochs}
\end{figure}

\begin{table}[h]
\centering
\caption{Zero-shot capabilities under different number of pre-training datasets.}
\label{tab:zero-shot}
\resizebox{0.7\columnwidth}{!}{
\begin{tabular}{ccccccc} 
\toprule
\multirow{2}{*}{Number} & \multicolumn{2}{c}{Instruments} & \multicolumn{2}{c}{Arts} & \multicolumn{2}{c}{Games}  \\ 
\cmidrule(lr){2-3}\cmidrule(lr){4-5}\cmidrule(lr){6-7}
                        & HR@10            & NDCG@10          & HR@10            & NDCG@10          & HR@10            & NDCG@10           \\ 
\midrule
0                       & 0.00099          & 0.00046          & 0.00113          & 0.00052          & 0.00066          & 0.00031           \\
2                       & 0.00240          & 0.00137          & 0.00140          & 0.00063          & 0.00109          & \textbf{0.00058}  \\
4                       & 0.00310          & 0.00170          & 0.00298          & 0.00132          & 0.00054          & 0.00027           \\
6                       & \textbf{0.00345} & \textbf{0.00171} & \textbf{0.00311} & \textbf{0.00138} & \textbf{0.00116} & 0.00047           \\
\bottomrule
\end{tabular}
}
\end{table}

\subsection{Zero-shot capability}

% 表现出了初步的zero-shot的能力,但是还很弱。模型很小,
We investigate whether models pre-trained on the source domain dataset have zero-shot capabilities, as shown in Table \ref{tab:zero-shot}. Here, "Number" represents the number of pre-training datasets, with "0" indicating model parameters randomly initialized. We use $\text{NIG}_1$ to evaluate performance. The results demonstrate that pre-trained models exhibit preliminary zero-shot capabilities on the Instruments and Arts datasets, although they are still weak. However, this capability is not evident on the Games dataset. We attribute this primarily to the scarcity of pre-training data and the limited parameters of the model, resulting in insufficient generalization. In the future, we aim to delve deeper into this phenomenon.

\section{Conclusion}

In this paper, we propose a novel approach named MQL4GRec, which transforms item content from different domains and modalities into a unified quantitative language to facilitate the effective transfer of recommendation knowledge. 
We first train a quantitative translator for each modality, converting items into the quantitative language and breaking down the barriers between them. 
Then, we design a series of quantitative language generation tasks aiming at endowing quantitative language with rich semantic information and prior knowledge.
Finally, we transfer the source domain and multimodal recommendation knowledge to the recommendation tasks through pre-training and fine-tuning.
Our proposed MQL4GRec achieves superior performance compared to the baseline method.
Moreover, MQL4GRec possesses strong scalability and potential as it does not rely on traditional item IDs and bridges the gap between different domains and modalities. We believe this represents a significant step towards universal recommendation models.


% \subsubsection*{Author Contributions}
% If you'd like to, you may include  a section for author contributions as is done
% in many journals. This is optional and at the discretion of the authors.

\subsubsection*{Acknowledgments}
This work was supported by National Key Research and Development Program of China (2021YFF1201200), NSFC (62276277), Guangdong Basic and Applied Basic Research Foundation (2022B1515120059).

% \newpage


\documentclass{article} % For LaTeX2e
\usepackage{iclr2025_conference,times}

% Optional math commands from https://github.com/goodfeli/dlbook_notation.
%%%%% NEW MATH DEFINITIONS %%%%%

\usepackage{amsmath,amsfonts,bm}
\usepackage{derivative}
% Mark sections of captions for referring to divisions of figures
\newcommand{\figleft}{{\em (Left)}}
\newcommand{\figcenter}{{\em (Center)}}
\newcommand{\figright}{{\em (Right)}}
\newcommand{\figtop}{{\em (Top)}}
\newcommand{\figbottom}{{\em (Bottom)}}
\newcommand{\captiona}{{\em (a)}}
\newcommand{\captionb}{{\em (b)}}
\newcommand{\captionc}{{\em (c)}}
\newcommand{\captiond}{{\em (d)}}

% Highlight a newly defined term
\newcommand{\newterm}[1]{{\bf #1}}

% Derivative d 
\newcommand{\deriv}{{\mathrm{d}}}

% Figure reference, lower-case.
\def\figref#1{figure~\ref{#1}}
% Figure reference, capital. For start of sentence
\def\Figref#1{Figure~\ref{#1}}
\def\twofigref#1#2{figures \ref{#1} and \ref{#2}}
\def\quadfigref#1#2#3#4{figures \ref{#1}, \ref{#2}, \ref{#3} and \ref{#4}}
% Section reference, lower-case.
\def\secref#1{section~\ref{#1}}
% Section reference, capital.
\def\Secref#1{Section~\ref{#1}}
% Reference to two sections.
\def\twosecrefs#1#2{sections \ref{#1} and \ref{#2}}
% Reference to three sections.
\def\secrefs#1#2#3{sections \ref{#1}, \ref{#2} and \ref{#3}}
% Reference to an equation, lower-case.
\def\eqref#1{equation~\ref{#1}}
% Reference to an equation, upper case
\def\Eqref#1{Equation~\ref{#1}}
% A raw reference to an equation---avoid using if possible
\def\plaineqref#1{\ref{#1}}
% Reference to a chapter, lower-case.
\def\chapref#1{chapter~\ref{#1}}
% Reference to an equation, upper case.
\def\Chapref#1{Chapter~\ref{#1}}
% Reference to a range of chapters
\def\rangechapref#1#2{chapters\ref{#1}--\ref{#2}}
% Reference to an algorithm, lower-case.
\def\algref#1{algorithm~\ref{#1}}
% Reference to an algorithm, upper case.
\def\Algref#1{Algorithm~\ref{#1}}
\def\twoalgref#1#2{algorithms \ref{#1} and \ref{#2}}
\def\Twoalgref#1#2{Algorithms \ref{#1} and \ref{#2}}
% Reference to a part, lower case
\def\partref#1{part~\ref{#1}}
% Reference to a part, upper case
\def\Partref#1{Part~\ref{#1}}
\def\twopartref#1#2{parts \ref{#1} and \ref{#2}}

\def\ceil#1{\lceil #1 \rceil}
\def\floor#1{\lfloor #1 \rfloor}
\def\1{\bm{1}}
\newcommand{\train}{\mathcal{D}}
\newcommand{\valid}{\mathcal{D_{\mathrm{valid}}}}
\newcommand{\test}{\mathcal{D_{\mathrm{test}}}}

\def\eps{{\epsilon}}


% Random variables
\def\reta{{\textnormal{$\eta$}}}
\def\ra{{\textnormal{a}}}
\def\rb{{\textnormal{b}}}
\def\rc{{\textnormal{c}}}
\def\rd{{\textnormal{d}}}
\def\re{{\textnormal{e}}}
\def\rf{{\textnormal{f}}}
\def\rg{{\textnormal{g}}}
\def\rh{{\textnormal{h}}}
\def\ri{{\textnormal{i}}}
\def\rj{{\textnormal{j}}}
\def\rk{{\textnormal{k}}}
\def\rl{{\textnormal{l}}}
% rm is already a command, just don't name any random variables m
\def\rn{{\textnormal{n}}}
\def\ro{{\textnormal{o}}}
\def\rp{{\textnormal{p}}}
\def\rq{{\textnormal{q}}}
\def\rr{{\textnormal{r}}}
\def\rs{{\textnormal{s}}}
\def\rt{{\textnormal{t}}}
\def\ru{{\textnormal{u}}}
\def\rv{{\textnormal{v}}}
\def\rw{{\textnormal{w}}}
\def\rx{{\textnormal{x}}}
\def\ry{{\textnormal{y}}}
\def\rz{{\textnormal{z}}}

% Random vectors
\def\rvepsilon{{\mathbf{\epsilon}}}
\def\rvphi{{\mathbf{\phi}}}
\def\rvtheta{{\mathbf{\theta}}}
\def\rva{{\mathbf{a}}}
\def\rvb{{\mathbf{b}}}
\def\rvc{{\mathbf{c}}}
\def\rvd{{\mathbf{d}}}
\def\rve{{\mathbf{e}}}
\def\rvf{{\mathbf{f}}}
\def\rvg{{\mathbf{g}}}
\def\rvh{{\mathbf{h}}}
\def\rvu{{\mathbf{i}}}
\def\rvj{{\mathbf{j}}}
\def\rvk{{\mathbf{k}}}
\def\rvl{{\mathbf{l}}}
\def\rvm{{\mathbf{m}}}
\def\rvn{{\mathbf{n}}}
\def\rvo{{\mathbf{o}}}
\def\rvp{{\mathbf{p}}}
\def\rvq{{\mathbf{q}}}
\def\rvr{{\mathbf{r}}}
\def\rvs{{\mathbf{s}}}
\def\rvt{{\mathbf{t}}}
\def\rvu{{\mathbf{u}}}
\def\rvv{{\mathbf{v}}}
\def\rvw{{\mathbf{w}}}
\def\rvx{{\mathbf{x}}}
\def\rvy{{\mathbf{y}}}
\def\rvz{{\mathbf{z}}}

% Elements of random vectors
\def\erva{{\textnormal{a}}}
\def\ervb{{\textnormal{b}}}
\def\ervc{{\textnormal{c}}}
\def\ervd{{\textnormal{d}}}
\def\erve{{\textnormal{e}}}
\def\ervf{{\textnormal{f}}}
\def\ervg{{\textnormal{g}}}
\def\ervh{{\textnormal{h}}}
\def\ervi{{\textnormal{i}}}
\def\ervj{{\textnormal{j}}}
\def\ervk{{\textnormal{k}}}
\def\ervl{{\textnormal{l}}}
\def\ervm{{\textnormal{m}}}
\def\ervn{{\textnormal{n}}}
\def\ervo{{\textnormal{o}}}
\def\ervp{{\textnormal{p}}}
\def\ervq{{\textnormal{q}}}
\def\ervr{{\textnormal{r}}}
\def\ervs{{\textnormal{s}}}
\def\ervt{{\textnormal{t}}}
\def\ervu{{\textnormal{u}}}
\def\ervv{{\textnormal{v}}}
\def\ervw{{\textnormal{w}}}
\def\ervx{{\textnormal{x}}}
\def\ervy{{\textnormal{y}}}
\def\ervz{{\textnormal{z}}}

% Random matrices
\def\rmA{{\mathbf{A}}}
\def\rmB{{\mathbf{B}}}
\def\rmC{{\mathbf{C}}}
\def\rmD{{\mathbf{D}}}
\def\rmE{{\mathbf{E}}}
\def\rmF{{\mathbf{F}}}
\def\rmG{{\mathbf{G}}}
\def\rmH{{\mathbf{H}}}
\def\rmI{{\mathbf{I}}}
\def\rmJ{{\mathbf{J}}}
\def\rmK{{\mathbf{K}}}
\def\rmL{{\mathbf{L}}}
\def\rmM{{\mathbf{M}}}
\def\rmN{{\mathbf{N}}}
\def\rmO{{\mathbf{O}}}
\def\rmP{{\mathbf{P}}}
\def\rmQ{{\mathbf{Q}}}
\def\rmR{{\mathbf{R}}}
\def\rmS{{\mathbf{S}}}
\def\rmT{{\mathbf{T}}}
\def\rmU{{\mathbf{U}}}
\def\rmV{{\mathbf{V}}}
\def\rmW{{\mathbf{W}}}
\def\rmX{{\mathbf{X}}}
\def\rmY{{\mathbf{Y}}}
\def\rmZ{{\mathbf{Z}}}

% Elements of random matrices
\def\ermA{{\textnormal{A}}}
\def\ermB{{\textnormal{B}}}
\def\ermC{{\textnormal{C}}}
\def\ermD{{\textnormal{D}}}
\def\ermE{{\textnormal{E}}}
\def\ermF{{\textnormal{F}}}
\def\ermG{{\textnormal{G}}}
\def\ermH{{\textnormal{H}}}
\def\ermI{{\textnormal{I}}}
\def\ermJ{{\textnormal{J}}}
\def\ermK{{\textnormal{K}}}
\def\ermL{{\textnormal{L}}}
\def\ermM{{\textnormal{M}}}
\def\ermN{{\textnormal{N}}}
\def\ermO{{\textnormal{O}}}
\def\ermP{{\textnormal{P}}}
\def\ermQ{{\textnormal{Q}}}
\def\ermR{{\textnormal{R}}}
\def\ermS{{\textnormal{S}}}
\def\ermT{{\textnormal{T}}}
\def\ermU{{\textnormal{U}}}
\def\ermV{{\textnormal{V}}}
\def\ermW{{\textnormal{W}}}
\def\ermX{{\textnormal{X}}}
\def\ermY{{\textnormal{Y}}}
\def\ermZ{{\textnormal{Z}}}

% Vectors
\def\vzero{{\bm{0}}}
\def\vone{{\bm{1}}}
\def\vmu{{\bm{\mu}}}
\def\vtheta{{\bm{\theta}}}
\def\vphi{{\bm{\phi}}}
\def\va{{\bm{a}}}
\def\vb{{\bm{b}}}
\def\vc{{\bm{c}}}
\def\vd{{\bm{d}}}
\def\ve{{\bm{e}}}
\def\vf{{\bm{f}}}
\def\vg{{\bm{g}}}
\def\vh{{\bm{h}}}
\def\vi{{\bm{i}}}
\def\vj{{\bm{j}}}
\def\vk{{\bm{k}}}
\def\vl{{\bm{l}}}
\def\vm{{\bm{m}}}
\def\vn{{\bm{n}}}
\def\vo{{\bm{o}}}
\def\vp{{\bm{p}}}
\def\vq{{\bm{q}}}
\def\vr{{\bm{r}}}
\def\vs{{\bm{s}}}
\def\vt{{\bm{t}}}
\def\vu{{\bm{u}}}
\def\vv{{\bm{v}}}
\def\vw{{\bm{w}}}
\def\vx{{\bm{x}}}
\def\vy{{\bm{y}}}
\def\vz{{\bm{z}}}

% Elements of vectors
\def\evalpha{{\alpha}}
\def\evbeta{{\beta}}
\def\evepsilon{{\epsilon}}
\def\evlambda{{\lambda}}
\def\evomega{{\omega}}
\def\evmu{{\mu}}
\def\evpsi{{\psi}}
\def\evsigma{{\sigma}}
\def\evtheta{{\theta}}
\def\eva{{a}}
\def\evb{{b}}
\def\evc{{c}}
\def\evd{{d}}
\def\eve{{e}}
\def\evf{{f}}
\def\evg{{g}}
\def\evh{{h}}
\def\evi{{i}}
\def\evj{{j}}
\def\evk{{k}}
\def\evl{{l}}
\def\evm{{m}}
\def\evn{{n}}
\def\evo{{o}}
\def\evp{{p}}
\def\evq{{q}}
\def\evr{{r}}
\def\evs{{s}}
\def\evt{{t}}
\def\evu{{u}}
\def\evv{{v}}
\def\evw{{w}}
\def\evx{{x}}
\def\evy{{y}}
\def\evz{{z}}

% Matrix
\def\mA{{\bm{A}}}
\def\mB{{\bm{B}}}
\def\mC{{\bm{C}}}
\def\mD{{\bm{D}}}
\def\mE{{\bm{E}}}
\def\mF{{\bm{F}}}
\def\mG{{\bm{G}}}
\def\mH{{\bm{H}}}
\def\mI{{\bm{I}}}
\def\mJ{{\bm{J}}}
\def\mK{{\bm{K}}}
\def\mL{{\bm{L}}}
\def\mM{{\bm{M}}}
\def\mN{{\bm{N}}}
\def\mO{{\bm{O}}}
\def\mP{{\bm{P}}}
\def\mQ{{\bm{Q}}}
\def\mR{{\bm{R}}}
\def\mS{{\bm{S}}}
\def\mT{{\bm{T}}}
\def\mU{{\bm{U}}}
\def\mV{{\bm{V}}}
\def\mW{{\bm{W}}}
\def\mX{{\bm{X}}}
\def\mY{{\bm{Y}}}
\def\mZ{{\bm{Z}}}
\def\mBeta{{\bm{\beta}}}
\def\mPhi{{\bm{\Phi}}}
\def\mLambda{{\bm{\Lambda}}}
\def\mSigma{{\bm{\Sigma}}}

% Tensor
\DeclareMathAlphabet{\mathsfit}{\encodingdefault}{\sfdefault}{m}{sl}
\SetMathAlphabet{\mathsfit}{bold}{\encodingdefault}{\sfdefault}{bx}{n}
\newcommand{\tens}[1]{\bm{\mathsfit{#1}}}
\def\tA{{\tens{A}}}
\def\tB{{\tens{B}}}
\def\tC{{\tens{C}}}
\def\tD{{\tens{D}}}
\def\tE{{\tens{E}}}
\def\tF{{\tens{F}}}
\def\tG{{\tens{G}}}
\def\tH{{\tens{H}}}
\def\tI{{\tens{I}}}
\def\tJ{{\tens{J}}}
\def\tK{{\tens{K}}}
\def\tL{{\tens{L}}}
\def\tM{{\tens{M}}}
\def\tN{{\tens{N}}}
\def\tO{{\tens{O}}}
\def\tP{{\tens{P}}}
\def\tQ{{\tens{Q}}}
\def\tR{{\tens{R}}}
\def\tS{{\tens{S}}}
\def\tT{{\tens{T}}}
\def\tU{{\tens{U}}}
\def\tV{{\tens{V}}}
\def\tW{{\tens{W}}}
\def\tX{{\tens{X}}}
\def\tY{{\tens{Y}}}
\def\tZ{{\tens{Z}}}


% Graph
\def\gA{{\mathcal{A}}}
\def\gB{{\mathcal{B}}}
\def\gC{{\mathcal{C}}}
\def\gD{{\mathcal{D}}}
\def\gE{{\mathcal{E}}}
\def\gF{{\mathcal{F}}}
\def\gG{{\mathcal{G}}}
\def\gH{{\mathcal{H}}}
\def\gI{{\mathcal{I}}}
\def\gJ{{\mathcal{J}}}
\def\gK{{\mathcal{K}}}
\def\gL{{\mathcal{L}}}
\def\gM{{\mathcal{M}}}
\def\gN{{\mathcal{N}}}
\def\gO{{\mathcal{O}}}
\def\gP{{\mathcal{P}}}
\def\gQ{{\mathcal{Q}}}
\def\gR{{\mathcal{R}}}
\def\gS{{\mathcal{S}}}
\def\gT{{\mathcal{T}}}
\def\gU{{\mathcal{U}}}
\def\gV{{\mathcal{V}}}
\def\gW{{\mathcal{W}}}
\def\gX{{\mathcal{X}}}
\def\gY{{\mathcal{Y}}}
\def\gZ{{\mathcal{Z}}}

% Sets
\def\sA{{\mathbb{A}}}
\def\sB{{\mathbb{B}}}
\def\sC{{\mathbb{C}}}
\def\sD{{\mathbb{D}}}
% Don't use a set called E, because this would be the same as our symbol
% for expectation.
\def\sF{{\mathbb{F}}}
\def\sG{{\mathbb{G}}}
\def\sH{{\mathbb{H}}}
\def\sI{{\mathbb{I}}}
\def\sJ{{\mathbb{J}}}
\def\sK{{\mathbb{K}}}
\def\sL{{\mathbb{L}}}
\def\sM{{\mathbb{M}}}
\def\sN{{\mathbb{N}}}
\def\sO{{\mathbb{O}}}
\def\sP{{\mathbb{P}}}
\def\sQ{{\mathbb{Q}}}
\def\sR{{\mathbb{R}}}
\def\sS{{\mathbb{S}}}
\def\sT{{\mathbb{T}}}
\def\sU{{\mathbb{U}}}
\def\sV{{\mathbb{V}}}
\def\sW{{\mathbb{W}}}
\def\sX{{\mathbb{X}}}
\def\sY{{\mathbb{Y}}}
\def\sZ{{\mathbb{Z}}}

% Entries of a matrix
\def\emLambda{{\Lambda}}
\def\emA{{A}}
\def\emB{{B}}
\def\emC{{C}}
\def\emD{{D}}
\def\emE{{E}}
\def\emF{{F}}
\def\emG{{G}}
\def\emH{{H}}
\def\emI{{I}}
\def\emJ{{J}}
\def\emK{{K}}
\def\emL{{L}}
\def\emM{{M}}
\def\emN{{N}}
\def\emO{{O}}
\def\emP{{P}}
\def\emQ{{Q}}
\def\emR{{R}}
\def\emS{{S}}
\def\emT{{T}}
\def\emU{{U}}
\def\emV{{V}}
\def\emW{{W}}
\def\emX{{X}}
\def\emY{{Y}}
\def\emZ{{Z}}
\def\emSigma{{\Sigma}}

% entries of a tensor
% Same font as tensor, without \bm wrapper
\newcommand{\etens}[1]{\mathsfit{#1}}
\def\etLambda{{\etens{\Lambda}}}
\def\etA{{\etens{A}}}
\def\etB{{\etens{B}}}
\def\etC{{\etens{C}}}
\def\etD{{\etens{D}}}
\def\etE{{\etens{E}}}
\def\etF{{\etens{F}}}
\def\etG{{\etens{G}}}
\def\etH{{\etens{H}}}
\def\etI{{\etens{I}}}
\def\etJ{{\etens{J}}}
\def\etK{{\etens{K}}}
\def\etL{{\etens{L}}}
\def\etM{{\etens{M}}}
\def\etN{{\etens{N}}}
\def\etO{{\etens{O}}}
\def\etP{{\etens{P}}}
\def\etQ{{\etens{Q}}}
\def\etR{{\etens{R}}}
\def\etS{{\etens{S}}}
\def\etT{{\etens{T}}}
\def\etU{{\etens{U}}}
\def\etV{{\etens{V}}}
\def\etW{{\etens{W}}}
\def\etX{{\etens{X}}}
\def\etY{{\etens{Y}}}
\def\etZ{{\etens{Z}}}

% The true underlying data generating distribution
\newcommand{\pdata}{p_{\rm{data}}}
\newcommand{\ptarget}{p_{\rm{target}}}
\newcommand{\pprior}{p_{\rm{prior}}}
\newcommand{\pbase}{p_{\rm{base}}}
\newcommand{\pref}{p_{\rm{ref}}}

% The empirical distribution defined by the training set
\newcommand{\ptrain}{\hat{p}_{\rm{data}}}
\newcommand{\Ptrain}{\hat{P}_{\rm{data}}}
% The model distribution
\newcommand{\pmodel}{p_{\rm{model}}}
\newcommand{\Pmodel}{P_{\rm{model}}}
\newcommand{\ptildemodel}{\tilde{p}_{\rm{model}}}
% Stochastic autoencoder distributions
\newcommand{\pencode}{p_{\rm{encoder}}}
\newcommand{\pdecode}{p_{\rm{decoder}}}
\newcommand{\precons}{p_{\rm{reconstruct}}}

\newcommand{\laplace}{\mathrm{Laplace}} % Laplace distribution

\newcommand{\E}{\mathbb{E}}
\newcommand{\Ls}{\mathcal{L}}
\newcommand{\R}{\mathbb{R}}
\newcommand{\emp}{\tilde{p}}
\newcommand{\lr}{\alpha}
\newcommand{\reg}{\lambda}
\newcommand{\rect}{\mathrm{rectifier}}
\newcommand{\softmax}{\mathrm{softmax}}
\newcommand{\sigmoid}{\sigma}
\newcommand{\softplus}{\zeta}
\newcommand{\KL}{D_{\mathrm{KL}}}
\newcommand{\Var}{\mathrm{Var}}
\newcommand{\standarderror}{\mathrm{SE}}
\newcommand{\Cov}{\mathrm{Cov}}
% Wolfram Mathworld says $L^2$ is for function spaces and $\ell^2$ is for vectors
% But then they seem to use $L^2$ for vectors throughout the site, and so does
% wikipedia.
\newcommand{\normlzero}{L^0}
\newcommand{\normlone}{L^1}
\newcommand{\normltwo}{L^2}
\newcommand{\normlp}{L^p}
\newcommand{\normmax}{L^\infty}

\newcommand{\parents}{Pa} % See usage in notation.tex. Chosen to match Daphne's book.

\DeclareMathOperator*{\argmax}{arg\,max}
\DeclareMathOperator*{\argmin}{arg\,min}

\DeclareMathOperator{\sign}{sign}
\DeclareMathOperator{\Tr}{Tr}
\let\ab\allowbreak


\usepackage{hyperref}
\usepackage{url}
\usepackage{cleveref}
\usepackage{booktabs}
\usepackage{multirow}
\usepackage{subcaption}
\usepackage{adjustbox} % To adjust table sizes
\usepackage{float}

% \iclrfinalcopy

% For table
\usepackage{multirow}

% For figures
\usepackage{graphicx}
% \usepackage[table]{xcolor}
\title{Improved Training Technique for Latent Consistency Models}

% Authors must not appear in the submitted version. They should be hidden
% as long as the \iclrfinalcopy macro remains commented out below.
% Non-anonymous submissions will be rejected without review.
\iclrfinalcopy

\author{Quan Dao$^{*\dagger}$\\
Rutgers University \\
\texttt{quan.dao@rutgers.edu} \\ \And 
Khanh Doan$^{*}$\\
Movian AI, Vietnam \\
\texttt{dnkhanh.k63.bk@gmail.com} \\ \And
Di Liu\\
Rutgers University \\
\texttt{di.liu@rutgers.edu} \\   \And
Trung Le\\
Monash University \\
\texttt{trunglm@monash.edu} \\   \And
Dimitris Metaxas\\
Rutgers University \\
\texttt{dnm@cs.rutgers.edu} \\
}


% The \author macro works with any number of authors. There are two commands
% used to separate the names and addresses of multiple authors: \And and \AND.
%
% Using \And between authors leaves it to \LaTeX{} to determine where to break
% the lines. Using \AND forces a linebreak at that point. So, if \LaTeX{}
% puts 3 of 4 authors names on the first line, and the last on the second
% line, try using \AND instead of \And before the third author name.

\newcommand{\fix}{\marginpar{FIX}}
\newcommand{\new}{\marginpar{NEW}}
\newcommand{\khanh}[1]{\textcolor{orange}{[Khanh: #1]}}
\newcommand{\quan}[1]{\textcolor{red}{[Quan: #1]}}
\newcommand{\diliu}[1]{\textcolor{purple}{[Di Liu: #1]}}
\newcommand{\trung}[1]{\textcolor{cyan}{[Trung: #1]}}
\newcommand{\metaxas}[1]{\textcolor{blue}{[Metaxas: #1]}}
\newcommand{\minisection}[1]{\noindent{\textbf{#1}}}


%\iclrfinalcopy % Uncomment for camera-ready version, but NOT for submission.
\begin{document}


\maketitle
\def\thefootnote{\textsuperscript{$*$}}\footnotetext{Equal contributions.}
\def\thefootnote{\textsuperscript{$\dagger$}}\footnotetext{Project Lead \& Corresponding Author.}

\begin{abstract}
Consistency models are a new family of generative models capable of producing high-quality samples in either a single step or multiple steps. Recently, consistency models have demonstrated impressive performance, achieving results on par with diffusion models in the pixel space. However, the success of scaling consistency training to large-scale datasets, particularly for text-to-image and video generation tasks, is determined by performance in the latent space. In this work, we analyze the statistical differences between pixel and latent spaces, discovering that latent data often contains highly impulsive outliers, which significantly degrade the performance of iCT in the latent space. To address this, we replace Pseudo-Huber losses with Cauchy losses, effectively mitigating the impact of outliers. Additionally, we introduce a diffusion loss at early timesteps and employ optimal transport (OT) coupling to further enhance performance. Lastly, we introduce the adaptive scaling-$c$ scheduler to manage the robust training process and adopt Non-scaling LayerNorm in the architecture to better capture the statistics of the features and reduce outlier impact. With these strategies, we successfully train latent consistency models capable of high-quality sampling with one or two steps, significantly narrowing the performance gap between latent consistency and diffusion models. The implementation is released here: \url{https://github.com/quandao10/sLCT/}
\end{abstract}

\section{Introduction}
In recent years, generative models have gained significant prominence, with models like ChatGPT excelling in language generation and Stable Diffusion \citep{rombach2021highresolution}. In computer vision, the diffusion model \citep{song2020score, song2019generative, ho2020denoising, sohl2015deep} has quickly popularized and dominated the Adversarial Generative Model (GAN) \citep{goodfellow2014generative}. It is capable of generating high-quality diverse images that beat SoTA GAN models \citep{dhariwal2021diffusion}. Additionally, diffusion models are easier to train, as they avoid the common pitfalls of training instability and the need for meticulous hyperparameter tuning associated with GANs. The application of diffusion spans the entire computer vision field, including text-to-image generation \citep{rombach2021highresolution, gu2022vector}, image editing \citep{meng2021sdedit, cyclediffusion, huberman2024edit, han2024proxedit, he2024dice}, text-to-3D generation \citep{poole2022dreamfusion, wang2024prolificdreamer}, personalization \citep{ruiz2022dreambooth, van2023anti, kumari2023multi} and control generation \citep{zhang2023adding, brooks2022instructpix2pix, zhangli2024layout}. Despite their powerful capabilities, they require thousands of function evaluations for sampling, which is computationally expensive and hinders their application in the real world. Numerous efforts have been made to address this sampling challenge, either by proposing new training frameworks \citep{xiao2021tackling, rombach2021highresolution} or through distillation techniques \citep{meng2023distillation, yin2024one, sauer2023adversarial, dao2024self}. However, methods like \citep{xiao2021tackling} suffer from low recall due to the inherent challenges of GAN training, while \citep{rombach2021highresolution} still requires multi-step sampling. Distillation-based approaches, on the other hand, rely heavily on pretrained diffusion models and demand additional training.

Recently, \citep{song2023consistency} introduced a new family of generative models called the consistency model. Compared to the diffusion model \citep{song2019generative, song2020score, ho2020denoising}, the consistency model could both generate high-quality samples in a single step and multi-steps. The consistency model could be obtained by either consistency distillation (CD) or consistency training (CT). In previous work \citep{song2023consistency}, CD significantly outperforms CT. However, the CD requires additional training budget for using pretrained diffusion, and its generation quality is inherently limited by the pretrained diffusion. Subsequent research \citep{song2023improved} improves the consistency training procedure, resulting in performance that not only surpasses consistency distillation but also approaches SoTA performance of diffusion models. Additionally, several works \citep{kim2023consistency, geng2024consistency} have further enhanced the efficiency and performance of CT, achieving significant results. However, all of these efforts have focused exclusively on pixel space, where data is perfectly bounded. In contrast, most large-scale applications of diffusion models, such as text-to-image or video generation, operate in latent space \citep{rombach2021highresolution, gu2022vector}, as training on pixel space for large-scale datasets is impractical. Therefore, to scale consistency models for large datasets, the consistency must perform effectively in latent space. This work addresses the key question: How well can consistency models perform in latent space? To explore this, we first directly applied the SoTA pixel consistency training method, iCT \citep{song2023improved}, to latent space. The preliminary results were extremely poor, as illustrated in \cref{fig:qualitative_ict}, motivating a deeper investigation into the underlying causes of this suboptimal performance. We aim to improve CT in latent space, narrowing the gap between the performance of latent consistency and diffusion.

We first conducted a statistical analysis of both latent and pixel spaces. Our analysis revealed that the latent space contains impulsive outliers, which, while accounting for a very small proportion, exhibit extremely high values akin to salt-and-pepper noise. We also drew a parallel between Deep Q-Networks (DQN) and the Consistency Model, as both employ temporal difference (TD) loss. This could lead to training instability compared to the Kullback-Leibler (KL) loss used in diffusion models. Even in bounded pixel space, the TD loss still contains impulsive outliers, which \citep{song2023improved} addressed by proposing the use of Pseudo-Huber loss to reduce training instability. As shown in \cref{fig:impulsive_noise}, the latent input contains extremely high impulsive outliers, leading to very large TD values. Consequently, the Pseudo-Huber loss fails to sufficiently mitigate these outliers, resulting in poor performance as demonstrated in \cref{fig:qualitative_ict}. To overcome this challenge, we adopt Cauchy loss, which heavily penalizes extremely impulsive outliers. Additionally, we introduce diffusion loss at early timesteps along with optimal transport (OT) matching, both of which significantly enhance the model's performance. Finally, we propose an adaptive scaling $c$ schedule to effectively control the robustness of the model, and we incorporate Non-scaling LayerNorm into the architecture. With these techniques, we significantly boost the performance of latent consistency model compared to the baseline iCT framework and bridge the gap between the latent diffusion and consistency training.

\section{Related Works} \label{related}
% \subsection{Diffusion Model and Fast Sampling Technique}
% Diffusion models \citep{song2020score, song2019generative, ho2020denoising} have recently been raised as the most powerful generative model and outperform GAN \citep{goodfellow2014generative} in many applications. Diffusion models can generate high-fidelity images and possess good mode coverage, allowing diverse samples compared to GAN. However, diffusion models require many function evaluations (NFEs) during inference time. This drawback hinders its application in the real world. Many works are trying to tackle this drawback and achieve promising results. They can be divided into two main research categories: training from scratch and building upon pretrained diffusion models. Following the first category, there are several works, such as DDGAN \citep{xiao2021tackling}, LDM \citep{rombach2021highresolution}, and VQDiff \citep{gu2022vector}. DDGAN \citep{xiao2021tackling} proposes to use a larger step size in the forward process to reduce the NFEs; they use GAN models to implicitly learn the backward transition. Even though DDGAN \citep{xiao2021tackling} requires only a few sampling timesteps, it still suffers from low recall due to mode collapse from GAN. LDM \citep{rombach2021highresolution} does not directly reduce the number of sampling time steps; instead, it compresses the image to latent with a much smaller resolution. By training on latent space, LDM \citep{rombach2021highresolution} can significantly reduce both time and memory budget, and the inference is much faster than other pixel diffusion models. LDM \citep{rombach2021highresolution} has become the core technique in many large-scale diffusion models. Most real-world applications rely on LDM since it allows to scale diffusion models up on enormous high-resolution training datasets, which is impractical if training pixel diffusion models. In the second category, several works such as \citep{lu2022dpm, zhang2022fast} propose the high-order solver during inference. These works could successfully reduce sample NFEs to 10 without any training. However, they cannot sample with little NFEs such as 1 or 2. The other line of work is a distillation-based method. Progressive distillation \citep{salimans2022progressive} proposes to progressively distill diffusion model; each stage distills to reduce the sampling NFEs by half. This technique is costly since it requires to train many stages. Later works such as Guided-Distill \citep{meng2023distillation}, Swiftbrush \citep{nguyen2024swiftbrush}, DMD \citep{yin2024one}, and UFOGEN \citep{xu2024ufogen} manage to distill diffusion into few-step generation without compromising generative quality. The major drawback of these techniques is that additional training is required.
% Furthermore, some techniques, such as Swiftbrush \citep{nguyen2024swiftbrush} and DMD \citep{yin2024one}, do not have few-step sampling. Other methods, such as UFOGEN \citep{xu2024ufogen} and ADD \citep{sauer2023adversarial}, require training GAN, which could lead to training instability and low mode coverage. A standout among these techniques is the consistency model. The consistency model \citep{song2023consistency} is defined based on probability flow ODE (PF-ODE), allowing single- and multi-step sampling. The consistency model could be achieved via training from scratch and distillation from the diffusion model.

% \subsection{Consistency Model}

Consistency model \citep{song2023consistency, song2023improved} proposes a new type of generative model based on PF-ODE, which allows 1, 2 or multi-step sampling. The consistency model could be obtained by either training from scratch using an unbiased score estimator or distilling from a pretrained diffusion model. Several works improve the training of the consistency model. ACT \citep{kong2023act}, CTM \citep{kim2023consistency} propose to use additional GAN along with consistency objective. While these methods could improve the performance of consistency training, they require an additional discriminator, which could need to tune the hyperparameters carefully. MCM \citep{heek2024multistep} introduces multistep consistency training, which is a combination of TRACT \citep{berthelot2023tract} and CM \citep{song2023consistency}. MCM increases the sampling budget to 2-8 steps to tradeoff with efficient training and high-quality image generation. ECM \citep{geng2024consistency} initializes the consistency model by pretrained diffusion model and fine-tuning it using the consistency training objective. ECM vastly achieves improved training times while maintaining good generation performance. However, ECM requires pretrained diffusion model, which must use the same architecture as the pretrained diffusion architecture. Although these works successfully improve the performance and efficiency of consistency training, they only investigate consistency training on pixel space. As in the diffusion model, where most applications are now based on latent space, scaling the consistency training \citep{song2023consistency, song2023improved} to text-to-image or higher resolution generation requires latent space training. Otherwise, with pretrained diffusion model, we could either finetune consistency training \citep{geng2024consistency} or distill from diffusion model \citep{song2023consistency, luo2023latent}. CM \citep{song2023consistency} is the first work proposing consistency distillation (CD) on pixel space. LCM \citep{luo2023latent} later applies consistency technique on latent space and can generate high-quality images within a few steps. However, LCM's generated images using 1-2 steps are still blurry \citep{luo2023latent}. Recent works, such as Hyper-SD \cite{ren2024hyper} and TCD \cite{zheng2024trajectory}, have introduced notable improvements to latent consistency distillation. TCD \cite{zheng2024trajectory} employed CTM \cite{kim2023consistency} instead of CD \cite{song2023consistency}, significantly enhancing the performance of the distilled student model. Building on this, Hyper-SD \cite{ren2024hyper} divided the Probability Flow ODE (PF-ODE) into multiple components inspired by Multistep Consistency Models (MCM) \cite{heek2024multistep}, and applied TCD \cite{zheng2024trajectory} to each segment. Subsequently, Hyper-SD \cite{ren2024hyper} merged these segments progressively into a final model, integrating human feedback learning and score distillation \cite{yin2024one} to optimize one-step generation performance.

\section{Preliminaries} \label{sec:bg}
Denote $\pdata(\rvx_0)$ as the data distribution, the forward diffusion process gradually adds Gaussian noise with monotonically increasing standard deviation $\sigma(t)$ for $t \in \{0,1,\dots,T\}$ such that $p_t(\rvx_t|\rvx_0) = \gN(\rvx_0, \sigma^2(t)\mI)$ and $\sigma(t)$ is handcrafted such that $\sigma(0) = \sigma_{\min}$ and $\sigma(T)=\sigma_{\max}$. By setting $\sigma(t) = t$, the probability flow ODE (PF-ODE) from \citep{Karras2022edm} is defined as:
\begin{equation}
    \frac{\rd\rvx_t}{\rd t} = -t\nabla_{\rvx_t} \log p_t(\rvx_t) = \frac{\left( \rvx_t - \vf(\rvx_t, t) \right)}{t},  \label{eq:pf_ode}
\end{equation}
where $\vf:(\rvx_t, t) \rightarrow \rvx_0$ is the denoising function which directly predicts clean data $\rvx_0$ from given perturbed data $\rvx_t$. 
\citep{song2023consistency} defines consistency model based on PF-ODE in \cref{eq:pf_ode}, which builds a bijective mapping $\vf$ between noisy distribution $p(\rvx_t)$ and data distribution $\pdata(\rvx_0)$. The bijective mapping $\vf:(\rvx_t, t) \rightarrow \rvx_0$ is termed the consistency function. A consistency model $\vf_\theta(\rvx_t, t)$ is trained to approximate this consistency function $\vf(\rvx_t, t)$. The previous works \citep{song2023consistency, song2023improved, Karras2022edm} impose the boundary condition by parameterizing the consistency model as:
\begin{equation}
    \vf_\theta(\rvx_t, t) = c_{skip}(t)\rvx_t + c_{out}(t)\mF_\theta(\rvx_t, t), \label{eq:cm_param}
\end{equation}
where $\mF_\theta(\rvx_t, t)$ is a neural network to train. Note that, since $\sigma(t) = t$, we hereafter use $t$ and $\sigma$ interchangeably. $c_{skip}(t)$ and $c_{out}(t)$ are time-dependent functions such that $c_{skip}(\sigma_{\min}) = 1$ and $c_{out}(\sigma_{\max}) = 0$.

To train or distill consistency model, \citep{song2023consistency, song2023improved, Karras2022edm} firstly discretize the PF-ODE using a sequence of noise levels $\sigma_{\min} = t_{\min} = t_1 < t_2 < \dots < t_{N} = t_{\max} = \sigma_{\max}$, where $t_i = \left( t_{\min}^{1/\rho} + \frac{i-1}{N-1}(t_{\max}^{1/\rho
} - t_{\min}^{1/\rho})\right)^\rho$ and $\rho = 7$. 

\textbf{Consistency Distillation} Given the pretrained diffusion model $\vs_\phi(\rvx_t, t) \approx \nabla_{\rvx_t} \log p_t(\rvx_t)$, the consistency model could be distilled from the pretrained diffusion model using the following CD loss:
\begin{equation}
    \gL_{\text{CD}}(\theta, \theta^-) = \E\left[ \lambda(t_i)d(\vf_\theta(\rvx_{t_{i+1}}, t_{i+1}), \vf_{\theta^{-}}(\Tilde{\rvx}_{t_i}, t_{i})) \right], \label{loss:cd}
\end{equation}
where $\rvx_{t_{i+1}} = \rvx_0 + t_{i+1} \rvz$ with the $\rvx_0 \sim \pdata(\rvx_0)$ and $\rvz \sim \gN(0, \mI)$ and $\rvx_{t_i} = \rvx_{t_{i+1}} - (t_{i}-t_{i+1})t_{i+1} \nabla_{\rvx_{t_{i+1}}} \log p_{t_{i+1}}(\rvx_{t_{i+1}}) = \rvx_{t_{i+1}} - (t_{i}-t_{i+1})t_{i+1}\vs_\phi(\rvx_{t_{i+1}}, t_{i+1})$. 

\textbf{Consistency Training}
The consistency model is trained by minimizing the following CT loss:
\begin{equation}
    \gL_{\text{CT}}(\theta, \theta^-) = \E\left[ \lambda(t_i)d(\vf_\theta(\rvx_{t_{i+1}}, t_{i+1}), \vf_{\theta^{-}}(\rvx_{t_i}, t_{i})) \right], \label{loss:ct}
\end{equation}
where $\rvx_{t_i} = \rvx_0 + t_{i} \rvz$ and $\rvx_{t_{i+1}} = \rvx_0 + t_{i+1} \rvz$ with the same $\rvx_0 \sim \pdata(\rvx_0)$ and $\rvz \sim \gN(0, \mI)$

In \cref{loss:cd} and \cref{loss:ct}, $\vf_\theta$ and $\vf_{\theta^-}$ are referred to as the online network and the target network, respectively. The target's parameter $\theta^-$ is obtained by applying the Exponential Moving Average (EMA) to the student's parameter $\theta$ during the training and distillation as follows:
\begin{equation}
    \theta^- \leftarrow \text{stopgrad}(\mu\theta^- + (1-\mu)\theta), \label{ema}
\end{equation}
with $0\leq\mu<1$ as the EMA decay rate,  weighting function $\lambda(t_i)$ for each timestep $t_i$, and $d(\cdot, \cdot)$ is a predefined metric function. 

In CM \citep{song2023consistency}, the consistency training still lags behind the consistency distillation and diffusion models. iCT \citep{song2023improved} later propose several improvements that significantly boost the training performance and efficiency. First, the EMA decay rate $\mu$ is set to $0$ for better training convergence. Second, the Fourier scaling factor of noise embedding and the dropout rate are carefully examined. Third, iCT introduces Pseudo-Huber losses to replace $L_2$ and LPIPS since LPIPS introduces the undesirable bias in generative modeling \citep{song2023improved}. Furthermore, the Pseudo-Huber is more robust to outliers since it imposes a smaller penalty for larger errors than the $L_2$ metric. Fourth, iCT proposes an exp curriculum for total discretization steps N, which doubles N after a predefined number of training iterations. Moreover, uniform weighting $\lambda(t_i) = 1$ is replaced by $\lambda(t_i)=1/(t_{i+1}-t_i)$. Finally, iCT adopts a discrete Lognormal distribution for timestep sampling as EDM \citep{Karras2022edm}. With all these improvements, CT is now better than CD and performs on par with the diffusion models in pixel space.

\section{Method}
\label{method}
In this paper, we first investigate the underlying reason behind the performance discrepancy between latent and pixel space using the same training framework in \cref{sec:analysis}. Based on the analysis, we find out the root of unsatisfied performance on latent space could be attributed to two factors: the impulsive outlier and the unstable temporal difference (TD) for computing consistency loss. To deal with impulsive outliers of TD on pixel space, \citep{song2023improved} proposes the Pseudo-Huber function as training loss. For the latent space, the impulsive outlier is even more severe, making Pseudo-Huber loss not enough to resist the outlier. Therefore,  \cref{sec:cauchy} introduces Cauchy loss, which is more effective with extreme outliers. In the next \cref{sec:diff_loss} and \cref{sec:ot}, we propose to use diffusion loss at early timesteps and OT matching for regularizing the overkill effect of consistency at the early step and training variance reduction, respectively. Section \ref{sec:c} designs an adaptive scheduler of scaling $c$ to control the robustness of the proposed loss function more carefully, leading to better performance. Finally, in \cref{sec:norm}, we investigate the normalization layers of architecture and introduce Non-scaling LayerNorm to both capture feature statistic better and reduce the sensitivity to outliers.

\subsection{Analysis of latent space} \label{sec:analysis}

We first reimplement the iCT model \citep{song2023improved} on the latent dataset CelebA-HQ $32 \times 32 \times 4$ and pixel dataset Cifar-10 $32 \times 32 \times 3$. Hereafter, we refer to the latent iCT model as iLCT. We find that iCT framework works well on pixel datasets as claim \citep{song2023improved}. However, it produces worse results on latent datasets as in \cref{fig:qualitative_ict} and \cref{tab:main_exp}. The iLCT gets a very high FID above 30 for both datasets, and the generative images are not usable in the real world. This observation raises concern about the sensitivity of CT algorithm with training data, and we should carefully examine the training dataset. In addition, we notice that the DQN and CM use the same TD loss, which update the current state using the future state. Furthermore, they also possess the training instability. This motivates to carefully examine the behavior of TD loss with different training data.


While the pixel data lies within the range $[-1, 1]$ after being normalized, the range of latent data varies depending on the encoder model, which is blackbox and unbound. After normalizing latent data using mean and variance, we observe that the latent data contains high-magnitude values. We call them the impulsive outliers since they account for small probability but are usually very large values. In the bottom left of \cref{fig:impulsive_noise}, the impulsive outlier of latent data is red, spanning from $-9$ to $7$, while the first and third quartiles are just around $-1.4$ and $1.4$, respectively. We evaluate how the iCT will be affected by data outliers by analyzing the temporal difference $\text{TD} = f_\theta(\rvx_{t_{i+1}}, t_{i+1})-f_{\theta^-}(\rvx_{t_i}, t_{i})$. In the top right of \cref{fig:impulsive_noise}, the impulsive outliers of pixel TD range from -1.5 to 1.7, which are not too far from the interquartile range compared to latent TD. The impulsive outliers of latent TD range is much wider from -3.2 to 5. iCT uses Pseudo-Huber loss instead of $L_2$ loss since the Huber is less sensitive to outliers, see \cref{fig:loss}. However, for latent data, the Huber's reduction in sensitivity to outliers is not enough. This indicates that even using Pseudo-Huber loss, the iLCT training on latent space could still be unstable and lead to worse performance, which matches our experiment results on iLCT. Based on the above analysis, we hypothesize that the TD value statistic highly depends on the training data statistic.

\begin{figure}[!t]
    \centering
    \includegraphics[width=0.85\linewidth]{figures/impulsive_noise.pdf}
    \caption{\textbf{Box and Whisker Plot:} Impulsive noise comparison between pixel and latent spaces. The right column shows the statistics of TD values at 21 discretization steps. Other discretization steps exhibit same behavior, where impulsive outliers are consistently present regardless of the total discretization steps. The blue boxes represent interquartile ranges of the data, while the green and orange dashed lines indicate inner and outer fences, respectively. Outliers are marked with red dots.}
    \label{fig:impulsive_noise}
\end{figure}

%To understand the root of TD's impulsive outlier, we look into Deep Q Learning (DQN) from Reinforcement Learning (RL). There is a strong correlation between DQN and CM. While the DQN uses Q-value of future state as the ground truth for Q-value of the current state, the CM updates the current timesteps $f(\rvx_{t_{i+1}}, t_{i+1})$ using the smaller timesteps $f(\rvx_{t_{i}}, t_{i})$. This loss type is called the temporal difference in RL. For stable training, DQN uses target network $\theta^-$ to estimate Q-value of future state, and CM similarly adopts the same technique for consistency loss. The target network $\theta^-$ could be updated differently, such as Polyak-averaging, periodic, and standard TD updates \citep{lee2019target}. The Polyak-averaging update is simply EMA update using in \citep{song2023consistency}, and standard update corresponds to \citep{song2023improved} which $\theta^- \leftarrow \theta$ every iteration. The periodic update is a standard update but after every fixed number of iterations. In RL, Polyak-averaging and periodic updates are more stable but slowly convergent \citep{lee2019target}. Even using these stable target updates, there is still instability in TD training. Since the target network needs to change along with the online model, the target value of TD can never be fixed, which makes the loss highly oscillate. Therefore, even though the pixel data is very well-bounded within [-1, 1], the CM training is still affected by impulsive outliers due to the nature of TD loss.

To mitigate the impact of impulsive outliers, we could use more stable target updates like Polyak or periodic in TD loss \cite{lee2019target}, but they lead to very slow convergence, as shown in \citep{song2023consistency}. Even though CM is initialized by a pretrained diffusion model, the Polyak update still takes a long time to converge. Therefore, using Polyak or periodic updates is computationally expensive, and we keep the standard target update as in \citep{song2023improved}. Another direction is using a special metric for latent like LPIPS on pixel space \citep{song2023consistency}. \citep{kang2024diffusion2gan} proposes the E-LatentLPIPS as a metric for distillation and performs well on distillation tasks. However, this requires training a network as a metric and using this metric during the training process will also increase the training budget. To avoid the overhead of the training, we seek a simple loss function like Pseudo-Huber but be more effective with outliers. We find that the Cauchy loss function \citep{black1996robust, barron2019general} could be a promising candidate in place of Pseudo-Huber for latent space.
\subsection{Cauchy Loss against Impulsive Outlier} \label{sec:cauchy}
In this section, we introduce the Cauchy loss \citep{black1996robust, barron2019general} function to deal with extreme impulsive outliers. The Cauchy loss function has the following form:
\begin{equation}
    d_{\text{Cauchy}}(\rvx, \rvy)=  \log \left(1+\frac{||\rvx-\rvy||_2^2}{2c^2}\right), \label{loss:cauchy}
\end{equation}
and we also consider two additional robust losses, which are Pseudo-Huber \citep{song2023improved, barron2019general} and Geman-McClure \citep{geman1986bayesian, barron2019general}
\begin{equation}
    d_{\text{Pseudo-Huber}}(\rvx, \rvy)= \sqrt{||\rvx-\rvy||_2^2 + c^2} - c, \label{loss:huber}
\end{equation}
\begin{equation}
    d_{\text{Geman-McClure}}(\rvx, \rvy)= \frac{2||\rvx-\rvy||_2^2}{||\rvx-\rvy||_2^2 + 4c^2}, \label{loss:gm}
\end{equation}
where $c$ is the scaling parameter to control how robust the loss is to the outlier. We analyze their robustness behavior against outliers. As shown in \cref{fig:loss_val}, the Pseudo-Huber loss linearly increases like $L_1$ loss for the large residuals $\rvx-\rvy$. In contrast, the Cauchy loss only grows logarithmically, and the Geman-McClure suppresses the loss value to $1$ for the outliers. 

The Pseudo-Huber loss works well if the residual value does not grow too high and, therefore, has a good performance on the pixel space. However, for the latent space, as shown in the bottom right of \cref{fig:impulsive_noise}, the TD suffers from extremely high values coming from the impulsive outlier in the latent dataset, the Cauchy loss could be more suitable since it significantly dampens the influence of extreme outliers. Otherwise, even Geman-McClure is very highly effective for removing outlier effects than two previous losses; it gives a gradient $0$ for high TD value and completely ignores the impulsive outliers as \cref{fig:loss_derivative}. This is unexpected behavior because even though we call the high-value latent impulsive outlier, they actually could encode important information from original data. Completely ignoring them could significantly hurt the performance of training model. Based on this analysis, we choose Cauchy loss as the default loss for latent CM for the rest of the paper. The loss ablation is provided in \cref{tab:ablate_robust}.


\begin{figure}[!ht]
    \centering
    \begin{subfigure}[t]{0.40\textwidth}
        \centering
        \includegraphics[width=1.0\textwidth]{figures/func.png}
        \caption{Robust Loss}
        \label{fig:loss_val}
    \end{subfigure}%
    ~ 
    \begin{subfigure}[t]{0.40 \textwidth}
        \centering
        \includegraphics[width=1.0\textwidth]{figures/derivative.png}
        \caption{Derivative of Robust Loss}
        \label{fig:loss_derivative}
    \end{subfigure}
    \caption{Analysis of robust loss: Pseudo-Huber, Cauchy, and Geman-McClure}
    \label{fig:loss}
\end{figure}
% \vspace{-3mm}


\subsection{Diffusion Loss at small timestep} \label{sec:diff_loss}
For small noise level $\sigma$, the ground truth of $f(\rvx_\sigma, \sigma)$ can be well approximated by $\rvx_0$, but this does not hold for large noise levels. Therefore, for low-level noise, the consistency objective seems to be overkill and harms the model's performance since instead of optimizing $f_\theta(\rvx_\sigma, \sigma)$ to approximated ground truth $\rvx_0$, the consistency objective optimizes through a proxy estimator $f_{\theta^-}(\rvx_{<\sigma}, <\sigma)$ leading to error accumulation over timestep. To regularize this overkill, we propose to apply an additional diffusion loss on small noise level as follows:

\begin{equation}
    L_{diff} = ||f_\theta(\rvx_{t_i}, t_i) - \rvx_0||^2_2 \quad \forall i \leq \text{int(N $\cdot$ r)}, \label{loss:diff}
\end{equation}

where N is the number of training discretization steps and $r\in[0;1]$ is the diffusion threshold, and we heuristicly choose $r=0.25$. We do not apply diffusion loss for large noise levels since $f(\rvx_\sigma, \sigma)$ will differ greatly from the target $\rvx_0$, leading to very high $L_2$ diffusion loss. This could harm the training consistency process, misleading to the wrong solution. We provide the ablation study in \cref{tab:diff_loss}. Furthermore, CTM \citep{kim2023consistency} also proposes to use diffusion loss, but they use them on both high and low-level noise, which is different from us. 

\subsection{OT matching reduces the variance} \label{sec:ot}
% \vspace{-5mm}
In this section, we adopt the OT matching technique from previous works \citep{pooladian2023multisample, lee2023minimizing}. \citep{pooladian2023multisample} proposes to use OT to match noise and data in the training batch, such as the moving $L_2$ cost is optimal. On the other hand, \citep{lee2023minimizing} introduces $\beta\text{VAE}$ for creating noise corresponding to data and train flow matching on the defined data-noise pairs. By reassigning noise-data pairs, these works significantly reduce the variance during the diffusion/flow matching training process, leading to a faster and more stable training process. According to \citep{zhang2023emergence}, the consistency training and diffusion models produce highly similar images given the same noise input. Therefore, the final output solution of the consistency and diffusion models should be close to each other. Since OT matching helps reduce the variance during training diffusion, it could be useful to reduce the variance of consistency training. In our implementation, we follow \citep{pooladian2023multisample, tong2023improving} using the POT library to map from noise to data in the training batch. The overhead caused by minibatch OT is relatively small, only around $0.93\%$ training time, but gains significant performance improvement as shown in \cref{tab:strategy}.

\subsection{Adaptive $c$ scheduler} \label{sec:c}

% \begin{figure}[h!]
%     \centering
%     \includegraphics[width=0.5\linewidth]{figures/C_by_NFE.pdf}
%     \caption{Our robust adaptive $c$ scheduler}
%     \label{fig:proposed_c}
% \end{figure}

\begin{figure}[h!]
    \centering
    \includegraphics[width=0.8\linewidth]{figures/C_merge.pdf}
    \caption{Model convergence plot on different $c$ schedule. (Left) Our proposed $c$ values. Performance on FID (Middle) and Recall (Right) of our proposed $c$ in comparison with different choices.}
    \label{fig:fid_vary_c}
\end{figure}
% \vspace{-5mm}

In this section, we examine the choice of scaling parameter $c$ in robust loss functions. The scaling parameter controls the robustness level, which is very important for model performance. The previous work \citep{song2023improved} proposes to use fixed constant $c_0 = 0.00054\sqrt{d}$, where $d$ is the dimension of data. We find that using this simple fixed $c$ is not yet optimal for the training consistency model. Especially in this paper, we follow the Exp curriculum specified by \cref{exp_cur} in \citep{song2023improved}, which doubles the total discretization step after a defined number of training iterations. 
\begin{equation}
    \text{NFE}(k)=\min \left(s_0 2^{\left\lfloor\frac{k}{K^{\prime}}\right\rfloor}, s_1\right)+1, \quad K^{\prime}=\left\lfloor\frac{K}{\log _2\left\lfloor s_1 / s_0\right\rfloor+1}\right\rfloor, \label{exp_cur}
\end{equation}
where $k$ is current training iteration, $K$ is total training iteration and $s_0 = 10, s_1=640$. During training, we notice that the variance of TD is significantly reduced as doubling total discretization steps using \cref{exp_cur}. Since the more discretization steps, the closer distance of $\rvx_{t_i}$ and $\rvx_{t_{i+1}}$, the TD value's range between them should be smaller. However, the impulsive outlier still exists regardless of the number of discretization steps. Intuitively, we propose a heuristic adaptive $c$ scheduler where the $c$ is scaled down proportional to the reduction rate of TD variance as the number of discretization steps increases. We plot our $c$ scheduler versus discretization steps in \cref{fig:fid_vary_c} and we fit the $c$ scheduler to get the scheduler equation as following:

\begin{equation}
    c = \exp(-1.18 * \log(\text{NFE}(k) - 1) - 0.72) \label{eq:c_scheduler}
\end{equation}

\subsection{Non-scaling Layernorm} \label{sec:norm}
As mentioned in \cref{sec:analysis}, the statistic of training data could play an important role in the success of consistency training. Furthermore, in architecture design, the normalization layer specifically handles the statistics of input, output, and hidden features. In this section, we investigate the normalization layer choice for consistency training, which is sensitive to training data statistics. 

Currently, both \citep{song2023improved, song2023consistency} use the UNet architecture from \citep{dhariwal2021diffusion}. In UNet \citep{dhariwal2021diffusion}, GroupNorm is used in every layer by default. The GroupNorm only captures the statistics over groups of local channels, while the LayerNorm further captures the statistics' overall features. Therefore, LayerNorm is better at capturing fine-grained statistics over the entire feature. We further carry out the experiments for other types of normalization, such as LayerNorm, InstanceNorm, RMSNorm in \cref{tab:norm_layer} and observe that the GroupNorm and InstanceNorm perform relatively well compared to others, especially LayerNorm. This could be due to that they are less sensitive to the outliers since they only capture the statistic over groups of channels. Therefore, the impulsive features only affect the normalization of a group containing them. For the LayerNorm, the impulsive features could negatively impact the overall features's normalization. We further look into the LayerNorm implementation and suspect that the scaling term could significantly amplify the outliers across features by serving as a shared parameter. This observation is also mentioned in \citep{wei2022outlier} for LLM quantization. In implementation, we set the \textbf{scaling term of LayerNorm to $1$} and \textbf{disabled the gradient update} for it \eqref{operation:layernorm}. We refer to it as Non-scaling LayerNorm (NsLN) as \citep{wei2022outlier}.

\begin{equation}
    \text{LN}_{\gamma, \beta}(\rvx) = \frac{\rvx - u(\rvx)}{\sqrt{\sigma^{2}(\rvx) + \epsilon}} \cdot \gamma + \beta, \quad
    \text{NsLN}_{\beta}(\rvx) = \frac{\rvx - u(\rvx)}{\sqrt{\sigma^{2}(\rvx) + \epsilon}} + \beta, \label{operation:layernorm}
\end{equation}

where $u(\rvx)$ and $\sigma^{2}(\rvx)$ are mean and variance of $\rvx$.

% \subsection{Improve Consistency Distillation}

% \vspace{-15mm}
\section{Experiment} \label{exp}

\subsection{Performance of our training technique} \label{exp:main}
% \vspace{-3mm}
\begin{table}[t]
    \centering
    \begin{tabular}{cc}
        \begin{minipage}[c]{0.58\textwidth}
            \centering
            \begin{subtable}[t]{\textwidth}
                \resizebox{\textwidth}{!}{%
                \begin{tabular}{l c c c c c}
                    \toprule
                    Model & NFE$\downarrow$ & FID$\downarrow$ & Recall$\uparrow$ & Epochs & Total Bs\\
                    \midrule 
                    \multicolumn{5}{c}{\textbf{Pixel Diffusion Model}}\\
                    \midrule
                    WaveDiff \citep{phung2023wavediff} & 2 & 5.94 & 0.37 & 500 & 64\\
                    Score SDE \citep{song2020score} & 4000 & 7.23 & - & ~6.2K & - \\
                    DDGAN \citep{xiao2021tackling} & 2 & 7.64 & 0.36 & 800 & 32 \\
                    RDUOT \citep{dao2024high} & 2 & 5.60 & 0.38 & 600 & 24 \\
                    RDM \citep{teng2023relay} & 270 & 3.15 & 0.55 & 4K & - \\
                    UNCSN++ \citep{kim2021soft} & 2000 & 7.16 & - & - & -\\
                    \midrule 
                    \multicolumn{5}{c}{\textbf{Latent Diffusion Model}}\\
                    \midrule
                    LFM-8 \citep{dao2023flow} & 85 & 5.82 & 0.41 & 500 & 112\\ 
                    LDM-4 \citep{rombach2021highresolution} & 200 & 5.11 & 0.49 & 600 &48 \\
                    LSGM \citep{vahdat2021score} & 23 & 7.22 & - & 1K &-\\
                    DDMI \citep{park2024ddmi} & 1000 & 7.25 & - & - &-\\
                    
                    DIMSUM \citep{phung2024dimsum} & 73 & 3.76  & 0.56 & 395 &32\\
                    $\text{LDM-8}^\dagger$ & 250 & {8.85}  & - & 1.4K &128\\
                    
                    \midrule
                    \multicolumn{5}{c}{\textbf{Latent Consistency Model}}\\
                    \midrule
                    iLCT \citep{song2023improved} & 1 & 37.15 & 0.12 & 1.4K &128\\
                    iLCT \citep{song2023improved} & 2 & 16.84 & 0.24 & 1.4K &128\\
                    Ours  & 1 & 7.27 & 0.50 & 1.4K &128\\
                    Ours  & 2 & 6.93 & 0.52 & 1.4K &128\\
                    \bottomrule
                \end{tabular}%
                }
            \caption{CelebA-HQ}
            \label{tab:celeb}
            \end{subtable}
        \end{minipage}
        \hfill
        \begin{minipage}[c]{0.42\textwidth}
            \centering
            \begin{subtable}[t]{\textwidth}
                \resizebox{\textwidth}{!}{%
                \begin{tabular}{l c c c c c}
                    \toprule
                    Model & NFE$\downarrow$ & FID$\downarrow$ & Recall$\uparrow$ & Epochs & Total Bs\\
                    \midrule 
                    \multicolumn{5}{c}{\textbf{Pixel Diffusion Model}}\\
                    \midrule
                    WaveDiff \citep{phung2023wavediff} & 2 & 5.94 & 0.37 & 500 & 64\\
                    Score SDE \citep{song2020score} & 4000 & 7.23 & - &6.2K & -\\
                    DDGAN \citep{xiao2021tackling} & 2 & 5.25 & 0.36 & 500 & 32\\
                    \midrule 
                    \multicolumn{5}{c}{\textbf{Latent Diffusion Model}}\\
                    \midrule
                    LFM-8 \citep{dao2023flow} & 90 & 7.70 & 0.39 & 90 &112\\
                    LDM-8 \citep{rombach2021highresolution} & 400 & 4.02 & 0.52 & 400 &96\\
                    $\text{LDM-8}^\dagger$ & 250 & {10.81} & - & 1.8K &256\\
                    \midrule
                    \multicolumn{5}{c}{\textbf{Latent Consistency Model}}\\
                    \midrule
                    iLCT \citep{song2023improved} & 1 &52.45  &0.11  & 1.8K &256\\
                    iLCT \citep{song2023improved} & 2 &24.67  &0.17  & 1.8K &256\\
                    Ours  & 1 &8.87  &0.47  & 1.8K &256\\
                    Ours  & 2 &7.71  &0.48  & 1.8K &256\\
                    \bottomrule
                \end{tabular}%
                }
            \caption{LSUN Church}
            \label{tab:lsun}
            \end{subtable}
            \hfill
            \begin{subtable}[t]{\textwidth}
                \resizebox{\textwidth}{!}{%
                \begin{tabular}{l c c c c c}
                    \toprule
                    Model & NFE$\downarrow$ & FID$\downarrow$ & Recall$\uparrow$ & Epochs &Total Bs \\
                    \midrule 
                    \multicolumn{5}{c}{\textbf{Latent Diffusion Model}}\\
                    \midrule
                    LFM-8 \citep{dao2023flow} & 84 & 8.07 & 0.40 & 700 &128\\
                    LDM-4 \citep{rombach2021highresolution} & 200 & 4.98 & 0.50 & 400 &42\\
                    $\text{LDM-8}^\dagger$ & 250 &{10.23} & - & 1.4K &128\\
                    \midrule
                    \multicolumn{5}{c}{\textbf{Latent Consistency Model}}\\
                    \midrule
                    iLCT \citep{song2023improved} & 1 & 48.82  & 0.15 & 1.4K &128 \\
                    iLCT \citep{song2023improved} & 2 & 21.15 & 0.19 & 1.4K &128\\
                    Ours  & 1 & 8.72  &0.42 & 1.4K &128\\
                    Ours  & 2 & 8.29  &0.43  & 1.4K &128\\
                    \bottomrule
                \end{tabular}%
                }
            \caption{FFHQ}
            \label{tab:ffhq}
            \end{subtable}
        \end{minipage}
    \end{tabular}
    \caption{Our performance on CelebA-HQ, LSUN Church, FFHQ datasets at resolution $256 \times 256$. ($\dagger$) means training on our machine with the same diffusion forward and equivalent architecture.}
    \label{tab:main_exp}
\end{table}


\minisection{Experiment Setting:}
We measure the performance of our proposed technique on three datasets: CelebA-HQ \citep{celeba}, FFHQ \citep{karras2019style}, and LSUN Church \citep{lsun}, at the same resolution of $256 \times 256$. Following LDM \citep{rombach2021highresolution}, we use pretrained VAE KL-8 \footnote{https://huggingface.co/stabilityai/sd-vae-ft-ema} to obtain latent data with the dimensionality of $32 \times 32 \times 4$. We adopt the OpenAI UNet architecture \citep{dhariwal2021diffusion} as the default architecture throughout the paper. Furthermore, we use the variance exploding (VE) forward process for all the consistency and diffusion experiments following \citep{song2023consistency, song2023improved}. The baseline iCT is self-implemented based on official implementation CM \citep{song2023consistency} and iCT \citep{song2023improved}. We refer to this baseline as iLCT. Furthermore, we also train the latent diffusion model for each dataset using the same VE forward noise process for fair comparisons with our technique. This LDM model is referred to as $\text{LDM-8}^{\dagger}$ in \cref{tab:main_exp}. All three frameworks, including ours, iLCT, and $\text{LDM-8}^{\dagger}$, use the same architecture.

\minisection{Evaluation:} During the evaluation, we first generate 50K latent samples and then pass them through VAE's decoder to obtain the pixel images. We use two well-known metrics, Fréchet Inception Distance (FID) \citep{fid} and Recall \citep{kynkaanniemi2019improved}, for measuring the performance of the model given the training data and 50K generated images. 

\minisection{Model Performance:} We report the performance of our model across all three datasets in \cref{tab:main_exp}, primarily to compare it with the baseline iLCT \citep{song2023improved} and LDM \citep{rombach2021highresolution}. For both 1 and 2 NFE sampling, we observe that the FIDs of iLCT for all datasets are notably high (over 30 for 1-NFE sampling and over 16 for 2-NFE sampling), consistent with the qualitative results shown in \cref{fig:qualitative_ict}, where the generated image is unrealistic and contain many artifacts. This poor performance of iLCT in latent space is expected, as the Pseudo-Huber training losses are insufficient in mitigating extreme impulsive outliers, as discussed in \cref{sec:analysis} and \cref{sec:cauchy}. In contrast, our proposed framework demonstrates significantly better FID and Recall than iLCT. Specifically, we achieve 1-NFE sampling FIDs of 7.27, 8.87, and 8.29 for CelebA-HQ, LSUN Church, and FFHQ, respectively. For 2-NFE sampling, our FID scores improve across all three datasets. Notably, our 1-NFE sampling outperforms $\text{LDM-8}^{\dagger}$, using the same noise scheduler and architecture. However, our models still exhibit higher FIDs compared to LDM \citep{rombach2021highresolution} and LFM \citep{dao2023flow}. In contrast, we only need 1 or 2 timestep sampling, whereas they require multiple timesteps for high-fidelity generation.
 It's important to note that we employ the VE forward process, whereas these other methods use VP and flow-matching forward processes. Furthermore, the qualitative results of our framework, as shown in \cref{fig:qualitative_1nfe}, highlight our ability to generate high-quality images.

\begin{figure}[ht]
\centering
    \begin{subfigure}[b]{0.3\textwidth}
    \centering
    \includegraphics[width=\textwidth]{figures/lct_celeba.pdf}
    \caption{CelebA-HQ}
    \label{fig:qualitative_celeba}
    \end{subfigure}
    \hfill
    \begin{subfigure}[b]{0.3\textwidth}
    \centering
    \includegraphics[width=\textwidth]{figures/lct_church.pdf}
    \caption{LSUN Church}
    \label{fig:qualitative_lsun_church}
    \end{subfigure}
    \hfill
    \begin{subfigure}[b]{0.3\textwidth}
    \centering
    \includegraphics[width=\textwidth]{figures/ffhq.pdf}
    \caption{FFHQ}
    \label{fig:qualitative_ffhq}
    \end{subfigure}
    \caption{Our qualitative results using 1-NFE at resolution $256 \times 256$}
    \label{fig:qualitative_1nfe}
\end{figure}

\begin{figure}[ht]
\centering
    \begin{subfigure}[b]{0.3\textwidth}
    \centering
    \includegraphics[width=\textwidth]{figures/lct_celeba_baseline.pdf}
    \caption{CelebA-HQ}
    \label{fig:qualitative_ict_celeba}
    \end{subfigure}
    \hfill
    \begin{subfigure}[b]{0.3\textwidth}
    \centering
    \includegraphics[width=\textwidth]{figures/lsun_church.pdf}
    \caption{LSUN Church}
    \label{fig:qualitative_ict_lsun_church}
    \end{subfigure}
    \hfill
    \begin{subfigure}[b]{0.3\textwidth}
    \centering
    \includegraphics[width=\textwidth]{figures/lct_ffhq_baseline.pdf}
    \caption{FFHQ}
    \label{fig:qualitative_ict_ffhq}
    \end{subfigure}
    \caption{iLCT qualitative results using 1-NFE at resolution  $256 \times 256$}
    \label{fig:qualitative_ict}
\end{figure}


\subsection{Ablation of proposed framework} \label{exp:ablation}

We ablate our proposed techniques on the CelebA-HQ $256\times256$ dataset, with all FID and Recall metrics measured using 1-NFE sampling. All models are trained for 1,400 epochs with the same hyperparameters. As shown in \cref{tab:strategy}, replacing Pseudo-Huber losses with Cauchy losses makes our model's training less sensitive to impulsive outliers, resulting in a significant FID reduction from $37.15$ to $13.02$. This demonstrates the effectiveness of Cauchy losses in handling extremely high-value outliers, as discussed in \cref{sec:cauchy}. Additionally, applying diffusion loss at small timesteps further reduces FID by approximately 4 points to $9.11$, as this loss term stabilizes the training process at small timesteps, as described in \cref{sec:diff_loss}. Introducing OT coupling during minibatch training reduces training variance, improving the FID to $8.89$. Notably, by replacing the fixed scaling term $c=c_0$, \citep{song2023improved} with an adaptive scaling schedule, our model achieves an additional FID reduction of more than 1 point, reaching $7.76$, highlighting the importance of the scaling term $c$ in robustness control. Finally, we propose using NsLN, which removes the scaling term from LayerNorm to handle outliers more effectively. NsLN captures feature statistics while mitigating the negative impact of outliers, resulting in our best FID of $7.27$.

\minisection{Robustness Loss} \label{exp:ablation:robust_loss}
To analyze the impact of different robust loss functions, we conduct an ablation study using our best settings but replace the Cauchy loss with alternatives such as L2, E-LatentLPIPS \cite{kang2024diffusion2gan}, the Huber and the Geman-McClure loss. The results, shown in \cref{tab:ablate_robust}, indicate that both Huber and Geman-McClure underperform compared to the Cauchy loss when applied in the latent space. This is because the Huber loss remains too sensitive to extremely impulsive outliers, while the Geman-McClure loss tends to ignore such outliers entirely, leading to a loss of important information. This behavior is also discussed in \cref{sec:cauchy}.

% \vspace{-2mm}
\begin{table}[h!]
    \centering
    \begin{tabular}{cc}
        \begin{minipage}[c]{0.40\textwidth}
            \centering
            
            \begin{subtable}[t]{\textwidth}
                \centering
                \begin{tabular}{lcc}
                    \toprule
                    Framework                      & FID $\downarrow$   & Recall $\uparrow$   \\
                    \midrule
                    iLCT                           & 37.15              & 0.12                \\
                    \midrule
                    Cauchy                         & 13.02              & 0.36                \\
                    + Diff                         & 9.11               & 0.41                \\
                    + OT                           & 8.89               & 0.42                \\
                    + Scaled $c$                   & 7.76               & 0.47                \\
                    + NsLN       & \textbf{7.27}               &\textbf{0.50}                \\
                    % \rowcolor{pink!60}+ NsLN       & 7.27               & 0.50                \\
                    \bottomrule
                \end{tabular}
                \caption{Components of proposed framework}
                \label{tab:strategy}
            \end{subtable}
            \hfill
            % \vspace{2mm}
            \begin{subtable}[t]{\textwidth}
                \centering
                \begin{tabular}{lcc}
                    \toprule
                    $r$           & FID $\downarrow$   & Recall $\uparrow$   \\
                    \midrule
                    1.0                    & 7.47               & 0.49                \\
                    0.6                      & 7.33               & 0.49                \\
                    % \rowcolor{pink!60}0.25   & 7.27               & 0.50                \\
                    0.25   & \textbf{7.27}               & \textbf{0.50}                \\
                    \bottomrule
                \end{tabular}
                \caption{Threshold using Diffusion loss}
                \label{tab:diff_loss}
            \end{subtable}
            
        \end{minipage}
        \hfill
        \begin{minipage}[c]{0.40\textwidth}
            \centering
            \begin{subtable}[t]{\textwidth}
                \centering
                \begin{tabular}{lcc}
                    \toprule
                    Loss                        & FID $\downarrow$   & Recall $\uparrow$   \\
                    \midrule
                     L2                          & 50.40              & 0.04                \\
                    E-LatentLPIPS               & 11.49              & 0.47                \\
                    \midrule
                    Huber                       & 9.97               & 0.44                \\
                    Geman McClure               & 11.28              & 0.44                \\
                    % \rowcolor{pink!60} Cauchy   & 7.27               & 0.50                \\
                    Cauchy   & \textbf{7.27}               & \textbf{0.50}                \\
                    \bottomrule
                \end{tabular}
                \caption{Robust losses.}
                \label{tab:ablate_robust}
            \end{subtable}
            \hfill
            % \vspace{2mm}
            \begin{subtable}[t]{\textwidth}
                \centering
                \begin{tabular}{lcc}
                    \toprule
                    Norm layer                             & FID $\downarrow$   & Recall $\uparrow$   \\
                    \midrule
                    $\text{GN}$                           & 7.76               & 0.47                \\
                    \midrule
                    % \midrule
                    IN                                      & 8.47               & 0.43                \\
                    % $\text{IN}^\dagger$                     & 8.03               & 0.46                \\
                    % \midrule
                    LN                                      & 9.05               & 0.46                \\
                    % $\text{LN}^\dagger$                     & 7.92               & 0.47                \\
                    % \midrule
                    RMS                                     & 8.96               & 0.46                \\
                    % $\text{RMS}^\dagger$                    & 7.62               & 0.47                \\
                    % \midrule
                    NsLN                 &\textbf{7.27}               &\textbf{0.50}                \\
                    % \rowcolor{pink!60} NsLN                 & 7.27               & 0.50                \\
                    % \rowcolor{white}$\text{NsLN}^\dagger$   & 7.64               & 0.47                \\
                    \bottomrule
                \end{tabular}
                \caption{Norm Layer}
                \label{tab:norm_layer}
            \end{subtable}
        \end{minipage}
    \end{tabular}
    \caption{Ablation Studies on CelebA-HQ $256\times256$ dataset at epoch 1400}
    \label{tab:ablation}
\end{table}

\minisection{Diffusion Threshold} \label{exp:ablation:diff_loss}
In this section, we explore the impact of varying the threshold for applying the diffusion loss function in combination with the consistency loss. We observe that using the diffusion loss at every timestep improves consistency training; however, it underperforms compared to applying the diffusion loss selectively at smaller timesteps such as $r=0.25$ as shown in \cref{tab:diff_loss}. This suggests that applying diffusion losses primarily at small noise levels improves performance as discussed \cref{sec:diff_loss}. At larger timesteps, the diffusion loss may conflict with the consistency loss, potentially guiding the model toward incorrect solutions, thereby reducing overall performance.



\minisection{Scaling term $c$ scheduler} \label{exp:ablation:vary_c}
In this section, we compare the performance of our adaptive scaling $c$ scheduler with the fixed scaling $c$ scheduler proposed in \citep{song2023improved}. Our model demonstrates better convergence with the proposed adaptive $c$ scheduler. The rationale behind this improvement lies in the fact that, as the discretization steps increases using the exponential curriculum, the value of the TD scales down. Despite the reduced TD value, impulsive outliers still persist. A fixed large scaling $c$ is not effective in handling these outliers. To address this, we scale $c$ down as discretization steps increases, which leads to better performance, as shown in \cref{fig:fid_vary_c}.


\minisection{Normalizing Layer} \label{exp:ablation:norm_layer}
We denote GN, IN, LN, RMS, and NsLN as GroupNorm, InstanceNorm, LayerNorm, RMSNorm, and Non-scaling LayerNorm, respectively. The baseline UNet architecture from \citep{dhariwal2021diffusion} uses GroupNorm by default. We replace the normalization layers in the baseline with each of these types and train the model on CelebA-HQ using the best settings. The results are reported in \cref{tab:norm_layer}. GN and IN only capture local statistics, making them more robust to outliers, as outliers in one region do not affect others. In contrast, LN captures statistics from all features, making it more vulnerable to outliers because an outlier affects all features through a shared scaling term. By removing the scaling term in LN, we obtain NsLN, which is both effective in capturing feature statistics and resistant to outliers. As shown in \cref{tab:norm_layer}, NsLN outperforms the second-best GN by 0.5 FID and significantly outperforms LN.
\section{Conclusion}
CT is highly sensitive to the statistical properties of the training data. In particular, when the data contains impulsive noise, such as latent data, CT becomes unstable, leading to poor performance. In this work, we propose using the Cauchy loss, which is more robust to outliers, along with several improved training strategies to enhance model performance. As a result, we can generate high-fidelity images from latent CT, effectively bridging the gap between latent diffusion models and consistency models. Future work could explore further improvements to the architecture, specifically by investigating normalization methods that reduce the impact of outliers. For example, removing the scaling term from group normalization or instance normalization may help mitigate outlier effects. Another promising future direction is the integration of this technique with Consistency Trajectory Models (CTM) \cite{kim2023consistency}, as CTM has demonstrated improved performance compared to traditional Consistency Models (CM) \cite{song2023consistency}.
\section*{Acknowledgements}
Research funded by research grants to Prof. Dimitris Metaxas from NSF: 2310966, 2235405, 2212301, 2003874, 1951890, AFOSR 23RT0630, and NIH 2R01HL127661.


\bibliography{iclr2025_conference}
\bibliographystyle{iclr2025_conference}

\newpage
\appendix
\section{Appendix}
We provide additional uncurated samples of our models for three datasets: CelebaA-HQ (\ref{fig:appendix:celeba_onestep}, \ref{fig:appendix:celeba_twostep}), LSUN Church (\ref{fig:appendix:lsun_onestep}, \ref{fig:appendix:lsun_twostep}), and FFHQ (\ref{fig:appendix:ffhq_onestep}, \ref{fig:appendix:ffhq_twostep}). We also provide additional uncurated samples of our models on CelebaA-HQ trained with L2 loss (\ref{fig:appendix:celeba_onestep_ilct_l2}) and E-LatentLPIPS loss (\ref{fig:appendix:celeba_onestep_ilct_elatentlpips}).

\begin{figure}[h]
\centering
    \includegraphics[width=0.6\textwidth]{figures/lct_celeba_more.pdf}
    \caption{One-step samples on CelebA-HQ $256 \times 256$}
    \label{fig:appendix:celeba_onestep}
\end{figure}

\begin{figure}[h]
\centering
    \includegraphics[width=0.6\textwidth]{figures/lct_celeba_more_2step.pdf}
    \caption{Two-step samples on CelebA-HQ $256 \times 256$}
    \label{fig:appendix:celeba_twostep}
\end{figure}

\begin{figure}[h]
\centering
    \includegraphics[width=0.6\textwidth]{figures/lct_lsun_more.pdf}
    \caption{One-step samples on LSUN Church $256 \times 256$}
    \label{fig:appendix:lsun_onestep}
\end{figure}

\begin{figure}[h]
\centering
    \includegraphics[width=0.6\textwidth]{figures/lct_lsun_more_2step.pdf}
    \caption{Two-step samples on LSUN Church $256 \times 256$}
    \label{fig:appendix:lsun_twostep}
\end{figure}

\begin{figure}[h]
\centering
    \includegraphics[width=0.6\textwidth]{figures/lct_ffhq_more.pdf}
    \caption{One-step samples on FFHQ $256 \times 256$}
    \label{fig:appendix:ffhq_onestep}
\end{figure}

\begin{figure}[h]
\centering
    \includegraphics[width=0.6\textwidth]{figures/lct_ffhq_more_2step.pdf}
    \caption{Two-step samples on FFHQ $256 \times 256$}
    \label{fig:appendix:ffhq_twostep}
\end{figure}


\begin{figure}[h]
\centering
    \includegraphics[width=0.6\textwidth]{figures/ilct_l2_celeba.pdf}
    \caption{One-step samples on CelebA-HQ $256 \times 256$ (L2 loss)}
    \label{fig:appendix:celeba_onestep_ilct_l2}
\end{figure}

\begin{figure}[h]
\centering
    \includegraphics[width=0.6\textwidth]{figures/ilct_latentlpips_celeba.pdf}
    \caption{One-step samples on CelebA-HQ $256 \times 256$ (E-LatentLPIPS loss)}
    \label{fig:appendix:celeba_onestep_ilct_elatentlpips}
\end{figure}

\end{document}

\bibliography{iclr2025_conference}
\bibliographystyle{iclr2025_conference}

\newpage
\appendix
\section{Dataset Statistics}
\label{dataset}

\begin{table}[h]
\small
\centering
\caption{Statistics of the preprocessed datasets. “\textbf{Avg}. \textit{len}” represents the average length of item sequences.}
\label{tab:preprocess}
\resizebox{0.65\columnwidth}{!}{
\begin{tabular}{lrrrrr} 
\toprule
\textbf{Datasets} & \textbf{\#Users} & \textbf{\#Items} & \textbf{\#Interactions} & \textbf{Sparsity} & \textbf{Avg.}\textit{ len}  \\ 
\midrule
Pet               & 183697           & 31986            & 1571284                 & 99.97\%           & 8.55                        \\
Cell              & 123885           & 38298            & 873966                  & 99.98\%           & 7.05                        \\
Automotive        & 105490           & 39537            & 845454                  & 99.98\%           & 8.01                        \\
Tools             & 144326           & 41482            & 1153959                 & 99.98\%           & 8.00                        \\
Toys              & 135748           & 47520            & 1158602                 & 99.98\%           & 8.53                        \\
Sports            & 191920           & 56395            & 1504646                 & 99.99\%           & 7.84                        \\ 
\midrule
Instruments       & 17112            & 6250             & 136226                  & 99.87\%           & 7.96                        \\
Arts              & 22171            & 9416             & 174079                  & 99.92\%           & 7.85                        \\
Games             & 42259            & 13839            & 373514                  & 99.94\%           & 8.84                        \\
\bottomrule
\end{tabular}
}
\end{table}

To evaluate the performance of the proposed approach, we conduct experiments in the pre-trained source and target domain settings.
We use six categories from from the Amazon Product Reviews dataset \citep{ni2019justifying} for pre-training, including \textit{“Pet Supplies”}, \textit{"Cell Phones and Accessories"}, \textit{“Automotive”}, \textit{“Tools and Home Improvement”}, \textit{“Toys and Games”}, \textit{“Sports and Outdoors”}, and three categories for sequential recommendation tasks, including \textit{“Musical Instruments”}, \textit{“Arts Crafts and Sewing”}, \textit{“Video Games”}.

Each item in the dataset is associated with a title, a description, and an image. Following previous work \citep{tiger}, we first filter out unpopular users and items with less than five interactions. Then, we create user behavior sequences based on the chronological order. The maximum item sequence length is uniformly set to 20 to meet all baseline requirements. The statistics of our preprocessed datasets are shown in Table \ref{tab:preprocess}.

\section{Baselines}
\label{baseline}

We compare the proposed approach with the following baseline methods: 

\setlist[itemize]{leftmargin=1.5em}
\begin{itemize}
    \item \textbf{GRU4Rec}~\citep{hidasi2015session} introduces Gating Recurrent Unit (GRU) to model user action sequences for session-based recommendations.
    \item \textbf{SASRec}~\citep{sasrec} uses a directional self-attentive model to capture item correlations within a sequence.
    \item \textbf{BERT4Rec}~\citep{Sun2019BERT4RecSR} employs a bi-directional self-attentive model with the cloze objective for modeling user behavior sequences.
    \item \textbf{FDSA}~\citep{fdsa} uses a self-attentive model to capture item and feature transition patterns.
    \item \textbf{S}$^3$\textbf{-Rec}~\citep{s3rec} pre-trains sequential models with mutual information maximization to learn the correlations among attributes, items, subsequences, and sequences.
    \item \textbf{VQ-Rec} \citep{hou2023learning} learns vector-quantized item representations for transferable sequential recommenders.
    \item \textbf{MISSRec} \citep{wang2023missrec} is a multi-modal pre-training and transfer learning framework for sequential recommendation.
    \item \textbf{P5-CID} \citep{p5, p5id} organizes multiple recommendation tasks in a text-to-text format and models different tasks uniformly using the T5 model. Here, we employ P5 with collaborative indexing as the baseline.
    \item \textbf{VIP5} \citep{geng2023vip5} is a multimodal foundation model considering visual, textual, and personalization modalities under the P5 recommendation paradigm, to unify various modalities and recommendation tasks.
    \item \textbf{TIGER} \citep{tiger} adopts the generative retrieval paradigm for sequential recommendation and introduces a semantic ID to uniquely identify items.
\end{itemize}

\section{Implementation Details.}
\label{detail}
To obtain textual representations, we employ LLaMA to encode the title and description of the item as its embedding and use mean pooling to aggregate multiple representations. To obtain visual representations, we utilize CLIP's \citep{clip} image branch as an encoder to encode the images of items, and we employ ViT-L/14 as the backbone. 
Both the encoder and decoder of RQ-VAE are implemented as Multi-Layer Perceptrons (MLPs) with ReLU activation functions.
The level of codebooks is set to 4, with each level consisting of 256 codebook vectors, and each vector has a dimension of 32.  
The model is optimized using the AdamW optimizer, employing a learning rate of 0.001 and a batch size of 1024.

Following previous work \citep{tiger}, we use the T5 \citep{t5} framework to implement our transformer based encoder-decoder architecture. We use 4 layers each for the transformer-based encoder and decoder models with 6 self-attention heads of dimension 64 in each layer. The MLP and the input dimension was set as 1024 and 128, respectively. 
The number of prompt tokens for every task is set to 4.
We employ the AdamW \citep{adamw} optimizer for model optimization, setting the weight decay to 0.01. 
During pre-training, we utilize a batch size of 4096 with a learning rate set to 0.001.
For alignment tuning, we employ a batch size of 512 with a maximum learning rate of 5e-4, and utilize a cosine scheduler with warm-up to adjust the learning rate. 
% More implementation details can be found in Appendix \ref{more_detail}.

Our experiments utilize the Tesla V100 GPU. For pretraining, we use four cards, and for fine-tuning, we use two cards. Since the model has around 13 million parameters, there is still a substantial amount of GPU memory remaining.

% \section{More Implementation Details}
% \label{more_detail}

% To obtain textual representations, we employ LLaMA to encode the title and description of the item as its embedding and use mean pooling to aggregate multiple representations. To obtain visual representations, we utilize CLIP's \citep{clip} image branch as an encoder to encode the images of items, and we employ ViT-L/14 as the backbone. 
% Both the encoder and decoder of RQ-VAE are implemented as Multi-Layer Perceptrons (MLPs) with ReLU activation functions.
% The level of codebooks is set to 4, with each level consisting of 256 codebook vectors, and each vector has a dimension of 32.  
% The model is optimized using the AdamW optimizer, employing a learning rate of 0.001 and a batch size of 1024.

% Our experiments utilize the Tesla V100 GPU. For pretraining, we use four cards, and for fine-tuning, we use two cards. Since the model has around 13 million parameters, there is still a substantial amount of GPU memory remaining.


\section{More Ablation Studies}
\label{more_ablation}

\subsection{Quantitative language generation tasks.}
\label{more_QLG}

We have supplemented Table \ref{tab:more_QLG} with more detailed ablation experiments of quantitative language generation tasks without pre-training. 
We study the impact of each task on recommendation performance through a combination of different tasks. We find that: 1) the AIG task always results in a significant performance improvement; 2) the QLA task needs to be paired with the AIG task in order to achieve better results.
This suggests that quantitative language serves as a bridge for knowledge transfer in recommendations, but we need to design appropriate quantitative language tasks.


\begin{table}
\small
\centering
\caption{Detailed ablation study of various quantitative language generation tasks without pre-training}
\label{tab:more_QLG}
\resizebox{0.85\columnwidth}{!}{
\begin{tabular}{llcccccc} 
\toprule
\multirow{2}{*}{Modal} & \multirow{2}{*}{Tasks}      & \multicolumn{2}{c}{Instruments}   & \multicolumn{2}{c}{Arts}          & \multicolumn{2}{c}{Games}          \\ 
\cmidrule(l){3-8}
                       &                             & HR@10           & NDCG@10         & HR@10           & NDCG@10         & HR@10           & NDCG@10          \\ 
\midrule
\multirow{6}{*}{Text}  & $\text{NIG}_1$              & 0.1277          & 0.0987          & 0.1163          & 0.0844          & 0.0885          & 0.0473           \\
                       & $\text{NIG}_1 \text{ + QLA}$ & 0.1275          & 0.0986          & 0.1166          & 0.0831          & 0.0871          & 0.0465           \\
                       & NIG                         & 0.1275          & 0.0986          & 0.1205          & 0.0877          & 0.0928          & 0.0493           \\
                       & NIG + QLA                   & 0.1263          & 0.0983          & 0.1204          & 0.0867          & 0.0919          & 0.0492           \\
                       & NIG + AIG                   & \uline{0.1279}  & \uline{0.0987}  & \uline{0.1249}  & \uline{0.0895}  & \uline{0.1002}  & \uline{0.0529}   \\
                       & NIG + AIG + QLA             & \textbf{0.1282} & \textbf{0.0993} & \textbf{0.1293} & \textbf{0.0913} & \textbf{0.1010} & \textbf{0.0531}  \\ 
\midrule\midrule
\multirow{6}{*}{Image} & $\text{NIG}_2$              & 0.1243          & 0.0968          & 0.1117          & 0.0812          & 0.0881          & 0.0478           \\
                       & $\text{NIG}_2 \text{ + QLA}$ & 0.1237          & 0.0978          & 0.1143          & 0.0826          & 0.0877          & 0.0468           \\
                       & NIG                         & 0.1262          & 0.0986          & 0.1158          & 0.0848          & 0.0899          & 0.0487           \\
                       & NIG + QLA                   & 0.1265          & 0.0988          & 0.1164          & 0.0849          & 0.0945          & 0.0505           \\
                       & NIG + AIG                   & \textbf{0.1299} & \uline{0.0998}  & \uline{0.1218}  & \uline{0.0878}  & \uline{0.1002}  & \uline{0.0534}   \\
                       & NIG + AIG + QLA             & \uline{0.1280}  & \textbf{0.1001} & \textbf{0.1259} & \textbf{0.0901} & \textbf{0.1017} & \textbf{0.0540}  \\
\bottomrule
\end{tabular}
}
\end{table}


\subsection{Pre-training.}
\label{more_pre-training}

We further provide the results of using $\text{NIG}_2$ to evaluate the performance of recommendations in Table \ref{tab:more_pretraining}.
From the results, it can be seen that:
1) On the Instruments and Arts datasets, both pre-training and Quantitative Language Generation (QLG) tasks are useful for improving recommendation performance. This indicates that quantitative language can migrate recommendation knowledge from the source domain and other modalities to the target task.
2) On the Games dataset, QLG with pre-training impairs performance, which might be due to some conflict of recommendation knowledge between the source domain and another modality. In the future, we will further explore this phenomenon.

\begin{table}
\small
\centering
\caption{Detailed ablation study of pre-training and quantitative language generation tasks.}
\label{tab:more_pretraining}
\resizebox{0.95\columnwidth}{!}{
\begin{tabular}{llcccccc} 
\toprule
\multirow{2}{*}{Modal} & \multirow{2}{*}{Methods}                  & \multicolumn{2}{c}{Instruments}                     & \multicolumn{2}{c}{Arts}                   & \multicolumn{2}{c}{Games}                            \\ 
\cmidrule(l){3-8}
                       &                                           & HR@10                    & NDCG@10                  & HR@10                    & NDCG@10         & HR@10                    & NDCG@10                   \\ 
\midrule
\multirow{4}{*}{Text}  & $\text{(0) NIG}_1$                        & 0.1277                   & 0.0987                   & 0.1163                   & 0.0844          & 0.0885                   & 0.0473                    \\
                       & (1) QLG~                                  & 0.1282                   & 0.0993                   & 0.1293                   & 0.0913          & 0.1010                   & 0.0531                    \\
                       & $\text{(2) NIG}_1 \text{w/ pre-training}$ & 0.1334                   & 0.1043                   & 0.1305                   & \textbf{0.0959} & 0.0950                   & 0.0508                    \\
                       & (3) QLG~~w/ pre-training                  & \uline{0.1362}           & \uline{0.1051}           & \uline{0.1314}           & 0.0944          & 0.0995                   & 0.0521                    \\ 
\midrule
\multirow{3}{*}{Image} & $\text{(4) NIG}_2$                        & 0.1243                   & 0.0968                   & 0.1117                   & 0.0812          & 0.0881                   & 0.0478                    \\
                       & (5) QLG~ ~                                & 0.1280                   & 0.1001                   & 0.1259                   & 0.0901          & \uline{0.1017}           & \uline{0.0540}            \\
                       & (6) QLG~~w/ pre-training    & 0.1322                   & 0.1029                   & 0.1265                   & 0.0914          & 0.0987                   & 0.0526                    \\ 
\midrule
All                    & (7) MQL4GRec                  & \textbf{\textbf{0.1375}} & \textbf{\textbf{0.1060}} & \textbf{\textbf{0.1327}} & \uline{0.0950}  & \textbf{\textbf{0.1033}} & \textbf{\textbf{0.0548}}  \\
\bottomrule
\end{tabular}
}
\end{table}

\section{Discussion}
\label{discussion}

\subsection{Handling Collisions.}
\label{app:handling}

To address the issue of item collisions, some methods \citep{tiger, p5id} append an additional identifier to the item indices, which may introduce semantically unrelated distributions.
LC-Rec \citep{zheng2023adapting} introduces a uniform distribution constraint to prevent multiple items from clustering in the same leaf node. 
Although the LC-Rec achieves better performance in handling collisions compared to previous approaches, it has an inherent problem: it cannot completely resolve item collisions when item modal content is identical or when the number of collisions exceeds the size of the last level's codebook. This leads to another issue: \textbf{multiple items sharing the same indices results in an unfair comparison of performance}.

In contrast, our method of dealing with collisions is more rational and can essentially solve the aforementioned problems. From the experimental results, our approach achieved similar outcomes to LC-Rec, hence we did not use a dedicated table to list the experimental results.

\subsection{Limitations.}
\label{Limitations}

Although our method achieves state-of-the-art performance, there are still some inherent limitations. For example: 1) The inference time is longer compared to traditional recommendation methods. This is an inherent flaw of generative recommendation systems, as such methods typically employ beam search and auto-regressive techniques to generate the next token.
2) Our method requires item content information, and the scenario where item content is missing has not yet been studied in the paper. This is an issue that we need to further analyze in our next steps.


\end{document}
