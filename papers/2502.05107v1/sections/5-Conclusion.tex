\section{Conclusion and Discussion}
In this paper, we introduce 3DMolFormer for structure-based drug discovery, a dual-channel transformer-based framework designed to process parallel sequences of tokens and numerical values representing pocket-ligand complexes. Through self-supervised large-scale pre-training and supervised fine-tuning, 3DMolFormer can accurately and efficiently predict the binding poses of ligands to protein pockets. Furthermore, through reinforcement learning fine-tuning, 3DMolFormer can generate drug candidates that exhibit high binding affinities for a given protein target, along with favorable drug-likeness and synthesizability. Above all, 3DMolFormer is the first machine learning framework that can simultaneously address both protein-ligand docking and pocket-aware 3D drug design, and it outperforms previous baselines in both tasks.

It is noteworthy that many recent deep learning models for 3D molecules, such as Uni-Mol, Pocket2Mol, TargetDiff, and DecompDiff, which serve as baselines in our experiments, adhere to the concept of "equivariance" introduced by geometric deep learning~\citep{Equivariance,Equivariance2}. However, the 3DMolFormer model does not explicitly enforce SE(3)-symmetry. It appears that through the normalization of 3D coordinates and random rotations during data augmentation, 3DMolFormer has acquired the SE(3)-equivariance by training on a sufficiently large and diverse dataset. This approach aligns with recent successful methods in the field, including AlphaFold3~\citep{AlphaFold3}, which also does not rely on SE(3)-equivariant architectures.

Admittedly, our approach still has some limitations. First, 3DMolFormer does not account for the flexibility of proteins during ligand binding, which may affect the accuracy of subsequent binding affinity prediction. Second, protein-ligand binding is a dynamic process, but 3DMolFormer struggles to capture this dynamism effectively. Finally, 3DMolFormer does not consider environmental factors such as temperature and pH, which can significantly influence the 3D conformation of the binding complex. These issues represent core challenges in current computational methods for structure-based drug discovery, and we look forward to future work addressing these limitations. Furthermore, the implementation details in 3DMolFormer have the potential to be further optimized, for example, advanced methods of multi-objective reinforcement learning~\citep{MORL} may be introduced into the drug design process.