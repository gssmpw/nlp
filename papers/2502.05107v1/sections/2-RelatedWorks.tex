\section{Related Works}
\begin{figure}[t]
    \centering
    \includegraphics[width=\textwidth]{imgs/PocketSeq.pdf}
    \vspace{-0.5cm}
    \caption{The parallel sequence of a protein pocket with 3D coordinates.}
    \label{PocketSeq}
    \vspace{-0.2cm}
\end{figure}
\begin{figure}[t]
    \centering
    \includegraphics[width=\textwidth]{imgs/LigandSeq.pdf}
    \vspace{-0.5cm}
    \caption{The parallel sequence of a small molecule ligand with 3D coordinates.}
    \label{LigandSeq}
    \vspace{-0.2cm}
\end{figure}

\paragraph{Molecular Pre-training}
% representation learning: cannot generate
The success of large-scale pre-training has extended from NLP to the field of drug discovery~\citep{ChemPretrainingSurvey, chen2023structure}. Many studies focus on molecular representation learning, which maps molecular structures to informative embeddings for downstream predictive tasks~\cite{PhysChem,GEM,DrugClip}. Several representation learning methods for protein-ligand binding have been proposed, including InteractionGraphNet~\citep{InteractionGraphNet} and BindNet~\citep{BindNet}, with Uni-Mol~\citep{Uni-Mol} collecting and pre-training on extensive 3D datasets of proteins and small molecules, achieving high accuracy in protein-ligand docking. Furthermore, models such as MolGPT~\citep{MolGPT}, Chemformer~\citep{Chemformer}, and BindGPT~\citep{BindGPT} utilize pre-training to enhance molecular distribution learning, enabling applications in generative tasks.

\paragraph{Protein-ligand Docking}
Protein-ligand docking encompasses three sequential tasks: binding site prediction, binding pose prediction, and binding affinity prediction, with binding pose prediction being the most critical in structure-based drug discovery~\citep{SBDDsurvey2}. Traditional search-based methods typically employ combinatorial optimization techniques to identify the best binding poses (known as targeted docking) within a given protein pocket, using tools such as AutoDock4~\citep{AutoDock4}, AutoDock Vina~\citep{AutoDockVina,AutoDockVina2}, and Smina~\citep{Smina}, which are widely used in practical virtual screening. Recently, deep learning approaches have been introduced for this task, exemplified by DeepDock~\citep{DeepDock} and Uni-Mol~\citep{Uni-Mol}. Additionally, various deep learning techniques for blind docking have emerged, which simultaneously predict binding sites and poses~\citep{EquiBind,TankBind,E3Bind,FABind,DiffDock,DiffDock-L}. However, blind docking methods are primarily hindered by inaccuracies in binding site prediction, making direct comparisons with targeted docking methods less meaningful. Moreover, some end-to-end approaches that predict binding affinity without 3D poses fail to provide the crucial structural information required in SBDD~\citep{AffinityPredSurvey}.

\paragraph{Pocket-aware 3D Drug Design}
Drug design is the ultimate goal of molecular design. Currently, most machine learning methods focus on generating 1D SMILES strings or 2D molecular graphs~\citep{SMILES-LSTM,LIMO,GEAM}, with reinforcement learning being a popular paradigm~\citep{Reinvent,GCPN,GEGL,RationaleRL,MolGym,FREED}. However, these approaches can only output discrete information about atoms and chemical bonds, lacking the capability to generate 3D coordinate values, thus limiting their application in SBDD. In contrast, pocket-aware 3D drug design explicitly utilizes the 3D structures of protein targets to generate \textit{de novo} small molecules with high binding affinity. Various machine learning techniques have been applied to pocket-aware 3D drug design, including genetic algorithms (e.g., AutoGrow~\citep{AutoGrow4}), variational autoencoders (e.g., liGAN~\citep{liGAN}), autoregressive models (e.g., AR~\citep{AR}, Pocket2Mol~\citep{Pocket2Mol}, Lingo3DMol~\citep{Lingo3DMol}), and flow models (GraphBP~\citep{GraphBP}). Recently, diffusion models have achieved state-of-the-art performance in this task, including DiffSBDD~\citep{DiffSBDD}, TargetDiff~\citep{TargetDiff}, and DecompDiff~\citep{DecompDiff}. Notably, some studies have developed transformer-based 3D drug design models. The XYZ-transformer~\citep{XYZtransformer} directly uses 3D coordinate values (retaining three decimal places) as tokens, while BindGPT~\citep{BindGPT} decomposes the integer and decimal parts of coordinates into two tokens to reduce vocabulary size. Token-Mol~\citep{Token-Mol}, on the other hand, employs torsion angles of small molecules instead of coordinate values to shorten sequence lengths. However, these methods represent values using discrete tokens, which disrupts the continuity of coordinates.