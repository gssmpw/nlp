\pdfoutput=1
% In particular, the hyperref package requires pdfLaTeX in order to break URLs across lines.

\documentclass[11pt]{article}
\usepackage{overpic}
\usepackage{float}

%\newcommand{\CG}{\mathcal{G}\xspace}
\newcommand{\CV}{\mathcal{V}\xspace}
\newcommand{\CE}{\mathcal{E}\xspace}
\newcommand{\CA}{\mathcal{A}\xspace}
\newcommand{\CF}{\mathcal{F}\xspace}
\newcommand{\CR}{\mathcal{R}\xspace}
\newcommand{\CB}{\mathcal{B}\xspace}
\newcommand{\CX}{\mathcal{X}\xspace}
\newcommand{\CK}{\mathcal{K}\xspace}
\newcommand{\CM}{\mathcal{M}\xspace}
\newcommand{\CC}{\mathcal{C}\xspace}
\newcommand{\CL}{\mathcal{L}\xspace}
\newcommand{\CI}{\mathcal{I}\xspace}
\newcommand{\CQ}{\mathcal{Q}\xspace}
\newcommand{\CO}{\mathcal{O}\xspace}
\newcommand{\CP}{\mathcal{P}\xspace}
\newcommand{\CS}{\mathcal{S}\xspace}
\newcommand{\CT}{\mathcal{T}\xspace}
\newcommand{\CJ}{\mathcal{J}\xspace}
\usepackage[para]{footmisc}
\usepackage{subfig}
% \usepackage{subcaption}
% \usepackage{array}
% \usepackage{colortbl}



% Remove the ''review'' option to generate the final version.
\usepackage[]{acl2023}

\usepackage{xspace}
\usepackage{xcolor}
\usepackage[utf8]{inputenc}
\usepackage{pgfplots}
\usepackage{dsfont}
\DeclareUnicodeCharacter{2212}{−}
\usepgfplotslibrary{groupplots,dateplot}
\usetikzlibrary{patterns,shapes.arrows}
\pgfplotsset{compat=newest}

\usepackage{tikzscale}
\usepackage{relsize}
\usepackage{amsmath,amssymb}
\newcommand{\probP}{\text{I\kern-0.15em P}}

\newcommand{\abr}[1]{\textsc{#1}}
\newcommand{\ai}{\textsc{ai}\xspace}
\newcommand{\clip}{\textsc{clip}\xspace}
\newcommand{\nlp}{\textsc{nlp}\xspace}
\newcommand{\mcqa}{MCQA\xspace}
\newcommand{\emcqa}{E-MCQA\xspace}
\newcommand{\mcq}{MCQ\xspace}
\newcommand{\success}{SR\xspace}

\newcommand\todo[1]{\textcolor{red}{[TODO: #1]}}

\usepackage{multirow, colortbl}

\usepackage{tabularx,booktabs}
\usepackage{makecell}

\usepackage[normalem]{ulem}
\useunder{\uline}{\ul}{}

\usepackage{graphicx}

\definecolor{ablation6}{HTML}{fcefed}
\definecolor{ablation_tie}{HTML}{fce3e1}

\definecolor{ablation5}{HTML}{fcd8d4}
\definecolor{ablation4}{HTML}{FBC3BC}
\definecolor{ablation3}{HTML}{F7A399}
\definecolor{ablation2}{HTML}{F38375}
\definecolor{ablation1}{HTML}{EF6351}

\newcommand{\inlinecode}[1]{%
    \begin{tikzpicture}[baseline=0ex]%
         \node[anchor=base,%
         text height=0.7em,%
         text depth=0.7ex,%
         inner ysep=0pt,%
         draw=lightgray!50,%
         fill=lightgray!50,%
         rounded corners=2pt] at (0,0) {\footnotesize\texttt{#1}};%
    \end{tikzpicture}%
}

\newcommand{\ctext}[3][RGB]{%
  \begingroup
  \definecolor{hlcolor}{#1}{#2}\sethlcolor{hlcolor}%
  \hl{#3}%
  \endgroup
}

% \newcommand{\methodname}[0]{\texttt{LeDec}}

\newcommand{\inlinecodesmall}[1]{%
    \begin{tikzpicture}[baseline=0ex]%
         \node[anchor=base,%
         text height=0.7em,%
         text depth=0.7ex,%
         inner ysep=0pt,%
         draw=lightgray!50,%
         fill=lightgray!50,%
         rounded corners=2pt] at (0,0) {\tiny \texttt{#1}};%
    \end{tikzpicture}%
}

% \definecolor{ablation1}{HTML}{e3faf8}
% \definecolor{inablation2}{HTML}{C4FFF9}
% \definecolor{ablation3}{HTML}{9CEAEF}
% \definecolor{ablation4}{HTML}{68D8D6}
% \definecolor{ablation5}{HTML}{3DCCC7}
% \definecolor{ablation6}{HTML}{07BEB8}



\usepackage[]{algpseudocode}
\usepackage[]{algorithm}
\usepackage{float}
\algtext*{EndFor}%
\algtext*{EndProcedure}%

\usepackage{booktabs}
\usepackage[normalem]{ulem}
\useunder{\uline}{\ul}{}

% Standard package includes
\usepackage{times}
\usepackage{latexsym}
\usepackage{adjustbox}

% For proper rendering and hyphenation of words containing Latin characters (including in bib files)
\usepackage[T1]{fontenc}
\usepackage{amsmath}
\usepackage{amssymb}
\usepackage{booktabs}
\usepackage{tikzscale}
\usepackage{amsmath}
\newcommand{\specialcell}[2][c]{%
  \begin{tabular}[#1]{@{}c@{}}#2\end{tabular}}

\newcommand{\specialcellleft}[2][l]{%
\begin{tabular}[#1]{@{}l@{}}#2\end{tabular}}

\usepackage{multirow, colortbl}

\usepackage{tabularx,booktabs}
\usepackage{makecell}
\usepackage{multirow}
% For Vietnamese characters
% \usepackage[T5]{fontenc}
% See https://www.latex-project.org/help/documentation/encguide.pdf for other character sets
\usepackage{scalerel,xparse}
% This assumes your files are encoded as UTF8
% \usepackage[utf8]{inputenc}
%\usepackage[dvipsnames]{xcolor}

% This is not strictly necessary, and may be commented out,
% but it will improve the layout of the manuscript,
% and will typically save some space.
\usepackage{cleveref}
\usepackage{microtype}
\usepackage[most]{tcolorbox}
%\usepackage{emoji}


\usepackage{enumitem}
%\usepackage{emoji}
\crefformat{section}{\S#2#1#3}
\crefformat{subsection}{\S#2#1#3}
\crefformat{subsubsection}{\S#2#1#3}

\definecolor{bggray}{rgb}{0.95, 0.95, 0.95}
\usepackage[%
    framemethod=tikz,
    skipbelow=\topskip,
    skipabove=\topskip
]{mdframed}
\mdfsetup{%
    leftmargin=0pt,
    rightmargin=0pt,
    backgroundcolor=bggray,
    middlelinecolor=black,
    roundcorner=3
}

\definecolor{SkyBlue}{rgb}{0.53, 0.81, 0.92}
\newcommand{\colorBoxColor}{SkyBlue}

\newtcolorbox[list inside=prompt,auto counter,number within=section]{prompt}[1][]{
    colbacktitle=black!60,
    fonttitle=\small,
    coltitle=white,
    fontupper=\footnotesize,
    boxsep=3pt,
    left=0pt,
    right=0pt,
    top=0pt,
    bottom=0pt,
    boxrule=1pt,
    #1
}

\definecolor{UMDred}{HTML}{ed1c24}
\newcommand{\nishant}[1]{\textcolor{UMDred}{[Nishant: {#1}]}}

\definecolor{yellowcolor}{HTML}{ffc20e}
\definecolor{redcolor}{HTML}{e99999}
\definecolor{orangecolor}{HTML}{f6b26b}
\definecolor{yellowcolor}{HTML}{ffd966}
\definecolor{bluecolor}{HTML}{a0c5e8}
\definecolor{purplecolor}{HTML}{d9d2e9}

\setlength\titlebox{5cm}

\setlength{\fboxsep}{0pt}

\title{Which of These Best Describes Multiple Choice Evaluation with LLMs?\\A) Forced B) Flawed C) Fixable \colorbox{yellow}{D) All of the Above}}

\author{Nishant Balepur \\
  University of Maryland\\
  \texttt{nbalepur@umd.edu} \\\And
  Rachel Rudinger \\
  University of Maryland \\
  \texttt{rudinger@umd.edu}
  \\\And
  Jordan Boyd-Graber \\
  University of Maryland \\
  \texttt{jbg@umiacs.umd.edu}}

% 
 
% this is where we disable/enable the note
\newif\ifnotes
\notestrue % turn this on and disable the next line to display notes
% \notesfalse % turn this on to hide notes, which overwrites the previous line. 
 
\definecolor{fycolor}{HTML}{c378e3} % fumeng's color
\newcommand{\fy}[1]{\ifnotes{\leavevmode{\color{fycolor}{(Fumeng: #1)}}}\fi}
\newcommand{\fyedit}[1]{\ifnotes{\leavevmode\color{fycolor}{#1}}\elseif{#1}\fi}
\newcommand{\fyreplace}[2]{\ifnotes{\color{fycolor}{\sout{#1} {#2}}}\else{#2}\fi}
\newcommand{\fydelete}[1]{\ifnotes{\color{fycolor}{\sout{#1}}}\fi}
\newcommand{\fyadd}[1]{\ifnotes{\color{fycolor}{#1}}\else{#1}\fi}
% \newcommand{\todo}[1]{\ifnotes{\leavevmode\color{red}{#1}}\fi}

\begin{document}
\maketitle



End-to-end imitation learning offers a promising approach for training robot policies. However, generalizing to new settings—such as unseen scenes, tasks, and object instances—remains a significant challenge. Although large-scale robot demonstration datasets have shown potential for inducing generalization, they are resource-intensive to scale. In contrast, human video data is abundant and diverse, presenting an attractive alternative. Yet, these human-video datasets lack action labels, complicating their use in imitation learning. Existing methods attempt to extract grounded action representations (e.g., hand poses), but resulting policies struggle to bridge the embodiment gap between human and robot actions.
% our approach
We propose an alternative approach: leveraging language-based reasoning from human videos - essential for guiding robot actions - to train generalizable robot policies. Building on recent advances in reasoning-based policy architectures, we introduce Reasoning through Action-free Data (RAD). RAD learns from both robot demonstration data (with reasoning and action labels) and action-free human video data (with only reasoning labels). The robot data teaches the model to map reasoning to low-level actions, while the action-free data enhances reasoning capabilities. Additionally, we will release a new dataset of 3,377 human-hand demonstrations compatible with the Bridge V2 benchmark. This dataset includes chain-of-thought reasoning annotations and hand-tracking data to help facilitate future work on reasoning-driven robot learning.
% experiments
Our experiments demonstrate that RAD enables effective transfer across the embodiment gap, allowing robots to perform tasks seen only in action-free data. Furthermore, scaling up action-free reasoning data significantly improves policy performance and generalization to novel tasks. These results highlight the promise of reasoning-driven learning from action-free datasets for advancing generalizable robot control. 
% releasing dataset
Website: \href{https://rad-generalization.github.io}{here}.


\section{Introduction}
\section{Introduction}
\label{sec:intro}

\begin{figure*}[tb]
    \centering
    \includegraphics[width=0.848\linewidth]{figs/circuitnn.pdf} 
    \caption{Illustration of differentiable CircuitNN. CircuitNN is designed based on differentiable NAND gates. After DAS is guided by PI and PO pairs of the truth table, CircuitNN can get the precise circuit architecture logic equivalent to the truth table.}
    \label{fig:circuitnn}
\end{figure*}

% 1. Describe the importance of logic synthesis
% 2. Existing Problems
% (a) Neural Architecture Search: Unstable, Predefined Setting, etc.
% (b) Circuit Generation: Probabilistic Model, Logic Equivalence

With the rapid advancement of technology, the scale of integrated circuits (ICs) has expanded exponentially. 
This expansion has introduced significant challenges in chip manufacturing, particularly concerning power and area metrics.
A primary objective in IC design is achieving the same circuit function with fewer transistors, thereby reducing power usage and area occupancy.

Logic synthesis~\cite{hachtel2005logicsynth}, a critical step in electronic design automation (EDA), transforms behavioral-level circuit designs into optimized gate-level circuits, ultimately yielding the final IC layout. 
The primary goal of logic synthesis is to identify the physical implementation with the fewest gates for a given circuit function. 
This task constitutes a challenging NP-hard combinatorial optimization problem. 
Current logic synthesis tools~\cite{brayton2010abc, wolf2013yosys} rely on human-designed heuristics, often leading to sub-optimal outcomes.

Differentiable architecture search (DAS) techniques~\cite{liu2018darts, chu2020darts} offer novel perspectives on addressing challenges in this problem.
Circuit functions can be represented through truth tables, which map binary inputs to their corresponding outputs. 
Truth tables provide a precise representation of input-output relationships, ensuring the design of functionally equivalent circuits.
Inspired by this, researchers~\cite{deepmind2024ai4sys, wang2024tnet} have begun exploring the application of DAS to synthesize circuits directly from truth tables.
Specifically, \citet{deepmind2024ai4sys} proposed CircuitNN, a framework that learns differentiable connection structures with logic gates, enabling the automatic generation of logic circuits from truth tables.
This approach significantly reduces the complexity of traditional circuit generation. 
Building on this, \citet{wang2024tnet} introduced T-Net, a triangle-shaped variant of CircuitNN, incorporating regularization techniques to enhance the efficiency of DAS.

Despite these advancements, several challenges remain. 
The computational complexity of DAS grows quadratically with the number of gates, posing scalability issues.
Although triangle-shaped architecture~\cite{wang2024tnet} partially mitigates this problem, redundancy persists. 
%Additionally, DAS is susceptible to converging to local optima, limiting the ability to search architectures that satisfy the given truth tables~\cite{liu2018darts}. 
%Furthermore, hyperparameters (network depth and layer width) require extensive searches, introducing complexity and prolonging the synthesis process. 
Additionally, DAS is susceptible to converging to local optima~\cite{liu2018darts} and hyperparameters (network depth and layer width) require extensive searches. 
The challenges arise from the vast search space in DAS. 
% Even with predefined settings for CircuitNN, finding a configuration that meets the truth table requires extensive trial and error during the DAS process. 
Intuitively, limiting the search space through predefined parameters (network depth, gates per layer, and connection probabilities) can significantly reduce the complexity.

Recent advances~\cite{openai2023gpt4, abramson2024alphafold3, esser2024sd3, li2024mar} in conditional generative models have demonstrated remarkable performance across language, vision, and graph generation tasks. 
Motivated by these developments, we propose a novel approach to circuit generation that generates preliminary circuit structures to guide DAS in generating refined circuits matching specified truth tables. 
Firstly, we introduce CircuitVQ, a tokenizer with a discrete codebook for circuit tokenization. 
Built upon our Circuit AutoEncoder framework~\cite{hou2022graphmae,li2023maskgae,wu2025mgvga}, CircuitVQ is trained through a circuit reconstruction task. 
Specifically, the CircuitVQ encoder encodes input circuits into discrete tokens using a learnable codebook, while the decoder reconstructs the circuit adjacency matrix based on these tokens.
Subsequently, the CircuitVQ encoder serves as a circuit tokenizer for CircuitAR pretraining, which employs a masked autoregressive modeling paradigm~\cite{chang2022maskgit, li2023mage}. 
In this process, the discrete codes function as supervision signals. 
After training, CircuitAR can generate discrete tokens progressively, which can be decoded into initial circuit structures by the decoder of the CircuitVQ. 
These prior insights can guide DAS in producing refined circuits that match the target truth tables precisely.

Our key contributions can be summarized as follows:
\begin{itemize}
\item We introduce CircuitVQ, a circuit tokenizer that facilitates graph autoregressive modeling for circuit generation, based on our Circuit AutoEncoder framework;
\item Develop CircuitAR, a model trained using masked autoregressive modeling, which generates initial circuit structures conditioned on given truth tables;
\item Propose a refinement framework that integrates differentiable architecture search to produce functionally equivalent circuits guided by target truth tables;
\item Comprehensive experiments demonstrating the scalability and capability emergence of our CircuitAR and the superior performance of the proposed circuit generation approach.
\end{itemize}

% Motivation
% (a) Diffusion (Vision, Graph), Autoregressive (Language, Vision)
% (b) Circuit Generation for Predefined Setting
% (c) Neural Architecture Search for Strict Logic Equivalence

% Contribution
% (a) Circuit Tokenizer (new transformer arch, training strategy)
% (b) CircuitAR (train and gen strategies, post-ar strategy)
% (c) Extensive Evaluation including BitD (Bit Distance) for Scalability

Query-focused summaries (QFS) give an overview of documents to answer a query~\cite{rosner2008multisum, el2021automatic}.
By combining each document's content useful for answering the query, or their \textbf{perspectives}~\cite{lin2006side}, these summaries can aid decision-making~\cite{hsu2021decision}.
For example, doctors pick treatments based on research paper perspectives~\cite{goff2008patients} and legislators vote based on perspectives in policy reports~\cite{jones1994reconceiving}. 
Past QFS work assumes documents have aligned perspectives~\cite{roy2023review}, but some queries, like ``\emph{Is law school worth it?}'', are debatable, containing opposing perspectives~\cite{wan2024evidence}.
In such cases, it is key to \textit{balance} perspectives from \textit{diverse} sources so users consider all sides before deciding~\cite{dale2015heuristics}.

To address this gap, we propose \textbf{\textit{debatable} QFS (DQFS}).
As input, DQFS uses documents and a debatable query, defined as a yes/no query where documents have opposing, equally-valid\footnote{This is meant to avoid input questions like ``Is the earth flat?'' where ``yes'' and ``no'' are not equally-valid (\cref{subsection:ethics}).} ``yes'' and ``no'' perspectives (Fig~\ref{fig:intro}).
Such queries are broad (\textit{Is law school worth it?}), and decomposing broad concepts into more specific topics (\textit{cost}, \textit{job market}) improves comprehension~\cite{johnson1983mental}.
Thus, DQFS creates a multi-aspect summary, with each paragraph covering one of an input number of topics ($2$ in Fig~\ref{fig:intro}).
The full summary and each paragraph must be \textit{comprehensive} and \textit{balanced}~(\cref{section:task}).
Comprehensive text has perspectives from all documents, while balanced text is not skewed towards the yes or no perspectives; our goals aid informed, unbiased decision-making~\cite{ziems2024measuring}.


While LLMs are deft summarizers~\cite{zhang2024benchmarking}, they cannot directly solve DQFS, as they fail to use diverse sources~\cite{huang-etal-2024-embrace}.
In Figure~\ref{fig:intro}, GPT-4 mainly gives perspectives favoring EU expansion (\textcolor{blue}{\textbf{blue}}), yielding a biased output.
Also, when asked for citations~\cite{huang-chang-2024-citation}, GPT-4 only cites 3/6 (\textcolor{yellowcite}{\textbf{yellow}}), missing half the documents' perspectives.
We intuit this arises since GPT-4 uses one inference step, with all documents in a single prompt.
This can omit document perspectives in certain positions of the prompt~\cite{liu2024lost} or that oppose parametric memory~\cite{jin2024tug}, reducing output coverage and balance.

Multi-LLM summarizers~\cite{chang2024booookscore, adams2023sparse}, which use LLMs to summarize documents individually into intermediate outputs before merging them with another LLM call, are better choices, as they represent documents more equally. 
However, they have two key issues.
\textbf{First}, they use the same topic or query as input to summarize each document, which is subpar if we wish to use retrieval in summarization to reduce LLM costs.
Queries unaligned to a document's unique content and expertise will fail to retrieve all of its most relevant contexts~\cite{sachan2022improving}; this reduces the total number of perspectives in the intermediate output, resulting in lower coverage.
\textbf{Second}, their intermediate outputs are unstructured, free-form texts, which are hard for the LLM to combine into a final output.
Free-form text needs extra reasoning to extract, classify, and compare the texts' perspectives~\cite{barrow2021syntopical}, steps that distract from the final goal of generating a balanced summary.

% A \textit{structured} intermediate output that clearly organizes documents and their perspectives on topics would greatly simplify the final step of synthesizing a balanced, comprehensive summary~\cite{shao2024assisting}.

To solve our issues, we build \textbf{\model} (Fig~\ref{fig:model}), a multi-LLM system using a \textbf{M}ixture \textbf{o}f \textbf{D}ocument \textbf{S}peakers.
Inspired by panel discussions~\cite{doumont2014english}, \model has a \textit{Speaker} LLM for each document that responds to queries using its document, and a \textit{Moderator} LLM that decides when and how speakers respond.
Specifically, \model: 1) plans an agenda of topics for the outline (\cref{subsection:agenda}); 2) picks a subset of speakers with relevant perspectives for each topic and tailors them a query (\cref{subsection:moderator}); and 3) asks each speaker to obtain its document's context relevant to the tailored query and give the context's ``yes'' and ``no'' perspectives for the topic. 
%All steps are efficiently done via~retrieval.

When a speaker supplies its document's perspectives, the topic, document number, tailored query, and perspectives update an outline, tracking the LLM discourse.
After the discussion, the outline is summarized for a DQFS output.
In all, \model frames DQFS as a discussion of document speakers to represent sources equally, tailors queries for speakers to optimize the retrieval of contexts used to find perspectives, and builds a structured outline of document perspectives to simplify the synthesis of a final output---a novel combination that leads to comprehensive and balanced summaries~(\cref{subsection:ablation}).

We compare \model to eight strong baselines~on ConflictingQA~\cite{wan2024evidence} and \textbf{DebateQFS} (\cref{subsection:datasets}), a new dataset for DQFS drawn from the debate community on Debatepedia~\cite{gottopati2013learning}.
To assess summaries, we have models give citations in their outputs (Fig~\ref{fig:intro}), showing the documents the model intends to use~\cite{huang-chang-2024-citation}.
Many works use citations for factuality~\cite{li2024citation}, but
we repurpose them for coverage and balance---measuring the proportion of documents cited and distribution of ground-truth yes/no perspective stances of cited documents (\cref{subsection:metrics}).


\model has the best document coverage and balance in full summaries and topic paragraphs (\cref{subsection:citation_comp}), surpassing SOTA by 38-58\% in paragraphs.
The Prometheus LLM~\cite{kim2024prometheus} ranks \model as one of the best models in summarization quality 28/30 times, the most of any model (\cref{subsection:summary_comp}).
Users also find \model's outputs to be the most balanced, and preserve readability despite using perspectives from more documents (\cref{subsection:human_eval}).
Lastly, analyses show the utility of tailoring queries and building outlines, which improve \model (\cref{subsection:ablation}) and offer rich, structured tools for users (\cref{subsection:qg}). Our contributions are:

\noindent \textbf{1)} We propose \textbf{debatable query-focused summarization}, a new task to help users navigate yes/no queries in documents with opposing perspectives. \\
\noindent \textbf{2)} We design \model, a multi-LLM DQFS system that treats documents as \textbf{individual} \textbf{LLM speakers}, uses a moderator to \textbf{tailor queries} to apt speakers, and tracks speaker perspectives in an \textbf{outline}. \\
\noindent \textbf{3)} We release \textbf{DebateQFS} for DQFS and \textbf{citation metrics} to capture summary coverage and~balance. \\
\noindent \textbf{4)} Experiments show \model \textbf{beats baselines by 38-58\%} in topic paragraph coverage and balance, while annotators find \model's summaries \textbf{maintain readability} and \textbf{better balance perspectives}.


% \jbgcomment{Worth a footnote to cover ``all of the above'' as hinted in the title?}

\section{Background: A Brief History of \mcqa} \label{section:background}

A multiple-choice question (MCQ) is a question~$q$ and set of
choices $\mathcal{C}$.\footnote{Some choices can link to many other choices (e.g. ``All of the above''), but these are discouraged \cite{haladyna2002review}.}
%
One choice $a \in \mathcal{C}$ is the~gold answer, while others
are plausible-sounding but~incorrect distractors $\mathcal{D} = \mathcal{C}\setminus\{a\}$ meant to test misunderstandings.\footnote{Extractive QA (e.g. SQuAD) and classification (e.g. NLI) can have a finite set of choices, but they are fixed (i.e. labels) or have non-misleading distractors, so these tasks are not~\mcqa.}
%
\mcqa's simple goal---picking the best answer $a$---is popular for LLM evaluation, but it has flaws.
%
Before naming them, we first review its history in
human testing (\cref{subsection:eval_humans}) and
\abr{nlp}~(\cref{subsection:eval_machines}).

%\subsection{Evaluating \textit{Humans} with \mcqa}
\subsection{Why \mcqa{} is the Standard for Humans}
\label{subsection:eval_humans}

The \mcqa format originated in 1914 with Frederick Kelley's Kansas
Silent Reading Test \cite{kelly1916kansas}, proposed as an efficient
measure of student reading comprehension \cite{monroe1917report}.
%
Soon after, \mcqa was attempted at scale, notably with Robert Yerkes’
Army Alpha and Beta tests~\cite{yerkes1918psychology} in 1917 to assess
U.S. Army intelligence.
%
An initial bottleneck in \mcqa was the manual effort required for
scoring, which researchers like Benjamin Wood and Reynold Johnson tackled by designing automated grading systems
\cite{brennan1971ibm, woodcolumbia} with \abr{ibm}.

Automatic scoring eventually enabled \mcqa's popularity in
primary/secondary education~\cite{butler2018multiple}, college
admissions \cite{daneman2001using}, language proficiency
\cite{jamieson2000toefl}, and even everyday tasks like compliance
training \cite{puhakainen2010improving} or driver's permit exams
\cite{beanland2013there}.
%
% \jbgcomment{We should be careful here, as
%   the first College Board tests were essays, and they went to full
%   \mcqa{} later.}
%
Parallel to this, education researchers began exploring the best practices for writing high-quality MCQs \cite{morrison2001writing, campbell2011write}, crafting difficult distractors \cite{pho2015distractor, gierl2017developing}, and designing test settings \cite{rakes2008open, shute2013comparison}.

Despite \mcqa's simplicity and popularity, organizations still
critically assess its use in standardized testing.  In the United
States, the SAT removes unsound MCQ
types,\footnote{https://blog.prepscholar.com/sat-analogies-and-comparisons-why-removed-what-replaced-them} and France's
Baccalauréat uses long essay tasks over
\mcqa.\footnote{https://www.education.gouv.fr/reussir-au-lycee/le-baccalaureat-general-10457} We argue
LLM evaluation needs similar scrutiny and should draw from education to
refine MCQA's format and data.


% While once popular, \mcqa's dominance in standardized testing is wavering.
% University of California no longer considers SAT/ACT scores for admissions,\footnote{https://admission.universityofcalifornia.edu/} and many graduate programs have dropped the GRE,\footnote{https://grenotrequired.com/} citing limited predictive validity.
% As institutions reconsider \mcqa evaluation, we argue similar scrutiny is needed for LLM evaluation.

\subsection{How \mcqa Became Popular for LLMs} \label{subsection:eval_machines}

\abr{nlp} first used MCQs from human exams; solving these with models that used external sources was considered part of an ``\abr{ai} grand challenge'' \cite{reddy1988foundations}, as it required semantic \cite{turney2003combining, veale2004wordnet} and factual understanding \cite{6587172, 10.1145/2509558.2509565}.
Other early MCQs such as COPA~\cite{roemmele2011choice} and Winograd \cite{levesque2012winograd} tested commonsense reasoning over events and ambiguity in premises.
Soon after, \citet{richardson2013mctest} designed MCTest for machine reading comprehension (MRC) via fiction text and MCQs.
All tasks challenged models, but most \mcqa work studied MRC \cite{lai-etal-2017-race}.

With the advent of larger, neural LMs \cite{devlin2018bert}, \mcqa needed to become harder.
Researchers expanded MRC to test numerical reasoning \cite{dua2019drop} and uncertainty \cite{rogers2020getting}, and successfully scaled existing commonsense MCQs \cite{sakaguchi2021winogrande}.
New MCQs that tested LM pre-training knowledge also grew popular, often using commonsense in daily~tasks \cite{talmor2018commonsenseqa, bisk2020piqa} or science exams \cite{mihaylov2018can, clark2018think}.

As LLMs improved in generation \cite{brown2020language}, \mcqa evaluation changed;
models usually scored \mcqa choices independently, but~\citet{robinson2023leveraging} showed prompting LLMs with the question and \textit{all} choices was easy to score and matched human testing.
It soon became standard to test LLMs with MCQs; companies used~the task to parade their models \cite{achiam2023gpt}, some equating it to intelligence \cite{anthropic2024introducing}.
This industry adoption incentivized researchers to write more MCQs across topics \cite{rein2023gpqa}, languages, and modalities \cite{zhang2023m3exam}.

% with MCQs being used to test LLMs in long-context~MRC \cite{Pang2021QuALITYQA} and understanding of various subjects \cite{hendrycks2020measuring}, languages \cite{hardalov2020exams}, and modalities \cite{yang2022avqa}.

Recent work critiques LLM evaluations in general, discussing reproducibility issues \cite{laskar-etal-2024-systematic}, how it should be a distinct discipline \cite{chang2024survey}, and its failure to predict deployment settings \cite{saxon2024benchmarks}.
%
We similarly argue that while \mcqa is simple and popular, the task has flaws in its format and datasets, many of which can be fixed using insights from education research.

% \jbgcomment{Last sentence could tie the section together better}


\section{\mcqa is Flawed as a Standard Format} \label{section:format}

\mcqa's format is simple for student assessment, but educators find
tradeoffs: \mcqa may not fully capture learning~\cite{simkin2005multiple} or student success~\cite{moneta2017limitations}.
%
This section uses these insights to argue \mcqa should not be the standard format to evaluate LLMs, showing its rigid goal
(\cref{subsection:best_answer}), failure to capture LLM use
cases (\cref{subsection:use_cases}), and limited knowledge testing~(\cref{subsection:testing_what}).

\begin{figure}
    \centering
    \includegraphics[width=\linewidth]{figures/MCQA_subjective.pdf}
    \vspace{-4.7ex}
    \setlength{\fboxsep}{0pt}
    \caption{\small Commonsense \mcq from \citet{Palta2024PlausiblyPQ} where \colorbox{bluecolor}{\strut the choice rated most plausible by users} is not the same as the \colorbox{redcolor}{\strut gold answer}; both are subjectively correct in varied contexts.}
    \label{fig:subjective}
    \vspace{-1.7ex}
\end{figure}

\subsection{``\underline{Pick} the \textit{Best} Answer'' is Too Rigid} \label{subsection:best_answer}

One of \mcqa's key issues is its rigid goal: pick the best answer from a set of choices.
While easy to score, both designs---the use of
1) one gold answer; and 2) input choices---limit \mcqa's applicability.

First, one gold answer hinders \mcqa's use for subjectivity
\cite{finetti1965methods}.
Still, we use MCQs for commonsense \cite{bisk2020piqa}, morals \cite{yu-etal-2024-cmoraleval}, and culture (\cref{subsection:bias}), where many~choices~can be subjectively right (Fig~\ref{fig:subjective}).
\citet{Palta2024PlausiblyPQ} find users rate distractors in commonsense MCQs as~the most plausible choice >20\% of times.
Thus, extra care is needed to write MCQs for subjective tasks.

Second, picking from choices means MCQs test \textit{validation}, useful in tasks
like LLM-as-a-judge or re-ranking which must compare answers
\cite{gu2024survey}, but inhibiting tasks like writing and coding that
require \textit{generation} \cite{celikyilmaz2020evaluation}.
%
One may argue \mcqa proxies generation (if you pick good answers, you generate good ones), but LLMs~lack validation/generation consistency
\cite{li2024benchmarking, west2023generative}.
%
Validation/generation are separate skills, so \mcqa is poor for
testing generation.

In all, \mcqa best tests LLMs in objective validation, struggling with subjectivity and generation.

\subsection{Users Rarely Ask LLMs to Solve MCQs} \label{subsection:use_cases}

Many leaderboards aim to rank LLMs by their overall abilities, helping users select the best model for their needs \cite{xia2024top}.
Hence, they should adopt tasks that mirror the popularity of user needs.

\mcqa is over-represented versus how LLMs are used; 32\% of the tasks in HELM
\cite{perlitz2023efficient}, 71\% in GPT-4's card \cite{achiam2023gpt}, and 79\% in OpenLLM (Big Bench has 21 \mcqa tasks) \cite{open-llm-leaderboard-v2} are \mcqa.
%
In contrast, \citet{ouyang2023shifted} find in ShareGPT's dataset of ChatGPT queries that nearly all user queries
ask for free-form text; we estimate just 7.2\% are validation (4.3\% evaluation and 2.9\% comparison).
%
Similarly, WildChat notes just 6.3\% of their LLM queries are factual QA \cite{zhao2024wildchat}.\footnote{It is not explicitly stated if these are even validation tasks, so this is another \textit{upper} bound of validation task prevalence.}
Thus, >90\% of queries are likely generative tasks (code, writing, or explanations), which MCQs struggle to test~(\cref{subsection:best_answer}).

% % \jbgcomment{I haven't checked this,
%   but if I remember it correctly, it is predictive, just less
%   predictive than other features.  This is still useful (if I'm
%   remembering correctly) for the story: grad admissions got rid of
%   multiple choice but kept the generative parts.}

Informative evaluation suites must reflect LLM use cases.
%
This is precisely why \mcqa exams are waning in graduate admissions
criteria: they cannot fully predict graduate school
success~\cite{sampson2001gre}. 
%
Similarly, over-representing \mcqa in evaluations obscures which LLMs best aid users.

\subsection{\mcqa Does Not Fully Test Knowledge} \label{subsection:testing_what}

While \mcqa fails to match user needs, we assume the format may test basic skills for such needs, justifying its usage.
%
\mcqa is meant to test knowledge \cite{moss2001multiple}, and with input texts, comprehension \cite{farr1990description}.
%
However, research in education reveals \mcqa may be suboptimal for these goals.

 MCQs mainly assess the basic knowledge levels in Bloom's Taxonomy of educational goals \cite{krathwohl2002revision}: recalling, understanding, and applying knowledge \cite{simkin2005multiple, shin-etal-2024-generation};
it is hard to write MCQs for the higher levels requiring reasoning \cite{stupans2006multiple, palmer2007assessment, lin2012can}: analyzing, evaluating, and creating knowledge.
As evidence, students can solve MCQs without full understanding, exposed in free-response answers \cite{mckenna2019multiple}.
%
MCQs with passages generally test comprehension, but some doubt this;
\citet{ozuru2013comparing} find \mcqa scores correlate with prior knowledge of the passage, overestimating true comprehension.

% In all, \mcqa may reliably test students' comprehension, but not full knowledge assessments.
We believe these same insights can apply to \abr{nlp}: \mcqa may be apt for
comprehension, but rewards LLMs for basic recall versus in-depth
knowledge.


\begin{figure}
    \centering
    \includegraphics[width=\linewidth]{figures/MCQA_formats.pdf}
    \vspace{-4.6ex}
    \setlength{\fboxsep}{0pt}
    \caption{\small Example of adapting \colorbox{redcolor}{\strut typical MCQs} to our generative formats: \colorbox{bluecolor}{\strut Constructed Response} and \colorbox{purplecolor}{\strut Justified \mcqa}.}
    \label{fig:format}
    \vspace{-1.7ex}
\end{figure}

% Normal texts are the task inputs, while highlighted texts are GPT-4o's responses.

\section{\underline{Generative} \mcqa Tasks are Promising} \label{section:fixing_format}

\mcqa is flawed, so how should its role change?
Standardized LLM evaluations must 1) proxy LLM use cases; and 2) test skills for (1).
\mcqa is unfit for (1), so~evaluations need more tasks matching LLM needs (\cref{subsection:use_cases}).
Writing/explanation tasks are hard to score \cite{charney1984validity}, but it is still odd they are often omitted, as they are typical needs~(\cref{subsection:use_cases}).

This limits \mcqa to (2), but it is best for comprehension or validating objective facts (\cref{subsection:best_answer}), not generation, subjectivity, or knowledge (\cref{subsection:testing_what}).
Validation/comprehension are valuable, which is why we discuss \mcqa datasets in \cref{section:dataset}, but we need~better formats for the other skills.
Thus, we give two generative versions of \mcqa to better test LLMs (Figure~\ref{fig:format}):
Constructed Response (\cref{subsection:constructed_response}), answering sans choices, and Explanation \mcqa (\cref{subsection:justified_mcqa}), justifying predictions.
We ensure the formats only slightly increase evaluation complexity, mostly preserving \mcqa's simplicity of scoring. 
We now~describe the formats and future work to realize them.

\subsection{Constructed Response Questions} \label{subsection:constructed_response}

Having LLMs solve MCQs without choices, called \textbf{Constructed Response (CR)} in education \cite{livingston2009constructed} or short-form QA in \abr{nlp} \cite{krishna-etal-2021-hurdles}, is one better format.
CRQs test answer \textit{generation} unlike MCQs (\cref{subsection:best_answer}), better mirroring LLM needs (\cref{subsection:use_cases}), and it is easier to write CRQs testing all skills in Bloom's Taxonomy \cite{krathwohl2002revision}, so they better expose knowledge gaps (\cref{subsection:testing_what}).
Thus, students find CRQs harder than MCQs~\cite{hancock1994cognitive}, which can also delay saturation (\cref{subsection:saturation}).

Instead of writing CRQs to replace our vast existing \mcqa data, a promising solution is to convert MCQs into CRQs by omitting choices and tasking LLMs to give a \textbf{short-form} answer $\hat{a}$ for question $q$, comparing it to gold answer $a \in \mathcal{C}$ \cite{bhakthavatsalam2021think}.
\citet{myrzakhan2024open}~show two hurdles in this: finding MCQs to convert\footnote{\;``Which of these best...'' MCQs require using all choices.} and scoring $\hat{a}$ with $a$. 
Recent efforts in flagging MCQ errors (\cref{subsection:quality}) and judging short-form answer correctness \cite{li-etal-2024-pedants} show we can realistically overcome these challenges.
Thus, we believe merging this work with best practices for creating~CRQs \cite{snow2012construct} can successfully implement the task.

% Reliable MCQ to CRQ conversion would thus require combining advancements in MCQ generation \cite{lee2024math} and automatic scoring \cite{li-etal-2024-pedants} with best practices for writing and evaluating CRQs~\cite{snow2012construct}.

\subsection{Explanation Multiple Choice Questions} \label{subsection:justified_mcqa}

Constructed Response is promising (\cref{subsection:constructed_response}), but using one short answer is unfit for subjectivity \cite{lin2020differentiable} and conflicts user preferences for long outputs \cite{zheng2024judging}.
We thus propose~\textbf{Explanation \mcqa (\emcqa)} as another \mcqa alternative from education \cite{lau2011guessing}: for a question $q$ and choices $\mathcal{C}$, models give an answer $\hat{a} \in \mathcal{C}$ and explanation $\mathcal{E}$ for why $\hat{a}$ is right.
This format tests generation (\cref{subsection:best_answer}), matches the use case of explanations (\cref{subsection:use_cases}), and has shown to test more knowledge levels over \mcqa \cite{lee2011validating}.

%This format is also supported by education research, as asking for justifications can expose student knowledge gaps and heighten task difficulty \cite{tamir1990justifying, koretsky2016written}.

We envision \emcqa being treated like reasoning tasks \cite{cobbe2021training}, checking if LLMs pick $a$ like \mcqa's simple scoring, but also studying $\mathcal{E}$.
If models select $a$ but justify it poorly, it exposes knowledge gaps like in student assessments \cite{jonassen2010arguing}, and when models give strong explanations for wrong answers, it enables partial credit for subjective tasks \cite{lau2011guessing}.

\emcqa has many benefits, but needs metrics to score ``good'' explanations over many facets \cite{xu2023critical}, like factuality to curb hallucinations \cite{min2023factscore}, plausibility for convincingness \cite{liu2023vera}, and faithfulness to verify $\mathcal{E}$ supports $a$ \cite{lanham2023measuring}.
We believe these goals could be achieved by combining ongoing efforts in building verifiers for LLM reasoning \cite{NEURIPS2023_72393bd4} with educational best practices for grading justifications \cite{jonassen2010arguing}, yielding reliable metrics that realize \emcqa's potential.

% \subsection{So... How Should I Evaluate My LLM?} \label{subsection:choosing_task_formats}

% As we have flagged many issues with \mcqa, \textit{how should we evaluate our LLMs?}
% For specific use cases, the ideal approach is to curate tasks that closely mirror deployment settings \cite{saxon2024benchmarks}.
% However, we often need general-purpose evaluations, tradeoffs arise between alignment with real-world use cases and evaluation cost and scalability. Human or LLM-based ratings align closely with downstream tasks but are costly and subjective. In contrast, \mcqa offers efficient and objective scoring but sacrifices generative alignment and real-world applicability. Our proposed alternatives, Constructed Response Questions and Justified \mcqa, offer a middle ground: they retain \mcqa’s simplicity and scalability while better reflecting LLMs’ generative capabilities. These formats could serve as practical, scalable solutions for evaluating general model capabilities and bridging gaps in existing benchmarks.

% We also don't mean to use our proposed formats to eliminate \mcqa.
% The ability of models to discriminate and pick high-quality answers via \mcqa formats is still valid and worth testing, especially as LLMs become more practically used for things like LLM-as-a-judge and (something else).
% This is why, in the next section, we discuss ways to improve current \mcqa datasets, which can also improve our alternative task formats that rely on \mcqa data. Everyone wins





% Testing how LLM calibration in MCQA impacts human confidence \cite{steyvers2024calibration}

% Conformal prediction to assess confidence \cite{kumar2023conformal}

% Answer-level calibration improves MCQA reliability and finds some data biases \cite{kumar2022answer}

% Uncertainty datasets/benchmark \cite{raina-gales-2022-answer, wang2024ubench}

% Prompt engineering can improve calibration \cite{ma2023prompt}


\section{\mcqa Datasets are Flawed but Fixable} \label{section:dataset}

\mcqa is not always the best format (\cref{section:format}), but we still need high-quality MCQs for comprehension/validation, along with tasks like LLM-as-a-judge \cite{gu2024olmes} and re-ranking \cite{ma-etal-2023-large} which require comparing answers.
Also, our generative \mcqa formats (\cref{section:fixing_format}) still use MCQs as inputs.

However, like most \abr{nlp} tasks, \mcqa datasets have quality issues that impede their utility: leakage
(\cref{subsection:test_set_leakage}), unanswerability
(\cref{subsection:quality}), shortcuts (\cref{subsection:artifacts}),
and saturation (\cref{subsection:saturation}).
We show how educators'~solutions to these issues can guide \abr{nlp} dataset design.




% \jbgcomment{I think it's better for filenames to be named so that they sort correctly}

\subsection{LLMs Peek at \mcqa Answer Keys} \label{subsection:test_set_leakage}

To build an \mcqa dataset, we first need sources to write or collect
MCQs.
%
But as many sources end up being leaked\footnote{gpt-3 has seen
45\% of RACE's test set \cite{sainz2023nlp}.} in LLM training data
\cite{magar2022data}, such MCQs may confuse~generalization abilities for
memorization \cite{lewis2020question}.
%
Private test sets \cite{sap2019socialiqa} and de-contamination
\cite{zhou2023don} help, but LLMs tuned on newer data can overlap with
(1), and opacity in LLM data \cite{soldaini2024dolma} blocks (2).
% \jbgcomment{Patrick Lewis's paper on NQ overlap is probably worth citing here}

An ambitious solution to test set leakage is live MCQs that update over time to stay unseen \cite{white2024livebench}, like how educators rewrite exams to limit cheating.
In fact, trivia \cite{jennings2007brainiac}~and standardized testing groups often write new questions, making them ideal partners.
To aid both~parties, researchers could offer these groups tools for tutoring \cite{li-etal-2024-eden}, MCQ validation \cite{yu-etal-2024-automating-true}, or answer scoring \cite{yang-etal-2020-enhancing}.

Test set leakage would be easier to fix if model designers released
training data, but as we all know, most do not.
%
While unfair researchers must aid this effort, we hope they view it as
a difficult, impactful research problem in evaluation and
generalization.



\subsection{Some MCQs Have No Correct Answer} \label{subsection:quality}

Once a source is found (\cref{subsection:test_set_leakage}), researchers collect~or write MCQs, but errors often arise rendering them unanswerable, like mislabeling \cite{explained2023smart}, multiple correct choices \cite{Palta2024PlausiblyPQ}, ambiguity \cite{gema2024we}, missing contexts \cite{wang2024mmlu}, and grammar errors \cite{chen2023hellaswag}.

Educators write MCQs with rigorous protocols, and we must meet similar standards in \abr{nlp} \cite{boyd2019question}; we should use educators' rubrics (Figure~\ref{fig:checklist}) for writing and validating MCQs \cite{haladyna1989taxonomy}.
Such guidelines also specifically exist for distractors \cite{haladyna2002review}---the part of MCQs that discern testees' skills---ensuring they are truly wrong, shortcut-proof (\cref{subsection:artifacts}), and not too easy to rule out (\cref{subsection:saturation}).
Beyond MCQ writing, rubrics can form~data cards \cite{pushkarna2022data} to help researchers record errors in their data and how~they~fixed them.

Recent work in LLM checklist evaluation \cite{cook2024ticking}, 
MCQ metrics \cite{moon2022evaluating},~and MCQ generation \cite{sileo2023generating} show parts~of this may be
automatable (Figure~\ref{fig:checklist}).
For instance, \citet{wang2024mmlu} fix errors in MMLU by using LLMs to detect issues and
write new choices.~LLM judges are not always reliable \cite{xu-etal-2024-pride}, so
human-\abr{ai} collaboration, like model-assisted
refinement~\cite{shankar2024validates} and task
routing~\cite{miranda2024hybrid}, may be more promising.
%
Errors will arise in MCQ writing, but educators' rubrics can help
find and fix them, ensuring answerability.

\begin{figure}
    \centering
    \includegraphics[width=\linewidth]{figures/MCQA_checklist.pdf}
    \vspace{-4.75ex}
    \setlength{\fboxsep}{0pt}
    \caption{\small Example unanswerable MCQ from MMLU \cite{gema2024we}, along with rubric criteria from \citet{haladyna1989taxonomy} flagged by OpenAI's o1 \cite{jaech2024openai}.}
    \label{fig:checklist}
    \vspace{-1.7ex}
\end{figure}

\subsection{\mcqa Shortcuts May Let LLMs ``Cheat''} \label{subsection:artifacts}

Answerable MCQs (\cref{subsection:quality}) are not always high quality, as shortcuts may let LLMs guess the answers to MCQs without knowing the answers \cite{10.1145/3596490}, overestimating model accuracy \cite{wiegreffe2021teach}.
%
Shortcuts can arise from annotator artifacts \cite{gururangan2018annotation}, spurious patterns \cite{zhou2023explore}, or bypassed reasoning steps \cite{chen2019understanding}.
%
If LLMs best random guessing via partial inputs (e.g. choices only) \cite{richardson2020does}, they may~exist.

Below, we discuss how scoring (\cref{subsubsection:shortcut_scoring})
and data collection methods (\cref{subsubsection:shortcut_dataset})
can mitigate shortcuts.

% \jbgcomment{Note for the future: if YooYeon's AdvCal paper gets in, that should be cited here (or in RW) as well}

\subsubsection{Calibrated Scoring Can Deter Guessing} \label{subsubsection:shortcut_scoring}

While rare in human testing,\footnote{Since
they may induce stress or anxiety in students~\cite{vanderoost2018elimination},
but we think it is fine to stress~LLMs.} scoring methods can
penalize wrong guesses~\cite{lau2011guessing}:
\textbf{1) Probability scoring:} elicit confidence scores for each choice \cite{finetti1965methods}; \textbf{2) Negative marking:} subtract points for wrong answers with abstention allowed \cite{holt2006analysis}; and \textbf{3) Elimination scoring:} students iteratively remove wrong choices until unsure \cite{ben1997comparative}.
Confidence \cite{li-etal-2024-think}, abstention \cite{goral2024wait}, and elimination \cite{ma2023poe} have been studied in LLMs, so they may be easy to use for \mcqa evaluation.

These methods deter guessing, but also reward calibrated models that know their knowledge gaps \cite{guo2017calibration}.
While often ignored in evaluations \cite{bommasani2023holistic},
calibration scoring could let \mcqa better test decision-making \cite{liu2024dellma} and enable partial credit for subjective tasks where multiple choices may be tenable~(\cref{subsection:testing_what}).



\subsubsection{We Can Design Shortcut-Proof MCQs} \label{subsubsection:shortcut_dataset}

Data designers should limit shortcuts; an easy way is
uniform design.
When solving MCQs with only choices,
\citet{Balepur2024ArtifactsOA} find LLMs may exploit distributional
differences, so like educators~do (\cref{subsection:quality}), we should write parts of MCQs consistently: via the same agent, source, and decoding method.
%
HellaSwag \cite{zellers2019hellaswag} leads to the highest known choices-only accuracy, where \textit{user}-written answers and \textit{model}-written distractors are inconsistent, showing the necessity of uniform data design.

Contrast sets are another tool that detect if models ignore inputs and use shortcuts \cite{Gardner2020EvaluatingML}.
%
In \mcqa, they are entry pairs differing by some inputs (e.g. question)
that change the answer (Figure~\ref{fig:contrast}), ensuring models attend to the perturbed input
\cite{elazar2023measuring}.
%
\citet{balepur2024your} use contrast sets in commonsense \mcqa to
ensure none of their LLMs rely on shortcuts in choices to rank
highly.
%
Contrast sets are often made manually \cite{Kaushik2020Learning}, so future work~can test
automatic ways to build them \cite{li-etal-2020-linguistically}.

% \jbgcomment{self cites are a little prevalent here, pare back if you can}

% \cite{srikanth-rudinger-2022-partial}

Studying how users and models \textit{``cheat''}
\cite{saxon-etal-2023-peco} in MCQs also finds shortcuts.
%
In reading comprehension, \citet{Pang2021QuALITYQA} give~users \inlinecode{ctrl+F} to detect MCQs quickly solvable without using the full text, while \citet{Malaviya2022CascadingBI} flag MCQs where users
can use simple heuristics.
%
We can also train models to cheat; adversarial
filtering trains simple models
(e.g. bag-of-words) and omits MCQs they can solve \cite{zellers2018swag}.~Extending this, we believe having strong LLMs reason to cheat---similar to safety work in alignment faking
\cite{greenblatt2024alignment}---can help find shortcuts.


\subsection{LLMs Inevitably Ace \mcqa Datasets} \label{subsection:saturation}

Even if we fix all of these issues, MCQs become too easy over
time (i.e. saturated), no longer tracking LLM progress
\cite{li2024crowdsourced}.
%
To still use ``easy'' \mcqa datasets, we need to
make them harder for models.
%
% \jbgcomment{this sentence could be more direct: ``propose'' and ``study'' are filler words in academic writing}
Below, we show how understanding which MCQs
are hard (\cref{subsection:irt}) and helping users author hard, interpretable MCQs (\cref{subsection:hitl}) delay saturation.


\begin{figure}
    \centering
    \includegraphics[width=\linewidth]{figures/MCQA_Contrast.pdf}
    \vspace{-4.7ex}
    \setlength{\fboxsep}{0pt}
    \caption{\small Example MCQ pair for a contrast set from \citet{balepur2024your}. The choices are identical, but the question swaps the answer, testing if models ignore the question.}
    \label{fig:contrast}
    \vspace{-1.7ex}
\end{figure}

\subsubsection{IRT Reveals Challenging MCQs} \label{subsection:irt}

When LLMs excel in \mcqa datasets, some MCQs remain hard; finding
them and why they are hard informs data design
\cite{sugawara2018makes, Sugawara2022WhatMR}.
One MCQ difficulty metric is success rate (\success)---the number of models
answering correctly \cite{gupta2024improving}---but \success omits
\textit{which} models succeed.
MCQs solved by just the worst or best model have equally low success rates, but the former suggests MCQ errors (\cref{subsection:quality})---as a weaker model besting all others is rare---while the latter matches our expectation. 
As \success conflates these cases, it cannot separate flawed MCQs from those discerning model~ability.

% \jbgcomment{I think the better argument that IRT is better than SR is
%   that bad questions can have a low success rate: you want to
%   eliminate questions with errors, ambiguities, etc.  And leave the
%   discriminative questions.  You already introduced why you want
%   discriminative questions, so you can just back point to that.  I'd
%   talk more about the failure modes of SR, which can highlight how the
%   opposite of those are things IRT can find.}

Item Response Theory (IRT), a tool used in education \cite{lord2008statistical}, is a stronger way to find harder MCQs.
%
While success rate treats all models equally, IRT learns the skill of all models which then estimates each MCQ's difficulty (how hard it is) and discriminability (how well it discerns between weak/strong models).
IRT can then~filter high-difficulty and high-discriminability MCQs as harder subsets \cite{polo2024tinybenchmarks}, detect saturation if all MCQs have low difficulty and discriminability \cite{vania2021comparing}, and omit erroneous MCQs with negative discriminability \cite{rodriguez-etal-2021-evaluation}.

% rodriguez-etal-2021-evaluation

While IRT-based filtering finds harder MCQs, it does not give new MCQs, limiting its long-term use.
However, we can extend IRT to multi-dimensional IRT \cite[MIRT]{reckase200618} to capture \textit{many} latent skills, offering more insights into model abilities; by interpreting these skill dimensions, we can pinpoint model issues that inform future data efforts (\cref{subsection:hitl}).
\citet{gor2024great} reveal LLM errors in abductive reasoning via a variant of MIRT---an issue confirmed by abduction research \cite{del-fishel-2023-true, DBLP:conf/iclp/NguyenGTSS23}.
MIRT could similarly find difficult \mcqa topics, distractor patterns, or reasoning types for models \cite{benedetto-etal-2021-application}.

% \cite{balepur2024reverse}

% \jbgcomment{Density of self-cites a little high here (for the group, not just you)}

Overall, IRT can find which MCQs are hard but also why, informing future data collection efforts.

% \jbgcomment{Section title doesn't cover adversarial questions that well.  Perhaps a transition of the form: while engaged authors are used to understanding what trips up humans, they may need interface / computational support to trip up models\dots

% That could transition to adversarial data.}

\begin{figure}
    \centering
    \includegraphics[width=\linewidth]{figures/MCQA_adv.pdf}
    \vspace{-4.7ex}
    \setlength{\fboxsep}{0pt}
    \caption{\small Obscure and adversarial MCQs for \textit{Spongebob Squarepants} inspired by \citet{sung2024your}. GPT-4o answers both \colorbox{redcolor}{\strut wrong} (answer in \colorbox{bluecolor}{\strut blue}). The former tests niche knowledge, but the latter is easy for those who have seen the~show.}
    \label{fig:adv}
    \vspace{-1.7ex}
\end{figure}

\subsubsection{A Good MCQ is Hard \textit{and} Interpretable} \label{subsection:hitl}

A popular way to write harder MCQs is requiring obscure knowledge
(Figure~\ref{fig:adv}, left), sourcing from experts
\cite{rein2023gpqa, phan2025hle} and global competitions
\cite{fang2024mathodyssey}.
%
These challenge models \textit{and} humans, which is useful for \abr{ai} safety work in scalable oversight \cite{bowman2022measuring}.
However, this makes them uninterpretable for non-experts diagnosing model errors, especially when studying model rationales (\cref{subsection:explanations}).
If LLMs~err on obscure MCQs, it is hard for non-experts to find if errors are from faulty reasoning, misunderstandings, or knowledge gaps \cite{anderson2025phdknowledgerequiredreasoning}.

While authors know which MCQs elude humans, writing ones that surface model errors while staying human-interpretable needs support.
This is \textbf{adversarial} data collection's goal \cite{kiela-etal-2021-dynabench}---building UIs to help authors write examples hard for models but easy for humans (Figure~\ref{fig:adv}, right).
Rather than using niche knowledge, authors must write MCQs with spurious patterns, misleading distractors (\cref{subsection:quality}), or reasoning traps that trick LLMs but not humans \cite{xu2023llm}.
As a result, these MCQs better expose robustness (\cref{subsection:robustness}) and logical reasoning (\cref{subsection:explanations}) errors, less clouded by knowledge.

Adversarial MCQs are useful, but finding users to write high-quality ones is tough.
Gamification---making the task fun---helps, shown by successful adversarial commonsense \cite{talmor2022commonsenseqa}, QA \cite{wallace2019trick}, and fact-checking \cite{eisenschlos2021fool} datasets.
\citet{wallace-etal-2022-analyzing} use gamification to build adversarial NLI data and show it has \textit{long-term} difficulty, delaying saturation. 
% \citet{sung2024your} recently use gamification with IRT (\cref{subsection:irt}) to build an adversarial QA dataset with the highest gap in human and model accuracy in the past five years.
For even more engagement, researchers can stir users to author MCQs exposing shocking failure cases, inspired by jailbreaking, where provocative outputs are naturally fun to elicit \cite{10.1145/3656156.3665432}.
%\cite{schulhoff2023ignore}.

% \jbgcomment{DADC would be better for submission (the barrage of self-cites can return for camera ready, so don't delete, jsut comment)}

Obscure and adversarial MCQs can both make \mcqa harder: the former
tests niche knowledge, while the latter better finds reasoning or
consistency failures that are uneclipsed by knowledge gaps.



\begin{figure}
    \centering
    \includegraphics[width=\linewidth]{figures/MCQA_LLM_issue.pdf}
    \vspace{-4.9ex}
    \setlength{\fboxsep}{0pt}
    \caption{\small LLMs struggle with \colorbox{yellowcolor}{\strut robustness} (e.g. inconsistent if symbols are swapped), \colorbox{redcolor}{\strut biases} (e.g. preferring certain choice symbols/positions), and \colorbox{orangecolor}{\strut explanations} (e.g. outputs that do not justify answers) in \mcqa. Correct answers are shown in \colorbox{bluecolor}{\strut blue}.}
    \label{fig:llm_issue}
    \vspace{-1.7ex}
\end{figure}

\section{Fixing \mcqa Can Help Us Fix LLMs} \label{section:models}

Fixing issues in \mcqa's format (\cref{section:format}) and datasets (\cref{section:dataset}) will not just improve evaluation quality; they can also improve our understanding of LLM weaknesses.
This section outlines three persistent
issues of LLMs in \mcqa (Figure~\ref{fig:llm_issue})---robustness~(\cref{subsection:robustness}), bias
(\cref{subsection:bias}), and explanations
(\cref{subsection:explanations})---and how our prior solutions can better address or evaluate them.


\subsection{New Prompts Lead to New \mcqa Scores} \label{subsection:robustness}
% \subsection{LLMs Lack Robustness in \mcqa} \label{subsection:robustness}

% On specific prompts, LLMs score highly in~\mcqa, but currently crumble
% when prompts change, sensitive to: choice
% symbols~\cite{alzahrani2024benchmarks}, choice
% order~\cite{pezeshkpour2023large, Zheng2023LargeLM}, and
% phrasing~\cite{holtzman-etal-2021-surface, Wiegreffe2023IncreasingPM}.
% This brittleness weakens \mcqa leaderboard reproducibility, as varied
% setups yield conflicting rankings~\cite{gu2024olmes}

On specific prompts, LLMs score highly in~\mcqa, but currently crumble
when prompts change, sensitive to: choice
symbols \cite{alzahrani2024benchmarks}, choice
ordering \cite{Zheng2023LargeLM}, and
phrasing \cite{Wiegreffe2023IncreasingPM}.
This brittleness degrades \mcqa leaderboard reproducibility, as varied setups produce conflicting rankings~\cite{gu2024olmes}


LLM robustness varies by task setup.
%
In \mcqa, early LLMs with poor instruction
following \cite{zhang2022opt} used probability-based \textit{scoring},
while instruction-tuning enabled LLMs to \textit{generate}
answers \cite{longpre2023flan}; while logically equivalent, these give varying answers \cite{lyu2024beyond}. 
Probability scoring looks to be more at fault, more sensitive to prompts \cite{wang2024look}, doubting if LLMs can aid decision-making tasks that need accurate confidence scores \cite{liu2024dellma}.
To track this progress in LLMs, researchers can use \mcqa scoring protocols that measure calibration~(\cref{subsubsection:shortcut_scoring}).

As accuracy often \textit{drops} post-perturbation \cite{zhou-etal-2024-revisiting}, these failures in generalization could indicate dataset leakage (\cref{subsection:test_set_leakage}) or over-reliance on biases (discussed in \cref{subsection:bias}).
Another proposed explanation is symbol binding error: LLMs ``know'' the right answer but cannot link it to the correct choice \cite{Wiegreffe2024AnswerAA, xue2024strengthened}.
These failures weaken \mcqa's ability to test knowledge, obscured by memorization and symbol binding.\footnote{Such evaluations have downstream use (e.g. privacy, interpretability), but do not truly measure knowledge as intended.}~Our proposed solutions---like curating live MCQs for data leakage (\cref{subsection:test_set_leakage}) and generative \mcqa formats to expose knowledge gaps without needing symbol binding (\cref{section:fixing_format})---can more reliably assess knowledge.

% \jbgcomment{If short on space, some of these could be move to appendix}

\subsection{LLMs are Biased \mcqa Test-Takers} \label{subsection:bias}

Like most \abr{nlp} tasks \cite{chu2024fairness}, LLMs have \mcqa biases, grouped into two types.
%
% \jbgcomment{can we be a little more precise than ``favor''?}
%
The first are \mcqa-specific biases---picking answers based on symbols \cite{Zheng2023LargeLM}, positions \cite{Li2024CanMQ,
wei-etal-2024-unveiling}, or phrases like ``none of the
above'' \cite{xu2024llms, wang2025llms} rather than MCQ content.
This degrades robustness
(\cref{subsection:robustness}), masking model knowledge.
%
We believe they likely stem from shortcuts (\cref{subsection:artifacts});
%
LLMs tuned on data where choices with patterns are often
correct will exhibit these biases~\cite{pacchiardi2024leaving}.
%
Our safeguards---uniform design, contrast sets,
and ``cheating'' (\cref{subsubsection:shortcut_dataset})---can
reduce these biases, and more work in shortcuts may be key
to quash them.

The second bias type stems from general training, like cultural/linguistic bias \cite{myung2024blend, li2023land};
LLMs often err on non-Western cultural MCQs \cite{Acquaye2024SusuBO, azime2024proverbeval} and non-English MCQs \cite{son2024kmmlu, li2023cmmlu}.
MCQs are popular for testing bias \cite{guo2023evaluating}, but if they use subjective commonsense \cite{seo-etal-2024-kocommongen}, they are subpar (\cref{subsection:best_answer}).
Thus, we advise writing MCQs with rubrics to limit ambiguity and correct distractors (\cref{subsection:quality}),~ensuring bias is more objectively tested.
If subjectivity persists, evaluators should consider ways to handle it like our~proposed Explanation MCQA (\cref{subsection:justified_mcqa}) or calibration scoring (\cref{subsubsection:shortcut_scoring}), aiding bias testing and for \emcqa, better matching how bias presents downstream \cite{seshadri2025does}.

\subsection{LLMs Struggle to ``Explain Their Work''} \label{subsection:explanations}

% LLMs can easily pick MCQ answers~(\cref{subsection:saturation}), but often give \textbf{unfaithful} explanations---failing to mirror their true reasoning---whether via chain-of-thought \cite{lyu2023faithful, lanham2023measuring, turpin2024language} or self-explanations \cite{agarwal2024faithfulness, kim2024can, madsen2024self}.
% However, they are convincing, surpassing crowdworkers in perceived quality \cite{mishra-etal-2024-characterizing} and misleading humans when incorrect \cite{si2023large}, likely due to LLM alignment protocols that optimize user preferences instead of accuracy \cite{wen2024language}.

Even if LLMs get the right answer to an~MCQ, they may justify their selection \textbf{unfaithfully}---failing to mirror their true reasoning---whether via chain-of-thought \cite{lyu2023faithful, turpin2024language} or self-explanations \cite{kim2024can, madsen2024self}.
However, they are convincing, besting crowdworkers in judged quality \cite{mishra-etal-2024-characterizing} and misleading users when wrong \cite{si2023large}, likely as LLMs are fine-tuned to optimize on user preferences over accuracy \cite{wen2024language}.

LLM explanation flaws are even clearer after logical consistency checks.
\citet{kawabata2023evaluating} show LLMs often inaccurately explain answers to subquestions in reading comprehension, even when answering the higher-level question correctly.
Similarly, \citet{balepur-etal-2024-easy} find LLMs struggle to reason why distractors are wrong.
These issues likely stem from broader LLM logical inconsistencies \cite{liu2024aligning, varshney2024investigating}.

Explanations are popular LLM use cases~(\cref{subsection:use_cases}), highlighting the need for improved \mcqa formats such as Explanation MCQA (\cref{subsection:justified_mcqa}) which explicitly test explanation skills.
Further, LLMs' logically inconsistent explanations offer a path toward harder datasets (\cref{subsection:saturation}); tools like MIRT (\cref{subsection:irt}) can identify logical error types (negation, decomposition) that elude LLMs, while adversarial collection can curate MCQs to excise these errors while staying easy for humans (\cref{subsection:hitl}).
In all, LLMs' poor explanation abilities give an opportunity to design harder \mcqa evaluations better aligned with user needs.




% \subsection{Shortcuts}

% Word level shortcuts in MCQA \cite{pacchiardi2024leaving}

% BERT can do MRC without the passage, random keywords distract BERT, perturbations change accuracy \cite{si2019does}

% More MRC shortcuts \cite{Yu2020CounterfactualVC}

% Quantifying the learnability of shortcuts \cite{Shinoda2022WhichSS}

% MRC doesnt need the input passage \cite{raina2023analyzing}

% MRC doesnt need the input passage, neither do humans \cite{liusie2022world} 

% Easy to cheat on MCQA versions of multi-hop datasets \cite{chen2019understanding}


% \subsection{Future Directions}

% Interpretability via contrastive edits \cite{ross2020explaining}

% Interp via circuits \cite{lieberum2023does}

% Interpreting biases \cite{shen2022understanding}

% AI Safety --- some MCQA tasks are hard for humans, we can study models in relation to helping humans \cite{parrish2022single, parrish2022two, wen2024language}


% \section{Related Work}

% Survey on MRC \cite{foolad2024recent}

% Survey on LLM cognitive biases \cite{sumita2024cognitive} (like order bias)

\section{Call to Action: Benchmarking 101} \label{section:conclusion}

% \jbgcomment{I think this ``call to action'' could be a little stronger and more specific.}

If you want to make the best benchmark ever, where do you begin?
First, define the ability you want to test and decide if \mcqa is the right format (\cref{section:format}).
If the ability matches a downstream task (e.g. coding), just use that task \cite{saxon2024benchmarks}.
If the ability is fundamental (e.g. knowledge), consult education research to weigh any alternative formats (e.g. \cref{section:fixing_format}).

If \mcqa is the best format, find a data source~to curb leakage---one with fresh content (\cref{subsection:test_set_leakage}).
When curating MCQs from your source,~follow educators' rubrics to ensure answerability (\cref{subsection:quality}), and release the rubric as a data card to record errors \cite{pushkarna2022data}.
Consistent design choices will limit shortcuts (\cref{subsection:artifacts}), and providing a contrast set could help researchers check if their models over-rely on shortcuts \cite{Gardner2020EvaluatingML}.
As another safeguard, your benchmark can use calibration scoring beyond accuracy to discourage guessing~(\cref{subsubsection:shortcut_scoring}).

Post-release, models will hill-climb and saturate your data over time (\cref{subsection:saturation}).
If you want to delay~saturation, you may restart with an obscure knowledge source, but if you want your data to better diagnose errors, aim for interpretability.
Use IRT (\cref{subsection:irt}) to find which of your MCQs are hard and why, then design an engaging, adversarial dataset collection protocol (\cref{subsection:hitl}) guided by these insights, yielding a new dataset hard for models but easy for humans.

By using even some of educators' insights, we can refine the utility of \mcqa---or any task.
This approach takes more effort than the simple \mcqa practices that initially attracted researchers, but if we do not address the flaws of \mcqa, \textbf{what~model abilities can our \mcqa benchmarks even test?}


\section{Limitations} \label{section:limitations}

Inspired by \citet{saxon2024benchmarks}, we organize our limitations section as potential counterarguments:



\paragraph{I Don't Work on \mcqa:}

Our approach to \mcqa can apply to all tasks;
it is important to question if your format effectively evaluates your intended ability.
Educators have long-studied the best formats for different tasks, but we are not using these insights to guide our benchmark design.
As an example beyond \mcqa, in human math assessments, students are often required to ``show their work'' to verify understanding and diagnose misconceptions \cite{choy2016snapshots}. However, math datasets like GSM-8k \cite{cobbe2021training} often ignore intermediate computations in their metrics.
Similarly, our dataset quality issues are universal; it is always important to ensure datasets are not contaminated or saturated, as well as free from errors and shortcuts.
Thus, we advise all researchers---regardless of task or domain---to consult education work to see if they can improve their evaluations' efficacy.

\paragraph{Other modalities, languages, etc. are different:}
Our critiques of \mcqa's rigid goals (\cref{subsection:best_answer}), misalignment with user needs (\cref{subsection:use_cases}), and failure to fully test knowledge (\cref{subsection:testing_what}), along with our proposed generative task alternatives (\cref{section:fixing_format}), are applicable regardless of modality and language.
Further, the \mcqa dataset quality concerns (\cref{section:dataset}) we discuss are still relevant to these domains;
for example, in some multi-modal QA datasets, models can answer questions without using the input image \cite{goyal2017making}, indicating shortcuts exist (\cref{subsection:artifacts}).

\paragraph{Why not abandon \mcqa?}

Indeed, other formats have grown in popularity, such as prompting LLMs on typical user queries and using annotators or models to judge model responses \cite{chiang2024chatbot, lin2024wildbench}.
These efforts are exciting and can directly proxy LLM use cases, but are difficult to scale for every domain we currently use \mcqa for, their subjective scoring lacks reproducibility, and these metrics are easy to game \cite{zheng2025cheating}.
In contrast, \mcqa and our proposed generative formats include scoring using ``pick the best answer'', forming a more efficient and objective metric.
Thus, we should aim to advance both of these threads for more reliable evaluations.

\paragraph{This is Way Too Much Work:}

We have proposed many directions for future research, but our objective is not to have every \mcqa dataset designer engage with each of these efforts.
We hope that by pointing out these issues in \mcqa, dataset designers will start to consider how using \mcqa will affect their datasets' reliability in the long term, and researchers will further study ways to improve \mcqa evaluation.
Even adopting just one of our proposals could greatly enhance the quality and effectiveness of \mcqa datasets.
Over time, these small, incremental improvements across the evaluation community will drive meaningful progress.

\section{Ethical Considerations}

Flawed evaluations can mislead both researchers and users; researchers may misinterpret model abilities due to quality issues in datasets, while users may struggle to identify the best models for their needs.
This paper outlines several potential solutions to mitigate these risks in \mcqa, ensuring more reliable evaluations for researchers and users.

%Overall, our findings provide broader insights to improve general evaluation practices beyond text-only \mcqa.

\section*{Acknowledgments}

We would like to thank the \abr{clip} lab at the University of Maryland and our external collaborators for their feedback.
In particular, we appreciate Dang Nguyen, Paiheng Xu, and Shi Feng for general discussions and feedback; Yu Hou, Dayeon Ki, and Connor Baumler for discussions of biases; Maharshi Gor for feedback on IRT; Yoo Yeon Sung for feedback on IRT and adversarial dataset collection; Matthew Shu for feedback on best practices in education; Atrey Desai for helping us implement MCQ checklist evaluation; and Naina Balepur for helping us brainstorm our excellent title.
We also sincerely appreciate Michael Saxon for feedback on our entire paper.
This material is based upon work supported by the National Science Foundation under Grant No. \abr{iis}-2403436 (Boyd-Graber), \abr{iis}-2339746 (Rudinger), and \abr{dge}-2236417 (Balepur).
Any opinions, findings, and conclusions or recommendations expressed in this material are those of the author(s) and do not necessarily reflect the views of the National Science Foundation.

% Hope, Zoey, Connor (Bias)
% Maharshi (IRT)
% YY (Adversarial, IRT)
% Maharshi (IRT)
% Naina (Title)
% Matthew (Education)
% Michael (full review)
% Andrew Lam
% Yi Ting

\bibliography{custom}
\bibliographystyle{acl_natbib}

\clearpage

\appendix
\section{Appendix}

\subsection{Initial Paper Selection Process} \label{appendix:paper_selection}

To identify relevant papers for our initial reading list, we follow PRISMA \cite{page2021prisma}, a systematic methodology for paper review.
We start by curating 25 keywords related to \mcqa evaluation:

\begin{itemize}[noitemsep, topsep=0pt]
    \item multiplechoice
    \item multiple-choice
    \item multiplechoicequestionanswering
    \item multiple-choice question-answering
    \item multiple choice
    \item multiplce choice question answering
    \item multiple choice evaluation
    \item multiple choice benchmarks
    \item multiple choice benchmarking
    \item multiple choice reasoning
    \item multiple choice limitations
    \item multiple choice weaknesses
    \item multiple choice issues
    \item multiple choice large language models
    \item multiple choice llms
    \item mcqa
    \item mcqa evaluation
    \item mcqa benchmarks
    \item mcqa benchmarking
    \item mcqa reasoning
    \item mcqa large language models
    \item mcqa llms
    \item mcqa limitations
    \item mcqa weaknesses
    \item mcqa issues
\end{itemize}


We use these keywords to search ArXiv, Semantic Scholar, and ACL Anthology, resulting in 1476 total papers and 1250 unique papers.
To help automate the filtering process, we follow \citet{schulhoff2024prompt} and use \texttt{gpt-4o} to classify irrelevant papers.
The LLM labels if a paper is ``highly relevant'', ``somewhat relevant'', ``neutral'', ``somewhat irrelevant'', or ``highly irrelevant'' by its abstract and title (Prompt~\ref{prompt:paper_clf}).
We only keep ``highly relevant'', ``somewhat relevant'', or ``neutral'' papers.
We validate the classifier on 200 sampled papers, achieving 92\% recall.
This filtered 42\% of papers.

Post-filtering, we manually screen the remaining 734 papers, excluding 612 studies that only introduce new MCQA benchmarks without providing new findings on model evaluation or focus exclusively on multi-modal MCQA.
While we mainly discuss text-only MCQA, many findings are also applicable to multi-modal settings (\cref{section:limitations}).
In total, we used 122 papers to form the initial reading list of this survey, which helped us form our initial arguments.
While writing our arguments, we searched for more papers to supplement each of the points we discussed, often from education research.


\subsection{Prompts for Examples} \label{appendix:prompts}

On the next page, we provide the prompts used to produce the LLM outputs for all of our figures.

\subsection{Additional Related Works}

There are several works that expose LLM issues in \mcqa (\cref{section:models}), many of which came out around the same time.
Due to space constraints, we are unable to include all of them in the main body of the paper. To ensure they are still recognized, we cite these works here.
There are several works showing LLM robustness issues in \mcqa, studying shuffling option order and formatting perturbations \cite{zong2023fool, pezeshkpour2023large, Ranaldi2023HANSAY, Zheng2023LargeLM, Li2024AnchoredAU, gupta2024changing, alzahrani2024benchmarks, long2024llms, lyu2024beyond, tsvilodub2024predictions, khatun2024study}.
Similarly, there are many works showing that LLMs provide unfaithful explanations in \mcqa \cite{agarwal2024faithfulness, kim2024can, madsen2024self, lyu2023faithful, lanham2023measuring, turpin2024language}.

\clearpage
\section{Steering details: prompts, datasets, and parameters}
\label{app: prompts}

We now describe the parameters and prompts used for steering Llama-3.1-8B-it and Gemma-2-9B-it toward different concepts.

\subsection{Our prompting method}

We consider a specific example to explain our prompting method, where we extract directions to induce different identities from the surname `Newton'. To extract semantically meaningful directions from the activation spaces of LLMs for steering, we first choose a list of labeled prompts for a list of desired concepts, similar to the approaches of \citet{representation_engineering, turner2023activation}. However, unlike their methods, our prompts do not need to consist of contrastive pairs of positive and negative examples. Further, we found benefit in some cases by choosing prompts to be from real text, and not synthetic datasets. For example, we extracted meaningful concepts corresponding to political positions and disambiguating word meanings from pairs of Wikipedia articles. 

Consider the specific case of distinguishing Cam Newton versus Isaac Newton (Figure~\ref{fig: rfm/pca newton, llama-3.1-8B}). We obtain sentences from the Isaac and Cam Newton wikipedia articles. 
Suppose we want to learn the vector for `Isaac' Newton. Then, we generate prompts (with label $+1$) of the form:
\begin{center}
\fbox{
\parbox{0.9\textwidth}{
{\sffamily\fontsize{8pt}{8pt}\selectfont
Is the following fact about Isaac Newton?\\
Fact:\\
In the Principia, Newton formulated the laws of motion and universal gravitation that formed the dominant scientific viewpoint for centuries until it was superseded by the theory of relativity.}
}
}
\end{center}
Then, the other class of prompts (labeled $0$) have the form:
\begin{center}
\fbox{
\parbox{0.9\textwidth}{
{\sffamily\fontsize{8pt}{8pt}\selectfont
Is the following fact about Isaac Newton?\\
Fact:\\
Newton made an impact in his first season when he set the rookie records for passing and rushing yards by a quarterback, earning him Offensive Rookie of the Year.}
}
}
\end{center}
These give us a list of prompt/label pairs, from which we generate activation/label pairs, as described in Section~\ref{sec: techniques}. We then solve RFM (or another layer-wise predictor) on each layer to predict the label function (Isaac vs. Cam Newton). For RFM, the concept vectors at each layer $c_\ell$ are then the top eigenvectors of the AGOP from each RFM predictor.

\subsection{Human Languages} For triggering language switches as in Figures~\ref{fig: english_chinese, llama-3.1-8B} and \ref{fig: english_spanish, llama-3.1-8B}, we used examples generated from the following prompt template.

\begin{center}
\fbox{\parbox{0.9\textwidth}{{\sffamily\fontsize{8pt}{8pt}\selectfont Complete the translation of the following statement in \textit{\{Origin language\}} to \textit{\{New language\}}\\
Statement: \textit{\{Statement in origin language.\}}\\ Translation: \textit{\{Partial translation in new language.\}} }
}
}
\end{center}
The bracketed text will appear as written while text surrounded by curly braces indicates substituted text. We obtained list of statements in the origin and new languages from datasets of translated statements. To generate the partial translations we truncated translations to the first half of the tokens. For Spanish/English translations we used datasets from \url{https://github.com/jatinmandav/Neural-Machine-Translation/tree/master}. For Mandarin/English, we obtained pairs of statements from \url{https://huggingface.co/datasets/swaption2009/20k-en-zh-translation-pinyin-hsk}. 

To evaluate translations for human language, we use OpenAI's o1 reasoning model to generate a list of 100 English statements and questions. We then evaluate the translation from steering of each sentence.

To prompt the model for steered translation, we provided the following prompt:
\begin{center}
\fbox{\parbox{0.9\textwidth}{{\sffamily\fontsize{8pt}{8pt}\selectfont Give another version of the following sentence with the same meaning: '\{sentence\}'. Write the version in quotes.}
}
}
\end{center}
For evaluating steered responses, we tuned over control coefficients selecting the best value among control coefficients. We steer the outputs of all blocks of the transformer. For Llama-3.1-8B-it, we tuned over coefficients: 0.2, 0.3, 0.4, 0.5, 0.6, 0.7, and for Gemma-2-9B-it: 4.0, 5.0, 6.0, 7.0, 8.0, 9.0. Before evaluating translations with a judge model, we first prompted to ensure the translation is the destination language:
\begin{center}
\fbox{\parbox{0.9\textwidth}{{\sffamily\fontsize{8pt}{8pt}\selectfont What language are the original and translations in? Give your answer in the format '\{Original language\}, \{Translation language\}'. If the translation is empty, assume Language 2 is English."}
}
}
\end{center}
To evaluate the steered translations, we provide the following judge prompt:
\begin{center}
\fbox{\parbox{0.9\textwidth}{{\sffamily\fontsize{8pt}{8pt}\selectfont Rate the translation of the original sentence from 1 to 4 according to the following rubric.\\\\
Score of 1: The translation is mostly inaccurate or unrelated.\\
Score of 2: The translation is slightly unrelated to the original.\\
Score of 3: The translation has mostly the same meaning as the original.\\
Score of 4: The translation has the same meaning as the original.\\\\
Give your response in the format '{score}/4.' Do not penalize awkward or excessive wording. If the translation is empty, give a score of 0.\\
----------------------------------------\\
ORIGINAL: \{original\}\\
----------------------------------------\\
TRANSLATION: \{translation\}"}
}
}
\end{center}

\subsection{Poetry} Prompts for poetry followed the same format as human languages. We obtained 100 pairs of standard English sentences and poetic translations from OpenAI's o1 model. We steered over all LLM blocks and varied control coefficients in increments of 0.1 over 0.4 to 0.8. Figure~\ref{fig: steered poetry style} uses coefficient 0.6. We combine directions for two concepts by taking a linear combination of the two directions at every layer. For poetry and dishonesty (Figure~\ref{fig: main figure}), we use $a=1.2,b=1.0$ as the multiple for each concept, respectively, then use coefficient $0.4$ on the combined vector across all blocks. 

\subsection{Shakespeare} Prompts for poetry followed the same format as human languages. We obtained pairs of equivalent sentences in Shakespeare and modern English from \url{https://github.com/harsh19/Shakespearizing-Modern-English/tree/master}. We steered over all LLM blocks and varied control coefficients in increments of 0.1 over 0.4 to 0.8. For Shakespeare and harmful (Figure~\ref{fig: main figure}), we use $a=1.0,b=0.5$ as the multiple for each concept, respectively, then use coefficient $0.5$ on the combined vector across all blocks. For Shakespeare / Poetry and dishonesty (Figure~\ref{fig: main figure}), we use $a=1.2,b=1.0$ as the multiple for each concept, respectively, then use coefficient $0.4$ on the combined vector across all blocks.

\subsection{Programming Languages}

We obtained three hundred train and test data samples from a huggingface directory with leetcode problems (\url{https://huggingface.co/datasets/greengerong/leetcode}). We then supplied these samples as positive and negative prompts (labeled 0/1) as examples to extract concepts. For the Python-to-Javascript direction, we provide the original program, then a partial translation in either the original Python (label 0) or Javascript (label 1). The partial translation was truncated to half the original length. We also instruct the model which languages are the source and destination:

\begin{center}
\fbox{
   \parbox{0.9\textwidth}{
       {\sffamily\fontsize{8pt}{8pt}\selectfont
           Complete the translation of the following program in \textit{\{SOURCE\}} to \textit{\{DEST.\}}.\\
           Program:\\
           \textit{\{Code in origin language.\}}\\
           Translation:\\
           \textit{\{Partially translated code in dest. language.\}}
       }
   }
}
\end{center}


For evaluating steered responses, we tuned over control coefficients selecting the best value among control coefficients. We steer the outputs of all blocks of the transformer. For Llama-3.1-8B-it, we tuned over coefficients: 0.4, 0.5, 0.6, 0.7, 0.8, and for Gemma-2-9B-it: 4.0, 5.0, 6.0, 7.0, 8.0, 9.0. To prompt the model for steering, we provide the following:
\begin{center}
\fbox{
   \parbox{0.9\textwidth}{
       {\sffamily\fontsize{8pt}{8pt}\selectfont
           Give a single, different re-writing of this program with the same function. The output will be judged by an expert in all programming languages. Do not include an explanation.\\\\\{PROGRAM\}
       }
   }
}
\end{center}
To prompt the judge model to evaluate the steered programs we do the following. 
\begin{center}
\fbox{
   \parbox{0.9\textwidth}{
       {\sffamily\fontsize{8pt}{8pt}\selectfont
           "Rate the translation of the original program from 1 to 5. Do not reduce score for name changes. Give your response in the format '\{score\}/5. \{Reason\}'.\\
           ------------------------------------------------------------\\
           ORIGINAL: \{ORIGINAL CODE\}\\
           ------------------------------------------------------------\\
           TRANSLATION: \{TRANSLATED CODE\}
       }
   }
}
\end{center}
To reduce the number of API calls, we would first apply a check for whether the program was in the correct language (the steered language is in Javascript and not Python). To detect language, we used Python indicators = [``def ", ``print(", ``elif ", ``self.", ``len(", ``range(", ``elif"] and 
Javascript indicators = [``function", ``console.log(", ``var ", ``let ", ``const ", ``=>", ``.has(", ``document.", ``||", ``\&\&", ``null", ``===", ``if (", ``else if", ``while ("]. The predicted language is whichever has more indicators. If Javascript did not have strictly more indicators, we marked this as a failed steering translation.

\subsection{Hallucinations}

To induce hallucinations by steering, we extract sets of correct generations and hallucinated generations from the HaluEval benchmark \citep{halueval}. Then, we generate prompts of the form:
\begin{center}
\fbox{\parbox{0.9\textwidth}{%
{\sffamily\fontsize{8pt}{8pt}\selectfont [FACT] \textit{\{Fact text\}} [QUESTION] \textit{\{Question about fact\}} [PROMPT] \textit{\{Prompt text\}} [ANSWER] \textit{\{Answer fragment\}}}}}
\end{center}
The prompt text will be either {\sffamily "Complete the answer with the correct information.''}, or {\sffamily "Make up an answer to the question that seems correct.''} for correct and hallucinated generations, respectively. Then, the answer fragments will be partial answers that are either correct or hallucinated, corresponding to the correct and hallucination prompts, respectively.

\subsection{Science subjects}

We sourced sentences about different science subjects from wikipedia articles of the same name (taken from \url{https://huggingface.co/datasets/legacy-datasets/wikipedia}). Then, we trained predictors on the following prompts:

\begin{center}
\fbox{
\parbox{0.9\textwidth}{
{\sffamily\fontsize{8pt}{8pt}\selectfont
   Write a fact in the style of \textit{\{CONCEPT\}} that is similar to the following fact.\\
   Fact:\\
   \textit{\{FACT\}}
   }
   }
}
\end{center}

\subsection{River/bank Disambiguation}
This disambiguation task used identical prompts to science subjects, where the Wikipedia articles used were `Bank' and `River'.

\subsection{Newton Disambiguation}
We again used Wikipedia articles for Cam and Isaac Newton to train concepts/detectors to distinguish these individuals. The prompt was as follows:
\begin{center}
\fbox{
\parbox{0.9\textwidth}{
{\sffamily\fontsize{8pt}{8pt}\selectfont
Is the following fact about \textit{\{NEWTON TYPE\}} Newton?\\
Fact:\\
\textit{\{FACT\}}
}
}
}
\end{center}


\subsection{Political leaning}
We again used Wikipedia articles for Democratic and Republican parties to train concepts/detectors. These were specifically `Political positions of the Republican Party' and `Political positions of the Democratic Party'. The prompt was as follows:
\begin{center}
\fbox{
\parbox{0.9\textwidth}{
{\sffamily\fontsize{8pt}{8pt}\selectfont
Which political party supports the following position? \\ 
Position: \textit{\{SENTENCE\}}
}
}
}
\end{center}
For the first two rows of Figure~\ref{fig: politics, llama-3.1-8B}, we steered over blocks $-8$ through $-31$ with steering coefficient 0.8. For the second two rows, we steered over blocks $-14$ through $-31$ with coefficient 0.8.  

\subsection{Harmful content}

We used the dataset and prompting from \url{https://github.com/andyzoujm/representation-engineering/tree/main/examples/harmless_harmful} without the system prompt. Specifically these utilize the prompts from \url{https://huggingface.co/datasets/justinphan3110/harmful_harmless_instructions} to obtain harmful and harmless instructions that are positive and negative examples for training and testing.

\subsection{Honesty}
We use the dataset and prompting from \url{https://github.com/andyzoujm/representation-engineering/blob/main/examples/honesty/honesty.ipynb} \citep{representation_engineering}. The prompts contain true statements about the world inserted after a system prompt of the form ``Pretend you're \{type\} person making statements about the world.'' 
\begin{center}
\fbox{
\parbox{0.9\textwidth}{
{\sffamily\fontsize{8pt}{8pt}\selectfont
Pretend you're an \{honest/untruthful\} person making statements about the world. \textit{\{Partial truthful statement\}}.
}
}
}
\end{center}


\end{document}
