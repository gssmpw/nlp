\subsection{Some MCQs Have No Correct Answer} \label{subsection:quality}

Once a source is found (\cref{subsection:test_set_leakage}), researchers collect~or write MCQs, but errors often arise rendering them unanswerable, like mislabeling \cite{explained2023smart}, multiple correct choices \cite{Palta2024PlausiblyPQ}, ambiguity \cite{gema2024we}, missing contexts \cite{wang2024mmlu}, and grammar errors \cite{chen2023hellaswag}.

Educators write MCQs with rigorous protocols, and we must meet similar standards in \abr{nlp} \cite{boyd2019question}; we should use educators' rubrics (Figure~\ref{fig:checklist}) for writing and validating MCQs \cite{haladyna1989taxonomy}.
Such guidelines also specifically exist for distractors \cite{haladyna2002review}---the part of MCQs that discern testees' skills---ensuring they are truly wrong, shortcut-proof (\cref{subsection:artifacts}), and not too easy to rule out (\cref{subsection:saturation}).
Beyond MCQ writing, rubrics can form~data cards \cite{pushkarna2022data} to help researchers record errors in their data and how~they~fixed them.

Recent work in LLM checklist evaluation \cite{cook2024ticking}, 
MCQ metrics \cite{moon2022evaluating},~and MCQ generation \cite{sileo2023generating} show parts~of this may be
automatable (Figure~\ref{fig:checklist}).
For instance, \citet{wang2024mmlu} fix errors in MMLU by using LLMs to detect issues and
write new choices.~LLM judges are not always reliable \cite{xu-etal-2024-pride}, so
human-\abr{ai} collaboration, like model-assisted
refinement~\cite{shankar2024validates} and task
routing~\cite{miranda2024hybrid}, may be more promising.
%
Errors will arise in MCQ writing, but educators' rubrics can help
find and fix them, ensuring answerability.

\fbox{\begin{minipage}{38em}

\subsubsection*{Scaling Law Reproducilibility Checklist}\label{sec:checklist}


\small

\begin{minipage}[t]{0.48\textwidth}
\raggedright
\paragraph{Scaling Law Hypothesis (\S\ref{sec:power-law-form})}

\begin{itemize}[leftmargin=*]
    \item What is the form of the power law?
    \item What are the variables related by (included in) the power law?
    \item What are the parameters to fit?
    \item On what principles is this form derived?
    \item Does this form make assumptions about how the variables are related?
    % \item How are each of these variables counted? (For example, how is compute cost/FLOPs counted, if applicable? How are parameters of the model counted?)
    % \item Are code/code snippets provided for calculating these variables if applicable? 
\end{itemize}


\paragraph{Training Setup (\S\ref{sec:model_training})}
\begin{itemize}[leftmargin=*]
    \item How many models are trained?
    \item At which sizes?
    \item On how much data each? On what data? Is any data repeated within the training for a model?
    \item How are model size, dataset size, and compute budget size counted? For example, how are parameters of the model counted? Are any parameters excluded (e.g., embedding layers)?
    \item Are code/code snippets provided for calculating these variables if applicable?
    % embedding  For example, how is compute cost counted, if applicable? 
    \item How are hyperparameters chosen (e.g., optimizer, learning rate schedule, batch size)? Do they change with scale?
    \item What other settings must be decided (e.g., model width vs. depth)? Do they change with scale?
    \item Is the training code open source?
    % \item How is the correctness of the scaling law considered SHOULD WE?
\end{itemize}

\end{minipage}
\begin{minipage}[t]{0.48\textwidth}
\raggedright


\paragraph{Data Collection(\S\ref{sec:data})}
\begin{itemize}[leftmargin=*]
    \item Are the model checkpoints provided openly?
    % \item Are these checkpoints modified in any way before evaluation? (say, checkpoint averaging)
    % \item If the above is done, is code for modifying the checkpoints provided?
    \item How many checkpoints per model are evaluated to fit each scaling law?
    \item What evaluation metric is used? On what dataset?
    \item Are the raw evaluation metrics modified, e.g., through loss interpolation, centering around a mean, scaling logarithmically, etc?
    \item If the above is done, is code for modifying the metric provided? 
\end{itemize}

\paragraph{Fitting Algorithm (\S\ref{sec:opt})}
\begin{itemize}[leftmargin=*]
    \item What objective (loss) is used?
    \item What algorithm is used to fit the equation?
    \item What hyperparameters are used for this algorithm?
    \item How is this algorithm initialized?
    \item Are all datapoints collected used to fit the equations? For example, are any outliers dropped? Are portions of the datapoints used to fit different equations?
    \item How is the correctness of the scaling law considered? Extrapolation, Confidence Intervals, Goodness of Fit?
\end{itemize}

\end{minipage}

% \paragraph{Other}
% \begin{itemize}
%     \item Is code for 
% \end{itemize}

\end{minipage}}