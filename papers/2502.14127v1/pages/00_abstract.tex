\begin{abstract} {
Multiple choice question answering (\mcqa) is popular for LLM evaluation due to its simplicity and human-like testing, but we argue for its reform.
%
We first reveal flaws in~\mcqa's format, as it struggles to: 1) test generation/subjectivity; 2) match LLM use cases; and 3) fully~test knowledge.
%
We instead advocate for generative formats based on human testing---where LLMs construct
and explain answers---better~capturing user needs and knowledge while remaining easy to score.
We then show even when \mcqa is a useful format, its datasets suffer from: leakage; unanswerability; shortcuts; and saturation.
In each issue, we give fixes from education, like rubrics to guide MCQ writing; scoring methods to bridle guessing; and Item Response Theory to build harder MCQs.
Lastly, we discuss LLM errors in \mcqa---robustness, biases, and unfaithful explanations---showing how our prior solutions better measure or address these issues.
While we do not need to desert \mcqa, we encourage more efforts in refining the task based on educational testing, advancing evaluations.
}
\end{abstract}
