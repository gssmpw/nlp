\section{Conclusions and Future Work}
\label{sec:conclusion}

In this paper, we introduce \textit{Survival Game}. It is inspired by Natural Selection and quantifies the intelligence of any subject in any task. It demonstrates three advantages.
\begin{itemize}
	\item Firstly, the test offers a clear physical meaning and a well-defined mathematical framework. It categorizes intelligence into three levels and enables a deep understanding of the subject's intelligence. 
	\item Secondly, \textit{Survival Game} provides a roadmap for the future development of AI. It shows a clear relationship to the scale of AI systems, which enables us to project the time to reach high-level intelligence.
	\item Finally, \textit{Survival Game} helps reveal the nature of human tasks and AI. It suggests that human tasks exhibit criticality properties and AI's superficial manner to solve human tasks. 
\end{itemize}


We believe that future efforts should focus on three key areas.
Firstly, since current AI technologies are mostly at the Limited Level, we need to identify appropriate application scenarios and design an effective human supervision framework.
Secondly, since \textit{Survival Game} connects the model size to the intelligence levels, we can use \textit{Survival Game} to plan the roadmap of future AI development. 
Finally, linking \textit{Survival Game} to SOC theory shows promising results and more efforts shall be made in this direction. This will not only help design more effective AI models but also deepen our understanding of humans ourselves.


