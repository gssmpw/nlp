\section{Related Work}
\label{sec:related_work}
Detection of spoofing based on properties of the received signal is explored in ____. Changes in the acquisition matrix (e.g., the shape of the acquisition peak, number of peaks per acquisition channel) of the \gls{gnss} signal are generally good indicators of the presence of adversarial signals. While such an approach is highly effective, it requires direct access to the acquisition stage of the \gls{gnss} receiver, unavailable in commercial \gls{cots} receivers. Alternatives, such as ____, use multiple channels to acquire separate peaks in the same acquisition space, with the drawback that reduced number of signals can be tracked at the same time. 

Transmission origin estimation based on the received signal power and on the receiver's \gls{agc} provide an indicative figure of the quality of the received signals ____. However, changes in the \gls{agc} are often hard to relate to adversarial manipulation or variations in the environment of a mobile antenna (subject to time-varying multipath). Techniques generally referred to as \gls{sqm}, while providing immediate insight on the structure and quality of the \gls{gnss} signal quality, tend to perform poorly in a dynamic scenario. Similarly, metrics based on Doppler or pseudorange plausibility monitoring are effective and relatively low cost in their evaluation ____, but can be thwarted by improved attacker hardware (e.g., more accurate clock distribution at the transmitter front-end) and better adversarial strategy (e.g., precise code phase alignment of the spoofed signals to the legitimate ones, ____). 

In this context, two interesting recent improvements allowed more advanced spoofing countermeasures to be deployed in civilian COTS systems. First, inertial sensors improved in stability and accuracy even at the lower end of the segment as long as the integration time is short. Second, more feature-rich \gls{gnss} receivers are increasingly integrated in platforms providing additional sensors, computational power, and connectivity. %Specifically, mobile devices are now capable of advanced \gls{gnss} measurements which can be coupled with \gls{imu} inertial sensing.
This generally includes so-called raw measurements obtained by the \gls{gnss} receiver tracking loops and consists of the raw observables without any processing from the \gls{gnss} receiver's \gls{pnt} engine. Techniques based on validation of the Doppler shift of the received signal often allow detection of spoofed satellite signal, but the attacker can circumvent such detection using better and more stable reference sources at the adversarial transmitter ____. Similarly, pseudorange measurement bounding also proved effective in detecting spoofed signals but with the limitation that often such detection system is dependent on a first acquisition in a benign scenario to establish a baseline ____. Such measurements are increasingly available even on mobile devices thanks to the Android Raw GNSS Measurements API, allowing hardening portable receivers ____

Work on the GNSS-INS fusion shows that the current state of the inertial navigation quality is sufficient to improve the quality of GNSS-only measurements in a benign scenario, for a gamut of mobile platforms ____. Such devices can detect spoofing based on the inconsistency of the dynamics, e.g. when the spoofer causes rapid changes in the \gls{pnt} solution beyond the dynamics achievable by the mobile system ____. Hypothesis testing based on incongruities of the \gls{imu} measured acceleration and the \gls{pnt} provided by the \gls{gnss} receiver reliably detect spoofing attacks but do not provide any further information on (the complexity of) the attack, relying on the navigation processor outcome. A traditional approach relies on innovation testing while performing joint navigation and estimation using a Kalman filter (or other variations). While this greatly benefits robotics and autonomous systems, the improvements to navigation in \gls{gnss} denied conditions are limited, and low-cost \gls{imu}s cannot provide reliable inertial navigation. Ultimately, the strongest limitation is the quality of the \gls{imu} sensors used for recovering from \gls{gnss} spoofing and jamming, with \gls{imu} errors degrading the solution usually within a few minutes of the loss of \gls{gnss} lock. Accumulation of the integration error will grow in an unbound manner over time, making the innovation test result meaningless for anti-spoofing purposes. 
Also, integration window-based methods are generally slow in detecting an adversary as the innovation residual needs to increase beyond the confidence the filter has in the estimated covariance of the GNSS measurement. Practically, a subtle adversary slowly drifting the \gls{pnt} solution might not be detected until it causes major \gls{pnt} solution disruption.

%At the same time, new capabilities are available at the \gls{gnss} receiver as lower-level information is available. This generally includes so-called raw measurements obtained by the \gls{gnss} receiver tracking loops and consists of the raw observables without any processing from the \gls{gnss} receiver's \gls{pnt} engine. Techniques based on validation of the Doppler shift of the received signal often allow detection of spoofed satellite signal, but the attacker can circumvent such detection using better and more stable reference sources at the adversarial transmitter ____. Similarly, pseudorange measurement bounding also proved effective in detecting spoofed signals but with the limitation that often such detection system is dependent on a first acquisition in a benign scenario to establish a baseline ____. Such measurements are increasingly available even on mobile devices thanks to the Android Raw GNSS Measurements API, allowing hardening portable receivers ____

Carrier phase measurements can provide considerable improvements to the quality and accuracy of the \gls{pnt} solution due to the much higher resolution of the carrier information, compared to code-based ranging. \gls{imu} measurements in tight GNSS-INS integration help resolve the integer ambiguity problem in differential \gls{gnss} systems where a joint baseline estimation with a reference station allows reliable spoofing detection even in a multipath-challenged environment (e.g., urban canyons). The main advantage consists in the dual robustness effect against the environment and potential attackers, but this requires external reference stations, limiting the applicability to scenarios where this is available. In the context ofthe  recent development of autonomous vehicular and aerial platforms, carrier phase measurements play a critical role in providing centimeter-level accurate positioning and enhancing spoofing countermeasures. As shown in ____, the high resolution of the carrier phase information can be evaluated against high-frequency antenna motion to detect adversarial signals originating from a single transmitter. 
Specifically, high-frequency antenna motion can be leveraged to detect spoofed satellite signals ____, specifically in the case where the antenna dynamics are unidirectional and can be determined by a mechanization model. The latter can be complex to extract for moving antennas, where the amplitude and frequency of the motion can be arbitrary and multi-directional in space.