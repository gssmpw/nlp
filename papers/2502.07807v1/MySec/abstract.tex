

\begin{abstract}
    Collaborative perception (CP) is a promising method for safe connected and autonomous driving, which enables multiple connected and autonomous vehicles (CAVs) to share sensing information with each other to enhance perception performance. For example, occluded objects can be detected, and the sensing range can be extended. However, compared with single-agent perception, the openness of a CP system makes it more vulnerable to malicious agents and attackers, who can inject malicious information to mislead the perception of an ego CAV, resulting in severe risks for the safety of autonomous driving systems. To mitigate the vulnerability of CP systems, we first propose a new paradigm for malicious agent detection that effectively identifies malicious agents at the feature level without requiring verification of final perception results, significantly reducing computational overhead. Building on this paradigm, we introduce CP-GuardBench, the first comprehensive dataset provided to train and evaluate various malicious agent detection methods for CP systems. Furthermore, we develop a robust defense method called CP-Guard+, which enhances the margin between the representations of benign and malicious features through a carefully designed mixed contrastive training strategy. Finally, we conduct extensive experiments on both CP-GuardBench and V2X-Sim, and the results demonstrate the superiority of CP-Guard+.
\end{abstract}