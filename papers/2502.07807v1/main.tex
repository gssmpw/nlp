%%%%%%%% ICML 2025 EXAMPLE LATEX SUBMISSION FILE %%%%%%%%%%%%%%%%%

\documentclass{article}

% Recommended, but optional, packages for figures and better typesetting:
\usepackage{microtype}
\usepackage{graphicx}
\usepackage{subcaption} 
\usepackage{booktabs} % for professional tables
\usepackage{xcolor} % for defining colors

% hyperref makes hyperlinks in the resulting PDF.
% If your build breaks (sometimes temporarily if a hyperlink spans a page)
% please comment out the following usepackage line and replace
% \usepackage{icml2025} with \usepackage[nohyperref]{icml2025} above.
\usepackage{multirow}

\definecolor{c1}{HTML}{15258E}
\definecolor{c2}{HTML}{000000}
\usepackage[colorlinks,citecolor=c1]{hyperref}
\usepackage{float}
\usepackage{makecell}

% Attempt to make hyperref and algorithmic work together better:
\newcommand{\theHalgorithm}{\arabic{algorithm}}

% Use the following line for the initial blind version submitted for review:
% \usepackage{icml2025}

% If accepted, instead use the following line for the camera-ready submission:
\usepackage[accepted]{icml2025}

% For theorems and such
\usepackage{amsmath}
\usepackage{amssymb}
\usepackage{mathtools}
\usepackage{amsthm}

% if you use cleveref..
\usepackage[capitalize,noabbrev]{cleveref}

%%%%%%%%%%%%%%%%%%%%%%%%%%%%%%%%
% THEOREMS
%%%%%%%%%%%%%%%%%%%%%%%%%%%%%%%%
\theoremstyle{plain}
\newtheorem{theorem}{Theorem}[section]
\newtheorem{proposition}[theorem]{Proposition}
\newtheorem{lemma}[theorem]{Lemma}
\newtheorem{corollary}[theorem]{Corollary}
\theoremstyle{definition}
\newtheorem{definition}[theorem]{Definition}
\newtheorem{assumption}[theorem]{Assumption}
\theoremstyle{remark}
\newtheorem{remark}[theorem]{Remark}

%%%%%%%%%%%% choose font %%%%%%%%%%%%
%%%%%%%%%%%% choose font %%%%%%%%%%%%
% \usepackage{charter}
% \usepackage{times}
% \usepackage{mathptmx}
% \usepackage{mathpazo}
% \usepackage{nimbussans}
% \usepackage{helvet}
% \renewcommand{\familydefault}{\sfdefault}
% \usepackage{courier}
% \usepackage{bookman}
%%%%%%%%%%%% choose font %%%%%%%%%%%%

% Todonotes is useful during development; simply uncomment the next line
%    and comment out the line below the next line to turn off comments
%\usepackage[disable,textsize=tiny]{todonotes}
\usepackage[textsize=tiny]{todonotes}


% The \icmltitle you define below is probably too long as a header.
% Therefore, a short form for the running title is supplied here:
\icmltitlerunning{CP-Guard+}

\begin{document}

\twocolumn[
\icmltitle{CP-Guard+: A New Paradigm for Malicious Agent Detection and Defense in Collaborative Perception}

% It is OKAY to include author information, even for blind
% submissions: the style file will automatically remove it for you
% unless you've provided the [accepted] option to the icml2025
% package.

% List of affiliations: The first argument should be a (short)
% identifier you will use later to specify author affiliations
% Academic affiliations should list Department, University, City, Region, Country
% Industry affiliations should list Company, City, Region, Country

% You can specify symbols, otherwise they are numbered in order.
% Ideally, you should not use this facility. Affiliations will be numbered
% in order of appearance and this is the preferred way.
\icmlsetsymbol{equal}{*}

\begin{icmlauthorlist}
\icmlauthor{Senkang Hu}{cityu,equal}
\icmlauthor{Yihang Tao}{cityu,equal}
\icmlauthor{Zihan Fang}{cityu}
\icmlauthor{Guowen Xu}{uestc}
\icmlauthor{Yiqin Deng}{cityu}
\icmlauthor{Sam Kwong}{lnu}
\icmlauthor{Yuguang Fang}{cityu}
%\icmlauthor{}{sch}
%\icmlauthor{}{sch}
\end{icmlauthorlist}

\icmlaffiliation{cityu}{Department of Computer Science, City University of Hong Kong.}
\icmlaffiliation{uestc}{School of Computer Science and Engineering, University of Electronic Science and Technology of China.}
\icmlaffiliation{lnu}{Department of Computing and Decision Sciences, Lingnan University.}

\icmlcorrespondingauthor{Yuguang Fang}{my.Fang@cityu.edu.hk}

% You may provide any keywords that you
% find helpful for describing your paper; these are used to populate
% the "keywords" metadata in the PDF but will not be shown in the document
\icmlkeywords{Machine Learning, ICML}

\vskip 0.3in
]

% this must go after the closing bracket ] following \twocolumn[ ...

% This command actually creates the footnote in the first column
% listing the affiliations and the copyright notice.
% The command takes one argument, which is text to display at the start of the footnote.
% The \icmlEqualContribution command is standard text for equal contribution.
% Remove it (just {}) if you do not need this facility.

% \printAffiliationsAndNotice{}  % leave blank if no need to mention equal contribution
\printAffiliationsAndNotice{\icmlEqualContribution} % otherwise use the standard text.

\begin{abstract}
    Collaborative perception (CP) is a promising method for safe connected and autonomous driving, which enables multiple vehicles to share sensing information to enhance perception performance. However, compared with single-vehicle perception, the openness of a CP system makes it more vulnerable to malicious attacks that can inject malicious information to mislead the perception of an ego vehicle, resulting in severe risks for safe driving. To mitigate such vulnerability, we first propose a new paradigm for malicious agent detection that effectively identifies malicious agents at the feature level without requiring verification of final perception results, significantly reducing computational overhead. Building on this paradigm, we introduce CP-GuardBench, the first comprehensive dataset provided to train and evaluate various malicious agent detection methods for CP systems. Furthermore, we develop a robust defense method called CP-Guard+, which enhances the margin between the representations of benign and malicious features through a carefully designed Dual-Centered Contrastive Loss (DCCLoss). Finally, we conduct extensive experiments on both CP-GuardBench and V2X-Sim, and demonstrate the superiority of CP-Guard+. 
\end{abstract}

\section{Introduction}
Implicit Neural Representations (INRs), which fit the target function using only input coordinates, have recently gained significant attention.
%
By leveraging the powerful fitting capability of Multilayer Perceptrons (MLPs), INRs can implicitly represent the target function without requiring their analytical expressions. 
%
The versatility of MLPs allows INRs to be applied in various fields, including inverse graphics~\citep{mildenhall2021nerf, barron2023zip, martin2021nerf}, image super-resolution~\citep{chen2021learning, yuan2022sobolev, gao2023implicit}, 
image generation~\citep{skorokhodov2021adversarial}, and more~\citep{chen2021nerv, strumpler2022implicit, shue20233d}.
%
\begin{figure}
    \includegraphics[width=0.5\textwidth]{Image/Fig2.pdf}
    \caption{As illustrated at the circled blue regions and green regions, it can be observed that even with well-chosen standard deviation/scale, as experimented in \autoref{figure:combined}, the results are still unsatisfactory. However, using our proposed method, the noise is significantly alleviated while further enhancing the high-frequency details.}
    \label{fig:var}
    \vspace{-10pt}
\end{figure}

\begin{figure*}[!ht]
    \centering
    \begin{minipage}[b]{0.25\textwidth}
        \centering
        \includegraphics[width=1.\textwidth]{Image/fig_cropped.pdf} % 替换为你的小图文件
        \label{figure:small_image}
        \vspace{-20pt}
    \end{minipage}%
    \hfill
    \begin{minipage}[b]{0.75\textwidth}
        \centering
        \includegraphics[width=1.\textwidth]{Image/psnr_trends_rff_pe_simplified.pdf} % 替换为你的大图文件
        \vspace{-20pt}
        \label{figure:large_image}
        
    \end{minipage}
    \caption{We test the performance of MLPs with Random Fourier Features (RFF) and MLPs with Positional Encoding (PE) on a 1024-resolution image to better distinguish between high- and low-frequency regions, as demonstrated on the left-hand side of this figure. We find that the performance of MLPs+RFF degrades rapidly with increasing standard deviation compared with MLPs+PE. Since positional encoding is deterministic, scale=512 can be considered to have standard deviation around 121.}
    \label{figure:combined}
    \vspace{-10pt}
\end{figure*}
Varying the sampling standard deviation/scale may lead to degradation results, as shown in \autoref{figure:combined}.
%
However, MLPs face a significant challenge known as the spectral bias, where low-frequency signals are typically favored during training~\citep{rahaman2019spectral}. 
A common solution is to map coordinates into the frequency domain using Fourier features, such as Random Fourier Features and Positional Encoding, which can be understood as manually set high-frequency correspondence prior to accelerating the learning of high-frequency targets.~\citep{tancik2020fourier}. 
This embeddings widely applied to the INRs for novel view synthesis~\citep{mildenhall2021nerf,barron2021mip}, dynamic scene reconstruction~\citep{pumarola2021d}, object tracking~\citep{wang2023tracking}, and medical imaging~\citep{corona2022mednerf}.
% \begin{figure}[!h]
%     \centering
%     \includegraphics[width=1.\textwidth]{Image/psnr_trends_rff_pe_simplified.pdf}
%     \caption{This figure shows the change of PSNR on the whole, low-frequency region, and high-frequency region of the image fitting by using two Fourier Features Embedding with varying scale of variance: (Right) Positional Encoding (PE) (Left) Random Fourier Features (RFF). Both PE and RFF will degrade the low-frequency regions of the target image when variance increases.}
%     \vspace{-20pt} 
%     \label{figure:stats}
% \end{figure}


Although many INRs' downstream application scenarios use this encoding type, it has certain limitations when applied to specific tasks.
%
It depends heavily on two key hyperparameters: the sampling standard deviation/scale (available sampling range of frequencies) and the number of samples.
%
Even with a proper choice of sampling standard deviation/scale, the output remains unsatisfactory, as shown in \autoref{fig:var}: Noisy low-frequency regions and degraded high-frequency regions persist with well chosen sampling standard deviation/scale with the grid-searched standard deviation/scale, which may potentially affect the performance of the downstream applications resulting in noisy or coarse output.
%
However, limited research has contributed to explaining the reason and finding a proper frequency embeddings for input~\citep{landgraf2022pins, yuce2022structured}.

In this paper, we aim to offer a potential explanation for the high-frequency noise and propose an effective solution to the inherent drawbacks of Fourier feature embeddings for INRs.
%
Firstly, we hypothesize that the noisy output arises from the interaction between Fourier feature embeddings and multi-layer perceptrons (MLPs). We argue that these two elements can enhance each other's representation capabilities when combined. However, this combination also introduces the inherent properties of the Fourier series into the MLPs.
%
To support our hypothesis, we propose a simple theorem stating that the unsampled frequency components of the embeddings establish a lower bound on the expected performance. This underpins our hypothesis, as the primary fitting error in finitely sampled Fourier series originates from these unsampled frequencies.

Inspired by the analysis of noisy output and the properties of Fourier series expansion, we propose an approach to address this issue by enabling INRs to adaptively filter out unnecessary high-frequency components in low-frequency regions while enriching the input frequencies of the embeddings if possible.
%
To achieve this, we employ bias-free (additive term-free) MLPs. These MLPs function as adaptive linear filters due to their strictly linear and scale-invariant properties~\citep{mohan2019robust}, which preserves the input pattern through each activation layer and potentially enhances the expressive capability of the embeddings.
%
Moreover, by viewing the learning rate of the proposed filter and INRs as a dynamically balancing problem, we introduce a custom line-search algorithm to adjust the learning rate during training. This algorithm tackles an optimization problem to approximate a global minimum solution. Integrating these approaches leads to significant performance improvements in both low-frequency and high-frequency regions, as demonstrated in the comparison shown in \autoref{fig:var}.
%
Finally, to evaluate the performance of the proposed method, we test it on various INRs tasks and compare it with state-of-the-art models, including BACON~\citep{lindell2022bacon}, SIREN~\citep{sitzmann2020implicit}, GAUSS~\citep{ramasinghe2022beyond} and WIRE~\citep{saragadam2023wire}. 
The experimental results prove that our approach enables MLPs to capture finer details via Fourier Features while effectively reducing high-frequency noise without causing oversmoothness.
%
To summarize, the following are the main contributions of this work:
\begin{itemize}
    \item From the perspective of Fourier features embeddings and MLPs, we hypothesize that the representation capacity of their combination is also the combination of their strengths and limitations. A simple lemma offers partial validation of this hypothesis.

    
    \item  We propose a method that employs a bias-free MLP as an adaptive linear filter to suppress unnecessary high frequencies. Additionally, a custom line-search algorithm is introduced to dynamically optimize the learning rate, achieving a balance between the filter and INRs modules.

    \item To validate our approach, we conduct extensive experiments across a variety of tasks, including image regression, 3D shape regression, and inverse graphics. These experiments demonstrate the effectiveness of our method in significantly reducing noisy outputs while avoiding the common issue of excessive smoothing.
\end{itemize}

\vspace{-3mm}
\section{Preliminaries}
\label{sec:preliminaries}
\subsection{Formulation of Collaborative Perception}
\label{sec:formulation}

% \vspace{-3mm}
In this section, we formulate collaborative perception and give the pipeline of our CP system. Specifically, let $\mathcal{X}^N$ denote the set of $N$ CAVs in the CP system. CAVs in $\mathcal{X}$ can be divided into two categories: the ego CAV and helping CAVs. The ego CAV is the one that needs to perceive its surrounding environment, while helping CAVs are the ones that send their complementary sensing information to the ego CAV to help it enhance its perception performance.
Thus, each CAV can be an ego one and helping one, depending on its role in a perception process. We assume that each CAV is equipped with a feature encoder $f_\mathtt{{encoder}}(\cdot)$, a feature aggregator $f_\mathtt{{agg}}(\cdot)$, and a feature decoder $f_\mathtt{{decoder}}(\cdot)$. For the $i$-th CAV in the set $\mathcal{X}$, the raw observation is denoted as $\mathbf{O}_i$ (such as camera images and LiDAR point clouds), and the final perception results are denoted as $\mathbf{Y}_i$. The CP pipeline of the $i$-th CAV can be described as follows.
\begin{enumerate}
    % \vspace{-3mm}
    \setlength{\itemsep}{0pt}
    \setlength{\parskip}{0pt}
    \setlength{\parsep}{0pt}
    \item \textit{Observation Encoding}: Each CAV encodes its raw observation $\mathbf{O}_j$ into an initial feature map $\mathbf{F}_j = f_\mathtt{{encoder}}(\mathbf{O}_j)$, where $j \in \mathcal{X}^N$.
    \item \textit{Intermediate Feature Transmission}: Helping CAVs transmit their intermediate features to the ego CAV: $\mathbf{F}_{j\rightarrow i}=\mathbf{\Gamma}_{j\rightarrow i}(\mathbf{F}_j),\  j\in \mathcal{X}^N, j\neq i,$
    where $\mathbf{\Gamma}_{j\rightarrow i}(\cdot)$ denotes a transmitter that conveys the $j$-th CAV's intermediate feature $\mathbf{F}_j$ to the ego CAV, while performing a spatial transformation. $\mathbf{F}_{j\rightarrow i}$ is the spatially aligned feature in the $i$-th CAV's coordinate.
    \item \textit{Feature Aggregation}: The ego CAV receives all the intermediate features and fuses them into a unified observational feature $\mathbf{F}_\mathtt{fused}=f_\mathtt{agg}(\mathbf{F}_{0\rightarrow i}, \{\mathbf{F}_{j\rightarrow i}\}_{j\neq i,\  j\in \mathcal{X}^N})$.
    \item \textit{Perception Decoding}: Finally, the ego CAV decodes the unified observational feature $\mathbf{F}_\mathtt{fused}$ into the final perception results $\mathbf{Y}=f_\mathtt{decoder}(\mathbf{F}_\mathtt{fused})$.
    % \vspace{-4mm}
\end{enumerate}


\begin{figure*}[t]
    % \vspace{-5mm}
    \centering
    % \fbox{\rule{0pt}{1.8in} \rule{0.9\linewidth}{0pt}}
    \includegraphics[width=.9\linewidth]{fig/CPDataGenerationPipeline.png}
    \vspace{-3mm}
    \caption{\textbf{Automatic Data Generation and Annotation Pipeline.} We first train a robust LiDAR collaborative object detector. Then, we discard the detection head and decoder and only keep the backbone as the intermediate feature generator. The data generation pipeline is shown in (a), (b), and (c), where (a) is the intermediate feature generation, (b) is the attack implementation, and (c) is the pair generation and saving.}
    \label{fig:data_generation}
    \vspace{-5mm} 
\end{figure*}



\subsection{Adversarial Threat Model}

Our focus is on the operation of an  intermediate-fusion collaboration scheme, where an attacker introduces designed adversarial perturbations into the intermediate features to mislead the perception of the ego CAV. Since an attacker participates in the collaborative system with local perception model installation, we assume they have white-box access to the model parameters. The attack procedure in each frame follows four sequential phases.
\begin{enumerate}
    \vspace{-4mm}
    \setlength{\itemsep}{0pt}
    \setlength{\parskip}{0pt}
    \setlength{\parsep}{0pt}
    \item \textit{Local Perception Phase}: All agents, including the malicious one, process their sensing data independently and extract intermediate features using feature encoders. 
    \vspace{-1mm}
    \begin{equation}
    \mathbf{F}_k = f_\mathtt{encoder}(\mathbf{O}_k), \quad k \in \mathcal{X}^N
    \end{equation}
    \vspace{-1mm}
    This phase operates in parallel without inter-agent communication.

    \item \textit{Feature Communication Phase}: All agents broadcast their extracted features through the network. Malicious agent $k$ collects feature information $\{\mathbf{F}_{j\rightarrow i}\}$ from other agents. Feature-level transmission ensures minimal communication overhead compared to raw sensor data exchange.

    \item \textit{Attack Generation Phase}: A malicious agent executes the attack by first perturbing its local features and then propagating them through the collaborative perception pipeline described in Section \ref{sec:formulation}.
    The attacker aims to optimize the perturbation $\delta$ through an iterative process. The optimization objective is formulated as:
    \vspace{-1mm}
    \begin{equation}
        \vspace{-2mm}
        \begin{aligned}
        \mathop{\arg\max}_{\delta} \mathcal{L}(\mathbf{Y}^\delta, \mathbf{Y}^\mathtt{gt}),
        \quad \mathtt{s.t.}\quad  \|\delta\|\leq \Delta
        \end{aligned}
    \end{equation}
    where $\Delta$ bounds the perturbation magnitude to maintain attack stealthiness. The total loss function is designed to aggregate adversarial losses over all object proposals, targeting both classification and localization aspects:
    \vspace{-1mm}
    \begin{equation}
        \vspace{-2mm}
        \mathcal{L}(\mathbf{Y}^\delta, \mathbf{Y}^\mathtt{gt}) = \sum_{p \in \mathbf{Y}^\delta} \mathcal{L}_\mathtt{adv}(p, p^\mathtt{gt})
    \end{equation}
    For each proposal $p$ with the highest confidence class $c = \mathop{\arg\max}\{p_i\}$, we leverage a class-specific adversarial loss following \citep{tuAdversarialAttacksMultiAgent2021}:
    \begin{equation*}
        % \vspace{-2mm}
        \mathcal{L}_\mathtt{adv}(p', p) = \begin{cases}
            -\log(1 - p'_c)\cdot\eta & c \neq k,\ p_c > \tau_1\\
            -\lambda p'_c\log(1 - p'_c) & c = k,\ p_c > \tau_2\\
            0 & \text{otherwise}
        \end{cases}
    \end{equation*}
    where $\eta$ represents the IoU between perturbed and original proposals to consider spatial accuracy, $\tau_1$ and $\tau_2$ are confidence thresholds for different attack scenarios, $\lambda$ balances the importance of different attack objectives, and $k$ denotes the background class.

    \item \textit{Defense and Final Perception Phase}: The ego vehicle integrates all received feature information, including potentially corrupted ones, to complete the final object detection task. Note that we focus exclusively on CP-specific vulnerabilities, excluding physical sensor attacks (e.g., LiDAR or GPS spoofing), which are general threats to CAVs. We also assume communication channels are secured with proper cryptographic protection.
\end{enumerate}



\section{CP-GuardBench}




To facilitate feature-level malicious agent detection in CP systems, we propose to develop CP-GuardBench, the first benchmark for malicious agent detection in CP systems. It provides a comprehensive dataset for training and evaluating malicious agent detection methods. 
In this section, we will introduce the details of CP-GuardBench, including the automatic data generation and annotation pipeline in Section \ref{sec:data_generation}, and the data visualization and statistics in Section \ref{sec:data_statistics}.

\subsection{Automatic Data Generation and Annotation}
\label{sec:data_generation}

We build CP-GuardBench based on one of the most widely used datasets in the CP field, V2X-Sim \citep{liV2XSimMultiAgentCollaborative2022}, which is a comprehensive simulated multi-agent perception dataset for V2X-aided autonomous driving. 
In this section, we introduce the automatic data generation and annotation pipeline of CP-GuardBench. The pipeline is shown in Figure \ref{fig:data_generation}. It consists of three steps: 1) intermediate feature generation, 2) attack implementation, and 3) pair generation and saving. 

Specifically, we first train a robust LiDAR collaborative object detector, which consists of a convolutional backbone, a convolutional decoder, and a prediction head for classification and regression \citep{Luo_2018_CVPR}. As for the fusion method, we adopt the mean fusion method to fuse the intermediate features from different collaborators.
Subsequently, the backbone is retained for extracting intermediate features, which are then transmitted and utilized by an ego CAV as supplementary information. 


Secondly, the attacks are implemented and applied to the intermediate features. 
The detection head and decoder are then frozen to generate the attacked detection results and optimize the adversarial perturbations. 
As shown in Figure \ref{fig:data_generation}, several iterations are required to optimize the perturbations, and the loss function differs for different attack types.
In our CP-GuardBench, we consider five types of attacks, including Projected Gradient Descent (PGD) \citep{madry2018towards}, Carini \& Wagner (C\&W) attack \citep{carlini2017evaluatingrobustnessneuralnetworks}, Basic Iterative Method (BIM) \citep{kurakin2017adversarialexamplesphysicalworld}, Guassian Noise Perturbation (GN), and Fast Gradient Sign Method (FGSM) \citep{goodfellow2015explainingharnessingadversarialexamples}. The implementation details can be found in Appendix \ref{app:attack_details}.



In the generation of attack data, we randomly choose one of the attacks mentioned above and generate the corresponding attack data in each iteration.
Finally, the perturbed features will be annotated with the corresponding attack type and saved for later use. 
\vspace{-3mm}




\subsection{Data Visualization and Statistics}
\label{sec:data_statistics}
% \vspace{-2mm}





We visualize the samples of the generated data in Figures \ref{fig:data_statistics}(a), (b), (c), and (d). We observe that attacks are so stealthy that it is very hard to see the difference with the naked eye, which poses a great challenge to address the detection of malicious agents.



To construct CP-GuardBench, we randomly sample 9000 frames from V2X-Sim and generate 42200 feature-label pairs. The data is then split into training, validation, and test sets with a ratio of 8:1:1. 
The data statistics are shown in Figures \ref{fig:data_statistics}(e), (f), and (g). Figure \ref{fig:data_statistics}(e) illustrates the distribution of the number of collaborators, which is the number of agents that collaboratively perceive environments. 
The number of collaborators ranges from 3 to 6, with the most common scenario being 4 collaborators, accounting for 46.0\% of the total data. 5 and 6 collaborators are also common, accounting for 29.9\% and 19.5\% of the total data, respectively. 
Regarding the distribution of attack types, as depicted in Figure \ref{fig:data_statistics}(f), we observe that the attack types are evenly distributed, with each type accounting for approximately 20\% of the total data. This is due to the random selection of one attack type in each iteration. Figure \ref{fig:data_statistics}(g) illustrates the attack ratio, which represents the ratio of the number of attackers to the total number of agents in a collaboration network. The maximum attack ratio exceeds 0.3, the minimum is 0, and the average attack ratio is 0.18.

% \vspace{-3mm}

\begin{figure}[t]
    \centering
    % \includegraphics[width=0.5\linewidth]{}
    % \caption{Caption}
    % \label{fig:enter-label}
    % \fbox{\rule{0pt}{1in} \rule{0.9\linewidth}{0pt}}
    \includegraphics[width=1\linewidth]{fig/data_statistics2.png}
    \vspace{-8mm}
    \caption{\textbf{Visualization and Statistics of CP-GuardBench. (a), (b), (c)  and (d) are visualization}, which visualize the normal intermediate features and the adversarial examples perturbed by different malicious agents. We can see the adversarial examples are almost identical to the normal examples, which indicates the challenges in detecting malicious agents. \textbf{(e), (f), (g) and (h) are the statistics} of CP-GuardBench, including the number of collaborators, attack ratio and attack types.} 
    \label{fig:data_statistics}
    \vspace{-8mm} 
\end{figure}
\vspace{-3mm}

\section{CP-Guard+}

In this section, we present our CP-Guard+, a tailored defense method for CP scenarios that effectively detects malicious agents. It consists of two techniques: 1) Residual Latent Feature Learning, which learns the residual features of benign and malicious agents, and 2) Dual Centered Contrastive Loss (DCCLoss), which clusters the representation of benign features into a compact space and ensures that the representation of malicious features is as distant from the benign space as possible.


\subsection{Residual Latent Feature Learning}
In CP scenarios, the dynamic environment causes noisy, non-stationary data distributions. Directly detecting malicious agents can be challenging due to this noise, as object detectors' feature maps often mix foreground and background information.

To tackle this challenge, we propose to learn residual latent features instead of directly learning the features of benign or malicious agents. By focusing on the differences between the collaborators' feature maps and the ego agent's feature maps, the model can better distinguish between benign and malicious agents. This mechanism is also inspired by the idea that the collaborators' intermediate feature maps will achieve a consensus rather than conflict with the ego CAV's intermediate feature maps.

Specifically, given the collaborators' intermediate feature maps $\{\mathbf{F}_{j\rightarrow i}\}_{j\neq i,\  j\in \mathcal{X}^N}$ and the ego CAV's intermediate feature maps $\mathbf{F}_i$, the residual feature is obtained as: $\mathbf{F}^\mathtt{res}_{j\rightarrow i}= \mathbf{F}_i - \mathbf{F}_{j\rightarrow i}.$ 

Then, we can leverage the residual latent features to detect malicious agents by modeling the detection problem as a binary classification task. A binary classifier $f_{\mathtt{cls}}(\mathbf{x}; \theta)$ is trained on the residual latent features to distinguish between benign (labeled 0) and malicious (labeled 1) agents. The model is optimized using the cross-entropy loss $\mathcal{L}_\mathtt{CE}$:
\begin{equation}
    \min_\theta \mathcal{L}_\mathtt{CE} (f_{\mathtt{cls}}(\mathbf{F}^\mathtt{res}_{j\rightarrow i}; \theta), y_{j\rightarrow i})
\end{equation}
where $y_{j\rightarrow i}$ is the ground truth label of residual feature $\mathbf{F}^\mathtt{res}_{j\rightarrow i}$.


% \begin{equation}
% \mathcal{L}_{\text{res}} = -\frac{1}{N} \sum_{i = 1}^{N} (y_i \log(p_i) + (1 - y_i) \log(1 - p_i)),
% \end{equation}

% where $N$ is the number of samples, $y_i$ is the ground truth label, $p_i$ is the predicted probability, and $\theta$ is the classifier's parameters that need to be optimized.


% \begin{table*}[t]
%     \caption{\textbf{Performance Evaluation of CP-Guard+ on CP-GuardBench.} We report the average accuracy, true positive rate (TPR), false positive rate (FPR), precision, and F1 score of CP-Guard+ on CP-GuardBench with different attack methods and perturbation budgets $\Delta = 0.25, 0.5, 0.75, 1.0$. }
%     \label{tab:quantitative_results_cpguardbench}
%     \centering 
    
%     \renewcommand\arraystretch{1.0}
%     \resizebox{1\linewidth}{!}{
%     \small
%     \begin{tabular}{p{1.2cm}|ccccc|ccccc}
%         \toprule

%         \multirow{2}{*}{ \textbf{Metrics} } & \multicolumn{5}{c|}{$\Delta = 0.10$} & \multicolumn{5}{c}{$\Delta = 0.25$} \\
        
%         & Accuracy $\uparrow$ & TPR $\downarrow$  & FPR $\downarrow$ & Precision $\uparrow$ & F1 Score $\uparrow$ & Accuracy $\uparrow$ & TPR $\uparrow$ & FPR $\downarrow$ & Precision $\uparrow$ & F1 Score $\uparrow$ \\ \midrule

%         PGD&98.77&100.00&1.54&94.19&97.01&98.83&100.00&1.46&94.52&97.18\\
%         BIM&98.90&100.00&1.37&94.81&97.33&98.90&100.00&1.37&94.79&97.32\\
%         C\&W &98.60&99.30&1.58&94.02&96.59&97.96&100.00&2.56&90.38&95.19\\
        
%         FGSM&91.64&64.41&1.46&91.79&75.70&97.53&93.22&1.37&94.50&93.86\\
%         GN &90.95&60.34&1.29&92.23&72.95&97.19&92.12&1.54&93.73&92.92\\ \midrule 
%         Average &95.78&84.81&1.44&93.43&87.93&98.08&97.07&1.66&93.45&95.29\\ \bottomrule\toprule

%         \multirow{2}{*}{ \textbf{Metrics} } & \multicolumn{5}{c|}{$\Delta = 0.50$} & \multicolumn{5}{c}{$\Delta = 0.75$} \\
        
%         & Accuracy $\uparrow$ & TPR $\uparrow$ & FPR $\downarrow$  & Precision $\uparrow$ & F1 Score $\uparrow$ & Accuracy $\uparrow$ & TPR $\uparrow$ & FPR $\downarrow$ & Precision $\uparrow$ & F1 Score $\uparrow$ \\ \midrule

       
%         PGD&98.83&100.00&1.46&94.55&97.20&98.60&100.00&1.75&93.50&96.64\\
%         BIM&98.90&100.00&1.37&94.81&97.33&98.66&100.00&1.67&93.73&96.76\\
%         C\&W &97.30&100.00&3.41&88.46&93.88&96.43&100.00&4.42&84.34&91.50\\
%         FGSM&98.63&98.63&1.37&94.75&96.66&98.83&100.00&1.46&94.48&97.16\\
%         GN &98.49&98.27&1.45&94.35&96.27&98.90&99.96&1.28&95.08&97.32\\ \midrule
%         Average &98.43&99.38&1.81&93.38&96.27&98.53&99.93&1.82&93.20&96.43\\ \bottomrule
%     \end{tabular}
%     }
%     \vspace{-5mm}
% \end{table*}

\begin{table*}[t]
    \caption{\textbf{Performance Evaluation of CP-Guard+ on CP-GuardBench.} We report the average accuracy, true positive rate (TPR), false positive rate (FPR), precision, and F1 score of CP-Guard+ on CP-GuardBench with different attack methods and perturbation budgets $\Delta = 0.10, 0.25, 0.5, 0.75$. }
    \label{tab:quantitative_results_cpguardbench}
    \centering 
    
    \renewcommand\arraystretch{1.0}
    \resizebox{1\linewidth}{!}{
    \small
    \begin{tabular}{p{1.2cm}|ccccc|ccccc}
        \toprule

        \multirow{2}{*}{ \textbf{Metrics} } & \multicolumn{5}{c|}{$\Delta = 0.10$} & \multicolumn{5}{c}{$\Delta = 0.25$} \\
        
        & Accuracy $\uparrow$ & TPR $\uparrow$  & FPR $\downarrow$ & Precision $\uparrow$ & F1 Score $\uparrow$ & Accuracy $\uparrow$ & TPR $\uparrow$ & FPR $\downarrow$ & Precision $\uparrow$ & F1 Score $\uparrow$ \\ \midrule

        PGD&99.79&98.97&0.00&100.00&99.48&99.93&100.00&0.09&99.66&99.83\\
        BIM&99.93&100.00&0.09&99.66&99.83&100.00&100.00&0.00&100.00&100.00\\
        C\&W &99.73&98.96&0.09&99.65&99.30&99.86&100.00&0.17&99.32&99.66\\
        
        FGSM&91.23&56.80&0.09&99.40&72.29&98.77&94.24&0.09&99.64&96.86\\
        GN &91.02&55.63&0.09&99.39&71.33&98.49&92.88&0.09&99.64&96.14\\ \midrule 
        Average &96.34&82.07&0.07&99.62&88.45&99.41&97.42&0.09&99.65&98.50\\ \bottomrule\toprule

        \multirow{2}{*}{ \textbf{Metrics} } & \multicolumn{5}{c|}{$\Delta = 0.50$} & \multicolumn{5}{c}{$\Delta = 0.75$} \\
        
        & Accuracy $\uparrow$ & TPR $\uparrow$ & FPR $\downarrow$  & Precision $\uparrow$ & F1 Score $\uparrow$ & Accuracy $\uparrow$ & TPR $\uparrow$ & FPR $\downarrow$ & Precision $\uparrow$ & F1 Score $\uparrow$ \\ \midrule

       
        PGD&99.86&100.00&0.17&99.32&99.66&99.86&100.00&0.17&99.33&99.66\\
        BIM&99.93&100.00&0.09&99.66&99.83&99.93&100.00&0.09&99.66&99.83\\
        C\&W &99.86&100.00&0.17&99.32&99.66&99.93&100.00&0.09&99.66&99.83\\
        FGSM&99.93&100.00&0.09&99.66&99.83&99.79&99.66&0.17&99.32&99.49\\
        GN &99.86&100.00&0.17&99.32&99.66&99.79&99.65&0.17&99.31&99.48\\ \midrule
        Average &99.89&100.00&0.14&99.46&99.73&99.86&99.86&0.14&99.46&99.66\\ \bottomrule
    \end{tabular}
    }
    \vspace{-5mm}
\end{table*}

\begin{table*}[t]
    \caption{\textbf{Comparative results of CP-Guard+ on the V2X-Sim Dataset.} We report the AP@0.5 and AP@0.7 with different perturbation budgets $\Delta$ and number of malicious agents $N_\text{mal}$. }
    \label{tab:quantitative_results_v2xsim}
    \centering 
    
    \renewcommand\arraystretch{0.9}
    \resizebox{1\linewidth}{!}{
    \small
    \begin{tabular}{c|c|cc|cc|cc|cc}
        \toprule
        % \multirow{2}{*}{ \textbf{Method} } & \multicolumn{8}{c}{ \textbf{IoU} } \\

        \multirow{2}{*}{ {\makecell{Attack\\Method}} }&\multirow{2}{*}{ {\makecell{Defense\\Method}} } & \multicolumn{2}{c|}{$\Delta = 0.25, N_\text{mal} = 1$ } & \multicolumn{2}{c|}{$\Delta = 0.5, N_\text{mal} = 1$} & \multicolumn{2}{c|}{$\Delta = 0.25, N_\text{mal} = 2$} & \multicolumn{2}{c}{$\Delta = 0.5, N_\text{mal} = 2$}\\
               && AP@0.5 & AP@0.7  & AP@0.5 & AP@0.7 & AP@0.5 & AP@0.7  & AP@0.5 & AP@0.7 \\
        \midrule
        \multirow{4}{*}{PGD } & No Defense & 29.73 & 28.47  & 11.35 & 11.17 & 12.69 & 12.42 & 1.69 & 1.65 \\
        & MADE & 64.63 & 45.22 & 64.81 & 44.89 & 62.45 & 43.49 & 63.04 & 43.77 \\
        & ROBOSAC & 62.13 & 42.90 & 63.67 & 43.79 & 59.01 & 40.03 & 59.97 & 40.44 \\
        & CP-Guard+ & \textbf{72.89} & \textbf{71.45} & \textbf{69.50} & \textbf{68.56} & \textbf{69.50} & \textbf{67.92} & \textbf{66.09} & \textbf{64.82} \\ 
        \midrule

        \multirow{4}{*}{C\&W } 
        & No Defense & 19.03 & 16.58 & 4.69 & 3.78 & 19.03 & 16.58 & 0.71 & 0.58 \\
        & MADE & 65.26 & 45.24 & \textbf{64.74} & 45.65 & 63.41 & 44.28  & \textbf{62.86} & 42.93 \\
        & ROBOSAC & 61.83 & 42.01 & 62.47 & 42.80 & 59.39 & 39.94  & 59.83 & 39.82 \\
        & CP-Guard+ & \textbf{69.41} & \textbf{66.86} & {60.64} & \textbf{55.41} & \textbf{64.17} & \textbf{61.73}  & {58.54} & \textbf{53.15} \\
        \midrule

        \multirow{4}{*}{BIM } 
        & No Defense & 26.69 & 25.71 & 10.05 & 9.89 & 11.59 & 11.38 & 1.37 & 1.33 \\ 
        & MADE & 66.11 & 45.94 & 65.51 & 45.47 &  64.36 & 43.89  & 63.56 & 44.09 \\ 
        & ROBOSAC & 62.69 & 43.80  & 63.78 & 43.66 & 59.10 & 39.74  & 59.29 & 39.89 \\ 
        & CP-Guard+ & \textbf{73.35} & \textbf{71.46} & \textbf{66.83} & \textbf{66.05} & \textbf{70.91} & \textbf{69.11}  & \textbf{66.30} & \textbf{64.62} \\ 
        \midrule

        \multirow{4}{*}{Average} 
        & No Defense & 25.15 & 23.59 & 8.70 & 8.28 & 14.44 & 13.46 & 1.27 & 1.19 \\ 
        & MADE & 65.33 & 45.47 & 65.02 & 45.34 & 63.41 & 43.89 & 63.15 & 43.60 \\ 
        & ROBOSAC & 62.21 & 42.90 & 63.31 & 43.42 & 59.37 & 39.90 & 59.70 & 40.05 \\ 
        & CP-Guard+ & \textbf{71.88} & \textbf{69.92} & \textbf{65.66} & \textbf{63.34} & \textbf{68.19} & \textbf{66.25} & \textbf{63.64} & \textbf{60.86}\\ 
        \midrule
        \multicolumn{2}{c|}{Upper-bound} & 79.94 & 78.40& 79.94 & 78.40 & 79.94 & 78.40 & 79.94 & 78.40 \\ 
        \bottomrule
    \end{tabular}
    }
    \vspace{-5mm}
\end{table*}


\subsection{Dual-Centered Contrastive Loss}

In practice, attackers can continuously design new attacks to manipulate the victim's intermediate feature maps. Therefore, we need a model that is resistant to unseen attacks, which means the model should cluster the representation of benign features into a more compact space and ensure that the representation of malicious features is as distant from the benign space as possible.

To tackle this challenge, we propose a Dual-Centered Contrastive Loss (DCCLoss), which is a contrastive learning-based objective function. It explicitly models the distribution relationship between benign and malicious features, enhancing the positive pairs (features of the same class) to their corresponding center closer, thereby enhancing the internal consistency of both benign and malicious features. Meanwhile, negative pairs (features from different classes) will be pushed away from each other's centers, ensuring maximal separation between benign and malicious features in the feature space. In this way, the robustness of the model against unseen attacks is improved.

% To achieve both feature compactness and separation simultaneously, we design a contrastive learning-based objective function. The core ideas of TCCLoss are clustering positive pairs (features of the same class) to their corresponding center closer, thereby enhancing the internal consistency of both benign and malicious features. Meanwhile, negative pairs (features from different classes) will be pushed away from each other's centers, ensuring maximal separation between benign and malicious features in the feature space.

Specifically, we first leverage the output of the penultimate fully connected layer $f_{\text{cls}}$ to obtain the one-dimensional vector $\{\mathcal{V}_i\}_{i=0,1,\ldots, N-1}$ of residual features $\{\mathbf{F}^\mathtt{res}_{j\rightarrow i}\}_{j\neq i,\  j\in \mathcal{X}^N}$. Then, we introduce two feature centers in DCCLoss: 
\begin{enumerate}
    \vspace{-3mm}
    \setlength{\itemsep}{0pt}
    \setlength{\parskip}{0pt}
    \setlength{\parsep}{0pt}
    % \item Global center ($\mathbf{c}_{\mathtt{glb}}$): Represents the overall distribution center of all features (both benign and malicious), capturing the global structure of the feature space. 
    % \vspace{-3mm}
    % \begin{equation}
    %     % \mathbf{c}_\mathtt{glb} = \frac{1}{N}\sum_{i=0}^{N} \frac{\mathcal{V}_i}{\|\mathcal{V}_i\|}
    %     \mathbf{c}_x = \frac{1}{N}\sum_{i=0}^{N}{\mathcal{V}_i}
    % \end{equation}
    % \vspace{-3mm}
    \item Benign feature center ($\mathbf{c}_{\mathtt{b}}$): represents the center of all benign features, ensuring a compact distribution of benign features.
    \item Malicious feature center ($\mathbf{c}_{\mathtt{mal}}$): represents the center of all malicious features, ensuring maximal separation between malicious and benign feature distributions.
    \vspace{-3mm}
    
    % \vspace{-3mm}
    % \item Malicious feature center ($\mathbf{c}_{\mathtt{mal}}$): Represents the center of all malicious features, ensuring maximal separation between malicious and benign feature distributions. Given the malicious vector set $\{\mathcal{V}_\mathtt{mal}\}$ and the number of malicious vectors $N_\mathtt{mal}$, the malicious feature center is computed as:
    % \vspace{-3mm}
    % \begin{equation}
    %     \mathbf{c}_\mathtt{mal} = \frac{1}{N_\mathtt{mal}}\sum_{\mathcal{V}_i\in \{\mathcal{V}_\mathtt{mal}\}}^{N_\mathtt{mal}} \mathcal{V}_i
    % \end{equation}
\end{enumerate}
The benign feature center $\mathbf{c}_\mathtt{b}$ and the malicious feature center $\mathbf{c}_\mathtt{mal}$ are computed by averaging the vectors of benign and malicious features, respectively:
\vspace{-2mm}
\begin{equation}
    \mathbf{c}_\mathtt{b} = \frac{1}{N_\mathtt{b}}\sum_{\mathcal{V}_i\in \{\mathcal{V}_\mathtt{b}\}}^{N_\mathtt{b}} \mathcal{V}_i, \quad  \mathbf{c}_\mathtt{mal} = \frac{1}{N_\mathtt{mal}}\sum_{\mathcal{V}_i\in \{\mathcal{V}_\mathtt{mal}\}}^{N_\mathtt{mal}} \mathcal{V}_i.
\end{equation}
where $N_\mathtt{b}$ and $N_\mathtt{mal}$ are the numbers of benign and malicious vectors, respectively, and $\mathcal{V}_\mathtt{b}$ and $\mathcal{V}_\mathtt{mal}$ are the sets of benign and malicious vectors, respectively.

Moreover, denote $(\mathcal{V}_m, \mathcal{V}_n)$ as a pair of features, which is a positive pair if they are from the same class (both benign or malicious) and a negative pair otherwise. We have the DCCLoss of one feature pair $\ell(\mathcal{V}_m, \mathcal{V}_n)$ as:
\begin{equation}
    \begin{aligned}
    &\ell\left(\mathcal{V}_m, \mathcal{V}_n\right)=-\log \\
    &\frac{\exp\left((\mathcal{V}_m-\mathbf{c}^{(m)})\odot  (\mathcal{V}_n-\mathbf{c}^{(n)})/\tau\right)}
        {\sum_{o=1, o\neq m}^{N} \mathbb{I}\cdot\exp\left((\mathcal{V}_m-\mathbf{c}^{(m)})\odot  (\mathcal{V}_o-\mathbf{c}^{(o)})/\tau\right)},\\
    & \text{where}\ \mathcal{V}_x - \mathbf{c}^{(x)}= 
    \left\{\begin{aligned}
        &\mathcal{V}_x - \mathbf{c}_\mathtt{b}, & \text{if } \mathcal{V}_x \in \{\mathcal{V}_\mathtt{b}\},\\ 
        &\mathcal{V}_x - \mathbf{c}_\mathtt{mal}, & \text{if } \mathcal{V}_x \in \{\mathcal{V}_\mathtt{mal}\},
    \end{aligned}\right.
    \end{aligned}
\end{equation}
where $\mathbb{I}(\mathcal{V}_m, \mathcal{V}_o)$ is an indicator function that returns one or zero for positive or negative pairs, respectively. $\tau$ is a temperature parameter and $\odot$ denotes the cosine similarity, where $\mathcal{V}_m\odot \mathcal{V}_n = \frac{\mathcal{V}_m^\top  \mathcal{V}_n}{\|\mathcal{V}_m\|\|\mathcal{V}_n\|}$.
The final DCCLoss is the average of $\ell$ of all positive pairs.
\begin{equation}
    \mathcal{L}_\mathtt{DCCLoss}= \sum_{m=1}^{N} \sum_{n=m+1}^{N} \frac{\left(1-\mathbb{I}(\mathcal{V}_m, \mathcal{V}_n)\right)\ell\left(\mathcal{V}_m, \mathcal{V}_n\right)}{C(N,2)},
    \label{eq:contrastive_loss}
\end{equation}
where $C(N,2)= \binom{N}{2}={N!}/({2!(N-2)!})$. 
During training, we use the combination of cross entropy loss and Eq. \ref{eq:contrastive_loss} to optimize the model:
\begin{equation}
    \mathcal{L} = \mathcal{L}_\mathtt{CE} + \alpha\cdot \mathcal{L}_\mathtt{DCCLoss}
    \label{eq:mixed_loss}
\end{equation}
where $\alpha$ is a hyperparameter to balance the two losses.

\textbf{Discussion of DCCLoss.} In so doing, the first term $\mathcal{L}_\mathtt{CE}$ quantifies the difference between the true distribution and the predicted distribution from the model, thereby penalizing the confidence in wrong predictions. The second term $\mathcal{L}_\mathtt{DCCLoss}$ contributes significantly to the learning process. Standard contrastive loss attempts to maximize the distance between negative pairs, which may cause the features of benign samples to gradually drift away from their center. However, DCCLoss calculates the distance using the feature center as a reference point, thus avoiding this issue. The optimization goal of DCCLoss is to maximize the angular distance between negative pairs to enhance feature discriminability while maintaining the compactness of benign sample features, keeping them as close to the feature center as possible. 
In other words, the introduction of dual-center modeling optimizes the distributional relationship between benign and malicious features, making them more separable and making the distributions of benign and malicious features more compact on its own, respectively. This helps resolve the distribution overlap problem and enhances the model's ability to detect unseen attacks.







\section{Experiments}
\label{sec:experiments}
The experiments are designed to address two key research questions.
First, \textbf{RQ1} evaluates whether the average $L_2$-norm of the counterfactual perturbation vectors ($\overline{||\perturb||}$) decreases as the model overfits the data, thereby providing further empirical validation for our hypothesis.
Second, \textbf{RQ2} evaluates the ability of the proposed counterfactual regularized loss, as defined in (\ref{eq:regularized_loss2}), to mitigate overfitting when compared to existing regularization techniques.

% The experiments are designed to address three key research questions. First, \textbf{RQ1} investigates whether the mean perturbation vector norm decreases as the model overfits the data, aiming to further validate our intuition. Second, \textbf{RQ2} explores whether the mean perturbation vector norm can be effectively leveraged as a regularization term during training, offering insights into its potential role in mitigating overfitting. Finally, \textbf{RQ3} examines whether our counterfactual regularizer enables the model to achieve superior performance compared to existing regularization methods, thus highlighting its practical advantage.

\subsection{Experimental Setup}
\textbf{\textit{Datasets, Models, and Tasks.}}
The experiments are conducted on three datasets: \textit{Water Potability}~\cite{kadiwal2020waterpotability}, \textit{Phomene}~\cite{phomene}, and \textit{CIFAR-10}~\cite{krizhevsky2009learning}. For \textit{Water Potability} and \textit{Phomene}, we randomly select $80\%$ of the samples for the training set, and the remaining $20\%$ for the test set, \textit{CIFAR-10} comes already split. Furthermore, we consider the following models: Logistic Regression, Multi-Layer Perceptron (MLP) with 100 and 30 neurons on each hidden layer, and PreactResNet-18~\cite{he2016cvecvv} as a Convolutional Neural Network (CNN) architecture.
We focus on binary classification tasks and leave the extension to multiclass scenarios for future work. However, for datasets that are inherently multiclass, we transform the problem into a binary classification task by selecting two classes, aligning with our assumption.

\smallskip
\noindent\textbf{\textit{Evaluation Measures.}} To characterize the degree of overfitting, we use the test loss, as it serves as a reliable indicator of the model's generalization capability to unseen data. Additionally, we evaluate the predictive performance of each model using the test accuracy.

\smallskip
\noindent\textbf{\textit{Baselines.}} We compare CF-Reg with the following regularization techniques: L1 (``Lasso''), L2 (``Ridge''), and Dropout.

\smallskip
\noindent\textbf{\textit{Configurations.}}
For each model, we adopt specific configurations as follows.
\begin{itemize}
\item \textit{Logistic Regression:} To induce overfitting in the model, we artificially increase the dimensionality of the data beyond the number of training samples by applying a polynomial feature expansion. This approach ensures that the model has enough capacity to overfit the training data, allowing us to analyze the impact of our counterfactual regularizer. The degree of the polynomial is chosen as the smallest degree that makes the number of features greater than the number of data.
\item \textit{Neural Networks (MLP and CNN):} To take advantage of the closed-form solution for computing the optimal perturbation vector as defined in (\ref{eq:opt-delta}), we use a local linear approximation of the neural network models. Hence, given an instance $\inst_i$, we consider the (optimal) counterfactual not with respect to $\model$ but with respect to:
\begin{equation}
\label{eq:taylor}
    \model^{lin}(\inst) = \model(\inst_i) + \nabla_{\inst}\model(\inst_i)(\inst - \inst_i),
\end{equation}
where $\model^{lin}$ represents the first-order Taylor approximation of $\model$ at $\inst_i$.
Note that this step is unnecessary for Logistic Regression, as it is inherently a linear model.
\end{itemize}

\smallskip
\noindent \textbf{\textit{Implementation Details.}} We run all experiments on a machine equipped with an AMD Ryzen 9 7900 12-Core Processor and an NVIDIA GeForce RTX 4090 GPU. Our implementation is based on the PyTorch Lightning framework. We use stochastic gradient descent as the optimizer with a learning rate of $\eta = 0.001$ and no weight decay. We use a batch size of $128$. The training and test steps are conducted for $6000$ epochs on the \textit{Water Potability} and \textit{Phoneme} datasets, while for the \textit{CIFAR-10} dataset, they are performed for $200$ epochs.
Finally, the contribution $w_i^{\varepsilon}$ of each training point $\inst_i$ is uniformly set as $w_i^{\varepsilon} = 1~\forall i\in \{1,\ldots,m\}$.

The source code implementation for our experiments is available at the following GitHub repository: \url{https://anonymous.4open.science/r/COCE-80B4/README.md} 

\subsection{RQ1: Counterfactual Perturbation vs. Overfitting}
To address \textbf{RQ1}, we analyze the relationship between the test loss and the average $L_2$-norm of the counterfactual perturbation vectors ($\overline{||\perturb||}$) over training epochs.

In particular, Figure~\ref{fig:delta_loss_epochs} depicts the evolution of $\overline{||\perturb||}$ alongside the test loss for an MLP trained \textit{without} regularization on the \textit{Water Potability} dataset. 
\begin{figure}[ht]
    \centering
    \includegraphics[width=0.85\linewidth]{img/delta_loss_epochs.png}
    \caption{The average counterfactual perturbation vector $\overline{||\perturb||}$ (left $y$-axis) and the cross-entropy test loss (right $y$-axis) over training epochs ($x$-axis) for an MLP trained on the \textit{Water Potability} dataset \textit{without} regularization.}
    \label{fig:delta_loss_epochs}
\end{figure}

The plot shows a clear trend as the model starts to overfit the data (evidenced by an increase in test loss). 
Notably, $\overline{||\perturb||}$ begins to decrease, which aligns with the hypothesis that the average distance to the optimal counterfactual example gets smaller as the model's decision boundary becomes increasingly adherent to the training data.

It is worth noting that this trend is heavily influenced by the choice of the counterfactual generator model. In particular, the relationship between $\overline{||\perturb||}$ and the degree of overfitting may become even more pronounced when leveraging more accurate counterfactual generators. However, these models often come at the cost of higher computational complexity, and their exploration is left to future work.

Nonetheless, we expect that $\overline{||\perturb||}$ will eventually stabilize at a plateau, as the average $L_2$-norm of the optimal counterfactual perturbations cannot vanish to zero.

% Additionally, the choice of employing the score-based counterfactual explanation framework to generate counterfactuals was driven to promote computational efficiency.

% Future enhancements to the framework may involve adopting models capable of generating more precise counterfactuals. While such approaches may yield to performance improvements, they are likely to come at the cost of increased computational complexity.


\subsection{RQ2: Counterfactual Regularization Performance}
To answer \textbf{RQ2}, we evaluate the effectiveness of the proposed counterfactual regularization (CF-Reg) by comparing its performance against existing baselines: unregularized training loss (No-Reg), L1 regularization (L1-Reg), L2 regularization (L2-Reg), and Dropout.
Specifically, for each model and dataset combination, Table~\ref{tab:regularization_comparison} presents the mean value and standard deviation of test accuracy achieved by each method across 5 random initialization. 

The table illustrates that our regularization technique consistently delivers better results than existing methods across all evaluated scenarios, except for one case -- i.e., Logistic Regression on the \textit{Phomene} dataset. 
However, this setting exhibits an unusual pattern, as the highest model accuracy is achieved without any regularization. Even in this case, CF-Reg still surpasses other regularization baselines.

From the results above, we derive the following key insights. First, CF-Reg proves to be effective across various model types, ranging from simple linear models (Logistic Regression) to deep architectures like MLPs and CNNs, and across diverse datasets, including both tabular and image data. 
Second, CF-Reg's strong performance on the \textit{Water} dataset with Logistic Regression suggests that its benefits may be more pronounced when applied to simpler models. However, the unexpected outcome on the \textit{Phoneme} dataset calls for further investigation into this phenomenon.


\begin{table*}[h!]
    \centering
    \caption{Mean value and standard deviation of test accuracy across 5 random initializations for different model, dataset, and regularization method. The best results are highlighted in \textbf{bold}.}
    \label{tab:regularization_comparison}
    \begin{tabular}{|c|c|c|c|c|c|c|}
        \hline
        \textbf{Model} & \textbf{Dataset} & \textbf{No-Reg} & \textbf{L1-Reg} & \textbf{L2-Reg} & \textbf{Dropout} & \textbf{CF-Reg (ours)} \\ \hline
        Logistic Regression   & \textit{Water}   & $0.6595 \pm 0.0038$   & $0.6729 \pm 0.0056$   & $0.6756 \pm 0.0046$  & N/A    & $\mathbf{0.6918 \pm 0.0036}$                     \\ \hline
        MLP   & \textit{Water}   & $0.6756 \pm 0.0042$   & $0.6790 \pm 0.0058$   & $0.6790 \pm 0.0023$  & $0.6750 \pm 0.0036$    & $\mathbf{0.6802 \pm 0.0046}$                    \\ \hline
%        MLP   & \textit{Adult}   & $0.8404 \pm 0.0010$   & $\mathbf{0.8495 \pm 0.0007}$   & $0.8489 \pm 0.0014$  & $\mathbf{0.8495 \pm 0.0016}$     & $0.8449 \pm 0.0019$                    \\ \hline
        Logistic Regression   & \textit{Phomene}   & $\mathbf{0.8148 \pm 0.0020}$   & $0.8041 \pm 0.0028$   & $0.7835 \pm 0.0176$  & N/A    & $0.8098 \pm 0.0055$                     \\ \hline
        MLP   & \textit{Phomene}   & $0.8677 \pm 0.0033$   & $0.8374 \pm 0.0080$   & $0.8673 \pm 0.0045$  & $0.8672 \pm 0.0042$     & $\mathbf{0.8718 \pm 0.0040}$                    \\ \hline
        CNN   & \textit{CIFAR-10} & $0.6670 \pm 0.0233$   & $0.6229 \pm 0.0850$   & $0.7348 \pm 0.0365$   & N/A    & $\mathbf{0.7427 \pm 0.0571}$                     \\ \hline
    \end{tabular}
\end{table*}

\begin{table*}[htb!]
    \centering
    \caption{Hyperparameter configurations utilized for the generation of Table \ref{tab:regularization_comparison}. For our regularization the hyperparameters are reported as $\mathbf{\alpha/\beta}$.}
    \label{tab:performance_parameters}
    \begin{tabular}{|c|c|c|c|c|c|c|}
        \hline
        \textbf{Model} & \textbf{Dataset} & \textbf{No-Reg} & \textbf{L1-Reg} & \textbf{L2-Reg} & \textbf{Dropout} & \textbf{CF-Reg (ours)} \\ \hline
        Logistic Regression   & \textit{Water}   & N/A   & $0.0093$   & $0.6927$  & N/A    & $0.3791/1.0355$                     \\ \hline
        MLP   & \textit{Water}   & N/A   & $0.0007$   & $0.0022$  & $0.0002$    & $0.2567/1.9775$                    \\ \hline
        Logistic Regression   &
        \textit{Phomene}   & N/A   & $0.0097$   & $0.7979$  & N/A    & $0.0571/1.8516$                     \\ \hline
        MLP   & \textit{Phomene}   & N/A   & $0.0007$   & $4.24\cdot10^{-5}$  & $0.0015$    & $0.0516/2.2700$                    \\ \hline
       % MLP   & \textit{Adult}   & N/A   & $0.0018$   & $0.0018$  & $0.0601$     & $0.0764/2.2068$                    \\ \hline
        CNN   & \textit{CIFAR-10} & N/A   & $0.0050$   & $0.0864$ & N/A    & $0.3018/
        2.1502$                     \\ \hline
    \end{tabular}
\end{table*}

\begin{table*}[htb!]
    \centering
    \caption{Mean value and standard deviation of training time across 5 different runs. The reported time (in seconds) corresponds to the generation of each entry in Table \ref{tab:regularization_comparison}. Times are }
    \label{tab:times}
    \begin{tabular}{|c|c|c|c|c|c|c|}
        \hline
        \textbf{Model} & \textbf{Dataset} & \textbf{No-Reg} & \textbf{L1-Reg} & \textbf{L2-Reg} & \textbf{Dropout} & \textbf{CF-Reg (ours)} \\ \hline
        Logistic Regression   & \textit{Water}   & $222.98 \pm 1.07$   & $239.94 \pm 2.59$   & $241.60 \pm 1.88$  & N/A    & $251.50 \pm 1.93$                     \\ \hline
        MLP   & \textit{Water}   & $225.71 \pm 3.85$   & $250.13 \pm 4.44$   & $255.78 \pm 2.38$  & $237.83 \pm 3.45$    & $266.48 \pm 3.46$                    \\ \hline
        Logistic Regression   & \textit{Phomene}   & $266.39 \pm 0.82$ & $367.52 \pm 6.85$   & $361.69 \pm 4.04$  & N/A   & $310.48 \pm 0.76$                    \\ \hline
        MLP   &
        \textit{Phomene} & $335.62 \pm 1.77$   & $390.86 \pm 2.11$   & $393.96 \pm 1.95$ & $363.51 \pm 5.07$    & $403.14 \pm 1.92$                     \\ \hline
       % MLP   & \textit{Adult}   & N/A   & $0.0018$   & $0.0018$  & $0.0601$     & $0.0764/2.2068$                    \\ \hline
        CNN   & \textit{CIFAR-10} & $370.09 \pm 0.18$   & $395.71 \pm 0.55$   & $401.38 \pm 0.16$ & N/A    & $1287.8 \pm 0.26$                     \\ \hline
    \end{tabular}
\end{table*}

\subsection{Feasibility of our Method}
A crucial requirement for any regularization technique is that it should impose minimal impact on the overall training process.
In this respect, CF-Reg introduces an overhead that depends on the time required to find the optimal counterfactual example for each training instance. 
As such, the more sophisticated the counterfactual generator model probed during training the higher would be the time required. However, a more advanced counterfactual generator might provide a more effective regularization. We discuss this trade-off in more details in Section~\ref{sec:discussion}.

Table~\ref{tab:times} presents the average training time ($\pm$ standard deviation) for each model and dataset combination listed in Table~\ref{tab:regularization_comparison}.
We can observe that the higher accuracy achieved by CF-Reg using the score-based counterfactual generator comes with only minimal overhead. However, when applied to deep neural networks with many hidden layers, such as \textit{PreactResNet-18}, the forward derivative computation required for the linearization of the network introduces a more noticeable computational cost, explaining the longer training times in the table.

\subsection{Hyperparameter Sensitivity Analysis}
The proposed counterfactual regularization technique relies on two key hyperparameters: $\alpha$ and $\beta$. The former is intrinsic to the loss formulation defined in (\ref{eq:cf-train}), while the latter is closely tied to the choice of the score-based counterfactual explanation method used.

Figure~\ref{fig:test_alpha_beta} illustrates how the test accuracy of an MLP trained on the \textit{Water Potability} dataset changes for different combinations of $\alpha$ and $\beta$.

\begin{figure}[ht]
    \centering
    \includegraphics[width=0.85\linewidth]{img/test_acc_alpha_beta.png}
    \caption{The test accuracy of an MLP trained on the \textit{Water Potability} dataset, evaluated while varying the weight of our counterfactual regularizer ($\alpha$) for different values of $\beta$.}
    \label{fig:test_alpha_beta}
\end{figure}

We observe that, for a fixed $\beta$, increasing the weight of our counterfactual regularizer ($\alpha$) can slightly improve test accuracy until a sudden drop is noticed for $\alpha > 0.1$.
This behavior was expected, as the impact of our penalty, like any regularization term, can be disruptive if not properly controlled.

Moreover, this finding further demonstrates that our regularization method, CF-Reg, is inherently data-driven. Therefore, it requires specific fine-tuning based on the combination of the model and dataset at hand.
\section{Conclusion}
In this work, we propose a simple yet effective approach, called SMILE, for graph few-shot learning with fewer tasks. Specifically, we introduce a novel dual-level mixup strategy, including within-task and across-task mixup, for enriching the diversity of nodes within each task and the diversity of tasks. Also, we incorporate the degree-based prior information to learn expressive node embeddings. Theoretically, we prove that SMILE effectively enhances the model's generalization performance. Empirically, we conduct extensive experiments on multiple benchmarks and the results suggest that SMILE significantly outperforms other baselines, including both in-domain and cross-domain few-shot settings.


\bibliography{ref, ref2}
\bibliographystyle{icml2025}


\newpage
\appendix
% \subsection{Lloyd-Max Algorithm}
\label{subsec:Lloyd-Max}
For a given quantization bitwidth $B$ and an operand $\bm{X}$, the Lloyd-Max algorithm finds $2^B$ quantization levels $\{\hat{x}_i\}_{i=1}^{2^B}$ such that quantizing $\bm{X}$ by rounding each scalar in $\bm{X}$ to the nearest quantization level minimizes the quantization MSE. 

The algorithm starts with an initial guess of quantization levels and then iteratively computes quantization thresholds $\{\tau_i\}_{i=1}^{2^B-1}$ and updates quantization levels $\{\hat{x}_i\}_{i=1}^{2^B}$. Specifically, at iteration $n$, thresholds are set to the midpoints of the previous iteration's levels:
\begin{align*}
    \tau_i^{(n)}=\frac{\hat{x}_i^{(n-1)}+\hat{x}_{i+1}^{(n-1)}}2 \text{ for } i=1\ldots 2^B-1
\end{align*}
Subsequently, the quantization levels are re-computed as conditional means of the data regions defined by the new thresholds:
\begin{align*}
    \hat{x}_i^{(n)}=\mathbb{E}\left[ \bm{X} \big| \bm{X}\in [\tau_{i-1}^{(n)},\tau_i^{(n)}] \right] \text{ for } i=1\ldots 2^B
\end{align*}
where to satisfy boundary conditions we have $\tau_0=-\infty$ and $\tau_{2^B}=\infty$. The algorithm iterates the above steps until convergence.

Figure \ref{fig:lm_quant} compares the quantization levels of a $7$-bit floating point (E3M3) quantizer (left) to a $7$-bit Lloyd-Max quantizer (right) when quantizing a layer of weights from the GPT3-126M model at a per-tensor granularity. As shown, the Lloyd-Max quantizer achieves substantially lower quantization MSE. Further, Table \ref{tab:FP7_vs_LM7} shows the superior perplexity achieved by Lloyd-Max quantizers for bitwidths of $7$, $6$ and $5$. The difference between the quantizers is clear at 5 bits, where per-tensor FP quantization incurs a drastic and unacceptable increase in perplexity, while Lloyd-Max quantization incurs a much smaller increase. Nevertheless, we note that even the optimal Lloyd-Max quantizer incurs a notable ($\sim 1.5$) increase in perplexity due to the coarse granularity of quantization. 

\begin{figure}[h]
  \centering
  \includegraphics[width=0.7\linewidth]{sections/figures/LM7_FP7.pdf}
  \caption{\small Quantization levels and the corresponding quantization MSE of Floating Point (left) vs Lloyd-Max (right) Quantizers for a layer of weights in the GPT3-126M model.}
  \label{fig:lm_quant}
\end{figure}

\begin{table}[h]\scriptsize
\begin{center}
\caption{\label{tab:FP7_vs_LM7} \small Comparing perplexity (lower is better) achieved by floating point quantizers and Lloyd-Max quantizers on a GPT3-126M model for the Wikitext-103 dataset.}
\begin{tabular}{c|cc|c}
\hline
 \multirow{2}{*}{\textbf{Bitwidth}} & \multicolumn{2}{|c|}{\textbf{Floating-Point Quantizer}} & \textbf{Lloyd-Max Quantizer} \\
 & Best Format & Wikitext-103 Perplexity & Wikitext-103 Perplexity \\
\hline
7 & E3M3 & 18.32 & 18.27 \\
6 & E3M2 & 19.07 & 18.51 \\
5 & E4M0 & 43.89 & 19.71 \\
\hline
\end{tabular}
\end{center}
\end{table}

\subsection{Proof of Local Optimality of LO-BCQ}
\label{subsec:lobcq_opt_proof}
For a given block $\bm{b}_j$, the quantization MSE during LO-BCQ can be empirically evaluated as $\frac{1}{L_b}\lVert \bm{b}_j- \bm{\hat{b}}_j\rVert^2_2$ where $\bm{\hat{b}}_j$ is computed from equation (\ref{eq:clustered_quantization_definition}) as $C_{f(\bm{b}_j)}(\bm{b}_j)$. Further, for a given block cluster $\mathcal{B}_i$, we compute the quantization MSE as $\frac{1}{|\mathcal{B}_{i}|}\sum_{\bm{b} \in \mathcal{B}_{i}} \frac{1}{L_b}\lVert \bm{b}- C_i^{(n)}(\bm{b})\rVert^2_2$. Therefore, at the end of iteration $n$, we evaluate the overall quantization MSE $J^{(n)}$ for a given operand $\bm{X}$ composed of $N_c$ block clusters as:
\begin{align*}
    \label{eq:mse_iter_n}
    J^{(n)} = \frac{1}{N_c} \sum_{i=1}^{N_c} \frac{1}{|\mathcal{B}_{i}^{(n)}|}\sum_{\bm{v} \in \mathcal{B}_{i}^{(n)}} \frac{1}{L_b}\lVert \bm{b}- B_i^{(n)}(\bm{b})\rVert^2_2
\end{align*}

At the end of iteration $n$, the codebooks are updated from $\mathcal{C}^{(n-1)}$ to $\mathcal{C}^{(n)}$. However, the mapping of a given vector $\bm{b}_j$ to quantizers $\mathcal{C}^{(n)}$ remains as  $f^{(n)}(\bm{b}_j)$. At the next iteration, during the vector clustering step, $f^{(n+1)}(\bm{b}_j)$ finds new mapping of $\bm{b}_j$ to updated codebooks $\mathcal{C}^{(n)}$ such that the quantization MSE over the candidate codebooks is minimized. Therefore, we obtain the following result for $\bm{b}_j$:
\begin{align*}
\frac{1}{L_b}\lVert \bm{b}_j - C_{f^{(n+1)}(\bm{b}_j)}^{(n)}(\bm{b}_j)\rVert^2_2 \le \frac{1}{L_b}\lVert \bm{b}_j - C_{f^{(n)}(\bm{b}_j)}^{(n)}(\bm{b}_j)\rVert^2_2
\end{align*}

That is, quantizing $\bm{b}_j$ at the end of the block clustering step of iteration $n+1$ results in lower quantization MSE compared to quantizing at the end of iteration $n$. Since this is true for all $\bm{b} \in \bm{X}$, we assert the following:
\begin{equation}
\begin{split}
\label{eq:mse_ineq_1}
    \tilde{J}^{(n+1)} &= \frac{1}{N_c} \sum_{i=1}^{N_c} \frac{1}{|\mathcal{B}_{i}^{(n+1)}|}\sum_{\bm{b} \in \mathcal{B}_{i}^{(n+1)}} \frac{1}{L_b}\lVert \bm{b} - C_i^{(n)}(b)\rVert^2_2 \le J^{(n)}
\end{split}
\end{equation}
where $\tilde{J}^{(n+1)}$ is the the quantization MSE after the vector clustering step at iteration $n+1$.

Next, during the codebook update step (\ref{eq:quantizers_update}) at iteration $n+1$, the per-cluster codebooks $\mathcal{C}^{(n)}$ are updated to $\mathcal{C}^{(n+1)}$ by invoking the Lloyd-Max algorithm \citep{Lloyd}. We know that for any given value distribution, the Lloyd-Max algorithm minimizes the quantization MSE. Therefore, for a given vector cluster $\mathcal{B}_i$ we obtain the following result:

\begin{equation}
    \frac{1}{|\mathcal{B}_{i}^{(n+1)}|}\sum_{\bm{b} \in \mathcal{B}_{i}^{(n+1)}} \frac{1}{L_b}\lVert \bm{b}- C_i^{(n+1)}(\bm{b})\rVert^2_2 \le \frac{1}{|\mathcal{B}_{i}^{(n+1)}|}\sum_{\bm{b} \in \mathcal{B}_{i}^{(n+1)}} \frac{1}{L_b}\lVert \bm{b}- C_i^{(n)}(\bm{b})\rVert^2_2
\end{equation}

The above equation states that quantizing the given block cluster $\mathcal{B}_i$ after updating the associated codebook from $C_i^{(n)}$ to $C_i^{(n+1)}$ results in lower quantization MSE. Since this is true for all the block clusters, we derive the following result: 
\begin{equation}
\begin{split}
\label{eq:mse_ineq_2}
     J^{(n+1)} &= \frac{1}{N_c} \sum_{i=1}^{N_c} \frac{1}{|\mathcal{B}_{i}^{(n+1)}|}\sum_{\bm{b} \in \mathcal{B}_{i}^{(n+1)}} \frac{1}{L_b}\lVert \bm{b}- C_i^{(n+1)}(\bm{b})\rVert^2_2  \le \tilde{J}^{(n+1)}   
\end{split}
\end{equation}

Following (\ref{eq:mse_ineq_1}) and (\ref{eq:mse_ineq_2}), we find that the quantization MSE is non-increasing for each iteration, that is, $J^{(1)} \ge J^{(2)} \ge J^{(3)} \ge \ldots \ge J^{(M)}$ where $M$ is the maximum number of iterations. 
%Therefore, we can say that if the algorithm converges, then it must be that it has converged to a local minimum. 
\hfill $\blacksquare$


\begin{figure}
    \begin{center}
    \includegraphics[width=0.5\textwidth]{sections//figures/mse_vs_iter.pdf}
    \end{center}
    \caption{\small NMSE vs iterations during LO-BCQ compared to other block quantization proposals}
    \label{fig:nmse_vs_iter}
\end{figure}

Figure \ref{fig:nmse_vs_iter} shows the empirical convergence of LO-BCQ across several block lengths and number of codebooks. Also, the MSE achieved by LO-BCQ is compared to baselines such as MXFP and VSQ. As shown, LO-BCQ converges to a lower MSE than the baselines. Further, we achieve better convergence for larger number of codebooks ($N_c$) and for a smaller block length ($L_b$), both of which increase the bitwidth of BCQ (see Eq \ref{eq:bitwidth_bcq}).


\subsection{Additional Accuracy Results}
%Table \ref{tab:lobcq_config} lists the various LOBCQ configurations and their corresponding bitwidths.
\begin{table}
\setlength{\tabcolsep}{4.75pt}
\begin{center}
\caption{\label{tab:lobcq_config} Various LO-BCQ configurations and their bitwidths.}
\begin{tabular}{|c||c|c|c|c||c|c||c|} 
\hline
 & \multicolumn{4}{|c||}{$L_b=8$} & \multicolumn{2}{|c||}{$L_b=4$} & $L_b=2$ \\
 \hline
 \backslashbox{$L_A$\kern-1em}{\kern-1em$N_c$} & 2 & 4 & 8 & 16 & 2 & 4 & 2 \\
 \hline
 64 & 4.25 & 4.375 & 4.5 & 4.625 & 4.375 & 4.625 & 4.625\\
 \hline
 32 & 4.375 & 4.5 & 4.625& 4.75 & 4.5 & 4.75 & 4.75 \\
 \hline
 16 & 4.625 & 4.75& 4.875 & 5 & 4.75 & 5 & 5 \\
 \hline
\end{tabular}
\end{center}
\end{table}

%\subsection{Perplexity achieved by various LO-BCQ configurations on Wikitext-103 dataset}

\begin{table} \centering
\begin{tabular}{|c||c|c|c|c||c|c||c|} 
\hline
 $L_b \rightarrow$& \multicolumn{4}{c||}{8} & \multicolumn{2}{c||}{4} & 2\\
 \hline
 \backslashbox{$L_A$\kern-1em}{\kern-1em$N_c$} & 2 & 4 & 8 & 16 & 2 & 4 & 2  \\
 %$N_c \rightarrow$ & 2 & 4 & 8 & 16 & 2 & 4 & 2 \\
 \hline
 \hline
 \multicolumn{8}{c}{GPT3-1.3B (FP32 PPL = 9.98)} \\ 
 \hline
 \hline
 64 & 10.40 & 10.23 & 10.17 & 10.15 &  10.28 & 10.18 & 10.19 \\
 \hline
 32 & 10.25 & 10.20 & 10.15 & 10.12 &  10.23 & 10.17 & 10.17 \\
 \hline
 16 & 10.22 & 10.16 & 10.10 & 10.09 &  10.21 & 10.14 & 10.16 \\
 \hline
  \hline
 \multicolumn{8}{c}{GPT3-8B (FP32 PPL = 7.38)} \\ 
 \hline
 \hline
 64 & 7.61 & 7.52 & 7.48 &  7.47 &  7.55 &  7.49 & 7.50 \\
 \hline
 32 & 7.52 & 7.50 & 7.46 &  7.45 &  7.52 &  7.48 & 7.48  \\
 \hline
 16 & 7.51 & 7.48 & 7.44 &  7.44 &  7.51 &  7.49 & 7.47  \\
 \hline
\end{tabular}
\caption{\label{tab:ppl_gpt3_abalation} Wikitext-103 perplexity across GPT3-1.3B and 8B models.}
\end{table}

\begin{table} \centering
\begin{tabular}{|c||c|c|c|c||} 
\hline
 $L_b \rightarrow$& \multicolumn{4}{c||}{8}\\
 \hline
 \backslashbox{$L_A$\kern-1em}{\kern-1em$N_c$} & 2 & 4 & 8 & 16 \\
 %$N_c \rightarrow$ & 2 & 4 & 8 & 16 & 2 & 4 & 2 \\
 \hline
 \hline
 \multicolumn{5}{|c|}{Llama2-7B (FP32 PPL = 5.06)} \\ 
 \hline
 \hline
 64 & 5.31 & 5.26 & 5.19 & 5.18  \\
 \hline
 32 & 5.23 & 5.25 & 5.18 & 5.15  \\
 \hline
 16 & 5.23 & 5.19 & 5.16 & 5.14  \\
 \hline
 \multicolumn{5}{|c|}{Nemotron4-15B (FP32 PPL = 5.87)} \\ 
 \hline
 \hline
 64  & 6.3 & 6.20 & 6.13 & 6.08  \\
 \hline
 32  & 6.24 & 6.12 & 6.07 & 6.03  \\
 \hline
 16  & 6.12 & 6.14 & 6.04 & 6.02  \\
 \hline
 \multicolumn{5}{|c|}{Nemotron4-340B (FP32 PPL = 3.48)} \\ 
 \hline
 \hline
 64 & 3.67 & 3.62 & 3.60 & 3.59 \\
 \hline
 32 & 3.63 & 3.61 & 3.59 & 3.56 \\
 \hline
 16 & 3.61 & 3.58 & 3.57 & 3.55 \\
 \hline
\end{tabular}
\caption{\label{tab:ppl_llama7B_nemo15B} Wikitext-103 perplexity compared to FP32 baseline in Llama2-7B and Nemotron4-15B, 340B models}
\end{table}

%\subsection{Perplexity achieved by various LO-BCQ configurations on MMLU dataset}


\begin{table} \centering
\begin{tabular}{|c||c|c|c|c||c|c|c|c|} 
\hline
 $L_b \rightarrow$& \multicolumn{4}{c||}{8} & \multicolumn{4}{c||}{8}\\
 \hline
 \backslashbox{$L_A$\kern-1em}{\kern-1em$N_c$} & 2 & 4 & 8 & 16 & 2 & 4 & 8 & 16  \\
 %$N_c \rightarrow$ & 2 & 4 & 8 & 16 & 2 & 4 & 2 \\
 \hline
 \hline
 \multicolumn{5}{|c|}{Llama2-7B (FP32 Accuracy = 45.8\%)} & \multicolumn{4}{|c|}{Llama2-70B (FP32 Accuracy = 69.12\%)} \\ 
 \hline
 \hline
 64 & 43.9 & 43.4 & 43.9 & 44.9 & 68.07 & 68.27 & 68.17 & 68.75 \\
 \hline
 32 & 44.5 & 43.8 & 44.9 & 44.5 & 68.37 & 68.51 & 68.35 & 68.27  \\
 \hline
 16 & 43.9 & 42.7 & 44.9 & 45 & 68.12 & 68.77 & 68.31 & 68.59  \\
 \hline
 \hline
 \multicolumn{5}{|c|}{GPT3-22B (FP32 Accuracy = 38.75\%)} & \multicolumn{4}{|c|}{Nemotron4-15B (FP32 Accuracy = 64.3\%)} \\ 
 \hline
 \hline
 64 & 36.71 & 38.85 & 38.13 & 38.92 & 63.17 & 62.36 & 63.72 & 64.09 \\
 \hline
 32 & 37.95 & 38.69 & 39.45 & 38.34 & 64.05 & 62.30 & 63.8 & 64.33  \\
 \hline
 16 & 38.88 & 38.80 & 38.31 & 38.92 & 63.22 & 63.51 & 63.93 & 64.43  \\
 \hline
\end{tabular}
\caption{\label{tab:mmlu_abalation} Accuracy on MMLU dataset across GPT3-22B, Llama2-7B, 70B and Nemotron4-15B models.}
\end{table}


%\subsection{Perplexity achieved by various LO-BCQ configurations on LM evaluation harness}

\begin{table} \centering
\begin{tabular}{|c||c|c|c|c||c|c|c|c|} 
\hline
 $L_b \rightarrow$& \multicolumn{4}{c||}{8} & \multicolumn{4}{c||}{8}\\
 \hline
 \backslashbox{$L_A$\kern-1em}{\kern-1em$N_c$} & 2 & 4 & 8 & 16 & 2 & 4 & 8 & 16  \\
 %$N_c \rightarrow$ & 2 & 4 & 8 & 16 & 2 & 4 & 2 \\
 \hline
 \hline
 \multicolumn{5}{|c|}{Race (FP32 Accuracy = 37.51\%)} & \multicolumn{4}{|c|}{Boolq (FP32 Accuracy = 64.62\%)} \\ 
 \hline
 \hline
 64 & 36.94 & 37.13 & 36.27 & 37.13 & 63.73 & 62.26 & 63.49 & 63.36 \\
 \hline
 32 & 37.03 & 36.36 & 36.08 & 37.03 & 62.54 & 63.51 & 63.49 & 63.55  \\
 \hline
 16 & 37.03 & 37.03 & 36.46 & 37.03 & 61.1 & 63.79 & 63.58 & 63.33  \\
 \hline
 \hline
 \multicolumn{5}{|c|}{Winogrande (FP32 Accuracy = 58.01\%)} & \multicolumn{4}{|c|}{Piqa (FP32 Accuracy = 74.21\%)} \\ 
 \hline
 \hline
 64 & 58.17 & 57.22 & 57.85 & 58.33 & 73.01 & 73.07 & 73.07 & 72.80 \\
 \hline
 32 & 59.12 & 58.09 & 57.85 & 58.41 & 73.01 & 73.94 & 72.74 & 73.18  \\
 \hline
 16 & 57.93 & 58.88 & 57.93 & 58.56 & 73.94 & 72.80 & 73.01 & 73.94  \\
 \hline
\end{tabular}
\caption{\label{tab:mmlu_abalation} Accuracy on LM evaluation harness tasks on GPT3-1.3B model.}
\end{table}

\begin{table} \centering
\begin{tabular}{|c||c|c|c|c||c|c|c|c|} 
\hline
 $L_b \rightarrow$& \multicolumn{4}{c||}{8} & \multicolumn{4}{c||}{8}\\
 \hline
 \backslashbox{$L_A$\kern-1em}{\kern-1em$N_c$} & 2 & 4 & 8 & 16 & 2 & 4 & 8 & 16  \\
 %$N_c \rightarrow$ & 2 & 4 & 8 & 16 & 2 & 4 & 2 \\
 \hline
 \hline
 \multicolumn{5}{|c|}{Race (FP32 Accuracy = 41.34\%)} & \multicolumn{4}{|c|}{Boolq (FP32 Accuracy = 68.32\%)} \\ 
 \hline
 \hline
 64 & 40.48 & 40.10 & 39.43 & 39.90 & 69.20 & 68.41 & 69.45 & 68.56 \\
 \hline
 32 & 39.52 & 39.52 & 40.77 & 39.62 & 68.32 & 67.43 & 68.17 & 69.30  \\
 \hline
 16 & 39.81 & 39.71 & 39.90 & 40.38 & 68.10 & 66.33 & 69.51 & 69.42  \\
 \hline
 \hline
 \multicolumn{5}{|c|}{Winogrande (FP32 Accuracy = 67.88\%)} & \multicolumn{4}{|c|}{Piqa (FP32 Accuracy = 78.78\%)} \\ 
 \hline
 \hline
 64 & 66.85 & 66.61 & 67.72 & 67.88 & 77.31 & 77.42 & 77.75 & 77.64 \\
 \hline
 32 & 67.25 & 67.72 & 67.72 & 67.00 & 77.31 & 77.04 & 77.80 & 77.37  \\
 \hline
 16 & 68.11 & 68.90 & 67.88 & 67.48 & 77.37 & 78.13 & 78.13 & 77.69  \\
 \hline
\end{tabular}
\caption{\label{tab:mmlu_abalation} Accuracy on LM evaluation harness tasks on GPT3-8B model.}
\end{table}

\begin{table} \centering
\begin{tabular}{|c||c|c|c|c||c|c|c|c|} 
\hline
 $L_b \rightarrow$& \multicolumn{4}{c||}{8} & \multicolumn{4}{c||}{8}\\
 \hline
 \backslashbox{$L_A$\kern-1em}{\kern-1em$N_c$} & 2 & 4 & 8 & 16 & 2 & 4 & 8 & 16  \\
 %$N_c \rightarrow$ & 2 & 4 & 8 & 16 & 2 & 4 & 2 \\
 \hline
 \hline
 \multicolumn{5}{|c|}{Race (FP32 Accuracy = 40.67\%)} & \multicolumn{4}{|c|}{Boolq (FP32 Accuracy = 76.54\%)} \\ 
 \hline
 \hline
 64 & 40.48 & 40.10 & 39.43 & 39.90 & 75.41 & 75.11 & 77.09 & 75.66 \\
 \hline
 32 & 39.52 & 39.52 & 40.77 & 39.62 & 76.02 & 76.02 & 75.96 & 75.35  \\
 \hline
 16 & 39.81 & 39.71 & 39.90 & 40.38 & 75.05 & 73.82 & 75.72 & 76.09  \\
 \hline
 \hline
 \multicolumn{5}{|c|}{Winogrande (FP32 Accuracy = 70.64\%)} & \multicolumn{4}{|c|}{Piqa (FP32 Accuracy = 79.16\%)} \\ 
 \hline
 \hline
 64 & 69.14 & 70.17 & 70.17 & 70.56 & 78.24 & 79.00 & 78.62 & 78.73 \\
 \hline
 32 & 70.96 & 69.69 & 71.27 & 69.30 & 78.56 & 79.49 & 79.16 & 78.89  \\
 \hline
 16 & 71.03 & 69.53 & 69.69 & 70.40 & 78.13 & 79.16 & 79.00 & 79.00  \\
 \hline
\end{tabular}
\caption{\label{tab:mmlu_abalation} Accuracy on LM evaluation harness tasks on GPT3-22B model.}
\end{table}

\begin{table} \centering
\begin{tabular}{|c||c|c|c|c||c|c|c|c|} 
\hline
 $L_b \rightarrow$& \multicolumn{4}{c||}{8} & \multicolumn{4}{c||}{8}\\
 \hline
 \backslashbox{$L_A$\kern-1em}{\kern-1em$N_c$} & 2 & 4 & 8 & 16 & 2 & 4 & 8 & 16  \\
 %$N_c \rightarrow$ & 2 & 4 & 8 & 16 & 2 & 4 & 2 \\
 \hline
 \hline
 \multicolumn{5}{|c|}{Race (FP32 Accuracy = 44.4\%)} & \multicolumn{4}{|c|}{Boolq (FP32 Accuracy = 79.29\%)} \\ 
 \hline
 \hline
 64 & 42.49 & 42.51 & 42.58 & 43.45 & 77.58 & 77.37 & 77.43 & 78.1 \\
 \hline
 32 & 43.35 & 42.49 & 43.64 & 43.73 & 77.86 & 75.32 & 77.28 & 77.86  \\
 \hline
 16 & 44.21 & 44.21 & 43.64 & 42.97 & 78.65 & 77 & 76.94 & 77.98  \\
 \hline
 \hline
 \multicolumn{5}{|c|}{Winogrande (FP32 Accuracy = 69.38\%)} & \multicolumn{4}{|c|}{Piqa (FP32 Accuracy = 78.07\%)} \\ 
 \hline
 \hline
 64 & 68.9 & 68.43 & 69.77 & 68.19 & 77.09 & 76.82 & 77.09 & 77.86 \\
 \hline
 32 & 69.38 & 68.51 & 68.82 & 68.90 & 78.07 & 76.71 & 78.07 & 77.86  \\
 \hline
 16 & 69.53 & 67.09 & 69.38 & 68.90 & 77.37 & 77.8 & 77.91 & 77.69  \\
 \hline
\end{tabular}
\caption{\label{tab:mmlu_abalation} Accuracy on LM evaluation harness tasks on Llama2-7B model.}
\end{table}

\begin{table} \centering
\begin{tabular}{|c||c|c|c|c||c|c|c|c|} 
\hline
 $L_b \rightarrow$& \multicolumn{4}{c||}{8} & \multicolumn{4}{c||}{8}\\
 \hline
 \backslashbox{$L_A$\kern-1em}{\kern-1em$N_c$} & 2 & 4 & 8 & 16 & 2 & 4 & 8 & 16  \\
 %$N_c \rightarrow$ & 2 & 4 & 8 & 16 & 2 & 4 & 2 \\
 \hline
 \hline
 \multicolumn{5}{|c|}{Race (FP32 Accuracy = 48.8\%)} & \multicolumn{4}{|c|}{Boolq (FP32 Accuracy = 85.23\%)} \\ 
 \hline
 \hline
 64 & 49.00 & 49.00 & 49.28 & 48.71 & 82.82 & 84.28 & 84.03 & 84.25 \\
 \hline
 32 & 49.57 & 48.52 & 48.33 & 49.28 & 83.85 & 84.46 & 84.31 & 84.93  \\
 \hline
 16 & 49.85 & 49.09 & 49.28 & 48.99 & 85.11 & 84.46 & 84.61 & 83.94  \\
 \hline
 \hline
 \multicolumn{5}{|c|}{Winogrande (FP32 Accuracy = 79.95\%)} & \multicolumn{4}{|c|}{Piqa (FP32 Accuracy = 81.56\%)} \\ 
 \hline
 \hline
 64 & 78.77 & 78.45 & 78.37 & 79.16 & 81.45 & 80.69 & 81.45 & 81.5 \\
 \hline
 32 & 78.45 & 79.01 & 78.69 & 80.66 & 81.56 & 80.58 & 81.18 & 81.34  \\
 \hline
 16 & 79.95 & 79.56 & 79.79 & 79.72 & 81.28 & 81.66 & 81.28 & 80.96  \\
 \hline
\end{tabular}
\caption{\label{tab:mmlu_abalation} Accuracy on LM evaluation harness tasks on Llama2-70B model.}
\end{table}

%\section{MSE Studies}
%\textcolor{red}{TODO}


\subsection{Number Formats and Quantization Method}
\label{subsec:numFormats_quantMethod}
\subsubsection{Integer Format}
An $n$-bit signed integer (INT) is typically represented with a 2s-complement format \citep{yao2022zeroquant,xiao2023smoothquant,dai2021vsq}, where the most significant bit denotes the sign.

\subsubsection{Floating Point Format}
An $n$-bit signed floating point (FP) number $x$ comprises of a 1-bit sign ($x_{\mathrm{sign}}$), $B_m$-bit mantissa ($x_{\mathrm{mant}}$) and $B_e$-bit exponent ($x_{\mathrm{exp}}$) such that $B_m+B_e=n-1$. The associated constant exponent bias ($E_{\mathrm{bias}}$) is computed as $(2^{{B_e}-1}-1)$. We denote this format as $E_{B_e}M_{B_m}$.  

\subsubsection{Quantization Scheme}
\label{subsec:quant_method}
A quantization scheme dictates how a given unquantized tensor is converted to its quantized representation. We consider FP formats for the purpose of illustration. Given an unquantized tensor $\bm{X}$ and an FP format $E_{B_e}M_{B_m}$, we first, we compute the quantization scale factor $s_X$ that maps the maximum absolute value of $\bm{X}$ to the maximum quantization level of the $E_{B_e}M_{B_m}$ format as follows:
\begin{align}
\label{eq:sf}
    s_X = \frac{\mathrm{max}(|\bm{X}|)}{\mathrm{max}(E_{B_e}M_{B_m})}
\end{align}
In the above equation, $|\cdot|$ denotes the absolute value function.

Next, we scale $\bm{X}$ by $s_X$ and quantize it to $\hat{\bm{X}}$ by rounding it to the nearest quantization level of $E_{B_e}M_{B_m}$ as:

\begin{align}
\label{eq:tensor_quant}
    \hat{\bm{X}} = \text{round-to-nearest}\left(\frac{\bm{X}}{s_X}, E_{B_e}M_{B_m}\right)
\end{align}

We perform dynamic max-scaled quantization \citep{wu2020integer}, where the scale factor $s$ for activations is dynamically computed during runtime.

\subsection{Vector Scaled Quantization}
\begin{wrapfigure}{r}{0.35\linewidth}
  \centering
  \includegraphics[width=\linewidth]{sections/figures/vsquant.jpg}
  \caption{\small Vectorwise decomposition for per-vector scaled quantization (VSQ \citep{dai2021vsq}).}
  \label{fig:vsquant}
\end{wrapfigure}
During VSQ \citep{dai2021vsq}, the operand tensors are decomposed into 1D vectors in a hardware friendly manner as shown in Figure \ref{fig:vsquant}. Since the decomposed tensors are used as operands in matrix multiplications during inference, it is beneficial to perform this decomposition along the reduction dimension of the multiplication. The vectorwise quantization is performed similar to tensorwise quantization described in Equations \ref{eq:sf} and \ref{eq:tensor_quant}, where a scale factor $s_v$ is required for each vector $\bm{v}$ that maps the maximum absolute value of that vector to the maximum quantization level. While smaller vector lengths can lead to larger accuracy gains, the associated memory and computational overheads due to the per-vector scale factors increases. To alleviate these overheads, VSQ \citep{dai2021vsq} proposed a second level quantization of the per-vector scale factors to unsigned integers, while MX \citep{rouhani2023shared} quantizes them to integer powers of 2 (denoted as $2^{INT}$).

\subsubsection{MX Format}
The MX format proposed in \citep{rouhani2023microscaling} introduces the concept of sub-block shifting. For every two scalar elements of $b$-bits each, there is a shared exponent bit. The value of this exponent bit is determined through an empirical analysis that targets minimizing quantization MSE. We note that the FP format $E_{1}M_{b}$ is strictly better than MX from an accuracy perspective since it allocates a dedicated exponent bit to each scalar as opposed to sharing it across two scalars. Therefore, we conservatively bound the accuracy of a $b+2$-bit signed MX format with that of a $E_{1}M_{b}$ format in our comparisons. For instance, we use E1M2 format as a proxy for MX4.

\begin{figure}
    \centering
    \includegraphics[width=1\linewidth]{sections//figures/BlockFormats.pdf}
    \caption{\small Comparing LO-BCQ to MX format.}
    \label{fig:block_formats}
\end{figure}

Figure \ref{fig:block_formats} compares our $4$-bit LO-BCQ block format to MX \citep{rouhani2023microscaling}. As shown, both LO-BCQ and MX decompose a given operand tensor into block arrays and each block array into blocks. Similar to MX, we find that per-block quantization ($L_b < L_A$) leads to better accuracy due to increased flexibility. While MX achieves this through per-block $1$-bit micro-scales, we associate a dedicated codebook to each block through a per-block codebook selector. Further, MX quantizes the per-block array scale-factor to E8M0 format without per-tensor scaling. In contrast during LO-BCQ, we find that per-tensor scaling combined with quantization of per-block array scale-factor to E4M3 format results in superior inference accuracy across models. 


\end{document}


% This document was modified from the file originally made available by
% Pat Langley and Andrea Danyluk for ICML-2K. This version was created
% by Iain Murray in 2018, and modified by Alexandre Bouchard in
% 2019 and 2021 and by Csaba Szepesvari, Gang Niu and Sivan Sabato in 2022.
% Modified again in 2023 and 2024 by Sivan Sabato and Jonathan Scarlett.
% Previous contributors include Dan Roy, Lise Getoor and Tobias
% Scheffer, which was slightly modified from the 2010 version by
% Thorsten Joachims & Johannes Fuernkranz, slightly modified from the
% 2009 version by Kiri Wagstaff and Sam Roweis's 2008 version, which is
% slightly modified from Prasad Tadepalli's 2007 version which is a
% lightly changed version of the previous year's version by Andrew
% Moore, which was in turn edited from those of Kristian Kersting and
% Codrina Lauth. Alex Smola contributed to the algorithmic style files.
