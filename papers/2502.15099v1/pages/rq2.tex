
\begin{figure*}[t]
    \centering
    \vspace{10pt}
     \includegraphics[scale=0.3]{figures/roles.png}
    \caption{Examples of Images in Different Functional Roles}
    \label{fig:roles}
%\vspace{-15pt}
\end{figure*}



\subsection{Functional Roles of Images (RQ2)}
\label{rq2}

%Textual step-to-reproduce (S2R)  are widely used in bug reproduction tasks due to their clear and explicit depiction of actions and target UI elements at each step. Many studies have found S2R valuable and designed techniques to extract the most crucial information from the S2Rs, which are actions and target, and then used the extracted action-target  to facilicate bug reproduction.. When it comes to images, our primary goal is to explore the functional roles and characteristics of images within bug reports. Also, We aim to identify what specific types of information within images are most useful, and which elements should be focused on and extracted to leverage them  in assisting with automated bug reproduction effectively.

%
%The methodology for classifying image roles in bug reports primarily relies on their placement within the report, with contextual relevance serving as a secondary criterion. For instance, an image located under a section titled “Observed Behavior” would be classified as an OB image, without needing to delve deeply into its contextual details. Contextual analysis is applied when a clearly defined section title is absent or when an image may need to be further classified into subcategories. This method ensures that each image is categorized based on its location and contribution to understanding the bug, providing a structured way to interpret visual information in the report.

%
Bug reports typically consist of several key components that describe an issue, each serving a specific purpose: Steps to Reproduce (S2Rs), Observed Behavior (OB), and Expected Behavior (EB)~\cite{chaparroAssessingQualitySteps2019,chaparroDetectingMissingInformation2017}. %S2Rs specify the actions needed to replicate the issue, OB details the symptoms observed, and EB outlines the intended behavior of the application. Thus
Following this structure, images in bug reports can be classified into four main categories: \textit{S2R}, \textit{OB}, \textit{EB}, and \textit{Others}. The \textit{Others} category includes images that do not fit into the first three categories (e.g., screenshots of code). 
%
Additionally, we found that images in \textit{S2R} category could be further categorized into three distinct types based on their characteristics: \textit{S2R$_{standalone}$}, \textit{S2R$_{context}$}, and \textit{S2R$_{outcome}$}.  
Just like textual descriptions in bug reports, bug reproduction tools that can correctly classify images into their respective functional roles are essential for successful bug reproduction.


\subsubsection{
\textbf{\textit{S2R$_{standalone}$ image}}} 
An \textit{S2R\_standalone image} is a self-contained visual representation of an S2R that requires no accompanying text. 
%
These images are typically annotated to emphasize target user interface (UI) elements and the actions to be performed. 
%
For instance, in Fig. ~\ref{fig:roles}-(a), the first S2R is depicted entirely through an annotated image: the target element "an.db" is circled in red, and the action "click" is clearly indicated next to it. This effectively communicates the reproduction step "click on an.db." Images that fully capture an entire step-to-reproduce (S2R) without any accompanying text are important for effective bug reproduction. 

Neglecting such images could lead to missing steps, a major factor impacting the performance of automated bug reproduction processes. %This issue remains a significant challenge that existing tools are striving to address~\cite{??}. Therefore, paying attention to this type of image is essential for developing comprehensive and flexible bug reproduction tools.


\subsubsection{
\textbf{\textit{S2R$_{context}$ image}}} 
An \textit{S2R$_{context}$ image}  complements textual S2R instructions by providing additional visual context. 
Typically, it accompanies text in a pattern, such as "textual S2R + image". 
The textual S2R offers the primary guidance for reproducing the step, while the image adds crucial visual cues, forming a complete S2R together. 
%
These images often highlight target UI elements or required text inputs (e.g. Figure~\ref{fig:roles}-(b)).
%
Users tend to include images to depict target elements that are difficult to describe, especially when the element is uncommon or lacks a clear label—such as a menu icon or a "more options" button. 
%
When images are used to show required text input, it often indicates that the input is complex or specially formatted, not just simple text that can be easily typed in the bug report.
%
For example, in the bug report shown in Figure~\ref{fig:roles}-(b)), the third S2R states, “Click on the tabs display point,” followed by an annotated image that indicates which UI element is the "tabs display point," helping the developer quickly and accurately locate the target element. %Figure Figure~\ref{fig:roles}-(x) presents an example where the required text input for bug reproduction is provided directly in the image.

Such images reduce ambiguity in S2R instructions by providing visual cues of the target, enhancing clarity and understanding. This addresses another main challenge in existing work, where the bug report written by the user might be ambiguous and not clear due to their writing habits and lack of technical background.


\subsubsection{
\textbf{\textit{S2R$_{outcome}$ image}}} 
This type of image captures the state of the UI after a S2R has been replayed.  Unlike other S2R images that either represent S2Rs or are used to complement S2Rs, \textit{S2R$_{outcome}$ images} focus solely on displaying the results following the execution of the S2R.
%
In manual bug reproduction, an \textit{S2R$_{outcome}$ image} is primarily used to verify whether an S2R has been successfully replayed by comparing the image with the actual UI page. This visual confirmation allows developers to ensure that the action produced the expected outcome, thereby validating the accuracy of the reproduction step.
%

However, in automated bug reproduction, existing tools typically do not perform any verification of S2R replaying.  Instead, these tools rely solely on the presence of the target of the next S2R to advance through the reproduction steps. For example, given two steps, $S1$ and $S2$, after performing $S1$, the tool searches for the target element of the subsequent S2R, $S2$, within the current UI state. If the element for $S2$ is not found, the tool may backtrack to a previous UI state to attempt replaying $S1$ again. If the target element is found, the tool proceeds to replay $S2$ and continues with the subsequent steps.
%
This approach assumes that the mere presence of next target elements is sufficient to determine the success of each step. Consequently, the utility and necessity of \textit{S2R$_{outcome}$ images}  in automated bug reproduction remain unexplored. While these images undeniably offer valuable visual confirmation in manual processes, their integration into automated systems has not been considered. Determining whether incorporating \textit{S2R$_{outcome}$ images} would enhance automated bug reproduction requires experimental validation and further investigation. 
%Figure~\ref{fig:roles}-(c)  shows a bug report where the second step clearly indicates the action "scroll" and the endpoint which is sufficient to carry out the action. The screenshot further aids in understanding the result.
%\commentty{Describe how does it differ from S2R in designing bug reproduction approaches. I assume S2R\_result images can be used to check whether a S2R is replayed successfully.}
\begin{tcolorbox}[colback=blue!5, colframe=black, boxrule=0.5pt]
\textbf{Finding 3:} 
%
S2R images fall into three subcategories—context, standalone, and outcome—based on how they convey information. S2R$_{context}$ and S2R$_{standalone}$ images reduce ambiguity by clarifying reproduction steps and providing visual cues to support textual guidance. 
S2R$_{outcome}$ images provide essential visual confirmation in bug reproduction but are currently overlooked by existing tools.
\end{tcolorbox}


\subsubsection{
\textbf{\textit{OB image}}} 
\textit{OB image} visually illustrates the reported bug symptom,  validating the existence of the bug. This saves developers from spending time questioning the existence of the bug when they are unable to reproduce it.
%
From the perspective of bug reproduction, OB images serve dual purposes: visually illustrating bug symptoms and aiding the step replay. OB images clearly convey the bug symptoms, helping developers comprehend the issue and recognize when the bug is triggered during manual reproduction. Moreover, OB images provide additional context and visual cues to guide navigation to the target page where bugs occur. 

For example, the images in Figure~\ref{fig:roles}-(d) shows OB images, illustrating the non-crash bug symptoms. 
%
In  Figure~\ref{fig:roles}-(d), the bug report shows the OB image with short description but not   S2Rs for reproducing the bug. Nevertheless, the image provides useful context: aside from displaying the overlap bug symptom, the image reveals that this issue occurred on the calculator page. The digits shown on the screen imply that to replicate the overlap, it might be necessary to input an equation in the calculator. Thus, beyond highlighting the bug symptom, \textit{OB images} offers clues about the specific page where the overlap took place.


To be clear, although \textit{OB images} and \textit{S2R$_{outcome}$ images} share similarities, as both capture the UI results of specific steps in the reproduction process, the key distinction lies in whether the image displays the reported bug symptom. An \textit{OB image} is essentially a subset of \textit{S2R$_{outcome}$ images}, specifically capturing the result of the final S2R where the bug symptom is visible. In contrast, \textit{S2R\_outcome images} may depict the result of any step along the reproduction process and do not show the bug symptom. Therefore, if an image is an \textit{S2R\_result images} of the last S2R, it is classified as an \textit{OB images} rather than an \textit{S2R\_outcome images}.


\subsubsection{
\textbf{\textit{EB image}}} 
An \textit{EB image}  provides a visual reference of the expected or correct behavior,  often captured from a previous version of the application. In our dataset, we found that bugs related to issues like padding, text alignment, and color settings (including dark and day modes) frequently include EB images to help developers visually compare and identify the discrepancies. Additionally, EB images often appear alongside OB images for direct comparison, as illustrated in Figure~\ref{fig:roles}-(e).

Despite this benefit, existing automated tools ignore not only EB images but also any textual information related to EB. This is because EB information serves as a reference for how the application should behave, rather than explicitly showing a failure. This implicit nature makes it challenging for automated tools to utilize, unlike S2Rs (which provide direct instructions for reproducing the bug) or OB images (which display the exact symptoms of the issue). 
%
However, we should not overlook the potential of leveraging EB images to enhance automated bug reproduction. One possible integration is to use EB images in conjunction with OB images for automated oracle checking. By comparing EB images (representing correct behavior) with OB images (displaying bug symptoms) and the actual UI page, automated tools can more effectively identify discrepancies. Another approach is to view the EB image as the inverse of the OB image, suggesting that the EB image also describes the same UI page. This perspective allows EB images to implicitly provide information about the UI page (e.g., the target state) where the bug occurs. 

\subsubsection{
\textbf{\textit{Others image}}} 
These images can be classified as follows: (i) non-UI images, such as screenshots of source code or error logs; (ii) combined screenshots, which are single images created by merging multiple screenshots, making them challenging to interpret due to the diverse information they contain and the lack of clear context, and the difficulty of separating them; and (iii) UI images that are not directly related to OB, EB, or S2R, such as a screenshot of the settings screen that simply shows the application's configuration (e.g., Figure~\ref{fig:roles}-(f)).
%
Although not directly illustrating the bug, these images (e.g., logs) can be valuable by revealing settings or conditions impacting the application’s behavior.


%\commentty{This kind of images could also be useful, .e.g, logs. Need to discuss.}


\begin{table}[h]
%\vspace{-10pt}
\centering
\caption{Distribution of Image Roles Across Single-Image and Multiple-Image Bug Reports}
\label{tab:rq2}
\small
\begin{tabular}{|l|c|c|c|}
\hline
 \rowcolor{gray!45}     Image Roles       & Sgl-Img BR & Multi-Img BR &  \textbf{Overall}\\ \hline\hline
 
\textit{S2R\_standalone}   & 0  & 2.60\% &  0.82\%\\ \hline
\textit{S2R\_context}   & 1.79\%  &  20.78\%  &   7.76\%\\ \hline
\textit{S2R\_outcome}   & 1.19\%  & 10.39\% &  4.08\% \\ \hline
\textit{OB}  &  93.45\% &  90.91\% & 92.65\% \\ \hline
\textit{EB}  &  2.38\%  &  46.75\% & 16.33\% \\ \hline
\textit{Others}   & 1.19\%  & 20.78\% & 7.35\% \\ \hline


\end{tabular}
\vspace{-10pt}
\end{table}


Table~\ref{tab:rq2} shows distribution of image roles
across image bug reports. The percentages refer to the proportion of bug reports containing images in each functional role, rather than the percentage of each image type over the total number of images.
%
In the Overall column, the majority of bug reports(92.65\%) contain at least one image serving the OB role, followed by EB images and S2R images. This distribution emphasizes the popularity of using images to visually present observed and expected outcomes, likely due to the variety of non-crash symptoms and expected behaviors that are more effectively conveyed through visuals rather than textual descriptions. Images provide a clearer representation of the issues and the desired results, especially in cases where complex visual elements are involved, making them a valuable resource in the bug reporting and reproduction process.

We further examine the functional roles of images in bug reports by differentiating between those that contain a single image and those with multiple images.  We hypothesize that this separation will aid in classifying images into their respective functional roles more effectively. For instance, if most bug reports with a single image predominantly feature OB images, it would be simpler to categorize these images accordingly. As the result, the last two columns of Table~\ref{tab:rq2} suggest the following:

First, \textit{OB} are dominant in both single-image and multiple-image bug reports, with 93.45\% of single-image reports and 90.91\% of multiple-image reports containing at least one OB image. This highlights the popularity of using images to illustrate observed behavior, underscoring their value in effectively communicating the symptoms of a bug. Additionally, we can infer that when a bug report includes only one image, it is most likely to be an \textit{OB} image, suggesting that capturing the observed behavior is often the primary purpose of single-image bug reports.

Second, apart from \textit{OB} images, the proportion of all other roles increases significantly in multiple-image reports. This can be attributed to two main reasons: (1) when a bug report includes more than one image, there is greater diversity in the roles these images can fulfill, and (2) in multiple-image bug reports, each image can serve a different role, leading to overlapping classifications. Consequently, the cumulative percentage for each role exceeds 100\% as individual reports can contain multiple images fulfilling multiple functional roles. This highlights how multiple-image bug reports are often more comprehensive, capturing various aspects of the bug, such as observed behavior, expected behavior, and contextual information, which collectively enhance the clarity and reproducibility of the bug report.

%Third, S2R images also have a notable presence, with S2R\_standalone accounting for 7.76\% and S2R\_context for 4.08\%. These types of images are more likely to appear in multiple-image reports, as the steps required to trigger a bug are rarely captured in a single step.


\begin{tcolorbox}[colback=blue!5, colframe=black, boxrule=0.5pt]
\textbf{Finding 4:} 
%
First, the majority of bug reports (92.65\%) contain at least one OB image. This trend is especially strong in single-image reports, where 93.45\% are OB images, suggesting that when only one image is included, it is most likely to represent the observed issue. 
%
Second, multiple-image reports contain diverse images serving different roles, which implies that if future work aims to leverage these images to facilitate automated bug reproduction, it will be crucial to understand how to identify and differentiate these roles effectively.
\end{tcolorbox}








%

 

%Table~\ref{tab:roles} shows the distribution of images in different roles. The majority of the images are OB images. The majority of the images are OB images. The majority of the images are OB images. The majority of the images are OB images.

%\subsection{Representativeness of RQ1}
%\label{represent}
%\sidong{Are these datasets also been evaluated in RQ2, if so, I suggest adding this section to Section 2.1. If not, I still suggest moving this to the beginning of Section 3.1.}
%To validate the representativeness of our study, we analyzed two third-party datasets of Android app bugs, AndroR2~\cite{wendland2021andror2, johnson2022empirical} and RegDroid~\cite{xiong2023empirical}. The two datasets were not constructed by the authors of this paper, and neither of them focuses on bug reports that contain images. 
%AndroR2 focuses on manually reproduced bug reports, including 80 reproducible bug reports—50 crash reports and 130 non-crash reports. RegDroid, on the other hand, studies functional bugs in Android apps, analyzing 399 non-crash reports.




%Our analysis involved two primary steps. First, we determined the percentage of bug reports in each dataset that included images. Second, we applied the same methodologies used in RQ1.1 to label the roles of the images and in RQ1.2 to assess their placements. We then compared the results to those in our own dataset to determine if the findings from the third-party datasets align with ours. This analysis will help assess the representativeness of our findings.


