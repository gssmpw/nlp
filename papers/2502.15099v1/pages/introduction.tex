\section{Introduction}
\label{sec:intro}

%Automated bug reproduction significantly speeds up the identification and reproduction of bugs, reducing the time required for manual testing and debugging.

In the mobile app marketplace, debugging and problem resolution are critical. Research shows that 88\% of app users are likely to abandon an app if they encounter persistent issues, emphasizing the importance of prompt resolution to retain users\cite{applause}. One significant challenge developers face is reproducing bugs reported by users, who often provide insufficient information, such as the exact sequence of their interactions\cite{johnson2022empirical, moran2015auto, bettenburg2008makes, ambriola1997processing}. To tackle this issue, the software engineering community is increasingly focused on automating the bug reproduction process~\cite{fazzini2018automatically, zhao2019recdroid, zhang2024mobile, zhang2023automatically, feng2024prompting, wang2024feedback, huang2023context, huang2024crashtranslator, feng2022gifdroid}.



Bug reproduction relies on the information provided in bug reports to replicate reported issues. 
%
From a functional perspective, a bug report typically includes essential components such as Steps to Reproduce (S2R), Expected Behavior (EB), and Observed Behavior (OB).  
%
In terms of media types, it may feature text descriptions, images, and videos/GIFs. 
%
In this paper, \textit{images} refer to UI page \emph{screenshots} or any \emph{static visual information} provided by the user, which may be beneficial for bug reproduction process.

Table~\ref{table:toolsummary} summarizes the recent state-of-the-art automated bug reproduction tools, highlighting the specific bug report information they utilize and the main techniques employed in their approaches. 
Textual Steps to Reproduce (S2Rs) stand out as the most commonly used information employed by the majority of the works~\cite{fazzini2018automatically, zhao2019recdroid, zhang2024mobile, zhang2023automatically, feng2024prompting}, as they provide the most straightforward instructions needed to trigger the bug. However, other efforts~\cite{wang2024feedback, feng2022gifdroid, huang2024crashtranslator} have shown that information beyond textual S2Rs is critical for bug reproduction, yielding promising results. For instance, Feng et al.~\cite{feng2022gifdroid,feng2023read,feng2022gifdroid1} employ videos and GIFs in bug reports for bug reproduction, while Huang et al.~\cite{huang2024crashtranslator} utilize error log. Similarly, Wang et al.~\cite{huang2023context} leverages the entire textual bug report, including the title and other elements, demonstrating that each piece of information in bug reports can potentially supplement the S2Rs and improve the performance of automated bug reproduction.

\begin{table}[t]
\vspace{10pt}
\centering
\begin{threeparttable}
\caption{\textbf{Summary of Recent Android Bug Reproduction Works}}
\label{table:toolsummary}
\scriptsize
\begin{tabular}{|l||r|c|p{0.8cm}|p{0.825cm}|}
\hline
%\rowcolor{gray!45} 
\textbf{Tool Name} & \textbf{Venue} & \textbf{Input}  & \textbf{Open Source} & \textbf{Dataset Size} \\ \hline \hline

\textbf{Yakusu}~\cite{fazzini2018automatically}   &  ISSTA' 18  & S2Rs        & $\checkmark$ & 60 \\ \hline
\textcolor{red}{$\star$} \textbf{ReCDroid}~\cite{zhao2019recdroid}    &  ICSE' 19  &    S2Rs       &   $\checkmark$    & 51 \\ \hline
%\textbf{MaCa} ~\cite{lee2022light} &  ASE' 22  &   S2Rs        &    Machine Learning   & $\checkmark$  &\\ \hline
\textbf{GIFdroid}~\cite{feng2022gifdroid}&  ICSE' 22  &    GIF \textbackslash Videos      &   $\checkmark$ &   61 \\ \hline
\textcolor{red}{$\star$} \textbf{ReproBot}~\cite{zhang2023automatically}   &  ISSTA' 23  &  S2Rs         & $\checkmark$ & 77  \\ \hline
\textbf{ScopeDroid}~\cite{huang2023context}    & ICSE' 23   &   S2Rs    &  $\times$   &  102   \\ \hline
\textcolor{red}{$\star$} \textbf{AdbGPT}~\cite{feng2024prompting}   &  ICSE' 24  &    S2Rs         & $\checkmark$ &  48   \\ \hline
\textbf{CrashTranslator}~\cite{huang2024crashtranslator}   & ICSE' 24   &  Error Log       & $\checkmark$   & 75  \\ \hline
%\textcolor{red}{$\star$} 
\textbf{Roam}~\cite{zhang2024mobile}   & FSE' 24   &   S2Rs      & $\checkmark$ &   72    \\ \hline
\textcolor{red}{$\star$} \textbf{ReBL}~\cite{wang2024feedback}   & ISSTA' 24   &   Whole bug report        & $\checkmark$ &  96  \\ \hline
% Add more rows as needed
\end{tabular}
\begin{tablenotes}
\footnotesize
\item[1.] \textcolor{red}{$\star$} indicates the tool is selected as baseline for this study.
\end{tablenotes}
\end{threeparttable}
%\caption{Summary of Existing Bug Reproduction Works}
%\label{table:toolsummary}
\vspace{-10pt}
\end{table}

%Figure~\ref{fig:br} shows a real-world bug report that includes two images which provide essential context for illustrating the issues and guiding the reproduction. 
However, none of these approaches consider the images provided in bug reports, and the potential for leveraging these images to assist in automated bug reproduction remains unexplored, even though many bug reports include images. For instance, RegDroid~\cite{xiong2023empirical}, which is the most comprehensive dataset of Android functional bugs, features images in 41.35\% (165/399) of its bug reports. 
Similarly, AndroR2~\cite{johnson2022empirical, wendland2021andror2}, the widely-studied dataset for Android bug reports, contains images in 23.33\% (42/180) of its reports. 
%Notably, although the authors of AndroR2 did not explicitly analyze the impact of images on manual bug reproduction, we found that certain bug reports would be challenging to manually reproduce without considering the information provided by images.
%7\% (3/50) crash reports and 30\% (39/130) non-crash. 
%AndroR2 is the largest publicly available collection of manually reproducible Android bug reports, which is widely adopted by existing bug reproduction tools~\cite{feng2024prompting, zhang2024mobile, huang2024crashtranslator, wang2024feedback, huang2023context, zhang2023automatically} for evaluation.
%
%\commentdbw{compared with GIfDroid, define images in bug reports, difference between gif and static image and possible funcitonaliteis of images}
%Additionally, we crawled 95,281 bug reports using the GitHub API, of which 16,310 included at least one image. Among these reports, 17.02\% contained at least one image.
% 
To address this gap, we conducted an empirical study to analyze images in bug reports. Our study is guided by the following primary research questions:


%Steps to reproduce can offer explicit target actions to be extracted for reproduction,  videos convey the entire process of interactions leading to the bug, logcat exception messages clearly indicate the activity where the bug occurred. In contrast, the types of images in bug reports are diverse—they are not limited to UI screenshots but may include diagrams and logs, which complicates automated interpretation. Furthermore, even if the image is a UI screenshot, it captures only a single state in the bug reproduction process. This raises the question: what is the role of such image serving in bug report and what information should be extracted from such screenshots to assist in reproduction? To address this gap, we conducted an empirical study analyzing bug reports containing images. Our study is guided by the following primary research questions:

%\commentzx{the following sentence seems duplicate the previous paragraph}
%While images are frequently present in bug reports and might provide essential information, existing automated bug reproduction tools do not take them into account.  This raises several key questions: What information do images in bug reports provide? Are they necessary for automated bug reproduction? If so, how can they be used effectively to enhance current tools or inspire the development of new approaches?
%To address this gap, we conducted an empirical study that analyzes bug reports containing images. Our study is guided by two primary research questions:

\noindent
\textbf{RQ1: How do the number and types of images vary across bug reports?}
%
This research question addresses both quantitative (e.g., single-image vs. multiple-image reports) and categorical (e.g., UI screenshots vs. non-UI screenshots) aspects of image usage in bug reports. 
%
Understanding these patterns is essential for assessing how images contribute to bug reports. This foundation can inform the design of automated bug reproduction tools by identifying image patterns that are likely to be effective in automation or pinpointing specific types of images with the greatest potential to enhance automation processes.

%By analyzing the number of images per report, we aim to uncover patterns such as the prevalence of single versus multiple-image reports and calculate the average number of images across the dataset. Additionally, examining image types—for example, screenshots compared to other visual formats—allows us to categorize how different images are utilized in reporting.

%reveal trends that enhance the understanding of images in bug reports.


%In contrast, plain images without any accompanying explanations may require additional effort to interpret, potentially leading to confusion and inefficiency. 
%

%\emph{Documentation} refers to whether images are accompanied by text or annotations that provide explanation and clarity. Plain images without accompanying text or annotations are considered poorly documented, as understanding their context or related information requires reviewing the entire bug report, making the process less efficient and more prone to confusion.
%\emph{Documentation} concernsthe placement of images within bug reports, considering whether images are positioned correctly in relation to the corresponding textual descriptions. The focus is on how misalignment may hinder the effective interpretation of images and the subsequent challenge it poses for automated tools to utilize these visuals in reproducing bugs accurately.
\begin{figure*}
    \centering
    \includegraphics[width=0.98\linewidth, height=5cm]{figures/workflow.png}
    \caption{Overview of our study}
    \label{fig:ovwerview}
\end{figure*}

\noindent
\textbf{RQ2: What are the functional roles of images in bug reports?} 
Textual information in bug reports is typically organized into functional categories, such as Steps-to-Reproduce (S2Rs) and Observed Behavior (OB)~\cite{chaparroAssessingQualitySteps2019,chaparroDetectingMissingInformation2017},
each emphasizing different key information and requiring distinct strategies to effectively utilize the information for bug reproduction. For instance, textual S2Rs incorporate specific action verbs (e.g., “click”) and nouns related to UI elements (e.g., “menu”), which have influenced the creation of various tools~\cite{zhao2019recdroid, feng2024prompting, zhang2023automatically}. These tools utilize S2Rs through a phase known as S2R Entity Extraction. During this phase, crucial entities are extracted from S2Rs, identifying important action-noun pairs to facilitate automated bug reproduction.
%
Similarly, when using crash logs, identifying key information like the activity name where the bug occurred and the specific error exception is crucial for accurately reproducing the bug.
%
Thus, this research question aims to examine the functional roles that images play in bug reports, identifying patterns within these roles and analyzing the specific information emphasized for each role. This analysis can inspire future research by providing a foundation for leveraging images more effectively: it clarifies how to differentiate between image roles and, for each role, highlights the essential information that should be prioritized or extracted to assist in bug reproduction.

\noindent
\textbf{RQ3: Are images in bug reports documented?} 
While a plain image can convey information, supporting text is essential for a complete understanding, especially when images may serve various functional roles (as explored in RQ2). Correctly identifying these roles for reproduction requires adequate documentation.
%
Therefore, this research question investigates the extent to which images in bug reports are documented. If documentation is present, it examines its form—such as annotations or accompanying text. When documentation is absent, we need to consider alternative ways to interpret and utilize the image effectively to understand its purpose and relevance.
%
%\emph{Documented images} refers to images that include annotations or are accompanied by explanatory text that indicates the purpose or context of the image.
%
%Documentation provides context and clarity, enabling developers to understand the reported issue more efficiently.  


 
%\commentzx{The description of RQ sounds broader than what we try to explore in this RQ. Finding "limitations" sounds like we are evaluating from every aspects, but essentially we are trying to see their limitation on leveraging the image information. I suggest to rephrase the text to make it more tighten to the following sentences.}
\noindent
\textbf{RQ4: What types of images have the most significant impact on the effectiveness of automated bug reproduction?} 
%To understand the impact of images on automated bug reproduction, it is essential to evaluate the performance of existing tools, which primarily focus on textual descriptions, with bug reports that include images. Specifically, 
This research question explores the role of images in automated bug reproduction by distinguishing between image types that contribute positively to the process, those with negligible impact, and those that may even hinder effectiveness. We aim to understand why certain images aid or obstruct automation, and to uncover opportunities to leverage these insights—whether to refine existing tools or to design new ones that make more strategic use of image.


%Through a comprehensive analysis of advanced bug reproduction tools, we examine how the omission of images from bug reports affects the automated reproduction process, identify the types of images that lead to reproduction failures, and analyze the overall influence of images on automated bug reproduction.

Our study highlights the importance of images in automated bug reproduction and presents several key findings regarding the effective use of different image types in bug reports.
In summary, this paper makes the following contributions:
%

%\begin{itemize}
%    \item  Images in bug reports fulfill various roles that highlight unique types of information, requiring different approaches to leverage them effectively. While most images function as Observed Behavior (OB) images, displaying the issue visually, they also illustrate Expected Behavior (EB) and Steps to Reproduce (S2R).
%     \item   In cases where a bug report contains only one image, it is invariably an Observed Behavior (OB) image. This indicates that users prioritize directly showcasing the issue when limited to a single image, likely to clearly present the core symptom.
%    \item S2R images are crucial for replicating the steps that lead to a bug, and they serve multiple functions. We categorize S2R images into three types: (1) those that depict an entire sequence of steps, (2) those that provide key information to complement textual S2R descriptions, and (3) those that serve as a visual verification of the result after executing the S2R. Unlike OB images, which mainly illustrate the bug itself, S2R images are vital for accurately replicating the bug-triggering process.
%    \item Observed Behavior (OB) images hold potential value for addressing ongoing challenges in functional bug verification and validation.
%\end{itemize}
%\commentty{summarize the findings using bullet points.}


%we found that (1) majority images are UI screenshot. (2) there are six different images role, different pattern and each provide different key information.  (3) documentation is important to improve the interpretation of images.we also found that images may not be necessary if textual step-to-reproduce (S2R) descriptions are sufficiently detailed. However, missing steps captured in S2R images can lead to failures or reduced performance. Additionally, images depicting the observed behavior (OB) and expected behavior (EB) are crucial for improving the accuracy of bug symptom verification.

%Based on these findings, we outline future research opportunities that guide the direction of automated bug report reproduction. We also developed a proof-of-concept tool, \Name{}, built on the latest bug reproduction tool, ReBL~\cite{add citation}, to integrate images into our framework. Our approach involved modifying existing tools to process both text and images, replacing the traditional API with a multimodal model. This adaptation enabled us to evaluate the combined input of text and visuals, resulting in a successful reproduction rate of xx\% for bug reports.  
%\begin{figure}[t]
 %   \centering
 %   \includegraphics[width=0.75\linewidth, height=10cm]{figures/br.png}
%    \caption{Real World Bug Report Example}
 %   \label{fig:br}
%\end{figure}


\begin{itemize}

    \item This study represents the first systematic study on the images within bug reports. It investigates the frequency of image inclusion, the functional roles these images serve, how they are documented, and evaluates their impact on the performance of current automated bug reproduction tools.
    
    \item Several key findings offer valuable insights and directions for future research on leveraging images in bug reports to enhance automated bug reproduction and new research direction.

    %\item Informed by key findings, we developed a proof-of-concept automated bug reproduction tool, which is the first approach to consider images in bug reports. 
    
    \item We have made the replication package available~\cite{replication}.
\end{itemize}

%\commentty{Add a summary to refer to subsequent sections.}
 


 