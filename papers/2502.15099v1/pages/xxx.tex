\subsubsection{Documentation of Images} 

%\commentdbw{placement? explanation? documentation?}} 

Unlike GIFs or videos, which capture the entire process of triggering a bug (including every UI page, actions, and transitions between pages), images are limited to presenting a snapshot of a specific UI state. Just as roles need to be identified in textual descriptions (e.g., determining whether a sentence is a S2R), fully leveraging image information for bug reproduction requires labeling the functional roles of images 
(i.e., the six categories described in Section 3.1). This helps facilitate the bug reproduction process. For instance, existing tools use S2R sentences as input to guide reproduction; similarly, images need analogous roles to be effectively utilized. 

We analyze how images are documented in bug reports to ascertain their functional roles. A well-documented image includes sufficient context to identify its function, such as the caption of the section where it's placed or the accompanying text, which clearly communicates its role within the report. Identifying an image’s appropriate role can often be achieved using simple regular expressions. For example, an OB (Observed Behavior) image appearing under a section titled "Observed Behavior" or "Actual Behavior" is considered well-aligned because its purpose aligns with that of the section, both aiming to illustrate the observed issue.
%
In contrast, if an image is placed out of its appropriate section, it requires additional effort to determine the role of the image,
such as using classification techniques. 

%In such cases, 
%readers must interpret the whole bug report to accurately identify whether the image is related to OB, EB, or S2Rs. If it is related to S2Rs, further interpretation is needed to determine which specific S2R it relates to.

Table ~\ref{tab:location} presents the well-documented image of each role. OB Image has the lowest proportion, with xx\% of OB images being incorrectly placed.

%Additionally, among the total xx wrong placement images, xx\% were OB iamge, and xx\% are the single image in the bug report.This is  likely occurs because the template provides a designated area for images, while the latter may be due to the perception that placing images at the end of the document makes it easier to organize.

%We find that xx\% (xxx/xxx) of wrongly placed images are located in a section called 'Screenshots.' Of these, xx\%(xxx/xxx) contain only one image, while xx\% have more than one image. When there is only one image, xx\% are OB images. However, when there are multiple images, it becomes not only challenging to identify their roles but also difficult to understand their relationships.





%Considering that user writing habits and preferences can lead to inconsistencies that make bug reports less readable and harder to understand, we also examine whether the images are positioned correctly. Correct positioning is defined, for example, as an OB Image appearing in the OB section of the report. Images placed outside their designated sections are deemed incorrectly positioned.


\begin{tcolorbox}[colback=gray!5, colframe=black, ]
{\bf Finding 4:}  {
%
XX\% of poorly documented images are placed separately at the end  or in the “screenshot” section of the bug report without any description, with the majority (xx\%) of them are OB Image. Therefore, approaches on identifying specific roles of images are needed
}
\end{tcolorbox}


%
\begin{table}[h]
    \centering
    \scriptsize
    \hspace*{\fill}
    \begin{minipage}{0.45\linewidth}
        \centering
        \begin{threeparttable}
        \caption{Distribution of Bug Reports Containing Images Across Different Functionalities} 
        \label{tab:roles}
        \begin{tabular}{l|l|*{6}{>{\centering\arraybackslash}p{0.325cm}}}
            \toprule
              & \# BR & S2R\_s & S2R\_c & S2R\_r & OB & EB & Others \\ \hline
            \textbf{Our Dataset}  & -    &  0.4\%   & 9.8\% & 2.9\%       & 94.7\%    & 18.0\% & 6.5\%     \\
            AndroR2   & 36 & 0\%  & 11.1\%      & 8.3\%    & 100\% & 5.6\%  & 5.6\%      \\
            RegDroid  & 162 & 0.6\%  & 4.3\%    & 9.3\%    & 87\%    & 11.1\%  & 8.6\%     \\
            \bottomrule
        \end{tabular}
        
        \begin{tablenotes}
            \footnotesize
            \item[1] \tiny S2R\_s = S2R\_standalone, S2R\_c = S2R\_context, S2R\_r = S2R\_result
        \end{tablenotes}
        \end{threeparttable}
    \end{minipage}%
    \hfill
    \begin{minipage}{0.45\linewidth}
        \centering
        \vspace{-10pt}
        % \begin{threeparttable}
        \caption{Distribution of Well-Documented Images Across  Functionalities}
        \label{tab:location}
        \begin{tabular}{l|l|*{4}{>{\centering\arraybackslash}p{0.3cm}}}
            \toprule
              & \# Img & S2R  & OB & EB & Others\\ \hline
            \textbf{OurDataset}      &  & -  & -          & -      \\
            AndroR2  & & -  & -        & -    & -    \\
            RegDroid & & -  & -       & -    & -   \\
            \bottomrule
        \end{tabular}
         %\begin{tablenotes}
        %    \item[1] \tiny S2R = the sum of all three kinds of S2R image
       % \end{tablenotes}
       % \end{threeparttable}
    \end{minipage}
    \hspace*{\fill}
\end{table}


\subsubsection{Generalizability of RQ2}
\label{represent}
\commentty{need more explanation.}
Generalizability is crucial for RQ₂, which focuses on analyzing the utilization of images in bug reports within a randomly sampled dataset. To ensure that our findings are not limited to the specific characteristics of this initial sample, we validated our results by applying the same analysis to two independent third-party datasets 
 AndroR2~\cite{wendland2021andror2, johnson2022empirical} and RegDroid~\cite{xiong2023empirical}.  
 
 %The two datasets were not constructed by the authors of this paper, and the presence of images in bug reports was not one of the criteria in their construction, nor were images in bug reports studied in subsequent analyses. \commentty{what is the purpose of the above sentence?}
%In contrast, RQ2 does not require generalizability because its goal is to evaluate the impact of images through controlled experiments using existing bug reproduction tools, with and without image information. This approach is designed to measure the effectiveness and efficiency under specific conditions, providing insights into how images influence tool performance rather than establishing a broad understanding of common practices or patterns.
AndroR2 contains 30\% (39/130) of bug reports with images,though three of these reports either no longer have access to the images or the images are externally hosted. For RegDroid, 41.85\% (167/399) of bug reports include images, but five of these reports lack access to the images. We focused on bug reports with accessible images, which includes 36 from AndroR2 and 162 from RegDroid. We applied the same methodologies (as described in Section~\ref{method}) used in RQ1 to classify the images in the bug reports and assess their placements. The results were then compared to our own dataset to determine if the findings from these third-party datasets align with ours. This analysis helps assess the representativeness of our findings.

%
Table~\ref{tab:roles} and Table~\ref{tab:location} present the results of classifying image roles and assessing image placement for our dataset and the two third-party datasets. The findings reveal a consistent trend: most bug reports feature OB images, with S2R images being the second most prevalent. The similar distribution of image roles across all three datasets indicates that our study's findings are generalizable beyond the randomly sampled dataset. By classifying and evaluating images in AndroR2 and RegDroid, we demonstrate that the identified patterns are consistent across different data sources and contexts. This cross-validation confirms the robustness of our findings regarding the utilization of images in bug reports, thereby enhancing the validity and reliability of our conclusions.





\ignore{
\subsubsection{Methodology}
Two authors independently classified the role of images and then cross-validated their results through comparison and discussion. When consensus could not be reached, an additional author was consulted to ensure agreement on the final categorization. The followed the follwing steps to classif

\begin{itemize}[leftmargin=0.5cm]
\item  The placement of images of bug reports often reveals their roles, particularly in reports that adhere to well-defined templates. Structured bug reports clearly organize information into titled sections, making categorization straightforward. For example, For example, an image located within the OB section can be distinctly classified as an OB image. 

\item Contextual analysis is essential for classifying images in unstructured bug reports and S2R images. This is because unstructured bug reports lack titled sections, and S2R images fall into three subroles, as introduced in section ~\ref{}. By examining the surrounding text and overall content, the role and relevance of each image can be accurately determined, ensuring proper classification.

\end{itemize}
}