\section{Empirical Study}


\subsection{Quantity and Types of Images (RQ1)}
This research question examines patterns in the quantity and types of images used in bug reports, offering insights into their distribution and usage. We analyze the average number of images per report, categorize them by type (e.g., UI screenshots vs. other images), and identify how many reports contain single or multiple images. This analysis provides a foundation for further studies to explore the potential of using image usage in bug reports for automated bug production.
%

In analyzing the 367 bug reports in Dataset$_1$, 70.77\% of reports contained exactly one image, while 29.23\% included multiple images, with the number of images per report ranging from 1 to 7. On average, there were 1.48 images per bug report. Of these, 95.18\% were UI screenshots, while a smaller portion (4.82\%) of these images included pictures of code snippets, crash logs, or photos of the UI screen taken by another device. 


%\begin{table}[t]
%\vspace{-10pt}
%\centering
%\caption{Distribution of Image Types Across Single-Image and Multiple-Image Bug Reports}
%\label{tab:rq1}
%\small
%\begin{tabular}{|l|c|c|c|}
%\hline
% \rowcolor{gray!45}            & Sgl-Img BR & Multi-Img BR &  \textbf{Overall}\\ \hline\hline
 
%Screenshot   &   &  &  \\ \hline
%Non-Screenshot   &   & &      \\ \hline
%\textbf{Overall}   &   &   &    \\ \hline
%\end{tabular}

%\end{table}


\begin{tcolorbox}[colback=blue!5, colframe=black, boxrule=0.5pt]
%
\textbf{Finding 2:} 70.77\% of reports included one image, while 29.23\% had multiple images (ranging from 1 to 7 per report). Of all images, 95.18\% were UI screenshots, and 4.82\% were non-UI (e.g., code snippets, error logs).

%
\end{tcolorbox}

Since the majority of reports contain a single image, it is crucial to understand the functional role of this image in single-image reports. What kind of information does it provide, and could it effectively be leveraged in automated bug reproduction? Conversely, in reports with multiple images, it is important to consider whether these images serve the same functional role or convey different aspects of the bug, as well as to examine the logical sequence between images. With these questions in mind, we will further explore the functional roles of images in RQ2.