

\subsection{Documentation of Images (RQ3)}
%
During the classification of the functional roles of images (RQ2), it is evident that static images provide limited information on their own. To accurately differentiate the roles of these images, it is necessary to consider the accompanying textual information in the bug report. This additional context is important for interpreting the intent and relevance of the images. Therefore, documentation becomes a key aspect. We studies the documentation of images—specifically, those that include annotations or are accompanied by explanatory text. Such documentation helps indicate the purpose or context of the image, enhancing its utility in understanding and reproducing the bug.
%An image was considered documented if it met at least one of the following criteria:
An image was considered documented following this criteria: we first determined whether an image contained Explanatory Text. If it did, we classified it under this category. If an image lacked explanatory text, we then assessed whether it was placed under a section with descriptive subsection titles. If an image met neither of these criteria, it was categorized as a plain image.
\begin{itemize}[left=0.1cm]
    \item \textbf{Explanatory Text (38.06\%).} Descriptive text accompanying the image that explains what the image shows or how it relates to the issues being reported.

    \item \textbf{Descriptive Subsection Titles (5.17\%).} Clear and informative headings above the image that indicate what the image is illustrating. Titles like "Observed Behavior" are helpful, whereas generic titles like "Screenshot" do not provide meaningful context. (e.g., plain images but placed under a subsection with a descriptive title)

\end{itemize}

\noindent
Of the images in bug reports, 43.23\% were identified as documented images, while 56.77\% were categorized as plain images. Plain images typically occur in two scenarios: (1) multiple consecutive images are placed together without accompanying text to clarify their differences or relationships, and (2) many bug reports use a template with sections labeled “Screenshots” or “Other.” Although these sections are well-intentioned, users often upload screenshots without additional explanation, making it difficult to interpret the images and challenging to identify which parts of the text they are intended to support. 





\begin{tcolorbox}[colback=blue!5, colframe=black, boxrule=0.5pt]
\textbf{Finding 5:}
%
Documented images (43.23\%) with explanatory text or descriptive titles enhance clarity, while plain images (56.77\%). If one intends to use images in automated bug reproduction, this lack of clear context complicates the classification process, making it challenging to accurately interpret each image's functional role.
%
\end{tcolorbox}


