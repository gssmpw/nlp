\documentclass[11pt]{article}
\pdfoutput=1
\usepackage[final]{acl}

% \usepackage{subfig}
\usepackage{subcaption}

% Font packages
\usepackage{times}
\usepackage{latexsym}
\usepackage[T1]{fontenc}
\usepackage[utf8]{inputenc}
\usepackage{microtype}
\usepackage{inconsolata} 
\usepackage{csquotes}
\usepackage{enumitem}

% Math packages
\usepackage{amsmath}
\usepackage{amssymb}  % Added for math symbols
\usepackage{amsthm}
\usepackage{mathtools}

% Table packages
\usepackage{booktabs}
\usepackage{multirow}
\usepackage{multicol}
\usepackage{tabularx}
\usepackage{tabulary}
\usepackage{adjustbox}
\usepackage{siunitx}
\usepackage{makecell}

% Graphics and color packages
\usepackage{graphicx}
\usepackage{bm}
\usepackage{xcolor}
\usepackage{xcolor,colortbl}

% TikZ and tcolorbox packages
\usepackage[most]{tcolorbox} 
\usepackage{cleveref}
\usepackage{makecell}

% Define tcolorbox styles
\newtcbtheorem[auto counter, number within=section, 
crefname={example}{Example},
Crefname={Example}{Example}]
{exmp}{Exam\smash{p}le} 
{colback=pale_red!5, colframe=DarkPurple, left=.02in, right=.02in,bottom=.02in, top=.02in}{exmp}  

\newtcbtheorem[auto counter, 
crefname={prompt}{Prompt},
Crefname={Prompt}{Prompt}]
{prompt}{Prom\smash{p}t} 
{colback=pale_red!5, colframe=DarkPurple, left=.02in, right=.02in,bottom=.02in, top=.02in}{prompt}  

% \usepackage{enumitem}
\usepackage{listings}
\lstset{
    language=Python,
    basicstyle=\ttfamily\footnotesize,
    breaklines=True,
    showstringspaces=False,
    frame=none,
    keywordstyle=\color{blue},
    commentstyle=\color{gray},
    stringstyle=\color{green!50!black},
    morekeywords={import, from, as, def, return, in, if, else, elif, for, while, try, except, with, class, self, True, False, None, and, or, not, is, pass},
} 










% Define colors
\definecolor{Violet}{RGB}{148,0,211}    
\definecolor{DarkPurple}{RGB}{75,0,130} 
\definecolor{lightergray}{RGB}{230,230,230}
\definecolor{DarkRed}{RGB}{130,25,0}
\definecolor{PurpleRed}{RGB}{204,0,102}
\definecolor{DarkGreen}{RGB}{30,130,30}
\definecolor{DarkBlue}{RGB}{0,0,250}
\definecolor{DarkYellow}{RGB}{255,128,0}
\definecolor{light-gray}{gray}{0.95}
\definecolor{lightgreen}{RGB}{231,255,219}
\definecolor{lightred}{RGB}{252,231,234}
\definecolor{lightyellow}{RGB}{250,253,191}
\definecolor{lightpurple}{RGB}{229,204,255}
\definecolor{lightblue}{RGB}{229,246,254}
\definecolor{value-modification}{RGB}{250, 217, 86}
\definecolor{digit-expansion}{RGB}{216, 194, 104}
\definecolor{integer-decimal-fraction}{RGB}{240, 133, 51}
\definecolor{semantic-paraphrasing}{RGB}{85, 157, 63}
\definecolor{complexity-increasing}{RGB}{58, 120, 175}
\definecolor{question-transformation}{RGB}{174, 205, 225}
\definecolor{interference-injection}{RGB}{255,204,229}
\definecolor{remove-constrain}{RGB}{204,204,255}

% Custom commands
\newcommand{\xmark}{\textcolor{red}{\ding{55}}} 
\newcommand{\cmark}{\textcolor{DarkGreen}{\ding{51}}}
\newcommand{\addmark}{\textcolor{DarkYellow}{\ding{59}}}
\newcommand{\editmark}{\textcolor{blue}{\ding{34}}}

\definecolor{pale_green}{rgb}{0.55,0.75,0.60}
\definecolor{pale_red}{rgb}{0.90,0.61,0.58}
\definecolor{pale_yellow}{rgb}{0.95,0.92,0.72}

% AIbox definition
\tcbset{
  aibox/.style={
    width=\textwidth,
    top=0pt, bottom=0pt, left=5pt, right=5pt,
    colback=white,
    colframe=black,
    colbacktitle=black,
    enhanced,
    center,
    attach boxed title to top left={yshift=-0.1in,xshift=0.15in},
    boxed title style={boxrule=0pt,colframe=white,},
  }
}
\newtcolorbox{AIbox}[2][]{aibox,title=#2,#1}

\addtocontents{toc}{\protect\setcounter{tocdepth}{-1}} 





\newcounter{testexample}
\usepackage{tcolorbox}
\usepackage{xcolor}

\def\exampletext{Definition} % If English

\NewDocumentEnvironment{testexample}{ O{} }
{
\colorlet{colexam}{red!45!black} % Global example color
\newtcolorbox[use counter=testexample]{testexamplebox}{%
    % Example Frame Start
    empty,% Empty previously set parameters
    % Removed title line: title={\exampletext: #1},% use \thetcbcounter to access the testexample counter text
    % Removed title related options
    % Removed: attach boxed title to top left,
    % Removed: minipage boxed title,
    % Removed: boxed title style={empty,size=minimal,toprule=0pt,top=4pt,left=3mm,overlay={}},
    % Removed: coltitle=colexam,fonttitle=\bfseries,
    before=\par\medskip\noindent,parbox=false,boxsep=0pt,left=3mm,right=0mm,top=5pt,breakable,pad at break=0mm,
       before upper=\csname @totalleftmargin\endcsname0pt, % Use instead of parbox=true. This ensures parskip is inherited by box.
    % Handles box when it exists on one page only
    overlay unbroken={\draw[colexam,line width=.5pt] ([xshift=-0pt]frame.north west) -- ([xshift=-0pt]frame.south west); },
    % Handles multipage box: first page
    overlay first={\draw[colexam,line width=.5pt] ([xshift=-0pt]frame.north west) -- ([xshift=-0pt]frame.south west); },
    % Handles multipage box: middle page
    overlay middle={\draw[colexam,line width=.5pt] ([xshift=-0pt]frame.north west) -- ([xshift=-0pt]frame.south west); },
    % Handles multipage box: last page
    overlay last={\draw[colexam,line width=.5pt] ([xshift=-0pt]frame.north west) -- ([xshift=-0pt]frame.south west); },%
    }
\begin{testexamplebox}}
{\end{testexamplebox}\endlist}

\newcommand{\xw}[1]{{\small\color{blue}{\bf xw:} #1}}



\title{Towards Reasoning Ability of Small Language Models}
% \title{From Large to Small: Towards Reasoning in Small and Compressed Language Models}

\author{
 \textbf{Gaurav Srivastava\textsuperscript{1}},
 \textbf{Shuxiang Cao\textsuperscript{2}},
 \textbf{Xuan Wang\textsuperscript{1}}
\\
\\
 \textsuperscript{1}Department of Computer Science, Virginia Tech, Blacksburg, VA, USA,
 \\
 \textsuperscript{2}Department of Physics, Clarendon Laboratory, University of Oxford, OX1 3PU, UK
\\
\\
 \normalsize{
        \texttt{(\href{gks@vt.edu}{gks}, \href{xuanw@vt.edu}{xuanw})@vt.edu; (\href{shuxiang.cao@physics.ox.ac.uk}{shuxiang.cao})@physics.ox.ac.uk}
     % \href{mailto:gks@vt.edu}{gks@vt.edu}
 }
}

\begin{document}
\maketitle


\begin{abstract}



Reasoning has long been viewed as an emergent property of large language models (LLMs), appearing at or above a certain scale ($\sim$100B parameters). However, recent studies challenge this assumption, showing that small language models (SLMs) can also achieve competitive reasoning performance. SLMs are increasingly favored for their efficiency and deployability. However, there is a lack of systematic study on the reasoning abilities of diverse SLMs, including those trained from scratch or derived from LLMs through quantization, pruning, and distillation. This raises a critical question: \emph{Can SLMs achieve reasoning abilities comparable to LLMs?} In this work, we systematically \textbf{survey, benchmark, and analyze} \textbf{72} SLMs from \textbf{six} model families across \textbf{14} reasoning benchmarks. For reliable evaluation, we examine \textbf{four} evaluation methods and compare \textbf{four} LLM judges against human evaluations on \textbf{800} data points. We repeat all experiments \textbf{three} times to ensure a robust performance assessment. Additionally, we analyze the impact of different prompting strategies in small models. Beyond accuracy, we also evaluate model robustness under \textbf{adversarial conditions} and \textbf{intermediate reasoning steps}. Our findings challenge the assumption that scaling is the only way to achieve strong reasoning. Instead, we foresee a future where SLMs with strong reasoning capabilities can be developed through structured training or post-training compression. They can serve as efficient alternatives to LLMs for reasoning-intensive tasks. \footnote{All model responses, evaluation results, and GPT-based judgments will be released.}

\end{abstract}


\section{Introduction}
\label{sec:introduction}
The business processes of organizations are experiencing ever-increasing complexity due to the large amount of data, high number of users, and high-tech devices involved \cite{martin2021pmopportunitieschallenges, beerepoot2023biggestbpmproblems}. This complexity may cause business processes to deviate from normal control flow due to unforeseen and disruptive anomalies \cite{adams2023proceddsriftdetection}. These control-flow anomalies manifest as unknown, skipped, and wrongly-ordered activities in the traces of event logs monitored from the execution of business processes \cite{ko2023adsystematicreview}. For the sake of clarity, let us consider an illustrative example of such anomalies. Figure \ref{FP_ANOMALIES} shows a so-called event log footprint, which captures the control flow relations of four activities of a hypothetical event log. In particular, this footprint captures the control-flow relations between activities \texttt{a}, \texttt{b}, \texttt{c} and \texttt{d}. These are the causal ($\rightarrow$) relation, concurrent ($\parallel$) relation, and other ($\#$) relations such as exclusivity or non-local dependency \cite{aalst2022pmhandbook}. In addition, on the right are six traces, of which five exhibit skipped, wrongly-ordered and unknown control-flow anomalies. For example, $\langle$\texttt{a b d}$\rangle$ has a skipped activity, which is \texttt{c}. Because of this skipped activity, the control-flow relation \texttt{b}$\,\#\,$\texttt{d} is violated, since \texttt{d} directly follows \texttt{b} in the anomalous trace.
\begin{figure}[!t]
\centering
\includegraphics[width=0.9\columnwidth]{images/FP_ANOMALIES.png}
\caption{An example event log footprint with six traces, of which five exhibit control-flow anomalies.}
\label{FP_ANOMALIES}
\end{figure}

\subsection{Control-flow anomaly detection}
Control-flow anomaly detection techniques aim to characterize the normal control flow from event logs and verify whether these deviations occur in new event logs \cite{ko2023adsystematicreview}. To develop control-flow anomaly detection techniques, \revision{process mining} has seen widespread adoption owing to process discovery and \revision{conformance checking}. On the one hand, process discovery is a set of algorithms that encode control-flow relations as a set of model elements and constraints according to a given modeling formalism \cite{aalst2022pmhandbook}; hereafter, we refer to the Petri net, a widespread modeling formalism. On the other hand, \revision{conformance checking} is an explainable set of algorithms that allows linking any deviations with the reference Petri net and providing the fitness measure, namely a measure of how much the Petri net fits the new event log \cite{aalst2022pmhandbook}. Many control-flow anomaly detection techniques based on \revision{conformance checking} (hereafter, \revision{conformance checking}-based techniques) use the fitness measure to determine whether an event log is anomalous \cite{bezerra2009pmad, bezerra2013adlogspais, myers2018icsadpm, pecchia2020applicationfailuresanalysispm}. 

The scientific literature also includes many \revision{conformance checking}-independent techniques for control-flow anomaly detection that combine specific types of trace encodings with machine/deep learning \cite{ko2023adsystematicreview, tavares2023pmtraceencoding}. Whereas these techniques are very effective, their explainability is challenging due to both the type of trace encoding employed and the machine/deep learning model used \cite{rawal2022trustworthyaiadvances,li2023explainablead}. Hence, in the following, we focus on the shortcomings of \revision{conformance checking}-based techniques to investigate whether it is possible to support the development of competitive control-flow anomaly detection techniques while maintaining the explainable nature of \revision{conformance checking}.
\begin{figure}[!t]
\centering
\includegraphics[width=\columnwidth]{images/HIGH_LEVEL_VIEW.png}
\caption{A high-level view of the proposed framework for combining \revision{process mining}-based feature extraction with dimensionality reduction for control-flow anomaly detection.}
\label{HIGH_LEVEL_VIEW}
\end{figure}

\subsection{Shortcomings of \revision{conformance checking}-based techniques}
Unfortunately, the detection effectiveness of \revision{conformance checking}-based techniques is affected by noisy data and low-quality Petri nets, which may be due to human errors in the modeling process or representational bias of process discovery algorithms \cite{bezerra2013adlogspais, pecchia2020applicationfailuresanalysispm, aalst2016pm}. Specifically, on the one hand, noisy data may introduce infrequent and deceptive control-flow relations that may result in inconsistent fitness measures, whereas, on the other hand, checking event logs against a low-quality Petri net could lead to an unreliable distribution of fitness measures. Nonetheless, such Petri nets can still be used as references to obtain insightful information for \revision{process mining}-based feature extraction, supporting the development of competitive and explainable \revision{conformance checking}-based techniques for control-flow anomaly detection despite the problems above. For example, a few works outline that token-based \revision{conformance checking} can be used for \revision{process mining}-based feature extraction to build tabular data and develop effective \revision{conformance checking}-based techniques for control-flow anomaly detection \cite{singh2022lapmsh, debenedictis2023dtadiiot}. However, to the best of our knowledge, the scientific literature lacks a structured proposal for \revision{process mining}-based feature extraction using the state-of-the-art \revision{conformance checking} variant, namely alignment-based \revision{conformance checking}.

\subsection{Contributions}
We propose a novel \revision{process mining}-based feature extraction approach with alignment-based \revision{conformance checking}. This variant aligns the deviating control flow with a reference Petri net; the resulting alignment can be inspected to extract additional statistics such as the number of times a given activity caused mismatches \cite{aalst2022pmhandbook}. We integrate this approach into a flexible and explainable framework for developing techniques for control-flow anomaly detection. The framework combines \revision{process mining}-based feature extraction and dimensionality reduction to handle high-dimensional feature sets, achieve detection effectiveness, and support explainability. Notably, in addition to our proposed \revision{process mining}-based feature extraction approach, the framework allows employing other approaches, enabling a fair comparison of multiple \revision{conformance checking}-based and \revision{conformance checking}-independent techniques for control-flow anomaly detection. Figure \ref{HIGH_LEVEL_VIEW} shows a high-level view of the framework. Business processes are monitored, and event logs obtained from the database of information systems. Subsequently, \revision{process mining}-based feature extraction is applied to these event logs and tabular data input to dimensionality reduction to identify control-flow anomalies. We apply several \revision{conformance checking}-based and \revision{conformance checking}-independent framework techniques to publicly available datasets, simulated data of a case study from railways, and real-world data of a case study from healthcare. We show that the framework techniques implementing our approach outperform the baseline \revision{conformance checking}-based techniques while maintaining the explainable nature of \revision{conformance checking}.

In summary, the contributions of this paper are as follows.
\begin{itemize}
    \item{
        A novel \revision{process mining}-based feature extraction approach to support the development of competitive and explainable \revision{conformance checking}-based techniques for control-flow anomaly detection.
    }
    \item{
        A flexible and explainable framework for developing techniques for control-flow anomaly detection using \revision{process mining}-based feature extraction and dimensionality reduction.
    }
    \item{
        Application to synthetic and real-world datasets of several \revision{conformance checking}-based and \revision{conformance checking}-independent framework techniques, evaluating their detection effectiveness and explainability.
    }
\end{itemize}

The rest of the paper is organized as follows.
\begin{itemize}
    \item Section \ref{sec:related_work} reviews the existing techniques for control-flow anomaly detection, categorizing them into \revision{conformance checking}-based and \revision{conformance checking}-independent techniques.
    \item Section \ref{sec:abccfe} provides the preliminaries of \revision{process mining} to establish the notation used throughout the paper, and delves into the details of the proposed \revision{process mining}-based feature extraction approach with alignment-based \revision{conformance checking}.
    \item Section \ref{sec:framework} describes the framework for developing \revision{conformance checking}-based and \revision{conformance checking}-independent techniques for control-flow anomaly detection that combine \revision{process mining}-based feature extraction and dimensionality reduction.
    \item Section \ref{sec:evaluation} presents the experiments conducted with multiple framework and baseline techniques using data from publicly available datasets and case studies.
    \item Section \ref{sec:conclusions} draws the conclusions and presents future work.
\end{itemize}
\section{Related Works}
\label{sec:related_works}


\noindent\textbf{Diffusion-based Video Generation. }
The advancement of diffusion models \cite{rombach2022high, ramesh2022hierarchical, zheng2022entropy} has led to significant progress in video generation. Due to the scarcity of high-quality video-text datasets \cite{Blattmann2023, Blattmann2023a}, researchers have adapted existing text-to-image (T2I) models to facilitate text-to-video (T2V) generation. Notable examples include AnimateDiff \cite{Guo2023}, Align your Latents \cite{Blattmann2023a}, PYoCo \cite{ge2023preserve}, and Emu Video \cite{girdhar2023emu}. Further advancements, such as LVDM \cite{he2022latent}, VideoCrafter \cite{chen2023videocrafter1, chen2024videocrafter2}, ModelScope \cite{wang2023modelscope}, LAVIE \cite{wang2023lavie}, and VideoFactory \cite{wang2024videofactory}, have refined these approaches by fine-tuning both spatial and temporal blocks, leveraging T2I models for initialization to improve video quality.
Recently, Sora \cite{brooks2024video} and CogVideoX \cite{yang2024cogvideox} enhance video generation by introducing Transformer-based diffusion backbones \cite{Peebles2023, Ma2024, Yu2024} and utilizing 3D-VAE, unlocking the potential for realistic world simulators. Additionally, SVD \cite{Blattmann2023}, SEINE \cite{chen2023seine}, PixelDance \cite{zeng2024make} and PIA \cite{zhang2024pia} have made significant strides in image-to-video generation, achieving notable improvements in quality and flexibility.
Further, I2VGen-XL \cite{zhang2023i2vgen}, DynamicCrafter \cite{Xing2023}, and Moonshot \cite{zhang2024moonshot} incorporate additional cross-attention layers to strengthen conditional signals during generation.



\noindent\textbf{Controllable Generation.}
Controllable generation has become a central focus in both image \citep{Zhang2023,jiang2024survey, Mou2024, Zheng2023, peng2024controlnext, ye2023ip, wu2024spherediffusion, song2024moma, wu2024ifadapter} and video \citep{gong2024atomovideo, zhang2024moonshot, guo2025sparsectrl, jiang2024videobooth} generation, enabling users to direct the output through various types of control. A wide range of controllable inputs has been explored, including text descriptions, pose \citep{ma2024follow,wang2023disco,hu2024animate,xu2024magicanimate}, audio \citep{tang2023anytoany,tian2024emo,he2024co}, identity representations \citep{chefer2024still,wang2024customvideo,wu2024customcrafter}, trajectory \citep{yin2023dragnuwa,chen2024motion,li2024generative,wu2024motionbooth, namekata2024sg}.


\noindent\textbf{Text-based Camera Control.}
Text-based camera control methods use natural language descriptions to guide camera motion in video generation. AnimateDiff \cite{Guo2023} and SVD \cite{Blattmann2023} fine-tune LoRAs \cite{hu2021lora} for specific camera movements based on text input. 
Image conductor\cite{li2024image} proposed to separate different camera and object motions through camera LoRA weight and object LoRA weight to achieve more precise motion control.
In contrast, MotionMaster \cite{hu2024motionmaster} and Peekaboo \cite{jain2024peekaboo} offer training-free approaches for generating coarse-grained camera motions, though with limited precision. VideoComposer \cite{wang2024videocomposer} adjusts pixel-level motion vectors to provide finer control, but challenges remain in achieving precise camera control.

\noindent\textbf{Trajectory-based Camera Control.}
MotionCtrl \cite{Wang2024Motionctrl}, CameraCtrl \cite{He2024Cameractrl}, and Direct-a-Video \cite{yang2024direct} use camera pose as input to enhance control, while CVD \cite{kuang2024collaborative} extends CameraCtrl for multi-view generation, though still limited by motion complexity. To improve geometric consistency, Pose-guided diffusion \cite{tseng2023consistent}, CamCo \cite{Xu2024}, and CamI2V \cite{zheng2024cami2v} apply epipolar constraints for consistent viewpoints. VD3D \cite{bahmani2024vd3d} introduces a ControlNet\cite{Zhang2023}-like conditioning mechanism with spatiotemporal camera embeddings, enabling more precise control.
CamTrol \cite{hou2024training} offers a training-free approach that renders static point clouds into multi-view frames for video generation. Cavia \cite{xu2024cavia} introduces view-integrated attention mechanisms to improve viewpoint and temporal consistency, while I2VControl-Camera \cite{feng2024i2vcontrol} refines camera movement by employing point trajectories in the camera coordinate system. Despite these advancements, challenges in maintaining camera control and scene-scale consistency remain, which our method seeks to address. It is noted that 4Dim~\cite{watson2024controlling} introduces absolute scale but in  4D novel view synthesis (NVS) of scenes.



\section{Experimental settings}
\label{sec:experimental-setup}
%We use two types of evaluation: offline and online. For offline
%evaluation,
\myparagraph{Datasets}
We use datasets which are reports of ranking
competitions \cite{raifer2017information,Mordo+al:25a}. In these
competitions, students were assigned to queries and had to produce
documents that would be highly ranked. Before the first round the students were provided with an example of a document relevant to the query. In each of the following rounds, the students observed past rankings for their queries and could modify their documents to potentially improve their next round ranking.

The first dataset, \firstmention{\firstDataset}, is the result of
ranking competitions held for $31$ queries from the TREC9-TREC12 Web
tracks \cite{raifer2017information}. Five to six students competed for each query. The undisclosed ranking function
was LambdaMART \cite{burges2010lambdamart} applied with various hand-crafted features. Following
Goren et al. \cite{goren2020ranking}, whose document modification
approach, \firstmention{\sentReplace}, serves as a baseline\footnote{We found that using LambdaMART instead of SVMrank as originally proposed \cite{goren2020ranking} yields improved performance.}, we use
round 7 for evaluation \cite{raifer2017information}. \sentReplace is a state-of-the-art feature-based supervised method for ranking-incentivized document modification. It
replaces a sentence in the document with another sentence to
improve ranking and to maintain content quality and faithfulness to
the original document.

The second dataset, \firstmention{\secondDataset}, is a report of
ranking competitions \cite{Mordo+al:25a} where the undisclosed ranking
function was the cosine between the E5 embedding vectors \cite{Wang+al:24a} of a document
and a query\footnote{The intfloat/e5-large-unsupervised version from
  the Hugging Face repository
  (\url{https://huggingface.co/intfloat/e5-large-unsupervised}).}. The competitions were run for 7 rounds with $15$
queries from the Web tracks of TREC9-TREC12; 4 players were competing
for each query \cite{wang2022text}.
%Our best performing document modification
%strategies (prompts) were used as bots in some rounds of these
%competitions for online evalution. (See more details below.)
%For
%offline evaluation,
We used round $4$ for evaluation to allow the document
modification methods to have enough history of past rankings. 

For both datasets just described, we apply the different document
modification methods, henceforth referred to as \firstmention{bots},
upon each of the documents in the ranked list for a query in the
specified round (except for the highest ranked document). For each
selected document, we induce a ranking using the same ranker used in
the competitions over its modified version and the original next-round
versions of the other documents (of students) from the round. We use
the evaluation measures described below upon the resultant ranking. We
average the evaluation results across all documents we modified per
round and over queries.

\myparagraph{Evaluation measures}
%We analyzed the performance of a document modification method using various evaluation measures, categorized into three primary groups: ranking properties, faithfulness properties and Quality and Relevance properties. All measures were computed per player and her document for a given query. The results were averaged over queries and grouped by the player type (student, baseline\footnote{i.e. the method of replacing paragraphs \cite{goren2020ranking}.}, a static document or one of the \bt s). For the online evaluation, the measures were also averaged over rounds.
To evaluate rank promotion (demotion) of documents as a result of
modification, we follow Goren et al. \cite{goren2020ranking} and
report \firstmention{Scaled Promotion}: the increase (decrease) of rank in the next round with respect to the current round normalized by the maximum possible rank promotion (demotion).



\omt{
%\begin{block}{Candidate Faithfulness at 1}
$CF@1(d_{curr},d_{next})=\frac{1}{n} \cdot \Sigma_{i=1}^{n} \mathbf{1}{\{ TT(d{curr},d_{next_{i}}) \geq 0.5 \}}$
%\end{block}

%\begin{block}{Normalized Candidate Faithfulness at 1}
$NCF@1(d_{curr},d_{next})=\frac{CF@1(d_{curr},d_{next})}{CF@1(d_{curr},d_{curr})}$
%\end{block}


%\begin{block}{Environmental Faithfulness at 10}
$EF@10(d_{next})=\frac{1}{2 \cdot 10} \cdot \Sigma_{i=1}^{10} [\mathbf{1}{\{ TT(d{\text{top}i}, d{\text{next}}) \geq 0.5 \}} + \mathbf{1}{\{ TT(d{\text{next}}, d_{\text{top}_i}) \geq 0.5 \}}]$
%\end{block}

%\begin{block}{Normalized Environmental Faithfulness at 10}
$NEF@10(d_{curr},d_{next})=\frac{EF@10(d_{next})}{EF@10(d_{curr})}$
%\end{block}
}


To evaluate the faithfulness of a modified document ($\dn$) to its
original (current) version ($\dc$), we compare
the two documents using Gekhman's et al. \cite{gekhman2023trueteacher}
natural language inference (NLI) approach. Specifically, we estimate
whether one document (denoted {\em hypothesis}) is entailed from the other
document (denoted {\em premise}) while preserving factual consistency. The estimate is the \trueteacher{} (TrueTeacher)
measure: \trueteacher $(premise, hypothesis)$ is the
output of the model in the range [0,1]; higher scores indicate stronger factual alignment.

To apply the TrueTeacher model, we first compute the average number of sentences in the modified document that are entailed\footnote{Entailment is determined by a threshold of $0.5$ for the TT score \cite{gekhman2023trueteacher}.} by the current document, which we refer to as raw faithfulness (RF):
%\trueteacher{} score between
%the current document ($\dc$) and all ($n$) sentences in the modified
%document ($\dn$):
$RF(\dn,\dc) \definedas \frac{1}{n} \sum_{i=1}^n \delta[\trueteacher
  (\dc, \dni) \ge 0.5];$ $d^{i}$ is the i'th sentence in document $d$;
$\delta$ is Kronecker's indicator function. Since $RF(\dc,\dc)$ is not
necessarily $1$, we normalize the raw
faithfulness to yield our \firstmention{\normFaith} measure: $\frac
{RF(\dn,\dc)}{RF(\dc,\dc)}$. 

Using LLMs to modify documents raises a concern
about hallucinations \cite{shuster2021retrieval}. We hence measure the
extent to which the content in the modified document is ``faithful''
to that in the entire corpus\footnote{For a corpus we use all the
  documents in all rounds prior to the round on which evaluation is
  performed.}. To that end, we treat the current document as a query,
and retrieve the top-$k$\footnote{We set $k=10$ in our experiments.}
documents in the corpus; $\topRet$ denotes the retrieved set. Retrieval is based on using cosine to compare a query
embedding and the document embedding. We use two types of embeddings:
E5 \cite{Wang+al:24a} and TF.IDF.  We define raw corpus faithfulness
(RCF) as: $RCF (\dn) \definedas \frac {1}{2k} \sum_{d \in \topRet}
(RF(\dn,d) + RF (d,\dn))$. The normalized corpus faithfulness
measure we use is: $CF (\dn) \definedas
\frac{RCF(\dn)}{RCF(\dc)}$. Using the E5 and TF.IDF embeddings results
in the \firstmention{\normCorpFaithE} and
\firstmention{\normCorpFaithT} normalized corpus faithfulness
measures, respectively.

Statistically significant differences are determined using the two-tailed paired permutation test
  with 100,000 random permutations and $p < 0.05$.

\omt{
The Normalized Candidate Faithfulness $NCF@1(\dc, \dn)$: the
normalization of the Candidate Faithfulness $CF@1(\dc, \dn)$ by the
self-consistency score: $\frac{CF@1 (\dc, \dn)}{CF@1(\dc, \dc)}$;
(iii) Environmental Faithfulness at 10 $EF@10(\dn)$: This metric
measures how much the generated document ($\dn$) maintains contextual
consistency with the broader corpus. Specifically, it measures the
similarity of $\dn$ to the top 10 documents most similar to it in the
corpus.  The corpus includes all the documents (across all queries)
available up to the test round. Two approaches are employed to compute
the similarity. The first approach is based on the (unsupervised) E5
\cite{wang2022text} representation with the cosine similarity
metric. The second approach is based on the TF.IDF
\cite{sparck1972statistical, salton1975vector} representation with the
cosine similarity metric. This metric is then calculated as follows:
$\frac{1}{2*10} \sum_{i=1}^{10} (\trueteacher(\dn, d_{top_{i}}) +
\trueteacher(d_{top_i}, \dn))$. Where $d_{top_{i}}$ represents the
$i$-th document, while ordering the documents with respect to the
similarity to $\dc$. The two approaches yield two variants of this
metric: $EF@10$\_dense and $EF@10$\_sparse, for the E5 and TF.IDF
representations, respectively; (iv) The Normalized Environmental
Faithfulness at 10 $NEF@10(\dn)$: The normalization of EF@10 by the
EF@10 of the current document: $\frac{EF@10(\dn)}{EF@10(\dc)}$. These
measures collectively provide a comprehensive framework for assessing
faithfulness. They evaluate the consistency of the modified document
not only with respect to the current document but also in relation to
other documents in the corpus.



\myparagraph{Relevance and Quality scores} The third category of evaluation measures focuses on the relevance and quality of documents. Both quality and relevance scores are assigned by crowdsourcing annotators via the Connect platform on CloudResearch \cite{noauthor_introducing_2024}, assessing the document's content quality and its relevance to the query\footnote{These evaluations of relevance and quality are conducted exclusively in the online evaluation setting.}. A document's quality or relevance score is set to 1 if at least three out of five English-speaking annotators marked it as valid or relevant to the query; otherwise, the score is set to 0. We report the ratio of documents that received a quality or relevance score of 1.
}


\myparagraph{Instantiating bots} For LLM we use Chat-GPT 4o
\cite{achiam2023gpt}. As described in Section
\ref{sec:bots}, there are a few parameters affecting the
instantiation of specific prompts. The number of queries is set to a
value in $\{1, 2\}$.  The number of examples per query is selected
from $\{1, 2, 3\}$. The number of past ranks (i.e., rounds) in the
Temporal prompt is selected from $\{2,3\}$. Using these
parameter values, and the other binary decision factors that affect
instantiation (see Section \ref{sec:bots}), results in $192$
different bots (prompts). In addition, we set the LLM's temperature parameter which controls potential drift to values in $\{0, 0.5, 1, 1.5, 2\}$ \cite{peeperkorn_is_2024}. 

\myparagraph{Rank promotion performance of bots} In terms of Scaled
Promotion, we found\footnote{Actual numbers are omitted due to space
  considerations and as they convery no additional insight.} that the
Pairwise bots (with random selection of document pairs) and the
Listwise bots were the best performing for both the \firstDataset and
\secondDataset datasets; the same specific instantiation of each of these two bots was
always among the top-3 performing bots for both datasets. This finding attests to the
rank-promotion effectiveness of these types of bots (prompts) for
different rankers (LambdaMART and E5). The Temporal bots (prompts), which provide rank-changes information along rounds, were less
effective (in terms of Scaled Promotion) than the Pairwise and
Listwise bots, but were more effective than the Pointwise bots. 

In what follows, we present the evaluation of the two
bots which posted for both datasets Scaled Promotion among the best three:\footnote{These bots were also the best performing in the online evaluation presented below.} 
%For efficiency considerations, we use LLama-2 with $13$B parameters
%\cite{touvron2023llama} to select the best performing
%configurations. The selection is performed with the \firstDataset
%dataset based on the scaled promotion evaluation measure. The best
%performing configurations for which we will report performance over the evaluation datasets are:

\begin{itemize}
\item Pairwise, where only the given query is included, one
  random pair of documents for each of the three last rounds is
  provided as examples, the current rank of the document is not
  used, and the temperature is set to $0.5$.
\item Listwise, where only the given query is included, two previous rounds are used, the current rank is not used, and the temperature is set to $0$.
\end{itemize}
Appendix \ref{appendix_prompt} provides the prompts for these bots.
%The fact that the pairwise and listwise approaches are the most
%effective is conceptually consistent of findings in work on using LLMs
%to induce ranking where the merits of pairwise and listwise approaches
%have been demonstrated \cite{ma2023zero,qin2023large}. For evaluation
%over \firstDataset and \secondDataset we use




\omt{
%The goal of the first phase is to identify a representative prompt for each class of prompts--that is, the prompt whose resultant agents maximize a metric related to ranking promotion. We conduct a comprehensive grid search over the 225 configurations described in Section \ref{sec:bots}, using Dataset 1. For this phase, we utilized Llama-2 with 13B parameters due to its availability \cite{touvron2023llama}. From each configuration, we constructed five agents, each with a different temperature setting for the probability model of the LLM\footnote{All other parameters of the LLM were fixed.}. The temperatures used were \{0, 0.5, 1, 1.5, 2\} and were selected based on the work of Peeperkorn et al \cite{peeperkorn_is_2024}. The selected agents compete for round 7, as was the case in Goren et al. \cite{goren2020ranking}. We do not report the detailed results of this phase due to space limitations in the paper.
}


\myparagraph{Online evaluation} The evaluation performed over the
\firstDataset and \secondDataset datasets is offline and therefore
spans a single round: the students who competed in the competition did
not respond to rankings induced over the documents we modify here. We
therefore also performed online evaluation where our instantiated
prompts competed as bots against students. We organized a ranking
competition\footnote{The competition was approved by institution and international ethics committees.} similar to that of Mordo et al. \cite{Mordo+al:25a} using 15
queries from TREC9-TREC12\footnote{These are different queries than
  those used in the \secondDataset dataset: 21, 55, 61, 64, 74, 75, 83, 96, 124,
  144, 161, 164, 166, 170, 194.}. In contrast to Mordo et al.'s
competitions \cite{Mordo+al:25a}, each game included 5 players: two-three
students, one of the two bots discussed above (Pairwise or
Listwise), and one or two static documents were created using a procedure similar to the one in Raifer et al. \cite{raifer2017information}: first, we used the query in the English Wikipedia search engine and selected a highly ranked page. We then extracted a candidate paragraph from this page, with a length of up to 150 words. Three annotators assessed the relevance of the passages, and we repeated the extraction process for each query until at least two annotators judged a paragraph as relevant. The selected paragraph was then used as a static document for the query for all students.

The students were not aware that they were competing
against bots. We applied our bots in rounds 5\footnote{Due to
  technical issues, we could not run the bots at round 4 as in the offline evaluation.}, 6 and 7 and report the
average performance over these three rounds.

We had documents in the online
evaluation judged for relevance and quality using crowdsourcing
annotators on the Connect platform of CloudResearch
\cite{noauthor_introducing_2024}. Following past work on ranking
competitions \cite{raifer2017information,goren2020ranking}, a
document's quality grade is set to $1$ if at least three out of five
English-speaking annotators marked it as valid (as opposed to keyword
stuffed or useless) and to $0$ otherwise. The relevance grade was $1$ if the document was
marked relevant by at least three annotators and $0$ otherwise. 







\endinput


We adopt an evaluation approach similar to that of Goren et al. \cite{goren2020ranking}. Two evaluation settings are considered: (i) Offline evaluation, where we leveraged existing datasets from ranking competitions, and (ii) Online evaluation, where a set of \bt s, each with a specific \contextualized, participate as a player in an ongoing ranking competition. The offline evaluation is run only for a single round, since the students did not respond to rankings that included the documents produced by our \bt s. In the online setting, other players may modify their documents simultaneously while our agents make their own modifications. In this section, we begin by describing the datasets used in our experiments (Section \ref{sec_datasets}). We then present the setups for both offline and online evaluations (Section \ref{sec_exp_set}). Finally, we outline the evaluation measures employed to assess performance of the \bt s and compare their performances against other types of agents (Section \ref{sec:eval-measures}).







\subsection{Datasets}\label{sec_datasets}
To perform offline and online evaluation, we deployed our approach in three different ranking competitions: one competition with feature-based ranking function, and two other utilizing transformer-based ranking function. The datasets employed in our experiments are as follows:

\myparagraph{Dataset 1} The first competition utilized for offline evaluation and comparison with the baseline model proposed by Goren et al. \cite{goren2020ranking}. It was organized by Raifer et al. \cite{raifer2017information}. In this competition, students enrolled in a course served as authors of documents and were assigned to 31 queries from the TREC9-TREC12 Web tracks. Each query defined a repeated-ranking-game. Students were incentivized with course grade bonuses and were asked to modify their documents for 8 rounds so that their document will be highly ranked for the played query. We selected round 7 for the offline evaluation, following Goren et al. \cite{goren2020ranking}. A total of 31 repeated-games (one per query) were conducted. A LambdaMART ranking function was applied \cite{burges2010lambdamart}.

\myparagraph{Dataset 2} This dataset used for offline evaluation and hyper-parameter tuning for the online evaluation. It was sourced from a ranking competition conducted by Mordo et al. \cite{div}. It involved a competition with 15 queries\footnote{From the TREC9-TREC12 Web Track as well.}, 7 rounds, and 4 players per game. In contrast to the competition described by Raifer et al. \cite{raifer2017information}, a transformer based ranking function was applied: the (unsupervised) E5 \footnote{The intfloat/e5-large-unsupervised version from the Hugging Face repository was used
  (\url{https://huggingface.co/intfloat/e5-large-unsupervised}).} \cite{wang2022text}. We focus on round 4, as it is the first round where we can apply our \bt s (recall that our \bt s require the context of previous rounds to modify a document).

\myparagraph{Dataset 3} We organize a ranking competition using 15 queries\footnote{From TREC9-TREC12; Different queries comparing to those used in Dataset 2: [21, 55, 61, 64, 74, 75, 83, 96, 124, 144, 161, 164, 166, 170, 194].}. The setup of this competition was similar to that of Mordo et al \cite{div}, with the following key difference: each games included 5 players. From round 5 \footnote{We initially planned to introduce our \bt s in round 4 as in Dataset 2; however, due to experimental constraints, we began their application in round 5.} of the competition, the players in each group consisted of: one \bt, two or three students and one or two planted documents\footnote{The same document as the initial document every participant started with.}. From the perspective of the students, the inclusion of \bt s did not alter the structure or appearance of the competition, preserving the integrity of the evaluation.

\subsection{Experimental setting}\label{sec_exp_set}

Our document modification approach operates as follows: first, the ranking for a given query is observed. Next, the approach modifies a specific document with the aim that the resulting document will be ranked higher in the next round of the game. In dynamic (online) settings, other documents may also be modified simultaneously, influencing the subsequent ranking. We design two evaluation paradigms—online and offline—both of which simulate a dynamic setting. The approach introduced by Goren et al. \cite{goren2020ranking} serves as a baseline to our approach.

\myparagraph{Offline evaluation}
The offline evaluation is divided into three phases. In each phase, we evaluated an \bt{} using a similar approach employed by Goren et al. \cite{goren2020ranking}: (i) Select a round and a game (query); (ii) Modify a document using the tested modification method (baseline \cite{goren2020ranking} or \bt{} with a specific prompt), excluding the top-ranked document. The exclusion of top-ranked documents is attributed to previous findings that their authors tend to avoid modifying their documents \cite{raifer2017information}. (iii) Evaluate the performance of the agent with respect to all other documents in the ranked list. (iv) Iterate over all the documents in the selected round and query and average the computed measure. 

% The performance of each metric is computed as the average over queries.

The goal of the first phase is to identify a representative prompt for each class of prompts--that is, the prompt whose resultant agents maximize a metric related to ranking promotion. We conduct a comprehensive grid search over the 225 configurations described in Section \ref{sec:bots}, using Dataset 1. For this phase, we utilized Llama-2 with 13B parameters due to its availability \cite{touvron2023llama}. From each configuration, we constructed five agents, each with a different temperature setting for the probability model of the LLM\footnote{All other parameters of the LLM were fixed.}. The temperatures used were \{0, 0.5, 1, 1.5, 2\} and were selected based on the work of Peeperkorn et al \cite{peeperkorn_is_2024}. The selected agents compete for round 7, as was the case in Goren et al. \cite{goren2020ranking}. We do not report the detailed results of this phase due to space limitations in the paper.

%In the second phase, we evaluated the performance of each representative prompt (Identified in Phase 1) on the same dataset and specifically on round 7, incorporating two key modifications: (i) we replaced Llama-2 with Chat-GPT 4o \cite{achiam2023gpt} as the latter demonstrated superior performance across multiple benchmarks\footnote{\url{https://docsbot.ai/models/compare/gpt-4o/llama-2-chat-13b}}. (ii) We included a baseline model introduced by Goren et al. \cite{goren2020ranking}, which modifies documents by replacing passages. Our implementation of the baseline consist of a primary difference: we replaced RankSVM \cite{joachims2002ranksvm} with LambdaMART \cite{burges2010lambdamart} due to its superior performance on Dataset 1. Details regarding the reproducibility process are omitted.

In the third phase, we evaluated the performance of each representative prompt on Dataset 2, which contains data from a ranking competition with a transformer-based ranking function. The evaluation procedure mirrored that of the second phase. We focused on round 4, as the round for evaluation.

\myparagraph{Online evaluation}
We adopted the two best performing \bt s (in terms of ranking promotion metric) evaluated on the transformer-based competition (Dataset 2) and assigned them as players in a similar ranking competition with different queries (resulting in Dataset 3). These \bt s joined the competition in round 5 and competed for the highest ranking in rounds 5, 6 and 7. Recall that the decision to introduce the \bt s in round 5, rather than at the beginning, was based on their dependency on past rankings, which were integrated into the \contextualized s to guide document modifications. The selection of round 5 over round 4 was due to experimental constraints.

% This article addresses the challenge of white-hat ranking-incentivized modifications, building on the work of Goren et al. \cite{goren2020ranking}, who explored this topic in offline and online competition settings on a ranking competition dataset comprised by Raifer et al. \cite{raifer2017information}. In addition to closely mimicking their approach, which utilized the feature-based LambdaMART ranker \cite{burges2010ranknet} and an LTR-based baseline, referred as the "feature-based" setting in this article, we adopt a transformer-based ranker—specifically, the E5 model introduced by Wang et al. \cite{wang2024multilingual}, both for offline and online evaluation. The E5 settings are referred to as the offline and online "transformer-based" settings in this article.

% Our study investigates the effectiveness of ranking-incentivized modifications within a comparable framework while leveraging the advantages of transformer-based models. The primary goal of this research is to evaluate strategies for rank promotion, focusing on leveraging large language models (LLMs) to implement modifications using various few-shot \cite{brown2020language} contextual approaches. By incorporating LLM-based methodologies, we aim to assess their ability to generate high-quality, contextually relevant modifications that adhere to the principles of white-hat ranking practices.


% \paragraph*{Dataset Creation}
% For the feature-based setting, we utilized the dataset created by Raifer et al. \cite{raifer2017information}, similarly to Goren et al. \cite{goren2020ranking}.

% In the transformer-based settings, we implemented and evaluated our \bt s within a ranking environment inspired by the 'ranking competition' framework introduced by Raifer et al. \cite{raifer2017information}. Similar to their approach, our ranking competition focused on optimizing documents for a black-box ranker. However, we introduced several adjustments to align with our experimental objectives and constraints.

% First, we utilized a subset of 15 queries derived from the TREC ClueWeb09 dataset \cite{clueweb09}, whereas Raifer et al. \cite{raifer2017information} used a broader set. This choice enabled us to conduct multiple experiments in parallel while maintaining a manageable workload and scalability. Additionally, we adhered to the original competition's guidelines, requiring concise, 150-word plain English submissions without links, special characters, or HTML tags, to ensure methodological consistency.

% The competition was structured into "matches" and "rounds." A match refers to a grouping of four competitors who worked on a single query. In each match, participants edited their texts to achieve the highest possible ranking for the query. A round is one iteration of competition during which all matches were conducted simultaneously for a specific query set. Each round provided participants with feedback, allowing them to see their own rankings as well as those of their competitors.

% The competition was divided into two parts, with each part consisting of seven rounds. In the first part, participants worked with 15 queries \cite{partA2024}. In the second part, these queries were replaced with 15 different ones, also sourced from ClueWeb09 \cite{clueweb09} \cite{partB2024 (TBA)}. The grouping of competitors and conditions remained fixed across rounds. We used the first part of the competition for offline evaluation and the second part for online evaluation, enabling a thorough analysis of our methodology under both controlled and dynamic conditions.

% By tailoring the competition to our needs while preserving its core principles, we ensured both comparability to prior work and the validity of our findings in the context of scalable and rigorous experimentation.


% \paragraph*{Offline Evaluation}
% The feature-based setting we developed was heavily influenced by the offline setting described in detail by Goren et al.\ \cite{goren2020ranking}. The evaluation of this setting was carried out meticulously, adhering closely to the methodology outlined in Goren et al.'s \cite{goren2020ranking} offline evaluation section.

% To rigorously assess our LLM-based ranking-incentivized modification methodology in the offline transformer-based setting, we constructed an evaluation setting inspired by the offline setting introduced by Goren et al.\ \cite{goren2020ranking}. Our experiments were conducted on documents initially ranked 2nd, 3rd, and 4th in the fifth round of a competition designed to rank documents against a shared set of 15 queries. These ranks were chosen deliberately, as they represented non-top-performing documents, providing a meaningful opportunity to evaluate the potential for improvement when modifications were applied, as suggested by Goren et al.\ \cite{goren2020ranking}.

% In each round of evaluation, four documents were subjected to ranking. Three of these were unaltered, human-authored documents selected from previous rankings in the competition. The fourth document was a modified version, generated by applying our LLM-based methodology. By incorporating multiple initial ranking positions and diverse queries, we ensured that the evaluation was not overly influenced by specific document characteristics or query types. This approach enhanced the generalizability of our findings.


% \paragraph*{Online Evaluation}
% To closely mimic the online experiment described by Goren et al.\ \cite{goren2020ranking}, we conducted the second part of the ranking competition using the same participant pool but a different set of 15 queries sourced from ClueWeb09 \cite{clueweb09}.

% For the online evaluation, we selected the two \bt{} methods that achieved the highest scores in the offline evaluation during one of our early experiments, the results of which are not depicted in this paper. We used "scaled rank promotion" as the metric for determining their performance. These \bt{}s were introduced into the competition starting from the 4th round of the second part, as their methodology required contextual information derived from at least three prior rounds to function effectively. The bots competed against the same set of human participants, with groupings and conditions held constant across all seven rounds of this phase. This ensured consistency in the evaluation and allowed for a direct comparison between \bt{}- and human-authored rankings.

% From the perspective of the human participants, the inclusion of \bt s did not alter the structure or appearance of the competition. Each round followed the same format and task descriptions as in the first part of the competition. This design ensured that human participants approached their ranking tasks without being influenced by knowledge of the \bt s' involvement, preserving the integrity of the evaluation.

% \paragraph*{Ranker and Document Embeddings}
% In the feature-based setting, we employed the same methodology and features previously utilized by Raifer et al.\ \cite{raifer2017information} and Goren et al.\ \cite{goren2020ranking}. Specifically, we used the exact trained LambdaMART model that was trained and used by Goren et al.\ \cite{goren2020ranking}. The method for text embeddings for this setting replicated Goren et al.'s \cite{goren2020ranking} approach - incorporating 25 content-based features. These features were selected either from those used in Microsoft's learning-to-rank datasets \cite{mslr}, or as query-independent measures of document quality. Notably, these included stopword-based metrics and the entropy of term distribution within a document, both of which have been proven effective in web retrieval scenarios.

% For the transformer-based offline and online settings, we utilized E5 as the ranker and the E5 embedder for the document embeddings.


% \paragraph*{Baseline}
% To establish a robust baseline, we implemented a ranking passage pairs approach, closely mirroring the methodology described in Goren et al. \cite{goren2020ranking}, with the primary difference being the replacement of RankSVM \cite{joachims2002ranksvm} with LambdaMART \cite{burges2010lambdamart}. LambdaMART was selected based on its superior empirical performance in prior evaluations. The dataset, features, and labels remained identical to the original setup. Labels for the passage pairs were generated using a dual-objective harmonic mean approach introduced in Goren et al. \cite{goren2020ranking}, integrating rank-promotion and local coherence objectives, where rank-promotion labels ranged from 0 to 4 based on positional improvement in rankings, and coherence labels quantified semantic similarity to maintain content quality. The harmonic mean was computed with $\beta = 1$, assigning equal weight to both objectives.

% Training was conducted on 57 documents extracted from Round 6 of the original competition dataset introduced by Raifer et al. \cite{raifer2017information}. The model’s performance was subsequently evaluated on 124 experimental settings, derived from Round 7 of the ranking competition, spanning documents ranked 2–5 across 31 queries. The validation set was configured similarly to Goren et al.'s \cite{goren2020ranking} procedure. Both training and validation utilized NDCG@1 as the evaluation metric, contrasting with the NDCG@5 used by Goren et al. \cite{goren2020ranking}, to align with the goal of selecting the top sentence-swapped document.

% We implemented LambdaMART \cite{burges2010lambdamart} using its default parameter settings for several features, specifically: \begin{itemize} \item \textbf{Minimum leaf support:} Minimum number of samples each leaf must contain, set to 1 (default). \item \textbf{Number of threshold candidates for tree splitting:} Set to 256 (default). \item \textbf{Early stopping rounds:} Set to 100 (default). \end{itemize}

% We conducted a grid search to tune other hyper-parameters, exploring different configurations. This grid search included: \begin{itemize} 
%     \item \textbf{Number of Trees:} 50, 500, 1000, 1200. 
%     \item \textbf{Number of Leaves per Tree:} 10, 50, 100. 
%     \item \textbf{Learning Rate (Shrinkage Value):} 0.01, 0.1, 0.2. 
% \end{itemize}


\begin{table*}[t]
\centering
\fontsize{11pt}{11pt}\selectfont
\begin{tabular}{lllllllllllll}
\toprule
\multicolumn{1}{c}{\textbf{task}} & \multicolumn{2}{c}{\textbf{Mir}} & \multicolumn{2}{c}{\textbf{Lai}} & \multicolumn{2}{c}{\textbf{Ziegen.}} & \multicolumn{2}{c}{\textbf{Cao}} & \multicolumn{2}{c}{\textbf{Alva-Man.}} & \multicolumn{1}{c}{\textbf{avg.}} & \textbf{\begin{tabular}[c]{@{}l@{}}avg.\\ rank\end{tabular}} \\
\multicolumn{1}{c}{\textbf{metrics}} & \multicolumn{1}{c}{\textbf{cor.}} & \multicolumn{1}{c}{\textbf{p-v.}} & \multicolumn{1}{c}{\textbf{cor.}} & \multicolumn{1}{c}{\textbf{p-v.}} & \multicolumn{1}{c}{\textbf{cor.}} & \multicolumn{1}{c}{\textbf{p-v.}} & \multicolumn{1}{c}{\textbf{cor.}} & \multicolumn{1}{c}{\textbf{p-v.}} & \multicolumn{1}{c}{\textbf{cor.}} & \multicolumn{1}{c}{\textbf{p-v.}} &  &  \\ \midrule
\textbf{S-Bleu} & 0.50 & 0.0 & 0.47 & 0.0 & 0.59 & 0.0 & 0.58 & 0.0 & 0.68 & 0.0 & 0.57 & 5.8 \\
\textbf{R-Bleu} & -- & -- & 0.27 & 0.0 & 0.30 & 0.0 & -- & -- & -- & -- & - &  \\
\textbf{S-Meteor} & 0.49 & 0.0 & 0.48 & 0.0 & 0.61 & 0.0 & 0.57 & 0.0 & 0.64 & 0.0 & 0.56 & 6.1 \\
\textbf{R-Meteor} & -- & -- & 0.34 & 0.0 & 0.26 & 0.0 & -- & -- & -- & -- & - &  \\
\textbf{S-Bertscore} & \textbf{0.53} & 0.0 & {\ul 0.80} & 0.0 & \textbf{0.70} & 0.0 & {\ul 0.66} & 0.0 & {\ul0.78} & 0.0 & \textbf{0.69} & \textbf{1.7} \\
\textbf{R-Bertscore} & -- & -- & 0.51 & 0.0 & 0.38 & 0.0 & -- & -- & -- & -- & - &  \\
\textbf{S-Bleurt} & {\ul 0.52} & 0.0 & {\ul 0.80} & 0.0 & 0.60 & 0.0 & \textbf{0.70} & 0.0 & \textbf{0.80} & 0.0 & {\ul 0.68} & {\ul 2.3} \\
\textbf{R-Bleurt} & -- & -- & 0.59 & 0.0 & -0.05 & 0.13 & -- & -- & -- & -- & - &  \\
\textbf{S-Cosine} & 0.51 & 0.0 & 0.69 & 0.0 & {\ul 0.62} & 0.0 & 0.61 & 0.0 & 0.65 & 0.0 & 0.62 & 4.4 \\
\textbf{R-Cosine} & -- & -- & 0.40 & 0.0 & 0.29 & 0.0 & -- & -- & -- & -- & - & \\ \midrule
\textbf{QuestEval} & 0.23 & 0.0 & 0.25 & 0.0 & 0.49 & 0.0 & 0.47 & 0.0 & 0.62 & 0.0 & 0.41 & 9.0 \\
\textbf{LLaMa3} & 0.36 & 0.0 & \textbf{0.84} & 0.0 & {\ul{0.62}} & 0.0 & 0.61 & 0.0 &  0.76 & 0.0 & 0.64 & 3.6 \\
\textbf{our (3b)} & 0.49 & 0.0 & 0.73 & 0.0 & 0.54 & 0.0 & 0.53 & 0.0 & 0.7 & 0.0 & 0.60 & 5.8 \\
\textbf{our (8b)} & 0.48 & 0.0 & 0.73 & 0.0 & 0.52 & 0.0 & 0.53 & 0.0 & 0.7 & 0.0 & 0.59 & 6.3 \\  \bottomrule
\end{tabular}
\caption{Pearson correlation on human evaluation on system output. `R-': reference-based. `S-': source-based.}
\label{tab:sys}
\end{table*}



\begin{table}%[]
\centering
\fontsize{11pt}{11pt}\selectfont
\begin{tabular}{llllll}
\toprule
\multicolumn{1}{c}{\textbf{task}} & \multicolumn{1}{c}{\textbf{Lai}} & \multicolumn{1}{c}{\textbf{Zei.}} & \multicolumn{1}{c}{\textbf{Scia.}} & \textbf{} & \textbf{} \\ 
\multicolumn{1}{c}{\textbf{metrics}} & \multicolumn{1}{c}{\textbf{cor.}} & \multicolumn{1}{c}{\textbf{cor.}} & \multicolumn{1}{c}{\textbf{cor.}} & \textbf{avg.} & \textbf{\begin{tabular}[c]{@{}l@{}}avg.\\ rank\end{tabular}} \\ \midrule
\textbf{S-Bleu} & 0.40 & 0.40 & 0.19* & 0.33 & 7.67 \\
\textbf{S-Meteor} & 0.41 & 0.42 & 0.16* & 0.33 & 7.33 \\
\textbf{S-BertS.} & {\ul0.58} & 0.47 & 0.31 & 0.45 & 3.67 \\
\textbf{S-Bleurt} & 0.45 & {\ul 0.54} & {\ul 0.37} & 0.45 & {\ul 3.33} \\
\textbf{S-Cosine} & 0.56 & 0.52 & 0.3 & {\ul 0.46} & {\ul 3.33} \\ \midrule
\textbf{QuestE.} & 0.27 & 0.35 & 0.06* & 0.23 & 9.00 \\
\textbf{LlaMA3} & \textbf{0.6} & \textbf{0.67} & \textbf{0.51} & \textbf{0.59} & \textbf{1.0} \\
\textbf{Our (3b)} & 0.51 & 0.49 & 0.23* & 0.39 & 4.83 \\
\textbf{Our (8b)} & 0.52 & 0.49 & 0.22* & 0.43 & 4.83 \\ \bottomrule
\end{tabular}
\caption{Pearson correlation on human ratings on reference output. *not significant; we cannot reject the null hypothesis of zero correlation}
\label{tab:ref}
\end{table}


\begin{table*}%[]
\centering
\fontsize{11pt}{11pt}\selectfont
\begin{tabular}{lllllllll}
\toprule
\textbf{task} & \multicolumn{1}{c}{\textbf{ALL}} & \multicolumn{1}{c}{\textbf{sentiment}} & \multicolumn{1}{c}{\textbf{detoxify}} & \multicolumn{1}{c}{\textbf{catchy}} & \multicolumn{1}{c}{\textbf{polite}} & \multicolumn{1}{c}{\textbf{persuasive}} & \multicolumn{1}{c}{\textbf{formal}} & \textbf{\begin{tabular}[c]{@{}l@{}}avg. \\ rank\end{tabular}} \\
\textbf{metrics} & \multicolumn{1}{c}{\textbf{cor.}} & \multicolumn{1}{c}{\textbf{cor.}} & \multicolumn{1}{c}{\textbf{cor.}} & \multicolumn{1}{c}{\textbf{cor.}} & \multicolumn{1}{c}{\textbf{cor.}} & \multicolumn{1}{c}{\textbf{cor.}} & \multicolumn{1}{c}{\textbf{cor.}} &  \\ \midrule
\textbf{S-Bleu} & -0.17 & -0.82 & -0.45 & -0.12* & -0.1* & -0.05 & -0.21 & 8.42 \\
\textbf{R-Bleu} & - & -0.5 & -0.45 &  &  &  &  &  \\
\textbf{S-Meteor} & -0.07* & -0.55 & -0.4 & -0.01* & 0.1* & -0.16 & -0.04* & 7.67 \\
\textbf{R-Meteor} & - & -0.17* & -0.39 & - & - & - & - & - \\
\textbf{S-BertScore} & 0.11 & -0.38 & -0.07* & -0.17* & 0.28 & 0.12 & 0.25 & 6.0 \\
\textbf{R-BertScore} & - & -0.02* & -0.21* & - & - & - & - & - \\
\textbf{S-Bleurt} & 0.29 & 0.05* & 0.45 & 0.06* & 0.29 & 0.23 & 0.46 & 4.2 \\
\textbf{R-Bleurt} & - &  0.21 & 0.38 & - & - & - & - & - \\
\textbf{S-Cosine} & 0.01* & -0.5 & -0.13* & -0.19* & 0.05* & -0.05* & 0.15* & 7.42 \\
\textbf{R-Cosine} & - & -0.11* & -0.16* & - & - & - & - & - \\ \midrule
\textbf{QuestEval} & 0.21 & {\ul{0.29}} & 0.23 & 0.37 & 0.19* & 0.35 & 0.14* & 4.67 \\
\textbf{LlaMA3} & \textbf{0.82} & \textbf{0.80} & \textbf{0.72} & \textbf{0.84} & \textbf{0.84} & \textbf{0.90} & \textbf{0.88} & \textbf{1.00} \\
\textbf{Our (3b)} & 0.47 & -0.11* & 0.37 & 0.61 & 0.53 & 0.54 & 0.66 & 3.5 \\
\textbf{Our (8b)} & {\ul{0.57}} & 0.09* & {\ul 0.49} & {\ul 0.72} & {\ul 0.64} & {\ul 0.62} & {\ul 0.67} & {\ul 2.17} \\ \bottomrule
\end{tabular}
\caption{Pearson correlation on human ratings on our constructed test set. 'R-': reference-based. 'S-': source-based. *not significant; we cannot reject the null hypothesis of zero correlation}
\label{tab:con}
\end{table*}

\section{Results}
We benchmark the different metrics on the different datasets using correlation to human judgement. For content preservation, we show results split on data with system output, reference output and our constructed test set: we show that the data source for evaluation leads to different conclusions on the metrics. In addition, we examine whether the metrics can rank style transfer systems similar to humans. On style strength, we likewise show correlations between human judgment and zero-shot evaluation approaches. When applicable, we summarize results by reporting the average correlation. And the average ranking of the metric per dataset (by ranking which metric obtains the highest correlation to human judgement per dataset). 

\subsection{Content preservation}
\paragraph{How do data sources affect the conclusion on best metric?}
The conclusions about the metrics' performance change radically depending on whether we use system output data, reference output, or our constructed test set. Ideally, a good metric correlates highly with humans on any data source. Ideally, for meta-evaluation, a metric should correlate consistently across all data sources, but the following shows that the correlations indicate different things, and the conclusion on the best metric should be drawn carefully.

Looking at the metrics correlations with humans on the data source with system output (Table~\ref{tab:sys}), we see a relatively high correlation for many of the metrics on many tasks. The overall best metrics are S-BertScore and S-BLEURT (avg+avg rank). We see no notable difference in our method of using the 3B or 8B model as the backbone.

Examining the average correlations based on data with reference output (Table~\ref{tab:ref}), now the zero-shoot prompting with LlaMA3 70B is the best-performing approach ($0.59$ avg). Tied for second place are source-based cosine embedding ($0.46$ avg), BLEURT ($0.45$ avg) and BertScore ($0.45$ avg). Our method follows on a 5. place: here, the 8b version (($0.43$ avg)) shows a bit stronger results than 3b ($0.39$ avg). The fact that the conclusions change, whether looking at reference or system output, confirms the observations made by \citet{scialom-etal-2021-questeval} on simplicity transfer.   

Now consider the results on our test set (Table~\ref{tab:con}): Several metrics show low or no correlation; we even see a significantly negative correlation for some metrics on ALL (BLEU) and for specific subparts of our test set for BLEU, Meteor, BertScore, Cosine. On the other end, LlaMA3 70B is again performing best, showing strong results ($0.82$ in ALL). The runner-up is now our 8B method, with a gap to the 3B version ($0.57$ vs $0.47$ in ALL). Note our method still shows zero correlation for the sentiment task. After, ranks BLEURT ($0.29$), QuestEval ($0.21$), BertScore ($0.11$), Cosine ($0.01$).  

On our test set, we find that some metrics that correlate relatively well on the other datasets, now exhibit low correlation. Hence, with our test set, we can now support the logical reasoning with data evidence: Evaluation of content preservation for style transfer needs to take the style shift into account. This conclusion could not be drawn using the existing data sources: We hypothesise that for the data with system-based output, successful output happens to be very similar to the source sentence and vice versa, and reference-based output might not contain server mistakes as they are gold references. Thus, none of the existing data sources tests the limits of the metrics.  


\paragraph{How do reference-based metrics compare to source-based ones?} Reference-based metrics show a lower correlation than the source-based counterpart for all metrics on both datasets with ratings on references (Table~\ref{tab:sys}). As discussed previously, reference-based metrics for style transfer have the drawback that many different good solutions on a rewrite might exist and not only one similar to a reference.


\paragraph{How well can the metrics rank the performance of style transfer methods?}
We compare the metrics' ability to judge the best style transfer methods w.r.t. the human annotations: Several of the data sources contain samples from different style transfer systems. In order to use metrics to assess the quality of the style transfer system, metrics should correctly find the best-performing system. Hence, we evaluate whether the metrics for content preservation provide the same system ranking as human evaluators. We take the mean of the score for every output on each system and the mean of the human annotations; we compare the systems using the Kendall's Tau correlation. 

We find only the evaluation using the dataset Mir, Lai, and Ziegen to result in significant correlations, probably because of sparsity in a number of system tests (App.~\ref{app:dataset}). Our method (8b) is the only metric providing a perfect ranking of the style transfer system on the Lai data, and Llama3 70B the only one on the Ziegen data. Results in App.~\ref{app:results}. 


\subsection{Style strength results}
%Evaluating style strengths is a challenging task. 
Llama3 70B shows better overall results than our method. However, our method scores higher than Llama3 70B on 2 out of 6 datasets, but it also exhibits zero correlation on one task (Table~\ref{tab:styleresults}).%More work i s needed on evaluating style strengths. 
 
\begin{table}%[]
\fontsize{11pt}{11pt}\selectfont
\begin{tabular}{lccc}
\toprule
\multicolumn{1}{c}{\textbf{}} & \textbf{LlaMA3} & \textbf{Our (3b)} & \textbf{Our (8b)} \\ \midrule
\textbf{Mir} & 0.46 & 0.54 & \textbf{0.57} \\
\textbf{Lai} & \textbf{0.57} & 0.18 & 0.19 \\
\textbf{Ziegen.} & 0.25 & 0.27 & \textbf{0.32} \\
\textbf{Alva-M.} & \textbf{0.59} & 0.03* & 0.02* \\
\textbf{Scialom} & \textbf{0.62} & 0.45 & 0.44 \\
\textbf{\begin{tabular}[c]{@{}l@{}}Our Test\end{tabular}} & \textbf{0.63} & 0.46 & 0.48 \\ \bottomrule
\end{tabular}
\caption{Style strength: Pearson correlation to human ratings. *not significant; we cannot reject the null hypothesis of zero corelation}
\label{tab:styleresults}
\end{table}

\subsection{Ablation}
We conduct several runs of the methods using LLMs with variations in instructions/prompts (App.~\ref{app:method}). We observe that the lower the correlation on a task, the higher the variation between the different runs. For our method, we only observe low variance between the runs.
None of the variations leads to different conclusions of the meta-evaluation. Results in App.~\ref{app:results}.
\section{Conclusion}
In this work, we propose a simple yet effective approach, called SMILE, for graph few-shot learning with fewer tasks. Specifically, we introduce a novel dual-level mixup strategy, including within-task and across-task mixup, for enriching the diversity of nodes within each task and the diversity of tasks. Also, we incorporate the degree-based prior information to learn expressive node embeddings. Theoretically, we prove that SMILE effectively enhances the model's generalization performance. Empirically, we conduct extensive experiments on multiple benchmarks and the results suggest that SMILE significantly outperforms other baselines, including both in-domain and cross-domain few-shot settings.
\section{Limitation}
The use of 3D-printed PLA for structural components improves improving ease of assembly and reduces weight and cost, yet it causes deformation under heavy load, which can diminish end-effector precision. Using metal, such as aluminum, would remedy this problem. Additionally, \robot relies on integrated joint relative encoders, requiring manual initialization in a fixed joint configuration each time the system is powered on. Using absolute joint encoders could significantly improve accuracy and ease of use, although it would increase the overall cost. 

%Reliance on commercially available actuators simplifies integration but imposes constraints on control frequency and customization, further limiting the potential for tailored performance improvements.

% The 6 DoF configuration provides sufficient mobility for most tasks; however, certain bimanual operations could benefit from an additional degree of freedom to handle complex joint constraints more effectively. Furthermore, the limited torque density of commercially available proprioceptive actuators restricts the payload and torque output, making the system less suitability for handling heavier loads or high-torque applications. 

The 6 DoF configuration of the arm provides sufficient mobility for single-arm manipulation tasks, yet it shows a limitation in certain bimanual manipulation problems. Specifically, when \robot holds onto a rigid object with both hands, each arm loses 1 DoF because the hands are fixed to the object during grasping. This leads to an underactuated kinematic chain which has a limited mobility in 3D space. We can achieve more mobility by letting the object slip inside the grippers, yet this renders the grasp less robust and simulation difficult. Therefore, we anticipate that designing a lightweight 3 DoF wrist in place of the current 2 DoF wrist allows a more diverse repertoire of manipulation in bimanual tasks.

Finally, the limited torque density of commercially available proprioceptive actuators restricts the performance. Currently, all of our actuators feature a 1:10 gear ratio, so \robot can handle up to 2.5 kg of payload. To handle a heavier object and manipulate it with higher torque, we expect the actuator to have 1:20$\sim$30 gear ratio, but it is difficult to find an off-the-shelf product that meets our requirements. Customizing the actuator to increase the torque density while minimizing the weight will enable \robot to move faster and handle more diverse objects.

%These constraints highlight opportunities for improvement in future iterations, including alternative materials for enhanced rigidity, custom actuator designs for higher control precision and torque density, the adoption of absolute joint encoders, and optimized configurations to balance dexterity and weight.



\section*{Ethics Statement}
This study evaluates small language models using standardized benchmarks and publicly available datasets, ensuring transparency and reproducibility. No private or sensitive data was used, and all models were assessed under fair conditions. We acknowledge potential biases in LLM-based evaluations and encourage further research for mitigation.

\section*{Acknowledgements}
This work was supported by the NSF NAIRR Pilot with PSC Neocortex and NCSA Delta, Cisco Research, Amazon, Schmidt Science, the Commonwealth Cyber Initiative, the Amazon–Virginia Tech Center for Efficient and Robust Machine Learning, the Sanghani Center for AI and Data Analytics at Virginia Tech, the Virginia Tech Innovation Campus, Children's National Hospital, and the Fralin Biomedical Research Institute at Virginia Tech. S. C. acknowledges support from Schmidt Science. The views, findings, conclusions, and recommendations expressed in this work are those of the authors and do not necessarily reflect the opinions of the funding agencies.

% This work is sponsored by the NSF NAIRR Pilot with PSC Neocortex and NCSA Delta, Cisco Research, Amazon, Schmidt Science, Commonwealth Cyber Initiative, Amazon + Virginia Tech Center for Efficient and Robust Machine Learning, Sanghani Center for AI and Data Analytics (Virginia Tech), Virginia Tech Innovation Campus, Children’s National Hospital and Fralin Biomedical Research Institute (Virginia Tech). S. C. acknowledges the support from Schmidt Science. Any opinions, findings, and conclusions or recommendations expressed in this work are those of the author(s) and do not necessarily reflect the views of the funding agencies.







\bibliography{custom}


\clearpage
\appendix

\subsection{Lloyd-Max Algorithm}
\label{subsec:Lloyd-Max}
For a given quantization bitwidth $B$ and an operand $\bm{X}$, the Lloyd-Max algorithm finds $2^B$ quantization levels $\{\hat{x}_i\}_{i=1}^{2^B}$ such that quantizing $\bm{X}$ by rounding each scalar in $\bm{X}$ to the nearest quantization level minimizes the quantization MSE. 

The algorithm starts with an initial guess of quantization levels and then iteratively computes quantization thresholds $\{\tau_i\}_{i=1}^{2^B-1}$ and updates quantization levels $\{\hat{x}_i\}_{i=1}^{2^B}$. Specifically, at iteration $n$, thresholds are set to the midpoints of the previous iteration's levels:
\begin{align*}
    \tau_i^{(n)}=\frac{\hat{x}_i^{(n-1)}+\hat{x}_{i+1}^{(n-1)}}2 \text{ for } i=1\ldots 2^B-1
\end{align*}
Subsequently, the quantization levels are re-computed as conditional means of the data regions defined by the new thresholds:
\begin{align*}
    \hat{x}_i^{(n)}=\mathbb{E}\left[ \bm{X} \big| \bm{X}\in [\tau_{i-1}^{(n)},\tau_i^{(n)}] \right] \text{ for } i=1\ldots 2^B
\end{align*}
where to satisfy boundary conditions we have $\tau_0=-\infty$ and $\tau_{2^B}=\infty$. The algorithm iterates the above steps until convergence.

Figure \ref{fig:lm_quant} compares the quantization levels of a $7$-bit floating point (E3M3) quantizer (left) to a $7$-bit Lloyd-Max quantizer (right) when quantizing a layer of weights from the GPT3-126M model at a per-tensor granularity. As shown, the Lloyd-Max quantizer achieves substantially lower quantization MSE. Further, Table \ref{tab:FP7_vs_LM7} shows the superior perplexity achieved by Lloyd-Max quantizers for bitwidths of $7$, $6$ and $5$. The difference between the quantizers is clear at 5 bits, where per-tensor FP quantization incurs a drastic and unacceptable increase in perplexity, while Lloyd-Max quantization incurs a much smaller increase. Nevertheless, we note that even the optimal Lloyd-Max quantizer incurs a notable ($\sim 1.5$) increase in perplexity due to the coarse granularity of quantization. 

\begin{figure}[h]
  \centering
  \includegraphics[width=0.7\linewidth]{sections/figures/LM7_FP7.pdf}
  \caption{\small Quantization levels and the corresponding quantization MSE of Floating Point (left) vs Lloyd-Max (right) Quantizers for a layer of weights in the GPT3-126M model.}
  \label{fig:lm_quant}
\end{figure}

\begin{table}[h]\scriptsize
\begin{center}
\caption{\label{tab:FP7_vs_LM7} \small Comparing perplexity (lower is better) achieved by floating point quantizers and Lloyd-Max quantizers on a GPT3-126M model for the Wikitext-103 dataset.}
\begin{tabular}{c|cc|c}
\hline
 \multirow{2}{*}{\textbf{Bitwidth}} & \multicolumn{2}{|c|}{\textbf{Floating-Point Quantizer}} & \textbf{Lloyd-Max Quantizer} \\
 & Best Format & Wikitext-103 Perplexity & Wikitext-103 Perplexity \\
\hline
7 & E3M3 & 18.32 & 18.27 \\
6 & E3M2 & 19.07 & 18.51 \\
5 & E4M0 & 43.89 & 19.71 \\
\hline
\end{tabular}
\end{center}
\end{table}

\subsection{Proof of Local Optimality of LO-BCQ}
\label{subsec:lobcq_opt_proof}
For a given block $\bm{b}_j$, the quantization MSE during LO-BCQ can be empirically evaluated as $\frac{1}{L_b}\lVert \bm{b}_j- \bm{\hat{b}}_j\rVert^2_2$ where $\bm{\hat{b}}_j$ is computed from equation (\ref{eq:clustered_quantization_definition}) as $C_{f(\bm{b}_j)}(\bm{b}_j)$. Further, for a given block cluster $\mathcal{B}_i$, we compute the quantization MSE as $\frac{1}{|\mathcal{B}_{i}|}\sum_{\bm{b} \in \mathcal{B}_{i}} \frac{1}{L_b}\lVert \bm{b}- C_i^{(n)}(\bm{b})\rVert^2_2$. Therefore, at the end of iteration $n$, we evaluate the overall quantization MSE $J^{(n)}$ for a given operand $\bm{X}$ composed of $N_c$ block clusters as:
\begin{align*}
    \label{eq:mse_iter_n}
    J^{(n)} = \frac{1}{N_c} \sum_{i=1}^{N_c} \frac{1}{|\mathcal{B}_{i}^{(n)}|}\sum_{\bm{v} \in \mathcal{B}_{i}^{(n)}} \frac{1}{L_b}\lVert \bm{b}- B_i^{(n)}(\bm{b})\rVert^2_2
\end{align*}

At the end of iteration $n$, the codebooks are updated from $\mathcal{C}^{(n-1)}$ to $\mathcal{C}^{(n)}$. However, the mapping of a given vector $\bm{b}_j$ to quantizers $\mathcal{C}^{(n)}$ remains as  $f^{(n)}(\bm{b}_j)$. At the next iteration, during the vector clustering step, $f^{(n+1)}(\bm{b}_j)$ finds new mapping of $\bm{b}_j$ to updated codebooks $\mathcal{C}^{(n)}$ such that the quantization MSE over the candidate codebooks is minimized. Therefore, we obtain the following result for $\bm{b}_j$:
\begin{align*}
\frac{1}{L_b}\lVert \bm{b}_j - C_{f^{(n+1)}(\bm{b}_j)}^{(n)}(\bm{b}_j)\rVert^2_2 \le \frac{1}{L_b}\lVert \bm{b}_j - C_{f^{(n)}(\bm{b}_j)}^{(n)}(\bm{b}_j)\rVert^2_2
\end{align*}

That is, quantizing $\bm{b}_j$ at the end of the block clustering step of iteration $n+1$ results in lower quantization MSE compared to quantizing at the end of iteration $n$. Since this is true for all $\bm{b} \in \bm{X}$, we assert the following:
\begin{equation}
\begin{split}
\label{eq:mse_ineq_1}
    \tilde{J}^{(n+1)} &= \frac{1}{N_c} \sum_{i=1}^{N_c} \frac{1}{|\mathcal{B}_{i}^{(n+1)}|}\sum_{\bm{b} \in \mathcal{B}_{i}^{(n+1)}} \frac{1}{L_b}\lVert \bm{b} - C_i^{(n)}(b)\rVert^2_2 \le J^{(n)}
\end{split}
\end{equation}
where $\tilde{J}^{(n+1)}$ is the the quantization MSE after the vector clustering step at iteration $n+1$.

Next, during the codebook update step (\ref{eq:quantizers_update}) at iteration $n+1$, the per-cluster codebooks $\mathcal{C}^{(n)}$ are updated to $\mathcal{C}^{(n+1)}$ by invoking the Lloyd-Max algorithm \citep{Lloyd}. We know that for any given value distribution, the Lloyd-Max algorithm minimizes the quantization MSE. Therefore, for a given vector cluster $\mathcal{B}_i$ we obtain the following result:

\begin{equation}
    \frac{1}{|\mathcal{B}_{i}^{(n+1)}|}\sum_{\bm{b} \in \mathcal{B}_{i}^{(n+1)}} \frac{1}{L_b}\lVert \bm{b}- C_i^{(n+1)}(\bm{b})\rVert^2_2 \le \frac{1}{|\mathcal{B}_{i}^{(n+1)}|}\sum_{\bm{b} \in \mathcal{B}_{i}^{(n+1)}} \frac{1}{L_b}\lVert \bm{b}- C_i^{(n)}(\bm{b})\rVert^2_2
\end{equation}

The above equation states that quantizing the given block cluster $\mathcal{B}_i$ after updating the associated codebook from $C_i^{(n)}$ to $C_i^{(n+1)}$ results in lower quantization MSE. Since this is true for all the block clusters, we derive the following result: 
\begin{equation}
\begin{split}
\label{eq:mse_ineq_2}
     J^{(n+1)} &= \frac{1}{N_c} \sum_{i=1}^{N_c} \frac{1}{|\mathcal{B}_{i}^{(n+1)}|}\sum_{\bm{b} \in \mathcal{B}_{i}^{(n+1)}} \frac{1}{L_b}\lVert \bm{b}- C_i^{(n+1)}(\bm{b})\rVert^2_2  \le \tilde{J}^{(n+1)}   
\end{split}
\end{equation}

Following (\ref{eq:mse_ineq_1}) and (\ref{eq:mse_ineq_2}), we find that the quantization MSE is non-increasing for each iteration, that is, $J^{(1)} \ge J^{(2)} \ge J^{(3)} \ge \ldots \ge J^{(M)}$ where $M$ is the maximum number of iterations. 
%Therefore, we can say that if the algorithm converges, then it must be that it has converged to a local minimum. 
\hfill $\blacksquare$


\begin{figure}
    \begin{center}
    \includegraphics[width=0.5\textwidth]{sections//figures/mse_vs_iter.pdf}
    \end{center}
    \caption{\small NMSE vs iterations during LO-BCQ compared to other block quantization proposals}
    \label{fig:nmse_vs_iter}
\end{figure}

Figure \ref{fig:nmse_vs_iter} shows the empirical convergence of LO-BCQ across several block lengths and number of codebooks. Also, the MSE achieved by LO-BCQ is compared to baselines such as MXFP and VSQ. As shown, LO-BCQ converges to a lower MSE than the baselines. Further, we achieve better convergence for larger number of codebooks ($N_c$) and for a smaller block length ($L_b$), both of which increase the bitwidth of BCQ (see Eq \ref{eq:bitwidth_bcq}).


\subsection{Additional Accuracy Results}
%Table \ref{tab:lobcq_config} lists the various LOBCQ configurations and their corresponding bitwidths.
\begin{table}
\setlength{\tabcolsep}{4.75pt}
\begin{center}
\caption{\label{tab:lobcq_config} Various LO-BCQ configurations and their bitwidths.}
\begin{tabular}{|c||c|c|c|c||c|c||c|} 
\hline
 & \multicolumn{4}{|c||}{$L_b=8$} & \multicolumn{2}{|c||}{$L_b=4$} & $L_b=2$ \\
 \hline
 \backslashbox{$L_A$\kern-1em}{\kern-1em$N_c$} & 2 & 4 & 8 & 16 & 2 & 4 & 2 \\
 \hline
 64 & 4.25 & 4.375 & 4.5 & 4.625 & 4.375 & 4.625 & 4.625\\
 \hline
 32 & 4.375 & 4.5 & 4.625& 4.75 & 4.5 & 4.75 & 4.75 \\
 \hline
 16 & 4.625 & 4.75& 4.875 & 5 & 4.75 & 5 & 5 \\
 \hline
\end{tabular}
\end{center}
\end{table}

%\subsection{Perplexity achieved by various LO-BCQ configurations on Wikitext-103 dataset}

\begin{table} \centering
\begin{tabular}{|c||c|c|c|c||c|c||c|} 
\hline
 $L_b \rightarrow$& \multicolumn{4}{c||}{8} & \multicolumn{2}{c||}{4} & 2\\
 \hline
 \backslashbox{$L_A$\kern-1em}{\kern-1em$N_c$} & 2 & 4 & 8 & 16 & 2 & 4 & 2  \\
 %$N_c \rightarrow$ & 2 & 4 & 8 & 16 & 2 & 4 & 2 \\
 \hline
 \hline
 \multicolumn{8}{c}{GPT3-1.3B (FP32 PPL = 9.98)} \\ 
 \hline
 \hline
 64 & 10.40 & 10.23 & 10.17 & 10.15 &  10.28 & 10.18 & 10.19 \\
 \hline
 32 & 10.25 & 10.20 & 10.15 & 10.12 &  10.23 & 10.17 & 10.17 \\
 \hline
 16 & 10.22 & 10.16 & 10.10 & 10.09 &  10.21 & 10.14 & 10.16 \\
 \hline
  \hline
 \multicolumn{8}{c}{GPT3-8B (FP32 PPL = 7.38)} \\ 
 \hline
 \hline
 64 & 7.61 & 7.52 & 7.48 &  7.47 &  7.55 &  7.49 & 7.50 \\
 \hline
 32 & 7.52 & 7.50 & 7.46 &  7.45 &  7.52 &  7.48 & 7.48  \\
 \hline
 16 & 7.51 & 7.48 & 7.44 &  7.44 &  7.51 &  7.49 & 7.47  \\
 \hline
\end{tabular}
\caption{\label{tab:ppl_gpt3_abalation} Wikitext-103 perplexity across GPT3-1.3B and 8B models.}
\end{table}

\begin{table} \centering
\begin{tabular}{|c||c|c|c|c||} 
\hline
 $L_b \rightarrow$& \multicolumn{4}{c||}{8}\\
 \hline
 \backslashbox{$L_A$\kern-1em}{\kern-1em$N_c$} & 2 & 4 & 8 & 16 \\
 %$N_c \rightarrow$ & 2 & 4 & 8 & 16 & 2 & 4 & 2 \\
 \hline
 \hline
 \multicolumn{5}{|c|}{Llama2-7B (FP32 PPL = 5.06)} \\ 
 \hline
 \hline
 64 & 5.31 & 5.26 & 5.19 & 5.18  \\
 \hline
 32 & 5.23 & 5.25 & 5.18 & 5.15  \\
 \hline
 16 & 5.23 & 5.19 & 5.16 & 5.14  \\
 \hline
 \multicolumn{5}{|c|}{Nemotron4-15B (FP32 PPL = 5.87)} \\ 
 \hline
 \hline
 64  & 6.3 & 6.20 & 6.13 & 6.08  \\
 \hline
 32  & 6.24 & 6.12 & 6.07 & 6.03  \\
 \hline
 16  & 6.12 & 6.14 & 6.04 & 6.02  \\
 \hline
 \multicolumn{5}{|c|}{Nemotron4-340B (FP32 PPL = 3.48)} \\ 
 \hline
 \hline
 64 & 3.67 & 3.62 & 3.60 & 3.59 \\
 \hline
 32 & 3.63 & 3.61 & 3.59 & 3.56 \\
 \hline
 16 & 3.61 & 3.58 & 3.57 & 3.55 \\
 \hline
\end{tabular}
\caption{\label{tab:ppl_llama7B_nemo15B} Wikitext-103 perplexity compared to FP32 baseline in Llama2-7B and Nemotron4-15B, 340B models}
\end{table}

%\subsection{Perplexity achieved by various LO-BCQ configurations on MMLU dataset}


\begin{table} \centering
\begin{tabular}{|c||c|c|c|c||c|c|c|c|} 
\hline
 $L_b \rightarrow$& \multicolumn{4}{c||}{8} & \multicolumn{4}{c||}{8}\\
 \hline
 \backslashbox{$L_A$\kern-1em}{\kern-1em$N_c$} & 2 & 4 & 8 & 16 & 2 & 4 & 8 & 16  \\
 %$N_c \rightarrow$ & 2 & 4 & 8 & 16 & 2 & 4 & 2 \\
 \hline
 \hline
 \multicolumn{5}{|c|}{Llama2-7B (FP32 Accuracy = 45.8\%)} & \multicolumn{4}{|c|}{Llama2-70B (FP32 Accuracy = 69.12\%)} \\ 
 \hline
 \hline
 64 & 43.9 & 43.4 & 43.9 & 44.9 & 68.07 & 68.27 & 68.17 & 68.75 \\
 \hline
 32 & 44.5 & 43.8 & 44.9 & 44.5 & 68.37 & 68.51 & 68.35 & 68.27  \\
 \hline
 16 & 43.9 & 42.7 & 44.9 & 45 & 68.12 & 68.77 & 68.31 & 68.59  \\
 \hline
 \hline
 \multicolumn{5}{|c|}{GPT3-22B (FP32 Accuracy = 38.75\%)} & \multicolumn{4}{|c|}{Nemotron4-15B (FP32 Accuracy = 64.3\%)} \\ 
 \hline
 \hline
 64 & 36.71 & 38.85 & 38.13 & 38.92 & 63.17 & 62.36 & 63.72 & 64.09 \\
 \hline
 32 & 37.95 & 38.69 & 39.45 & 38.34 & 64.05 & 62.30 & 63.8 & 64.33  \\
 \hline
 16 & 38.88 & 38.80 & 38.31 & 38.92 & 63.22 & 63.51 & 63.93 & 64.43  \\
 \hline
\end{tabular}
\caption{\label{tab:mmlu_abalation} Accuracy on MMLU dataset across GPT3-22B, Llama2-7B, 70B and Nemotron4-15B models.}
\end{table}


%\subsection{Perplexity achieved by various LO-BCQ configurations on LM evaluation harness}

\begin{table} \centering
\begin{tabular}{|c||c|c|c|c||c|c|c|c|} 
\hline
 $L_b \rightarrow$& \multicolumn{4}{c||}{8} & \multicolumn{4}{c||}{8}\\
 \hline
 \backslashbox{$L_A$\kern-1em}{\kern-1em$N_c$} & 2 & 4 & 8 & 16 & 2 & 4 & 8 & 16  \\
 %$N_c \rightarrow$ & 2 & 4 & 8 & 16 & 2 & 4 & 2 \\
 \hline
 \hline
 \multicolumn{5}{|c|}{Race (FP32 Accuracy = 37.51\%)} & \multicolumn{4}{|c|}{Boolq (FP32 Accuracy = 64.62\%)} \\ 
 \hline
 \hline
 64 & 36.94 & 37.13 & 36.27 & 37.13 & 63.73 & 62.26 & 63.49 & 63.36 \\
 \hline
 32 & 37.03 & 36.36 & 36.08 & 37.03 & 62.54 & 63.51 & 63.49 & 63.55  \\
 \hline
 16 & 37.03 & 37.03 & 36.46 & 37.03 & 61.1 & 63.79 & 63.58 & 63.33  \\
 \hline
 \hline
 \multicolumn{5}{|c|}{Winogrande (FP32 Accuracy = 58.01\%)} & \multicolumn{4}{|c|}{Piqa (FP32 Accuracy = 74.21\%)} \\ 
 \hline
 \hline
 64 & 58.17 & 57.22 & 57.85 & 58.33 & 73.01 & 73.07 & 73.07 & 72.80 \\
 \hline
 32 & 59.12 & 58.09 & 57.85 & 58.41 & 73.01 & 73.94 & 72.74 & 73.18  \\
 \hline
 16 & 57.93 & 58.88 & 57.93 & 58.56 & 73.94 & 72.80 & 73.01 & 73.94  \\
 \hline
\end{tabular}
\caption{\label{tab:mmlu_abalation} Accuracy on LM evaluation harness tasks on GPT3-1.3B model.}
\end{table}

\begin{table} \centering
\begin{tabular}{|c||c|c|c|c||c|c|c|c|} 
\hline
 $L_b \rightarrow$& \multicolumn{4}{c||}{8} & \multicolumn{4}{c||}{8}\\
 \hline
 \backslashbox{$L_A$\kern-1em}{\kern-1em$N_c$} & 2 & 4 & 8 & 16 & 2 & 4 & 8 & 16  \\
 %$N_c \rightarrow$ & 2 & 4 & 8 & 16 & 2 & 4 & 2 \\
 \hline
 \hline
 \multicolumn{5}{|c|}{Race (FP32 Accuracy = 41.34\%)} & \multicolumn{4}{|c|}{Boolq (FP32 Accuracy = 68.32\%)} \\ 
 \hline
 \hline
 64 & 40.48 & 40.10 & 39.43 & 39.90 & 69.20 & 68.41 & 69.45 & 68.56 \\
 \hline
 32 & 39.52 & 39.52 & 40.77 & 39.62 & 68.32 & 67.43 & 68.17 & 69.30  \\
 \hline
 16 & 39.81 & 39.71 & 39.90 & 40.38 & 68.10 & 66.33 & 69.51 & 69.42  \\
 \hline
 \hline
 \multicolumn{5}{|c|}{Winogrande (FP32 Accuracy = 67.88\%)} & \multicolumn{4}{|c|}{Piqa (FP32 Accuracy = 78.78\%)} \\ 
 \hline
 \hline
 64 & 66.85 & 66.61 & 67.72 & 67.88 & 77.31 & 77.42 & 77.75 & 77.64 \\
 \hline
 32 & 67.25 & 67.72 & 67.72 & 67.00 & 77.31 & 77.04 & 77.80 & 77.37  \\
 \hline
 16 & 68.11 & 68.90 & 67.88 & 67.48 & 77.37 & 78.13 & 78.13 & 77.69  \\
 \hline
\end{tabular}
\caption{\label{tab:mmlu_abalation} Accuracy on LM evaluation harness tasks on GPT3-8B model.}
\end{table}

\begin{table} \centering
\begin{tabular}{|c||c|c|c|c||c|c|c|c|} 
\hline
 $L_b \rightarrow$& \multicolumn{4}{c||}{8} & \multicolumn{4}{c||}{8}\\
 \hline
 \backslashbox{$L_A$\kern-1em}{\kern-1em$N_c$} & 2 & 4 & 8 & 16 & 2 & 4 & 8 & 16  \\
 %$N_c \rightarrow$ & 2 & 4 & 8 & 16 & 2 & 4 & 2 \\
 \hline
 \hline
 \multicolumn{5}{|c|}{Race (FP32 Accuracy = 40.67\%)} & \multicolumn{4}{|c|}{Boolq (FP32 Accuracy = 76.54\%)} \\ 
 \hline
 \hline
 64 & 40.48 & 40.10 & 39.43 & 39.90 & 75.41 & 75.11 & 77.09 & 75.66 \\
 \hline
 32 & 39.52 & 39.52 & 40.77 & 39.62 & 76.02 & 76.02 & 75.96 & 75.35  \\
 \hline
 16 & 39.81 & 39.71 & 39.90 & 40.38 & 75.05 & 73.82 & 75.72 & 76.09  \\
 \hline
 \hline
 \multicolumn{5}{|c|}{Winogrande (FP32 Accuracy = 70.64\%)} & \multicolumn{4}{|c|}{Piqa (FP32 Accuracy = 79.16\%)} \\ 
 \hline
 \hline
 64 & 69.14 & 70.17 & 70.17 & 70.56 & 78.24 & 79.00 & 78.62 & 78.73 \\
 \hline
 32 & 70.96 & 69.69 & 71.27 & 69.30 & 78.56 & 79.49 & 79.16 & 78.89  \\
 \hline
 16 & 71.03 & 69.53 & 69.69 & 70.40 & 78.13 & 79.16 & 79.00 & 79.00  \\
 \hline
\end{tabular}
\caption{\label{tab:mmlu_abalation} Accuracy on LM evaluation harness tasks on GPT3-22B model.}
\end{table}

\begin{table} \centering
\begin{tabular}{|c||c|c|c|c||c|c|c|c|} 
\hline
 $L_b \rightarrow$& \multicolumn{4}{c||}{8} & \multicolumn{4}{c||}{8}\\
 \hline
 \backslashbox{$L_A$\kern-1em}{\kern-1em$N_c$} & 2 & 4 & 8 & 16 & 2 & 4 & 8 & 16  \\
 %$N_c \rightarrow$ & 2 & 4 & 8 & 16 & 2 & 4 & 2 \\
 \hline
 \hline
 \multicolumn{5}{|c|}{Race (FP32 Accuracy = 44.4\%)} & \multicolumn{4}{|c|}{Boolq (FP32 Accuracy = 79.29\%)} \\ 
 \hline
 \hline
 64 & 42.49 & 42.51 & 42.58 & 43.45 & 77.58 & 77.37 & 77.43 & 78.1 \\
 \hline
 32 & 43.35 & 42.49 & 43.64 & 43.73 & 77.86 & 75.32 & 77.28 & 77.86  \\
 \hline
 16 & 44.21 & 44.21 & 43.64 & 42.97 & 78.65 & 77 & 76.94 & 77.98  \\
 \hline
 \hline
 \multicolumn{5}{|c|}{Winogrande (FP32 Accuracy = 69.38\%)} & \multicolumn{4}{|c|}{Piqa (FP32 Accuracy = 78.07\%)} \\ 
 \hline
 \hline
 64 & 68.9 & 68.43 & 69.77 & 68.19 & 77.09 & 76.82 & 77.09 & 77.86 \\
 \hline
 32 & 69.38 & 68.51 & 68.82 & 68.90 & 78.07 & 76.71 & 78.07 & 77.86  \\
 \hline
 16 & 69.53 & 67.09 & 69.38 & 68.90 & 77.37 & 77.8 & 77.91 & 77.69  \\
 \hline
\end{tabular}
\caption{\label{tab:mmlu_abalation} Accuracy on LM evaluation harness tasks on Llama2-7B model.}
\end{table}

\begin{table} \centering
\begin{tabular}{|c||c|c|c|c||c|c|c|c|} 
\hline
 $L_b \rightarrow$& \multicolumn{4}{c||}{8} & \multicolumn{4}{c||}{8}\\
 \hline
 \backslashbox{$L_A$\kern-1em}{\kern-1em$N_c$} & 2 & 4 & 8 & 16 & 2 & 4 & 8 & 16  \\
 %$N_c \rightarrow$ & 2 & 4 & 8 & 16 & 2 & 4 & 2 \\
 \hline
 \hline
 \multicolumn{5}{|c|}{Race (FP32 Accuracy = 48.8\%)} & \multicolumn{4}{|c|}{Boolq (FP32 Accuracy = 85.23\%)} \\ 
 \hline
 \hline
 64 & 49.00 & 49.00 & 49.28 & 48.71 & 82.82 & 84.28 & 84.03 & 84.25 \\
 \hline
 32 & 49.57 & 48.52 & 48.33 & 49.28 & 83.85 & 84.46 & 84.31 & 84.93  \\
 \hline
 16 & 49.85 & 49.09 & 49.28 & 48.99 & 85.11 & 84.46 & 84.61 & 83.94  \\
 \hline
 \hline
 \multicolumn{5}{|c|}{Winogrande (FP32 Accuracy = 79.95\%)} & \multicolumn{4}{|c|}{Piqa (FP32 Accuracy = 81.56\%)} \\ 
 \hline
 \hline
 64 & 78.77 & 78.45 & 78.37 & 79.16 & 81.45 & 80.69 & 81.45 & 81.5 \\
 \hline
 32 & 78.45 & 79.01 & 78.69 & 80.66 & 81.56 & 80.58 & 81.18 & 81.34  \\
 \hline
 16 & 79.95 & 79.56 & 79.79 & 79.72 & 81.28 & 81.66 & 81.28 & 80.96  \\
 \hline
\end{tabular}
\caption{\label{tab:mmlu_abalation} Accuracy on LM evaluation harness tasks on Llama2-70B model.}
\end{table}

%\section{MSE Studies}
%\textcolor{red}{TODO}


\subsection{Number Formats and Quantization Method}
\label{subsec:numFormats_quantMethod}
\subsubsection{Integer Format}
An $n$-bit signed integer (INT) is typically represented with a 2s-complement format \citep{yao2022zeroquant,xiao2023smoothquant,dai2021vsq}, where the most significant bit denotes the sign.

\subsubsection{Floating Point Format}
An $n$-bit signed floating point (FP) number $x$ comprises of a 1-bit sign ($x_{\mathrm{sign}}$), $B_m$-bit mantissa ($x_{\mathrm{mant}}$) and $B_e$-bit exponent ($x_{\mathrm{exp}}$) such that $B_m+B_e=n-1$. The associated constant exponent bias ($E_{\mathrm{bias}}$) is computed as $(2^{{B_e}-1}-1)$. We denote this format as $E_{B_e}M_{B_m}$.  

\subsubsection{Quantization Scheme}
\label{subsec:quant_method}
A quantization scheme dictates how a given unquantized tensor is converted to its quantized representation. We consider FP formats for the purpose of illustration. Given an unquantized tensor $\bm{X}$ and an FP format $E_{B_e}M_{B_m}$, we first, we compute the quantization scale factor $s_X$ that maps the maximum absolute value of $\bm{X}$ to the maximum quantization level of the $E_{B_e}M_{B_m}$ format as follows:
\begin{align}
\label{eq:sf}
    s_X = \frac{\mathrm{max}(|\bm{X}|)}{\mathrm{max}(E_{B_e}M_{B_m})}
\end{align}
In the above equation, $|\cdot|$ denotes the absolute value function.

Next, we scale $\bm{X}$ by $s_X$ and quantize it to $\hat{\bm{X}}$ by rounding it to the nearest quantization level of $E_{B_e}M_{B_m}$ as:

\begin{align}
\label{eq:tensor_quant}
    \hat{\bm{X}} = \text{round-to-nearest}\left(\frac{\bm{X}}{s_X}, E_{B_e}M_{B_m}\right)
\end{align}

We perform dynamic max-scaled quantization \citep{wu2020integer}, where the scale factor $s$ for activations is dynamically computed during runtime.

\subsection{Vector Scaled Quantization}
\begin{wrapfigure}{r}{0.35\linewidth}
  \centering
  \includegraphics[width=\linewidth]{sections/figures/vsquant.jpg}
  \caption{\small Vectorwise decomposition for per-vector scaled quantization (VSQ \citep{dai2021vsq}).}
  \label{fig:vsquant}
\end{wrapfigure}
During VSQ \citep{dai2021vsq}, the operand tensors are decomposed into 1D vectors in a hardware friendly manner as shown in Figure \ref{fig:vsquant}. Since the decomposed tensors are used as operands in matrix multiplications during inference, it is beneficial to perform this decomposition along the reduction dimension of the multiplication. The vectorwise quantization is performed similar to tensorwise quantization described in Equations \ref{eq:sf} and \ref{eq:tensor_quant}, where a scale factor $s_v$ is required for each vector $\bm{v}$ that maps the maximum absolute value of that vector to the maximum quantization level. While smaller vector lengths can lead to larger accuracy gains, the associated memory and computational overheads due to the per-vector scale factors increases. To alleviate these overheads, VSQ \citep{dai2021vsq} proposed a second level quantization of the per-vector scale factors to unsigned integers, while MX \citep{rouhani2023shared} quantizes them to integer powers of 2 (denoted as $2^{INT}$).

\subsubsection{MX Format}
The MX format proposed in \citep{rouhani2023microscaling} introduces the concept of sub-block shifting. For every two scalar elements of $b$-bits each, there is a shared exponent bit. The value of this exponent bit is determined through an empirical analysis that targets minimizing quantization MSE. We note that the FP format $E_{1}M_{b}$ is strictly better than MX from an accuracy perspective since it allocates a dedicated exponent bit to each scalar as opposed to sharing it across two scalars. Therefore, we conservatively bound the accuracy of a $b+2$-bit signed MX format with that of a $E_{1}M_{b}$ format in our comparisons. For instance, we use E1M2 format as a proxy for MX4.

\begin{figure}
    \centering
    \includegraphics[width=1\linewidth]{sections//figures/BlockFormats.pdf}
    \caption{\small Comparing LO-BCQ to MX format.}
    \label{fig:block_formats}
\end{figure}

Figure \ref{fig:block_formats} compares our $4$-bit LO-BCQ block format to MX \citep{rouhani2023microscaling}. As shown, both LO-BCQ and MX decompose a given operand tensor into block arrays and each block array into blocks. Similar to MX, we find that per-block quantization ($L_b < L_A$) leads to better accuracy due to increased flexibility. While MX achieves this through per-block $1$-bit micro-scales, we associate a dedicated codebook to each block through a per-block codebook selector. Further, MX quantizes the per-block array scale-factor to E8M0 format without per-tensor scaling. In contrast during LO-BCQ, we find that per-tensor scaling combined with quantization of per-block array scale-factor to E4M3 format results in superior inference accuracy across models. 






\end{document}
