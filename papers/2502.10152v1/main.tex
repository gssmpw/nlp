\documentclass[12pt]{article}
\usepackage{geometry}

% Packages
\usepackage[utf8]{inputenc}
\usepackage{amsmath}
\usepackage{amsfonts}
\usepackage{amssymb}
\usepackage[normalem]{ulem}
\usepackage{booktabs}
\usepackage{amsthm}
\usepackage{authblk}
\usepackage{hyperref}

\DeclareMathOperator*{\argmax}{arg\,max}
\DeclareMathOperator*{\argmin}{arg\,min}
\newtheorem{definition}{Definition}
\newtheorem{proposition}{Proposition}
\newtheorem{corollary}{Corollary}
\newtheorem{theorem}{Theorem}

% Bibliography setup
\usepackage[style=numeric]{biblatex}
\addbibresource{biblio.bib}

\geometry{margin=1.11in}

% Title and authors
\title{Characterization of Logarithmic Fekete Critical Configurations of at Most Six Points in All Dimensions}

\author[1]{Diego Armentano}
\author[2]{Leandro Bentancur}
\author[1,3]{Federico Carrasco}
\author[3]{\\Marcelo Fiori}
\author[3]{Matías Valdés\thanks{mvaldes@fing.edu.uy}}
\author[4]{Mauricio Velasco}

\affil[1]{\footnotesize Facultad de Ciencias Económicas y Administración, Universidad de la República, Montevideo, Uruguay}
\affil[2]{Facultad de Ciencias, Universidad de la República, Montevideo, Uruguay}
\affil[3]{Facultad de Ingeniería, Universidad de la República, Montevideo, Uruguay}
\affil[4]{Departamento de Informática, Universidad Católica del Uruguay, Montevideo, Uruguay}

\date{}

\begin{document}

\maketitle

\begin{abstract}
    We consider the logarithmic Fekete problem, which consists of placing a fixed number of points on the unit sphere in $\mathbb{R}^d$, in such a way that the product of all pairs of mutual Euclidean distances is maximized or, equivalently, so that their logarithmic energy is minimized. Using tools from Computational Algebraic Geometry, we find and classify all critical configurations for this problem when considering at most six points in every dimension $d$. Our results discover some previously unknown optimal configurations and give the first reported case of a spurious local minimum for the Fekete problem.
\end{abstract}

\section{Introduction}
	
	Consider the problem of placing $n$ different points on the unit sphere $S^{d-1} \subset \mathbb{R}^d$, in such a way that the product of their mutual Euclidean distances is maximized:
	\begin{equation} \label{eq:fekete_prod}
		\argmax_{w=(w_1,\dots, w_n) \in (S^{d-1})^n } \prod_{i=1}^{n} \prod_{j=i+1}^{n} \| w_i - w_j \|^2 .
	\end{equation}
    Taking logarithm in the objective function, we obtain an equivalent problem, known as the logarithmic Fekete problem:
	\begin{equation} \label{eq:fekete_log}
		\argmin_{ w \in (S^{d-1})^n } - \sum_{i=1}^{n} \sum_{j=i+1}^{n} \log \left( \|w_i - w_j \|^2 \right) .
	\end{equation}

    This is considered a highly non-trival optimization problem, with exact solutions known only for a few values of $n$.
    Indeed, Smale’s 7th problem, listed among the key open problems for the 21st century by
    Steve Smale, asks whether it is possible to find $n$ points on the sphere $S^2$ in polynomial time in $n$, so
    that its logarithmic energy differs from the optimal value by at most $c\ln n$,
    for a universal constant $c$ \cite{smale_problems}. A key difficulty of this problem is that the optimal value remains unknown to logarithmic precision \cite{betermin2018renormalized,carlos_fatima}.
    
    There are several lines of research related to this problem. Among them, we highlight two. 
    One focuses on constructing random or deterministic configurations with ``good'' energy values for an arbitrary number of points, usually large or asymptotic \cite{ABS,alishahi2015spherical,beltran2018diamond}. The other aims to characterize optimal configurations for specific values of $n$, typically small ones. Our work falls within this latter setting.
    
    While the solution to this problem is known only for a few values of $n$, namely, $n \leq 6$ and $n = 12$, even less is understood about the critical configurations.
    For example, even for the values of $n$ where the solution is known, there are no results concerning possible local minima that are not global, or the existence of saddle-points (other than the case when all points lie in an Equator).
    
	In this work we find and classify all the critical configurations of the logarithmic Fekete problem, for the case of at most six points, and for spheres of all possible dimensions. In particular, we find the previously unknown solution for six points on the sphere $S^3$. For this case we also show that there exists a local minima that is not global, and that this is the only case where this happens for configurations of up to six points.
    
	Our strategy for determining all critical configurations of Problem~\eqref{eq:fekete_log} proceeds as follows.
	First we define a system of polynomial equations, associated to the critical configurations.
	Then we count the number of complex solutions of the polynomial system, by computing a Gr{\"o}bner basis for its ideal. We call this quantity the ``expected number of solutions''. Observe that regardless of whether the ideal is radical, this is an upper bound on the number of real solutions of the problem.
	Simultaneously, we find as many solutions of the polynomial system as we can, and compare the number of found solutions with the upper bound. The proof of the Theorem is achieved because we are able to find enough solutions so as to exactly match the upper bound, guaranteeing that our list of critical configurations is exhaustive.
	The construction of solutions is done using different approaches. In the case of four and five points, we are able to find all solutions by considering natural symmetric configurations.  For six points, imagination is not enough, and we use Gr{\"o}bner bases to find additional solutions. 
    The only problem with the above strategy is that the locus of critical points in $(S^{d-1})^n$ does not form a zero-dimensional variety because it is invariant under the action of the orthogonal group. In order to recover finiteness, we reformulate the problem modulo the orthogonal group action. Doing so recovers finiteness and leads to a simpler formulation, which allows us to carry out the desired classification.
    The source code for reproducing our results is available at: \url{www.github.com/matiasvd}.

\section{Related work}

	For $S^1$, the solution of Problem \eqref{eq:fekete_log} is known for any number of points, and is given by $n$ equidistributed points \cite[Theorem 2.3.3]{borodachov2019}.
    We denote this configuration as $n$-gon or ``Equator''.
	For $S^2$, the solutions are known only for up to six points \cite{andreev1996, kolushov1997, dragnev2002}, and for twelve points \cite{andreev1996}.
	Table~\ref{tab:known_solutions} lists the known solutions for up to six points, for all possible dimensions of the sphere.
	We use the term $(n-1)$-simplex for the configuration that has $n$ points on the sphere, all at the same distance. Note that the 3-simplex is the regular Tetrahedron.
    One of our contributions is to give the solution in $S^3$ for six points, which is the only unknown case in Table~\ref{tab:known_solutions}.
	
	\begin{table}
    \centering
		\caption{Known optimal configurations of the logarithmic Fekete problem for at most six points.}
		\label{tab:known_solutions}
		\begin{tabular}{ccccc}
			\toprule
			$n$ & $S^1$ & $S^2$ & $S^3$ & $S^4$ \\
			\midrule
			3 & 2-simplex & N/A & N/A & N/A \\
			4 & 4-gon & Tetrahedron \cite{kolushov1997} & N/A & N/A \\
			5 & 5-gon & Bi Piramid \cite{dragnev2002} & 4-simplex \cite{kolushov1997} & N/A \\
			6 & 6-gon & Octahedron \cite{kolushov1997} & ? & 5-simplex \cite{kolushov1997} \\
			\bottomrule
		\end{tabular}
	\end{table}

    Regarding critical configurations of Problem \eqref{eq:fekete_log}, little is known about them; even for the cases where the solutions are known.
    One thing that is known is that the problem has no local maxima \cite[Corollary 1.3]{beltran2013}, and that critical configurations always have center of mass zero \cite[Proposition 2]{dragnev2002}.
    It is also known that the global minima is not always unique. In particular, if $q=p^l$, with $p>2$ prime and $l \geq 1$, taking $n=(q+1)(q^3+1)$ points on $S^{d-1}$, $d=q \frac{q^3+1}{q+1}$, there are $\lfloor (l-1)/2 \rfloor$ essentially different global minima \cite[Section 1]{ballinger2009}.
    On the other hand, it is not known if the problem has local minima that are not global (spurious local minima). Also, the only saddle point that is mentioned in the literature is the Equator \cite[Section 2]{shub1993}.
    Other unknown property is whether every saddle may be classified using the Hessian. This is particularly important for numerical optimization \cite{ge2015, jin2017}. For 7 points on $S^2$, there is a critical configuration that the Hessian does not classify \cite[Remark 6.3]{constantineau2023}, although it is believed that it is a global minimum.
    Finally, based on numerical experiments, it is conjectured that the number of spurious local minima increases ``dramatically'' with the number of points \cite[Section 4]{rakhmanov1995}, although no spurious local minima are known.
    In this work we show that, for up to six points and all sphere dimensions: there is a spurious local minima in $S^3$, all global minima are essentially unique, there are many other types of saddles other than the Equator, and the Hessian always classifies.
    

\section{System of critical configurations}

	The Lagrangian of Problem \eqref{eq:fekete_log} is:
	$$ L(w, \lambda) := - \sum_{i=1}^{n} \sum_{j=i+1}^{n} \log \left( \|w_i - w_j \|^2 \right) + \sum_{i=1}^{n} \lambda_i \left( \| w_i \|^2 - 1 \right) .$$
	A critical configuration is a sequence $(w_1,\ldots,w_n)$ on the product of spheres, for which there exist multipliers $( \lambda_1,\ldots,\lambda_n )$, such that the Lagrangian has null gradient with respect to $w$:
	$$ \frac{\partial L(w,\lambda)}{\partial w_k} = - \sum_{j=1, j \neq k}^{n} \frac{ 2 \left( w_k - w_j \right) }{ \| w_k - w_j \|^2 } + 2 \lambda_k w_k = \vec{0} , \quad \forall \ k=1,\ldots,n .$$
	Taking dot product with $w_k$, and using that $\|w_k - w_j \|^2 = 2 - 2 w_k^T w_j$ on the sphere, we obtain the values of the multipliers, which happen to be all equal and constant:
    $$ \lambda_k = \frac{n-1}{2} , \quad \forall \ k=1,\ldots,n .$$
	Thus, the critical configurations are those points on the sphere
	\begin{equation} \label{eq:shpere}
		\|w_k\|^2 = 1, \quad \forall \ k=1,\ldots,n ,
	\end{equation}
	which also satisfy the null gradient condition:
	\begin{equation} \label{eq:lagrange_gradient_null}
		\frac{n-1}{2} w_k - \sum_{j=1, j \neq k}^{n} \frac{ w_k - w_j }{ \| w_k - w_j \|^2 } = \vec{0}, \quad \forall \ k=1,\ldots,n .\\
	\end{equation}
    
    An interesting property of the critical configurations of Problem \eqref{eq:fekete_log}, is that their center of mass must be null:
	\begin{equation} \label{eq:center_mass_null}
		\sum_{k=1}^{n} w_k = \vec{0} .
	\end{equation}
	This is obtained by summing the equations \eqref{eq:lagrange_gradient_null} and noting that:
	$$ \sum_{k=1}^{n} \sum_{j=1, j \neq k}^{n} \frac{ w_k - w_j }{ \| w_k - w_j \|^2 } = \vec{0} .$$

    Searching for critical configurations of the Fekete problem means finding solutions to the system of polynomial equations defined by \eqref{eq:shpere} and \eqref{eq:lagrange_gradient_null}, to which we will add \eqref{eq:center_mass_null}, which follows from the previous ones and does not modify the set of solutions.
    
\section{Removing orthogonal symmetry}

    An important property of the system of equations defined in the preceding section is that the system is invariant under isometries of the sphere.
    More precisely, if $Q \in \mathbb{R}^{d \times d}$ is an orthogonal matrix, and $(w_1, \ldots, w_n)$ is a critical configuration, then $(Q w_1, \ldots, Q w_n)$ is also a critical configuration.
    Thus, given a critical configuration, we obtain an infinite number of them by applying rotations.
    As our method relies on counting solutions, we need a system with a finite number of solutions.
	For this, we consider another system of equations, where variables are the dot products of pairs of points: $$ x_{ij} := w_i^T w_j, \quad i \neq j .$$
	Note that these variables represent the cosine of the angles between pairs of points.
	To obtain the new system of equations, we first take dot product in equation \eqref{eq:lagrange_gradient_null}, with respect to each point $w_i$:
	$$ \sum_{j=1, j \neq k}^{n} \frac{ w_i^T w_k - w_i^T w_j }{ \| w_k - w_j \|^2 } = \frac{n-1}{2} w_i^T w_k, \quad \forall \ k,i=1,\ldots,n .$$
    As $w_l^T w_l = 1, \forall \ l$, we have: $\| w_k - w_j \|^2 = 2 (1-w_k^T w_j)$. Using this, the equations may be written in the new variables as:
    \begin{equation} \label{eq:lagrange_gradient_null_xij}
		\sum_{j=1, j \neq k}^{n} \frac{ x_{ik} - x_{ij} }{ 1 - x_{kj} } = (n-1) x_{ik}, \quad \forall \ k \neq i .
	\end{equation}
    Note that for $k=i$ the equation is trivial. We then consider Equation \eqref{eq:center_mass_null}, were we also take dot product with respect to each point $w_j$:
	$$ \sum_{k=1}^{n} w_j^T w_k = 0, \quad \forall \ j=1,\ldots,n .$$	
	Using $w_j^T w_j = 1$, the equations in the new variables are:
	\begin{equation} \label{eq:center_mass_null_xij}
		1 + \sum_{k=1, k \neq j}^{n} x_{jk} = 0, \quad \forall \ j=1,\ldots,n .	
	\end{equation}
	Equations \eqref{eq:lagrange_gradient_null_xij} and \eqref{eq:center_mass_null_xij} determine the new system of equations in the variables $x_{ij}$. The condition $w_i \in S^{d-1}$ of equation \eqref{eq:shpere} is now implicit, and we will not define variables $x_{ii}$.

\section{Relation between solutions of both systems} \label{Sec. Relation between solutions of both systems}
    
    Any solution in the variables $w_i$ has an associated solution $x_{ij} = w_i^T w_j$. The reciprocal is also true, provided that we consider complex solutions. That is: given a solution $x_{ij} \in \mathbb{C}$, there is a solution $w_i \in \mathbb{C}^d$, with $x_{ij}=w_i^T w_j$. To see this, we first define two matrices.

    
    \begin{definition}[Dot product matrix] \label{def:dot_product_matrix}
        Let $x_{ij} \in \mathbb{C}, \ 1 \leq i < j \leq n$. Define $X \in \mathbb{C}^{n \times n}$, symmetric, with $X_{ij} = x_{ij}, \ \forall \ i < j$, $X_{ii}=1, \ \forall \ i$.
    \end{definition}
    
    \begin{definition}[Cartesian coordinates matrix]
        Given $n$ vectors $w_i \in \mathbb{C}^d$, define $W \in \mathbb{C}^{d \times n}$, with $w_i$ as column $i$.
    \end{definition}
    
    We now use a corollary of the following well-known result.
    
    \begin{proposition}[Autonne-Takagi factorization {\cite[Corollary 2.6.6]{horn2013}}]
        If $X \in \mathbb{C}^{n \times n}$ is symmetric, there is a unitary $P \in \mathbb{C}^{n \times n}$, and a nonnegative diagonal matrix $D \in \mathbb{R}^{n \times n}$, such that: $X = P^T D P$.
        Furthermore, the entries of $D$ are the singular values of $X$.
    \end{proposition}
    
    \begin{corollary} \label{cor:complex_solution}
        If $X \in \mathbb{C}^{n \times n}$ is symmetric with rank $d$, and ones on its diagonal, there exists $W \in \mathbb{C}^{d \times n}$, such that: $X=W^T W$, and $w_i^T w_i = 1, \ \forall \ i$.
    \end{corollary}
    
    \begin{proof}
        As $X$ is symmetric: $X = P^T D P$. Take: $W = \sqrt{ \hat{D} } P$; where $\hat{D} \in \mathbb{R}^{d \times n}$ is the submatrix of $D$ with the positive singular values.
    \end{proof}

    We call $X$ the dot product matrix, as we may write $X = W^T W$.
    % Observe this is not a dot product if $W$ is complex.
    Thus, given a solution $X \in \mathbb{C}^{n \times n}$ of rank $d$, we obtain a configuration $(w_1, ..., w_n)$ given by the columns of $W \in \mathbb{C}^{d \times n}$. As the relation between both equation systems is obtained by taking dot product with $w_i$, $i=1,...,n$, and this set is a generator of $\mathbb{C}^d$, we get that both systems are equivalent.
    Thus, we do not lose or introduce solutions in $\mathbb{C}$.
    The next result characterizes the correspondence between real solutions of both systems, which is the ones we are really interested in.
    
    \begin{proposition}[{\cite[Corollary 2.5.11]{horn2013}}] \label{prop:x_es_dot_prod}
    	A matrix $X \in \mathbb{R}^{n \times n}$ is symmetric and positive semi-definite if and only if there exists $W \in \mathbb{R}^{d \times n}$, such that $X=W^T W$.
    	In this case, if $X = Q^T D Q$ is an orthogonal eigenvalue factorization of $X$, we can take $W := \sqrt{D} Q$.
    \end{proposition}
    
    Note that when the previous result applies, $W$ has columns with unit 2-norm, if and only if $X=W^T W$ has ones on its diagonal.


\section{Polynomial system of equations}

	To apply tools from Algebraic Geometry, we need to express the new equations as a system of polynomial equations.
    The condition that points are pairwise distinct is: $x_{kj} \neq 1, \ \forall \ k \neq j$. This may be written introducing auxiliary variables $z_{kj}$, such that:
    \begin{equation} \label{eq:definition_zij}
		z_{kj} \left( 1 - x_{kj} \right) = 1, \quad \forall \ k \neq j .
	\end{equation}
	Using these variables, the equations in \eqref{eq:lagrange_gradient_null_xij} can be rewritten as polynomial equations:
	\begin{equation} \label{eq:lagrange_gradient_null_xij_pol}
		\sum_{j=1, j \neq k}^{n} \left( x_{ik} - x_{ij} \right) z_{kj} = (n-1) x_{ik}, \quad \forall \ k \neq i .
	\end{equation}
	Note that $z_{ij}=z_{ji}$, for all $i \neq j$.
	From now on, we will work with the overdetermined system of polynomial equations given by \eqref{eq:center_mass_null_xij} (null center of mass), \eqref{eq:definition_zij} (auxiliary variables), and \eqref{eq:lagrange_gradient_null_xij_pol} (null gradient of the Lagrangian); expressed in the variables $x_{ij}$ and $z_{ij}$, for $i < j$.
    This system has $m := 2 \binom{n}{2}$ variables, and $n + \binom{n}{2} + n(n-1)$ equations. Table \ref{tab:variables_equations_number} shows these numbers for some values of $n$.
	
	\begin{table}
    \centering
	\caption{Number of variables and equations of the polynomial system given by Equations \eqref{eq:center_mass_null_xij}, \eqref{eq:definition_zij} and \eqref{eq:lagrange_gradient_null_xij_pol}, with $x_{ij}=x_{ji}$.}
	\label{tab:variables_equations_number}
	\begin{tabular}{cccccccc}
		\toprule
		 & $n$ & 3 & 4 & 5 & 6 & 7 & 8 \\
		\midrule
		variables & $2 \binom{n}{2}$ & 6 & 12 & 20 & 30 & 42 & 56 \\
		equations & $n(3n-1)/2$ & 12 & 22 & 35 & 51 & 70 & 92 \\
		\bottomrule
	\end{tabular}
	\end{table}


\section{Counting complex solutions}

	To apply our method, we need to verify that the system of polynomial equations \eqref{eq:center_mass_null_xij}, \eqref{eq:definition_zij} and \eqref{eq:lagrange_gradient_null_xij_pol} has a finite number of solutions, and to count this number.
    We first define the polynomial ideal $I \subset \mathbb{Q}[z_{ij},x_{ij}]$, generated by the polynomials of the equations. The set of complex solutions of these equations is denoted by $V(I) \subset \mathbb{C}^m$, where $m$ is the number of variables. 
    We then use the following result, which allows us to compute an upper bound on the true number of critical points via Gr\"obner bases.
    
	\begin{theorem}[{\cite[Finiteness Th. p. 39, Corollary 2.5 Ch. 4]{cox2005}}]
		Let $I$ be an ideal of the ring $R = \mathbb{C}[x_1,\ldots,x_m]$, and $G$ a Gr{\"o}bner basis of $I$, w.r.t. a graded monomial order.
		The set of solutions $V(I) \subset \mathbb{C}^m$ is finite, if and only if, for every variable $x_i$, there exists $\alpha_i \in \mathbb{Z}^{+}$, such that $x_i^{\alpha_i} \in LT(G)$; where $LT(G)$ denotes the set of leading terms of the elements of $G$.
		In this case, the number of complex solutions of $V(I)$, counted with multiplicity, is the number of monomials of $R$ that are not in the ideal $\left\langle LT(G) \right\rangle$ (not divisible by any element of $LT(G)$).
	\end{theorem}

	To calculate a Gr{\"o}bner basis we use \emph{msolve} \cite{berthomieu2021}, with graded reverse lexicographic monomial order (grevlex).
    We then pass this basis to \emph{Macaulay2} (\emph{M2}) \cite{M2} and use it to verify that the number of solutions is finite and to obtain the number of solutions $(\deg I)$.
    % We then pass this basis to \emph{Macaulay2} (\emph{M2}) \cite{M2}, using the command \textit{forceGB} to let \emph{M2} know that this is a Gr{\"o}bner basis. Within \emph{M2}, we use the commands \textit{dim} and \textit{degree} to verify that the number of solutions is finite and to obtain the number of solutions, respectively.    
	Both \emph{msolve} and \emph{M2} are open source software.
	\emph{Msolve} implements the F4 algorithm \cite{faugere1999}, with parallel algorithms for solving the linear systems generated by F4.
	% \textcolor{blue}{We use version 0.73 of \emph{msolve} binaries from the project website, with AVX512 instructions. Calculations are done with 20 threads of an Intel Xeon-Gold 6138 2GHz.}
    Table \ref{tab:complex_solutions_number} shows the number of ``expected solutions'' of the polynomial system for different number of points.

    \begin{table}
    \centering
		\caption{Expected solutions of polynomial system for $n \leq 6$. Time and RAM of \emph{msolve} to calculate Gr\"obner basis (grevlex).}
		\label{tab:complex_solutions_number}
		\begin{tabular}{ccccccccc}
			\toprule
            & \multicolumn{2}{c}{Time (s) \emph{msolve}} & & \multicolumn{2}{c}{Basis} & \multicolumn{2}{c}{\emph{M2}} \\
			$n$ & Total & Lift to $\mathbb{Q}$ & RAM & \# pols & size & $\dim I$ & $\deg I$ \\
			\midrule
			% 3 &  &  &  & & 0 & \textcolor{red}{1}  \\
			4 & 0.70 & 0.10 & 3.7 MB & 15 & 600 B & 0 & 4 \\
			5 & 0.68 & 0.25 & 3.7 MB & 130 & 77 KB & 0 & 38  \\
			6 & 3963 & 3469 & 37 GB & 2473 & 1.2 GB & 0 & 938 \\
			\bottomrule
		\end{tabular}
	\end{table}

\section{Counting permutations}

	Once we know the number of expected solutions of our polynomial system, we need to find explicit solutions by other methods, until we match the number of expected solutions.
    
	Notice that given $X=W^T W$, and a permutation matrix $P$, then $P^T X P = (WP)^T (WP)$. Since $WP$ is acting by permutations on the points of the original problem on the sphere, and this last problem is invariant by permutations, we have the following result.

	\begin{proposition}
		Let $(X,Z)$ be a solution of the polynomial system given by \eqref{eq:center_mass_null_xij}, \eqref{eq:definition_zij} and \eqref{eq:lagrange_gradient_null_xij_pol}; where $Z \in \mathbb{C}^{n \times n}$ is symmetric, with $Z_{ij} = z_{ij}, \ \forall \ i < j$, $Z_{ii}=0, \ \forall \ i$.
		For every permutation matrix $P \in \mathbb{R}^{n \times n}$, the conjugate by $P$: $\left( P^T X P, P^T Z P \right)$, is also a solution.
	\end{proposition}
        
	\begin{corollary}
		If $X$ is a solution to the polynomial system, the number of different solutions given by conjugations of $X$ by permutations, is the size of the orbit of $X$ under conjugation by permutations.
	\end{corollary}

    Now we give a method to count the different permuted solutions. Instead of calculating $|Orb(X)|$ directly, we calculate the size of the orbit stabilizer subgroup $|Stab(X)|$, as it is easier to implement.
	Both sizes are related by the Orbit Stabilizer Theorem: $$ |Orb(X)| = \frac{|S_n|}{|Stab(X)|} = \frac{n!}{|Stab(X)|} .$$
	To calculate $|Stab(X)|$ for a given $X$, we iterate through all permutation matrices $P$, to count the permutations that satisfy: $X = P^T X P$.


\section{Energies and optimal configurations}

	The product energy, as defined in equation \eqref{eq:fekete_prod}, may be written as: $$ E := \prod_{i=1}^{n} \prod_{j=i+1}^{n} \|w_i - w_j\|^2 = 2^{\binom{n}{2}} \prod_{i=1}^{n} \prod_{j=i+1}^{n} \left( 1 - x_{ij} \right) .$$
    An optimal configuration in $S^{d-1}$, is a critical configuration with the maximum product energy, within all critical configurations of $S^{d-1}$.    
    A configuration $W$ is in $S^{d-1} \subset \mathbb{R}^d$ whenever $d \geq rk(W)$. For example, the Equator has $rk(W)=2$, and it can be embedded in any $S^{d-1}$, with $d \geq 2$.  
    Thus, an optimal configuration in $S^{d-1}$ is a critical configuration $W \in \mathbb{R}^{d \times n}$ with the maximum product energy, within all critical configurations with $rk(X) \leq d$, $X=W^T W$.

\section{Four and five points}

	We now apply our method for the case of four and five points on the sphere. The case of six points will be treated separately as it will require some additional ideas.
	We need to match the number of solutions of Table \ref{tab:complex_solutions_number} with the solutions we may find explicitly.	
	
    \textbf{Four points.} In this case we have at least two natural candidates for critical configurations: the regular Tetrahedron and the Equator.
	In the Tetrahedron all pairs of points have the same angle, with inner products: $w_i^T w_j = -\frac{1}{3}, \ \forall \ i \neq j$. The dot product matrix is:
	$$ X_{\text{Tet.}} = \begin{pmatrix}
	   1 & \theta & \theta & \theta \\
	   \theta & 1 & \theta & \theta \\
	   \theta & \theta & 1 & \theta \\
	   \theta & \theta & \theta & 1 \\
	\end{pmatrix}, \quad \theta := -\frac{1}{3} .$$
    % For the Equator, we consider the following cartesian coordinates: $$ w_1 = \left( 1,0,0 \right), \ w_2 = \left( 0,1,0 \right), \ w_3 = \left( -1,0,0 \right), \ w_4 = \left( 0,-1,0 \right) .$$
    For the Equator, an associated dot product matrix is:
	$$ X_{\text{Eq.}} = \begin{pmatrix}
		1 & 0 & -1 & 0 \\
		0 & 1 & 0 & -1 \\
		-1 & 0 & 1 & 0 \\
		0 & -1 & 0 & 1 \\
	\end{pmatrix} .$$
    To verify that each $X$ is a solution, we replace its values in the polynomial equations using \emph{M2}.
    Table \ref{tab:orbit_energy_rank_n4} shows the size of the orbit of each $X$ by conjugation. Remember that the elements of each orbit are different solutions of the polynomial system. Their sum matches the number of expected solutions. This proves that for $n=4$ the only kind of critical configurations are the Tetrahedron and the Equator.
    Table \ref{tab:orbit_energy_rank_n4} also shows the Energy and rank associated to each critical configuration. We see that the optimal configurations for $n=4$ are: the Equator in $S^1$ and the Tetrahedron in $S^2$.

    \begin{table}
    \centering
		\caption{$n=4$. Orbit size, energy and rank.}
        \label{tab:orbit_energy_rank_n4}
		\begin{tabular}{cccc}
			Conf. & $|\text{Orb}(X)|$ & Energy $E / 2^{15}$ & $\text{rank}(X)$ \\
			\midrule
			Tetrahedron & 1 & $\simeq 5.619$ & 3 \\
			\midrule
			Equator & 3 & 4.0 & 2 \\
			\midrule
			Found solutions & 4 & & \\
			\midrule
			Expected solutions & 4 & & \\            
			\bottomrule
		\end{tabular}
	\end{table}

    \textbf{Five points.} We have at least three candidates for critical configurations in $S^2$: the Equator and two other denoted 1:3:1 and 1:4. (F\"oppl notation \cite{foppl1912}).
	Configuration 1:3:1 has two points forming a dipole, and three other points equidistributed in a plane through the origin, orthogonal to the dipole. The dot product matrix is:
	$$ X_{1:3:1} = \begin{pmatrix}
		1 & A & 0 & 0 & 0 \\
		A & 1 & 0 & 0 & 0 \\
		0 & 0 & 1 & B & B \\
		0 & 0 & B & 1 & B \\
		0 & 0 & B & B & 1
	\end{pmatrix}, \quad \begin{array}{l} A := -1, \\ B := -\frac{1}{2} \end{array} .$$
	Configuration 1:4 may be represented as the north pole, and four other points equidistributed in a plane orthogonal to the $z$-axis. Using the condition of null center of mass, this plane is $z = -\frac{1}{4}$.
	The associated dot product matrix is:
	$$ X_{1:4} = \begin{pmatrix}
        1 & A & A & A & A \\
        A & 1 & B & C & B \\
        A & B & 1 & B & C \\
        A & C & B & 1 & B \\
        A & B & C & B & 1 \\
	\end{pmatrix}, \quad \begin{array}{l} A := -\frac{1}{4}, \\ B := \frac{1}{16}, \\ C := -\frac{7}{8} \end{array} .$$
    For the Equator, the dot product matrix is:
    $$ X_{\text{Eq.}} = \begin{pmatrix}
	   1 & A & B & B & A \\
	   A & 1 & A & B & B \\
	   B & A & 1 & A & B \\
	   B & B & A & 1 & A \\
	   A & B & B & A & 1
	\end{pmatrix}, \quad \begin{array}{l} A := \frac{-1+\sqrt{5}}{4}, \\ B := \frac{-1-\sqrt{5}}{4} \end{array} .$$
	We also consider a critical configuration on $S^3$, given by the 4-simplex; where any pair of points have the same dot product $\theta$. The condition of null center of mass implies: $\theta = -\frac{1}{4}$.
    
	Table \ref{tab:orbit_energy_rank_n5} shows the size of the orbit of each $X$ by conjugation. Their sum matches the number of expected solutions. This proves that for $n=5$ the only kind of critical configurations are those described in this section.
    Table \ref{tab:orbit_energy_rank_n5} also shows the Energy and rank associated to each critical configuration.
    We see that the optimal configurations for $n=5$ are: the Equator in $S^1$, 1:3:1 in $S^2$, and the 4-simplex in $S^3$. This agrees with already known results. In future sections we will classify the rest of the critical configurations.
    
	\begin{table}
    \centering
		\caption{$n=5$. Orbit size, energy and rank.}
		\label{tab:orbit_energy_rank_n5}
		\begin{tabular}{cccc}
			Configuration & $|\text{Orb}(X)|$ & Energy $E / 2^{15}$ & $\text{rank}(X)$ \\
			\midrule
            4-simplex & 1 & $\simeq 9.313$ & 4 \\
			\midrule
			1:3:1 & 10 & 6.75 & 3 \\
            1:4 & 15 & $\simeq 6.63$ & 3 \\
			\midrule
			Equator & 12 & $\simeq 3.052$ & 2 \\
			\midrule
			Found solutions & 38 & & \\
			\midrule
			Expected solutions & 38 & & \\
			\bottomrule
		\end{tabular}
	\end{table}	


\section{Six points}

	For six points, the number of expected solutions is 938 (Table \ref{tab:complex_solutions_number}).
	On the other hand, we may imagine the following critical configurations on $S^2$: the Equator, 1:5 and 1:4:1; with dot product matrices:
	$$ X_{1:5} = \begin{pmatrix}
        1 & A & A & A & A & A \\
        A & 1 & B & C & C & B \\
        A & B & 1 & B & C & C \\
        A & C & B & 1 & B & C \\
        A & C & C & B & 1 & B \\
        A & B & C & C & B & 1
	\end{pmatrix}, \quad \begin{array}{l} A := -\frac{1}{5}, \\ B := \frac{ -5 + 6 \sqrt{5} }{25}, \\ C := \frac{ -5 - 6 \sqrt{5} }{25} \end{array} .$$
    $$ X_{1:4:1} = \begin{pmatrix}
        1 & A & 0 & 0 & 0 & 0 \\
        A & 1 & 0 & 0 & 0 & 0 \\
        0 & 0 & 1 & 0 & A & 0 \\
        0 & 0 & 0 & 1 & 0 & A \\
        0 & 0 & A & 0 & 1 & 0 \\
        0 & 0 & 0 & A & 0 & 1
    \end{pmatrix}, \quad A := -1 .$$
    $$ X_{\text{Eq.}} = \begin{pmatrix}
        1 & A & B & C & B & A \\
        A & 1 & A & B & C & B \\
        B & A & 1 & A & B & C \\
        C & B & A & 1 & A & B \\
        B & C & B & A & 1 & A \\
        A & B & C & B & A & 1
	\end{pmatrix}, \quad \begin{array}{l} A := \frac{1}{2}, \\ B := -\frac{1}{2}, \\ C := -1 \end{array} $$
    We also consider the 5-simplex on $S^4$, where all dot products are $\theta = -\frac{1}{5}$.
    Finally, we consider configuration 3:3 on $S^2$, which consists of two parallel planes, each with three equidistributed points, and with the two triangles in phase with each other. The case where triangles have a phase shift of $\frac{\pi}{3}$ is the same as configuration 1:4:1.
    The analysis of 3:3 (with null phase shift) will reveal another critical configuration, which we call ``real conjugate'' of 3:3.

\subsection{Configuration 3:3 and its real conjugate}

    Consider configuration 3:3 with null phase shift. This configuration is critical when the triangles are placed at $z = \pm z_0$, where $z_0 := \sqrt{ \frac{ -3 + 2 \sqrt{6} }{5} } \simeq 0.62$.
    To express it in cartesian coordinates, we define the radius of each triangle $R := \sqrt{1-z_0^2}$, and take:
    $$ w_1 = \left( R, 0, -z_0 \right), \ w_2 = \left( -\frac{R}{2}, \frac{R \sqrt{3}}{2}, -z_0 \right), \ w_3 = \left( -\frac{R}{2}, -\frac{R \sqrt{3}}{2}, -z_0 \right) .$$
    $$ w_4 = \left( R, 0, z_0 \right), \ w_5 = \left( -\frac{R}{2}, \frac{R \sqrt{3}}{2}, z_0 \right), \ w_6 = \left( -\frac{R}{2}, -\frac{R \sqrt{3}}{2}, z_0 \right) .$$
    The corresponding dot product matrix is:        
	$$ X_{3:3} = \begin{pmatrix}
		1 & A & A & B & C & C \\
		A & 1 & A & C & B & C \\
		A & A & 1 & C & C & B \\
	    B & C & C & 1 & A & A \\
        C & B & C & A & 1 & A \\
		C & C & B & A & A & 1
	\end{pmatrix}, \quad \begin{array}{l}
    A := \frac{ -7 + 3 \sqrt{6} }{5} \\
    B := \frac{ 11 - 4 \sqrt{6} }{5} \\
    C := \frac{ -1 - \sqrt{6} }{5}
	\end{array} .$$

    To verify that $X_{3:3}$ is critical, we replace its values in the polynomial equations using \emph{M2}.
    As the matrix depends on $\sqrt{6}$, we define a polynomial ring with coefficients in $k:=\mathbb{Q}[t]/ (t^2-6)$. This introduces a variable $t$, such that $t^2=6$.
    We then replace $\sqrt{6}$ by $t$ in $X_{3:3}$, and check that its entries verify the equations in $k[x_{ij},z_{ij}]$.
    As this is true for every $t$ such that $t^2=6$, it implies that if we change $\sqrt{6}$ by $-\sqrt{6}$ in $X_{3:3}$, we also obtain a solution to the polynomial equations. We call this the ``real conjugate'' of 3:3, with matrix:
	$$ X_{\overline{3:3}} = \begin{pmatrix}
		1 & \overline{A} & \overline{A} & \overline{B} & \overline{C} & \overline{C} \\
		\overline{A} & 1 & \overline{A} & \overline{C} & \overline{B} & \overline{C} \\
		\overline{A} & \overline{A} & 1 & \overline{C} & \overline{C} & \overline{B} \\
	    \overline{B} & \overline{C} & \overline{C} & 1 & \overline{A} & \overline{A} \\
        \overline{C} & \overline{B} & \overline{C} & \overline{A} & 1 & \overline{A} \\
		\overline{C} & \overline{C} & \overline{B} & \overline{A} & \overline{A} & 1
    \end{pmatrix}, \quad \begin{array}{l}
    \overline{A} := \frac{ -7 - 3 \sqrt{6} }{5} \simeq -2.87 \\
    \overline{B} := \frac{ 11 + 4 \sqrt{6} }{5} \simeq 4.16 \\
    \overline{C} := \frac{ -1 + \sqrt{6} }{5} \simeq 0.29
	\end{array} .$$
    As $X_{\overline{3:3}}$ has entries with absolute value greater than one, it is not the dot product of points on a unit sphere (it corresponds to a complex critical configuration).
    Note also that we already encountered configurations with non-rational values: the Equator for $n=5$, and 1:5; both depending on $\sqrt{5}$. However, in those cases, changing $\sqrt{5}$ by $-\sqrt{5}$ gives a matrix in the same orbit as the original.
    
	Table \ref{tab:solutions_permutations_n6_inicial} shows the orbit size of each of the ``imaginable'' solutions. Their sum is 268, which is far from the expected 938 solutions.
	As we will show, this is because the polynomial system has other types of solutions, which we are not able to imagine a priori.
		
	\begin{table}
    \centering
		\caption{$n=6$. Number of ``imaginable'' solutions.}
		\label{tab:solutions_permutations_n6_inicial}
		\begin{tabular}{cc}
			\toprule
			Configuration & $|\text{Orb}(X)|$ \\
			\midrule
			Equator & 60 \\
			1:5 & 72 \\
			1:4:1 & 15 \\
			3:3 & 60 \\
			3:3 real conj. & 60 \\
			5-simplex & 1 \\
			\midrule
			Found solutions & 268 \\
			\midrule
			Expected solutions & 938 \\
			\bottomrule
		\end{tabular}
	\end{table}	


\subsection{Possible values for the dot product}

	To identify new solutions, we first find all possible values for one of the variables, say $x_{45}$, and then replace each value in the ideal equations, to find the solutions associated with that value.
	Note that, each time we fix a value for $x_{45}$, we have to solve a polynomial system with much fewer solutions than the original.
    Also, as the problem is invariant under permutations, knowing the values of $x_{45}$ is the same as knowing the possible values of any variable $x_{ij}$.
    
	To find all possible values for $x_{45}$, we calculate a Gr{\"o}bner basis with a monomial order that eliminates all the variables, except $x_{45}$. This gives a reduced Gr{\"o}bner basis for the ideal $I \cap \mathbb{Q}[x_{45}]$. As this is a principal ideal domain, its basis is a set with only one polynomial. The roots of this polynomial are the possible values for $x_{45}$.   
	We use \emph{M2} to calculate the factors of the generator of $I \cap \mathbb{Q}[x_{45}]$. Table \ref{tab:generator_factors_n6} shows these factors and their roots.
    Calculating the Gr\"obner basis with \emph{msolve} takes 10 hours and 190 GB of RAM, using 20 threads.

	\begin{table}
    \centering
	\caption{$n=6$. Factors of the generator of $I \cap \mathbb{Q}[x_{45}]$.}
	\label{tab:generator_factors_n6}
	\begin{tabular}{ccc}
		\toprule
		& Generator Factor & Roots \\
		\midrule
		1 & $ x_{45}$ & 0 \\
		2 & $(x_{45} + 1)^2$ & -1 \\
		3 & $ 2 x_{45} - 1 $ & 1/2  \\
		4 & $ 2 x_{45} + 1 $ & -1/2 \\
		5 & $ ( 5 x_{45} - 1 )^2 $ & 1/5 \\
		6 & $ 5 x_{45} + 1 $ & -1/5 \\
		7 & $ ( 5 x_{45} + 7 )^2 $ & -7/5 \\
		8 & $ ( 5 x_{45}^2 + 1 )^2 $ & $\pm \frac{i}{\sqrt{5}}$ \\
		9 & $ 5 x_{45}^2 - 22 x_{45} + 5 $ & $\frac{ 11 \pm 4 \sqrt{6} }{5}$ \\
		10 & $ 5 x_{45}^2 + 2 x_{45} - 1 $ & $\frac{ -1 \pm \sqrt{6} }{5}$ \\
		11 & $ 5 x_{45}^2 + 14 x_{45} - 1 $ & $\frac{ -7 \pm 3 \sqrt{6} }{5}$ \\
		12 & $ 25 x_{45}^2 + 28 x_{45} + 19 $ & $\frac{ -14 \pm i 3 \sqrt{31} }{25}$ \\
		13 & $ 125 x_{45}^2 + 50 x_{45} - 31 $ & $\frac{ -5 \pm 6 \sqrt{5} }{25}$ \\
		14 & $ 100 x_{45}^4 + 95 x_{45}^3 - 21 x_{45}^2 - 22 x_{45} + 10 $ & 4 complex \\
		15 & $ 250 x_{45}^4 + 110 x_{45}^3 - 21 x_{45}^2 - 19 x_{45} + 4 $ & 4 complex \\
		16 & $ 400 x_{45}^4 + 488 x_{45}^3 - 111 x_{45}^2 - 196 x_{45} + 67 $ & 4 complex \\
		17 & $3 x_{45} + 1$ & -1/3 \\
		18 & $5 x_{45} + 4$ & -4/5 \\
		19 & $10 x_{45} - 1$ & 1/10 \\
		20 & $25 x_{45} - 1$ & 1/25 \\
		21 & $25 x_{45} + 11$ & -11/25 \\
		22 & $25 x_{45} + 23$ & -23/25 \\
		\bottomrule
	\end{tabular}
	\end{table}	


\subsection{Minimal primes and new solutions} \label{sec:minprimes_proc}

	For each factor $h_i$ of the generator of the ideal $I \cap \mathbb{Q}[x_{45}]$, denote by $\sqrt{h_i}$ its squarefree part. We consider the ideal $I_i := I + ( \sqrt{h_i} )$, obtained by adding $\sqrt{h_i}$ to the generators of $I$.
    For each $I_i$ we calculate a Gr\"obner basis with \emph{msolve}, using grevlex monomial order, and then use it as input of \emph{M2} to factorize $I_i$ into its minimal primes: $$ I_i = J_1 \cap J_2 \cap \ldots \cap J_k ;$$ where each $J_l$ is a minimal prime ideal of $I_i$.
	The solutions of $I_i$ are: $$ V(I_i) = V(J_1) \cup V(J_2) \cup \ldots \cup V(J_k) .$$
	Thus, we may find the solutions $V(I_i)$ by solving the polynomial system of each ideal $J_l$.
	Appendix \ref{sec:minprimes_sols} shows the results for the factors that give new solutions. There are seven new solutions, some of them with complex coordinates.
    Execution time and RAM is negligible, except for Factor 12, where \emph{msolve} takes 253s and 16 GB.
	

\subsection{Orbits including new solutions}

	Table \ref{tab:solutions_permutations_n6} shows the critical configurations identified for six points and the size of each orbit by conjugation. There are still 90 solutions remaining to match the 938 expected solutions.
	As we show in the next section, this is because ``Complex 1'' has multiplicity greater than one. This, together with the value of expected number of solutions, implies that ``Complex 1'' has multiplicty exactly two, and that we have found all critical configurations for six points.

	\begin{table}
    \centering
		\caption{$n=6$. Orbit size and expected number of solutions.}
		\label{tab:solutions_permutations_n6}
		\begin{tabular}{cc}
			\toprule
			Configuration & $|\text{Orb}(X)|$ \\
			\midrule
			Equator & 60 \\
			1:5 & 72 \\
			1:4:1 & 15 \\
			3:3 (and real ``conjugate'') & 60 + 60 \\
			Complex 1 & 90 \\
			Complex 2 & $360 = 180 \times 2$ \\
			New1 & 15 \\
			New2 & 45 \\
			New3 & 60 \\
			New4 & 10 \\
			5-simplex & 1 \\
			\midrule
			Found solutions & 848 \\
			\midrule
			Expected solutions & 938 \\
			\midrule
			Difference & $938 - 848 = 90$ \\
			\bottomrule
		\end{tabular}
	\end{table}	


\subsection{Multiplicity of ``Complex 1''}
	
	\begin{proposition}[{\cite[Theorem 5.1]{hartshorne2013algebraic}}]
		Consider a polynomial ideal $I \subset \mathbb{Q}[x_1,\ldots,x_m]$, generated by polynomials $\{ g_1, \ldots, g_r \}$.
		Define the Jacobian matrix of the ideal generators as: $$ J (x) \in \mathbb{R}^{r \times m} \quad / \quad J(x)_{ij} := \frac{\partial g_i}{\partial x_j}(x), \quad \forall \ x \in \mathbb{C}^m .$$
		Let $\hat{x}$ be a solution to the system of equations associated with the ideal generators.
		This solution has multiplicity greater than one, if and only if, the matrix $J(\hat{x})$ is rank deficient.
	\end{proposition}
  
	For ``Complex 1'', $J(\hat{x},\hat{z})$ has size $30 \times 51$ and rank 29. This implies that ``Complex 1'' has multiplicity greater than one, as we wanted to prove. Calculations are done with \emph{M2} code \emph{solMultiplicty.m2}.


\subsection{Cartesian coordinates on the sphere} \label{sec:cartesian_coords_sphere}
    
	Table \ref{tab:eigenvalues_x_n6} shows the eigenvalues of each solution.
    When $X$ is real, symmetric and positive semi-definite, we use Proposition \ref{prop:x_es_dot_prod} to obtain a critical configuration on the unit sphere (Appendix \ref{sec:cartesian_coord_new}). 
    For the other cases, although we cannot obtain a real solution on the sphere, we can obtain a complex solution applying Corollary \ref{cor:complex_solution}.
    This shows that Problem \eqref{eq:fekete_log} has three complex critical configurations.

	\begin{table}
    \centering
		\caption{$n=6$. Properties of solution matrix $X$.}
		\label{tab:eigenvalues_x_n6}
		\begin{tabular}{ccccc}
			Conf. & $X$ real & $X$ eigenvalues $\neq 0$ & $X \succeq 0$ & $rk(X)$ \\
			\midrule
			Equator & yes & 3 $(\times 2)$ & yes & 2 \\
			1:5 & yes & $\frac{6}{5}$ $(\times 1)$, $\frac{12}{5}$ $(\times 2)$ & yes & 3 \\
			1:4:1 & yes & 2 $(\times 3)$ & yes & 3 \\
			3:3 ($\sqrt{6}$) & yes & $\simeq 2.28 (\times 1)$, $\simeq 1.86 (\times 2)$ & yes & 3 \\
			3:3 ($-\sqrt{6}$) & yes & $\simeq -9.48 (\times 1)$, $\simeq 7.74 (\times 2)$ & no & 3 \\
			Complex 1 & no & $\frac{6}{5} (\times 1)$, $\frac{12}{5} (\times 2)$ & yes & 3 \\
			Complex 2 & no & 3 complex different & no & 3 \\
			New1 & yes & $\frac{4}{3} (\times 3)$, $2 (\times 1)$ & yes & 4 \\
			New2 & yes & $\frac{6}{5} (\times 2)$, $\frac{9}{5} (\times 2)$ & yes & 4 \\
			New3 & yes & $\frac{6}{5} (\times 1)$, $\frac{36}{25} (\times 2)$, $\frac{48}{25} (\times 1)$ & yes & 4 \\
			New4 & yes & $\frac{3}{2} (\times 4)$ & yes & 4 \\
            5-simplex & yes & $\frac{6}{5} (\times 5)$ & yes & 5 \\
			\bottomrule
		\end{tabular}
	\end{table}
    
    
\subsection{Energies and solution in $S^3$ for six points}

	Table \ref{tab:energy_rank_n6} shows the Energy and rank associated to each critical configuration that can be expressed as $X = W^T W$, with $W \in \mathbb{R}^{rk(X) \times n}$. 
    We conclude that the optimal configurations for $n=6$ are: the Equator in $S^1$, 1:4:1 in $S^2$, New4 in $S^3$, and the 5-simplex in $S^4$. The novelty here is the optimal configuration for $S^3$.
    
	\begin{table}
    \centering
		\caption{$n=6$. Energy, rank and orbit size.}
		\label{tab:energy_rank_n6}
		\begin{tabular}{cccc}
			Configuration & Energy $E / 2^{15}$ & $rk(X)$ & $|\text{Orb}(X)|$ \\
			\midrule
            5-simplex & $\simeq 15.41$ & 5 & 1 \\
			\midrule
            New4 & $\simeq 11.39$ & 4 & 10 \\
            New1 & $\simeq 11.24 $ & 4 & 15 \\
			New3 & $\simeq 11.17 $ & 4 & 60 \\
			New2 & $\simeq 10.97 $ & 4 & 45 \\
			\midrule
			1:4:1 & $ 8.00 $ & 3 & 15 \\
            3:3 ($\sqrt{6}$) & $ \simeq 6.62 $ & 3 & 60 \\
            3:3 ($-\sqrt{6}$) & $\notin S^2$ & 3 & 60 \\
            1:5 & $ \simeq 5.05 $ & 3 & 72 \\
            Complex 1 & non real coord. & 3 & 90 \\
            Complex 2 & non real coord. & 3 & 180 \\
			\midrule
			Equator & $ \simeq 1.42 $ & 2 & 60 \\
			\bottomrule
		\end{tabular}
	\end{table}	

    For each sphere $S^{d-1}$, the optimal configuration $X$ has the maximum number of conjugate symmetries (minimum orbit size).
    This is also true for $n=4$ and $n=5$, as can be seen in Tables \ref{tab:orbit_energy_rank_n4} and \ref{tab:orbit_energy_rank_n5}.
    Also, within each sphere $S^{d-1}$, the product energy increases as the conjugate symmetries increase (decreases with the orbit size); the only exception being New2 and New3.    

\section{Geometric interpretation on $S^3$} \label{sec: Geometric interpretation on S^3}

	New4, which is the solution on $S^3 \subset \mathbb{R}^4$ for six points, consists of two equilateral triangles, each inscribed in a copy of $S^1$ lying in orthogonal spaces.
    In particular, from the cartesian coordinates given in \eqref{eq:cartesian_new4}, we see that points $w_0$, $w_1$ and $w_2$ form an equilateral triangle, inscribed in an $S^1$. This is because they have the same pairwise angles: $\cos^{-1}(-1/2) = \frac{2 \pi}{3}$, and belong to the same plane through the origin: $w_0 + w_1 = -w_2$. The same happens with points $w_3$, $w_4$ and $w_5$, which form an equilateral triangle, inscribed in another $S^1$. Finally, these circumferences lie in orthogonal spaces: $$ w_i^T w_j = 0, \ \forall \ i \in \{0,1,2\}, \ j \in \{3,4,5\} .$$

    For the other critical configurations, they can be viewed as given by the fibration of $S^3$ by spheres. 
    New1 is, in fact, analogous to the 1:4:1 configuration of $S^2$; It has the poles and the optimal configuration for 4 points in the equatorial sphere.
    New3 is, in turn, analogous to the 1:5 configuration of $S^2$; It has a pole and the optimal configuration for 5 points in the corresponding sphere.
    New2 has no analogous configuration on $S^2$. It has 4 points on the Equator of a sphere and the optimal configuration for 2 points in another sphere, with the peculiarity that the line that passes through these two points is orthogonal to the plane of the 4 point Equator.

\section{Classification of critical configurations}

	We now classify all critical configurations, using the Hessian of the Lagrangian.
	First consider the case of the unit sphere $S^2 \subset \mathbb{R}^3$.
	The gradient of the Lagrangian of Problem \eqref{eq:fekete_log} has coordinate functions:
	$$ \frac{\partial L}{\partial w_k} = - \sum_{j=1, j \neq k}^{n} \frac{ 2 \left( w_k - w_j \right) }{ \| w_k - w_j \|^2 } + (n-1) w_k, \quad \forall \ k=1,\ldots,n .$$
	These may be written as:
	$$ \frac{\partial L}{\partial w_k} = \sum_{j=1, j \neq k}^{n} f(w_k-w_j) + (n-1) w_k, \quad \forall \ k=1,\ldots,n .$$
	where we define the auxiliary function $f: \mathbb{R}^3 \to \mathbb{R}^3$, such that $$ f(w) := - \frac{2 w}{ \| w \|^2 } = -\frac{ 2 (x,y,z) }{ x^2 + y^2 + z^2 }, \quad w := (x,y,z) .$$
    The Jacobian of $f$ is:
	$$ J_f(w) := \frac{-2}{\| w \|_2^4} \begin{pmatrix}
		-x^2 + y^2 + z^2 & -2xy & -2xz \\
		 -2xy & x^2 - y^2 + z^2 & -2yz \\
		-2xz & -2yz & x^2 + y^2 - z^2 \\
	\end{pmatrix} .$$
	Using this matrix, we can write the Hessian $H_L$ of the Lagrangian as $n^2$ blocks of $3 \times 3$ matrices, given by: 
	$$ (H_L)_{ii} := \frac{\partial L}{\partial w_i^2} = \sum_{j=1, j \neq i} J_f(w_i-w_j) + (n-1) I_3, \quad \forall \ i .$$	
	$$ (H_L)_{ij} := \frac{\partial L}{\partial w_i w_j} = - J_f(w_i-w_j), \quad \forall \ i \neq j .$$
	These expressions also apply to points on $S^{d-1} \subset \mathbb{R}^d$, with $H_L \in \mathbb{R}^{d n \times d n}$.
    As we are interested in the eigenvalues of $H_L$ associated with directions of the tangent space of the product of spheres, we project $H_L$ onto this tangent space. The projected matrix is: $$ h_L := V^T H_L V, \quad h_L \in \mathbb{R}^{n(d-1) \times n(d-1)} ;$$
    where the columns of $V$ are an orthonormal basis of the tangent space.
    This basis is obtained as the orthogonal complement of a basis of the normal space: $B_n = \{v_1,\ldots,v_n\}$; where $v_i = \left( \vec{0}, w_i, \vec{0} \right) \in \mathbb{R}^{nd}$.
    We use the following result to classify the critical configurations.

    \begin{proposition}
        Consider a critical configuration $w$ of Problem \eqref{eq:fekete_log}. Denote by $h_L$ the Hessian of the Lagrangian at $w$, projected onto the tangent space of the product of $n$ spheres $S^{d-1}$.
        \begin{enumerate}
            \item if $h_L$ has at least one negative eigenvalue, $w$ is a saddle point.

            \item if $h_L$ has eigenvalue zero with multiplicty exactly $d(d-1)/2$, and all other eigenvalues of $h_L$ are positive, then $w$ is a local minimum (possibly global).
            
            \item If $w$ is a saddle point in $S^k$, it is also a saddle in $S^l$, $\forall \ l \geq k$.
        \end{enumerate}
    \end{proposition}

    \begin{proof}
    \begin{enumerate}
        \item Problem \eqref{eq:fekete_log} does not admit local maxima.
        The existence of a negative eigenvalue implies the critical configuration is not a local minima, and so it must be a saddle.
        
        \item As the energy remains constant under rotations, $h_L$ will always have zero as an eigenvalue, with multiplicity at least the dimension of the orthogonal group: $|O(d)| = d(d-1)/2$. When this inequality is exact, zero is associated to rotations only. If the other eigenvalues are positive, then the Hessian classifies and the critical configuration is a local minima.

        \item If $w$ is a saddle point in $S^k$, there exists a local direction in which the energy value increases. This direction remains if we increase the dimension of the sphere.
    \end{enumerate}

    \end{proof}

	The eigenvalues of each $h_L$ are given in Appendix \ref{sec:eigenvalues_hessian}.
    Table \ref{tab:classification_n6} shows the classification of each critical configuration.
    All configurations are saddle points or global minima, except for ``New1'', which is a local minima in $S^3$, that is not a global minima.
    Also, every saddle point has a negative direction associated with the projected Hessian.
    Note that the Equator is a global minima in $S^1$, and a saddle point in $S^2$. This also happens with other configurations.
    
	\begin{table}
    \centering
	\caption{$n=6$. Classification. S is Saddle, GM is global minima, and SM is local minima that is not global (spurious).}
	\label{tab:classification_n6}
	\begin{tabular}{ccccc}
		Configuration & $S^1$ & $S^2$ & $S^3$ & $S^4$ \\
		\midrule
		Equator & GM & S & S & S \\
		\midrule
		1:5 & - & S & S & S \\
		1:4:1 & - & GM & S & S \\
		3:3 ($\sqrt{6}$) & - & S & S & S \\
		\midrule
		New1 & - & - & SM & S \\
		New2 & - & - & S & S \\
		New3 & - & - & S & S \\
		New4 & - & - & GM & S \\
		\midrule
		5-simplex & - & - & - & GM \\
		\bottomrule
	\end{tabular}
	\end{table}	


\section{Conclusions and Future Work}
    
    In this work we characterize the critical configurations of the logarithmic Fekete problem for up to six points in all dimensions.    
    We proceed by expressing the set of critical configurations up to the action of the orthogonal group as a zero dimensional variety. Simultaneously, we construct several candidates for critical configurations. The proof of the classification theorem is achieved by showing that the number of candidates matches de degree of the variety, when the candidates are counted with multiplicity. Our computation of the degree relies on Gr\"obner bases.
    
    In particular, this classification reveals several novel results for six points in $S^3$: a spurious local minimum and the global minimum configuration.
    Our approach is in principle applicable to any number of points. However, in practice there seems to be a computational bottleneck in the construction of the Gröbner bases of the ideal of the critical configurations variety.
    The extension of this method to the case of $n=7$ points is the subject of ongoing work.


\section*{Acknowledgements}

    We thank Pedro Raigorodsky and Carlos Beltr\'an for useful discussions during the completion of this work.
	We thank ClusterUY for the infrastructure for the numerical experiments.
	Mat\'ias Vald\'es acknowledges support from a PhD grant from Agencia Nacional de Investigaci\'on e Innovaci\'on (ANII) and Comisi\'on Acad\'emica de Posgrado (CAP). Leandro Bentancur acknowledges support from a PhD grant from CAP. Marcelo Fiori and Mauricio Velasco were partially supported by ANII grant FCE-1-2023-1-176172.



%%
%% Print the bibliography
%%
\printbibliography

%%
%% If your work has an appendix, this is the place to put it.
\newpage
\appendix

\section{Solutions of minimal primes for $n=6$} \label{sec:minprimes_sols}

In this appendix we describe the new solutions obtained with the procedure of Section \ref{sec:minprimes_proc}, for the case of six points.

\subsection{Factor $5 x_{45} - 1$ - Complex 1}
	
	We obtain 12 minimal primes, and a new kind of solution, which we call ``Complex 1'', as some of its coordinates are complex.
	$$ X_{C1} = \begin{pmatrix}
	       1 & -1 & -x_{15} & x_{15} & x_{15} & -x_{15} \\
            & 1 & x_{15} & -x_{15} & -x_{15} & x_{15} \\
            & & 1 & x_{45} & x_{45} & x_{25} \\
            &  & & 1 & x_{25} & x_{45} \\
            &  & & & 1 & x_{45} \\
            &  & & & & 1 \\
	\end{pmatrix}, \quad \begin{array}{l} x_{45} = \frac{1}{5}, \\ x_{25} = -\frac{7}{5}, \\ x_{15} = \pm \frac{i}{\sqrt{5}} \end{array} .$$

\subsection{Factor $25 x_{45}^2 + 28 x_{45} + 19$ - Complex 2}
	
    We obtain 3 minimal primes and a new kind of solution, which we call ``Complex 2'', as some of its coordinates are complex.
    \begin{gather*}
        X_{C2} = \begin{pmatrix}
            1 & A & x_{13} & B & C & C \\
    		& 1 & B & x_{13} & C & C \\
    		& & 1 & D & x_{35} & x_{35} \\
    		  & & & 1 & x_{35} & x_{35} \\
    		& & & & 1 & x_{45} \\
    		& & & & & 1 \\
    	\end{pmatrix} ;\\
        \left\lbrace \begin{array}{l}
            25 x_{45}^2+28x_{45}+19 = 0, \quad 8 x_{13}^2-5x_{13}x_{45}+x_{13}-x_{45}-3 = 0 , \\
            20 x_{35}^2+10x_{35}x_{45}+10x_{35}-x_{45}-3 = 0, \\
            A := 2 x_{35} + \frac{3}{8} x_{45} + \frac{1}{8}, \quad B := -x_{13} + \frac{5}{8} x_{45} -\frac{1}{8}, \\
            C := -x_{35} - \frac{1}{2} x_{45} - \frac{1}{2}, \quad D := -2 x_{35} - \frac{5}{8} x_{45} - \frac{7}{8} .
        \end{array} \right. .
    \end{gather*}
    
\subsection{Factor $3 x_{45} + 1$ - New1}

    We obtain 6 minimal primes, and a new kind of solution, ``New1''.
    $$ X_{N1} = \begin{pmatrix}
    	1 & x_{45} & x_{02} & x_{02} & x_{45} & x_{45} \\
    	& 1 & x_{02} & x_{02} & x_{45} & x_{45} \\
    	& & 1 & x_{23} & x_{02} & x_{02} \\
    	& & & 1 & x_{02} & x_{02} \\
    	& & & & 1 & x_{45} \\
    	& & & & & 1 \\
    \end{pmatrix}, \quad \begin{array}{l} x_{45} = -1/3, \\ x_{02} = 0, \\ x_{23} = - 1 \end{array} .$$

\subsection{Factor $5 x_{45} + 4$ - New2}

We obtain 6 minimal primes, and a new kind of solution, ``New2''.
$$ X_{N2} = \begin{pmatrix}
	1 & x_{45} & x_{02} & x_{02} & x_{04} & x_{04} \\
	& 1 & x_{02} & x_{02} & x_{04} & x_{04} \\
	& & 1 & x_{02} & x_{02} & x_{02} \\
	& & & 1 & x_{02} & x_{02} \\
	& & & & 1 & x_{45} \\
	& & & & & 1 \\
\end{pmatrix}, \quad \begin{array}{l} x_{45} = -4/5, \\ x_{04} = 1/10, \\ x_{02} = -1/5 \end{array} .$$


\subsection{Factor $25 x_{45} - 1$ - New3}

We obtain 24 minimal primes, and a new kind of solution, ``New3''.
$$ X_{N3} = \begin{pmatrix}
	1 & x_{45} & x_{45} & x_{03} & x_{45} & x_{05} \\
	& 1 & x_{12} & x_{03} & x_{12} & x_{45} \\
	& & 1 & x_{03} & x_{12} & x_{45} \\
	& & & 1 & x_{03} & x_{03} \\
	& & & & 1 & x_{45} \\
	& & & & & 1 \\
\end{pmatrix}, \quad \begin{array}{l} x_{45} = 1/25, \\ x_{05} = -23/25, \\ x_{12} = -11/25, \\ x_{03} = -1/5 \end{array} .$$


\subsection{Factor $x_{45}$ - New4}

We obtain 26 minimal primes, and a new kind of solution, ``New4''.
$$ X_{N4} = \begin{pmatrix}
	1 & x_{01} & x_{45} & x_{45} & x_{45} & x_{01} \\
	& 1 & x_{45} & x_{45} & x_{45} & x_{01} \\
	& & 1 & x_{01} & x_{01} & x_{45} \\
	& & & 1 & x_{01} & x_{45} \\
	& & & & 1 & x_{45} \\
	& & & & & 1 \\
\end{pmatrix}, \quad \begin{array}{l} x_{45} = 0, \\ x_{01} = -1/2 \end{array} .$$


\section{New real solutions for $n=6$} \label{sec:cartesian_coord_new}

In this appendix we give cartesian coordinates on $S^3$ for the new real solutions of six points, as described in Section \ref{sec:cartesian_coords_sphere}.

$$ W_{\text{N1}} = \frac{1}{3} \begin{pmatrix}
    0 & 0 &  -\sqrt{6} & \sqrt{6} & 0 & 0 \\
    0 & 0 & -\sqrt{2} & -\sqrt{2} & 2\sqrt{2} & 0 \\
    0 & 3 & -1 & -1 & -1 & 0 \\
    3 & 0 & 0 & 0 & 0 & -3
\end{pmatrix} .$$

	$$ W_{\text{N2}} = \frac{1}{10} \begin{pmatrix}
    -3\sqrt{10} & 3\sqrt{10} & 0 & 0 & 0 & 0 \\
    0 & 0 & -3\sqrt{10} & 3\sqrt{10} & 0 & 0 \\
    0 & 0 & 0 & 0 & 2\sqrt{15} & -2\sqrt{15} \\
    \sqrt{10} & \sqrt{10} & \sqrt{10} & \sqrt{10} & -2\sqrt{10} & -2\sqrt{10}
\end{pmatrix}.$$

	$$ W_{\text{N3}} = \frac{1}{5} \begin{pmatrix}
    0 & 0 & -3\sqrt{2} & 3\sqrt{2} & 0 & 0 \\
    0 & 0 & -\sqrt{6} & -\sqrt{6} & 2\sqrt{6} & 0 \\
    0 & 2\sqrt{6} & 0 & 0 & 0 & - 2\sqrt{6} \\
    5 & -1 & -1 & -1 & -1 & -1
\end{pmatrix} .$$

	\begin{equation} \label{eq:cartesian_new4}
		W_{\text{N4}} = \frac{1}{2} \begin{pmatrix}
        -\sqrt{3} & \sqrt{3} & 0 & 0 & 0 & 0 \\
        -1 & -1 & 2 & 0 & 0 & 0 \\
        0 & 0 & 0 & -\sqrt{3} & \sqrt{3} & 0 \\
        0 & 0 & 0 & -1 & -1 & 2
    \end{pmatrix}.
	\end{equation}

Based on the geometrical interpretation of section \ref{sec: Geometric interpretation on S^3}, and inspired by the F\"oppl notation, we propose the following notation for these new configurations:
\begin{align*}
    \text{(New1)} \quad & 1:3_{S^2}:1 \\
    \text{(New2)} \quad & 2_{S^2}:4_{\text{Eq.}} \\
    \text{(New3)} \quad & 1:5_{S^2} \\
    \text{(New4)} \quad & 3_{S^1} \times 3_{S^1} ;
\end{align*}
where the last one uses $\times$ since the configuration lies in $S^1 \times S^1$.

 %    \begin{equation} \label{eq:cartesian_complex1}
	% 	W_{\text{C1}} = \begin{pmatrix}
 %            1 & 0 & 0 \\
 %            -1 & 0 & 0 \\
 %            -\frac{ \sqrt{5} i }{5} & 0 & -\frac{ \sqrt{30} }{5} \\
 %            \frac{ \sqrt{5} i }{5} & -\frac{ \sqrt{30} }{5} & 0 \\
 %            \frac{ \sqrt{5} i }{5} &  \frac{ \sqrt{30} }{5} & 0 \\
 %            -\frac{ \sqrt{5} i }{5} & 0 &  \frac{ \sqrt{30} }{5}
 %        \end{pmatrix}
	% \end{equation}


\section{Projected Hessian eigenvalues} \label{sec:eigenvalues_hessian}
    
    The eigenvalues of the projected Hessian are given in Tables \ref{tab:hess_eigenvalues_n4}, \ref{tab:hess_eigenvalues_n5} and \ref{tab:hess_eigenvalues_n6}.
    The associated code can be found in the \emph{SymPy} notebook \emph{``classifyHessian''}.
    For 1:5 and 3:3 we could not calculate the exact eigenvalues, and we report the eigenvalues calculated with floating point arithmetic. To prove that these configurations are saddle, we give feasible directions where the quadratic form of the Hessian is negative.
    To find a negative direction for 1:5, we move the first two points of the roots of unity, one upward and the other downward. The associated value of the quadratic form of the Hessian is:
    $$ v^T H_L v = \frac{ -263 \sqrt{5} }{625} + \frac{ 13 \sqrt{5} \sqrt{ 2 \sqrt{5} + 6 } }{625} + \frac{185}{625} \simeq -0.494 < 0 .$$
    To find a negative direction for 3:3, we move the three points of the upper hemisphere in the direction of a counterclockwise rotation with respect to the z-axis. The associated value is:
    $$ v^T H_L v = 36 - 15 \sqrt{6} \simeq -0.742 < 0 .$$
    
    % \begin{table*}
    \begin{table}
    \centering
    \caption{$n=4$. Eigenvalues of the Hessian of the Lagrangian, projected onto the tangent space of the product of spheres.}
    \label{tab:hess_eigenvalues_n4}
    \begin{tabular}{cccc}
    	$\mathbb{R}^d$ & Conf. & eig.: mult. & $|O(d)|$ \\
    	\midrule
    	$\mathbb{R}^2$ & Equator & 4:1, 3:2, 0:1 & 1 \\
    	\midrule
        $\mathbb{R}^3$ & Equator & 4:1, 3:3, 0:3, -1:1 & 3 \\
        & Tetrahedron & $\frac{3}{2}$:2, 3:3, 0:3 & \\
        \bottomrule
    \end{tabular}
    \end{table}
    % \end{table*}
    

    % \begin{table*}
    \begin{table}
    \centering
    \caption{$n=5$. Eigenvalues of the Hessian of the Lagrangian, projected onto the tangent space of the product of spheres.}
    \label{tab:hess_eigenvalues_n5}
    \begin{tabular}{cccc}
    	$\mathbb{R}^d$ & Conf. & eig.: mult. & $|O(d)|$ \\
    	\midrule
    	$\mathbb{R}^2$ & Equator & 6:2, 4:2, 0:1 & 1 \\
    	\midrule
        $\mathbb{R}^3$ & Equator & 6:2, 4:3, 0:3, -2:2 & 3 \\
        & 1:4 & $\frac{64}{15}$:1, $-\frac{4}{15}$:1, 2:2, 4:3, 0:3 \\
        & 1:3:1 & $\frac{7}{2}$:2, $\frac{1}{2}$:2, 4:3, 0:3 \\
        \midrule
        $\mathbb{R}^4$ & 1:3:1 & $\frac{7}{2}$:2, $\frac{1}{2}$:2, 4:4, 0:6, -1:1 & 6 \\
        & 4-simplex & 4:4, $\frac{8}{5}$:5, 0:6 \\
        \bottomrule
    \end{tabular}
    \end{table}
    % \end{table*}
    
    \begin{table}
    \centering
    % \begin{table*}
    \caption{$n=6$. Eigenvalues of the Hessian of the Lagrangian, projected onto the tangent space of the product of spheres. Values with (*) are calculated with floating point arithmetic.}
    \label{tab:hess_eigenvalues_n6}
    \begin{tabular}{cccc}
    	$\mathbb{R}^d$ & Conf. & eig.: mult. & $|O(d)|$ \\
    	\midrule
    	$\mathbb{R}^2$ & Equator & 9:1, 8:2, 5:2, 0:1 & 1 \\
    	\midrule
        $\mathbb{R}^3$ & Equator & 9:1, 8:2, 5:3, 0:3, -4:1, -3:2 & 3 \\
        & 1:5 (*) & 5:3, 0:3, 2.5:2, -1.25:2, 6.25:2 \\
        & 3:3 (*) & 5:3, 0:3, 3.55:2, 1.45:2, 5.80:1, -0.80:1 \\
        & 1:4:1 & 5:3, 4:3, 1:3, 0:3 \\
        \midrule
        $\mathbb{R}^4$ & 1:4:1 & 5:4, 4:3, 1:3, 0:6, -1:2 & 6 \\
        & New1 & $\frac{3}{2}$:2, $\frac{10}{3}$:3, $\frac{1}{3}$:3, 5:4, 0:6 \\
        & New2 & $\frac{5}{3}$:1, $\frac{40}{9}$:1, $\frac{-5}{18}$:2, 5: 4, $\frac{25}{12}$:4, 0:6 \\
        & New3 & $\frac{13}{6}$:1, $\frac{-5}{24}$:1, $\frac{11}{6}$:2, $\frac{175}{48}$:2, $\frac{25}{48}$:2, 5:4, 0:6 \\
        & New4 & 5:4, 3:4, $\frac{1}{2}$:4, 0:6 \\
        \midrule
        $\mathbb{R}^5$ & New1 & $\frac{3}{2}$:2, $\frac{10}{3}$:3, $\frac{1}{3}$:3, 5:5, 0:10, -1:1 & 10 \\
         & New4 & 5:5, 3:4, $\frac{1}{2}$:4, 0:10, -1:1 &  \\
        & 5-simplex & 5:5, $\frac{5}{3}$:9, 0:10 \\
        \bottomrule
    \end{tabular}
    % \end{table*}
    \end{table}
    
\end{document}
\endinput
%%
%% End of file `sample-sigconf- latex.tex'.
