\section{Related work}
For $S^1$, the solution of Problem \eqref{eq:fekete_log} is known for any number of points, and is given by $n$ equidistributed points \cite[Theorem 2.3.3]{borodachov2019}.
    We denote this configuration as $n$-gon or ``Equator''.
	For $S^2$, the solutions are known only for up to six points ____, and for twelve points ____.
	Table~\ref{tab:known_solutions} lists the known solutions for up to six points, for all possible dimensions of the sphere.
	We use the term $(n-1)$-simplex for the configuration that has $n$ points on the sphere, all at the same distance. Note that the 3-simplex is the regular Tetrahedron.
    One of our contributions is to give the solution in $S^3$ for six points, which is the only unknown case in Table~\ref{tab:known_solutions}.
	
	\begin{table}
    \centering
		\caption{Known optimal configurations of the logarithmic Fekete problem for at most six points.}
		\label{tab:known_solutions}
		\begin{tabular}{ccccc}
			\toprule
			$n$ & $S^1$ & $S^2$ & $S^3$ & $S^4$ \\
			\midrule
			3 & 2-simplex & N/A & N/A & N/A \\
			4 & 4-gon & Tetrahedron ____ & N/A & N/A \\
			5 & 5-gon & Bi Piramid ____ & 4-simplex ____ & N/A \\
			6 & 6-gon & Octahedron ____ & ? & 5-simplex ____ \\
			\bottomrule
		\end{tabular}
	\end{table}

    Regarding critical configurations of Problem \eqref{eq:fekete_log}, little is known about them; even for the cases where the solutions are known.
    One thing that is known is that the problem has no local maxima \cite[Corollary 1.3]{beltran2013}, and that critical configurations always have center of mass zero \cite[Proposition 2]{dragnev2002}.
    It is also known that the global minima is not always unique. In particular, if $q=p^l$, with $p>2$ prime and $l \geq 1$, taking $n=(q+1)(q^3+1)$ points on $S^{d-1}$, $d=q \frac{q^3+1}{q+1}$, there are $\lfloor (l-1)/2 \rfloor$ essentially different global minima \cite[Section 1]{ballinger2009}.
    On the other hand, it is not known if the problem has local minima that are not global (spurious local minima). Also, the only saddle point that is mentioned in the literature is the Equator \cite[Section 2]{shub1993}.
    Other unknown property is whether every saddle may be classified using the Hessian. This is particularly important for numerical optimization ____. For 7 points on $S^2$, there is a critical configuration that the Hessian does not classify \cite[Remark 6.3]{constantineau2023}, although it is believed that it is a global minimum.
    Finally, based on numerical experiments, it is conjectured that the number of spurious local minima increases ``dramatically'' with the number of points \cite[Section 4]{rakhmanov1995}, although no spurious local minima are known.
    In this work we show that, for up to six points and all sphere dimensions: there is a spurious local minima in $S^3$, all global minima are essentially unique, there are many other types of saddles other than the Equator, and the Hessian always classifies.