\section{Empirical Validation on Real Benchmark} \label{sec:benchmark}


\paragraph{Setup.} As done in the pilot experiments (Figure \ref{fig:pilot_study} and \ref{fig:pilot_study_synthetic}), we used LLaVA v1.5 \cite{liu2024improved} and LLaVA NeXT \cite{liu2024llavanext} in 7B and 13B sizes and evaluated them on the LLaVA-Bench COCO and LLaVA-Bench Wild \cite{liu2023visual} datasets for assessing open-ended generation. To comprehensively examine diverse types of shifts, we further simulate synthetic distribution shifts as well as natural distribution shifts. For synthetic shifts, we consider 7 visual scenarios (1 ID case $+$ 2 synthetic perturbation types at 3 severity levels), and 5 text scenarios (1 ID case $+$ 2 synthetic perturbation types at 2 severity levels), resulting in $7 \times5 = 35$ synthetic scenarios, where 1 scenario is ID and the other 34 are OOD cases.
For natural shifts, we use 4 visual scenarios (1 ID + 3 OOD difficulty levels) and 7 text scenarios (1 ID-English $+$ 6 languages), yielding $4 \times 7 = 28$ natural scenarios. \textbf{This comprehensive design covers a total of 34 synthetic and 27 natural shifts}.
A summary of the OOD construction strategies is in Table \ref{tab:shift_scenarios}. 
\begin{table}[t]
\vspace{-0.5em}
\caption{\textbf{Summary of distribution shift scenarios.} }
\centering
\small
\begin{tabular}{@{}ll@{}}
\toprule
 Type &
  Strategy (\# of category) \\ \midrule
\begin{tabular}[c]{@{}l@{}}Synthetic visual shift\\ (COCO Images) \end{tabular} &
  \begin{tabular}[c]{@{}l@{}}Perturbation (2): Defocus blur, frost\\ Severity (3): Weak, Normal, Strong \end{tabular} \\ \midrule
  Synthetic text shift &
  \begin{tabular}[c]{@{}l@{}}Perturbation (2): Typo,  Word Replacement\\ Severity (2): Weak, Strong\end{tabular}  \\ \midrule
\begin{tabular}[c]{@{}l@{}}Natural visual shift \\ (in-the-wild Images) \end{tabular} & 
  \begin{tabular}[c]{@{}l@{}}LLaVA-Bench in-the-wild \\ Split (3): Easy, Normal, Hard\end{tabular}  \\ \midrule
  Natural text shift & 
  Translation (6): GE, CH, KO, EL, AR, HI \\ \bottomrule
\end{tabular} \label{tab:shift_scenarios}
\vspace{-0.5em}
\end{table}

\paragraph{Estimation of MI and JSD.} 
For the empirical realization of our theoretical statements, we adopt a popular neural estimator for MI, CLUB \cite{cheng2020club} to compute empirical EMID, and a JS divergence estimator, RJSD \cite{hoyos2023representation}, to compute empirical EMID and its upper-bound (see Appendix~\ref{appendix:implementation_details} for details). Experiments with 23 alternative implementations derived consistent conclusions (Please refer to Table \ref{tab:ablation_emiwr} and \ref{tab:ablation_emidup}).

\paragraph{Correlation between win rate and EMI.}
We first conduct the Spearman correlation analysis and Kendall's tau analysis between the win rate and our empirical estimates of EMI. In Table \ref{tab:wr_emi_corr}, we can see that EMI estimates exhibit a strong correlation with win rate, both in terms of absolute coefficient and $p$-value, across all models. This empirical evidence validates the theoretical connection between EMI and win rate discussed in Theorem \ref{Main-thm1-thm}. Therefore, our EMI can be used as a reliable and cost-efficient alternative to win rate for MLLM evaluation with theoretical guarantees.

\begin{table}[t]
\vspace{-0.6em}
\caption{\textbf{Spearman rank correlation and Kendall's tau between win rate and EMI.} We conduct correlation analysis between the win rate (Eq. \eqref{eq:win_rate}) and EMI (Eq. \eqref{eq:emi}) on 34 synthetic and 27 natural distribution shifts across four MLLMs.
}
\centering
{\footnotesize
\begin{tabular}{@{}c|l|cccc@{}}
\toprule
& & \multicolumn{2}{c}{Spearman} & \multicolumn{2}{c}{Kendall} \\ 
& Model & $\rho$ & $p$-val & $\tau$ & $p$-val \\ \midrule
& LLaVA v1.5 7B & 0.794 & $<$0.001 & 0.604 & $<$0.001              \\
& LLaVA v1.5 13B & 0.652 & $<$0.001 & 0.483 & $<$0.001              \\
& LLaVA NeXT 7B & 0.738 & $<$0.001 & 0.564 & $<$0.001              \\
\multirow{-4}{*}{\rotatebox[origin=c]{90}{Synthetic}} & LLaVA NeXT 13B & 0.726 & $<$0.001 & 0.527 & $<$0.001 \\ \midrule
& LLaVA v1.5 7B & 0.610 & 0.001 & 0.450 & 0.001 \\
& LLaVA v1.5 13B & 0.720 & $<$0.001 & 0.575 & $<$0.001              \\
& LLaVA NeXT 7B & 0.593 & 0.001 & 0.435 & 0.001 \\
\multirow{-4}{*}{\rotatebox[origin=c]{90}{Natural}} & LLaVA NeXT 13B & 0.457 & 0.014 & 0.321 & 0.017 \\ \bottomrule
\end{tabular} \label{tab:wr_emi_corr}
}
\vspace{-0.6em}
\end{table}

\begin{table}[!thb]
\vspace{-0.6em}
\caption{\textbf{Pearson correlation analysis between EMID and its upper bound.} We provide Pearson $r$ and $p$-value between the empirical estimates of EMID (Eq. \eqref{eq:emid}) and its upper bound (Eq. \eqref{eq:emid_bound_simple}) on 34 synthetic and 27 natural shifts scenarios per model. %, and observe strong and significant correlations.
}
\centering
{\scriptsize
\begin{tabular}{@{}l|cccc@{}}
\toprule
               & \multicolumn{2}{c}{Synthetic} & \multicolumn{2}{c}{Natural}  \\ 
Model          & Pearson $r$  & $p$-val & Pearson $r$ & $p$-val \\ \midrule
LLaVA v1.5 7B  & 0.755      & $<$0.001 & 0.553     & 0.003            \\
LLaVA v1.5 13B & 0.785      & $<$0.001 & 0.638     & $<$0.001 \\
LLaVA v1.5 7B  & 0.742      & $<$0.001 & 0.594     & 0.001            \\
LLaVA v1.5 13B & 0.807      & $<$0.001 & 0.550     & 0.003            \\ \midrule
All models     & 0.746      & $<$0.001 & 0.565     & $<$0.001 \\ \bottomrule
\end{tabular} \label{tab:emid_ub_pearson}
}
\vspace{-0.6em}
\end{table}
\paragraph{Verification of bound.} We now validate our main Theorem. Figure \ref{fig:emid_ub_scatter} (left two) shows the scatter plots comparing EMID with its upper bound across four models, each evaluated over 34 and 27 synthetic and natural distribution shifts. We see a clear trend between EMID and its upper bound in the synthetic shift where we could directly control the severity of shifts. While the natural shift setup is noisier, a similar overall trend is observed. Meanwhile, our bounds depend on the distributional discrepancy between the model's response $\hat{y}$ and the ground truth response $y$. Thus, they naturally induce different bounds for each MLLM. The right panel of Figure \ref{fig:emid_ub_scatter} presents model-wise plots with linear regression coefficients, where we observe that each model has a different degree of performance sensitivity against shifts. Pearson correlation analysis results in Table \ref{tab:emid_ub_pearson} further confirm statistically significant correlations between EMID and its upper bound, supporting the validity of the theorem.

\paragraph{Partial bound analysis.} It is common that we can not access the ground truth response $Y$ from our evaluation dataset in advance. Then, one may want to use the EMID upper bound as an estimator of the maximum risk of MLLM given two datasets, i.e., $\max \text{EMID}(P_{\mathbf{X}Y},Q_{\mathbf{X}Y};\theta)$, by neglecting the output-related term $\Delta$. In Figure \ref{fig:emid_partialub_scatter}, we investigate whether the summation of two JS divergence terms can still be predictive for EMID. Although the trends become loose compared to the full bound due to the non-optimality of MLLM parameters, the partial upper bound still has moderate correlations (denoted by Pearson $r$) with EMID.
\begin{figure}[!ht]
    \vspace{-0.65em}
    \centering
    \includegraphics[width=0.98\linewidth]{rawfiles/emid_ub_jsdvtonly.pdf}
    \vspace{-0.7em}
    \caption{\textbf{Scatter plot with regression line between empirical estimates of EMID and partial components of its upper bound.} We remove the $\Delta$ term of bound (Eq. \eqref{eq:emid_bound_simple}) and only use the estimates of JSD terms over visual and text inputs. 
    }
    \label{fig:emid_partialub_scatter}
    % \vspace{-1em}
\end{figure}
