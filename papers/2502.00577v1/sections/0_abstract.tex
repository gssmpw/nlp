\begin{abstract}
Multimodal large language models (MLLMs) have shown promising capabilities but struggle under distribution shifts, where evaluation data differ from instruction tuning distributions. 
Although previous works have provided empirical evaluations, we argue that establishing a formal framework that can characterize and quantify the risk of MLLMs is necessary to ensure the safe and reliable application of MLLMs in the real world. 
By taking an information-theoretic perspective, we propose the first theoretical framework that enables the quantification of the maximum risk of MLLMs under distribution shifts. Central to our framework is the introduction of Effective Mutual Information (EMI), a principled metric that quantifies the relevance between input queries and model responses. We derive an upper bound for the EMI difference between in-distribution (ID) and out-of-distribution (OOD) data, connecting it to visual and textual distributional discrepancies.
Extensive experiments on real benchmark datasets, spanning 61 shift scenarios empirically validate our theoretical insights. 
\end{abstract}