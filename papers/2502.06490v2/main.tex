\documentclass[journal]{IEEEtran}
% \documentclass[lettersize,journal]{IEEEtran}
% \documentclass[10pt,journal,compsoc]{IEEEtran}
\usepackage{amsmath,amsfonts}
\usepackage{algorithmic}
\usepackage{algorithm}
\usepackage{array}
\usepackage[caption=false,font=normalsize,labelfont=sf,textfont=sf]{subfig}
\usepackage{textcomp}
\usepackage{stfloats}
\usepackage{url}
\usepackage{verbatim}
\usepackage{makecell}
\usepackage{graphicx}
\usepackage{multirow}
\usepackage{multicol}
\usepackage{booktabs}
\usepackage{xcolor}
\usepackage{hyperref}
\usepackage{cite}

\usepackage{enumitem}
\usepackage{amssymb}
\usepackage{bm}
\usepackage{subcaption} % Use subcaption for subfigures

\usepackage{tikz}
\usetikzlibrary{mindmap}
\usepackage{smartdiagram}
\usesmartdiagramlibrary{additions}
\usepackage{forest}
\usetikzlibrary{shadows}
\usepackage{relsize}
% Colors images
\usepackage{color}
\definecolor{lightcoral}{rgb}{0.94, 0.5, 0.5}
\definecolor{lightgreen}{rgb}{0.56, 0.93, 0.56}
\definecolor{lightyellow}{rgb}{0.94, 0.84, 0.6}
\definecolor{brightlavender}{rgb}{0.75, 0.58, 0.89}

\definecolor{skyblue}{rgb}{0.53, 0.81, 0.92}
\definecolor{peachpuff}{rgb}{1.0, 0.85, 0.73}
\definecolor{goldenrod}{rgb}{0.85, 0.65, 0.13}
\definecolor{orchid}{rgb}{0.85, 0.44, 0.84}
\definecolor{salmon}{rgb}{0.98, 0.5, 0.45}
\definecolor{turquoise}{rgb}{0.25, 0.88, 0.82}
\definecolor{plum}{rgb}{0.87, 0.63, 0.87}
\definecolor{khaki}{rgb}{0.94, 0.9, 0.55}
\definecolor{slateblue}{rgb}{0.42, 0.35, 0.8}
\definecolor{forestgreen}{rgb}{0.13, 0.55, 0.13}
\definecolor{midnightblue}{rgb}{0.1, 0.1, 0.44}
\definecolor{lightsteelblue}{rgb}{0.69, 0.77, 0.87}

\definecolor{limegreen}{rgb}{0.2, 0.8, 0.2}
\definecolor{palegreen}{rgb}{0.6, 0.98, 0.6}
\definecolor{springgreen}{rgb}{0.0, 1.0, 0.5}
\definecolor{mediumseagreen}{rgb}{0.24, 0.7, 0.44}
\definecolor{seagreen}{rgb}{0.18, 0.55, 0.34}
\definecolor{yellowgreen}{rgb}{0.6, 0.8, 0.2}
\definecolor{olivedrab}{rgb}{0.42, 0.56, 0.14}
\definecolor{darkseagreen}{rgb}{0.56, 0.74, 0.56}
\definecolor{lightseagreen}{rgb}{0.13, 0.7, 0.67}
\definecolor{forestgreen}{rgb}{0.13, 0.55, 0.13}
\definecolor{darkolivegreen}{rgb}{0.33, 0.42, 0.18}
\definecolor{greenyellow}{rgb}{0.68, 1.0, 0.18}
\definecolor{chartreuse}{rgb}{0.5, 1.0, 0.0}
\definecolor{mintgreen}{rgb}{0.6, 1.0, 0.6}

\hyphenation{op-tical net-works semi-conduc-tor IEEE-Xplore}
% updated with editorial comments 8/9/2021

\begin{document}

\title{Recent Advances in Discrete Speech Tokens: A Review}
% ,~\IEEEmembership{Student Member,~IEEE}
\author{Yiwei Guo, Zhihan Li, Hankun Wang,  Bohan Li, Chongtian Shao, Hanglei Zhang, Chenpeng Du, Xie Chen, Shujie Liu, Kai Yu
% ,~\IEEEmembership{Senior Member,~IEEE}
        % <-this % stops a space
% \author{Zhihan Li,~\IEEEmembership{Student Member,~IEEE}}
\thanks{Corresponding Author: Kai Yu. Email: kai.yu@sjtu.edu.cn}% <-this % stops a space
\thanks{Yiwei Guo, Zhihan Li, Bohan Li, Chongtian Shao, Hanglei Zhang, Hankun Wang, Chenpeng Du, Xie Chen and Kai Yu are with the MoE Key Lab of Artificial Intelligence, AI Institute; X-LANCE Lab, Department of Computer Science and Engineering, Shanghai Jiao Tong University,
Shanghai, China. Email: yiwei.guo@sjtu.edu.cn}% <-this % stops a space
% \thanks{Shuai Wang is with Shenzhen Research Institute of Big Data, Shenzhen, China.}
\thanks{Shujie Liu is with Microsoft Research Asia (MSRA), Beijing 100080, China. 
% Email: shujliu@microsoft.com
}
% \thanks{Manuscript received April 19, 2021; revised August 16, 2021.}
}

% The paper headers
% \markboth{Journal of \LaTeX\ Class Files,~Vol.~14, No.~8, August~2021}%
% {Shell \MakeLowercase{\textit{et al.}}: A Sample Article Using IEEEtran.cls for IEEE Journals}

% \IEEEpubid{0000--0000/00\$00.00~\copyright~2021 IEEE}
% Remember, if you use this you must call \IEEEpubidadjcol in the second
% column for its text to clear the IEEEpubid mark.

\maketitle

\begin{abstract}
% With the rapid development of speech generation in the era of large language models (LLMs), discrete speech tokens have emerged as a fundamental representation for speech.
% These tokens are discrete, short and compact representations not only favorable for transmission and storage, but also convenient for incorporating speech into the language modeling paradigm.
% There are two major categories of discrete speech tokens, acoustic and semantic tokens, each of which has evolved into rich research fields with various motivations, designs and methods.
% This survey provides a comprehensive overview of the existing taxonomy and recent advances in building discrete speech tokens, takes an in-depth look at the pros and cons in each research direction, and presents unified experimental comparisons for different types of tokens.
% We also discuss and summarize existing challenges in academia, with which we anticipate to provide inspirations for the future development of discrete speech tokens. 
The rapid advancement of speech generation technologies in the era of large language models (LLMs) has established discrete speech tokens as a foundational paradigm for speech representation. These tokens, characterized by their discrete, compact, and concise nature, are not only advantageous for efficient transmission and storage, but also inherently compatible with the language modeling framework, enabling seamless integration of speech into text-dominated LLM architectures. Current research categorizes discrete speech tokens into two principal classes: \textit{acoustic} tokens and \textit{semantic} tokens, each of which has evolved into a rich research domain characterized by unique design philosophies and methodological approaches. This survey systematically synthesizes the existing taxonomy and recent innovations in discrete speech tokenization, conducts a critical examination of the strengths and limitations of each paradigm, and presents systematic experimental comparisons across token types. Furthermore, we identify persistent challenges in the field and propose potential research directions, aiming to offer actionable insights to inspire future advancements in the development and application of discrete speech tokens.
\end{abstract}

\begin{IEEEkeywords}
Discrete speech tokens, neural audio codec, speech tokenizer, speech LLMs, spoken language modeling, speech generation, acoustic tokens, semantic tokens
\end{IEEEkeywords}

\IEEEpubidadjcol

\documentclass[../main.tex]{subfiles}
\graphicspath{{../images/}}
\makeatletter
\def\input@path{{../images/}}
\makeatother
\begin{document}
\section{Introduction}
\begin{figure}
\centering
\begin{tikzpicture}
\node[inner sep=0pt] (ws) at (0, 0) {
\includegraphics[height=.4\textwidth, trim={10cm 0 10cm 0},clip]{world_space.png}};
\node[inner sep=0pt] (cs) at (6,0) {\includegraphics[height=.4\textwidth, trim={10cm 1cm 10cm 4cm},clip]{conf_space.png}};
\end{tikzpicture}
\vspace{-5pt}
\label{fig:pbrm_intro}
\caption{\textbf{Left}: Shows world space obstacles as grey spheres. Robots start and goal configuration is colored red and green, respectively. Configurations along the computed path are colored transparent blue. \textbf{Right:} Mapped world space scenario to configuration space. Obstacle region is the grey mesh. Red spheres are collision-free regions computed by the neural SCDF. The optimized shortest path in the convex corridor is the blue curve.}
\vspace{-25pt}
\end{figure}
Motion planning is the problem of finding a collision-free trajectory that connects a given start and goal configuration. The planning takes place in the configuration space of the robot. For single body robots, like mobile robots or drones, the configuration space and the world space are usually the same. This simplifies the planning, since explicit obstacle representations are available which enables geometrical tools like separating hyperplanes, smallest distance to obstacles etc., to be used when designing motion planning algorithms. For multi-body robots like manipulators, the situation is completely different. The world space obstacles are usually mapped to non-convex regions, and to make the problem even harder, the mapping is usually not known. Forming explicit representations of the obstacle region in the configuration space is usually too expensive or intractable. Despite all of this, sampling based planners are used with great success, which mainly is due to their use of implicit representations of the obstacle region. The basic idea is to construct a graph in the configuration space that covers and connects the collision-free region. From this graph, a path can be extracted that connects a given start and goal configuration. The approach is computationally expensive, since the graph is constructed with the smallest geometrical building block available, points, which represents a collision-check. Furthermore, the extracted paths from the graph are non-smooth and jagged due to the stochastic nature of the approach. This adds an additional post-processing step to the process, where the paths are shortcutted and smoothened, before the path can be used for tracking. Clearly a lot of time is invested to form this graph and produce smooth paths. Thus, if the obstacles start to move, then all of this work is done in no use, since all points that make up this graph need to be re-verified, which is simply too time consuming to be done in real time.
\\\\
In this work, we want to address the existing drawbacks of the sampling based planners. Our main contribution is an improved motion planner where each vertex in the graph covers a collision-free region in the form of a sphere instead of a point and where the edges are formed with neighboring intersecting spheres. This representation has the advantage of instead of returning piecewise linear paths, returning a sequence of overlapping spheres, i.e. a convex corridor, that connects a given start and goal configuration, illustrated in Figure \ref{fig:pbrm_intro}. This convex corridor allows us to use convex optimization to produce smooth trajectories, instead of computationally expensive post-processing methods. The representation further allows us to estimate the coverage of the collision-free space, which gives us awareness and feedback in the offline roadmap construction phase. Finally, our representation is simple to adapt to moving obstacles, simply requery for the new radii and recheck for intersections. 
\\\\
The spherical collision-free regions are formed using a signed distance function (SDF), which is a function that returns the smallest distance from an arbitrary point to the boundary of an obstacle. As the name implies, the distance is signed, thus if the point is inside the obstacle it is negative otherwise positive. If the distance is positive, a sphere with radius equal to the distance is guaranteed to cover a collision-free region. Using an SDF in motion planning is not new, but what is novel about our approach is that we express the distance in the configuration space instead of the world space and by doing so allows us to form these convex collision-free regions. We refer to the resulting SDF as a signed configuration distance function (SCDF). Computing an SCDF analytically is non-trivial, our approach is therefore to parameterize the SCDF with a deep neural network and learn the mapping by supervised learning. Our resulting neural SCDF can compute distances for different parameter values of obstacle shapes and we also show how multiple distances can be combined, thus making our approach flexible.
\section{Related work}
Motion planning algorithms can roughly be divided into three families, grid-based, sampling based and optimization based methods. Grid-based methods (GBM) discretize the planning space from which a graph is then compiled. A standard search method is A$^\star$ \citep{a_star}, which is classified as an \textit{informed} search method, since it employs a heuristic function to speed up the search. A$^\star$ guarantees to return an optimal path at the level of discretization used. GBMs usually discretize the planning space by a regular lattice and this limits the GBMs to problems with low dimensionality due to the curse of dimensionality. Thus, GBMs are usually limited to single-body robots where the degrees of freedom (DOF) are low. To overcome the inherent scaling problem with the GBMs, stochastic methods are usually used for multi-body robots. These methods are termed as sampling-based methods (SBM) and core members within this family are the rapidly-exploring random trees (RRT) \citep{rrt} and the probabilistic roadmap (PRM) \citep{prm}. RRT grows a tree from the start configuration and explores the collision-free region in a rapid way until it is able to connect to the goal region. RRT is usually improved by bi-directional planning \citep{rrt_connect}, i.e. an additional tree is grown from the goal configuration and the trees are tested for connection after any tree has been expanded. RRT is a single-query method, thus it searches for a path from scratch each time it is queried. Contrary to this, PRM is a multi-query method, which solves for multiple queries without starting from scratch. PRM does this by creating a roadmap (graph) that covers the collision-free space as an offline step. The graph is then used to solve for multiple queries. PRMs are used in cases where the environment does not change since the extra offline step is too computationally costly and needs to be re-done if the environment is changed. In our work, we address this inherent issue by using a different roadmap representation. Our vertices in the graph cover a collision-free region in the form of spheres and we form the edges by checking for intersecting spheres. If something in the environment changes, we recompute the spheres radii and recheck the intersections, without relying on collision detection. We use a trained neural network to compute the sphere radius, therefore querying for the radius can be done fast, hence our representation enables the PRM for dynamic environments.
\\\\
In the recent decades, optimization based methods (OBM) \citep{chomp, schulman, itomp, stomp} have been introduced as an alternative to SBM for multi-body robots. Like the SBM, the OBMs scale well to higher dimensional problems and produce smoother motion. It is common to use a SDF in the optimization since it is a smooth function, thus enabling gradient-based methods. However, the standard way of expressing the SDF is in world space. The distance therefore needs to be mapped to the configuration space by the forward kinematics. This mapping makes the optimization problem a non-linear program (NLP), which is computationally expensive to solve. Recently, a different approach has been proposed. In \cite{mp_gcs} motion planning is formulated as a convex optimization problem by using the graph of convex sets framework \citep{gcs}. The underlying idea is to decompose the collision-free space into intersecting convex sets from which a convex optimization problem is formulated. In cases where an explicit representation of the obstacles in the configuration space exists, like for single-body robots, creating collision-free convex regions can be done fast \citep{iris}. For multi-body robots, this is non-trivial. Existing work does this successfully \citep{iris_nlp, iris_c} by an optimization based approach, but the methods are still too time consuming to be used in the presence of moving obstacles. Our approach is instead to use deep learning to learn an SDF expressed in the configuration space. With this, we can query for shortest distances to the collision boundary, which allows us to expand spherical regions which are collision-free. Our approach is fast and therefore enables our suggested roadmap planner to be used in dynamic environments.
\\\\
Recent research has focused on learning collision detection \citep{fk_kernel_distance, diffco, graphdistnet} by predicting the signed distance between the robot links and the surrounding obstacles in the world space. The learned SDF is used in trajectory optimization but since the distance is expressed in the world space, the problem becomes an NLP and therefore takes a long time to solve. We take a novel approach and suggest to instead express the signed distance in the configuration space. This allows us to improve the PRM at the same time as it enables convex optimization for trajectory optimization, which runs faster and is more reliable than NLP solvers. In \cite{cspf} a learned signed distance function in the configuration space is proposed similar to our approach. However, their approach is restricted to point cloud representations, while we propose to represent the obstacles as parameterized geometric shapes, e.g. spheres. Furthermore, we also show how to use our learned SCDF to improve an existing roadmap planner.
\section{Problem formulation}
A robot is located in the world space, $\W \subset \R^3 $. The unique location of the robot is given by its configuration $\q \in \C$, where $\C$ is the configuration space. The set of points covered by the robots bodies at a certain configuration is expressed as $\B(\q) \subset \W$. The robot is surrounded by $\NrObst$ obstacles $\O = \bigcup_{i=1}^{\NrObst} \O_i$, where  $\O_i \subset \W$. The representation of the obstacle in the configuration space is the set $\C\O_i = \{\q \in \C \: |\: \B(\q) \cap \O_i \neq \emptyset \}$. The obstacle space is formed as $\Co = \bigcup_{i=1}^{\NrObst} \C \O_i$. The complement is referred to as the free space, $\Cf = \C \setminus \Co$. The path planning problem is a tuple, ($\Cf$, $\qStart$, $\qGoal$), where we want to connect a query pair, consisting of a start, $\qStart$, and goal configuration, $\qGoal$, with a geometric path, $\q(s): [0, 1] \mapsto \Cf$, such that $\q(0)=\qStart$ and $\q(1)=\qGoal$, or report correctly when such a path does not exist.
\end{document}


\begin{figure*}[t]
\begin{center}
\includegraphics[width=.85\linewidth]{fig_overview_v3.pdf}
\end{center}
\caption{
FastAtlas Overview: In each frame, we compute charts spanning fully or partially visible triangles (a), determine texture space bounding boxes for the visible portions of the view-space projections of each chart, and tightly pack these boxes into atlases (b, here $2K \times 2K$). We simultaneously bijectively parameterize and shade the charts into their atlas boxes, obtaining high quality texture space shading (c), and use this shading to render the shaded frames (d).}
\label{fig:overview}
\label{fig:alg_overview}
\end{figure*}

\section{Overview}
\label{sec:overview}
Our work has two core contributions: a real-time, GPU-based algorithm for tight packing of general parameterized charts into compact atlases; and a real-time TSS method that
utilizes this packing.  

\paragraph*{FastAtlas Packing.}
FastAtlas runs entirely on the GPU as a series of compute shaders. It takes the bounding boxes of parameterized charts as input, and packs them into an atlas (Fig~\ref{fig:overview}b, Sec.~\ref{sec:pack}). As such, the only input it requires are the dimensions of the bounding boxes.
Its outputs are deterministic; identical input charts are packed into identical atlases. This is critical for TSS and similar applications, as it ensures that consecutive frames taken from the same camera view have the same shading. Even minute shading differences across such frames can cause sampling jitter, leading to undesirable flicker \cite{baker2012rock}. 
While prior methods such as \cite{mueller2018shading,hladky2019tessellated,hladky2021snakebinning,Neff2022MSA} cap the dimensions of the charts that can be packed as-is for a given atlas size, and scale down all charts that exceed these dimensions, we scale all charts by the same factor, and do so only when strictly necessary to achieve packing success (Figs~\ref{fig:atlas},~\ref{fig:sas_issues}). 

\paragraph*{TSS using FastAtlas.}
Our end-to-end TSS atlas generation method combines the packing method above with a novel approach for computing seamless per-frame charts. 
We define our charts as the connected components of the visible surfaces in each frame (Fig.~\ref{fig:overview}a), and efficiently compute them using a parallel union-find algorithm (Sec.~\ref{sec:visible}). Since the boundaries of these charts coincide with the contours of the rendered surface, they are {\em invisible} to the viewer. This approach 
eliminates the artifacts caused by shading discontinuities along visible seams (Fig.~\ref{fig:seams}). 

\begin{parWithWrapFigure}
\begin{wrapfigure}{l}{.27\columnwidth}%
\includegraphics[width=\linewidth]{fig_inset_view_plane.pdf}%
\end{wrapfigure}
We bijectively parametrize the {\em visible portions} of our charts by projecting them to view space (inset). This maps a constant number of texels to each pixel in the final rendered output, evenly distributing residual undersampling error across all image pixels. While conceptually straightforward, efficiently parameterizing charts containing partially visible triangles using viewspace projection is non-trivial, as the visible portions may no longer be triangular (e.g. green triangle in the inset); applying naive projection to triangles with vertices behind the camera may produce ill-posed results. Clipping triangles before projection is both computationally expensive and significantly complicates downstream operations. We avoid explicit clipping by observing that all that is required for atlas packing is the dimensions of, potentially conservative, bounding boxes of these projected visible portions. We compute such bounding boxes without explicit chart clipping by adapting a conservative screen coverage estimator \shortcite{Blinn:CalculatingScreenCoverage} (Sec.~\ref{sec:box}). We then pack the computed boxes using FastAtlas. 
\end{parWithWrapFigure}

Finally, we shade the visible portion of each chart into its corresponding atlas bounding box (Fig~\ref{fig:overview}c). 
The resulting texture is then used during rasterization as a standard texture map (Fig. ~\ref{fig:overview}d). 
Our framework is compatible with all existing approaches for texture space shading, including forward shading, raytraced illumination, or deferred shading in texture space \cite{baker:2016}. In the examples shown, we use the standard forward shading based rendering pipeline included in the G3D Innovation Engine \cite{G3D17}, a commercial grade renderer.


\section{Pre-requisites: Discrete Representation Learning}
\label{sec:prereq}

Discrete speech tokens are obtained through the quantization of continuous representations, which is usually achieved by offline clustering or online vector quantization algorithms.
% In this section, we briefly introduce the commonly used quantization methods in current discrete speech tokens as a preliminary.
This section provides a concise overview of the existing quantization methods commonly used in discrete speech tokens.

Denote $\bm x\in \mathbb R^d$ as a vector in the $d$-dimensional continuous space. A quantization process $q$ transforms $\bm x$ into a discrete \textit{token} in a finite set, i.e. $q(\bm x): \mathbb R^d\to\{1,2,...,V\}$ where $V$ is the \textit{vocabulary size}.
The output tokens are sometimes referred to as \textit{indexes} in the finite $V$-cardinal set.
The function $q$ is usually associated with a \textit{codebook} $\mathcal C=\{\bm c_1,\bm c_2,...,\bm c_V\}$ where every \textit{code-vector} $\bm c_i\in\mathbb R^d$ corresponds to the $i$-th token. 
The code-vectors are representations of tokens in the original $d$-dimensional space.
As $V$ elements can be encoded using $\lceil \log_2 V\rceil$ raw bits\footnote{We use $\lceil z\rceil$ to denote the ceiling of a scalar $z$, i.e., the smallest integer greater than or equal to $z$. 
Similarly, $\lfloor z\rfloor$ denotes the floor of $z$, i.e., the largest integer less than or equal to $z$.}, quantization often compresses the cost for data storage and transmission to a great extent.

\vspace{-0.15in}
\subsection{Offline Clustering}
\vspace{-0.05in}
Clustering is a simple approach for quantization. 
Given a dataset $X=\{\bm x_1,\bm x_2,...\bm x_N\}$, a clustering algorithm aims to assign each sample $\bm x_i$ to a group such that some cost is minimized.
The most frequently used clustering method for discrete speech tokens is k-means clustering~\cite{IKOTUN2023178}, e.g. in GSLM~\cite{lakhotia2021generative}.
K-means is a clustering algorithm based on Euclidean distances.
Its training process iteratively assigns each data sample to the nearest centroid, and moves cluster centroids till convergence, with a pre-defined number of clusters.
After training, the centroids form the codebook, and new data can be quantized to the index of the nearest centroid in this Voronoi partition.
In practice, centroids are usually initialized with the k-means++ algorithm~\cite{kmeans++} for better convergence.

Hierarchical agglomerative clustering has also been used in discrete speech tokens, which iteratively merges the closest clusters.
It is usually applied after k-means to reduce the number of clusters~\cite{cho2024sd,baade2024syllablelm}.
Other clustering algorithms are less explored in the context of discrete speech tokens.

\vspace{-0.12in}
\subsection{Vector Quantization}
\vspace{-0.04in}

\IEEEpubidadjcol

Clustering is often an isolate process, thus cannot be optimized together with other neural network modules.
Instead, vector quantization (VQ)~\cite{gray1984vector} enables a learnable network module that allows gradients to pass through when producing discrete representations.
Autoencoders with a VQ module is termed VQ-VAE~\cite{VQVAE}.
% There are multiple ways a VQ module can be realized:
There are multiple VQ methods:

\subsubsection{K-means VQ}
Like k-means clustering, k-means VQ method finds the code-vector closest to the input, i.e. 
\begin{equation}
    q(\bm x)=\underset{i\in \{1,2,...,V\}}{\arg\min} \|\bm x-\bm c_i\|^2.
\end{equation}
% \footnote{Note that $q$ outputs the code-vector now instead of the codebook index. Although it is a slight abuse of notation, the essence of $q$ is not changed.}
Then, code-vector $\bm c_k\triangleq\bm c_{q(\mathbf x)}$ is fed to subsequent networks.
As the $\min$ operation is not differentiable, straight-through estimators (STEs)~\cite{bengio2013estimating} are usually applied to graft gradients, i.e. $\operatorname{STE}(\bm c_k,\bm x)=\bm x+\operatorname{sg}(\bm c_k-\bm x)$ where $\operatorname{sg(\cdot)}$ stops tracking gradients.
In this way, the input value to subsequent networks is still $\bm c_k$, but gradients are grafted to $\bm x$ in back propagation.
% In this way, the loss function is still calculated by the value of $q(\bm x)$, but gradients that should be placed on $q(\bm x)$ is now grafted to $\bm x$ itself.

Auxiliary loss functions are often used together with k-means VQ~\cite{VQVAE}: commitment loss $\mathcal L_{\text{cmt}}=\|\operatorname{sg}(\bm c_k)-\bm x\|^2$ and codebook loss $\mathcal L_{\text{code}}=\|\operatorname{sg}(\bm x)-\bm c_k\|^2$.
The commitment loss pushes the continuous input $\bm x$ towards the closest codebook entry, while the codebook loss does the opposite and updates the code-vector $\bm c_k$.
The two loss terms are weighted by different factors to put different optimization strengths on $\bm x$ and $\bm c_k$, as pushing $\bm c_k$ towards $\bm x$ is an easier task.
It is also common to replace $\mathcal L_{\text{code}}$ with exponential moving average (EMA) to update the codebook instead~\cite{razavi2019generating}, which does not rely on explicit loss functions.
% Although EMA does not rely on explicit loss functions, it achieves a similar goal that gradually merges the continuous input $\bm x$ into the code-vector $\bm c_k$.
% that $\bm x$ is quantized to.

VQ in high-dimensional spaces is known to suffer from codebook collapse,  where the codebook usage is highly imbalanced~\cite{lancucki2020robust,dhariwal2020jukebox}.
To improve the utilization of codebook, random replacement (as known as \textit{codebook expiration}) can be applied~\cite{dhariwal2020jukebox} on code-vectors that have remained inactive for a long time. 
Other solutions include additional auxiliary constraints such as entropy penalty~\cite{chang2022maskgit,yu2024language}, factorized codebook lookup in low-dimensional space~\cite{yu2022vectorquantized}, and adding a linear projection to update all code-vectors together~\cite{zhu2024addressing}. 
% SimVQ~\cite{zhu2024addressing} explains that the independent updating of some code-vectors causes them to occupy the entire space, which leads to codebook collapse, and using linear layers to update all code-vectors simultaneously can help mitigate this problem.
% SimVQ~\cite{zhu2024addressing} states that codebook collapse stems from the independent optimization trajectories of code-vectors that lead to  spatial dominance, and proposes to add a simple linear projection layer that simultaneously updates all code-vectors per step to mitigate this problem.

\subsubsection{Gumbel VQ}
Instead of quantizing by Euclidean distance, another choice is by probability. 
Gumbel VQ~\cite{jang2017categorical} uses Gumbel-Softmax as a proxy distribution for traditional Softmax to allow differentiable sampling.
Given input $\bm x$ and a codebook of size $V$, a transform $h(\cdot)$ is applied on $\bm x$ into $V$ logits: $\bm l=h(\bm x)\in \mathbb R^V$.
In inference, quantization is performed by choosing the index with the largest logit, i.e. $q(\bm x)=\arg\max_i \left\{\bm l^{(i)}\right\}$.
In training, samples are drawn from the categorical distribution implied by $\bm l$ for the subsequent neural networks.
To achieve efficient sampling and let gradients pass through, Gumbel trick is used:
\begin{align}
    &\bm u\in \mathbb R^V\sim \operatorname{Uniform}(0, 1),\bm v=-\log(-\log(\bm u)) \label{eq:gumbel-noise} \\
    &\bm s=\operatorname{Softmax}((\bm l+\bm v)/\tau) \label{eq:gumbel-softmax}
\end{align}
where Eq.\eqref{eq:gumbel-noise} samples Gumbel noise $\bm v$ element-wise, and Eq.\eqref{eq:gumbel-softmax} calculates Gumbel-Softmax distribution $\bm s$ with a temperature $\tau$.
The forward pass simply use $j=\arg\max_i \{\bm s^{(i)}\}$ as the sampled index, but the true gradient of Gumbel-Softmax is used in backward pass.
In other words, the gradient on the one-hot distribution corresponding to $j$ is grafted to $\bm s$ as an approximate.
The temperature $\tau$ balances the approximation accuracy and gradient variances.
% : a lower $\tau$ results in sharper $\bm s$ and thus more accurate gradient estimate, but higher gradient variances~\cite{jang2017categorical}. 
% In practice, $\tau$ is usually annealed from high to low~\cite{jang2017categorical,vq-wav2vec}.
The transform $h(\cdot)$ is usually parameterized as neural networks, or negatively proportional to Euclidean distances~\cite{jiang2023latent}.

After quantization, code-vector $\bm c_k$ with $k=q(\bm x)$ is fed to subsequent networks.
Gumbel VQ does not require additional losses, since code-vectors can be directly learned with gradients and do not need to be pushed towards $\bm x$.

\subsubsection{Finite Scalar Quantization}
% \textcolor{gray}{VQ in the high-dimensional space is known to suffer from codebook collapse, a phenomenon where only a portion of codebook is active or the codebook usage is highly imbalanced~\cite{lancucki2020robust,dhariwal2020jukebox}.}
As mentioned before, VQ methods based on code-vector assignment usually suffer from codebook collapse. 
Despite many efforts, this remains a crucial challenge.
Finite scalar quantization (FSQ)~\cite{mentzer2024finite} is an alternative to perform quantization in scalar domain.
FSQ quantizes each dimension of a vector $\bm x$ into $L$ levels.
% \footnote{$L$ should better be odd, if following the original FSQ paper~\cite{mentzer2024finite}.}
For the $i$-th dimension $\bm x^{(i)}$, FSQ transforms the values to into limited range and then rounds to integers, i.e. 
\begin{equation}
    q\left(\bm x^{(i)}\right)=\operatorname{round}\left(\lfloor L/2\rfloor\tanh\left(\bm x^{(i)}\right)\right).
\end{equation}
The quantized values are thus integers ranging from $-\lfloor L/2 \rfloor$ to $\lfloor L/2 \rfloor$\footnote{Following \cite{mentzer2024finite}, this is the symmetric case for $L$ being odd. When $L$ is even, there is an offset before rounding to obtain asymmetric quantized values.}.
For a $d$-dimensional vector $\bm x$, there are $L^d$ possible quantization outcomes.
% Common choices are $L\ge 5$ and $d\le 10$, so FSQ usually has a much smaller hidden dimension than VQ (where usually $d\ge 100$).
Hence, FSQ usually requires a much smaller hidden dimension than VQ.
STE is applied to pass gradients.
As quantization is simply done via rounding to integers, there is no explicit codebooks associated with the FSQ process.
% The range of $q$ here is also not a categorical set where indexes cannot be numerically compared as that of VQ, but an ordered set of integers.
% This indicates a different approach when using FSQ instead of VQ for generative tasks.
FSQ is reported to have  better codebook usage\footnote{Although there is no longer a codebook associated with code-vectors, codebook usage can still be measured among all possible $V=L^d$ outcomes.} especially for a large $V$ compared to VQ methods, without the need for auxiliary losses.

\subsubsection{Other VQ Tricks}
In many cases, a single VQ module suffers from a highly-limited representation space, thus results in poor performance compared to continuous counterparts. 
There are some widely-used VQ tricks that introduce multiple quantizers to refine the quantized space, as shown in Fig.\ref{fig:gvq-rvq}:

\begin{figure}
    \centering
    \includegraphics[width=0.53\linewidth]{figs/GVQ.png}
    \includegraphics[width=0.45\linewidth]{figs/RVQ.png}
    \caption{Diagram of GVQ (left) and RVQ (right).}
    \label{fig:gvq-rvq}
\end{figure}

\begin{enumerate}[leftmargin=5mm]
    \item \textit{Grouped VQ (GVQ)}, also known as \textit{product quantization}~\cite{product_quantization}. It groups the input vector $\bm x$ by dimensions and apply VQ on different parts of $\bm x$ independently. They can have different or shared codebooks. The VQ outputs are then concatenated along dimensions to match that of $\bm x$. For instance, GVQ is used in neural word embeddings~\cite{9164982} and speech self-supervised learning models~\cite{vq-wav2vec,baevski2020wav2vec} to achieve efficient quantization.
    \item \textit{Residual VQ (RVQ)}, also known as \textit{multi-stage quantization}~\cite{multiple-stage-vector-quantization}. It adopts a serial approach that iteratively quantizes the residual of the last quantizer.
    Similar to GVQ, RVQ also has multiple quantizers.
    For the $i$-th quantizer $q_i$ with input $\bm x_i$ and output code-vector $\bm c_{k}$, the residual is defined as $\bm x_{i+1}=\bm x_i-\bm c_{k}$.
    The outputs from all $q_i$ are finally summed as the quantized result of $\bm x$.
    In this way, information in the codebooks is supposed to follow a coarse-to-fine order, and more details in the original $\bm x$ can be preserved than a plain quantizer.
    It is used in various speech codecs~\cite{zeghidour2021soundstream,encodec,kumar2024high}, for instance.
\end{enumerate}
GVQ and RVQ can be flexibly combined to form GRVQ~\cite{yang2023hifi} that applies RVQ on each GVQ branch for better codebook utilization.
% A variant of RVQ is cross-scale VQ (CSVQ)~\cite{jiang22_interspeech} where residuals are not defined as the quantization margins directly, but instead 
A network can also contain multiple VQ modules at different places, like cross-scale VQ (CSVQ)~\cite{jiang22_interspeech} where every decoder layer has a quantizer inside.

Note that RVQ naturally produces an order of importance in residual layers, while all quantizers in GVQ are equally important.
Such order of importance can also be enforced in GVQ by a ``nested dropout'' trick~\cite{rippel2014learning}.
% There is also a ``nested dropout'' trick~\cite{rippel2014learning} that assigns an importance order to GVQ, by manually define an order of quantizers and randomly dropping-out the last few quantizers in training.



\vspace{-0.03in}
\section{Speech Tokenization Methods: Acoustic Tokens}
\label{sec:acoustic}

% \begin{table*}[]
% \centering
% \caption{A summary of famous acoustic speech tokens (neural speech codecs). 
% Italic ``\textit{C,T,U}'' denote CNN, Transformer or U-Net-based generator architecture in Fig.\ref{fig:generator}.
% Symbols `/' and `-' denote ``or'' and ``to'' for different model versions, and ``+'' means different configurations in different VQ streams in a single model. 
% $Q,F,V$ mean number of quantizers, frame rate and vocabulary size of each quantizer respectively. 
% For example, ``$Q=2$, $V$=8192+($2^{12}$-$2^{15}$)'' in SemantiCodec means one of the two VQ streams has 8192 possible codes, and the other can vary from $2^{12}$ to $2^{15}$ in different configurations.
% Bitrates are computed by $\frac1{1000}\sum_{i=1}^Q F_i\lceil \log_2 V_i\rceil$ kbps, without entropy coding. }
% \label{tab:acoustic-metadata}
% \resizebox{\textwidth}{!}{
% % if necessary, use "\makecell{A\\B}" to create line break in a cell
% \begin{tabular}{@{}lccccccc@{}}
% \toprule
% \textbf{{Acoustic Speech Tokens}} & \textbf{Model Framework} & \textbf{{Sampling Rate}} & \textbf{Quantization} & $Q$ & \textbf{$F$} & \textbf{$V$} & \textbf{{Bitrate (kbps)}}  \\ \midrule
% \multicolumn{6}{l}{\textbf{\textit{General-purpose acoustic tokens}}} \\
% SoundStream~\cite{zeghidour2021soundstream} & VQ-GAN (\textit{C}) & 24kHz & RVQ & max 24 & 75Hz & 1024 & max 18.00 \\
% EnCodec~\cite{encodec} & VQ-GAN (\textit{C})& 24kHz & RVQ & max 32 & 75Hz & 1024 & max 24.00  \\
% % LMCodec~\cite{LMCodec} & VQ-GAN & 16kHz & RVQ & max 24 & 50Hz & 1024 & max  \\
% TF-Codec~\cite{jiang2023latent} & VQ-GAN (\textit{C}) & 16kHz & GVQ & 3-32 & 25Hz & 512 / 1024 & 0.68-8.00 \\
% Disen-TF-Codec~\cite{jiang2023disentangled} & VQ-GAN (\textit{C}) & 16kHz & GVQ & 2 / 6 & 25Hz & 256 / 1024 & {0.40 / 1.50} \\
% AudioDec~\cite{audiodec} & VQ-VAE (\textit{C})+GAN& 48kHz & RVQ & 8 & 160Hz & 1024 & 12.80 \\
% HiFi-Codec\cite{yang2023hifi} & VQ-GAN (\textit{C})& 16 / 24kHz & GRVQ & 4 & 50-100Hz & 1024 & 2.00-4.00 \\
% DAC~\cite{kumar2024high} & VQ-GAN (\textit{C})& 44.1kHz & RVQ & 9 & 86Hz & 1024 & 7.74 \\
% LaDiffCodec~\cite{yang2024generative} & Latent diffusion & 16kHz & RVQ & 3 / 6 & 50Hz & 1024 & 1.50 / 3.00 \\
% {FreqCodec}~\cite{du2024funcodec} & VQ-GAN (\textit{C})& 16kHz & RVQ & max 32 & 50Hz & 1024 & max 16.00 \\
% TiCodec~\cite{ticodec} & VQ-GAN (\textit{C})& 24kHz & RVQ, GVQ & 1-4 & 75Hz & 1024 & 0.75-3.00  \\
% APCodec~\cite{APCodec} & VQ-GAN (\textit{C})& 48kHz & RVQ & 4 & 150Hz &1024 & 6.00 \\
% % FACodec~\cite{facodec} & 6 & 80Hz & 1024 & 4.80  \\
% % SSVC~\cite{SSVC} & 4 & 50Hz & 512 & 1.80 \\
% % SpeechTokenizer~\cite{zhang2024speechtokenizer} & 8 & 50Hz & 1024 & 4.00 \\
% {SRCodec~\cite{zheng2024srcodec}} & VQ-GAN (\textit{C}) & 16kHz & GRVQ   & 2-8 & 50Hz & 512+1024 & 0.95-3.80\\
% SQ-Codec~\cite{yang24l_interspeech} & VQ-GAN (\textit{C})& 16kHz & FSQ & 32 & 50Hz &19 & 8.00 \\
% Single-Codec~\cite{singlecodec} & VQ-GAN (\textit{T+C})& 24kHz & VQ & 1 & 23Hz & 8192 & 0.30  \\
% ESC~\cite{gu2024esc}  & VQ-GAN (\textit{U})& 16kHz & GVQ & max 18 & 50Hz & 1024 & max 9.00 \\
% CoFi-Codec~\cite{guo2024speaking} & VQ-GAN (\textit{U}) & 16kHz & GVQ & 3 & 8.33+25+50Hz & 16384 & 1.17 \\
% HILCodec~\cite{ahn2024hilcodec} & VQ-GAN (\textit{C})& 24kHz & RVQ & 2-12 & 75Hz & 1024 & 1.50-9.00 \\
% SuperCodec~\cite{zheng2024supercodec} & VQ-GAN (\textit{C})& 16kHz & RVQ & 2-12 & 50Hz & 1024 & 1.00-6.00\\
% SNAC~\cite{Siuzdak_SNAC_Multi-Scale_Neural_2024} & VQ-GAN (\textit{C})& 24kHz & RVQ & 3 &12+23+47Hz & 4096 & 0.98 \\
% dMel~\cite{bai2024dmel} & {Quantizer only} & 16kHz & FSQ & 80 & 80Hz & 16 & 25.60 \\
% WavTokenizer~\cite{ji2024wavtokenizer} & VQ-GAN (\textit{C})& 24kHz & VQ & 1 & 40 / 75Hz & 4096 & 0.48 / 0.90  \\
% BigCodec~\cite{xin2024bigcodec} & VQ-GAN (\textit{C})& 16kHz & VQ & 1 & 80Hz & 8192 & 1.04 \\
% LFSC~\cite{casanova2024low} & VQ-GAN (\textit{C})& 22.05kHz &  FSQ & 8 & 21.5Hz & 2016 & 1.89 \\
% NDVQ~\cite{niu2024ndvq} & VQ-GAN (\textit{C})& 24kHz &  RVQ & max 32 & 75Hz & 1024 & max 24.00 \\
% VRVQ~\cite{chae2024variable} & VQ-GAN (\textit{C})& 44.1kHz & RVQ & 8 & 86Hz & 1024 & {0.26 + max 6.89}\\
% % \textcolor{red}{SimVQ}~\cite{zhu2024addressing} & \\
% TS3-Codec~\cite{wu2024ts3codectransformerbasedsimplestreaming} & VQ-GAN (\textit{T})& 16kHz & VQ & 1 & 40 / 50Hz & $2^{16}$ / $2^{17}$ & 0.64-0.85 \\
% Stable-Codec~\cite{parker2024scalingtransformerslowbitratehighquality} & VQ-GAN (\textit{T})& 16kHz & FSQ  & 6 / 12 & 25Hz & 5 / 6 & 0.40 / 0.70 \\
% FreeCodec~\cite{zheng2024freecodecdisentangledneuralspeech} & VQ-GAN (\textit{C+T})& 16kHz & VQ & 1+1 & 50+7Hz & 256 & 0.45  \\
% \midrule
% \multicolumn{6}{l}{\textbf{\textit{Mixed-objective acoustic tokens: semantic distillation}}} \\
% {Siahkoohi et al.}~\cite{siahkoohi22_interspeech} & VQ-GAN (\textit{C})& 16kHz & RVQ & 2+1 / 2+2 / 6 & 25+50Hz & 64 & 0.60 / 0.90 / 1.80\\
% SpeechTokenizer~\cite{zhang2024speechtokenizer} & VQ-GAN (\textit{C})& 16kHz & RVQ & 8 & 50Hz & 1024 & 4.00 \\
% SemantiCodec~\cite{liu2024semanticodec} & Latent diffusion & 16kHz & VQ & 2 & 12.5-50Hz & {8192+($2^{12}$-$2^{15}$)} & 0.31-1.40 \\
% LLM-Codec~\cite{yang2024uniaudio15} & VQ-GAN (\textit{C})& 16kHz & RVQ & 3 & 8.33+16.67+33.33Hz & 3248+32000+32000& 0.85 \\
% X-Codec~\cite{ye2024codec} & VQ-GAN (\textit{C})& 16kHz & RVQ & max 8 & 50Hz & 1024 & max 4.00 \\
% SoCodec~\cite{guo2024socodec} & VQ-GAN (\textit{C})& 16kHz & GVQ & 1 / 4 / 8 & 25 / 8.3 / 4.2Hz & 16384 & 0.35 / 0.47 \\
% Mimi~\cite{kyutai2024moshi} & VQ-GAN (\textit{C+T})& 24kHz & RVQ & 8 & 12.5Hz & 2048 & 1.10 \\
% \midrule
% \multicolumn{6}{l}{\textbf{\textit{Mixed-objective acoustic tokens: disentanglement}}} \\
% SSVC~\cite{SSVC} & VQ-GAN (\textit{C})& 24kHz & RVQ & 4 & 50Hz & 512 & 1.80 \\
% PromptCodec~\cite{pan2024promptcodec} & VQ-GAN (\textit{C})& 24kHz & GRVQ & 1-4 & 75Hz & 1024 & 0.75-3.00 \\
% FACodec~\cite{facodec} & VQ-GAN (\textit{C})& 16kHz & RVQ & 1+2+3 & 80Hz & 1024 & 4.80 \\
% LSCodec~\cite{guo2024lscodec} & VQ-VAE (\textit{C+T})+GAN& 24kHz &  VQ & 1 & 25 / 50Hz & 1024 / 300& 0.25 / 0.45 \\
% SD-Codec~\cite{bie2024learning} & VQ-GAN (\textit{C})& 16kHz &  RVQ & 12 & 50Hz & 1024 & 6.00 \\
% \bottomrule
% \end{tabular}
% }
% % \vspace{-0.15in}
% \end{table*}


\begin{table*}[]
\centering
\caption{A summary of famous acoustic speech tokens (neural speech codecs). 
Italic ``\textit{C,T,U}'' denote CNN, Transformer or U-Net-based generator architecture in Fig.\ref{fig:generator}.
Symbols `/' and `-' denote ``or'' and ``to'' for different model versions, and ``+'' means different configurations in different VQ streams in a single model. 
$Q,F,V$ mean number of quantizers, frame rate and vocabulary size of each quantizer respectively. 
For example, ``$Q=2$, $V$=8192+($2^{12}$-$2^{15}$)'' in SemantiCodec means one of the two VQ streams has 8192 possible codes, and the other can vary from $2^{12}$ to $2^{15}$ in different configurations.
Bitrates are computed by $\frac1{1000}\sum_{i=1}^Q F_i\lceil \log_2 V_i\rceil$ kbps, without entropy coding. }
\label{tab:acoustic-metadata}
\resizebox{\textwidth}{!}{
% if necessary, use "\makecell{A\\B}" to create line break in a cell
\begin{tabular}{@{}lccccccc@{}}
\toprule
\textbf{{Acoustic Speech Tokens}} & \textbf{Model Framework} & \textbf{\makecell{Sampling\\Rate (kHz)}} & \textbf{\makecell{Quantization\\Method}} & \textbf{$Q$} & \textbf{$F$ (Hz)} & \textbf{$V$} & \textbf{{Bitrate (kbps)}}  \\ \midrule
\multicolumn{6}{l}{\textbf{\textit{General-purpose acoustic tokens}}} \\
SoundStream~\cite{zeghidour2021soundstream} & VQ-GAN (\textit{C}) & 24 & RVQ & max 24 & 75 & 1024 & max 18.00 \\
EnCodec~\cite{encodec} & VQ-GAN (\textit{C})& 24 & RVQ & max 32 & 75 & 1024 & max 24.00  \\
TF-Codec~\cite{jiang2023latent} & VQ-GAN (\textit{C}) & 16 & GVQ & 3-32 & 25 & 512 / 1024 & 0.68-8.00 \\
Disen-TF-Codec~\cite{jiang2023disentangled} & VQ-GAN (\textit{C}) & 16 & GVQ & 2 / 6 & 25 & 256 / 1024 & {0.40 / 1.50} \\
AudioDec~\cite{audiodec} & VQ-VAE (\textit{C})+GAN& 48 & RVQ & 8 & 160 & 1024 & 12.80 \\
HiFi-Codec\cite{yang2023hifi} & VQ-GAN (\textit{C})& 16 / 24 & GRVQ & 4 & 50-100 & 1024 & 2.00-4.00 \\
DAC~\cite{kumar2024high} & VQ-GAN (\textit{C})& 44.1 & RVQ & 9 & 86 & 1024 & 7.74 \\
LaDiffCodec~\cite{yang2024generative} & Latent diffusion & 16 & RVQ & 3 / 6 & 50 & 1024 & 1.50 / 3.00 \\
{FreqCodec}~\cite{du2024funcodec} & VQ-GAN (\textit{C})& 16 & RVQ & max 32 & 50 & 1024 & max 16.00 \\
TiCodec~\cite{ticodec} & VQ-GAN (\textit{C})& 24 & RVQ, GVQ & 1-4 & 75 & 1024 & 0.75-3.00  \\
APCodec~\cite{APCodec} & VQ-GAN (\textit{C})& 48 & RVQ & 4 & 150 &1024 & 6.00 \\
SRCodec~\cite{zheng2024srcodec} & VQ-GAN (\textit{C}) & 16 & GRVQ   & 2-8 & 50 & 512+1024 & 0.95-3.80\\
SQ-Codec~\cite{yang24l_interspeech} & VQ-GAN (\textit{C})& 16 & FSQ & 32 & 50 &19 & 8.00 \\
Single-Codec~\cite{singlecodec} & VQ-GAN (\textit{T+C})& 24 & VQ & 1 & 23 & 8192 & 0.30  \\
ESC~\cite{gu2024esc}  & VQ-GAN (\textit{U})& 16 & GVQ & max 18 & 50 & 1024 & max 9.00 \\
CoFi-Codec~\cite{guo2024speaking} & VQ-GAN (\textit{U}) & 16 & GVQ & 3 & 8.33+25+50 & 16384 & 1.17 \\
HILCodec~\cite{ahn2024hilcodec} & VQ-GAN (\textit{C})& 24 & RVQ & 2-12 & 75 & 1024 & 1.50-9.00 \\
SuperCodec~\cite{zheng2024supercodec} & VQ-GAN (\textit{C})& 16 & RVQ & 2-12 & 50 & 1024 & 1.00-6.00\\
SNAC~\cite{Siuzdak_SNAC_Multi-Scale_Neural_2024} & VQ-GAN (\textit{C})& 24 & RVQ & 3 &12+23+47 & 4096 & 0.98 \\
% dMel~\cite{bai2024dmel} & {Quantizer only} & 16 & FSQ & 80 & 80 & 16 & 25.60 \\
WavTokenizer~\cite{ji2024wavtokenizer} & VQ-GAN (\textit{C})& 24 & VQ & 1 & 40 / 75 & 4096 & 0.48 / 0.90  \\
BigCodec~\cite{xin2024bigcodec} & VQ-GAN (\textit{C})& 16 & VQ & 1 & 80 & 8192 & 1.04 \\
LFSC~\cite{casanova2024low} & VQ-GAN (\textit{C})& 22.05 &  FSQ & 8 & 21.5 & 2016 & 1.89 \\
NDVQ~\cite{niu2024ndvq} & VQ-GAN (\textit{C})& 24 &  RVQ & max 32 & 75 & 1024 & max 24.00 \\
VRVQ~\cite{chae2024variable} & VQ-GAN (\textit{C})& 44.1 & RVQ & 8 & 86 & 1024 & {0.26 + max 6.89}\\
% \textcolor{red}{SimVQ}~\cite{zhu2024addressing} & \\
TS3-Codec~\cite{wu2024ts3codectransformerbasedsimplestreaming} & VQ-GAN (\textit{T})& 16 & VQ & 1 & 40 / 50 & $2^{16}$ / $2^{17}$ & 0.64-0.85 \\
Stable-Codec~\cite{parker2024scalingtransformerslowbitratehighquality} & VQ-GAN (\textit{T})& 16 & FSQ  & 6 / 12 & 25 & 5 / 6 & 0.40 / 0.70 \\
FreeCodec~\cite{zheng2024freecodecdisentangledneuralspeech} & VQ-GAN (\textit{C+T})& 16 & VQ & 1+1 & 50+7 & 256 & 0.45  \\
\midrule
\multicolumn{6}{l}{\textbf{\textit{Mixed-objective acoustic tokens: semantic distillation}}} \\
{Siahkoohi et al.}~\cite{siahkoohi22_interspeech} & VQ-GAN (\textit{C})& 16 & RVQ & 2+1 / 2+2 / 6 & 25+50 & 64 & 0.60 / 0.90 / 1.80\\
SpeechTokenizer~\cite{zhang2024speechtokenizer} & VQ-GAN (\textit{C})& 16 & RVQ & 8 & 50 & 1024 & 4.00 \\
SemantiCodec~\cite{liu2024semanticodec} & Latent diffusion & 16 & VQ & 2 & 12.5-50 & {8192+($2^{12}$-$2^{15}$)} & 0.31-1.40 \\
LLM-Codec~\cite{yang2024uniaudio15} & VQ-GAN (\textit{C})& 16 & RVQ & 3 & 8.33+16.67+33.33 & 3248+32000+32000& 0.85 \\
X-Codec~\cite{ye2024codec} & VQ-GAN (\textit{C})& 16 & RVQ & max 8 & 50 & 1024 & max 4.00 \\
SoCodec~\cite{guo2024socodec} & VQ-GAN (\textit{C})& 16 & GVQ & 1 / 4 / 8 & 25 / 8.3 / 4.2 & 16384 & 0.35 / 0.47 \\
Mimi~\cite{kyutai2024moshi} & VQ-GAN (\textit{C+T})& 24 & RVQ & 8 & 12.5 & 2048 & 1.10 \\
X-Codec 2.0~\cite{ye2025llasa} & VQ-GAN (\textit{C+T}) & 16 & FSQ & 8 & 50 & 4 & 0.80 \\
\midrule
\multicolumn{6}{l}{\textbf{\textit{Mixed-objective acoustic tokens: disentanglement}}} \\
SSVC~\cite{SSVC} & VQ-GAN (\textit{C})& 24 & RVQ & 4 & 50 & 512 & 1.80 \\
PromptCodec~\cite{pan2024promptcodec} & VQ-GAN (\textit{C})& 24 & GRVQ & 1-4 & 75 & 1024 & 0.75-3.00 \\
FACodec~\cite{facodec} & VQ-GAN (\textit{C})& 16 & RVQ & 1+2+3 & 80 & 1024 & 4.80 \\
LSCodec~\cite{guo2024lscodec} & VQ-VAE (\textit{C+T})+GAN& 24 &  VQ & 1 & 25 / 50 & 1024 / 300& 0.25 / 0.45 \\
SD-Codec~\cite{bie2024learning} & VQ-GAN (\textit{C})& 16 &  RVQ & 12 & 50 & 1024 & 6.00 \\
\bottomrule
\end{tabular}
}
\end{table*}

Acoustic tokens, also known as \textit{speech codecs}, refer to the discrete representations optimized mainly for signal compression and reconstruction.
The audio codec technology arises long ago.
Traditional codecs, including 
% AAC~\cite{bosi1997iso}, 
MP3~\cite{rfc5219}, Opus~\cite{Valin2012DefinitionOT} and EVS~\cite{dietz2015overview}, typically take advantage of signal processing algorithms to improve quality and lower the bitrate.

In the deep learning era, numerous codec models based on neural networks have emerged.
% These codec models typically have a encoder-decoder framework that involves a quantization module in the middle.
% These models typically train an encoder that compresses speech signals, and a decoder that recovers speech signals, with a quantizer between the two.
These models typically consist of an encoder that compresses speech signals and a decoder that reconstructs the speech signals, with a quantizer situated between the two.
The quantizer is also parameterized and jointly trained with the whole network in an end-to-end manner. 
% As the purpose of acoustic tokens is signal reconstruction, current acoustic tokens models almost all rely on generative adversarial network (GAN) training criterion, i.e. the encoder-decoder generator tries to fool a set of discriminators by making the reconstructed signals as close to the original as possible.
% Those VQ-VAE model with GAN training criterion is typically referred to as VQ-GAN~\cite{esser2021taming}.
The codebook indices produced by the quantizer are referred to as acoustic tokens.
To improve the representation ability of discrete VQ spaces and thus obtain better codec performance, RVQ, GVQ, GRVQ and FSQ tricks are commonly applied in the quantization module.
% FSQ has also been integrated in acoustic tokens.

We list the VQ method, number of quantizers $Q$, frame rate $F$, vocabulary size $V$ for each quantizer, and the resulting bitrate of existing neural acoustic speech tokens in Table.\ref{tab:acoustic-metadata}.

\subsection{Model Architectures}
\label{sec:acoustic-arch}

\begin{figure}
    \centering
    % \includegraphics[width=0.8\linewidth]{figs/acoustic1.png}
    % \includegraphics[width=0.8\linewidth]{figs/acoustic2.png}
    \includegraphics[width=0.9\linewidth]{figs/acoustic.png}
    \caption{Neural architectures of acoustic tokens.
    % Upper: \textbf{VQ-GAN} type where the quantization module is placed between an encoder and a decoder; Bottom: \textbf{latent diffusion} type where quantized tokens condition the diffusion process towards a latent space learned by an autoencoder.
    Note that inputs and outputs can be waveforms, frequency-domain features or even SSL features depending on purpose and design.}
    \label{fig:acoustic-paradigms}
\end{figure}


Although acoustic codec models differ from one to one regarding their purposes, most of them share a similar encoder-quantizer-decoder framework.
With audio clip $\bm x$ that can either be time-domain sampling points, frequency-domain features or even other machine learning features, an encoder $f_\theta(\cdot)$ transforms it to $f_\theta(\bm x)$ in a continuous latent vector space. 
% For waveform inputs, $f_\theta(\cdot)$ will also downsample
The encoder $f_\theta(\cdot)$ will usually perform downsampling to reduce the temporal length of the input signals, especially for waveform inputs.
A VQ module $q_\phi(\cdot)$ discretizes $f_\theta(\bm x)$ into tokens and corresponding codebook vectors $\bm c$.
A decoder $g_\psi(\cdot)$ then uses $\bm c$ to reconstruct $\hat {\bm x}$, and a certain distance metric of $d(\bm x, \hat{\bm x})$ is usually optimized.
There are two major paradigms for designing the encoder, decoder, and quantizers, which can be summarized as diagrams in Fig.\ref{fig:acoustic-paradigms}.
% \textcolor{red}{Maybe we can merge U-Net into VQ-GAN, since they have nothing different in essence.}

\begin{figure}
    \centering
    % \includegraphics[width=0.8\linewidth]{figs/CNN.png}
    % \includegraphics[width=0.8\linewidth]{figs/transformer.png}
    % \includegraphics[width=0.7\linewidth]{figs/unet.png}
    \includegraphics[width=0.9\linewidth]{figs/acoustic-archs.png}
    \caption{Major generator (VQ-VAE) architectures of VQ-GAN-based acoustic tokens. 
    % Upper: \textbf{CNN-based}; Middle: \textbf{Transformer-based}; Bottom: \textbf{U-Net-based}. 
    ``Q.'' and ``Trans'' are short for quantizer and Transformer, respectively.}
    \label{fig:generator}
    \vspace{-0.2in}
\end{figure}

\subsubsection{VQ-GAN}
VQ-GAN~\cite{esser2021taming} is a very commonly adopted framework of acoustic tokens that trains a VQ-VAE with GAN objectives. 
Besides the original reconstruction and VQ objectives in a VQ-VAE, VQ-GAN uses discriminators $d_\xi(\bm x, \hat{\bm x})$ to distinguish real and reconstructed data that adversarially train the generator network composed of $f_\theta,q_\phi$, and $g_\psi$. In acoustic tokens, there are usually multiple discriminators, e.g. multi-resolution and multi-scale STFT discriminators from the neural vocoder researches~\cite{kumar2019melgan,jang21_interspeech}.
The generator architecture of VQ-GAN-based acoustic tokens has multiple choices, with the three most representative ones visualized in Fig.\ref{fig:generator}: CNN-based, Transformer-based, and U-Net-based.

The CNN-based generator is the most widely used architecture so far in acoustic tokens.
SoundStream~\cite{zeghidour2021soundstream} and EnCodec~\cite{encodec} are two famous early neural acoustic tokens that operate in an end-to-end VQ-GAN manner.
% SoundStream is also the basis for Lyra V2 codec\footnote{\url{https://github.com/google/lyra}}.
They receive time-domain waveforms as inputs and directly reconstruct waveforms.
Their encoder and decoder have a mirrored architecture to perform down and up-samplings.
In SoundStream, the encoder and decoder are purely constructed by convolutional neural networks (CNNs) while EnCodec augments them with an LSTM.
The CNN encoder down-samples the waveform to a high-dimensional embedding sequence, whose frame rate is determined by the sampling rate, CNN kernel sizes and strides at a fixed ratio.
The continuous embeddings are passed to an RVQ quantizer, and the quantized vectors are summed before being transformed to the waveform domain by the CNN decoder.
% Multi-resolution and multi-scale STFT discriminators are applied to distinguish between real and reconstructed speech.
The training criteria include reconstruction loss (in the time and frequency domain), adversarial loss, feature matching loss, and quantization losses for RVQ layers.
To allow for a flexible choice of bitrates, structured dropout is adopted where the number of codebooks in the RVQ module can be randomly chosen~\cite{zeghidour2021soundstream}, such that only a portion of quantizers in front are activated during training.
The acoustic tokens can consequently reside in variable bitrates depending on the chosen number of RVQ quantizers.
The inputs and outputs of the codec model can also be frequency-domain features like magnitude and phase spectra for reducing computation burden~\cite{du2024funcodec}.
There, the convolution kernels are typically 2D instead of 1D in the time-domain codecs.
% \textcolor{red}{Shall training losses be expressed in detail?}

Later, Transformers~\cite{transformer} have been adopted, e.g. in Single-Codec~\cite{singlecodec} and Mimi~\cite{kyutai2024moshi}.
They can be directly applied to frequency-domain inputs and outputs.
When operating on waveform-domain inputs or outputs, a CNN~\cite{kyutai2024moshi} or patchifying~\cite{wu2024ts3codectransformerbasedsimplestreaming,parker2024scalingtransformerslowbitratehighquality} operation is usually added before or after the Transformer blocks.
In Mimi, a shallow Transformer layer is added after the CNN-based encoder, and vice versa in its decoder.
Recently, some propose to use purely Transformer-based backbone and discard the CNN blocks, e.g. TS3-Codec~\cite{wu2024ts3codectransformerbasedsimplestreaming}.
As Transformers demonstrate superior modeling ability and scaling property, these works prove to outperform CNN-based codecs either with less computation~\cite{wu2024ts3codectransformerbasedsimplestreaming} or larger scale~\cite{parker2024scalingtransformerslowbitratehighquality}.
However, to ensure stream-ability, an attention mask should be employed~\cite{kyutai2024moshi}.
The encoder and decoder can also be designed to be different. 
For example, Single-Codec~\cite{singlecodec} uses Conformer~\cite{conformer} encoder and CNN decoder, while LSCodec~\cite{guo2024lscodec} uses the reverse configuration.

% While the most acoustic tokens contain only one quantizer (can be RVQ or GVQ though)
Though RVQ or GVQ is usually applied, most acoustic tokens contain only one quantization module as a whole.
However, there are also U-Net-based codecs where multiple quantizers are employed, e.g. CoFi-Codec~\cite{guo2024speaking} and ESC~\cite{gu2024esc}.
Each sub-encoder or decoder in the U-Net can be a CNN or Transformer.
This offers a more flexible control of the resolution of each VQ stream (Section \ref{sec:multi-resolution}).

It is also noteworthy that training a separate vocoder on top of existing acoustic tokens may result in improved audio quality than the original decoded outputs, since reconstructing waveform alone may be simpler than optimizing VQ representation and reconstruction at the same time.
This is exemplarily verified in AudioDec~\cite{audiodec}, MBD~\cite{san2023discrete} and Vocos~\cite{siuzdak2024vocos}.
Therefore, some acoustic tokens directly simplify the VQ-GAN training objective back to the original VQ-VAE, where the discrete tokens are obtained first by a simple reconstruction loss, and a vocoder is trained as an additional stage, like AudioDec~\cite{audiodec} and LSCodec~\cite{guo2024lscodec}.
These works are denoted as ``VQ-VAE+GAN'' in Table \ref{tab:acoustic-metadata}.
% \textcolor{red}{TODO: maybe add some explanation? Is this because vocoder is larger than decoder?}

\subsubsection{Latent diffusion} 
Different from VQ-GAN which uses GAN to generate waveforms or frequency features, some codecs also use latent diffusion~\cite{ho2020denoising,song2021scorebased,rombach2022high} as an alternative.
These codecs use discretized tokens as a condition to generate some latent acoustic space, e.g. from a pretrained continuous speech autoencoder.
Since diffusion models are strong generative models, acoustic tokens of this type does not need discriminators and adversarial training like VQ-GAN.
For instance, LaDiffCodec~\cite{yang2024generative} uses EnCodec tokens to condition the diffusion process from Gaussian noise to the latent space in a pretrained and frozen waveform autoencoder.
This is to bridge the gap of reconstruction quality between discrete and continuous representations and improve the codec performance compared to the original acoustic tokens.
Inference efficiency is a major concern of these models unless specifically optimized in limited sampling steps.
% By transforming the discrete codes from EnCodec to a properly-shaped continuous latent space, it improves the codec performance compare to original the EnCodec model.
% SemantiCodec~\cite{liu2024semanticodec} also belongs to this type, which will be detailed in Section \ref{sec:acoustic-distillation}.
% \textcolor{red}{Shall training losses be expressed in detail?}

% \paragraph{U-Net}\textcolor{red}{A separate paragraph or an affiliation in VQ-GAN?}

\vspace{-0.1in}
\subsection{General-Purpose Acoustic Tokens}

\label{sec:acoustic-general}

\subsubsection{Motivation}

In this section, we describe the most common type of neural acoustic tokens (speech codecs) that are designed only with the objective of speech signal reconstruction.
Those acoustic tokens are optimized towards better signal or perceptual quality under bitrates as low as possible. 

\subsubsection{Approaches}

% \textcolor{red}{Should we organize this subsection using architectures in Fig 1?}

% \textcolor{red}{HILCodec?}

\paragraph{Advanced VQ methods and model architectures}

Based on SoundStream and EnCodec, more codecs with advanced VQ methods, network structure, or optimization strategies have been researched with depth.
% HiFi-Codec~\cite{yang2023hifi} applies GRVQ on codecs to reduce the number of codebooks.
% SRCodec~\cite{zheng2024srcodec} proposes a dual attention mechanism and a split residual VQ strategy, which can also be regarded as GRVQ with interactions between groups.
% , and also launches the AcademiCodec\footnote{\url{https://github.com/yangdongchao/AcademiCodec}} project to facilitate codec research.
As an example, DAC~\cite{kumar2024high} achieves remarkable reconstruction quality by adding periodic inductive bias, better discriminators, modified loss functions, and a better VQ mechanism from ViT-VQGAN~\cite{yu2022vectorquantized} to improve codebook usage. 
Specifically, it performs L2-normed code lookup in a low-dimensional space (e.g. 8 or 32) instead of a high-dimensional space like 1024.
% Its VQ tricks are reported to improve code usage.
Other architectural improvements include using frequency-domain inputs~\cite{APCodec,ai24b_interspeech,singlecodec}, variance-constrained residual blocks~\cite{ahn2024hilcodec}, multi-filter bank discriminator~\cite{ahn2024hilcodec}, selective down-sampling back-projection~\cite{zheng2024supercodec}, etc.
% APCodec~\cite{APCodec} uses amplitude and phase spectra as inputs and outputs of the VQ-GAN codec model.
% Its encoder and decoder are based on an improved ConvNeXt v2 network~\cite{woo2023convnext}, and it achieves fast and low-latency compression for 48kHz audio.
% Based on APCodec, \cite{ai24b_interspeech} later reduces the necessary bitrate to 1kbps for high sampling rate scenarios by introducing additional bandwidth reduction and recovery modules before and after VQ-GAN.
% HILCodec~\cite{ahn2024hilcodec} proposes spectrogram blocks, variance-constrained residual blocks and a multi-filter bank discriminator to achieve high fidelity and lightweight streaming codec.
% SuperCodec~\cite{zheng2024supercodec} improves the traditional CNN blocks in encoder and decoder into a selective down-sampling back-projection network for better performance.

% ~\cite{kyutai2024moshi} to improve subjective perception: 1) introducing causal Transformers to both its encoder and decoder; 2) not applying VQ with a certain probability; 3) pure adversarial training without reconstruction losses.
% Meanwhile, TF-Codec introduces learnable frequency input compression and bitrate-controllable quantization to optimize with varying bitrates.

Several training tricks are explored, such as not applying VQ with a certain probability and pure adversarial training proposed in Moshi~\cite{kyutai2024moshi}.
Also, the training of neural speech codecs does not need to be end-to-end, i.e. the learning of VQ representations and signal reconstruction can be separated.
\cite{audiodec,du2024apcodec+} adopt a two-stage training process that introduces adversarial losses and an additional vocoder after training only with metric losses, to achieve improved quality.
Additional training criteria around the VQ module are proposed for better VQ utilization, such as delicate code-vector replacement strategy, codebook balancing loss, and similarity loss between consecutive RVQ layers proposed in ERVQ~\cite{zheng2024ervq}.
% ERVQ~\cite{zheng2024ervq} proposes a more delicate code-vector replacement strategy and a codebook balancing loss to enhance the VQ usage.
% It also applies a similarity loss after consecutive RVQ layers to encourage each RVQ layer to focus on different speech features.

% It achieves low latency while improving quality.
% APCodec+~\cite{du2024apcodec+} introduces this two-stage training process into APCodec, and declares that using adversarial loss throughout the entire process yields better performance.

% \textcolor{red}{SRCodec}.

% Although the aforementioned works all use regular GVQ or RVQ quantizers, 
Other VQ methods besides GVQ or RVQ also exist in speech codecs.
NDVQ~\cite{niu2024ndvq} improves the capacity of RVQ space by changing codebook {vectors} to parameterized Gaussian {distributions}.
% Instead of quantizing the continuous input to the closest codebook entry in each RVQ layer, NDVQ performs quantization by choosing the mean and variance with the greatest probability density.
% A sample is then drawn from the chosen Gaussian distribution as the VQ output, with a reparameterization technique.
FSQ has also been introduced to several speech codecs, like SQ-Codec~\cite{yang24l_interspeech} where scalar rounding is applied to each of its 32-dimensional latent space.
Stable-Codec~\cite{parker2024scalingtransformerslowbitratehighquality} adopts FSQ in a Transformer-based architecture, exhibiting strong scalability to large model sizes up to 950M parameter count.
It also explores a flexible post-training quantization level adjustment technique and residual FSQ strategy.
% dMel~\cite{bai2024dmel} directly quantizes mel-filterbanks per dimension with evenly-paced boundaries between the minimum and maximum of filterbank values.
% This is similar to FSQ but is parameter-free.

Note that most acoustic tokens require multiple quantizers, but \textbf{single-codebook} codecs have also been explored.
Single-Codec~\cite{singlecodec} designs an encoder consisting of Conformer and bidirectional LSTM to better compress mel spectrogram inputs.
% It accomplishes codec using only a \textbf{single codebook}.
WavTokenizer~\cite{ji2024wavtokenizer} and BigCodec~\cite{xin2024bigcodec} further explores single-codebook codec modeling with better network designs or larger parameter count.
TS3-Codec~\cite{wu2024ts3codectransformerbasedsimplestreaming} adopts a fully Transformer design that leads to a better single-codebook codec with fewer computation overhead.
LSCodec~\cite{guo2024lscodec} also achieves single-codebook coding with speaker disentanglement (Section \ref{sec:acoustic-disen}).
These single-codebook codecs with remarkably low bitrates offer great benefit to downstream speech generation models on simplicity and efficiency.

\paragraph{Temporal redundancy reduction}

Instead of capturing all the information through VQ layers like the previously mentioned codecs, some researchers have attempted to reduce the redundant bitrate of time-varying VQ codes.
One reasonable method is to encode the global information in speech, e.g. speaker timbre and channel effects, by a global encoder instead of the time-varying codes.
% The global information in speech includes speaker identity, channel effect and so on, and it does not need to be repetitively encoded by time-varying discrete tokens.
% The global information in speech, which includes speaker identity, channel effects, and other attributes, does not need to be repetitively encoded by time-varying discrete tokens. 
Disen-TF-Codec~\cite{jiang2023disentangled} is the first to explore VQ-GAN codec models with an additional global encoder that aids the codec decoder. 
In Disen-TF-Codec, the global features are designed to be sequential to adapt to speaker changes during transmission.
In TiCodec~\cite{ticodec}, the global tokens are time-invariant and vector-quantized instead.
They are extracted from different segments of an utterance in conjunction with time-varying tokens.
% By this design, TiCodec tries to minimize the global information in its RVQ time-varying codes.
Similar global encoders are also seen in \cite{guo2024socodec,guo2024speaking,singlecodec}.
% In \cite{guo2024socodec,guo2024speaking}, an ECAPA-TDNN~\cite{desplanques20_interspeech} reference encoder is employed.
% The introduction of a global encoder also facilitates the development of single-codebook codecs, such as Single-Codec~\cite{singlecodec}.
FreeCodec~\cite{zheng2024freecodecdisentangledneuralspeech} further incorporates a prosody encoder~\cite{ren2022prosospeech} that compresses the low-frequency range of mel spectrograms into a low frame rate VQ sequence to assist in reconstruction.

% This shows potential for compression acoustic information to a \textbf{single codebook} without repetitive RVQ process.

Another typical example of temporal redundancy reduction is predictive coding, as seen in TF-Codec~\cite{jiang2023latent}.
This approach captures temporal-varying information in the latent space by autoregressive prediction, which significantly reduces redundancy and entropy in the residual part for  quantization.
LMCodec~\cite{LMCodec} employs autoregressive prediction from coarse codes (first RVQ levels) to fine codes (last RVQ levels)~\cite{borsos2023audiolm}, enabling the transmission of fewer codes.

\paragraph{Multi-resolution and variable-bitrate coding}
\label{sec:multi-resolution}

% Instead of uni-resolution tokens where all the quantizers share the same temporal frequency of typically 25-86Hz, it is reasonable to design multi-resolution codecs since there are simultaneously fast and slow information streams in speech.
Rather than relying solely on uni-resolution tokens, where all quantizers share the same temporal frequency, it is reasonable to design multi-resolution codecs, because speech contains both fast and slow information streams.
For instance, many vowels exhibit slowly changing characteristics, while events such as explosive consonants and background noises require fine-grained modeling. Therefore, incorporating multiple temporal resolutions in codecs is likely to reduce the necessary bitrate.
% Therefore, incorporating multiple temporal resolutions in codecs is likely to reduce the necessary bitrate.
% Lots of vowels exhibit slowly changing characteristics, while events like explosive consonants and background noises may require finer-grained modeling.
% Hence, designing multiple temporal resolutions in codec is likely to decrease the necessary bitrate.

% Multi-resolution acoustic tokens have been investigated in contrast to the previous uni-resolution tokens.
% In multi-resolution acoustic tokens, different token streams have different frame rates for modeling spoken information from coarse to fine.

% Siahkoohi et al.~\cite{siahkoohi22_interspeech} 
SNAC~\cite{Siuzdak_SNAC_Multi-Scale_Neural_2024} is a notable multi-resolution acoustic token.
It follows the DAC~\cite{kumar2024high} architecture, but in each RVQ layer, residuals are downsampled before codebook look-up and upsampled afterward.
This enables SNAC to have three RVQ streams at a frame rate of 12, 23, 47Hz respectively.
% For example, with a CNN downsampling factor of 512 on 24kHz waveforms, the first quantizer further downsamples the sequence by a factor of 4, and then upsamples the 12Hz quantized vectors by the same factor to compute quantization residuals.
% The downsampling factors for the second and third quantizers are 2 and 1, respectively.
% This design enables SNAC to outperform codecs with uni-resolution RVQ, especially under low bitrates.
Similarly, CoFi-Codec~\cite{guo2024speaking} achieves multi-resolution coding by GVQ quantizers within its U-Net-based architecture.
LLM-Codec~\cite{yang2024uniaudio15} also adopts this idea to achieve very low frame rates with semantic distillation (Section \ref{sec:acoustic-distillation}).
% , CoFi-Codec~\cite{guo2024speaking} uses a U-Net architecture, where each encoder is a CNN with a specific downsampling rate.
% The decoders follow a mirrored procedure, and quantizers are placed between each encoder-decoder pair with at a specific resolution.
% This results in a multi-resolution representation, where, at each scale, GVQ is applied to the residual between the encoder and decoder hidden embeddings.
% In contrast, ESC~\cite{gu2024esc} changes the frequency resolution in each layer rather than the time resolution.

% Apart from multiple temporal resolutions, it is also valuable to notice the different information intensities in different speech frames.
In addition to multiple temporal resolutions, it is also feasible to consider the varying information intensities across different speech frames. 
% Some frames carry critical information, while others may be less informative (e.g., silences). 
This observation motivates the design of codecs to allocate different numbers of quantizers for different speech frames.
% Some frames carry important information while others may be obscure (e.g. silences).
% This inspires codec to allocate different numbers of quantizers for different speech frames.
As an example, VRVQ~\cite{chae2024variable} automatically selects the number of RVQ quantizers per frame by a predictor that is jointly trained with the whole network.
% In VRVQ, a predictor receives the encoder's hidden embeddings and outputs an importance map for each frame ranging from 0 to 1.
% This importance map determines the number of quantizers $Q$ for each frame and masks the quantizers beyond the first $Q$ quantizers.
% Since this masking process is not differentiable, surrogate functions are introduced to train the importance map predictor.

\subsubsection{Challenges}
Despite the emergence of single-codebook and low-bitrate codecs~\cite{singlecodec,ji2024wavtokenizer,xin2024bigcodec,guo2024lscodec}, achieving ideal reconstruction quality with a highly limited VQ space remains a challenging problem. 
% Besides, as acoustic tokens try to encode all necessary information for signal recovery, they may be redundant and too complex for downstream modeling.
Additionally, as acoustic tokens aim to encode all necessary information for signal recovery, they may become redundant and overly complex for downstream modeling.
While scaling up the model size or switching to non-causal networks has been shown to improve performance~\cite{singlecodec,xin2024bigcodec,parker2024scalingtransformerslowbitratehighquality}, these approaches may also compromise streamability or efficiency.
Furthermore, simply introducing global encoders like \cite{jiang2023disentangled,ticodec,guo2024speaking} does not guarantee disentanglement (Section \ref{sec:acoustic-disen}) and may still result in redundancy within the time-varying codes.

\subsection{Acoustic Tokens with Semantic Distillation}
\label{sec:acoustic-distillation}

\begin{figure}
    \centering
    \includegraphics[width=0.8\linewidth]{figs/acoustic-distill1.png}
    \includegraphics[width=0.8\linewidth]{figs/acoustic-distill2.png}
    \includegraphics[width=0.8\linewidth]{figs/acoustic-distill3.png}
    \caption{Different semantic distillation methods in acoustic tokens. Gray color indicates frozen during training.}
    \label{fig:acoustic-distill}
    \vspace{-0.2in}
\end{figure}

\subsubsection{Motivation}
Acoustic tokens are a convenient choice for spoken language models, as they can be directly converted back to waveforms without the need for extra vocoders.
However, if reconstruction is the sole objective of these tokens, their representation space may become overly complex and overly focused on acoustic details, in contrast to natural language tokens that primarily carry semantic information.
A natural improvement is to incorporate speech semantic features either from speech self-supervised learning (SSL) models, supervised models, or even text transcriptions.
Since speech SSL models aim to capture high-level phonetic or semantic information without external supervision~\cite{mohamed2022self}, integrating SSL features does not impose additional data requirements for injecting semantic information into the training process. 
Acoustic tokens with criteria beyond reconstruction are sometimes referred to as having a ``mixed objective''~\cite{cui2024recent}.
Given that the primary purpose of these models remains acoustic reconstruction in these models, we continue to refer to them as acoustic tokens.
The process of introducing semantic information into acoustic tokens is termed \textbf{semantic distillation}, with approaches summarized in Fig. \ref{fig:acoustic-distill}.

\subsubsection{Approaches}

\paragraph{Semantic feature guidance} 
The earliest effort in semantic distillation is to guide some RVQ layers in acoustic tokens towards semantic features, which are typically SSL features. 
Since information in RVQ naturally follows a coarse-to-fine order, guiding early RVQ layers towards semantic-oriented features helps establish and reinforce a semantic-to-acoustic information hierarchy.
For example, SpeechTokenizer~\cite{zhang2024speechtokenizer} uses a HuBERT~\cite{hsu2021hubert} SSL model to guide the first RVQ layer in EnCodec.
This ensures that the first RVQ layer contains more semantic information, thereby pushing acoustic details to the subsequent RVQ layers. 
This distillation is implemented either by regressing the first RVQ output to continuous HuBERT embeddings or by classifying it into discrete HuBERT tokens.
LLM-Codec alternatively uses Whisper~\cite{whisper} and T5~\cite{raffel2020exploring} as semantic teachers.
Mimi~\cite{kyutai2024moshi} uses a WavLM~\cite{chen2022wavlm} teacher and applies distillation to a specialized VQ module rather than the first RVQ layer.
% uses a WavLM~\cite{chen2022wavlm} model as a semantic teacher and designs a specialized VQ module for distillation, rather than using the first RVQ layer. 
% It claims to achieve a better semantic-acoustic trade-off compared to forcing acoustic information into the residual of the semantic quantizer.
Since SSL feature guidance occurs only during the training stage, it does not incur additional inference costs.
It has been reported that TTS language models trained with such acoustic tokens exhibit better robustness than those with unguided tokens~\cite{zhang2024speechtokenizer}.

\paragraph{Fixed semantic codebook} A more direct approach to achieve semantic distillation is to integrate semantic knowledge into the codebook of quantizers. 
% The encoder is then tasked with transforming the original speech into this semantic codebook space, while the decoder must learn to recover acoustics from this semantic space and the residuals.
This forces the quantization space itself to be more semantic-related.
This method is proposed in LLM-Codec~\cite{yang2024uniaudio15} where all three RVQ codebooks are initiated from the token embedding module of LLaMa-2~\cite{touvron2023llama2} and remain frozen during training.
% where pretrained automatic speech recognition (ASR) model Whisper~\cite{whisper}, text language model T5~\cite{raffel2020exploring}, and the LLM LLaMa-2~\cite{touvron2023llama,touvron2023llama2} are employed as semantic teachers.
% LLM-Codec comprises three RVQ layers where all codebooks are initiated from the token embedding module of LLaMa-2 and remain frozen during training.
% Specifically, the first RVQ codebook is constructed by selecting common words and average their corresponding sub-word embeddings from LLaMa-2.
% The rest two codebooks directly utilize the entire vocabulary space of LLaMa-2.
% Input vector sequences are downsampled at different rates before entering the first and second quantizers.
% The outputs of T5 and Whisper encoders are used to guide the first and second RVQ layers, respectively.
This approach not only reduces the bitrate of the codec but also significantly enhances the semantic representation ability of LLM-Codec.

\paragraph{Semantic features as inputs or outputs} 
Semantic features can also be compressed together with the acoustic features. 
This requires the encoder and quantizer to construct a shared acoustic and semantic space that balances the two information sources. 
The first attempt in this direction is made in \cite{siahkoohi22_interspeech} where Conformer representations from a pretrained wav2vec 2.0~\cite{baevski2020wav2vec} are combined with CNN encoder outputs for quantization.
SemantiCodec~\cite{liu2024semanticodec} quantizes AudioMAE~\cite{huang2022masked} SSL features
% \footnote{In fact, a stack of discretized and continuous AudioMAE features.} 
without relying on acoustic inputs. 
The quantized SSL features then serve as a condition for acoustic reconstruction using latent diffusion, which resembles a vocoder that transforms semantic inputs into acoustic outputs.
% SoCodec~\cite{guo2024socodec} also directly quantizes HuBERT features and reconstruct, but incorporates a global acoustic condition to aid reconstruction.
% With a downsampling semantic encoder, it remarkably explores a frame shift up to 240ms.
Providing aligned phoneme sequences instead of SSL features to the quantizer has also shown benefits on reducing bitrates~\cite{du2024funcodec}.
% Additionally, it has also been reported to reduce bitrate when aligned phoneme sequences are added to the encoder output before RVQ~\cite{du2024funcodec}.
% The Mimi tokenizer, proposed in the speech-to-speech LLM Moshi~\cite{kyutai2024moshi}, relies on WavLM~\cite{chen2022wavlm} for semantic distillation. It also regresses the output of a VQ layer to WavLM embeddings, but separates this VQ layer with the rest RVQ layers, differently with SpeechTokenizer.

Moreover, semantic features can also serve as outputs, thereby reinforcing the constraint that semantic information be compressed into the discrete latent space.
% For instance, SoCodec quantizes HuBERT 
For instance, \cite{guo2024socodec,ye2024codec} combine hidden HuBERT embeddings with acoustic features before RVQ and jointly optimizes acoustic and semantic reconstruction objectives.
X-Codec 2.0~\cite{ye2025llasa} improves it by using w2v-BERT 2.0~\cite{barrault2023seamless} and FSQ.
% Then, these tokens can compress semantic features directly.

\subsubsection{Challenges}
Guiding part of the RVQ layers towards semantic features does not guarantee that acoustic information is encoded in the remaining layers, as shown by the degraded VC performance in SpeechTokenizer~\cite{zhang2024speechtokenizer}.
It may impose a greater challenge for the VQ layer to encode both acoustic and semantic information if semantic features serve as inputs as well.
% Fixing a semantic codebook could also negatively impact the acoustic reconstruction ability, since the VQ representation space is too restricted.
Additionally, fixing a semantic codebook could negatively impact acoustic reconstruction ability, as the VQ representation space becomes overly restricted.

\subsection{Acoustic Tokens with Disentanglement}
\label{sec:acoustic-disen}
\subsubsection{Motivation}
Another line of mixed-objective acoustic tokens is {disentanglement}.
A prominent research direction is the disentanglement of speaker timbre information, as this is a global trait among all the speech information aspects.
% It is redundant to encode speaker information into every token timestep, hence information in acoustic tokens will be more compact and the necessary bitrate will be lower if the global speaker timbre is removed.
Encoding speaker information into every token timestep is redundant; thus, removing the global speaker timbre can make the information in acoustic tokens more compact and reduce the necessary bitrate. 
Speaker-decoupled speech tokens can alleviate the modeling burden for downstream tasks. For example, a TTS model using these tokens can achieve independent control over prosody and speaker identity.
% A speaker-decoupled speech token will ease the modeling burden for downstream tasks.
% For example, a TTS model with those tokens can achieve prosody and speaker control independently.
The disentanglement of speaker timbre also enables an acoustic token to perform voice conversion (VC), as timbre from the target speaker can be easily combined with the speaker-agnostic content tokens from the source speech.

Note that in Section \ref{sec:acoustic}, it is mentioned that some codecs introduce a global encoder to reduce the necessary bitrate of time-variant tokens~\cite{jiang2023disentangled,ticodec,singlecodec,zheng2024freecodecdisentangledneuralspeech}.
They have already demonstrated some ability to decouple global speaker timbre and local contents, albeit in an \textbf{implicit} manner through the natural information bottleneck from VQ.
In this section, we elaborate on \textbf{explicit} methods, which involve specialized training techniques and criteria to achieve disentanglement.

\subsubsection{Approaches}
\paragraph{Gradient reversal layer (GRL)} The GRL technique~\cite{drl} is commonly used for disentanglement. Suppose speaker information needs to be disentangled, and a classifier (or speaker verifier, etc.) $s_\mu(\cdot)$ receives some latent feature $\bm h$ from the acoustic token to perform speaker discriminative tasks. 
GRL operates by negating the gradient sign before $s_\mu(\cdot)$, thereby forcing $\bm h$ to fool the speaker classifier while the classifier itself improves, similar to adversarial training.

SSVC~\cite{SSVC} is one of the pioneering efforts in this direction.
% , which is the basis for BASE-TTS~\cite{lajszczak2024base}.
SSVC attempts to decouple content and speaker representations from WavLM features.
The content branch is quantized via RVQ, and the speaker branch is trained using a contrastive loss to produce speaker embeddings.
Disentanglement is enforced by a GRL between the speaker embeddings produced from the speaker branch and the content representations.
% SSVC designs two coupled regressors from WavLM: a speaker regressor and a content regressor.
% These regressors are essentially attention modules on every WavLM layer.
% The speaker regressor is used to train a speaker extractor by contrastive loss to produce discriminative speaker embeddings.
% The output from the content regressor is quantized by RVQ before being combined with speaker embeddings and reconstructed into waveforms.
% SSVC trains a VQ-VAE on WavLM embeddings with RVQ, where the encoder is simply an attention module on every WavLM layer, and the decoder is a BigVGAN vocoder~\cite{lee2023bigvgan}.
% It jointly trains a speaker extractor based on the WavLM embeddings using contrastive loss, similar to GE2E~\cite{wan2018generalized} in speaker verification.
% The speaker extractor outputs discriminative speaker embeddings, which is fed into the BiVGAN vocoder in its VQ-VAE.
% Disentanglement is enforced by a GRL on a cosine distance loss between the speaker extractor outputs from the speaker regressor and content regressor.
Similarly, PromptCodec minimizes an SSIM loss~\cite{wang2004image} between content and speaker representations, with the help of a pretrained speaker verification model.

Such GRL technique is not limited to disentangling speaker timbre alone.
FACodec~\cite{facodec} employs supervised decoupling to factorize speech into speaker timbre, content, prosody, and acoustic detail information.
The timbre extractor in FACodec is optimized via a speaker classification loss.
% For prosody, content, and detail aspects, different RVQ modules are applied respectively before supervised decoupling.
For the other components -- prosody, content, and acoustic detail -- separate RVQ modules are applied prior to the supervised decoupling process.
For each component, some supervision signal with the desired information is applied, and GRL is employed to other non-related information components.
% the normalized F0. In the prosody branch, normalized F0 is predicted, and GRL is applied to the frame-aligned phonemes. Meanwhile, in the acoustic detail branch, GRL is performed using both phonemes and F0.
% The decoder of FACodec receives all four information branches and re-combines them to reconstruct speech.
These three quantized features are then combined before applying GRL with the speaker information. 
Finally, the decoder integrates all four information branches to reconstruct the speech signal.

\paragraph{Perturbation}
For speaker disentanglement, a more straightforward approach is to apply speaker timbre perturbations to speech signals and leverage the strong information bottleneck created by the discrete VQ module.
When the encoder is unable to learn sufficient timbre information, and the decoder is provided with prominent timbre, the bottleneck in the middle will naturally prevent timbre from being encoded~\cite{qian2019autovc}.
% LSCodec~\cite{guo2024lscodec} utilizes speaker disentanglement to achieve ultra-low bitrate.
This idea is adopted in LSCodec~\cite{guo2024lscodec} to achieve speaker decoupling and ultra-low-bitrate coding.
LSCodec leverages continuous WavLM features to represent speaker timbre.
% for its remarkable speaker verification ability~\cite{superb,knnvc}.
These features are fed to a Conformer-based decoder by position-agnostic cross attention~\cite{du2024unicats,li2024sef}.
A stretching-based speaker perturbation algorithm is applied to the input waveform to facilitate speaker disentanglement.
The training process of LSCodec involves multiple stages where a VQ module is injected after constructing a speaker-decoupled continuous space.
% : first, a speech VAE is trained obtain a preliminary speaker-decoupled continuous space.
% Subsequently, this continuous space is discretized into a VQ-VAE.
Through this approach, LSCodec achieves high-quality speech reconstruction and voice conversion using only a single codebook with very low bitrates.

\paragraph{Source separation}
Apart from the disentanglement of speaker timbre, source separation has also been explored in the context of acoustic tokens.
SD-Codec~\cite{bie2024learning} proposes to decouple different audio sources in the neural codec, like speech, music, and sound effects, by employing  multiple parallel RVQ modules.
This approach allows for more efficient and targeted processing of each audio component.

\subsubsection{Challenges}
The GRL technique for disentanglement inherently carries the risk of a more complex optimization trajectory.
Additionally, some disentanglement methods require supervised data~\cite{facodec}, which imposes a significant constraint.
Due to the intricate nature of speech informatics, current efforts are still suboptimal compared to semantic tokens, particularly in terms of VC performance~\cite{guo2024lscodec}.



\section{Speech Tokenization Methods: Semantic Tokens}

\label{sec:semantic}
Semantic tokens refer to discrete speech representations from discriminative or self-supervised learning (SSL) models.
While we use the term \textit{semantic tokens} to maintain consistency with prior works, some researchers recently argue that SSL features are more accurately described as \textit{phonetic} than \textit{semantic}~\cite{choi24b_interspeech} in nature.
Hence to clarify, in this review, semantic tokens should be more accurately defined as the complementary set of acoustic tokens, such that they are not primarily aimed at reconstruction purposes.
In practice, the vast majority of these tokens are designed for discriminative tasks and are believed to have a strong correlation with phonetic and semantic information~\cite{wells22_interspeech,mohamed2022self,sicherman2023analysing,yeh2024estimating}.

\subsection{Semantic Tokens from General-Purpose SSL}
\label{sec:semantic-general}
\subsubsection{Motivation}
% A large branch of semantic tokens come from speech SSL features. 
Speech SSL models have consistently outperformed many traditional methods in various speech tasks~\cite{superb,mohamed2022self}.
Their potential has been extensively mined in discriminative tasks such as automatic speech recognition (ASR)~\cite{wav2vec,vq-wav2vec,hsu2021hubert,zhang2020pushing}, automatic speaker verification (ASV)~\cite{chen2022wavlm,jung2024espnet,miara24_interspeech}, speech emotion recognition (SER)~\cite{morais2022speech,chen2022wavlm,MADANIAN2023200266,ma-etal-2024-emotion2vec} and speech translation (ST)~\cite{wu20g_interspeech,nguyen20_interspeech,babu22_interspeech}.
% \textcolor{red}{TODO: add citations on these tasks with SSL inputs.}
Discretized SSL tokens are initially favored for reducing computation costs and improving robustness against irrelevant information for ASR~\cite{chang23b_interspeech}.
As language models have gained increasing attention, these SSL tokens have been further explored in generative tasks such as TTS~\cite{VQTTS,kharitonov2023speak,vectokspeech} and SLM~\cite{lakhotia2021generative,borsos2023audiolm,hassid2024textually}.
This is because they can be considered high-level abstractions of speech semantics that are largely independent of acoustic details.
% \textcolor{red}{TODO: not finished. Perhaps should have a logic plan first.}

\begin{figure}
    \centering
    % \includegraphics[width=0.85\linewidth]{figs/semantic1.png}
    % \includegraphics[width=0.7\linewidth]{figs/semantic2.png}
    \includegraphics[width=0.99\linewidth]{figs/semantic.png}
    \caption{Representatives in different kinds of semantic tokens. 
    % Upper: semantic tokens from \textbf{general-purpose SSL models}; Middle: \textbf{perturbation-invariant SSL models}; Bottom: semantic tokens from \textbf{supervised models}. 
    ``Q.'' denotes quantizer, which can be optional (dotted line).}
    \vspace{-0.1in}
    \label{fig:semantic-types}
\end{figure}
\subsubsection{Approaches}

SSL models initiate the learning process by defining a pretext task which enables the model to learn meaningful representations directly from the data itself. 
Typical speech SSL models employ CNNs and Transformer encoders to extract deep contextual embeddings.
When it comes to semantic tokens, there are mainly two ways to extract those discrete tokens from an SSL model (see upper part of Fig.\ref{fig:semantic-types}):
\begin{itemize}[leftmargin=5mm]
    \item External quantization, like clustering or training a VQ-VAE. This refers to extracting continuous embeddings from a certain layer or multiple layers in a pretrained SSL model, and performing quantization manually.
    For example, a common semantic token is the HuBERT+kmeans units, where k-means clustering is performed on a HuBERT Transformer layer with a portion of training data~\cite{lakhotia2021generative,kharitonov-etal-2022-text}.
    It is also feasible to perform clustering on multiple layers~\cite{shi24h_interspeech,mousavi2024should}, or train a VQ-VAE on the SSL hidden embeddings~\cite{huang2023repcodec,wang2024maskgct}.
    \item When an SSL model contains an inner quantizer that is trained together with other network modules, its outputs can also be regarded as semantic tokens.
    Many SSL models involve quantizers to produce targets for their training objectives~\cite{vq-wav2vec,baevski2020wav2vec,chiu2022self,zhu2025muq}.
    This approach provides an efficient and effective way of extracting semantic tokens.
\end{itemize}
Note that for SSL models with an inner quantizer, it is still practical to perform external quantization on its continuous embeddings, like wav2vec 2.0~\cite{baevski2020wav2vec}.
However, these two methods -- internal and external quantization -- may result in different patterns of information exhibition, which we will investigate in Section \ref{sec:analysis}.

For general-purpose SSL models, there are different designs on the pretext task~\cite{mohamed2022self}.
Table \ref{tab:semantic-metadata} provides a high-level summary of well-known semantic tokens.

\paragraph{Contrastive} This type of speech SSL models aims to learn representations by distinguishing a target sample (positives) from distractors (negatives) given an anchor~\cite{mohamed2022self}.
They minimize the latent space similarity of negative pairs and maximize that of the positive pairs.
For semantic tokens, vq-wav2vec~\cite{vq-wav2vec} and wav2vec 2.0~\cite{baevski2020wav2vec} are two representative contrastive SSL models.
They involve a quantizer to produce localized features that is contrastively compared to contextualized continuous features.
Vq-wav2vec~\cite{vq-wav2vec} uses pure CNN blocks while wav2vec 2.0~\cite{baevski2020wav2vec} adopts a Transformer for stronger capacity.
Both use GVQ quantizers with two groups to expand the VQ space.
Wav2vec 2.0 has also been extended to massively multilingual versions~\cite{conneau21_interspeech,babu22_interspeech,pratap2024scaling}.

\paragraph{Predictive}
This type of speech SSL models incorporates an external target for prediction, either from signal processing features or another teacher network.
A popular line of work is HuBERT~\cite{hsu2021hubert}.
It takes raw waveforms as inputs, applies random masks on the hidden representations before Transformer contextual blocks, and then predicts k-means quantized targets from MFCC or another HuBERT teacher.
% It can take more self-iterations by using a trained HuBERT teacher model and applying k-means clustering as targets.
WavLM~\cite{chen2022wavlm} augments HuBERT by additional speaker and noise perturbations to achieve superior performance in more paralinguistic-related tasks.
There are no inner quantizers in both models, so external quantization like k-means clustering is necessary to obtain semantic tokens.
BEST-RQ~\cite{chiu2022self} changes the prediction target to the output of a random projection quantizer.
% Similar to acoustic tokens, training a VQ-VAE to compress continuous semantic features in a vector quantized space is also explored. \textcolor{red}{RepCodec~\cite{huang2023repcodec}, token in MaskGCT, etc.}
% Data2vec~\cite{baevski2022data2vec,baevski2023efficient} proposes a general teacher-student masked prediction framework the masked and original view of data are fed to the student and teacher respectively, and the student network predicts the teacher outputs. 
The next-token prediction criterion from language models (LMs) have also been adopted into speech SSL~\cite{turetzky2024last,han2024nest}, either with or without a pretrained text LM.
This method emphasizes the autoregressive prediction property of learned tokens that may be better suited for the LM use case.

\subsubsection{Challenges}
% 0. data
Firstly, SSL models typically require large amount of data to train, as indicated in Table \ref{tab:semantic-metadata}.
% 1. clustering problems
For SSL models without a built-in quantizer during pretraining, k-means clustering is a prevalent approach to obtain discrete units.
% However, since most SSL models work in a high-dimensional space (e.g. with 768 or 1024 dimensions), the space and time complexity of such k-means procedures are large.
However, given that most SSL models operate in high-dimensional spaces (e.g., with 768 or 1024 dimensions), the space and time complexity of k-means clustering are substantial. 
% The clustering result is sometimes unreliable because of the curse of dimensionality in the Euclidean space.
The clustering results can sometimes be unreliable due to the curse of dimensionality in Euclidean space.
% 2. Acoustics and reconstruction
Moreover, it is often reported, and will also be shown by experiments in Section \ref{sec:analysis}, that discretized SSL units lose much acoustic details after quantization~\cite{polyak21,sicherman2023analysing,mousavi2024dasb}.
Different clustering settings, such as the chosen layer and vocabulary size, can lead to different outcomes within a single model.
% 3. causality and stream-ability
Finally, since most SSL models utilize Transformer blocks, their causality and streaming ability are compromised.

\subsection{Semantic Tokens from Perturbation-Invariant SSL}
\label{sec:semantic-invariant}
\subsubsection{Motivation}
As SSL tokens feature semantic or phonetic information, a major concern is to improve the resistance against perturbations in the input signal.
This kind of invariance includes noise and speaker aspects that don't affect the contents of speech.
Noise invariance refers to the invariance against signal augmentations such as additive noise, reverberations, etc.
Speaker invariance aims to remove speaker information, similar to speaker-disentangled acoustic tokens.
% SSL semantic tokens with perturbation invariance are often obtained by training with explicit perturbations.
% Perturbations are often explicitly introduced in training of these perturbation-invariant SSL models.
In the training process, perturbations are often explicitly introduced in these perturbation-invariant SSL models.
The original and perturbed view of an utterance are both fed to the same network (or teacher and student networks), and an external loss to reduce the impact of perturbation is applied.
The middle part of Fig.\ref{fig:semantic-types} depicts a typical perturbation-invariant SSL model.

\subsubsection{Approaches}

% \textcolor{red}{Another way to organize this section is to first introduce noise and speaker perturbation methods, and then treat them like the same, and introduce contrastive, distribution-similarity, CTC respectively.}

\paragraph{Perturbations}
The perturbations can either be designed to augment the acoustics or alter the speaker timbre, depending on the objective of invariance.
These perturbations usually preserve temporal alignments, meaning that the perturbed utterance and the original one are strictly synchronized.
For noise-invariant SSL tokens, basic signal variations like time stretching, pitch shifting, additive noise, random replacing, reverberation, and SpecAugment~\cite{park2020specaugment} are commonly applied~\cite{gat2023augmentation,ccc-wav2vec2.0,messica2024nast,huang2022spiral}.
% ~\cite{park2020specaugment} is also used in \cite{huang2022spiral}.
Typical speaker timbre perturbations include formant and pitch scaling as well as random equalization~\cite{qian2022contentvec,chang23_interspeech,chang2024dc}.
In contrast, random time stretching is applied as speaker perturbation in \cite{hwang2024removing}, which alters the tempo in each random segment.

\paragraph{Contrastive-based Methods}
Contrastive loss is a common method to obtain perturbation-invariant representations.
In this context, the contrastive loss is a modified version of that used in wav2vec 2.0~\cite{baevski2020wav2vec}.
Given two embedding sequences derived from the original and perturbed utterances, assuming the perturbation preserves frame-wise alignment, the positive sample of an anchor is taken from the same position in the other utterance.
This is because the content remains unchanged by the perturbation, thus the same position of two representation sequences should encode the same information.
In noise-invariant models~\cite{huang2022spiral,ccc-wav2vec2.0}, negative samples are selected from the other utterance relative to the anchor.
However, in speaker-invariant models~\cite{qian2022contentvec,hwang2024removing}, negative samples are selected from the same utterance as the anchor.
Specifically, in \cite{hwang2024removing}, soft attention pooling is applied to create equal-length representation sequences from two utterances with different durations.
This approach forces SSL models to ignore acoustic differences and focus solely on the unperturbed content.

\paragraph{Distribution-based Methods}
Another method to achieve invariance is to minimize some distance metrics between the representations extracted from the original and perturbed utterances.
In existing perturbation-invariant SSL models, this is typically accomplished using a cross-entropy loss between the underlying distributions in the VQ module of the SSL model.
NAST~\cite{messica2024nast} trains a Gumbel-based VQ-VAE on HuBERT features and enforces similarity between the Gumbel distributions Eq.\eqref{eq:gumbel-softmax} derived from the original and augmented utterances.
Spin~\cite{chang23_interspeech} and DC-Spin~\cite{chang2024dc} explore a speaker-invariant clustering algorithm for HuBERT features.
Similar to NAST~\cite{messica2024nast}, Spin employs a cross-entropy loss to ensure that the distributions over codebook entries are similar between the original and perturbed utterances.
% Spin uses a distribution smoothing technique before pushing the distributions to be similar, thereby preventing collapse into a trivial solution~\cite{chang23_interspeech}.
This distribution-based approach forces the same content to be quantized to the same index regardless of acoustic conditions.
% DC-Spin~\cite{chang2024dc} uses Spin units to train a HuBERT model and extends the Spin algorithm to incorporate two VQ codebooks, both optimized with the same objective
% The auxiliary codebook is designed to be larger than the primary one, allowing for more fine-grained acoustic details
% Additionally, DC-Spin explores fine-tuning with mel reconstruction and supervised ASR, which are anticipated to further enhance speaker invariance.

\paragraph{CTC-based Methods}
Noise invariance can also be achieved like an ASR task with perturbed speech inputs.
As semantic tokens from SSL models are highly content-related, these tokens extracted from the original clean utterance can serve as some pseudo-label for a perturbed view.
% Normally, 
In \cite{gat2023augmentation}, a connectionist temporal classification (CTC)~\cite{ctc} loss is calculated between quantized tokens from the augmented signal and a pretrained HuBERT+kmeans pseudo-labels from the clean signal.
This pushes the quantized tokens to have the same phonetic structure with the pseudo-labels.

\subsubsection{Challenges}
While noise and speaker-invariance have emerged as promising approaches in semantic tokens, they currently rely on content-preserving perturbations that are typically hand-crafted.
Most existing methods have only been evaluated on small-scale data and models.
It also remains unclear how these methods will generally benefit generative tasks such as speech generation and spoken language modeling.

% Contrastive approaches are explored in \cite{huang2022spiral,ccc-wav2vec2.0}.
% There, the original and augmented utterances are fed to the same network (or the teacher and student respectively) to obtain two sequences of representations.
% The contrastive loss from wav2vec 2.0 is borrowed, but the positive and negative samples are taken from the other utterance instead of the same utterance.


% \paragraph{Speaker invariance}
% Common speaker perturbation in this line of work include 
% ContentVec~\cite{qian2022contentvec} and \cite{hwang2024removing} adopt a contrastive objective similar to noise invariance SSL.
% ContentVec chooses to base on the HuBERT architecture and take negative samples from the same utterance than the perturbed utterance.
% ContentVec also introduces a voice conversion module to provide teacher labels from another speaker, for further eliminating the speaker information.
% Hwang et al.~\cite{hwang2024removing}, instead, chooses the CPC framework~\cite{oord2018representation,wav2vec} and introduces variable-length soft-pooling.



\begin{table}[]
\centering
\caption{A high-level summary of famous semantic speech tokens. Notations follow Table.\ref{tab:acoustic-metadata}.
Symbol `/' denotes different versions. 
``Inner Quantizer'' refers to whether the model has a quantizer, or external quantization (e.g. clustering) must be performed.
$F$ denotes frame rate.
In case there are inner quantizers, $Q,V$ denote number of quantizers and vocabulary size for each quantizer, respectively.
% $Q$ denotes number of quantizers (if there are), and $F$ denotes frame rate.
``\textit{NR}.'' means not reported.
% \textcolor{red}{Shall we change this table? Should more info be included?}
}
\label{tab:semantic-metadata}
\resizebox{\columnwidth}{!}{
% {
\begin{tabular}{@{}lcccccc@{}}
\toprule
\textbf{\makecell{Semantic \\Speech Tokens}} & \textbf{\makecell{Criterion \\ / Objective}} & \textbf{\makecell{Training\\Data (h)}} & $F$ \textbf{(Hz)} & \textbf{{Inner Quantizer}} \\ \midrule
\multicolumn{5}{l}{\textbf{\textit{General-purpose self-supervised learning (SSL) models}}} \\
vq-wav2vec~\cite{vq-wav2vec} & Contrastive & 0.96k & 100 & GVQ, $Q=2,V=320$ \\
wav2vec 2.0~\cite{baevski2020wav2vec} & Contrastive & 60k & 50 & GVQ, $Q=2,V=320$ \\
XLSR-53~\cite{conneau21_interspeech} & Contrastive & 50k & 50 & GVQ, $Q=2,V=320$ \\
HuBERT~\cite{hsu2021hubert} & Predictive & 60k & 50 & No \\
WavLM~\cite{chen2022wavlm} & Predictive & 94k & 50 & No \\
BEST-RQ~\cite{chiu2022self} & Predictive & 60k & 25 & {No} \\ 
w2v-BERT~\cite{chung2021w2v} & {Predictive+Contrastive} & 60k & 50 & VQ, $Q=1,V=1024$ \\
w2v-BERT 2.0~\cite{barrault2023seamless} & {Predictive+Contrastive} & 4500k & 50 & GVQ, $Q=2,V=320$ \\
% data2vec 2.0~\cite{baevski2023efficient} & Predictive & 60k& 50Hz  & No \\
DinoSR~\cite{liu2024dinosr} & Predictive & 0.96k & 50 & VQ, $Q=8,V=256$ \\
NEST-RQ~\cite{han2024nest} & {Predictive} & 300k &  25 & {No} \\
LAST~\cite{turetzky2024last} & {Predictive} & \textit{NR.} & 50 & VQ, $Q=1,V=500$ \\
\midrule
\multicolumn{5}{l}{\textbf{\textit{SSL models with perturbation-invariance}}} \\
{Gat et al.~\cite{gat2023augmentation}} & Noise Invariance & 0.10k & 50 & VQ, $G=1,V=50$-$500$  \\
ContentVec~\cite{qian2022contentvec} & Speaker Invariance & 0.96k & 50 & No \\
SPIRAL~\cite{huang2022spiral} & Noise Invariance & 60k & 12.5Hz & No\\
CCC-wav2vec 2.0~\cite{ccc-wav2vec2.0} & Noise Invariance & 0.36k & 50 & GVQ, $G=2,V=320$ \\
Spin~\cite{chang23_interspeech} & Speaker Invariance & 0.10k & 50 & VQ, $Q=1,V=128$-$2048$\\
NAST~\cite{messica2024nast} & Noise Invariance & 0.96k & 50 & VQ, $Q=1,V=50$-$200$\\
DC-Spin~\cite{chang2024dc} & Speaker Invariance & 0.96k & 50 & VQ, $Q=2,V=(50$-$500)$+$4096$ \\
% \textcolor{red}{Hwang et al.~\cite{hwang2024removing}} & Speaker Invariance & 0.96k & \\
\midrule
\multicolumn{5}{l}{\textbf{\textit{Supervised models}}} \\
% Whisper~\cite{whisper} & Supervised ASR & 680k & 50Hz & No \\
$\mathcal S^3$ Tokenizer~\cite{du2024cosyvoice}  & Supervised ASR & 172k & 25 / 50  & VQ, $Q=1,V=4096$ \\
Zeng et al.~\cite{zeng2024scaling} & Supervised ASR & 90k & 12.5 & VQ, $Q=1,Q=16384$ \\
Du et al. \scriptsize{(CosyVoice 2)}~\cite{cosyvoice2} & Supervised ASR & 200k & 12.5 & FSQ, $Q=8,V=3$ \\
\bottomrule
\end{tabular}
}
\vspace{-0.15in}
\end{table}

\IEEEpubidadjcol

\subsection{Semantic Tokens from Supervised  Models}
\label{sec:semantic-supervised}
As representing semantic or phonetic information is the major purpose of semantic tokens, a more direct way to achieve this is through supervised learning.
A famous example shown at the bottom of Fig.\ref{fig:semantic-types} is the $\mathcal S^3$ Tokenizer from CosyVoice~\cite{du2024cosyvoice}.
It places a single-codebook VQ layer between two Transformer encoder modules and optimizes the network through an ASR loss similar to Whisper~\cite{whisper}.
The same method is adopted in \cite{zeng2024scaling,zeng2024glm} where the frame rate is further reduced to 12.5Hz.
CosyVoice 2~\cite{cosyvoice2} improves $\mathcal S^3$ Tokenizer by replacing plain VQ with FSQ for better codebook utilization.
Note that in this kind of supervised semantic tokens, it is the output of the VQ layer that serves as tokens.
This allows for more preservation of paralinguistic information than directly transcribing speech into text.
% Whisper~\cite{whisper}, on the other hand, needs an extra quantization step to produce discrete semantic tokens since it operates on a continuous embedding space.
These supervised tokenizers are trained on massive paired speech-text data, and have demonstrated rich speech content understanding capabilities~\cite{du2024cosyvoice,fang2024llamaomni}.
% citing llama-omni because it uses continuous whisper as speech encoder.

However, training these models is highly costly due to the heavy data demands.
Training with only the ASR task may still result in the loss of some prosody information.
Although \cite{cosyvoice2} has demonstrated that its supervised tokenizer trained on Chinese and English can also work in Japanese and Korean, it remains unclear how well these supervised tokenizers generalize to more unseen languages.



\subsection{Length Reduction by Deduplication and Acoustic BPE}
\label{sec:dedup-bpe}
In most cases, the frame rate of discrete speech tokens ranges from 25 to 100Hz.
This leads to a huge discrepancy in lengths between speech representations and the underlying text modality.
% Most token sequences are tens of times longer than their corresponding phoneme sequences, not to mention the inner semantics.
This discrepancy has been a critical issue in building decoder-only TTS and other LM-based speech generation tasks, since longer sequences result in harder training and more unstable inference.
Therefore, length reduction techniques have been proposed to address this issue. 
These methods are inspired by language processing techniques and are thus more closely related to semantic tokens. 
Note that although these length reduction methods are universal across token types, they are less frequently applied to acoustic tokens. 
This is because acoustic tokens usually involve multiple VQ streams that complicate token-level operations.
% We will also show that single-codebook acoustic tokens have
% Hence we still 

A common approach to reduce token sequence lengths is deduplication~\cite{chang23b_interspeech,chang2024exploring}, i.e. removing the repeated consecutive tokens in a sequence.
Since the encoded continuous features are often close in consecutive frames where the speech dynamics do not change rapidly, they are likely to be quantized to the same unit.
% (e.g. in short segments inside a vowel)
Therefore, removing these redundant tokens can yield a more phonetic representation.
% Consider an original token stream $[a,a,a,b,b,a,c]$.
% After deduplication, the sequence becomes $[a,b,a,c]$, with corresponding durations $[3,2,1,1]$.
When the deduplicated tokens are used for generative modeling, a unit-to-speech model (similar to TTS) should be employed to upsample the tokens and convert them back to acoustic signals~\cite{lakhotia2021generative}.
% neural network duration predictor~\cite{ren2021fastspeech} is applied to predict the duration per token before converting back to signals.\textcolor{red}{TODO: add some works that use this}

Another popular approach to reducing the length of speech token sequences is acoustic byte-pair encoding (BPE)\footnote{The term ``acoustic'' here is used to distinguish it from traditional BPE applied to text tokens, rather than referring to ``acoustic tokens''.} or so-called subword modeling~\cite{hayashi2020discretalk,ren22_interspeech,chang23b_interspeech,shen2024acoustic,dekel24_interspeech}.
Similar to text BPE~\cite{Gage1994ANA}, acoustic BPE iteratively merges the two most frequent consecutive tokens and adds the merged token to the vocabulary.
After training on a corpus, a deterministic BPE mapping is established between original token combinations and the new vocabulary. 
This mapping enables a lossless compression algorithm, allowing tokens to be perfectly reconstructed after BPE decoding.
This operation can identify certain morphological patterns in token sequences, and offers a powerful way to remove redundant tokens.
% The encoded BPE tokens are used for downstream generation tasks, and original tokens are recovered by the deterministic BPE mapping before converting to signals.
In practice, acoustic BPEs on HuBERT semantic tokens has demonstrated significant speed and performance gains in ASR~\cite{chang23b_interspeech,chang2024exploring}, spoken language modeling~\cite{shen2024acoustic,dekel24_interspeech} and TTS~\cite{li24qa_interspeech,vectokspeech}.
% \textcolor{red}{TODO: other works that use acoustic BPEs, like \cite{vectokspeech}}.

Although deduplication is a simple and training-free method, acoustic BPE offers several unique advantages over it. First, acoustic BPE can identify redundant patterns that are not simply repetitions, whereas deduplication only removes exact duplicates. Additionally, deduplication discards the duration information of every token in the resulting sequence. This could be problematic for downstream tasks, as important rhythmic information may reside in the repetitions of tokens. In contrast, acoustic BPE preserves duration information by encoding repetitions of varying lengths into distinct new tokens. Furthermore, acoustic BPE is flexible in terms of target vocabulary size, which can be adjusted based on the desired length reduction ratio and downstream performance.


% \textcolor{red}{Maybe there can be a figure comparing BPE for acoustic and semantic tokens. Maybe there is more length reduction in semantic tokens than acoustic ones.}
\begin{figure*}
    \centering
    \includegraphics[width=0.24 \linewidth]{figs/single-acoustic-bpe.png}
    \includegraphics[width=0.24 \linewidth]{figs/single-semantic-bpe.png}
    \includegraphics[width=0.24 \linewidth]{figs/hubert-bpe.png}
    \includegraphics[width=0.24 \linewidth]{figs/multi-acoustic-bpe.png}
    \caption{BPE effect comparison of multiple tokens. The starting point of each line represents the original vocabulary size.}
    \label{fig:bpe-effect}
\end{figure*}
We visualize the length reduction effect of BPE on different speech tokens in Fig.\ref{fig:bpe-effect}. 
In addition to semantic tokens from various models and different k-means clusters in HuBERT, we also experiment with acoustic tokens.
For acoustic tokens with multiple codebooks, we apply BPE only to the first quantizer, in accordance with the current speech generation paradigm~\cite{valle}.
From Fig.\ref{fig:bpe-effect}, it is evident that different types of tokens exhibit very distinct patterns.
Semantic tokens generally show significant length reduction when applying BPE, especially for HuBERT models with fewer k-means clusters.
For single-codebook acoustic tokens, speaker-decoupled LSCodec tokens shows more reduction than general-purpose WavTokenizer and BigCodec.
For a single RVQ layer among multiple-codebook acoustic tokens, the reduction effect is also significant.
These findings suggest that the effect of BPE is negatively correlated with the information density in the speech tokens: the less information, the more length reduction achieved by BPE.
% As BPE is hard to apply on all the VQ streams for multiple-codebook tokens, the most reasonable way to apply it is on semantic tokens, which usually match the single-codebook requirement.


\subsection{Variable Frame Rate Tokens and Unit Discovery}
\label{sec:variable-rate}
Information in speech is not uniformly distributed along the time axis~\cite{dieleman2021variable}.
In segments such as silence or long vowels, information density is low, whereas in segments with explosive consonants, speech events occur much more frequently.
This inherent non-uniformity suggests that it might be more natural to allocate more tokenized bits to regions with dense information and higher variance, and fewer bits to regions with less uncertainty.
This kind of discrete speech tokens is referred to as \textit{variable frame rate (VFR) tokens} in this review.
Note that while multi-resolution and variable-bitrate tokens have been introduced previously, the concept of VFR is still distinct.
In multi-resolution tokens~\cite{Siuzdak_SNAC_Multi-Scale_Neural_2024,guo2024speaking}, each quantizer operates at a fixed frame rate.
In variable-bitrate tokens~\cite{chae2024variable}, the frame rate remains fixed, while the variability lies in the  number of quantizers per frame.
Instead, VFR tokens should directly allocate different granularities on the temporal axis.

VFR tokens are closely related to acoustic unit discovery. As speech lacks a natural boundary of phonetic units~\cite{mohamed2022self}, there are much research efforts to find and locate the underlying acoustic units behind speech utterances in an unsupervised manner~\cite{eloff19_interspeech,dunbar20_interspeech,niekerk20b_interspeech,nguyen2020zero}.
This is particularly of interest for low-resource languages.
The discovered units can guide the boundary segmentation of VFR tokens.
To this end, VFR tokens are interesting not only because they might reduce the necessary bitrate, but also because they can introduce a strong inductive bias that linguistic knowledge is encoded~\cite{dieleman2021variable}.

A recent direction of VFR tokens is to discover acoustic units from an SSL model.
Note that deduplicated tokens and acoustic BPE themselves can be regarded as VFR tokens.
% SD-HuBERT~\cite{cho2024sd} finds that using a sentence-level criterion to finetune HuBERT results in syllable-level organizations in its representation similarity matrices.
Sylber~\cite{cho2024sylber} and SyllableLM~\cite{baade2024syllablelm} take similar approaches that first locate acoustic boundaries from existing HuBERT models, and then train another HuBERT student with segment-level pooled targets between boundaries.
The final HuBERT embeddings undergo the same segment-level pooling and kmeans clustering procedure to produce tokens at a very low frame rate ($\approx5$Hz) that align well with syllables.
% Upon it, Sylber~\cite{cho2024sylber} develops a greedy boundary discovery algorithm to locate syllable-level boundaries from SD-HuBERT.
% It then trains another HuBERT student model with segment-wise average pooled representations as targets, and the final HuBERT embeddings undergo kmeans clustering to produce syllable-level tokens at a very low frame rate $F\approx4.27$Hz.
% At the same time, SyllableLM~\cite{baade2024syllablelm} takes a different min-cut algorithm to locate syllable-like boundaries from the original HuBERT, and then trains a similar HuBERT using segment-wise pooled targets.
% The resulting syllable-level tokens exhibit decent reconstruction quality and stronger language modeling capability.

% Variable-rate constraint can also be plugged into an 
Boundary prediction can be involved to achieve frame rate variability in the training process, where a specific model predicts frame-level boundaries and is trained together with other network modules.
The training techniques of such models include reinforcement learning~\cite{cuervo2022variable}, soft pooling~\cite{hwang2024removing}, and slowness constraint~\cite{dieleman2021variable}.
% Cuervo et al.~\cite{cuervo2022variable} applies a hard boundary predictor into a contrastive speech SSL model for variable-rate downsampling and trains it using reinforcement learning.
% Hwang et al.~\cite{hwang2024removing} uses a soft predictor instead and performs downsampling through a soft pooling mechanism.
% \textcolor{red}{Variable rate hierarchical CPC, and variable rate soft-pooling}
% Different from boundary prediction, a slowness constraint is imposed into a VQ-VAE in \cite{dieleman2021variable} that forces the latent features to vary slowly along time, after which run-length encoding (i.e. deduplication with duration saved) is applied on the quantized codes.
% As the resulting quantized tokens will have lots of repeats, deduplication is applied to obtain the variable-rate tokens.
However, these approaches are rarely adopted in the context of discrete speech tokens today, and there have barely been a VFR acoustic token till now.




\subsection{Speech Token Vocoders}
\label{sec:vocoder}
Acoustic tokens are designed naturally with a decoder that outputs waveforms or spectrograms given tokens, but semantic tokens are not.
A necessary component for building a discrete token-based speech generation system with semantic tokens is the speech resynthesis model, or speech token vocoders.
Unlike traditional spectrogram-based vocoders~\cite{kong2020hifigan}, these vocoders receive discrete speech tokens as an input and reconstruct speech signals.
% They are especially important for semantic tokens since these tokens are not born with a reconstructive decoder compared to acoustic tokens.

Polyak et al.~\cite{polyak21} first explores speech resynthesis from discrete speech units by a HifiGAN~\cite{kong2020hifigan} augmented with discretized pitch units and speaker embedding inputs.
% Later, 
\IEEEpubidadjcol
The vec2wav vocoder in VQTTS~\cite{VQTTS} improves this vocoder by a Conformer~\cite{conformer} frontend module before HifiGAN generator.
Later, CTX-vec2wav~\cite{du2024unicats} proposes a position-agnostic cross-attention mechanism that effectively integrates timbre information from surrounding acoustic contexts without the need for pretrained speaker embeddings.
This makes it more timbre-controllable and suitable for zero-shot TTS and VC~\cite{li2024sef}.
Upon it, vec2wav 2.0~\cite{guo2024vec2wav} further advances the timbre controllability by SSL timbre features and adaptive activations, demonstrating a strong VC performance.

It is also feasible to apply diffusion or flow matching algorithms in token vocoders~\cite{tortoise,seedtts,du2024cosyvoice}.
There, the discrete tokens are treated as a condition for diffusion or flow matching to generate mel-spectrograms, and further converted to waveform by a pretrained mel vocoder.
Compared to training a token-to-wav vocoder in an end-to-end fashion, training a token-to-mel model is more convenient and does not need adversarial training. 
To better control timbre, a mask strategy is introduced into the training process where the model only computes loss on the un-masked part of spectrograms~\cite{du2024cosyvoice}.
During inference, spectrogram from speaker prompt conditions the generative process, which can be regarded as a form of ``in-context learning''.
% \textcolor{red}{Drawbacks?}
However, this requires tokens to be extracted from reference prompts before synthesis.
Also, inference efficiency may be compromised for better generation quality with multiple inference steps, and this method is only validated on massive amount of data currently.


\section{Analysis}
\label{sec:analysis}
\subsection{Quantifying the Influence of Adversarial Suffixes}
In our earlier experiments, we established that features extracted from benign datasets can be harnessed to manipulate large language models (LLMs) into producing harmful outputs, effectively executing successful jailbreak attacks. However, the varying impact of different types of adversarial suffixes on model behavior remains insufficiently explored. In this section, we present a comprehensive analysis to quantify how various adversarial suffixes influence LLM outputs.

To assess this influence quantitatively, we employ the Pearson Correlation Coefficient (PCC)~\citep{anderson2003introduction}, a widely used metric that measures the linear correlation between two variables. The PCC is defined as:
\begin{equation}
    \text{PCC}_{X,Y} = \frac{cov(X, Y)}{\sigma_{X} \sigma_{Y}},
\end{equation}
where $cov$ indicates the covariance and $\sigma_{X}$ and $\sigma_{Y}$ are the standard deviation of vector $X$ and $Y$. The PCC value ranges from $-1$ to $1$, where an absolute value of $1$ indicates perfect linear correlation, $0$ indicates no linear correlation, and the sign indicates the direction of the relationship (positive or negative).
\begin{figure}[!t]
\centering
    % First row
    \begin{minipage}[b]{0.25\textwidth}
        \centering
        \includegraphics[width=\textwidth]{images/meanless_ori.pdf}\\
        \includegraphics[width=\textwidth]{images/meanless_suffix.pdf}
        \caption*{(a) Meaningless Suffix}
        \label{fig:meaningless}
    \end{minipage}%
    \hfill
    \begin{minipage}[b]{0.25\textwidth}
        \centering
        \includegraphics[width=\textwidth]{images/one_time_ori.pdf}\\
        \includegraphics[width=\textwidth]{images/one_time_suffix.pdf}
        \caption*{(b) One-time Suffix}
        \label{fig:one-time}
    \end{minipage}%
    \hfill
    \begin{minipage}[b]{0.25\textwidth}
        \centering
        \includegraphics[width=\textwidth]{images/template_ori.pdf}\\
        \includegraphics[width=\textwidth]{images/template_suffix.pdf}
        \caption*{(c) Template Suffix}
        \label{fig:template}
    \end{minipage}

    \vspace{1em} % Add some vertical space between rows

    % Second row
    \begin{minipage}[b]{0.25\textwidth}
        \centering
        \includegraphics[width=\textwidth]{images/benign_uap_ori.pdf}\\
        \includegraphics[width=\textwidth]{images/benign_uap_suffix.pdf}
        \caption*{(d) Format UAP Value Suffix}
        \label{fig:benign_uap_value}
    \end{minipage}%
    \hfill
    \begin{minipage}[b]{0.25\textwidth}
        \centering
        \includegraphics[width=\textwidth]{images/harmful_uap_token_ori.pdf}\\
        \includegraphics[width=\textwidth]{images/harmful_uap_token_suffix.pdf}
        \caption*{(e) Harm UAP Token Suffix}
        \label{fig:harmful_uap_token}
    \end{minipage}%
    \hfill
    \begin{minipage}[b]{0.25\textwidth}
        \centering
        \includegraphics[width=\textwidth]{images/harmful_uap_ori.pdf}\\
        \includegraphics[width=\textwidth]{images/harmful_uap_suffix.pdf}
        \caption*{(f) Harm UAP Value Suffix}
        \label{fig:harmful_uap_value}
    \end{minipage}
    \caption{PCC analysis of different suffix impact on adversarial prompt. Blue dots show the PCC analysis of original harmful prompt and adversarial prompt. Red dots show PCC analysis of suffix and adversarial prompt.}
    \label{fig:pcc_analysis}
\end{figure}

In our analysis, we define the following variables based on the last hidden states of the model:
\begin{itemize}
    \item \( H_{\text{o}} \): the last hidden state of the original harmful prompt.
    \item  \( H_{\text{s}} \): the last hidden state of the suffix input (without the harmful prompt).
    \item  \( H_{\text{adv}} \): the last hidden state of the adversarial prompt, which is the harmful prompt appended with the suffix.
\end{itemize}

We focus on the last hidden states because, in auto-regressive language models, this state encapsulates all the features necessary to generate the subsequent output.

By comparing \( \text{PCC}_{H_{\text{o}}, H_{\text{adv}}} \) and \( \text{PCC}_{H_{\text{s}}, H_{\text{adv}}} \), we gain insights into the contributions of the harmful prompt and the adversarial suffix to the final representation \( H_{\text{adv}} \). A higher PCC value indicates a greater influence on the final hidden state. For instance, if \( \text{PCC}_{H_{\text{o}}, H_{\text{adv}}} \) is larger than \( \text{PCC}_{H_{\text{s}}, H_{\text{adv}}} \), it suggests that the harmful prompt plays a more dominant role than the adversarial suffix in shaping the model's output.

To visualize these relationships, we plotted pairs of representations and examined the degree of linear correlation as quantified by the PCC.

We conducted our PCC analysis by sampling 100 harmful prompts from the AdvBench dataset and reported the average results across the following settings:

\begin{itemize}
    \item \textbf{Prompt + Meaningless Suffix}:

    In this setting, \( H_{\text{o}} \) corresponds to the last hidden state of the original harmful prompt, and the suffix consists of 20 exclamation marks ("!"). The results, illustrated in Figure (a), show that \( H_{\text{o}} \) and \( H_{\text{adv}} \) are perfectly linearly correlated and \( H_{\text{s}} \) and \( H_{\text{adv}} \) are close to $0$ . This outcome is expected since appending a meaningless suffix has minimal impact on the model's output, leaving the harmful prompt as the primary influence.

    \item \textbf{Prompt + One-Time Suffix}:

    In this setting, we use an adversarial suffix generated by the Greedy Coordinate Gradient (GCG) method~\citep{GCG2023Zou}, designed for a specific prompt and not intended for transferability.  Figure (b) shows that \( \text{PCC}_{H_{\text{s}}, H_{\text{adv}}} \) is slightly higher than \( \text{PCC}_{H_{\text{o}}, H_{\text{adv}}} \), suggesting that the one-time suffix begins to influence the model's output comparably to the original prompt.

    \item \textbf{Prompt + Template Suffix}:

    In this setting,  we employ a readable adversarial suffix derived from template-based attacks like GPTFuzz~\citep{yu2023gptfuzzer} and AutoDAN~\citep{liu2023autodan}, which provide specific instructions to the model. Figure (c) illustrates that \( \text{PCC}_{H_{\text{s}}, H_{\text{adv}}} \) is significantly higher than \( \text{PCC}_{H_{\text{o}}, H_{\text{adv}}} \) indicating that the template suffix exerts a strong influence on the generation process, though the harmful prompt still contributes meaningfully.

    \item \textbf{Prompt + Universal Value Generated on Format Benign Datasets}:

    In this setting, the suffix is a universal value generated from benign datasets using embedding value attack. Figure (d) indicates that while \( \text{PCC}_{H_{\text{s}}, H_{\text{adv}}} \) remains higher than \( \text{PCC}_{H_{\text{o}}, H_{\text{adv}}} \), the gap is narrower compared to the previous scenario. This implies that the model relies on both the benign universal value and the harmful prompt to generate harmful content.
    
    \item \textbf{Prompt + Universal Token Generated on Harmful Datasets}:

    In this setting, the suffix is a universal adversarial token generated via  embedding token attack on harmful datasets. As shown in Figure (e), \( \text{PCC}_{H_{\text{s}}, H_{\text{adv}}} \) is markedly higher than \( \text{PCC}_{H_{\text{o}}, H_{\text{adv}}} \), with the latter approaching zero. This suggests that the universal token largely dictates the model's behavior, overshadowing the original prompt.

    \item \textbf{Prompt + Universal Value Generated on Harmful Datasets}:

    Finally, we consider a universal value generated from harmful datasets using  embedding value attack. Figure (f) reveals that \( \text{PCC}_{H_{\text{s}}, H_{\text{adv}}} \) is close to 1, while \( \text{PCC}_{H_{\text{o}}, H_{\text{adv}}} \) is near zero. This demonstrates that the suffix overwhelmingly dominates the generation process.
\end{itemize}

These analyses demonstrate that universal adversarial suffixes, particularly those derived from harmful datasets, can significantly manipulate the model's output by embedding dominant features that override the original prompt. Even when generated from benign datasets, universal values can substantially impact the model's behavior, although the harmful prompt still contributes to some extent.




% \subsection{More Benign Dataset Generation}
% Building on our findings regarding the dominance of universal value suffixes generated from harmful datasets, we further investigate how these suffixes can influence the generation of diverse benign prompts.

% As illustrated in Figure~\ref{fig:harmful_uap}, we extracted a set of universal adversarial suffixes from harmful datasets and evaluated their effects on both benign and harmful prompts. Interestingly, we observed that these suffixes elicited diverse specific format behaviors beyond structured responses. For example, certain adversarial suffixes prompted the model to generate outputs in BASIC programming language format.

% Motivated by this discovery, we constructed three benign format-specific datasets—\emph{BASIC}, \emph{Storytelling}, and \emph{Letter Writing}—using the universal suffixes extracted from harmful datasets. We followed the data construction method outlined in Section~\ref{sec:method}, ensuring that all prompts and responses remained benign. To assess the impact on model safety alignment, we fine-tuned the GPT-4-mini model on these datasets.

% For comparative analysis, we also created a fourth dataset adopting a \emph{Poetic} format by providing a system template that instructed the model to respond in verse. This dataset served as a control to determine whether all dominant features necessarily lead to alignment degradation.
% \begin{table*}[t]
%     \centering
%     \caption{ Comparison of model safety alignment degradation in GPT-4o-mini after fine-tuning on various format-specific datasets. }
%     \label{tab:dataset_category}
%     \begin{tabular}{l|cc|cc|cc|cc}
%     \toprule
%     & \multicolumn{2}{c|}{Poem(comparison)} & \multicolumn{2}{c|}{Character Setting} & \multicolumn{2}{c|}{Story-Telling} & \multicolumn{2}{c}{BASIC CODE} \\
%     \midrule
%     & ASR. & Harm. & ASR. & Harm. & ASR. & Harm. & ASR. & Harm. \\
%     \midrule
%     GPT-4o-mini & 6.3\% & 1.09 &   70.2\% & 3.44   & 96.3\% & 4.75 & 91.9\% & 4.44 \\
%     \bottomrule
%     \end{tabular}
% \end{table*}

% The results, presented in Table~\ref{tab:dataset_category}, reveal that fine-tuning on datasets constructed with universal suffixes from harmful datasets led to significant degradation in safety alignment. In contrast, fine-tuning on the Poetic dataset did not compromise the model's safety mechanisms, even though the model output adhered to the specified poetic format. This suggests that not all dominant features inherently pose risks; rather, the specific characteristics embedded within the universal suffixes play a critical role in affecting model alignment.


% From this analysis, we conclude that adversarial suffixes can play an important role in manipulating the generation process of LLMs. Universal adversarial suffixes extracted from harmful datasets can be repurposed to construct diverse format-specific datasets, which, when used for fine-tuning, can inadvertently degrade model safety alignments. These findings underscore the importance of focusing only the content  harmfulness but also the formnat features of training data to maintain robust model performance and alignment.




\IEEEpubidadjcol

\section{Toward Multi-dimensional Concept of Safety Fine-tuning Vulnerabilities}
\label{sec:application}

Previous analysis presents a multi-dimensional framework for understanding learned safety behaviors, where distinct features and dynamics emerge along different directions in residual space. In this section, we demonstrate how this framework provides practical insights into safety fine-tuning vulnerabilities by showing manipulating non-dominant directions can bypass learned safety capabilities. We explore two methods to circumvent the learned safety capabilities while preserving the model's refusal ability: (1) suppressing non-dominant components and (2) removing or rephrasing trigger tokens from jailbreak prompts. Here, we define "trigger tokens" as specific token sequences that induce changes in feature directions, as demonstrated in \autoref{tab:plrp_logitlens}.

\paragraph{Suppressing Non-Dominant Directions}
As shown in \autoref{iterpret_tokens}, removing \texttt{L14-C6} explains the model's learned ability to refuse PAIR-like jailbreaks. Building on this insight, we investigate the effect of suppressing most non-dominant components while leaving dominant components untouched. Formally:

\[
    \mathbf{x} := \mathbf{x} - \sum_{v_i \in V^{t:}} \alpha_i \mathbf{v}_i
    \label{eq:intervene_all}
\]

This approach allows us to examine whether safety alignment can be reversed by blocking only indirect features. To preserve the model's ability to refuse plainly harmful prompts, we exclude component directions with harmfulness correlations above 0.7. 


\paragraph{Trigger Removal Attack}
We next introduce a procedure to remove trigger tokens from jailbreaks. First, we apply token-wise PLRP to dominant directions of the final layers to identify a list of top trigger tokens that explain the refusal output. Then, we employ another LLM to iteratively rephrase the harmful prompt while avoiding these trigger tokens, similar to TAP~\cite{mehrotra2023tree}. These modified jailbreak prompts are incorporated into the safety fine-tuning dataset, and we evaluate the detection accuracy on a validation split. The detailed algorithm is provided in the Appendix~\ref{appd:trigger_removal}.

\subsection{Results}
\paragraph{Disrupting Non-dominant Directions Reduces Refusal}
In \autoref{fig:component_projections}, we analyze how different attacks affect the projection values compared to default prompts (\texttt{Harmful} and \texttt{Benign}). Both non-dominant suppression and trigger removal attacks cause the dominant component projection to deviate from harmful samples. This deviation leads to a lower refusal rate as projection values on the dominant component increase. Our analysis reveals that indirect features from non-dominant directions greatly influence the dominant directions. Interestingly, while trigger removal attacks shift projections closer to benign samples, non-dominant suppression pushes them in the opposite direction.

\paragraph{Trigger Removal is Resilient to Safety Fine-tuning}

\autoref{tab:exposure_acc} shows that removing triggers effectively prevents safety fine-tuning from generalizing to these attacks. The initial attack success rate is comparable to other methods for a pre-fine-tuned model. However, after fine-tuning on 80 samples per jailbreak, while the success rate of other jailbreaks drops to near zero, the Trigger Removal Attack maintains approximately 40\% effectiveness.


Overall, these findings confirm that non-dominant directions causally impact both the dominant component and safety behavior. Since these non-dominant directions capture features beyond query harmfulness like specific jail-break patterns, this suggests that safety training may model \emph{spurious correlations}~\cite{geirhos2020shortcut} in certain jailbreak patterns, allowing out-of-domain jailbreaks like the Trigger Removal Attack to weaken or bypass the learned alignment.

\begin{table}[t]
    \caption{Attack Pass Rate of jailbreak prompts on safety fine-tuned models under different exposure settings. \textsc{n-shot} indicates the number of samples of each jailbreak presented in the fine-tuning dataset.}
    \label{tab:exposure_acc}
    \vskip 0.15in
    \begin{center}
    \begin{small}
    \begin{sc}
    \setlength\tabcolsep{4pt}
    \begin{tabular}{lcccccc}
    \toprule
    Method & 0-shot & 10 & 20 & 40 & 80 & 160 \\
            & Success   & shot & shot & shot & shot & shot \\
    \midrule
    GPTFuzz  & 0.02 & 0.02 & 0.02 & 0.03 & 0.03 & 0.03 \\
    Flip     & 0.78 & 0.12 & 0.22 & 0.03 & 0.03 & 0.03 \\
    Pair     & 0.82 & 0.75 & 0.45 & 0.17 & 0.12 & 0.05 \\
    ReNellm  & 0.61 & 0.00 & 0.00 & 0.00 & 0.00 & 0.00 \\
    \midrule
    \begin{tabular}[c]{@{}l@{}} Trigger \\ Removal \end{tabular}     & 0.77 & 0.78 & 0.62 & 0.52 & 0.42 & 0.30 \\
    \bottomrule
    \end{tabular}
    \end{sc}
    \end{small}
    \end{center}
    \vskip -0.2in
\end{table}


\renewcommand\cellset{\renewcommand\arraystretch{0.7}}
\begin{figure*}
    \centering
    \resizebox{\textwidth}{!}{
    \begin{tabular}{c|c|c}
    \toprule
    \scriptsize \textbf{} & \small \textbf{$2\times$ length extrapolation} & 
    \small \makecell{\textbf{$2\times$ spatial extrapolation}}  \\ \midrule 
    \multirow{2}{*}{\makecell[t]{\small \textbf{Normal} \\ \textbf{length}}} 
    &
    \begin{minipage}{0.75\textwidth}
    \centering
\includegraphics[width=0.95\textwidth]{images/challenge/ref_vid.pdf}
    %\vspace{.1cm}
    \end{minipage}
    & 
    \begin{minipage}{0.2\textwidth}
    \centering   
    \includegraphics[height=0.25\textwidth]{images/challenge/ref_img.pdf}
    %%\vspace{.1cm}
    \end{minipage}
    \\ 
    & \small{Video of $49$ frames} &  \small{Image of 1K resolution} \\ \midrule
    % Extrapolation 
    \multirow{2}{*}{\makecell[t]{\small\textbf{PE}}} &
    \begin{minipage}{0.75\textwidth}
    \centering
    \includegraphics[width=0.95\textwidth]{images/challenge/PE_vid.pdf}
    %\vspace{.1cm}
    \end{minipage}
    & 
    \begin{minipage}{0.15\textwidth}
    \centering   
    \includegraphics[height=0.68\textwidth]{images/challenge/PE_img.pdf}
    %\vspace{.1cm}
    \end{minipage}
    \\
     &  \small{(a) Temporal repetition} & \small{(d) Spatial repetition} \\ 
    % Interpolation 
    \multirow{2}{*}{\makecell[t]{\small\textbf{PI}}} &
    \begin{minipage}{0.75\textwidth}
    %\vspace{.1cm}
    \centering
    \includegraphics[width=0.95\textwidth]{images/challenge/PI_vid.pdf}
    %\vspace{.1cm}
    \end{minipage}
    & 
    \begin{minipage}{0.15\textwidth}
    \centering 
    %\vspace{.1cm}
    \includegraphics[width=0.68\textwidth]{images/challenge/PI_img.pdf}
    %\vspace{.1cm}
    \end{minipage}
    \\
     &  \small{(b) Slower motion} & \small{(e)  Blurred details
} \\ 
    % Time-Aware RoPE 
    \multirow{2}{*}{\makecell[t]{\small\textbf{NTK}}} &
    \begin{minipage}{0.75\textwidth}
     %\vspace{.1cm}
     \centering
    \includegraphics[width=0.95\textwidth, height=50pt]{images/challenge/TA_vid.pdf}
     %\vspace{.1cm}
     \end{minipage}
    & 
    \begin{minipage}{0.15\textwidth}
    \centering   
     %\vspace{.1cm}
    \includegraphics[width=0.68\textwidth]{images/challenge/TA_img.pdf}
     %\vspace{.1cm}
    \end{minipage}
    \\
     &  \small{(c) Temporal repetition} & \small{(f) Spatial repetition} \\ \bottomrule
    \end{tabular}
    }
    \caption{\textbf{Visualization of existing methods for 2$\times$ extrapolation in video and image generation.} The base models CogVideoX-5B~\cite{yang2024cogvideox} and Lumina-Next~\cite{zhuo2024lumina} are trained to sample videos of up to 49 frames and images of up to 1K resolution, respectively. Existing methods lead to \textit{temporal repetition} or \textit{slower motion} in video extrapolation and \textit{spatial repetition} or \textit{blurred content} in image extrapolation, respectively. Please refer to Appendix~\ref{sec: existing failure} for more results and details. 
    }
    \label{fig:challenge}
    %%\vspace{-0.5cm}
\end{figure*}

\vspace{-0.1in}
\section*{Acknowledgments}
We thank Haoran Wang, Jingyu Zhou, and Shuai Wang for their contribution in a tutorial related to this review paper.
% This should be a simple paragraph before the References to thank those individuals and institutions who have supported your work on this article.
\vspace{-0.1in}

\bibliographystyle{IEEEtran}
\bibliography{refs}
% \printbibliography


% \newpage

% \section{Biography Section}
% If you have an EPS/PDF photo (graphicx package needed), extra braces are
%  needed around the contents of the optional argument to biography to prevent
%  the LaTeX parser from getting confused when it sees the complicated
%  $\backslash${\tt{includegraphics}} command within an optional argument. (You can create
%  your own custom macro containing the $\backslash${\tt{includegraphics}} command to make things
%  simpler here.)
 
% \vspace{11pt}

% \bf{If you include a photo:}\vspace{-33pt}
% \begin{IEEEbiography}[{\includegraphics[width=1in,height=1.25in,clip,keepaspectratio]{author_photos/ywg.jpg}}]{Yiwei Guo}
% % Use $\backslash${\tt{begin\{IEEEbiography\}}} and then for the 1st argument use $\backslash${\tt{includegraphics}} to declare and link the author photo.
% % Use the author name as the 3rd argument followed by the biography text.
% received his B.E. degree of Artificial Intelligence from Shanghai Jiao Tong University, China, in 2023. He is currently a Ph.D. student in Computer Science and Engineering in Shanghai Jiao Tong University, under the supervision of Prof. Kai Yu. 
% His current research interests include text-to-speech synthesis and generative models.
% \end{IEEEbiography}

% \begin{IEEEbiography}[{\includegraphics[width=1in,height=1.25in,clip,keepaspectratio]{author_photos/xiechen.jpg}}]{Xie Chen} ?

% \end{IEEEbiography}

% \begin{IEEEbiography}[{\includegraphics[width=1in,height=1.25in,clip,keepaspectratio]{author_photos/kaiyu.jpg}}]{Kai Yu} ?

% \end{IEEEbiography}

% \vspace{11pt}

% \bf{If you will not include a photo:}\vspace{-33pt}
% \begin{IEEEbiographynophoto}{John Doe}
% Use $\backslash${\tt{begin\{IEEEbiographynophoto\}}} and the author name as the argument followed by the biography text.
% \end{IEEEbiographynophoto}


\vfill

\end{document}


