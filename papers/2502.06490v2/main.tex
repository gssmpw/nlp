\documentclass[journal]{IEEEtran}
% \documentclass[lettersize,journal]{IEEEtran}
% \documentclass[10pt,journal,compsoc]{IEEEtran}
\usepackage{amsmath,amsfonts}
\usepackage{algorithmic}
\usepackage{algorithm}
\usepackage{array}
\usepackage[caption=false,font=normalsize,labelfont=sf,textfont=sf]{subfig}
\usepackage{textcomp}
\usepackage{stfloats}
\usepackage{url}
\usepackage{verbatim}
\usepackage{makecell}
\usepackage{graphicx}
\usepackage{multirow}
\usepackage{multicol}
\usepackage{booktabs}
\usepackage{xcolor}
\usepackage{hyperref}
\usepackage{cite}

\usepackage{enumitem}
\usepackage{amssymb}
\usepackage{bm}
\usepackage{subcaption} % Use subcaption for subfigures

\usepackage{tikz}
\usetikzlibrary{mindmap}
\usepackage{smartdiagram}
\usesmartdiagramlibrary{additions}
\usepackage{forest}
\usetikzlibrary{shadows}
\usepackage{relsize}
% Colors images
\usepackage{color}
\definecolor{lightcoral}{rgb}{0.94, 0.5, 0.5}
\definecolor{lightgreen}{rgb}{0.56, 0.93, 0.56}
\definecolor{lightyellow}{rgb}{0.94, 0.84, 0.6}
\definecolor{brightlavender}{rgb}{0.75, 0.58, 0.89}

\definecolor{skyblue}{rgb}{0.53, 0.81, 0.92}
\definecolor{peachpuff}{rgb}{1.0, 0.85, 0.73}
\definecolor{goldenrod}{rgb}{0.85, 0.65, 0.13}
\definecolor{orchid}{rgb}{0.85, 0.44, 0.84}
\definecolor{salmon}{rgb}{0.98, 0.5, 0.45}
\definecolor{turquoise}{rgb}{0.25, 0.88, 0.82}
\definecolor{plum}{rgb}{0.87, 0.63, 0.87}
\definecolor{khaki}{rgb}{0.94, 0.9, 0.55}
\definecolor{slateblue}{rgb}{0.42, 0.35, 0.8}
\definecolor{forestgreen}{rgb}{0.13, 0.55, 0.13}
\definecolor{midnightblue}{rgb}{0.1, 0.1, 0.44}
\definecolor{lightsteelblue}{rgb}{0.69, 0.77, 0.87}

\definecolor{limegreen}{rgb}{0.2, 0.8, 0.2}
\definecolor{palegreen}{rgb}{0.6, 0.98, 0.6}
\definecolor{springgreen}{rgb}{0.0, 1.0, 0.5}
\definecolor{mediumseagreen}{rgb}{0.24, 0.7, 0.44}
\definecolor{seagreen}{rgb}{0.18, 0.55, 0.34}
\definecolor{yellowgreen}{rgb}{0.6, 0.8, 0.2}
\definecolor{olivedrab}{rgb}{0.42, 0.56, 0.14}
\definecolor{darkseagreen}{rgb}{0.56, 0.74, 0.56}
\definecolor{lightseagreen}{rgb}{0.13, 0.7, 0.67}
\definecolor{forestgreen}{rgb}{0.13, 0.55, 0.13}
\definecolor{darkolivegreen}{rgb}{0.33, 0.42, 0.18}
\definecolor{greenyellow}{rgb}{0.68, 1.0, 0.18}
\definecolor{chartreuse}{rgb}{0.5, 1.0, 0.0}
\definecolor{mintgreen}{rgb}{0.6, 1.0, 0.6}

\hyphenation{op-tical net-works semi-conduc-tor IEEE-Xplore}
% updated with editorial comments 8/9/2021

\begin{document}

\title{Recent Advances in Discrete Speech Tokens: A Review}
% ,~\IEEEmembership{Student Member,~IEEE}
\author{Yiwei Guo, Zhihan Li, Hankun Wang,  Bohan Li, Chongtian Shao, Hanglei Zhang, Chenpeng Du, Xie Chen, Shujie Liu, Kai Yu
% ,~\IEEEmembership{Senior Member,~IEEE}
        % <-this % stops a space
% \author{Zhihan Li,~\IEEEmembership{Student Member,~IEEE}}
\thanks{Corresponding Author: Kai Yu. Email: kai.yu@sjtu.edu.cn}% <-this % stops a space
\thanks{Yiwei Guo, Zhihan Li, Bohan Li, Chongtian Shao, Hanglei Zhang, Hankun Wang, Chenpeng Du, Xie Chen and Kai Yu are with the MoE Key Lab of Artificial Intelligence, AI Institute; X-LANCE Lab, Department of Computer Science and Engineering, Shanghai Jiao Tong University,
Shanghai, China. Email: yiwei.guo@sjtu.edu.cn}% <-this % stops a space
% \thanks{Shuai Wang is with Shenzhen Research Institute of Big Data, Shenzhen, China.}
\thanks{Shujie Liu is with Microsoft Research Asia (MSRA), Beijing 100080, China. 
% Email: shujliu@microsoft.com
}
% \thanks{Manuscript received April 19, 2021; revised August 16, 2021.}
}

% The paper headers
% \markboth{Journal of \LaTeX\ Class Files,~Vol.~14, No.~8, August~2021}%
% {Shell \MakeLowercase{\textit{et al.}}: A Sample Article Using IEEEtran.cls for IEEE Journals}

% \IEEEpubid{0000--0000/00\$00.00~\copyright~2021 IEEE}
% Remember, if you use this you must call \IEEEpubidadjcol in the second
% column for its text to clear the IEEEpubid mark.

\maketitle

\begin{abstract}
% With the rapid development of speech generation in the era of large language models (LLMs), discrete speech tokens have emerged as a fundamental representation for speech.
% These tokens are discrete, short and compact representations not only favorable for transmission and storage, but also convenient for incorporating speech into the language modeling paradigm.
% There are two major categories of discrete speech tokens, acoustic and semantic tokens, each of which has evolved into rich research fields with various motivations, designs and methods.
% This survey provides a comprehensive overview of the existing taxonomy and recent advances in building discrete speech tokens, takes an in-depth look at the pros and cons in each research direction, and presents unified experimental comparisons for different types of tokens.
% We also discuss and summarize existing challenges in academia, with which we anticipate to provide inspirations for the future development of discrete speech tokens. 
The rapid advancement of speech generation technologies in the era of large language models (LLMs) has established discrete speech tokens as a foundational paradigm for speech representation. These tokens, characterized by their discrete, compact, and concise nature, are not only advantageous for efficient transmission and storage, but also inherently compatible with the language modeling framework, enabling seamless integration of speech into text-dominated LLM architectures. Current research categorizes discrete speech tokens into two principal classes: \textit{acoustic} tokens and \textit{semantic} tokens, each of which has evolved into a rich research domain characterized by unique design philosophies and methodological approaches. This survey systematically synthesizes the existing taxonomy and recent innovations in discrete speech tokenization, conducts a critical examination of the strengths and limitations of each paradigm, and presents systematic experimental comparisons across token types. Furthermore, we identify persistent challenges in the field and propose potential research directions, aiming to offer actionable insights to inspire future advancements in the development and application of discrete speech tokens.
\end{abstract}

\begin{IEEEkeywords}
Discrete speech tokens, neural audio codec, speech tokenizer, speech LLMs, spoken language modeling, speech generation, acoustic tokens, semantic tokens
\end{IEEEkeywords}

\IEEEpubidadjcol

\section{Introduction}
\label{sec:introduction}
The business processes of organizations are experiencing ever-increasing complexity due to the large amount of data, high number of users, and high-tech devices involved \cite{martin2021pmopportunitieschallenges, beerepoot2023biggestbpmproblems}. This complexity may cause business processes to deviate from normal control flow due to unforeseen and disruptive anomalies \cite{adams2023proceddsriftdetection}. These control-flow anomalies manifest as unknown, skipped, and wrongly-ordered activities in the traces of event logs monitored from the execution of business processes \cite{ko2023adsystematicreview}. For the sake of clarity, let us consider an illustrative example of such anomalies. Figure \ref{FP_ANOMALIES} shows a so-called event log footprint, which captures the control flow relations of four activities of a hypothetical event log. In particular, this footprint captures the control-flow relations between activities \texttt{a}, \texttt{b}, \texttt{c} and \texttt{d}. These are the causal ($\rightarrow$) relation, concurrent ($\parallel$) relation, and other ($\#$) relations such as exclusivity or non-local dependency \cite{aalst2022pmhandbook}. In addition, on the right are six traces, of which five exhibit skipped, wrongly-ordered and unknown control-flow anomalies. For example, $\langle$\texttt{a b d}$\rangle$ has a skipped activity, which is \texttt{c}. Because of this skipped activity, the control-flow relation \texttt{b}$\,\#\,$\texttt{d} is violated, since \texttt{d} directly follows \texttt{b} in the anomalous trace.
\begin{figure}[!t]
\centering
\includegraphics[width=0.9\columnwidth]{images/FP_ANOMALIES.png}
\caption{An example event log footprint with six traces, of which five exhibit control-flow anomalies.}
\label{FP_ANOMALIES}
\end{figure}

\subsection{Control-flow anomaly detection}
Control-flow anomaly detection techniques aim to characterize the normal control flow from event logs and verify whether these deviations occur in new event logs \cite{ko2023adsystematicreview}. To develop control-flow anomaly detection techniques, \revision{process mining} has seen widespread adoption owing to process discovery and \revision{conformance checking}. On the one hand, process discovery is a set of algorithms that encode control-flow relations as a set of model elements and constraints according to a given modeling formalism \cite{aalst2022pmhandbook}; hereafter, we refer to the Petri net, a widespread modeling formalism. On the other hand, \revision{conformance checking} is an explainable set of algorithms that allows linking any deviations with the reference Petri net and providing the fitness measure, namely a measure of how much the Petri net fits the new event log \cite{aalst2022pmhandbook}. Many control-flow anomaly detection techniques based on \revision{conformance checking} (hereafter, \revision{conformance checking}-based techniques) use the fitness measure to determine whether an event log is anomalous \cite{bezerra2009pmad, bezerra2013adlogspais, myers2018icsadpm, pecchia2020applicationfailuresanalysispm}. 

The scientific literature also includes many \revision{conformance checking}-independent techniques for control-flow anomaly detection that combine specific types of trace encodings with machine/deep learning \cite{ko2023adsystematicreview, tavares2023pmtraceencoding}. Whereas these techniques are very effective, their explainability is challenging due to both the type of trace encoding employed and the machine/deep learning model used \cite{rawal2022trustworthyaiadvances,li2023explainablead}. Hence, in the following, we focus on the shortcomings of \revision{conformance checking}-based techniques to investigate whether it is possible to support the development of competitive control-flow anomaly detection techniques while maintaining the explainable nature of \revision{conformance checking}.
\begin{figure}[!t]
\centering
\includegraphics[width=\columnwidth]{images/HIGH_LEVEL_VIEW.png}
\caption{A high-level view of the proposed framework for combining \revision{process mining}-based feature extraction with dimensionality reduction for control-flow anomaly detection.}
\label{HIGH_LEVEL_VIEW}
\end{figure}

\subsection{Shortcomings of \revision{conformance checking}-based techniques}
Unfortunately, the detection effectiveness of \revision{conformance checking}-based techniques is affected by noisy data and low-quality Petri nets, which may be due to human errors in the modeling process or representational bias of process discovery algorithms \cite{bezerra2013adlogspais, pecchia2020applicationfailuresanalysispm, aalst2016pm}. Specifically, on the one hand, noisy data may introduce infrequent and deceptive control-flow relations that may result in inconsistent fitness measures, whereas, on the other hand, checking event logs against a low-quality Petri net could lead to an unreliable distribution of fitness measures. Nonetheless, such Petri nets can still be used as references to obtain insightful information for \revision{process mining}-based feature extraction, supporting the development of competitive and explainable \revision{conformance checking}-based techniques for control-flow anomaly detection despite the problems above. For example, a few works outline that token-based \revision{conformance checking} can be used for \revision{process mining}-based feature extraction to build tabular data and develop effective \revision{conformance checking}-based techniques for control-flow anomaly detection \cite{singh2022lapmsh, debenedictis2023dtadiiot}. However, to the best of our knowledge, the scientific literature lacks a structured proposal for \revision{process mining}-based feature extraction using the state-of-the-art \revision{conformance checking} variant, namely alignment-based \revision{conformance checking}.

\subsection{Contributions}
We propose a novel \revision{process mining}-based feature extraction approach with alignment-based \revision{conformance checking}. This variant aligns the deviating control flow with a reference Petri net; the resulting alignment can be inspected to extract additional statistics such as the number of times a given activity caused mismatches \cite{aalst2022pmhandbook}. We integrate this approach into a flexible and explainable framework for developing techniques for control-flow anomaly detection. The framework combines \revision{process mining}-based feature extraction and dimensionality reduction to handle high-dimensional feature sets, achieve detection effectiveness, and support explainability. Notably, in addition to our proposed \revision{process mining}-based feature extraction approach, the framework allows employing other approaches, enabling a fair comparison of multiple \revision{conformance checking}-based and \revision{conformance checking}-independent techniques for control-flow anomaly detection. Figure \ref{HIGH_LEVEL_VIEW} shows a high-level view of the framework. Business processes are monitored, and event logs obtained from the database of information systems. Subsequently, \revision{process mining}-based feature extraction is applied to these event logs and tabular data input to dimensionality reduction to identify control-flow anomalies. We apply several \revision{conformance checking}-based and \revision{conformance checking}-independent framework techniques to publicly available datasets, simulated data of a case study from railways, and real-world data of a case study from healthcare. We show that the framework techniques implementing our approach outperform the baseline \revision{conformance checking}-based techniques while maintaining the explainable nature of \revision{conformance checking}.

In summary, the contributions of this paper are as follows.
\begin{itemize}
    \item{
        A novel \revision{process mining}-based feature extraction approach to support the development of competitive and explainable \revision{conformance checking}-based techniques for control-flow anomaly detection.
    }
    \item{
        A flexible and explainable framework for developing techniques for control-flow anomaly detection using \revision{process mining}-based feature extraction and dimensionality reduction.
    }
    \item{
        Application to synthetic and real-world datasets of several \revision{conformance checking}-based and \revision{conformance checking}-independent framework techniques, evaluating their detection effectiveness and explainability.
    }
\end{itemize}

The rest of the paper is organized as follows.
\begin{itemize}
    \item Section \ref{sec:related_work} reviews the existing techniques for control-flow anomaly detection, categorizing them into \revision{conformance checking}-based and \revision{conformance checking}-independent techniques.
    \item Section \ref{sec:abccfe} provides the preliminaries of \revision{process mining} to establish the notation used throughout the paper, and delves into the details of the proposed \revision{process mining}-based feature extraction approach with alignment-based \revision{conformance checking}.
    \item Section \ref{sec:framework} describes the framework for developing \revision{conformance checking}-based and \revision{conformance checking}-independent techniques for control-flow anomaly detection that combine \revision{process mining}-based feature extraction and dimensionality reduction.
    \item Section \ref{sec:evaluation} presents the experiments conducted with multiple framework and baseline techniques using data from publicly available datasets and case studies.
    \item Section \ref{sec:conclusions} draws the conclusions and presents future work.
\end{itemize}

\section{Overview}

\revision{In this section, we first explain the foundational concept of Hausdorff distance-based penetration depth algorithms, which are essential for understanding our method (Sec.~\ref{sec:preliminary}).
We then provide a brief overview of our proposed RT-based penetration depth algorithm (Sec.~\ref{subsec:algo_overview}).}



\section{Preliminaries }
\label{sec:Preliminaries}

% Before we introduce our method, we first overview the important basics of 3D dynamic human modeling with Gaussian splatting. Then, we discuss the diffusion-based 3d generation techniques, and how they can be applied to human modeling.
% \ZY{I stopp here. TBC.}
% \subsection{Dynamic human modeling with Gaussian splatting}
\subsection{3D Gaussian Splatting}
3D Gaussian splatting~\cite{kerbl3Dgaussians} is an explicit scene representation that allows high-quality real-time rendering. The given scene is represented by a set of static 3D Gaussians, which are parameterized as follows: Gaussian center $x\in {\mathbb{R}^3}$, color $c\in {\mathbb{R}^3}$, opacity $\alpha\in {\mathbb{R}}$, spatial rotation in the form of quaternion $q\in {\mathbb{R}^4}$, and scaling factor $s\in {\mathbb{R}^3}$. Given these properties, the rendering process is represented as:
\begin{equation}
  I = Splatting(x, c, s, \alpha, q, r),
  \label{eq:splattingGA}
\end{equation}
where $I$ is the rendered image, $r$ is a set of query rays crossing the scene, and $Splatting(\cdot)$ is a differentiable rendering process. We refer readers to Kerbl et al.'s paper~\cite{kerbl3Dgaussians} for the details of Gaussian splatting. 



% \ZY{I would suggest move this part to the method part.}
% GaissianAvatar is a dynamic human generation model based on Gaussian splitting. Given a sequence of RGB images, this method utilizes fitted SMPLs and sampled points on its surface to obtain a pose-dependent feature map by a pose encoder. The pose-dependent features and a geometry feature are fed in a Gaussian decoder, which is employed to establish a functional mapping from the underlying geometry of the human form to diverse attributes of 3D Gaussians on the canonical surfaces. The parameter prediction process is articulated as follows:
% \begin{equation}
%   (\Delta x,c,s)=G_{\theta}(S+P),
%   \label{eq:gaussiandecoder}
% \end{equation}
%  where $G_{\theta}$ represents the Gaussian decoder, and $(S+P)$ is the multiplication of geometry feature S and pose feature P. Instead of optimizing all attributes of Gaussian, this decoder predicts 3D positional offset $\Delta{x} \in {\mathbb{R}^3}$, color $c\in\mathbb{R}^3$, and 3D scaling factor $ s\in\mathbb{R}^3$. To enhance geometry reconstruction accuracy, the opacity $\alpha$ and 3D rotation $q$ are set to fixed values of $1$ and $(1,0,0,0)$ respectively.
 
%  To render the canonical avatar in observation space, we seamlessly combine the Linear Blend Skinning function with the Gaussian Splatting~\cite{kerbl3Dgaussians} rendering process: 
% \begin{equation}
%   I_{\theta}=Splatting(x_o,Q,d),
%   \label{eq:splatting}
% \end{equation}
% \begin{equation}
%   x_o = T_{lbs}(x_c,p,w),
%   \label{eq:LBS}
% \end{equation}
% where $I_{\theta}$ represents the final rendered image, and the canonical Gaussian position $x_c$ is the sum of the initial position $x$ and the predicted offset $\Delta x$. The LBS function $T_{lbs}$ applies the SMPL skeleton pose $p$ and blending weights $w$ to deform $x_c$ into observation space as $x_o$. $Q$ denotes the remaining attributes of the Gaussians. With the rendering process, they can now reposition these canonical 3D Gaussians into the observation space.



\subsection{Score Distillation Sampling}
Score Distillation Sampling (SDS)~\cite{poole2022dreamfusion} builds a bridge between diffusion models and 3D representations. In SDS, the noised input is denoised in one time-step, and the difference between added noise and predicted noise is considered SDS loss, expressed as:

% \begin{equation}
%   \mathcal{L}_{SDS}(I_{\Phi}) \triangleq E_{t,\epsilon}[w(t)(\epsilon_{\phi}(z_t,y,t)-\epsilon)\frac{\partial I_{\Phi}}{\partial\Phi}],
%   \label{eq:SDSObserv}
% \end{equation}
\begin{equation}
    \mathcal{L}_{\text{SDS}}(I_{\Phi}) \triangleq \mathbb{E}_{t,\epsilon} \left[ w(t) \left( \epsilon_{\phi}(z_t, y, t) - \epsilon \right) \frac{\partial I_{\Phi}}{\partial \Phi} \right],
  \label{eq:SDSObservGA}
\end{equation}
where the input $I_{\Phi}$ represents a rendered image from a 3D representation, such as 3D Gaussians, with optimizable parameters $\Phi$. $\epsilon_{\phi}$ corresponds to the predicted noise of diffusion networks, which is produced by incorporating the noise image $z_t$ as input and conditioning it with a text or image $y$ at timestep $t$. The noise image $z_t$ is derived by introducing noise $\epsilon$ into $I_{\Phi}$ at timestep $t$. The loss is weighted by the diffusion scheduler $w(t)$. 
% \vspace{-3mm}

\subsection{Overview of the RTPD Algorithm}\label{subsec:algo_overview}
Fig.~\ref{fig:Overview} presents an overview of our RTPD algorithm.
It is grounded in the Hausdorff distance-based penetration depth calculation method (Sec.~\ref{sec:preliminary}).
%, similar to that of Tang et al.~\shortcite{SIG09HIST}.
The process consists of two primary phases: penetration surface extraction and Hausdorff distance calculation.
We leverage the RTX platform's capabilities to accelerate both of these steps.

\begin{figure*}[t]
    \centering
    \includegraphics[width=0.8\textwidth]{Image/overview.pdf}
    \caption{The overview of RT-based penetration depth calculation algorithm overview}
    \label{fig:Overview}
\end{figure*}

The penetration surface extraction phase focuses on identifying the overlapped region between two objects.
\revision{The penetration surface is defined as a set of polygons from one object, where at least one of its vertices lies within the other object. 
Note that in our work, we focus on triangles rather than general polygons, as they are processed most efficiently on the RTX platform.}
To facilitate this extraction, we introduce a ray-tracing-based \revision{Point-in-Polyhedron} test (RT-PIP), significantly accelerated through the use of RT cores (Sec.~\ref{sec:RT-PIP}).
This test capitalizes on the ray-surface intersection capabilities of the RTX platform.
%
Initially, a Geometry Acceleration Structure (GAS) is generated for each object, as required by the RTX platform.
The RT-PIP module takes the GAS of one object (e.g., $GAS_{A}$) and the point set of the other object (e.g., $P_{B}$).
It outputs a set of points (e.g., $P_{\partial B}$) representing the penetration region, indicating their location inside the opposing object.
Subsequently, a penetration surface (e.g., $\partial B$) is constructed using this point set (e.g., $P_{\partial B}$) (Sec.~\ref{subsec:surfaceGen}).
%
The generated penetration surfaces (e.g., $\partial A$ and $\partial B$) are then forwarded to the next step. 

The Hausdorff distance calculation phase utilizes the ray-surface intersection test of the RTX platform (Sec.~\ref{sec:RT-Hausdorff}) to compute the Hausdorff distance between two objects.
We introduce a novel Ray-Tracing-based Hausdorff DISTance algorithm, RT-HDIST.
It begins by generating GAS for the two penetration surfaces, $P_{\partial A}$ and $P_{\partial B}$, derived from the preceding step.
RT-HDIST processes the GAS of a penetration surface (e.g., $GAS_{\partial A}$) alongside the point set of the other penetration surface (e.g., $P_{\partial B}$) to compute the penetration depth between them.
The algorithm operates bidirectionally, considering both directions ($\partial A \to \partial B$ and $\partial B \to \partial A$).
The final penetration depth between the two objects, A and B, is determined by selecting the larger value from these two directional computations.

%In the Hausdorff distance calculation step, we compute the Hausdorff distance between given two objects using a ray-surface-intersection test. (Sec.~\ref{sec:RT-Hausdorff}) Initially, we construct the GAS for both $\partial A$ and $\partial B$ to utilize the RT-core effectively. The RT-based Hausdorff distance algorithms then determine the Hausdorff distance by processing the GAS of one object (e.g. $GAS_{\partial A}$) and set of the vertices of the other (e.g. $P_{\partial B}$). Following the Hausdorff distance definition (Eq.~\ref{equation:hausdorff_definition}), we compute the Hausdorff distance to both directions ($\partial A \to \partial B$) and ($\partial B \to \partial A$). As a result, the bigger one is the final Hausdorff distance, and also it is the penetration depth between input object $A$ and $B$.


%the proposed RT-based penetration depth calculation pipeline.
%Our proposed methods adopt Tang's Hausdorff-based penetration depth methods~\cite{SIG09HIST}. The pipeline is divided into the penetration surface extraction step and the Hausdorff distance calculation between the penetration surface steps. However, since Tang's approach is not suitable for the RT platform in detail, we modified and applied it with appropriate methods.

%The penetration surface extraction step is extracting overlapped surfaces on other objects. To utilize the RT core, we use the ray-intersection-based PIP(Point-In-Polygon) algorithms instead of collision detection between two objects which Tang et al.~\cite{SIG09HIST} used. (Sec.~\ref{sec:RT-PIP})
%RT core-based PIP test uses a ray-surface intersection test. For purpose this, we generate the GAS(Geometry Acceleration Structure) for each object. RT core-based PIP test takes the GAS of one object (e.g. $GAS_{A}$) and a set of vertex of another one (e.g. $P_{B}$). Then this computes the penetrated vertex set of another one (e.g. $P_{\partial B}$). To calculate the Hausdorff distance, these vertex sets change to objects constructed by penetrated surface (e.g. $\partial B$). Finally, the two generated overlapped surface objects $\partial A$ and $\partial B$ are used in the Hausdorff distance calculation step.

\section{Pre-requisites: Discrete Representation Learning}
\label{sec:prereq}

Discrete speech tokens are obtained through the quantization of continuous representations, which is usually achieved by offline clustering or online vector quantization algorithms.
% In this section, we briefly introduce the commonly used quantization methods in current discrete speech tokens as a preliminary.
This section provides a concise overview of the existing quantization methods commonly used in discrete speech tokens.

Denote $\bm x\in \mathbb R^d$ as a vector in the $d$-dimensional continuous space. A quantization process $q$ transforms $\bm x$ into a discrete \textit{token} in a finite set, i.e. $q(\bm x): \mathbb R^d\to\{1,2,...,V\}$ where $V$ is the \textit{vocabulary size}.
The output tokens are sometimes referred to as \textit{indexes} in the finite $V$-cardinal set.
The function $q$ is usually associated with a \textit{codebook} $\mathcal C=\{\bm c_1,\bm c_2,...,\bm c_V\}$ where every \textit{code-vector} $\bm c_i\in\mathbb R^d$ corresponds to the $i$-th token. 
The code-vectors are representations of tokens in the original $d$-dimensional space.
As $V$ elements can be encoded using $\lceil \log_2 V\rceil$ raw bits\footnote{We use $\lceil z\rceil$ to denote the ceiling of a scalar $z$, i.e., the smallest integer greater than or equal to $z$. 
Similarly, $\lfloor z\rfloor$ denotes the floor of $z$, i.e., the largest integer less than or equal to $z$.}, quantization often compresses the cost for data storage and transmission to a great extent.

\vspace{-0.15in}
\subsection{Offline Clustering}
\vspace{-0.05in}
Clustering is a simple approach for quantization. 
Given a dataset $X=\{\bm x_1,\bm x_2,...\bm x_N\}$, a clustering algorithm aims to assign each sample $\bm x_i$ to a group such that some cost is minimized.
The most frequently used clustering method for discrete speech tokens is k-means clustering~\cite{IKOTUN2023178}, e.g. in GSLM~\cite{lakhotia2021generative}.
K-means is a clustering algorithm based on Euclidean distances.
Its training process iteratively assigns each data sample to the nearest centroid, and moves cluster centroids till convergence, with a pre-defined number of clusters.
After training, the centroids form the codebook, and new data can be quantized to the index of the nearest centroid in this Voronoi partition.
In practice, centroids are usually initialized with the k-means++ algorithm~\cite{kmeans++} for better convergence.

Hierarchical agglomerative clustering has also been used in discrete speech tokens, which iteratively merges the closest clusters.
It is usually applied after k-means to reduce the number of clusters~\cite{cho2024sd,baade2024syllablelm}.
Other clustering algorithms are less explored in the context of discrete speech tokens.

\vspace{-0.12in}
\subsection{Vector Quantization}
\vspace{-0.04in}

\IEEEpubidadjcol

Clustering is often an isolate process, thus cannot be optimized together with other neural network modules.
Instead, vector quantization (VQ)~\cite{gray1984vector} enables a learnable network module that allows gradients to pass through when producing discrete representations.
Autoencoders with a VQ module is termed VQ-VAE~\cite{VQVAE}.
% There are multiple ways a VQ module can be realized:
There are multiple VQ methods:

\subsubsection{K-means VQ}
Like k-means clustering, k-means VQ method finds the code-vector closest to the input, i.e. 
\begin{equation}
    q(\bm x)=\underset{i\in \{1,2,...,V\}}{\arg\min} \|\bm x-\bm c_i\|^2.
\end{equation}
% \footnote{Note that $q$ outputs the code-vector now instead of the codebook index. Although it is a slight abuse of notation, the essence of $q$ is not changed.}
Then, code-vector $\bm c_k\triangleq\bm c_{q(\mathbf x)}$ is fed to subsequent networks.
As the $\min$ operation is not differentiable, straight-through estimators (STEs)~\cite{bengio2013estimating} are usually applied to graft gradients, i.e. $\operatorname{STE}(\bm c_k,\bm x)=\bm x+\operatorname{sg}(\bm c_k-\bm x)$ where $\operatorname{sg(\cdot)}$ stops tracking gradients.
In this way, the input value to subsequent networks is still $\bm c_k$, but gradients are grafted to $\bm x$ in back propagation.
% In this way, the loss function is still calculated by the value of $q(\bm x)$, but gradients that should be placed on $q(\bm x)$ is now grafted to $\bm x$ itself.

Auxiliary loss functions are often used together with k-means VQ~\cite{VQVAE}: commitment loss $\mathcal L_{\text{cmt}}=\|\operatorname{sg}(\bm c_k)-\bm x\|^2$ and codebook loss $\mathcal L_{\text{code}}=\|\operatorname{sg}(\bm x)-\bm c_k\|^2$.
The commitment loss pushes the continuous input $\bm x$ towards the closest codebook entry, while the codebook loss does the opposite and updates the code-vector $\bm c_k$.
The two loss terms are weighted by different factors to put different optimization strengths on $\bm x$ and $\bm c_k$, as pushing $\bm c_k$ towards $\bm x$ is an easier task.
It is also common to replace $\mathcal L_{\text{code}}$ with exponential moving average (EMA) to update the codebook instead~\cite{razavi2019generating}, which does not rely on explicit loss functions.
% Although EMA does not rely on explicit loss functions, it achieves a similar goal that gradually merges the continuous input $\bm x$ into the code-vector $\bm c_k$.
% that $\bm x$ is quantized to.

VQ in high-dimensional spaces is known to suffer from codebook collapse,  where the codebook usage is highly imbalanced~\cite{lancucki2020robust,dhariwal2020jukebox}.
To improve the utilization of codebook, random replacement (as known as \textit{codebook expiration}) can be applied~\cite{dhariwal2020jukebox} on code-vectors that have remained inactive for a long time. 
Other solutions include additional auxiliary constraints such as entropy penalty~\cite{chang2022maskgit,yu2024language}, factorized codebook lookup in low-dimensional space~\cite{yu2022vectorquantized}, and adding a linear projection to update all code-vectors together~\cite{zhu2024addressing}. 
% SimVQ~\cite{zhu2024addressing} explains that the independent updating of some code-vectors causes them to occupy the entire space, which leads to codebook collapse, and using linear layers to update all code-vectors simultaneously can help mitigate this problem.
% SimVQ~\cite{zhu2024addressing} states that codebook collapse stems from the independent optimization trajectories of code-vectors that lead to  spatial dominance, and proposes to add a simple linear projection layer that simultaneously updates all code-vectors per step to mitigate this problem.

\subsubsection{Gumbel VQ}
Instead of quantizing by Euclidean distance, another choice is by probability. 
Gumbel VQ~\cite{jang2017categorical} uses Gumbel-Softmax as a proxy distribution for traditional Softmax to allow differentiable sampling.
Given input $\bm x$ and a codebook of size $V$, a transform $h(\cdot)$ is applied on $\bm x$ into $V$ logits: $\bm l=h(\bm x)\in \mathbb R^V$.
In inference, quantization is performed by choosing the index with the largest logit, i.e. $q(\bm x)=\arg\max_i \left\{\bm l^{(i)}\right\}$.
In training, samples are drawn from the categorical distribution implied by $\bm l$ for the subsequent neural networks.
To achieve efficient sampling and let gradients pass through, Gumbel trick is used:
\begin{align}
    &\bm u\in \mathbb R^V\sim \operatorname{Uniform}(0, 1),\bm v=-\log(-\log(\bm u)) \label{eq:gumbel-noise} \\
    &\bm s=\operatorname{Softmax}((\bm l+\bm v)/\tau) \label{eq:gumbel-softmax}
\end{align}
where Eq.\eqref{eq:gumbel-noise} samples Gumbel noise $\bm v$ element-wise, and Eq.\eqref{eq:gumbel-softmax} calculates Gumbel-Softmax distribution $\bm s$ with a temperature $\tau$.
The forward pass simply use $j=\arg\max_i \{\bm s^{(i)}\}$ as the sampled index, but the true gradient of Gumbel-Softmax is used in backward pass.
In other words, the gradient on the one-hot distribution corresponding to $j$ is grafted to $\bm s$ as an approximate.
The temperature $\tau$ balances the approximation accuracy and gradient variances.
% : a lower $\tau$ results in sharper $\bm s$ and thus more accurate gradient estimate, but higher gradient variances~\cite{jang2017categorical}. 
% In practice, $\tau$ is usually annealed from high to low~\cite{jang2017categorical,vq-wav2vec}.
The transform $h(\cdot)$ is usually parameterized as neural networks, or negatively proportional to Euclidean distances~\cite{jiang2023latent}.

After quantization, code-vector $\bm c_k$ with $k=q(\bm x)$ is fed to subsequent networks.
Gumbel VQ does not require additional losses, since code-vectors can be directly learned with gradients and do not need to be pushed towards $\bm x$.

\subsubsection{Finite Scalar Quantization}
% \textcolor{gray}{VQ in the high-dimensional space is known to suffer from codebook collapse, a phenomenon where only a portion of codebook is active or the codebook usage is highly imbalanced~\cite{lancucki2020robust,dhariwal2020jukebox}.}
As mentioned before, VQ methods based on code-vector assignment usually suffer from codebook collapse. 
Despite many efforts, this remains a crucial challenge.
Finite scalar quantization (FSQ)~\cite{mentzer2024finite} is an alternative to perform quantization in scalar domain.
FSQ quantizes each dimension of a vector $\bm x$ into $L$ levels.
% \footnote{$L$ should better be odd, if following the original FSQ paper~\cite{mentzer2024finite}.}
For the $i$-th dimension $\bm x^{(i)}$, FSQ transforms the values to into limited range and then rounds to integers, i.e. 
\begin{equation}
    q\left(\bm x^{(i)}\right)=\operatorname{round}\left(\lfloor L/2\rfloor\tanh\left(\bm x^{(i)}\right)\right).
\end{equation}
The quantized values are thus integers ranging from $-\lfloor L/2 \rfloor$ to $\lfloor L/2 \rfloor$\footnote{Following \cite{mentzer2024finite}, this is the symmetric case for $L$ being odd. When $L$ is even, there is an offset before rounding to obtain asymmetric quantized values.}.
For a $d$-dimensional vector $\bm x$, there are $L^d$ possible quantization outcomes.
% Common choices are $L\ge 5$ and $d\le 10$, so FSQ usually has a much smaller hidden dimension than VQ (where usually $d\ge 100$).
Hence, FSQ usually requires a much smaller hidden dimension than VQ.
STE is applied to pass gradients.
As quantization is simply done via rounding to integers, there is no explicit codebooks associated with the FSQ process.
% The range of $q$ here is also not a categorical set where indexes cannot be numerically compared as that of VQ, but an ordered set of integers.
% This indicates a different approach when using FSQ instead of VQ for generative tasks.
FSQ is reported to have  better codebook usage\footnote{Although there is no longer a codebook associated with code-vectors, codebook usage can still be measured among all possible $V=L^d$ outcomes.} especially for a large $V$ compared to VQ methods, without the need for auxiliary losses.

\subsubsection{Other VQ Tricks}
In many cases, a single VQ module suffers from a highly-limited representation space, thus results in poor performance compared to continuous counterparts. 
There are some widely-used VQ tricks that introduce multiple quantizers to refine the quantized space, as shown in Fig.\ref{fig:gvq-rvq}:

\begin{figure}
    \centering
    \includegraphics[width=0.53\linewidth]{figs/GVQ.png}
    \includegraphics[width=0.45\linewidth]{figs/RVQ.png}
    \caption{Diagram of GVQ (left) and RVQ (right).}
    \label{fig:gvq-rvq}
\end{figure}

\begin{enumerate}[leftmargin=5mm]
    \item \textit{Grouped VQ (GVQ)}, also known as \textit{product quantization}~\cite{product_quantization}. It groups the input vector $\bm x$ by dimensions and apply VQ on different parts of $\bm x$ independently. They can have different or shared codebooks. The VQ outputs are then concatenated along dimensions to match that of $\bm x$. For instance, GVQ is used in neural word embeddings~\cite{9164982} and speech self-supervised learning models~\cite{vq-wav2vec,baevski2020wav2vec} to achieve efficient quantization.
    \item \textit{Residual VQ (RVQ)}, also known as \textit{multi-stage quantization}~\cite{multiple-stage-vector-quantization}. It adopts a serial approach that iteratively quantizes the residual of the last quantizer.
    Similar to GVQ, RVQ also has multiple quantizers.
    For the $i$-th quantizer $q_i$ with input $\bm x_i$ and output code-vector $\bm c_{k}$, the residual is defined as $\bm x_{i+1}=\bm x_i-\bm c_{k}$.
    The outputs from all $q_i$ are finally summed as the quantized result of $\bm x$.
    In this way, information in the codebooks is supposed to follow a coarse-to-fine order, and more details in the original $\bm x$ can be preserved than a plain quantizer.
    It is used in various speech codecs~\cite{zeghidour2021soundstream,encodec,kumar2024high}, for instance.
\end{enumerate}
GVQ and RVQ can be flexibly combined to form GRVQ~\cite{yang2023hifi} that applies RVQ on each GVQ branch for better codebook utilization.
% A variant of RVQ is cross-scale VQ (CSVQ)~\cite{jiang22_interspeech} where residuals are not defined as the quantization margins directly, but instead 
A network can also contain multiple VQ modules at different places, like cross-scale VQ (CSVQ)~\cite{jiang22_interspeech} where every decoder layer has a quantizer inside.

Note that RVQ naturally produces an order of importance in residual layers, while all quantizers in GVQ are equally important.
Such order of importance can also be enforced in GVQ by a ``nested dropout'' trick~\cite{rippel2014learning}.
% There is also a ``nested dropout'' trick~\cite{rippel2014learning} that assigns an importance order to GVQ, by manually define an order of quantizers and randomly dropping-out the last few quantizers in training.



\vspace{-0.03in}
\section{Speech Tokenization Methods: Acoustic Tokens}
\label{sec:acoustic}

% \begin{table*}[]
% \centering
% \caption{A summary of famous acoustic speech tokens (neural speech codecs). 
% Italic ``\textit{C,T,U}'' denote CNN, Transformer or U-Net-based generator architecture in Fig.\ref{fig:generator}.
% Symbols `/' and `-' denote ``or'' and ``to'' for different model versions, and ``+'' means different configurations in different VQ streams in a single model. 
% $Q,F,V$ mean number of quantizers, frame rate and vocabulary size of each quantizer respectively. 
% For example, ``$Q=2$, $V$=8192+($2^{12}$-$2^{15}$)'' in SemantiCodec means one of the two VQ streams has 8192 possible codes, and the other can vary from $2^{12}$ to $2^{15}$ in different configurations.
% Bitrates are computed by $\frac1{1000}\sum_{i=1}^Q F_i\lceil \log_2 V_i\rceil$ kbps, without entropy coding. }
% \label{tab:acoustic-metadata}
% \resizebox{\textwidth}{!}{
% % if necessary, use "\makecell{A\\B}" to create line break in a cell
% \begin{tabular}{@{}lccccccc@{}}
% \toprule
% \textbf{{Acoustic Speech Tokens}} & \textbf{Model Framework} & \textbf{{Sampling Rate}} & \textbf{Quantization} & $Q$ & \textbf{$F$} & \textbf{$V$} & \textbf{{Bitrate (kbps)}}  \\ \midrule
% \multicolumn{6}{l}{\textbf{\textit{General-purpose acoustic tokens}}} \\
% SoundStream~\cite{zeghidour2021soundstream} & VQ-GAN (\textit{C}) & 24kHz & RVQ & max 24 & 75Hz & 1024 & max 18.00 \\
% EnCodec~\cite{encodec} & VQ-GAN (\textit{C})& 24kHz & RVQ & max 32 & 75Hz & 1024 & max 24.00  \\
% % LMCodec~\cite{LMCodec} & VQ-GAN & 16kHz & RVQ & max 24 & 50Hz & 1024 & max  \\
% TF-Codec~\cite{jiang2023latent} & VQ-GAN (\textit{C}) & 16kHz & GVQ & 3-32 & 25Hz & 512 / 1024 & 0.68-8.00 \\
% Disen-TF-Codec~\cite{jiang2023disentangled} & VQ-GAN (\textit{C}) & 16kHz & GVQ & 2 / 6 & 25Hz & 256 / 1024 & {0.40 / 1.50} \\
% AudioDec~\cite{audiodec} & VQ-VAE (\textit{C})+GAN& 48kHz & RVQ & 8 & 160Hz & 1024 & 12.80 \\
% HiFi-Codec\cite{yang2023hifi} & VQ-GAN (\textit{C})& 16 / 24kHz & GRVQ & 4 & 50-100Hz & 1024 & 2.00-4.00 \\
% DAC~\cite{kumar2024high} & VQ-GAN (\textit{C})& 44.1kHz & RVQ & 9 & 86Hz & 1024 & 7.74 \\
% LaDiffCodec~\cite{yang2024generative} & Latent diffusion & 16kHz & RVQ & 3 / 6 & 50Hz & 1024 & 1.50 / 3.00 \\
% {FreqCodec}~\cite{du2024funcodec} & VQ-GAN (\textit{C})& 16kHz & RVQ & max 32 & 50Hz & 1024 & max 16.00 \\
% TiCodec~\cite{ticodec} & VQ-GAN (\textit{C})& 24kHz & RVQ, GVQ & 1-4 & 75Hz & 1024 & 0.75-3.00  \\
% APCodec~\cite{APCodec} & VQ-GAN (\textit{C})& 48kHz & RVQ & 4 & 150Hz &1024 & 6.00 \\
% % FACodec~\cite{facodec} & 6 & 80Hz & 1024 & 4.80  \\
% % SSVC~\cite{SSVC} & 4 & 50Hz & 512 & 1.80 \\
% % SpeechTokenizer~\cite{zhang2024speechtokenizer} & 8 & 50Hz & 1024 & 4.00 \\
% {SRCodec~\cite{zheng2024srcodec}} & VQ-GAN (\textit{C}) & 16kHz & GRVQ   & 2-8 & 50Hz & 512+1024 & 0.95-3.80\\
% SQ-Codec~\cite{yang24l_interspeech} & VQ-GAN (\textit{C})& 16kHz & FSQ & 32 & 50Hz &19 & 8.00 \\
% Single-Codec~\cite{singlecodec} & VQ-GAN (\textit{T+C})& 24kHz & VQ & 1 & 23Hz & 8192 & 0.30  \\
% ESC~\cite{gu2024esc}  & VQ-GAN (\textit{U})& 16kHz & GVQ & max 18 & 50Hz & 1024 & max 9.00 \\
% CoFi-Codec~\cite{guo2024speaking} & VQ-GAN (\textit{U}) & 16kHz & GVQ & 3 & 8.33+25+50Hz & 16384 & 1.17 \\
% HILCodec~\cite{ahn2024hilcodec} & VQ-GAN (\textit{C})& 24kHz & RVQ & 2-12 & 75Hz & 1024 & 1.50-9.00 \\
% SuperCodec~\cite{zheng2024supercodec} & VQ-GAN (\textit{C})& 16kHz & RVQ & 2-12 & 50Hz & 1024 & 1.00-6.00\\
% SNAC~\cite{Siuzdak_SNAC_Multi-Scale_Neural_2024} & VQ-GAN (\textit{C})& 24kHz & RVQ & 3 &12+23+47Hz & 4096 & 0.98 \\
% dMel~\cite{bai2024dmel} & {Quantizer only} & 16kHz & FSQ & 80 & 80Hz & 16 & 25.60 \\
% WavTokenizer~\cite{ji2024wavtokenizer} & VQ-GAN (\textit{C})& 24kHz & VQ & 1 & 40 / 75Hz & 4096 & 0.48 / 0.90  \\
% BigCodec~\cite{xin2024bigcodec} & VQ-GAN (\textit{C})& 16kHz & VQ & 1 & 80Hz & 8192 & 1.04 \\
% LFSC~\cite{casanova2024low} & VQ-GAN (\textit{C})& 22.05kHz &  FSQ & 8 & 21.5Hz & 2016 & 1.89 \\
% NDVQ~\cite{niu2024ndvq} & VQ-GAN (\textit{C})& 24kHz &  RVQ & max 32 & 75Hz & 1024 & max 24.00 \\
% VRVQ~\cite{chae2024variable} & VQ-GAN (\textit{C})& 44.1kHz & RVQ & 8 & 86Hz & 1024 & {0.26 + max 6.89}\\
% % \textcolor{red}{SimVQ}~\cite{zhu2024addressing} & \\
% TS3-Codec~\cite{wu2024ts3codectransformerbasedsimplestreaming} & VQ-GAN (\textit{T})& 16kHz & VQ & 1 & 40 / 50Hz & $2^{16}$ / $2^{17}$ & 0.64-0.85 \\
% Stable-Codec~\cite{parker2024scalingtransformerslowbitratehighquality} & VQ-GAN (\textit{T})& 16kHz & FSQ  & 6 / 12 & 25Hz & 5 / 6 & 0.40 / 0.70 \\
% FreeCodec~\cite{zheng2024freecodecdisentangledneuralspeech} & VQ-GAN (\textit{C+T})& 16kHz & VQ & 1+1 & 50+7Hz & 256 & 0.45  \\
% \midrule
% \multicolumn{6}{l}{\textbf{\textit{Mixed-objective acoustic tokens: semantic distillation}}} \\
% {Siahkoohi et al.}~\cite{siahkoohi22_interspeech} & VQ-GAN (\textit{C})& 16kHz & RVQ & 2+1 / 2+2 / 6 & 25+50Hz & 64 & 0.60 / 0.90 / 1.80\\
% SpeechTokenizer~\cite{zhang2024speechtokenizer} & VQ-GAN (\textit{C})& 16kHz & RVQ & 8 & 50Hz & 1024 & 4.00 \\
% SemantiCodec~\cite{liu2024semanticodec} & Latent diffusion & 16kHz & VQ & 2 & 12.5-50Hz & {8192+($2^{12}$-$2^{15}$)} & 0.31-1.40 \\
% LLM-Codec~\cite{yang2024uniaudio15} & VQ-GAN (\textit{C})& 16kHz & RVQ & 3 & 8.33+16.67+33.33Hz & 3248+32000+32000& 0.85 \\
% X-Codec~\cite{ye2024codec} & VQ-GAN (\textit{C})& 16kHz & RVQ & max 8 & 50Hz & 1024 & max 4.00 \\
% SoCodec~\cite{guo2024socodec} & VQ-GAN (\textit{C})& 16kHz & GVQ & 1 / 4 / 8 & 25 / 8.3 / 4.2Hz & 16384 & 0.35 / 0.47 \\
% Mimi~\cite{kyutai2024moshi} & VQ-GAN (\textit{C+T})& 24kHz & RVQ & 8 & 12.5Hz & 2048 & 1.10 \\
% \midrule
% \multicolumn{6}{l}{\textbf{\textit{Mixed-objective acoustic tokens: disentanglement}}} \\
% SSVC~\cite{SSVC} & VQ-GAN (\textit{C})& 24kHz & RVQ & 4 & 50Hz & 512 & 1.80 \\
% PromptCodec~\cite{pan2024promptcodec} & VQ-GAN (\textit{C})& 24kHz & GRVQ & 1-4 & 75Hz & 1024 & 0.75-3.00 \\
% FACodec~\cite{facodec} & VQ-GAN (\textit{C})& 16kHz & RVQ & 1+2+3 & 80Hz & 1024 & 4.80 \\
% LSCodec~\cite{guo2024lscodec} & VQ-VAE (\textit{C+T})+GAN& 24kHz &  VQ & 1 & 25 / 50Hz & 1024 / 300& 0.25 / 0.45 \\
% SD-Codec~\cite{bie2024learning} & VQ-GAN (\textit{C})& 16kHz &  RVQ & 12 & 50Hz & 1024 & 6.00 \\
% \bottomrule
% \end{tabular}
% }
% % \vspace{-0.15in}
% \end{table*}


\begin{table*}[]
\centering
\caption{A summary of famous acoustic speech tokens (neural speech codecs). 
Italic ``\textit{C,T,U}'' denote CNN, Transformer or U-Net-based generator architecture in Fig.\ref{fig:generator}.
Symbols `/' and `-' denote ``or'' and ``to'' for different model versions, and ``+'' means different configurations in different VQ streams in a single model. 
$Q,F,V$ mean number of quantizers, frame rate and vocabulary size of each quantizer respectively. 
For example, ``$Q=2$, $V$=8192+($2^{12}$-$2^{15}$)'' in SemantiCodec means one of the two VQ streams has 8192 possible codes, and the other can vary from $2^{12}$ to $2^{15}$ in different configurations.
Bitrates are computed by $\frac1{1000}\sum_{i=1}^Q F_i\lceil \log_2 V_i\rceil$ kbps, without entropy coding. }
\label{tab:acoustic-metadata}
\resizebox{\textwidth}{!}{
% if necessary, use "\makecell{A\\B}" to create line break in a cell
\begin{tabular}{@{}lccccccc@{}}
\toprule
\textbf{{Acoustic Speech Tokens}} & \textbf{Model Framework} & \textbf{\makecell{Sampling\\Rate (kHz)}} & \textbf{\makecell{Quantization\\Method}} & \textbf{$Q$} & \textbf{$F$ (Hz)} & \textbf{$V$} & \textbf{{Bitrate (kbps)}}  \\ \midrule
\multicolumn{6}{l}{\textbf{\textit{General-purpose acoustic tokens}}} \\
SoundStream~\cite{zeghidour2021soundstream} & VQ-GAN (\textit{C}) & 24 & RVQ & max 24 & 75 & 1024 & max 18.00 \\
EnCodec~\cite{encodec} & VQ-GAN (\textit{C})& 24 & RVQ & max 32 & 75 & 1024 & max 24.00  \\
TF-Codec~\cite{jiang2023latent} & VQ-GAN (\textit{C}) & 16 & GVQ & 3-32 & 25 & 512 / 1024 & 0.68-8.00 \\
Disen-TF-Codec~\cite{jiang2023disentangled} & VQ-GAN (\textit{C}) & 16 & GVQ & 2 / 6 & 25 & 256 / 1024 & {0.40 / 1.50} \\
AudioDec~\cite{audiodec} & VQ-VAE (\textit{C})+GAN& 48 & RVQ & 8 & 160 & 1024 & 12.80 \\
HiFi-Codec\cite{yang2023hifi} & VQ-GAN (\textit{C})& 16 / 24 & GRVQ & 4 & 50-100 & 1024 & 2.00-4.00 \\
DAC~\cite{kumar2024high} & VQ-GAN (\textit{C})& 44.1 & RVQ & 9 & 86 & 1024 & 7.74 \\
LaDiffCodec~\cite{yang2024generative} & Latent diffusion & 16 & RVQ & 3 / 6 & 50 & 1024 & 1.50 / 3.00 \\
{FreqCodec}~\cite{du2024funcodec} & VQ-GAN (\textit{C})& 16 & RVQ & max 32 & 50 & 1024 & max 16.00 \\
TiCodec~\cite{ticodec} & VQ-GAN (\textit{C})& 24 & RVQ, GVQ & 1-4 & 75 & 1024 & 0.75-3.00  \\
APCodec~\cite{APCodec} & VQ-GAN (\textit{C})& 48 & RVQ & 4 & 150 &1024 & 6.00 \\
SRCodec~\cite{zheng2024srcodec} & VQ-GAN (\textit{C}) & 16 & GRVQ   & 2-8 & 50 & 512+1024 & 0.95-3.80\\
SQ-Codec~\cite{yang24l_interspeech} & VQ-GAN (\textit{C})& 16 & FSQ & 32 & 50 &19 & 8.00 \\
Single-Codec~\cite{singlecodec} & VQ-GAN (\textit{T+C})& 24 & VQ & 1 & 23 & 8192 & 0.30  \\
ESC~\cite{gu2024esc}  & VQ-GAN (\textit{U})& 16 & GVQ & max 18 & 50 & 1024 & max 9.00 \\
CoFi-Codec~\cite{guo2024speaking} & VQ-GAN (\textit{U}) & 16 & GVQ & 3 & 8.33+25+50 & 16384 & 1.17 \\
HILCodec~\cite{ahn2024hilcodec} & VQ-GAN (\textit{C})& 24 & RVQ & 2-12 & 75 & 1024 & 1.50-9.00 \\
SuperCodec~\cite{zheng2024supercodec} & VQ-GAN (\textit{C})& 16 & RVQ & 2-12 & 50 & 1024 & 1.00-6.00\\
SNAC~\cite{Siuzdak_SNAC_Multi-Scale_Neural_2024} & VQ-GAN (\textit{C})& 24 & RVQ & 3 &12+23+47 & 4096 & 0.98 \\
% dMel~\cite{bai2024dmel} & {Quantizer only} & 16 & FSQ & 80 & 80 & 16 & 25.60 \\
WavTokenizer~\cite{ji2024wavtokenizer} & VQ-GAN (\textit{C})& 24 & VQ & 1 & 40 / 75 & 4096 & 0.48 / 0.90  \\
BigCodec~\cite{xin2024bigcodec} & VQ-GAN (\textit{C})& 16 & VQ & 1 & 80 & 8192 & 1.04 \\
LFSC~\cite{casanova2024low} & VQ-GAN (\textit{C})& 22.05 &  FSQ & 8 & 21.5 & 2016 & 1.89 \\
NDVQ~\cite{niu2024ndvq} & VQ-GAN (\textit{C})& 24 &  RVQ & max 32 & 75 & 1024 & max 24.00 \\
VRVQ~\cite{chae2024variable} & VQ-GAN (\textit{C})& 44.1 & RVQ & 8 & 86 & 1024 & {0.26 + max 6.89}\\
% \textcolor{red}{SimVQ}~\cite{zhu2024addressing} & \\
TS3-Codec~\cite{wu2024ts3codectransformerbasedsimplestreaming} & VQ-GAN (\textit{T})& 16 & VQ & 1 & 40 / 50 & $2^{16}$ / $2^{17}$ & 0.64-0.85 \\
Stable-Codec~\cite{parker2024scalingtransformerslowbitratehighquality} & VQ-GAN (\textit{T})& 16 & FSQ  & 6 / 12 & 25 & 5 / 6 & 0.40 / 0.70 \\
FreeCodec~\cite{zheng2024freecodecdisentangledneuralspeech} & VQ-GAN (\textit{C+T})& 16 & VQ & 1+1 & 50+7 & 256 & 0.45  \\
\midrule
\multicolumn{6}{l}{\textbf{\textit{Mixed-objective acoustic tokens: semantic distillation}}} \\
{Siahkoohi et al.}~\cite{siahkoohi22_interspeech} & VQ-GAN (\textit{C})& 16 & RVQ & 2+1 / 2+2 / 6 & 25+50 & 64 & 0.60 / 0.90 / 1.80\\
SpeechTokenizer~\cite{zhang2024speechtokenizer} & VQ-GAN (\textit{C})& 16 & RVQ & 8 & 50 & 1024 & 4.00 \\
SemantiCodec~\cite{liu2024semanticodec} & Latent diffusion & 16 & VQ & 2 & 12.5-50 & {8192+($2^{12}$-$2^{15}$)} & 0.31-1.40 \\
LLM-Codec~\cite{yang2024uniaudio15} & VQ-GAN (\textit{C})& 16 & RVQ & 3 & 8.33+16.67+33.33 & 3248+32000+32000& 0.85 \\
X-Codec~\cite{ye2024codec} & VQ-GAN (\textit{C})& 16 & RVQ & max 8 & 50 & 1024 & max 4.00 \\
SoCodec~\cite{guo2024socodec} & VQ-GAN (\textit{C})& 16 & GVQ & 1 / 4 / 8 & 25 / 8.3 / 4.2 & 16384 & 0.35 / 0.47 \\
Mimi~\cite{kyutai2024moshi} & VQ-GAN (\textit{C+T})& 24 & RVQ & 8 & 12.5 & 2048 & 1.10 \\
X-Codec 2.0~\cite{ye2025llasa} & VQ-GAN (\textit{C+T}) & 16 & FSQ & 8 & 50 & 4 & 0.80 \\
\midrule
\multicolumn{6}{l}{\textbf{\textit{Mixed-objective acoustic tokens: disentanglement}}} \\
SSVC~\cite{SSVC} & VQ-GAN (\textit{C})& 24 & RVQ & 4 & 50 & 512 & 1.80 \\
PromptCodec~\cite{pan2024promptcodec} & VQ-GAN (\textit{C})& 24 & GRVQ & 1-4 & 75 & 1024 & 0.75-3.00 \\
FACodec~\cite{facodec} & VQ-GAN (\textit{C})& 16 & RVQ & 1+2+3 & 80 & 1024 & 4.80 \\
LSCodec~\cite{guo2024lscodec} & VQ-VAE (\textit{C+T})+GAN& 24 &  VQ & 1 & 25 / 50 & 1024 / 300& 0.25 / 0.45 \\
SD-Codec~\cite{bie2024learning} & VQ-GAN (\textit{C})& 16 &  RVQ & 12 & 50 & 1024 & 6.00 \\
\bottomrule
\end{tabular}
}
\end{table*}

Acoustic tokens, also known as \textit{speech codecs}, refer to the discrete representations optimized mainly for signal compression and reconstruction.
The audio codec technology arises long ago.
Traditional codecs, including 
% AAC~\cite{bosi1997iso}, 
MP3~\cite{rfc5219}, Opus~\cite{Valin2012DefinitionOT} and EVS~\cite{dietz2015overview}, typically take advantage of signal processing algorithms to improve quality and lower the bitrate.

In the deep learning era, numerous codec models based on neural networks have emerged.
% These codec models typically have a encoder-decoder framework that involves a quantization module in the middle.
% These models typically train an encoder that compresses speech signals, and a decoder that recovers speech signals, with a quantizer between the two.
These models typically consist of an encoder that compresses speech signals and a decoder that reconstructs the speech signals, with a quantizer situated between the two.
The quantizer is also parameterized and jointly trained with the whole network in an end-to-end manner. 
% As the purpose of acoustic tokens is signal reconstruction, current acoustic tokens models almost all rely on generative adversarial network (GAN) training criterion, i.e. the encoder-decoder generator tries to fool a set of discriminators by making the reconstructed signals as close to the original as possible.
% Those VQ-VAE model with GAN training criterion is typically referred to as VQ-GAN~\cite{esser2021taming}.
The codebook indices produced by the quantizer are referred to as acoustic tokens.
To improve the representation ability of discrete VQ spaces and thus obtain better codec performance, RVQ, GVQ, GRVQ and FSQ tricks are commonly applied in the quantization module.
% FSQ has also been integrated in acoustic tokens.

We list the VQ method, number of quantizers $Q$, frame rate $F$, vocabulary size $V$ for each quantizer, and the resulting bitrate of existing neural acoustic speech tokens in Table.\ref{tab:acoustic-metadata}.

\subsection{Model Architectures}
\label{sec:acoustic-arch}

\begin{figure}
    \centering
    % \includegraphics[width=0.8\linewidth]{figs/acoustic1.png}
    % \includegraphics[width=0.8\linewidth]{figs/acoustic2.png}
    \includegraphics[width=0.9\linewidth]{figs/acoustic.png}
    \caption{Neural architectures of acoustic tokens.
    % Upper: \textbf{VQ-GAN} type where the quantization module is placed between an encoder and a decoder; Bottom: \textbf{latent diffusion} type where quantized tokens condition the diffusion process towards a latent space learned by an autoencoder.
    Note that inputs and outputs can be waveforms, frequency-domain features or even SSL features depending on purpose and design.}
    \label{fig:acoustic-paradigms}
\end{figure}


Although acoustic codec models differ from one to one regarding their purposes, most of them share a similar encoder-quantizer-decoder framework.
With audio clip $\bm x$ that can either be time-domain sampling points, frequency-domain features or even other machine learning features, an encoder $f_\theta(\cdot)$ transforms it to $f_\theta(\bm x)$ in a continuous latent vector space. 
% For waveform inputs, $f_\theta(\cdot)$ will also downsample
The encoder $f_\theta(\cdot)$ will usually perform downsampling to reduce the temporal length of the input signals, especially for waveform inputs.
A VQ module $q_\phi(\cdot)$ discretizes $f_\theta(\bm x)$ into tokens and corresponding codebook vectors $\bm c$.
A decoder $g_\psi(\cdot)$ then uses $\bm c$ to reconstruct $\hat {\bm x}$, and a certain distance metric of $d(\bm x, \hat{\bm x})$ is usually optimized.
There are two major paradigms for designing the encoder, decoder, and quantizers, which can be summarized as diagrams in Fig.\ref{fig:acoustic-paradigms}.
% \textcolor{red}{Maybe we can merge U-Net into VQ-GAN, since they have nothing different in essence.}

\begin{figure}
    \centering
    % \includegraphics[width=0.8\linewidth]{figs/CNN.png}
    % \includegraphics[width=0.8\linewidth]{figs/transformer.png}
    % \includegraphics[width=0.7\linewidth]{figs/unet.png}
    \includegraphics[width=0.9\linewidth]{figs/acoustic-archs.png}
    \caption{Major generator (VQ-VAE) architectures of VQ-GAN-based acoustic tokens. 
    % Upper: \textbf{CNN-based}; Middle: \textbf{Transformer-based}; Bottom: \textbf{U-Net-based}. 
    ``Q.'' and ``Trans'' are short for quantizer and Transformer, respectively.}
    \label{fig:generator}
    \vspace{-0.2in}
\end{figure}

\subsubsection{VQ-GAN}
VQ-GAN~\cite{esser2021taming} is a very commonly adopted framework of acoustic tokens that trains a VQ-VAE with GAN objectives. 
Besides the original reconstruction and VQ objectives in a VQ-VAE, VQ-GAN uses discriminators $d_\xi(\bm x, \hat{\bm x})$ to distinguish real and reconstructed data that adversarially train the generator network composed of $f_\theta,q_\phi$, and $g_\psi$. In acoustic tokens, there are usually multiple discriminators, e.g. multi-resolution and multi-scale STFT discriminators from the neural vocoder researches~\cite{kumar2019melgan,jang21_interspeech}.
The generator architecture of VQ-GAN-based acoustic tokens has multiple choices, with the three most representative ones visualized in Fig.\ref{fig:generator}: CNN-based, Transformer-based, and U-Net-based.

The CNN-based generator is the most widely used architecture so far in acoustic tokens.
SoundStream~\cite{zeghidour2021soundstream} and EnCodec~\cite{encodec} are two famous early neural acoustic tokens that operate in an end-to-end VQ-GAN manner.
% SoundStream is also the basis for Lyra V2 codec\footnote{\url{https://github.com/google/lyra}}.
They receive time-domain waveforms as inputs and directly reconstruct waveforms.
Their encoder and decoder have a mirrored architecture to perform down and up-samplings.
In SoundStream, the encoder and decoder are purely constructed by convolutional neural networks (CNNs) while EnCodec augments them with an LSTM.
The CNN encoder down-samples the waveform to a high-dimensional embedding sequence, whose frame rate is determined by the sampling rate, CNN kernel sizes and strides at a fixed ratio.
The continuous embeddings are passed to an RVQ quantizer, and the quantized vectors are summed before being transformed to the waveform domain by the CNN decoder.
% Multi-resolution and multi-scale STFT discriminators are applied to distinguish between real and reconstructed speech.
The training criteria include reconstruction loss (in the time and frequency domain), adversarial loss, feature matching loss, and quantization losses for RVQ layers.
To allow for a flexible choice of bitrates, structured dropout is adopted where the number of codebooks in the RVQ module can be randomly chosen~\cite{zeghidour2021soundstream}, such that only a portion of quantizers in front are activated during training.
The acoustic tokens can consequently reside in variable bitrates depending on the chosen number of RVQ quantizers.
The inputs and outputs of the codec model can also be frequency-domain features like magnitude and phase spectra for reducing computation burden~\cite{du2024funcodec}.
There, the convolution kernels are typically 2D instead of 1D in the time-domain codecs.
% \textcolor{red}{Shall training losses be expressed in detail?}

Later, Transformers~\cite{transformer} have been adopted, e.g. in Single-Codec~\cite{singlecodec} and Mimi~\cite{kyutai2024moshi}.
They can be directly applied to frequency-domain inputs and outputs.
When operating on waveform-domain inputs or outputs, a CNN~\cite{kyutai2024moshi} or patchifying~\cite{wu2024ts3codectransformerbasedsimplestreaming,parker2024scalingtransformerslowbitratehighquality} operation is usually added before or after the Transformer blocks.
In Mimi, a shallow Transformer layer is added after the CNN-based encoder, and vice versa in its decoder.
Recently, some propose to use purely Transformer-based backbone and discard the CNN blocks, e.g. TS3-Codec~\cite{wu2024ts3codectransformerbasedsimplestreaming}.
As Transformers demonstrate superior modeling ability and scaling property, these works prove to outperform CNN-based codecs either with less computation~\cite{wu2024ts3codectransformerbasedsimplestreaming} or larger scale~\cite{parker2024scalingtransformerslowbitratehighquality}.
However, to ensure stream-ability, an attention mask should be employed~\cite{kyutai2024moshi}.
The encoder and decoder can also be designed to be different. 
For example, Single-Codec~\cite{singlecodec} uses Conformer~\cite{conformer} encoder and CNN decoder, while LSCodec~\cite{guo2024lscodec} uses the reverse configuration.

% While the most acoustic tokens contain only one quantizer (can be RVQ or GVQ though)
Though RVQ or GVQ is usually applied, most acoustic tokens contain only one quantization module as a whole.
However, there are also U-Net-based codecs where multiple quantizers are employed, e.g. CoFi-Codec~\cite{guo2024speaking} and ESC~\cite{gu2024esc}.
Each sub-encoder or decoder in the U-Net can be a CNN or Transformer.
This offers a more flexible control of the resolution of each VQ stream (Section \ref{sec:multi-resolution}).

It is also noteworthy that training a separate vocoder on top of existing acoustic tokens may result in improved audio quality than the original decoded outputs, since reconstructing waveform alone may be simpler than optimizing VQ representation and reconstruction at the same time.
This is exemplarily verified in AudioDec~\cite{audiodec}, MBD~\cite{san2023discrete} and Vocos~\cite{siuzdak2024vocos}.
Therefore, some acoustic tokens directly simplify the VQ-GAN training objective back to the original VQ-VAE, where the discrete tokens are obtained first by a simple reconstruction loss, and a vocoder is trained as an additional stage, like AudioDec~\cite{audiodec} and LSCodec~\cite{guo2024lscodec}.
These works are denoted as ``VQ-VAE+GAN'' in Table \ref{tab:acoustic-metadata}.
% \textcolor{red}{TODO: maybe add some explanation? Is this because vocoder is larger than decoder?}

\subsubsection{Latent diffusion} 
Different from VQ-GAN which uses GAN to generate waveforms or frequency features, some codecs also use latent diffusion~\cite{ho2020denoising,song2021scorebased,rombach2022high} as an alternative.
These codecs use discretized tokens as a condition to generate some latent acoustic space, e.g. from a pretrained continuous speech autoencoder.
Since diffusion models are strong generative models, acoustic tokens of this type does not need discriminators and adversarial training like VQ-GAN.
For instance, LaDiffCodec~\cite{yang2024generative} uses EnCodec tokens to condition the diffusion process from Gaussian noise to the latent space in a pretrained and frozen waveform autoencoder.
This is to bridge the gap of reconstruction quality between discrete and continuous representations and improve the codec performance compared to the original acoustic tokens.
Inference efficiency is a major concern of these models unless specifically optimized in limited sampling steps.
% By transforming the discrete codes from EnCodec to a properly-shaped continuous latent space, it improves the codec performance compare to original the EnCodec model.
% SemantiCodec~\cite{liu2024semanticodec} also belongs to this type, which will be detailed in Section \ref{sec:acoustic-distillation}.
% \textcolor{red}{Shall training losses be expressed in detail?}

% \paragraph{U-Net}\textcolor{red}{A separate paragraph or an affiliation in VQ-GAN?}

\vspace{-0.1in}
\subsection{General-Purpose Acoustic Tokens}

\label{sec:acoustic-general}

\subsubsection{Motivation}

In this section, we describe the most common type of neural acoustic tokens (speech codecs) that are designed only with the objective of speech signal reconstruction.
Those acoustic tokens are optimized towards better signal or perceptual quality under bitrates as low as possible. 

\subsubsection{Approaches}

% \textcolor{red}{Should we organize this subsection using architectures in Fig 1?}

% \textcolor{red}{HILCodec?}

\paragraph{Advanced VQ methods and model architectures}

Based on SoundStream and EnCodec, more codecs with advanced VQ methods, network structure, or optimization strategies have been researched with depth.
% HiFi-Codec~\cite{yang2023hifi} applies GRVQ on codecs to reduce the number of codebooks.
% SRCodec~\cite{zheng2024srcodec} proposes a dual attention mechanism and a split residual VQ strategy, which can also be regarded as GRVQ with interactions between groups.
% , and also launches the AcademiCodec\footnote{\url{https://github.com/yangdongchao/AcademiCodec}} project to facilitate codec research.
As an example, DAC~\cite{kumar2024high} achieves remarkable reconstruction quality by adding periodic inductive bias, better discriminators, modified loss functions, and a better VQ mechanism from ViT-VQGAN~\cite{yu2022vectorquantized} to improve codebook usage. 
Specifically, it performs L2-normed code lookup in a low-dimensional space (e.g. 8 or 32) instead of a high-dimensional space like 1024.
% Its VQ tricks are reported to improve code usage.
Other architectural improvements include using frequency-domain inputs~\cite{APCodec,ai24b_interspeech,singlecodec}, variance-constrained residual blocks~\cite{ahn2024hilcodec}, multi-filter bank discriminator~\cite{ahn2024hilcodec}, selective down-sampling back-projection~\cite{zheng2024supercodec}, etc.
% APCodec~\cite{APCodec} uses amplitude and phase spectra as inputs and outputs of the VQ-GAN codec model.
% Its encoder and decoder are based on an improved ConvNeXt v2 network~\cite{woo2023convnext}, and it achieves fast and low-latency compression for 48kHz audio.
% Based on APCodec, \cite{ai24b_interspeech} later reduces the necessary bitrate to 1kbps for high sampling rate scenarios by introducing additional bandwidth reduction and recovery modules before and after VQ-GAN.
% HILCodec~\cite{ahn2024hilcodec} proposes spectrogram blocks, variance-constrained residual blocks and a multi-filter bank discriminator to achieve high fidelity and lightweight streaming codec.
% SuperCodec~\cite{zheng2024supercodec} improves the traditional CNN blocks in encoder and decoder into a selective down-sampling back-projection network for better performance.

% ~\cite{kyutai2024moshi} to improve subjective perception: 1) introducing causal Transformers to both its encoder and decoder; 2) not applying VQ with a certain probability; 3) pure adversarial training without reconstruction losses.
% Meanwhile, TF-Codec introduces learnable frequency input compression and bitrate-controllable quantization to optimize with varying bitrates.

Several training tricks are explored, such as not applying VQ with a certain probability and pure adversarial training proposed in Moshi~\cite{kyutai2024moshi}.
Also, the training of neural speech codecs does not need to be end-to-end, i.e. the learning of VQ representations and signal reconstruction can be separated.
\cite{audiodec,du2024apcodec+} adopt a two-stage training process that introduces adversarial losses and an additional vocoder after training only with metric losses, to achieve improved quality.
Additional training criteria around the VQ module are proposed for better VQ utilization, such as delicate code-vector replacement strategy, codebook balancing loss, and similarity loss between consecutive RVQ layers proposed in ERVQ~\cite{zheng2024ervq}.
% ERVQ~\cite{zheng2024ervq} proposes a more delicate code-vector replacement strategy and a codebook balancing loss to enhance the VQ usage.
% It also applies a similarity loss after consecutive RVQ layers to encourage each RVQ layer to focus on different speech features.

% It achieves low latency while improving quality.
% APCodec+~\cite{du2024apcodec+} introduces this two-stage training process into APCodec, and declares that using adversarial loss throughout the entire process yields better performance.

% \textcolor{red}{SRCodec}.

% Although the aforementioned works all use regular GVQ or RVQ quantizers, 
Other VQ methods besides GVQ or RVQ also exist in speech codecs.
NDVQ~\cite{niu2024ndvq} improves the capacity of RVQ space by changing codebook {vectors} to parameterized Gaussian {distributions}.
% Instead of quantizing the continuous input to the closest codebook entry in each RVQ layer, NDVQ performs quantization by choosing the mean and variance with the greatest probability density.
% A sample is then drawn from the chosen Gaussian distribution as the VQ output, with a reparameterization technique.
FSQ has also been introduced to several speech codecs, like SQ-Codec~\cite{yang24l_interspeech} where scalar rounding is applied to each of its 32-dimensional latent space.
Stable-Codec~\cite{parker2024scalingtransformerslowbitratehighquality} adopts FSQ in a Transformer-based architecture, exhibiting strong scalability to large model sizes up to 950M parameter count.
It also explores a flexible post-training quantization level adjustment technique and residual FSQ strategy.
% dMel~\cite{bai2024dmel} directly quantizes mel-filterbanks per dimension with evenly-paced boundaries between the minimum and maximum of filterbank values.
% This is similar to FSQ but is parameter-free.

Note that most acoustic tokens require multiple quantizers, but \textbf{single-codebook} codecs have also been explored.
Single-Codec~\cite{singlecodec} designs an encoder consisting of Conformer and bidirectional LSTM to better compress mel spectrogram inputs.
% It accomplishes codec using only a \textbf{single codebook}.
WavTokenizer~\cite{ji2024wavtokenizer} and BigCodec~\cite{xin2024bigcodec} further explores single-codebook codec modeling with better network designs or larger parameter count.
TS3-Codec~\cite{wu2024ts3codectransformerbasedsimplestreaming} adopts a fully Transformer design that leads to a better single-codebook codec with fewer computation overhead.
LSCodec~\cite{guo2024lscodec} also achieves single-codebook coding with speaker disentanglement (Section \ref{sec:acoustic-disen}).
These single-codebook codecs with remarkably low bitrates offer great benefit to downstream speech generation models on simplicity and efficiency.

\paragraph{Temporal redundancy reduction}

Instead of capturing all the information through VQ layers like the previously mentioned codecs, some researchers have attempted to reduce the redundant bitrate of time-varying VQ codes.
One reasonable method is to encode the global information in speech, e.g. speaker timbre and channel effects, by a global encoder instead of the time-varying codes.
% The global information in speech includes speaker identity, channel effect and so on, and it does not need to be repetitively encoded by time-varying discrete tokens.
% The global information in speech, which includes speaker identity, channel effects, and other attributes, does not need to be repetitively encoded by time-varying discrete tokens. 
Disen-TF-Codec~\cite{jiang2023disentangled} is the first to explore VQ-GAN codec models with an additional global encoder that aids the codec decoder. 
In Disen-TF-Codec, the global features are designed to be sequential to adapt to speaker changes during transmission.
In TiCodec~\cite{ticodec}, the global tokens are time-invariant and vector-quantized instead.
They are extracted from different segments of an utterance in conjunction with time-varying tokens.
% By this design, TiCodec tries to minimize the global information in its RVQ time-varying codes.
Similar global encoders are also seen in \cite{guo2024socodec,guo2024speaking,singlecodec}.
% In \cite{guo2024socodec,guo2024speaking}, an ECAPA-TDNN~\cite{desplanques20_interspeech} reference encoder is employed.
% The introduction of a global encoder also facilitates the development of single-codebook codecs, such as Single-Codec~\cite{singlecodec}.
FreeCodec~\cite{zheng2024freecodecdisentangledneuralspeech} further incorporates a prosody encoder~\cite{ren2022prosospeech} that compresses the low-frequency range of mel spectrograms into a low frame rate VQ sequence to assist in reconstruction.

% This shows potential for compression acoustic information to a \textbf{single codebook} without repetitive RVQ process.

Another typical example of temporal redundancy reduction is predictive coding, as seen in TF-Codec~\cite{jiang2023latent}.
This approach captures temporal-varying information in the latent space by autoregressive prediction, which significantly reduces redundancy and entropy in the residual part for  quantization.
LMCodec~\cite{LMCodec} employs autoregressive prediction from coarse codes (first RVQ levels) to fine codes (last RVQ levels)~\cite{borsos2023audiolm}, enabling the transmission of fewer codes.

\paragraph{Multi-resolution and variable-bitrate coding}
\label{sec:multi-resolution}

% Instead of uni-resolution tokens where all the quantizers share the same temporal frequency of typically 25-86Hz, it is reasonable to design multi-resolution codecs since there are simultaneously fast and slow information streams in speech.
Rather than relying solely on uni-resolution tokens, where all quantizers share the same temporal frequency, it is reasonable to design multi-resolution codecs, because speech contains both fast and slow information streams.
For instance, many vowels exhibit slowly changing characteristics, while events such as explosive consonants and background noises require fine-grained modeling. Therefore, incorporating multiple temporal resolutions in codecs is likely to reduce the necessary bitrate.
% Therefore, incorporating multiple temporal resolutions in codecs is likely to reduce the necessary bitrate.
% Lots of vowels exhibit slowly changing characteristics, while events like explosive consonants and background noises may require finer-grained modeling.
% Hence, designing multiple temporal resolutions in codec is likely to decrease the necessary bitrate.

% Multi-resolution acoustic tokens have been investigated in contrast to the previous uni-resolution tokens.
% In multi-resolution acoustic tokens, different token streams have different frame rates for modeling spoken information from coarse to fine.

% Siahkoohi et al.~\cite{siahkoohi22_interspeech} 
SNAC~\cite{Siuzdak_SNAC_Multi-Scale_Neural_2024} is a notable multi-resolution acoustic token.
It follows the DAC~\cite{kumar2024high} architecture, but in each RVQ layer, residuals are downsampled before codebook look-up and upsampled afterward.
This enables SNAC to have three RVQ streams at a frame rate of 12, 23, 47Hz respectively.
% For example, with a CNN downsampling factor of 512 on 24kHz waveforms, the first quantizer further downsamples the sequence by a factor of 4, and then upsamples the 12Hz quantized vectors by the same factor to compute quantization residuals.
% The downsampling factors for the second and third quantizers are 2 and 1, respectively.
% This design enables SNAC to outperform codecs with uni-resolution RVQ, especially under low bitrates.
Similarly, CoFi-Codec~\cite{guo2024speaking} achieves multi-resolution coding by GVQ quantizers within its U-Net-based architecture.
LLM-Codec~\cite{yang2024uniaudio15} also adopts this idea to achieve very low frame rates with semantic distillation (Section \ref{sec:acoustic-distillation}).
% , CoFi-Codec~\cite{guo2024speaking} uses a U-Net architecture, where each encoder is a CNN with a specific downsampling rate.
% The decoders follow a mirrored procedure, and quantizers are placed between each encoder-decoder pair with at a specific resolution.
% This results in a multi-resolution representation, where, at each scale, GVQ is applied to the residual between the encoder and decoder hidden embeddings.
% In contrast, ESC~\cite{gu2024esc} changes the frequency resolution in each layer rather than the time resolution.

% Apart from multiple temporal resolutions, it is also valuable to notice the different information intensities in different speech frames.
In addition to multiple temporal resolutions, it is also feasible to consider the varying information intensities across different speech frames. 
% Some frames carry critical information, while others may be less informative (e.g., silences). 
This observation motivates the design of codecs to allocate different numbers of quantizers for different speech frames.
% Some frames carry important information while others may be obscure (e.g. silences).
% This inspires codec to allocate different numbers of quantizers for different speech frames.
As an example, VRVQ~\cite{chae2024variable} automatically selects the number of RVQ quantizers per frame by a predictor that is jointly trained with the whole network.
% In VRVQ, a predictor receives the encoder's hidden embeddings and outputs an importance map for each frame ranging from 0 to 1.
% This importance map determines the number of quantizers $Q$ for each frame and masks the quantizers beyond the first $Q$ quantizers.
% Since this masking process is not differentiable, surrogate functions are introduced to train the importance map predictor.

\subsubsection{Challenges}
Despite the emergence of single-codebook and low-bitrate codecs~\cite{singlecodec,ji2024wavtokenizer,xin2024bigcodec,guo2024lscodec}, achieving ideal reconstruction quality with a highly limited VQ space remains a challenging problem. 
% Besides, as acoustic tokens try to encode all necessary information for signal recovery, they may be redundant and too complex for downstream modeling.
Additionally, as acoustic tokens aim to encode all necessary information for signal recovery, they may become redundant and overly complex for downstream modeling.
While scaling up the model size or switching to non-causal networks has been shown to improve performance~\cite{singlecodec,xin2024bigcodec,parker2024scalingtransformerslowbitratehighquality}, these approaches may also compromise streamability or efficiency.
Furthermore, simply introducing global encoders like \cite{jiang2023disentangled,ticodec,guo2024speaking} does not guarantee disentanglement (Section \ref{sec:acoustic-disen}) and may still result in redundancy within the time-varying codes.

\subsection{Acoustic Tokens with Semantic Distillation}
\label{sec:acoustic-distillation}

\begin{figure}
    \centering
    \includegraphics[width=0.8\linewidth]{figs/acoustic-distill1.png}
    \includegraphics[width=0.8\linewidth]{figs/acoustic-distill2.png}
    \includegraphics[width=0.8\linewidth]{figs/acoustic-distill3.png}
    \caption{Different semantic distillation methods in acoustic tokens. Gray color indicates frozen during training.}
    \label{fig:acoustic-distill}
    \vspace{-0.2in}
\end{figure}

\subsubsection{Motivation}
Acoustic tokens are a convenient choice for spoken language models, as they can be directly converted back to waveforms without the need for extra vocoders.
However, if reconstruction is the sole objective of these tokens, their representation space may become overly complex and overly focused on acoustic details, in contrast to natural language tokens that primarily carry semantic information.
A natural improvement is to incorporate speech semantic features either from speech self-supervised learning (SSL) models, supervised models, or even text transcriptions.
Since speech SSL models aim to capture high-level phonetic or semantic information without external supervision~\cite{mohamed2022self}, integrating SSL features does not impose additional data requirements for injecting semantic information into the training process. 
Acoustic tokens with criteria beyond reconstruction are sometimes referred to as having a ``mixed objective''~\cite{cui2024recent}.
Given that the primary purpose of these models remains acoustic reconstruction in these models, we continue to refer to them as acoustic tokens.
The process of introducing semantic information into acoustic tokens is termed \textbf{semantic distillation}, with approaches summarized in Fig. \ref{fig:acoustic-distill}.

\subsubsection{Approaches}

\paragraph{Semantic feature guidance} 
The earliest effort in semantic distillation is to guide some RVQ layers in acoustic tokens towards semantic features, which are typically SSL features. 
Since information in RVQ naturally follows a coarse-to-fine order, guiding early RVQ layers towards semantic-oriented features helps establish and reinforce a semantic-to-acoustic information hierarchy.
For example, SpeechTokenizer~\cite{zhang2024speechtokenizer} uses a HuBERT~\cite{hsu2021hubert} SSL model to guide the first RVQ layer in EnCodec.
This ensures that the first RVQ layer contains more semantic information, thereby pushing acoustic details to the subsequent RVQ layers. 
This distillation is implemented either by regressing the first RVQ output to continuous HuBERT embeddings or by classifying it into discrete HuBERT tokens.
LLM-Codec alternatively uses Whisper~\cite{whisper} and T5~\cite{raffel2020exploring} as semantic teachers.
Mimi~\cite{kyutai2024moshi} uses a WavLM~\cite{chen2022wavlm} teacher and applies distillation to a specialized VQ module rather than the first RVQ layer.
% uses a WavLM~\cite{chen2022wavlm} model as a semantic teacher and designs a specialized VQ module for distillation, rather than using the first RVQ layer. 
% It claims to achieve a better semantic-acoustic trade-off compared to forcing acoustic information into the residual of the semantic quantizer.
Since SSL feature guidance occurs only during the training stage, it does not incur additional inference costs.
It has been reported that TTS language models trained with such acoustic tokens exhibit better robustness than those with unguided tokens~\cite{zhang2024speechtokenizer}.

\paragraph{Fixed semantic codebook} A more direct approach to achieve semantic distillation is to integrate semantic knowledge into the codebook of quantizers. 
% The encoder is then tasked with transforming the original speech into this semantic codebook space, while the decoder must learn to recover acoustics from this semantic space and the residuals.
This forces the quantization space itself to be more semantic-related.
This method is proposed in LLM-Codec~\cite{yang2024uniaudio15} where all three RVQ codebooks are initiated from the token embedding module of LLaMa-2~\cite{touvron2023llama2} and remain frozen during training.
% where pretrained automatic speech recognition (ASR) model Whisper~\cite{whisper}, text language model T5~\cite{raffel2020exploring}, and the LLM LLaMa-2~\cite{touvron2023llama,touvron2023llama2} are employed as semantic teachers.
% LLM-Codec comprises three RVQ layers where all codebooks are initiated from the token embedding module of LLaMa-2 and remain frozen during training.
% Specifically, the first RVQ codebook is constructed by selecting common words and average their corresponding sub-word embeddings from LLaMa-2.
% The rest two codebooks directly utilize the entire vocabulary space of LLaMa-2.
% Input vector sequences are downsampled at different rates before entering the first and second quantizers.
% The outputs of T5 and Whisper encoders are used to guide the first and second RVQ layers, respectively.
This approach not only reduces the bitrate of the codec but also significantly enhances the semantic representation ability of LLM-Codec.

\paragraph{Semantic features as inputs or outputs} 
Semantic features can also be compressed together with the acoustic features. 
This requires the encoder and quantizer to construct a shared acoustic and semantic space that balances the two information sources. 
The first attempt in this direction is made in \cite{siahkoohi22_interspeech} where Conformer representations from a pretrained wav2vec 2.0~\cite{baevski2020wav2vec} are combined with CNN encoder outputs for quantization.
SemantiCodec~\cite{liu2024semanticodec} quantizes AudioMAE~\cite{huang2022masked} SSL features
% \footnote{In fact, a stack of discretized and continuous AudioMAE features.} 
without relying on acoustic inputs. 
The quantized SSL features then serve as a condition for acoustic reconstruction using latent diffusion, which resembles a vocoder that transforms semantic inputs into acoustic outputs.
% SoCodec~\cite{guo2024socodec} also directly quantizes HuBERT features and reconstruct, but incorporates a global acoustic condition to aid reconstruction.
% With a downsampling semantic encoder, it remarkably explores a frame shift up to 240ms.
Providing aligned phoneme sequences instead of SSL features to the quantizer has also shown benefits on reducing bitrates~\cite{du2024funcodec}.
% Additionally, it has also been reported to reduce bitrate when aligned phoneme sequences are added to the encoder output before RVQ~\cite{du2024funcodec}.
% The Mimi tokenizer, proposed in the speech-to-speech LLM Moshi~\cite{kyutai2024moshi}, relies on WavLM~\cite{chen2022wavlm} for semantic distillation. It also regresses the output of a VQ layer to WavLM embeddings, but separates this VQ layer with the rest RVQ layers, differently with SpeechTokenizer.

Moreover, semantic features can also serve as outputs, thereby reinforcing the constraint that semantic information be compressed into the discrete latent space.
% For instance, SoCodec quantizes HuBERT 
For instance, \cite{guo2024socodec,ye2024codec} combine hidden HuBERT embeddings with acoustic features before RVQ and jointly optimizes acoustic and semantic reconstruction objectives.
X-Codec 2.0~\cite{ye2025llasa} improves it by using w2v-BERT 2.0~\cite{barrault2023seamless} and FSQ.
% Then, these tokens can compress semantic features directly.

\subsubsection{Challenges}
Guiding part of the RVQ layers towards semantic features does not guarantee that acoustic information is encoded in the remaining layers, as shown by the degraded VC performance in SpeechTokenizer~\cite{zhang2024speechtokenizer}.
It may impose a greater challenge for the VQ layer to encode both acoustic and semantic information if semantic features serve as inputs as well.
% Fixing a semantic codebook could also negatively impact the acoustic reconstruction ability, since the VQ representation space is too restricted.
Additionally, fixing a semantic codebook could negatively impact acoustic reconstruction ability, as the VQ representation space becomes overly restricted.

\subsection{Acoustic Tokens with Disentanglement}
\label{sec:acoustic-disen}
\subsubsection{Motivation}
Another line of mixed-objective acoustic tokens is {disentanglement}.
A prominent research direction is the disentanglement of speaker timbre information, as this is a global trait among all the speech information aspects.
% It is redundant to encode speaker information into every token timestep, hence information in acoustic tokens will be more compact and the necessary bitrate will be lower if the global speaker timbre is removed.
Encoding speaker information into every token timestep is redundant; thus, removing the global speaker timbre can make the information in acoustic tokens more compact and reduce the necessary bitrate. 
Speaker-decoupled speech tokens can alleviate the modeling burden for downstream tasks. For example, a TTS model using these tokens can achieve independent control over prosody and speaker identity.
% A speaker-decoupled speech token will ease the modeling burden for downstream tasks.
% For example, a TTS model with those tokens can achieve prosody and speaker control independently.
The disentanglement of speaker timbre also enables an acoustic token to perform voice conversion (VC), as timbre from the target speaker can be easily combined with the speaker-agnostic content tokens from the source speech.

Note that in Section \ref{sec:acoustic}, it is mentioned that some codecs introduce a global encoder to reduce the necessary bitrate of time-variant tokens~\cite{jiang2023disentangled,ticodec,singlecodec,zheng2024freecodecdisentangledneuralspeech}.
They have already demonstrated some ability to decouple global speaker timbre and local contents, albeit in an \textbf{implicit} manner through the natural information bottleneck from VQ.
In this section, we elaborate on \textbf{explicit} methods, which involve specialized training techniques and criteria to achieve disentanglement.

\subsubsection{Approaches}
\paragraph{Gradient reversal layer (GRL)} The GRL technique~\cite{drl} is commonly used for disentanglement. Suppose speaker information needs to be disentangled, and a classifier (or speaker verifier, etc.) $s_\mu(\cdot)$ receives some latent feature $\bm h$ from the acoustic token to perform speaker discriminative tasks. 
GRL operates by negating the gradient sign before $s_\mu(\cdot)$, thereby forcing $\bm h$ to fool the speaker classifier while the classifier itself improves, similar to adversarial training.

SSVC~\cite{SSVC} is one of the pioneering efforts in this direction.
% , which is the basis for BASE-TTS~\cite{lajszczak2024base}.
SSVC attempts to decouple content and speaker representations from WavLM features.
The content branch is quantized via RVQ, and the speaker branch is trained using a contrastive loss to produce speaker embeddings.
Disentanglement is enforced by a GRL between the speaker embeddings produced from the speaker branch and the content representations.
% SSVC designs two coupled regressors from WavLM: a speaker regressor and a content regressor.
% These regressors are essentially attention modules on every WavLM layer.
% The speaker regressor is used to train a speaker extractor by contrastive loss to produce discriminative speaker embeddings.
% The output from the content regressor is quantized by RVQ before being combined with speaker embeddings and reconstructed into waveforms.
% SSVC trains a VQ-VAE on WavLM embeddings with RVQ, where the encoder is simply an attention module on every WavLM layer, and the decoder is a BigVGAN vocoder~\cite{lee2023bigvgan}.
% It jointly trains a speaker extractor based on the WavLM embeddings using contrastive loss, similar to GE2E~\cite{wan2018generalized} in speaker verification.
% The speaker extractor outputs discriminative speaker embeddings, which is fed into the BiVGAN vocoder in its VQ-VAE.
% Disentanglement is enforced by a GRL on a cosine distance loss between the speaker extractor outputs from the speaker regressor and content regressor.
Similarly, PromptCodec minimizes an SSIM loss~\cite{wang2004image} between content and speaker representations, with the help of a pretrained speaker verification model.

Such GRL technique is not limited to disentangling speaker timbre alone.
FACodec~\cite{facodec} employs supervised decoupling to factorize speech into speaker timbre, content, prosody, and acoustic detail information.
The timbre extractor in FACodec is optimized via a speaker classification loss.
% For prosody, content, and detail aspects, different RVQ modules are applied respectively before supervised decoupling.
For the other components -- prosody, content, and acoustic detail -- separate RVQ modules are applied prior to the supervised decoupling process.
For each component, some supervision signal with the desired information is applied, and GRL is employed to other non-related information components.
% the normalized F0. In the prosody branch, normalized F0 is predicted, and GRL is applied to the frame-aligned phonemes. Meanwhile, in the acoustic detail branch, GRL is performed using both phonemes and F0.
% The decoder of FACodec receives all four information branches and re-combines them to reconstruct speech.
These three quantized features are then combined before applying GRL with the speaker information. 
Finally, the decoder integrates all four information branches to reconstruct the speech signal.

\paragraph{Perturbation}
For speaker disentanglement, a more straightforward approach is to apply speaker timbre perturbations to speech signals and leverage the strong information bottleneck created by the discrete VQ module.
When the encoder is unable to learn sufficient timbre information, and the decoder is provided with prominent timbre, the bottleneck in the middle will naturally prevent timbre from being encoded~\cite{qian2019autovc}.
% LSCodec~\cite{guo2024lscodec} utilizes speaker disentanglement to achieve ultra-low bitrate.
This idea is adopted in LSCodec~\cite{guo2024lscodec} to achieve speaker decoupling and ultra-low-bitrate coding.
LSCodec leverages continuous WavLM features to represent speaker timbre.
% for its remarkable speaker verification ability~\cite{superb,knnvc}.
These features are fed to a Conformer-based decoder by position-agnostic cross attention~\cite{du2024unicats,li2024sef}.
A stretching-based speaker perturbation algorithm is applied to the input waveform to facilitate speaker disentanglement.
The training process of LSCodec involves multiple stages where a VQ module is injected after constructing a speaker-decoupled continuous space.
% : first, a speech VAE is trained obtain a preliminary speaker-decoupled continuous space.
% Subsequently, this continuous space is discretized into a VQ-VAE.
Through this approach, LSCodec achieves high-quality speech reconstruction and voice conversion using only a single codebook with very low bitrates.

\paragraph{Source separation}
Apart from the disentanglement of speaker timbre, source separation has also been explored in the context of acoustic tokens.
SD-Codec~\cite{bie2024learning} proposes to decouple different audio sources in the neural codec, like speech, music, and sound effects, by employing  multiple parallel RVQ modules.
This approach allows for more efficient and targeted processing of each audio component.

\subsubsection{Challenges}
The GRL technique for disentanglement inherently carries the risk of a more complex optimization trajectory.
Additionally, some disentanglement methods require supervised data~\cite{facodec}, which imposes a significant constraint.
Due to the intricate nature of speech informatics, current efforts are still suboptimal compared to semantic tokens, particularly in terms of VC performance~\cite{guo2024lscodec}.



\section{Speech Tokenization Methods: Semantic Tokens}

\label{sec:semantic}
Semantic tokens refer to discrete speech representations from discriminative or self-supervised learning (SSL) models.
While we use the term \textit{semantic tokens} to maintain consistency with prior works, some researchers recently argue that SSL features are more accurately described as \textit{phonetic} than \textit{semantic}~\cite{choi24b_interspeech} in nature.
Hence to clarify, in this review, semantic tokens should be more accurately defined as the complementary set of acoustic tokens, such that they are not primarily aimed at reconstruction purposes.
In practice, the vast majority of these tokens are designed for discriminative tasks and are believed to have a strong correlation with phonetic and semantic information~\cite{wells22_interspeech,mohamed2022self,sicherman2023analysing,yeh2024estimating}.

\subsection{Semantic Tokens from General-Purpose SSL}
\label{sec:semantic-general}
\subsubsection{Motivation}
% A large branch of semantic tokens come from speech SSL features. 
Speech SSL models have consistently outperformed many traditional methods in various speech tasks~\cite{superb,mohamed2022self}.
Their potential has been extensively mined in discriminative tasks such as automatic speech recognition (ASR)~\cite{wav2vec,vq-wav2vec,hsu2021hubert,zhang2020pushing}, automatic speaker verification (ASV)~\cite{chen2022wavlm,jung2024espnet,miara24_interspeech}, speech emotion recognition (SER)~\cite{morais2022speech,chen2022wavlm,MADANIAN2023200266,ma-etal-2024-emotion2vec} and speech translation (ST)~\cite{wu20g_interspeech,nguyen20_interspeech,babu22_interspeech}.
% \textcolor{red}{TODO: add citations on these tasks with SSL inputs.}
Discretized SSL tokens are initially favored for reducing computation costs and improving robustness against irrelevant information for ASR~\cite{chang23b_interspeech}.
As language models have gained increasing attention, these SSL tokens have been further explored in generative tasks such as TTS~\cite{VQTTS,kharitonov2023speak,vectokspeech} and SLM~\cite{lakhotia2021generative,borsos2023audiolm,hassid2024textually}.
This is because they can be considered high-level abstractions of speech semantics that are largely independent of acoustic details.
% \textcolor{red}{TODO: not finished. Perhaps should have a logic plan first.}

\begin{figure}
    \centering
    % \includegraphics[width=0.85\linewidth]{figs/semantic1.png}
    % \includegraphics[width=0.7\linewidth]{figs/semantic2.png}
    \includegraphics[width=0.99\linewidth]{figs/semantic.png}
    \caption{Representatives in different kinds of semantic tokens. 
    % Upper: semantic tokens from \textbf{general-purpose SSL models}; Middle: \textbf{perturbation-invariant SSL models}; Bottom: semantic tokens from \textbf{supervised models}. 
    ``Q.'' denotes quantizer, which can be optional (dotted line).}
    \vspace{-0.1in}
    \label{fig:semantic-types}
\end{figure}
\subsubsection{Approaches}

SSL models initiate the learning process by defining a pretext task which enables the model to learn meaningful representations directly from the data itself. 
Typical speech SSL models employ CNNs and Transformer encoders to extract deep contextual embeddings.
When it comes to semantic tokens, there are mainly two ways to extract those discrete tokens from an SSL model (see upper part of Fig.\ref{fig:semantic-types}):
\begin{itemize}[leftmargin=5mm]
    \item External quantization, like clustering or training a VQ-VAE. This refers to extracting continuous embeddings from a certain layer or multiple layers in a pretrained SSL model, and performing quantization manually.
    For example, a common semantic token is the HuBERT+kmeans units, where k-means clustering is performed on a HuBERT Transformer layer with a portion of training data~\cite{lakhotia2021generative,kharitonov-etal-2022-text}.
    It is also feasible to perform clustering on multiple layers~\cite{shi24h_interspeech,mousavi2024should}, or train a VQ-VAE on the SSL hidden embeddings~\cite{huang2023repcodec,wang2024maskgct}.
    \item When an SSL model contains an inner quantizer that is trained together with other network modules, its outputs can also be regarded as semantic tokens.
    Many SSL models involve quantizers to produce targets for their training objectives~\cite{vq-wav2vec,baevski2020wav2vec,chiu2022self,zhu2025muq}.
    This approach provides an efficient and effective way of extracting semantic tokens.
\end{itemize}
Note that for SSL models with an inner quantizer, it is still practical to perform external quantization on its continuous embeddings, like wav2vec 2.0~\cite{baevski2020wav2vec}.
However, these two methods -- internal and external quantization -- may result in different patterns of information exhibition, which we will investigate in Section \ref{sec:analysis}.

For general-purpose SSL models, there are different designs on the pretext task~\cite{mohamed2022self}.
Table \ref{tab:semantic-metadata} provides a high-level summary of well-known semantic tokens.

\paragraph{Contrastive} This type of speech SSL models aims to learn representations by distinguishing a target sample (positives) from distractors (negatives) given an anchor~\cite{mohamed2022self}.
They minimize the latent space similarity of negative pairs and maximize that of the positive pairs.
For semantic tokens, vq-wav2vec~\cite{vq-wav2vec} and wav2vec 2.0~\cite{baevski2020wav2vec} are two representative contrastive SSL models.
They involve a quantizer to produce localized features that is contrastively compared to contextualized continuous features.
Vq-wav2vec~\cite{vq-wav2vec} uses pure CNN blocks while wav2vec 2.0~\cite{baevski2020wav2vec} adopts a Transformer for stronger capacity.
Both use GVQ quantizers with two groups to expand the VQ space.
Wav2vec 2.0 has also been extended to massively multilingual versions~\cite{conneau21_interspeech,babu22_interspeech,pratap2024scaling}.

\paragraph{Predictive}
This type of speech SSL models incorporates an external target for prediction, either from signal processing features or another teacher network.
A popular line of work is HuBERT~\cite{hsu2021hubert}.
It takes raw waveforms as inputs, applies random masks on the hidden representations before Transformer contextual blocks, and then predicts k-means quantized targets from MFCC or another HuBERT teacher.
% It can take more self-iterations by using a trained HuBERT teacher model and applying k-means clustering as targets.
WavLM~\cite{chen2022wavlm} augments HuBERT by additional speaker and noise perturbations to achieve superior performance in more paralinguistic-related tasks.
There are no inner quantizers in both models, so external quantization like k-means clustering is necessary to obtain semantic tokens.
BEST-RQ~\cite{chiu2022self} changes the prediction target to the output of a random projection quantizer.
% Similar to acoustic tokens, training a VQ-VAE to compress continuous semantic features in a vector quantized space is also explored. \textcolor{red}{RepCodec~\cite{huang2023repcodec}, token in MaskGCT, etc.}
% Data2vec~\cite{baevski2022data2vec,baevski2023efficient} proposes a general teacher-student masked prediction framework the masked and original view of data are fed to the student and teacher respectively, and the student network predicts the teacher outputs. 
The next-token prediction criterion from language models (LMs) have also been adopted into speech SSL~\cite{turetzky2024last,han2024nest}, either with or without a pretrained text LM.
This method emphasizes the autoregressive prediction property of learned tokens that may be better suited for the LM use case.

\subsubsection{Challenges}
% 0. data
Firstly, SSL models typically require large amount of data to train, as indicated in Table \ref{tab:semantic-metadata}.
% 1. clustering problems
For SSL models without a built-in quantizer during pretraining, k-means clustering is a prevalent approach to obtain discrete units.
% However, since most SSL models work in a high-dimensional space (e.g. with 768 or 1024 dimensions), the space and time complexity of such k-means procedures are large.
However, given that most SSL models operate in high-dimensional spaces (e.g., with 768 or 1024 dimensions), the space and time complexity of k-means clustering are substantial. 
% The clustering result is sometimes unreliable because of the curse of dimensionality in the Euclidean space.
The clustering results can sometimes be unreliable due to the curse of dimensionality in Euclidean space.
% 2. Acoustics and reconstruction
Moreover, it is often reported, and will also be shown by experiments in Section \ref{sec:analysis}, that discretized SSL units lose much acoustic details after quantization~\cite{polyak21,sicherman2023analysing,mousavi2024dasb}.
Different clustering settings, such as the chosen layer and vocabulary size, can lead to different outcomes within a single model.
% 3. causality and stream-ability
Finally, since most SSL models utilize Transformer blocks, their causality and streaming ability are compromised.

\subsection{Semantic Tokens from Perturbation-Invariant SSL}
\label{sec:semantic-invariant}
\subsubsection{Motivation}
As SSL tokens feature semantic or phonetic information, a major concern is to improve the resistance against perturbations in the input signal.
This kind of invariance includes noise and speaker aspects that don't affect the contents of speech.
Noise invariance refers to the invariance against signal augmentations such as additive noise, reverberations, etc.
Speaker invariance aims to remove speaker information, similar to speaker-disentangled acoustic tokens.
% SSL semantic tokens with perturbation invariance are often obtained by training with explicit perturbations.
% Perturbations are often explicitly introduced in training of these perturbation-invariant SSL models.
In the training process, perturbations are often explicitly introduced in these perturbation-invariant SSL models.
The original and perturbed view of an utterance are both fed to the same network (or teacher and student networks), and an external loss to reduce the impact of perturbation is applied.
The middle part of Fig.\ref{fig:semantic-types} depicts a typical perturbation-invariant SSL model.

\subsubsection{Approaches}

% \textcolor{red}{Another way to organize this section is to first introduce noise and speaker perturbation methods, and then treat them like the same, and introduce contrastive, distribution-similarity, CTC respectively.}

\paragraph{Perturbations}
The perturbations can either be designed to augment the acoustics or alter the speaker timbre, depending on the objective of invariance.
These perturbations usually preserve temporal alignments, meaning that the perturbed utterance and the original one are strictly synchronized.
For noise-invariant SSL tokens, basic signal variations like time stretching, pitch shifting, additive noise, random replacing, reverberation, and SpecAugment~\cite{park2020specaugment} are commonly applied~\cite{gat2023augmentation,ccc-wav2vec2.0,messica2024nast,huang2022spiral}.
% ~\cite{park2020specaugment} is also used in \cite{huang2022spiral}.
Typical speaker timbre perturbations include formant and pitch scaling as well as random equalization~\cite{qian2022contentvec,chang23_interspeech,chang2024dc}.
In contrast, random time stretching is applied as speaker perturbation in \cite{hwang2024removing}, which alters the tempo in each random segment.

\paragraph{Contrastive-based Methods}
Contrastive loss is a common method to obtain perturbation-invariant representations.
In this context, the contrastive loss is a modified version of that used in wav2vec 2.0~\cite{baevski2020wav2vec}.
Given two embedding sequences derived from the original and perturbed utterances, assuming the perturbation preserves frame-wise alignment, the positive sample of an anchor is taken from the same position in the other utterance.
This is because the content remains unchanged by the perturbation, thus the same position of two representation sequences should encode the same information.
In noise-invariant models~\cite{huang2022spiral,ccc-wav2vec2.0}, negative samples are selected from the other utterance relative to the anchor.
However, in speaker-invariant models~\cite{qian2022contentvec,hwang2024removing}, negative samples are selected from the same utterance as the anchor.
Specifically, in \cite{hwang2024removing}, soft attention pooling is applied to create equal-length representation sequences from two utterances with different durations.
This approach forces SSL models to ignore acoustic differences and focus solely on the unperturbed content.

\paragraph{Distribution-based Methods}
Another method to achieve invariance is to minimize some distance metrics between the representations extracted from the original and perturbed utterances.
In existing perturbation-invariant SSL models, this is typically accomplished using a cross-entropy loss between the underlying distributions in the VQ module of the SSL model.
NAST~\cite{messica2024nast} trains a Gumbel-based VQ-VAE on HuBERT features and enforces similarity between the Gumbel distributions Eq.\eqref{eq:gumbel-softmax} derived from the original and augmented utterances.
Spin~\cite{chang23_interspeech} and DC-Spin~\cite{chang2024dc} explore a speaker-invariant clustering algorithm for HuBERT features.
Similar to NAST~\cite{messica2024nast}, Spin employs a cross-entropy loss to ensure that the distributions over codebook entries are similar between the original and perturbed utterances.
% Spin uses a distribution smoothing technique before pushing the distributions to be similar, thereby preventing collapse into a trivial solution~\cite{chang23_interspeech}.
This distribution-based approach forces the same content to be quantized to the same index regardless of acoustic conditions.
% DC-Spin~\cite{chang2024dc} uses Spin units to train a HuBERT model and extends the Spin algorithm to incorporate two VQ codebooks, both optimized with the same objective
% The auxiliary codebook is designed to be larger than the primary one, allowing for more fine-grained acoustic details
% Additionally, DC-Spin explores fine-tuning with mel reconstruction and supervised ASR, which are anticipated to further enhance speaker invariance.

\paragraph{CTC-based Methods}
Noise invariance can also be achieved like an ASR task with perturbed speech inputs.
As semantic tokens from SSL models are highly content-related, these tokens extracted from the original clean utterance can serve as some pseudo-label for a perturbed view.
% Normally, 
In \cite{gat2023augmentation}, a connectionist temporal classification (CTC)~\cite{ctc} loss is calculated between quantized tokens from the augmented signal and a pretrained HuBERT+kmeans pseudo-labels from the clean signal.
This pushes the quantized tokens to have the same phonetic structure with the pseudo-labels.

\subsubsection{Challenges}
While noise and speaker-invariance have emerged as promising approaches in semantic tokens, they currently rely on content-preserving perturbations that are typically hand-crafted.
Most existing methods have only been evaluated on small-scale data and models.
It also remains unclear how these methods will generally benefit generative tasks such as speech generation and spoken language modeling.

% Contrastive approaches are explored in \cite{huang2022spiral,ccc-wav2vec2.0}.
% There, the original and augmented utterances are fed to the same network (or the teacher and student respectively) to obtain two sequences of representations.
% The contrastive loss from wav2vec 2.0 is borrowed, but the positive and negative samples are taken from the other utterance instead of the same utterance.


% \paragraph{Speaker invariance}
% Common speaker perturbation in this line of work include 
% ContentVec~\cite{qian2022contentvec} and \cite{hwang2024removing} adopt a contrastive objective similar to noise invariance SSL.
% ContentVec chooses to base on the HuBERT architecture and take negative samples from the same utterance than the perturbed utterance.
% ContentVec also introduces a voice conversion module to provide teacher labels from another speaker, for further eliminating the speaker information.
% Hwang et al.~\cite{hwang2024removing}, instead, chooses the CPC framework~\cite{oord2018representation,wav2vec} and introduces variable-length soft-pooling.



\begin{table}[]
\centering
\caption{A high-level summary of famous semantic speech tokens. Notations follow Table.\ref{tab:acoustic-metadata}.
Symbol `/' denotes different versions. 
``Inner Quantizer'' refers to whether the model has a quantizer, or external quantization (e.g. clustering) must be performed.
$F$ denotes frame rate.
In case there are inner quantizers, $Q,V$ denote number of quantizers and vocabulary size for each quantizer, respectively.
% $Q$ denotes number of quantizers (if there are), and $F$ denotes frame rate.
``\textit{NR}.'' means not reported.
% \textcolor{red}{Shall we change this table? Should more info be included?}
}
\label{tab:semantic-metadata}
\resizebox{\columnwidth}{!}{
% {
\begin{tabular}{@{}lcccccc@{}}
\toprule
\textbf{\makecell{Semantic \\Speech Tokens}} & \textbf{\makecell{Criterion \\ / Objective}} & \textbf{\makecell{Training\\Data (h)}} & $F$ \textbf{(Hz)} & \textbf{{Inner Quantizer}} \\ \midrule
\multicolumn{5}{l}{\textbf{\textit{General-purpose self-supervised learning (SSL) models}}} \\
vq-wav2vec~\cite{vq-wav2vec} & Contrastive & 0.96k & 100 & GVQ, $Q=2,V=320$ \\
wav2vec 2.0~\cite{baevski2020wav2vec} & Contrastive & 60k & 50 & GVQ, $Q=2,V=320$ \\
XLSR-53~\cite{conneau21_interspeech} & Contrastive & 50k & 50 & GVQ, $Q=2,V=320$ \\
HuBERT~\cite{hsu2021hubert} & Predictive & 60k & 50 & No \\
WavLM~\cite{chen2022wavlm} & Predictive & 94k & 50 & No \\
BEST-RQ~\cite{chiu2022self} & Predictive & 60k & 25 & {No} \\ 
w2v-BERT~\cite{chung2021w2v} & {Predictive+Contrastive} & 60k & 50 & VQ, $Q=1,V=1024$ \\
w2v-BERT 2.0~\cite{barrault2023seamless} & {Predictive+Contrastive} & 4500k & 50 & GVQ, $Q=2,V=320$ \\
% data2vec 2.0~\cite{baevski2023efficient} & Predictive & 60k& 50Hz  & No \\
DinoSR~\cite{liu2024dinosr} & Predictive & 0.96k & 50 & VQ, $Q=8,V=256$ \\
NEST-RQ~\cite{han2024nest} & {Predictive} & 300k &  25 & {No} \\
LAST~\cite{turetzky2024last} & {Predictive} & \textit{NR.} & 50 & VQ, $Q=1,V=500$ \\
\midrule
\multicolumn{5}{l}{\textbf{\textit{SSL models with perturbation-invariance}}} \\
{Gat et al.~\cite{gat2023augmentation}} & Noise Invariance & 0.10k & 50 & VQ, $G=1,V=50$-$500$  \\
ContentVec~\cite{qian2022contentvec} & Speaker Invariance & 0.96k & 50 & No \\
SPIRAL~\cite{huang2022spiral} & Noise Invariance & 60k & 12.5Hz & No\\
CCC-wav2vec 2.0~\cite{ccc-wav2vec2.0} & Noise Invariance & 0.36k & 50 & GVQ, $G=2,V=320$ \\
Spin~\cite{chang23_interspeech} & Speaker Invariance & 0.10k & 50 & VQ, $Q=1,V=128$-$2048$\\
NAST~\cite{messica2024nast} & Noise Invariance & 0.96k & 50 & VQ, $Q=1,V=50$-$200$\\
DC-Spin~\cite{chang2024dc} & Speaker Invariance & 0.96k & 50 & VQ, $Q=2,V=(50$-$500)$+$4096$ \\
% \textcolor{red}{Hwang et al.~\cite{hwang2024removing}} & Speaker Invariance & 0.96k & \\
\midrule
\multicolumn{5}{l}{\textbf{\textit{Supervised models}}} \\
% Whisper~\cite{whisper} & Supervised ASR & 680k & 50Hz & No \\
$\mathcal S^3$ Tokenizer~\cite{du2024cosyvoice}  & Supervised ASR & 172k & 25 / 50  & VQ, $Q=1,V=4096$ \\
Zeng et al.~\cite{zeng2024scaling} & Supervised ASR & 90k & 12.5 & VQ, $Q=1,Q=16384$ \\
Du et al. \scriptsize{(CosyVoice 2)}~\cite{cosyvoice2} & Supervised ASR & 200k & 12.5 & FSQ, $Q=8,V=3$ \\
\bottomrule
\end{tabular}
}
\vspace{-0.15in}
\end{table}

\IEEEpubidadjcol

\subsection{Semantic Tokens from Supervised  Models}
\label{sec:semantic-supervised}
As representing semantic or phonetic information is the major purpose of semantic tokens, a more direct way to achieve this is through supervised learning.
A famous example shown at the bottom of Fig.\ref{fig:semantic-types} is the $\mathcal S^3$ Tokenizer from CosyVoice~\cite{du2024cosyvoice}.
It places a single-codebook VQ layer between two Transformer encoder modules and optimizes the network through an ASR loss similar to Whisper~\cite{whisper}.
The same method is adopted in \cite{zeng2024scaling,zeng2024glm} where the frame rate is further reduced to 12.5Hz.
CosyVoice 2~\cite{cosyvoice2} improves $\mathcal S^3$ Tokenizer by replacing plain VQ with FSQ for better codebook utilization.
Note that in this kind of supervised semantic tokens, it is the output of the VQ layer that serves as tokens.
This allows for more preservation of paralinguistic information than directly transcribing speech into text.
% Whisper~\cite{whisper}, on the other hand, needs an extra quantization step to produce discrete semantic tokens since it operates on a continuous embedding space.
These supervised tokenizers are trained on massive paired speech-text data, and have demonstrated rich speech content understanding capabilities~\cite{du2024cosyvoice,fang2024llamaomni}.
% citing llama-omni because it uses continuous whisper as speech encoder.

However, training these models is highly costly due to the heavy data demands.
Training with only the ASR task may still result in the loss of some prosody information.
Although \cite{cosyvoice2} has demonstrated that its supervised tokenizer trained on Chinese and English can also work in Japanese and Korean, it remains unclear how well these supervised tokenizers generalize to more unseen languages.



\subsection{Length Reduction by Deduplication and Acoustic BPE}
\label{sec:dedup-bpe}
In most cases, the frame rate of discrete speech tokens ranges from 25 to 100Hz.
This leads to a huge discrepancy in lengths between speech representations and the underlying text modality.
% Most token sequences are tens of times longer than their corresponding phoneme sequences, not to mention the inner semantics.
This discrepancy has been a critical issue in building decoder-only TTS and other LM-based speech generation tasks, since longer sequences result in harder training and more unstable inference.
Therefore, length reduction techniques have been proposed to address this issue. 
These methods are inspired by language processing techniques and are thus more closely related to semantic tokens. 
Note that although these length reduction methods are universal across token types, they are less frequently applied to acoustic tokens. 
This is because acoustic tokens usually involve multiple VQ streams that complicate token-level operations.
% We will also show that single-codebook acoustic tokens have
% Hence we still 

A common approach to reduce token sequence lengths is deduplication~\cite{chang23b_interspeech,chang2024exploring}, i.e. removing the repeated consecutive tokens in a sequence.
Since the encoded continuous features are often close in consecutive frames where the speech dynamics do not change rapidly, they are likely to be quantized to the same unit.
% (e.g. in short segments inside a vowel)
Therefore, removing these redundant tokens can yield a more phonetic representation.
% Consider an original token stream $[a,a,a,b,b,a,c]$.
% After deduplication, the sequence becomes $[a,b,a,c]$, with corresponding durations $[3,2,1,1]$.
When the deduplicated tokens are used for generative modeling, a unit-to-speech model (similar to TTS) should be employed to upsample the tokens and convert them back to acoustic signals~\cite{lakhotia2021generative}.
% neural network duration predictor~\cite{ren2021fastspeech} is applied to predict the duration per token before converting back to signals.\textcolor{red}{TODO: add some works that use this}

Another popular approach to reducing the length of speech token sequences is acoustic byte-pair encoding (BPE)\footnote{The term ``acoustic'' here is used to distinguish it from traditional BPE applied to text tokens, rather than referring to ``acoustic tokens''.} or so-called subword modeling~\cite{hayashi2020discretalk,ren22_interspeech,chang23b_interspeech,shen2024acoustic,dekel24_interspeech}.
Similar to text BPE~\cite{Gage1994ANA}, acoustic BPE iteratively merges the two most frequent consecutive tokens and adds the merged token to the vocabulary.
After training on a corpus, a deterministic BPE mapping is established between original token combinations and the new vocabulary. 
This mapping enables a lossless compression algorithm, allowing tokens to be perfectly reconstructed after BPE decoding.
This operation can identify certain morphological patterns in token sequences, and offers a powerful way to remove redundant tokens.
% The encoded BPE tokens are used for downstream generation tasks, and original tokens are recovered by the deterministic BPE mapping before converting to signals.
In practice, acoustic BPEs on HuBERT semantic tokens has demonstrated significant speed and performance gains in ASR~\cite{chang23b_interspeech,chang2024exploring}, spoken language modeling~\cite{shen2024acoustic,dekel24_interspeech} and TTS~\cite{li24qa_interspeech,vectokspeech}.
% \textcolor{red}{TODO: other works that use acoustic BPEs, like \cite{vectokspeech}}.

Although deduplication is a simple and training-free method, acoustic BPE offers several unique advantages over it. First, acoustic BPE can identify redundant patterns that are not simply repetitions, whereas deduplication only removes exact duplicates. Additionally, deduplication discards the duration information of every token in the resulting sequence. This could be problematic for downstream tasks, as important rhythmic information may reside in the repetitions of tokens. In contrast, acoustic BPE preserves duration information by encoding repetitions of varying lengths into distinct new tokens. Furthermore, acoustic BPE is flexible in terms of target vocabulary size, which can be adjusted based on the desired length reduction ratio and downstream performance.


% \textcolor{red}{Maybe there can be a figure comparing BPE for acoustic and semantic tokens. Maybe there is more length reduction in semantic tokens than acoustic ones.}
\begin{figure*}
    \centering
    \includegraphics[width=0.24 \linewidth]{figs/single-acoustic-bpe.png}
    \includegraphics[width=0.24 \linewidth]{figs/single-semantic-bpe.png}
    \includegraphics[width=0.24 \linewidth]{figs/hubert-bpe.png}
    \includegraphics[width=0.24 \linewidth]{figs/multi-acoustic-bpe.png}
    \caption{BPE effect comparison of multiple tokens. The starting point of each line represents the original vocabulary size.}
    \label{fig:bpe-effect}
\end{figure*}
We visualize the length reduction effect of BPE on different speech tokens in Fig.\ref{fig:bpe-effect}. 
In addition to semantic tokens from various models and different k-means clusters in HuBERT, we also experiment with acoustic tokens.
For acoustic tokens with multiple codebooks, we apply BPE only to the first quantizer, in accordance with the current speech generation paradigm~\cite{valle}.
From Fig.\ref{fig:bpe-effect}, it is evident that different types of tokens exhibit very distinct patterns.
Semantic tokens generally show significant length reduction when applying BPE, especially for HuBERT models with fewer k-means clusters.
For single-codebook acoustic tokens, speaker-decoupled LSCodec tokens shows more reduction than general-purpose WavTokenizer and BigCodec.
For a single RVQ layer among multiple-codebook acoustic tokens, the reduction effect is also significant.
These findings suggest that the effect of BPE is negatively correlated with the information density in the speech tokens: the less information, the more length reduction achieved by BPE.
% As BPE is hard to apply on all the VQ streams for multiple-codebook tokens, the most reasonable way to apply it is on semantic tokens, which usually match the single-codebook requirement.


\subsection{Variable Frame Rate Tokens and Unit Discovery}
\label{sec:variable-rate}
Information in speech is not uniformly distributed along the time axis~\cite{dieleman2021variable}.
In segments such as silence or long vowels, information density is low, whereas in segments with explosive consonants, speech events occur much more frequently.
This inherent non-uniformity suggests that it might be more natural to allocate more tokenized bits to regions with dense information and higher variance, and fewer bits to regions with less uncertainty.
This kind of discrete speech tokens is referred to as \textit{variable frame rate (VFR) tokens} in this review.
Note that while multi-resolution and variable-bitrate tokens have been introduced previously, the concept of VFR is still distinct.
In multi-resolution tokens~\cite{Siuzdak_SNAC_Multi-Scale_Neural_2024,guo2024speaking}, each quantizer operates at a fixed frame rate.
In variable-bitrate tokens~\cite{chae2024variable}, the frame rate remains fixed, while the variability lies in the  number of quantizers per frame.
Instead, VFR tokens should directly allocate different granularities on the temporal axis.

VFR tokens are closely related to acoustic unit discovery. As speech lacks a natural boundary of phonetic units~\cite{mohamed2022self}, there are much research efforts to find and locate the underlying acoustic units behind speech utterances in an unsupervised manner~\cite{eloff19_interspeech,dunbar20_interspeech,niekerk20b_interspeech,nguyen2020zero}.
This is particularly of interest for low-resource languages.
The discovered units can guide the boundary segmentation of VFR tokens.
To this end, VFR tokens are interesting not only because they might reduce the necessary bitrate, but also because they can introduce a strong inductive bias that linguistic knowledge is encoded~\cite{dieleman2021variable}.

A recent direction of VFR tokens is to discover acoustic units from an SSL model.
Note that deduplicated tokens and acoustic BPE themselves can be regarded as VFR tokens.
% SD-HuBERT~\cite{cho2024sd} finds that using a sentence-level criterion to finetune HuBERT results in syllable-level organizations in its representation similarity matrices.
Sylber~\cite{cho2024sylber} and SyllableLM~\cite{baade2024syllablelm} take similar approaches that first locate acoustic boundaries from existing HuBERT models, and then train another HuBERT student with segment-level pooled targets between boundaries.
The final HuBERT embeddings undergo the same segment-level pooling and kmeans clustering procedure to produce tokens at a very low frame rate ($\approx5$Hz) that align well with syllables.
% Upon it, Sylber~\cite{cho2024sylber} develops a greedy boundary discovery algorithm to locate syllable-level boundaries from SD-HuBERT.
% It then trains another HuBERT student model with segment-wise average pooled representations as targets, and the final HuBERT embeddings undergo kmeans clustering to produce syllable-level tokens at a very low frame rate $F\approx4.27$Hz.
% At the same time, SyllableLM~\cite{baade2024syllablelm} takes a different min-cut algorithm to locate syllable-like boundaries from the original HuBERT, and then trains a similar HuBERT using segment-wise pooled targets.
% The resulting syllable-level tokens exhibit decent reconstruction quality and stronger language modeling capability.

% Variable-rate constraint can also be plugged into an 
Boundary prediction can be involved to achieve frame rate variability in the training process, where a specific model predicts frame-level boundaries and is trained together with other network modules.
The training techniques of such models include reinforcement learning~\cite{cuervo2022variable}, soft pooling~\cite{hwang2024removing}, and slowness constraint~\cite{dieleman2021variable}.
% Cuervo et al.~\cite{cuervo2022variable} applies a hard boundary predictor into a contrastive speech SSL model for variable-rate downsampling and trains it using reinforcement learning.
% Hwang et al.~\cite{hwang2024removing} uses a soft predictor instead and performs downsampling through a soft pooling mechanism.
% \textcolor{red}{Variable rate hierarchical CPC, and variable rate soft-pooling}
% Different from boundary prediction, a slowness constraint is imposed into a VQ-VAE in \cite{dieleman2021variable} that forces the latent features to vary slowly along time, after which run-length encoding (i.e. deduplication with duration saved) is applied on the quantized codes.
% As the resulting quantized tokens will have lots of repeats, deduplication is applied to obtain the variable-rate tokens.
However, these approaches are rarely adopted in the context of discrete speech tokens today, and there have barely been a VFR acoustic token till now.




\subsection{Speech Token Vocoders}
\label{sec:vocoder}
Acoustic tokens are designed naturally with a decoder that outputs waveforms or spectrograms given tokens, but semantic tokens are not.
A necessary component for building a discrete token-based speech generation system with semantic tokens is the speech resynthesis model, or speech token vocoders.
Unlike traditional spectrogram-based vocoders~\cite{kong2020hifigan}, these vocoders receive discrete speech tokens as an input and reconstruct speech signals.
% They are especially important for semantic tokens since these tokens are not born with a reconstructive decoder compared to acoustic tokens.

Polyak et al.~\cite{polyak21} first explores speech resynthesis from discrete speech units by a HifiGAN~\cite{kong2020hifigan} augmented with discretized pitch units and speaker embedding inputs.
% Later, 
\IEEEpubidadjcol
The vec2wav vocoder in VQTTS~\cite{VQTTS} improves this vocoder by a Conformer~\cite{conformer} frontend module before HifiGAN generator.
Later, CTX-vec2wav~\cite{du2024unicats} proposes a position-agnostic cross-attention mechanism that effectively integrates timbre information from surrounding acoustic contexts without the need for pretrained speaker embeddings.
This makes it more timbre-controllable and suitable for zero-shot TTS and VC~\cite{li2024sef}.
Upon it, vec2wav 2.0~\cite{guo2024vec2wav} further advances the timbre controllability by SSL timbre features and adaptive activations, demonstrating a strong VC performance.

It is also feasible to apply diffusion or flow matching algorithms in token vocoders~\cite{tortoise,seedtts,du2024cosyvoice}.
There, the discrete tokens are treated as a condition for diffusion or flow matching to generate mel-spectrograms, and further converted to waveform by a pretrained mel vocoder.
Compared to training a token-to-wav vocoder in an end-to-end fashion, training a token-to-mel model is more convenient and does not need adversarial training. 
To better control timbre, a mask strategy is introduced into the training process where the model only computes loss on the un-masked part of spectrograms~\cite{du2024cosyvoice}.
During inference, spectrogram from speaker prompt conditions the generative process, which can be regarded as a form of ``in-context learning''.
% \textcolor{red}{Drawbacks?}
However, this requires tokens to be extracted from reference prompts before synthesis.
Also, inference efficiency may be compromised for better generation quality with multiple inference steps, and this method is only validated on massive amount of data currently.


\section{Analysis}
\label{sec:analysis}
In the following sections, we will analyze European type approval regulation\footnote{Strictly speaking, the German enabling act (AFGBV) does not regulate type-approval, but how test \& operating permits are issued for SAE-Level-4 systems. Type-approval regulation for SAE-Level-3 systems follows UN Regulation No. 157 (UN-ECE-ALKS) \parencite{un157}.} regarding the underlying notions of ``safety'' and ``risk''.
We will classify these notions according to their absolute or relative character, underlying risk sources, or underlying concepts of harm.

\subsection{Classification of Safety Notions}
\label{sec:safety-notions}
We will refer to \emph{absolute} notions of safety as conceptualizations that assume the complete absence of any kind of risk.
Opposed to this, \emph{relative} notions of safety are based on a conceptualization that specifically includes risk acceptance criteria, e.g., in terms of ``tolerable'' risk or ``sufficient'' safety.

For classifying notions of safety by their underlying risk (or rather ``hazard'') sources, and different concepts of harm, \Cref{fig:hazard-sources} provides an overview of our reasoning, which is closely in line with the argumentation provided by Waymo in \parencite{favaro2023}.
We prefer ``hazard sources'' over ``risk sources'', as a risk must always be related to a \emph{cause} or \emph{source of harm} (i.e., a hazard \parencite[p.~1, def. 3.2]{iso51}).
Without a concrete (scenario) context that the system is operating in, a hazard is \emph{latent}: E.g., when operating in public traffic, there is a fundamental possibility that a \emph{collision with a pedestrian} leads to (physical) harm for that pedestrian. 
However, only if an automated vehicle shows (potentially) hazardous behavior (e.g., not decelerating properly) \emph{and} is located near a pedestrian (context), the hazard is instantiated and leads to a hazardous event.
\begin{figure*}
    \includeimg[width=.9\textwidth]{hazard-sources0.pdf}
    \caption{Graphical summary of a taxonomy of risk related to automated vehicles, extended based on ISO 21448 (\parencite{iso21448}) and \parencite{favaro2023}. Top: Causal chain from hazard sources to actual harm; bottom: summary of the individual elements' contributions to a resulting risk. Graphic translated from \parencite{nolte2024} \label{fig:hazard-sources}}
\end{figure*}
If the hazardous event cannot be mitigated or controlled, we see a loss event in which the pedestrian's health is harmed.
Note that this hypothetical chain of events is summarized in the definition of risk:
The probability of occurrence of harm is determined by a) the frequency with which hazard sources manifest, b) the time for which the system operates in a context that exposes the possibility of harm, and c) by the probability with which a hazardous event can be controlled.
A risk can then be determined as a function of the probability of harm and the severity of the harm potentially inflicted on the pedestrian.

In the following, we will apply this general model to introduce different types of hazard sources and also different types of harm.
\cref{fig:hazard-sources} shows two distinct hazard sources, i.e., functional insufficiencies and E/E-failures that can lead to hazardous behavior.
ISO~21488 \parencite{iso21448} defines functional insufficiencies as insufficiencies that stem from an incomplete or faulty system specification (specification insufficiencies).
In addition, the standard considers insufficiencies that stem from insufficient technical capability to operate inside the targeted Operational Design Domain (performance insufficiencies).
Functional insufficiencies are related to the ``Safety of the Intended Functionality (SOTIF)'' (according to ISO~21448), ``Behavioral Safety'' (according to Waymo \parencite{waymo2018}), or ``Operational Safety'' (according to UN Regulation No. 157 \parencite{un157}).
E/E-Failures are related to classic functional safety and are covered exhaustively by ISO~26262 \parencite{iso2018}.
Additional hazard sources can, e.g., be related to malicious security attacks (ISO~21434), or even to mechanical failures that should be covered (in the US) in the Federal Motor Vehicle Safety Standards (FMVSS).

For the classification of notions of safety by the related harm, in \parencite{salem2024, nolte2024}, we take a different approach compared to \parencite{koopman2024}:
We extend the concept of harm to the violation of stakeholder \emph{values}, where values are considered to be a ``standard of varying importance among other such standards that, when combined, form a value pattern that reduces complexity for stakeholders [\ldots] [and] determines situational actions [\ldots].'' \parencite{albert2008}
In this sense, values are profound, personal determinants for individual or collective behavior.
The notion of values being organized in a weighted value pattern shows that values can be ranked according to importance.
For automated vehicles, \emph{physical wellbeing} and \emph{mobility} can, e.g., be considered values which need to be balanced to achieve societal acceptance, in line with the discussion of required tradeoffs in \cref{sec:terminology}.
For the analysis of the following regulatory frameworks, we will evaluate if the given safety or risk notions allow tradeoffs regarding underlying stakeholder values. 

\subsection{UN Regulation No. 157 \& European Implementing Regulation (EU) 2022/1426}
\label{sec:enabling-act}
UN Regulation No. 157 \parencite{un157} and the European Implementing Regulation 2022/1426 \parencite{eu1426} provide type approval regulation for automated vehicles equipped with SAE-Level-3 (UN Reg. 157) and Level 4 (EU 2022/1426) systems on an international (UN Reg. 157) and European (EU 2022/1426) level.

Generally, EU type approval considers UN ECE regulations mandatory for its member states ((EU) 2018/858, \parencite{eu858}), while the EU largely forgoes implementing EU-specific type approval rules, it maintains the right to alter or to amend UN ECE regulation \parencite{eu858}.

In this respect, the terminology and conceptualizations in the EU Implementing Act closely follow those in UN Reg. No. 157.
The EU Implementing Act gives a clear reference to UN Reg. No. 157 \parencite[][Preamble,  Paragraph 1]{eu1426}.
Hence, the documents can be assessed in parallel.
Differences will be pointed out as necessary.

Both acts are written in rather technical language, including the formulation of technical requirements (e.g., regarding deceleration values or speeds in certain scenarios).
While providing exhaustive definitions and terminology, neither of both documents provide an actual definition of risk or safety.
The definition of ``unreasonable'' risk in both documents does not define risk, but only what is considered \emph{unreasonable}. It states that the ``overall level of risk for [the driver, (only in UN Reg. 157)] vehicle occupants and other road users which is increased compared to a competently and carefully driven manual vehicle.''
The pertaining notions of safety and risk can hence only be derived from the context in which they are used.

\subsubsection{Absolute vs. Relative Notions of Safety}
In line with the technical detail provided in the acts, both clearly imply a \emph{relative} notion of safety and refer to the absence of \emph{unreasonable} risk throughout, which is typical for technical safety definitions.

Both acts require sufficient proof and documentation that the to-be-approved automated driving systems are ``free of unreasonable safety risks to vehicle occupants and other road users'' for type approval.\footnote{As it targets SAE-Level-3 systems, UN Reg. 157 also refers to the driver, where applicable.}
In this respect, both acts demand that the manufacturers perform verification and validation activities for performance requirements that include ``[\ldots] the conclusion that the system is designed in such a way that it is free from unreasonable risks [\ldots]''.
Additionally, \emph{risk minimization} is a recurring theme when it comes to the definition of Minimum Risk Maneuvers (MRM).

Finally, supporting the relative notions of safety and risk, UN Reg. 157 introduces the concept of ``reasonable foreseeable and preventable'' \parencite[Article 1, Clause 5.1.1.]{un157} collisions, which implies that a residual risk will remain with the introduction of automated vehicles.
\parencite[][Appendix 3, Clause 3.1.]{un157} explicitly states that only \emph{some} scenarios that are unpreventable for a competent human driver can actually be prevented by an automated driving system.
While this concept is not applied throughout the EU Implementing Act, both documents explicitly refer to \emph{residual} risks that are related to the operation of automated driving systems (\parencite[][Annex I, Clause 1]{un157}, \parencite[][Annex II, Clause 7.1.1.]{eu1426}).

\subsubsection{Hazard Sources}
Hazard sources that are explicitly differentiated in UN Reg. 157 and (EU) 2022/1426 are E/E-failures that are in scope of functional safety (ISO~26262) and functional insufficiencies that are in scope of behavioral (or ``operational'') safety (ISO~21448).
Both documents consistently differentiate both sources when formulating requirements.

While the acts share a common definition of ``operational'' safety (\parencite[][Article 2, def. 30.]{eu1426}, \parencite[][Annex 4, def. 2.15.]{un157}), the definitions for functional safety differ.
\parencite{un157} defines functional safety as the ``absence of unreasonable risk under the occurrence of hazards caused by a malfunctioning behaviour of electric/electronic systems [\ldots]'', \parencite{eu1426} drops the specification of ``electric/electronic systems'' from the definition.
When taken at face value, this definition would mean that functional safety included all possible hazard sources, regardless of their origin, which is a deviation from the otherwise precise usage of safety-related terminology.

\subsubsection{Harm Types}
As the acts lack explicit definitions of safety and risk, there is no consistent and explicit notion of different harm types that could be differentiated.

\parencite{un157} gives little hints regarding different considered harm types.
``The absence of unreasonable risk'' in terms of human driving performance could hence be related to any chosen performance metric that allows a comparison with a competent careful human driver including, e.g., accident statistics, statistics about rule violations, or changes in traffic flow.

In \parencite{eu1426}, ``safety'' is, implicitly, attributed to the absence of unreasonable risk to life and limb of humans.
This is supported by the performance requirements that are formulated:
\parencite[][Annex II, Clause 1.1.2. (d)]{eu1426} demands that an automated driving system can adapt the vehicle behavior in a way that it minimizes risk and prioritizes the protection of human life.

Both acts demand the adherence to traffic rules (\parencite[][Annex 2, Clause 1.3.]{eu1426}, \parencite[][Clause 5.1.2.]{un157}).
\parencite[][Annex II, Clause 1.1.2. (c)]{eu1426} also demands that an automated driving system shall adapt its behavior to surrounding traffic conditions, such as the current traffic flow.
With the relative notion of risk in both acts, the unspecific clear statement that there may be unpreventable accidents \parencite{un157}, and a demand of prioritization of human life in \parencite{eu1426}, both acts could be interpreted to allow developers to make tradeoffs as discussed in \cref{sec:terminology}.


\subsubsection{Conclusion}
To summarize, the UN Reg. 157 and the (EU) 2022/1426 both clearly support the technical notion of safety as the absence of unreasonable risk.
The notion is used consistently throughout both documents, providing a sufficiently clear terminology for the developers of automated vehicles.
Uncertainty remains when it comes to considered harm types: Both acts do not explicitly allow for broader notions of safety, in the sense of \parencite{koopman2024} or \parencite{salem2024}.
Finally, a minor weak spot can be seen in the definition of risk acceptance criteria: Both acts take the human driving performance as a baseline.
While (EU) 2022/1426 specifies that these criteria are specific to the systems' Operational Design Domain \parencite[][Annex II, Clause 7.1.1.]{eu1426}, the reference to the concrete Operational Design Domain is missing in UN Reg. 157.
Without a clearly defined notion of safety, however, it remains unclear, how aspects beyond net accident statistics (which are given as an example in \parencite[][Annex II, Clause 7.1.1.]{eu1426}), can be addressed practically, as demanded by \parencite{koopman2024}.

\subsection{German Regulation (StVG \& AFGBV)}
\label{sec:afgbv}
The German L3 (Automated Driving Act) and L4 (Act on Autonomous Driving) Acts from 2017 and 2021,\footnote{Formally, these are amendments to the German Road Traffic Act (StVG): 06/21/2017, BGBl. I p. 1648, 07/12/2021 BGBl. I p. 3108.} respectively, provide enabling regulation for the operation of SAE-Level-3 and 4 vehicles on German roads.
The German Implementing Regulation (\parencite{afgbv}, AFGBV) defines how this enabling regulation is to be implemented for granting testing permits for SAE-Level-3 and -4 and driving permits for SAE-Level-3 and -4 automated driving systems.\footnote{Note that these permits do not grant EU-wide type approval, but serve as a special solution for German roads only. At the same time, the AFGBV has the same scope as (EU) 2022/1426.}
With all three acts, Germany was the first country to regulate the approval of automated vehicles for a domestic market.
All acts are subject to (repeated) evaluation until the year 2030 regarding their impact on the development of automated driving technology.
An assessment of the German AFGBV and comparisons to (EU) 2022/1426 have been given in \cite{steininger2022} in German.

Just as for UN Reg. 157 and (EU) 2022/1426, neither the StVG nor the AFGBV provide a clear definition of ``safety'' or ``risk'' -- even though the "safety" of the road traffic is one major goal of the StVG and StVO.
Again, different implicit notions of both concepts can only be interpreted from the context of existing wording.
An additional complication that is related to the German language is that ``safety'' and ``security'' can both be addressed as ``Sicherheit'', adding another potential source of unclarity.
Literal Quotations in this section are our translations from the German act.

\subsubsection{Absolute vs. Relative Notions of Safety}
For assessing absolute vs. relative notions of safety in German regulation, it should be mentioned that the main goal of the German StVO is to ensure the ``safety and ease of traffic flow'' -- an already diametral goal that requires human drivers to make tradeoffs.\footnote{For human drivers, this also creates legal uncertainty which can sometimes only be settled in a-posteriori court cases.}
While UN and EU regulation clearly shows a relative notion of safety\footnote{And even the StVG contains sections that use wording such as ``best possible safety for vehicle occupants'' (§1d (4) StVG) and acknowledges that there are unavoidable hazards to human life (§1e (2) No. 2c)).}, the German AFGBV contains ambiguous statements in this respect:
Several paragraphs contain a demand for a hazard free operation of automated vehicles.
§4 (1) No. 4 AFGBV, e.g., states that ``the operation of vehicles with autonomous driving functions must neither negatively impact road traffic safety or traffic flow, nor endanger the life and limb of persons.''
Additionally, §6 (1) AFGBV states that the permits for testing and operation have to be revoked, if it becomes apparent that a ``negative impact on road traffic safety or traffic flow, or hazards to the life and limb of persons cannot be ruled out''.
The same wording is used for the approval of operational design domains regulated in §10 (1) No. 1.
A particularly misleading statement is made regarding the requirements for technical supervision instances which are regulated in §14 (3) AFGBV which states that an automated vehicle has to be  ``immediately removed from the public traffic space if a risk minimal state leads to hazards to road traffic safety or traffic flow''.
Considering the argumentation in \cref{sec:terminology}, that residual risks related to the operation of automated driving systems are inevitable, these are strong statements which, if taken at face value, technically prohibit the operation of automated vehicles.
It suggests an \emph{absolute} notion of safety that requires the complete absence of risk.  
The last statement above is particularly contradictory in itself, considering that a risk \emph{minimal} state always implies a residual risk.

In addition to these absolute safety notions, there are passages which suggest a relative notion of safety:
The approval for Operational Design Domains is coupled to the proof that the operation of an automated vehicle ``neither negatively impacts road traffic safety or traffic flow, nor significantly endangers the life and limb of persons beyond the general risk of an impact that is typical of local road traffic'' (§9 (2) No. 3 AFGBV).
The addition of a relative risk measure ``beyond the general risk of an impact'' provides a relaxation (cf. also \cite{steininger2022}, who criticizes the aforementioned absolute safety notion) that also yields an implicit acceptance criterion (\emph{statistically as good as} human drivers) similar to the requirements stated in UN Reg. 157 and (EU) 2022/1426.

Additional hints for a relative notion of safety can be found in Annex 1, Part 1, No. 1.1 and Annex 1, Part 2, No. 10.
Part 1, No 1.1 specifies collision-avoidance requirements and acknowledges that not all collisions can be avoided.\footnote{The same is true for Part 2, No. 10, Clause 10.2.5.}
Part 2, No. 10 specifies requirements for test cases.
It demands that test cases are suitable to provide evidence that the ``safety of a vehicle with an autonomous driving function is increased compared to the safety of human-driven vehicles''.
This does not only acknowledge residual risks, but also yields an acceptance criterion (\emph{better} than human drivers) that is different from the implied acceptance criterion given in §9 (2) No. 3 AFGBV.

\subsubsection{Hazard Sources}
Regarding hazard sources, Annex 1 and 3 AFGBV explicitly refer to ISO~26262 and ISO~21448 (or rather its predecessor ISO/PAS~21448:2019).
However, regarding the discussion of actual hazard sources, the context in which both standards are mentioned is partially unclear:
Annex 1, Clause 1.3 discusses requirements for path and speed planning.
Clause 1.3 d) demands that in intersections, a Time to Collision (TTC) greater than 3 seconds must be guaranteed.
If manufacturers deviate from this, it is demanded that ``state-of-the-art, systematic safety evaluations'' are performed.
Fulfillment of the state of the art is assumed if ``the guidelines of ISO~26262:2018-12 Road Vehicles -- Functional Safety are fulfilled''.
Technically, ISO~26262 is not suitable to define the state of the art in this context, as the requirements discussed fall in the scope of operational (or behavioral) safety (ISO~21448).
A hazard source ``violated minimal time to collision'' is clearly a functional insufficiency, not an E/E-failure.

Similar unclarity presents itself in Annex 3, Clause 1 AFGBV: 
Clause 1 specifies the contents of the ``functional specification''.
The ``specification of the functionality'' is an artifact which is demanded in ISO~21448:2022 (Clause 5.3) \parencite{iso21448}.
However, Annex 3, Clause 1 AFGBV states that the ``functional specification'' is considered to comply to the state of the art, if the ``functional specification'' adheres to ISO~26262-3:2018 (Concept Phase).
Again, this assumes SOTIF-related contents as part of ISO~26262, which introduces the ``Item Definition'' as an artifact, which is significantly different from the ``specification of the functionality'' which is demanded by ISO~21448.
Finally, Annex 3, Clause 3 AFGBV demands a ``documentation of the safety concept'' which ``allows a functional safety assessment''.
A safety concept that is related to operational / behavioral safety is not demanded.
Technically, the unclarity with respect to the addressed harm types lead to the fact that the requirements provided by the AFGBV do not comply with the state of the art in the field, providing questionable regulation.

\subsubsection{Harm Types}
Just like UN Reg. 157 and (EU) 2022/1426, the German StVG and AFGBV do not explicitly differentiate concrete harm types for their notions of safety.
However, the AFGBV mentions three main concerns for the operation of automated vehicles which are \emph{traffic flow} (e.g., §4 (1) No. 4 AFGBV), compliance to \emph{traffic law} (e.g., §1e (2) No. 2 StVG), and the \emph{life and limb of humans} (e.g., §4 (1) No. 4 AFGBV).

Again, there is some ambiguity in the chosen wording:
The conflict between traffic flow and safety has already been argued in \cref{sec:terminology}.
The wording given in §4 (1) No. 4 and §6 (1) AFGBV  demand to ensure (absolute) safety \emph{and} traffic flow at the same time, which is impossible (cf. \cref{sec:terminology}) from an engineering perspective.
§1e (2) No. 2 StVG defines that ``vehicles with an autonomous driving function must [\ldots] be capable to comply to [\ldots] traffic rules in a self-contained manner''.
Taken at face value, this wording implies that an automated driving system could lose its testing or operating permit as soon as it violates a traffic rule.
A way out could be provided by §1 of the German Traffic Act (StVO) which demands careful and considerate behavior of all traffic participants and by that allows judgement calls for human drivers.
However, if §1 is applicable in certain situations is often settled in court cases. 
For developers, the application of §1 StVO during system design hence remains a legal risk.

While there are rather absolute statements as mentioned above, sections of the AFGBV and StVG can be interpreted to allow tradeoffs:
§1e (2) No. 2 b) demands that a system,  ``in case of an inevitable, alternative harm to legal objectives, considers the significance of the legal objectives, where the protection of human life has highest priority''.
This exact wording \emph{could} provide some slack for the absolute demands in other parts of the acts, enabling tradeoffs between (tolerable) risk and mobility as discussed in \cref{sec:terminology}.
However, it remains unclear if this interpretation is legally possible.

\subsubsection{Conclusion}
Compared to UN Reg. 157 and (EU) 2022/1426, the German StVG and AFGBV introduce openly inconsistent notions of safety and risk which are partially directly contradictory:
The wording partially implies absolute and relative notions of safety and risk at the same time.
The implied validation targets (``better'' or ``as good as'' human drivers) are equally contradictory. 
The partially implied absolute notions of safety, when taken at face value, prohibit engineers from making the tradeoffs required to develop a system that is safe and provides customer benefit at the same time. 
In consequence, the wording in the acts is prone to introducing legal uncertainty.
This uncertainty creates additional clarification need and effort for manufacturers and engineers who design and develop SAE-Level-3 and -4 automated driving systems. The use of undefined legal terms not only makes it more difficult for engineers to comply with the law, but also complicates the interpretation of the law and leads to legal uncertainty.

\subsection{UK Automated Vehicles Act 2024 (2024 c. 10)}
The UK has issued a national enabling act for regulating the approval of automated vehicles on the roads in the UK.
To the best of our knowledge, concrete implementing regulation has not been issued yet.
Regarding terminology, the act begins with a dedicated terminology section to clarify the terms used in the act \parencite[Part 1, Chapter 1, Section 1]{ukav2024}.
In that regard, the act defines a vehicle to drive ```autonomously' if --- (a)
it is being controlled not by an individual but by equipment of the vehicle, and (b) neither the vehicle nor its surroundings are being monitored by an individual with a view to immediate intervention in the driving of the vehicle.''
The act hence covers SAE-Level-3 to SAE-Level-5 automated driving systems.

\subsubsection{Absolute vs. Relative Notions of Safety}
While not providing an explicit definition of safety and risk, the UK Automated Vehicles Act (``UK AV Act'') \parencite{ukav2024} explicitly refers to a relative notion of safety.
Part~1, Chapter~1, Section~1, Clause (7)~(a) defines that an automated vehicle travels ```safely' if it travels to an acceptably safe standard''.
This clarifies that absolute safety is not achievable and that acceptance criteria to prove the acceptability of residual risk are required, even though a concrete safety definition is not given.
The act explicitly tasks the UK Secretary of State\footnote{Which means, that concrete implementation regulation needs to be enacted.} to install safety principles to determine the ``acceptably safe standard'' in Part~1, Chapter~1, Section~1, Clause (7)~(a).
In this respect, the act also provides one general validation target as it demands that the safety principles must ensure that ``authorized automated vehicles will achieve a level of safety equivalent to, or higher than, that of careful and competent human drivers''.
Hence, the top-level validation risk acceptance criterion assumed for UK regulation is ``\emph{at least as good} as human drivers''.

\subsubsection{Hazard Sources}
The UK AV Act contains no statements that could be directly related to different hazard sources.
Note that, in contrast to the rest of the analyzed documents, the UK AV Act is enabling rather than implementing regulation.
It is hence comparable to the German StVG, which does not refer to concrete hazard sources as well.

\subsubsection{Types of Harm}
Even though providing a clear relative safety notion, the missing definition of risk also implies a lack of explicitly differentiable types of harm.
Implicitly, three different types of harm can be derived from the wording in the act.
This includes the harm to life and limb of humans\footnote{Part~1, Chapter~3, Section~25 defines ``aggravated offence where death or serious injury occurs'' \parencite{ukav2024}.}, the violation of traffic rules\footnote{Part~1, Chapter~1, Clause~(7)~(b) defines that an automated vehicle travels ```legally' if it travels with an acceptably low risk of committing a traffic infraction''}, and the cause of inconvenience to the public \parencite[Part~1, Chapter~1, Section~58, Clause (2)~(d)]{ukav2024}.

The act connects all the aforementioned types of harm to ``risk'' or ``acceptable safety''.
While the act generally defines criminal offenses for providing ``false or misleading information about safety'', it also acknowledges possible defenses if it can be proven that ``reasonable precautions'' were taken and that ``due diligence'' was exercised to ``avoid the commission of the offence''.
This statement could enable tradeoffs within the scope of ``reasonable risk'' to the life and limb of humans, the violation of traffic rules, or to the cause of inconvenience to the public, as we argued in \cref{sec:terminology}.

\subsubsection{Conclusion}
From the set of reviewed documents, the current UK AV Act is the one with the most obvious relative notions of safety and risk and the one that seems to provide a legal framework for permitting tradeoffs.
In our review, we did not spot major inconsistency beyond a missing definitions of safety and risk\footnote{Note that with the Office for Product Safety and Standards (OPSS), there is a British government agency that maintains an exhaustive and widely focussed ``Risk Lexicon'' that provides suitable risk definitions. For us, it remains unclear, to what extent this terminology is assumed general knowledge in British legislation.}.
The general, relative notion of safety and the related alleged ability for designers to argue well-founded development tradeoffs within the legal framework could prove beneficial for the actual implementation of automated driving systems.
While the act thus appears as a solid foundation for the market introduction of automated vehicles, without accompanying implementing regulation, it is too early to draw definite conclusions.

\IEEEpubidadjcol

\section{Application: Harnessing the Linearity}
\label{sec:application}
Leveraging the \emph{linearity} of DMD operator, as well as the intuition of bases exposed by the spectral decomposition, we have developed several novel applications that extend the capabilities of our Koopman-based reduced-order simulation pipeline. In this section, we explore these applications, demonstrating that our method's unique strengths translate into practical tools for graphics and simulation.

\subsection{Direct Editing Temporal Dynamics}
\label{sec:editing}
\begin{figure}[!ht]
    \centering
    \includegraphics[width=1\columnwidth]{figure/karman_vortex_street_editing.pdf}
    \caption{\textbf{Editing temporal dynamics of K\'arm\'an Vortex Street with the Koopman Operator Approximation}. The modifications are applied to the DMD basis coefficients: (a) Scaling the modulus of the DMD basis by factors of 0.5, 1.0, and 1.5, affecting overall amplitude; (b) Adjusting the real part of $\bm{\Omega}$, influencing growth and decay rates of modal contributions; (c) Modifying the imaginary part, altering phase dynamics and wave propagation characteristics. }
    \label{fig:karman_editing}
    \Description{}
\end{figure}


\begin{figure*}[!ht]
    \centering
    \includegraphics[width=1\linewidth]{figure/reversibility.pdf}
    \caption{\textbf{Reversibility of Flows with Inversed DMD Operator}. We compare the reconstruction of two distinct fluid flows using Dynamic Mode Decomposition (DMD). The top row in each panel shows the velocity L2-norm of the field used to train the DMD, while the second and third rows depict the temporal evolution of the reconstructed flow fields as applied to an initial density field. The forward-time training phase is followed by a backward-time testing phase to assess predictive accuracy when advecting backward in time. The bottom plots show the evolution of kinetic energy over time. From the buoyant case, we observe the inverted DMD operator $\bm{A^{-1}}$ can still reasonably trace backward in time without compromising much visual quality. The vortical case exhibits a more challenging example where the symmetry should be reconstructed backwards in time. We see that the inverse operator indeed recovers this symmetry, with some acceptable levels of incurred noise. Bottom plots show the evolution of the total kinetic energy over time, demonstrating that our inverse operator actually correctly reverses the arrow of time, reversing the dissipation-related entropy increase over time. Decreasing kinetic energy also validates the \emph{physical plausibility} of our result.}
    \label{fig:reverse_simulation}
    \Description{}
\end{figure*}


Since our method approximates \refeq{eqn:euler_equations} with a linear operator in the full space, this allows us to transform the operator acting on the velocity field into the evolution of different modes under a linear operator. Therefore, we can directly edit the temporal dynamics of the fluid system by modifying the modes of the reduced \koopman{} $\bm{\hat{K}}$:
we set $t_0$ to be the initial time, $\bm{\Omega} = \nicefrac{\log(\bm\Lambda)}{\Delta t}$, where $\Delta t$ is the time step of the dataset. With this, we can rewrite \refeq{eqn:reduced_koopman_simulation} in the following form:
\begin{equation}
    \begin{aligned}
    \bm{u}(t_0 + k\Delta t) &= \bm{\Phi}\exp(\bm{\Omega} t) \bm{z}(t_0) \\
    &= \bm{\Phi}\exp{\left(k(\log(r) + i\theta)\right)} \bm{z}(t_0) \\
    &= \sum_{i = 1}^{n} {w_i} \bm{\Phi_i} r_i^k \left(\cos(k\theta_i) + \sin(k\theta_i)\right) \bm{z_i}(t_0)\\
    \end{aligned}
    \label{eqn:edit_temporal}
\end{equation}
% explanation for the formula
where ${w_i}$ is a user-defined scalar weight, $r_i = \sqrt{\Re(\lambda_i)^2 + \Im(\lambda_i)^2}$ is the \emph{modulus} and $\theta_i = \arctan\left(\Im(\lambda_i), \Re(\lambda_i)\right)$ is the \emph{phase} of the $i$-th eigenvalue $\lambda_i$ in the diagonal \emph{complex} eigenvalue matrix $\bm{\Lambda}$. Notice that this implies that the modes of the spectral decomposition represent different scales of vorticity, completing the physical intuition of the reduced space modes.
% show the benefits of our method for artist to edit

As shown in \refeq{eqn:edit_temporal}, our method decomposes a simulation sequence into modes with different growth/decay rates and frequencies.
The growth/decay rate of a mode is reflected in $r_i$, where a larger $r_i$ indicates a higher growth rate (or a lower decay rate), and vice versa.
The frequency of a mode is represented by the absolute value of $\theta_i$, with a larger absolute value corresponding to a higher frequency mode, and vice versa.
Furthermore, the different modes are decoupled, allowing for the adjustment of the relative proportions between modes.
As a result, these properties provide the artist with powerful tools to edit the simulation playback. The artist can modify the overall velocity field by adjusting the proportion ($w_i$), growth/decay rate ($r_i$), and frequency ($\theta_i$) of specific modes.
% explanation for what we actually do in code
In the experiments, we directly adjust the real part of $\bm{\Omega_i}$ to control $r_i$, modify the imaginary part of $\bm{\Omega_i}$ to control $\theta_i$, and vary the modulus of $\bm{\Phi_i}$ to control $w_i$.
% explanation for what we did to edit in karman vortex street scene
\paragraph{Editing the K\'arm\'an Vortex Street}
The first example is editing on the classic K\'arm\'an vortex street. We filter the imaginary part of $\bm{\Omega}$ and cluster modes with an absolute value smaller than $0.01$ as \emph{low-frequency cluster}, and the rest as \emph{high-frequency cluster}.
The low-frequency mode manifests as a laminar flow, with its phase changing very slowly over time. The high-frequency mode is represented by vortical structures distributed on both sides of the cylinder, where the phase of this mode changes relatively quickly over time.
As seen in \reffig{fig:karman_editing}, when we adjust the modulus of the high-frequency cluster from $0.5$ to $1.5$, the intensity of the vortices increases, which is as we expected. When we set the real part of $\bm{\Omega}$ to $0.5$, it can be observed that the high-frequency motion decays faster than user input. When we set the real part of $\bm{\Omega}$ to $1.5$, it can be observed that the high-frequency motion decays slower than user input. Similarly, when we tune the imaginary part of $\bm{\Omega}$ from $0.5$ to $1.5$, we could observe the oscillation frequency of the fluid trail transitions from slow to fast compared to user input.
% explanation for what we did to edit in 3D plume scene
\paragraph{Editing the Plume with Bunny}
To evaluate the editing capability of our method, we scale our editing scenario to 3D. With the same filtering procedure as in the K\'arm\'an vortex street example, we set the low-frequency cluster to high-frequency cluster ratio to $4:1$, $2:1$, $1:2$, and $1:4$, and compared the results with the user input. From the results, we observe that when the proportion of low-frequency cluster is increased, with a ratio of $4:1$, the top of the plume lacks "wrinkles" and appears more "fluffy". This is because the velocity field is dominated by smoother, lower-frequency modes than the original user input. Conversely, when the proportion of high-frequency cluster is increased, with ratios of $1:4$, the plume developes more detailed plume structure around the top, as the velocity field now emphasizes more high-frequency details compared to the user input.

\subsection{Reversibility of the Reduced Simulation}
Although physically-based fluid simulations have the capability to generate stunning visuals, when artists aim to direct the fluid's evolution toward a predefined target shape, challenges arise. It is a long standing problem in the community that people aim to enable users with \emph{spatial control}. In this example, we aim to enable users to do \emph{temporal control}, motived by a prior work \citet{oborn2018time}. Compared to previous work \shortcite{oborn2018time} where the authors employ a self-attraction force to replace the arbitrary external forces, providing a stable, physics-motivated, but time-consuming approach, we propose a data-driven, fast, and easy to implement method to address the same problem.

\label{sec:reversibility}

We observe that that given $\bm{\tilde{K}} = \bm{\Phi} \bm{\Lambda} \bm{\Phi}^+$, we could easily compute the \emph{inverse} of the truncated \koopman{} $\bm{\tilde{K}}^{-1} = (\bm{\Phi} \bm{\Lambda} \bm{\Phi}^+)^{-1} = \bm{\Phi} \bm{\Lambda}^{-1} \bm{\Phi}^+$, which is essentially the approximate inverse time evolution $\bm{f}^{-1}(\bm u)$ of the fluid system. This allows us to reverse the simulation by applying the inverse truncated \koopman{} to the current state of the fluid system:
\begin{equation}
    \label{eqn:reverse_simulation}
    \begin{aligned}
        \bm{u}(t) &= \bm{A}^{-1} \bm{u}(t + \Delta t), \\
        \bm{u}(t) &= \bm{\Phi} \bm{\Lambda}^{-1}\bm{\Phi}^+ \bm{u}(t + \Delta t), \\
        \bm{u}(t) &= \bm{\Phi} \bm{\Lambda}^{-1} \bm{z}(t + \Delta t).
    \end{aligned}
\end{equation}

Similar to \refeq{eqn:reduced_koopman_projection}, we could train the reduced \koopman{} on the forward simulation data and then apply the inverse reduced \koopman{} to reverse the simulation, given a state of the fluid system.


\begin{figure*}[!ht]
    \centering
    \includegraphics[width=1\linewidth]{figure/upsample.pdf}
    \caption{\textbf{Upsampling and Generalization to Unseen Sequences with Trained DMD Operator}. Two different input low-resolution fluid simulations (bunny and strawberry) are upscaled using the same DMD operator trained on a different velocity field. Initial velocity fields are seeded as moving down based on the input density field.    
    Naive application of DMD shown in each middle column, and our \emph{augmented DMD upresolution} method shown on the right columns. 
    Schematic of our method presented on the far right. At each frame, we project the low-resolution artist-directed input into the low-order bases of our reduced representation, using these to replace the low-order terms of the DMD field. Notice that naive application of DMD simply moves towards the known input training data, while our augmented field matches the low-resolution input more closely, with extra high-order detail gained from the DMD operator.}
    \label{fig:upsample}
    \Description{}
\end{figure*}


% first explanation for buoyant reversibility
\paragraph{Reversibility of Buoyant Flow}
We experiment our approach on a simple buoyant flow setup (\reffig{fig:reverse_simulation}, left). Our dataset was initialized with a \textit{qian}, a density field shaped like a round coin with a square hole, with the density value set to $1$. A density value of $1$ density field was driven by a velocity field where an upwards velocity of $0.3$ is set within the qian and downwards elsewhere. We run the simulation for $300$ frames to construct the dataset, and trained the DMD operator on this dataset. The inverse operator $\bm{\tilde{K}}^{-1}$ was then applied to the initial velocity field of the dataset at $t=0$ (frame $0$). By iteratively applying the inverse operator, we obtained the velocity fields for the preceding frames, starting from frame $-1$, frame $-2$, and all the way back to frame $-300$.
% stability
When examining the evolution of the density field from frame -300 to frame 300, it is evident that the velocity field remains consistently upward and smooth, indicating that our method is both reasonable and effective.
% energy
Further analysis of the energy of the velocity field obtained through the inverse process and the velocity field from the dataset reveals a downward trend in energy, with a smooth and reasonable curve, consistent with fluids with dissipative properties. This demonstrates that our inverse operator has the ability to predict a \emph{physically-plausible} velocity field prior to the dataset.

% second explanation for vortical reversibility
\paragraph{Reversibility of Vortical Flow}
To challenge the method with a scene of nontrivial vortical structure, we initialized a vortex sheet by placing four vortices at the corners of the domain (\reffig{fig:reverse_simulation}, right). We generated the dataset using the same procedure as in the previous experiment, resulting in a collection of $500$ frames. Subsequently, we constructed the inverse operator to recover the velocity fields preceding the dataset.
% stability
The results show that the density field (counterclockwise) and the dataset (clockwise) rotate in the opposite direction, which indicates that the velocity field predicted by the inverse operator is correct. This is because the vortex sheet velocity field continuously rotates in a clockwise direction, and by examining the density field from frame -500 to frame 500, we observe that the field indeed undergoes continuous clockwise rotation.
% energy
From the energy field analysis, the results show that, except for the significant energy fluctuation between frames -500 and -450, the energy consistently decreases in the remaining frames, with a consistent slope. This further demonstrates the robustness of our method.

\subsection{Upsampling with Reduced Koopman Operator}

The scale of the imaginary part of eigenvalues in $\bm{\Lambda}$ encode different scales of turbulent modes, enabling us to use a trained DMD operator to add in secondary motion to an existing fluid simulation. This is particularly useful for \emph{upscaling} a low-resolution fluid, simulated using stable fluid for example, leveraging the DMD basis to add in turbulent modes that were too small for the low-res sim to capture. This upscaling problem has been explored in prior work \cite{kim2008wavelet, nielsen2009guiding}, but we show that due to the linearity of the Koopman operator, and the physical intuition on each of its reduced bases, this upscaling is essentially attained for \emph{free}, amounting to nothing more than a linear combination of two matrix multiplications. 

\subsubsection{Evolution} \label{sec:upres_direct}

Suppose we have frames of a low-res input velocity field $\{\bm{L}_0, \bm{L}_1, \bm{L}_2, \dots, \bm{L}_T\}$, a high-res initial condition $H_0$. Additionally, we have some DMD basis $\bm{\Phi}$ trained on some high-res simulation distinct from the low-res simulation, with corresponding eigenvalues $\bm{\Lambda}$, sorted by the length of their imaginary parts in increasing order. At the first frame, we can generate the reduced-space initial condition by simply using our basis mapping $R_0 = \bm{\Phi}^TH_0$.

Now, for every subsequent frame $t$, we generate $R_t$ by first applying the DMD evolution on the previous reduced space frame to produce an intermediate state $R^*_t=\bm{\Lambda}R_{t-1}$. We also produce a representation of the current frame of the low-res input in reduced space $P_t = \bm{\Phi}^TL_t$. Now, we have a representation of the \emph{current} frame of the low-res input, and the DMD \emph{time evolution} of the \emph{previous} reduced space frame. We want to keep the low-order bulk flow of the low-res input, and augment it with the high-order turbulent flow learned by the DMD basis. To that end, we split each reduced-space vector into a low-order and high-order part: $R^*_t = \left[R_t^{*L}\ R_t^{*H}\right]$, $P_t=\left[P_t^L\ P_t^H\right]$. Now, we take only the low-order modes of the input flow, and the high-order modes of the DMD-evolved flow, to produce our new reduced space velocity field $R_t=\left[P_t^L\ R_t^{*H}\right]$. From here, we can just apply the basis to return to high-resolution full-space $H_t=\bm{\Phi}R_t$.

We note that the composition operators here are linear. We can simply represent them with selection matrices $S^H$, $S_L$, for the high- and low-order bases respectively, such that $R_t=S^LP_t + S^HR_t^*$. Since the DMD operator is also linear, we note that this entire upscaling method is linear by construction.

Results are shown on Figure \ref{fig:upsample}. We see that even if the initial velocity field is significantly different from the input field, the low-order basis is able to capture the bulk flow of the low-resolution input, and modify the DMD-produced field accordingly. In particular, we note that naively applying the DMD operator, without passing the low-resolution input field into the low-order bases, ends up reconstructing the original training set, rather than a velocity field directed by our input. This is demonstrated by the results for the two initial conditions being very similar, whereas our augmented field matches the input much closer.

\subsubsection{Projection}

The above governs the time evolution of the velocity field. In some cases, where the input velocity field differs significantly from the training data used for the DMD basis, the above as written will still produce velocity fields that are unacceptably different from the input velocity field. This is largely representation error, fields that are far away from the training data are less representable by the reduced space. In these cases, we can again leverage our input low-res field, this time as a constraint. 

Essentially, we would like to project our velocity field $\bm{H}_t$ onto the space of velocity fields that are identical to the input low-res field when downsampled to that resolution. This can be represented as an equality-constrained quadratic problem,
\begin{align}
    &\argmin_x \frac{1}{2}(\bm{x}-\bm{H}_t)^T(\bm{x}-\bm{H}_t) \\
    &\text{subject to } \bm{Ax} = \bm{L}_t,
\end{align}
where $\bm{A}$ is a downsampling operator that converts from high-res to low-res. 
Notice that because the downsampling operator does not change for the duration of the simulation. Thus, the KKT (Karush-Kuhn-Tucker) matrix can be precomputed making the projection a single matrix multiply during runtime.

\begin{wrapfigure}{r}{0.5\columnwidth}
    \vspace{-2pt}
    \includegraphics[width=0.5\columnwidth]{figure/qr.pdf}
    \hspace{5pt}
    \label{fig:qp_project}
\end{wrapfigure}

As a sanity check, we show the effect of this projection here: it is apparent with the projection step,we can recover fields that are much closer to the input, yet retaining extra high-order detail. And of course, because these are all linear, linear combinations of the direct and projected fields can be taken. In particular, because the basis functions of reduced space are orthogonal, a diagonal matrix of linear weights can be taken, preferring projected for low-order modes and direct for high-order modes for example.

\begin{comment}
Given a high-resolution DMD matrix $A \in \mathbb{R}^{N_{hi} \times N_{hi}}$ trained on high-dimensional data, we reconstruct a high-resolution sequence $\bm{x}_{hi}(t) \in \mathbb{R}^{N_{hi}}$ using an initial high-resolution frame $\bm{x}_{hi}(0)$ and subsequent low-resolution frames $\bm{x}_{lo}(t) \in \mathbb{R}^{N_{lo}}$, where $N_{lo} < N_{hi}$. The matrix $A$ is structured as
\begin{equation}
    \label{eqn:slice_A}
    A = \begin{bmatrix} A_{ll} & A_{lh} \\ A_{hl} & A_{hh} \end{bmatrix}
\end{equation}, with $A_{ll} \in \mathbb{R}^{N_{lo} \times N_{lo}}$ map low frequency component to, $A_{lh} \in \mathbb{R}^{N_{lo} \times (N_{hi} - N_{lo})}$, $A_{hl} \in \mathbb{R}^{(N_{hi} - N_{lo}) \times N_{lo}}$, and $A_{hh} \in \mathbb{R}^{(N_{hi} - N_{lo}) \times (N_{hi} - N_{lo})}$.

Starting from the initial condition $\bm{x}_{hi}(0)$, the high-resolution state at time $t + \Delta t$ is updated using:
\begin{equation}
    \label{eqn:upsampling_advect}
    \bm{x}_{hi}(t + \Delta t) = A \bm{x}_{hi}(t) + \begin{bmatrix} \bm{x}_{lo}(t + \Delta t) - A_{ll} \bm{x}_{lo}(t) \\ A_{hl} \left( \bm{x}_{lo}(t + \Delta t) - A_{ll} \bm{x}_{lo}(t) \right) \end{bmatrix}.
\end{equation}

In this equation, $A \bm{x}_{hi}(t)$ evolves the high-resolution dynamics. The term $\bm{x}_{lo}(t + \Delta t) - A_{ll} \bm{x}_{lo}(t)$ represents the correction to the low-frequency component, and $A_{hl} \left( \bm{x}_{lo}(t + \Delta t) - A_{ll} \bm{x}_{lo}(t) \right)$ in-paints the missing high-frequency details. This process ensures the reconstructed high-resolution sequence remains consistent with the initial frame and the low-resolution input while leveraging the full dynamics encoded in $A$.

\end{comment}

\section{Challenges when Guiding with History}
\label{sec:history_guidance_challenges}

% \begin{figure*}[!t]
%     \centering    
%     \includegraphics[width=.84\linewidth]{assets/detailed_architecture.pdf}
%     \caption{\textbf{Detailed architecture illustration} of \sname if it is incoporated with W.A.L.T~\citep{gupta2023photorealistic}. 
%     } 
%     \label{fig:detailed_architecture}
% \end{figure*}
\section{Sampling Procedure}
\label{appen:sampling}
We provide detailed sampling procedure of \sname in Algorithm~\ref{algo:sampling}.

\begin{algorithm}[h!]
\begin{spacing}{1.05}
\caption{\sname Diffusion}\label{algo:sampling}
\begin{algorithmic}[1]
\For{$n=1$ to $N$} \Comment{{\it Autoregressively generate $n$-th latent vector $\bz^n$.}}
\State Sample the random noise $\bz_T^{n} \sim p(\bz_T)$.
\For{$i$ in $\{0, \ldots, M-1\}$}
\State Compute the score $\bm{\epsilon}_i \leftarrow D_{\bm{\theta}}(\bz_{i}^{n}, t;\, \bh^{n-1}, \bc)$.
\State Compute $\bz_{i+1}^n \leftarrow \bz_i^n + (t_{i+1} - t_{i})\bm{\epsilon}_i$ \Comment{{\it Euler solver; can be different with other solvers.}}
\EndFor
\State $\bh^{n} = \mathrm{HiddenState}\big(D_{\bm{\theta}} (\bz_M^{n}, 0;\, \bh^{i-1}, \bc)\big)$ \Comment{{\it Compute memory latent vector.}}
\EndFor
\State Decode $[\bx^1,\ldots,\bx^{N}]$ from generated latent vectors $[\bz^1,\ldots,\bz^{N}]$.
\State Output the generated video $[\bx^1,\ldots,\bx^{N}]$.
\end{algorithmic}
\end{spacing}
\end{algorithm}
 


Video diffusion models are conditional diffusion models  $p(\bx|\bc)$, where $\bx$ denotes frames to be generated, and $\bc$ represents the conditioning (e.g. text prompt, or a few observed prior frames). For simplicity, we refer to the latter as \emph{history}, even when the observed images could be e.g. a subset of keyframes that are spaced across time. Our discussion of $\bc$ will focus exclusively on history conditioning and exclude text or other forms of conditioning in notation. Formally, let $\bx_{\cT}$ denote a $T$-frame video clips with indices $\cT = \{1, 2, \ldots, T\}$. Define $\cH \subset \cT$ as the indices of history frames used for conditioning, and $\cG = \cT \setminus \cH$ as the indices of the frames to be generated. Our objective is to model the conditional distribution $p(\xG | \xH)$ with a diffusion model. 

We aim to extend classifier-free guidance (CFG) to this setting. Since the history $\xH$ serves as conditioning, sampling can be performed by estimating the following score: 
\begin{equation} 
\label{eq:history_guidance}
\score p_k(\xGk)
+ \omega \big[\score p_k(\xGk|\xH)  - \score p_k(\xGk)\big].
\end{equation}
This approach differs from conventional CFG in two ways: 1) The generation $\xG$ and conditioning history $\xH$ belong to the same signal $\bx_{\cT}$, differing only in their indices $\cG, \cH\subset \cT$; thus, the generated $\xG$ can be reused as conditioning $\xH$ for generating subsequent frames. 2) The history $\xH$ can be any subset of $\cT$, allowing its length to vary. Guiding with history, therefore, requires a model that can estimate both conditional and unconditional scores given arbitrary subsets of video frames. Below, we analyze how these differences present challenges for implementation within the current paradigm of video diffusion models (VDMs).

\textbf{Architectures with fixed-length conditioning.}
As shown in \cref{fig:architecture-conventional}, DiT~\cite{peebles2023scalable} or U-Net-based diffusion models~\cite{bao2023all,rombach2022high} typically inject conditioning using AdaLN~\cite{peebles2023scalable, perez2018film} layers or by concatenating the conditioning with noisy input frames along the channel dimension.
This design constrains conditioning to a fixed-size vector. While some models adopt sequence encoders for variable-length conditioning (e.g., for text inputs), these encoders are often pre-trained~\cite{yang2024cogvideox} and cannot share parameters with the diffusion model to encode history frames. %
Consequently, guidance has been limited to fixed-length and generally short history~\cite{blattmann2023stable, xing2023dynamicrafter, yang2024cogvideox, watson2024controlling}.

\textbf{Framewise Binary Dropout performs poorly.} 
Classifier-free guidance is typically implemented using a single network that jointly represents the conditional and unconditional models. These are trained via \emph{binary dropout}, where the conditioning variable $\bc$ is randomly masked during training  with a certain probability. 
History guidance can, in principle, be achieved by randomly dropping out subsets of history frames during training.
However, our ablations (Sec.~\ref{sec:exp_ablation}) reveal that this approach performs poorly. We hypothesize that this is due to inefficient token utilization: although the model processes all $|\cT|$ frames via attention, only a random subset of $|\cG|$ frames contribute to the loss. This becomes more pronounced as videos grow longer, making framewise binary dropout a suboptimal choice.





\vspace{-0.1in}
\section*{Acknowledgments}
We thank Haoran Wang, Jingyu Zhou, and Shuai Wang for their contribution in a tutorial related to this review paper.
% This should be a simple paragraph before the References to thank those individuals and institutions who have supported your work on this article.
\vspace{-0.1in}

\bibliographystyle{IEEEtran}
\bibliography{refs}
% \printbibliography


% \newpage

% \section{Biography Section}
% If you have an EPS/PDF photo (graphicx package needed), extra braces are
%  needed around the contents of the optional argument to biography to prevent
%  the LaTeX parser from getting confused when it sees the complicated
%  $\backslash${\tt{includegraphics}} command within an optional argument. (You can create
%  your own custom macro containing the $\backslash${\tt{includegraphics}} command to make things
%  simpler here.)
 
% \vspace{11pt}

% \bf{If you include a photo:}\vspace{-33pt}
% \begin{IEEEbiography}[{\includegraphics[width=1in,height=1.25in,clip,keepaspectratio]{author_photos/ywg.jpg}}]{Yiwei Guo}
% % Use $\backslash${\tt{begin\{IEEEbiography\}}} and then for the 1st argument use $\backslash${\tt{includegraphics}} to declare and link the author photo.
% % Use the author name as the 3rd argument followed by the biography text.
% received his B.E. degree of Artificial Intelligence from Shanghai Jiao Tong University, China, in 2023. He is currently a Ph.D. student in Computer Science and Engineering in Shanghai Jiao Tong University, under the supervision of Prof. Kai Yu. 
% His current research interests include text-to-speech synthesis and generative models.
% \end{IEEEbiography}

% \begin{IEEEbiography}[{\includegraphics[width=1in,height=1.25in,clip,keepaspectratio]{author_photos/xiechen.jpg}}]{Xie Chen} ?

% \end{IEEEbiography}

% \begin{IEEEbiography}[{\includegraphics[width=1in,height=1.25in,clip,keepaspectratio]{author_photos/kaiyu.jpg}}]{Kai Yu} ?

% \end{IEEEbiography}

% \vspace{11pt}

% \bf{If you will not include a photo:}\vspace{-33pt}
% \begin{IEEEbiographynophoto}{John Doe}
% Use $\backslash${\tt{begin\{IEEEbiographynophoto\}}} and the author name as the argument followed by the biography text.
% \end{IEEEbiographynophoto}


\vfill

\end{document}


