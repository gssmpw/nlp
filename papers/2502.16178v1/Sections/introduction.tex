\section{Introduction}
Online tutoring and learning have become an increasingly important and widely adopted form of education~\cite{paudel2021online,doi:10.3102/0034654315581420,dumford2018online,10.1145/3613904.3641985}. Specifically, the outbreak of the COVID-19 pandemic was a pivotal moment accelerating the transition to online learning, allowing people to observe its inherent benefits, scalability, and flexibility for online education~\cite{aristovnik2023impact, thomas2023so, whenthetutorbecomes}. This shift has led to broader acceptance and adoption of online learning which demands a larger amount of online-teaching tutors~\cite{barno2024scaling,whenthetutorbecomes,thomas2023so,paudel2021online}. To fill this gap, more novice tutors and part-time tutors are recruited. Many of these new online-teaching tutors are not professionally trained~\cite{thomas2023so}, and may lack sufficient knowledge to handle certain situations with students, such as managing technical issues, fostering student engagement, and planning courses~\cite{vlachopoulos2021quality}.
  
This trend has prompted researchers to explore innovative approaches to train and prepare tutors for the challenges of online learning environments~\cite{thomas2023so,rosenberg2021addressing}. A prominent method is scenario-based tutor training, which simulates real-world teaching scenarios to provide educators with practical training opportunities~\cite{developmentofscenario}. Notable examples include clinical simulations proposed by \citet{dotger2013had} and the PLUS system developed by \citet{personalizedlearning}. The emergence of large language models (LLMs) has introduced more effective and intelligent methods for scenario-based tutor training~\cite{jin2024teach,gpteach, lee23generative}. Systems like GPTeach~\cite{gpteach} and studies by \citet{lee23generative} have demonstrated the potential of LLMs in simulating realistic training scenarios for tutor development. While these works highlight the feasibility and benefits of scenario-based training, they fall short in addressing the critical aspect of designing and authoring targeted scenarios, such as those focusing on student engagement challenges. Therefore, we aimed to utilize LLMs to develop an effective scenario-based training system tailored for novice tutors to practice engaging students in online learning.

To identify the pressing needs of online tutors and inform the design of effective scenario-based training solutions, we conducted a formative study comprising of two surveys. The first survey included $N_1 = 86$ tutors, and the second survey included $N_2 = 102$ tutors, both from JANN\footnote{https://jann.mx}. JANN is an existing online learning platform that connects thousands of volunteer tutors with K-12 students for free online math sessions in Mexico. The first survey aimed to investigate the challenges tutors encounter when teaching online. The results revealed that \textbf{engagement issues} stand out to be the most significant challenge with tutors expressing a strong need for strategies and training to address this problem effectively.
To gain a deeper insight into novice tutors' problems with student engagement, we performed the second survey with open-ended questions which aimed to understand the specific scenarios in which student disengagement manifests and how tutors have been addressing these challenges. We used a thematic analysis method~\cite{joffe2011thematic} to analyze the result and identified four themes to represent different forms of student disengagement: \textit{Lack of Interest and Engagement}, \textit{Lack of Confidence}, \textit{Varying Learning Speeds}, and \textit{Fatigue and Focus Issues}. The findings from our two rounds of surveys provided valuable guidance for designing a scenario-based training system specifically targeting student engagement issues.

Based on these survey results, the literature review on the effectiveness of scenario-based methods~\cite{clark2009accelerating,preservice,grossman2008back,adapting,scenariobased,prospectsforchange} and the feasibility of using large language models (LLMs) to simulate students~\cite{gpteach}, we designed our system, \textit{TutorUp}, as illustrated in Fig. ~\ref{fig:system}. Focusing on providing training for novice tutors to address student engagement problems, \textit{TutorUp} leverages GPT-4o~\cite{openai2023chatgpt4} to simulate student conversations, presenting common scenarios of engagement challenges in online learning. Tutors can interact with simulated students by typing instructions, which provides a realistic teaching scenario for practice. Additionally, we provide both immediate and asynchronous feedback to help novice tutors practice engaging students with effective strategies. 

To assess the usefulness and usability of \textit{TutorUp}, we conducted a within-subjects user study with $16$ participants, who are novice tutors conducting remote tutoring.  
The contributions of this work are summarized as follows:
\begin{itemize}
\item \textbf{Survey of Challenges in Remote Tutoring}: We present findings from a formative study involving two rounds of surveys ($N1 = 86$, $N2 = 102$) with tutors teaching online. We identify student engagement as a central factor for learning success, outline common scenarios where disengagement can occur, and curate teaching strategies based on tutor feedback and review of learning science literature.
%
\item \textbf{LLM-based System for Practicing Engagement Strategies}: Based on the requirements identified from our survey, we design and evaluate \textit{TutorUp}—a training system that aids tutors in learning strategies to promote student engagement. The system offers scenario-based training by simulating real teaching situations using LLM-based student agents and provides feedback referencing user inputs and established teaching strategies.
%
\item \textbf{User Study with Online Tutors}: We present results from a within-subject user study ($N = 16$) comparing \textit{TutorUp} to a baseline system. Participants rated \textit{TutorUp} significantly higher in terms of effectiveness and usability. Evaluations of conversational transcripts suggest that \textit{TutorUp} improved the acquisition and application of engagement strategies.
\end{itemize}


