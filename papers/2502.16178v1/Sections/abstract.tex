% This abstract is deprecated

%\textbf{Abstract: }

% With the rise of online learning, many novice tutors who are often volunteers become online teachers. However, many lack experience in addressing the challenges that come with remote online tutoring, particularly the problem of keeping individual student engaged. While there are training materials that provide guidance on which teaching strategies can promote student engagement, practicing these strategies in live tutoring scenarios with real students is costly and may cause harm. In response, we introduce TutorUp, an LLM-driven system that allows tutors to practice teaching simulated disengaged students and receive feedback. The student disengagement scenarios in the TutorUp system are based on a formative study, which included surveys and interviews (N=108) exploring student engagement issues in online learning. Particularly, we use a prompt scheme that "tells the story" of online session dialogues to ensure realistic simulations. LLMs-powered personalized feedback is provided both immediately and asynchronously. In a within-subject evaluation (N=16), participants rated our system significantly higher than the baseline across various metrics. Compared to the baseline system, TutorUp allows tutors to more effectively learn and apply strategies for engaging students. \todo{Add a sentence on the wider implications of this work.}
