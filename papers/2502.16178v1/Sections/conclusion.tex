\section{Conclusion}
% This work presents TutorUp, an scenario-based tutor training system that focused on student online engagement problems. It utilized LLMs to present scenarios by simulating disengaged students for tutor to practise teaching. To provide an effective training, we introduced 1) reactive disengaged scenarios, 2) BigPicture-Character prompting pipeline and 3) immediate and asynchronous feedback scheme to facilitate tutor training process. Our user study with 16 novice tutors showed that TutorUp provides an effective training for novice tutors to deal with students online engagement problems, with useful strategies learnt and practiced.  We anticipate that our work can provide experiences on creating simulated scenario-based training systems and support for using LLMs to simulate population with  specific features. 

This work introduced \textit{TutorUp}, a scenario-based tutor training system designed to address student engagement challenges in online learning environments. \textit{TutorUp} leverages large language models (LLMs) to simulate disengaged students, allowing tutors to practice teaching strategies in a safe environment that mimics real teaching scenarios. To enhance training effectiveness, we implemented three key components: (1) reactive disengagement scenarios, (2) a BigPicture-Character prompting pipeline, and (3) a system that provides immediate and asynchronous feedback to support the training process.
%
Our user study with 16 novice tutors demonstrated that \textit{TutorUp} effectively equips tutors with practical strategies for managing student engagement challenges in online settings. Tutors were able to learn and apply useful strategies through the system. We hope that our work will contribute valuable insights for the development of scenario-based training systems and the utility of LLMs to simulate populations with specific characteristics.
\label{sec:conclusion}
