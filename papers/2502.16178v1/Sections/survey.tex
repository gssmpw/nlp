\section{Formative Study: Understanding Challenges and Scenarios of Engagement in Remote Tutoring}
\label{sec:formative_study}

\subsection{Study Context}

We conducted a formative study including two rounds of surveys to gain a deeper understanding of the specific challenges tutors face in online teaching (first survey) and inform the design of effective scenario-based training
solutions (second survey). From the initial survey, we identified managing student engagement as the biggest challenge. Focusing on this issue, the second survey investigated the scenarios of student disengagement and how tutors address them. All survey questions are provided in Appendix \ref{apdx:sureveyquestions}. 

We worked closely with tutors on the JANN platform to complete surveys for this formative study. JANN is a non-profit online platform that matches volunteers with groups of kids to teach them. Tutors receive no payment, and the majority are college students who volunteer to fulfill a social service requirement for obtaining a university degree in Mexico. The requirements to become a tutor on this platform are minimal: a brief interview (1-3 minutes) and completion of a basic teaching training (3 hours). As a result, tutors' experience varies widely, with many having no prior teaching experience. This makes effective training tools and materials essential for JANN, which strongly motivates our work.


\aptLtoX{\begin{table*}
\centering
\caption{Reactive Scenario Themes. Based on tutors' descriptions of how engagement related challenges surface in their classes, we simulate scenarios allowing tutors to practice teaching strategies promoting student engagement.}
\label{tab:reactiv_scenarios} % Label for referencing the table
\begin{tabular}{p{150pt}p{150pt}p{150pt}}
\toprule
Scenario Theme                  & Description                                                                                              & Reactive Scenario                                                         \\
\midrule
Lack of Interest and Engagement & Students are disengaged in class, showing little interest or participation in learning activities.       & Students don't attend the class~                                          \\
                                &                                                                                                          & Students don't respond to asked questions/group messages~                 \\
                                &                                                                                                          & Students don't show reaction to the taught content (never ask questions)~ \\
                                &                                                                                                          & Students not paying attention to the lecture content                      \\
                                &                                                                                                          & Students did not want to participate                                      \\
                                \hline
Lack of Confidence              & Students lack confidence in their abilities, which affects their participation and performance in class. & Students have low self-condidence                                      \\
                                &                                                                                                          & Students are afraid to make mistakes                                                   \\
                                &                                                                                                          & Students know the answer to something but don't reply~                    \\
                                \hline
Varying Learning Paces          & Students have varying learning speeds, with faster learners feeling bored and slower ones struggling to keep up.                & Some students learn faster and get bored                                  \\
                                &                      & Students making very basic errors like subtraction or multiplication      \\
                                \hline
Fatigue and Focus Issues        & Students appear tired and unable to concentrate, affecting their learning and engagement.                & Students appear tired in the class                                        \\
\bottomrule
\end{tabular}
\end{table*}}{\begin{table*}
\centering
\caption{Reactive Scenario Themes. Based on tutors' descriptions of how engagement related challenges surface in their classes, we simulate scenarios allowing tutors to practice teaching strategies promoting student engagement.}
\label{tab:reactiv_scenarios} % Label for referencing the table

\begin{tblr}{
  width = \linewidth,
  colspec = {Q[35]Q[45]Q[60]},
  row{1} = {c},
  cell{2}{1} = {r=5}{},
  cell{2}{2} = {r=5}{},
  cell{7}{1} = {r=3}{},
  cell{7}{2} = {r=3}{},
  cell{10}{1} = {r=2}{},
  cell{10}{2} = {r=2}{},
  hline{1-2,7,10,12-13} = {-}{},
}
Scenario Theme                  & Description                                                                                              & Reactive Scenario                                                         \\
Lack of Interest and Engagement & Students are disengaged in class, showing little interest or participation in learning activities.       & Students don't attend the class~                                          \\
                                &                                                                                                          & Students don't respond to asked questions/group messages~                 \\
                                &                                                                                                          & Students don't show reaction to the taught content (never ask questions)~ \\
                                &                                                                                                          & Students not paying attention to the lecture content                      \\
                                &                                                                                                          & Students did not want to participate                                      \\
Lack of Confidence              & Students lack confidence in their abilities, which affects their participation and performance in class. & Students have low self-condidence                                      \\
                                &                                                                                                          & Students are afraid to make mistakes                                                   \\
                                &                                                                                                          & Students know the answer to something but don't reply~                    \\
Varying Learning Paces          & Students have varying learning speeds, with faster learners feeling bored and slower ones struggling to keep up.                & Some students learn faster and get bored                                  \\
                                &                      & Students making very basic errors like subtraction or multiplication      \\
Fatigue and Focus Issues        & Students appear tired and unable to concentrate, affecting their learning and engagement.                & Students appear tired in the class                                        
\end{tblr}
\end{table*}}


\subsection{Survey 1: Overall Challenges in Online Tutoring}
We conducted a web-based survey investigating the challenges JANN's tutors face.
The survey included two open-ended questions, focusing on the challenges they faced in the past and present. 


We received responses from 86 tutors, including 47 females and 39 males, of whom 17 are full-time college students. Tutors' average age was 24 years and median age was 22 years.
%They provided valuable insights into the challenges of online teaching.
Two researchers conducted a thematic analysis~\cite{joffe2011thematic} of the survey responses using the Affinity Diagramming~\cite{kawakita1991original} method to classify and interpret the data. Specifically, one researcher performed the initial classification of all responses, while the other reviewed the categorization to ensure accuracy. When encountering conflicts, the reviewer labeled them and the two researchers discussed their interpretations and referred to the original responses until they reached a consensus. After the classification, they also discussed the names of the categories. 

From the challenges identified in the responses, we highlighted issues such as class preparation, tutor confidence, and maintaining engagement. Among these we found that maintaining student engagement is the most crucial issue, closely tied to factors such as scheduling (e.g., students not attending), tutor skills and mindset (e.g., nervousness and lack of practice), and the teaching process (e.g., keeping students' attention). Overall, these analysis results motivate us to design a tutor training system with strategies to help tutors address these student engagement challenges. 


\subsection{Survey 2: Scenarios of Engagement Issues in Online Learning}
\label{subsec:survey2}

Through a literature review, we identified scenario-based training as an effective method to train tutors as it could provide tutors with realistic scenario simulation for practicing~\cite{teachingearly, prospectsforchange}. Thus, we conducted the second survey specifically focusing on identifying the exact scenarios tutors faced and the strategies they employed to manage student engagement. We designed three open-ended questions asking about tutors' observations of student engagement and their strategies for addressing it. To make our survey rigorous and effective, we conducted a comprehensive review of the relevant literature on student engagement~\cite{abou2021emergency,motivation,enhancing,sevenprinciples,casestudy}. From this review, we identified and summarized potential strategies for mitigating engagement problems, which we then categorized into ten distinct categories (e.g., show empathy to students and set academic goals) with specific instances. The full summary of the strategy can be found in Appendix \ref{apdx:strategies}. Based on these strategies, we formulated specific questions for tutors, inquiring whether they have tried these strategies and, if so, how and when they implemented them.

The questions in this second round of the survey are thus structured into two main parts:
\begin{enumerate}
    \item \textbf{Open-ended Questions:}
    \begin{enumerate}%[label=\textbf{Q\arabic*:}]
        \item[\textbf{Q1:}] \textit{What types of student engagement problems have you encountered?}
        \item[\textbf{Q2:}] \textit{Why do you think students are disengaged?}
        \item[\textbf{Q3:}] \textit{What strategies have you used to increase students' engagement and when did you use them?}
    \end{enumerate}
    \item \textbf{Strategy-specific Questions:}
    \begin{enumerate}%[label=\textbf{Q\arabic*:}, resume]
        \item[\textbf{Q4:}] \textit{Have you employed a specific strategy (e.g., set time limitation)?}
        \item[\textbf{Q5:}] \textit{If yes, how and when did you implement this strategy?}
    \end{enumerate}
\end{enumerate}

We received valid responses from 102 tutors on JANN, including 60 females and 42 males. The average age was 27 years and the median age was 24 years. Additionally, 91 had no prior online tutoring experience before they started at JANN, while the other 17 had an average of 18 hours of past tutoring experience. 

The survey analysis involved two main steps, identifying and matching scenario-strategy pairs and clustering scenarios using affinity diagramming ~\cite{kawakita1991original}, conducted by the same two researchers from the first survey. First, scenarios were identified from the responses and matched with corresponding strategies. One researcher summarized the scenarios and matched all responses, while the second reviewed them. For open-ended questions ($N=3$), scenarios involving student disengagement and related strategies were identified to form scenario-strategy pairs. For strategy-specific questions ($N=10$), scenarios were extracted based on the provided strategies to form scenario-strategy pairs. For example, a response to the strategy-specific question ``Did you discuss behavioral expectations with your student?'' such as ``The student did not want to participate'' was matched to form the pair: ``Student did not want to participate - Behavioral expectations.'' In cases of conflict, the researchers first discussed their differing interpretations and reviewed the original responses until consensus was reached. Finally, the scenario-strategy pairs were grouped and labeled to identify common scenarios and strategies, a classification step similar to the first survey. This analysis resulted in summarized scenario types with corresponding strategies to address each scenario. For example, in a scenario type where students appear tired in class, matched strategies such as ``explaining to reduce students' burden'' and ``adding attractive activities'' can be used to address the issue. % and thus are matched to it 


From the analysis result, we identified that some scenarios were specifically centered on student-related factors, such as knowledge level (e.g., fast learners getting bored), personality traits (e.g., unconfident, shy), or classroom behavior (e.g., tiredness, don't respond). Tutors addressed these situations by implementing strategies tailored to the specific disengagement issues of students.
Others were more general, focusing on broader classroom contexts, such as ``at the beginning'' or ``always'' where teachers proactively maintained engagement without targeting specific student behaviors. 
% \deleted{Based on the responses, we categorized both the scenarios and the corresponding strategies. We found that the responses primarily addressed two types of scenarios and their associated strategies: \textit{preventive scenarios} and \textit{reactive scenarios}.} 
These two types of scenarios align with the Proactive-Reactive behavioral patterns in student-teacher relationships~\cite{yucel2010analysis}. Based on this alignment, we formulated the scenarios and their associated strategies into two overarching types:
    
\textbf{Proactive Scenarios}: \textit{Tutors implement strategies to proactively maintain engagement across various contexts.} In the survey, when answering the ``how and when do you use this strategy'' questions, some tutors provided general scenarios, such as ``at the beginning'', ``whenever students are asked to do exercises,'' or even ``always.'' We observed that these strategies were not aimed at addressing specific disengagement issues but at preventing disengagement and promoting overall student engagement to maintain focus.

\textbf{Reactive Scenarios}: \textit{Tutors take strategies to address specific instances of disengagement as they occur.} In contrast to proactive strategies, some tutors reported taking specific measures in response to particular student issues. For example, in response to the question ``Why do you think students are disengaged?'', tutors mentioned scenarios such as "some students learn faster and get bored" or ``students know the answer but don't reply.'' In these cases, tutors reacted to students' behaviors and employed targeted strategies to promote engagement. From tutors' responses, we identified reactive situations where specific disengagement issues arose. Out of 1,326 responses (102 participants answering 13 questions), 138 were identified as describing reactive scenarios. 
In this analysis process, reactive scenarios were categorized into four themes: \textit{1) lack of interest and engagement, 2) lack of confidence, 3) varying learning paces, 4) fatigue and focus issues.} Tab.~\ref{tab:reactiv_scenarios} shows description and examples of reactive scenarios for each theme. 

Overall, reactive disengagement scenarios are more challenging for novice tutors, as they require instant and context-specific interventions~\cite{park2022frustration}. Therefore, reactive scenarios were the primary focus of this study.


\subsection{Design Requirements:}
\subsubsection{General Requirements}
Here we summarize key insights from the formative study, emphasizing the need for a tutor training system focused on student engagement in online learning. To ensure the system is effective and constructive, we propose two main design requirements to support tutors:
\begin{enumerate}%[label=R\arabic*]
    \item[R1] \textbf{Simulated Online Learning Scenarios}: \\
    The system should simulate realistic online learning scenarios, allowing tutors to practice their teaching skills in an environment where reactive situations are presented.

    \item[R2] \textbf{Feedback and Improvement Strategies}: \\
    The system should provide feedback to assess tutors' performances and offer insights and strategies for improvements.
\end{enumerate}


% \usepackage{tabularray}

% \usepackage{tabularray}
% \begin{table}
% \centering
% \begin{tblr}{
%   width = \linewidth,
%   colspec = {Q[15]Q[15]Q[15]},
%   column{1} = {c},
%   cell{1}{1} = {r=6}{},
%   cell{1}{2} = {c},
%   cell{2}{2} = {r=5}{c},
%   cell{7}{1} = {r=4}{},
%   cell{8}{2} = {r=3}{c},
%   cell{11}{1} = {r=3}{},
%   cell{11}{2} = {c},
%   cell{12}{2} = {r=2}{c},
%   cell{14}{1} = {r=2}{},
%   cell{14}{2} = {c},
%   cell{15}{2} = {c},
%   vline{2-3} = {1,7,11,14}{},
%   vline{3} = {2,8,12,15}{},
%   hline{1,7,11,14,16} = {-}{},
%   hline{2,8,12,15} = {2-3}{},
% }
% Lack of Interest and Engagement & Description       & Lack of Interest and Engagement  Students are disengaged in class, showing little interest or participation in learning activities.~            \\
%                                 & Reactive Scenario & Students don't attend the class                                                                                                                 \\
%                                 &                   & Students don't respond to asked questions/group messages                                                                                        \\
%                                 &                   & Students don't show reaction to the taught content (never ask questions)                                                                        \\
%                                 &                   & Students not paying attention to the lecture content                                                                                            \\
%                                 &                   & Students did not want to participate~                                                                                                           \\
% Lack of Confidence              & Description       & Students lack confidence in their abilities, which affects their participation and performance in class.  Students have low self-condidence~~~~ \\
%                                 & Reactive Scenario & Students have low self-condidence                                                                                                               \\
%                                 &                   & Students are afraid to make mistakes~                                                                                                           \\
%                                 &                   & Students know the answer to something but don't reply                                                                                           \\
% Varying Learning Paces~         & Description       & Students have varying learning speeds, with faster learners feeling bored and slower ones struggling to keep up.                                \\
%                                 & Reactive Scenario & Some students learn faster and get bored~                                                                                                       \\
%                                 &                   & Students making very basic errors like subtraction or multiplication~                                                                           \\
% Fatigue and Focus Issues        & Description       & Students appear tired and unable to concentrate, affecting their learning and engagement.                                                       \\
%                                 & Reactive Scenario & Students appear tired in the class                                                                                                              
% \end{tblr}
% \end{table}

\subsubsection{Design Iteration}
\label{subsec:iteration}
We invited five experienced Math tutors from JANN (2 male, 3 female, aged 24 to 30, with an average age of 26.6) to participate in design iteration interviews to gather their insights and suggestions for updating our system. Their online tutoring experience ranged from 4 to 6 months, with 3 having additional teaching backgrounds in private tutoring and formal school environments (average 3.3 years).  These interviews were conducted via Zoom, where tutors used our system under our instruction. When trying our system, they were asked to think aloud and express their ideas. Follow-up questions on students and feedback design were also included. They generally found the simulated scenarios realistic and the feedback valuable for practicing engagement strategies. However, they suggested improvements including that the strategies matched to the scenarios were incomplete. We asked them to help refine the strategy matching using the ten categories identified in Section \ref{subsec:survey2}, which improved the scenario-strategy alignment (see the complete results in supplementary materials). We also integrated other valuable feedback to enhance system effectiveness and user experience (details in Section \ref{sec:system_design}).

