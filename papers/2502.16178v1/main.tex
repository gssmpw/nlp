%%
%% This is file `sample-manuscript.tex',
%% generated with the docstrip utility.
%%
%% The original source files were:
%%
%% samples.dtx  (with options: `all,proceedings,bibtex,manuscript')
%% 
%% IMPORTANT NOTICE:
%% 
%% For the copyright see the source file.
%% 
%% Any modified versions of this file must be renamed
%% with new filenames distinct from sample-manuscript.tex.
%% 
%% For distribution of the original source see the terms
%% for copying and modification in the file samples.dtx.
%% 
%% This generated file may be distributed as long as the
%% original source files, as listed above, are part of the
%% same distribution. (The sources need not necessarily be
%% in the same archive or directory.)
%%
%%
%% Commands for TeXCount
%TC:macro \cite [option:text,text]
%TC:macro \citep [option:text,text]
%TC:macro \citet [option:text,text]
%TC:envir table 0 1
%TC:envir table* 0 1
%TC:envir tabular [ignore] word
%TC:envir displaymath 0 word
%TC:envir math 0 word
%TC:envir comment 0 0
%%
%%
%% The first command in your LaTeX source must be the \documentclass
%% command.
%%
%% For submission and review of your manuscript please change the
%% command to \documentclass[manuscript, screen, review]{acmart}.
%%
%% When submitting camera ready or to TAPS, please change the command
%% to \documentclass[sigconf]{acmart} or whichever template is required
%% for your publication.
%%
%%
% \documentclass[manuscript,screen,review]{acmart}
%\documentclass[manuscript]{acmart}  % anonymous
\documentclass[sigconf]{acmart}


%%
%% \BibTeX command to typeset BibTeX logo in the docs
\AtBeginDocument{%
  \providecommand\BibTeX{{%
    Bib\TeX}}}

%% Rights management information.  This information is sent to you
%% when you complete the rights form.  These commands have SAMPLE
%% values in them; it is your responsibility as an author to replace
%% the commands and values with those provided to you when you
%% complete the rights form.


% copy from email
\copyrightyear{2025} 
\acmYear{2025} 
\setcopyright{cc}
\setcctype{by}
\acmConference[CHI '25]{CHI Conference on Human Factors in Computing Systems}{April 26-May 1, 2025}{Yokohama, Japan}
\acmBooktitle{CHI Conference on Human Factors in Computing Systems (CHI '25), April 26-May 1, 2025, Yokohama, Japan}
\acmDOI{10.1145/3706598.3713589}
\acmISBN{979-8-4007-1394-1/25/04}

% below are the original


% \setcopyright{acmlicensed}
% \copyrightyear{2018}
% \acmYear{2018}
% \acmDOI{XXXXXXX.XXXXXXX}

%% These commands are for a PROCEEDINGS abstract or paper.
% \acmConference[Conference acronym 'XX]{Make sure to enter the correct
%   conference title from your rights confirmation emai}{June 03--05,
%   2018}{Woodstock, NY}
%%
%%  Uncomment \acmBooktitle if the title of the proceedings is different
%%  from ``Proceedings of ...''!
%%
%%\acmBooktitle{Woodstock '18: ACM Symposium on Neural Gaze Detection,
%%  June 03--05, 2018, Woodstock, NY}
% \acmISBN{978-1-4503-XXXX-X/18/06}


%%
%% Submission ID.
%% Use this when submitting an article to a sponsored event. You'll
%% receive a unique submission ID from the organizers
%% of the event, and this ID should be used as the parameter to this command.
%%\acmSubmissionID{123-A56-BU3}

%%
%% For managing citations, it is recommended to use bibliography
%% files in BibTeX format.
%%
%% You can then either use BibTeX with the ACM-Reference-Format style,
%% or BibLaTeX with the acmnumeric or acmauthoryear sytles, that include
%% support for advanced citation of software artefact from the
%% biblatex-software package, also separately available on CTAN.
%%
%% Look at the sample-*-biblatex.tex files for templates showcasing
%% the biblatex styles.
%%

%%
%% The majority of ACM publications use numbered citations and
%% references.  The command \citestyle{authoryear} switches to the
%% "author year" style.
%%
%% If you are preparing content for an event
%% sponsored by ACM SIGGRAPH, you must use the "author year" style of
%% citations and references.
%% Uncommenting
%% the next command will enable that style.
%%\citestyle{acmauthoryear}



%%%%%%%%%%%%%%%%%%%%%%%%%%%%%%%%%%%%%%%%%%%%%%%%%%%%%%%%%%%%%%%%%%
% CUSTOM COMMANDS AND PACKAGES
%%%%%%%%%%%%%%%%%%%%%%%%%%%%%%%%%%%%%%%%%%%%%%%%%%%%%%%%%%%%%%%%%%
\let\comment\undefined%added to fix error
% \PassOptionsToPackage{normalem}{ulem}%added to fix error
% \usepackage{changes}


%\let\uline\relax
\let\uuline\relax
\let\uwave\relax
\let\sout\relax
\let\xout\relax
\let\dotuline\relax
\let\dashuline\relax
\let\dashuline\relax
\let\ULdepth\relax
\let\ULset\relax


\usepackage{acmart-taps}
\usepackage{xcolor}
\usepackage{tabularray}
\usepackage{subcaption}
\usepackage{enumitem}
\usepackage{booktabs}   
\usepackage{array}      
\usepackage{arydshln}   
\usepackage{caption}   
\usepackage{color}

% \usepackage[margin=1in]{geometry}

%\newcommand{\todo}[1]{\textcolor{red}{[*** #1 ***]}}
% black blue


%%%%%%%%%%%%%%%%%%%%%%%%%%%%%%%%%%%%%%%%%%%%%%%%%%%%%%%%%%%%%%%%%%


\sloppy
%%
%% end of the preamble, start of the body of the document source.
\begin{document}

%%
%% The "title" command has an optional parameter,
%% allowing the author to define a "short title" to be used in page headers.
% \title{TutorUp: A Scenario-Based Learning Environment to Facilitate the Adoption of Effective Engagement Strategies in Remote Tutoring}
\title[TutorUp: Training Tutors to Address Engagement Challenges in Online Learning]{TutorUp: What If Your Students Were Simulated? Training Tutors to Address Engagement Challenges in Online Learning}
% %%
% %% The "author" command and its associated commands are used to define
% %% the authors and their affiliations.
% %% Of note is the shared affiliation of the first two authors, and the
% %% "authornote" and "authornotemark" commands
% %% used to denote shared contribution to the research.
% \author{Ben Trovato}
% \authornote{Both authors contributed equally to this research.}
% \email{trovato@corporation.com}
% \orcid{1234-5678-9012}
% \author{G.K.M. Tobin}
% \authornotemark[1]
% \email{webmaster@marysville-ohio.com}
% \affiliation{%
%   \institution{Institute for Clarity in Documentation}
%   \city{Dublin}
%   \state{Ohio}
%   \country{USA}
% }

\author{Sitong Pan}
\affiliation{%
  \institution{Zhejiang University}
  \city{Hangzhou}
  \country{China}}
\email{pstong@zju.edu.cn}

\author{Robin Schmucker}
 \affiliation{%
 \institution{Carnegie Mellon University}
 \city{Pittsburgh}
 \state{PA}
 \country{USA}
}
\email{rschmuck@cs.cmu.edu}


\author{Bernardo Garcia Bulle Bueno}
\affiliation{%
\institution{Massachusetts Institute of Technology}
\city{Cambridge}
\state{MA}
\country{USA}}
\email{bernard0@mit.edu}


\author{Salome Aguilar Llanes}
\affiliation{%
\institution{Massachusetts Institute of Technology}
\city{Cambridge}
\state{MA}
\country{USA}}
\email{saloagui@mit.edu}

\author{Fernanda Albo Alarcón}
\affiliation{%
  % \institution{Jóvenes Ayudando a Niñas y Niños}
  \institution{Autonomous Technological Institute of Mexico}
  \city{Mexico City}
  \country{Mexico}}
\email{fer.albo.al@gmail.com}

 \author{Hangxiao Zhu}
\affiliation{%
   \institution{Texas A\&M University}
   \city{College Station}
   \state{TX}
   \country{USA}
   }
\email{hangxiao@tamu.edu}

 \author{Adam Teo}
\affiliation{%
   \institution{Texas A\&M University}
   \city{College Station}
   \state{TX}
   \country{USA}
   }
\email{adamt321@tamu.edu}

\author{Meng Xia}
\affiliation{%
   \institution{Texas A\&M University}
   \city{College Station}
   \state{TX}
   \country{USA}
   }
\email{mengxia@tamu.edu}


% %%
% %% By default, the full list of authors will be used in the page
% %% headers. Often, this list is too long, and will overlap
% %% other information printed in the page headers. This command allows
% %% the author to define a more concise list
% %% of authors' names for this purpose.
\renewcommand{\shortauthors}{Pan et al.}

%%
%% The abstract is a short summary of the work to be presented in the
%% article.
\begin{abstract}
%With the rise of online learning, many novice tutors become online teachers. However, many lack experience in facilitating student engagement during remote tutoring, and practicing with real students is costly and potentially harmful. In response, we introduce TutorUp, an LLMs-based system that enables tutors to practice engagement strategies with simulated students in different teaching scenarios. These scenarios mimic real world teaching settings and are based on a formative study, which included surveys and interviews (N=108) exploring student engagement issues in online learning. Particularly, to promote realistic simulations, we employ a prompting strategy that "tells the story" of online session dialogues. We further employ LLMs to provide immediate and asynchronous feedback by referencing user inputs and established teaching strategies derived from the literature. In a within-subject evaluation (N=16), participants rated our system significantly higher than the baseline across effectiveness and usability. Compared to the baseline system, TutorUp allows tutors to more effectively learn and apply strategies for engaging students. % 
% With the rise of online learning, many novice tutors become online teachers. However, many lack experience in facilitating student engagement in remote settings. In response, we introduce TutorUp, a Large Language Model (LLM)-based system that enables tutors to practice engagement strategies with simulated students in scenario-based training. These scenarios mimic real teaching situations based on a formative study (N=108), which included surveys and interviews exploring student engagement issues in online learning. To promote immersive simulations, we employ a prompting strategy that iteratively "tells the story" of online learning dialogues. Additionally, TutorUp provides immediate and asynchronous feedback by referencing user inputs and teaching strategies from the learning science literature. In a within-subject evaluation (N=16), participants rated TutorUp significantly higher than a baseline system across effectiveness and usability. Our findings suggest that TutorUp allows tutors to learn and apply strategies for engaging students more effectively, contributing to improved practices in online tutoring.
% Alternative version
With the rise of online learning, many novice tutors lack experience engaging students remotely. We introduce \textit{TutorUp}, a Large Language Model (LLM)-based system that enables novice tutors to practice engagement strategies with simulated students through scenario-based training. Based on a formative study involving two surveys ($N_1=86$, $N_2=102$) on student engagement challenges, we summarize scenarios that mimic real teaching situations. To enhance immersion and realism, we employ a prompting strategy that simulates dynamic online learning dialogues. \textit{TutorUp} provides immediate and asynchronous feedback by referencing tutor-students online session dialogues and evidence-based teaching strategies from learning science literature. In a within-subject evaluation ($N=16$), participants rated \textit{TutorUp} significantly higher than a baseline system without simulation capabilities regarding effectiveness and usability. Our findings suggest that \textit{TutorUp} provides novice tutors with more effective training to learn and apply teaching strategies to address online student engagement challenges. 
\end{abstract}
% helps tutors learn and apply strategies for more effective student engagement, contributing to improved practices in online tutoring.
%%
%% The code below is generated by the tool at http://dl.acm.org/ccs.cfm.
%% Please copy and paste the code instead of the example below.
%%

\begin{CCSXML}
<ccs2012>
   <concept>
       <concept_id>10003120.10003121.10011748</concept_id>
       <concept_desc>Human-centered computing~Empirical studies in HCI</concept_desc>
       <concept_significance>500</concept_significance>
       </concept>
   <concept>
       <concept_id>10010405.10010489</concept_id>
       <concept_desc>Applied computing~Education</concept_desc>
       <concept_significance>500</concept_significance>
       </concept>
 </ccs2012>
\end{CCSXML}

\ccsdesc[500]{Human-centered computing~Empirical studies in HCI}
\ccsdesc[500]{Applied computing~Education}


%%
%% Keywords. The author(s) should pick words that accurately describe
%% the work being presented. Separate the keywords with commas.
\keywords{Remote Tutoring, Tutor Training, Interactive Learning Environments, Conversational Agents, Large Language Models, Student Engagement}

%\received{20 February 2007}
%\received[revised]{12 March 2009}
%\received[accepted]{5 June 2009}

%%
%% This command processes the author and affiliation and title
%% information and builds the first part of the formatted document.
\maketitle
% \begin{abstract}
Retrieval-Augmented Generation (RAG) is often used with Large Language Models (LLMs) to infuse domain knowledge or user-specific information. In RAG, given a user query, a retriever extracts chunks of relevant text from a knowledge base. These chunks are sent to an LLM as part of the input prompt. Typically, any given chunk is repeatedly retrieved across user questions. However, currently, for every question, attention-layers in LLMs fully compute the key values (KVs) repeatedly for the input chunks, as state-of-the-art methods cannot reuse KV-caches when chunks appear at arbitrary locations with arbitrary contexts. Naive reuse leads to output quality degradation.  This leads to potentially redundant computations on expensive GPUs and increases latency. In this work, we propose \sys, a system for managing and reusing precomputed KVs corresponding to the text chunks (we call \textit{chunk-caches}) in RAG-based systems. We present how to identify \hl{\textit{chunk-caches} that are reusable}, how to efficiently perform a small fraction of recomputation to \textit{fix} the cache to maintain output quality, and how to efficiently store and evict \textit{chunk-caches} in the hardware for maximizing reuse while masking any overheads. With real production workloads as well as synthetic datasets, we show that \sys reduces redundant computation by \textbf{51\%} over SOTA prefix-caching and \textbf{75\%} over full recomputation.
\hl{Additionally, with continuous batching on a real production workload, we get a \textbf{1.6$\times$} speedup in throughput and a \textbf{2$\times$} reduction in end-to-end response latency over prefix-caching while maintaining quality, for both the \llama-3-8B and \llama-3-70B models. 
}
\end{abstract}







\documentclass[../main.tex]{subfiles}
\graphicspath{{../images/}}
\makeatletter
\def\input@path{{../images/}}
\makeatother
\begin{document}
\section{Introduction}
\begin{figure}
\centering
\begin{tikzpicture}
\node[inner sep=0pt] (ws) at (0, 0) {
\includegraphics[height=.4\textwidth, trim={10cm 0 10cm 0},clip]{world_space.png}};
\node[inner sep=0pt] (cs) at (6,0) {\includegraphics[height=.4\textwidth, trim={10cm 1cm 10cm 4cm},clip]{conf_space.png}};
\end{tikzpicture}
\vspace{-5pt}
\label{fig:pbrm_intro}
\caption{\textbf{Left}: Shows world space obstacles as grey spheres. Robots start and goal configuration is colored red and green, respectively. Configurations along the computed path are colored transparent blue. \textbf{Right:} Mapped world space scenario to configuration space. Obstacle region is the grey mesh. Red spheres are collision-free regions computed by the neural SCDF. The optimized shortest path in the convex corridor is the blue curve.}
\vspace{-25pt}
\end{figure}
Motion planning is the problem of finding a collision-free trajectory that connects a given start and goal configuration. The planning takes place in the configuration space of the robot. For single body robots, like mobile robots or drones, the configuration space and the world space are usually the same. This simplifies the planning, since explicit obstacle representations are available which enables geometrical tools like separating hyperplanes, smallest distance to obstacles etc., to be used when designing motion planning algorithms. For multi-body robots like manipulators, the situation is completely different. The world space obstacles are usually mapped to non-convex regions, and to make the problem even harder, the mapping is usually not known. Forming explicit representations of the obstacle region in the configuration space is usually too expensive or intractable. Despite all of this, sampling based planners are used with great success, which mainly is due to their use of implicit representations of the obstacle region. The basic idea is to construct a graph in the configuration space that covers and connects the collision-free region. From this graph, a path can be extracted that connects a given start and goal configuration. The approach is computationally expensive, since the graph is constructed with the smallest geometrical building block available, points, which represents a collision-check. Furthermore, the extracted paths from the graph are non-smooth and jagged due to the stochastic nature of the approach. This adds an additional post-processing step to the process, where the paths are shortcutted and smoothened, before the path can be used for tracking. Clearly a lot of time is invested to form this graph and produce smooth paths. Thus, if the obstacles start to move, then all of this work is done in no use, since all points that make up this graph need to be re-verified, which is simply too time consuming to be done in real time.
\\\\
In this work, we want to address the existing drawbacks of the sampling based planners. Our main contribution is an improved motion planner where each vertex in the graph covers a collision-free region in the form of a sphere instead of a point and where the edges are formed with neighboring intersecting spheres. This representation has the advantage of instead of returning piecewise linear paths, returning a sequence of overlapping spheres, i.e. a convex corridor, that connects a given start and goal configuration, illustrated in Figure \ref{fig:pbrm_intro}. This convex corridor allows us to use convex optimization to produce smooth trajectories, instead of computationally expensive post-processing methods. The representation further allows us to estimate the coverage of the collision-free space, which gives us awareness and feedback in the offline roadmap construction phase. Finally, our representation is simple to adapt to moving obstacles, simply requery for the new radii and recheck for intersections. 
\\\\
The spherical collision-free regions are formed using a signed distance function (SDF), which is a function that returns the smallest distance from an arbitrary point to the boundary of an obstacle. As the name implies, the distance is signed, thus if the point is inside the obstacle it is negative otherwise positive. If the distance is positive, a sphere with radius equal to the distance is guaranteed to cover a collision-free region. Using an SDF in motion planning is not new, but what is novel about our approach is that we express the distance in the configuration space instead of the world space and by doing so allows us to form these convex collision-free regions. We refer to the resulting SDF as a signed configuration distance function (SCDF). Computing an SCDF analytically is non-trivial, our approach is therefore to parameterize the SCDF with a deep neural network and learn the mapping by supervised learning. Our resulting neural SCDF can compute distances for different parameter values of obstacle shapes and we also show how multiple distances can be combined, thus making our approach flexible.
\section{Related work}
Motion planning algorithms can roughly be divided into three families, grid-based, sampling based and optimization based methods. Grid-based methods (GBM) discretize the planning space from which a graph is then compiled. A standard search method is A$^\star$ \citep{a_star}, which is classified as an \textit{informed} search method, since it employs a heuristic function to speed up the search. A$^\star$ guarantees to return an optimal path at the level of discretization used. GBMs usually discretize the planning space by a regular lattice and this limits the GBMs to problems with low dimensionality due to the curse of dimensionality. Thus, GBMs are usually limited to single-body robots where the degrees of freedom (DOF) are low. To overcome the inherent scaling problem with the GBMs, stochastic methods are usually used for multi-body robots. These methods are termed as sampling-based methods (SBM) and core members within this family are the rapidly-exploring random trees (RRT) \citep{rrt} and the probabilistic roadmap (PRM) \citep{prm}. RRT grows a tree from the start configuration and explores the collision-free region in a rapid way until it is able to connect to the goal region. RRT is usually improved by bi-directional planning \citep{rrt_connect}, i.e. an additional tree is grown from the goal configuration and the trees are tested for connection after any tree has been expanded. RRT is a single-query method, thus it searches for a path from scratch each time it is queried. Contrary to this, PRM is a multi-query method, which solves for multiple queries without starting from scratch. PRM does this by creating a roadmap (graph) that covers the collision-free space as an offline step. The graph is then used to solve for multiple queries. PRMs are used in cases where the environment does not change since the extra offline step is too computationally costly and needs to be re-done if the environment is changed. In our work, we address this inherent issue by using a different roadmap representation. Our vertices in the graph cover a collision-free region in the form of spheres and we form the edges by checking for intersecting spheres. If something in the environment changes, we recompute the spheres radii and recheck the intersections, without relying on collision detection. We use a trained neural network to compute the sphere radius, therefore querying for the radius can be done fast, hence our representation enables the PRM for dynamic environments.
\\\\
In the recent decades, optimization based methods (OBM) \citep{chomp, schulman, itomp, stomp} have been introduced as an alternative to SBM for multi-body robots. Like the SBM, the OBMs scale well to higher dimensional problems and produce smoother motion. It is common to use a SDF in the optimization since it is a smooth function, thus enabling gradient-based methods. However, the standard way of expressing the SDF is in world space. The distance therefore needs to be mapped to the configuration space by the forward kinematics. This mapping makes the optimization problem a non-linear program (NLP), which is computationally expensive to solve. Recently, a different approach has been proposed. In \cite{mp_gcs} motion planning is formulated as a convex optimization problem by using the graph of convex sets framework \citep{gcs}. The underlying idea is to decompose the collision-free space into intersecting convex sets from which a convex optimization problem is formulated. In cases where an explicit representation of the obstacles in the configuration space exists, like for single-body robots, creating collision-free convex regions can be done fast \citep{iris}. For multi-body robots, this is non-trivial. Existing work does this successfully \citep{iris_nlp, iris_c} by an optimization based approach, but the methods are still too time consuming to be used in the presence of moving obstacles. Our approach is instead to use deep learning to learn an SDF expressed in the configuration space. With this, we can query for shortest distances to the collision boundary, which allows us to expand spherical regions which are collision-free. Our approach is fast and therefore enables our suggested roadmap planner to be used in dynamic environments.
\\\\
Recent research has focused on learning collision detection \citep{fk_kernel_distance, diffco, graphdistnet} by predicting the signed distance between the robot links and the surrounding obstacles in the world space. The learned SDF is used in trajectory optimization but since the distance is expressed in the world space, the problem becomes an NLP and therefore takes a long time to solve. We take a novel approach and suggest to instead express the signed distance in the configuration space. This allows us to improve the PRM at the same time as it enables convex optimization for trajectory optimization, which runs faster and is more reliable than NLP solvers. In \cite{cspf} a learned signed distance function in the configuration space is proposed similar to our approach. However, their approach is restricted to point cloud representations, while we propose to represent the obstacles as parameterized geometric shapes, e.g. spheres. Furthermore, we also show how to use our learned SCDF to improve an existing roadmap planner.
\section{Problem formulation}
A robot is located in the world space, $\W \subset \R^3 $. The unique location of the robot is given by its configuration $\q \in \C$, where $\C$ is the configuration space. The set of points covered by the robots bodies at a certain configuration is expressed as $\B(\q) \subset \W$. The robot is surrounded by $\NrObst$ obstacles $\O = \bigcup_{i=1}^{\NrObst} \O_i$, where  $\O_i \subset \W$. The representation of the obstacle in the configuration space is the set $\C\O_i = \{\q \in \C \: |\: \B(\q) \cap \O_i \neq \emptyset \}$. The obstacle space is formed as $\Co = \bigcup_{i=1}^{\NrObst} \C \O_i$. The complement is referred to as the free space, $\Cf = \C \setminus \Co$. The path planning problem is a tuple, ($\Cf$, $\qStart$, $\qGoal$), where we want to connect a query pair, consisting of a start, $\qStart$, and goal configuration, $\qGoal$, with a geometric path, $\q(s): [0, 1] \mapsto \Cf$, such that $\q(0)=\qStart$ and $\q(1)=\qGoal$, or report correctly when such a path does not exist.
\end{document}

\section{Related Work}
Alongside a discussion of what is meant by LLM harmfulness,
this section covers two distinct strands of related work: measuring types of harm in LLMs, and LLMs for diverse annotation tasks. %First,

%Different kinds of 
Diverse undesirable LLM outputs, from toxic language to privacy invasion, have been discussed in the observed \cite{banko-etal-2020-unified}. Here we review the ones we include in our definition of ``harm.'' %definition. Plus, we review LLMs as judges. 
Toxic content can be elicited from both generative  \cite{deshpande2023toxicity} and masked LLMs \cite{ousidhoum-etal-2021-probing}. 
%Among ways 
To measure toxic or hateful language, some use APIs such as PerspectiveAPI \cite{lees2022new} or HateBERT \cite{caselli-etal-2021-hatebert}. \citet{openai2024gpt4technicalreport} report that GPT4 produces toxic content 0.78\% of the time, versus 6.48\% in GPT3.5.
%as opposed to GPT3.5 with 6.48\%. On the other hand,
\citet{dubey2024llama} report that llama3-70B produces harmful content 5\% of the time, %whereas the 405B model generates harm 3\% of the time. 
compared to 3\% in the 405B model.
Instead of %single value classifiers to measure harm, 
reporting an absolute rate, we focus on relative harmfulness of different LLMs. %, so we point to recent work on LLMs for annotation.

The first category of harm we consider is social stereotyping and bias. %discrimination. It has been shown that 
LLMs can perpetuate social bias based on gender, race, religion etc. \cite{lin-etal-2022-gendered,bender2021dangers,field-etal-2021-survey,gupta-etal-2024-sociodemographic,andriushchenko2024agentharm,mazeika2024harmbench}. This can marginalize these groups more, and results in less fair model performance. \citet{guo2024hey} designed a competition to elicit biased output from LLMs to assess the perception of bias from non-expert users. %The first part of our work is similar to this analysis, but 
We also intentionally elicit harmful output, going %we look at other types of harms besides bias.
beyond social bias.

%When the models become stronger, they become more robust to jailbreaking attacks to elicit harmful content. However, there are datasets that can still jailbreak models to produce harmful content \cite{andriushchenko2024agentharm,mazeika2024harmbench}.

Our second category of harm is offensiveness and toxicity, which %. As opposed to stereotyping or social discrimination, this harm 
%is more subjective and harder to define than the previous category, so there 
lacks an established definition due to its greater subjectivity \cite{dev-etal-2022-measures,korre-etal-2023-harmful}. We include hate speech (HS) and abusive language as toxic content. HS can be defined as expressions of offensive and discriminatory discourse towards a group or an individual based on characteristics such as race, religion, nationality, or other group characteristics \cite{john2000hate,jahan2023systematic,basile2019semeval,davidson2017automated}. It includes racism, negative stereotyping, and sexist language. On the other hand, abusive language is content with inappropriate words such as profanity or disrespectful terms. It also includes psychological threats such as humiliation. %or constant criticism. %Toxic content can be elicited from both generative models \cite{deshpande2023toxicity} and masked language models \cite{ousidhoum-etal-2021-probing}.

%In addition to obvious toxic content, LLMs can generate diverse implicit toxic outputs using reinforcement learning with favoring toxic content in the reward function \cite{wen-etal-2023-unveiling}.  Regarding the subjectivity of this task, \cite{korre-etal-2023-harmful} reannotate the existing datasets with different definitions of toxicity and show that broader definitions result in more robust annotations, but interannotator agreements are still lower than 0.5. \cite{dev-etal-2022-measures} also point out the lack of definition for bias and harm in general and propose a framework to guide researchers during the development of bias measures.

Harm can be implicit, such as privacy invasion
%We are also interested in privacy invasion,
where there is 
leakage of personal information. %leakage from the model. 
%LLMs can memorize details of the training data and then leak private information such as 
This includes social security numbers, phone numbers, or bank account information \cite{carlini2021extracting,brown2022does}. 
%There are several frameworks to test the privacy of LLMs \cite{li2024llm} and generate data for personal attribute inference \cite{yukhymenko2024synthetic,kim2024propile}.

%Our definition of harm includes hate speech (HS) as well. HS can be defined as \textcolor{red}{expressions of} hatred towards a social group, the humiliation of the members of a group, or %communication disparaging  extreme disparagement of a person or a group based on race, color, ethnicity, gender, sexual orientation, nationality, religion, or other group characteristics .

For data annotation, LLMs
%Besides text generation, 
%LLMs have been used to annotate data because they 
can %be comparable to 
replace humans for some tasks, %and make the annotation process faster and cheaper 
with gains in efficiency and economy \cite{tan2024large}. They have been used for sociological annotations such as for classification of stance, bots or humor  \cite{ziems2024can,zhu2023can}. For tasks such as topic and frame detection or sentence segmentation they can surpass crowd-workers
%Some works show that they can surpass crowd-workers for some tasks such as topic and frame detection or sentence segmentation %into research aspects 
\cite{he2024if,gilardi2023chatgpt}. Some have argued that human-LLM collaboration results in more reliable annotation \cite{he2024if,zhang2023llmaaa,kim2024meganno+}. In addition to more objective tasks,
%LLMs have been used to annotate data %even 
they have been applied to subjective annotations such as offensiveness and abusiveness \cite{pavlovic-poesio-2024-effectiveness,zhu2023can,he2023annollm}, %. For example, LLMs are used as judges to rank responses from different LLMs 
or to rank outputs from different LLMs based on helpfulness, accuracy, or relevance \cite{zheng2023judging,lin2024wildbench,dubois2024length}. These works tend to focus on human-large LLM interactions, whereas we focus on single-turn responses from smaller LLMs. We inspire from \citet{zheng2023judging} but we only measure harm instead of overall performance. Plus, we use 3 LLMs to evaluate smaller LLMs.
Our survey is designed to explore the perceptions and attitudes of Black Americans regarding AAE representation in chat-based AI systems across a variety of settings, ranging from professional to personal. For each setting, we gauge how and when participants want an LLM (or chatbot) to use AAE versus MUSE. We aimed to provide sufficient detail on the settings to make them more relatable and easier to comprehend \cite{lenzner2012effects}. The settings are selected to give a more complete picture of Black American's every-day experiences and preferences \cite{Maedche2019AI-Based}. For each scenario presented, study participants were asked to choose from the answer choices seen in \autoref{table:Vignette Choices} for how they would want such an LLM to interact:

%The specific settings we chose are described below, and cover both professional and personal contexts (as AI is used in both~\cite{Garrotes2021THE}), as well as use-cases the cover both LLM-generated continuations of one's own text as and LLM responses in an assistant-like scenario.

%Since use of these AI technologies is widespread in people' work and private lives \cite{Garrotes2021THE}, we explore both formal (professional) and informal (personal) settings. 


\begin{enumerate}[nolistsep,noitemsep]
\item \textit{AI Assistants} (professional and personal).
These LLM-response settings reflect the use of an AI assistant for helping with either professional or personal tasks, and whether such an assistant should address the user in AAE.


\item\textit{Customer Bot}. This LLM-response setting reflects the use of a text-based chatbot agent for quick assistance, and whether the agent should continue the interaction after greeting the user in AAE.

\item \textit{Email and SMS Autocomplete}. These LLM-continuation settings reflect the use of an LLM to autocomplete a user's own writing for emails or text messages.  % if AAE speakers want these systems to maintain a consistent tone from that with which they started the conversation. 

\item \textit{Educational Avatar}. This LLM-response setting reflects the use of AAE by an avatar in an education platform and whether this could impact learning experience.
% \hal{TODO fill in}.
%As inclusive educational materials enhance learning outcomes,  accessibility and provide overall support for better learning experiences \cite{baker2020linguistic}, we also present this setting.
\end{enumerate}

% \hal{there are a lot of mismatches that we need to clean up here (section 3.1). the above set does not match the set in Figure 1, which also doesn't match the set in 4.1.1. the terminology is also inconsistent (including at least Educational Content Delivery vs Educational Avatar). what's going on?} \kac{I wrote one part and the other was written by Christabel. We used different terminologies. I think the one in 4 is valid. Will check against the data to see where the  mismatch is coming from}

\noindent



%SS: Commented out since we don't discuss in Results.Participants were also asked to rank their preferences for when text-based generative AI should communicate in or understand AAE across each scenario on a scale from 1 (highest priority) to 3 (lowest priority). This segment aimed to compare Black Americans' differing preferences towards AI generation of AAE across distinct contexts, considering them relative to each other, to potentially provide insights into the broader societal expectations and acceptance for AAE in these technologies.
% We also asked participants to rank these settings to provid a clear understanding of where AAE/AAL speakers prioritize the use of AAE.
%The final portion of the survey asked participants to consider the potential benefits and dangers they perceive from the use of AAE by these AI systems, which helps us understand the broader implications of Generative AI's role in linguistic representation and inclusivity. 

%\begin{itemize}
 %   \item Always MUSE - \textit{Choice 0}
  %  \item Option to manually switch between AAE and MUSE - \textit{Choice 1}
   % \item Automatic detection and adaptation to AAE or MUSE - \textit{Choice 2}
    %\item Always AAE - \textit{Choice 3}
    %\item No preference as long as the system is effective - \textit{Choice 4}
%\end{itemize}

\begin{table}[t]
\centering
\footnotesize
%\rowcolors{2}{gray!10}{}
%\resizebox{0.98\linewidth}{!}{
\renewcommand{\arraystretch}{0.85} % Reduce row height
\setlength{\tabcolsep}{4pt}
\begin{tabular}{ @{~}l @{~~~} p{55mm} @{~} }
  \toprule			
%  \textbf{Option} & \textbf{Desired LLM Behavior} \\ 		
%  \midrule			
AlwaysMUSE &	LLM should always use MUSE.	\\
AlwaysAAE &	LLM should always use AAE. \\[0.5em]
AutoDetect &	LLM should automatically detect/adapt to\\&\quad the user's language variety. \\
UserOption &	LLM should provide an option to switch\\&\quad between AAE and MUSE. \\[0.5em]
NoPreference	&	No preference as long as the system is\\&\quad effective. \\
  \bottomrule			
\end{tabular}	
%}
%\end{adjustbox}
\caption{The set of possible choices for the preferences survey, which asked Black Americans about the contexts or scenarios in which they would prefer to have language model-based AI technologies generate AAE vs MUSE.} %Scenarios they considered ranged from email continuations in professional settings to educational avatars.}
%\hal{numbering doesn't match}
\label{table:Vignette Choices}
\end{table}






\section{System}\label{sec:system}
We consider systems in the form
%
\begin{subequations}\label{eq:system}
	\begin{align}
		\label{eq:system:x0}
		x(t) &= x_0(t), & t &\in (-\infty, t_0], \\
		%
		\label{eq:system:x}
		\dot x(t) &= f(x(t), z(t)), & t &\in [t_0, t_f],
	\end{align}
\end{subequations}
%
where $t \in \R$ is time, $t_0, t_f \in \R$ are the initial and final time, $x: \R \rightarrow \R^{n_x}$ is the state, and $x_0: \R \rightarrow \R^{n_x}$ is the initial state function. Furthermore, $f: \R^{n_x} \times \R^{n_z} \rightarrow \R^{n_x}$ is the right-hand side function, and the memory state, $z: \R \rightarrow \R^{n_z}$, is given by the convolution
%
\begin{subequations}\label{eq:system:delay}
	\begin{align}
		\label{eq:system:z}
		z(t) &= \int\limits_{-\infty}^t \alpha(t - s) \odot r(s) \incr s, \\
		%
		\label{eq:system:r}
		r(t) &= h(x(t)),
	\end{align}
\end{subequations}
%
where $r: \R \rightarrow \R^{n_z}$ is the delayed variable, and each element of $\alpha: \Rnn \rightarrow \Rnn^{n_z}$ is a \emph{regular} kernel (see Definition~\ref{def:regular:kernel}). Furthermore, $h: \R^{n_x} \rightarrow \R^{n_z}$ is the memory function. We assume that $f$ and $h$ are differentiable in their arguments, and we refer to the paper by Ponosov et al.~\cite[Thm.~1]{Ponosov:etal:2004} for more details on the existence and uniqueness of solutions to the initial value problem~\eqref{eq:system}--\eqref{eq:system:delay}. See also the book by Hale and Lunel~\cite{Hale:Lunel:1993}.
%
\begin{definition}\label{def:regular:kernel}
	A scalar-valued kernel, $\alpha: \Rnn \rightarrow \Rnn$, is \emph{regular} if it satisfies the following properties.
	%
	\begin{enumerate}
		\item It is non-negative and bounded, i.e., $0 \leq \alpha(t) \leq K$ for all $t \in \Rnn$ and for some finite $K \in \Rp$.
		%
		\item It is continuous, i.e., for all $\epsilon \in \Rp$ and $t \in \Rnn$, there exists a $\delta \in \Rp$ such that $|\alpha(s) - \alpha(t)| < \epsilon$ for all $s \in \Rnn$ satisfying $|s - t| < \delta$.
		%
		\item It is normalized such that
	\end{enumerate}
	%
	\begin{align}\label{eq:kernel:normalization}
		\int\limits_0^\infty \alpha(t) \incr t &= 1.
	\end{align}
\end{definition}
%
For a given system of DDEs with distributed time delays, each element of $\alpha$ may not satisfy~\eqref{eq:kernel:normalization}. However, as they are assumed to be nonzero and non-negative, it is straightforward to normalize them. Next, we present a few well-known corollaries about the steady states of~\eqref{eq:system:x}--\eqref{eq:system:delay} and their stability.
%
\begin{corollary}\label{thm:steady:state}
	A state $\bar x \in \R^{n_x}$ is a steady state of the system~\eqref{eq:system:x}--\eqref{eq:system:delay} if
	%
	\begin{align}\label{eq:steady:state}
		0 &= f(\bar x, \bar z), &
		\bar z &= \bar r = h(\bar x).
	\end{align}
\end{corollary}

\begin{proof}
	In steady state, $x(t) = \bar x$ for all $t$. Consequently, $r(t) = \bar r = h(\bar x)$ and
	%
	\begin{align}
		z(t)
		&= \int_{-\infty}^t \alpha(t - s) \odot \bar r \incr s
		 = \int_{-\infty}^t \alpha(t - s) \incr s \odot \bar r
		 = \bar r,
	\end{align}
	%
	for all $t$, where we have used the property~\eqref{eq:kernel:normalization} of each element of $\alpha$.
\end{proof}
%
\begin{corollary}\label{thm:stability}
	The system~\eqref{eq:system:x}--\eqref{eq:system:delay} is locally asymptotically stable around a steady state, $\bar x$, satisfying~\eqref{eq:steady:state} if $\real \lambda < 0$ for all $\lambda \in \C$ that satisfy the characteristic equation
	%
	\begin{align}\label{eq:characteristic:equation}
		\det\left(F - \lambda I + G \int_0^\infty e^{-\lambda s} \diag \alpha(s) \incr s H\right) = 0,
	\end{align}
	%
	where $I \in \R^{n_x \times n_x}$ is an identity matrix.
	%
	The matrices $F \in \R^{n_x \times n_x}$, $G \in \R^{n_x \times n_z}$, and $H \in \R^{n_z \times n_x}$ are the Jacobians of the right-hand side function and the delay function evaluated in the steady state:
	%
	\begin{align}\label{eq:jacobians}
		F &= \pdiff{f}{x}(\bar x, \bar z), &
		G &= \pdiff{f}{z}(\bar x, \bar z), &
		H &= \pdiff{h}{x}(\bar x).
	\end{align}
	%
\end{corollary}

\begin{proof}
	The linearized system corresponding to~\eqref{eq:system:x}--\eqref{eq:system:delay} describes the evolution of the deviation variable $X: \R \rightarrow \R^{n_x}$:
	%
	\begin{align}\label{eq:linearized:system}
		\dot X(t) &= F X(t) + G \int_{-\infty}^t \alpha(t - s) \odot H X(s) \incr s, &
		X(t) &= x(t) - \bar x.
	\end{align}
	%
	See, e.g., \cite{Cushing:1975, Cushing:1977, Miller:1972} for proofs of the condition~\eqref{eq:characteristic:equation} for asymptotic stability of the linearized system in~\eqref{eq:linearized:system}.
\end{proof}

\section{User evaluation with frequent users of mobile ASR: Lab study and online survey }
To evaluate the usability of our approach, we decided to conduct an in-person lab evaluation of the SpeechCompass phone case and the speech-to-text application (described in Section~\ref{subsection:app}), with frequent users of mobile transcription technology. We first conducted a large-scale online pilot study to inform the design of the in-person lab evaluation, which we conducted with eight deaf or hard-of-hearing participants, set up to mimic a realistic conversation scenario. 

\begin{figure*}
  \centering
  \includegraphics[width=0.75\linewidth]{images/second_study.pdf}
  \caption{Participants' preferences for different visualization techniques in the online survey. A) Results indicating how valuable the specific indicator would be for the user. B) Preferences for the specific indicators for speech direction.} 
  \label{fig:user_preferences_online} 
\end{figure*}


\subsection{Large-scale, online survey (n=494)} In this survey, we use screenshots of our interactive UI prototypes to solicit initial user
feedback on the potential for our proposed approach, to guide the design of a more realistic in-person lab study.

The study was conducted using the same Google Surveys deployment and screening methodology as for the foundational study, detailed in Section 3. The participants were shown different UI renderings and were asked to rate them. The large-scale online survey could only show static images of the interfaces, due to limitations of the survey tool. Out of 985 respondents we focus our analysis on the 494 participants who use captioning technology multiple times per week or more frequently. 

As shown in Figure~\ref{fig:user_preferences_online}A, the colored text was found to be valuable by 60\% of participants. Glyph indicators for speech direction, which included arrow and circle+line indicators, were found valuable by 70\%. The Edge indicator and the mini map had a less positive reception. 

To better understand which glyph indicators were favored, we also asked targeted questions about them, as shown in Figure~\ref{fig:user_preferences_online}B. \emph{Circle + line} was preferred by 13.1\% more respondents than the \emph{highlight box} (45.1\% vs 32.0\%), and the \emph{arrow} was preferred by 21.9\% more respondents than the \emph{circle + line} (51.2\% vs 29.3\%).


\subsection{Lab study (n=8)}
\alex{explain and emphasize intention}
We recruited 8 participants from our institution who were frequent users of captioning technology. Five were female, three were male, and all were deaf or hard of hearing. One participant was 25--34 years old, four were 34--44, one was 45--53, and two were 65+ (we are only allowed to collect age ranges at our institution). 


% setup: https://docs.google.com/document/d/1akr5HVMgJb8Kd9KaEZJcdXn2S0IbHhd8JdBPTE0TiA0/edit?usp=sharing
The study took place in a quiet lab over approximately 60 minutes and used the phone-case prototype (Figure~\ref{fig:pcb_design}) with our mobile ASR application (Figure~\ref{fig:phone_interfaces}). First, the participant was introduced to the technology, prototype, and the purpose of the study. Then, the participant was asked to fill out a background survey, which included demographic questions and their current use and experienced challenges with transcription technology. Afterward, the participant was introduced to different visualization scenarios with the SpeechCompass application. The participant used the SpeechCompass transcription while sitting between the two experimenters, as they all sat around a small table with the SpeechCompass phone case in the center. In each of the seven conditions, which ran for 5 minutes, the experimenters sat across from each other and had short conversations about different topics. The participants were instructed to turn off hearing aid devices if they used any, and were asked to use the SoundCompass UI and transcript to follow the conversation. The experimenters' casual conversations included topics like weekend plans, hobbies, and the weather. The seven conditions, which used the ASR, diarization, and localization functionality for different visualization techniques, are shown in Figure~\ref{fig:ui_options} and presented with more UI context in Figure~\ref{fig:phone_interfaces}. The conditions were:
\begin{enumerate}
    \item \textbf{Transcription only}. The transcribed text is shown in white on a black background. 
    
    \item \textbf{Edge indicator}. A circle (``dot'') that moves around the edge of the screen to point to the currently active speaker. The color of the dot changes based on the direction. 
    
    \item \textbf{Arrow indicator}. A glyph using a colored arrow next to a white text block. The glyph points in the direction of the associated speech. 
    
    \item \textbf{Circle + line indicator}. A glyph using a circle with a directional line next to a white text block. The glyph points in the direction of the speech associated with the text. 
    
    \item \textbf{Mini map}. A colored circle with a smaller circle (``dot'') moves around its edge to point to the currently active speaker. The color of the dot changes based on the direction. 
    
    \item \textbf{Colored text}. The text is colored based on the direction that the associated speech was coming from. 
    
    \item \textbf{Everything on}. All indicators are turned on (except the Circle + line, as it couldn't be used simultaneously with the arrow). 
\end{enumerate}

%five isolated visualization techniques, baseline with just text transcription (no speaker information), and with all visualization turned one. Minimap was shown with an arrow, since we envisioned it would be combined with other techniques. 
After participants had completed all conditions, they filled out a form that asked them to rate how desirable each of the five visual indicator styles (\textit{Edge indicator}, \textit{Arrow}, \textit{Circle  + line}, \textit{Colored map}, and \textit{Colored text}) were on a 7-point Likert scale, from \emph{-3: Strongly dislike} to \emph{+3: Strongly like}. Finally, they were asked to rate the overall value of directional feedback to the transcription experience, how strongly they would recommend these features to users of mobile captioning, and whether they had any general free-form feedback about SpeechCompass. 

\begin{figure*}
  \centering
  \includegraphics[width=0.65\linewidth]{images/study_setup.png}
  \caption{Examples of seven visualization scenarios that participants experienced in the in-person study.} 
  \label{fig:ui_options} 
\end{figure*}

%After running the scenarios, participants filled out the second part of the survey, which asked them to rate each scenario and overall impression on a scale from -3:strongly dislike to +3:strongly like. Finally, the participants filled out free form feedback about the study. 

\begin{figure*}
  \centering
  \includegraphics[width=0.65\linewidth]{images/box_plot_in_person_study_results.png}
  \caption{Boxplots of results of the in-person study. A) Participants' preferences for different visualization techniques. B) Overall opinions about augmented mobile ASR application.\alex{love these plots -- maybe to B you could also add the question about multi-people conversations as the leftmost, since it is also on same scale?} } 
  \label{fig:user_preferences} 
\end{figure*}

\subsection{Results}
Mobile transcription apps (e.g., Android Live Transcribe) were the most used communication technology for the participants. Specifically, three used them multiple times per day, one used them daily, three used them multiple times per week, and one used them rarely. 

75\% of participants frequently experienced the scenario where multiple people would get mixed up in the transcript (two multiple times per day, two daily, two multiple times per week). All participants agreed that it was challenging to participate in conversations when speech was combined from multiple people. 
%Similarly to the online survey, we asked participants to select the biggest challenges they experienced in their use of transcription technology (same options as in Figure~\ref{fig: survey-challenges}). where the majority (6/8) selected \textit{"Have to look away from the person I am talking to"}.  
\\

A Kruskal-Wallis (KW) test found a significant effect
on participant preferences for visualization techniques (P=.014).
The post-hoc pair-wise analyses using the Wilcoxon test with Bonferroni correction did, however, not show statistical significance between any pairs.
Of the five visual indicator styles that participants experienced, \emph{Colored text} was the most well-received (mean ($\bar{x})=2.625$), as it was rated positively by all the participants. %, with six strong like (+3), one like (+2), and one slight like (+1). 
The \emph{Arrow} indicator was also well-received ($\bar{x}=1.125$), with six positive, one negative, and one neutral participant.
%(one strong like (+3), three like (+2) and one slight like (+1)) and one dislike (-1) and one neutral (0)). 
Several participants noted that \emph{Arrow} and \emph{Colored text} worked well together: \emph{"Arrows + color seem to be most easier way to indicate the direction." (P2)} and \emph{"The combination of the colored text with the arrow was the most effective for me." (P7)}.

The other indicator styles received more mixed feedback. The feedback for both \emph{Edge indicator} ($\bar{x}=0.25$) and \emph{Circle + line} ($\bar{x}=-0.125$) was split between four negative and four positive participants. 
Some participants were concerned that \emph{Edge indicator} was distracting and not sufficiently discreet: \textit{"I do prefer the tool be as discrete as possible and would perhaps choose to avoid bright colored things moving around since this would be eye-catching and this kind of attention is often undesired" (P3)} and 
\textit{"Indicator moving around the edge was distracting and causing a bit of eye strain" (P2)}.
On the other hand, another participant found this style particularly useful: \textit{"the color dot moving to the speaker direction worked REALLY well" (P1)}. 
For \emph{Circle + line}, some participants struggled with its legibility: \textit{"If the analog direction indicators were larger (and translucent, or set behind)" (P8)} and \textit{"The lines in a circle were a bit slower and not as accurate (buggy)" (P5)}.
The \emph{Mini map} was rated positively by five participants and negatively by three. The most favorable participant stated: \emph{"this is also great for environmental awareness for those with single-sided hearing or no hearing at all." (P3)} and a participant who disliked the \emph{Edge indicator} commented: \emph{"steady map in the corner worked a bit better (P5)"}.

Overall, all participants agreed with the value of directional feedback ($\bar{x}=2.88$, seven Strongly agree:+3 and one Agree:+2) and would recommend these features to other users of captioning technology ($\bar{x}=2.63$, five Strongly agree:+3 and three Agree:+2): \textit{"I really liked that almost immediately I could tell that there was a speaker change, so that as soon as the text started to show up, I could better contextualize that text as attributed to a new speaker." (P1)}, \textit{"I'm very happy to see this tool being developed, it's a great addition to other speech recognition tools!" (P3)}, and \textit{"This prototype is definitely a life changer and I strongly believe that it will improve the quality of access to communication with speakers for many users" (P6)}.

\subsection{Discussion}
Consistent with the large-scale survey, the value of the diarization and localization features was immediate to all users. The participants were asked if directional guidance would be valuable in their mobile transcription experience. All eight users agreed. Also, all eight users would recommend this feature to mobile captioning users. 

While the large-scale survey helped inform our testing and exclude conditions (e.g., \emph{Highlight box}), the lab study allowed us to more rigorously evaluate the techniques in a realistic scenario. This difference became significant for the \emph{Edge indicator} and \emph{Mini map}, where issues, such as discreetness and distracting aspects, became evident during live usage. 

The results suggest that the combination of \textit{Colored text} and \textit{Arrow} would meet the preferences of most users, thanks to the balance of directional encoding and clarity. The arrow has redundant benefits too, since colored text might not always be reliably visible depending on lighting and screen conditions (e.g., strong sunlight, or dim display) and might also not be usable for colorblind users. The mixed feedback for other techniques indicates that the interface may also benefit from mechanisms that would allow users to customize the visualization style. Such customization could also apply to rendering properties, such as color, transparency, and line thickness, as some participants found \textit{Circle + line} particularly difficult to interpret. In both the large-scale survey and the in-person lab study, the \textit{Arrow} was preferred over \textit{Circle + line}. Through more customization options and extended usage in their daily lives, participants will be able to provide more nuanced feedback about these techniques. 


% Edge indicator and mini map had a less positive reception. However, they were rated more positively than those in the in-person study. Since participants didn't experience the working prototype, the discreet and distracting aspects that were observed in the in-person study were not captured. 

% In both online and in-person study, the arrow directional glyph was preferred to circle+line.



% This dichotomy demonstrates that users should be given a way to customize their experience. For example, the edge indicator received strong likes and dislikes from different participants. 


% This indicates that the interface designers should make the directional glyphs as easy to read as possible.


% The results of the online survey followed what was observed in the in-person study. Edge indicator and mini map had a less positive reception. However, they were rated more positively than those in the in-person study. Since participants didn't experience the working prototype, the discreet and distracting aspects that were observed in the in-person study were not captured. In both online and in-person study, the arrow directional glyph was preferred to circle+line.

% As indicated in the survey, the value of the diarization and localization features was immediate to all users. The participants were asked if directional feedback is valuable in their mobile transcription experience. All eight users agreed. Also, all eight users would recommend this feature to mobile captioning users. 


% \textit{"I really liked that almost immediately I could tell that there was a speaker change, so that as soon as the text started to show up, I could better contextualize that text as attributed to a new speaker." (P1)}

% P3
% Arrows + color seem to be most easier way to indicate the direction.
% \emph{"Arrows + color seem to be most easier way to indicate the direction." (P2)}
% P4
% \textit{"I'm very happy to see this tool being developed, it's a great addition to other speech recognition tools!" (P3)
% }
% \textit{"it was great to see so many options being offered" (P3)
% }
% P6 
% \textit{"This prototype is definitely a life changer and I strongly beleve that it will improve the quality of access to communication with speakers for many users" (P6)}

% P8
% The combination of the colored text with the arrow was the most effective for me.

% \emph{"The combination of the colored text with the arrow was the most effective for me." (P7)}
\section{Evaluation}
We provide three sets of insights into this section, organised as \textit{findings (F*)}. We quantitatively study the effect of the adversarial and counterfactual perturbations on the performance of informal reasoners and autoformalisation methods. Then, we dive deeper into method variants. Finally, 
we analyse the nature of formalisation errors made by the models.

\subsection{Robustness Analysis}
\paragraph{\textbf{\emph{F1: Noise perturbations have a stronger effect on formalisation methods than informal \ac{LLM} reasoners.}}}
Table~\ref{tab:distraction_k4_formalisation} shows that, on average, the accuracy of both direct and \ac{CoT} informal reasoning remains between $73\%$ and $74\%$ in the face of added noise. While the autoformalisation method performs similarly to informal reasoners on the original dataset, its performance decreases between $4\%$ and $11\%$. The accuracy drops especially with logical (L) and tautological (T) distractions, whose logical language formats trick the \ac{LLM} into formalizing the noisy clauses. On the other hand, the linguistically complex and more natural sentences of encyclopedic distractions show a minor effect, suggesting that \acp{LLM} successfully avoids formalizing the more complicated sentences.

\paragraph{\textbf{\emph{F2: All \ac{LLM}-based reasoning methods suffer a drop for counterfactual perturbations.}}} % influence .}}}
Table~\ref{tab:distraction_k4_formalisation} shows that counterfactual statements cause a significant decrease in performance for both the informal reasoners and autoformalisation methods of between $12\%$ and $13\%$ on average. 
Moreover, this observation also holds for all tested models, i.e., none are robust towards counterfactual perturbations across every evaluated dimension. Even the strongest model, GPT 4o-mini, yields a performance of 63-68\%, which is relatively close to the random performance of 50\%. The high impact of counterfactual statements (the single ``not'' inserted) could be due to the inability of \acp{LLM} to overwrite prior knowledge with explicitly stated information or memorization of the answers. We study the error sources further in §\ref{subsec:errors}.  

\noindent \paragraph{\textbf{\emph{F3: Introducing multiple noise sentences has an effect only for logical distractions.}}}
We show the impact of introducing between one and four sentences for the two top-performing autoformalisation models in Figure~\ref{fig:length_distraction}. The figure shows similar trends with and without counterfactual perturbations.
As additional logical distractions are introduced, the model performance consistently decreases. Tautological (T) distractions lead to a decline in accuracy with a single disruptive sentence, yet adding more noise does not worsen the outcome. 
The tautological corpus introduces truth constants for all sentences as a persistent unseen logical construct. Given that this leads only to a decrease for a single occurrence, we can assume that a model can consistently handle the same unseen logical construct. In contrast, the logical corpus increases the chance of adding text, requiring new, previously unseen reasoning constructs for each added sentence. The impact of encyclopedic noise remains negligible, generalising F1 to $k$ sentences. Similarly, counterfactual perturbations remain much more effective for all settings, generalising F2.

\begin{table}[!t]
\small
\setlength{\modelspacing}{2pt}
\setlength{\tabcolsep}{1.7pt} % Default value: 6pt
\setlength{\belowrulesep}{4pt}
\begin{threeparttable}
    \centering
    \begin{tabular}{cc l r rrr @{\quad} rrrr}
\toprule
\multirow{2}{*}{} & \multirow{2}{*}{} & Reasoning & \multirow{2}{*}{O} & \multicolumn{3}{c}{Distraction} & \multicolumn{4}{c}{Counterfactual} \\
 & & Format & & E& L & T & $\text{O}_C$ & $\text{E}_C$& $\text{L}_C$ & $\text{T}_C$\\
\midrule
\multirow{6}{*}{\rotatebox{90}{Gemma-2}} & \multirow{3}{*}{\rotatebox{90}{9b}}
   & Informal (direct) & \textbf{0.78} & \textbf{0.80} & \textbf{0.79} & \textbf{0.77} & 0.58 & 0.52 & 0.50 & 0.59 \\
 & & Informal (CoT) & 0.72 & 0.78 & 0.73 & 0.76 & 0.61 & \textbf{0.57} & \textbf{0.60} & \textbf{0.66} \\
 & & Formal (FOL) & 0.62 & 0.58 & 0.52 & 0.53 & \textbf{0.63} & 0.52 & 0.46 & 0.46 \\[\modelspacing]
\cmidrule{2-11}
 & \multirow{3}{*}{\rotatebox{90}{27b}} 
   & Informal (direct) & 0.71 & 0.69 & \textbf{0.66} & \textbf{0.68} & 0.59 & 0.51 & 0.54 & 0.59 \\
 & & Informal (CoT) & 0.66 & 0.65 & 0.64 & 0.63 & 0.62 & 0.58 & \textbf{0.62} & \textbf{0.64} \\
 & & Formal (FOL) & \textbf{0.74} & \textbf{0.74} & 0.61 & 0.61 & \underline{\textbf{0.72}} & \underline{\textbf{0.67}} & 0.58 & 0.51 \\[\modelspacing]
\midrule
\multirow{6}{*}{\rotatebox{90}{Mistral}} & \multirow{3}{*}{\rotatebox{90}{7B}} 
   & Informal (direct) & 0.77 & \textbf{0.77} & 0.75 & \textbf{0.79} & \textbf{0.63} & \textbf{0.54} & \textbf{0.54} & \textbf{0.66} \\
 & & Informal (CoT) & \textbf{0.79} & 0.75 & \textbf{0.77} & 0.78 & 0.55 & 0.52 & \textbf{0.54} & 0.58 \\
 & & Formal (FOL) & 0.62 & 0.58 & 0.54 & 0.57 & 0.50 & \textbf{0.54} & 0.51 & 0.52 \\[\modelspacing]
\cmidrule{2-11}
 & \multirow{3}{*}{\rotatebox{90}{Small}} 
   & Informal (direct) & \textbf{0.77} & \textbf{0.76} & \textbf{0.76} & \textbf{0.75} & 0.61 & 0.51 & 0.56 & 0.59 \\
 & & Informal (CoT) & 0.72 & 0.72 & 0.72 & 0.71 & \textbf{0.62} & \textbf{0.59} & \textbf{0.62} & \textbf{0.68} \\
 & & Formal (FOL) & 0.68 & 0.59 & 0.53 & 0.64 & 0.54 & 0.55 & 0.49 & 0.51 \\[\modelspacing]
\midrule
\multirow{6}{*}{\rotatebox{90}{Llama-3.1}} & \multirow{3}{*}{\rotatebox{90}{8B}} 
   & Informal (direct) & 0.63 & 0.61 & 0.64 & 0.66 & 0.61 & \textbf{0.62} & 0.59 & 0.61 \\
 & & Informal (CoT) & 0.73 & \textbf{0.73} & \textbf{0.71} & \textbf{0.72} & \textbf{0.62} & 0.59 & \textbf{0.61} & \textbf{0.65} \\
 & & Formal (FOL) & \textbf{0.77} & 0.71 & 0.63 & 0.52 & 0.60 & 0.58 & 0.55 & 0.52 \\[\modelspacing]
\cmidrule{2-11}
 & \multirow{3}{*}{\rotatebox{90}{70B}} 
   & Informal (direct) & 0.77 & 0.74 & 0.74 & 0.73 & 0.62 & 0.53 & 0.56 & 0.64 \\
 & & Informal (CoT) & \textbf{0.78} & \textbf{0.75} & \textbf{0.76} & \textbf{0.76} & 0.64 & 0.61 & \textbf{0.66} & \underline{\textbf{0.73}} \\
 & & Formal (FOL) & 0.74 & 0.73 & 0.71 & 0.71 & \textbf{0.66} & \textbf{0.62} & 0.59 & 0.57 \\[\modelspacing]
 \midrule
\multirow{3}{*}{\rotatebox{90}{GPT}} & \multirow{3}{*}{\rotatebox{90}{4o-mini}} 
   & Informal (direct) & 0.78 & 0.77 & 0.79 & 0.79 & 0.64 & 0.61 & 0.61 & 0.63 \\
 & & Informal (CoT) & 0.80 & 0.80 & \underline{\textbf{0.81}} & \underline{\textbf{0.82}} & \textbf{0.68} & \textbf{0.63} & \underline{\textbf{0.68}} & \textbf{0.64} \\
 & & Formal (FOL) & \underline{\textbf{0.84}} & \underline{\textbf{0.82}} & 0.73 & 0.79 & 0.63 & 0.62 & 0.57 & 0.54 \\[\modelspacing]
 \midrule
\multicolumn{2}{c}{\multirow{3}{*}{\textbf{Avg}}} 
 & Informal (direct) & 0.74 & 0.73 & 0.73 & 0.73 & 0.61 & 0.55 & 0.56 & 0.62 \\
 & & Informal (CoT) & 0.74 & 0.74 & 0.73 & 0.74 & 0.62 & 0.58 & 0.62 & 0.65 \\
  & & Formal (FOL) & 0.72 & 0.68 &	0.61 & 0.62 & 0.61 & 0.59 & 0.54 & 0.52 \\
\bottomrule
\end{tabular}
\caption{Accuracies of informal and autoformalisation-based deductive reasoners. The best overall model per dataset is underlined; the best model version is marked in bold.}
\label{tab:distraction_k4_formalisation}
\end{threeparttable}
\end{table} 

\begin{figure}[!t]
    \centering
    \scriptsize
    \begin{tikzpicture}
        \begin{axis}[name=gpt,
            title={GPT-4o-mini},
            width=0.6\linewidth,
            height=0.6\linewidth,
            xlabel={\# Noise sentences},
            ylabel={Accuracy},
            xmin=-0.1, xmax=4.1,
            ymin=0.5, ymax=0.9,
            xtick={1,2,4},
            ytick={0.55, 0.6, 0.65, 0.75, 0.8, 0.85},
            title style={yshift=-0.6em},
            legend style={at={(1,-0.15)},
	           anchor=north,legend columns=-1},
            x label style={at={(axis description cs:1,-0.05)},anchor=north},
            y label style={at={(axis description cs:-0.15,0.5)},anchor=south},
            ymajorgrids=true,
            grid style=dashed,
        ]
            \addplot[color=blue, mark=square,]
                coordinates {
                (0,0.848076939582825)(1,0.823076903820038)(2,0.826923072338104)(4,0.821153819561005)
                };
            \addplot[color=red, mark=triangle,]
                coordinates {
                (0,0.848076939582825)(1,0.817307710647583)(2,0.801923096179962)(4,0.759615361690521)
                };
            \addplot[color=green, mark=diamond,] 
                coordinates {
                (0,0.848076939582825)(1,0.767307698726654)(2,0.769230782985687)(4,0.803846180438995)
                };
            \addplot[color=blue, mark=square*] 
                coordinates {
                (0,0.627777755260468)(1,0.622222244739533)(2,0.600000023841858)(4,0.633333325386047)
                };
            \addplot[color=red, mark=triangle*,] 
                coordinates {
                (0,0.627777755260468)(1,0.611111104488373)(2,0.611111104488373)(4,0.594444453716278)
                };
            \addplot[color=green, mark=diamond*,] 
                coordinates {
                (0,0.627777755260468)(1,0.572222232818604)(2,0.538888871669769)(4,0.555555582046509)
                };
                \legend{E,L,T,$\text{E}_C$, $\text{L}_C$ , $\text{T}_C$}
        \end{axis}

        \begin{axis}[name=llama, at={($(gpt.east)+(0.1cm,0)$)},anchor=west,
            title={Llama 3.1 70b},
            width=0.6\linewidth,
            height=0.6\linewidth,
            xmin=-0.1,, xmax=4.1,
            ymin=0.5, ymax=0.9,
            xtick={1,2,4},
            ytick={0.55, 0.6, 0.65, 0.75, 0.8, 0.85},
            title style={yshift=-0.6em},
            yticklabel=\empty,
            ymajorgrids=true,
            grid style=dashed,
        ]
            \addplot[color=blue, mark=square,]
                coordinates {
                (0,0.838461518287659)(1,0.817307710647583)(2,0.805769205093384)(4,0.817307710647583)
                };
            \addplot[color=red, mark=triangle,]
                coordinates {
                (0,0.838461518287659)(1,0.819230794906616)(2,0.803846180438995)(4,0.771153867244721)
                };
            \addplot[color=green, mark=diamond,]
                coordinates {
                (0,0.838461518287659)(1,0.803846180438995)(2,0.807692289352417)(4,0.805769205093384)
                };
            \addplot[color=blue, mark=square*]
                coordinates {
                (0,0.627777755260468)(1,0.622222244739533)(2,0.577777802944183)(4,0.594444453716278)
                };
            \addplot[color=red, mark=triangle*,]
                coordinates {
                (0,0.627777755260468)(1,0.583333313465118)(2,0.561111092567444)(4,0.577777802944183)
                };
            \addplot[color=green, mark=diamond*,]
                coordinates {
                (0,0.627777755260468)(1,0.627777755260468)(2,0.566666662693024)(4,0.577777802944183)
                };
        \end{axis}
    \end{tikzpicture}
    \caption{Influence of the number of noisy sentences for FOL.}
    \label{fig:length_distraction}
\end{figure}



\subsection{Impact of Method Design}
\paragraph{\textbf{\emph{F4: \ac{CoT} prompting is most impactful when both noise and counterfactual perturbations are applied.}}}
The accuracies for the individual \acp{LLM} in Table~\ref{tab:distraction_k4_formalisation} show that the impact of \ac{CoT} is negligible for noise-only datasets (first four columns). Meanwhile, the benefit from \ac{CoT} is most pronounced in the datasets that combine noise and counterfactual perturbations.
The better-performing informal prompting strategy for a model remains stable for all types of distractions. Still, the decline in performance due to counterfactuals leads to a less consistent preference for a specific prompting style.

\paragraph{\textbf{\emph{F5: The best-performing grammar differs per model and is unstable across data versions.}}}

The evaluation of different logical forms for formal \ac{LLM}-based reasoning in Table~\ref{tab:distraction_k4_logical_form} shows the preference of some models for specific syntactic formats.
Llama 3.1 70B has a considerable improvement of $12\%$ with TPTP syntax on the original set, while Llama 3.1 8B benefits from the R-FOL syntax. However, all grammars show a declining accuracy trend and increased syntax errors for noise perturbations, where the best grammar loses its advantage over the rest. 
When comparing the grammars on the counterfactual partitions, we observe that TPTP is consistently more robust than the standard first-order logic grammar. Here, GPT 4o-mini shows a reduction from $O$ to $O_C$ of $20\%$ for FOL and only $12\%$ for the TPTP grammar. Since this does not correlate with fewer syntax errors, the formalisation in TPTP prevents semantical errors for counterfactual premises. 
A positive reading of these results, especially the minor differences between FOL and R-FOL, is that autoformalisation \acp{LLM} can adapt to the grammar syntax prescribed in the prompt without further loss in performance.

\begin{table}[!t]
\small
\setlength{\modelspacing}{2pt}
\setlength{\tabcolsep}{1.7pt} % Default value: 6pt
\setlength{\belowrulesep}{4pt}
\begin{threeparttable}
    \centering
    \begin{tabular}{cc l r rrr @{\quad} rrrr}
\toprule
\multirow{2}{*}{} & \multirow{2}{*}{} & Grammar & \multirow{2}{*}{O} & \multicolumn{3}{c}{Distraction} & \multicolumn{4}{c}{Counterfactual} \\
 & & Syntax & & E& L & T & $\text{O}_C$ & $\text{E}_C$& $\text{L}_C$ & $\text{T}_C$\\
\midrule
\multirow{6}{*}{\rotatebox{90}{Llama-3.1}} & \multirow{3}{*}{\rotatebox{90}{8B}} 
   & FOL & 0.77 & \textbf{0.71} & 0.61 & \textbf{0.53} & 0.58 & \textbf{0.55} & 0.52 & \textbf{0.56} \\
 & & R-FOL & \textbf{0.78} & 0.69 & \textbf{0.62} & \textbf{0.53} & 0.58 & \textbf{0.55} & \textbf{0.54} & 0.52 \\
 & & TPTP & 0.73 & 0.67 & 0.55 & 0.51 & \textbf{0.68} & 0.54 & 0.46 & 0.51 \\[\modelspacing]
\cmidrule{2-11}
 & \multirow{3}{*}{\rotatebox{90}{70B}} 
   & FOL & 0.76 & 0.73 & 0.71 & \textbf{0.72} & 0.67 & 0.57 & 0.63 & 0.56 \\
 & & R-FOL & 0.76 & 0.73 & 0.67 & 0.71 & 0.64 & 0.57 & 0.53 & 0.64 \\
 & & TPTP & \underline{\textbf{0.88}} & \underline{\textbf{0.84}} & \underline{\textbf{0.81}} & \textbf{0.72} & \underline{\textbf{0.81}} & \underline{\textbf{0.68}} & \underline{\textbf{0.67}} & \underline{\textbf{0.68}} \\[\modelspacing]
\midrule
\multirow{3}{*}{\rotatebox{90}{GPT}} & \multirow{3}{*}{\rotatebox{90}{4o-mini}} 
   & FOL & \textbf{0.84} & \textbf{0.82} & \textbf{0.72} & \underline{\textbf{0.78}} & 0.64 & \textbf{0.63} & \textbf{0.61} & 0.51 \\
 & & R-FOL & \textbf{0.84} & 0.77 & 0.70 & \underline{\textbf{0.78}} & \textbf{0.72} & 0.56 & 0.54 & \textbf{0.63} \\
 & & TPTP & 0.83 & \textbf{0.82} & 0.71 & 0.71 & 0.69 & \textbf{0.63} & 0.57 & 0.57 \\
\bottomrule
\end{tabular}
\caption{Accuracies of different formalisation grammars for autoformalisation.}
\label{tab:distraction_k4_logical_form}
\end{threeparttable}
\end{table} 

\paragraph{\textbf{\emph{F6: Feedback does not help \acp{LLM} self-correct to mitigate robustness issues.}}}
\autoref{tab:distraction_k4_feedback} shows the results with different error recovery mechanisms. The results indicate that no feedback strategy emerges as a winner in the different datasets. 
All feedback variants reduce syntax errors for noise perturbations, but given the lack of a consistent increase in accuracy, the corrected formalisations are most likely to contain semantic errors still. 
The type of feedback message only has a minor influence on correcting syntax errors, whereas Llama 3.1 70b and GPT 4o-mini correct slightly more syntax errors with specific error messages. This finding aligns with \cite{huang2023large}, who also found that \acp{LLM} cannot consistently self-correct their reasoning after receiving relevant feedback.

\begin{table}[!ht]
\small
\setlength{\modelspacing}{2pt}
\setlength{\tabcolsep}{1.7pt} % Default value: 6pt
\setlength{\belowrulesep}{4pt}
\begin{threeparttable}
    \centering
    \begin{tabular}{cc l r rrr @{\quad} rrrr}
\toprule
\multirow{2}{*}{} & \multirow{2}{*}{} & \multirow{2}{*}{Feedback} & \multirow{2}{*}{O} & \multicolumn{3}{c}{Distraction} & \multicolumn{4}{c}{Counterfactual} \\
 & & & & E& L & T & $\text{O}_C$ & $\text{E}_C$& $\text{L}_C$ & $\text{T}_C$\\
\midrule
\multirow{8}{*}{\rotatebox{90}{Llama-3.1}} & \multirow{4}{*}{\rotatebox{90}{8B}} 
   & No recovery & 0.77 & \textbf{0.72} & 0.62 & 0.53 & 0.59 & 0.58 & 0.56 & \textbf{0.56} \\
 & & Error type & \textbf{0.79} & 0.71 & 0.63 & \textbf{0.56} & \textbf{0.66} & 0.54 & 0.52 & 0.51 \\
 & & Error message & 0.78 & 0.71 & \textbf{0.67} & 0.55 & 0.59 & 0.53 & \underline{\textbf{0.64}} & 0.49 \\
 & & Warning & 0.74 & 0.66 & 0.58 & 0.55 & 0.55 & \textbf{0.60} & 0.49 & 0.49 \\[\modelspacing]
\cmidrule{2-11}
 & \multirow{4}{*}{\rotatebox{90}{70B}} 
   & No recovery & \textbf{0.77} & \textbf{0.72} & \textbf{0.73} & 0.71 & \textbf{0.64} & 0.59 & \textbf{0.61} & 0.56 \\
 & & Error type & 0.72 & 0.70 & 0.72 & \textbf{0.73} & 0.62 & 0.56 & 0.60 & 0.58 \\
 & & Error message & 0.71 & 0.70 & \textbf{0.73} & 0.71 & \textbf{0.64} & 0.59 & 0.54 & \underline{\textbf{0.64}} \\
 & & Warning & 0.69 & \textbf{0.72} & 0.72 & 0.72 & 0.62 & \underline{\textbf{0.65}} & \textbf{0.61} & 0.63 \\[\modelspacing]
\midrule
\multirow{4}{*}{\rotatebox{90}{GPT}} & \multirow{4}{*}{\rotatebox{90}{4o-mini}} 
   & No recovery & \underline{\textbf{0.84}} & \underline{\textbf{0.82}} & 0.73 & 0.79 & 0.64 & \textbf{0.62} & 0.56 & \textbf{0.56} \\
 & & Error type & 0.83 & 0.79 & 0.74 & 0.76 & 0.67 & 0.57 & 0.56 & \textbf{0.56} \\
 & & Error message & \underline{\textbf{0.84}} & 0.78 & \underline{\textbf{0.77}} & \underline{\textbf{0.80}} & 0.62 & 0.59 & 0.56 & \textbf{0.56} \\
 & & Warning & \underline{\textbf{0.84}} & 0.75 & 0.73 & 0.76 & \underline{\textbf{0.70}} & 0.61 & \textbf{0.61} & 0.55 \\
 \bottomrule
\end{tabular}
\caption{Accuracies of error recovery strategies.}
\label{tab:distraction_k4_feedback}
\end{threeparttable}
\end{table} 

\subsection{Error Analysis}
\label{subsec:errors}
\paragraph{\textbf{\emph{F7: Autoformalisation increases syntax errors for noise perturbations.}}}
The low performance for noise perturbations correlates with more syntax errors for all models and distraction categories (cf. execution rates in Table~\ref{tab:appendix_k4_formalisation_exec}). The three worst-performing models (both Mistral models, Gemma-2 9b) generate, at best, for $37\%$  and, at worst, for only $4\%$ of the samples, a valid logical form.
Gemma-2 9b and Llama3.1 8b produce more syntax errors than the larger counterparts, suggesting that larger models are more robust towards noise perturbations. 
The accuracy of syntactically valid samples is higher than the informal reasoning methods for most distractions (Table~\ref{tab:appendix_k4_formalisation_vacc}), motivating informal reasoning as a backup strategy for formal reasoning. The error message feedback reveals two common syntax errors: 1) errors by models with an initial low execution rate exhibit issues with the template structure, including using incorrect keywords or adding conversational phrases;
2) perturbation-related errors, the most common of which is using undefined truth constants as part of tautological distractions. 

\paragraph{\textbf{\emph{F8: Autoformalisation increases semantic errors for counterfactuals.}}}
Unlike the introduced noise, counterfactual perturbations do not lead to more syntax errors. The execution rate in Table~\ref{tab:appendix_k4_formalisation_exec} is stable or improves for counterfactuals. However, we see a drop in accuracy for the counterfactual column $\text{O}_C$ in Table~\ref{tab:distraction_k4_formalisation} and can conclude that the number of logical forms with semantic errors has to increase. This suggests that the introduced negation is not correctly formalised. Looking at the warnings generated by the feedback mechanism, for GPT 4o-mini, $161$ warning messages are generated on the unperturbed data. $54$ of these were fixed with a single iteration. Not considering predicates and individuals as part of the context is the most frequent warning across all models. 
This work identifies signal collapse as a critical bottleneck in one-shot neural network pruning. Performance loss in pruned networks is due to \textbf{signal collapse} in addition to the removal of critical parameters. We propose \textbf{REFLOW} (\textbf{Re}storing \textbf{F}low of \textbf{Low}-variance signals), a simple yet effective method that mitigates signal collapse without computationally expensive weight updates. By focusing on signal preservation, REFLOW highlights the importance of mitigating signal collapse in sparse networks and enables magnitude pruning to match or surpass state-of-the-art one-shot pruning methods such as CHITA, CBS, and WF.

REFLOW consistently achieves state-of-the-art accuracy across diverse architectures, restoring ResNeXt-101 from under 4.1\% to 78.9\% top-1 accuracy at 80\% sparsity on ImageNet. Its lightweight design makes it a practical solution for both research and deployment, delivering high-quality sparse models without the overhead of traditional approaches. These findings challenge the traditional emphasis on weight selection strategies and underscore the critical role of signal propagation for achieving high-quality sparse networks in the context of one-shot pruning.



\section*{Conclusion}
This paper aims to enhance our understanding of the computational complexity of computing various Shapley value variants. We found that for various ML models --- including decision trees, regression tree ensembles, weighted automata, and linear regression --- both local and global interventional and baseline SHAP can be computed in polynomial time under HMM modeled distributions. This extends popular algorithms, such as TreeSHAP, beyond their empirical distributional scope. We also establish strict complexity gaps between the various SHAP variants (baseline, interventional, and conditional) and prove the intractability of computing SHAP for tree ensembles and neural networks in simplified scenarios. Overall, we present SHAP as a versatile framework whose complexity depends on four key factors: \begin{inparaenum}[(i)] \item model type, \item SHAP variant, \item distribution modeling approach, \item and local vs. global explanations\end{inparaenum}. We believe this perspective provides deeper insight into the computational complexity of SHAP, paving the way for future work.




%We believe that our framework provides a more intricate understanding of SHAP computation complexity across different models, distributions, and variants, paving the way for further research.

Our work opens promising directions for future research. First, expanding our computational analysis to other SHAP-related metrics, such as asymmetric SHAP~\citep{frye20} and SAGE~\citep{covert2020understanding}, would be valuable. Additionally, we aim to explore more expressive distribution classes and relaxed assumptions beyond those in Section \ref{sec:tractable} while maintaining tractable SHAP computation. Finally, when exact computation is intractable (Section \ref{sec:intractable}), investigating the approximability of SHAP metrics through approximation and parameterized complexity theory~\citep{downey2012parameterized} is an important direction.

%Our work opens several promising avenues for future research on the computational properties of explainable AI methods, with a particular focus on SHAP. First, it would be interesting to broaden the computational analysis conducted in this work to include other popular SHAP-related metrics in the literature, such as asymmetric SHAP \cite{frye20} and SAGE \cite{covert2020understanding}. Also, in the future, we aim to explore more expressive distribution classes and relaxed distributional assumptions—extending beyond those examined in Section \ref{sec:tractable} —that still yield tractable SHAP computation. Finally, when exact computation proves intractable (Section \ref{sec:intractable}), it is worthwhile to theoretically investigate the question of the approximability of computing the SHAP metrics across various configurations, through the lens of approximation and parametrized complexity theory \cite{arora2009computational}.

%This paper aims to deepen our understanding of the computational complexity involved in obtaining different Shapley value variants. We found that for a variety of ML models, including decision trees, tree ensembles for regression, weighted automata, and linear regression models — computing both local and global interventional and baseline SHAP can be done in polynomial time when distributions are modeled by HMMs. This extends the distributional scope of popular algorithms like TreeSHAP, which is limited to empirical distributions. Additionally, we demonstrate a strict complexity gap between SHAP variants, showing that interventional and baseline SHAP can be strictly easier to compute than conditional SHAP. Despite these positive results, we uncovered intractability for various SHAP variants in neural networks and tree ensembles. Finally, we provided generalized complexity relations across SHAP variants. We believe that our framework offers a deeper understanding of the complexity involved in computing SHAP across various variants, models, distributions, as well as in both local and global computations, laying the groundwork for future research.


% \input{Sections/template_text}


\bibliographystyle{ACM-Reference-Format}
\bibliography{bibliography}

\newpage
\appendix




\section{Ten Strategies Summarized from Literature Review }
\label{apdx:strategies}
Strategies with multiple instances are input to prompt Immediate Feedback and Asynchronous Feedback in \textit{TutorUp} and serve as feedback for Baseline System.
% \usepackage{color}
% \usepackage{tabularray}

\aptLtoX{\begin{table}[H]
\centering
\caption{Strategies summarized from Literature Review with two instances for each.}
\begin{tabular}{p{125pt}p{125pt}}
\toprule
\textbf{Strategy Category}                                 & \textbf{Strategy Instance}                                            \\
\midrule
Show empathy and respect toward students                   & Refer to students by name~                                            \\
                                                           & Demonstrate concern for their student ~                               \\
                                                           \hline
Promote peer interaction                                   & Encourages contact between students and faculty~                      \\
                                                           & Develops reciprocity and cooperation among students.~                 \\
                                                           \hline
Give prompt, correct, positive, and personalized Feedback~ & Encouragement to positive behavior ~                                  \\
                                                           & Be Free with Praise and Constructive in Criticism~                    \\
                                                           \hline
Promote persistence                                        & Don‘t allow students to give up, scaffold until they succeed ~        \\
\hline
                                                           & Treat each student as capable - believe that students want to learn ~ \\
                                                           \hline
Maintain active learning                                   & Encourage challenging discussions, team projects, and peer critiques  \\
\hline
                                                           & Expect active participation ~                                         \\
Set Clear goals~                                           & Show the Need for the Lesson~                                         \\
                                                           & Group work                                                            \\
                                                           \hline
Give autonomy to students                                  & Allow students to participate in decision-making and assessment ~     \\
\hline
                                                           & Take feedback from students~                                          \\
Promote group work                                         & Assign Responsibilities~                                              \\
                                                           & Ask students to answer each others questions ~                        \\
                                                           \hline
Set time constraint                                        & Emphasizes time on task.~                                             \\
                                                           & Ask for how many time students need                                   \\
                                                           \hline
Set behavioral expectations                                & Set clear expectation on student behavior                             \\
                                                           & Insist that students show respect and care for one another~~          \\
                                                           \bottomrule
\end{tabular}
\end{table}}{\begin{table}[H]
\centering
\caption{Strategies summarized from Literature Review with two instances for each.}
\begin{tblr}{
  width = \linewidth,
  colspec = {Q[431]Q[506]},
  cell{2}{1} = {r=2}{},
  cell{4}{1} = {r=2}{},
  cell{6}{1} = {r=2}{},
  cell{8}{1} = {r=2}{},
  cell{10}{1} = {r=2}{},
  cell{12}{1} = {r=2}{},
  cell{14}{1} = {r=2}{},
  cell{16}{1} = {r=2}{},
  cell{18}{1} = {r=2}{},
  cell{20}{1} = {r=2}{},
  hline{1-2,4,6,8,10,12,14,16,18,20,22} = {-}{},
}
\textbf{Strategy Category}                                 & \textbf{Strategy Instance}                                            \\
Show empathy and respect toward students                   & Refer to students by name~                                            \\
                                                           & Demonstrate concern for their student ~                               \\
Promote peer interaction                                   & Encourages contact between students and faculty~                      \\
                                                           & Develops reciprocity and cooperation among students.~                 \\
Give prompt, correct, positive, and personalized Feedback~ & Encouragement to positive behavior ~                                  \\
                                                           & Be Free with Praise and Constructive in Criticism~                    \\
Promote persistence                                        & Don‘t allow students to give up, scaffold until they succeed ~        \\
                                                           & Treat each student as capable - believe that students want to learn ~ \\
Maintain active learning                                   & Encourage challenging discussions, team projects, and peer critiques  \\
                                                           & Expect active participation ~                                         \\
Set Clear goals~                                           & Show the Need for the Lesson~                                         \\
                                                           & Group work                                                            \\
Give autonomy to students                                  & Allow students to participate in decision-making and assessment ~     \\
                                                           & Take feedback from students~                                          \\
Promote group work                                         & Assign Responsibilities~                                              \\
                                                           & Ask students to answer each others questions ~                        \\
Set time constraint                                        & Emphasizes time on task.~                                             \\
                                                           & Ask for how many time students need                                   \\
Set behavioral expectations                                & Set clear expectation on student behavior                             \\
                                                           & Insist that students show respect and care for one another~~          
\end{tblr}
\end{table}}

% Here we have space to include additional information and materials. For example survey questions and detailed descriptions of the individual prompts used by the system.



\section{Evaluation results in tabular form}
\label{apdx:evaluation}
% \subsection{Evaluation results in tabular form}

These tables contain the same information as the bar charts in Fig. \ref{fig:participant_ratings} and Fig. \ref{fig:expert_ratings}.
% \label{fig:participant_ratings}
% \label{fig:expert_ratings}

% \usepackage{tabularray}
\aptLtoX{\begin{table}[H]
\centering
\caption{Statistics on Qualitative Assessment.}
\label{tab:qualitativeResult}
\begin{tabular}{llp{40pt}p{40pt}p{40pt}p{40pt}}
\toprule
Measurement           &      & \textbf{Use Strategies Appropriately} & \textbf{Use Strategies Effectively} & \textbf{Students are More Engaged.} & \textbf{Strategies Accessible for Students} \\
\midrule
Baseline              & Mean & 2.06                               & 1.75                                & 1.84                                & 2.00                                        \\
                      & SD   & 0.60                               & 0.45                                & 0.44                                & 0.52                                        \\
                      \hline
TutorUp               & Mean & 2.47                               & 2.09                                & 2.06                                & 2.34                                        \\
                      & SD   & 0.53                               & 0.43                                & 0.54                                & 0.48                                        \\
                      \hline
Shapiro-Wilk          &      & 0.05                               & 0.34                                & 0.12                                & 0.65                                        \\
\hline
Paired Samples t Test &      & 0.03                               & 0.09                                & 0.23                                & 0.10  \\
\bottomrule                                      
\end{tabular}
\end{table}}{\begin{table}[H]
\centering
\caption{Statistics on Qualitative Assessment.}
\label{tab:qualitativeResult}
\begin{tblr}{
  width = \linewidth,
  colspec = {Q[92]Q[71]Q[181]Q[173]Q[190]Q[225]},
  cells = {c},
  cell{1}{1} = {c=2}{0.162\linewidth},
  cell{2}{1} = {r=2}{},
  cell{4}{1} = {r=2}{},
  cell{6}{1} = {c=2}{0.162\linewidth},
  cell{7}{1} = {c=2}{0.162\linewidth},
  hline{1-2,4,6-8} = {-}{},
}
Measurement           &      & \textbf{Use Strategies Appropriately} & \textbf{Use Strategies Effectively} & \textbf{Students are More Engaged.} & \textbf{Strategies Accessible for Students} \\
Baseline              & Mean & 2.06                               & 1.75                                & 1.84                                & 2.00                                        \\
                      & SD   & 0.60                               & 0.45                                & 0.44                                & 0.52                                        \\
TutorUp               & Mean & 2.47                               & 2.09                                & 2.06                                & 2.34                                        \\
                      & SD   & 0.53                               & 0.43                                & 0.54                                & 0.48                                        \\
Shapiro-Wilk          &      & 0.05                               & 0.34                                & 0.12                                & 0.65                                        \\
Paired Samples t Test &      & 0.03                               & 0.09                                & 0.23                                & 0.10                                        
\end{tblr}
\end{table}}


\aptLtoX{\begin{table}[!h]
\centering
\caption{Statistics on User Study Result.}
\label{tab:userstudyResult}
\begin{tabular}{llp{30pt}p{30pt}p{30pt}p{30pt}p{30pt}p{30pt}p{30pt}p{30pt}p{30pt}p{30pt}p{30pt}}
\toprule
Aspect        &      & Relevance and Motivation &       & System Effectiveness &       & Skill Improvement and Confidence &       &       &       & System Usability &       &       \\
\midrule
Question      &      & Q1                       & Q2    & Q3                   & Q4    & Q5                               & Q6    & Q7    & Q8    & Q9               & Q10   & Q11   \\
\hline
Baseline      & Mean & 3.94                     & 3.69  & 3.44                 & 3.69  & 3.88                             & 3.63  & 2.94  & 4.06  & 4.06             & 2.63  & 3.94  \\
              & SD   & 1.12                     & 0.95  & 1.31                 & 0.95  & 1.09                             & 1.09  & 1.24  & 0.85  & 1.00             & 1.31  & 1.31  \\
              \hline
TutorUp       & Mean & 4.63                     & 4.75  & 4.44                 & 4.69  & 4.63                             & 4.63  & 4.31  & 4.81  & 4.56             & 3.88  & 4.63  \\
              & SD   & 0.50                     & 0.45  & 0.73                 & 0.48  & 0.50                             & 0.81  & 0.79  & 0.40  & 0.81             & 1.15  & 1.15  \\
              \hline
Binomial Test &      & .035                    & .000 & .001                & .001 & .020                            & .003 & .000 & .004 & .089            & .002 & .010 \\
\bottomrule
\end{tabular}
\end{table}}{\begin{table}[!h]
\centering
\caption{Statistics on User Study Result.}
\label{tab:userstudyResult}
\begin{tblr}{
  width = \linewidth,
  colspec = {Q[67]Q[56]Q[92]Q[92]Q[79]Q[79]Q[65]Q[65]Q[65]Q[65]Q[56]Q[56]Q[54]},
  cells = {c},
  cell{1}{1} = {c=2}{0.123\linewidth},
  cell{1}{3} = {c=2}{0.184\linewidth},
  cell{1}{5} = {c=2}{0.158\linewidth},
  cell{1}{7} = {c=4}{0.26\linewidth},
  cell{1}{11} = {c=3}{0.166\linewidth},
  cell{2}{1} = {c=2}{0.123\linewidth},
  cell{3}{1} = {r=2}{},
  cell{5}{1} = {r=2}{},
  cell{7}{1} = {c=2}{0.123\linewidth},
  hline{1-3,5,7-8} = {-}{},
}
Aspect        &      & Relevance and Motivation &       & System Effectiveness &       & Skill Improvement and Confidence &       &       &       & System Usability &       &       \\
Question      &      & Q1                       & Q2    & Q3                   & Q4    & Q5                               & Q6    & Q7    & Q8    & Q9               & Q10   & Q11   \\
Baseline      & Mean & 3.94                     & 3.69  & 3.44                 & 3.69  & 3.88                             & 3.63  & 2.94  & 4.06  & 4.06             & 2.63  & 3.94  \\
              & SD   & 1.12                     & 0.95  & 1.31                 & 0.95  & 1.09                             & 1.09  & 1.24  & 0.85  & 1.00             & 1.31  & 1.31  \\
TutorUp       & Mean & 4.63                     & 4.75  & 4.44                 & 4.69  & 4.63                             & 4.63  & 4.31  & 4.81  & 4.56             & 3.88  & 4.63  \\
              & SD   & 0.50                     & 0.45  & 0.73                 & 0.48  & 0.50                             & 0.81  & 0.79  & 0.40  & 0.81             & 1.15  & 1.15  \\
Binomial Test &      & .035                    & .000 & .001                & .001 & .020                            & .003 & .000 & .004 & .089            & .002 & .010 
\end{tblr}
\end{table}}


\section{Survey Questions}
\label{apdx:sureveyquestions}
\subsection{First Survey Questions}
\begin{enumerate}
    \item What difficulties or challenging situations did you face when you started giving the tutor's?
    \item What difficulties and challenging situations do you currently face during the tutor’s?

\end{enumerate}

\subsection{Second Survey Questions}

\textbf{Open-ended Questions: }

\begin{enumerate}
    \item What types of student engagement problems have you encountered?
    \item Why do you think students are disengaged?
    \item Which strategies have you used to increase students' engagement and when did you use them? Please list all strategies and the situations when you use them:
\end{enumerate}

\textbf{Strategy-specific Questions: }

\begin{enumerate}
    \item Did you show empathy and respect towards your students to improve students' engagement (e.g., referring to students by their names or demonstrating concerns for their well being)? How and when?
    \item Did you promote social community and belonging among your students to improve students' engagement (e.g., let students introduce themselves or facilitate warm social interaction between students)? How and when?
    \item Did you discuss behavioral expectations with your students to improve students' engagement (e.g., don't interrupt each other when talking or start tutoring sessions on time)? How and when?
    \item Did you discuss academic goals with your students to improve students' engagement? (e.g., master certain subject specific skills or personal long term goals that students want to achieve (go to university))? How and when?
    \item Did you promote active learning in your tutoring sessions to improve students' engagement (e.g., let students work on problems or let them explain a concept to you)? How and when?
    \item Did you give prompt and positive feedback to students to improve students' engagement?(e.g., when they make progress towards solving a problem or when they show good behavior)? How and when?
    \item Did you encourage students to be persistent to improve students' engagement(e.g., don't allow students to give up or scaffold until they succeed)? How and when?
    \item Did you give autonomy to your students to improve students' engagement (e.g., let students participate when making decisions)? How and when?
    \item Did you set time constraints for activities inside your tutoring sessions to improve students' engagement (e.g., asking students to solve a practice problem in 5 minutes)? How and when?
    \item Did you present yourself to your students as a positive role model to improve students' engagement (show a growth mindset or start sessions on time)? How and when?
\end{enumerate}
    




\end{document}
\endinput
%%
%% End of file `sample-manuscript.tex'.
