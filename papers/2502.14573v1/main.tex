\documentclass{article} % For LaTeX2e
\usepackage{iclr2025_conference,times}

% Optional math commands from https://github.com/goodfeli/dlbook_notation.
%%%%% NEW MATH DEFINITIONS %%%%%

% \usepackage{amsmath,amsfonts,bm}
\usepackage{amsmath,amsfonts}

\usepackage{pifont}


\newcommand{\R}{\mathbb{R}}


\def\va{{\mathbf{a}}}
\def\vg{{\mathbf{g}}}

% Sets
\def\sR{\mathbb{R}}
\def\sC{\mathbb{C}}
\def\sZ{\mathbb{Z}}
\def\sN{\mathbb{N}}
\def\sQ{\mathbb{Q}}

\def\sS{\mathcal{S}}



% Vectors
\def\vzero{{\mathbf{0}}}
\def\vone{{\mathbf{1}}}
\def\vmu{{\mathbf{\mu}}}
\def\vtheta{{\mathbf{\theta}}}
\def\va{{\mathbf{a}}}
\def\vb{{\mathbf{b}}}
\def\vc{{\mathbf{c}}}
\def\vd{{\mathbf{d}}}
\def\ve{{\mathbf{e}}}
\def\vf{{\mathbf{f}}}
\def\vg{{\mathbf{g}}}
\def\vh{{\mathbf{h}}}
\def\vi{{\mathbf{i}}}
\def\vj{{\mathbf{j}}}
\def\vk{{\mathbf{k}}}
\def\vl{{\mathbf{l}}}
\def\vm{{\mathbf{m}}}
\def\vn{{\mathbf{n}}}
\def\vo{{\mathbf{o}}}
\def\vp{{\mathbf{p}}}
\def\vq{{\mathbf{q}}}
\def\vr{{\mathbf{r}}}
\def\vs{{\mathbf{s}}}
\def\vt{{\mathbf{t}}}
\def\vu{{\mathbf{u}}}
\def\vv{{\mathbf{v}}}
\def\vw{{\mathbf{w}}}
\def\vx{{\mathbf{x}}}
\def\vy{{\mathbf{y}}}
\def\vz{{\mathbf{z}}}
\def\vzeta{{\mathbf{\zeta}}}

% Matrix
\def\mA{{\mathbf{A}}}
\def\mB{{\mathbf{B}}}
\def\mC{{\mathbf{C}}}
\def\mD{{\mathbf{D}}}
\def\mE{{\mathbf{E}}}
\def\mF{{\mathbf{F}}}
\def\mG{{\mathbf{G}}}
\def\mH{{\mathbf{H}}}
\def\mI{{\mathbf{I}}}
\def\mJ{{\mathbf{J}}}
\def\mK{{\mathbf{K}}}
\def\mL{{\mathbf{L}}}
\def\mM{{\mathbf{M}}}
\def\mN{{\mathbf{N}}}
\def\mO{{\mathbf{O}}}
\def\mP{{\mathbf{P}}}
\def\mQ{{\mathbf{Q}}}
\def\mR{{\mathbf{R}}}
\def\mS{{\mathbf{S}}}
\def\mT{{\mathbf{T}}}
\def\mU{{\mathbf{U}}}
\def\mV{{\mathbf{V}}}
\def\mW{{\mathbf{W}}}
\def\mX{{\mathbf{X}}}
\def\mY{{\mathbf{Y}}}
\def\mZ{{\mathbf{Z}}}
\def\mBeta{{\mathbf{\beta}}}
\def\mPhi{{\mathbf{\Phi}}}
\def\mLambda{{\mathbf{\Lambda}}}
\def\mSigma{{\mathbf{\Sigma}}}


% Expectation
% \def\eE{\mathop{\mathbb{E}}\limits}
\def\eE{\mathbb{E}}

% Probability
\def\pP{\mathbb{P}}

% Tilde
\def\tf{\tilde{f}}
\def\tS{\tilde{S}}
\def\wtF{\widetilde{\mathcal{F}}}
\def\whR{\widehat{R}}
\def\tvx{\tilde{\mathbf{x}}}
\def\ty{\tilde{y}}


\def\defeq{\overset{\textup{def}}{=}}
% \def\defeq{\overset{.}{=}}
\def\defone{\overset{\text{\ding{172}}}{=}}
\def\deftwo{\overset{\text{\ding{173}}}{=}}
\def\leqone{\overset{\text{\ding{172}}}{\leq}}
\def\leqtwo{\overset{\text{\ding{173}}}{\leq}}
\def\leqthree{\overset{\text{\ding{174}}}{\leq}}
\def\leqfour{\overset{\text{\ding{175}}}{\leq}}
\def\eqone{\overset{\text{\ding{172}}}{=}}
\def\eqtwo{\overset{\text{\ding{173}}}{=}}
\def\eqthree{\overset{\text{\ding{174}}}{=}}
\def\eqfour{\overset{\text{\ding{175}}}{=}}
\def\geqfive{\overset{\text{\ding{176}}}{\geq}}

\usepackage{hyperref}
\usepackage{url}

% Custom Packages
\usepackage{kotex}
\usepackage{comment}
\usepackage{graphicx}
\usepackage{amsmath}
\usepackage{multirow}
\usepackage{enumitem}
\usepackage{amsthm}
\usepackage{booktabs}
\usepackage{subcaption}

% Custom function
\newcommand{\tabref}[1]{Table~\ref{#1}}
% \newcommand{\Tabref}[1]{Table~\ref{#1}}
% \newcommand{\equref}[1]{Eq.~\ref{#1}}
% \newcommand{\figref}[1]{Fig.~\ref{#1}}
% \newcommand{\secref}[1]{Sec.~\ref{#1}}
\newcommand{\inc}[1]{{\footnotesize \textcolor{blue}{($+$#1\%)}}}
\newcommand{\dec}[1]{{\footnotesize \textcolor{red}{($-$#1\%)}}}
\newcommand{\eqq}[1]{{\footnotesize ($+$#1\%)}}
\newcommand{\mathintab}[1]{\texorpdfstring{$#1$}{TEXT}}
% \algnewcommand\INPUT{\item[\textbf{Input:}]}%
% \algnewcommand\OUTPUT{\item[\textbf{Output:}]}%


\title{Self-supervised Monocular Depth Estimation Robust to Reflective Surface Leveraged by Triplet Mining}

% Authors must not appear in the submitted version. They should be hidden
% as long as the \iclrfinalcopy macro remains commented out below.
% Non-anonymous submissions will be rejected without review.

\author{Wonhyeok Choi$^{1,}$\thanks{These authors contributed equally.} , Kyumin Hwang$^{1,}$\footnotemark[1] , Wei Peng$^{2}$, Minwoo Choi$^{1}$, Sunghoon Im$^{1,}$\thanks{S. Im is the corresponding author.} \\
Electrical Engineering and Computer Science$^1$, Psychiatry and Behavioral Sciences$^2$\\
Daegu Gyeongbuk Institute of Science and Technology$^1$, Stanford University$^2$\\
South Korea$^1$, USA$^2$\\
{\ttfamily \small \{smu06117,kyumin,subminu,sunghoonim\}@dgist.ac.kr$^1$,
wepeng@stanford.edu$^2$}}

% The \author macro works with any number of authors. There are two commands
% used to separate the names and addresses of multiple authors: \And and \AND.
%
% Using \And between authors leaves it to \LaTeX{} to determine where to break
% the lines. Using \AND forces a linebreak at that point. So, if \LaTeX{}
% puts 3 of 4 authors names on the first line, and the last on the second
% line, try using \AND instead of \And before the third author name.

\newcommand{\fix}{\marginpar{FIX}}
\newcommand{\new}{\marginpar{NEW}}

\iclrfinalcopy % Uncomment for camera-ready version, but NOT for submission.
\begin{document}


\maketitle

\begin{abstract}
Self-supervised monocular depth estimation (SSMDE) aims to predict the dense depth map of a monocular image, by learning depth from RGB image sequences, eliminating the need for ground-truth depth labels.
Although this approach simplifies data acquisition compared to supervised methods, it struggles with reflective surfaces, as they violate the assumptions of Lambertian reflectance, leading to inaccurate training on such surfaces.
To tackle this problem, we propose a novel training strategy for an SSMDE by leveraging triplet mining to pinpoint reflective regions at the pixel level, guided by the camera geometry between different viewpoints.
The proposed reflection-aware triplet mining loss specifically penalizes the inappropriate photometric error minimization on the localized reflective regions while preserving depth accuracy in non-reflective areas.
We also incorporate a reflection-aware knowledge distillation method that enables a student model to selectively learn the pixel-level knowledge from reflective and non-reflective regions. This results in robust depth estimation across areas.
Evaluation results on multiple datasets demonstrate that our method effectively enhances depth quality on reflective surfaces and outperforms state-of-the-art SSMDE baselines.
% It also incorporates our reflective-aware knowledge distillation method that enables a student model to achieve improvements in both areas simultaneously.
\end{abstract}

%%%%%%%%%
\section{Introduction}\label{sec:intro}

In computational finance, Monte Carlo simulations are used extensively to estimate the expected value of financial payoffs based on the solution of stochastic differential equations (SDEs) which model the evolution of stock prices, interest rates, exchange rates and other quantities \cite{glasserman04}.  Monte Carlo methods are very general and flexible, but for high accuracy it requires generating a large number of costly SDE path approximations, which has motivated research into a number of variance reduction or, equivalently, cost reduction techniques. One such method is
Multilevel Monte Carlo (MLMC), which was proposed in \cite{GILES2008} and was adapted for various applications that are summarised in \cite{Giles_overview17} and successfully combined with other methods such as quasi-Monte Carlo methods. The main idea of MLMC is to approximate the payoff using different time stepping resolutions when numerically solving the underlying SDE and to generate an optimal number of samples on each level, such that the overall computational cost is minimised subject to the desired bound on the variance. %, such that the total computational cost is minimised. 
The computational savings come from the fact that most samples are computed on the coarser levels and hence are less expensive while only a few samples from the finest levels are required \cite{GILES2008}.


Among the directions in which the computational cost 
of MLMC methods could further be reduced, an important avenue is the use of lower precision calculations, especially for the first Monte Carlo levels where the targeted accuracy is relatively low. 
 An overview of the research on mixed precision for the standard Monte Carlo (MC) framework is provided in \cite{ChowMixedPrecisionStandardMC} but only a few references study the potential of low precision computation in the MLMC framework \cite{Rounding_error_oliver}. To the best of our knowledge, the only MLMC framework with customised precision in the literature is \cite{brugger2014mixed}, but they use a uniform precision for all operations on each Monte Carlo level instead of optimising 
 the precision of each intermediary variable to reduce as much as possible the cost of path generation.
 
An important motivation for an MLMC framework with variable precision would be performing the low precision computations on reconfigurable hardware devices such as Field Programmable Gate Arrays (FPGAs). FPGAs contain customizable logic blocks and connectors that make it easy to adapt the digital circuit architecture for a specific application, leading to a highly parallel and optimised implementation. Therefore they are successfully exploited in applications that require high speed and have high computational workload, such as signal processing \cite{woods2008fpga}, and real time applications like high frequency trading \cite{HFT1,HFT2}. That is why a number of previous works in hardware architecture design implemented the MLMC algorithm to price financial options using FPGAs as accelerators, which resulted in improved speed and power efficiency compared to full CPU architectures \cite{Schryver2013AMM}. The paper \cite{lindsey2016domain} also proposed 
a Domain Specific Language to automate the configuration of FPGAs for this specific application. However, only \cite{brugger2014mixed} proposed a heuristic to reduce the precision in calculations.

In addition, all aforementioned works considered that the random number generation (RNG) is performed in single or double precision. Yet in most cases an important portion of the workload in the overall MLMC simulation comes from the RNG and in \cite{brugger2014mixed} this limited the total computational savings.
To reduce the cost of MLMC simulations in particular those based on the Geometric Brownian Motion (GBM), \cite{approximateICDF_Oliver, NestedOliver} have proposed to use approximate random numbers that are generated by applying an approximation of the inverse CDF to uniform random numbers. In \cite{NestedOliver}, the authors proposed a way to integrate these lower precision random variables into a \textit{nested} MLMC framework and completed a numerical analysis to bound the resulting error at each MC level by a product of the time step and the error in the random number approximation. The same authors show in \cite{approximateICDF_Oliver} that using approximate random variables reduces the cost of path generation by a factor 7.


In this paper we propose a nested MLMC framework that combines the use of approximate random normal variables and lower precision calculations to reduce the computational cost of MLMC even further than \cite{brugger2014mixed,NestedOliver}. We illustrate the efficiency of our framework in Matlab, after making several assumptions on the cost of operations and size of the errors that we carefully justify. We focus on the case of GBM and use the approximate RNG methods presented in \cite{approximateICDF_Oliver} as well as a new slightly modified method that combines CDF inversion and the central limit theorem. To choose the precision of the variables in the low precision path generation, we introduce a novel method to optimise the bit-widths. This optimisation is performed before the main path generation loop is executed and is based on a linear model of the payoff error  
due to rounding when computing in low precision. The error model relies on algorithmic differentiation in a similar manner to \cite{unifying-bwoptim,bitwidth-AD,ADAPT}. The bit-width optimisation procedure can be performed off-line, so this stage can be excluded from the on-line time complexity of our framework. The user specified desired accuracy is then enforced by calculating on-line the number of samples that need to be generated.

In terms of hardware design, we suggest implementing the low precision path generation on FPGAs and the full-precision ones on a CPU or GPU. 
The FPGA offers enough flexibility to define a separate bit-width for every variable in the low precision path generation, and can be reconfigured periodically to update the bit-widths when the market parameters have changed considerably. 


The paper is organized as follows : \Cref{sec:MLMC} introduces MLMC and nested MLMC to make clear the estimator that is implemented in our framework. Then in \Cref{sec:RNG} we detail the methods that could be used to obtain approximate random normally distributed numbers very cheaply for the low precision path generation. In \Cref{sec:error_model} and \Cref{sec:costModel} we propose an error model and a cost model (resp.) that we then use to formulate the optimisation problem that is solved to obtain the optimal bit-widths of fixed point variables in \Cref{sec:optimisation}. Finally we summarise our results and future directions in \Cref{sec:conclusion}.




\section{Related Work}

\subsection{First-order logic for natural entailment}

Since the start of the RTE challenge \citep{rte}, multiple works have attempted using FOL representations to solve natural language entailment. These methods first obtain the syntactic/semantic parse tree and apply a rule-based transformation to get the FOL representation \citep{bos-markert-2005-recognising, bos-nli}. However, it was repeatedly shown that these FOL representations are not empirically effective in solving natural language entailment. For instance, \citet{bos-nli} reported that FOL representations translated from the discourse representation structure (DRS) yield only 1.9\% recall in detecting the entailment in the single-premise RTE benchmark \citep{rte}.

Independently from these works, multi-premise logical entailment benchmarks \citep{tafjord-etal-2021-proofwriter, logicnli, folio} were developed to evaluate the reasoning ability of generative models. These benchmarks adopt the classic 3-way entailment label classification format (\textit{entailment, contradiction, neutral}) of single-premise RTE tasks, in which both the NL sentences and their gold FOL representations point to the same entailment label. 

Recent works have applied LLMs to obtain FOL representations for these multi-premise logical entailment tasks \citep{logiclm, linc, divide-and-translate}, fueled by the code generation ability of LLMs. While they achieve significant performance in synthetic, controlled logical reasoning benchmarks, whether they can generalize to natural entailment has remained unanswered. Furthermore, \citet{linc} observed that LLMs are highly susceptible to \textit{arbitrariness}, as they fail to produce coherent predicate names or numbers of arguments even when generating FOL representations of premises and hypotheses in a single inference.

\subsection{Executable semantic representations}

Apart from FOL, a stream of research focuses on the \textit{executability} of semantic representations. From this perspective, semantic representations are \textit{program codes} that can be executed to solve downstream tasks, such as query intent analysis \citep{spider, dligach-etal-2022-exploring} and question answering \citep{semparse-qa}. The performance of the semantic parser is directly assessed by the accuracy of execution results for the downstream tasks, rather than the similarity between the prediction and the reference parse.

To improve the execution accuracy that is often non-differentiable, reinforcement learning (RL) and its variants have been applied to train neural semantic parsers \citep{cheng-etal-2019-learning, cheng-lapata-2018-weakly}. Using only the input sentence and the desired execution result, these methods learn to maximize the probability of the representations that lead to the correct execution result. However, these approaches are not directly applicable to EPF, as EPF requires taking account of \textit{interactions between premises and hypotheses} during execution (\textit{i.e.} theorem proving) while these methods assume that sentences are isolated.





\section{Methodology}
\paragraph{Preliminaries.}
We primarily focus on the homologous model merging, in which $\boldsymbol{\theta}_i$ all come from the same base model $\boldsymbol{\theta}_{\rm{base}}$. Given $K$ tasks $\{T_1,T_2,\cdots,T_K\}$ and $K$ corresponding fine-tuned models with parameters $\{\boldsymbol{\theta}_1,\boldsymbol{\theta}_2,\cdots,\boldsymbol{\theta}_K\}$, model merging aims to combine $K$ fine-tuned models into one single model simultaneously performing on $\{T_1,T_2,\cdots,T_K\}$ without post-training~\cite{method_p1_1,method_p1_2}.
Task vector~\cite{ilharco2023editing,yang2024adamerging} is a key element in merging method which could enhances the base model‘s ability or enable the model to handle other tasks. Specifically, for task $T_i$, the task vector $\boldsymbol\tau_i\in \mathbb{R}^D$ is defined as the vector obtained by subtracting the SFT weights $\boldsymbol{\theta}_i$ from the base model weight
$\boldsymbol{\theta}_{\rm{base}}$, \emph{i.e.}, $\boldsymbol\tau_i=\boldsymbol{\theta}_i-\boldsymbol{\theta}_{\rm{base}}$. The merged model could be denoted as $\boldsymbol{\theta}_m=\boldsymbol{\theta}_{\rm{base}}+\sum_i \lambda_i\boldsymbol{\tau}_i$, which $\lambda_i$ is the scaling factor measuring the importance of task vector. For clarification, we also denote the neuron set in $\boldsymbol{\theta}_i$ as $\mathcal{N}_i$, the neuron set in $\boldsymbol{\tau}_i$ as $\mathcal{T}_i$.



\begin{algorithm}[!ht]
    \caption{LED-Merging}
    \label{alg1}
    \begin{algorithmic}[1]
        \REQUIRE  base model $\boldsymbol{\theta}_{\rm{base}}$, SFT models $\{\boldsymbol{\theta}_{i}\mid i\in [K]\}$, mask ratios \{$r_{i} \mid i\in [K]\}$, scaling factors $\{\lambda_i\mid i\in[K]\}$, location datasets $\{\mathcal{X}_{i}\mid i\in[K]\}$
        \ENSURE merged parameter $\boldsymbol{\theta}_{m}$
        \STATE $\mathcal{M}\leftarrow\phi$
        \STATE $\boldsymbol{\theta}_{m}\leftarrow \boldsymbol{\theta}_{\rm{base}}$
        \FOR{$i\in [K]$}
        \STATE $I(\boldsymbol{\theta}_i)=\mathbb{E}_{x\sim \mathcal{X}_i}|\boldsymbol{\theta}_{i}\odot \nabla_{\boldsymbol{\theta}_i}\mathcal{L}(x)|$
        \STATE $I(\boldsymbol{\theta}_{\rm{base}})=\mathbb{E}_{x\sim \mathcal{X}_i}|\boldsymbol{\theta}_{\rm{base}}\odot \nabla_{\boldsymbol{\theta}_{\rm{base}}}\mathcal{L}(x)|$
        
        \STATE calculate $\mathcal{T}^{r_i}_{i}$ following Equation \ref{vote}
        \STATE  $\mathcal{M}\leftarrow \mathcal{M}\cup\{\mathcal{T}^{r_i}_i\}$
       
        
   
        
        
        \ENDFOR  
        \FOR{$i\in [K]$}
        
        \STATE calculate $\text{Disjoint}(\mathcal{T}_i^{r_i})$ use Equation~\ref{disjoint_safety}
        \STATE $\boldsymbol{m}_i \leftarrow \boldsymbol{0}$
        \FOR{$d\in \mathcal{T}_i^{r_i}$}
        \STATE $\boldsymbol{m}_{i,d}=1$
        \ENDFOR
        \STATE $\boldsymbol{\theta}_{m}\leftarrow \boldsymbol{\theta}_{m}+\lambda_i \boldsymbol{\tau}_i\odot \boldsymbol{m}_{i}$
        \ENDFOR
    \end{algorithmic}
\end{algorithm}
    %\vspace{-5pt}
\begin{figure*}[h!]
    \centering
    \includegraphics[width=\linewidth]{figs/pipeline_v2.pdf}
    \vspace{-40mm}
    \caption{Overview of our two-stage training pipeline {\ours}.}
    \label{fig:pipeline}
\end{figure*}


\paragraph{LED-Merging: Location, Election, and Disjoint Merging}
To address the neuron misidentification and interference issues in existing model merging methods, we propose LED-Merging (Location, Election, and Disjoint Merging). Specifically, previous studies \cite{modelstock, ilharco2023editing, tiesmerging} fail to accurately identify safety-related neurons in task vectors with a single magnitude score, namely \textit{neuron misidentification}. Meanwhile, there exists an interference between safety-related and utility-related task vector neurons during the merging process, namely \textit{neuron interference}. To address neuron misidentification, we first locate important neurons both in the base and fine-tuned models and then elect neurons from the task vector considering these two scores together. Subsequently, to mitigate the interference, we introduce a disjoint step, isolating these important neurons so that they influence different base neurons. The whole process is illustrated in Figure~\ref{fig:method}. 




In the location and election step, we consider the importance score from base and fine-tuned models simultaneously to locate task-specific neurons. In this way, it is more accurate than relying on the magnitude score alone because task-specific neurons with high importance score in the fine-tuned model may not necessarily score high in the base model, and vice versa.

{\textbf{Location}}.  We first calculate importance scores for each neuron in a base/fine-tuned model. Given a location dataset $\mathcal{X}_i=\{(x,y)_k\}$, where $x$ is the question and $y$ is the answer, we calculate the importance scores for the weight $\boldsymbol{\theta}_i\in\mathbb{R}^D$ in any  layer as follows~\cite{snip,spareseGPT,sun2024a}:
\begin{equation}
    I(\boldsymbol{\theta}_i)=\mathbb{E}_{x\sim \mathcal{X}_i}[\boldsymbol{\theta}_i\odot \nabla _{\boldsymbol{\theta}_i}\mathcal{L}(x)],
    \label{location}
\end{equation}
which $\mathcal{L}(x)=-\log p(y\mid x)$ is the conditional negative log-likelihood loss. We choose the SNIP score~\cite{snip} because it balances computational efficiency and performance~\cite{cq}. Please refer to Sec.~\ref{sec:ablation} for the comparison between different location methods. After computing importance scores, we choose top-$r_i$ neurons as the important neuron subset $\mathcal{N}_{i}^{r_i}$ from $I(\boldsymbol{\theta}_i)$.
 
 % After computing locating scores, we select the neurons scoring both high in base and fine-tuned models as important neurons in task vectors. Then in the disjoint step,  with preventing  polysemantic neurons  from receiving gradient updates towards different directions,
 % we use set difference to isolate the safety   and utility-related neurons  and construct corresponding masks for merging process,

{\textbf{Election}}. A natural question is how to select important neurons in the task vector $\boldsymbol{\tau}_i$ based on $I(\boldsymbol{\theta}_{\rm{base}})$ and $I(\boldsymbol{\theta}_{i})$. The important neurons in the base model may be different from neurons in the fine-tuned model. Therefore, we introduce the following election strategy to select neurons with high scores in both base and fine-tuned models:
\begin{equation}
    \mathcal{T}_i^{r_i}=\mathcal{N}_i^{r_i}\cap \mathcal{N}_{\rm{base}}^{r_i}.
    \label{vote}
\end{equation}
\emph{Remark}. We compare different choosing methods, including scoring low or high in base or fine-tuned model in Section~\ref{sec:ablation} and find that Equation \ref{vote} achieves the best performance.





{\textbf{Disjoint}}. As important neurons from different task vectors may conflict with each other at the same position, we use the set difference to disjoint the neurons from others to prevent interference:
\begin{equation}
    \text{Disjoint}(\mathcal{T}^{r_i}_{i})=\mathcal{T}^{r_i}_{i}-\mathop{\cup}\limits_{{J}\subsetneqq [K],|J|\geq 2}\mathop{\cap}\limits_{j\in {J}}\mathcal{T}^{r_j}_{j}.
    \label{disjoint_safety}
\end{equation}

Next, we construct a mask $\boldsymbol{m}_i\in\mathbb{R}^D$ to implement disjoint in the merging process. Specifically, this mask $\boldsymbol{m}_i$ is used to select neurons from $\mathcal{T}_i$. The mask ratio is $r_i$, where $r\in(0,1]$. The mask $\boldsymbol{m}_i$ can be derived from:
\begin{equation}
    \boldsymbol{m}_{i,d}=\begin{aligned} &\left\{ \begin{array}{ll} 1, & \text{if } d\in \text{Disjoint}(\mathcal{T}_{i}^{r_i}), \\ 0, & \text{otherwise}. \end{array} \right. \end{aligned}
    \label{mask_safety}
\end{equation}


% \subsection{Merging Models with Masks}
{\textbf{Merging}}. The final
merged task vector $\boldsymbol{\tau}_m$ is as follows:
\begin{equation}
    \boldsymbol{\tau}_m= \sum_i \lambda_i\boldsymbol{\tau}_{i}\odot\boldsymbol{m}_i.
    \label{merged_task_vector}
\end{equation}
We summarize the workflow in Algorithm \ref{alg1}.




\section{Experiments}
\label{sec:experiments}

\begin{figure*}[t]
\vspace{-6mm}
    \centering
    \includegraphics[width=0.8\linewidth]{figs/compare.pdf}
    \vspace{-4mm}
    \caption{\textbf{Qualitative comparison} with the baseline for generating a sequence of novel view images.  
    The results demonstrate that our method synthesizes more consistent multi-view images compared to our baseline model (Zero123). In addition, compared to SyncDreamer, our method visually maintains better similarity to the conditioned image and appears more natural.}
    \label{fig:sota_compare}
\vspace{-5mm}
\end{figure*}

\subsection{Experimental Setups}
\textbf{Dataset.}
Following previous work~\cite{zero123, SyncDreamer}, we evaluate our work on the Google Scanned Object (GSO)~\cite{GSO} dataset to verify the zero-shot novel view image synthesis capability. 
We also provide results for additional datasets in the Supplementary Material.
Specifically, we randomly select 30 objects from the GSO dataset with various object categories. 
Unlike recent approaches~\cite{mvdream, SyncDreamer} that aim to enhance the consistency of novel view synthesis models by generating multiple fixed-view images, our method can generate images from any camera pose and any number of views. Therefore, we conduct experiments under different camera pose settings to validate our approach:
specifically, 
1) \textit{16-views with free camera pose}: for each object, we circularly render 16 views with the elevation angles ranging in $[-10\degree, 40\degree]$ and the azimuth angles are evenly distributed in $[0\degree, 360\degree]$. 
2) \textit{16-views with fixed camera pose}: We maintain a constant elevation angle of $30\degree$ and uniformly sample azimuth angles (same as SyncDreamer~\cite{SyncDreamer}).
3) \textit{32-views with free camera pose}: Similar to the first setting, but we sample 32 views.
It's important to note that our method does not require additional training or fine-tuning on any datasets.

\noindent\textbf{Metrics.}
To validate the effectiveness of our method, we mainly evaluate it based on three criteria:
1) \textit{Quality Score}. We evaluate the image quality of synthesized multi-view images by measuring their similarity with ground truth images. Following prior research~\cite{zero123, sparsefusion}, we report the similarity between the synthesized images and the ground truth images with standard metrics: PSNR, SSIM~\cite{ssim}, and LPIPS~\cite{lpips}.
2) \textit{Multi-view Consistency Score}. As the primary goal of our work is to improve the consistency of generated images, we also employ the 3D consistency score~\cite{3dim} to verify the consistency among the synthesized images. Specifically, we train an Instant-NGP~\cite{instant_ngp} with the input image and part of the synthesized novel view images of our model and evaluate the similarity between the remaining synthesized images and the rendered images of Instant-NGP. For the synthesized multi-view images of each object, we allocate $3/4$ for training and reserve the remaining $1/4$ for validation.
Intuitively, if the consistency of synthesized images is improved, the NeRF-like model will train a better object representation, and the re-rendered images will agree more with the validation images.
3) \textit{Input Consistency Score}. To assess the faithfulness of synthesized images in preserving the identity of the input condition image, we introduce the input consistency score. This score calculates the similarity of each synthesized image with the input condition image, utilizing the LPIPS metric.

In addition, we use synthesized multi-view images to train a neural 3D reconstruction model (NeuS~\cite{neus}) and report commonly used Chamfer Distances (CD) and Volume IoUs between the trained 3D model and the ground truth.

\noindent\textbf{Baselines.}
Given that our main goal is to improve the consistency of the trained baseline model without further fine-tuning, we mainly compare our approach with the used baseline model Zero123~\cite{zero123}. Additionally, we compare our method to the SOTA approaches such as PGD~\cite{tseng2023consistent} and SyncDreamer~\cite{SyncDreamer} using the same Zero123 base model.

\noindent\textbf{Implementation Details.}
We use the official checkpoint provided by Zero123~\cite{zero123}, which is trained on objaverse~\cite{objaverse} for 165,000 steps. We inject our epipolar attention layer after step $T=4$ and layer $L=10$ by default. We find that feature fusion weight $\alpha=0.5$, and the number of context views $M=2$ work better.

\begin{table}[t]
\centering
\caption{Comparison of multi-view consistency, image quality, and input consistency of synthesized multi-view images at the 16-view setting with free camera pose.}
\label{tab:view16_free_compare}
\vspace{-2mm}
\scalebox{0.6}{
\begin{tabular}{c ccc ccc c}
\toprule
              & \multicolumn{3}{c}{Multi-view Consistency} & \multicolumn{3}{c}{Quality Score} & \multicolumn{1}{c}{Input Consis.} \\
              \cmidrule(lr){2-4} \cmidrule(lr){5-7} \cmidrule(lr){8-8}
              & PSNR$\uparrow$  & SSIM$\uparrow$ & LPIPS$\downarrow$ 
              & PSNR$\uparrow$  & SSIM$\uparrow$ & LPIPS$\downarrow$ 
              & LPIPS$\downarrow$ 
              \\ \midrule

Zero123
& 15.225        & 0.645       & 0.408
& 14.255        & 0.747       &	0.208
& 0.303         
\\
SyncDreamer
& 14.830        & 0.626       & 0.434
& 12.650        & 0.713       &	0.254
& 0.317         
\\
Ours 
& \best{18.300}	& \best{0.734}	& \best{0.355}
& \best{14.947}	& \best{0.763}	& \best{0.191}
& \best{0.282}
\\

\bottomrule
\end{tabular}
}
\end{table}

\begin{table}[t]
\vspace{-1mm}
\centering
\caption{Comparison of multi-view consistency, image quality, and input consistency at the 16-view setting with fixed camera pose as SyncDreamer~\cite{SyncDreamer}.}
\label{tab:view16_fxied_compare}
\vspace{-3mm}
\scalebox{0.6}{
\begin{tabular}{c ccc ccc c}
\toprule
              & \multicolumn{3}{c}{Multi-view Consistency} & \multicolumn{3}{c}{Quality Score} & \multicolumn{1}{c}{Input Consis.} \\
              \cmidrule(lr){2-4} \cmidrule(lr){5-7} \cmidrule(lr){8-8}
              & PSNR$\uparrow$  & SSIM$\uparrow$ & LPIPS$\downarrow$ 
              & PSNR$\uparrow$  & SSIM$\uparrow$ & LPIPS$\downarrow$ 
              & LPIPS$\downarrow$ 
              \\ \midrule

Zero123
& 16.556        & 0.682       & 0.378
& 14.592        & 0.750       &	0.207
& 0.305         
\\
SyncDreamer
& \best{22.424}        & \best{0.812}       & \best{0.268}
& 15.269        & 0.749       &	0.196
& 0.300         
\\
Ours 
& 21.151	& 0.780	& 0.302
& \best{15.293}	& \best{0.764}	& \best{0.184}
& \best{0.287}
\\

\bottomrule
\end{tabular}
}
\vspace{-4mm}
\end{table}


\subsection{Comparison With Baseline Models}
The quantitative comparison on three settings are shown in Tab.~\ref{tab:view16_free_compare}, Tab.~\ref{tab:view16_fxied_compare}, and Tab.~\ref{tab:view32_free_compare}. The qualitative comparison is shown in Fig.~\ref{fig:sota_compare}.

\begin{table}[t]
\centering
\caption{Comparison of multi-view consistency and image quality scores of synthesized multi-view images at the 32-view setting with free camera pose.}
\vspace{-3mm}
\label{tab:view32_free_compare}
\scalebox{0.7}{
\begin{tabular}{c ccc ccc}
\toprule
              & \multicolumn{3}{c}{Multi-view Consistency} & \multicolumn{3}{c}{Quality Score} \\
              \cmidrule(lr){2-4} \cmidrule(lr){5-7}
              & PSNR$\uparrow$  & SSIM$\uparrow$ & LPIPS$\downarrow$ 
              & PSNR$\uparrow$  & SSIM$\uparrow$ & LPIPS$\downarrow$ 
              \\ \midrule

Zero123
& 16.515        & 0.694       & 0.378
& 15.142        & 0.733       &	0.211
\\
PGD~\cite{tseng2023consistent}
& 18.481        & 0.720       & 0.343
& 15.281        & 0.739       &	0.205
\\
Ours 
& \best{20.655}	& \best{0.792}	& \best{0.305}
& \best{15.268}	& \best{0.742}	& \best{0.203}
\\

\bottomrule
\end{tabular}
}
\vspace{-3mm}
\end{table}

\begin{table*}
  [t]
  \centering
  \resizebox{\textwidth}{!}{%
  \begin{tabular}{cccccccccccc}
    \toprule \multicolumn{2}{c}{Components}                                                             & \multicolumn{5}{c}{Re-executability Rate (\%)} & \multicolumn{5}{c}{Readability (\#)} \\
    \cmidrule(lr){1-2} \cmidrule(lr){3-7} \cmidrule(lr){8-12}        \hspace{8pt}\labelemoji\hspace{8pt}                                                                & \hspace{8pt}\toolemoji\hspace{8pt}                                      & O0                                 & O1             & O2             & O3             & AVG            & O0             & O1             & O2             & O3             & AVG            \\
    \hline
    \rowcolor[rgb]{0.93,0.93,0.93}\multicolumn{12}{c}{\textbf{Initialize with LLM4Decompile-End-6.7B~\citep{llm4decompile}}}   \\
    \xmark                                                                                              & \xmark                                    & 69.51                              & 46.95          & 50.61          & 46.34          & 53.35          & 3.98 & 3.41 & 3.44 & 3.38 & 3.55 \\
    \cmark                                                                                              & \xmark                                    & 75.61                              & 50.61          & 50.00          & 50.00          & 56.55          & 4.01 & 3.44 & 3.39 & \textbf{3.49} & 3.58 \\
    \xmark                                                                                              & \cmark                                    & 83.54                     & \textbf{56.10}          & 51.22          & 50.61 & 60.37 & 4.05 & 3.51 & 3.51 & 3.42 & 3.62 \\
    \cmark                                                                                              & \cmark                                    & \textbf{85.37}                            & \textbf{56.10}                     & \textbf{51.83} & \textbf{52.43}          & \textbf{61.43} & \textbf{4.13} & \textbf{3.60} & \textbf{3.54} & \textbf{3.49} & \textbf{3.69} \\

    \rowcolor[rgb]{0.93,0.93,0.93}\multicolumn{12}{c}{\textbf{Initialize with Deepseek-Coder-6.7B-base~\citep{deepseekcoder}}} \\
    \xmark                                                                                              & \xmark                                    & 59.15                              & 35.98          & 39.02          & 37.80          & 42.99          & 3.71 & 3.05 & 3.16 & 3.05 & 3.24 \\
    \cmark                                                                                              & \xmark                                    & 66.46                              & 41.46          & 38.41          & 36.59          & 45.73          & 3.76 & 3.17 & \textbf{3.21} & 3.08 & 3.31 \\
    \xmark                                                                                              & \cmark                                    & 70.73                              & 39.63          & 39.02          & 40.24          & 47.41          & 3.90 & 3.17 & 3.08 & 3.11 & 3.31 \\
    \cmark                                                                                              & \cmark                                    & \textbf{79.88}                     & \textbf{45.73} & \textbf{43.90} & \textbf{42.68} & \textbf{53.05} & \textbf{3.96} & \textbf{3.21} & 3.18 & \textbf{3.19} & \textbf{3.38} \\
    \bottomrule
  \end{tabular}%
  }
  \caption{The ablation study of different methods across four optimization levels
  (O0, O1, O2, O3), as well as their average scores (AVG). The results in bold represent the optimal performance. The ~\labelemoji~ and ~\toolemoji~ means Relabedling and Function Call. \textbf{Bold} denotes the best performance.}
  \label{tab:ablation}
\end{table*}



\begin{figure*}[ht]
    \centering
    \begin{minipage}{0.65\textwidth}
        \centering
        \includegraphics[width=0.95\linewidth]{figs/ablation.pdf}
        \vspace{-2mm}
        \captionof{figure}{Qualitative Comparison for different design choices. Our method, employing multi-view epipolar attention, demonstrates the best consistency.}
        \label{fig:ablation}
    \end{minipage}\hfill
    \begin{minipage}{0.33\textwidth}
        \centering
        \includegraphics[width=0.8\linewidth]{figs/neus_ver.pdf}
        \vspace{-3mm}
        \caption{Our method shows better direct 3D reconstruction~\cite{neus}.}
        \label{fig:neus}
    \end{minipage}
    \vspace{-5mm}
\end{figure*}

\noindent\textbf{Multi-view Consistency.}
Tab.~\ref{tab:view16_fxied_compare} presents the 3D consistency scores compared to our baseline model (Zero123) and SyncDreamer. The results indicate a significant improvement across all three metrics achieved by our method when compared with Zero123.
While our method exhibits a marginally lower numerical consistency score compared to SyncDreamer, it enables the synthesis of images with arbitrary camera poses.	
This capability is illustrated in Tab.~\ref{tab:view16_free_compare}, where our method consistently enhances consistency with changes in camera pose settings, whereas SyncDreamer fails to do so and exhibits inferior results compared to Zero123.
Furthermore, our method facilitates the synthesis of multi-view images with any number of camera views. This versatility is demonstrated in Tab.~\ref{tab:view32_free_compare}, where our method continues to achieve significant improvements in consistency scores, while SyncDreamer is unable to operate under such conditions.	

Meanwhile, Fig.~\ref{fig:sota_compare} provides a qualitative comparison with the baseline. While both our method and SyncDreamer enhance consistency, our method visually preserves better similarity to the input image, including color and texture details. The input consistency score further corroborates this.

\noindent\textbf{Image Quality.}
While our primary goal centers around enhancing the consistency of synthesized multi-view images, we also evaluate the image quality by comparing the similarity with the ground truth images. The results shown in Tab.~\ref{tab:view16_free_compare}, Tab.~\ref{tab:view16_fxied_compare}, and Tab.~\ref{tab:view32_free_compare} indicate that our method also enhances the image quality under different settings besides improving the consistency.
Moreover, our method shows better image quality compared with SyncDreamer even in the 16-view setting with fixed camera pose.

\noindent\textbf{Input Consistency.}
Input consistency terms whether the results align with the input image.
Fig.~\ref{fig:sota_compare} illustrates that both our method and SyncDreamer enhance multi-view consistency. However, the color and texture details of SyncDreamer's results diverge from the input image and appear visually unnatural.
This discrepancy is evident in the input consistency score presented in Tab.~\ref{tab:view16_fxied_compare}, indicating lower similarity with the condition image in the SyncDreamer results.	

\subsection{Ablation Study}
The overall quantitative results are shown in Tab.~\ref{tab:ablation}, and the qualitative comparisons are shown in Fig.~\ref{fig:ablation}.

\noindent \textbf{Full Attention \vs Epipolar Attention.}
The results presented in Tab.\ref{tab:ablation} and Fig.\ref{fig:ablation} demonstrate that our epipolar attention mechanism can synthesize more consistent multi-view images compared with full attention. Furthermore, our epipolar attention achieves a greater performance improvement compared to full attention when using multiple reference images. This could be attributed to the fact that our epipolar attention more effectively localizes target information, as depicted in Fig.~\ref{fig:full_attn_compare}, thereby reducing noise from the reference images. In the multi-view setting, where multiple reference images are utilized, this noise reduction becomes particularly crucial.
Moreover, it is noteworthy that the epipolar attention mechanism consumes less GPU memory compared to our baseline, as discussed in Sec.~\ref{sec:attn_analysis}.

\noindent \textbf{Attending Single-View \vs Multi-View.}
Applying the epipolar attention significantly improves the consistency between the input and target views. However, the consistency between different views in the unobserved regions of the input view is not well preserved.
After implementing our epipolar attention in the multi-view setting, the consistency across the generated multi-view images is further improved. The last row in Tab.~\ref{tab:ablation} shows that after applying our multi-view epipolar attention, the consistency score is further improved compared with the single-view setting. Besides, the qualitative result in Fig.~\ref{fig:ablation} also shows better consistency among different target views.



\begin{table}[t]
\centering
\vspace{-1mm}
\caption{Comparison of 3D reconstruction results. Our method significantly improves the reconstruction quality.}
\vspace{-3mm}
\label{tab:neus}
\scalebox{0.7}{
\begin{tabular}{c cc}
\toprule
              &  Chamfer Dist.$\downarrow$  & Volume IoU$\uparrow$
\\ \midrule

            Zero123         & 0.017         & 0.819    \\
            SyncDreamer     & \best{0.013}         & \best{0.847}    \\
            Ours            & 0.014	& 0.842 \\

\bottomrule
\end{tabular}
}
\vspace{-5mm}
\end{table}


\vspace{-2mm}
\subsection{Downstream Application}
\vspace{-2mm}
To demonstrate the effectiveness of our method, we also applied it to the downstream 3D reconstruction task. Specifically, we trained the NeuS model~\cite{neus} directly using images synthesized by our method, Zero123, and SyncDreamer, respectively.
The quantitative results in Tab.~\ref{tab:neus} show that the consistent multi-view images synthesized by our method can significantly improve the 3D reconstruction quality.
Additionally, our method exhibits similar performance to SyncDreamer which requires time-consuming re-training.
The qualitative results in Fig.~\ref{fig:neus} show that it is challenging to train the NeuS model directly due to the lack of consistency in the images generated by Zero123. In contrast, our method generates more consistent multi-view images and, therefore, better reconstructs the geometry and texture details.
We show improvements on other downstream applications such as image-to-3D in the Supplementary Material.



%\section{Limitation}
\section{Conclusion}
This paper addresses the intricate challenge of self-supervised monocular depth estimation on reflective surfaces.
Our method employs a novel metric learning approach, centered around a reflection-aware triplet mining loss. 
This novel loss function significantly improves depth prediction accuracy by accurately identifying reflective regions on a per-pixel basis and effectively adjusting the minimization of photometric errors, which are typically problematic on reflective surfaces. 
It also preserves high-frequency details on non-reflective surfaces by selectively regulating photometric error minimization based on reflection region selection.
Moreover, we introduce a reflection-aware knowledge distillation method, enabling a student model to enhance performance in both reflective and non-reflective surfaces.
This method leverages the strengths of different teaching networks to produce a more robust and versatile student model.
Experimental evaluations conducted on the indoor scene datasets demonstrate our method consistently enhances depth performance across various architectural frameworks.
These results underscore the robustness and versatility of our approach, marking it as a valuable contribution to the field of self-supervised monocular depth estimation.
% This is the pioneering work of an end-to-end SSMDE training strategy that can significantly enhance the depth performances of reflective surfaces while preserving the high-frequency details on non-reflective surfaces.

% \clearpage

% \begin{acknowledgement}
\section*{Acknowledgement}
This work was supported by Korea Research Institute for defense Technology planning and advancement through Defense Innovation Vanguard Enterprise Project, funded by Defense Acquisition Program Administration (R230206) and the National Research Foundation of Korea (NRF) grant funded by the Korea government (MSIT) (No. RS-2023-00210908).
% \end{acknowledgement}
 
\bibliography{egbib.bib}
\bibliographystyle{iclr2025_conference}
% \bibliographystyle{plainnat}

\clearpage
\setcounter{page}{1}
\maketitlesupplementary


\renewcommand{\thetable}{S\arabic{table}}
\renewcommand{\thefigure}{S\arabic{figure}}

\clearpage
\appendix

\onecolumn 

\section{Appendix}
\tableofcontents 
\clearpage


\begin{table*}[t]
\centering
\small
\renewcommand{\arraystretch}{1.2}
\setlength{\tabcolsep}{6pt}
\begin{tabular*}{\textwidth}{@{\extracolsep{\fill}} 
  >{\centering\arraybackslash}m{3cm}  % Sampling Mode
  >{\centering\arraybackslash}m{1.5cm} % Stride
  >{\centering\arraybackslash}m{3cm}   % Extrapolation Factor
  >{\centering\arraybackslash}m{2cm}   % Total NFE
  >{\centering\arraybackslash}m{3cm}   % Sampling Time [s]
  >{\centering\arraybackslash}m{1cm}   % FVD subcolumn 1
  >{\centering\arraybackslash}m{1cm}}  % FVD subcolumn 2
\toprule
\makecell{\textbf{Sampling}\\\textbf{Mode}} & 
\makecell{\textbf{Stride}} & 
\makecell{\textbf{Extrapolation}\\\textbf{Factor}} & 
\makecell{\textbf{Total}\\\textbf{NFE}} & 
\makecell{\textbf{Sampling}\\\textbf{Time [s]}} & 
\multicolumn{2}{c}{\textbf{FVD$\downarrow$}} \\
\cmidrule(rr){6-7}
 & & & & & \textbf{DMLab} & \textbf{FFS} \\
 \midrule
\rowcolor{gray!8}Diffusion Forcing~\cite{chen2024diffusionforcing} & $s=k-m$ & $1\times$ & $286 / 266$ & $45.32$ / $52.26$ & $60.30$ & $51.90$ \\
Rolling Diffusion~\cite{ruhe2024rollingdiffusionmodels} & $s=k-m$ & $1\times$ & $500$ / $500$ & $79.24$ / $98.23$ & \textbf{52.43} & \textbf{45.51} \\
\rowcolor{gray!8}\textit{MaskFlow} (MGM-Style) & $s=k-m$ & $1\times$ & \textbf{20} / \textbf{20} & \textbf{3.17 / 3.93} & $53.17$ & $45.92$ \\
\midrule
Diffusion Forcing~\cite{chen2024diffusionforcing} & $s=k-m$ & $2\times$ & $858$ / $798$ & $135.97$ / $156.78$ & $175.01$ & $144.43$ \\
\rowcolor{gray!8}Rolling Diffusion~\cite{ruhe2024rollingdiffusionmodels} & $s=k-m$ & $2\times$ & $896$ / $788$ & $141.99 / 154.81$ & 201.70 & 72.49 \\
\textit{MaskFlow} (MGM-Style) & $s=k-m$ & $2\times$ & \textbf{60} / \textbf{60} & \textbf{9.51} / \textbf{9.30} & $188.02$ &  $59.93$ \\
\rowcolor{gray!8}\textit{MaskFlow} (MGM-Style) & $s=1$ & $2\times$ & $740$ / $340$ & 117.27 / 66.80 & \textbf{50.87} & \textbf{30.43} \\
\midrule
Diffusion Forcing~\cite{chen2024diffusionforcing} & $s=k-m$ & $5\times$ & $2{,}002$ / $1{,}596$ & $317.27$ / $313.56$ & $232.89$ & $272.14$ \\
\rowcolor{gray!8}Rolling Diffusion~\cite{ruhe2024rollingdiffusionmodels} & $s=k-m$ & $5\times$ & $2{,}084$ / $1{,}652$ & $330.27$ / $324.56$ & $338.34$ & $248.13$ \\
\textit{MaskFlow} (MGM-Style) & $s=k-m$ & $5\times$ & \textbf{140} / \textbf{120} & \textbf{22.19 / 23.58} & $334.15$ & $108.74$ \\
\rowcolor{gray!8}\textit{MaskFlow} (MGM-Style) & $s=1$ & $5\times$ & $2{,}900$ / $1{,}300$ & 100.09/379.91 & \textbf{181.11} & \textbf{103.69} \\
\bottomrule
\end{tabular*}
\caption{\textbf{MGM Style sampling is much faster without sacrificing quality.} We report the total number of function evaluations (NFE), sampling time (in seconds), and FVD for various sampling methods and extrapolation factors across both datasets.}
\label{tab:speed_comparison}
\end{table*}



\subsection{Additional Related Work}

\paragraph{Masked Diffusion Models.} 
Limitations of autoregressive models for probabilistic language modeling have recently sparked increasing interest in masked diffusion models. Recent works like \cite{shi2024simplifiedgeneralizedmaskeddiffusion} and \cite{sahoo2024simpleeffectivemaskeddiffusion} have aligned masked generative models with the design space of diffusion models by formulating continuous-time forward and sampling processes. Works like \cite{nie2024scalingmaskeddiffusionmodels} and \cite{gong2024scalingdiffusionlanguagemodels} also demonstrate the significant scaling potential of MDM for language tasks, indicating that this masked modeling paradigm can rival autoregressive approaches for modalities beyond language such as protein co-design \cite{campbell2024generative} and vision.

\subsection{Computation of NFE for Different Sampling Methods}

Our sampling speed evaluations are determined by computing the required number of chunks 
\[
\ell = \left\lceil \frac{L - k}{s} \right\rceil + 1,
\]
to generate a video of total length \(L\), where \(k\) is the chunk size and \(s\) is the stride with which the chunk start is shifted. The overall number of function evaluations (NFEs) is then obtained by multiplying \(\ell\) with the number of sampling steps required to generate one chunk. We apply this methodology for all chunkwise-autoregressive approaches.

\begin{itemize}
    \item \textbf{MGM-Style Sampling:} In this method each chunk is generated in $20$ forward passes, so that the total NFE is
    \[
    \text{NFE}_{\mathrm{MGM}} = \ell \times 20.
    \]

    \item \textbf{FM-Style Sampling:} Here we generate each chunk in $250$ forward passes:
    \[
    \text{NFE}_{\mathrm{FM}} = \ell \times 250.
    \]
    
    \item \textbf{Diffusion Forcing with Pyramid Scheduling:} Here, we apply $250$ sampling timesteps per frame but begin unmasking earlier frames as the denoising process proceeds. For a chunk of \(k\) frames, we generate a scheduling matrix with 
    \[
    H = 250 + (k-1) + 1 = k + 250
    \]
    rows and \(k\) columns. Each entry in the scheduling matrix is computed as
    \[
    \text{scheduling\_matrix}[i,j] = 250 + j - i,\quad \text{for } i=0,\ldots,H-1 \text{ and } j=0,\ldots,k-1,
    \]
    and then clipped to the interval \([0,249]\). Since we iterate through each of the $H$ rows of the denoising matrix in each chunk we effectively compute
    \[
    \text{NFE}_{\text{DiffusionForcing}} = k + 250.
    \] 
    
    \item \textbf{RDM Sampling:} This approach proceeds in three stages:
    \begin{enumerate}
        \item \textit{Initialization (Init-Schedule):} The initial window of \(k\) frames is processed using a fixed schedule that applies $T=250$ forward passes to bring the window to its rolling state.
        
        \item \textit{Sliding Window Handling:} After initialization, the window is shifted by one frame at a time. For each shift, an inner loop is executed that updates the denoising levels until the first non-context frame (i.e., the frame immediately following the \(m\) context frames) is fully denoised (i.e., reaches a value of 1). This inner loop requires $\left\lceil \frac{T}{k-m} \right\rceil$ forward passes per window shift. As the window is shifted \((L - k)\) times, this stage contributes roughly \((L - k) \times \left\lceil \frac{T}{k-m} \right\rceil\) forward passes.
        
        \item \textit{Final Window Processing:} Once the sliding window stage is complete, the final (partial) window is further refined until all frames are fully denoised. This final stage requires additional $250$ forward passes.
    \end{enumerate}
    
    Thus, the total NFE for RDM is given by
    \[
    \text{NFE}_{\mathrm{Rolling}} = 250 \; (\text{init-schedule}) + (L - k) \times \left\lceil \frac{T}{k-m} \right\rceil\ \; (\text{sliding}) + 250 \; (\text{final window}).
    \]
\end{itemize}




\subsection{Training \& Implementation Details}

All FFS models were trained on 4 H100 GPUs with a local batch size of $4$. We run training for a total of $200{,}000$ steps and use a sigmoid scheduler that determines the per-frame masking ratio for a sampled masking level $t^k$. We use an AdamW optimizer with a learning rate of $1e-4$ and $\beta_1 = 0.9$ and $\beta_2 = 0.999$. We additionally incorporate a frame-level loss weighting mechanism based that is also based on \(t^k\). We adopt \emph{fused}-SNR loss weighting from \cite{hang2023efficient,chen2024diffusionforcing} and derive it for discrete flow matching. Let

\[
\text{SNR}(t) \;=\; \frac{\kappa(t)^2}{\,1 - \kappa(t)^2\,},
\]

where \(\kappa(t)\) is the masking schedule. The \emph{fused}-SNR mechanism smoothes SNR values across time steps in a video by computing an exponentially decaying SNR from previous frames (or tokens). We refer the reader to~\cite{chen2024diffusionforcing} for full details.


\begin{algorithm}[!ht]
\caption{\textbf{FM-Style Sampling with Context Frames for a Single Chunk}}
\label{alg:fmsampling}
\begin{algorithmic}[1]
\REQUIRE 
   $p(\mathbf{x}_1 | \mathbf{x}_t, \mathbf{t};\theta)$, 
   $t$, 
   context frames $\mathbf{c} = (c^1,\dots,c^m)$, 
   fully masked frame \([M]\) (i.e., a frame where every token equals the mask token \(M\)),
   $t \in [0,1]$, 
   $\Delta t$

\STATE $\mathbf{x}_t \,\gets\, (\,c^1,\dots,c^m,\,[M],\dots,[M])$
\STATE $t \,\gets\, 0$
\STATE $\mathbf{t} \gets (1,\dots,1,0,\dots0)$

\WHILE{$t \,\le\, 1 - \Delta t$}
    \STATE $u_t(\mathbf{x}_t) 
        \;=\; 
        \frac{t}{1-t}
        \Bigl[
          p_\theta(\mathbf{x}_1 \mid \mathbf{x}_t,\,\mathbf{t}) 
          \;-\; 
          \delta_{\mathbf{x}_t}
        \Bigr]$
    \STATE $p_\theta\!\bigl(\mathbf{x}_1 \mid \mathbf{x}_{t+\Delta t},\,\mathbf{t}+\Delta t\bigr)
        \;=\;
        \mathrm{Cat}\!\Bigl[\,
          \delta_{\mathbf{x}_t}
          \;+\; 
          u_t(\mathbf{x}_t)\,\Delta t
        \Bigr]$
    \STATE \textbf{For each token} $n$ in $\mathbf{x}_t$: 
    \STATE \quad 
    $
       x_{t+\Delta t}^{n} \gets
       \begin{cases}
          x_t^{n}, & \text{if } x_t^{n} \neq M,\\
          p(\cdot | \mathbf{x}_{t+\Delta t},\,\mathbf{t}+\Delta t; \theta), & \text{if } x_t^{n} = M.
       \end{cases}
    $
\STATE $t \gets t + \Delta t$
\STATE $\mathbf{t} \gets \mathbf{t} + \Delta t$

\ENDWHILE
\STATE \textbf{return} $\mathbf{x}_t$
\end{algorithmic}
\end{algorithm}

\begin{algorithm}[ht]
\caption{\textbf{MGM-Style Sampling for a Single Chunk}}
\label{alg:mgm_chunk_unmasking_revised}
\begin{algorithmic}[1]
\REQUIRE 
  Network $p(\mathbf{x}_1 \mid \mathbf{x}_t, \mathbf{t}; \theta)$,  
  context frames $\mathbf{c} = (c^1,\dots,c^m)$,  
  masked frame $[M]$ (i.e., every token equals $M$),    
  total unmasking steps $T$
\STATE \textbf{Initialize:}\\
$\mathbf{x}_t \;\leftarrow\; (\mathbf{c},\, [M],\dots,[M])$\\
$\mathbf{t} \;\leftarrow\; (\underbrace{1,\dots,1}_{m},\, \underbrace{0,\dots,0}_{k-m})$
\STATE Define the set of masked token indices in $\mathbf{x}_t$:\\
$\mathcal{M} \;\triangleq\; \{\, n \mid x_t^n = M \,\}.$
\FOR{$i=1$ \textbf{to} $T$}
    \STATE Compute token-wise logits:\\
    $\boldsymbol{\lambda} \;\leftarrow\; p(\mathbf{x}_1 \mid \mathbf{x}_t, \mathbf{t}; \theta).$
    \STATE \textbf{For each token} $n \in \mathcal{M}$: \\
    sample $\hat{x}_t^n \sim \mathrm{Cat}\Bigl(\mathrm{Softmax}\bigl(\boldsymbol{\lambda}^n\bigr)\Bigr)$ \\
    and compute the confidence score 
    $C_n \;=\; \mathrm{Softmax}\bigl(\boldsymbol{\lambda}^n\bigr)_{\hat{x}_t^n}.$ \\
    \STATE \textbf{Define the confidence threshold:}\\
    Let $\alpha$ denote the desired fraction of masked tokens to update in each iteration (e.g. $\alpha = 1/T$). \\
    
    Then set 
    $\tau_c \;=\; \min\Bigl\{ c \in [0,1] \;\Bigm|\; \Bigl|\{ j \in \mathcal{M} \mid C_j \ge c \}\Bigr| \ge \Bigl\lceil \alpha\,|\mathcal{M}| \Bigr\rceil \Bigr\}.$ \\
    
    (That is, $\tau_c$ is chosen as the minimum confidence such that at least $\lceil \alpha\,|\mathcal{M}| \rceil$ tokens have confidence scores at or above $\tau_c$, thereby selecting the top $\lceil \alpha\,|\mathcal{M}| \rceil$ tokens.)
    \STATE \textbf{For each token} $n \in \mathcal{M}$ with $C_n \ge \tau_c$, update:\\
    $x_t^n \;\leftarrow\; \hat{x}_t^n.$
    \STATE Update the set of masked indices:\\
    $\mathcal{M} \;\leftarrow\; \{\, n \mid x_t^n = M \,\}.$
    \IF{$\mathcal{M} = \varnothing$}
         \STATE \textbf{break}
    \ENDIF
\ENDFOR
\STATE \textbf{return} $\mathbf{x}_t$.
\end{algorithmic}
\end{algorithm}


\subsection{Baseline Details}

The two most comparable works to our method are \citet{chen2024diffusionforcing} and \citet{ruhe2024rollingdiffusionmodels}. Both of these techniques propose novel sampling methods that can be rolled out to long video lengths, and also apply frame-specific noise levels. Both of these approaches are diffusion-based and operate on continuous representations, whereas we operate on discrete tokens and use masking. We re-implement both the pyramid sampling scheme proposed in Diffusion Forcing and the Rolling Diffusion sampling method in our discrete setting. This allows us to compare the baseline sampling methods to MaskFlow on the same model backbones. 
%
To isolate the effect of our chunkwise autoregressive sampling methodology on performance from the effects of tokenization, we reimplement both the pyramid sampling scheme proposed in Diffusion Forcing and the Rolling Diffusion sampling method for our discrete setting. This allows us to compare the baseline sampling methods on the same timestep-dependent model backbone. 
%
Although it is conceivable that Rolling Diffusion sampling may perform better when applied to a model explicitly trained using the progressive noise schedule suggested in \citet{ruhe2024rollingdiffusionmodels}, we believe this comparison is still fair. Our training methodology does not inject any inductive bias by way of the masking level into the model, so there is no obvious advantage that our sampling should have over other methods. 
We provide a comprehensive evaluation of performance and sampling efficiency across both datasets and different sampling modes.


\subsection{Dataset Details}

\paragraph{Deepmind Lab.} The Deepmind Lab (DMLab) navigation dataset contains $64 \times 64$ resolution videos of random walks in a 3D maze environment. We use the total 625 videos with frame length 300 frames, and randomly sample sequences of 36 consecutive frames from each video during training. We upscale video frames to a resolution of $256 \times 256$ before tokenizing them similar to our approach for FaceForensics. We disregard the provided actions, focusing on action-unconditional video generation. We use $m=12$ and $s=24$ for the DMLab full sequence generation experiments unless stated otherwise.

\paragraph{FaceForensics.} FaceForensics (FFS) is a dataset that contains $150\times150$ images of deepfake faces, totaling 704 videos with varying number of frames at 8 frames-per-second. We upsample the resolution to $256 \times 256$, before encoding individual frames using the image-based tokenizer SD-VQGAN \cite{rombach2022high_latentdiffusion_ldm}. While image-based tokenizers have shown to lead to flickering issues, we observe high-reconstruction quality (reconstruction FVD $\approx 8$ on FFS) on our datasets and thus leave work on video tokenization to other works. After tokenization, we train on encoded frame sequences of 16 frames, each consisting of token grids with dimensionality $32 \times 32$. We generally use $m=2$ ground-truth context frames for conditioning, and $s=14$.

\subsection{Further Quantitative Results}

\paragraph{Our chunkwise autoregressive MGM-style sampling is preferable to full sequence training in settings with limited hardware.} To evaluate our method for long video generation against a longer training window baseline, we compare the performance of a frame-level masking model trained on $16$ frames with full sequence generation of a constant-masking level model trained on $32$ frames with similar batch size and on similar hardware. In Table ~\ref{tab:longer_train_window_baseline} we show that iterative rollout of our MGM-style sampling outperforms full sequence generation even when the full sequence model is trained on a longer window.

\begin{table}[ht]
    \centering
    \normalsize
    \resizebox{0.48\textwidth}{!}{%
    \begin{tabular}{l|cccc}
    \toprule
    \makecell{\textbf{Sampling} \\ \textbf{Mode}} 
    & \makecell{\textbf{Training} \\ \textbf{Window}} 
    & \makecell{\textbf{Sampling} \\ \textbf{Window}} 
    & \makecell{\textbf{Total} \\ \textbf{NFE}}
    & \makecell{\textbf{FVD} $\downarrow$} \\
    \midrule
    FM-Style (bs=2) & 32 & 32 & 250 & 253.08 \\
    \midrule
    \textit{MaskFlow} (MGM-Style) (bs=2) & 16 & 32 & 60 & 192.76 \\
    \rowcolor{gray!8}\textit{MaskFlow} (MGM-Style) (bs=4) & 16 & 32 & 60 & \textbf{59.93} \\
    \bottomrule
    \end{tabular}
    }
    \caption{\textbf{Our MGM-style sampling is more efficient and generates better results over baseline for larger training windows}. We train a constant masking ratio model on larger window sizes with similar batch size on similar hardware, and compare full sequence generation to generating the same length using our chunkwise MGM-style sampling.}
    \label{tab:longer_train_window_baseline}
\end{table}

\begin{table}[ht]
    \centering
    \normalsize
    \resizebox{0.48\textwidth}{!}{%
    \begin{tabular}{l|ccrr}
        \toprule
        & \makecell{\textbf{Extrapolation} \\ \textbf{Factor}}
        & \makecell{\textbf{Sampling} \\ \textbf{Stride}}
        & \makecell{\textbf{Total} \\ \textbf{NFE}}
        & \makecell{\textbf{FVD} $\downarrow$} \\
        \midrule
        FaceForensics   & $2\times$  & $s=14$ (\textit{full sequence}) & \textbf{60} & 59.93 \\
        \rowcolor{gray!8}FaceForensics   & $2\times$  & $s=1$ (\textit{autoregressive})  & 340 & \textbf{30.43} \\
        \midrule
        FaceForensics   & $5\times$  & $s=14$ (\textit{full sequence}) & \textbf{120} & 108.74 \\
        \rowcolor{gray!8}FaceForensics & $5\times$  & $s=1$ (\textit{autoregressive})  & 1,300 & \textbf{103.69} \\
        \midrule
        FaceForensics   & $10\times$ & $s=14$ (\textit{full sequence}) & \textbf{240} & 214.39 \\
       \rowcolor{gray!8} FaceForensics   & $10\times$ & $s=1$ (\textit{autoregressive})  & 2,900 & \textbf{165.02} \\
        \midrule
        \midrule
        DMLab & $2\times$  & $s=24$ (\textit{full sequence}) & \textbf{60} & 188.22 \\
        \rowcolor{gray!8}DMLab & $2\times$  & $s=1$ (\textit{autoregressive})  & 740  & \textbf{50.87} \\
        \midrule
        DMLab & $5\times$  & $s=24$ (\textit{full sequence}) & \textbf{140}  & 334.15 \\
       \rowcolor{gray!8} DMLab & $5\times$  & $s=1$ (\textit{autoregressive})  & 2,900 & \textbf{181.11} \\
        \bottomrule
    \end{tabular}
    }
    \caption{\textbf{Autoregressive sampling outperforms full sequence sampling on timestep-dependent models at the cost of higher NFE.}}
    \label{tab:autoregression_dependent}
\end{table}


\begin{figure*}[ht!]
    \centering
    \includegraphics[width=0.7\textwidth]{figpaper/realestate.pdf}\hfill
    \caption{\textbf{Further visualizations on the Realestate10K \cite{zhou2018stereo} dataset.} Models trained on chunk size $k = 16$ with $4$ H100 GPUs. Due to computational limitations, we cannot  provide further analyses on this larger, more compute intensive dataset.}
    \label{fig:faces_comparison}
\end{figure*}




\begin{table}[ht]
    \centering
    \normalsize
    \resizebox{0.58\textwidth}{!}{%
    \begin{tabular}{l|ccrr}
        \toprule
        & \makecell{\textbf{Extrapolation} \\ \textbf{Factor}}
        & \makecell{\textbf{Sampling} \\ \textbf{Stride}}
        & \makecell{\textbf{Total} \\ \textbf{NFE}}
        & \makecell{\textbf{FVD} $\downarrow$} \\
        \midrule
        FaceForensics   & $2\times$  & $s=14$ (\textit{full sequence}) & \textbf{60} & 109.96 \\
        FaceForensics   & $2\times$  & $s=1$ (\textit{autoregressive}) & 340 & \textbf{43.91} \\
        \midrule
        FaceForensics   & $5\times$  & $s=14$ (\textit{full sequence}) & \textbf{120} & \textbf{137.66} \\
        FaceForensics & $5\times$  & $s=1$ (\textit{autoregressive})  & 1,300 & 193.90 \\
        \midrule
        FaceForensics   & $10\times$ & $s=14$ (\textit{full sequence}) & \textbf{240} & \textbf{174.92} \\
        FaceForensics   & $10\times$ & $s=1$ (\textit{autoregressive})  & 2,900 & 293.16 \\
        \midrule
        \midrule
        DMLab & $2\times$  & $s=24$ (\textit{full sequence}) & \textbf{60} & 219.33 \\
        DMLab & $2\times$  & $s=1$ (\textit{autoregressive})  & 740  & \textbf{42.53} \\
        \midrule
        DMLab & $5\times$  & $s=24$ (\textit{full sequence}) & \textbf{140}  & 402.73 \\
        DMLab & $5\times$  & $s=1$ (\textit{autoregressive})  & 2,900 & \textbf{80.56} \\
        \bottomrule
    \end{tabular}
    }
    \caption{\textbf{Autoregressive sampling outperforms full sequence sampling on timestep-independent models at the cost of higher NFE.} Performance improvement on DMLab is substantial.}
    \label{tab:autoregression_independent}
\end{table}

\newpage

\subsection{Further Qualitative Results}

\begin{figure*}[ht!]
    \centering
    \includegraphics[width=0.48\textwidth]{figpaper/faces1.pdf}\hfill
    \includegraphics[width=0.48\textwidth]{figpaper/faces2.pdf}
    \caption{\textbf{Visualizations of FaceForensics generation results with different context frames.}}
    \label{fig:faces_comparison}
\end{figure*}
















\end{document}
