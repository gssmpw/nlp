\documentclass[sigconf]{acmart}
%%
%% \BibTeX command to typeset BibTeX logo in the docs
\AtBeginDocument{%
  \providecommand\BibTeX{{%
    Bib\TeX}}}


\usepackage{colortbl}
\usepackage{enumitem}
\usepackage{hyperref}
\usepackage{url}
\usepackage{wrapfig}
\usepackage{booktabs}
\newcommand{\todo}[1]{{\color{blue}{#1}}}
\usepackage{amsfonts}       
\usepackage{amsmath}
\usepackage{graphicx}
\usepackage{amsthm}
% \usepackage{amssymb}
\usepackage{subfigure}
\usepackage{pifont}
\usepackage{mathtools}
\usepackage{stmaryrd}
\usepackage[T1]{fontenc}
\usepackage{multirow}
\newcommand{\goodname}{\textsf{Wildflare GuardRail}}
\usepackage[vlined,linesnumbered,ruled,resetcount]{algorithm2e}
\usepackage{wrapfig}
\usepackage{lipsum}  % For generating filler text
\usepackage{booktabs}

% \usepackage{table}
% \usepackage{ragged2e}
% \usepackage{xcolor}         % colors
% \usepackage{pifont}
\newcommand{\cmark}{\ding{51}}
\newcommand{\xmark}{\ding{55}}
\definecolor{redx}{RGB}{180,0,0}
\definecolor{greenx}{RGB}{0,180,0}
\newcommand{\redcmark}{\color{redx}\ding{51}}
\newcommand{\greenxmark}{\color{greenx}\ding{55}}
\newcommand{\redxmark}{\color{redx}\ding{55}}
\newcommand{\greencmark}{\color{greenx}\ding{51}}
\definecolor{redx}{RGB}{180,0,0}
\definecolor{greenx}{RGB}{0,180,0}
\usepackage{flushend}
\usepackage{algpseudocode}  % For pseudocode formatting
\newcommand{\B}{\vspace*{-\smallskipamount}}
\newcommand{\BB}{\vspace*{-\medskipamount}}
\newcommand{\BBB}{\vspace*{-\bigskipamount}}
\newcommand{\grounding}{{\textsf{Grounding}}}
\newcommand{\detection}{{\textsf{Safety Detector}}}
\newcommand{\fixing}{{\textsf{Repairer}}}
\newcommand{\customization}{{\textsf{Customizer}}}
\newtheorem{definition}{Definition}
\newtheorem{example}{Example}
% \usepackage{algorithm}      % For the algorithm floating environment
\usepackage{algpseudocode}
\usepackage{amsmath}       % For \mathcal, \mathbf, etc.  

\usepackage{quoting}


\begin{document}



\title{Bridging the Safety Gap: A Guardrail Pipeline for Trustworthy LLM Inferences}

\author{Shanshan Han}
\affiliation{%
  \institution{University of California, Irvine}
  \city{Irvine}
  \state{California}
  \country{USA}
}
\email{shanshan.han@uci.edu}

\author{Salman Avestimehr}
\affiliation{%
  \institution{University of Southern California}
  \city{Los Angeles}
  \state{California}
  \country{USA}}
\email{avestime@usc.edu}


\author{Chaoyang He}
\affiliation{%
  \institution{TensorOpera AI}
  \city{Palo Alto}
  \state{California}
  \country{USA}}
\email{ch@tensoropera.com}

\renewcommand{\shortauthors}{Han et al.}



\begin{abstract}
Large language model (LLM)-based agents have shown promise in tackling complex tasks by interacting dynamically with the environment. 
Existing work primarily focuses on behavior cloning from expert demonstrations and preference learning through exploratory trajectory sampling. However, these methods often struggle in long-horizon tasks, where suboptimal actions accumulate step by step, causing agents to deviate from correct task trajectories.
To address this, we highlight the importance of \textit{timely calibration} and the need to automatically construct calibration trajectories for training agents. We propose \textbf{S}tep-Level \textbf{T}raj\textbf{e}ctory \textbf{Ca}libration (\textbf{\model}), a novel framework for LLM agent learning. 
Specifically, \model identifies suboptimal actions through a step-level reward comparison during exploration. It constructs calibrated trajectories using LLM-driven reflection, enabling agents to learn from improved decision-making processes. These calibrated trajectories, together with successful trajectory data, are utilized for reinforced training.
Extensive experiments demonstrate that \model significantly outperforms existing methods. Further analysis highlights that step-level calibration enables agents to complete tasks with greater robustness. 
Our code and data are available at \url{https://github.com/WangHanLinHenry/STeCa}.
\end{abstract}

\maketitle




\section{Introduction}


\section{Introduction}

Despite the remarkable capabilities of large language models (LLMs)~\cite{DBLP:conf/emnlp/QinZ0CYY23,DBLP:journals/corr/abs-2307-09288}, they often inevitably exhibit hallucinations due to incorrect or outdated knowledge embedded in their parameters~\cite{DBLP:journals/corr/abs-2309-01219, DBLP:journals/corr/abs-2302-12813, DBLP:journals/csur/JiLFYSXIBMF23}.
Given the significant time and expense required to retrain LLMs, there has been growing interest in \emph{model editing} (a.k.a., \emph{knowledge editing})~\cite{DBLP:conf/iclr/SinitsinPPPB20, DBLP:journals/corr/abs-2012-00363, DBLP:conf/acl/DaiDHSCW22, DBLP:conf/icml/MitchellLBMF22, DBLP:conf/nips/MengBAB22, DBLP:conf/iclr/MengSABB23, DBLP:conf/emnlp/YaoWT0LDC023, DBLP:conf/emnlp/ZhongWMPC23, DBLP:conf/icml/MaL0G24, DBLP:journals/corr/abs-2401-04700}, 
which aims to update the knowledge of LLMs cost-effectively.
Some existing methods of model editing achieve this by modifying model parameters, which can be generally divided into two categories~\cite{DBLP:journals/corr/abs-2308-07269, DBLP:conf/emnlp/YaoWT0LDC023}.
Specifically, one type is based on \emph{Meta-Learning}~\cite{DBLP:conf/emnlp/CaoAT21, DBLP:conf/acl/DaiDHSCW22}, while the other is based on \emph{Locate-then-Edit}~\cite{DBLP:conf/acl/DaiDHSCW22, DBLP:conf/nips/MengBAB22, DBLP:conf/iclr/MengSABB23}. This paper primarily focuses on the latter.

\begin{figure}[t]
  \centering
  \includegraphics[width=0.48\textwidth]{figures/demonstration.pdf}
  \vspace{-4mm}
  \caption{(a) Comparison of regular model editing and EAC. EAC compresses the editing information into the dimensions where the editing anchors are located. Here, we utilize the gradients generated during training and the magnitude of the updated knowledge vector to identify anchors. (b) Comparison of general downstream task performance before editing, after regular editing, and after constrained editing by EAC.}
  \vspace{-3mm}
  \label{demo}
\end{figure}

\emph{Sequential} model editing~\cite{DBLP:conf/emnlp/YaoWT0LDC023} can expedite the continual learning of LLMs where a series of consecutive edits are conducted.
This is very important in real-world scenarios because new knowledge continually appears, requiring the model to retain previous knowledge while conducting new edits. 
Some studies have experimentally revealed that in sequential editing, existing methods lead to a decrease in the general abilities of the model across downstream tasks~\cite{DBLP:journals/corr/abs-2401-04700, DBLP:conf/acl/GuptaRA24, DBLP:conf/acl/Yang0MLYC24, DBLP:conf/acl/HuC00024}. 
Besides, \citet{ma2024perturbation} have performed a theoretical analysis to elucidate the bottleneck of the general abilities during sequential editing.
However, previous work has not introduced an effective method that maintains editing performance while preserving general abilities in sequential editing.
This impacts model scalability and presents major challenges for continuous learning in LLMs.

In this paper, a statistical analysis is first conducted to help understand how the model is affected during sequential editing using two popular editing methods, including ROME~\cite{DBLP:conf/nips/MengBAB22} and MEMIT~\cite{DBLP:conf/iclr/MengSABB23}.
Matrix norms, particularly the L1 norm, have been shown to be effective indicators of matrix properties such as sparsity, stability, and conditioning, as evidenced by several theoretical works~\cite{kahan2013tutorial}. In our analysis of matrix norms, we observe significant deviations in the parameter matrix after sequential editing.
Besides, the semantic differences between the facts before and after editing are also visualized, and we find that the differences become larger as the deviation of the parameter matrix after editing increases.
Therefore, we assume that each edit during sequential editing not only updates the editing fact as expected but also unintentionally introduces non-trivial noise that can cause the edited model to deviate from its original semantics space.
Furthermore, the accumulation of non-trivial noise can amplify the negative impact on the general abilities of LLMs.

Inspired by these findings, a framework termed \textbf{E}diting \textbf{A}nchor \textbf{C}ompression (EAC) is proposed to constrain the deviation of the parameter matrix during sequential editing by reducing the norm of the update matrix at each step. 
As shown in Figure~\ref{demo}, EAC first selects a subset of dimension with a high product of gradient and magnitude values, namely editing anchors, that are considered crucial for encoding the new relation through a weighted gradient saliency map.
Retraining is then performed on the dimensions where these important editing anchors are located, effectively compressing the editing information.
By compressing information only in certain dimensions and leaving other dimensions unmodified, the deviation of the parameter matrix after editing is constrained. 
To further regulate changes in the L1 norm of the edited matrix to constrain the deviation, we incorporate a scored elastic net ~\cite{zou2005regularization} into the retraining process, optimizing the previously selected editing anchors.

To validate the effectiveness of the proposed EAC, experiments of applying EAC to \textbf{two popular editing methods} including ROME and MEMIT are conducted.
In addition, \textbf{three LLMs of varying sizes} including GPT2-XL~\cite{radford2019language}, LLaMA-3 (8B)~\cite{llama3} and LLaMA-2 (13B)~\cite{DBLP:journals/corr/abs-2307-09288} and \textbf{four representative tasks} including 
natural language inference~\cite{DBLP:conf/mlcw/DaganGM05}, 
summarization~\cite{gliwa-etal-2019-samsum},
open-domain question-answering~\cite{DBLP:journals/tacl/KwiatkowskiPRCP19},  
and sentiment analysis~\cite{DBLP:conf/emnlp/SocherPWCMNP13} are selected to extensively demonstrate the impact of model editing on the general abilities of LLMs. 
Experimental results demonstrate that in sequential editing, EAC can effectively preserve over 70\% of the general abilities of the model across downstream tasks and better retain the edited knowledge.

In summary, our contributions to this paper are three-fold:
(1) This paper statistically elucidates how deviations in the parameter matrix after editing are responsible for the decreased general abilities of the model across downstream tasks after sequential editing.
(2) A framework termed EAC is proposed, which ultimately aims to constrain the deviation of the parameter matrix after editing by compressing the editing information into editing anchors. 
(3) It is discovered that on models like GPT2-XL and LLaMA-3 (8B), EAC significantly preserves over 70\% of the general abilities across downstream tasks and retains the edited knowledge better.



\section{Related Work}
\section{Related Work}

\subsection{Personalization and Role-Playing}
Recent works have introduced benchmark datasets for personalizing LLM outputs in tasks like email, abstract, and news writing, focusing on shorter outputs (e.g., 300 tokens for product reviews \citep{kumar2024longlamp} and 850 for news writing \citep{shashidhar-etal-2024-unsupervised}). These methods infer user traits from history for task-specific personalization \citep{sun-etal-2024-revealing, sun-etal-2025-persona, pal2024beyond, li2023teach, salemi2025reasoning}. In contrast, we tackle the more subjective problem of long-form story writing, with author stories averaging 1500 tokens. Unlike prior role-playing approaches that use predefined personas (e.g., Tony Stark, Confucius) \citep{wang-etal-2024-rolellm, sadeq-etal-2024-mitigating, tu2023characterchat, xu2023expertprompting}, we propose a novel method to infer story-writing personas from an author’s history to guide role-playing.


\subsection{Story Understanding and Generation}  
Prior work on persona-aware story generation \citep{yunusov-etal-2024-mirrorstories, bae-kim-2024-collective, zhang-etal-2022-persona, chandu-etal-2019-way} defines personas using discrete attributes like personality traits, demographics, or hobbies. Similarly, \citep{zhu-etal-2023-storytrans} explore story style transfer across pre-defined domains (e.g., fairy tales, martial arts, Shakespearean plays). In contrast, we mimic an individual author's writing style based on their history. Our approach differs by (1) inferring long-form author personas—descriptions of an author’s style from their past works, rather than relying on demographics, and (2) handling long-form story generation, averaging 1500 tokens per output, exceeding typical story lengths in prior research.



\begin{figure*}
  \centering
  \includegraphics[width=0.92\textwidth]{figures/overview.png}
  \caption{Overview.}
  \label{fig: system_overview}
\end{figure*}

\section{\goodname~Overview}


% \vspace{-1em}
\section{Design Overview}
\label{sec:overview}
% \vspace{-0.2em}

In this section, we first present our new HBD architecture \sys{} guided by the design principles outlined above. We then provide an overview of its key components.


\begin{figure*}[ht]
    \centering
    \begin{subfigure}[b]{0.45\textwidth}
        \centering
        \includegraphics[height=80pt]{figs/design/transceiver.pdf}
        \caption{Components of OCS transceivers.}
        \label{figure:design:transceiver:component}
    \end{subfigure}
    \hspace{10pt}
    \begin{subfigure}[b]{0.45\textwidth}
        \centering
        \includegraphics[height=80pt]{figs/design/phase-shifter-v2.pdf}
        \caption{Zoom into OCS MZI switch matrix.}
        \label{figure:design:transceiver:ocs}
    \end{subfigure}
    \vspace{-10pt}
    \caption{Design of OCS Transceivers. The core component is OCS integrated in transceivers.}
    \label{figure:design:transceiver}
    \vspace{-15pt}
\end{figure*}

\para{Transceiver-centric HBD architecture}.
As discussed in \S\ref{sec:background:hbd} and summarized in \tabref{tab:hbd-compare}, existing architectures face a fundamental tradeoff among scalability, cost, and fault isolation. The GPU-centric architecture offers high scalability and low cost connectivity but suffers from a large fault explosion radius. In contrast, the switch-centric architecture improves fault isolation by leveraging centralized switches to confine failures to the node level. However, this comes at the cost of reduced scalability and higher connection overhead. The GPU-switch hybrid architecture takes a middle-ground approach but still suffers from significant fault propagation. As a result, no existing architecture fully meets all requirements.

Our key insight is that \textit{connectivity and dynamic switching can be unified at the transceiver level} using Optical Circuit Switching (OCS). By embedding OCS within each transceiver, we can enable reconfigurable point-to-multipoint connectivity, effectively combining both connectivity and switching at the optical layer. This represents a fundamental departure from conventional designs, where transceivers are limited to static point-to-point links and rely on high-radix switches for dynamic switching. We refer to this novel design as the \textit{transceiver-centric HBD architecture}. 

We realize this design with \sys{}, which has three key components as shown in \figref{fig:overview}.


\para{Design 1: Silicon Photonics based OCS transceiver (OCSTrx) (\secref{sec:design:docs}).} 
To enable large-scale deployment, we require a low-cost, low-power transceiver with Optical Circuit Switching (OCS) support. Unlike prior high-radix switches solutions that rely on MEMS-based switching~\cite{urata2022missionapollo, mem-optical-switches}, we leverage advances in Silicon Photonics (SiPh), which offer a simpler structure, lower cost, and reduced power consumption—making them well-suited for commercial transceivers.

Our SiPh-based OCS transceiver (OCSTrx), shown on the left of \figref{fig:overview}, provides two types of communication paths: i) \textit{Cross-lane loopback path (path 3)}, enabling direct GPU-to-GPU communication within the node, which can be used to construct dynamic size topologies; ii) \textit{Dual external paths (path 1\&2)}, connecting to external nodes. All these paths utilize time-division bandwidth allocation, featuring sub-1ms switching latency. With this capability, our \ocstrx \xspace allows dynamic reallocation of full GPU bandwidth to an active external path rather than splitting bandwidth across multiple paths. This eliminates redundant link waste—for instance, activating one external path completely disables the other, ensuring efficient bandwidth utilization.


\para{Design 2: Reconfigurable K-Hop Ring topology (\secref{section:design:topology}).}
With \ocstrx~ that provides reconfigurable connections at the transceiver, the next challenge is designing the topology. A naive starting point is the the full-mesh topology~\cite{fullmesh} which can provide full connectivity among all nodes using \ocstrx~. However, full-mesh design requires $O(N^2)$ links, inducing prohibitive complexity and cost. To reduce costs while maintaining near-ideal fault tolerance and performance, we prune the full-mesh topology into a K-Hop Ring topology based on traffic locality and fault non-locality (Details in~\S\ref{section:design:topology}). Combining the reconfigurability of \ocstrx{}, we propose a \textit{reconfigurable K-Hop Ring topology}, shown in the middle of \figref{fig:overview}, which consists of two key parts:

i) \textit{Intra-node topology:} dynamic GPU-granular ring construction is enabled by activating loopback paths. For example, while $N_1$-$N_3$ physically form a line topology, activating loopback paths creates a ring between $N_1$'s GPUs (1–4) and $N_3$'s GPUs (1–4). This mechanism allows for the construction of arbitrary-sized rings at any location, supporting optimal TP group sizes for different models while effectively minimizing resource fragmentation.

ii) \textit{Inter-node fault isolation: } dual external paths connect to primary and secondary neighbors (e.g., 2-Hop Ring). When a node fails (e.g., $N_2$), its neighbor ($N_1$) activates the backup path ($N_1$-$N_3$) to bypass the fault while maintaining full bandwidth, approaching node-level fault explosion radius. \S\ref{section:design:topology} generalizes this design to $K>2$.


\para{Design 3: HBD-DCN Orchestration Algorithm (\secref{sec:design:orch}).}
Designing an optimal HBD topology is crucial, but end-to-end training performance also depends on the efficient coordination between HBD and DCN. For instance, improper orchestration of TP groups can cause DP traffic to span across ToRs, resulting in DCN congestion. However, existing methods lack the ability to jointly optimize HBD and DCN coordination to alleviate congestion and enhance communication efficiency.
To address this, we propose the HBD-DCN Orchestrator, as shown on the right side of \figref{fig:overview}. The orchestrator takes three inputs: the user-defined job scale and parallelism strategy, the DCN topology and traffic pattern, and the real-time HBD fault pattern. It then generates the TP placement scheme, which maximizes GPU utilization and minimizes cross-ToR communication within the DCN.






\vspace{-1em}

\section{\goodname~\detection}\label{sec:safety_detector}


\detection~addresses unsafe inputs and inappropriate LLM responses to ensure that both the user queries provided to the models and the LLM outputs are safe and free from misinformation. 
\subsection{Unsafe Input Detection}\label{sec: unsafe_input_detection}

We developed a model to detect unsafe contents in user queries before they are processed by LLMs for inference.
While existing approaches categorize unsafe content into various types (e.g.,  toxicity, prompt injection, stereotypes, harassment, threats, identity attacks, and violence)~\citep{openai-data-paper,Wang2023DecodingTrustAC,Detoxify}, our method employs a unified, binary classification model finetuned based on our opensourced LLM~\citep{fox}, classifying content as  safe or unsafe.


This strategy offers several key advantages, as follows: \textit{i}) By fine-tuning our base model, which has been trained on vast amounts of data, the classification model can leverage pre-existing knowledge relevant to safety detection.
\textit{ii}) A binary classification of ``safe'' and ``unsafe'' is both efficient and sufficient for LLM services, as any unsafe query should be rejected, regardless of the specific risk.
\textit{iii}) This approach avoids the complexities and potential inaccuracies of categorizing overlapping or ambiguous types of unsafe content in some publicly available datasets. For example, toxicity toward minority groups could also be classified as bias, but current datasets may inadequately capture such nuances.
\textit{iv}) Using straightforward code logic, we can transform public datasets for safety detection into clear safe/unsafe labels, minimizing ambiguity and ensuring high-quality training data.





The biggest challenge in training such model is the discrepancy between the training data and real-world user query distributions, where using traditional datasets alone can result in poor performance due to their divergence from actual user queries~\citep{openai-data-paper}.
To mitigate these issues, we integrated data of various domains and contexts to better simulate the variety of unsafe queries that users might submit.
We crafted a training dataset
by combining samples randomly selected from 15 public datasets, as will be introduced in Table~\ref{tab:exp_datasets} in \S\ref{sec: exp}. 
Such a dataset captures  diverse contents in user inputs in practice, thus can be more representative on potential real-world inputs. 


\subsection{Hallucination Detection and Reasoning}\label{sec: hallucination_detection_and_reasoning}


\begin{algorithm}[!t]
\textbf{Inputs:} 
$\mathcal{D}$: a training dataset that contains ``context'', ``inputs'', ``llm\_answer'', and ``labels'' for hallucination; %\textit{response\_template}: for indicating the start of the response, e.g., ``{\#\#\# Response:}''; 
\textit{\text{prompt}\_\text{template}}: for formulating the hallucination detection data, see Figure~\ref{fig: training_data_example}; \textit{\text{GPT}\_\textit{reasoning}\_\text{template}}: for generating prompts for GPT API, see Figure~\ref{fig: training_data_example}.

\textbf{Outputs:} $\mathcal{D}_t$: the training dataset.


\nl{\bf Function $\boldsymbol{\mathit{process\_data}(\mathcal{D})}$} \nllabel{ln:function_process_data}
\Begin{

\nl $\mathcal{D}_t\leftarrow \phi$

\nl \For{$d \in $ $\mathcal{D}$}{

\nl \eIf{$\mathit{is}\_\mathit{hallucination}$($d$)}{

\nl $\mathit{halu}\_\mathit{reason}\leftarrow\mathit{GPT}\_\mathit{API}$(\textit{GPT}\_\textit{reasoning}\_\textit{template}($d$[``$\mathit{question}$''], $d$[``$\mathit{context}$''], $d$[``$\mathit{llm}\_\mathit{answer}$'']))

\nl $\mathit{response}\leftarrow$ ``Yes, '' + $\mathit{halu}\_\mathit{reason}$


\nl $d^\prime\leftarrow$\textit{prompt}\_\textit{template}($d$[``$\mathit{question}$''], $d$[``$\mathit{context}$''], $d$[``$\mathit{llm}\_\mathit{answer}$''], $\mathit{response}$)

}{

\nl $d^\prime\leftarrow$\textit{prompt}\_\textit{template}($d$[``$\mathit{question}$''], $d$[``$\mathit{context}$''], $d$[``$\mathit{llm}\_\mathit{answer}$''], ``$\mathit{No.}$'')


}

\nl $\mathcal{D}_t$.$\mathit{add}(d^\prime)$

}

\nl \textbf{return $\mathcal{D}_t$}
}

\caption{Hallucination detection training data processing.}
\label{alg:hallucination_data_processing}
\end{algorithm}







Hallucinations occur when the LLM generates responses that is inaccurate, fabricated, or irrelevant~\citep{filippova2020controlled, maynez2020faithfulness,huang2023survey,rawte2023survey}.
Despite appearing coherent and plausible, hallucinated LLM responses are unreliable, often containing fabricated, misleading information that is  
divergent from the user input, thus fail to meet users' expectations and severely undermine the trustworthiness and utility of the LLM applications.
While grounding can mitigate hallucinations by contextualizing user inputs and enriching the informativeness of user queries, it cannot eliminate hallucinations entirely. 
This is because hallucinations stem from nearly every aspects of LLM training and inference, such as low-quality training data~\citep{lin2021truthfulqa,kang2023impact} and %LLM memorizing training data~\citep{lin-etal-2022-truthfulqa}, 
randomness of sampling strategies~\citep{chuang2023dola}, and moreover, the very nature probabilistic properties of LLMs. 

Effectively handling hallucinations in LLM responses is both crucial and challenging for producing high-quality LLM responses. 
Existing works that detect presence of hallucinations are insufficient~\citep{manakul2023selfcheckgpt,liu2021token}. To provide high-quality responses to users,  we should handle the detected hallucinations properly, i.e., obtaining the explanations for the hallucinations in the LLM responses and further, fixing the hallucinated responses if possible. 


To this end, we propose utilizing our own LLM, Fox-1, as base model~\citep{fox} to finetune a 
hallucination detection model  for detecting hallucinated content and providing explanations, and further, facilitating the subsequent \fixing~in \S\ref{sec:fixing}. The design of the model has the following advantages: \textit{i}) \textit{classification}: it identifies the presence of hallucinations in the LLM output; and \textit{ii}) \textit{reasoning}: it generates explanations for the hallucinated contents, offering insights for the subsequent correction in \fixing; \textit{iii}) \textit{simultaneous classification and reasoning}: it process \textit{i}) and \textit{ii}) at the same time, which saves computation cost and improves efficiency; and \textit{iv}) \textit{vast pre-training data}: it leverages pre-existing knowledge
on hallucination in the base model, which may potentially benefit hallucination detection and reasoning.



\begin{figure*}
  \centering
  \includegraphics[width=\textwidth]{figures/training_data_example.pdf}
  \caption{Prompt templates and sample training data for hallucination detection and reasoning.}
  \label{fig: training_data_example}
\end{figure*}

\textbf{Training. }
We feed our base model with hallucination dataset to train a model for both detecting and reasoning for the hallucination. %Thus, the model should have text generation capabilities. 
However, public available datasets for hallucinated LLM responses are mainly classification datasets with texts and labels, e.g., HaluEval~\citep{li2023halueval}. To address this, we utilize the GPT4 API~\citep{openai-data-paper} to generate explanations for hallucinated contents, and
define a prompt template
to create structured prompts based on the classification data to make it suitable for classification and reasoning simultaneously. 
We demonstrate the prompt templates and sample training data in Figure~\ref{fig: training_data_example}, and summarize data processing in Algorithm~\ref{alg:hallucination_data_processing}. 







\begin{algorithm}[!t]
\textbf{Inputs:}
$\mathcal{M}$: hallucination detection model; 
$\mathit{tokenizer}$: tokenizer for $\mathcal{M}$; $\mathit{q}$: a query submitted by users;
$\mathit{context}$: the context to answer the question; retrieved from vector data storage;
$a$: the answer returned by an LLM for the question; \textit{inference}\_\textit{prompt}\_\textit{template}: see Figure~\ref{fig: training_data_example}. 



\nl{\bf Function $\boldsymbol{\mathit{inference}(\mathcal{M}, \mathit{q}, \mathit{context}, a, k)}$} \nllabel{ln:inference}
\Begin{


\nl $\mathit{prompt}\leftarrow$\textit{inference}\_\textit{prompt}\_\textit{template}($\mathit{q}$, $\mathit{context}$, $a$)

\nl $\mathit{tokenized}\_\mathit{prompt}\leftarrow \mathit{tokenizer}$($\mathit{prompt}$)

\nl $\mathit{halu}\_\mathit{res}\leftarrow$ 
$\mathcal{M}$.$\mathit{generate}$($\mathit{tokenized}\_\mathit{prompt}$)

\nl $\mathit{first}\_\mathit{word}\_\mathit{logits}\leftarrow \mathit{halu}\_\mathit{res}$.$\mathit{logits}$[0], 

\nl$\mathit{results}\leftarrow\mathit{softmax}$($\mathit{first}\_\mathit{word}\_\mathit{logits}$)

\nl $\mathit{top\_k\_probs}\leftarrow \mathit{top}$($\mathit{results}$, $k$)

\nl $P_{\mathit{halu}}(a)\leftarrow \mathit{compute}\_\mathit{halu}\_\mathit{prob}(\mathit{top}\_k\_\mathit{probs})$ 

% \\\Comment{See Definition~\ref{def:halu_prob}}


\nl \lIf{$P(a)\geq 0.5$}{\textbf{return} True}

\nl{\textbf{return} False}
}

\caption{Hallucination detection model inference.}
\label{alg:inference}
\end{algorithm}


\textbf{Inference. }
We expect the LLM to directly output results whether the LLM response contains hallucinations, \textit{i}.\textit{e}., the first token of outputs to be ``Yes'' or ``No'' as detection results, according to the formatted data sample in Figure~\ref{fig: training_data_example}. However, the first token of the LLM response is probabilistic due to the self-autoregressive nature of decoder-based text generation LLMs. 
To obtain desired outputs, we formulate the text-generation outputs by utilizing the top-$k$ first tokens (and their possibilities) of the outputs to generate classification results. By default, $k$ is 10. 

\begin{definition}[Probability of hallucination]
Let $a$ be an LLM answer, let $\{t_1, ..., t_{k}\}$ be the top-k potential first token, and let $\{p_1, ..., p_{k}\}$  be their top-k probabilities. Let $T$ be a tokenization function, and let $T(\text{"Yes"})$ and $T(\text{"No"})$ be the tokens corresponding to ``Yes'' and ``No'', respectively. The probability of hallucination in $a$ is
$P_\mathit{halu}(a) = \frac{\sum_{i=1}^{k} P(t_i | t_i\in T(\text{"Yes"}))}{\sum_{i=1}^{k} P(t_i | t_i\in T(\text{"Yes"})) + \sum_{i=1}^{k} P(t_i | t_i\in T(\text{"No"}))}
$   
\end{definition}\label{def:halu_prob}

Detection results with $P_{\mathit{halu}}(*)\geq0.5$ indicate the content is classified as ``hallucinated''; otherwise, the content is ``safe''.  The detailed procedure of  inference is described in Algorithm~\ref{alg:inference}.


\section{\goodname~\grounding}\label{sec:grounding}






\goodname~\grounding~enhances the contextual richness and informativeness of user queries by leveraging external knowledge in vector database. Thus, LLMs can utilize such contextual knowledge to generate high-quality outputs, particularly by grounding user queries before they are passed to the LLMs for inference.

To support similarity search over the knowledge data, \goodname~creates vector indexes by vectorizing plaintext knowledge.% This involves vectorizing entire knowledge entries to create vector indexes. 
\goodname~employs two primary methods for indexing: \textit{i}) \textit{{Whole Knowledge Index}} that creates indexes based on each entire data entry in the datasets; and \textit{ii}) \textit{{Key Information Index}} that indexes only the key information in each data entry, i.e., questions in QA datasets. 
Whole Knowledge Index reflects the data distribution and ensurers that the indexed data captures the contextual variety and complexity found in real-world queries, while Key Information Index 
focuses on the core information of each data entry, thus facilitates efficient retrieval of relevant data. 
We evaluate the effectiveness of indexes with \textit{callback}, i.e., the probability of successfully retrieving the original records from a dataset using Top-$k$ queries. 
We experimentally evaluate the indexing methods in \S\ref{sec: exp}.


\begin{definition}[Callback]
    Let $D_v$ be a vector data storage that contains $n$ records, let $Q$ be a plaintext user query set, and let $I(Q)$ be the vector index created based on $Q$. For each query $q\in Q$, let $I_q$ be the vector index created based on $q$, and let $D_v(I_q)$ denote the set of Top-$k$ records returned by querying $D_v$ with $I(q)$, and let $r_q$ denote the most relevant record of $q$ in $D_v$. 
    The callback for Top-$k$ queries on the query set $Q$ is defined as:
$$C_k(Q) = \frac{1}{|Q|} \sum_{q \in Q} [r_q \in D_v(I_q)]$$
where $[\cdot]$ is Iverson Bracket Notation~\citep{iverson1962programming}, equal to 1 if the condition inside is true, and 0 otherwise.
\end{definition}

To ensure effective and informative grounding, 
the distribution of the index should closely align with query patterns, i.e., query distributions. 
By grounding user queries with knowledge retrieved with a proper index, 
the LLMs can generate contextually appropriate responses, and further, reduce hallucinations and improve the quality of the responses. 



\begin{figure*}
  \centering
  \includegraphics[width=0.86\textwidth]{figures/fixing_data_example.pdf}
  \caption{Prompt templates and sample training data for \fixing.}
  \label{fig: training_data_example_fixing}
\end{figure*}

\section{\goodname~\customization}\label{sec:customization}

\goodname~\customization~utilizes lightweight wrappers to flexibly edit or customize LLM outputs to fix some small errors or enhancing the format of the answer. The wrappers integrate code-based rules, APIs, web searches, and small models to efficiently handle editing and customization tasks according to user-defined protocols. \goodname~\customization~ offers several key advantages. It facilitates rapid development and deployment of user-defined protocols, which crucial in production environments where real-time adjustments are necessary. In scenarios where training or fine-tuning LLMs is unfeasible due to time or resource constraints, this method provides an alternative for immediate output customization. Moreover, the wrappers enable flexible incorporation  of various tools and data sources, which enhances the applicability of  \goodname~and reduces resource-intensive LLM calls. 



\begin{example}[Warning URLs]\label{example:waring_urls}
The objective was to detect if LLM outputs contain URLs and prepend a warning message of the unsafe URLs at the beginning of the LLM outputs. \customization~should check the safety of the URLs founded,  i.e., whether they are malicious or unreachable, and includes such information in the warning if they were unsafe.
\customization~utilizes a regular expression pattern 
to identify URLs within the text. Upon URLs founded, \customization~calls APIs for detecting phishing URLs, such as Google SafeBrowsing~\citep{google-safe-browsing}, and assess the accessibility of the benign URL by issuing web requests. Malicious URLs, as well as unreachable URLs that return status codes of 4XX, are added in the warning at the beginning of the LLM outputs.
\end{example}


Note that the task in Example~\ref{example:waring_urls} cannot be achieved through prompt engineering when querying LLMs, as the warning must appear at the beginning, and LLMs generate content token by token, making later content unpredictable.
We use the following example to illustrate this property, and experimentally evaluate the efficiency of \customization~wrappers in \textbf{Exp 4} in~\S\ref{sec: exp}. 


\begin{example}
\label{ex:llm-limitation}
We present a concrete demonstration of tasks that cannot be reliably accomplished through prompt engineering alone, due to the token-by-token generation mechanism inherent in LLMs. This sequential generation process fundamentally precludes anticipatory knowledge of future token occurrences during text production.
Consider the following prompt submitted to GPT-4:



\begingroup
{
% \addtolength\leftmargini{-0.2cm}
\begin{quote}
\ttfamily
\noindent Write an English poem about a rabbit; please include information at the beginning of the poem about how many times the word ``rabbit'' appears in the poem.
\end{quote}
}
\endgroup

The generated response (shown below) claims four occurrences of ``rabbit,'' while actual analysis reveals five occurrences:
\begingroup
{ 
\begin{quote}
\ttfamily
(The word ``rabbit'' appears 4 times in this poem.) 

In a field where soft grasses sway and dip, 
a \textcolor{red}{rabbit} hopped beneath the moon's silver whip. 
With ears like twin sails in the cool night air, 
each twitch and turn done with utmost care. \\

This \textcolor{red}{rabbit}, cloaked in twilight's gentle hug, 
nestled in shadows, a cozy, earthen snug. 
Dreams of dandelion fields in sunny sprawl, 
where no shadows loom, nor fears enthrall. \\

With a nose that twitches at secrets untold, 
and paws that dance over leaves of gold, 
the \textcolor{red}{rabbit} finds joy in the simplest of things—the warmth of the sun, the freedom of springs. \\

But as dawn breaks and stars bid goodbye, 
back to the burrow, this \textcolor{red}{rabbit} must fly. 
For the world is wide and wolves roam free,
yet in his burrow, safe as can be, this \textcolor{red}{rabbit} shall be.
\end{quote}
}
\endgroup

This discrepancy arises from the LLM's inability to maintain an accurate token count during generation while simultaneously composing coherent text. The example underscores the necessity of post-inference wrappers to detect and correct such errors in model outputs.
\end{example}






\section{\goodname~\fixing}\label{sec:fixing}


\goodname~\fixing~addresses errors in the LLM  outputs that are challenging to resolve through editing with wrappers in~\customization, particularly, hallucinated content. \fixing~analyzes and corrects the hallucinated output based on the reason for the hallucinations generated by the hallucination detection model. 


\goodname~\fixing~takes several key inputs, including the user's original query, the context retrieved with \grounding, the hallucinated responses generated by the LLM, as well as the reason for hallucination. 
Given these inputs, \fixing~corrects the flawed output according to the hallucination reason.
To enable \fixing~to handle hallucinations effectively, we  leverage the same hallucination detection dataset as \detection, i.e., HaluEval~\citep{li2023halueval}, that contains user questions, contexts, hallucinated LLM answers, and correct answers. 
We also designed a customized data template that incorporates the information. The data templates for training, inference, as well as an example for the training data, are demonstrated in Figure~\ref{fig: training_data_example_fixing}. 




\section{Experiments}\label{sec: exp}

\section{Experiments}

\subsection{Datasets}

\textbf{MSMARCO}.
We utilized the MS MARCO Passage Ranking dataset as the data source to evaluate the ability of our method to improve document rankings under more challenging topic-query tasks. Specifically, we assessed whether our method could significantly enhance the ranking of documents by the retrieval model within a RAG system.

To construct topic-lists for evaluation, we applied a K-means clustering algorithm to group similar queries, forming topics that each contained a series of related queries. To further evaluate the performance of our method under extreme topic-query scenarios, we applied an intra-topic similarity filtering process. Only topics with queries exhibiting high semantic diversity and containing a sufficient number of queries were retained.

This process resulted in 29 topics, with each topic containing an average of 22.28 queries. The average similarity score within each topic was approximately 0.5, indicating sufficient diversity among queries to ensure a rigorous evaluation. This curated dataset enabled us to test the robustness of our method in handling highly diverse and challenging topic-query tasks within a RAG system.

\textbf{PROCON}.
To conduct our experiments, we utilized controversial topic data scraped from the PROCON.ORG website. This dataset includes over 80 topics covering various domains, such as society, health, government, and education. Each topic is discussed from two stance labels \{\textit{PRO (support), CON (oppose)}\}, with passages arguing from these perspectives.

To simulate real-world user interactions with a RAG system, we instructed a large language model (GPT-4o) to act as a user and generate 40 potential sub-queries for each topic. These sub-queries were designed to reflect the diverse questions and concerns users might raise when exploring a specific controversial topic. 

After generating the sub-queries, we applied a similarity filtering process to ensure diversity by retaining only those with a similarity score below approximately 0.85. The filtering step effectively removed redundant queries while preserving a wide range of perspectives. As a result, the final set of topic-queries achieved an average similarity score of approximately 0.7, ensuring that the queries were sufficiently diverse yet semantically relevant. This process provided a robust and balanced sub-queries set for evaluation.


\subsection{Experiment Details}
The specific setting details for the Topic-queries RAG manipulation experiment are as follows:

(1) Black-box RAG. We represent the black-box RAG process as \( \text{RAG}_{\text{black}} \). The RAG framework is Conversational RAG from LangChain. The LLMs adopted in RAG are the open-source models Meta-Llama-3.1-8B-Instruct (Llama3.1), Qwen-2.5-7B-Instruct(Qwen2.5). The system prompt and additional detailed descriptions are provided in Appendix~\ref{exp-detail}.

(2) Retrieval Model Specification. We benchmark three dominant dense retrievers—Contriever \cite{gao2021unsupervised}, DPR \cite{karpukhin-etal-2020-dense}, and ANCE \cite{xiong2020approximate}.By convention, we use dot product between the embedding vectors of questions(queries) and candidate documents as their similarity score \(R\) in the ranking. 


\label{opinion-classfication}
(3) Opinion classification. We use Qwen2.5-Instruct-72B as the opinion classifier. Qwen2.5-Instruct-72B, due to its large parameter size, is capable of accurately identifying and classifying opinions within text.

(4) Experimental parameters. In knowledge-guided attack process, we set the maximum editing distance $\epsilon$ to 0.2, the semantic similarity threshold $\lambda$ to 0.85, and the number of iterations $N$ to 5. For adversarial trigger generation, we used a beam size of 3, top-$k$ of 10, a batch size of 32, a temperature of 1.0, a learning rate of 0.005, and a sequence length of 10. In RAG\textsubscript{black}, $k$ (the number of retrieved documents) is set to 3, with the LLM temperature also fixed at 1.0 to mirror real-world conditions.

(5) Poisoned Target. For the PROCON dataset, to investigate the manipulation performance under more challenging conditions, we performed relevance ranking for the documents with respect to each topic-query set $Q$ and the target stance $S_t$ . From the ranked list, we selected the last five documents (i.e., those with the lowest relevance) as the target poisoned documents.
For the MS MARCO dataset, we utilized the top-1000 relevance-ranked passage list for each topic-query set. From this list, we selected the passage with the lowest average rank as the target passage. This approach ensures that the evaluation focuses on passages that are least relevant to the target queries, thus providing a more rigorous benchmark for the proposed method.

(6) Experimental environment. All our experiments are conducted in Python 3.8 environment and run on a NVIDIA DGX H100 GPU. 

\subsection{Research Questions}

We propose four research questions to evaluate the effectiveness of our method in the topic-queries task, focusing on black-box NRM attacks and opinion manipulation to RAGs.

\textbf{RQ1}: Can Topic-FlipRAG significantly enhance the rankings of target documents in the RAG retriever for topic-queries?

\textbf{RQ2}: To what extent does Topic-FlipRAG affect the answers generated by the target RAG systems?

\textbf{RQ3}: Does topic-oriented opinion manipulation significantly impact users' perceptions of controversial topics?

\textbf{RQ4}: How robust does Topic-FlipRAG against exisiting mitigation mechanism?

\subsection{Baseline Settings}
To assess the effectiveness of our proposed method, we compare it against adversarial attack baselines designed for black-box, topic-oriented RAG scenarios, ensuring minimal modifications to the original documents. We exclude BadRAG\cite{xue2024badrag}, a backdoor RAG attack limited to white-box scenarios, and topic-IR-attack\cite{liu2023topic}, as its incomplete implementation prevents reliable replication.
For the selected baseline methods, we adapted them to meet the requirements of our task while preserving their core components. A brief overview of the baseline methods is provided below, with detailed descriptions available in Appendix~\ref{baselines-details}.

\textbf{PoisonedRAG.}
Zou et al.\cite{zou2024poisonedrag} propose an approach adaptable to both black-box and white-box settings. For our task, we employ its black-box strategy by inserting a randomly chosen query from the topic-queries set \( Q \) at the beginning of each document.

\textbf{PAT.}
This gradient-based adversarial retrieval attack uses a pairwise loss function to generate triggers that meet fluency and coherence constraints. We adapt PAT to produce triggers \( T_{\text{pat}} \) for target documents within the topic-queries set, evaluating their effectiveness under black-box conditions.


\textbf{Collision.}
This method generates adversarial paragraphs (collisions) via gradient-based optimization to produce content semantically aligned with the target query. In a topic-queries context, we create collisions for the entire topic-queries set and examine their transfer performance on black-box RAG retrievers.

These baseline methods provide benchmarks for comparing the efficacy of our approach in a fully black-box, topic-oriented RAG attack scenario.

\subsection{Evaluation Metrics}

For \textbf{RQ1}, we focus on ranking manipulation. We measure the average proportion of target opinions in top-3 rankings before and after manipulation (\(\text{Top3}_{\text{ori}}, \text{Top3}_{\text{att}}\)) and define top3-v as their difference. We also employ the Ranking Attack Success Rate (RASR), reflecting how often target documents are successfully boosted, and Boost Rank (BRank), denoting the average rank improvement for all target documents. Lastly, we report the proportion of target documents in the Top-50 and Top-500 positions to indicate how effectively they are pushed toward higher rankings.

\textbf{top3-v.} Computed by subtracting \(\text{Top3}_{\text{ori}}\) from \(\text{Top3}_{\text{att}}\), top3-v ranges from -1 to 1. A positive value signifies a successful increase of the target opinion in top-3 results, while a negative value indicates a detrimental effect.

\textbf{Ranking Attack Success Rate (RASR).} RASR captures how frequently target documents are successfully boosted in each query’s ranking. Higher values indicate greater attack effectiveness.

\textbf{Boost Rank (BRank).} BRank is the average rank improvement for all target documents under each query. A target document contributes negatively if its rank is unintentionally lowered.

\textbf{Top-50, Top-500.} These metrics represent the percentage of target documents that move into specific ranking thresholds in the MS MARCO Dataset after manipulation. Higher percentages imply more effective promotion of target documents. 


For \textbf{RQ2}, we employ Average Stance Variation (ASV) to assess how significantly our opinion manipulation influences the LLM’s responses in a black-box RAG. To address the natural variability of controversial topics and the inherent instability of large language models, we also propose Real Adjusted ASV (\(\Delta\)-ASV).

\textbf{Average Stance Variation (ASV).}
ASV is defined as the absolute difference between the manipulated opinion score and the original opinion score assigned to an LLM response (0 = opposing, 1 = neutral, 2 = supporting). A higher ASV signifies a more pronounced shift in polarity and hence greater manipulation effectiveness.

\textbf{Real Adjusted ASV ($\Delta$-ASV)}. To account for the inherent variability of controversial topics and the instability of large language models, we measure the baseline ASV in a clean state, denoted as ASV\textsubscript{clean} (calculated without adversarial manipulation). $\Delta$-ASV is then obtained by subtracting ASV\textsubscript{clean} from the manipulated ASV, i.e., \( \text{$\Delta$-ASV} = \text{ASV} - \text{ASV\textsubscript{clean}} \). This adjustment ensures that $\Delta$-ASV reflects the true impact of adversarial manipulation by eliminating the influence of natural stance variation. It reflects the extent to which the polarity of the RAG-system outputs is affected by the manipulation.  A positive $\Delta$-ASV indicates a significant shift in the opinion polarity due to manipulation, with larger values representing greater manipulation effectiveness.


\section{Conclusion}\label{sec:conclusion}
\section{Conclusion }
This paper introduces the Latent Radiance Field (LRF), which to our knowledge, is the first work to construct radiance field representations directly in the 2D latent space for 3D reconstruction. We present a novel framework for incorporating 3D awareness into 2D representation learning, featuring a correspondence-aware autoencoding method and a VAE-Radiance Field (VAE-RF) alignment strategy to bridge the domain gap between the 2D latent space and the natural 3D space, thereby significantly enhancing the visual quality of our LRF.
Future work will focus on incorporating our method with more compact 3D representations, efficient NVS, few-shot NVS in latent space, as well as exploring its application with potential 3D latent diffusion models.



\bibliographystyle{ACM-Reference-Format}
\bibliography{reference}



\end{document}
