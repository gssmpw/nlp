\documentclass[sigconf]{acmart}
%%
%% \BibTeX command to typeset BibTeX logo in the docs
\AtBeginDocument{%
  \providecommand\BibTeX{{%
    Bib\TeX}}}


\usepackage{colortbl}
\usepackage{enumitem}
\usepackage{hyperref}
\usepackage{url}
\usepackage{wrapfig}
\usepackage{booktabs}
\newcommand{\todo}[1]{{\color{blue}{#1}}}
\usepackage{amsfonts}       
\usepackage{amsmath}
\usepackage{graphicx}
\usepackage{amsthm}
% \usepackage{amssymb}
\usepackage{subfigure}
\usepackage{pifont}
\usepackage{mathtools}
\usepackage{stmaryrd}
\usepackage[T1]{fontenc}
\usepackage{multirow}
\newcommand{\goodname}{\textsf{Wildflare GuardRail}}
\usepackage[vlined,linesnumbered,ruled,resetcount]{algorithm2e}
\usepackage{wrapfig}
\usepackage{lipsum}  % For generating filler text
\usepackage{booktabs}

% \usepackage{table}
% \usepackage{ragged2e}
% \usepackage{xcolor}         % colors
% \usepackage{pifont}
\newcommand{\cmark}{\ding{51}}
\newcommand{\xmark}{\ding{55}}
\definecolor{redx}{RGB}{180,0,0}
\definecolor{greenx}{RGB}{0,180,0}
\newcommand{\redcmark}{\color{redx}\ding{51}}
\newcommand{\greenxmark}{\color{greenx}\ding{55}}
\newcommand{\redxmark}{\color{redx}\ding{55}}
\newcommand{\greencmark}{\color{greenx}\ding{51}}
\definecolor{redx}{RGB}{180,0,0}
\definecolor{greenx}{RGB}{0,180,0}
\usepackage{flushend}
\usepackage{algpseudocode}  % For pseudocode formatting
\newcommand{\B}{\vspace*{-\smallskipamount}}
\newcommand{\BB}{\vspace*{-\medskipamount}}
\newcommand{\BBB}{\vspace*{-\bigskipamount}}
\newcommand{\grounding}{{\textsf{Grounding}}}
\newcommand{\detection}{{\textsf{Safety Detector}}}
\newcommand{\fixing}{{\textsf{Repairer}}}
\newcommand{\customization}{{\textsf{Customizer}}}
\newtheorem{definition}{Definition}
\newtheorem{example}{Example}
% \usepackage{algorithm}      % For the algorithm floating environment
\usepackage{algpseudocode}
\usepackage{amsmath}       % For \mathcal, \mathbf, etc.  

\usepackage{quoting}


\begin{document}



\title{Bridging the Safety Gap: A Guardrail Pipeline for Trustworthy LLM Inferences}

\author{Shanshan Han}
\affiliation{%
  \institution{University of California, Irvine}
  \city{Irvine}
  \state{California}
  \country{USA}
}
\email{shanshan.han@uci.edu}

\author{Salman Avestimehr}
\affiliation{%
  \institution{University of Southern California}
  \city{Los Angeles}
  \state{California}
  \country{USA}}
\email{avestime@usc.edu}


\author{Chaoyang He}
\affiliation{%
  \institution{TensorOpera AI}
  \city{Palo Alto}
  \state{California}
  \country{USA}}
\email{ch@tensoropera.com}

\renewcommand{\shortauthors}{Han et al.}



\begin{abstract}
\begin{abstract}
Out-of-distribution (OOD) detection and OOD generalization are widely studied in Deep Neural Networks (DNNs), yet their relationship remains poorly understood. We empirically show that the degree of Neural Collapse (NC) in a network layer is inversely related with these objectives: stronger NC improves OOD detection but degrades generalization, while weaker NC enhances generalization at the cost of detection. This trade-off suggests that a single feature space cannot simultaneously achieve both tasks. To address this, we develop a theoretical framework linking NC to OOD detection and generalization. We show that entropy regularization mitigates NC to improve generalization, while a fixed Simplex Equiangular Tight Frame (ETF) projector enforces NC for better detection. Based on these insights, we propose a method to control NC at different DNN layers. In experiments, our method excels at both tasks across OOD datasets and DNN architectures. 

\begin{comment}   

Out-of-distribution (OOD) detection and OOD generalization are critical for deploying machine learning models in real-world scenarios. While substantial progress has been made in addressing these problems independently, few works have attempted to tackle them jointly. However, existing methods often rely on auxiliary OOD training data and primarily focus on covariate-shifted OOD data that share labels with in-distribution (ID) data. In contrast, we tackle the more realistic and challenging task of jointly detecting and generalizing to semantic OOD data with disjoint labels from the ID data, without auxiliary OOD training data.
Achieving both objectives simultaneously is inherently difficult due to a fundamental conflict — OOD generalization requires enhanced transferability, while OOD detection necessitates the inhibition of transfer.
To address this, we leverage insights from neural collapse (NC) — a phenomenon in deep networks where top-layer representations suppress feature variability and adopt a Simplex Equiangular Tight Frame (ETF) structure, impairing transferability. By controlling NC, we unify OOD detection and generalization: preventing NC enhances OOD transfer while inducing NC improves OOD detection.
Our proposed method excels at both tasks across various OOD datasets and architectures. 

\end{comment}


\end{abstract}
\end{abstract}

\maketitle




\section{Introduction}


\section{Introduction}

% State of the world (robots for creative activites)
The term ``robot,'' originally signifying `forced labor,' has long been associated with labor and work. Robots have demonstrated their utility in various automated productive and social contexts, where the primary goals are improving productivity, safety, and fostering social interactions with humans~\cite{simoes2022designing, weidemann2021role, honig2018understanding}. However, an increasing number of cases feature using of robots in creative settings. Unlike productive contexts, where the focus is on efficiency and task completion~\cite{arents2022smart}, or social contexts, where communication and trust are prioritized~\cite{nam2020trust, saunderson2019robots}, creative environments prioritize artistic innovation and expression~\cite{hsueh2024counts}. This shift fundamentally alters the dynamics of human-robot interaction, redefining the roles and expectations for both humans and robots.

For instance, robots’ social behaviors are leveraged to support the generation and expression of creative ideas~\cite{hu2021exploring, sandoval2022human, alves2020creativity}, and programmable robotic movements and trajectories are employed to inspire artistic activities such as sketching~\cite{lin2020your}. These studies often engage participants from creative fields who possess limited prior experience with robotics, and are typically conducted in short-term, experimental settings. Consequently, the findings from these studies remain constrained since much can be learned from professional practitioners' experiences to inform system design such as digital fabrication~\cite{hirsch2023nothing}. There is a notable gap in research examining the long-term, active, and practical experience of integrating robotic systems into the creative processes. As a result, the deeper insights into how robots facilitate and shape creative processes, beyond simply augmenting human creativity, remain underexplored. In this study, we aim to better understand the impacts of robots on creative processes and outcomes.

As early as Leonardo da Vinci's 16th century ``Automaton,'' artists have explored the creative affordances of robotic systems~\cite{shanken2002cybernetics, pagliarini2009development, jeon2017robotic}. The artistic creation process typically encompasses various stages, including the exploration of materials and techniques, ongoing experimentation and iteration, and the continual refinement of the artists' insights into their creative subjects~\cite{lewis2023art, sturdee2022state}. Therefore, investigating the artistic process involving robots offers an opportunity to gain deeper insights into robots' creative potential. Robotic art, in particular, provides a compelling case for this exploration.

We define robotic art as artworks that utilize robotic or automated machines to create artistic experiences and tangible artifacts. One example is robotic installation art, in which robots are programmed to follow specific rules that embody the artist’s expression (\autoref{fig:teaser} (a)). Another example is responsive art, in which robots react to their environment, with behaviors that change over time or in response to spectators (\autoref{fig:teaser} (b)). Additionally, there are robotic creators, which possess a degree of agency, allowing them to collaborate with human artists and produce works that extend beyond mere replication of human-created art (\autoref{fig:teaser} (c) and (d)). As such, robotic art becomes a rich case for exploring human-machine interactions in creative contexts. Gaining a deeper understanding of how robots facilitate artistic expression can provide insights for designing computing systems to support creative activities~\cite{gomez2021robot}.

% Therefore, we did...
We draw on semi-structured, in-depth interviews with renowned professional robotic artists to investigate the use of robots in artistic practice. Specifically, our goal is to understand how artistic exploration of robotic systems challenges conventional assumptions about the functions of robots, such as their roles in automating repetitive tasks or serving human needs. We also explore the implications of robots in the artistic process and examine how creativity may emerge within robotic art. To address these interrelated inquiries, our study focuses on the practice of robotic art, posing the research question: \textit{How do robotic artists utilize robots in their artistic practice?} We approach this inquiry through the perspectives and experiences of robotic artists, who creatively design, modify, and repurpose robotic systems for artistic expression and exploration.

% The key findings are...
Our findings highlight the social, material, and temporal dimensions of artists' practices that shape their creativity and artistic outcomes. The creation of robotic art is largely a social process, as artists receive both explicit and implicit feedback through the audience's reactions and reception of their work. Simultaneously, the embodiment and malfunctions inherent to robotic systems drive artistic experimentation. The temporal processes of creation and exhibition, beyond just the final product, further enhance the creative value. Our empirical analysis presents how creativity emerges through the interplay of social, material, and temporal interactions among artists, robots, audiences, and the environment.

% The contributions of this work are...
We make two main contributions to HCI in this study. 
First, we elucidate the interactive mechanisms among key actors---human creators, machines, audiences, and environments---within the practice of robotic art, a topic that remains underexplored in HCI. Our findings reveal the significance of sociality (e.g., interactions between artists and audiences), materiality (e.g., the embodiment and malfunctions of robots), and temporality (e.g., the processes of creation and exhibition) in shaping creative values. We propose that these three facets are central to the creative process and facilitate the emergence of creativity in robotic art.
Second, drawing from the findings, we offer implications for \textit{socially informed}, \textit{material-attentive}, and \textit{process-oriented} creation with computing systems. We suggest leveraging these three aspects to enhance creativity and the creative experience. Specifically, we discuss the value of incorporating implicit audience feedback, designing with technical malfunctions, and focusing on the post-creation process to foster alternative creative experiences with machines~\cite{alter2010designing, juarez2022glitch}.






\section{Related Work}
\paragraph{Uncertainty-based hallucination detection methods.}
Various approaches have been proposed to detect hallucinated content in LLMs generation.
Unlike other methods that require external knowledge sources for fact-checking~\citep{gou2024critic, chen-etal-2024-complex, min-etal-2023-factscore, huo2023retrieving}, uncertainty-based approaches are reference-free and rely only on LLM internal states or behaviors to determine hallucination~\citep{10.1145/3703155}. 
For instance, sampling-based approaches generate multiple responses and measure the diversity in meaning among them~\citep{fomicheva-etal-2020-unsupervised, kuhn2023semantic, lin2024generating}, while density-based approaches approximate the training data distribution and provide probabilities or unnormalized scores to assess how likely a generated response belongs to the distribution~\citep{yoo-etal-2022-detection, ren2023outofdistribution, vazhentsev-etal-2023-hybrid}.

In this paper, we focus on uncertainty quantification methods that rely on token-level likelihood or entropy~\citep{guerreiro-etal-2023-looking, malinin2021uncertainty}. 
Recent works have explored refining likelihood estimation by incorporating semantic relationships or reweighting token importance. For instance, Claim-Conditioned Probability (CCP)~\citep{fadeeva-etal-2024-fact} was introduced to recalculate likelihood according to semantical equivalence; while \citet{zhang-etal-2023-enhancing-uncertainty} and \citet{duan-etal-2024-shifting} adjust token weights to better convey meaning in uncertainty aggregation. \emph{Although these approaches leverage token-level information, they are typically evaluated at the sentence level, raising questions about their reliability}. To address this, we conduct a comprehensive analysis of entity-level hallucination detection for finer-grained performance insights.


\paragraph{Fine-grained hallucination detection benchmark.}

Most hallucination detection benchmarks are in sentence or paragraph level. For example, CoQA~\citep{reddy-etal-2019-coqa}, TriviaQA~\citep{joshi-etal-2017-triviaqa}, TruthfulQA~\citep{lin-etal-2022-truthfulqa}, and HaluEval~\citep{li-etal-2023-halueval}. These benchmarks classify each generated response as either hallucinated or correct. However, instance-level detection cannot pinpoint specific hallucinated content, which is crucial for correcting misinformation~\citep{cattan2024localizingfactualinconsistenciesattributable}. This limitation becomes particularly problematic in long-form text, where a single response often combines supported and unsupported information, making binary quality judgments inadequate~\citep{min-etal-2023-factscore}.

To address these challenges, recent works have advanced benchmarks for more granular hallucination detection. For example, \citet{min-etal-2023-factscore} introduced \textsc{FActScore}, which decomposes LLM-generated text into atomic facts---short sentences conveying a single piece of information---for more precise evaluation. In parallel, \citet{cattan2024localizingfactualinconsistenciesattributable} introduced \textsc{QASemConsistency}, decomposing LLM generated text with QA-SRL, a semantic formalism, to form simple QA pairs, where each QA pair represent one verifiable fact. \emph{However, these methods do not enable entity-level hallucination detection, as they lack explicit entity-level labeling (hallucinated or not) in the original generated text}.  
Beyond decomposition-based approaches, datasets like \textsc{HaDes}~\citep{liu-etal-2022-token} and CLIFF~\citep{cao-wang-2021-cliff} create token-level hallucinated content by perturbing human-written text, allowing token-level annotation on the same text. These perturbed hallucinated content, however, could be unrealistic, biased, and overly synthetic due to the limitations of models they used to perturb words. 
To bridge this gap, we create a new dataset with entity-level hallucination labels on the same LLMs generated text. This allows us to evaluate uncertainty-based hallucination detection approaches on a finer-grained level and analyze their reliability.








\begin{figure*}
  \centering
  \includegraphics[width=0.92\textwidth]{figures/overview.png}
  \caption{Overview.}
  \label{fig: system_overview}
\end{figure*}

\section{\goodname~Overview}
\goodname~enhances safety of LLM inputs and outputs while improving their quality. Specifically, it achieves two goals, 1) all user inputs are safe, contextually grounded, and effectively processed, such that the inputs to the LLMs are of high-quality and informative; and 2) the output generated by the LLMs are evaluated and enhanced, such that the outputs passed to users can be both relevant and of high quality. 
The pipeline can be partitioned into two parts, including 
1) processing before LLM inference that enhances user queries, and 2) processing after LLM inference that detects undesired content and handle them properly. We overview our pipeline in Figure~\ref{fig: system_overview}.


\noindent\underline{\textit{Pre-inference processing. }}
Before sending user queries to LLMs, \goodname~detects if there are any safety issues in the queries with \detection~and ground the queries with context knowledge with \grounding. 
\detection~monitors user inputs to identify and reject queries that might be unsafe. The monitoring includes typical safety checks, including toxicity, stereotypes, threats, obscenities, prompt injection attacks, etc. Any form of unsafe content will lead to the queries being rejected. 
Inputs that pass this initial safety check are grounded with context with \grounding, where the user query is contextualized and enhanced with relevant knowledge retrieved from the vector data storage. By equipping the query with some context knowledge, the LLM can do inference with enriched information, thus can reduce hallucinations when generating responses. The details of \detection~ and \grounding~will be introduced in \S\ref{sec:safety_detector} and \S\ref{sec:grounding}, respectively.




\noindent\underline{\textit{Post-inference processing. }}
Upon LLM finishing inference, \detection~detects safety issues in the LLM outputs, specifically, hallucinations. This is because LLM applications typically leverages well-developed LLMs or APIs, such as LLaMA~\citep{touvron2023llama} and ChatGPT API~\citep{openai-data-paper}, which are generally safe and less likely to generate toxic or other unsafe content, while hallucinations occur frequently. \detection~identifies hallucinations and provides reasons for the hallucinations, such that \goodname~can utilize the reasoning for later refinement of the LLM outputs. To achieve goal, \goodname~employs a text generation model to generate explainable results, and adjusts the loss function during training to ensure the model to produce classification results. 
After \detection~finishes detection, \fixing~fixes the problematic content or aligns the outputs with some rule-based wrappers to meet user expectations. 
If the outputs are difficult to fix, e.g., hallucinated responses, 
\fixing~will call a fixing model to fix the answers. Details about \fixing~can be found in \S\ref{sec:fixing}.



\vspace{-1em}

\section{\goodname~\detection}\label{sec:safety_detector}


\detection~addresses unsafe inputs and inappropriate LLM responses to ensure that both the user queries provided to the models and the LLM outputs are safe and free from misinformation. 
\subsection{Unsafe Input Detection}\label{sec: unsafe_input_detection}

We developed a model to detect unsafe contents in user queries before they are processed by LLMs for inference.
While existing approaches categorize unsafe content into various types (e.g.,  toxicity, prompt injection, stereotypes, harassment, threats, identity attacks, and violence)~\citep{openai-data-paper,Wang2023DecodingTrustAC,Detoxify}, our method employs a unified, binary classification model finetuned based on our opensourced LLM~\citep{fox}, classifying content as  safe or unsafe.


This strategy offers several key advantages, as follows: \textit{i}) By fine-tuning our base model, which has been trained on vast amounts of data, the classification model can leverage pre-existing knowledge relevant to safety detection.
\textit{ii}) A binary classification of ``safe'' and ``unsafe'' is both efficient and sufficient for LLM services, as any unsafe query should be rejected, regardless of the specific risk.
\textit{iii}) This approach avoids the complexities and potential inaccuracies of categorizing overlapping or ambiguous types of unsafe content in some publicly available datasets. For example, toxicity toward minority groups could also be classified as bias, but current datasets may inadequately capture such nuances.
\textit{iv}) Using straightforward code logic, we can transform public datasets for safety detection into clear safe/unsafe labels, minimizing ambiguity and ensuring high-quality training data.





The biggest challenge in training such model is the discrepancy between the training data and real-world user query distributions, where using traditional datasets alone can result in poor performance due to their divergence from actual user queries~\citep{openai-data-paper}.
To mitigate these issues, we integrated data of various domains and contexts to better simulate the variety of unsafe queries that users might submit.
We crafted a training dataset
by combining samples randomly selected from 15 public datasets, as will be introduced in Table~\ref{tab:exp_datasets} in \S\ref{sec: exp}. 
Such a dataset captures  diverse contents in user inputs in practice, thus can be more representative on potential real-world inputs. 


\subsection{Hallucination Detection and Reasoning}\label{sec: hallucination_detection_and_reasoning}


\begin{algorithm}[!t]
\textbf{Inputs:} 
$\mathcal{D}$: a training dataset that contains ``context'', ``inputs'', ``llm\_answer'', and ``labels'' for hallucination; %\textit{response\_template}: for indicating the start of the response, e.g., ``{\#\#\# Response:}''; 
\textit{\text{prompt}\_\text{template}}: for formulating the hallucination detection data, see Figure~\ref{fig: training_data_example}; \textit{\text{GPT}\_\textit{reasoning}\_\text{template}}: for generating prompts for GPT API, see Figure~\ref{fig: training_data_example}.

\textbf{Outputs:} $\mathcal{D}_t$: the training dataset.


\nl{\bf Function $\boldsymbol{\mathit{process\_data}(\mathcal{D})}$} \nllabel{ln:function_process_data}
\Begin{

\nl $\mathcal{D}_t\leftarrow \phi$

\nl \For{$d \in $ $\mathcal{D}$}{

\nl \eIf{$\mathit{is}\_\mathit{hallucination}$($d$)}{

\nl $\mathit{halu}\_\mathit{reason}\leftarrow\mathit{GPT}\_\mathit{API}$(\textit{GPT}\_\textit{reasoning}\_\textit{template}($d$[``$\mathit{question}$''], $d$[``$\mathit{context}$''], $d$[``$\mathit{llm}\_\mathit{answer}$'']))

\nl $\mathit{response}\leftarrow$ ``Yes, '' + $\mathit{halu}\_\mathit{reason}$


\nl $d^\prime\leftarrow$\textit{prompt}\_\textit{template}($d$[``$\mathit{question}$''], $d$[``$\mathit{context}$''], $d$[``$\mathit{llm}\_\mathit{answer}$''], $\mathit{response}$)

}{

\nl $d^\prime\leftarrow$\textit{prompt}\_\textit{template}($d$[``$\mathit{question}$''], $d$[``$\mathit{context}$''], $d$[``$\mathit{llm}\_\mathit{answer}$''], ``$\mathit{No.}$'')


}

\nl $\mathcal{D}_t$.$\mathit{add}(d^\prime)$

}

\nl \textbf{return $\mathcal{D}_t$}
}

\caption{Hallucination detection training data processing.}
\label{alg:hallucination_data_processing}
\end{algorithm}







Hallucinations occur when the LLM generates responses that is inaccurate, fabricated, or irrelevant~\citep{filippova2020controlled, maynez2020faithfulness,huang2023survey,rawte2023survey}.
Despite appearing coherent and plausible, hallucinated LLM responses are unreliable, often containing fabricated, misleading information that is  
divergent from the user input, thus fail to meet users' expectations and severely undermine the trustworthiness and utility of the LLM applications.
While grounding can mitigate hallucinations by contextualizing user inputs and enriching the informativeness of user queries, it cannot eliminate hallucinations entirely. 
This is because hallucinations stem from nearly every aspects of LLM training and inference, such as low-quality training data~\citep{lin2021truthfulqa,kang2023impact} and %LLM memorizing training data~\citep{lin-etal-2022-truthfulqa}, 
randomness of sampling strategies~\citep{chuang2023dola}, and moreover, the very nature probabilistic properties of LLMs. 

Effectively handling hallucinations in LLM responses is both crucial and challenging for producing high-quality LLM responses. 
Existing works that detect presence of hallucinations are insufficient~\citep{manakul2023selfcheckgpt,liu2021token}. To provide high-quality responses to users,  we should handle the detected hallucinations properly, i.e., obtaining the explanations for the hallucinations in the LLM responses and further, fixing the hallucinated responses if possible. 


To this end, we propose utilizing our own LLM, Fox-1, as base model~\citep{fox} to finetune a 
hallucination detection model  for detecting hallucinated content and providing explanations, and further, facilitating the subsequent \fixing~in \S\ref{sec:fixing}. The design of the model has the following advantages: \textit{i}) \textit{classification}: it identifies the presence of hallucinations in the LLM output; and \textit{ii}) \textit{reasoning}: it generates explanations for the hallucinated contents, offering insights for the subsequent correction in \fixing; \textit{iii}) \textit{simultaneous classification and reasoning}: it process \textit{i}) and \textit{ii}) at the same time, which saves computation cost and improves efficiency; and \textit{iv}) \textit{vast pre-training data}: it leverages pre-existing knowledge
on hallucination in the base model, which may potentially benefit hallucination detection and reasoning.



\begin{figure*}
  \centering
  \includegraphics[width=\textwidth]{figures/training_data_example.pdf}
  \caption{Prompt templates and sample training data for hallucination detection and reasoning.}
  \label{fig: training_data_example}
\end{figure*}

\textbf{Training. }
We feed our base model with hallucination dataset to train a model for both detecting and reasoning for the hallucination. %Thus, the model should have text generation capabilities. 
However, public available datasets for hallucinated LLM responses are mainly classification datasets with texts and labels, e.g., HaluEval~\citep{li2023halueval}. To address this, we utilize the GPT4 API~\citep{openai-data-paper} to generate explanations for hallucinated contents, and
define a prompt template
to create structured prompts based on the classification data to make it suitable for classification and reasoning simultaneously. 
We demonstrate the prompt templates and sample training data in Figure~\ref{fig: training_data_example}, and summarize data processing in Algorithm~\ref{alg:hallucination_data_processing}. 







\begin{algorithm}[!t]
\textbf{Inputs:}
$\mathcal{M}$: hallucination detection model; 
$\mathit{tokenizer}$: tokenizer for $\mathcal{M}$; $\mathit{q}$: a query submitted by users;
$\mathit{context}$: the context to answer the question; retrieved from vector data storage;
$a$: the answer returned by an LLM for the question; \textit{inference}\_\textit{prompt}\_\textit{template}: see Figure~\ref{fig: training_data_example}. 



\nl{\bf Function $\boldsymbol{\mathit{inference}(\mathcal{M}, \mathit{q}, \mathit{context}, a, k)}$} \nllabel{ln:inference}
\Begin{


\nl $\mathit{prompt}\leftarrow$\textit{inference}\_\textit{prompt}\_\textit{template}($\mathit{q}$, $\mathit{context}$, $a$)

\nl $\mathit{tokenized}\_\mathit{prompt}\leftarrow \mathit{tokenizer}$($\mathit{prompt}$)

\nl $\mathit{halu}\_\mathit{res}\leftarrow$ 
$\mathcal{M}$.$\mathit{generate}$($\mathit{tokenized}\_\mathit{prompt}$)

\nl $\mathit{first}\_\mathit{word}\_\mathit{logits}\leftarrow \mathit{halu}\_\mathit{res}$.$\mathit{logits}$[0], 

\nl$\mathit{results}\leftarrow\mathit{softmax}$($\mathit{first}\_\mathit{word}\_\mathit{logits}$)

\nl $\mathit{top\_k\_probs}\leftarrow \mathit{top}$($\mathit{results}$, $k$)

\nl $P_{\mathit{halu}}(a)\leftarrow \mathit{compute}\_\mathit{halu}\_\mathit{prob}(\mathit{top}\_k\_\mathit{probs})$ 

% \\\Comment{See Definition~\ref{def:halu_prob}}


\nl \lIf{$P(a)\geq 0.5$}{\textbf{return} True}

\nl{\textbf{return} False}
}

\caption{Hallucination detection model inference.}
\label{alg:inference}
\end{algorithm}


\textbf{Inference. }
We expect the LLM to directly output results whether the LLM response contains hallucinations, \textit{i}.\textit{e}., the first token of outputs to be ``Yes'' or ``No'' as detection results, according to the formatted data sample in Figure~\ref{fig: training_data_example}. However, the first token of the LLM response is probabilistic due to the self-autoregressive nature of decoder-based text generation LLMs. 
To obtain desired outputs, we formulate the text-generation outputs by utilizing the top-$k$ first tokens (and their possibilities) of the outputs to generate classification results. By default, $k$ is 10. 

\begin{definition}[Probability of hallucination]
Let $a$ be an LLM answer, let $\{t_1, ..., t_{k}\}$ be the top-k potential first token, and let $\{p_1, ..., p_{k}\}$  be their top-k probabilities. Let $T$ be a tokenization function, and let $T(\text{"Yes"})$ and $T(\text{"No"})$ be the tokens corresponding to ``Yes'' and ``No'', respectively. The probability of hallucination in $a$ is
$P_\mathit{halu}(a) = \frac{\sum_{i=1}^{k} P(t_i | t_i\in T(\text{"Yes"}))}{\sum_{i=1}^{k} P(t_i | t_i\in T(\text{"Yes"})) + \sum_{i=1}^{k} P(t_i | t_i\in T(\text{"No"}))}
$   
\end{definition}\label{def:halu_prob}

Detection results with $P_{\mathit{halu}}(*)\geq0.5$ indicate the content is classified as ``hallucinated''; otherwise, the content is ``safe''.  The detailed procedure of  inference is described in Algorithm~\ref{alg:inference}.


\section{\goodname~\grounding}\label{sec:grounding}






\goodname~\grounding~enhances the contextual richness and informativeness of user queries by leveraging external knowledge in vector database. Thus, LLMs can utilize such contextual knowledge to generate high-quality outputs, particularly by grounding user queries before they are passed to the LLMs for inference.

To support similarity search over the knowledge data, \goodname~creates vector indexes by vectorizing plaintext knowledge.% This involves vectorizing entire knowledge entries to create vector indexes. 
\goodname~employs two primary methods for indexing: \textit{i}) \textit{{Whole Knowledge Index}} that creates indexes based on each entire data entry in the datasets; and \textit{ii}) \textit{{Key Information Index}} that indexes only the key information in each data entry, i.e., questions in QA datasets. 
Whole Knowledge Index reflects the data distribution and ensurers that the indexed data captures the contextual variety and complexity found in real-world queries, while Key Information Index 
focuses on the core information of each data entry, thus facilitates efficient retrieval of relevant data. 
We evaluate the effectiveness of indexes with \textit{callback}, i.e., the probability of successfully retrieving the original records from a dataset using Top-$k$ queries. 
We experimentally evaluate the indexing methods in \S\ref{sec: exp}.


\begin{definition}[Callback]
    Let $D_v$ be a vector data storage that contains $n$ records, let $Q$ be a plaintext user query set, and let $I(Q)$ be the vector index created based on $Q$. For each query $q\in Q$, let $I_q$ be the vector index created based on $q$, and let $D_v(I_q)$ denote the set of Top-$k$ records returned by querying $D_v$ with $I(q)$, and let $r_q$ denote the most relevant record of $q$ in $D_v$. 
    The callback for Top-$k$ queries on the query set $Q$ is defined as:
$$C_k(Q) = \frac{1}{|Q|} \sum_{q \in Q} [r_q \in D_v(I_q)]$$
where $[\cdot]$ is Iverson Bracket Notation~\citep{iverson1962programming}, equal to 1 if the condition inside is true, and 0 otherwise.
\end{definition}

To ensure effective and informative grounding, 
the distribution of the index should closely align with query patterns, i.e., query distributions. 
By grounding user queries with knowledge retrieved with a proper index, 
the LLMs can generate contextually appropriate responses, and further, reduce hallucinations and improve the quality of the responses. 



\begin{figure*}
  \centering
  \includegraphics[width=0.86\textwidth]{figures/fixing_data_example.pdf}
  \caption{Prompt templates and sample training data for \fixing.}
  \label{fig: training_data_example_fixing}
\end{figure*}

\section{\goodname~\customization}\label{sec:customization}

\goodname~\customization~utilizes lightweight wrappers to flexibly edit or customize LLM outputs to fix some small errors or enhancing the format of the answer. The wrappers integrate code-based rules, APIs, web searches, and small models to efficiently handle editing and customization tasks according to user-defined protocols. \goodname~\customization~ offers several key advantages. It facilitates rapid development and deployment of user-defined protocols, which crucial in production environments where real-time adjustments are necessary. In scenarios where training or fine-tuning LLMs is unfeasible due to time or resource constraints, this method provides an alternative for immediate output customization. Moreover, the wrappers enable flexible incorporation  of various tools and data sources, which enhances the applicability of  \goodname~and reduces resource-intensive LLM calls. 



\begin{example}[Warning URLs]\label{example:waring_urls}
The objective was to detect if LLM outputs contain URLs and prepend a warning message of the unsafe URLs at the beginning of the LLM outputs. \customization~should check the safety of the URLs founded,  i.e., whether they are malicious or unreachable, and includes such information in the warning if they were unsafe.
\customization~utilizes a regular expression pattern 
to identify URLs within the text. Upon URLs founded, \customization~calls APIs for detecting phishing URLs, such as Google SafeBrowsing~\citep{google-safe-browsing}, and assess the accessibility of the benign URL by issuing web requests. Malicious URLs, as well as unreachable URLs that return status codes of 4XX, are added in the warning at the beginning of the LLM outputs.
\end{example}


Note that the task in Example~\ref{example:waring_urls} cannot be achieved through prompt engineering when querying LLMs, as the warning must appear at the beginning, and LLMs generate content token by token, making later content unpredictable.
We use the following example to illustrate this property, and experimentally evaluate the efficiency of \customization~wrappers in \textbf{Exp 4} in~\S\ref{sec: exp}. 


\begin{example}
\label{ex:llm-limitation}
We present a concrete demonstration of tasks that cannot be reliably accomplished through prompt engineering alone, due to the token-by-token generation mechanism inherent in LLMs. This sequential generation process fundamentally precludes anticipatory knowledge of future token occurrences during text production.
Consider the following prompt submitted to GPT-4:



\begingroup
{
% \addtolength\leftmargini{-0.2cm}
\begin{quote}
\ttfamily
\noindent Write an English poem about a rabbit; please include information at the beginning of the poem about how many times the word ``rabbit'' appears in the poem.
\end{quote}
}
\endgroup

The generated response (shown below) claims four occurrences of ``rabbit,'' while actual analysis reveals five occurrences:
\begingroup
{ 
\begin{quote}
\ttfamily
(The word ``rabbit'' appears 4 times in this poem.) 

In a field where soft grasses sway and dip, 
a \textcolor{red}{rabbit} hopped beneath the moon's silver whip. 
With ears like twin sails in the cool night air, 
each twitch and turn done with utmost care. \\

This \textcolor{red}{rabbit}, cloaked in twilight's gentle hug, 
nestled in shadows, a cozy, earthen snug. 
Dreams of dandelion fields in sunny sprawl, 
where no shadows loom, nor fears enthrall. \\

With a nose that twitches at secrets untold, 
and paws that dance over leaves of gold, 
the \textcolor{red}{rabbit} finds joy in the simplest of things—the warmth of the sun, the freedom of springs. \\

But as dawn breaks and stars bid goodbye, 
back to the burrow, this \textcolor{red}{rabbit} must fly. 
For the world is wide and wolves roam free,
yet in his burrow, safe as can be, this \textcolor{red}{rabbit} shall be.
\end{quote}
}
\endgroup

This discrepancy arises from the LLM's inability to maintain an accurate token count during generation while simultaneously composing coherent text. The example underscores the necessity of post-inference wrappers to detect and correct such errors in model outputs.
\end{example}






\section{\goodname~\fixing}\label{sec:fixing}


\goodname~\fixing~addresses errors in the LLM  outputs that are challenging to resolve through editing with wrappers in~\customization, particularly, hallucinated content. \fixing~analyzes and corrects the hallucinated output based on the reason for the hallucinations generated by the hallucination detection model. 


\goodname~\fixing~takes several key inputs, including the user's original query, the context retrieved with \grounding, the hallucinated responses generated by the LLM, as well as the reason for hallucination. 
Given these inputs, \fixing~corrects the flawed output according to the hallucination reason.
To enable \fixing~to handle hallucinations effectively, we  leverage the same hallucination detection dataset as \detection, i.e., HaluEval~\citep{li2023halueval}, that contains user questions, contexts, hallucinated LLM answers, and correct answers. 
We also designed a customized data template that incorporates the information. The data templates for training, inference, as well as an example for the training data, are demonstrated in Figure~\ref{fig: training_data_example_fixing}. 




\section{Experiments}\label{sec: exp}

\section{Experiments}

\subsection{Datasets}

\textbf{MSMARCO}.
We utilized the MS MARCO Passage Ranking dataset as the data source to evaluate the ability of our method to improve document rankings under more challenging topic-query tasks. Specifically, we assessed whether our method could significantly enhance the ranking of documents by the retrieval model within a RAG system.

To construct topic-lists for evaluation, we applied a K-means clustering algorithm to group similar queries, forming topics that each contained a series of related queries. To further evaluate the performance of our method under extreme topic-query scenarios, we applied an intra-topic similarity filtering process. Only topics with queries exhibiting high semantic diversity and containing a sufficient number of queries were retained.

This process resulted in 29 topics, with each topic containing an average of 22.28 queries. The average similarity score within each topic was approximately 0.5, indicating sufficient diversity among queries to ensure a rigorous evaluation. This curated dataset enabled us to test the robustness of our method in handling highly diverse and challenging topic-query tasks within a RAG system.

\textbf{PROCON}.
To conduct our experiments, we utilized controversial topic data scraped from the PROCON.ORG website. This dataset includes over 80 topics covering various domains, such as society, health, government, and education. Each topic is discussed from two stance labels \{\textit{PRO (support), CON (oppose)}\}, with passages arguing from these perspectives.

To simulate real-world user interactions with a RAG system, we instructed a large language model (GPT-4o) to act as a user and generate 40 potential sub-queries for each topic. These sub-queries were designed to reflect the diverse questions and concerns users might raise when exploring a specific controversial topic. 

After generating the sub-queries, we applied a similarity filtering process to ensure diversity by retaining only those with a similarity score below approximately 0.85. The filtering step effectively removed redundant queries while preserving a wide range of perspectives. As a result, the final set of topic-queries achieved an average similarity score of approximately 0.7, ensuring that the queries were sufficiently diverse yet semantically relevant. This process provided a robust and balanced sub-queries set for evaluation.


\subsection{Experiment Details}
The specific setting details for the Topic-queries RAG manipulation experiment are as follows:

(1) Black-box RAG. We represent the black-box RAG process as \( \text{RAG}_{\text{black}} \). The RAG framework is Conversational RAG from LangChain. The LLMs adopted in RAG are the open-source models Meta-Llama-3.1-8B-Instruct (Llama3.1), Qwen-2.5-7B-Instruct(Qwen2.5). The system prompt and additional detailed descriptions are provided in Appendix~\ref{exp-detail}.

(2) Retrieval Model Specification. We benchmark three dominant dense retrievers—Contriever \cite{gao2021unsupervised}, DPR \cite{karpukhin-etal-2020-dense}, and ANCE \cite{xiong2020approximate}.By convention, we use dot product between the embedding vectors of questions(queries) and candidate documents as their similarity score \(R\) in the ranking. 


\label{opinion-classfication}
(3) Opinion classification. We use Qwen2.5-Instruct-72B as the opinion classifier. Qwen2.5-Instruct-72B, due to its large parameter size, is capable of accurately identifying and classifying opinions within text.

(4) Experimental parameters. In knowledge-guided attack process, we set the maximum editing distance $\epsilon$ to 0.2, the semantic similarity threshold $\lambda$ to 0.85, and the number of iterations $N$ to 5. For adversarial trigger generation, we used a beam size of 3, top-$k$ of 10, a batch size of 32, a temperature of 1.0, a learning rate of 0.005, and a sequence length of 10. In RAG\textsubscript{black}, $k$ (the number of retrieved documents) is set to 3, with the LLM temperature also fixed at 1.0 to mirror real-world conditions.

(5) Poisoned Target. For the PROCON dataset, to investigate the manipulation performance under more challenging conditions, we performed relevance ranking for the documents with respect to each topic-query set $Q$ and the target stance $S_t$ . From the ranked list, we selected the last five documents (i.e., those with the lowest relevance) as the target poisoned documents.
For the MS MARCO dataset, we utilized the top-1000 relevance-ranked passage list for each topic-query set. From this list, we selected the passage with the lowest average rank as the target passage. This approach ensures that the evaluation focuses on passages that are least relevant to the target queries, thus providing a more rigorous benchmark for the proposed method.

(6) Experimental environment. All our experiments are conducted in Python 3.8 environment and run on a NVIDIA DGX H100 GPU. 

\subsection{Research Questions}

We propose four research questions to evaluate the effectiveness of our method in the topic-queries task, focusing on black-box NRM attacks and opinion manipulation to RAGs.

\textbf{RQ1}: Can Topic-FlipRAG significantly enhance the rankings of target documents in the RAG retriever for topic-queries?

\textbf{RQ2}: To what extent does Topic-FlipRAG affect the answers generated by the target RAG systems?

\textbf{RQ3}: Does topic-oriented opinion manipulation significantly impact users' perceptions of controversial topics?

\textbf{RQ4}: How robust does Topic-FlipRAG against exisiting mitigation mechanism?

\subsection{Baseline Settings}
To assess the effectiveness of our proposed method, we compare it against adversarial attack baselines designed for black-box, topic-oriented RAG scenarios, ensuring minimal modifications to the original documents. We exclude BadRAG\cite{xue2024badrag}, a backdoor RAG attack limited to white-box scenarios, and topic-IR-attack\cite{liu2023topic}, as its incomplete implementation prevents reliable replication.
For the selected baseline methods, we adapted them to meet the requirements of our task while preserving their core components. A brief overview of the baseline methods is provided below, with detailed descriptions available in Appendix~\ref{baselines-details}.

\textbf{PoisonedRAG.}
Zou et al.\cite{zou2024poisonedrag} propose an approach adaptable to both black-box and white-box settings. For our task, we employ its black-box strategy by inserting a randomly chosen query from the topic-queries set \( Q \) at the beginning of each document.

\textbf{PAT.}
This gradient-based adversarial retrieval attack uses a pairwise loss function to generate triggers that meet fluency and coherence constraints. We adapt PAT to produce triggers \( T_{\text{pat}} \) for target documents within the topic-queries set, evaluating their effectiveness under black-box conditions.


\textbf{Collision.}
This method generates adversarial paragraphs (collisions) via gradient-based optimization to produce content semantically aligned with the target query. In a topic-queries context, we create collisions for the entire topic-queries set and examine their transfer performance on black-box RAG retrievers.

These baseline methods provide benchmarks for comparing the efficacy of our approach in a fully black-box, topic-oriented RAG attack scenario.

\subsection{Evaluation Metrics}

For \textbf{RQ1}, we focus on ranking manipulation. We measure the average proportion of target opinions in top-3 rankings before and after manipulation (\(\text{Top3}_{\text{ori}}, \text{Top3}_{\text{att}}\)) and define top3-v as their difference. We also employ the Ranking Attack Success Rate (RASR), reflecting how often target documents are successfully boosted, and Boost Rank (BRank), denoting the average rank improvement for all target documents. Lastly, we report the proportion of target documents in the Top-50 and Top-500 positions to indicate how effectively they are pushed toward higher rankings.

\textbf{top3-v.} Computed by subtracting \(\text{Top3}_{\text{ori}}\) from \(\text{Top3}_{\text{att}}\), top3-v ranges from -1 to 1. A positive value signifies a successful increase of the target opinion in top-3 results, while a negative value indicates a detrimental effect.

\textbf{Ranking Attack Success Rate (RASR).} RASR captures how frequently target documents are successfully boosted in each query’s ranking. Higher values indicate greater attack effectiveness.

\textbf{Boost Rank (BRank).} BRank is the average rank improvement for all target documents under each query. A target document contributes negatively if its rank is unintentionally lowered.

\textbf{Top-50, Top-500.} These metrics represent the percentage of target documents that move into specific ranking thresholds in the MS MARCO Dataset after manipulation. Higher percentages imply more effective promotion of target documents. 


For \textbf{RQ2}, we employ Average Stance Variation (ASV) to assess how significantly our opinion manipulation influences the LLM’s responses in a black-box RAG. To address the natural variability of controversial topics and the inherent instability of large language models, we also propose Real Adjusted ASV (\(\Delta\)-ASV).

\textbf{Average Stance Variation (ASV).}
ASV is defined as the absolute difference between the manipulated opinion score and the original opinion score assigned to an LLM response (0 = opposing, 1 = neutral, 2 = supporting). A higher ASV signifies a more pronounced shift in polarity and hence greater manipulation effectiveness.

\textbf{Real Adjusted ASV ($\Delta$-ASV)}. To account for the inherent variability of controversial topics and the instability of large language models, we measure the baseline ASV in a clean state, denoted as ASV\textsubscript{clean} (calculated without adversarial manipulation). $\Delta$-ASV is then obtained by subtracting ASV\textsubscript{clean} from the manipulated ASV, i.e., \( \text{$\Delta$-ASV} = \text{ASV} - \text{ASV\textsubscript{clean}} \). This adjustment ensures that $\Delta$-ASV reflects the true impact of adversarial manipulation by eliminating the influence of natural stance variation. It reflects the extent to which the polarity of the RAG-system outputs is affected by the manipulation.  A positive $\Delta$-ASV indicates a significant shift in the opinion polarity due to manipulation, with larger values representing greater manipulation effectiveness.


\section{Conclusion}\label{sec:conclusion}
\section{Conclusion}
We introduced \methodname, an effective training framework defending against MIAs for LLMs. The extensive experiments demonstrate its robustness in protecting privacy while maintaining strong language modeling performance across various datasets and architectures. Although our study focuses on fine-tuning due to computational constraints, \methodname can be seamlessly applied to large-scale pretraining, as done in prior selective pretraining work~\cite{lin2024not}. By categorizing tokens and treating them appropriately, \methodname opens a novel pathway for MIA defense. Future work can explore improved token selection strategies and multi-objective training approaches.


\bibliographystyle{ACM-Reference-Format}
\bibliography{reference}



\end{document}
