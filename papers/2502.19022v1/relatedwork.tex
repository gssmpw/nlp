\section{Related Work}
BV-logic has been related to causality in quantum physics in at least two ways before: by connection to the causal set approach to quantum gravity \cite{blute_quantum_causal_dynamics,blute_logical_basis}, and by connection to the study of higher-order quantum theory and indefinite causal orders \cite{kissinger_caus,SimmonsKissinger2022}.
The former \cite{blute_quantum_causal_dynamics} introduced the study of \textit{discrete quantum causal dynamics} (DQCD) to track the causal evolution of density matrices through foliations of a classical spacetime.
Spacetime events are considered to be vertices of a directed acyclic graph (DAG) with the edges representing causal links between the events.
Quantum systems are then placed over these DAGs with their evolutions compatible with the causal structure.
It was later demonstrated that the structure of these quantum causal evolutions could be interpreted as models of the system BV with the intuitive interpretation of the connectives given in equation \eqref{eq:bv_logic_structs} \cite{blute_logical_basis}.
Two concrete examples of categories which gave semantics to this logical syntax were given: the probabilistic and quantum coherence spaces of \cite{girard_coherence_spaces}.
DQCD is, however, restricted to first order: it tracks the evolution of states through spacetime but does not carry a notion of event as an intervention or a hole, and likewise does not carry a notion of higher-order process. 


The other notable connection between BV-logic, causality and quantum theory was developed in \cite{kissinger_caus,SimmonsKissinger2022}.
This introduced the Caus-construction (as an instance of double gluing) and demonstrated that the higher-order processes of finite dimensional classical and quantum theory form BV-categories.
Within these categories are not only the expected first-order evolutions of systems but also the higher-order maps arising in the supermap framework.
What is surprising however is the specificity of the construction, with the Caus-construction only working in particular instances and not permitting the study of spatio-temporal correlations between systems from arbitrary physical theories.
This includes infinite dimensional quantum theory, in addition to a myriad of other exotic models stemming from Operational and Categorical Probabilistic Theories \cite{chiribella_purification,gogioso_cpt}.

Another approach to modelling quantum supermaps was presented in \cite{hefford_supermaps} developing on \cite{wilson_locality,wilson_polycategories}.
A general definition for higher-order transformations over any symmetric monoidal category was first presented in \cite{wilson_locality} and demonstrated to recover the usual definition in the case of finite dimensional quantum theory.
This definition was reinterpreted in \cite{hefford_supermaps} to show that the required algebraic laws are precisely those of a strong natural transformation between strong profunctors.
With this identification in place and by noting that strong endoprofunctors form a duoidal category \cite{garner,earnshaw,roman_thesis}, it was possible to endow the earlier models of \cite{wilson_locality,wilson_polycategories} with tensor products $\otimes$ and $\seq$ which faithfully model the causally ordered and non-ordered supermaps.
Lacking from this model was any richer logic: it was noted that the natural candidates for a negation or $\parr$ failed to be involutive and associative respectively, and thus failed to give a model of MLL or indeed BV.

The search for models of BV-logic leads us naturally to the study of its categorical semantics, which appears to have not been particularly deeply studied in the literature.
Although a notion of BV-category was suggested in \cite{blute_BV}, even the basic notions of BV-functors and BV-natural transformations are absent from the literature, and these are vital not only in the development of the theory of BV-categories themselves, but also in comparisons between physical models of spatio-temporal correlations exhibiting this logic.
We also note that, as far as we are aware there is yet to be a definitive notion of a model of BV-logic.
While this is not a topic we are going to address here, the BV-categories of \cite{blute_BV} can be placed on a firm algebraic footing and appear to fulfil at least the minimal of demands to interpret the system BV.
Sadly, no general methods for constructing BV-categories have appeared in the literature, and the only widely known non-trivial examples (where $\otimes$, $\seq$ and $\parr$ all differ) are those we have earlier discussed.
The closest we have found to a result in this direction appears in \cite{atkey_bv} where it is proven that the Chu construction \cite{chu,barr} sends duoidal pomonoids to BV-pomonoids.
To categorify this result would require the upgrading of pomonoids to monoidal categories while dealing with all the additional coherence data.

This leaves us with two thoughts: 1) that there should exist some deeper connection between monoidal categories and models of BV-logic, and 2) that such a connection would not only be a rich source of models of the system BV, but that this source could be used to develop the study of spatio-temporal correlations in less well understood domains such as infinite dimensional quantum theory or in generalised probabilistic theories.
Encouraged by the relevance of these thoughts to both physics and logic, we aim to study the question of how to construct from any monoidal category, its associated BV-category of events.