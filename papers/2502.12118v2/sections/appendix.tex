\part*{Appendices}













\textbf{A. Proofs from Section 5.}

\textbf{B. Additional Related Work.}

\textbf{C. Additional Experiments in the Didactic Setup.}

\textbf{D. Additional Experiments on MATH.}


\section{Proofs from Section~\ref{sec:theory-compare-vfree-vbased}}
\label{sec:proofs}



\subsection{Useful Lemmas}
\label{subsec:useful-lemmas}



For a pair of probability measures $P$ and $Q$, we define the total variation distance as $\tv{P}{Q}=\frac{1}{2}\int\abs{\mathrm{d}P-\mathrm{d}Q}$, and define the $\chi^2$-divergence by $\chisq{P}{Q}=\int\frac{(\mathrm{d}Q-\mathrm{d}Q)^2}{\mathrm{d}Q}$ if $P\ll Q$ and $\chi^2\paren{P \| Q}=+\infty$ otherwise. We define the KL divergence by $\kl{P}{Q}= \int{} \mathrm{d}P\log\paren{\frac{\mathrm{d}P}{\mathrm{d}Q}}$ if $P\ll Q$ and $\kl{P}{Q}=+\infty$ otherwise.


  \begin{lemma}[\cite{polyanskiy2014lecture}]
    \label{lem:pinsker}
    The following inequalities hold:
    \begin{itemize}
    \item $\tv{P}{Q}\leq \hell{P}{Q}\leq 2\tv{P}{Q}$.%
    \item $\frac{1}{6}\hell{P}{Q}
      \leq \chisq{P}{\frac{1}{2}(P+Q)}
      \leq  \hell{P}{Q} 
      $.
    \item $\tv{P}{Q} \leq \sqrt{\frac{1}{2} \kl{P}{Q}}$
    \end{itemize}
  \end{lemma}


\begin{lemma}[Change of measure \cite{polyanskiy2014lecture,foster2024behavior}] \label{lemma:CoM} Let $P$ and $Q$ be probability distributions over a measurable space $(\mathcal{Y}, \mathscr{F})$. Then for all functions $h: \mathcal{Y} \rightarrow \mathbb{R}$,
\label{lem:change-of-measure}
\begin{align}
    \left|\mathbb{E}_{P} [h(Y)]-\mathbb{E}_{Q} [h(Y)]\right| &\leq \sqrt{\Var_Q \left[h(Y)\right] \cdot \chisq{P}{Q}} \tag{$\chi^2$-CoM} \label{eq:chi2CoM}\\
    &\leq \sqrt{\frac{1}{2}\left(\mathbb{E}_P \left[h^2(Y )\right]+\mathbb{E}_Q \left[h^2(Y)\right]\right) \cdot D_{\mathrm{H}}^2(P,Q)} \label{eq:HellCoM}\tag{$\mathbb{H}$-CoM}
\end{align}
\end{lemma}

\begin{lemma}[Total expert heterogeneity]
  \label{lem:total-variance}
  For any policy $\pi$, recall the definition of heterogeneity in Definition~\ref{def:exp-heterogeneity}. For this definition of heterogeneity the following equivalance to the expected conditional variance of rewards is true:
  \begin{align*}
    \sigma^2_\pi = \E_{\bx \sim \rho} \Var_{\tau \sim \pi(\cdot \mid \bx)} \brck{r(\tau)}.  
  \end{align*}
\end{lemma}
\begin{proof}
Let us begin by recalling the definition of $\sigma^2_\pi$. 
 \begin{align*}    
    \sigma^2_{\pi} \eqdef \sum_{h=1}^{H} \E_{\bs_h \sim d^{\pi}_{h}}\brck{\Var_{\pi(\cdot \mid \bs_h)}\brck{Q_{\pi}(\bs_h, a_h)}}.
    \end{align*}
Now let us expand $\Var_{\pi(\cdot \mid \bs_h)}\brck{Q_{\pi}(\bs_h, a_h)}$ in the following way.
\begin{align*}
&\Var_{\pi}\brck{\sum_{h^\prime = h}^H r(\bs_{h^\prime}, a_{h^\prime}) \middle| \bs_h} \\
    &= \Var_{\pi}\brck{r(\bs_h, a_h) + \sum_{h^\prime = h+1}^H r(\bs_{h^\prime}, a_{h^\prime}) \middle| \bs_h} 
\end{align*}
\begin{align*}
     &= \E_{\pi}\brck{\paren{r(\bs_h, a_h) - V_\pi(\bs_h) + \sum_{h^\prime = h+1}^H r(\bs_{h^\prime}, a_{h^\prime})}^2 \middle| \bs_h} \\
    &=  \E_{\pi}\brck{\paren{r(\bs_h, a_h) + V_\pi(\bs_{h+1}) - V_\pi(\bs_h) + \sum_{h^\prime = h+1}^H r(\bs_{h^\prime}, a_{h^\prime}) - V_\pi(\bs_{h+1})}^2 \middle| \bs_h} \\
    &=  \E_{\pi}\brck{\paren{Q_\pi (\bs_h, a_h) - V_\pi(\bs_h) + \sum_{h^\prime = h+1}^H r(\bs_{h^\prime}, a_{h^\prime}) - V_\pi(\bs_{h+1})}^2 \middle| \bs_h}
\end{align*}
Breaking the above expectation into three terms by expanding the square, note that the third term is zero because, $\E_{\pi} \brck{Q_\pi(\bs_h, a_h) - V_\pi(\bs_{h+1}) \mid \bs_h} = 0$, for any state $\bs_h$ and in our autoregressive MDP with deterministic dynamics, $$Q_\pi(\bs_h, a_h) = r(\bs_h, a_h) + V_\pi(\bs_{h+1}),$$ also for every state $\bs_h$. Recall that, here the state $\bs_{h+1} = (\bs_h, a_h)$. Additionally, we also take the expecation over the state distribution of $\bs_h \sim d^\pi_h$, and since the equality is true individually for each value of $\bs_h$, it also holds under the expectation over $\bs_h$. This gives us the following.  
\begin{align*}
    & \E_{\bs_h \sim d^{\pi}_{h}}\brck{\E_{\pi}\brck{\paren{r(\bs_h, a_h) + V_\pi(\bs_{h+1}) - V_\pi(\bs_h) + \sum_{h^\prime = h+1}^H r(\bs_{h^\prime}, a_{h^\prime}) - V_\pi(\bs_{h+1})}^2 \middle| \bs_h}} \\
    &=   \E_{\bs_h \sim d^{\pi}_{h}}\brck{\E_{\pi}\brck{\paren{r(\bs_h, a_h) + V_\pi(\bs_{h+1}) - V_\pi(\bs_h)}^2 \middle| \bs_h} + \E_{\pi} \brck{\paren{ \sum_{h^\prime = h+1}^H r(\bs_{h^\prime}, a_{h^\prime}) - V_\pi(\bs_{h+1})}^2 \middle| \bs_h}} \\ 
    &\quad\quad + 2\cdot \E_{\bs_h \sim d^{\pi}_{h}}\brck{\E_{\pi}\brck{{r(\bs_h, a_h) + V_\pi(\bs_{h+1}) - V_\pi(\bs_h)} \middle| \bs_h} \cdot \E_{\pi} \brck{{ \sum_{h^\prime = h+1}^H r(\bs_{h^\prime}, a_{h^\prime}) - V_\pi(\bs_{h+1})} \middle| \bs_h}}  
\end{align*}
As we noted above, the third term in the summation above is zero. Thus,
\begin{align*}
    \E_{\bs_h \sim d^\pi_h} \brck{\Var_\pi \brck{ \sum_{h^\prime=h}^H r(\bs_{h^\prime}, a_{h^\prime}) \middle| \bs_h}} &= \E_{\bs_{h+1} \sim d^\pi_{h+1}}\brck{ \Var_\pi \brck{ \sum_{h^\prime=h+1}^H r(\bs_{h^\prime}, a_{h^\prime})} \middle| \bs_h} \\  
    & \quad\quad + \E_{\bs_{h} \sim d^\pi_h} \brck{ \Var_\pi \brck{Q_\pi (\bs_h, a_h)} \middle| \bs_h}   
\end{align*}
The above induction is true for all values of $h$.
Now, taking the sum over $h$, from $h=1$ to $h=H$ on both left and right sides of the equation and using the definition of $\sigma^2_\pi$, we get:
\begin{align*}
    \sigma_\pi^2 = \E_{\bs_1 \sim d^\pi_1} \left[ \Var_{\pi} \brck{\sum_{h=1}^H r(\bs_h, a_h) \middle| \bs_1} \right].  
\end{align*}
Recall from Section~\ref{sec:prelim} that the first state $\bs_1$ is simply the input prompt $\bx$. Thus $d_1^\pi$ is indpendent of $\pi$ and is simply the distribution over the input prompts $\bx$, which is defined as $\rho$. Plugging this into the above equation we get:
\begin{align*}
    \sigma^2 = \E_{\bx \sim \rho} \left[ \Var_{\pi} \brck{\sum_{h=1}^H r(\bs_h, a_h) \middle| \bx} \right] = \E_{\bx \sim \rho} \left[ \Var_{\tau \sim \pi(\cdot \mid \bx)} \brck{r(\tau)} \right].  
\end{align*}
\end{proof}


\begin{lemma} \label{lemma:Abound}
Consider a random variable $A$ which is almost surely non-negative and has mean $\mu$ and variance $\sigma^2$. For any $\theta \ge 0$,
\begin{align}
    \mathbb{E} \left[ \frac{\theta ( \mu - A )}{\sigma + \theta A} \right] \le 2 \theta^2
\end{align}
\end{lemma}
\begin{proof}
Let $f(\theta) = \mathbb{E} \left[ \frac{\theta ( \mu - A )}{\sigma + \theta A} \right]$. Observe that,
\begin{align*}
    f'(\theta) &= \mathbb{E} \left[ \frac{\mu - A}{\sigma + \theta A} \right] - \mathbb{E} \left[ \frac{\theta ( \mu - A ) A}{(\sigma + \theta A)^2} \right] \\
    f''(\theta) &= - 2 \mathbb{E} \left[ \frac{(\mu - A)A}{(\sigma + \theta A)^2} \right] + 2 \mathbb{E} \left[ \frac{\theta ( \mu - A ) A^2}{(\sigma + \theta A)^3} \right] \\
    &= 2\mathbb{E} \left[ \frac{\theta ( \mu - A ) A^2 - (\mu-A)A(\sigma + \theta A)}{(\sigma + \theta A)^3} \right] \\
    &= 2\sigma \mathbb{E} \left[ \frac{(A-\mu)A}{(\sigma + \theta A)^3} \right] \\
    &= 2\sigma \mathbb{E} \left[ \frac{\mu(A-\mu)}{(\sigma + \theta A)^3} \right] + 2\sigma \mathbb{E} \left[ \frac{(A-\mu)^2}{(\sigma + \theta A)^3} \right] \\
    &\le 2\sigma \mathbb{E} \left[ \frac{\mu(A-\mu)}{(\sigma + \theta A)^3} \right] + 2\sigma \mathbb{E} \left[ \frac{(A-\mu)^2}{\sigma^3} \right] \\
    &= 2\sigma \mathbb{E} \left[ \frac{\mu(A-\mu)}{(\sigma + \theta A)^3} \right] + 2
\end{align*}
Note that $\mu (A - \mu)$ and $(\sigma + \theta A)^3$ are both increasing functions in $A$, and therefore,
\begin{align*}
    \mathbb{E} \left[ \frac{\mu(A-\mu)}{(\sigma + \theta A)^3} \right] &\le 2\sigma \mathbb{E} \left[ \mu (A-\mu) \right] \mathbb{E} \left[ \frac{1}{(\sigma + \theta A)^3} \right] = 0.
\end{align*}
This results in the upper bound $\| f'' \|_\infty \le 2$. Since $f(0) = 0$ and $f'(0) = 0$, we have that,
\begin{align*}
    f(\theta) = \int_{0}^{\theta} f''(\alpha) \mathrm{d}\alpha \le 2 \theta^2.
\end{align*}
\end{proof}

\subsection{Lower bound on \texorpdfstring{$\sigma_e$}{}: Proof of Lemma~\ref{lem:expert-variance-lower-bound}}
\label{subsec:proof-expert-variance-lower-bound}

In this section, we show that for any base policy $\pibase$, and any expert policy $\piexp$ such that $\kl{\piexp}{\pibase} \le \kappa$,
\begin{align*}
    \sigma_e^2 \ge \sigma_b^2 - H \sigma_b \sqrt{\kappa/2}.
\end{align*}
Since $\kl{\cdot}{\cdot} \le \chi^2 (\cdot \| \cdot )$ pointwise, this implies the lower bound on $\sigma_e$ within the $\chi^2$ ball.

By definition, observe that,
\begin{align*}
    \sigma_\pi^2 &= \mathbb{E}_{\bx \sim \rho} [\Var_{\tau \sim \pi(\cdot \mid \bx)} \brck{r(\tau)}] \\
    &= \frac{1}{2} \mathbb{E}_{\bx \sim \rho} \left[ \mathbb{E}_{\tau, \tau' \sim \pi (\cdot \mid \bx)} [ ( r(\tau)-r(\tau'))^2 ] \right]
\end{align*}


Note that the squared Hellinger divergence $D_H^2$ satisfies $D_{\mathrm{H}}^2 (\cdot,\cdot) \le D_{\mathrm{KL}} (\cdot,\cdot)$ pointwise (cf. Lemma 2.4 in \citet{tsybakov2009nonparametric}). With the choice $Y = (\tau,\tau')$ in the change-of-measure argument in \cref{eq:HellCoM} of \Cref{lemma:CoM}, $h (Y) = ( r(\tau)-r(\tau'))^2$ and $P$ denote the distribution over trajectories $\pibase(\cdot\mid \bx)$ and $Q$ denote the distribution over trajectories induced by $\piexp (\cdot \mid \bx)$,
\begin{align}
| \Var_{\tau \sim \pibase (\cdot \mid \bx)} [ r (\tau)  ] - \Var_{\tau \sim \piexp (\cdot \mid \bx)} [ r (\tau)  ] | 
&\le \frac{1}{2} \sqrt{\frac{1}{2} \left(\mathbb{E}_P \left[h^2(Y )\right]+\mathbb{E}_Q \left[h^2(Y)\right]\right) \cdot \kl{(\tau_e,\tau_e')}{ (\tau_b, \tau_b')}} \nonumber\\
&\le \frac{1}{2} \sqrt{\left(\mathbb{E}_P \left[h^2(Y )\right]+\mathbb{E}_Q \left[h^2(Y)\right]\right) \cdot \kl{\tau_e}{\tau_b}} \label{eq:sq}
\end{align}
where in the last inequality, we use the fact that $\tau_e$ and $\tau_e'$ are i.i.d. $\sim \piexp(\cdot\mid\bx)$, and likewise $\tau_b$ and $\tau_b'$ are i.i.d. $\sim \pibase(\cdot\mid\bx)$, and the chain rule of KL divergence. What remains is to bound $\mathbb{E}_{\tau \sim \pi (\cdot \mid \bx)} [ ( r (\tau) - r (\tau'))^4 ]$ for $\pi = \piexp$ and $\pi = \pibase$. Since $|r(\tau) - r(\tau')| \le H$ almost surely,
\begin{align*}
    \mathbb{E}_{\tau \sim \pi (\cdot\mid\bx)} [ ( r (\tau) - r (\tau') )^4 ] &\le 2H^2 \Var_{\tau \sim \pi (\cdot\mid\bx)} [ r (\tau) ]
\end{align*}
Let's denote $A = \Var_{\tau \sim \piexp (\cdot\mid\bx)} [r(\tau)]$ and $B = \Var_{\tau \sim \pibase (\cdot\mid\bx)} [r(\tau)]$. Combining with \cref{eq:sq} and squaring, and denoting $\kl{\tau_e}{\tau_b} = \kappa_\bx$,
\begin{align}
    &\left( A - B \right)^2 \le \frac{H^2}{4} \left( A + B \right) \cdot \kappa_\bx \nonumber\\
    \implies &A^2 - \left( 2 B + \frac{\kappa_\bx H^2}{4} \right) A + \left( B^2 - \frac{\kappa_\bx H^2}{4} B \right)  \le 0 \label{eq:23}
\end{align}
This is a quadratic equation in $A$. Solving, we get,
\begin{align*}
    A &\ge \left( B + \frac{\kappa_\bx H^2}{8} \right) - \sqrt{\left( B + \frac{\kappa_\bx H^2}{8} \right)^2 - \left( B^2 - \frac{\kappa_\bx H^2}{4} B \right)} \\
    &= \left( B + \frac{\kappa_\bx H^2}{8} \right) - \sqrt{ \frac{\kappa_\bx H^2}{2} B + \frac{\kappa_\bx^2 H^4}{64}} \\
    &\ge B - H \sqrt{ \kappa_\bx B/2}
\end{align*}
where the last inequality uses the subadditivity of the $\sqrt{\cdot}$ function. This implies that,
\begin{align*}
    \Var_{\tau \sim \piexp (\cdot\mid\bx)} [r(\tau)] \ge \Var_{\tau \sim \pibase (\cdot\mid\bx)} [r(\tau)] - H \sqrt{ (\kappa_\bx/2) \Var_{\tau \sim \pibase (\cdot\mid\bx)} [r(\tau)] }
\end{align*}
Taking an expectation over $\bx \sim \rho$ on both sides,
and using Jensen's inequality,
\begin{align*}
    \sigma^2_e &\ge \sigma^2_b - H \mathbb{E}_{\bx \sim \rho} \left[ \sqrt{ (\kappa_\bx/2) \Var_{\tau \sim \pibase (\cdot\mid\bx)} [r(\tau)] } \right] \\
    &\ge \sigma^2_b - H \sqrt{ \E_{\bx \sim \rho} [\kappa_\bx/2] \mathbb{E}_{\bx \sim \rho} \left[ \Var_{\tau \sim \pibase (\cdot\mid\bx)} [r(\tau)] \right]} \\
    &= \sigma^2_b - H \sigma_b \sqrt{\kappa/2}
\end{align*}
Noting that $\mathbb{E}_{\bx \sim \rho} [\kappa_\bx] \le \kappa$. Solving for the larger root of the quadratic in \cref{eq:23}, we also arrive at the upper bound,
\begin{align}
    A &\le B + H \sqrt{\kappa_\bx B /2} + \frac{\kappa_\bx H^2}{4} \nonumber\\
    \implies \sigma_e^2 &\le \sigma_b^2 + H \sigma_b \sqrt{\kappa/2}+ \frac{\kappa H^2}{4}. \label{eq:var-ub}
\end{align}
which follows by taking an expectation over $\bx \sim \rho$.

\paragraph{Optimality of \Cref{lem:expert-variance-lower-bound}.} The above result is tight up to constants. Consider an autoregressive MDP with a single prompt, where picking action $a_0$ at time $1$ results in hitting a bi-level (so, regardless of future actions, a reward of $1$ is collected at each step) and picking action $a_1$ results in a reward of $0$ forever. $\pibase$ picks the first branch with probability $p$ and the second with probability $1-p$ at $t=1$. Then, $\sigma_b^2 = p (1-p) H^2$ and by scaling $p$ from $0$ to $1/2$, any $0 \le \sigma_b^2 \le H^2/4$ can be achieved. On the other hand, consider the policy $\piexp$ which plays $a_0$ with probability $p - \theta$ at $t=1$. Suppose $p$ is a constant. Then,
\begin{align*}
    \chi^2 (\piexp \| \pibase) &= \frac{(p - \theta)^2}{p} + \frac{1 - 2 (p - \theta) + (p-\theta)^2}{1-p} - 1 \\
    &= \frac{p^2 - 2p \theta + \theta^2}{p} + \frac{(1 - p)^2 + 2\theta(1-p) + \theta^2}{1-p} - 1 \\
    &= \frac{\theta^2}{p} + \frac{\theta^2}{1-p} \\
    &= \frac{\theta^2}{p(1-p)}
\end{align*}
Therefore, choosing $\theta = \min \{ p, \sqrt{\kappa p (1-p)} \}$, we get,
\begin{align*}
    \chi^2 (\piexp \| \pibase) \le \kappa
\end{align*}
And furthermore that, $\sigma_e^2 = (p-\theta) (1 - (p-\theta))H^2$ and therefore,
\begin{align*}
    \sigma_e^2 - \sigma_b^2 &= (p-\theta) (1 - (p-\theta))H^2 - p(1 - p)H^2 \\
    &= - (\theta + \theta^2 - 2p \theta) H^2,
\end{align*}
when $\theta = p$, we get $\sigma_e^2 = 0$. When $\theta = \sqrt{\kappa p (1-p)}$, this is assumed to be in the regime $\theta > p$ and so,
\begin{align*}
    \sigma_e^2 - \sigma_b^2 &\le - (\theta + p \theta - 2 p \theta) H^2 \\
    &\le - \frac{\theta}{2} H^2
\end{align*}
where in the last equation we recall the assumption that $p \le 1/2$. Plugging in $\theta$ and observing that $H^2 \theta = H \sigma_b \sqrt{\kappa}$ completes the proof.



\subsection{Proof of Theorem~\ref{thm:verifier-free-thm}}
\label{subsec:proof-verifier-free-thm}

We will state a slightly more formal version of \Cref{thm:verifier-free-thm} below in \Cref{subsubsec:5.4}. Prior to this, we introduce some relevant notation necessary to state the main result.

\subsubsection{Measure of complexity: \texorpdfstring{$L_k^\star$}{}}

Consider an arbitrary partitioning of the prompt space $\mathcal{X}$ into $k$ disjoint parts, denoted $\{ \mathcal{X}_i \}_{i=1}^k$. Let $\{ \mathcal{X}^\star_i \}_{i=1}^k$ denote the partitioning of the prompt space which maximizes,
\begin{align} \label{eq:Lstar}
    L(\{\mathcal{X}_i\}_{i=1}^k) \eqdef \min \left\{ \mathbb{E}_{\bx \sim \rho} [\sigma_{e,\bx} \mathbb{I} (\bx \in \cup_{i \in K} \mathcal{X}_i)] : K \subseteq [k] \text{ and } |K| \ge k/4 \right\}.
\end{align}
And let $L^\star_k = L(\{\mathcal{X}^\star_i\}_{i=1}^k)$. Our construction, and lower bounds derived therafter are stated in terms of $\{ \mathcal{X}_i^\star \}_{i=1}^k$ and $L^\star_k$. We devote the first part of this section toward interpretations of $L_k^\star$.

Recall that $\widetilde{\sigma}_e = \text{Median} ( \{ \sigma_{\pi,\bx} : \bx \in \mathcal{X} \} )$, $\overline{\sigma}_e = \mathbb{E}_{\bx \sim \rho}  [\sigma_{\piexp,\bx}]$ and $\sigma^2_e = \mathbb{E}_{\bx \sim \rho} [\sigma_{\piexp,\bx}^2]$. We will first show that $L_k^\star \gtrsim \widetilde{\sigma}_e$ always. Later, we will show that if $\sigma_e^2 \le c \overline{\sigma}_e^2$ for a sufficiently small constant $c > 1$, $L_k^\star \gtrsim \sigma_e$.

\subsubsection{Interpretations of, and bounds on \texorpdfstring{$L_k^\star$}{}}
\begin{lemma}
Consider any $8 \le k \le |\mathcal{X}|/4$. Then, $L_k^\star \ge \frac{1}{32} \widetilde{\sigma}_e$.
\end{lemma}
\begin{proof}
We will prove this statement in two assertions,
\begin{enumerate}
    \item When $k$ is a power of two, $L_{k/2}^\star \ge L_k^\star$.
    \item When $k$ is any power of two, and any $k/2 \le k' \le k$, $L_{k'}^\star \ge \frac{1}{2} L_{k}^\star$.
\end{enumerate}
For a subset $X \subseteq \mathcal{X}$, define its score $s(X) = \E_{\bx \sim \rho} [ \sigma_{e,\bx} \mathbb{I} (\bx \in X)]$. Assuming these two assertions, we will prove the main lemma first, and then come back to proving them.

\paragraph{Proof of the main lemma.} Consider $k$ as the largest power of $2$ between $|\mathcal{X}|/4$ and $|\mathcal{X}|/2$. For this choice, consider the partition of $\mathcal{X}$ into $k$ sets by choosing the first $k$ parts as singleton sets, consisting of the top $k$ prompts $\bx \in \mathcal{X}$ with the highest values of $\sigma_{e,\bx}$; the remaining prompts are distributed among sets in the partition arbitrarily. Notably, the score of each part $\mathcal{X}_i$ in this partition satisfies $s(\mathcal{X}_i) \ge \text{Median} ( \{ s( \{\bx\} ) : \bx \in \mathcal{X} \} )$; by implication, for any such value of $k$, 
\begin{align} \label{eq:median}
L_k^\star \ge \frac{k}{4} \text{Median} ( \{ s(\bx) : \bx \in \mathcal{X} \} ) \ge \frac{|\mathcal{X}|}{16} \text{Median} ( \{ s( \{ \bx \} ) : \bx \in \mathcal{X} \} ) = \frac{1}{16} \text{Median} ( \{ \sigma_{e,\bx} : \bx \in \mathcal{X} \} )
\end{align}
where the last equation uses the fact that $\rho$ is the uniform distribution over $\mathcal{X}$. Therefore, for any $k' \le k$, we have that $L_k^\star \ge \frac{1}{2} L_k^\star = \ge \frac{1}{32} \text{Median} ( \{ \sigma_{e,\bx} : \bx \in \mathcal{X} \} )$.

\paragraph{Proof of the first assertion.}  Consider the optimal partition which induces $L_k^\star$, $\{ \mathcal{X}^\star_i \}_{i=1}^k$, arranged in increasing order of scores. Note then, that $L^\star_k = \sum_{i=1}^{ k/4 } s(\mathcal{X}_i)$ Consider the partition of $\mathcal{X}$ into $k/2$ parts, as $\{ \mathcal{X}_1^\star \cup \mathcal{X}_2^\star, \mathcal{X}_3^\star \cup \mathcal{X}_4^\star, \cdots, \mathcal{X}_{k-1}^\star \cup \mathcal{X}_k^\star \}$. Since scores are additive, the $k/8$ parts with the lowest scores must be $\{ \mathcal{X}_i^\star \cup \mathcal{X}_{i+1}^\star \}_{i=1}^{k/8}$. This implies the first assertion.

\paragraph{Proof of the second assertion.} Consider the optimal partition which induces $L_k^\star$, $\{ \mathcal{X}^\star_i \}_{i=1}^k$. By dissolving the bottom $k-k'$ parts (in terms of score) of $\{ \mathcal{X}^\star_i \}_{i=1}^k$ and merging them with other parts, this results in a partitioning of $\mathcal{X}$ such that the sum of $k'/4$ worst scores of the parts must be at least $(k'/k) L_k^\star \ge L_k^\star/2$.


\end{proof}

\begin{lemma}
Suppose $\sigma_e^2 \le \frac{4}{3} \overline{\sigma}_e^2$, then $\widetilde{\sigma}_e \ge \frac{1}{10} \overline{\sigma}_e \ge \frac{1}{15} \sigma_e$.
\end{lemma}
\begin{proof}
By the Paley-Zygmund inequality,
\begin{align}
\Pr_{\bx \sim \rho} \left[ \sigma_{e,\bx} \ge \frac{1}{10} \overline{\sigma}_e \right] \ge \frac{4}{5} \times \frac{\overline{\sigma}_e^2}{\sigma_e^2}
\end{align}
When $\sigma_e^2 \le \frac{4}{3} \overline{\sigma}_e^2$, the LHS is at least $3/5$. This means that at least $3|\mathcal{X}|/5$ of the prompts satisfy $\sigma_{e,\bx} \ge \frac{1}{10} \overline{\sigma}_e$, and so $\text{Median} ( \{ \sigma_{e,\bx} : \bx \in \mathcal{X} \} ) \ge \frac{1}{10} \overline{\sigma}_e$.
\end{proof}

As a corollary of this lemma, we have that,

\begin{corollary}
Under the condition $\sigma_e^2 \le (4/3) \overline{\sigma}_e^2$, for every $k \le |\mathcal{X}|/4$, we have that $L_k^\star \ge c \sigma_e$ for some absolute constant $c > 0$.
\end{corollary}

Having introduced these interpretations of $L_k^\star$, we prove the following instance-dependent lower bound on the suboptimality of any verifier-free algorithm.


\subsubsection{Lower bounds on verifier-free approaches} \label{subsubsec:5.4}

Below we introduce the class of rewards for which we prove the instance-dependent lower bound in \Cref{thm:verifier-free-thm}.

\begin{definition}[Half-bi-level rewards] \label{def:R1/2}
Define the class of half-bi-level rewards, $\gR_{1/2}$, as those reward functions such that every trajectory contains a bi-level at or before time $t = \lfloor H/2 \rfloor$. Namely, for any trajectory $(s_1,a_1,\cdots,s_H,a_H)$, $r(s_t,a_t) = 1$ for every $t \ge \lfloor H/2 \rfloor$ for any reward $r \in \gR_{1/2}$.
\end{definition}

\begin{remark}
Although half-bi-level rewards are constrained to have all their bi-levels before time $H/2$, this does not preclude there from existing policies having high variance under rewards from this class. In particular, there exists a policy $\pi$ and a reward $r \in \mathcal{R}_{1/2}$ such that $\sigma_\pi^2 = H^2/16$.
\end{remark}

\begin{theorem}
Suppose $|\mathcal{X}| \ge 16$ and choose any $4 \le k \le |\mathcal{X}|/4$. Consider any autoregressive MDP and assume that $\rho = \text{Unif} (\mathcal{X})$. For any choice of reward $r \in \mathcal{R}_{1/2}$, base policy $\pibase$ and expert policy $\piexp \in \Pi_\varepsilon$, there exists an alternate family of expert policies $\Pi^\prime$ of size $\lceil 2^{k/4} \rceil$ and reward class $\mathcal{R}^\prime \subset \mathcal{R}$ (also of the same size), such that,
\begin{enumerate}
    \item $\piexp \in \Pi^\prime$ and $r \in \mathcal{R}^\prime$,
    \item $\Pi^\prime \subseteq \Pi_{\varepsilon^\prime}$ corresponds to a family of feasible expert policies with $\varepsilon^\prime = 3(1+\varepsilon) \cdot \max \left\{ \frac{H\sqrt{\varepsilon_{\text{stat}}}}{\sigma_{\min}}, \frac{H^2 \varepsilon_{\text{stat}}}{\sigma_{\min}^2} \right\}$.\\
    Here, $\sigma_{\min} = \min_{\bx \in \mathcal{X}} \sigma_{e,\bx}$.
    \item For every $r^\prime \in \mathcal{R}^\prime$ and policy $\pi^\prime \in \Pi^\prime$, $\sigma_{r^\prime}^2 (\pi^\prime) \le \sigma_e^2 + H \sigma_e \sqrt{\varepsilon_{\text{stat}}} + H^2 \varepsilon_{\text{stat}}$.
    \item For any realizable verifier-based learning algorithm, satisfying $\hat{\pi}^{\text{vf}}_n \in \Pi^\prime$,
    \begin{equation}
    \max_{\pi^\prime \in \Pi^\prime} \max_{r^\prime \in \mathcal{R}} \Pr \left( J_{r^\prime} (\pi^\prime) - J_{r^\prime} (\hat{\pi}^{\text{vf}}_n) \ge L_k^\star\sqrt{\varepsilon_{\text{stat}}} \right) \ge 1/8
\end{equation}
\end{enumerate}
Here, we define $\varepsilon_{\text{stat}} = \frac{\log(|\Pi^\prime|)}{16n}$ and assume that $n$ is sufficiently large so that $\varepsilon_{\text{stat}} \le \min_{\bx \in \mathcal{X}} \sigma_{e,\bx}^2 / (J_r(\piexp|\bx))^2$. 
\end{theorem}

\begin{proof}[Proof structure]
We define the alternate policy class $\Pi^\prime$ across \Cref{lem:packing} and \Cref{lemma:GV}, culminating in \Cref{subsubsec:PIR}. Property 2 (i.e., $\Pi^\prime \subseteq \Pi_{\varepsilon'}$) and Property 3 (i.e., the bound on the variance of policies in $\Pi^\prime$ on rewards in $\mathcal{R}^\prime$) are established in \Cref{lem:packing}.
\end{proof}

\begin{remark}
The results of \cite{foster2024behavior} establish a similar lower bound for autoregressive MDPs. However their construction specifically assumes, either $(i)$ there is a single prompt, or $(ii)$ the adversary constructing an alternate hard instance can change the initial state distribution $\rho$. This follows from the fact that their alternate policy is constructed in a way which does not preserve the initial state distribution of the MDP (cf. Lemma G.1 in their paper). 
\end{remark}


Our lower bound scales with $L^\star_k \gtrsim \widetilde{\sigma}_e$ where $\widetilde{\sigma}_e = \text{Median}( \{ \sigma_{e,\bx} : \bx \in \mathcal{X} \})$, rather than $\sigma_e$, as previous work \cite{foster2024behavior} hints in the case of a single prompt. In general, it turns out that it is not possible to have an instance-dependent lower bound that scales as $\Omega (\sigma_e \sqrt{\log(|\Pi|)/n})$. There exist a class of MDPs where verifier free approaches achieve an error of $\gO (\widetilde{\sigma}_e \sqrt{\log(|\Pi|)/n})$, even under the worst case choice of policy class, and improve over the suggested $\Theta (\sigma_e \sqrt{\log(|\Pi|)/n})$ instance-dependent error.  


\begin{theorem}
Consider an autoregressive MDP with $|\mathcal{A}| = 2$ and $H=1$. There exists an expert policy $\piexp$, such that \textbf{for any policy class} $\Pi \ni \piexp$ of size $|\Pi| \ge 2^{\Omega(|\mathcal{X}|)}$, there exists a verifier-free learner such that with probability at least $1-\delta$,
\begin{align*}
    \max_{r \in \mathcal{R}} J_r (\piexp) - J_r (\hat{\pi}_n^{\text{vf}}) &\le \widetilde{\gO}_{|\mathcal{X}|,\delta} \left( \widetilde{\sigma}_e \sqrt{\frac{\log (|\Pi|)}{n}} + \frac{\log(|\Pi|)}{n} \right) \\
    &= \widetilde{\Theta}_{|\mathcal{X}|,\delta} \left( \frac{\sigma_e}{\sqrt{|\mathcal{X}|} }\cdot \sqrt{\frac{\log (|\Pi|)}{n}} + \frac{\log(|\Pi|)}{n} \right)
\end{align*}
as long as $\delta \ge |\mathcal{X}|\exp ( - \frac{1}{2} \sqrt{n/|\mathcal{X}|})$.
\end{theorem}
\begin{proof}
WLOG, assume $\mathcal{A} = \{ 0,1 \}$. Consider the following expert: for the $i^{\text{th}}$ prompt, arranged in arbitrary order, let $\piexp (1|\bx_i) = \frac{1}{2i^2}$. Observe that,
\begin{align*}
    \widetilde{\sigma}_e &= \Theta \left( \frac{1}{|\mathcal{X}|} \right) \\
    \overline{\sigma}_e &= \mathbb{E}_\rho [\sigma_{e,\bx}] \le \mathbb{E}_\rho [\sqrt{\piexp (1|\bx)}] = \Theta \left( \frac{\log(|\mathcal{X}|)}{|\mathcal{X}|} \right) \\
    \sigma_e &= \sqrt{\mathbb{E}_\rho [\sigma_{e,\bx}^2]} \ge \sqrt{\frac{1}{2}\mathbb{E}_\rho [\piexp(1|\bx)]} = \Theta \left( \frac{1}{2 \sqrt{|\mathcal{X}|}} \right)
\end{align*}

For each action, construct the empirical distribution estimator, and return this policy as $\hat{\pi}_n^{\text{vf}} (0|\bx)$. Then, with probability at least $1-\delta$, conditioning on the number of samples $n_\bx$ observed with prompt $\bx$,
\begin{align*}
    | \hat{\pi}^{\text{vf}}_n (0|\bx) - \piexp (0|\bx) | \le \min \left\{ 1, \sqrt{\frac{\piexp (0|\bx) \log(2/\delta)}{n_\bx}} + \frac{ \log(2/\delta)}{n_\bx} \right\}
\end{align*}
Therefore, with probability at least $1 - \delta$,
\begin{align}
    \max_{r \in \mathcal{R}} J_r (\piexp) - J_r (\hat{\pi}^{\text{vf}}_n) &= \mathbb{E}_\rho \left[ \tv{\hat{\pi}^{\text{vf}}_n (\cdot|\bx)}{\piexp (\cdot|\bx)} \right] \nonumber\\
    &\le \mathbb{E}_\rho \left[ \left\{ 1, \sqrt{\frac{\piexp (0|\bx) \log(2|\mathcal{X}|/\delta)}{n_\bx}} + \frac{ \log(2/\delta)}{n_\bx} \right\} \right] \label{eq:090123}
\end{align}
With probability $1 - \delta$, we have that $n_\bx \ge \frac{n}{|\mathcal{X}|} - \sqrt{\frac{n}{|\mathcal{X}|} \log (1/\delta)}$ for every $\bx \in \mathcal{X}$. Assuming $\delta \ge |\mathcal{X}| \exp (-\frac{1}{2}\sqrt{n/|\mathcal{X}|})$, by union bounding, we have that with probability at least $1 - \delta$, for all $\bx \in \mathcal{X}$, $n_\bx \ge \frac{n}{2|\mathcal{X}|}$. Combining with \cref{eq:090123}, with probability at least $1-2\delta$,
\begin{align*}
    \max_{r \in \mathcal{R}} J_r (\piexp) - J_r (\hat{\pi}^{\text{vf}}_n) &\le 2 \sum_{\bx \in \mathcal{X}} \sqrt{\frac{\piexp (0|\bx) \log (|\mathcal{X}|/\delta)}{n |\mathcal{X}|}} + \frac{ \log(2/\delta)}{n}\\
    &\le 2\log(|\mathcal{X}|) \sqrt{\frac{\log (|\mathcal{X}|/\delta)}{n |\mathcal{X}|}} + 2\frac{|\mathcal{X}| \log(2/\delta)}{n} \\
    &\le 2\widetilde{\sigma}_e \cdot \sqrt{\log (|\Pi|) \frac{\log (|\mathcal{X}|/\delta)}{n}} + \frac{2\log(|\Pi|) \log(2/\delta)}{n}
\end{align*}
where the last inequality uses the fact that $|\Pi| \ge 2^{\Omega(|\mathcal{X}|)}$ and by construction, the value of $\widetilde{\sigma}_e$.

\end{proof}

\begin{lemma} \label{lemma:1-r}
For any reward $r \in \gR_{1/2}$, there exists another reward $\widetilde{r} \in \gR$ such that, for any policy $\pi \in \Pi$ and input distribution $\rho$,
\begin{align*}
    \E_{\rho, \pi} [r (\tau)] &= H -  \E_{\rho, \pi} [ \widetilde{r} (\tau)] \\
    \Var_{\rho,\pi} [ r (\tau) ] &= \Var_{\rho,\pi} [ \widetilde{r} (\tau) ]
\end{align*}
\end{lemma}
\begin{proof}
Consider the bi-level reward $r$, and consider the set of minimal states: $\cup_{\tau \in \mathcal{A}^H} \{ s_{t^\star} \text{ where } t^\star = \min \{ 1 \le t \le H : r(s_{t-1},a_t) > r(s_{t-2},a_{t-1}) \}$. These are the states where a bi-level may be first visited. For each such minimal state, the bi-level property implies that any trajectory which visits this state collects a reward of $1$ at every point in time regardless of the sequence of actions played. Based on this construction, we define the reward $\widetilde{r}$ as follows: for every minimal state $s$ which appears at time $t$, consider the subtree rooted at this node (i.e., the set of trajectories which visit this state). Delete this minimal state, and replace it by the set of all $2^{H-t}$ new minimal states corresponding to the set of all states in the subtree at depth $H-t$. Let $\widetilde{r}$ be induced by this new set of minimal states; moreover, it is feasible to construct this set because of the assumption that $r \in \mathcal{R}_{1/2}$: every minimal state appears at some value of $t \le H/2$.

Consider any trajectory $\tau$. Suppose this trajectory visits a bi-level at time $t \le H/2$. Now the same trajectory is guaranteed to visit a bi-level at time $H-t \ge H/2$. Thus, $\widetilde{r} (\tau) = H - r(\tau)$, and the assertions about $\E_{\rho,\pi} [\widetilde{r}(\tau)]$ and $\Var_{\rho,\pi} [ \widetilde{r} (\tau) ]$ follow suit.
\end{proof}






\begin{lemma}
\label{lem:packing}
For any policy $\pi$ and reward $r$, and $0 \leq \xi \leq \min_{\bx \in \mathcal{X}} \frac{\sigma_{e,\bx}^2}{4(J_r(\piexp|\bx))^2}$, there exists a class of $2^k$ policies, $\Pi_k = \{ \pi_{\bm{z}} : \bm{z} \in \{ 0,1 \}^k \}$ indexed by binary vectors, and a class of $2^k$ rewards indexed similarly as $\mathcal{R}_k = \{ r_{\bm{z}} : \bm{z} \in \{ 0,1 \}^k \}$, such that,
\begin{enumerate}
    \item For any $\bm{z}, \bm{z}' \in \{ 0,1 \}^k$, $\chisq{\pi_{\bm{z}}}{\pi_{\bm{z}'}} \le 8\xi$. Furthermore, $\chisq{\pi_{\bm{z}}}{\piexp} \le 8\xi$.
    \item $J_{r_{\bm{z}}} (\pi_{\bm{z}}) - J_{r_{\bm{z}}} (\pi_{\bm{z}'}) = \sqrt{\xi} \sum_{i=1}^k \mathbb{I} (\bm{z}_i \ne \bm{z}'_i (\bx)) \cdot \E_{\bx \sim \rho} [ \sigma_{e,\bx} \mathbb{I} (\bx \in \mathcal{X}^\star_i)]$,
    \item For every reward $r' \in \mathcal{R}_k$ and every $\pi' \in \Pi_k$: $\sigma_{e,\bx}^2 (\pi',r') \le \sigma_{e,\bx}^2 + H \sigma_{e,\bx} \sqrt{\xi} + H^2 \xi$.
    \item Recall that $\piexp \in \Pi_\varepsilon$, the $\varepsilon$-radius KL ball around $\pibase$. Then, every $\pi' \in \Pi_k$ belongs in the ball $\Pi_{\varepsilon'}$, where,
    \begin{align}
        \varepsilon' = 3 (1+\varepsilon) \cdot \max \left\{ \frac{\sqrt{\xi} H}{\sigma_{\min}} , \frac{\xi H^2}{\sigma_{\min}^2} \right\}.
    \end{align}
    and where $\sigma_{\min} = \min_{\bx \in \mathcal{X}} \sigma_{e,\bx}$.
\end{enumerate}
\end{lemma}
\begin{proof}
The policy $\pi_{\bm{z}}$ is defined as follows. For each $i \in [k]$ and $\bx \in \mathcal{X}_i$,
\begin{align}
    \pi_{\bm{z}} (\tau|\bx) \propto \begin{cases}
        (\sigma_{e,\bx} + \theta_\bx r(\tau))\piexp(\tau|\bx), \quad &\text{if } \bm{z}_i = 1 \\
        \piexp(\tau|\bx), &\text{otherwise.}
    \end{cases}
\end{align}
where $\theta_\bx \ge 0$ is a parameter to be determined later. Likewise, the reward $r_{\bm{z}}$ is defined as follows. For each $\bx \in \mathcal{X}_i$,
\begin{align}
    r_{\bm{z}} (\tau|\bx) \propto \begin{cases}
        r(\tau), \quad &\text{if } \bm{z}_i = 1 \\
        \widetilde{r} (\tau|\bx), &\text{otherwise.}
    \end{cases}
\end{align}
where $\widetilde{r}$ is the reward defined in \Cref{lemma:1-r}. Since we only care about values and variances, for all intents and purposes, $\widetilde{r}$ is the same as $1-r$ (which itself may not be a bi-level reward).

\paragraph{Assertion 1: Bounding the $\chi^2$-divergence between $\pi_{\bm{z}}$ and $\pi_{\bm{z}'}$.} Consider any pair of binary vectors $\bm{z},\bm{z}' \in \{ 0,1 \}^k$. If $\bm{z}_i = \bm{z}'_i$, then $\chisq{\pi_{\bm{z}} (\cdot|\bx)}{\pi_{\bm{z}'} (\cdot|\bx)} = 0$ for any $\bx \in \mathcal{X}_i$. Otherwise, if $\bm{z}_i = 1$ and $\bm{z}'_i = 0$, for any $\bx \in \mathcal{X}_i$,
\begin{align}
    \chisq{\pi_{\bm{z}} (\cdot|\bx)}{\pi_{\bm{z}'} (\cdot|\bx)} &= \chisq{\pi_{\bm{z}} (\cdot|\bx)}{\piexp(\cdot|\bx)} \nonumber\\
    &= \frac{\mathbb{E}_{\piexp} [(\sigma_{e,\bx} + \theta_\bx r(\tau))^2 | \bx ]}{\mathbb{E}_{\piexp} [\sigma_{e,\bx} + \theta_\bx r(\tau) | \bx ]^2} - 1 \nonumber\\
    &= \frac{\sigma_{e,\bx}^2 + 2 \theta_\bx \sigma_{e,\bx} J_r (\piexp | \bx) + \theta_\bx^2 ((J_r (\piexp | \bx))^2 + \sigma_{e,\bx}^2)}{(\sigma_e + \theta_\bx J_r(\piexp | \bx))^2} - 1 \nonumber\\
    &= \frac{\theta_\bx^2 \sigma_{e,\bx}^2}{(\sigma_{e,\bx} + \theta_\bx J_r(\piexp | \bx))^2} \nonumber\\
    &= \xi \label{eq:b1b'0}
\end{align}
where the last equation follows by choosing $\theta_\bx$ such that $\theta_\bx \sigma_{e,\bx} = \sqrt{\xi} (\sigma_{e,\bx} + \theta_\bx J_r(\piexp | \bx))$. There will always exist a feasible choice of $\theta_\bx \ge 0$ satisifying this equation as long as the condition $\sqrt{\xi} \le \sigma_{e,\bx} / J_r (\piexp | \bx)$ is satisfied, and under the stronger restriction $\sqrt{\xi} \le \sigma_{e,\bx} / 2 J_r (\piexp | \bx)$ we will have that $\theta_\bx \le 2\sqrt{\xi}$. On the other hand, if $\bm{z} (\bx) = 0$ and $\bm{z}' (\bx) = 1$, for any $\bx \in \mathcal{X}_i$,
\begin{align}
    \chisq{\pi_{\bm{z}} (\cdot|\bx)}{\pi_{\bm{z}'} (\cdot|\bx)} &= \chisq{\piexp(\cdot|\bx)}{\pi_{\bm{z}} (\cdot|\bx)} \nonumber\\
    &= \mathbb{E}_{\pi} [\sigma_{e,\bx} + \theta_\bx r(\tau) | \bx ] \cdot \mathbb{E}_{\pi} \left[ \frac{1}{\sigma_{e,\bx} + \theta_\bx r(\tau)} \middle| \bx \right] - 1 \nonumber\\
    &= \mathbb{E}_{\pi} \left[ \frac{\sigma_{e,\bx} + \theta_\bx J_r(\piexp | \bx)}{\sigma_{e,\bx} + \theta_\bx r(\tau)} \middle| \bx \right] - 1 \nonumber\\
    &= \mathbb{E}_{\pi} \left[ \frac{\theta_\bx ( J_r(\piexp | \bx) - r(\tau))}{\sigma_{e,\bx} + \theta_\bx r(\tau)} \middle| \bx \right] \nonumber\\
    &\overset{(i)}{\le} 2 \theta_\bx^2 \nonumber\\
    &\le 8 \xi \label{eq:b0b'1}
\end{align}
where $(i)$ follows from \Cref{lemma:Abound} and the last inequality relies on the choice of $\theta_\bx \le 2 \sqrt{\xi}$. Combining \cref{eq:b1b'0,eq:b0b'1} with an expectation over $\bx \sim \rho$ results in a proof of the first assertion.

\paragraph{Assertion 2: Bounding the value gap.} Observe that $J_r (\pi_{\bm{z}} | \bx) - J_r(\pi_{\bm{z}'} | \bx) = 0$ for any $\bx \in \mathcal{X}_i$ if $\bm{z}_i = \bm{z}'_i$. In case $\bm{z}_i = 1$ and $\bm{z}'_i = 0$ and any $\bx \in \mathcal{X}_i$, $r_{\bm{z}} (\tau) = r (\tau)$ for any $\tau$ which visits $\bx$ and,
\begin{align}
    J_{r_{\bm{z}}} (\pi_{\bm{z}} | \bx) - J_{r_{\bm{z}}} (\pi_{\bm{z}'} | \bx)
    &= \frac{\mathbb{E}_\pi [\sigma_{e} r (\tau) + \theta_\bx (r (\tau))^2 | \bx ]}{\mathbb{E}_\pi [\sigma_{e,\bx} + \theta_\bx r(\tau) | \bx ]} - J_{r_{\bm{z}}} (\piexp | \bx) \nonumber\\
    &= \frac{\sigma_{e,\bx} J_r (\piexp | \bx) + \theta_\bx (J_r(\piexp | \bx))^2 + \sigma_{e,\bx}^2)}{\sigma_{e,\bx} + \theta_\bx J_r(\piexp | \bx)} - J_r(\piexp | \bx) \nonumber\\
    &= \frac{\theta_\bx \sigma_{e,\bx}^2}{\sigma_{e,\bx} + \theta_\bx J_r(\piexp | \bx)} \nonumber\\
    &= \sigma_{e,\bx} \sqrt{\xi} \label{eq:883}
\end{align}
where the last equation follows by choice of $\theta_\bx$. When $\bm{z}_i = 0$ and $\bm{z}'_i = 1$, the same analysis results in the same bound $J_r (\pi_{\bm{z}} | \bx) - J_r(\pi_{\bm{z}'} | \bx) = \sigma_{e,\bx} \sqrt{\xi}$ for any $\bx \in \mathcal{X}_i$, and taking an expectation over $\bx \sim \rho$ proves the second assertion.

\paragraph{Assertion 3: Bound on variance of $\pi_{\bm{z}}$.} This follows from \Cref{eq:var-ub}, which bounds the variance of a policy which lies within a radius $\kappa$ $\chi^2$ ball of another: in particular, $\pi_{\bm{z}} (\cdot|\bx)$ lies in a $\xi$-sized KL ball around $\piexp (\cdot|\bx)$, which has variance $\sigma_{e,\bx}^2$, and taking an expectation over $\bx \sim \rho$. Note also that the reward $r_{\bm{z}}$ preserves variances across policies compared to $r$ (cf. \Cref{lemma:1-r} and the fact that $r_{\bm{z}}$ uses either $r$ or $\widetilde{r}$), so it suffices to carry out the variance computation under $r$.

\paragraph{Assertion 4: Bound on $\chisq{\pi}{\pibase}$ for $\pi \in \Pi_k$.} For any $\bm{z} \in \{ 0,1 \}^k$, note that $\pi_{\bm{z}}$ and $\piexp$ have density ratio upper bounded by,
\begin{align*}
    \left\| \frac{\pi_{\bm{z}} (\tau|\bx)}{\piexp(\tau|\bx)} \right\|_\infty &\le  \frac{\sigma_{e,\bx} + \theta_\bx H}{\sigma_{e,\bx} + \theta_\bx J_r (\piexp|\bx)} \\
    &\le 1+\frac{2\sqrt{\xi} H}{\sigma_{\min}}
\end{align*}
This upper bound on the density ratio implies that,
\begin{align*}
    \chisq{\pi_{\bm{z}}}{\pibase} &= \mathbb{E}_{\bx \sim \rho} \left[ \chisq{\pi_{\bm{z}}(\cdot|\bx)}{\pibase (\cdot|\bx)} \right] \\
    &\le \left( 1 + \frac{2\sqrt{\xi} H}{\sigma_{\min}} \right)^2  (1+\chisq{\piexp}{\pibase}) - 1 \\
    &\le 3 (1+\varepsilon) \cdot \max \left\{ \frac{\sqrt{\xi} H}{\sigma_{\min}} , \frac{\xi H^2}{\sigma_{\min}^2} \right\}
\end{align*}
\end{proof}


\begin{lemma} \label{lemma:GV}
There exists a subset $\mathcal{Z} \subseteq \{ 0,1 \}^k$ with $|\mathcal{Z}| = \lceil 2^{k/4} \rceil$ and such that every pair $\bm{z}, \bm{z}' \in \mathcal{Z}$ satisfies,
\begin{equation*}
    \sum_{i=1}^k \mathbb{I} (\bm{z}_i \ne \bm{z}'_i) \ge k/4
\end{equation*} 
\end{lemma}
\begin{proof}
This statement essentially follows from the Gilbert-Varshamov bound (cf. Theorem~5.2.6 in \cite{LingXing2004}).
\end{proof}


\subsubsection{Construction of policy class \texorpdfstring{$\Pi^\prime$}{} and reward class \texorpdfstring{$\mathcal{R}^\prime$}{}} \label{subsubsec:PIR}

Consider the set of policies $\Pi^\prime = \{ \pi_{\bm{z}} : \bm{z} \in \mathcal{Z} \} \subseteq \Pi_k$  and $\mathcal{R}^\prime = \{ r_{\bm{z}} : \bm{z} \in \mathcal{Z} \}$ (see the proof of \Cref{lem:packing} for a definition of $\pi_{\bm{z}}$, $\Pi_k$ and $\mathcal{R}_k$). By \Cref{lemma:GV}, $|\Pi^\prime| \approx 2^{k/4}$, and furthermore, for any $\bm{z}, \bm{z}' \in \mathcal{Z}$,
\begin{align} \label{eq:91222}
    J_{r_{\bm{z}}} (\pi_{\bm{z}}) - J_{r_{\bm{z}}} (\pi_{\bm{z}'}) \ge L_k^\star \sqrt{\xi} 
\end{align}
where $L_k^\star$ is defined in \Cref{eq:Lstar}. This bound follows from the first assertion in \Cref{lem:packing} and the fact that $\bm{z}$ and $\bm{z}'$ differ in at least $k/4$ coordinates; $L_k^\star$, by definition, captures the deviation for the worst-case choice of $k/4$ coordinates.

\begin{definition}[\cite{chen2016bayes,rajaraman2024statistical}]
The $\chi^2$-informativity is defined as,
\begin{align*}
I_{\chi^2}(X ; Y) \triangleq \inf _{Q_Y} \chi^2\left(P_{X Y} \| P_X \times Q_Y\right)    
\end{align*}
\end{definition}

\begin{theorem}
Consider the family of policies $\Pi^\prime$ defined above. Let $p_{\Pi^\prime}$ denote the uniform prior over them (alternately, the distribution over $\pi_{\bm{z}}$ for $\bm{z} \sim \text{Unif} (\mathcal{Z})$). Let the policy $\hat{\pi}$ be constructed via a dataset $D$ and assume that the verifier-free learner is realizable, satisfying $\hat{\pi}_n^{\text{vf}} \in \Pi^\prime$. Then,
\begin{equation*}
    \Pr ( J_{r_{\bm{z}}} (\pi_{\bm{z}}) - J_{r_{\bm{z}}} (\hat{\pi}_n^{\text{vf}}) \ge L_k^\star\sqrt{\xi} ) \ge 1 - \frac{1}{|\Pi^\prime|} \sqrt{I_{\chi^2} (\bm{z}; D) + 1}
\end{equation*}
\end{theorem}
\begin{proof}
Let $P$ be the joint distribution of $\bm{z}$ and $D$. Let $Q$ be the distribution $\text{Unif} (\mathcal{Z}) \times Q_{\text{data}}$ for a generic (arbitrary) data distribution $Q_{\text{data}}$. Let $T : (\bm{z}, D) \mapsto \mathbb{I} ( J_{r_{\bm{z}}} (\pi_{\bm{z}}) - J_{r_{\bm{z}}} (\hat{\pi}_n^{\text{vf}}) \ge L_k^\star \sqrt{\xi})$ be a generic map, and $P \circ T^{-1}$ and $Q \circ T^{-1}$ be the pushforward measures of $P$ and $Q$ by $T$. Letting $\mathcal{E} (\bm{z},D) = \{ J_{r_{\bm{z}}} (\pi_{\bm{z}}) - J_{r_{\bm{z}}} (\hat{\pi}_n^{\text{vf}}) \ge L_k^\star \sqrt{\xi}) \}$, the data-processing inequality gives,
\begin{align}
    \chisq{P}{Q} &\ge \chisq{P \circ T^{-1}}{Q \circ T^{-1}} \nonumber\\
    &= \frac{(P (\mathcal{E} (\bm{z},D)) - Q (\mathcal{E} (\bm{z},D)))^2}{Q(\mathcal{E} (\bm{z},D))(1 - Q(\mathcal{E} (\bm{z},D)))} \label{eq:443}
\end{align}
Let us assume that the learner's policy $\hat{\pi}$ is realizable, and satisfies $\hat{\pi} \in \Pi^\prime$.
By the product structure of $Q$, we have that,
\begin{align*}
    Q(\mathcal{E} (\bm{z},D)) \le \sup_{\pi \in \Pi^\prime} \Pr \left( J_{r_{\bm{z}}} (\pi_{\bm{z}}) - J_{r_{\bm{z}}} (\pi) \ge L_k^\star \sqrt{\xi} \right) = 1-\frac{1}{|\Pi^\prime|}.
\end{align*}
where the last inequality uses the fact that for any $\bm{z}' \ne \bm{z}$, $J_{r_{\bm{z}}} (\pi_{\bm{z}}) - J_{r_{\bm{z}}} (\pi_{\bm{z}'}) \ge L_k^\star \sqrt{\xi}$ (cf. \Cref{eq:91222}). Combining with \cref{eq:443}, rearranging, simplifying and taking the infimum over $Q_{\text{data}}$ completes the proof.
\end{proof}

\begin{lemma}
Consider any realizable verifier-free learner, satisfying $\hat{\pi}_n^{\text{vf}} \in \Pi^\prime$. Then,
\begin{align*}
    \Pr \left( J_{r_{\bm{z}}} (\pi_{\bm{z}}) - J_{r_{\bm{z}}} (\hat{\pi}_n^{\text{vf}}) \ge L_k^\star\sqrt{\frac{\log(|\Pi^\prime|)}{16n}} \right) \ge \frac{1}{4}
\end{align*}
\end{lemma}
\begin{proof}
Observe that,
\begin{align*}
    I_{\chi^2} (\bm{z}; D) + 1 &= \inf_{Q_{\text{data}}} \int \left[ \frac{(p_\Pi (\pi_{\bm{z}}))^2 \left( \prod_{\tau \in D} \pi_{\bm{z}} (\tau) \right)^2}{p_\Pi (\pi_{\bm{z}}) Q_{\text{data}} (D)} \right] dD d\pi \\
    &\overset{(i)}{\le} \int \left[ \frac{p_\Pi (\pi_{\bm{z}}) \left( \prod_{\tau \in D} \pi_{\bm{z}} (\tau) \right)^2}{\prod_{\tau \in D} \piexp (\tau)} \right] dD d\pi \\
    &= \int \left[ \frac{p_\Pi (\pi_{\bm{z}}) \left( \prod_{\tau \in D} \pi_{\bm{z}} (\tau) \right)^2}{\prod_{\tau \in D} \piexp (\tau)} \right] dD d\pi \\
    &= \mathbb{E}_{\pi \sim p_\Pi} [ (1+\chisq{\pi_{\bm{z}}}{\piexp})^n ] \\
    &\overset{(ii)}{\le} (1+8\xi)^n
\end{align*}
where in $(i)$ we choose $Q_{\text{data}}$ as the data distribution realized by $\piexp$ and in $(ii)$, we use the first assertion of \Cref{lem:packing}.
Choose $\xi = \varepsilon_{\text{stat}} = \frac{\log(
|\Pi^\prime|)}{16n}$, we get,
\begin{align*}
    \Pr \left( J_{r_{\bm{z}}} (\pi_{\bm{z}}) - J_{r_{\bm{z}}} (\hat{\pi}_n^{\text{vf}}) < L_k^\star\sqrt{\frac{\log(|\Pi^\prime|)}{n}} \right) \ge \frac{1}{4}
\end{align*}
\end{proof}






























































\subsection{Bounding the performance of \Cref{alg:algorithm_simple_vb}}
\label{subsec:proof-verifier-based-thm}

\subsubsection{Understanding the anti-concentration assumption}

Recall that the anticoncentration assumption controls the probability of the reward $r(\tau)$ for $\tau \sim \pibase(\cdot|\bx)$ of exceeding its mean by a margin of $\sqrt{\varepsilon}$ times its standard deviation. Namely,
{
    \setlength{\abovedisplayskip}{5pt}
    \setlength{\abovedisplayskip}{5pt}
    \begin{align*}
    \!\!\!c_\bx (\varepsilon) \eqdef \text{Pr}_{\pibase(\cdot|\bx)} \big( r(\tau) \ge \E_{\pibase(\cdot|\bx)}\brck{r(\tau)} + \sigma_{b, \bx} \sqrt{\varepsilon} \big).
    \end{align*}
}
The interpretation of $c_\bx (\varepsilon)$ is natural, as a prompt-conditional measure of anticoncentration of the rewards $r(\tau)$ collected by the base policy. However, as we discuss in the next lemma, the deviation term $\E_{\pibase(\cdot|\bx)}\brck{r(\tau)} + \sigma_{b, \bx} \sqrt{\varepsilon}$ serves a dual purpose: it precisely captures the maximum value achievable in a $\chi^2$ ball around $\pibase$ of radius $\varepsilon$.

\begin{lemma}[Characterizing the optimal value within the $\chi^2$ ball] \label{lemma:characterization} For a single prompt $\bx \in \mathcal{X}$, consider the set of policies $\Pi_{\varepsilon,\bx} = \{ \pi : \chisq{\pi(\cdot|\bs)}{\pibase(\cdot|\bx)} \le \varepsilon \}$. Then,
\begin{equation}
    \sup_{\pi \in \Pi_{\varepsilon,\bx}} \E_{\tau \sim \pi(\cdot|\bx)} [r(\tau)] \ge \E_{\pibase(\cdot|\bx)} [r(\tau)] + \sigma_{b,\bx} \sqrt{\varepsilon}.
\end{equation}
Furthermore, as long as $\varepsilon \le \frac{\sigma_{b,\bx}^2}{(J_r(\pibase|\bx))^2}$, this inequality is an equality.
\end{lemma}
\begin{proof}
Consider the candidate policy $\pi (\tau|\bx) \propto (\sigma_{b,\bx} + \theta r(\tau)) \pibase(\cdot|\bx)$ for $\theta$ to be chosen later. Mirroring the calculation in \Cref{eq:b1b'0} (with $\piexp$ replaced by $\pibase$), we see that,
\begin{align*}
    \chisq{\pi(\cdot|\bx)}{\pibase(\cdot|\bx)} = \frac{\theta^2 \sigma_{b,\bx}^2}{(\sigma_{b,\bx} + \theta J_r (\pibase|\bx))^2}
\end{align*}
The maximum achievable value of the $\chi^2$ divergence by this policy is $\frac{\sigma_{b,\bx}^2}{(J_r (\pibase|\bx))^2}$. Likewise, mirroring the calculation in \cref{eq:883},
\begin{align*}
    J_r (\pi|\bx) - J_r(\pibase|\bx) = \frac{\theta \sigma_{b,\bx}^2}{\sigma_\bx + \theta J_r(\pibase|\bx)} = \sigma_{b,\bx} \sqrt{\chisq{\pi(\cdot|\bx)}{\pibase(\cdot|\bx)}} = \sigma_{b,\bx} \sqrt{\varepsilon}
\end{align*}
Therefore, with the appropriate choice of $\theta$, this policy is a feasible policy achieving the supremum in the statement. What remains is to show that the supremum can be no larger. By \Cref{lemma:CoM}, with the choice of $Y = r(\tau)$, $P$ as the distribution over $\tau$ induced by $\pi(\cdot|\bx)$ and $Q$ the distribution over trajectories induced by $\pibase(\cdot|\bx)$. Then,
\begin{align*}
    \left| \mathbb{E}_{\tau \sim \pi(\cdot|\bx)} [r(\tau)] - \mathbb{E}_{\tau \sim \pibase(\cdot|\bx)} [r(\tau)] \right| &\le \sqrt{\Var_{\tau \sim \pibase(\cdot|\bx)} [r(\tau)] \cdot \chisq{\pi(\cdot|\bx)}{\pibase(\cdot|\bx)}} = \sigma_{b,\bx} \sqrt{\varepsilon}
\end{align*}
This shows that the supremizing value is exactly $\sigma_{b,\bx} \sqrt{\varepsilon}$.
\end{proof}


    \begin{property}[Regularity] \label{prop:reg}
Assume that for each $\bx \in \mathcal{X}$ that $J_r(\pibase|\bx) > 0$ and,
\begin{align*}
    \varepsilon_\bx \eqdef \chisq{\bar{\pi}_\kappa (\cdot|\bx)}{\pibase(\cdot|\bx)} \le \frac{\sigma_{b,\bx}^2}{(J_r(\pibase|\bx))^2}.
\end{align*}
where $\bar{\pi}_\kappa$ is any policy which collects the maximum value, while remaining within $\Pi_\kappa$.
\end{property}

\begin{lemma} \label{lemma:pi_lambda}
Suppose $\pibase$ is $c_0$-anticoncentrated for some problem horizon $h_0$ and assume that \Cref{prop:reg} holds true for the base policy at this value of $h_0$. Define a collection of parameters, $\lambda = \{ \lambda_\bx : \bx \in \mathcal{X} \}$ where $\mathbb{R} \ni \lambda_\bx \in (0,\sigma_b \sqrt{2/c_0}]$. Then, there exists a policy $\pi_c$ such that,
\begin{enumerate}
    \item Almost surely, $r(\tau) > 0$ for $\tau \sim \pi_c (\cdot|\bx)$ and any $\bx \in \mathcal{X}$.
    \item $\pi_c$ is no worse than $\piexp$. Namely, $J_r (\pi_c) \ge \sup_{\pi \in \Pi_\kappa} J_r(\pi) \ge J_r (\piexp) $.
    \item For every $\bx \in \mathcal{X}$, $\sup_{\tau : \Pr_{\pibase} (\tau|\bx) > 0} \frac{\Pr_{\pi_c} (\tau|\bx)}{\Pr_{\pibase} (\tau|\bx)} \le c_0^{-1}$ %
\end{enumerate}
\end{lemma}
\begin{proof}
Fix a prompt $\bx \in \mathcal{X}$. We will construct $\pi_c$ separately for each prompt and later argue about each of these three assertions. Since $\pibase$ is $c_0$-anticoncentrated for some problem horizon $h_0$, as long as $\varepsilon_\bx \eqdef \chisq{\bar{\pi}_\kappa (\cdot|\bx)}{\pibase(\cdot|\bx)} \le \frac{\sigma_{b,\bx}^2}{(J_r(\pibase|\bx))^2}$, by \Cref{lemma:characterization}, defining $\mathcal{T}$ as the set of trajectories $\{ r(\tau) \ge \sup_{\pi \in \Pi_{\varepsilon_\bx,\bx}} \mathbb{E}_{\tau\sim\pi(\cdot|\bx)} [r(\tau)] \}$. Then,
\begin{align} \label{eq:plb}
    \Pr_{\tau \sim \pibase(\cdot|\bx)} \left( \tau \in \mathcal{T} \right) \ge c_0
\end{align}

Consider the policy $\pi_c (\cdot|\bx)$ which is the mixture over the trajectories $\mathcal{T} = \{ \tau : r(\tau) \geq  \E_{\tau \sim \pibase(\cdot \mid \bx)}[r(\tau)] + \sigma_{b,\bx} \sqrt{\varepsilon_\bx}\}$ with mixture weights $w_\tau \propto \Pr_{\pibase (\cdot|\bx)} (\tau)$. Since the MDP is autoregressive (i.e., tree-like), $\pi_c (\cdot|\bx)$ corresponds to a simple policy (as opposed to a mixture over policies), since two trajectories in $\mathcal{T}$ can not visit the same state again after a different action is played between them, i.e., a breakpoint. This implies that the mixture of these two trajectories is the same as the policies which agrees with them until the breakpoint and picks one of the trajectories to follow at the breakpoint, proportional to its weight. The same argument applies when considering a mixture over more than two trajectories. Next, we prove the three assertions of this lemma.

\paragraph{Assertion 1: Rewards are strictly positive.} $\pi_c(\cdot|\bx)$ is only supported on trajectories which collect rewards which exceed $\sup_{\pi \in \Pi_{\varepsilon_\bx,\bx}} \mathbb{E}_{\tau\sim\pi(\cdot|\bx)} [r(\tau)] \ge \mathbb{E}_{\tau\sim\pibase(\cdot|\bx)} [r(\tau)]$. By \Cref{prop:reg}, we have that $\mathbb{E}_{\tau \sim\pibase(\cdot|\bx)} [r(\tau)] > 0$; this implies that the reward collected by every such trajectory is not only strictly positive, but must be at least $1$ (by the bi-level property of the rewards).


\paragraph{Assertion 2: Value bound.} $\pi_c (\cdot|\bx)$ is supported on trajectories which collect reward at least: $$\sup_{\pi \in \Pi_{\varepsilon_\bx,\bx}} \E_{\tau \sim \pi(\cdot|\bx)} [r(\tau)].$$ Thus, with probability $1$, for any trajectory $\tau$ sampled from $\pi_c (\cdot|\bx)$, $r(\tau) \ge \sup_{\pi \in \Pi_{\varepsilon_\bx,\bx}} \E_{\tau \sim \pi(\cdot|\bx)} [r(\tau)]$. Taking an expectation over $\tau \sim \pi_c (\cdot|\bx)$, we get, $\E_{\tau \sim \pi_c (\cdot|\bx)} [r(\tau)] \ge \sup_{\pi \in \Pi_{\varepsilon_\bx,\bx}} \E_{\tau \sim \pi(\cdot|\bx)} [r(\tau)]$. Further, taking an expectation over $\bx \sim \rho$,
\begin{align*}
    \E_{\rho, \pi_c} [r(\tau)] &\ge \E_{\bx \sim \rho} \left[ \sup_{\pi \in \Pi_{\varepsilon_\bx,\bx}} \E_{\tau \sim \pi(\cdot|\bx)} [r(\tau)] \right] \\
    &\ge \sup_{\pi \in \bigcap_{\bx \in \mathcal{X}} \Pi_{\varepsilon_\bx,\bx}} \E_{\rho,\pi} [r(\tau)] \\
    &= \sup_{\pi \in \Pi_{\kappa}} \E_{\rho,\pi} [r(\tau)]
\end{align*}
where the last equation follows by definition of $\varepsilon_\bx$ (cf. \Cref{prop:reg}).

\paragraph{Assertion 3: Bounds on coverage.} Note that $\pi_c (\cdot|\bx)$ is the policy $\sum_{\tau \in \mathcal{T}} w_\tau \delta_\tau$. In particular, for any trajectory $\tau$ in the support of $\pi_c (\cdot|\bx)$,
\begin{align}
    \frac{\Pr_{\pi_c} (\tau|\bx)}{\Pr_{\pibase} (\tau|\bx)} = \frac{w_\tau}{\Pr_{\pibase(\cdot|\bx)} (\tau)} = \frac{1}{\sum_{\tau \in \mathcal{T}} \Pr_{\pibase (\cdot|\bx)} (\tau)}
\end{align}
where the last equation follows by definition of $w_\tau$. By \cref{eq:plb}, $\sum_{\tau \in \mathcal{T}} \Pr_{\pibase(\cdot|\bx)} (\tau) \ge c_0$. This completes the proof of the last assertion.
\end{proof}

\begin{lemma} \label{lemma:pi_lambda_tilde}
Suppose $\pi_b$ is $c_0$-anticoncentrated for some problem horizon $h_0$ and assume that \Cref{prop:reg} holds true for the base policy $\pibase$ at this value of $h_0$. Consider the policy $\pi_c$ introduced in \Cref{lemma:pi_lambda} at this value $h_0$. For any horizon $H > h_0$, there exists a policy $\widetilde{\pi}_c$ which satisfies essentially the same conditions,
\begin{enumerate}
    \item Almost surely, $r(\tau) > 0$ for $\tau \sim \widetilde{\pi}_c (\cdot|\bx)$ for any $\bx \in \mathcal{X}$,
    \item $\widetilde{\pi}_c$ is no worse than $\piexp$ when deployed on horizon $H$. Namely, $J_r^H (\widetilde{\pi}_c) \ge \sup_{\pi \in \Pi_\kappa^H} J_r^H (\pi) \ge J_r^H (\piexp) $.
    \item $\sup_{\tau : \Pr_{\pibase} (\tau|\bx) > 0} \frac{\Pr_{\pi_c} (\tau|\bx)}{\Pr_{\pibase} (\tau|\bx)} \le c_0^{-1}$. %
\end{enumerate}
Here, we point out that the in the third assertion (coverage), $(a)$ trajectories $\tau$ are of length $H$, and $(b)$ the variance term $\sigma_b (h_0)$ that appears is that of the base policy evaluated on the horizon $h_0$. Everywhere, we take care to superscript $J_r$ and $\Pi_\kappa$ to indicate the horizon over which the policies are considered.
\end{lemma}
\begin{proof}
Consider the ``extension'' of $\pi_c$, defined till time $h_0$, by $\pibase$ (which we assume is defined for every $t \in \mathbb{N}$). Namely, consider the policy $\widetilde{\pi}_c$ which follows $\pi_c$ till time $h_0$ and plays actions according to $\pibase$ thereon.

The first three assertions follow from the fact that $\pi_c$ is only supported on trajectories with strictly positive reward. By the bi-level property, each of these trajectories collect $1$ unit of reward at every $t > h_0$. Thus, $J^H_r (\widetilde{\pi}_c) = J^{h_0}_r (\widetilde{\pi}_c) + (H-h_0)$, while $\sup_{\pi \in \Pi_\kappa^H} J_r^H (\pi) \le \sup_{\pi \in \Pi_\kappa^{h_0}} J_r^H (\pi) + (H-h_0)$. This follows from the fact that the supremizing policy for the $H$ horizon problem can be truncated to the first $h_0$ steps to result in a candidate policy in $\Pi_\kappa^{h_0}$; in the process the value of the policy decreases by at most $H-h_0$. The last assertion follows from the fact that $\widetilde{\pi}_c$ and $\pibase$ agree after time $h_0$, so the worst-case density ratio cannot increase as $H$ increases beyond $h_0$.
\end{proof}










\subsubsection{Analysis of \Cref{alg:algorithm_simple_vb}: Proof of \Cref{thm:verifier-based-thm}}

Below, we provide implementation details of \Cref{alg:algorithm_simple_vb} and a slightly more formal version of \Cref{thm:verifier-based-thm}. We will define the confidence set $\hat{R}_\gamma$ below, and choose $\gamma$ appropriately as any upper bound to $\est(\delta)$ (see \cref{eq:estoff}). One such upper bound is provided in \Cref{lemma:estoff}. For the purpose of this section, we will assume that \Cref{alg:algorithm_simple_vb} carries out least square estimation with respect to some reward class $\gR_{\text{vb}}$ such that $r$ belongs to this class, and may be a subset or superset of the set of all bi-level rewards, $\mathcal{R}$.

\begin{theorem}[Formal version of \Cref{thm:verifier-based-thm}]
Consider a bi-level reward $r$,  base policy $\pibase$ that is $c_0$-anticoncentrated at some horizon $h_0 \leq H$ and assume that \Cref{prop:reg} is satisfied at $h_0$. Suppose the verifier is used to label the cumulative reward of every trajectory and results in a dataset of noisy reward annotations, $\{ (\bx_i,\tau_i,y_i) \}_{i=1}^n$: assume that the reward annotations are of the form $y_i = r(\tau_i) + Z_i$ where the $Z_i$'s are independent and standard normal with trajectory level variance $\Var[Z_i] \le \sigma_{\text{noise}}^2$. Then, the policy $\hat{\pi}_n^{\mathrm{vb}}$ returned by \Cref{alg:algorithm_simple_vb},  
the suboptimality gap w.r.t. the best expert $\bar{\pi}_\kappa \in \Pi_\kappa$ satisfies: with probability $\ge 1 - \delta$,
\begin{align*}   
    &J_r(\bar{\pi}_\kappa) - J_r(\hat{\pi}_{n}^\mathrm{vb}) \; \lsim \;  \frac{(H + \sigma_{\text{noise}}^2) \log(\nicefrac{|\gR_{\text{vb}}|}{\delta})}{n c_0},
\end{align*}  
With independent $O(1)$-variance noise at steps of a trajectory, note that $\sigma_{\text{noise}}^2 \le O(H)$.
\end{theorem}


Below we instantiate the confidence set $\hat{R}_\gamma$ in \Cref{alg:algorithm_simple_vb}. Recall that we assume that \Cref{alg:algorithm_simple_vb} carries out least square estimation with respect to some reward class $\gR_{\text{vb}}$: with $\hat{r}_{\text{ls}}$ as the least squares estimator,
\begin{align*}
    \hat{r}_{\text{ls}} &\gets \inf_{r' \in \gR_{\text{vb}}} \frac{1}{n} \sum_{i=1}^n (r'(\tau_i) - y_i)^2 \\
    \widetilde{R}_\gamma &= \left\{ r' \in \gR_{\text{vb}} \middle| \frac{1}{n} \sum_{i=1}^n (r'(\tau_i) - \hat{r}_{\text{ls}} (\tau_i))^2 \le \gamma \right\} \\
    \hat{R}_\gamma &= \left\{ \{ \textsf{round} (r'(\cdot)) \} : r' \in \widetilde{R}_\gamma \right\}
\end{align*}
Where $\textsf{round} (r(\cdot))$ is the ``rounding'' of the reward $r$, for every $\tau$, $r(\tau)$ is rounded to the nearest integer, breaking ties arbitrarily. We define the offline estimation error of the least-squares estimator below. Define $\mathcal{E}_\delta$ as the event,
\begin{align} \label{eq:estoff}
    \frac{1}{n} \sum_{i=1}^n (\hat{r}_{\text{ls}} (\tau_i) - r (\tau_i))^2 \le \est (\delta)
\end{align}
And suppose $\Pr (\mathcal{E}_\delta) \ge 1-\delta$ where the probability is computed over the randomness of the training dataset $\{ (\bx_i,\tau_i) \}_{i=1}^n$.

The analysis of the verifier-based learner in \Cref{alg:algorithm_simple_vb} follows the standard analysis of pessimism-based algorithms. For an arbitrary comparator policy $\pi_c$,
\begin{align}
    J_r (\pi_c) - J_r ( \hat{\pi}_n^{\text{vb}} ) &\le J_r (\pi_c) - \min_{\hat{r} \in \hat{R}_\gamma} J_{\hat{r}} ( \hat{\pi}_n^{\text{vb}} ) \nonumber\\
    &\le J_r (\pi_c) - \min_{\hat{r} \in \hat{R}_\gamma} J_{\hat{r}} ( \pi_c ) \nonumber\\
    &\le \sup_{\hat{r} \in \hat{R}_\gamma} \mathbb{E}_{\rho, \pi_c} \left[ | r (\tau) - \hat{r} (\tau) | \right]  \label{eq:003}
\end{align}
With the choice of the comparator policy $\pi_c = \widetilde{\pi}_c$, as defined in \Cref{lemma:pi_lambda_tilde},
\begin{align*}
    \sup_{\pi \in \Pi_\kappa} J_r (\pi) - J_r ( \hat{\pi}_n^{\text{vb}} ) &\le \sup_{\hat{r} \in \hat{R}_\gamma} c_0^{-1} \mathbb{E}_{\rho, \pibase} \left[ | r (\tau) - \hat{r} (\tau) | \right].
\end{align*}
where note that the base policy is assumed to be $c_0$-anticoncentrated for the horizon $h_0$. The performance of the algorithm thus relies on establishing a generalization bound for the reward estimation problem, which is proved below in \Cref{theorem:finalerror}. In conjunction, this results in the upper bound: with probability $1-\delta$,
\begin{align*}
\sup_{\pi \in \Pi_\kappa} J_r (\pi) - J_r ( \hat{\pi}_n^{\text{vb}} ) &\le \gO \left( \frac{(H + \sigma_{\text{noise}}^2) \cdot \log (\nicefrac{|\mathcal{R}_{\text{vb}}|}{\delta})}{c_0 n} \right)
\end{align*}

\begin{theorem} \label{theorem:finalerror}
Recall that the reward annotations are of the form $y_i = r(\tau_i) + Z_i$ where the noise $Z_i$ is assumed to be independent and standard normal with trajectory level variance $\sigma_{\text{noise}}^2$.
Consider any $\delta \in (0,1)$. Then, with probability $1-\delta$, simultaneously for all $r' \in \hat{R}_{\gamma}$,
\begin{align*}
    \E_{\rho,\pibase} [|r(\tau) - r'(\tau)|] \le \gO \left( \frac{(H + \sigma_{\text{noise}}^2) \cdot \log (\nicefrac{|\mathcal{R}_{\text{vb}}|}{\delta})}{n} \right)
\end{align*}
Note that with independent noise at each step, $\sigma_{\text{noise}}^2 \le O(H)$.
\end{theorem}
\begin{proof}
This result is a direct combination of \Cref{lemma:estoff,lemma:gb}.
\end{proof}


\begin{lemma}[Lemma C.1 in \cite{foster2024online}] \label{lemma:estoff}
It suffices to choose,
\begin{align} \label{eq:estoff-bd}
    \est (\delta) = \frac{8 \sigma_{\text{noise}}^2 \log(\nicefrac{|\gR_{\text{vb}}|}{\delta})}{n}
\end{align}
to guarantee that $\Pr (\mathcal{E}_\delta) \ge 1 - \delta$.
\end{lemma}

\begin{lemma} \label{lemma:empl1bound}
With the choice $\gamma = \est (\delta)$, under the event $\mathcal{E}_\delta$, $r \in \hat{R}_\gamma$. Under the same event, for every reward $r'' \in \hat{R}_\gamma$,
\begin{align*}
    \frac{1}{n} \sum_{i=1}^n |r''(\tau_i) - r (\tau_i)| \le 16 \cdot \est(\delta)
\end{align*}
\end{lemma}
\begin{proof}
The first assertion follows by definition of $\widetilde{R}_\gamma$ and \Cref{eq:estoff}, and the fact that $r$ is a bi-level reward, so it is unperturbed by the $\textsf{round} (\cdot)$ operation. For the second assertion: under $\mathcal{E}_\delta$, for any reward $r' \in \widetilde{R}_\gamma$,
\begin{align} \label{eq:332}
    \frac{1}{n} \sum_{i=1}^n (r'(\tau_i) - r(\tau_i))^2 \le \frac{2}{n} \sum_{i=1}^n (r'(\tau_i) - \hat{r}_{\text{ls}} (\tau_i))^2 + (r (\tau_i) - \hat{r}_{\text{ls}}(\tau_i)) )^2 \le 4\est (\delta)
\end{align}
Consider the $r'' = \textsf{round} (r') \in \hat{R}_\gamma$, for this choice of reward, observe that $r'' (\tau) - r(\tau) \in \mathbb{Z}$, since both rewards only take integer values. Furthermore, $(a)$ if $|r' (\tau) - r (\tau)| < 1/2$, then we know that $r''(\tau) - r (\tau) = 0$ surely, and $(b)$ if $|r' (\tau) - r (\tau)| \ge 1/2$, then $|r''(\tau) - r (\tau)| \le 2 |r' (\tau) - r (\tau)|$. This implies,
\begin{align*}
    \frac{1}{n} \sum_{i=1}^n |r''(\tau_i) - r (\tau_i)| &= \frac{1}{n} \sum_{i=1}^n |r''(\tau_i) - r (\tau_i)| \cdot \mathbb{I} (|r' (\tau) - r (\tau)| > 1/2)\\
    &\le \frac{2}{n} \sum_{i=1}^n |r'(\tau_i) - r (\tau_i)| \cdot \mathbb{I} (|r' (\tau) - r (\tau)| > 1/2)\\
    &\le \frac{4}{n} \sum_{i=1}^n |r'(\tau_i) - r (\tau_i)|^2 \cdot \mathbb{I} (|r' (\tau) - r (\tau)| > 1/2)\\
    &\le 16 \cdot \est(\delta)
\end{align*}
where the last inequality follows from \cref{eq:332}.
\end{proof}

\subsubsection{Proof of Proposition~\ref{prp:verifier-accuracy}}

\begin{lemma}[Generalization bound for learning in $L_1$-error] \label{lemma:gb}
With probability $1-2\delta$, simultaneously for all $r' \in \hat{R}_{\gamma}$,
\begin{align*}
    \E_{\rho,\pibase} [|r(\tau) - r'(\tau)|] \le \gO \left( \frac{H \cdot \log (\nicefrac{|\mathcal{R}_{\text{vb}}|}{\delta})}{n} + \est (\delta) \right)
\end{align*}
\end{lemma}
\begin{proof}
For any fixed reward $r' \in \mathcal{R}_{\text{vb}}$, by Bernstein concentration, with probability $\ge 1 - \delta$,
\begin{align*}
    \E_{\rho,\pibase} [|r(\tau) - r'(\tau)|] - \frac{1}{n} \sum_{i=1}^n [|r(\tau_i) - r' (\tau_i)|] &\le \sqrt{\frac{\Var_{\rho,\pibase} [ |r(\tau) - r'(\tau)|] \cdot \log (\nicefrac{1}{\delta})}{n}} \\
    &\le \sqrt{\frac{\E_{\rho,\pibase} [ (r(\tau) - r'(\tau))^2] \cdot \log (\nicefrac{1}{\delta})}{n}} \\
    &\le \sqrt{\frac{H \cdot \E_{\rho,\pibase} [ |r(\tau) - r'(\tau)|] \cdot \log (\nicefrac{1}{\delta})}{n}}
\end{align*}
Union bounding over rewards in $\mathcal{R}_{\text{vb}}$, and choosing an arbitrary $r' \in \hat{R}_\gamma$, by \Cref{lemma:empl1bound}, with probability $\ge 1-2\delta$,
\begin{align*}
    \E_{\rho,\pibase} [|r(\tau) - r'(\tau)|] \le 16 \cdot \est (\delta) + \sqrt{\frac{H \cdot \E_{\rho,\pibase} [ |r(\tau) - r'(\tau)|] \cdot \log (\nicefrac{|\mathcal{R}_{\text{vb}}|}{\delta})}{n}}
\end{align*}
Solving the quadratic equation results in the upper bound: with probability $\ge 1 - 2\delta$,
\begin{align*}
    \forall r' \in \hat{R}_\gamma,\quad \E_{\rho,\pibase} [|r(\tau) - r'(\tau)|] \le \gO \left( \frac{H \cdot \log (\nicefrac{|\mathcal{R}_{\text{vb}}|}{\delta})}{n} + \est (\delta) \right)
\end{align*}

\end{proof}


















\subsection{Proof of Theorem~\ref{thm:vg-gap-lower-bound}}
\label{subsec:proof-vg-gap-lower-bound}

The proof of this result follows directly from the instance lower bound in Theorem~\ref{thm:verifier-free-thm} and suboptimality upper bound result in Theorem~\ref{thm:verifier-based-thm}. When, $\tilde{\sigma}_b =  \Omega(H)$, the lower bound on the suboptimality gap of any VF method scales as $\nicefrac{H\log(|\Pi|)}{n}$, with respect to any expert in a $O(1)$-$\chi^2$ ball around the base policy $\pibase$, where as if $\pibase$ is $c_0$ anti-concentrated, then there exists an algorithm that yields an upper bound on the suboptimality gap of $\nicefrac{H\log|\gR|}{n}$, with constant probability. Thus, in compliance with the definition of scaling test-time compute in Definition~\ref{def:h-alpha-scaling}, as we scale $n=\Omega(H)$, we get the result in Theorem~\ref{thm:main-theorem}.


As an example of such a $\pibase$, consider a single prompt, and a base policy that gets a reward of $1$ with probability $> \frac{3}{5}$ on any trajectory rolled out till horizon $H=H_0$, and that this mass remains constant as we scale $H \rightarrow \infty$, i.e., the fraction of in correct trajectories (in the set $\gS_{H_0}$) remain incorrect no matter how much we rollout $\pibase$. For this distribution, it is easy to see that $\widetilde{\sigma_b} = \Omega(H)$, but is $0.5$-anti-concentrated. 


\subsection{Analyzing Verifier Accuracy Under 0/1 Loss}
\label{subsec:proof-verifier-accuracy}



Consider the following modified version of   Algorithm~\ref{alg:algorithm_simple_vb}.



\begin{algorithm}[ht]
\caption{Simple Verifier-Based Algorithm with $\ell_{0/1}$ loss}
\begin{algorithmic}[1]
\REQUIRE Base policy $\pibase$, dataset $\{(\bx_i, \tau_i)\}_{i=1}^n$ of prompts $\bx_i \sim \rho$ and traces $\tau_i \sim \pibase(\cdot \mid \bx)$.
\STATE For every $\tau_i$  annotate $(\bx_i, \tau_i)$ with bi-level reward $r(\tau_i)$. %
\STATE Learn set of classifiers $\hat{R}_\gamma \subset \gR$ that are $\gamma$-optimal, \textit{i.e.}, 
{
\begin{align*}
    \hat{R}_\gamma \eqdef \cbrck{r' \in \gR \middle| \frac{1}{n}\sum\nolimits_{i=1}^n\ell_{0/1}(r'(\tau_i), r(\tau_i)) \leq \gamma}
\end{align*}
}\STATE Return any optimal pessimistic verifier-based policy,
{
\begin{align*}
    \vspace{-0.1cm}
    \setlength{\abovedisplayskip}{0pt}
    \setlength{\belowdisplayskip}{0pt}
    \hat{\pi}^\mathrm{vb}_n \in \argmax_{\pi \in \Pi} \min_{r \in \hat{R}_\gamma} J_r(\hat{\pi}).
\end{align*}
}
\end{algorithmic}
\end{algorithm}


\begin{proposition}[Verifier accuracy]
    \label{prp:verifier-accuracy-lzone}
    For any bi-level reward $r$, base policy $\pibase$, 
    there exists an algorithm querying the at most reward annotator $n$ times to learn $\hat{r} \in \gR$, s.t. w.p. $1-\delta$,  
    {
    \setlength{\abovedisplayskip}{3pt}
    \setlength{\belowdisplayskip}{3pt}
    \begin{align*}
        \E_{\rho,\pibase} [\ellzone(r(\tau), \hat{r}(\tau))] =  \widetilde{\gO}_n \paren{\frac{\log(\nicefrac{|\mathcal{R}|}{\delta}) \log H }{n}} \eqdef \gamma_{\text{stat}}.
    \end{align*}
    In Algorithm~\ref{alg:algorithm_simple_vb}, setting $\gamma = \gamma_{\text{stat}} \implies r \in \hat{R}_\gamma$ w.p. $\ge 1-\delta$.
    }
\end{proposition}



\begin{definition}[Graph dimension] Let $\mathcal{H}$ be a hypothesis class on an input space $\mathcal{X}$ and label space $\mathcal{Y}$. Let $S \subseteq \mathcal{X}$. The class $\mathcal{H}$ is said to $G$-shatter $S$ if there exists an $f : S \to \mathcal{Y}$ such that for every $T \subseteq S$, there is a $g \in \mathcal{H}$ such that $\forall x \in T,\ g(x) = f (x)$, and $\forall x \in S \setminus T$, $g(x) \ne f (x)$. The graph dimension of $\mathcal{H}$, denoted $d_G (\mathcal{H})$, is the maximal cardinality of a set that is $G$-shattered by $\mathcal{H}$.
\end{definition}

\begin{theorem}[Sample complexity of multiclass classification \cite{daniely2011multiclass}] \label{theorem:multiclass}
There exists an absolute constant $C > 0$ such that for every hypothesis class $\mathcal{H}$, given a $\mathcal{H}$-realizable i.i.d. dataset $D$ of size $n \ge n (\varepsilon)$, where,
\begin{align} \label{eq:nstar}
    n (\varepsilon) = C \left( \frac{d_G (\mathcal{H}) \log ( 1/\varepsilon ) + \log ( 1/\delta)}{\varepsilon} \right),
\end{align}
empirical risk minimization on $D$ with the hypothesis class $\mathcal{H}$ incurs $0$-$1$ loss of at most $\varepsilon$ with probability at least $1-\delta$.
\end{theorem}

\begin{lemma}[Upper bound on the graph dimension]
For any hypothesis class $\mathcal{H}$, $d_G (\mathcal{H}) \le \log_2 (|\mathcal{H}|)$.
\end{lemma}
\begin{proof}
For a set $S \subseteq \mathcal{X}$ to be $G$-shattered by $\mathcal{H}$ if there exists a function $f$ such that for any subset $T \subseteq S$ there exists an discriminator $g_T \in \mathcal{H}$ that agrees with $f$ on $T$ and disagrees with it on $S \setminus T$. Across different choices of the subset $T \subseteq S$, the discriminating $g_T$ cannot be the same: indeed for $T_1 \ne T_2 \subseteq S$, $g_{T_1}$ and $g_{T_2}$ must disagree on points in $(T_1 \setminus T_2) \cup (T_2 \setminus T_1)$, the symmetric difference of the two subsets. This is simply because on points in $T_1 \setminus T_2$, $g_{T_1}$ agrees with $f$ and $g_{T_2}$ disagrees with $f$, while on points in $T_2 \setminus T_1$, $g_{T_2}$ agrees with $f$ and $g_{T_1}$ disagrees with $f$. Since the map $T \to g_T$ is injective, and there are $2^{|S|}$ choices of $T$, this means that $S$ can only be $G$-shattered if $|\mathcal{H}| \ge 2^{|S|}$.
\end{proof}

\begin{theorem} \label{thm:reward-estimation}
Given a dataset of $n(\varepsilon)$ trajectories from $\pibase$, there exists an algorithm which calls the verifier $n(\varepsilon) \lceil \log_2 (H) \rceil$ times and learns a reward model such that,
\begin{align}
    \mathbb{E}_{\rho, \pibase} \left[ \mathbb{I} ( r(\tau) \ne \hat{r} (\tau)) \right] \le \varepsilon.
\end{align}
\end{theorem}
\begin{proof}
Recall that $\mathcal{R}$ is assumed to be a bi-level reward class.
For each $r \in \mathcal{R}$, consider the multiclass classifier $f_r : (\mathcal{S} \times \mathcal{A})^H \to [H+1]$ which maps a trajectory $\tau = \{ (s_1,a_1),\cdots,(s_H,a_H) \}$ to the value of $h \in [H]$ such that $h$ is the first point in the trajectory where $r(s_h,a_h) = 1$, i.e., the location of the bi-level in the trajectory. If the reward stays $0$ entirely through the trajectory, then $f_r (\tau) = H+1$. First, we relate the $0$-$1$ error of a reward estimator $\hat{r}$ to the multiclass classification error of $f_r$, assuming the labels come from $f_r$. Observe that,
\begin{align}
    \mathbb{E}_{\rho,\pibase} \left[ \mathbb{I} (r(\tau) \ne \hat{r} (\tau)) \right] \le \mathbb{E}_{\rho,\pibase} \left[ \mathbb{I} (f_r (\tau) \ne f_{\hat{r}} (\tau)) \right].
\end{align}
This follows from the fact that, if $r(\tau) \ne \hat{r} (\tau)$, then the bi-level in this trajectory $\tau$ is identified incorrectly, implying that $f_r (\tau) \ne f_{\hat{r}} (\tau)$. Recall that the expert dataset is composed of $n=n(\varepsilon)$ trajectories $D = \{ (\bx_i,\tau_i) \}_{i=1}^n$ for some $\varepsilon > 0$ (see \Cref{eq:nstar} for the definition of $n(\varepsilon)$). Using the verifier to annotate rewards, by a binary searching, the location of the bi-level in any of these $n$ trajectories may be located: thus with $n \lceil \log_2 (H) \rceil$ calls to the verifier, a dataset of $n$ examples may be constructed of the form $\{ (\tau_i,f_r (\tau)) \}_{i=1}^n$ for the ground truth reward $r$. By carrying out empirical risk minimization over the hypothesis class $\mathcal{F} = \{ f_r : r \in \mathcal{R} \}$ to learn a hypothesis $\hat{f}$, and invoking \Cref{theorem:multiclass}, with probability $\ge 1-\delta$,
\begin{align}
    \mathbb{E}_{\rho, \pibase} \left[ \mathbb{I} (f_r (\tau) \ne \hat{f} (\tau)) \right] \le \varepsilon.
\end{align}
\end{proof}









\subsection{Proof of Theorem~\ref{thm:verifier-free-thm} for the single problem instance}
\label{subsec:proof-verifier-free-thm-single-problem}

This result follows using a similar approach as the instance-dependent lower bound against behavior cloning proved in \cite{foster2024behavior}.
For the case, where we have a single prompt $\bx$, we use the following lemma to argue that given an expert policy $\pi_e$, we can always construct another policy $\tilde{\pi}_e$, and a pair of rewards $\{r, \tilde{r}\}$ that satisfy certain properties, while ensuring that each policy observes a variance of $\sigma^2$ in the range $(0, H^2/4]$ for either of the rewards.

Next, we consider the following inequality, which holds for any \(\Delta > 0\):

\[
\min_{\texttt{Alg}} \max_{\pi \in \{\pi_e, \tilde{\pi}_e\}} \max_{r \in \{r, \tilde{r}\}} \mathbb{P}\left[J_r(\pi) - J_r(\hat{\pi}) \geq \Delta \right]
\geq
\min_{\texttt{Alg}} \max_{\pi \in \{\pi_e, \tilde{\pi}_e\}} \mathbb{P}\left[\left|J_{r}(\pi) - J_{r}(\hat{\pi})\right| \geq \Delta \right].
\]

Here, \(J_r(\pi)\) denotes the expected reward under the reward function \(r\), and for convenience, we abbreviate \(J(\pi) \equiv J_{r}(\pi)\) going forward. Let \(\mathbb{P}_n^\pi\) represent the probability distribution of the offline imitation learning dataset when the data is collected under policy \(\pi\). By choosing \(\Delta = \frac{\left|J(\pi_e) - J(\tilde{\pi}_e)\right|}{2}\), and applying the standard Le Cam two-point argument, we can conclude that:

\[
\max\left\{
\mathbb{P}_n^{\pi_e}\left[\left|J(\pi_e) - J(\hat{\pi})\right| \geq \Delta\right],
\mathbb{P}_n^{\tilde{\pi}_e}\left[\left|J(\tilde{\pi}_e) - J(\hat{\pi})\right| \geq \Delta\right]
\right\}
\]

is bounded below by:

\[
\frac{1}{2} \left(
1 - \mathbb{P}_n^{\pi_e}\left[\left|J(\pi_e) - J(\hat{\pi})\right| < \Delta\right] +
\mathbb{P}_n^{\tilde{\pi}_e}\left[\left|J(\tilde{\pi}_e) - J(\hat{\pi})\right| \geq \Delta\right]
\right).
\]

This, in turn, is further bounded below by:

\[
\frac{1}{2} \left(
1 - \mathbb{P}_n^{\pi_e}\left[\left|J(\tilde{\pi}_e) - J(\hat{\pi})\right| \geq \Delta\right] +
\mathbb{P}_n^{\tilde{\pi}_e}\left[\left|J(\tilde{\pi}_e) - J(\hat{\pi})\right| \geq \Delta\right]
\right),
\]

and by a standard application of the data processing inequality for the total variation distance, we have:

\[
\frac{1}{2} \left(1 - \mathrm{D}_{\text{TV}}\left(\mathbb{P}_n^{\pi_e}, \mathbb{P}_n^{\tilde{\pi}_e}\right)\right).
\]

Utilizing the tensorization property of the Hellinger distance~\cite{wainwright2019high}, we further lower bound this by:

\[
\frac{1}{2} \left(1 - \sqrt{n \cdot \mathrm{D}_{\text{H}}\left(\mathbb{P}^{\pi_e}, \mathbb{P}^{\tilde{\pi}_e}\right)}\right).
\]

Next, we proceed to show the following key inequality:

\[
\omega_{\pi_e}(\varepsilon) \coloneqq \sup_{\pi} \left\{\left|J(\pi) - J(\pi_e)\right| \, \middle| \, \mathrm{D}_{\text{H}}\left(\mathbb{P}^{\pi_e}, \mathbb{P}^\pi\right) \leq \varepsilon^2\right\}
\geq \Omega(1) \cdot \sqrt{\sigma_{\pi_e}^2 \cdot \varepsilon^2},
\]

for any \(\varepsilon > 0\) sufficiently small. The final result follows by setting \(\varepsilon^2 \propto \frac{1}{n}\), and defining:

\[
\tilde{\pi}_e = \arg\max_{\pi} \left\{\left|J(\pi) - J(\pi_e)\right| \, \middle| \, \mathrm{D}_{\text{H}}\left(\mathbb{P}^{\pi_e}, \mathbb{P}^\pi\right) \leq \varepsilon^2\right\}.
\]

To prove this, we invoke the following technical lemma:

\begin{lemma}[Lemma G.1 in ~\citet{foster2024behavior}]
\label{lem:kl_dual}
For any distribution \(\mathbb{Q}\) and any function \(h\) satisfying \(\left|h\right| \leq R\) almost surely, it holds that for all \(0 \leq \varepsilon^2 \leq \frac{\mathrm{Var}_{\mathbb{Q}}[h]}{4R^2}\), there exists a distribution \(\mathbb{P}\) such that:
\begin{enumerate}
    \item \(\mathbb{E}_{\mathbb{P}}[h] - \mathbb{E}_{\mathbb{Q}}[h] \geq 2^{-3} \sqrt{\mathrm{Var}_{\mathbb{Q}}[h] \cdot \varepsilon^2}\),
    \item \(\mathrm{D}_{\text{KL}}(\mathbb{Q} \| \mathbb{P}) \leq \varepsilon^2\).
\end{enumerate}
\end{lemma}

In the case of stochastic policies \(\pi\) in the autoregressive Markov Decision Process \(\mathcal{M}^*\), these policies are equivalent to defining arbitrary joint distributions over the sequence \((a_1, \ldots, a_H)\) using Bayes' rule. Consequently, since \(J(\pi) = \mathbb{E}^\pi\left[\sum_{h=1}^H r_h\right]\), Lemma~\ref{lem:kl_dual} ensures that for any \(\varepsilon^2 \leq \frac{\mathrm{Var}^{\pi_e}\left[\sum_{h=1}^H r_h\right]}{4R^2}\), there exists a policy \(\tilde{\pi}_e\) such that:

\[
\mathrm{D}_{\text{H}}\left(\mathbb{P}^{\pi_e}, \mathbb{P}^{\tilde{\pi}_e}\right) \leq \mathrm{D}_{\text{KL}}\left(\mathbb{P}^{\pi_e}, \mathbb{P}^{\tilde{\pi}_e}\right) \leq \varepsilon^2,
\]

and:

\[
J(\tilde{\pi}_e) - J(\pi_e) \geq 2^{-3} \sqrt{\mathrm{Var}^{\pi_e}\left[\sum_{h=1}^H r_h\right] \cdot \varepsilon^2}.
\]

This establishes the desired inequality. Setting \(\varepsilon^2 = \frac{c}{n}\) for some constant \(c > 0\), we achieve 
$$\sqrt{n \cdot \mathrm{D}_{\text{H}}\left(\mathbb{P}^{\pi_e}, \mathbb{P}^{\tilde{\pi}_e}\right)} \leq \frac{1}{2},$$ 
which is valid provided that \(n \geq c' \cdot \frac{R^2}{\sigma_{\pi_e}^2}\).









\section{Additional Related Work}
\label{sec:additional-rel-work}


\textbf{Scaling test-time compute.} Recent works~\cite{sardana2023beyond,snell2024scaling} have shown that scaling test-time compute can improve performance at a rate faster than that afforded by traditional approaches of scaling data~\cite{li2024common} or model parameters~\cite{hoffmann2022training}, implying that training compute can often be traded off optimally for test-compute~\cite{villalobos2023trading,jones2021scaling}. There are two popular ways of spending test compute. First, is to autoregressively sample from the LLM long ``chains-of-thought''  that resemble linearized search traces~\cite{yao2023tree,gandhi2024stream} or an iterative refinement of answers~\cite{qu2024recursive,kumar2024training}. Second, is to explicitly implement search procedures~\cite{wu2024inference,beeching2024scalingtesttimecompute} with trained verifiers~\cite{cobbe2021gsm8k,setlur2024rewarding}. In our work, we empirically show that either of these approaches can scale well, and both theoretically and empirically examine a different and critical axis of separating these approaches: access to verification during training or inference. Additionally,  recent works~\cite{chen2024not,setlur2025opt} raise concerns about the unncessary wastage of test-time compute by sampling overly long responses for even simple questions~\cite{qwen2}. In our work, we use a ``bi-level'' reward formulation to capture what it means to efficiently use test-compute, and how to compare the asymptotic compute efficiency of verifier-free and verfier-based algorithms. 

\textbf{Access to verification.} We say that a finetuning algorithm has access to verification if it directly uses ground truth rewards to finetune LLMs, \textit{e.g.}, the 0/1 correctness labels on math solutions~\cite{uesato2022solving,bi2024deepseek}; or if it queries trained verifiers for collecting training data~\cite{hosseini2024v} and running search procedures at test-time~\cite{welleck2024decoding,chen2024more,chow2024inference}. The former approach of training LLMs to generate long ``chains of thought'' with final reward on-policy RL~\cite{MoonshotAI,deepseekai2025deepseekr1incentivizingreasoningcapability} has shown impressive gains on reasoning benchmarks. For off-policy RL algorithms~\cite{rafailov2023direct,zelikman2022star,singh2023beyond} that utilize verification, converting the same 0/1 rewards into value function based process verification has been shown to be critical~\cite{setlur2024rl}. 
Apart from these verification can also be generative~\cite{zhang2024generative} and implicit~\cite{yuan2024implicitprm} where the same LLM is trained to generate and self-verify responses iteratively.  In this work, we bucket all the above as verifier-based algorithms, and formally show that  the asymptotic performance of this class scales test-compute more efficiently than approaches that do not query any sort of rewards, highlighting the critical role played by access to verification. 




\textbf{Verifier-free algorithms.} Multiple works have proposed to scale test-time compute by finetuning pre-trained LLMs on manually stitched search traces~\cite{gandhi2024stream,nie2024evolve} that all lead to the correct solution. The goal here is to force the LLM to mimic known search procedures like Monte-Carlo tree search~\cite{yang2022chain,xie2024monte} or A$^\star$~\cite{lehnert2024beyond} on training questions, with the hope that the LLM learns to search for solutions on test problems too~\cite{sel2023algorithm}. Crucially these algorithms do not annotate search trajectories in the training data with any reward, and the LLM is forced to mimic multipe search traces that are ``heterogeneous'' in nature, \textit{i.e.}, different traces spending varying number of tokens (for search) to arrive at the same final solution. In our work, we analyze how this heterogeneous nature makes it hard for \emph{any} supervised finetuning algorithm to generalize, resulting in a poor test-time scaling law for these,  matching observations in practice ~\citep{kumar2024training,xiang2025towards}.







\section{Additional Experiments in the Didactic Setup}
\label{sec:additional-didactic}




\textbf{Details on the setup.} We generalize the planted subsequence problem from \cite{setlur2024rewarding}. The input prompt is a sequence of length $5$ with the tokens chosen randomly from the set $\{1,2, 3,\ldots,10\}$. We fix the unknown function to be $g(x)=2x+5$. We fix the vocabulary for the policy we are training to be the set $\gV \eqdef \{0, \ldots, 30\}$.  Here $0$ is treated as the padding token. 
Concretely, for an input problem  $\bx = (x_1,$$..$$, x_5)$, we say that a response $\by$ with $H$ tokens from the vocabulary $\gV$  
is a correct trace if there exists a \emph{gold} contiguous subsequence $(g(x_1),$$..,$$g(x_{5}))$ planted in $\by$. Here, the underlying mapping $g:$$[10]$$\mapsto$$[30]$ is fixed but unknown.  For a state $\bs \eqdef (\bx, a_1,$$ ..$$, a_h)$, the bi-level reward $r(\bs) = 1$ if and only if there exists some $h^\prime \leq h$ such that the last $5$ tokens before $h'$ \textit{i.e.},  $(a_{h^\prime - 4},$$..,$$a_{h^\prime})$
match the gold subsequence.  In order to use the same performance scale to compare  methods trained for different horizon $H$ values (test-time compute budget), we  $J_r(\pi)$ and divide it by the maximum reward of $H-4$. 


We wish to construct base policies $\pibase$ that: \textbf{(i)} differ in heterogeneity, and \textbf{(ii)} satisfy the anti-concentration condition. To do so, we finetune GPT2-xl~\cite{radford2019language} on samples obtained from a mixture of hand-designed ``procedural'' policies. Inspired from \citet{setlur2024rewarding}, a procedural policy $\mu_\gamma (\by^\star_{k+1} | \bs)$$\propto$ $\gamma$, when the last $k$ tokens in the state $\bs$, match the first $k$ tokens in the gold subsequence $\by^\star$.  Thus, the normalized return for $\mu_\gamma$$\rightarrow$$1$, as $\gamma$$\rightarrow$$\infty$. We vary the heterogeneity of $\pibase$ by finetuning GPT2-xl on data from a mixture of procedural policies with $\gamma $ $\in$ $\{5, 10, 20, 50, 100, 500\}$. 
Once the last $5$ tokens match the gold sequence, the procedural policy puts mass  $\propto \gamma$ on the padding token $0$. 
See Figure~\ref{fig:examples-didactic} for an illustration of data sampled from different procedural policies. 

For any compute budget $H$ (token length), we train separate SFT and RL policies, where SFT is run on traces that are $H$ tokens long. We also run RL on the same token budget, against a trained verifier. The verifier is trained on samples from the base policy. For this, we train a GPT2-xl transformer as a multiclass classifier, that takes in an $H$ length sequence and outputs a single value in $0$ to $H$ (i.e., it is an $H+1$-way classifier). 



\textbf{Experiment details.} For the RL runs, we use REINFORCE~\cite{ahmadian2024back} train for 20k iterations in both with a batch size of 64, and a constant learning rate of $1e-4$, with the Adam optimizer. The RL runs are initialized with the base policy, and to prevent reward hacking we also use a KL penalty (with weight $0.2$), in addition to the REINFORCE training objective. 
For every trace in a batch, we query the trained verifier, which outputs a value between $0$ and $H$, which directly tells us where the ``staircase'' appears in the bi-level reward. For example, a value of $2$ implies that the staircase appears on the second last token. We convert this outcome supervision into token-level 0/1 rewards and update the policy with the computed policy gradient. For SFT, we also use the Adam optimizer with a learning rate of $2e-4$, and a batch size of $64$. Similar to RL, we apply a KL regularization term in addition to the next token prediction loss (ignoring the padding token 0), where the strength of the KL term is the same as RL. SFT runs are also initialized with the base policy. Using the same hyperparameters, we obtain the base policy by running SFT on 200k data points sampled i.i.d. from  the uniform mixture over procedural policies outlined above. To collect training data for the verifier, we draw a random sample of $\nicefrac{n}{\log H}$ prompts in $\gD_\mathrm{tr}$, and then make $\log(H)$ calls on each of them to binary search for the token where the correct answer first appeared. This way, we only query reward annotator $n$ times. Finally, for our experiments, where we vary base and expert policy heterogeneity, we simply change $\gamma$ (reducing variance over it), in a way that the average performance of the base/expert policy remains roughtly the same. 


\begin{figure}
    \centering
    \includegraphics[width=0.55\linewidth]{figures/didactic-example.pdf}
    \caption{\textbf{\emph{Procedural policies for the generalized planted subsequence problem:}} For two values of $\gamma$: 10, and 1000, we show examples of two draws, over $H=10$ tokens from each. Here, the unknown mapping is $g(x)=2x+5$. When $\gamma$ is $1000$, the policy (over the first 5 tokens) is almost like a dirac delta distribution on the gold subsequence, followed by which it samples the padding tokens. On the other hand, when $\gamma=10$, it makes multiple attempts and completing the sequence. Once it fails, it makes a new attempt. In the second sample, we see that after a few tokens it gets the correct sequence, achieving a total bi-level reward of $3$, and normalizing it with $H-4$, we get a normalized reward of $0.5$.}
    \label{fig:examples-didactic}
\end{figure}

\begin{figure}[!ht]
    \centering
    \includegraphics[width=0.4\linewidth]{figures/toy_verifier_acc.pdf}
    \hspace{2em}
    \includegraphics[width=0.35\linewidth]{figures/toy_vary_sigmab_acc.pdf}
    \caption{\textbf{\emph{Accuracy of trained verifier:}}
    (Left) we plot the accuracy of the verifier (black line), as we scale the horizon (black line). We also plot the performance of RL with ground-truth (GT) bi-level rewards, and compare it with RL using the trained verifier. (Right) As we vary base policy heterogeneity we plot the accuracy of the verifier on two distributions: (i) on base policy $\pibase$, (ii) on policy learned by running RL $\hat{\pi}^\mathrm{vb}_n$.    }
    \label{fig:didactic-verifier-acc-1}
\end{figure}


\textbf{Accuracy of trained verifier.} In Figure~\ref{fig:didactic-verifier-acc-1}(left), we plot the accuracy of the verifier (black line), as we scale the horizon. We fix the data budget to $n=2^14$ here. Since, here budget implies a multi-class classification over more classes, the problem hardness increases for the verifier, which explains the performance drop. Initially, we do see an improvement with $H$, since the coverage over high reward trajectories improves with $H$, as we sample the base policy for longer. 
We also plot the upper bound on RL performance, where we train the RL policy with ground-truth staircase rewards. Looking at its performance, it is clear that across all horizons, RL with trained verifier mainly suffers from the inaccuracy of the trained verifier (i.e., reward hacking issues). In Figure~\ref{fig:didactic-verifier-acc-1}(right), we plot the accuracy of the learned verifier on two distributions (base policy), and the policy learned by RL. As we reduce base policy heterogeneity, it is easier to generalize on the base policy, but the verifier is inaccurate outside the narrow distribution of the base policy, making it more susceptible to reward hacking. As a result, we observe poor accuracy on the learned policy's distribution. 

\vspace{-1em}

\section{Additional Experiments on MATH}
\label{sec:additional-math}




\textbf{Experiment details.} We run all our training on the questions in the training set of MATH~\cite{hendrycksmath2021}, and run our test on the MATH500 evaluation benchmark. Any problem instance is determined by the data budget $n$ and compute budget $H$. For a fixed compute budget $H$, we run SFT on trajectories of search traces. A single search trace consists of multiple turns of responses. Each response is at most 512 tokens long. A trace can thus comprise of anywhere between $1$ to $H/512$ turns (sometimes more for shorter responses).  Here, all turns except the last one are responses that result in incorrect final answers for the problem. This is exactly the recipe followed by \citet{qu2024recursive} and \citet{snell2024scaling}, for learning to self-correct, revise responses, or more generally use sequential compute at test-time. See below for examples of search traces of SFT. For training the verifier, we simply use samples from the base LLM, and use 0/1 rewards to train a model of the same size as the policy with cross-entropy loss. We use a batchsize of 32 and learning rate of $1e-6$ for all our experiments. We run SFT and verifier training for 10000 iterations on each instance. We use a weight decay of 0.01 for training both. At test-time, we measure the performance of SFT using majority voting over sequential revisions, following~\citet{snell2024scaling}. We sample rollouts with temperature 1.0.
\begin{figure}[!ht]
    \centering
    \includegraphics[width=0.33\linewidth]{figures/ver_acc.pdf}
    \caption{\textbf{\emph{Accuracy of verifier:}} We plot the accuracy of verifier as we scale data, for Llama3-3b and 8b sized models. We find that the larger model generalizes better across  all data budgets.}
    \label{fig:verifier-acc}
\end{figure}

\textbf{Verifier accuracy.} In Figure~\ref{fig:verifier-acc} we plot the test accuracy of the trained verifier as we scale data. Interestingly, we find that the bigger model always generalizes better, and models of both sizes improve performance at the same rate, as we scale the data budget $n$.






\resizebox{0.99\linewidth}{!}{
\begin{analysisbox}[Search trace with sequential revisions for SFT (example with 2 episodes)]

\textbf{Query:} A farmer has three times as many chickens as cows on his farm. If he has 15 cows, how many chickens does he have?  

\textbf{Turn-1 Response:}  
Step 1: Let’s assume the number of cows as \( x \).  
Step 2: According to the given information, the farmer has three times as many chickens as cows. So, the number of chickens can be expressed as \( 3x \).  
Step 3: We are given that there are 15 cows, so let’s substitute \( x = 15 \) into the equation:  
\[
3x = 3(15) = 30
\]  
Answer: \(\boxed{30}\) INCORRECT

\textbf{Turn-3 Response:}  
Let’s solve the problem step by step:  
Step 1: Let’s denote the number of cows as \( x \).  
Step 2: We are given that the farmer has three times as many chickens as cows. So, the number of chickens can be expressed as \( 3x \).  
Step 3: We are also given that there are 15 cows, so we can substitute \( x = 15 \) into the expression for the number of chickens:  
\[
3 \times 15 = 45
\]  
Answer: \(\boxed{45}\)  CORRECT

\end{analysisbox}
}

\resizebox{0.99\linewidth}{!}{
\begin{analysisbox}[Search trace with sequential revisions for SFT (example with 3 episodes)]

\textbf{Query:} A rectangle’s length is five more than twice its width. If its perimeter is 58 cm, what is its width?  

\textbf{Turn-1 Response:}  
Step 1: Let’s assume the width of the rectangle as \( x \).  
Step 2: According to the given information, the length is five more than twice the width. So, the length can be expressed as \( 2x + 5 \).  
Step 3: The perimeter of a rectangle is given by the formula:  
\[
\text{Perimeter} = 2(\text{Length} + \text{Width})
\]  
Step 4: Substituting the given perimeter of 58 cm and our expressions for length and width:  
\[
2(x + 2x + 5) = 58
\]  
\[
x = \frac{53}{6}
\]  
Answer: \(\boxed{\frac{53}{6}}\) INCORRECT


\textbf{Turn-2 Response (Subtle Error Introduced):}  
Step 1: The width is \( x \) and the length is \( 2x + 5 \).  
Step 2: Using the perimeter formula:  
\[
2(x + 2x + 5) = 58
\]  
\[
6x + 5 = 58
\]  
\[
x = 7
\]  
Answer: \(\boxed{7}\) INCORRECT


\textbf{Turn-3 Response:}  
Let’s solve the problem step by step:  
Step 1: Let’s denote the width of the rectangle as \( x \).  
Step 2: The length is five more than twice the width, so it can be expressed as \( 2x + 5 \).  
Step 3: The perimeter formula for a rectangle is:  
\[
\text{Perimeter} = 2(\text{Length} + \text{Width})
\]  
Step 4: Substituting the given perimeter of 58 cm:  
\[
2(x + 2x + 5) = 58
\]  
\[
x = 8
\]  
Answer: \(\boxed{8}\) INCORRECT
\end{analysisbox}
}







