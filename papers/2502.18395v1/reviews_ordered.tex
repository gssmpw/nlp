
\section{Moral Stress and related Work (Section 2)}

R1.3a
3a. Beyond those particulars, I was not certain why the theory of moral stress was the most persuasive for this domain. Shilton's work on values is certainly relevant. I had hoped for a clearer sense of how this work built on Shilton's work. I'd also like to ask about the applicability of Miller et al.'s concepts of value tensions and of value dams and flows (Proc. GROUP 2007). On page 10, the paper asserts that "we found that these simmering emotions of uneasiness, discomfort and insecurity, are best characterised with the psychological concept of moral stress." However, there is no formal evidence of why that theory provides a good characterisation, and there is no comparison with other possible bases for describing and analyzing the experiences. The paper would be stronger if it provided a more comparative account of the theory of moral stress, and a rationale for why that theory is applicable to this situation. I expect that I will agree that moral stress is a good fit for the domain, but I wish that the paper made a better case for that applicability.

3b. In this regard, I think that the analysis could be stronger if we understood how the theory of moral stress was applied in that analysis. The last paragraph in Section 3.1 makes reference to open coding, coding for themes, and coding for language. It would be helpful to know what method was used in coding for themes. It would be very important to know how moral stress was used in these analyses. For example, were the themes directly derived and applied from the theory of moral stress - a kind of confirmatory analysis? Were the themes constructed "up" from the data, as implied by thematic analysis and as required in grounded theory - e.g., using moral stress as a set of sensitizing concepts? The paper would be stronger if the relationship of data to theory were made clearer.

R3.1
I think related work is adequate as is. However, if desired by the authors, it might be useful to also cite some prior research that looks at the (small p) politics of civic/public technology development.



\section{Team composition, set up and methods (Section 3)}
 
R1.1
1. I finally realized that the situation and position(ality) of the team were not clear to me. Is this team responsible for creating and implementing a particular project? Is this team responsible for proposing (or promoting?) a certain facilitation (or intervention?) of their method to other teams? How was this team formed, who gave this team its goal or mission, and what was the intended outcome of the team's work? The paper would be stronger if it were better contextualized and grounded in the team's situation.

 2a. I could not understand any specific aspects of the method that the team used - either for themselves or for other teams (see #1, above). I appreciated the critique of other papers' unclear descriptions of manifestos, workshops, toolkits, etc., but I thought that this paper was similarly unclear. What was the text of the Manifesto (with full de-identification, of course)? What tools were in the Toolkit(s)? How were those tools used in the Workshops? What is a map-template (mentioned only once, on page 9)? We know that there is a method that is based in both interpersonal dynamics and in artifacts, but we don't learn enough about the dynamics and artifacts. This weakness is further compounded by a lack of comparison to the manifestos, workshops, toolkits, etc., that were critiqued in the Related Work. The paper would be stronger if it provided more details about its own methods, and if it provided a comparison with other methods.

2b. Because of the lack of clarity in #1 (above), I could not tell who the stakeholders were for the team, its methods, and its outcomes. These aspects are important for our understanding of what the team was supposed to accomplish - i.e., its organizational goals and responsibilities. On page 6, we read that "The City and involved parties consider the Manifesto a successful result of a complex, collaborative, and inclusive process." Who were the interested parties? There are hints on page 14 that some parties may have been excluded (intentionally or unintentionally): Who decided which interested parties to include, and on what basis? It is only on page 15 that we learn that the project was a pilot, but we don't know what that means in practice, or whether and how the common organisational pressures for efficiency and completion were modified for the pilot. The paper would be stronger with a better organizational contextualization.

2c
There is an implicit value-judgement in the subsection on beyond organisation. The paper describes an accessibility project about obstacles to wheelchair-based travel on City streets. That's a laudible goal - but please search the ACM Digital Library for multiple, similar projects that should be cited and compared (' "street" AND "accessibility" '). However, the paper then makes an argument that certain forms of blockage were actually "minor digressions such as mis-parked bikes and misplaced flower pots... minor rule breaking... potential for misuse..." I'd like to ask the authors to reflect on their own values, and on how they unintentionally create a hierarchy of values about their own project. From the perspective of someone who uses a wheelchair, those "minor digressions" nonetheless block the sidewalk, and create impassible barriers. The only recourse is then to move the wheelchair into the street (perhaps bumping down and up a series of curbs). The result is multiple dangers to the wheelchair-user: The danger of taking a spill at each curb, and the danger of navigating the wheelchair on a road with cars and bicycles. In #2b (above), I asked about who was included in the collaborative and inclusive processes. I begin to suspect that few or no disabled people were part of the team or even part of the group that informed or advised the team. The paper would be stronger if it took up this complexity of value levers (see again Shilton) or value tensions and dams (see again Miller et al.). I urge the authors to reconsider their own values, and to open their discussion to broader and potentially disruptive views.

R2.1
The one methodological issue I would like the authors to address in revising the paper is what the role of the other authors was in the research process - from the way the paper is written, it is implied that the paper was prepared by a team while the fieldwork itself was conducted solely by the first author. It would be good to clarify this explicitly.

R3.2
When introducing the Team (Section 3.4, line 383) the “route-planner” project is mentioned, but not described. A short description might be useful to help the reader understand the context. A little more detail on the data analysis process might be helpful as well in Section 3.2.

\section{Findings Section 4}

R3.3
* Section 4.1.1 seems to be informed completely by Nadine’s view. Are there other “within team” perspectives that help tell this part of the story? (Or is it intentionally focusing just on Nadine’s experience as an Ethics Owner?)

In Section 4.1.2, the affective aspects came through less clearly for me. (The idea that there are tensions and friction was clear). I suspect this may be an example where the quotes in a longer conversational context or the tone of the speakers conveyed affect, which is not fully coming through in the text-based quotes. Helping explain the affective or stressful parts of this section may help the contribution come through more clearly.

\section{Discussion: Connect better to CHI literature (Section 5)}

R1.4
Finally, I think that the paper fails to make a strong connection to the CHI literature in the Discussion. We learn about how a team pursued a mission in an organisation, and we learn that the organisation made a seemingly simple mission into a series of complexities. That's a valuable story - but what contribution does it make to the CHI conference and the CHI readership? The paper makes very good use of the CHI literature in Related Work. I urge the authors to bring at least some of those citations back into the Discussion, so that there can be clear points of relevance and reference for the CHI audience.

Marco: Maybe also CSCW lit? 

R2.2
The paper gives it readers serious and significant food-for-thought, although it could perhaps do a little more to suggest what the community should do upon learning the lessons the paper delivers.

(2a) reflecting more on how the lessons from this study might guide is in situations where teams are less engaged in ethical work and less functional, and 

(2b) expanding the Discussion and/or Conclusion with a few lines to more clearly state what scholars in the HCI community might do upon reading this paper. The paper is effective in delivering its core message but, ironically, in its current form it might add to the moral stress of its readers by illustrating the difficulty of ethical interventions into organizational life but doing less to guide us forward.