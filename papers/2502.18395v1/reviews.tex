Reviews:
----------------------------------------------------------------

1AC review (reviewer 4) I can go with either Accept with Minor Revisions or Revise and Resubmit

1AC: The Meta-Review

My recommendation is ARR, reflecting my consideration that the suggested changes can be addressed by the authors in a revised version. The submission received two reviews from external experts, and one review from a knowledgeable internal reviewer. All reviewers see the paper as promising and recognize that once ready for publication, it will make a substantive, valuable contribution to the HCI community (not just within CHI but well beyond).

All reviewers would like to see a bit more detail about methods, including: 
\begin{itemize}
    \item the role of the different authors in fieldwork and paper writing (R2) and whether disabled people were part of the research team (R1),
    \item the project and organizational context (R1, R3),
    \item the concrete materials and the interpersonal dynamics (R1),
    \item and the analysis (R1, R3), in particular how the moral stress theory was applied (i.e. the concrete link between data and theory) (R1).
\end{itemize}

Regarding the theoretical base of the argument, R1 recommends to highlight the applicability of moral stress theory by comparing it with other lenses. Moreover, R1 suggests the authors might improve the submission by engaging more directly with Shilton’s and with Miller et al.’s work - especially in relation to the authors’ own positionalities and values. Here, I would like to call the authors' attention to reflect and act upon R1's comments about unintentionally creating a hierarchy of values within the project when talking about accessibility barriers as 'minor digressions' and the like.
In addition, R3 points to other useful related work that could be added if the authors consider it relevant.

Honoring the perspective of moral stress of the readers themselves, R2 wonders if the authors could go one step further by suggesting the community some action points that build on the paper’s learnings, especially in cases of less functional teams and/or teams less guided by ethical stances. Also pertaining to the Discussion, R1 requests the authors to bring back the most relevant related work from that section, to strengthen the link to the CHI literature.

Minor changes include R1’s recommendations to write down a summary of each theme in its own subsection, R2 and R3’s request to double check references, and R3’s comments on the Findings sections.

The authors will find more detailed comments in the thoughtful reviews, that I encourage them to take to heart. I thank the authors for a compelling submission in a very important topic and I look forward to reading a revised version.

----------------------------------------------------------------

2AC review (reviewer 1) I recommend Revise and Resubmit

This paper is very promising. I have some concerns about how the argument is structured, about the lack of detail of the description of the materials and dynamics, and about the "fit" to the CHI conference. I think that both of these concerns can be addressed during revise and resubmit. I hope to see an improved version of this paper at CHI 2025.

The paper makes a useful contribution to the CHI literature with its emphasis on moral stress. It is helpful and healthy to bring in concepts from related literatures - in this case, a branch of psychology. I believe that the concept of moral stress can be applied in other projects and situations that are addressed by CHI, CSCW, DIS, and GROUP. I look forward to this paper becoming influential in at least those four conferences.

As I began to read the findings, I was initially skeptical of the themes of within team, beyond team, and beyond organisation. I was very happy to find these three subsections to be quite convincing. I'd like to suggest that each of the three subsections could be improved by a brief summary of that subsection's theme, and the key aspects of the data that contribute to that each theme.

That being said, I'm not sure that the paper makes a strong enough case for adopting the theory of moral stress for these kinds of situations. I've been trying to understand why I have a problem here. I think there are three contributing issues for me:

 1. I finally realized that the situation and position(ality) of the team were not clear to me. Is this team responsible for creating and implementing a particular project? Is this team responsible for proposing (or promoting?) a certain facilitation (or intervention?) of their method to other teams? How was this team formed, who gave this team its goal or mission, and what was the intended outcome of the team's work? The paper would be stronger if it were better contextualized and grounded in the team's situation.

2a. I could not understand any specific aspects of the method that the team used - either for themselves or for other teams (see #1, above). I appreciated the critique of other papers' unclear descriptions of manifestos, workshops, toolkits, etc., but I thought that this paper was similarly unclear. What was the text of the Manifesto (with full de-identification, of course)? What tools were in the Toolkit(s)? How were those tools used in the Workshops? What is a map-template (mentioned only once, on page 9)? We know that there is a method that is based in both interpersonal dynamics and in artifacts, but we don't learn enough about the dynamics and artifacts. This weakness is further compounded by a lack of comparison to the manifestos, workshops, toolkits, etc., that were critiqued in the Related Work. The paper would be stronger if it provided more details about its own methods, and if it provided a comparison with other methods.

2b. Because of the lack of clarity in #1 (above), I could not tell who the stakeholders were for the team, its methods, and its outcomes. These aspects are important for our understanding of what the team was supposed to accomplish - i.e., its organizational goals and responsibilities. On page 6, we read that "The City and involved parties consider the Manifesto a successful result of a complex, collaborative, and inclusive process." Who were the interested parties? There are hints on page 14 that some parties may have been excluded (intentionally or unintentionally): Who decided which interested parties to include, and on what basis? It is only on page 15 that we learn that the project was a pilot, but we don't know what that means in practice, or whether and how the common organisational pressures for efficiency and completion were modified for the pilot. The paper would be stronger with a better organizational contextualization.

3a. Beyond those particulars, I was not certain why the theory of moral stress was the most persuasive for this domain. Shilton's work on values is certainly relevant. I had hoped for a clearer sense of how this work built on Shilton's work. I'd also like to ask about the applicability of Miller et al.'s concepts of value tensions and of value dams and flows (Proc. GROUP 2007). On page 10, the paper asserts that "we found that these simmering emotions of uneasiness, discomfort and insecurity, are best characterised with the psychological concept of moral stress." However, there is no formal evidence of why that theory provides a good characterisation, and there is no comparison with other possible bases for describing and analyzing the experiences. The paper would be stronger if it provided a more comparative account of the theory of moral stress, and a rationale for why that theory is applicable to this situation. I expect that I will agree that moral stress is a good fit for the domain, but I wish that the paper made a better case for that applicability.

3b. In this regard, I think that the analysis could be stronger if we understood how the theory of moral stress was applied in that analysis. The last paragraph in Section 3.1 makes reference to open coding, coding for themes, and coding for language. It would be helpful to know what method was used in coding for themes. It would be very important to know how moral stress was used in these analyses. For example, were the themes directly derived and applied from the theory of moral stress - a kind of confirmatory analysis? Were the themes constructed "up" from the data, as implied by thematic analysis and as required in grounded theory - e.g., using moral stress as a set of sensitizing concepts? The paper would be stronger if the relationship of data to theory were made clearer.

There is an implicit value-judgement in the subsection on beyond organisation. The paper describes an accessibility project about obstacles to wheelchair-based travel on City streets. That's a laudible goal - but please search the ACM Digital Library for multiple, similar projects that should be cited and compared (' "street" AND "accessibility" '). However, the paper then makes an argument that certain forms of blockage were actually "minor digressions such as mis-parked bikes and misplaced flower pots... minor rule breaking... potential for misuse..." I'd like to ask the authors to reflect on their own values, and on how they unintentionally create a hierarchy of values about their own project. From the perspective of someone who uses a wheelchair, those "minor digressions" nonetheless block the sidewalk, and create impassible barriers. The only recourse is then to move the wheelchair into the street (perhaps bumping down and up a series of curbs). The result is multiple dangers to the wheelchair-user: The danger of taking a spill at each curb, and the danger of navigating the wheelchair on a road with cars and bicycles. In #2b (above), I asked about who was included in the collaborative and inclusive processes. I begin to suspect that few or no disabled people were part of the team or even part of the group that informed or advised the team. The paper would be stronger if it took up this complexity of value levers (see again Shilton) or value tensions and dams (see again Miller et al.). I urge the authors to reconsider their own values, and to open their discussion to broader and potentially disruptive views.

Finally, I think that the paper fails to make a strong connection to the CHI literature in the Discussion. We learn about how a team pursued a mission in an organisation, and we learn that the organisation made a seemingly simple mission into a series of complexities. That's a valuable story - but what contribution does it make to the CHI conference and the CHI readership? The paper makes very good use of the CHI literature in Related Work. I urge the authors to bring at least some of those citations back into the Discussion, so that there can be clear points of relevance and reference for the CHI audience.

Summary

As I wrote at the beginning of the review, I think this project has much promise. In its present form, I think that the paper is not yet ready for the CHI readership. But that is what Revise and Resubmit is for! I hope that the authors can address at least some of the points that I have raised in this review. The paper has the potential to make a solid contribution to CHI. I hope that my comments will help to strengthen the paper toward that contribution.

----------------------------------------------------------------

reviewer review (reviewer 2) I can go with either Accept with Minor Revisions or Revise and Resubmit

Review (Round 1)

This paper brings the concept of moral stress into studies of ethics-in-practice, a line of research that has been actively developed within (and beyond) the HCI community over the past years. The authors discuss moral stress in connection to the efforts and difficulties of bridging the gap between AI ethics principles and practice, with a focus on the effects that interventions like toolkits and checklists have on technology practitioners. The paper presents findings from an ethnographic investigation a European city organization, highlighting the affective experience of committing to the integration of ethics interventions into technical and organizational practice. The authors argue that moral stress is to an extent inevitable - yet tends to go unaccounted for - and lay out some of the implications this should have for organizational practice.

Through the unique and thorough fieldwork presented, the paper makes a valuable empirical and conceptual contribution in following up prior work by illustrating moral stress as a core challenge to the successful integration of ethics tools in technical practice. In my assessment, this paper will be of high interest to other scholars working on ethics-in-practice and AI ethics more broadly. I consider it to make a further valuable move in highlighting that technology practitioners (at least in the studied case) are already concerned with the ethical implications of their work and in pointing out that well-intended ethics toolkits, checklists, and workshops risk dismissing this aspect of their expertise, along with failing to account for the emotional consequences of introducing such resources into organizational life. The paper gives it readers serious and significant food-for-thought, although it could perhaps do a little more to suggest what the community should do upon learning the lessons the paper delivers.

I consider the research in this paper to be of high quality. The fieldwork engagement is extensive and there is noteworthy merit also in gaining access to the studied field site. From the description in the paper, the research has also been conducted in line with good research practice, including appropriate respect and care toward the participants. I do not see reason to doubt that other researchers and practitioners can use these results confidently (although, of course, this should be done with understanding of the ethnographic nature of the presented research - focusing on a singular field site and not
 aiming at empirical generalizability). The one methodological issue I would like the authors to address in revising the paper is what the role of the other authors was in the research process - from the way the paper is written, it is implied that the paper was prepared by a team while the fieldwork itself was conducted solely by the first author. It would be good to clarify this explicitly.

In terms of previous work, the paper is well-situated within the relevant current HCI literature and it also does a nice job of providing a compact yet sufficient overview of how the challenges of moral conduct and ethics in practice have been studied from a psychological perspective in fields such as management studies and nursing studies. I recommend, however, that the authors look over their writing where they reference prior work with author names in-text, as I spotted a couple of minor mistakes (ref #49 is not solo-authored by Lindberg, and ref #33 should say Garrett et al.). These are, evidently, very easily to fixed.

When it comes to presentation clarity, the paper is written in a clear and consistent manner. The framing is clear throughout, and the language should be widely accessible. Personally, I found this paper to be a good read!

While the paper is, in my assessment, acceptable for publication after minor revisions, the two more substantive issues I would like the authors to consider are (1) reflecting more on how the lessons from this study might guide is in situations where teams are less engaged in ethical work and less functional, and (2) expanding the Discussion and/or Conclusion with a few lines to more clearly state what scholars in the HCI community might do upon reading this paper. The paper is effective in delivering its core message but, ironically, in its current form it might add to the moral stress of its readers by illustrating the difficulty of ethical interventions into organizational life but doing less to guide us forward.

To sum, I recommend accepting this paper for publication after relatively minor revisions.

----------------------------------------------------------------

reviewer review (reviewer 3) I can go with either Accept with Minor Revisions or Revise and Resubmit

This paper contributes empirical findings about the types of moral stress and anxiety involved in developing ethical sensitivity, using ethnographic findings of a technology team in a city/civic context in Europe. This is an important contribution, building on prior HCI research on ethics in design practice, but provides an original and significant contribution in both exploring this the public sector, and in focusing on the emotional and affective experience. Overall I find the research quality high, providing useful insights from ethnographic methods. Previous work is well reviewed, and the paper is well written with clear presentation.

I was very excited to see this work focusing on the emotional and affective experiences of addressing ethics, and excited by the public sector domain of the research. I found the paper a joy to read, with useful reflections about how organizational factors can lead to moral stress.

Some other aspects I quite enjoyed in the paper included: discussion of ethics as a relational practice (Sec 2.2); the particular site of study is very interesting as a place to study ethics in practice; integration of consideration of organizational structure and how different relationalities affect visibility of moral stress (Sec 4.1)discussion of translation work and the forms of hidden labor required (Sec 4.1.1); many of the quotes or vignettes of the field site were very evocative and generally helped convey the points well. All of these add up to an original and significant contribution in my opinion.

I would be ok if the paper was accepted as-is, but I think a few minor things could be improved if desired:

    == 1. Some clarification on methods/research context
    When introducing the Team (Section 3.4, line 383) the “route-planner” project is mentioned, but not described. A short description might be useful to help the reader understand the context. A little more detail on the data analysis process might be helpful as well in Section 3.2.

    == 2. Improving Findings
    Overall I found the Findings to make sense and can be a useful contribution as-is. I have a few optional suggestions (or clarification questions).

* Section 4.1.1 seems to be informed completely by Nadine’s view. Are there other “within team” perspectives that help tell this part of the story? (Or is it intentionally focusing just on Nadine’s experience as an Ethics Owner?)

In Section 4.1.2, the affective aspects came through less clearly for me. (The idea that there are tensions and friction was clear). I suspect this may be an example where the quotes in a longer conversational context or the tone of the speakers conveyed affect, which is not fully coming through in the text-based quotes. Helping explain the affective or stressful parts of this section may help the contribution come through more clearly.

  *  The subsubsections (within teams, beyond team, beyond organization) might work better as “subsections” rather than “subsubsections”

== 3. Related Work Suggestions
    I think related work is adequate as is. However, if desired by the authors, it might be useful to also cite some prior research that looks at the (small p) politics of civic/public technology development. (I’m most familiar with these examples in a US context as that is where my research is done, so these may not be the most useful citations to the authors; there may be more relevant European examples that I am not familiar with)

    * Kirsten Boehner and Carl DiSalvo. 2016. Data, Design and Civics: An Exploratory
                Study of Civic Tech. In Proceedings of the 2016 CHI Conference on
                Human Factors in Computing Systems (CHI '16).
                https://eur02.safelinks.protection.outlook.com/?url=https%3A%2F%2Fdoi.org%2F10.1145%2F2858036.2858326&data=05%7C02%7Csrr%40di.ku.dk%7C36497ffd812f46a49ff408dcfdef2c4a%7Ca3927f91cda14696af898c9f1ceffa91%7C0%7C0%7C638664450909788781%7CUnknown%7CTWFpbGZsb3d8eyJWIjoiMC4wLjAwMDAiLCJQIjoiV2luMzIiLCJBTiI6Ik1haWwiLCJXVCI6Mn0%3D%7C0%7C%7C%7C&sdata=9WHMi3Zwk4EeogkeLAugFnbKdGY0xwgCLpIgh8iYZsE%3D&reserved=0

    * Horgan, Leah, and Paul Dourish. 2018. “Ambiguity, Ambivalence, and Activism:
                Data Organizing Inside the Institution”. Krisis | Journal for
                Contemporary Philosophy 38 (1):72-84.
                https://eur02.safelinks.protection.outlook.com/?url=https%3A%2F%2Fdoi.org%2F10.21827%2Fkrisis.38.1.37183&data=05%7C02%7Csrr%40di.ku.dk%7C36497ffd812f46a49ff408dcfdef2c4a%7Ca3927f91cda14696af898c9f1ceffa91%7C0%7C0%7C638664450909945044%7CUnknown%7CTWFpbGZsb3d8eyJWIjoiMC4wLjAwMDAiLCJQIjoiV2luMzIiLCJBTiI6Ik1haWwiLCJXVCI6Mn0%3D%7C0%7C%7C%7C&sdata=0M43UlUU0mlGu%2FmsgjA8ZoPei2%2BiooC%2BNlZVnCEVf%2FE%3D&reserved=0.


    == 4. Miscellaneous Minor Things
    * Citations – technically, I believe the ACM format requests DOIs to be included when available; and citation [49] seems to be missing a journal/conference title.