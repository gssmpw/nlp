We are grateful to the reviewers for their clear and constructive feedback, which helped us sharpen the contribution and improve the paper. We have carefully attended to reviewer comments by making the following changes. 

Sharpening the contribution
Many criticisms stemmed from our unclear articulation of the paper's goals and contributions, and we thank R1 for pointing this out. We have revised the contribution statement in the introduction to clarify that this paper is neither an evaluation of a specific ethics toolkit nor an analysis of a particular city project. Our goal is to show the long-term effects of implementing tools such as those based on VSD in technical practice and to propose a conceptual framing to explain why efforts to design ethical reflection tools often fail to achieve long-term outcomes.

Strengthening the argumentation for moral stress as a framework
R1 suggested emphasizing why moral stress is an applicable theory in this context. We have revised sections 2.2 and 2.3 to better explain the core of this conceptual framework and how it relates to concepts such as VSD, value dams, and value levers in HCI. 

Methodology
All reviewers asked for more detail in the methods, which we appreciate as explaining an ethnographic process is often a challenge. We have added further detail on the data analysis, coding process (R1,R3) and task distribution within the research team (R2). 

Clarifying research context
We have struggled with presenting the context and tools in our study, especially given the strict anonymization requirements from the collaborating organization. We have restructured the research context section, adding clarification of the content of the ethics toolkit used by the teams (R1) and what role the toolkit and the workshops play in the team's work routines. We hope our edits make it clear that the mechanics of the toolkit and the intervention themselves are not under study here, but on the effects this effort to broaden the space for matters of concern for technical practice has on the team, demonstrating how the effort to design such interventions may overlook the very real negative emotional outcomes they might create even in what we observe as perhaps the best case scenario. 

Here we would like to respond directly to R1's critique of our positionality. The route-planner project gave us access to work with the team and we have added more detail to clarify some of the project specifics, such as that the project was conducted in collaboration with wheel chair users. However, the team under study was working on several projects simultaneously and the results we report came from our observations across these projects, as we followed the team and their daily work rather than focusing on a single project. We appreciate R1 calling us out on unreflexively implying value hierarchies in reporting on a particular tension the team encountered and have edited the paragraph to clarify the concerns without minimizing other's challenges. 

Findings
R1 and R3 noted that moral stress was less evident in the findings section. We have edited the section to highlight where and how moral stress manifested, and identify mitigation strategies we observed, both individual and organizational. We further address why neither value levers nor value dams are appropriate concepts to describe the team’s attempts to navigate a range of different ethical challenges, but potentially how and why things happen the way they do. We hope this section now provides clearer language to identify the socio-affective consequences of ethics interventions in technical practice. 

We agree with R3 that the within-team section is mostly based on the ethics owner perspective, but we felt it was impossible to expand the story there without making the paper overly long. We have made minor edits to expand on internal team dynamics without adding more quotes.

Discussion
Reviewers had several suggestions for the discussion, which we have addressed as follows:

1.	We have strengthened our engagement with the relevant HCI literature (R1), noting that many papers in HCI document what we consider moral stress without identifying it explicitly. This helps us to more clearly argue how HCI scholars and professionals may identify and address moral stress in the context of their research and engagement with technical practice (R2). We clarify that while creating healthy teams is not typically an HCI goal, awareness of its necessity for effective ethics interventions is critical in both intervention design and technical practice analysis.  
2.	We thank R3 for pointing out additional related work, particularly the Horgan and Dourish article, which helped us position some thoughts on what HCI scholars might do and point to some ways forward. 

We thank the reviewers for pointing to problems in references and minor mistakes. We have addressed these to the best of our ability and will continue to tinker with bibtex to ensure all references are correctly displayed.



%We are grateful to the reviewers for their clear and constructive feedback. We feel your feedback helped us sharpen the contribution and improve the paper. We have carefully attended to reviewer comments and made the following changes. 

%Sharpening the contribution - Many of the criticisms seemed to stem from our failure to clearly articulate the goals and contributions of the paper and we thank R1 for pointing this out. We have strengthened our contribution statement in the introduction to clarify that this paper is neither an evaluation of a particular ethics toolkit, nor an analysis of a particular city project. Our goal in this paper is to show what happens when ethics tools are implemented in technical practice long term and to propose a conceptual framing that helps explain why existing efforts to design tools for ethical reflection in technical practice rarely achieve the desired outcomes long term. 

%We have reworked the knowledge contributions listed in the introduction, to clarify that our intention with this paper is not to present another toolkit or method for the surfacing of ethical dilemmas or value tensions, but rather to report on the affective impact such toolkits and methods have on the professionals that are expected to adopt them. We hope that the re-formulation will resolve some questions from especialy reviewer 1 in regards to how the concept of moral stress builds on prior work within HCI, that contribute ethics oriented interventions, such as VSD. 

%Strengthening the argumentation for moral stress as a framework - R1 recommended that we highlight why moral stress is an applicable theory in this context. We have edited section 2.2 and restructured section 2.3 to better explain the core of this conceptual framework and how it relates to concepts such as VSD, value dams, and value levers in HCI. 

%We have also restructured the section of 2.3 to introduce the concept of moral stress with more importance (R1.3a). 

%Methodology - All reviewers asked for more detail in the methods, which we appreciate as explaining an ethnographic process is often a challenge. We provided additional detail on data analysis and coding process (R1,R3) and clarified task distribution within the research team (R2). 

% Clarifying research context - We have struggled with how to present the context and the tools that the teams use in our study, especially given the strict anonymization requirements from the organization we worked with. The manifesto and the toolkit are not particularly unique - and we have restructured the research context section, adding more detail to clarify the content of the ethics toolkit used by the teams (R1) as well as what role the toolkit and the workshops play in the team's routine work practices. We hope our edits make it clear that the mechanics of the toolkit and the intervention themselves are not under study here, rather our focus is on the effects this effort to broaden the space for matters of concern for technical practice has on the team, demonstrating how the effort to design such interventions may overlook the very real negative emotional outcomes these can create even in what we observe as perhaps the best case scenario. 

%Responding to a criticism from R1 about our own positionality, the route-planner project gave us the opportunity for entry to work with the team and we have added more detail to clarify some of the project specifics, such as the fact that the project was conducted in collaboration with wheel chair users. However, the team under study was working on several projects simultaneously and the results we report came from our observations across these because we followed the team and their daily work rather than focusing on a single project. We appreciate R1 calling us out on unreflexively implying value hierarchies in reporting on a particular tension the team encountered. We have clarified to make sure our meaning is clear without minimizing other's challenges. 

%We have also restructured the research context and data analysis sections. As the toolkit and manifesto themselves are not the focus of our research interest, we have moved the descriptions of these into the research context, rather than research findings. We hope this also helps with presenting the connection and positioning of the teams and the toolkit in a more straight forward way, as requested by reviewer 1 (R1.1,2a,2b). 
%We also want to respond to two specific concerns of reviewer 1 in regards to the research context. The first is the request on the consideration of salient values within both the observed technical team and the research team about accessibility concerns during the routeplanner project. While the research was conducted within the project context of developing a routeplanner for wheelchair uers, the outcomes and concrete value tensions the teams faced during this project are not the core of this papers contributions – as reviewer 1 correctly states, there is a strong research corpus that already reports on the political dimensions of accessibility projects within HCI and the dificulties around participation and involvement of disabled user groups. We added a few details to clarify some of the project specifics (such as the fact that the project was conducted in collaboration with wheel chair users), but want to highlight, that while we use the issues faced by the team in regards to the project, how they solve those issues is not the contribution of the paper, but rather the emotional and affective dimension of being faced with these dilemmas in the first place. 
%Reviewer 1 also requested more details on the manifesto and the toolkits concretely. Unfortunately we are unable to share more details without making the collaborating team easily identifyable through a simple google search. However, we also want to stress again that 

%Findings
%R1 and R3 commented on moral stress being less evident in the findings section. We have edited the findings sections to highlight where and how moral stress manifested and to more clearly identify mitigation strategies, both individual and organizational, that we had observed. 
%We further clarify why neither value levers nor value dams are appropriate concepts to describe not what happens as the team attempted to navigate through a range of different ethical challenges, but potentially how and why things happen the way they do. We hope this section now provides a much clearer language to identify the socio-affective consequences of ethics interventions in technical practice. 

%We agree with R3 that the within-team section is mostly based on the ethics owner perspective, but we felt it was impossible to expand the story there without making the paper overly long. We have made small edits to indicate the role the team dynamics played internally, but without additional quotes. 

%We have made some minor additions to the finding reports to highlight where the observed somatic and affective experiences mirror prior research and connect to the reported symptoms of moral stress. We hope that this strengthens our claim that the concept of moral stress is useful in characterising the impact the ethics intervention have on the team, as 

%Discussion
%Reviewers had several suggestions for the discussion section. We take these in turn. 

%First, we have strengthened our engagement with the relevant HCI literature (R1), pointing out that many existing papers essentially document what appear to be expressions of moral stress and attempts to mitigate it, without clearly identifying it. This allowed us to more clearly position what HCI scholars and professionals may be able to do in both identifying and addressing moral stress in the context of their research and engagement with technical practice (R2). While actually creating healthy and successful teams is not a typical goal of HCI research, we clarify that being aware that this is a de facto requirement for ethics interventions to work is key in the design of these interventions and in the analysis of technical practice more generally.  

%Second, we thank R3 for the additional related work. We were not aware of the Horgan and Dourish article and its relevance is undeniable especially in helping us position our thoughts on what HCI scholars might do and pointing to ways forward. 

%We thank the reviewers for pointing to problems in references and minor mistakes. We have addressed these to the best of our ability and will continue to tinker with bibtex to make sure all references are correctly displayed. 

%Lastly, we have reworked and deepened our discussion section. As reviewer 1 and 2 rightfully point out, we were missing the connection to some prior discussed HCI work, and the relevance of the CHI community. We have picked up on a collection of references we have introduced in the background section earlier, and added additional clarification on how we consider moral stress as a concept to connect to the motivation of HCI research to faciliate ethical consideration and responsible technical development in practice. We have also highlighted our suggestions for mitigations with additional references, and also added a clarifying paragraph that indicates that this paper is limited in mitigation strategies, and is opening up the field for further research into how mitigation strategies around moral stress from other fields can be carried over into HCI.  
