\section{Related Work}
\subsection{The Theory-Practice Gap in AI Ethics}
The gap between AI ethics principles and practice is well documented \cite{fjeld2020principled, mittelstadt2019principles} and there is consensus that crossing this gap is a non-trivial problem \cite{munn2023uselessness}. For example, in a review of 200 AI ethics guidelines and recommendations, Corrêa et al. \cite{correa2023worldwide} find that prescriptive normative claims are typically presented without considerations for how to achieve them. While there seems to be a convergence around which principles are most important, the interpretation and justifications of why and how these principles matter diverge widely \cite{jobin2019global}. Even the ACM code of ethics, while a decidedly important document, has little apparent impact on professionals in the tech community \cite{mcnamara2018does}. In a scoping review of responsible AI guidelines, Sadek et al. \cite{sadek2024challenges} note that the abstract nature of the principles and the lack of clear implementation procedures are important reasons for why the gap between principles and practice persists. 

In parallel to prescriptive principle-based approaches, we see a variety of materials targeting ethical concerns in technical practice, such as games \cite{ballard2019judgment, ali_ai_2023}, card-sets \cite{tkautz_cards_2021, artefact_tarot_nodate, calderon_blindspots_2021, ECCOLA}, toolkits \cite{tangible_ethical_nodate, gispen_ethics_2017, zou_design_2018}, and workshops \cite{33a_ai_2022, doteveryone_consequence_nodate, open_data_2021, hyper_island_unintended_nodate} that have been produced both by researchers and industry actors to operationalize ethical considerations into development processes. In the following we refer to such materials as ethics interventions. Reviews of these interventions consistently show that they tend to fall short of impact in the AI industry \cite{chivukula2021surveying, wong2023seeing, hagendorff2020ethics} as they do not account for the environments in which such tech practices are situated \cite{yildirim2023investigating, liao2020questioning, chivukula2020dimensions, winkler2021twenty}. For example, Morley and colleagues \cite{morley2020initial} demonstrate that almost all existing translational tools and methods are either too flexible (vulnerable to ethics washing) or too strict (unresponsive to context). The decoupling of policies, practices, and outcomes leads to practitioners facing major hurdles when trying to integrate AI ethics practices into development processes and organizational structures \cite{ali_ai_2023}. Interviews with industry leaders in particular report that lacking awareness and feasible practices are a challenge \cite{lindberg2024doing}, and that ethics efforts are perceived to be in tension with industry structures and values such as technological solutionism and market fundamentalism \cite{lindberg2024doing, metcalf2019owning}. 


\subsection{Ethics as a social and relational practice}
Where principles-based approaches often take their departure in philosophical considerations of ethical concerns, studies of ethics in practice in technical contexts demonstrate that technical practitioners are already ethically engaged through other means \cite{dindler2022engagements, widder2023s, wong2021tactics, ibanez2022operationalising, taylor2020constructing, deng_investigating_2023}. Shilton’s \cite{shilton2013values} notion of value levers for example describes how organizational and operational structures in design teams and labs influence the way values and ethical deliberations are being thematized, normalized, and included or excluded in daily routines. Shilton described different infrastructural aspects that support and embed discussions about values in the design of technology, calling attention to the fact that in a market driven design field, the constant pressure of technical innovation makes it more difficult for teams to make the time for a slow and deliberate value-driven design process. More recently, Lindberg, Karlström and Barbutiu \cite{lindberg2023cultivating} report on the persistent organisational barriers that make it difficult to create emotional buy in from practitioners to invest time and resources on ethics efforts, in particular where such efforts take more complex forms to encourage practitioners to reflect and engage with ethical implications of their work beyond "tick box exercises". 

Such research highlights that ethics in practice is not the same as ethics as normative inquiry. The modularized nature of the software development workflow, inherent to any AI development, makes it difficult to pin down where exactly accountabilities and responsibilities for principles and values are located among the developers and other stakeholders \cite{widder2023dislocated}. Organisational processes such as project planning, resource management, and team hierarchies strongly influence ethical decision-making dynamics \cite{devon_design_2004}. The central debate in tech ethics is therefore not whether or which ethics is desirable, but what “ethics” entails, who gets to define it, and what kind of impact those questions have in practice \cite{green_contestation_2021}. Wong and colleagues \cite{wong2023seeing} note that ethics toolkits often lack guidance around how to navigate labor, organizational, and institutional power dynamics as they relate to technical work. Currently, the gap between principles and practice requires a lot of invisible work from technologists to navigate hierarchies, market dynamics, and the lack of awareness in the rest of the organization, which mostly goes unacknowledged \cite{deng_investigating_2023, wang2023designing}. 

Demands and idealistic notions of ethical reflection and action then clash with the reality of everyday industry practice, in which tech logics and organisational structures stand in stark opposition to the necessary resources required for ethical reflection - such as uncertainty, criticality, and slowness \cite{stark2009creative}. Yet little research has explored the impact and affective experience of integrating ethics tools into environments where such clashes are inevitable. We follow scholars of ethics in practice \cite{drage2024engineers, powell2022addressing, raji2021you, shklovski2023nodes} who argue against a too individualized approach to ethics, and consider ethics as a relational practice, where responsibility lies between the technical expert and the organizational structures within which they operate \cite{kranakis2004fixing}. We also build on research that argues for considering ethics as a situated and lived experience \cite{shilton2018engaging}, suggesting that the emotional context of ethics is important for bridging the gap between principles and practice \cite{dunbar2005emotional}. Prior research that reports on concrete methods, tools, and specific case studies of their usage, such as value dams \cite{Miller2007ValueDams}, and a variety of similar ethics interventions developed in HCI, tell us how practitioners deal with concrete trade offs and argue that those skills are transferable to future projects. They do not report on the longer term shifts in mindset and how that makes affected practitioners feel about their work moving forward. We therefore investigate ethical considerations as emotional and embodied \cite{su2021critical}, especially where they challenge existing social and political norms in an organisational context, coming into conflict with the techno-optimism that often accompanies technology development \cite{gould2009moving}.

%Methods such as VSD provide practical and actionable approaches for addressing value tensions. VSD in itself however does not provide a lens to understand the emotional and affective impact their application has on the practitioners. This is where we suggest Moral stress as a beneficial lens - rather than evaluating the ways in which such tools provide solutions for ethical dilemmas, we argue that, since ideal solutions are impossible, we need to understand and account for the emotional fall out of that recognition. 
%Our study closely observes on-the-ground practices and raises the tensions and contradictions that challenge what technical practitioners seem to believe about ethical practice, and abstract ideals and ideas such as inclusion, and responsibility. Value Dams require resources and time to ask people, involve people, very different stakeholders. VSD is making ethical because of a moral process. The operationalisation of ethics justifies the outcome. What we talk about is something very different. The team is growing, this is growing pains, the team is deep in the distress phase, not mature in living with the discomforts. What they do might not result in what they want, and we observe them starting to reflect on what they want.}

%\revision{There are two opposing motivations in different ethics tools - tools that encourage reflection and critical questioning aim to open up technical practice to inquisitive dialogue about ethical implications, leading to increasing uncertainty. On the other hand, "tick box exercises" are used to close down dialogue about ethical implications - such engagement gives a stamp of approval to further technical development, enabling a faster moving on from ethical concerns to production or logistical challenges. In cultures which prize the later, closing down of ethical dialogue might be considered as more productive and efficient than the opening up of ethics conversations.}


\subsection{Emotional response, ethical sensitivity and moral stress}
Moral reasoning and ethical decision-making have an emotional component that has long been overlooked by responsible AI research efforts. Deep reflection that is required in design is a type of emotion work, that can come at the cost of feelings of guilt, self-blame, and emotional exhaustion \cite{ballam2019emotionwork}. Exploring ethical considerations in design through a soma design perspective, Garrett et al. \cite{garrett2023felt} and Popova et al. \cite{popova2022vulnerability} demonstrate that connecting ethical sensitivity to emotional experience can provide a generative approach to understanding the embodied perspective of ethical decision making, but also involves distancing and vulnerability as key emotional dimensions \cite{popova2023should}. Investigating an applied setting, Ma and colleagues report on the self-doubts, insecurities, and suffering faced by designers being limited in their decision-making power in addressing unfair business models through design decisions \cite{Ma2024SenseOfMorality}. Engaging with ethical considerations then, can be a challenging and vulnerable, emotion-laden process, but so far HCI has lacked the language to systematically describe the experience and its implications.

Outside of HCI, the challenges of moral conduct and ethics in practice have been well studied from a psychological perspective.
In fields such as management studies and nursing studies the fact that practitioners respond emotionally to moral quandaries and ethical dilemmas is established consensus. Studies from these fields provide us with defined vocabulary to identify internal processes around moral decision-making in practice. Jordan \cite{jordan2009social} for example defines the trait of having developed a situated attentiveness for the moral relevance of decisions and actions, in particular in connection to specific values, as \textit{moral awareness}. Based on a review of 108 studies of from fields such as medicine and accounting, Boyd and Shilton \cite{boyd2021adapting} review the concept of \textit{ethical sensitivity} for its relevance for technical practice. They conceptualise it as a three step process comprised of recognition (noticing an ethical problem, due to moral awareness), particularisation (understanding the situation that creates it) and judgment (being able to decide what to do about it).

Moral awareness and ethical sensitivity describe important parts of moral reasoning that precede morally motivated action and provide vocabulary to analyse the conduct and experiences of practitioners grappling with ethical challenges. An increase in ethical sensitivity and moral awareness is a common goal for practical ethics interventions in tech \cite{Boyd_2022_DesigningUp}. The assumption is that expanding the worldview of developers and designers will lead to engagement in deeper ethical reflection, which can result in increased ethical sensitivity, in turn leading to morally sound development efforts and responsible technology design \cite{vanhee2022ethical}.

However, increased ethical sensitivity and moral awareness can also have negative consequences \cite{weaver2007ethical}. An increase of self-reflective behaviour can make people more self-conscious and uncertain about their own actions, and a higher moral awareness is linked to higher levels of stress when roles and environment do not align \cite{Ames2020Antecedentsandconsequences}. Research from nursing studies reports that nurses suffer when the constraints of nursing work, such as the lack of time, supervisory reluctance or institutional policy, limit or prevent capacity for moral action\cite{corley2001development}. In management studies Reynolds and colleagues \cite{reynolds2012moral} identify suffering as a result of having to make morally relevant decisions, especially when there is a discrepancy between individual and organizational goals or views of the right actions. The discrepancy can come from having to choose between what are seen as right and wrong actions, but it can also occur when choosing between two legitimate but contradictory right options or two that appear equally wrong \cite{reynolds2012moral}.  

These adverse experiences as outcomes of increased ethical sensitivity are summarised by the concept of \textit{moral stress} \cite{reynolds2015recognition} (in nursing studies termed moral distress \cite{morley2019moraldistress, Lutzen2012Moraldistress, imbulana2021interventions}). Reynolds and colleagues define moral stress as "a psychological state (both cognitive and emotional) marked by anxiety and unrest due to an individual’s uncertainty about his or her ability to fulfill relevant moral obligations." \cite{reynolds2012moral}.
This type of stress arises from having to manage a mismatch between moral ideals and reality \cite{reynolds2012moral,reynolds2015recognition}. Emotionally, moral stress manifests outwardly through anxiety, frustration, and anger \cite{Lutzen2012Moraldistress}.
%Interventions that aim to operationalise moral decision making around technical systems neglect that it is still people that are required to take these decisions, even when there is a process provided to making them, and that even where value tensions are recognized and negotiated, this is unlikely to address the experience of stress. Those in the position to make decisions must shoulder the potentially negative outcomes of a decision that is beneficial for the system. Where such individualised burden is not recognised, moral stress can turn into frustration, anger, and potential cognitive burn out \cite{reynolds2015recognition}. 
%one thing to highlight - while moral stress manifests as psychological and physiological symptoms, its sources and comping mechanisms are relational and must engage institutional infrastructures. when the weight of the system becomes individualized (because people are  human and moral decisions are human) this is where moral stress becomes difficult - when people know what is good for the system but they still need to face the individual harm that is caused in favor of the system functioning. the physiological manifestation is through - anxiety, frustration and anger. 
 %that results from a disconnect between recognizing a potential ethical issue and a lack of capacity to act on it \cite{reynolds2012moral,morley2020initial}.
While there is disagreement about the precise conditions that cause moral stress, work load, time pressure, and role ambiguity are cited as the most common antecedents \cite{imbulana2021interventions,morley2019moraldistress}. When unaddressed, moral stress is reported to lower the quality of patient care or management decision-making, resulting in avoidance of morally relevant situations and even leading people to change workplace or occupation \cite{morley2019moraldistress}. Research suggests a range of organisational mitigation strategies in order to avoid cognitive and emotional burn out \cite{reynolds2015recognition}.

Despite the richness of research looking into operationalising ethical considerations into technical practice tools, few studies have explored the \textit{lived experience} of such operationalisation efforts. While the importance of emotions in ethical decision making is starting to be explored in HCI research, no conceptual bridge yet exists between operationalised ethics efforts and the affective experiences they trigger, especially where such experiences are negative. Ethics toolkits and similar materials to operationalize ethics are often expected to function like any other organizational process, while the surrounding power structures and required relational labor are overlooked \cite{wong2023seeing}. In addition, where ethical sensitivity has been researched in the field of technology design, it has been as a property of individuals, while technology design typically happens in teams and collaborative structures \cite{boyd2021adapting}. 

We present insights from an ethnographic study of technology development processes that have integrated ethics tools into everyday practice, paying attention to the embodied, situated, and emotional context of doing ethics with such tools. Leveraging the extant research on moral stress, we propose it as a valuable concept to provide vocabulary to help interpret our observations of ethical deliberation beyond acts of reflection, towards understanding the lived affective experience of practitioners. We see moral stress as an additional dimension contributing to our understanding of why well-intended ethics efforts struggle with sustainable adoption within industry.