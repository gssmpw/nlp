\subsection{Proof of Theorem \ref{theorem:concentration-block}: Concentration Under Polynomial Ergodicity}
\label{app:proof-concentration-block}
\begin{theoremapp*}[Concentration Under Polynomial Ergodicity]
Let \(\{x_t\}_{t=1}^n\) be a sequence of random variables generated by a Markov chain satisfying polynomial ergodicity as per Theorem \ref{theorem:polynomial-ergodicity} with rate \(\beta > 1\). Assume that each \(x_t\) has bounded variance, i.e., \(\text{Var}(x_t) \leq \sigma^2\) for some constant \(\sigma^2 > 0\), and that the rewards are bounded, i.e., \(|x_t| \leq M\) almost surely for some constant \(M > 0\). Then, for any \(\epsilon > 0\),
\[
\mathbb{P}\left(\left|\frac{1}{n}\sum_{t=1}^n x_t - \mathbb{E}[x_t]\right| > \epsilon\right) \leq 2\exp\left(-\frac{n\epsilon^2}{2C_\beta}\right),
\]
where \(C_\beta\) depends on the mixing rate \(\beta\) and the bound \(M\).
\end{theoremapp*}

\begin{proof}
We employ a block decomposition strategy and apply Bernstein's inequality \cite{vershynin2018high} for dependent sequences. The proof proceeds as follows:

\textbf{1) Define the Blocks:}

Partition the sequence \(\{x_t\}_{t=1}^n\) into non-overlapping blocks of size \(b\):
\[
Y_k = \sum_{t=(k-1)b + 1}^{kb} x_t, \quad k = 1, 2, \ldots, \left\lfloor \frac{n}{b} \right\rfloor.
\]
Let
\[
S_n = \sum_{k=1}^{\left\lfloor \frac{n}{b} \right\rfloor} Y_k
\]
be the sum of the block sums. Here, \(S_n\) aggregates the contributions from each block, thereby facilitating the analysis of dependencies between blocks.

\textbf{2) Bounding \(|Y_k|\):}

Since each \(x_t\) is bounded by \(M\), the sum of \(b\) such variables satisfies:
\[
|Y_k| \leq \sum_{t=(k-1)b + 1}^{kb} |x_t| \leq bM.
\]
Thus, each block sum \(Y_k\) is bounded by \(bM\).

\textbf{3) Covariance Between Blocks:}

From Lemma \ref{lemma:covariance-between-blocks}, the covariance between blocks \(Y_k\) and \(Y_j\) satisfies:
\[
|\text{Cov}(Y_k, Y_j)| \leq C b^2 |k - j|^{-\beta}.
\]
This bound reflects the inter-block dependence, as intra-block dependencies are encapsulated within the definition of each block \(Y_k\).

The total covariance across all block pairs is:
\[
\sum_{k=1}^{\left\lfloor \frac{n}{b} \right\rfloor} \sum_{j \neq k} |\text{Cov}(Y_k, Y_j)| \leq C b^2 \sum_{k=1}^{\left\lfloor \frac{n}{b} \right\rfloor} \sum_{j \neq k} |k - j|^{-\beta}.
\]
Approximating the double sum by an integral for large \(n/b\),
\begin{align*}
\sum_{k=1}^{\left\lfloor \frac{n}{b} \right\rfloor} \sum_{j \neq k} |k - j|^{-\beta} &\approx 2 \sum_{k=1}^{\left\lfloor \frac{n}{b} \right\rfloor} \sum_{m=1}^{\left\lfloor \frac{n}{b} \right\rfloor - k} m^{-\beta}\\
&\leq \frac{2}{\beta - 1} \left(\left\lfloor \frac{n}{b} \right\rfloor^{1 - \beta} - 1\right).
\end{align*}
For \(\beta > 1\),
\[
\sum_{k=1}^{\left\lfloor \frac{n}{b} \right\rfloor} \sum_{j \neq k} |\text{Cov}(Y_k, Y_j)| \leq \frac{2 C b^2}{\beta - 1}.
\]
Thus, the variance of \(S_n\) is:
\begin{align*}
\text{Var}(S_n) &= \sum_{k=1}^{\left\lfloor \frac{n}{b} \right\rfloor} \text{Var}(Y_k) + \sum_{k=1}^{\left\lfloor \frac{n}{b} \right\rfloor} \sum_{j \neq k} \text{Cov}(Y_k, Y_j)\\
&\leq C n b + \frac{2 C b^2}{\beta - 1}.
\end{align*}
Here, \(\text{Var}(Y_k)\) is bounded by \(C b\) assuming \(\text{Var}(x_t) \leq \sigma^2\), and there are \(\left\lfloor n/b \right\rfloor\) such blocks.

\textbf{4) Choosing Block Size \(b\):}

To balance the variance and covariance contributions, select \(b = n^{1/(\beta + 1)}\). This choice ensures that:
\[
C n b = C n^{(\beta + 2)/(\beta + 1)} \quad \text{and} \quad \frac{2 C b^2}{\beta - 1} = \frac{2 C n^{2/(\beta + 1)}}{\beta - 1}.
\]
Given that \(\beta > 1\), \(2/(\beta + 1) < (\beta + 2)/(\beta + 1)\), implying that the variance term \(C n b\) dominates for large \(n\). However, for the purpose of the concentration bound, we require the total variance \(\text{Var}(S_n)\) to scale linearly with \(n\). Therefore, we adjust the block size to ensure that:
\[
C n b + \frac{2 C b^2}{\beta - 1} \leq C' n,
\]
where \(C'\) is a constant. This is achievable by selecting \(b = n^{1/(\beta + 1)}\), leading to:
\[
\text{Var}(S_n) \leq C n.
\]

\textbf{5) Application of Bernstein's Inequality:}

Given that each block sum \(Y_k\) is bounded by \(bM\) and has variance \(\text{Var}(Y_k) \leq C b\), we can apply Bernstein's inequality to \(S_n\). Bernstein's inequality for sums of bounded random variables states that for any \(\epsilon > 0\),
\[
\mathbb{P}\left(|S_n - \mathbb{E}[S_n]| > \epsilon n\right) \leq 2\exp\left(-\frac{\epsilon^2 n^2}{2 \text{Var}(S_n) + \frac{2}{3} M_Y \epsilon n}\right),
\]
where \(M_Y = bM\) bounds \(|Y_k|\).

Substituting \(\text{Var}(S_n) \leq C n\) and \(M_Y = bM = n^{1/(\beta + 1)} M\), the inequality becomes:
\begin{align*}
&\mathbb{P}\left(\left|\frac{1}{n} S_n - \mathbb{E}\left[\frac{S_n}{n}\right]\right| > \epsilon\right)\leq
2\exp\left(-\frac{\epsilon^2 n^2}{2 C n + \frac{2}{3} M n^{1/(\beta + 1)} \epsilon n}\right).
\end{align*}
Simplifying the denominator:
\[
2 C n + \frac{2}{3} M n^{1 + 1/(\beta + 1)} \epsilon = 2 C n + \frac{2}{3} M \epsilon n^{(\beta + 2)/(\beta + 1)}.
\]
Since \(\beta > 1\), \((\beta + 2)/(\beta + 1) > 1\), and for large \(n\), the term \(2 C n\) dominates the denominator. Thus, for sufficiently large \(n\),
\[
\mathbb{P}\left(\left|\frac{1}{n} S_n - \mathbb{E}\left[\frac{S_n}{n}\right]\right| > \epsilon\right) \leq 2\exp\left(-\frac{\epsilon^2 n}{2 C_\beta}\right),
\]
where \(C_\beta = C\) encapsulates the constants from the variance and block size selection.

\textbf{6) Final Concentration Bound:}

Combining the above steps, we obtain:
\[
\mathbb{P}\left(\left|\frac{1}{n}\sum_{t=1}^n x_t - \mathbb{E}[x_t]\right| > \epsilon\right) \leq 2\exp\left(-\frac{n \epsilon^2}{2 C_\beta}\right),
\]
where \(C_\beta\) depends on the mixing rate \(\beta > 1\) and the bound \(M\) on \(x_t\).

\textbf{7) Limitation for \(\beta \leq 1\):}

It is important to note that Theorem \ref{theorem:concentration-block} requires \(\beta > 1\) to ensure the convergence of the covariance sum and the validity of the concentration bound. In cases where \(\beta \leq 1\), the dependencies between blocks decay too slowly, leading to potential divergence in covariance terms and invalidating the concentration bound.

\textbf{Conclusion:}

This bound demonstrates that the empirical average \(\frac{1}{n}\sum_{t=1}^n x_t\) concentrates around its expectation \(\mathbb{E}[x_t]\) with high probability, governed by the mixing rate \(\beta > 1\). The block decomposition effectively handles inter-block dependencies, while intra-block dependencies are managed through the block size selection and the boundedness of \(x_t\).
\end{proof}