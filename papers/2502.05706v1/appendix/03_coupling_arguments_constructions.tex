\subsection{Proof of Lemma \ref{lemma:coupling-argument-polynomial-mixing}: Coupling Argument under Polynomial Mixing}
\label{app:proof-coupling-argument-polynomial-mixing}
\begin{lemmaapp*}[Coupling Argument under Polynomial Mixing]
Let \(\{x_t\}_{t=1}^\infty\) and \(\{y_t\}_{t=1}^\infty\) be two Markov chains satisfying polynomial ergodicity as per Theorem \ref{theorem:polynomial-ergodicity} with rates \(\alpha_x\) and \(\alpha_y\), respectively. Assume both chains satisfy the block independence condition from Lemma \ref{lemma:dependent-blocks}. There exists a coupling such that for all \(t \geq 1\),
\[
\mathbb{P}(x_t \neq y_t) \leq C t^{-\beta},
\]
where \(C > 0\) and \(\beta = \min\{\alpha_x, \alpha_y\}\).
\end{lemmaapp*}
\begin{proof}
\;\newline
\paragraph{Objective:}  
Construct a coupling of two instances of the Markov chain $(x_t)$, denoted by $(x_t)$ and $(y_t)$, such that the probability that they differ at time $t$ decays polynomially with $t$.
\paragraph{1) Utilizing the Markov Property}
By the \textbf{Markov property}, the distribution of block \( B_i \) given block \( B_{i-1} \) depends solely on the last state of \( B_{i-1} \), denoted as \( x_{(i-1)b} \). Therefore, we have:
\[
\mathbb{P}(B_i \mid B_{i-1}) = P^b(x_{(i-1)b}, \cdot),
\]
where \( P^b \) denotes the \( b \)-step transition kernel of the Markov chain.
\paragraph{2) Relating Total Variation Distance to Coupling Probability}
Under the assumption that the Markov chain is in its \textbf{stationary distribution} \( \pi \), the marginal distribution of any block \( B_i \) is:
\[
\mathbb{P}(B_i) = \pi(\cdot).
\]
Consequently, the Total Variation (TV) Distance between the conditional distribution \( \mathbb{P}(B_i \mid B_{i-1}) \) and the marginal distribution \( \mathbb{P}(B_i) \) satisfies:
\[
\Bigl\| \mathbb{P}(B_i \mid B_{i-1}) - \mathbb{P}(B_i) \Bigr\|_{TV}
= \Bigl\| P^b(x_{(i-1)b}, \cdot) - \pi(\cdot) \Bigr\|_{TV}.
\]
\textbf{Justification:}  
This equality holds because \( \mathbb{P}(B_i \mid B_{i-1}) \) is precisely the distribution \( P^b(x_{(i-1)b}, \cdot) \), and \( \mathbb{P}(B_i) \) is the stationary distribution \( \pi(\cdot) \). Therefore, the TV Distance between these two distributions is exactly the TV Distance between \( P^b(x_{(i-1)b}, \cdot) \) and \( \pi(\cdot) \).
\paragraph{3) Applying the Polynomial Mixing Condition}
Given the \textbf{polynomial mixing condition}, for any state \( x \in \mathcal{X} \) and block size \( b \), the TV Distance between the \( b \)-step transition probability \( P^b(x, \cdot) \) and the stationary distribution \( \pi(\cdot) \) satisfies:
\[
\Bigl\| P^b(x, \cdot) - \pi(\cdot) \Bigr\|_{TV}
\leq C_0\,b^{-\alpha} \cdot (1 + V(x)),
\]
where:
\begin{itemize}
    \item \( C_0 > 0 \) is a constant independent of \( b \) and \( x \),
    \item \( \alpha > 0 \) characterizes the rate of mixing,
    \item \( V(x) \) is a Lyapunov function ensuring that states with higher \( V(x) \) are appropriately weighted.
\end{itemize}
\paragraph{4) Bounding the Probability of Non-Coupling}
To bound \( \mathbb{P}(x_t \neq y_t) \), we analyze the coupling process over successive blocks of size \( b \).
\newline\textbf{Partitioning Time into Blocks:}
Divide the time horizon into non-overlapping blocks of size \( b \):
\[
  B_i \;=\; (x_{(i-1)b+1}, \dots, x_{ib})
  \quad
  \text{for } i=1,2,\dots.
\]
\newline\textbf{Coupling at Each Block Boundary:}  
At the end of each block \( k \), the TV Distance between the distributions of \( x_{kb} \) and \( y_{kb} \) is bounded by:
\[
\Bigl\| P^b(x_{(k-1)b}, \cdot) - \pi(\cdot) \Bigr\|_{TV} \leq C_0\,b^{-\alpha} \cdot (1 + V(x_{(k-1)b})).
\]
This follows directly from the polynomial mixing condition applied to each block transition.
\newline\textbf{Using the Lyapunov Function for Uniform Bounds:}  
The Lyapunov drift condition ensures that:
\[
\mathbb{E}\left[V(x_{kb})\right]
\leq \frac{K}{1 - \lambda}
\]
for constants \( K > 0 \) and \( 0 < \lambda < 1 \). This implies that, on average, \( V(x_{kb}) \) is bounded, allowing us to replace \( V(x_{(k-1)b}) \) with a constant bound \( V_{\max} \) in expectation:
\[
\Bigl\| P^b(x_{(k-1)b}, \cdot) - \pi(\cdot) \Bigr\|_{TV} \leq C_0\,b^{-\alpha} \cdot (1 + V_{\max}).
\]
\newline\textbf{Aggregating Over Blocks to Bound \( \mathbb{P}(\tau > t) \):}  
The probability that the chains have not coupled by time \( t \) can be bounded by summing the bounds over all preceding blocks:
\[
\mathbb{P}(\tau > t) \leq \sum_{k=1}^{t/b} \Bigl\| P^b(x_{(k-1)b}, \cdot) - \pi(\cdot) \Bigr\|_{TV} \leq C_0\,b^{-\alpha} \cdot (1 + V_{\max}) \cdot \frac{t}{b}.
\]
Simplifying, we obtain:
\[
\mathbb{P}(\tau > t) \leq C \cdot t \cdot b^{-\alpha -1},
\]
where \( C = C_0 (1 + V_{\max}) \).
\newline\textbf{Choosing Block Size \( b \) Appropriately:}  
To ensure that \( \mathbb{P}(\tau > t) \leq C t^{-\alpha} \), choose \( b = t^{\gamma} \) for an appropriate \( \gamma \) that satisfies:
\[
t \cdot b^{-\alpha -1} = t \cdot t^{-\gamma(\alpha +1)} = t^{1 - \gamma(\alpha +1)} \leq t^{-\alpha}.
\]
Solving for \( \gamma \), we require:
\[
1 - \gamma(\alpha +1) \leq -\alpha \implies \gamma(\alpha +1) \geq \alpha +1 \implies \gamma \geq 1.
\]
Therefore, setting \( \gamma = 1 \) (i.e., \( b = t \)) suffices:
\[
\mathbb{P}(\tau > t) \leq C t^{-\alpha}.
\]
\paragraph{5) Concluding the Coupling Bound}
Combining the above steps, we conclude that:
\[
\mathbb{P}(x_t \neq y_t) = \mathbb{P}(\tau > t) \leq C t^{-\alpha},
\]
where \( C > 0 \) is a constant determined by \( C_0 \), \( V_{\max} \), and other constants from the Lyapunov drift condition.
\paragraph{6) Conclusion:}
The constructed coupling ensures that the probability of the coupled chains \( x_t \) and \( y_t \) differing at time \( t \) decays polynomially with \( t \), as required by the lemma. This completes the proof of Lemma \ref{lemma:coupling-argument-polynomial-mixing}.
\end{proof}