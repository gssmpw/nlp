\section{Stability and Mixing Conditions}
In reinforcement learning (RL), understanding the behavior of the underlying Markov decision processes (MDPs) is crucial for designing algorithms that converge to optimal solutions.

These properties help ensure that the system does not diverge or remain stuck in certain regions of the state space.
Instead, the process explores the environment sufficiently, allowing the agent to collect diverse experiences.
The stability conditions, like drift, quantify the tendency of the process to return to a stable region, while mixing conditions, like polynomial mixing, describe how quickly the influence of initial states diminishes over time.

Together, these conditions establish a foundation for analyzing the convergence of RL algorithms, especially in environments with complex dynamics or long-term dependencies.

\vspace{1em}
\begin{assumption}[Polynomial Mixing]
\label{ass:polynomial-mixing}
The Markov process \((x_t)\) satisfies polynomial mixing: for any bounded measurable functions \(f\) and \(g\),
\[
|\mathbb{E}[f(x_t)g(x_{t+k})] - \mathbb{E}[f(x_t)]\mathbb{E}[g(x_{t+k})]| \leq C k^{-\beta},
\]
for some constants \(C > 0\) and \(\beta > 1\).
\end{assumption}
\vspace{1em}
\begin{assumption}[Drift Condition]
\label{ass:drift-condition}
There exists a function \(V: S \to [1, \infty)\) and constants \(\lambda > 0\), \(b < \infty\), such that:
\[
\mathbb{E}[V(x_{t+1}) \mid x_t] \leq V(x_t) - \lambda W(x_t) + b,
\]
where \(W(x_t)\) is a positive function representing the distance of \(x_t\) from a stable region.
\end{assumption}
\vspace{1em}
\begin{lemma}[Polynomial Ergodicity]
\label{theorem:polynomial-ergodicity}
Under the above polynomial mixing \ref{ass:polynomial-mixing} and drift condition \ref{ass:drift-condition}, the Markov process satisfies:
\[
|\mathbb{P}(x_t \in A) - \pi(A)| \leq C (1 + V(x_0)) t^{-\beta},
\]
for some constants \(C > 0\), \(\pi\) is the stationary distribution,
 \( V: \mathcal{X} \to [1,\infty) \) is a Lyapunov function, and \( \beta > 1 \).
\end{lemma}

\begin{proof}
The proof follows from the drift condition and coupling arguments that ensure mixing rates of \(t^{-\beta}\). Details are deferred to Appendix~\ref{app:proof-polynomial-ergodicity}.
\end{proof}