% arXiv directives
\pdfoutput=1  % Forces PDF output

\documentclass{article}
\usepackage{graphicx} % Required for inserting images
\usepackage[accepted]{icml2025}
% \usepackage{icml2025}
\usepackage{svg}
\usepackage{hyperref}

%for text coloring
\usepackage{xcolor}
%for long comments
\usepackage{comment}

\usepackage{booktabs}

% For theorems and such
\usepackage{amsmath}
\usepackage{amssymb}
\usepackage{mathtools}
\usepackage{amsthm}

\newtheorem{theorem}{Theorem}
\newtheorem{lemma}{Lemma}
\newtheorem{definition}{Definition}

\usepackage[capitalize,noabbrev]{cleveref}

\usepackage{placeins}
\usepackage{makecell}
\usepackage{xcolor} % for prompts
\usepackage{tcolorbox} % for prompts
\tcbset{
  colframe=black!75,    % Box border color
  colback=gray!10,      % Box background color
  sharp corners=all,    % Use sharp corners for alignment
  boxrule=0.5mm,        % Border thickness
  left=4mm,             % Left padding
  right=4mm,            % Right padding
  top=2mm,              % Top padding
  bottom=2mm,           % Bottom padding
}


\begin{document}

\twocolumn[
\icmltitle{From Euler to AI: Unifying Formulas for Mathematical Constants}

\icmlsetsymbol{equal}{*}

\begin{icmlauthorlist}
\icmlauthor{Tomer Raz}{Technion}
\icmlauthor{Michael Shalyt}{Technion}
\icmlauthor{Elyasheev Leibtag}{Technion}
\icmlauthor{Rotem Kalisch}{Technion}
\icmlauthor{Yaron Hadad}{Technion}
\icmlauthor{Ido Kaminer}{Technion}


\end{icmlauthorlist}

\icmlaffiliation{Technion}{Technion - Israel Institute of Technology, Haifa 3200003, Israel}

\icmlcorrespondingauthor{Ido Kaminer}{kaminer@technion.ac.il}

% You may provide any keywords that you
% find helpful for describing your paper; these are used to populate
% the "keywords" metadata in the PDF but will not be shown in the document
\icmlkeywords{Machine Learning, ICML, Mathematics, Pi, Formula, Number Theory}

\vskip 0.3in
]

\printAffiliationsAndNotice{}

\begin{abstract}
\label{section-abstract}

The constant {\large $\pi$} has fascinated scholars for centuries, inspiring the derivation of countless formulas rooted in profound mathematical insight. This abundance of formulas raises a question: Are they interconnected, and can a unifying structure explain their relationships?

We propose a systematic methodology for discovering and proving formula equivalences, leveraging modern large language models, large-scale data processing, and novel mathematical algorithms. 
Analyzing 457,145 arXiv papers, over a third of the validated formulas for $\pi$ were proven to be derivable from a single mathematical object 
— including formulas by Euler, Gauss, Lord Brouncker, and newer ones from algorithmic discoveries by the Ramanujan Machine.

Our approach extends to other constants, such as $e$, $\zeta(3)$, and Catalan’s constant, proving its broad applicability. This work represents a step toward the automatic unification of mathematical knowledge, laying a foundation for AI-driven discoveries of connections across scientific domains.

\vspace{-10pt}
\end{abstract}

\section{Introduction}
\label{section-introduction}

The earliest rigorous approximation for $\pi$ dates back to Archimedes, around 250 BCE \cite{archimedess2020} establishing the bounds $\frac{223}{71} < \pi < \frac{22}{7}$. 
Modern algorithms for calculating $\pi$ employ more sophisticated formulas. For example, the Chudnovsky algorithm \cite{ChudnovskyAlgorithm}, derived from a formula by Ramanujan \cite{Ramanujan1914}, remains instrumental for precision records. Similarly, the BBP formula \cite{BBP1996} is notable for enabling computation of specific $\pi$ digits without requiring prior digits.
Such breakthroughs inspired fundamental advances in computer science, such as high-precision arithmetic \cite{BAILEY201210106}, evolutionary optimization \cite{koza1994genetic}, and elliptic curve cryptography \cite{10.1007/3-540-39799-X_31}.
Recent efforts led to the development of computer algorithms capable of generating numerous formula hypotheses for mathematical constants \citep{Raayoni2021, esma, doi:10.1073/pnas.2321440121}. 

\begin{figure}[ht]
    \centering
    \includegraphics[width=1\linewidth]{Images/Pi_Formulas_With_Background.pdf}
    \caption{Selected $\pi$ formulas discovered over the centuries.}
    \label{fig:pi-formulas}
    \vspace{-6pt}
\end{figure}

\begin{figure*}[h]
    \centering
    \vspace{-4pt}
    \includegraphics[width=0.8\linewidth]{Images/overview_figure.pdf}
    \vspace{-8pt}
    \caption{\textbf{Automated methodology for unifying mathematical knowledge.} A large corpus of mathematical formulas is processed and converted to executable code for validation, clustering, and generalization into a unifying mathematical structure.}
    \label{fig:Overview}
    \vspace{-10pt}
\end{figure*}

The plethora of results related to $\pi$ discovered over the centuries led to a persistent question: How are they all connected?
This question is important not only for preventing accidental rediscoveries (e.g., Lange's formula from 1999 \cite{Lange1999AnEC} had already been derived by Lord Brouncker in 1656 \cite{Osler15012010}). Many equivalent formulas appear vastly different at first glance. A simple example is Euler's continued fraction that provides equivalent representations of infinite sums \cite{euler1748introductio}. 
There exists a wide range of methods for proving equivalences, but these methods are often completely different from one another. 
This complex situation underscores the need for a systematic approach to unify these relationships.
In fact, any future large-scale effort to \textit{unify mathematical knowledge} will need to address this conceptual challenge
to deepen our understanding of the shared underlying structures.

In this work, we propose a system for the large-scale harvesting, identification, and unification of mathematical formulas (see Fig. \ref{fig:Overview}). This effort leverages recent advances in content understanding based on large language models (LLMs), the newly discovered concept of Conservative Matrix Fields (CMFs) \cite{doi:10.1073/pnas.2321440121}, and a novel mathematical algorithm that we call \textit{coboundary equivalence} for finding and proving relations between formulas (Section \ref{section-methodology-matching}). The combined system is based on several key steps: harvesting the relevant mathematical expressions from arXiv and other open sources, which are then extracted and validated in a multi-stage process (Section \ref{subsection-engineering}). This automatic process converts all formulas to a canonical form (Section \ref{section-formula-canonicalization}). The process is completed by a clustering stage in which our coboundary algorithm unifies the formulas in a CMF structure (Section \ref{section-methodology-matching}).

To demonstrate this methodology, we selected the symbolic case study of formulas explicitly calculating $\pi$. A total of 457,145 papers from arXiv (in multiple languages) were analyzed, chosen based on their labels (see Appendix \ref{appendix-engineering-article-retrieval}), encompassing 278 million mathematical expressions. 
After a ``harvesting'' process that identifies relevant formulas, 254 different $\pi$  formulas were validated, of which 37\% were found
to correspond to different trajectories within a single CMF. 
We expect near-future improvements of our algorithm to classify all the $\pi$ formulas into one or just a few unique CMFs that unify all knowledge about $\pi$ calculation.

The success of this study highlights the prospects of such algorithms to autonomously unify vast mathematical knowledge.
Beyond the example of $\pi$, we applied our algorithm to other mathematical constants like $e$, $\zeta(3)$, and Catalan’s constant, and to a variety of formula structures, showcasing its broad potential.

\section{Methodology for Formula Unification} 
\label{section-methodology}

\subsection{LLMs for Harvesting Mathematical Formulas}
\label{subsection-engineering}

\begin{figure}[th!]
    \centering
    \includegraphics[width=1\linewidth]{Images/engineering_figure.pdf}
    \label{fig:engineering-formula-extraction}
    \vspace{-20pt}
    \caption{\textbf{Pipeline for automated harvesting of mathematical formulas (left), exemplified using one of the analyzed $\pi$ formulas (right)}. (a) Equations are scraped from papers on arXiv. (b) Regular expressions on the $\text{\LaTeX }$ strings retrieve series and continued fractions that contain $\pi$ as the only irrational number (see \cref{appendix-engineering-formula-patterns}). (c) Few-shot classification using OpenAI's GPT-4o mini \cite{openai2024gpt4ocard} identifies formulas calculating the constant $\pi$. Then, OpenAI's GPT-4o identifies the formula type (series, continued fraction, or neither). (d) Extraction of the series' summand or the continued fraction's partial numerator and partial denominator, using GPT-4o. The formula is then converted to code. (e) Formula code is evaluated for validation using the integer relation finder algorithm PSLQ. (f) The formulas are converted to a canonical recurrence using RISC's tool for fitting recurrences \cite{kauers2022guessing} (\cref{section-formula-canonicalization}).}
    \label{fig:engineering-formula-extraction}
    \vspace{-14pt}
\end{figure}

\subsubsection{Retrieval and Extraction}

Our challenge lies in the natural language processing of formulas. We analyze the $\text{\LaTeX }$ source code of 457,145 arXiv articles by combining regular expressions and LLMs. From these files, we extract all mathematical expressions, resulting in 278,242,582 unique strings. Filtering for expressions containing the $\pi$ symbol identifies 121,684 $\pi$-related formulas.
The widespread use of the symbol $\pi$ in scientific literature means that most occurrences are unrelated to calculating the constant itself. To address this, each potential formula is classified as computing $\pi$ or not, using GPT-4o mini (chosen for its cost-effectiveness), reducing the number of candidates to 3389. 
Next, GPT-4o categorizes formulas by type: series, continued fraction, or neither.
Expressions including additional variables or symbols other than indices for summation or recurrence were mostly discarded, as the current algorithm has no reliable way of interpreting them—resulting in 847 formula candidates. Future work could analyze the text surrounding each equation and extract the definitions of additional symbols. To test the entire algorithm for such cases as well, we handled select cases manually.

Each formula, represented as a $\text{\LaTeX }$ string, must then be parsed into a Computer Algebra System (CAS) for further manipulations (in our case SymPy \cite{sympy}). Automatically extracting algebraic forms from $\text{\LaTeX }$ strings is especially complicated due to varied $\text{\LaTeX }$ patterns, which are difficult to systematically convert to executable code using a predefined logic.
LLMs help us overcome these obstacles by processing text contextually and attending to relevant parts of the text, solving the natural language processing task that may have required elaborate rules \cite{radford2019language, brown2020language}. 
Specifically, we use OpenAI's GPT-4o and GPT-4o mini \cite{openai2023gpt4} to translate relevant $\text{\LaTeX }$ into executable mathematical code (see \cref{appendix-engineering} for the exact prompts used).
Since there is no assurance that the LLM-generated formula code is correct, we apply a code-validation loop: code is executed and any errors are sent back to the LLM along with the faulty code to correct it, up to three times.

\subsubsection{Validation of the Formulas}

We validate that each extracted formula computes the constant $\pi$ by running the formula code to get a numerical approximation and then applying PSLQ, an integer relation algorithm \cite{PSLQ}. 
Limit values are not extracted directly from the $\text{\LaTeX }$ string for validation, since we found that GPT-4o got them wrong in some cases (see \cref{tab:appendix-engineering-llm-example-wrong-value}). Instead, the PSLQ approach fixes these critical GPT mistakes and reproduces the intended formulas.

Out of the 847 candidates, 412 were validated as $\pi$ formulas using PSLQ and 254 yielded recurrences of order 2 (\cref{section-formula-canonicalization}), which we analyzed and unified via the coboundary algorithm (\cref{tab:unified-formulas} and \cref{tab:collected-formulas}). The remaining validated recurrences, e.g., of order 3, can also be addressed as we show for special cases in Appendix \ref{appendix-risc-guess-results}.


\subsection{Mathematical Background}
\label{section-math-backround}
\subsubsection{Recurrences and Approximations}
%
This section introduces the key mathematical objects used by our algorithm. 
Each formula of a mathematical constant provides a sequence of rational numbers that converge to that constant (known as a \textit{Diophantine approximation}). Each intermediate element is an approximation of the constant. 
%
Formulas—including infinite sums, products, and continued fractions—can be converted into recurrences, providing a cohesive framework for unification.
%

A function \(u_n\) satisfies a recurrence of order $m$ if
$u_n = a_{1,n} u_{n-1} + a_{2,n} u_{n-2}+\ldots+ a_{m,n}u_{n-m}$,
which can be represented via the associated companion matrix:
\begin{equation} \label{def:companion-form}
\operatorname{CM}(n) \coloneqq
\begin{pmatrix}
0 & 0 & \dots & 0 & a_{m,n} \\
1 & 0 & \dots & 0 & a_{m-1,n} \\
0 & 1 & \dots & 0 & a_{m-2,n} \\
\vdots & \vdots & \ddots & \vdots & \vdots \\
0 & 0 & \dots & 1 & a_{1,n}
\end{pmatrix}
\end{equation} 
By incrementally multiplying the companion matrix over $n$ steps, we get the matrix: 
\begin{equation}\label{eq:step_matrix}
\prod_{i=1}^n\operatorname{CM}(i) =
\begin{pmatrix}
p_{1,n-m}  & \dots &  p_{1,n-1} & p_{1,n} \\
p_{2,n-m} & \dots &  p_{2,n-1} & p_{2,n} \\
p_{3,n-m}  &\dots &  p_{3,n-1} & p_{3,n} \\
\vdots & \ddots & \vdots & \vdots \\
p_{m,n-m} & \dots & p_{m,n-1} & p_{m,n}
\end{pmatrix}
\end{equation}
$p_{1,n},\ldots p_{m,n}$ are solutions to the recurrence for the initial conditions $p_{i,j}=\delta^j_i$. Other solutions for different initial conditions can be written as linear combinations of these.

Recurrences evaluate a desired constant \(L\) either directly $\lim_{n\to \infty} u_n=L$ (e.g., for recurrences representing infinite sums), or as a ratio
$\lim_{n\to \infty} \frac{p_n}{q_n}=L$ with $p_n$ and $q_n$ being solutions for the recurrence (e.g., for continued fractions).

In the special case of a second-order recurrence, $u_n= a_nu_{n-1}+b_nu_{n-2}$, any pair of solutions is associated with a formula in the form of a continued fraction:
\vspace{-2pt}
\begin{equation} \label{eq:cf_def}
 \cfrac{b_1}{a_1 + \cfrac{b_2}{\ddots + \cfrac{b_n}{a_n}}} = \frac{p_n}{q_n}
\vspace{-2pt}
\end{equation}
When the functions $a_n=a(n)$ and $b_n=b(n)$ are polynomials, the formula above is known as a Polynomial Continued Fraction (PCF), denoted by $\text{PCF}\left( a(n),b(n) \right)$.
%

\subsubsection{Metrics on Approximation Formulas}


%
\begin{figure*}[h!]
    \centering
    \vspace{-2pt}
    \includegraphics[width=0.8\linewidth]{Images/coboundary_schematic.pdf}
    \vspace{-12pt}
    \caption{ \textbf{Coboundary equivalence}: the mathematical framework connecting different formulas once cast into their canonical forms. (a) The coboundary condition $A(n) \cdot U(n+1) = U(n) \cdot B(n)$ recasts formulas as (b,c) parallel trajectories in a CMF. 
    (d) Example of two coboundary-equivalent formulas, presenting their coboundary matrices and limits. Coboundary equivalence shows the limits of the formulas are related by a Möbius transformation, such that proving the limit of one formula proves the other.}
    \label{fig:coboundary-schematic}
    \vspace{-10pt}
\end{figure*}
%

%
Since each formula converging to a constant \(L\) provides a converging sequence, the formula can be characterized by \textit{dynamical metrics} capturing properties such as its convergence rate. A recent paper \cite{BlindDelta} proposed to use such metrics for formula discovery and clustering. The metrics of \textit{convergence rate} and \textit{irrationality measure} are used in our algorithm.

The \textit{convergence rate} is defined as:
\vspace{-3pt}
\begin{equation}
    \label{def:convergent_ratio}
    r \coloneqq \lim_{n\to \infty} \frac{1}{n}\log \left| L-x_n \right|
\vspace{-1pt}
\end{equation}
The ratio of \(r\) (when they exist and are non-zero) between formulas hints that one may be a subsequence of the other. To see why, consider two series; one with summand $s(n)$ and the other with summand $s(3n-2) + s(3n-1) + s(3n)$. These series have the same limit, as the latter produces a subseries of the former, but its convergence rate is 3 times higher. Evidently, their summands do not appear directly equivalent. Our algorithm thus applies \textit{folding} (Appendix \ref{appendix-maths-fold}) to equate the \(r\) values of each pair of formulas that are candidates for equivalence.

The \textit{irrationality measure} of \(x_n = \frac{p_n}{q_n}\) is defined as the limit \(\delta = \lim_{n \to \infty} \delta_n\), where
\vspace{-3pt}
\begin{equation}
    \label{def:irrationlity_measure}
    \delta_n = -1-\frac{\log\left|L - \frac{p_n}{q_n}\right| }{\log \left|q_n \right|}
%\vspace{-2pt}
\end{equation}
The \(\delta\) measure does not distinguish between subsequences, it is used in number theory to prove the irrationality of constants. A constant is shown to be irrational if it admits an approximation with a positive irrationality measure. The \(\delta\) measure is invariant under coboundary transformations, making it a useful metric in our algorithm (see Fig.~\ref{fig:coboundary-steps}). Once two formulas share the same \( r \) and \(\delta\), we apply the coboundary algorithm to relate them.
%

\subsubsection{Coboundary Equivalence}


The concept of coboundary is a key part of our algorithm connecting formulas, finding non-trivial relations between formulas that often seem unrelated (see Appendix \ref{appendix-Coboundary-transform} for motivation and context on the coboundary definition).
%Recurrences often define formulas by ratios between two solutions, as in PCFs. In these cases, we focus on the ratios between elements in the incremented product matrix~\eqref{eq:step_matrix}. Consequently, the formula presented by the matrix \(\operatorname{CM}(n)\) and its scalar multiple \(f(n)\cdot \operatorname{CM}(n)\) will be the same. We incorporate this observation by considering these matrices as projective matrices.

Two matrices $A(n),B(n)\in \operatorname{PGL}_m\left(\mathbb{Q}(n)\right)$ are \textit{coboundary equivalent} if there exist a matrix $U(n)$ such that 
\vspace{-1pt}
\begin{equation}
    \label{def:cbdry_rational_matix}
     A(n) \cdot U(n+1)  =   U(n) \cdot B(n)
     \vspace{-1pt}
\end{equation}
We say that two recurrences are coboundary equivalent if their companion matrices \eqref{def:companion-form} are coboundary equivalent.
Consequently, the recurrences' step matrices \eqref{eq:step_matrix} satisfy:
\vspace{-2pt}
\[
\left(\prod_{i=1}^n A(i)\right)\cdot U(n+1)  = U(1)\cdot\left(\prod_{i=1}^n B(i)\right)
\vspace{-1pt}
\]

Since any matrix with rational function coefficients can be scaled to have polynomial coefficients, this equivalence translates to a matrix equation with polynomial entries.
We define a coboundary between matrices $A(n), B(n) \in \operatorname{GL}_m(\mathbb{Q}[n])$ if there exist a matrix $U(n) \in \operatorname{GL}_m(\mathbb{Q}[n])$ and polynomials $p_A(n), p_B(n) \in \mathbb{Q}[n]$ such that
\vspace{-1pt}
\begin{equation}
    \label{def:cbd_equiv_poly}
   p_A(n) \cdot A(n) \cdot U(n+1)  =  p_B(n) \cdot U(n) \cdot B(n) 
\vspace{-1pt}
\end{equation}
We say that two recurrences are \textit{coboundary equivalent} if a coboundary exists between their associated companion matrices (Fig. \ref{fig:coboundary-schematic}(a)). 

%
Finding a coboundary between two matrices is inherently a non-linear problem due to the flexibility in choosing the polynomials \(p_A\) and \(p_B\). 
Despite the non-linearity, we found a coboundary solver algorithm for the case of $m=2$ (see Appendix \ref{appendix-algs-coboundary-algorithm}).
%

\subsubsection{Conservative Matrix Fields (CMFs)}
\label{section-math-background-cmf}


The CMF is the mathematical structure that enables unifying formulas of the same constant. The CMF concept was originally found by generalizing PCFs \cite{doi:10.1073/pnas.2321440121}, and was later realized to be more general (Appendix~\ref{appendix-cmf}). To exemplify the concept, we focus on the CMF that encompasses many of the $\pi$ formulas. This CMF is 3D, i.e., consisting of three matrices \(M_\mathbf{x}, M_\mathbf{y}, M_\mathbf{z}\) with rational function entries in the variables $(x,y,z)$, satisfying:
\vspace{-2pt}
\begin{align*}
    M_\mathbf{x}(x, y, z) M_\mathbf{y}(x+1, y, z) &= M_\mathbf{y}(x, y, z) M_\mathbf{x}(x, y+1, z) \\
    M_\mathbf{x}(x, y, z) M_\mathbf{z}(x+1, y, z) &= M_\mathbf{z}(x, y, z) M_\mathbf{x}(x, y, z+1) \\
    M_\mathbf{y}(x, y, z) M_\mathbf{z}(x, y+1, z) &= M_\mathbf{z}(x, y, z) M_\mathbf{y}(x, y, z+1)
\vspace{-3pt}
\end{align*}
This property describes the path-independence of the transition between two points in a 3D lattice (illustrated in Fig.~~\ref{fig:coboundary-schematic}(b)), 
in analogy to a conservative vector field.

For an explanation of how formulas reside as directions within the CMF, see Appendix \ref{appendix-section-trj-mat}. A notable feature of the CMF is that pairs of formulas found to be parallel trajectories with different initial points correspond to two matrices that are coboundary equivalent (Appendix \ref{appendix-section-trj-mat}).

Many of the known $\pi$ formulas reside within a single CMF:
\vspace{-2pt}
\begin{equation}
    \label{def:the_pi_cmf}
    \begin{aligned}
M_\mathbf{x} &=
\begin{pmatrix}
    1 & y \\
    \frac{1}{x} & \frac{2x + y - 2z + 2}{x}
\end{pmatrix}
\\
M_\mathbf{y} &=
\begin{pmatrix}
    1 & x \\
    \frac{1}{y} & \frac{x + 2y - 2z + 2}{y}
\end{pmatrix}
\\
M_\mathbf{z} &=
\begin{pmatrix}
    \frac{z(-x - y + z)}{(y - z)(x - z)} & \frac{zxy}{(y - z)(x - z)} \\
    \frac{z}{(y - z)(x - z)} & \frac{-z^2}{(y - z)(x - z)}
\end{pmatrix}
\end{aligned}
\end{equation}
For further details on the construction, see Appendix \ref{appendix Hypergeomtric-CMF}.


%
\subsection{Formula Clustering and Unification}

\subsubsection{The Canonical Form}
\label{section-formula-canonicalization}

The first step in unification is converting each formula to a canonical form, i.e., the simplest linear recurrence with polynomial coefficients (\cref{appendix-canonical-form}). This canonical form captures a wide range of formulas, including infinite sums. 
Relying on automated algebraic capabilities for such a task is unpredictable. Thus, we opt to use a computational method for converting the formulas to polynomial recurrences: a Mathematica package by RISC \cite{kauers2022guessing} that fits polynomial-coefficient linear recurrences to each sequence of rational numbers, thus finding the minimal recurrence. 
Each series is converted to a recurrence using its first 200 partial sums. Out of 412 validated formulas (\cref{subsection-engineering}), 254 have representations as order-2 recurrences, which are equivalent to PCFs. 

The conversion to canonical forms automatically unifies multiple different formulas. Our harvesting process yields 93 different canonical forms for $\pi$ from 254 distinct arXiv sources (selected examples in \cref{tab:canonicalization-examples}).

\begin{table*}[th!]
    \vspace{-8pt}
    \caption{\textbf{Formula representations using canonical forms}. Formulas harvested from arXiv are automatically converted to their canonical forms. Certain formulas that look different actually have the same canonical form (1,2), proving they are identical. This representation leaves only less trivial connections for the later steps of the algorithm to find. Initial conditions needed for each PCF to compute every approximant of the original formula using \cref{eq:step_matrix} were found using the method described in \cref{appendix-algs-series-limit-to-cf-limit}.}
    \label{tab:canonicalization-examples}
    \centering
    \vspace{-8pt}
    \begin{center}
    \begin{tabular}{lcccccc}
        \toprule
        & Formula & Value & arXiv Source & Canonical Form (CF) & CF Value & Initial Conditions \\
        \midrule
        & $\sum_{n=0}^{\infty} \frac{n!}{\prod_{i = 1}^{n}(2i+1)}$
        & $\frac{\pi}{2}$
        & \cite{1806.03346}
        & $\text{PCF}(3n+1, n(1-2n))$
        & $\frac{2}{\pi}$
        &
        $\begin{bmatrix}
            0 & 1 \\
            1 & 1
        \end{bmatrix}$ \\
        2
        &
        $\sum_{n=1}^{\infty} \frac{2^n}{n{2n\choose n}}$
        & $\frac{\pi}{2}$
        & \cite{2010.05610}
        & $\text{PCF}(3n+1, n(1-2n))$
        & $\frac{2}{\pi}$
        &
        $\begin{bmatrix}
            0 & 1 \\
            1 & 1
        \end{bmatrix}$ \\
        3 % node 189 in pipcfs
        & $\sum_{n=0}^{\infty} \frac{(-1)^n}{2n+1}$
        & $\frac{\pi}{4}$
        & \cite{2404.15210}
        & $\text{PCF}(2, (2n-1)^2)$
        & $1+\frac{4}{\pi}$
        &
        $\begin{bmatrix}
            0 & 1 \\
            1 & 1
        \end{bmatrix}$ \\
        4 % 
        & $\sum_{n=1}^{\infty} \frac{(-1)^{n+1}}{n(n+1)(2n+1)}$
        & $\pi - 3$
        & \cite{2206.07174} 
        & $\text{PCF}(6, (2n+1)^2)$
        & $\frac{1}{\pi-3}$
        &
        $\begin{bmatrix}
            0 & 1 \\
            1 & 6
        \end{bmatrix}$ \\
        \midrule
        5 % 
        & $\sum_{n=1}^{\infty} \frac{4^n(12n-5)}{(2n-1){4n \choose 2n}}$
        & $\frac{3\pi + 4}{2}$
        & \cite{2204.08275}
        & *
        &
        $\frac{-42\pi - 196}{3\pi+4}$
        &
        $\begin{bmatrix}
            0 & 70 \\
            -1 & 15
        \end{bmatrix}$ \\
    \end{tabular}
    \begin{tabular}{lcccc}
        & & $\text{* PCF}(240 n^3 + 164 n^2 - 54 n - 29, -9216 n^6 + 12288 n^5 + 11264 n^4 - 15520 n^3 - 764 n^2 + 3802 n - 714)$ & & \\
        \bottomrule
    \end{tabular}
    \end{center}
    \vspace{-14pt}
\end{table*}

\subsubsection{Clustering Based on Dynamical Metrics}
\label{section-methodology-matching}

\begin{figure}[th!]
    \centering
    \includegraphics[width=0.8\linewidth]{Images/Coboundary_Steps.pdf}
    \vspace{-8pt}
    \caption{\textbf{The coboundary algorithm: connecting polynomial linear recurrences.} This algorithm is demonstrated here for polynomial continued fractions (PCFs) but can be generalized to any linear polynomial recurrence. (a) Compute the dynamical metrics \cite{BlindDelta} for the two PCFs (irrationality measures $\delta_A$, $\delta_B$ and the convergence rates ratio $r_A/r_B$). The $\delta$ metrics are used to identify possible connections, as only if $\delta_A = \delta_B$, the PCFs can be related via coboundary (in practice, we test for them to be within $0.05$ of each other). (b) ``Fold'' $\mathrm{PCF}_A$ by $r_B$ and $\mathrm{PCF}_B$ by $r_A$ (\cref{appendix-maths-fold}). (c) Solve for a general Möbius transform (a 2x2 matrix $U(1)$) that once applied to the limit of $\mathrm{PCF}_B$ equates it to the limit of $\mathrm{PCF}_A$. (d) Representing the PCFs in matrix form ($A (n)$ and $B (n)$), propagate the coboundary matrix via the relation $U(n+1) = A(n)^{-1} \cdot U(n) \cdot B(n)$ up to $U(N)$ ($N=40$ in our runs). (e) Assume the general form of $U(n)$ to have polynomial entries up to degree $N-1$ and solve for their coefficients using normalized $U(1, \ldots ,N)$. If such a solution is found and validated, the PCFs are coboundary-related. See \cref{appendix-algs-formula-matching} and \cref{appendix-algs-coboundary-algorithm} for more details.
    }
    \label{fig:coboundary-steps}
    \vspace{-8pt}
\end{figure}

Clustering formulas provides a heuristic as to which coboundary matches should be attempted.
Dynamical metrics that can be directly calculated from formulas are effective for clustering, indicating that formulas with the same metrics are likely to be related to the same constant \cite{BlindDelta}. We find that such metrics also indicate a more intricate connection between the formulas, enabling us to unify them in a systematic way that provides an analytical conversion between them.
Specifically, the irrationality measure is the most reliable indicator for whether two formulas can be matched. Formulas in canonical form are thus first compared to each other based on this metric (Fig. \ref{fig:coboundary-steps}a).
For every new formula that we attempt to unify, we first search for directions in the CMF that correspond to recurrences with the same $\delta$. Such a search can be improved by using gradient decent on the direction parameters, because $\delta$ is found to be continuous \cite{doi:10.1073/pnas.2321440121}.

We find that $\delta$ is not sufficient to imply equivalence, and thus complement it using the convergence rates. 
We analyze the convergence rates of the formulas and evaluate $\frac{r_A}{r_B}$. Then, canonical form A is folded by $r_B$ and canonical form B is folded by $r_A$ (Fig. \ref{fig:coboundary-steps}b).

\subsubsection{Matching by Coboundary Equivalence}

Having folded two formulas to make them converge at the same rate, the next step is finding the precise algebraic relation between them. The coboundary framework discovers such transformations using a novel algorithm that we developed.
A general coboundary equivalence (as in \cref{def:cbd_equiv_poly}) reads: $p_A(n) \cdot A(n) \cdot U(n+1) = p_B(n) \cdot U(n) \cdot B(n)$.
Given two order-2 recurrences, there are 6 unknown polynomials: $p_A(n), p_B(n)$ and the 4 entries of $U(n)$. The coboundary equation is nonlinear in the coefficients of the polynomials, making the problem hard to solve. Furthermore, the problem has an undefined number of variables; a degree for each of the polynomials must be chosen before a set of equations can be written.

We present a method for finding these connections without solving nonlinear equations, instead leveraging the recurrence limits to compute a sequence of empirical coboundary matrices, whose elements are fitted to rational functions. The algorithm relies on the observation that if two recurrences $A(n)$ and $B(n)$ are coboundary with matrix $U(n)$ and ``external'' polynomials $p_A(n),p_B(n)$, then their limits $L_A$ and $L_B$ are related by the relation $L_A = U(1)(L_B)$,
where $U(n)(\cdot)$ denotes the Möbius transform defined by $U(n)$ (\cref{def:Möbius}).
In the case of an order-2 recurrence (i.e., PCF), $U(1)$ is four integers solved for by equating coefficients of $\pi$ on both sides of the equation (Fig.~\ref{fig:coboundary-steps}c). Using the necessary condition for a coboundary equivalence, we propagate this empirical matrix further along and calculate $U(n+1) = A(n)^{-1}\cdot U(n) \cdot B(n)$ (Fig.~\ref{fig:coboundary-steps}d). Rational functions are fit to each of the components of $U$, after which $U$ is multiplied by the least common multiple of the denominators of the elements, leading to a polynomial $U(n)$. We can multiply $A(n) \cdot U(n+1)$ and $U(n) \cdot B(n)$ and compare. If these products, which are matrices, are the same up to a factor (a function of $n$ times the identity matrix), then we have rigorously proven a polynomial coboundary relation between the two canonical forms (Fig. \ref{fig:coboundary-steps}e).

Fig. \ref{fig:coboundary-steps} summarizes the flow of matching two canonical form formulas. Using this method, we find that formulas 1,2 and 5 from \cref{tab:canonicalization-examples} are equivalent and that formulas 3,4 are also equivalent. Refer to \cref{appendix-results-cmf-examples} for descriptions of how the algorithm is applied to these formulas, and refer to \cref{appendix-algs} for a listing of the algorithms involved.
The same procedure is applied to each canonical form formula, measuring its $\delta$ value and relying on its continuity as a function of direction in the CMF to locate worthy directions that yield potential formula pairs for the coboundary algorithm.
The matching algorithm is then applied between formulas and representative recurrences from the CMF. Finding a match between a formula and a CMF representative proves the formula is generated by the CMF. The full list of results is provided in \cref{tab:unified-formulas} and \cref{tab:collected-formulas}.
Selected results for $\pi$ are detailed in the following section.


\begin{figure*}[ht!]
    \centering
    \includegraphics[width=1\linewidth]{Images/cmf_unification.pdf}
    \vspace{-20pt}
    \caption{\textbf{Formula Unification by a Conservative Matrix Field (CMF)}. Numerous $\pi$ formulas harvested from the literature are automatically arranged as trajectories in a 3D CMF. These formulas include famous ones by Gauss, Euler,
    and Lord Brouncker. The full list of unified formulas and their canonical forms is available in \cref{tab:unified-formulas}. 
    Part of a single octant of the infinite 3D CMF is displayed. 
    Each cluster (large dashed circles) denotes formulas connected by coboundary, representing parallel trajectories or overlapping trajectories.
    The number at each cluster center is the $\delta$ of all formulas in this cluster. Initial conditions for example trajectories in the CMF appear in dark blue. Arrows show trajectory directions (clarified by light blue dots). Note that multiple formula clusters can have the same $\delta$ value without being coboundary, showing that sharing $\delta$ is not a sufficient condition for formulas being coboundary-related.}
    \label{fig:cmf-unification}
    \vspace{-10pt}
\end{figure*}

\section{Results}
\label{section-results}



\subsection{Equivalences Among Famous Formulas}

Our automated system proves previously unknown equivalences between formulas. Among the formulas connected are famous examples, such as 
one of Ramanujan's 1914 formulas, 
as well as Lord Brouncker's, Euler's, and Gauss's PCFs from the 17\textsuperscript{th}, 18\textsuperscript{th}, and 19\textsuperscript{th} centuries, respectively \cite{Ramanujan1914,Osler15012010, euler1748introductio, Gauss1813}.

The following series by Ramanujan \cite{Ramanujan1914}
\vspace{-1pt}
\begin{equation}
    \label{eq:ramanujan-1914}
    \begin{split}
        \frac{4}{\pi} =& \sum_{k=0}^\infty (-1)^k \frac{\left(\frac{1}{2}\right)_k \left(\frac{1}{4}\right)_k \left(\frac{3}{4}\right)_k}{(1)_k^3}\cdot \\ & \cdot (1123 + 21460k) \cdot \left(\frac{1}{882}\right)^{2k+1}
    \end{split}
\end{equation}
was proven by our algorithm to be equivalent to a newer series from a paper published in 2020 \cite{sun2020newseriespowerspi}:
\begin{equation}
    \label{eq:sun-series}
    \begin{split}
    &\frac{341446000}{\pi} = \\ &\sum_{k=0}^\infty \frac{ \binom{2k}{k}^2 \binom{4k}{2k}}{(k+1)(2k-1)(4k-1)(-2^{10} 21^4)^k} \cdot \\ &\cdot \left(1424799848k^2 + 1533506502k + 1086885699\right)
    \end{split}
\end{equation}
The equivalence of these two formulas (detailed in \cref{appendix-results-equivalence-examples}) demonstrates how two previously distinct mathematical expressions, discovered over a century apart, are now proven to be equivalent through a straightforward matrix multiplication, identified by an automated process.

Fig.~\ref{fig:coboundary-schematic}d proves the equivalence of another pair of famous formulas: (1) $\text{PCF}(2n+3, n(n+2))$, discovered in 2021 as one of the first $\pi$ formulas discovered by a computer \cite{Raayoni2021}. (2) $\text{PCF}(2n+1, n^2)$, discovered by Gauss in 1813 \cite{Gauss1813} and provided at the time an especially efficient way to calculate digits of $\pi$. These formulas are also unified in the CMF (Fig.~\ref{fig:cmf-unification}) in the cluster of trajectory $(1,1,2)$ and $\delta=-0.21$ ($\text{PCF}(2n+3, n(n+2))$ is not marked).
Fig.~\ref{fig:cmf-unification} highlights additional well-known formulas that were proved equivalent.

\subsection{Formulas Unified by a Conservative Matrix Field (CMF)}

Out of 93 canonical forms (corresponding to 254 validated formulas), 47 (95 validated formulas) were identified as trajectories within a single CMF. 
The trajectory-arrangement of these formulas is summarized in Fig.~\ref{fig:cmf-unification}, showing how the CMF unifies them, with different trajectory directions describing formulas of different dynamical metrics. 

Each cluster contains formulas equivalent via coboundary and fold operations, corresponding to parallel or overlapping trajectories in the CMF.
Trajectory $(1,0,0)$ with $\delta = -0.65$ describes many formulas discovered algorithmically in \cite{Raayoni2021}. 
Trajectory $(2,1,1)$ with $\delta = -0.48$ holds three infinite sums, all of which have factors $\frac{(-2)^k}{{4k \choose 2k}}$ in their terms. Trajectory $(1,1,1)$ with $\delta = -1.00$ describes, among others, famous slowly converging formulas by Lord Brouncker and Euler. 
The full list of the canonical forms captured by the CMF appears in \cref{tab:unified-formulas}. 
We expect additional formulas (\cref{tab:collected-formulas}) to be found in the CMF once the algorithms for finding trajectory candidates for matching are improved.



\subsection{Unification of Formulas Beyond $\pi$}
\label{subsection-results-other-constants}

Our method is applicable to mathematical constants other than $\pi$. Consider these two formulas for Ap\'{e}ry's constant, the Riemann zeta function value $\zeta(3)$:
\vspace{-4pt}
\begin{align}
    \label{eq:zeta3-def}
    \zeta(3) &= \sum_{n=1}^{\infty} \frac{1}{n^3} \\
    \label{eq:zeta3-fast}
    \frac{5}{4} - \zeta(3) &= \sum_{n=2}^{\infty} \frac{1}{n^3(n^2 - 1)} 
\end{align}
The second formula (\cref{eq:zeta3-fast}) \cite{kummer1837neue} has faster convergence rate compared to the classical definition of $\zeta(3)$ (\cref{eq:zeta3-def}). 
Our automatic procedure proves their equivalence by the coboundary transform (see \cref{appendix-results-equivalence-examples} for a complete exposition).

As another example, the following two PCFs for Catalan's constant \cite{catalan_representations} are also coboundary equivalent (see \cref{appendix-results-equivalence-examples} for the coboundary matrix).
\vspace{-2pt}
\begin{equation}
\label{eq:catalan1}
    \frac{1}{2 - 2G} = 3 + \cfrac{2^2}{1 + \cfrac{2^2}{3 + \cfrac{4^2}{1 + \cfrac{4^2}{\cdots}}}}
\end{equation}
\vspace{-2pt}
and
\begin{equation}
\label{eq:catalan2}
    \frac{1}{2G - 1} = \frac{1}{2} + \cfrac{1^2}{\frac{1}{2} + \cfrac{1\cdot2}{\frac{1}{2} + \cfrac{2^2}{\frac{1}{2} + \cfrac{2\cdot3}{\cdots}}}}
\end{equation}
\cref{appendix-results-equivalence-examples} also shows equivalence between $e$ formulas.

\section{Outlook}

Many of the examples analyzed in this work are second-order recurrences and series or continued fractions. However, the coboundary algorithm and all the other steps in our harvesting and clustering processes are applicable to any formula that generates a sequence of approximations for a given constant, including those involving deeper recurrences. Expanding the system to accommodate these cases is a promising direction for future work.

Currently, the harvesting step relies on the LLM's ability to interpret and contextualize mathematical $\text{\LaTeX }$ strings. 
This step introduces data loss and false negatives in formula classification.
Improvements in prompt engineering and in validation techniques will enhance the robustness of this LLM use.
As more advanced LLMs become available, this step will become increasingly reliable.

Going beyond $\pi$, we automatically identified equivalent formulas for $e$, $\zeta(3)$, and Catalan’s constant—demonstrating the generality of the approach. A natural next step is to conduct exhaustive searches for other well-known constants, as well as fundamental mathematical structures in fields such as physics and computer science. Successfully unifying formulas across disciplines could reveal scientific insights that would be difficult to uncover without large-scale automated analysis.

Looking forward, the same approach of collection, analysis, and organization of mathematical knowledge could help establish rigorous connections between different branches of mathematics.
From an even broader perspective, the methodology presented in this work could help develop more general frameworks for identifying connections between different scientific theories through their mathematical representations.
As the volume of information grows at an accelerating pace, finding automated ways to unify knowledge will become increasingly essential, especially with the goal of providing more intuitive understanding on complex concepts.

% Removed for the double-blind submission
\section*{Acknowledgments}

This research received support through Schmidt Sciences, LLC.

\newpage
\section*{Impact Statement}

This paper presents work whose goal is to advance the field
of AI for Mathematics. There are many potential societal
consequences of our work, none of which we feel must be
specifically highlighted here.


\bibliography{library,arxiv_library}
\bibliographystyle{icml2025}




\newpage

\appendix
\onecolumn

\section{Examples of Special Results}

\subsection{Examples for Unification by the Conservative Matrix Field (CMF)}
\label{appendix-results-cmf-examples}

The examples shown here were found completely automatically via the algorithms discussed in \cref{section-methodology}. The actions at each step are described as if being applied by a human for clarity, but we emphasize that the connections were made automatically (i.e., patterns of the following paragraphs could have been written a single time and details of specific proofs could be filled in by a computer as it runs the algorithms).

\subsubsection{Unification Example: Formulas 1, 2 and 5 from \cref{tab:canonicalization-examples}}

Formulas 1 and 2 have the same canonical form. Are there additional connections here? Could it be that formula 5 is also related to formulas 1 and 2, despite looking very different? Using the methodology showcased in Fig.~\ref{fig:coboundary-steps}, we attempt to answer this question. First, we compute the irrationality measure, the $\delta$ parameter, for each of the formulas, resulting in $-0.65$ for both canonical forms. The first empirically necessary condition for a coboundary equivalence has been passed. Next, we compute convergence rates (see \cref{appendix-algs-formula-matching} for the definition), resulting in $0.69$ and $1.38$ for formulas 1 and 5, respectively. These are suspiciously similar - dividing the two we get $\frac{r_1}{r_2} = \frac{1}{2}$. Folding (\cref{appendix-maths-fold}) canonical form 1 by 2 and canonical form 5 by 1 (meaning no change) results in
$$
\text{PCF}_{1,\text{folded}}
=\text{PCF}\left(60 n^{3} + 34 n^{2} - 11 n - 3, 2 n \left(- 288 n^{5} + 624 n^{4} - 230 n^{3} - 225 n^{2} + 158 n - 24\right)\right)
$$
Applying the coboundary finding algorithm (\cref{appendix-algs-coboundary-algorithm}), the following matrix and polynomials connecting $\text{PCF}_{1,\text{folded}}$ and $\text{PCF}_5$ are found:
$$
U(n) = 
\begin{pmatrix}
    48 n^{3} - 85 n^{2} + 28 n
    & 2304 n^{6} - 9792 n^{5} + 15440 n^{4} - 11100 n^{3} + 3586 n^{2} - 408 n
    \\
    -1
    & - 48 n^{3} + 200 n^{2} - 223 n + 51
\end{pmatrix}
$$
\begin{align*}
p_A(n) &=
12n - 7 \\
p_B(n) &= 
3n + 2
\end{align*}
This means that \cref{def:cbd_equiv_poly} should hold
$$
p_A(n)\cdot \text{CM}_{1,\text{folded}}
\cdot U(n+1) = p_B(n)\cdot U(n) \cdot \text{CM}_{5}
$$
(recall $\text{CM}$ is short for ``companion matrix", the matrix representing the recurrence of a PCF, see \cref{def:companion-form}). Multiplying out the terms, we indeed get the expected relation:
\begin{align*}
& (12n-7) \cdot 
\begin{pmatrix}
    0
    & 2 n \left(- 288 n^{5} + 624 n^{4} - 230 n^{3} - 225 n^{2} + 158 n - 24\right)
    \\
    1
    & 60 n^{3} + 34 n^{2} - 11 n - 3
\end{pmatrix}
\cdot \\
&
\begin{pmatrix}
48 n^{3} + 59 n^{2} + 2 n - 9
& 2304 n^{6} + 4032 n^{5} + 1040 n^{4} - 1180 n^{3} - 434 n^{2} + 88 n + 30 \\
-1
& - 48 n^{3} + 56 n^{2} + 33 n - 20
\end{pmatrix}
\end{align*}
$$=$$
\begin{align*}
& (3n + 2) \cdot
\begin{pmatrix}
48 n^{3} - 85 n^{2} + 28 n
& 2304 n^{6} - 9792 n^{5} + 15440 n^{4} - 11100 n^{3} + 3586 n^{2} - 408 n
\\
-1
& - 48 n^{3} + 200 n^{2} - 223 n + 51
\end{pmatrix}
\cdot \\
&
\begin{pmatrix}
    0
    & - 9216 n^{6} + 12288 n^{5} + 11264 n^{4} - 15520 n^{3} - 764 n^{2} + 3802 n - 714
    \\
    1
    & 240 n^{3} + 164 n^{2} - 54 n - 29
\end{pmatrix}
\end{align*}

We have therefore found a transformation from one canonical form to the other; Formulas 1,2 and 5 are equivalent. Having a proof of one formula out of these three immediately proves the other two.

Are these formulas contained in the CMF? Searching through the list of recurrences generated by the CMF (see \cref{append-algs-cmf-trajectories}), we find that the recurrence of trajectory $(1,0,0)$ in the matrix field also has $\delta = -0.65$. The recurrence generated by this direction is precisely the recurrence $\text{PCF}_1$, meaning a trivial coboundary equivalence exists between formulas 1,2 and the matrix field, concluding the search. Note that, in general, a nontrivial coboundary equivalence - meaning one with non-identity coboundary matrix - must also be found between a representative canonical form of the equivalent formulas, and the canonical form of the recurrence generated by the CMF. See \cref{appendix-results-cmf-examples-nontrivial-coboundary} for an example.

\subsubsection{Unification Example: Formulas 3 and 4 from \cref{tab:canonicalization-examples}}

A shortened version of the last proof, but applied to the case of formulas 3 and 4 goes like this:

The canonical forms of formulas 3 and 4 have $\delta = -1$ and convergence rates $0$ (they converge polynomially - very slowly). Given the similarity in $\delta$, we conjecture that the formulas are coboundary. Applying the coboundary-solving algorithm (\cref{appendix-algs-coboundary-algorithm}) yields the coboundary matrix:
$$
U(n) =
\begin{pmatrix}
4 n^{2} - 4 n + 1
& 8 n^{3} + 4 n^{2} - 10 n + 3 \\
2 n + 1
& 4 n^{2} + 8 n + 7
\end{pmatrix}
$$
and trivial external polynomials - $p_A(n)=1,p_B(n)=1$.
So formulas 3 and 4 are equivalent.

A trajectory in the CMF that generates a recurrence with similar $\delta$ is $(1,1,1)$. The recurrence generated is precisely the canonical form of formula 3, $\text{PCF}_3$, concluding the search.

The next example requires a nontrivial coboundary equivalence between a recurrence of the CMF and the formula of interest.


This example is implemented and available in the \href{https://colab.research.google.com/drive/13EC9hwEhoA_xvEu_7p_9wbIl2QjDknqC?authuser=1#scrollTo=i-3w8tyB2_-P}{results colab} as well as all other formulas we have presented.

\subsubsection{Unification Example: Cluster $(-1,3,3)$ ($\delta = -0.91$)}
\label{appendix-results-cmf-examples-nontrivial-coboundary}

The following example pertains to the $\delta=-0.91$ cluster in \cref{tab:unified-formulas}. Formula 33 from this cluster is shown in the table for the purpose of this example. We have $\text{PCF}_{33}$, taken from the CMF:

\begin{equation*}
    \begin{split}
    \text{PCF}_{33} =& \text{PCF}(- 7568 n^{5} - 11664 n^{4} + 6992 n^{3} + 6036 n^{2} - 279 n - 162, \\
    &- 24 n (2 n + 1) (4 n - 3) (4 n - 1) (6 n - 7) (6 n - 5) (22 n^{2} - 39 n - 1) (22 n^{2} + 49 n + 9))
    \end{split}
\end{equation*}
and
\begin{equation*}
    \begin{split}
    \text{PCF}_{34} =& \text{PCF}(3784 n^{4} + 156 n^{3} - 1942 n^{2} + 261 n + 45, \\
    &- 24 n (2 n - 3) (4 n - 3) (4 n - 1) (6 n - 5) (6 n - 1) (11 n - 14) (11 n + 8))
    \end{split}
\end{equation*}

which corresponds to the series

$$2\pi = \sum_{k=1}^{\infty} \frac{16^{k} (22 k^{2} - 17 k + 3) {\binom{4 k}{2 k}}}{k (4 k - 3) (4 k - 1) {\binom{3 k}{k}} {\binom{6 k}{3 k}}}$$

Computing convergence rates, we find both of these have $0.52$, so they are not folded. The canonical forms are then fed into the coboundary algorithm, resulting in
$$
U(n) =
\begin{pmatrix}
    U_{11}(n) & U_{12}(n) \\
    U_{21}(n) & U_{22}(n)
\end{pmatrix}
$$
\begin{align*}
    p_A(n) &=
    - 22 n^{2} + 61 n - 42
    \\
    p_B(n) &=
    44 n^{3} + 120 n^{2} + 67 n + 9
\end{align*}
where
\begin{align*}
    U_{11}(n) &= 
    1848 n^{5} - 7676 n^{4} + 10730 n^{3} - 5605 n^{2} + 682 n + 21
    \\
    U_{12}(n) &=
    6690816 n^{9} - 50485248 n^{8} + 157736064 n^{7} \\ 
    &- 262980096 n^{6} + 250327296 n^{5} - 133790880 n^{4} \\
    &+ 35803608 n^{3} - 3191832 n^{2} - 105840 n
    \\
    U_{21}(n) &= 1
    \\
    U_{22}(n) &=
    2860 n^{4} - 10680 n^{3} + 13481 n^{2} - 6348 n + 756
    \\
\end{align*}
showing that the series is contained within the CMF.

\subsubsection{Unification Example: Cluster $(0,0,1)$ ($\delta = -1.00$)}
\label{appendix-results-cmf-examples-shown-in-figure-cmf-unification}

Here we show the four formulas listed explicitly in Fig. ~\ref{fig:cmf-unification} are all equivalent, these correspond to indices 44-47 of \cref{tab:unified-formulas} (formula 48 is omitted for brevity but can be shown to be equivalent to the others in the same way - it is initially in the cluster of formulas 46,47, see below). Please refer to the top of \cref{appendix-results-cmf-examples} for more verbose explanations of the algorithm steps. 
All formulas have $\delta=-1.00$ of course, so they proceed to the convergence rate matching stage.

First, consider the two polynomial continued fractions (formulas 44 and 45):
\begin{align*}
    \text{PCF}_{44} &= \text{PCF}(1, n (n + 1)) \\
    \text{PCF}_{45} &= \text{PCF}(2, n^2)
\end{align*}

These have convergence rates $0$, so all combinations of folding up to 2 are passed to the next stage (\cref{appendix-algs-formula-matching}). Applying the coboundary algorithm (\cref{appendix-algs-coboundary-algorithm}), The two turn out to be coboundary to each other with no folds necessary:
$$U(n) = 
\begin{pmatrix}
    n & -n^2 \\
    -1 & n-1
\end{pmatrix}
$$
with trivial ``external" polynomials $p_A(n),p_B(n)=1$.

Next, consider formulas 46 and 47:
Applying the same steps as above shows they have convergence rates of $0$. Passing all four combinations of folding to the coboundary algorithm, we obtain a coboundary matrix relating the two formulas with no folds necessary:

$$
U(n) = 
\begin{pmatrix}
    4n^2 - 4n + 1 & 16n^2 - 16n^3 + 4n^2 \\
    -1 & -4n^2-4
\end{pmatrix}
$$
these too with trivial ``external" polynomials.

At this point there are two clusters. Can they be united?
Consider the pair 44, 46. Passing all four combinations for folds to the coboundary algorithm, a coboundary matrix comes up:
$$
U(n) = 
\begin{pmatrix}
4 n^{2} - 4 n + 2
& 16 n^{4} - 16 n^{3} + 4 n^{2}
\\
-1
& 1 - 4 n^{2}
\end{pmatrix}
$$
with external polynomials
$$p_A(n) = 2n-1$$
$$p_B(n) = 2n+1$$
In conclusion, we have found that formulas 44-47 are equivalent to each other. Only one need be proven to prove all of the others.


\subsection{Additional Equivalences Proven Automatically}
\label{appendix-results-equivalence-examples}

\subsubsection{Equivalence Example: Notable Formulas for $\pi$}

\cref{eq:ramanujan-1914} \cite{Ramanujan1914} and \cref{eq:sun-series} \cite{sun2020newseriespowerspi} are converted to recurrences, both of order 2, after which they are converted to canonical form PCFs, respectively:
\begin{align*}
\frac{239018472}{-3528 + 1123 \pi}
&=
\text{PCF}_\text{Ramanujan}
=
\text{PCF}(a_1(n), b_1(n))
\\
\frac{1047212167162854000}{-341446000 + 108685699 \pi}
&=
\text{PCF}_\text{Sun}
=
\text{PCF}(a_2(n), b_2(n))
\end{align*}
\begin{align*}
a_1(n)
&=
534215282560n^{4} + 1630601631968n^{3} + 1686512782328n^{2} + 618081838666n + 27955409115
\\
b_1(n)
&=
n^{3}(366856790423961600n^{5} + 588680355780034560n^{4} - 56045383774765056n^{3}
\\
&- 487988770034755584n^{2} - 247923828204062976n - 34298642100691584)
\\
a_2(n)
&=
534215282560n^{4} + 1630601631968n^{3} + 1686512782328n^{2} + 618081838666n + 27955409115
\\
b_2(n)
&=
n^{3}(366856790423961600n^{5} + 588680355780034560n^{4} - 56045383774765056n^{3} \\ &- 487988770034755584n^{2} - 247923828204062976n - 34298642100691584)
\end{align*}
The coboundary matching algorithm is then applied. The canonical forms share $\delta=-0.29$ so the pair proceeds to the next stage of matching. Since the formulas share a convergence coefficient of $13.56$, the recurrences are not folded and a coboundary matrix is attempted to be fit between the canonical forms, resulting in a hypothesis coboundary matrix of degree 10, coupled with external polynomials of degree 4.

$$
U(n) =
\begin{pmatrix}
    U_{11}(n) & U_{12}(n) \\
    U_{21}(n) & U_{22}(n)
\end{pmatrix}
$$
\begin{align*}
    p_A(n) &=
    11398398784 n^{4} - 19077544640 n^{3} + 9321191372 n^{2} - 1315967464 n - 20955
    \\
    p_B(n) &=
    171680 n^{3} + 395264 n^{2} + 290210 n + 67749
\end{align*}
where
\begin{align*}
    U_{11}(n) &= 
    28876576000 n^{5} - 61950059840 n^{4} + 1926362087953808 n^{3} - 1678583497631500 n^{2} - 139251745359750 n
    \\
    U_{12}(n) &=
    1024204559309528510398464000 n^{10} - 2119123722024588790327541760 n^{9}
    \\
    &+ 1056569453502166636426985472 n^{8} + 244974995622211634412208128 n^{7}
    \\
    &- 205564834935781598084742144 n^{6} - 7035268079364204755916288 n^{5}
    \\
    &+ 8470527814505833597769472 n^{4} + 134868258407972960640 n^{3}
    \\
    U_{21}(n) &= 42050 n - 29337
    \\
    U_{22}(n) &=
    1491444197503390771200 n^{6} - 926743682638889031168 n^{5} - 1329170087838044354112 n^{4}
    \\
    &+ 980655193799148492576 n^{3} - 117379649957600136708 n^{2} - 9013576532170267008 n - 143483055820335
\end{align*}

Multiplying out the terms of the coboundary condition (\cref{def:cbd_equiv_poly}) renders \cref{eq:ramanujan-1914} and \cref{eq:sun-series} equivalent.

\subsubsection{Equivalence Example: Formulas for $\zeta(3)$}
Converting the two formulas for $\zeta(3)$, \cref{eq:zeta3-fast} and \cref{eq:zeta3-def}, to canonical forms, respectively yields
\begin{align*}
    \frac{2}{5 - 4\zeta(3)} &= 12 + \cfrac{-48}{40 + \cfrac{-648}{98 + \cfrac{-3840}{\ddots + \cfrac{-n(n + 1)^{4}(n + 2)}{2n^{3} + 9n^{2} + 17n + 12 + \ddots}}}}
    \\
    \frac{\zeta(3)}{\zeta(3) - 1} &= 9 + \cfrac{-64}{35 + \cfrac{-729}{91 + \cfrac{-4096}{\ddots + \cfrac{-(n + 1)^{6}}{2n^{3} + 9n^{2} + 15n + 9 + \ddots}}}}
\end{align*}
Applying our methods, we find a coboundary matrix of degree 6 with linear external polynomials,
\begin{equation*}
U(n) =
\begin{pmatrix}
n^3 + n^2 + n + 1 &
n^6 + 5n^5 + 10n^4 + 10n^3 + 5n^2 + n \\
-1 &
-n^3 - 4n^2 - 5n
\end{pmatrix}
\end{equation*}
$$
p_A(n) = n
$$
$$
p_B(n) = n+1
$$
which are together a certificate of equivalence for the canonical forms, and hence for the original formulas too.

\subsubsection{Equivalence Example: Formulas for Catalan's constant G}
\cref{eq:catalan1} and \cref{eq:catalan2} are equivalent via coboundary matrix 
$$
U(n) = 
\begin{pmatrix}
4n^2 + 2n &
16n^4 \\
-1 &
-4n^2 + 2n -1
\end{pmatrix}
$$
and trivial external polynomials - $p_A(n)=1$ and $p_B(n)=1$. Note that \cref{eq:catalan1} and \cref{eq:catalan2} are not polynomial continued fractions in their current form due to a periodicity of 2 in the $a_n$, $b_n$ functions. To convert them into polynomial form, they are first inflated to make them integer continued fractions (only \cref{eq:catalan2}), then folded by 2 to make them polynomial, resulting in the canonical forms:
\begin{align*}
    \frac{1}{2 - 2G} &= 7 + \cfrac{-16}{23 + \cfrac{-256}{55 + \cfrac{-1296}{\ddots + \cfrac{-16n^{4}}{8n^{2} + 8n + 7 + \ddots}}}}
    \\
    \frac{2}{2G-1} &= 5 + \cfrac{-32}{25 + \cfrac{-384}{61 + \cfrac{-1728}{\ddots + \cfrac{16n^{3}(-n - 1)}{8n^{2} + 12n + 5 + \ddots}}}}
\end{align*}
which are indeed related by the coboundary matrix above.

\subsubsection{Equivalence Example: Formulas for $e$}
Consider these two polynomial continued fractions for $e$ \cite{Raayoni2021}:

\begin{align*}
    \frac{6e}{2e-3} &= 7 + \cfrac{-4}{14 + \cfrac{-20}{23 + \cfrac{-54}{\ddots + \cfrac{n^{2}(-n - 3)}{n(n + 6) + 7 + \ddots}}}} \\
    \frac{4e}{2e-1} &= 3 + \cfrac{-3}{7 + \cfrac{-16}{13 + \cfrac{-45}{\ddots + \cfrac{n^{2}(-n - 2)}{n(n + 3) + 3 + \ddots}}}}
\end{align*}

Applying our method, we find coboundary matrix
\[
U(n) =
\begin{pmatrix}
n^3 + 4n^2 + 6n + 6 & n^4 + 4n^3 + 4n^2 \\
-n - 1 & -n^2 - n + 2
\end{pmatrix}
\]
and external polynomials
$$
p_A(n) = n+2
$$
$$
p_B(n) = n+3
$$
proving that the two formulas are equivalent.

\subsection{Polynomial Recurrences for Formulas, Some of Order Greater than 2}
\label{appendix-risc-guess-results}

One of Ramanujan's 1914 formulas (shown in Fig. \ref{fig:engineering-formula-extraction}) is represented by the following order-2 polynomial recurrence using RISC's Mathematica package for finding minimal recurrences \cite{kauers2022guessing}

\begin{equation*}
\begin{aligned}
0 = &\left(-\frac{18426177}{3162112} 
- n \cdot \frac{1603904319}{63242240} 
- n^2 \cdot \frac{185504787}{3952640} 
- n^3 \cdot \frac{605532897}{12648448} \right. \\
&\left. - n^4 \cdot \frac{22985937}{790528} 
- n^5 \cdot \frac{83133297}{7905280} 
- n^6 \cdot \frac{2072547}{988160} 
- n^7 \cdot \frac{729}{4096} \right) f(n) \\
+ &\left(-\frac{569520571}{15810560} 
- n \cdot \frac{1927156365}{12648448} 
- n^2 \cdot \frac{1076882413}{3952640} 
- n^3 \cdot \frac{3379580191}{12648448} \right. \\
&\left. - n^4 \cdot \frac{122831663}{790528} 
- n^5 \cdot \frac{424008847}{7905280} 
- n^6 \cdot \frac{10066461}{988160} 
- n^7 \cdot \frac{3367}{4096} \right) f(1 + n) \\
+ &\left(\frac{40384}{965} 
+ n \cdot \frac{171504}{965} 
+ n^2 \cdot \frac{61640}{193} 
+ n^3 \cdot \frac{60808}{193} \right. \\
&\left. + n^4 \cdot \frac{35600}{193} 
+ n^5 \cdot \frac{61907}{965} 
+ n^6 \cdot \frac{23709}{1930} 
+ n^7 \right) f(2 + n).
\end{aligned}
\end{equation*}

Some formulas are generated by recurrences of higher order. The methods presented in this work can be generalized to higher degree recurrences. For example, these two series for Catalan's constant \cite{catalan_representations}

\[
G = \sum_{n=0}^{\infty} \frac{1}{2^{n+1}} \sum_{k=0}^{n} \binom{n}{k} \frac{(-1)^k}{(2k+1)^2}
\]
and
\[
2G = \sum_{n=0}^{\infty} \frac{2^n}{(2n+1) \binom{2n}{n}} \sum_{k=0}^{n} \frac{1}{2k+1},
\]

are given by the same recurrence of order 3:

\[
\begin{aligned}
0 = &\left(-\frac{3}{2} - \frac{5n}{4} - \frac{n^2}{4}\right) f(n) \\
&+ \left(\frac{21}{2} + \frac{29n}{4} + \frac{5n^2}{4}\right) f(1 + n) \\
&+ \left(-\frac{85}{4} - 13n - 2n^2\right) f(2 + n) \\
&+ \left(\frac{49}{4} + 7n + n^2\right) f(3 + n)
\end{aligned}
\]

meaning their recurrence matrices, which are in general companion matrices (\cref{def:companion-form}), are trivially coboundary, with the identity coboundary matrix. Other cases in which the recurrence is not precisely the same require a generalization of the coboundary algorithm \cref{appendix-algs-coboundary-algorithm} to solve for the coboundary matrix. An example is listed below. We hope the two recurrences will be found to be coboundary-equivalent in the future.

The series
\[
\frac{2}{\pi} = \sum_{k = 0}^\infty \sum_{i = 0}^k \binom{2k - 2i}{k - i}^2 \binom{2i}{i}^2 \cdot k \left(\frac{1}{32} \right)^k
\]
is given by the order-3 recurrence
\[
0=\left(-4 - 8n - 6n^2 - 2n^3 - \frac{n^4}{4} \right) f(n) + \left(\frac{81}{4} + \frac{173n}{4} + \frac{65n^2}{2} + \frac{21n^3}{2} + \frac{5n^4}{4} \right) f(n+1)
\]
\[
+ \left(-\frac{137}{4} - \frac{297n}{4} - \frac{111n^2}{2} - \frac{35n^3}{2} - 2n^4 \right) f(n+2) + \left(18 + 39n + 29n^2 + 9n^3 + n^4 \right) f(n+3)
\]

The series 
\[
\frac{2}{\pi} =\sum_{n = 0}^{\infty} (-1)^n \frac{(3 n + 1)}{32^n} 
\sum_{k = 0}^{n} \binom{2 n - 2 k}{n - k} \binom{2 k}{k} \binom{n}{k}^2
\]
is given by the order-3 recurrence
\[
0=\left(-\frac{35}{9} - \frac{26n}{3} - \frac{23n^2}{3} - \frac{121n^3}{36} - \frac{35n^4}{48} - \frac{n^5}{16} \right) f(n) 
\]
\[
+ \left(-\frac{365}{9} - \frac{181n}{2} - \frac{1879n^2}{24} - \frac{1589n^3}{48} - \frac{55n^4}{8} - \frac{9n^5}{16} \right) f(n+1) 
\]
\[
+ \left(-\frac{356}{9} - \frac{503n}{6} - \frac{1633n^2}{24} - \frac{1279n^3}{48} - \frac{81n^4}{16} - \frac{3n^5}{8} \right) f(n+2) 
\]
\[
+ \left(84 + 183n + 154n^2 + \frac{568n^3}{9} + \frac{38n^4}{3} + n^5 \right) f(n+3)
\]


\section{Algorithms}
\label{appendix-algs}
This appendix contains an in depth description of the algorithms discussed in \cref{section-methodology}. The algorithms are ordered top-down, from the highest level algorithm to the lowest. 

\subsection{The Coboundary Graph Growing Algorithm}
\label{appendix-algs-graph-growing}

Input: Initialized graph with no edges, where each node is a canonical-form recurrence for a formula. Each node has precomputed attributes: irrationality measure ($\delta$) and convergence rate (\cref{section-math-backround}). Each node has a `source' attribute with possible values: `cmf' or `formula', with some nodes being recurrences generated by the CMF (see \cref{append-algs-cmf-trajectories}). Each node has attribute `was\_hub' initialized to False.

Output: The graph as a forest, where each edge contains the rigorous transformation between the two nodes it connects and every CMF-generated recurrence is a root in this forest (but not all roots are CMF-generated). This forest is termed a \textit{coboundary graph}. Every tree groups formulas that are rigorously-equivalent together. The trees found by the algorithm are actually subgraphs of cliques; not all clique edges are computed during the matching phase to make the algorithm more efficient, but they all exist.

Steps:

1. Group nodes according to $\delta$: Initialize an empty list for each value between $-1.00$ and $0$ at intervals of $0.05$ including the edge values. For every group value $\delta_G$, insert every node that has $|\delta - \delta_G| < 0.05$ into the group's corresponding list. Note that a node will appear in two groups. This is intentional to prevent missing matches.

2. For every group of nodes indexed by $\delta_G$:
\begin{itemize}
    \item BFS variant on the nodes that are not from CMFs: Initialize a list containing all non-CMF nodes from the group $L_{\delta_G}$ and initialize an empty list $L$. A random ``hub" formula that is not CMF-generated is chosen from $L_{\delta_G}$. This hub is attempted to be matched to all the other formulas using \cref{appendix-algs-formula-matching}, with successful matches being stored as new edges containing the transformations. Every formula matched to the hub is removed from the list $L_{\delta_G}$. Nodes that fail to be matched are added to the list $L$. Once all connections have been attempted for the current node being processed, meaning $L_{\delta_G}$ is empty, the current hub's `was\_hub' attribute is set to 'true' and it is added to $L$.
    \item A new hub with `was\_hub' attribute `false' is then selected at random from $L$ and the matching process is repeated between this new ``hub" and the nodes in $L$.
    \item This process continues until all nodes except for one remain, or until the list of nodes that have attribute `was\_hub' == `false' is empty.
    \item The remaining nodes in the nonemtpy list (out of the two lists $L_{\delta_G}$ and $L$) are deemed final non-CMF ``hubs". They are currently the roots of a forest since a node is removed from the list as soon as it is matched with the current ``hub" (so it can have at most one connection with a node that was a current ``hub").
\end{itemize}

2. All CMF candidates from the same $\delta_G$ cluster are matched to the final non-CMF ``hubs" using a process similar to 1. If an edge is found, the hub and all nodes connected to it are found in the CMF.

3. Output the resulting graph.

\subsection{Matching Algorithm}
\label{appendix-algs-formula-matching}
A match is a binary relation between two formulas. To find one, formulas represented as canonical form recurrences (\cref{appendix-canonical-form}) are clustered based on dynamical metrics (\cref{section-methodology-matching}). Promising pairs, those with similar irrationality measure $\delta$ (\cref{def:irrationlity_measure}), are folded according to their convergence rates (\cref{def:convergent_ratio}), then sent to the coboundary solving algorithm (\cref{appendix-algs-coboundary-algorithm}). The result is a triple: the fold transform needed to be applied to each of the recurrences, and the coboundary matrix and polynomials as outputted by \cref{appendix-algs-coboundary-algorithm}.

Input: Two linear polynomial-coefficient recurrences with matrices $A(n)$ and $B(n)$.

Output: A list of three transformations connecting the recurrences (if found)
\begin{itemize}
    \item fold transformation for each of the recurrences
    \item coboundary transformation linking the two recurrences after they are folded
\end{itemize}

Steps:

1. If unknown, compute the approximate limit of each of the recurrences, using $4000$ terms.

2. Compute the convergence rates of each of the recurrences using \cref{def:convergent_ratio} at $n=2000$. If the convergence rate is less than $5\cdot10^{-2}$, set it to 0.

3. Compute the ratio of the convergence rates of the recurrences, when defined, as $R=\frac{|r_A|}{|r_B|}$. Assume this number can be approximated as a rational with low-enough denominator to good degree. Set $R=0$ if either of the convergence rates is $0$ (this is normally the case when $\delta = -1$, such PCFs converge polynomially).

4. Fold recurrence $A$ by $r_B$ and recurrence B by $r_A$ if $R \neq 0$. If $R=0$, all combinations of folding either of the recurrences by 2 are passed to the next phase for a total of 4 options: don't fold either, fold one of them (two options), or fold both. This is because the convergence rate does not contain fold information for PCFs with slow convergence.

5. Apply the coboundary solving algorithm \cref{appendix-algs-coboundary-algorithm} between the recurrences. If a coboundary transform is successfully found, output fold transforms from 4 and the coboundary transform.

\subsection{Coboundary Solving Algorithm}
\label{appendix-algs-coboundary-algorithm}

We aim to find a coboundary relation between two recurrences.
As discussed in \cref{section-methodology-matching}, solving explicit equations for a coboundary matrix symbolically constitutes a nonlinear problem. To see why, let us write the equations for the coboundary matrix and polynomials in full. We have, given two polynomial recurrence matrices $A(n)$ and $B(n)$ (not necessarily in companion form \cref{def:companion-form}

$$p_A(n)\cdot A(n)\cdot U(n+1) =p_B(n)\cdot U(n)\cdot B(n) \iff$$
$$
\sum_{i=0}^{d_{p_A}} p_{A,i}n^i \cdot
\begin{pmatrix}
    \sum_{i=0}^{d_{A_{11}}}A_{11,i}n^i & \sum_{i=0}^{d_{A_{12}}}A_{12,i}n^i \\
    \sum_{i=0}^{d_{A_{21}}}A_{21,i}n^i & \sum_{i=0}^{d_{A_{22}}}A_{22,i}n^i
\end{pmatrix}
\cdot
\begin{pmatrix}
    \sum_{i=0}^{d_{U_{11}}}U_{11,i}(n+1)^i & \sum_{i=0}^{d_{U_{12}}}U_{12,i}(n+1)^i \\
    \sum_{i=0}^{d_{U_{21}}}U_{21,i}(n+1)^i & \sum_{i=0}^{d_{U_{22}}}U_{22,i}(n+1)^i
\end{pmatrix}
=
$$
$$
\sum_{i=0}^{d_{p_B}} p_{B,i}n^i
\cdot
\begin{pmatrix}
    \sum_{i=0}^{d_{U_{11}}}U_{11,i}n^i & \sum_{i=0}^{d_{U_{12}}}U_{12,i}n^i \\
    \sum_{i=0}^{d_{U_{21}}}U_{21,i}n^i & \sum_{i=0}^{d_{U_{22}}}U_{22,i}n^i
\end{pmatrix}
\cdot
\begin{pmatrix}
    \sum_{i=0}^{d_{B_{11}}}B_{11,i}n^i & \sum_{i=0}^{d_{B_{12}}}B_{12,i}n^i \\
    \sum_{i=0}^{d_{B_{21}}}B_{21,i}n^i & \sum_{i=0}^{d_{B_{22}}}B_{22,i}n^i
\end{pmatrix}
$$
The unknowns of this equation are the coefficients of $p_A(n),p_B(n)$ and the coefficients of the four polynomials of $U(n)$. It is clear from the above that the coefficients are coupled, making the equations resulting from equating powers of $n$ nonlinear.

An empirical method that takes care of nonlinearity and details such as what the polynomial degrees are without requiring them as input would be much preferred. Luckily, such a method is possible: we use a necessary condition (\cref{lemma-necessary-condition-for-coboundary-matrix}) a coboundary matrix between two recurrences must obey to leverage information about the recurrences and reconstruct the coboundary matrix in full.

Input: Two order-2 recurrences that converge to the same irrational constant $\varphi$ (up to a Möbius transformation ~\eqref{def:Möbius}) and a requested depth up to which a coboundary matrix will be fit to ``measurements" of the coboundary matrix. Denote the recurrence matrices by $A(n)$ and $B(n)$, their limits by $L_A$ and $L_B$ (without loss of generality - integer Möbius transformations of $\varphi$) and the requested depth as $N$.

Output: A polynomial matrix and two additional polynomials, if found, satisfying the coboundary condition \cref{def:cbd_equiv_poly}.

1. Solve for the first coboundary matrix $U(1)$ (up to a multiplicative factor):
\begin{itemize}
    \item Write the equation $L_A = U (L_B)$, where $U(\cdot)$ is the Möbius transformation, for the four unknowns of the first coboundary matrix $ U = \begin{pmatrix} u_{11} & u_{12} \\ u_{21} & u_{22} \end{pmatrix} $.
    \item Writing the Möbius transformation explicitly, the equation reads $L_A = \frac{u_{11}L_B + u_{12}}{u_{21}L_B + u_{22}}$
    \item Without loss of generality, the above equation can be written as
    $\frac{\alpha \varphi + \beta}{\gamma \varphi + \delta} = \frac{a \varphi + b}{c \varphi + d}$, where $a, b, c, d$ depend on $\{u_{ij}\}_{i,j\in\{1,2\}}$ and $L_B$ and $L_A = \frac{\alpha \varphi + \beta}{\gamma \varphi + \delta}$.
    \item By equating coefficients of powers of $\varphi$ in the numerator and denominator independently, we obtain four equations for the four unknowns of $U$.  (This must hold assuming the irrational number does not solve a quadratic equation, a condition met by all non-algebraic constants like $\pi$ and $\zeta(3)$.) The solution is the four rational numbers of $U$ (up to a free factor).
\end{itemize}

2. Propagate the coboundary matrix to the requested depth $N$ using the necessary condition for a coboundary equivalence:
\begin{itemize}
    \item $ A(n)\cdot U(n+1) \propto U(n) \cdot B(n) \Longrightarrow U(n+1) \propto A(n)^{-1} \cdot U(n) \cdot B(n)$
    \item The resulting matrices are again the rational matrices $\{U_i\}_{i=2}^N$, known only up to independent multiplicative factors.
\end{itemize}

3. Divide each of the ``measured" $\{U(i)\}_{i=1}^N$ by, e.g., the $U_{11}$ element of each matrix.
\begin{itemize}
    \item (Take care to pick an element $U_{11}, U_{12}, U_{21}, U_{22}$ that does not zero out on $1, 2, \dots, N$. If all do then pick the element whose last 0 arrives at the earliest index and toss out all measurements preceeding this index.)
    \item The result is a new list of matrices.
    \item If a polynomial coboundary relation exists between the two recurrences, $p_A(n) \cdot A(n) \cdot \tilde{U}(n+1) = p_B(n) \cdot \tilde{U}(n) \cdot B(n)$ for some polynomials $p_A(n)$ and $p_B(n)$, then the matrices we have found are precisely the result of dividing $\tilde{U}(n)$ by its first element. Meaning our measurements should be of a rational matrix.
\end{itemize}

4. Fit rational functions to each of the elements of the measured $U$:
\begin{itemize}
    \item Writing a general $2\times2$ rational matrix requires $8$ polynomials. When equating the rational function of each element to the measurements of that element, the result is a system of linear equations for the coefficients of the numerator polynomial and the denominator polynomial.
    \item The result is a rational coboundary matrix \textit{hypothesis} $U_h(n)$. Hypothesis - because this is a rational fit to our empirical coboundary matrices.
\end{itemize}

5. The final stage of the algorithm is to verify the hypothesis.
\begin{itemize}
    \item Multiply out $A(n) \cdot U_h(n+1)$ and $U_h(n) \cdot B(n)$. If the resulting matrices differ by a multiplicative factor (not a matrix, a rational function), meaning the condition $A(n) \cdot U_h(n+1) \propto U_h(n) \cdot B(n)$ holds, then the coboundary matrix hypothesis $U_h(n)$ is a valid coboundary matrix.
    \item Multiply both sides of $A(n)\cdot U_h(n+1) \propto U_h(n) \cdot B(n)$ by the least common multiple of the denominators of $U_h(n+1), U_h(n)$ to convert the coboundary relation into a polynomial one of the form $p_A(n)\cdot A(n) \cdot U(n+1) = p_B(n) \cdot U(n) \cdot B(n)$, where $p_A(n), p_B(n)$ are polynomials.
\end{itemize}

6. Output $U(n)$, $p_A(n)$ and $p_B(n)$ (which are defined only if $U_h$ was valid).

Below we include a proof of the necessary condition a coboundary matrix must obey that was leveraged to create the algorithm above.

\begin{lemma}
\label{lemma-necessary-condition-for-coboundary-matrix}
    (A necessary condition on the coboundary equivalence matrix.) Let $L_A = \displaystyle \lim_{n\to \infty} PCF\left( a(n),b(n) \right)$ ,$L_B = \lim_{n\to \infty} PCF\left( c(n),d(n) \right)$ be converging PCFs with associated companion matrices     $A(n),B(n)\in \operatorname{PGL}_2\left(\mathbb{Q}(n)\right)$. If $A(n)$ is coboundary to $B(n)$, then $L_A$ and $L_B$ are related through a rational Möbius transformation, moreover, if $U(n)$ is the coboundary matrix then $L_A=U(1)(L_B)$ ($U(1)$ applied to $L_B$ as a Möbius transformation).
\end{lemma}
\begin{proof}
    Let $\mathcal{A}_n$ and $\mathcal{B}_n$ be the step matrices of the PCF recurrence.
    \[
    \mathcal{A}_n =\prod_{i=1}^n A(i)= 
\begin{pmatrix}
    p_{n-1} & p_n \\
    q_{n-1} & q_n
\end{pmatrix},
\quad  \mathcal{B}_n=\prod_{i=1}^n B_i= 
\begin{pmatrix}
    s_{n-1} & s_n \\
    t_{n-1} & t_n
\end{pmatrix}
    \]
   Let \[
   U(n+1) = \begin{pmatrix}
    \alpha(n) & \beta(n) \\
    \gamma(n) & \delta(n)
\end{pmatrix}
\] with $\alpha,\beta,\gamma,\delta$ polynomials.
Since $A(n)$ and $B(n)$ are coboundary equivalent
\[
\mathcal{A}_n \cdot U(n+1) = U_1 \cdot \mathcal{B}_n
\]
This implies equality when applying a factional linear transformation to $0$.
\begin{equation}\label{eq:apendix_proof}
\left(\begin{pmatrix}
    p_{n-1} & p_n \\
    q_{n-1} & q_n
\end{pmatrix} \cdot U(n+1)\right)(0) =\left( U(1) \cdot\begin{pmatrix}
    s_{n-1} & s_n \\
    t_{n-1} & t_n
\end{pmatrix}\right)(0)
\end{equation}
Taking the limit, Equation~\eqref{eq:apendix_proof} above yields
\[
\displaystyle \lim_{n\to \infty}  \frac{p_{n-1}\beta(n)+p_n\delta(n)}{q_{n-1}\beta(n)+q_n\delta(n)} = \lim_{n\to \infty} \frac{p_{n-1}}{q_{n-1}}\cdot \left(\frac{\frac{\beta(n)}{\delta(n)}+\frac{p_n}{p_{n-1}}}{\frac{\beta(n)}{\delta(n)}+\frac{q_n}{q_{n-1}}}\right)
\]
The matrix $A(n)$ is a companion matrix for a second-order linear recurrence, any solution for the recurrence is composed of two prime solutions $x_n,y_n$, where the growth of $x_n$ is dominant over the growth of $y_n$. Since $\frac{p_n}{q_n}$ converge, we know that $p_n$ and $q_n$ both have an $x_n$ component to then, making $\frac{p_n}{p_{n-1}}$ and $\frac{q_n}{q_{n-1}}$ both grow asymptotically the same, this readers $\displaystyle \lim_{n\to \infty} (\frac{\frac{\beta(n)}{\delta(n)}+\frac{p_n}{p_{n-1}}}{\frac{\beta(n)}{\delta(n)}+\frac{q_n}{q_{n-1}}})=1$ and we may conclude that 

\[
\lim_{n\to \infty}  \frac{p_{n-1}\beta(n)+p_n\delta(n)}{q_{n-1}\beta(n)+q_n\delta(n)}
= \lim_{n\to \infty} \frac{p_{n-1}}{q_{n-1}}\cdot \left(\frac{\frac{\beta(n)}{\delta(n)}+\frac{p_n}{p_{n-1}}}{\frac{\beta(n)}{\delta(n)}+\frac{q_n}{q_{n-1}}}\right) = L_A\cdot 1
 \]
Hence $L_A = U_1(L_B)$ as needed.
\end{proof}

This algorithm is also available on our \href{https://colab.research.google.com/drive/13EC9hwEhoA_xvEu_7p_9wbIl2QjDknqC?authuser=1#scrollTo=i-3w8tyB2_-P}{results colab}

\subsection{Conversion of Formulas to Canonical Form}
We use RISC's package to convert formulas into recurrences, after which the proper initial conditions needed for the recurrence to generate the series are found.

Our use of RISC's algorithm can be summarized by

Input: First $N$ approximants of a formula.

Output: Minimal polynomial recurrence.

As more numbers in a sequence are taken, the probability that a recurrence will produce the first terms of the rational sequence but not the rest decreases. Additionally, once such a recurrence is found for a formula with closed form, the series can be substituted in and shown to obey it, making the method robust.

Formulas given by a polynomial recurrence of order 2 can be represented as polynomial $2 \times 2$ companion matrices (polynomial continued fractions): given a recurrence $x_n$
$$c(n)x_n=a(n)x_{n-1}+b(n)x_{n-2}$$

the corresponding canonical form is
$$\text{PCF}\left( a(n), b(n)c(n-1)\right)$$
Which is achieved using inflation by $c(n)$ (see \ref{appendix-Coboundary-transform}).

Converting a general series to a continued fraction can be done using a technique devised by Euler, however, this technique relies on algebraic manipulation of the term of the series and also does not necessarily yield \textit{polynomial} recurrences, let alone recurrences of order 2. A formula can have a recurrence of order 2 that is not polynomial, but have minimal polynomial recurrence of order 3.

\subsubsection{Conversion of Series Limit to Polynomial Continued Fraction (PCF) Limit}
\label{appendix-algs-series-limit-to-cf-limit}

Input: Series, with partial sums (not summand) to index $n$ given by $S(n)$, and continued fraction $CF(a_n, b_n)$ from the recurrence fit to the series using RISC's tool \cite{kauers2022guessing}.

Output: $x \in \mathbb{Q}$ such that the initial conditions
$$\begin{pmatrix}
S_0 & xS_1 \\
1 & x
\end{pmatrix}$$
generate the partial sums of the entire series when the continued fraction's recurrence is applied from index 2.
By demanding the second convergent of the continued fraction is equal to the second partial sum of the series (recall our notation for the Möbius transform defined by matrix $U$ - $U(\cdot)$):
$$
\left(
\begin{pmatrix}
S_0 & xS_1 \\
1 & x
\end{pmatrix}
\cdot
\begin{pmatrix}
0 & b_2 \\
1 & a_2
\end{pmatrix}
\right)(0)
=
S_2
$$
we arrive at the equation
$$\frac{S_0b_2 + xS_1a_2}{b_2 + xa_2} = S_2$$
Solving for x, we obtain
$$x = \frac{-b_2}{a_2} \left( \frac{S_2 - S_0}{S_2 - S_1} \right)$$ 

Note that only the first three partial sums of the series $S_0$, $S_1$ and $S_2$ (more generally $n_0$, $n_0+1$, $n_0+2$, where $n_0$ is the start index of the series summation) and a single partial-numerator and partial-denominator pair of the continued fraction are needed. This is since the continued fraction values $\frac{p_n}{q_n}$ satisfy the recurrence, hence when equating the first two values to the values of the sum we get the exact same series.

The algorithm's result can be converted to initial conditions for when the continued fraction's recurrence is applied from any other index by multiplying by recurrence matrices of appropriate indices and their inverses, i.e.,

\begin{equation*}
    \begin{pmatrix}
    S_0 & xS_1 \\
    1 & x
    \end{pmatrix}
    \cdot
    \begin{pmatrix}
    0 & b_1 \\
    1 & a_1
    \end{pmatrix}^{-1}
    \cdot
    \begin{pmatrix}
    0 & b_1 \\
    1 & a_1
    \end{pmatrix}
    \cdot
    \begin{pmatrix}
    0 & b_2 \\
    1 & a_2
    \end{pmatrix}
    \cdot
    \begin{pmatrix}
    0 & b_3 \\
    1 & a_3
    \end{pmatrix}
    \dots
\end{equation*}

yields the series partial sums but using a product from $n=1$, using the same notation as \cref{eq:step_matrix}.

This method is convenient since there is no need to keep track of transformations applied to the series' recurrence in the process of conversion to continued fraction. In summary, the correct initial conditions for recreating the series from the recurrence are found using only the recurrence and first few partial sums of the series.

This algorithm is also available on our \href{https://colab.research.google.com/drive/13EC9hwEhoA_xvEu_7p_9wbIl2QjDknqC?authuser=1#scrollTo=i-3w8tyB2_-P}{results colab}.

\subsection{Recurrence Generation by Conservative Matrix Field (CMF)}
\label{append-algs-cmf-trajectories}

CMF Generation of a recurrence is made by selecting a starting point and a trajectory in the CMF. We calculate their corresponding trajectory matrix (see \ref{appendix-section-trj-mat}), and finally convert it to the canonical companion form (see \ref{appendix-second-order-recurrences}) in order to generate a recurrence.

It is worth mentioning that there are degrees of freedom in the calculation of the trajectory matrix, due to the conserving property of the CMF. The order of matrix multiplication can be selected, which is advantageous in preventing matrix singularities.

Our search space for formulas in the CMF defined in \ref{def:the_pi_cmf} consisted of the trajectory subspace $T_{10} =\{(a, b, c) \in \mathbb{Z}^3 : max\{|a|, |b|, |c|\} < 10\}$ and the initial position $p_0 = (\frac{1}{2},\frac{1}{2},\frac{1}{2})$. In the rare cases where $M(n)$ had a singularity at some point $p_0+kt$ (where $t \in T_{10}$), we have shifted the initial position to be $p' = p_0+(k+1)t$ to avoid it.


\section{Recurrences and Limit Preserving Transformations}
\label{appendix-maths}


\subsection{On Linear Recurrences}
%
Recurrences, also known as ``difference equations," are the discrete analog of differential equations. They play a prominent role in various areas of mathematics and science, including Newton's approximation algorithms, counting problems (combinatorics), special functions, and the modeling of economic and biological systems.
In this section, we revisit the notion of a linear recurrence that we introduced in \cref{section-math-backround}. For a comprehensive guide on the theory of difference equations, see \cite{alma990000938590203596}.


A function \(u_n\) satisfies a recurrence of order $m$ if it is a solution to the equation:
%
\[
u_n = a_{1,n} u_{n-1} + a_{2,n} u_{n-2}+\ldots a_{m,n}u_{n-m}
\]
%
It is customary to represent the recurrence via the associated companion matrix:
\begin{equation} 
\operatorname{CM}(n) \coloneqq
\begin{pmatrix}
0 & 0 & \dots & 0 & a_{m,n} \\
1 & 0 & \dots & 0 & a_{m-1,n} \\
0 & 1 & \dots & 0 & a_{m-2,n} \\
\vdots & \vdots & \ddots & \vdots & \vdots \\
0 & 0 & \dots & 1 & a_{1,n}
\end{pmatrix}
\end{equation} 


Observe that \([u_{n-m},\dots, u_{n-2}, u_{n-1}]\operatorname{CM}(n) = [u_{n-(m-1)},\dots,u_{n-1}, u_n]\). Thus, the companion matrix represents a single step in the recurrence at time \( n \).
By incrementally multiplying the companion matrix over $n$ steps, we get the matrix: 
\begin{equation}\label{eq:step_matrix-apend}M_n\coloneqq
\prod_{i=1}^n\operatorname{CM}(i) =
\begin{pmatrix}
p_{1,n-m}  & \dots &  p_{1,n-1} & p_{1,n} \\
p_{2,n-m} & \dots &  p_{2,n-1} & p_{2,n} \\
p_{3,n-m}  &\dots &  p_{3,n-1} & p_{3,n} \\
\vdots & \ddots & \vdots & \vdots \\
p_{m,n-m} & \dots & p_{m,n-1} & p_{m,n}
\end{pmatrix}
\end{equation}
which we call the $n$-th step matrix.
The functions $p_{1,n},\ldots p_{m,n}$ are solutions to the recurrence equation with initial conditions $p_{i,j}=\delta_i^j $, and any other solution is a linear combination of these.
Hence this matrix encapsulated all the information about the recurrence.
Explicitly to get a different set of solutions we multiply the step matrix $M_n$ from left by the initial condition matrix $U_\text{init}$ where each row represents the initial condition for a solution.


Alternatively, let \( A(n) \in \operatorname{PGL}_m\left( \mathbb{Q}(n)\right) \) be a matrix with coefficient functions. Similar to the process done for the companion matrix,  we incrementally multiply the matrix:
\[
\mathcal{A}_n \coloneqq \prod_{i=1}^n A(i)
\]
We regard $\mathcal{A}_n$ as a "cocycle" (see \ref{appendex- group cocycle} and \ref{appendex- dymanic cocycle}).  We are fundamentally interested in the situation where the matrix $A(n)$ is coboundary equivalent~\eqref{def:cbdry_rational_matix} to the companion matrix of a recurrence. 


In the context of Diophantine approximation formulas for constants, we examine the ratios of elements in the last column of the step matrix \( M_n \). Diophantine approximation formulas can also be derived by taking ratios of elements in \(\mathcal{A}_n\).

We will now provide a detailed analysis for the case of second-order recurrences. In this case, we utilize the fact that a 2-by-2 matrix acts by Möbius transformation (linear fractional transformation) on \(\mathbb{R} \cup \infty\), and we provide a complete characterization of coboundary equivalence between matrices.
%
\subsubsection{Second Order Recurrences}
\label{appendix-second-order-recurrences}
%
Recall that any 2-by-2 matrix \( M \) can perform a fractional linear transformation. The action of \( M \) on the number \( l \) is defined as:
\begin{equation}
    \label{def:Möbius}
\begin{pmatrix}
    a & b \\
    c & d
\end{pmatrix} (l) = \frac{a \cdot l + b}{c \cdot l + d}
\end{equation}
This is known as the Möbius transformation of the matrix acting on \( l \). It is customary to extend this action to $\mathbb{R}\cup \infty$ by defining $\begin{pmatrix}
    a & b \\
    c & d
\end{pmatrix} (\infty)=\frac{a}{c}$
%

%
Let \( a_n, b_n \in \mathbb{Q}(n) \) be rational functions, and consider the recurrence relation \( u_n = a_n u_{n-1} + b_n u_{n-2} \).
By incrementally multiplying the companion matrix \(\operatorname{CM}(n)\) and examining the ratio of the elements in the first (or second) column, we effectively apply a Möbius transformation. Explicitly, we have:
\[
M_N (0)  = \frac{p_n}{q_n}, \quad M_N (\infty) = \frac{p_{n-1}}{q_{n-1}}
\]

Alternatively, we can start with a general matrix \( A(n) \in \operatorname{PGL}_2\left( \mathbb{Q}(n)\right) \) and incrementally multiply to obtain the matrix:

\[
\mathcal{A}_n \coloneqq \prod_{i=1}^n A(i)
\]

We can then examine the ratios of elements in the first or second column of this matrix, or even apply it as a Möbius transformation at a value \( l \in \mathbb{Q} \). This can potentially provide a Diophantine approximation to some constant, similar to how a recurrence relation provides a formula via a Möbius transformation of the step matrices.

We call such a matrix \( A(n) \) a \textit{formula generating matrix} if, for any \( l \in \mathbb{Q} \cup \{\infty\} \), the sequence \(\mathcal{A}_n( l)\) provides an approximation formula for a constant \( L \).
%

Indeed as the following lemma asserts any such matrix is coboundary equivalent (see Definition~\eqref{def:cbdry_rational_matix}) to a companion matrix of a recurrence equation.
\begin{lemma}
    \label{lem:formula_genrating_iff_CM}
    A matrix $A(n) \in \operatorname{PGL}_2\left( \mathbb{Q}(n)\right)$ is a formula-generating matrix if and only if there exist a matrix $U(n) \in \operatorname{PGL}_2\left( \mathbb{Q}(n)\right)$ and a companion matrix $\operatorname{CM}(n)$ associated to a recurrence $u_n$, such that
    \[
    U(n)\cdot A(n) = \operatorname{CM}(n) \cdot U(n+1)
    \]
    
\end{lemma}
\begin{proof}
    First, assume 
    \[A(n) = \begin{pmatrix}
        \alpha(n) & \beta(n) \\
        \gamma(n) & \delta(n)
    \end{pmatrix}
    \]
    is a formula-generating matrix, and denote by $\mathcal{A}_n = \prod_{i=1}^n A(i)$.
    If $\gamma(n)=0$ then the bottom left entry in $\mathcal{A}_n$ will remain $0$ for any $n$, resulting in $\lim_{n\to \infty} \mathcal{A}_n.\infty = \infty$. In this case it must be that $\beta(n)$ is non zero since if it is zero then so is the top right entry of $\mathcal{A}_n$, resulting in $\lim_{n\to \infty} \mathcal{A}_n.0 = 0\ne \infty$.
    Hence we may assume that either $\gamma(n)$ or $\beta(n)$.
    
    If $\gamma(n)\ne 0$ then define 
    \[
    U(n) = \begin{pmatrix}
    \gamma(n) & -\alpha(n) \\
    0 & 1
    \end{pmatrix}
    \]
    And get that
    \[
    U(n)\cdot A(n)\cdot U^{-1}(n+1) = 
    \begin{pmatrix}
    0 &  -\alpha({n}) \delta(n) + \beta(n)\gamma({n}) \\
    \frac{\gamma({n})}{\gamma({n+1})} & \frac{\alpha(n+1)\gamma({n})}{\gamma({n+1})}+\delta({n})
    \end{pmatrix}
    \]
    And if $\gamma(n)=0$ define
    \[
    U(n) =\begin{pmatrix}
    1 & 0 \\
    \alpha({n-1}) & \beta({n-1})
    \end{pmatrix}
    \]
      And get that
    \[
    U(n)\cdot A(n)\cdot U^{-1}(n+1) =
    \begin{pmatrix}
    0 & 1 \\
    -\frac{\alpha(n)(\alpha(n-1)\beta(n)+\beta(n-1)\delta(n))}{\beta(n)})
+\alpha({n-1})\alpha(n) & \frac{\alpha(n-1)\beta(n)+\beta(n-1)\delta(n)}{\beta(n)})
    \end{pmatrix}
    \]
    Where these matrices are projectively equivalent to a matrix in a companion form. This transformation was originally shown in \cite{esma}.

    To conclude the proof it is left to show that companion form matrices are formula-generating.
    
    Let $\operatorname{CM}(n)= \begin{pmatrix}
        0 & b_n \\
        1 & a_n
    \end{pmatrix}$ 
    be a companion form matrix with $\displaystyle \lim_{n\to \infty} \frac{p_n}{q_n}=L$. 
    \[
    M_n(l) = \frac{p_{n-1}\cdot l+ p_n}{q_{n-1}\cdot l+q_n}
    = \frac{p_n}{q_n} \cdot \frac{\frac{p_{n-1}}{p_n}\cdot l+ 1}{\frac{q_{n-1}}{q_n}\cdot l+1} 
    \]
    And since the ratio of $\frac{p_n}{q_n}$ converges,  the growth rate of these solutions is the same, implying that $\displaystyle \lim_{n\to \infty} \left( \frac{\frac{p_{n-1}}{p_n}\cdot l+ 1}{\frac{q_{n-1}}{q_n}\cdot l+1}\right)  = 1$ making $\displaystyle \lim_{n\to \infty} M_n(l)=L$.
\end{proof}
%
%
\subsection{Coboundary transform}
\label{appendix-Coboundary-transform}
%
Given a recurrence relation $u_n$, we are interested in analyzing the dynamics of its solutions, which translates to analyzing the behavior of the companion matrix $\operatorname{CM}(n)$ increments, i.e., the behavior of the step matrix $M_n$ as $n$ approaches infinity.

We are interested in the ratio between solutions to the recurrence. Consequently, we look at these matrices as elements in the projective group $\operatorname{PGL}_m\left( \mathbb{Q}(n) \right)$
We will recall the definition of a coboundary equivalence:

    Two matrices $A(n),B(n)\in \operatorname{PGL}_m\left(\mathbb{Q}(n)\right)$ are said to be \textit{coboundary equivalent} if there exist a matrix $U(n)$ such that 
\begin{equation}
     A(n) \cdot U({n+1})  =   U(n) \cdot B(n) 
\end{equation}

From the equation above we see that $A(n) = U(n) \cdot B(n) \cdot U^{-1}(n+1) $ thus exhibiting a "telescoping effect" on the product, resulting in the equation:
\[
(\prod_{i=1}^n A(n))\cdot U(n+1)  = U(1)\cdot(\prod_{i=1}^n B(n)) 
\]
This suggests that these matrices share similar dynamics.

We denote $A(n)\sim B(n)$, to indicate that they are coboundary equivalent, and we often call the matrix $U(n)$ a \textit{coboundry transform} from $A(n)$ to $B(n)$.
%

\textbf{Index shift}:

Notice that the matrices $A(n)$ and $A(n+1)$ are coboundary equivalent by simply taking $U(n)=A(n)$. This implies that an index shift is a coboundary operation. Since coboundary is an equivalence, by transitivity, any integer index shift is also a coboundary operation.

\textbf{Inflation}:

Given a function $u_n$ satisfying the relation
\[
u_n = a_{1,n} u_{n-1} + a_{2,n} u_{n-2} + \ldots + a_{m,n} u_{n-m}
\]
We can define $\widetilde{u}_{n}= u_n \cdot \displaystyle \prod_{i=1}^n c_i$ for some function $c_n$.
The function $\widetilde{u}_{n}$ satisfied the relation
\[
\widetilde{u}_n = a_{1,n} c_n \widetilde{u}_{n-1} + a_{2,n} c_n c_{n-1} \widetilde{u}_{n-2} + \ldots + a_{m,n} c_n c_{n-1} \cdots c_{n-(m-1)}\widetilde{u}_{n-m}
\]

Let \(\operatorname{CM}(n)\) denote the companion form of $u_n$ and \(\widetilde{\operatorname{CM}}(n)\) denote the companion form of $\widetilde{u}_n$, then the matrix 

\[
U(n) = \begin{pmatrix}
   \displaystyle \prod_{i=1}^{m-1} c(n-i) & 0 & \dots & 0 & 0 \\
0 &\displaystyle \prod_{i=1}^{m-2} c(n-i)  & \dots & 0 & 0 \\
\vdots & \vdots & \ddots & \vdots & \vdots \\
0 & 0 & \dots & c(n-1)& 0 \\
0 & 0 & \dots & 0 &  1
\end{pmatrix}
\]

Is a coboundary transform between \(\operatorname{CM}(n)\) and  \(\widetilde{\operatorname{CM}}(n)\).
This process is called inflation and it allows one to construct a Polynomial Continued Fraction (PCF) from a recurrence given by rational function coefficients, (see \cite{BlindDelta}), this is done by taking $c_n $ to be the LCM of the denominators.
In the other direction starting from a $PCF\left(a(n),b(n)\right)$ and taking $C_n = n^{-\text{deg}(a)}$ the inflation by $c_n$ is called factorial reduction (see \cite{Arnold_BenDavid2024}).

\subsubsection{Canonical Form}
\label{appendix-canonical-form}
We have seen that a recurrence relation with rational function coefficients can be transformed via inflation to a recurrence with polynomial coefficients. We say that a recurrence 

\[ 
u_n = a_{1,n} u_{n-1} + a_{2,n} u_{n-2} + \ldots + a_{m,n} u_{n-m} 
\]

is in canonical form if the coefficients \( a_{i,n} \) are polynomials, and for any other recurrence with polynomial coefficients \({a'}_{i,n}\) which is coboundary equivalent to it, the degree of each \( a_{i,n} \) is less than or equal to the degree of the corresponding \({a'}_{i,n}\).


Lemma~\ref{lem:formula_genrating_iff_CM} asserts that in the case of second-order recurrences, any formula-generating matrix is equivalent to a companion matrix. In turn, this is equivalent to a companion matrix of a recurrence $u_n = a(n)u_{n-1}+b(n)u_{n-2}$ with polynomial coefficients through the process of inflation. We denote such recurrences by $PCF\left( a(n),b(n) \right)$ as they are associated with a polynomial continued fraction
\[
    M(n)(0)= \cfrac{b(1)}{
        a(1) + \cfrac{b(2)}{
            \begin{array}{cc} 
                a(2) \ + & \\  
                & \ddots
            \end{array}
            \begin{array}{c}
                \\ 
                + \ \cfrac{b(n)}{a(n)}
            \end{array}
        }
    } = \frac{p_n}{q_n}
\]
The canonical form of $PCF\left( n^2 + n + 1, n^4 + n^2 + 1\right)$, for example, is equal to $PCF\left(1,1\right)$, the simplest continued fraction for the Golden Ratio.
%
\subsubsection{Group Cocycle and Coboundary}
\label{appendex- group cocycle}
We now describe the mathematical context that motivates our definition of coboundary equivalence.

Let \(\Gamma\) be a group acting by automorphisms on a group \(G\) via the map:

\[
\varphi\colon \Gamma \to \operatorname{Aut}(G)
\]

A \(\Gamma\)-cocycle with respect to \(\varphi\) is a map \(\mathcal{M}\colon \Gamma \to G\) satisfying the cocycle condition:
\begin{equation}
\label{eq:cocycle_condition}
    \mathcal{M}(\gamma_1\gamma_2) = \mathcal{M}(\gamma_1)\cdot \varphi_{\gamma_1}(\mathcal{M}(\gamma_2)) 
\end{equation}
We declare two cocycles, \(\mathcal{M}\) and \(\mathcal{M}'\), to be coboundary equivalent if there exists an element \(g \in G\) such that:
\begin{equation}
\label{def:grou_cohomology_cbd_equiv}
g\cdot \mathcal{M}(\gamma)=\mathcal{M'}(\gamma)\cdot\varphi_\gamma(g)
\end{equation}
For further details, we refer the interested reader to Chapter 5 in \cite{serre1979galois}.

Here, we focus on the case where \(\Gamma = \mathbb{Z}\) and \(G = \operatorname{PGL}_m\left( \mathbb{Q}(n) \right)\). 

For any $k\in \Gamma$, and \(A(n) \in G\), the action map is defined by \(\varphi_m(A(n)) = A(n+k)\).
Since \(\mathbb{Z}\) is generated by a single element, any cocycle \(\mathcal{M}\colon \Gamma \to G\) is determined by the value \(\mathcal{M}_1(n) \in G\). 
For $k>0$ the cocycle condition implies the 
\[
\mathcal{M}_{k}(n) = \prod_{i=0}^{k-1} \mathcal{M}_1(n+i)
\]
Moreover since $ \mathcal{M}_{0}(n)=\operatorname{Id}_m $, have that $ \mathcal{M}_{k}(n) = \left(\mathcal{M}_{-k}(n-k)\right)^{-1}$

From the definition of the coboundary equivalence in the context of group cocycles, we get that two \(\Gamma\)-cocycles, \(\mathcal{M}\) and \(\mathcal{M}'\), are coboundary equivalent if there exists \(U(n) \in G\) such that:
\[
U(n) \cdot \mathcal{M}_1(n) = \mathcal{M}'_1(n) \cdot U(n+1)
\]
When regarding a matrix \(A(n) \in \operatorname{PGL}_m\left(\mathbb{Q}(n)\right)\) as the matrix $\mathcal{M}_1(n)$ generating the cocycle, this equivalence is precisely the definition given in Definition~\eqref{def:cbdry_rational_matix}.

\subsubsection{Cocyles in Dynamical Systems}
\label{appendex- dymanic cocycle}

Let \( T \) be a homeomorphism \( T\colon X \to X \) of some topological space. In dynamical systems, one studies the evolution of the system under the repeated application of the transformation \( T \).

A function 
\[
\mathcal{A}\colon \mathbb{Z} \times X \to \operatorname{GL}_m(\mathbb{R})
\]
satisfying
\[
\mathcal{A}(k_1+k_2, x) = \mathcal{A}(k_1, x) \mathcal{A}(k_2, T^{k_1}(x))
\]
is called a continuous linear cocycle (see \cite{bochi2024complete}).

Also in this context, a single matrix $A(n)\in \operatorname{GL}_m\left(\mathbb{Q}(n) \right)$ defines a cocycle. This is done by taking \( X = \mathbb{R} \), and constructing a cocycle by defining
\[
\mathcal{A}(k, x) = \prod_{i=0}^{k-1} A(x+i)
\]
Typically, it is assumed that the topological space \( X \) is a compact space or a probability measure space (see \cite{DuartePedro2016LEoL} for details on dynamic cocycles).
Our definition of coboundary equivalence~\eqref{def:cbdry_rational_matix} is not only mathematically natural but also captures the essence of our dynamic framework. The vast data collected on related formulas suggest that this notion is effective for equating formulas, as it preserves the measure of irrationality.
%
\subsection{Fold Transform}
\label{appendix-maths-fold}
%
Consider the matrix \( A(n) \) as encoding a process defined by incrementally multiplying the matrix. We might wish to associate a matrix with an accelerated process, one that goes ``k steps at a time". That is, a matrix \(A_{k\text{-fold}}(n) \) such that taking one step in \( A_{k\text{-fold}}(n) \) is equivalent to taking \( k \)-steps in \( A \). This requirement can be expressed as:
\[
\prod_{i=1}^{nk} A(i) = \prod_{i=1}^{n} A_{k\text{-fold}}(i)
\]
And the k-fold matrix is given by the product:
\[
A_{k\text{-fold}}(n) \coloneqq \prod_{j=1}^{k} A(kn-k+j)
\]
When the matrix \( A(n) \) is a trajectory matrix within CMF (as described in Appendix ~\ref{appendix-section-trj-mat}
), the \( k \)-folding of \( A(n) \) corresponds to the trajectory matrix associated with a direction that is \( k \)-times the original direction.

\section{The Conservative Matrix Field (CMF)}
\label{appendix-cmf}
In their paper \cite{doi:10.1073/pnas.2321440121} the authors define a CMF using matrices with polynomial coefficients, this is not enough for our purposes since the $\pi$-cmf ~\ref{def:the_pi_cmf} has rational function as coefficients. We redefine the CMF in a way that allows one to insert rational functions as coefficients.
%
\begin{definition}
    \label{def:CMF}
    A $d$-dimensional CMF of rank $m$ is defined by a collection of $d$-matrices $M_{\mathbf{x}_1},\ldots,M_{\mathbf{x}_d}$ in $\operatorname{PGL}_m\left( \mathbb{Q}(x_1,\ldots,x_d)\right)$ satisfying the conserving property, that is, for any pair $i\ne j$
    \[
    \begin{aligned}
    M_{\mathbf{x}_i}(x_1,\ldots,x_i,\ldots,x_j,\ldots,x_d)  &M_{\mathbf{x}_j} (x_1,\ldots,x_i+1,\ldots,x_j,\ldots,x_d)
    \\&=\\
    M_{\mathbf{x}_j}
    (x_1,\ldots,x_i,\ldots,x_j,\ldots,x_d)
    &M_{\mathbf{x}_i}(x_1,\ldots,x_i,\ldots,x_j+1,\ldots,x_d)
    \end{aligned}
    \]
\end{definition}
Envision a $d$-dimensional lattice where each edge has a displacement function representing the ``work" of moving between vertices. The conserving property is that: the work is \textit{path-independent}.
%.
%
\subsection{Trajectory matrices in a CMF}
\label{appendix-section-trj-mat}
 %
 Let $\mathcal{M}$, be a $d$-dimensional CMF of rank $m$. Given a point $x\in \mathbb{Q}^d$ and a direction $v\in \mathbb{Z}^d$ the evaluation map $v\mapsto \mathcal{M}_v(x)$ describes the displacement from point $x$ to point $x+v$.
    By the conserving property, the total work for a displacement along a broken path, say, first from $x$ to $x+v$ and then from $x+v$ to $(x+v)+w$, is equal to the total displacement from $x$ to $x+v+w$, in terms of the evaluation map this is translated to:
    \[
    \mathcal{M}_{v+w}(x) =\mathcal{M}_{v}(x) \mathcal{M}_{w}(x+v)  
    \]
 
 We can construct a general matrix representing the work going from the point $x+(n-1)v$ to the point $x+nv$, this matrix is in $\operatorname{PGL}_2\left(\mathbb{Q}(n)\right)$ and is given by
 \[
 T_{x,v}(n)=\mathcal{M}_v(x+(n-1)v)
 \]
We call this matrix the trajectory matrix associated with the point $x$ in direction $v$.

Note that by the conserving property  $\text{Id} =M_{-v+v}(x)=M_{-v}(x)M_{v}(x-v)$, this provides us with the identity
\[
M_{-v}(x)=M_{v}^{-1}(x-v)
\]

Taking a point $x'$ in the lattice $ x+\mathbb{Z}^d$  with the same direction $v$ yields the trajectory matrix 
\[
T_{x',v}(n)=\mathcal{M}_v(x'+(n-1)v)
\]
Since $x'$ is in the same lattice, the difference $w =x'-x$ is in $\mathbb{Z}^d$. The direction $w$ represents the direction from $x$ to $x'$. 
By the conserving property taking $U(n)=T_{x,w}(n)$, the trajectory matrix from $x$ in the direction $w$ is a coboundary transform between $T_{x,v}(n)$ and $T_{x',v}(n)$.

In terms of the evaluation functions we see explicitly :
\[
\mathcal{M}_{v+w}(x+(n-1)v) =\mathcal{M}_{v+w}(x+(n-1)v) \mathcal{M}_{w}(x+(n-1)v+v) =  T_{x,v}(n)T_{x,w}(n+1)
\]
and
\[
\mathcal{M}_{v+w}(x+(n-1)v) =\mathcal{M}_{w}(x+(n-1)v) \mathcal{M}_{v}(x+(n-1)v+(x'-x)) =  T_{x,w}(n) T_{x',v}(n)
\]

\subsection{The $\pi$-CMF}
\label{appendix Hypergeomtric-CMF}
The following three matrices describe a 3D rank 2 CMF. 
\begin{equation}
    \label{def:the_pi_cmf}
    \begin{aligned}
M_{\mathbf{x}} &=
\begin{pmatrix}
    1 & y \\
    \frac{1}{x} & \frac{2x + y - 2z + 2}{x}
\end{pmatrix}
\\
M_{\mathbf{y}} &=
\begin{pmatrix}
    1 & x \\
    \frac{1}{y} & \frac{x + 2y - 2z + 2}{y}
\end{pmatrix}
\\
M_{\mathbf{z}} &=
\begin{pmatrix}
    \frac{z(-x - y + z)}{(y - z)(x - z)} & \frac{zxy}{(y - z)(x - z)} \\
    \frac{z}{(y - z)(x - z)} & \frac{-z^2}{(y - z)(x - z)}
\end{pmatrix}
\end{aligned}
\end{equation}
The cocycle $\mathcal{M}\colon \mathbb{Z}^3\to \operatorname{PGL}_2\left(\mathbb{Q}(n)\right)$ is defined on the generators of $\mathbb{Z}^3$ as
\[
\mathcal{M}_{e_1} = M_{\mathbf{x}}, \quad \mathcal{M}_{e_2} = M_{\mathbf{y}}, \quad \mathcal{M}_{e_3} = M_{\mathbf{z}}
\]

\textbf{Example of a formula arising as a trajectory in the CMF.}

Let $x=(\frac{1}{2},\frac{-1}{2},\frac{3}{2})$ be an point in space defining the lattice $x+\mathbb{Z}^3$.
Directions between points in this lattice correspond to a vast collection of formulas for $\pi$ as seen in Fig.~\ref{fig:cmf-unification}. 

For completeness of the exposition, we take the most simple direction $e_3=(0,0,1)$ and show how the famous Euler formula $PCF\left(1,n(n+1)\right)=\frac{2}{\pi-2}$ sits as a trajectory matrix over the point $x$ in the direction $e_3$.

\[\operatorname{CM}(n) = \begin{pmatrix}
0 &n(n+1)\\
1 & 1
\end{pmatrix}
 \]
is the companion form of Euler's PCF.

The trajectory matrix is equal to
\[
T{x,e_3}(n)= \mathcal{M}_{e_3}(\frac{1}{2},\frac{-1}{2},\frac{3}{2}+(n-1) ) =
\begin{pmatrix}
    \frac{(2n+1)^2}{4n(n+1)} & \frac{-2n-1}{8n(n+1)} \\
    \frac{2n+1}{2n(n+1)} & -\frac{(2n+1)^2}{4n(n+1)}
\end{pmatrix}
\]
Following the process described in Lemma~\ref{lem:formula_genrating_iff_CM}  we define 
\[
U(n) = 
\begin{pmatrix}
  \frac{2n+1}{2n(n+1)} & - \frac{(2n+1)^2}{4n(n+1)}\\
0 & 1
\end{pmatrix}
\]
and get that
\[
E(n)=U(n)T(n)U^{-1}({n+1}) = 
\begin{pmatrix}
0 & \frac{n^2+2n+\frac{3}{4}}{n^2+3n+2}\\
1 & \frac{n+\frac{3}{2}}{n^2+3n+2}
\end{pmatrix} \cdot \frac{2n^2+5n+2}{n(2n+3)}
\]
Is in companion form. We let $I(n)$ be the inflation by $c_n=n^2+3n+2$  (see ~\ref{appendix-Coboundary-transform}) and we get the relation
\[
I(n) \cdot E(N)
= \operatorname{CM}(n)\cdot I({n+1}) 
\]
Concluding $T(n)\sim E(n)\sim \operatorname{CM}(n) $ which states the equivalence of the trajectory with Euler's PCF.


\section{Formula Harvesting Details} \label{appendix-engineering}

\subsection{Article Retrieval}
\label{appendix-engineering-article-retrieval}
arXiv's search API was not reliable for retrieving papers with $\pi$ formulas. Some simple queries such as ``formula for pi" or ``$\pi$ formula" returned few results (and if the search method is not set to word-for-word results are mostly irrelevant). Not knowing all patterns in which $\pi$ tends to be calculated in, we went with a more exhaustive approach.

457,145 articles from the following categories which were indexed in the arXiv metadata dataset \cite{arXivMetadata} as of 24 November, 2024, were scraped: math.CA, math.NT, math.PR, math.CO, math.GM, math.HO, cs.AI, cs.LG and cs.DC.

\subsection{$\text{\LaTeX }$ Equation Patterns and Preprocessing}
\label{appendix-engineering-equation-patterns}

The following unnumbered or inline $\text{\LaTeX }$ math environments were scraped from all articles:

\begin{itemize}
    \item \texttt{\$ $\cdot$ \$},
    \item \texttt{\$\$ $\cdot$ \$\$},
    \item \texttt{\textbackslash{}[ $\cdot$ \textbackslash{}]}
    \item \texttt{\textbackslash{}( $\cdot$ \textbackslash{})}
    \item \texttt{math}
\end{itemize}

The following $\text{\LaTeX }$ equation environments were scraped from all articles:

\begin{itemize}
    \item \texttt{equation}
    \item \texttt{align}
    \item \texttt{gather}
    \item \texttt{multline}
    \item \texttt{alignat}
    \item \texttt{eqnarray}
\end{itemize}
\texttt{}

Starred (\texttt{*}) versions of the latter equation environments were also collected, for a total of 17 environments.

Equation environments were kept so strings with multiple equations could be split into distinct formulas during preprocessing. Preprocessing was mainly aimed at removing text within equations and setting uniform symbols for objects like

\texttt{\textbackslash{} ddots}, \texttt{\textbackslash{} cdots}.

\subsection{Formula Retrieval Patterns}
\label{appendix-engineering-formula-patterns}

\begin{table}[h!]
    \caption{\textbf{$\text{\LaTeX }$ Formula Patterns.} Each pattern was paired with both ``$\backslash pi =$" and ``$= \backslash pi$", and the \texttt{\textbackslash cfrac}-based variants of the \texttt{\textbackslash frac}-containing regular expressions were also included, resulting in a total of 10 patterns.}
    \label{tab:latex-patterns}
    \vskip 0.15in
    \centering
    \begin{center}
    \begin{small}
    \begin{tabular}{lc}
        \toprule
        Pattern for - & Python \textbf{re} Pattern \\
        \midrule
        Series & \textbackslash{} sum \textbackslash{} s*\_\{(?s:.)*\}\textbackslash{} s*\textbackslash{} char`\^ \textbackslash{} s*
        \\ \\
        
        Sum of \textbackslash{} frac
        & (\textbackslash{} s*\textbackslash{} frac\textbackslash{} s*\{\textbackslash{} s*[\char`\^  \{\}]*\textbackslash{} s*\}\textbackslash{}s*\{\textbackslash{} s*[\char`\^  \{\}]*\textbackslash{} s*\}) \\ & ((?:\textbackslash{} s*\textbackslash{}+\textbackslash{} s*\textbackslash{} frac\textbackslash{} s*\{\textbackslash{} s*[\char`\^  \{\}]*\textbackslash{} s*\}\textbackslash{} s*\{\textbackslash{} s*[\char`\^  \{\}]*\textbackslash{} s*\})+) \\ \\
        
        Nested \textbackslash{} frac &
        (\textbackslash{} frac\textbackslash{} s*\{\textbackslash{} s*[\char`\^ \{\} ]*\textbackslash{} s*\}\textbackslash{} s*\{\textbackslash{} s*[\char`\^ \{\} ]*) \\ & ((?:\textbackslash{} frac\textbackslash{} s*\{\textbackslash{} s*[\char`\^ \{\} ]*\textbackslash{} s*\}\textbackslash{} s*\{\textbackslash{} s*[\char`\^ \{\} ]*)+\textbackslash{} s*\}\textbackslash{} s*\}) \\
        
        \bottomrule
   \end{tabular}
    \end{small}
    \end{center}
\end{table}

In addition to \textit{positive} regular expressions, presence of any of the following strings in preprocessed data halted processing:

\texttt{sqrt}, \texttt{tan}, \texttt{cos}, \texttt{sin}, \texttt{log}, \texttt{ln}, \texttt{zeta}, \texttt{pi\char`\^{}}

\subsection{Formula Classification and Extraction}

As a formula is extracted, information collected from previous prompts is appended to the following prompts to reinforce context. See \cref{appendix-engineering-prompts} for the prompts used. All prompts to the LLM use a temperature of 0 to promote consistency. The OpenAI GPT API supports a return format called Structured Outputs, which guarantees a json schema of choice is returned by the LLM. Formula candidates are passed to the LLM for zero-shot binary classification, into classes: formulas that compute the constant $\pi$ and formulas that do not. Candidates are then passed again for trinary classification into classes: series, continued fraction (the targets of this study), and other formulas (which are either not $\pi$ formulas or are but are not series or continued fractions). The third class is intentionally redundant, as some strings GPT-4o mini gets wrong may be correctly filtered out by GPT-4o: of 3389 strings classified as $\pi$-computing series or continued fractions by GPT-4o mini, 1711 were tossed by GPT-4o.

The vast majority of formulas classified as $\pi$ formulas by GPT-4o were series rather than nested fractions (1,613 vs 65). In some instances the LLM successfully generalized sequences of numbers to polynomial expressions, see \cref{tab:appendix-engineering-llm-polynomial-extraction}.

\begin{table}[h!]
    \caption{OpenAI's GPT-4o extracts the partial numerator and partial denominator of a continued fraction. Explanation is by the LLM. $\text{\LaTeX }$ taken from \cite{Raayoni2021}}
    \label{tab:appendix-engineering-llm-polynomial-extraction}
    \vskip 0.15in
    \centering
    \begin{center}
    \begin{small}
    % \begin{sc}
    \begin{tabular}{c|c}
        \toprule
       $\text{\LaTeX }$ & \texttt{3 + \textbackslash{}frac\{1 * 3\}\{5 + \textbackslash{}frac\{2 * 4\}\{7 + } \\
       & \texttt{\textbackslash{}frac\{3 * 5\}\{9 + \textbackslash{}frac\{4 * 6\}\{11 + \dots\}\}\}\} = } \\
       & \texttt{3\textbackslash{}frac\{1\}\{\_\{2\}F\_\{1\} ( 1,\textbackslash{}frac\{1\}\{2\}; \textbackslash{}frac\{5\}\{2\}; - 1 ) \} = }\\
       & \texttt{\textbackslash{}frac\{4\}\{\textbackslash{}pi - 2\}} \\
       \midrule
       Rendered & $ 3 + \frac{1 \cdot 3}{5 + \frac{2 \cdot 4}{7 + \frac{3 \cdot 5}{9 + \frac{4 \cdot 6}{11 + \dots}}}} = 3 \cdot \frac{1}{\, _2F_1\left( 1, \frac{1}{2}; \frac{5}{2}; -1 \right)} = \frac{4}{\pi - 2}$ \\
        \midrule
        Prompt & Identify the partial denominator an and partial numerator bn. \\ & Extract each of them and write them as proper SymPy expressions as a \\ & function of depth n. \\
        \midrule
        Output & $a_n$ = 2*n + 3, $b_n$ = n * (n + 2) \\
       \midrule
       Explanation & The partial denominator an is the sequence 5, 7, 9, 11,... which can be expressed as 2*n + 3.
       \\ & The partial numerator bn is given by the sequence
       1*3, 2*4, 3*5, 4*6,...
       \\ & which can be expressed as n*(n + 2). \\
       \bottomrule
   \end{tabular}
    % \end{sc}
    \end{small}
    \end{center}
\end{table}

Perhaps additional formulas could be gathered in the future by allowing the LLM to decide whether to call a polynomial fitting function on such sequences.

\subsection{Prompts}
\label{appendix-engineering-prompts}

We utilized the Structured Output feature of OpenAI's GPTs and used a temperature of 0 for all prompts.\footnote{\href{https://platform.openai.com/docs/guides/structured-outputs}{Structured outputs} OpenAI documentation.} The return schemas are included with the prompts. 
$\text{\LaTeX }$ from prompts is rendered for readability.

\subsubsection{Classification Prompts}
Initial classification by GPT-4o mini:

\noindent \textbf{System Message:}
\vspace{2mm} 

\begin{tcolorbox}
You are a model that classifies whether a $\text{\LaTeX }$ string is a formula that can be rearranged to calculate the constant \\ \texttt{\{constant\}}. 
Specifically, we are interested in continued fractions and series.
\end{tcolorbox}

\vspace{4mm} 

\noindent \textbf{User Message:}
\vspace{2mm}

\begin{tcolorbox}
Is this a continued fraction or a series that can be rearranged to calculate the constant \texttt{\{constant\}}? \texttt{\{latex\_string\}} \\

Structured output: boolean.
\end{tcolorbox}

\subsubsection{Extraction Prompts}

For added context during extraction, the second classification query shown in Fig.\ref{fig:engineering-formula-extraction} (c) is actually conducted during the extraction stage. Values in source strings were not used for formula validation, see \cref{appendix-engineering-future-work}.

\noindent \textbf{System Message:}
\vspace{2mm} 

\begin{tcolorbox}
You are a model that extracts formula information from a $\text{\LaTeX }$ string.\\

Your task is to: \\
a. Classify the type of formula: series, continued fraction, or neither. \\
b. Extract its components and identify the variable. \\

This information will be used to compute the formula later, so it is critical that the extracted value and components are accurate to ensure correctness.\\

You will be asked separately about each of the following steps: \\
Step 1. Classify the formula: Determine whether the $\text{\LaTeX }$ string represents a series or a continued fraction that can be rearranged to calculate the constant \texttt{\{constant\}}.\\
Step 2. Identify the formula type: Specify whether it is a series or a continued fraction.\\
Step 3. Extract the formula components: \\
\quad - For series: Identify the term and the start value. \\
\quad - For continued fractions: Identify the partial numerator and partial denominator.\\
Step 4. Identify the variable of the formula: Clearly state the variable used in the formula.
\end{tcolorbox}

\vspace{4mm}

\noindent \textbf{User Message: Step 1}
\vspace{2mm} 

% step 1

\begin{tcolorbox}
Step 1: \\
Is this formula a series or a continued fraction that can be rearranged to calculate the constant \texttt{\{constant\}}?\\

\texttt{\{latex\_string\}} \\

Structured output: boolean.
\end{tcolorbox}

\vspace{4mm} 

\noindent \textbf{User Message: Step 2}
\vspace{2mm} 

% step 2

\begin{tcolorbox}
Step 2: \\
Determine the type of formula. \\
Is this formula a continued fraction or a series?\\

\texttt{\{latex\_string\}} \\

Structured output: 'series' or 'cf' (for continued fraction).
\end{tcolorbox}

\vspace{4mm}

\noindent \textbf{User Message: Step 3}
\vspace{2mm} 

% step 3

The prompt in step 3 depends on the classification result.

If \texttt{formula\_type == 'continued fraction'}

($\text{\LaTeX }$ for one-shot example taken from \cite{Raayoni2021})

\begin{tcolorbox}
Step 3: \\
The formula is a continued fraction. Identify the following components: \\
1. The partial denominator (\texttt{an}) as a function of depth (\texttt{n}). \\
2. The partial numerator (\texttt{bn}) as a function of depth (\texttt{n}). \\
3. Any unknown variables or expressions (other than the depth \texttt{n}). \\

Write each component as a proper SymPy expression. For example: \\
The string 
\[
\forall z \in \mathbb{C}: \quad 1+\frac{1\cdot(2\cdot z-1)}{4+\frac{2\cdot(2\cdot z-3)}{7+\frac{3\cdot(2\cdot z-5)}{10+\frac{4\cdot(2\cdot z-7)}{13+\ldots}}}} = \frac{2^{2\cdot z +1}}{\pi\binom{2\cdot z}{z}}
\]
has the following: \\
- \texttt{an}: \texttt{'3*n + 1'} \\
- \texttt{bn}: \texttt{'n*(2*z - (2*n - 1))'} \\
- \texttt{unknowns}: \texttt{['z']} \\

The continued fraction is: \\
\texttt{\{latex\_string\}} \\

Structured output: \{'an': str, 'bn': str, 'unknowns': list[str]\}
\end{tcolorbox}

If \texttt{formula\_type == 'series'} 

($\text{\LaTeX }$ for one-shot example taken from \cite{1806.03346})

\begin{tcolorbox}
Step 3: \\
The formula is a series. Identify the following components: \\
1. The term as a SymPy expression. \\
2. The dummy variable. \\
3. The start value of the dummy variable. \\
4. Any unknown variables (other than the dummy variable). \\

For example: \\
The string 
\[
\pi = \frac{22}{7} - 24\sum_{n=2}^\infty \frac{(-1)^{n}}{(2n+1)(2n+2)(2n+3)(2n+4)(2n+5)}
\]
has the following: \\
- \texttt{Term}: \texttt{'(-1)**n / ((2*n + 1)*(2*n + 2)*(2*n + 3)*(2*n + 4)*(2*n + 5))'} \\
- \texttt{Dummy variable}: \texttt{'n'} \\
- \texttt{Start}: \texttt{'2'} \\
- \texttt{Unknowns}: \texttt{[]} \\

Pay attention to special symbols like \texttt{\_symbol} (e.g., \texttt{\textbackslash{}frac{1}{2})\_n} ), which often indicate a SymPy \texttt{RisingFactorial}. Another symbol to look out for is \texttt{H\_}, which often means a SymPy \texttt{harmonic}. \\

The series is: \\
\texttt{\{latex\_string\}} \\

Structured output: \{'term': str, 'dummy\_var': str, 'start': int, 'unknowns': list[str]\}.
\end{tcolorbox}

\vspace{4mm} 

\noindent \textbf{User Message: Step 4}
\vspace{2mm} 

% step 4

($\text{\LaTeX }$ for one-shot example taken from \cite{1806.03346})

\begin{tcolorbox}
Step 4: \\
Identify the variable used in the formula. \\
If the formula is a series, focus on the variable used in the outermost summation. \\
If the formula contains nested summations or other variables, ensure you extract only the variable from the outermost summation and exclude all others. \\

For example: \\
The string 
\[
\pi = \frac{22}{7} - 24\sum_{n=2}^\infty \frac{(-1)^n}{(2n+1)(2n+2)(2n+3)(2n+4)(2n+5)}
\]
has the outermost summation variable: \texttt{'n'}. \\

Extract the variable from the formula: \\
\texttt{\{latex\_string\}}. \\

Structured output: str.
\end{tcolorbox}

\subsubsection{Utility Prompts}
After extraction, each of the code components extracted in steps 3, 4 from the above prompts went through an execution test. Faulty code was sent back to GPT-4o for correction to SymPy code that runs properly. A total of only 16 code-correction iterations were needed during a run with 847 extractions, a testament to the LLM's ability to write executable code. We are confident that this stage could be removed in future runs with minimal consequences to the size of the formula dataset. Since the number of corrections is so low, the cost of leaving this stage in is minimal. In short, the correction stage is largely insignificant when using GPT-4o to derive executable code from $\text{\LaTeX }$.

The following prompts were used in the loop for up to three iterations:

\noindent \textbf{System Message:}
\vspace{2mm} 

\begin{tcolorbox}
You are a helpful assistant tasked with extracting mathematical expressions 
from strings and rewriting them in proper SymPy format.\\
Your output must be valid Python code that can be executed without errors.\\
Always focus on processing the original string provided and ensure the response 
contains only the corrected SymPy expression, formatted as executable Python code.
\end{tcolorbox}

\vspace{4mm} 

\noindent \textbf{User Message:}
\vspace{2mm} 

(Includes the Python error message from a failed execution attempt - e.)
\begin{tcolorbox}
The last attempt was invalid SymPy code: \texttt{\{str(e)[:50]\}}.

Last attempt: \\
\texttt{\{string\}}

Task: \\
1. Extract the expression from the \textbf{original string} below. \\
2. Rewrite it in proper SymPy format as valid, executable Python code. \\
3. Only return the corrected SymPy expression, formatted as valid Python code.

Original string: \\
\texttt{\{original\_string\}}

Process the \textbf{original string} and provide the corrected SymPy expression. \\

Structured output: str.
\end{tcolorbox}

\subsection{Formula Validation}

Validating collected formulas presents a challenge. The safest way to validate a formula converges to the expected constant is by computing the formula (trying to reconstruct a proof automatically could be an interesting aspect of research in LLMs). As exemplified in \cref{appendix-engineering-future-work}, GPT-4o sometimes missed free constants or multipliers when tested, or a simple arithmetic manipulation needed for the proposed value to equal the precise value of the series or continued fraction. PSLQ \cite{PSLQ} solves this by finding the correct Möbius transformation between the constant of interest and the formula's value, while relying only on the algebraic expression picked up by the LLM.

\subsection{Possible Improvements to Formula Harvesting Method}
\label{appendix-engineering-future-work}

\begin{table}[ht!]
    \caption{A typical extraction by OpenAI's GPT-4o: The LLM extracts the partial numerator and partial denominator of a continued fraction, but fails to find the correct value of the formula it collects, which needs to be isolated through simple arithmetic; the correct value is
    $\left( \frac{2}{1 - \frac{\pi}{4} \cdot \frac{3 \cdot 3 \cdot 5 \cdot 5}{2 \cdot 4 \cdot 4 \cdot 6}} - 1 \right) = \frac{256 + 75 \pi}{256 - 75 \pi}$. $\text{\LaTeX }$ taken from \cite{nimbran2018continuedfractionspig}.}
    \label{tab:appendix-engineering-llm-example-wrong-value}
    \vskip 0.15in
    \centering
    \begin{center}
    \begin{small}
    % \begin{sc}
    \begin{tabular}{c|c}
        \toprule
       $\text{\LaTeX }$ & \texttt{\textbackslash{}frac\{\textbackslash{}pi\}\{4\} = \textbackslash{}frac\{2\}\{3\} * \textbackslash{}frac\{4\}\{3\}} \\ 
       & \texttt{* \textbackslash{}frac\{4\}\{5\} * \textbackslash{}frac\{6\}\{5\} * [ 1 - \textbackslash{}cfrac\{2\}\{25 + \textbackslash{}cfrac\{1 * 3\}\{24 + } \\
       & \texttt{\textbackslash{}cfrac\{3 * 5\}\{24 + \textbackslash{}cfrac\{5 * 7\}\{24 + ... \}\}\}\}\} ]} \\
       \midrule
        Rendered & $\frac{\pi}{4} = \frac{2}{3} \cdot \frac{4}{3} \cdot \frac{4}{5} \cdot \frac{6}{5} \cdot \left( 1 - \cfrac{2}{25 + \cfrac{1 \cdot 3}{24 + \cfrac{3 \cdot 5}{24 + \cfrac{5 \cdot 7}{24 + \dots}}}} \right) $ \\
        \midrule
        Prompt & Extraction pipeline (\cref{appendix-engineering-prompts}) with an added prompt for identifying the value. \\
        \midrule
        Output & $a_n$ = 24, $b_n$ = (2n-1)*(2n+1), value = pi / 4. \\
       \bottomrule
   \end{tabular}
    % \end{sc}
    \end{small}
    \end{center}
\end{table}

The example in \cref{tab:appendix-engineering-llm-example-wrong-value} shows that despite collecting the correct partial numerator and partial denominator functions, the LLM did not do the algebra needed to derive the stand-alone value of the continued fraction; it collected the value of the original formula, which \textit{contains} the continued fraction.

Instead of forcing the responsibility of algebra onto the LLM, we choose to use an integer-relation finding algorithm to identify the values of collected formulas: given the prior belief (of the LLM) that a formula computes a constant and a nontrivial integer relation between a formula's empirical value and the constant of interest, the integer relation found is likely the correct one. This is a practical way to validate the formulas outputted by the LLM.

In the future, elements identified in the equation could be replaced by symbols (e.g. replace a series term with the placeholder "TERM") to help the LLM do arithmetic. Equations could also be converted to symbolic language at this point and solved rigorously. Additionally, more examples could be included in the prompt, but this would add to extraction costs.

The improvements discussed above pertain to the logic of the extraction method. Now that we have some labeled data, we could also move to optimize the prompts being used.
Also, maybe a method could be devised for optimization of the formula-processing architecture. For example, perhaps it would be better to first ascertain the variable of a formula and only then move on to collecting the expression of the formula. (Our logic was that the LLM might get ``confused" in cases where a string contains multiple variables if we do not first guide it to identify the expression of interest.) 


\section{Tables of Full Results}

All of the formulas presented in this section, as well as key algorithms presented in this paper, are available in our \href{https://colab.research.google.com/drive/13EC9hwEhoA_xvEu_7p_9wbIl2QjDknqC?authuser=1#scrollTo=i-3w8tyB2_-P}{results colab}. We encourage you to try some of them yourselves.

\subsection{Formulas Unified by the Conservative Matrix Field (CMF)}

\newcolumntype{P}[1]{>{\raggedright\arraybackslash}p{#1}}

\newcolumntype{C}[1]{>{\centering\arraybackslash}p{#1}}

Unified formulas and their sources are presented in Tables ~\ref{tab:unified-formulas} and ~\ref{tab:unified-formulas-sources}, respectively.

\begin{table}[th!]
    \caption{\textbf{Formulas unified by the Conservative Matrix Field (CMF) from Fig.\ref{fig:cmf-unification}}. Clusters of formulas harvested from the literature are given in terms of their corresponding trajectory in the CMF. Dashes indicate a formula's canonical form (CF) is the same as the formula collected. ``PCF" means polynomial continued fraction. Each row is a distinct canonical form. Note that some rows contain multiple source formulas, all of which have the same canonical form. Formula sources are shown in \cref{tab:unified-formulas-sources}. 48 formulas are shown below, these are the 47 formulas captured by the CMF with the addition of a PCF found only in the CMF, index 33, which supports an example in \cref{appendix-results-cmf-examples-nontrivial-coboundary}.
    }
    \label{tab:unified-formulas}
    \vskip 0.15in
    \centering
    \begin{center}
    \begin{tabular}{ccP{5cm}cP{4cm}c}
        \toprule
        Cluster
        & Formula
        & Value
        & Canonical Form (CF)
        & CF Value \\
        
        \midrule
        
        \makecell{$(1, 1, 2)$ \\ $\delta = -0.2$}
        & 1 % pipcfs index 0
        & $PCF(2n+5, n(n+4))$
        & $\frac{8}{3\pi-8}$
        & -
        & -
        \\
        
        & 2 % pipcfs index 1
        & $PCF(2n+1, n^2)$
        & $\frac{4}{\pi}$
        & -
        & -
        \\

        & 3 % pipcfs index 2
        & $PCF(2n+3, n(n+2))$
        & $\frac{4}{\pi-2}$
        & -
        & -
        \\

        & 4 % pipcfs index 3
        &
        $PCF(- 2 (4 n + 3) (6 n^{2} + 9 n + 2), - n^{2} (2 n + 1)^{2} (4 n - 3) (4 n + 5))$
        & $\frac{10}{\pi - 4}$
        & -  
        & - 
        \\
        
        \midrule

        \makecell{$(3,1,1)$ \\ $\delta = -0.45$}
        & 5 % pipcfs index 8
        & $\sum_{k=1}^{\infty} \frac{\left(-4\right)^{k} \left(7 k - 1\right) {\binom{2 k}{k}}}{k \left(2 k - 1\right) {\binom{3 k}{k}} {\binom{6 k}{3 k}}}$
        & $-\frac{\pi}{4}$
        &
        $PCF(728 n^{3} + 638 n^{2} + 9 n - 15, 6 n (2 n - 1) (6 n - 5) (6 n - 1) (7 n - 8) (7 n + 6))$
        & $-\frac{48}{\pi}$
        \\

        \midrule

        \makecell{$(2,1,1)$ \\ $\delta = -0.48$}
        & 6 % pipcfs index 9
        & $\sum_{k=1}^{\infty} \frac{\left(-2\right)^{k} \left(6 k - 1\right)}{k \left(2 k - 1\right) {\binom{4 k}{2 k}}}$
        & $\frac{\pi}{2}$
        &
        $PCF((3 n + 1) (28 n^{2} + 16 n - 3), n (2 n - 1) (4 n - 3) (4 n - 1) (6 n - 7) (6 n + 5))$
        & $- \frac{10}{\pi}$
        \\

        & 7 % pipcfs index 10
        & $\sum_{k=0}^{\infty} \frac{\left(-2\right)^{k} \left(30 k - 7\right)}{{\binom{4 k}{2 k}}}$
        & $- \frac{32}{3} - \frac{\pi}{2}$
        & $PCF(420 n^{3} - 1028 n^{2} + 675 n - 111, n (2 n - 1) (4 n - 7) (4 n - 5) (30 n - 67) (30 n - 7))$
        & $- \frac{4662}{3 \pi + 64}$
        \\

        & 8 % pipcfs index 11
        & $\sum_{k=1}^{\infty} \frac{\left(-2\right)^{k} k \left(126 k + 29\right)}{{\binom{4 k}{2 k}}}$
        & $\frac{50}{3} + \frac{11 \pi}{2}$
        & $PCF(1764 n^{4} + 8596 n^{3} + 14767 n^{2} + 10143 n + 2053, n (n + 2)^{2} (2 n + 3) (4 n + 1) (4 n + 3) (126 n + 29) (126 n + 281))$
        & $\frac{-36530 - 3372 \pi}{-15 + \pi}$
        \\

        \midrule

        \makecell{$(1,0,0)$ \\ $\delta = -0.65$}
        & 9
        & $PCF(3 n + 5, - n (2 n - 1))$
        & $\frac{48}{-320 + 105 \pi}$
        & -
        & -
        \\

        & 10
        & $PCF(3 (n + 3), - (n + 3) (2 n - 1))$
        & $\frac{32 - 15 \pi}{-96 + 30 \pi}$
        & -
        & -
        \\

        & 11
        & $PCF(3 n + 4, - n (2 n - 1))$
        & $\frac{12}{-44 + 15 \pi}$
        & -
        & -
        \\

        & 12
        & $PCF(3 n + 7, - (n + 2) (2 n - 1))$
        & $\frac{32 - 6 \pi}{-64 + 21 \pi}$
        & -
        & -
        \\

        & 13
        & $PCF(3 n + 5, - (n + 1) (2 n - 1))$
        & $\frac{4 - 3 \pi}{-20 + 6 \pi}$
        & -
        & -
        \\

        & 14
        & $PCF(3 n + 5, - n (2 n + 1))$
        & $- \frac{4}{-48 + 15 \pi}$
        & -
        & -
        \\

        & 15
        & $PCF(3 n + 7, - (n + 3) (2 n - 1))$
        & $3 + \frac{9 \pi}{8}$
        & -
        & -
        \\

        & 16
        & $PCF(3 n + 8, - (n + 3) (2 n - 1))$
        & $- \frac{9 \pi}{-32 + 9 \pi}$
        & -
        & -
        \\

        & 17
        & $PCF(3 (n + 1), - n (2 n - 1))$
        & $\frac{4}{-8 + 3 \pi}$
        & -
        & -
        \\

        & 18
        & $PCF(3 (n + 2), - (n + 2) (2 n - 1))$
        & $\frac{8}{-8 + 3 \pi}$
        & -
        & -
        \\

        & 19
        & $PCF(3 n + 4, - n (2 n + 1))$
        & $- \frac{2}{-10 + 3 \pi}$
        & -
        & -
        \\

        & 20
        & $PCF(3 n + 4, - (n + 1) (2 n - 1))$
        & $- \frac{\pi}{-4 + \pi}$
        & -
        & -
        \\

   \end{tabular}
    \end{center}
\end{table}

\begin{table*}[th!]
    \vskip 0.15in
    \centering
    \begin{center}
    \begin{tabular}{ccP{5cm}cP{5cm}c}
        % \toprule

        & 21
        & \makecell[l]{
            $\sum_{n=0}^{\infty} \frac{n!}{\prod_{i=1}^{n} \left(2 i + 1\right)}$
        \\
            $\sum_{j=0}^{\infty} \frac{2^{j + 1}}{\left(2 j + 1\right) {\binom{2 j}{j}}}$
        \\
            $\sum_{k=1}^{\infty} \frac{2^{k}}{k {\binom{2 k}{k}}}$
        }
        & \makecell[c]{
            $\frac{\pi}{2}$
        \\
            $\pi$
        \\
            $\frac{\pi}{2}$
        }
        & $PCF(3 n + 2, - n (2 n - 1))$
        & $\frac{2}{-2 + \pi}$
        \\

        & 22
        & $PCF(3 n + 5, - (n + 2) (2 n - 1))$
        & $\frac{2 \pi + 8}{\pi}$
        & -
        & -
        \\

        & 23
        & $PCF(3 n + 5, - n (2 n + 3))$
        & $\frac{6}{-8 + 3 \pi}$
        & -
        & -
        \\

        & 24
        & $PCF(3 (n + 1), - n (2 n + 1))$
        & $- \frac{2}{-4 + \pi}$
        & -
        & -
        \\

        & 25
        & $PCF(3 (n + 1), - (n + 1) (2 n - 1))$
        & $1 + \frac{\pi}{2}$
        & -
        & -
        \\

        & 26
        & $PCF(3 n + 1, - n (2 n - 1))$
        & $\frac{2}{\pi}$
        & -
        & -
        \\

        & 27
        & $PCF(3(n + 2), - (n + 3) (2 n - 1))$
        & $\frac{15 \pi + 48}{8 + 3 \pi}$
        & -
        & -
        \\

        & 28
        & $PCF(3 n + 4, - (n + 2) (2 n - 1))$
        & $\frac{12 + 4 \pi}{\pi + 4}$
        & -
        & -
        \\

        & 29
        & $\sum_{k=1}^{\infty} \frac{4^{k} \left(12 k - 5\right)}{\left(2 k - 1\right) {\binom{4 k}{2 k}}}$
        & $\frac{3\pi}{2} + 2$
        & $PCF(240 n^{3} + 164 n^{2} - 54 n - 29, - 2 (n + 1) (2 n - 1) (4 n - 3) (4 n - 1) (12 n - 17) (12 n + 7))$
        & $\frac{-196 - 42 \pi}{4 + 3 \pi}$
        \\

        & 30
        & $PCF(3 n + 2, - (n + 1) (2 n - 1))$
        & $\frac{\pi + 4}{2 + \pi}$
        & -
        & -
        \\

        & 31
        & $PCF(3 n + 5, - (n + 3) (2 n - 1))$
        & $\frac{84 + 27 \pi}{6 \pi + 20}$
        & -
        & -
        \\

        & 32
        & $PCF(3 n^{2} + 9 n + 5, - (n + 1)^{2} (n + 3)(2 n - 1))$
        & $\frac{3 \left(2 \pi + 7\right)}{5 + 2 \pi}$
        & -
        & -
        \\

    \midrule

        \makecell{$(-1,3,3)$ \\ $\delta = -0.91$}
        & 33
        & $PCF(- 7568 n^{5} - 11664 n^{4} + 6992 n^{3} + 6036 n^{2} - 279 n - 162, - 24 n (2 n + 1) (4 n - 3) (4 n - 1) (6 n - 7) (6 n - 5) (22 n^{2} - 39 n - 1) (22 n^{2} + 49 n + 9))$
        
        Note: this formula is a representative from the CMF and is not included in the count of unified formulas. It is here as an aid to \cref{appendix-results-cmf-examples-nontrivial-coboundary}.
        & $- \frac{648}{\pi + 4}$
        & -
        & -
        \\

        & 34
        & $\sum_{k=1}^{\infty} \frac{16^{k} (22 k^{2} - 17 k + 3) {\binom{4 k}{2 k}}}{k (4 k - 3) (4 k - 1) {\binom{3 k}{k}} {\binom{6 k}{3 k}}}$
        & $2\pi$
        & $PCF(3784 n^{4} + 156 n^{3} - 1942 n^{2} + 261 n + 45, - 24 n (2 n - 3) (4 n - 3) (4 n - 1) (6 n - 5) (6 n - 1) (11 n - 14) (11 n + 8))$
        & $\frac{96}{\pi}$
        \\

    \midrule

        \makecell{$(1,1,1)$ \\ $\delta = -1$}
        & 35
        & $PCF(10, (2 n - 1)^{2})$
        & $\frac{5 \pi + 16}{\pi}$
        & -
        & -
        \\

        & 36
        & \makecell[l]{$\sum_{n=1}^{\infty} 36 (-1)^{n} \cdot$ \\ $\cdot \frac{1}{(4 n^{2} + 5) ((2 n + 2)^{2} + 5) (- 2 n + (2 n + 1)^{3} - 1)}$}
        & $\frac{7}{9} - \frac{\pi}{4}$
        & $PCF(14 (4 n^{2} + 8 n + 9), (2 n + 1)^{2} (4 n^{2} + 5) (4 n^{2} + 8 n + 9))$
        & $\frac{36}{-28 + 9 \pi}$
        \\

        & 37
        & $\sum_{n=1}^{\infty} \frac{\left(-1\right)^{n - 1}}{n \left(2 n + 1\right) \left(2 n + 2\right) \left(2 n + 3\right) \left(4 n - 2\right)}$
        & $\frac{5}{36} - \frac{\pi}{24}$
        & $PCF(10, (2 n - 1) (2 n + 3))$
        & $- \frac{6}{-10 + 3 \pi}$
        \\

        & 38
        & \makecell[l]{$\sum_{n=2}^{\infty} \frac{(-1)^{n}}{(2 n + 1) (2 n + 2)} \cdot$ \\ $\cdot \frac{1}{(2 n + 3) (2 n + 4)(2 n + 5)}$}
        & $\frac{11}{84} - \frac{\pi}{24}$
        & $PCF(10, (2 n + 3) (2 n + 7))$
        & $\frac{658 - 210 \pi}{-330 + 105 \pi}$
        \\

        & 39
        & $PCF(6, (2 n - 1)^{2})$
        & $3 + \pi$
        & -
        & -
        \\
        
        & 40
        & \makecell[l]{
            $\sum_{n=1}^{\infty} \frac{(-1)^{n + 1}}{n (n + 1) (2 n + 1)}$
        }
        & \makecell[c]{
            $\pi - 3$
        }
        & $PCF(6, (2 n + 1)^{2})$
        & $\frac{1}{-3 + \pi}$
        \\
        % \bottomrule
    \end{tabular}
    \end{center}
\end{table*}

\begin{table*}[th!]
    \vskip 0.15in
    \centering
    \begin{center}
    % \begin{small}
    % \begin{sc}
    \begin{tabular}{ccP{5cm}cP{5cm}c}
        % \toprule

        & 41
        & $\sum_{n=1}^{\infty} \frac{(-1)^{n - 1}}{(2 n - 1) (2 n + 1) (2 n + 3)}$
        & $\frac{\pi}{8} - \frac{1}{3}$
        & $PCF(6, (2 n - 1) (2 n + 3))$
        & $\frac{3 \pi}{-8 + 3 \pi}$
        \\

        & 42
        & \makecell[l]{
            $\sum_{j=0}^{\infty} (\frac{(-1)^{j}}{(2 j + 2) (2 j + 3)} + \frac{(-1)^{j}}{(2 j + 1) (2 j + 2)})$
        \\
            $\sum_{k=1}^{\infty} \frac{(-1)^{k}}{(2 k - 1) (2 k + 1)}$
            }
        & \makecell[l]{
            $\frac{\pi}{2} - 1$
        \\
            $\frac{1}{2} - \frac{\pi}{4}$
        }
        & $PCF(4, (2 n - 1) (2 n + 1))$
        & $\frac{2 + \pi}{-2 + \pi}$
        \\

        & 43
        & $\sum_{k=0}^{\infty} \frac{(-1)^{k}}{2 k + 1}$
        & $\frac{\pi}{4}$
        & $PCF(2, (2 n - 1)^{2})$
        & $\frac{4}{\pi} + 1$
        \\

        \midrule

        \makecell{$(0,0,1)$ \\ $\delta = -1.00$}
        & 44
        & $PCF(1, n (n + 1))$
        & $\frac{2}{-2 + \pi}$
        & -
        & -
        \\

        & 45
        & $PCF(2, n^{2})$
        & $- \frac{2}{-4 + \pi}$
        & -
        & -
        \\

        & 46
        & \makecell[l]{
            $\sum_{n=0}^{\infty} \frac{2^{- 4 n - 4} {\binom{2 n}{n}} {\binom{2 n + 2}{n + 1}}}{(n + 1) (2 n + 1)}$
        \\
            $\sum_{n=0}^{\infty} \frac{4^{- 2 n} {\binom{2 n}{n}}^{2}}{(n + 1)^{2}}$
        }
        & \makecell[l]{
            $\frac{2}{\pi} - \frac{1}{2}$
        \\
            $\frac{16}{\pi} - 4$
        }
        & $PCF(8 n^{2} + 4 n + 5, - 4 n^{2} (2 n - 1)^{2})$
        & $- \frac{4}{-4 + \pi}$
        \\

        & 47
        & $\sum_{n=1}^{\infty} \frac{16^{n}}{n^{2} (2 n + 1)^{2} {\binom{2 n}{n}}^{2}}$
        & $4\pi - 12$
        & $PCF(8 n^{2} + 12 n + 9, - 4 n^{2} (2 n + 1)^{2})$
        & $\frac{1}{-3 + \pi}$
        \\

        & 48
        & $\sum_{n=1}^{\infty} \frac{({(\frac{1}{2})}_{n})^{2}}{(n + 1)!^{2}}$
        & $\frac{16}{\pi} - 5$
        & $PCF(8 n^{2} + 20 n + 17, - 4 (n + 1)^{2} (2 n + 1)^{2})$
        & $\frac{-16 + 4 \pi}{-16 + 5 \pi}$
        \\
         
        \bottomrule
    \end{tabular}
    % \end{sc}
    % \end{small}
    \end{center}
\end{table*}

\FloatBarrier

\begin{table*}[th!]
    \caption{arXiv sources of formulas unified by the CMF from \cref{tab:unified-formulas}.}
    \label{tab:unified-formulas-sources}
    \vskip 0.15in
    \centering
    \begin{center}
    % \begin{small}
    % \begin{sc}
    \begin{tabular}{cC{5cm}|cC{5cm}}
        \toprule
        & arXiv sources & & arXiv sources \\
        \midrule
        
        1
        & \cite{1907.00205}
        &
        25
        & \cite{1907.00205}
        \\

        2
        & \cite{1907.00205}
        &
        26
        & \cite{1907.00205}
        \\

        3
        & \cite{1907.00205}
        &
        27
        & \cite{1907.00205}
        \\

        4
        & \cite{2308.11829}
        &
        28
        & \cite{1907.00205}
        \\

        5
        & \cite{1407.8465},\cite{2307.03086}
        &
        29
        & \cite{2204.08275}
        \\

        6
        & \cite{2204.08275}
        &
        30
        & \cite{1907.00205}
        \\

        7
        & \cite{2204.08275}
        &
        31
        & \cite{1907.00205}
        \\

        8
        & \cite{2204.08275}
        &
        32
        & \cite{Arnold_BenDavid2024}
        \\

        9
        & \cite{1907.00205}
        &
        33
        & CMF representative
        \\

        10
        &\cite{1907.00205}
        &
        34
        & \cite{2310.03699}
        \\

        11
        &\cite{1907.00205}
        &
        35
        & \cite{1806.03346}
        \\

        12
        &\cite{1907.00205}
        &
        36
        & \cite{2405.11134}
        \\

        13
        &\cite{1907.00205}
        &
        37
        & \cite{1806.03346}
        \\

        14
        &\cite{1907.00205}
        &
        38
        & \cite{1806.03346}
        \\

        15
        &\cite{1907.00205}
        &
        39
        & \cite{1806.03346},\cite{1907.00205}
        \\

        16
        &\cite{1907.00205}
        &
        40
        & \cite{1601.03180},\cite{1806.08411},\cite{1912.03214},\cite{2206.07174}
        \\

        17
        &\cite{1907.00205}
        &
        41
        & \cite{1806.03346}
        \\

        18
        &\cite{1907.00205}
        &
        42
        & \cite{1806.03346},\cite{1906.09629},\cite{2305.14367}
        \\

        19
        &\cite{1907.00205}
        &
        43
        & \cite{1806.03346},\cite{1907.00205},\cite{2007.10161},\cite{2104.12412},\cite{2206.07174},\cite{2301.12945},\cite{2312.17402},\cite{2404.15210},\cite{2405.11134},\cite{2405.12248},\cite{2407.19885},\cite{2408.12096},\cite{2409.06658}
        \\

        20
        &\cite{1907.00205}
        &
        44
        & \cite{1806.03346}
        \\

        21
        & \cite{0707.2124},\cite{1806.03346},\cite{2010.05610},\cite{2112.00622}
        &
        45
        & \cite{1806.03346}
        \\

        22
        &\cite{1907.00205}
        &
        46
        & \cite{2111.10998},\cite{2204.04535}
        \\

        23
        &\cite{1907.00205}
        &
        47
        & \cite{1708.04269}
        \\

        24
        &\cite{1907.00205}
        &
        48
        & \cite{1804.08210}
        \\
        
    \bottomrule
    \end{tabular}
    % \end{sc}
    % \end{small}
    \end{center}
\end{table*}

\FloatBarrier

\subsection{Formulas Not Yet Unified by the Conservative Matrix Field (CMF)}
The following formulas (\cref{tab:collected-formulas}) were automatically collected, validated and clustered, resulting in rigorous equivalences. We believe finding the correct trajectories in the CMF will prove these formulas are also contained in it. arXiv sources are shown in \cref{tab:collected-formulas-sources}.

\begin{table}[th!]
    \caption{\textbf{Additional formulas for $\pi$ that were automatically harvested from arXiv and clustered rigorously}. This table complements \cref{tab:unified-formulas}; the formulas are mutually exclusive and were collected using the same harvesting pipeline. These formulas were then connected among themselves as described in \cref{appendix-algs-graph-growing}, resulting in clusters. Two notable examples are the $\delta = -0.29$ cluster which contains a Ramanujan 1914 \cite{Ramanujan1914} series along with a modern formula \cite{sun2020newseriespowerspi} (discussed in \cref{section-results}), and the $\delta=-0.74$ cluster which contains a BBP-type formula \cite{BBP1996} and an equivalent version that is a fold by 2 of the BBP.
    }
    \label{tab:collected-formulas}
    \vskip 0.15in
    \centering
    \begin{center}
    % \begin{small}
    % \begin{sc}
    \begin{tabular}{ccP{5cm}cP{5cm}c}
        \toprule
        Cluster
        & % Index
        & Formula
        & Value 
        & Canonical Form (CF) 
        & CF Value 
        \\

        \midrule

        $\delta = -0.29$
        & 1
        & \makecell[l]{$\sum_{k=0}^\infty \frac{(-1)^{k} 882^{-2k-1} (21460 k + 1123)}{k!^3}\cdot$ \\ $\cdot {(\frac{1}{4})}_{(k)} {(\frac{1}{2})}_{(k)} {(\frac{3}{4})}_{(k)}$}
        & $\frac{4}{\pi}$
        & $PCF(534215282560 n^{4} + 1630601631968 n^{3} + 1686512782328 n^{2} + 618081838666 n + 27955409115, 24893568 n^{3} (2 n + 1) (4 n + 1) (4 n + 3) (21460 n - 20337) (21460 n + 22583))$
        & $\frac{239018472}{-3528 + 1123 \pi}$
        \\

        & 2
        & \makecell[l]{$\sum_{k=0}^{\infty} \frac{(-199148544)^{- k} {\binom{2 k}{k}}^{2} {\binom{4 k}{2 k}}}{(k + 1) (2 k - 1) (4 k - 1)}\cdot$ \\ $\cdot(1424799848 k^{2} + 1533506502 k + $ \\ $108685699)$}
        & $\frac{341446000}{\pi}$
        & $PCF(35468306308982528 n^{5} + 180047738533689024 n^{4} + 332745102731042192 n^{3} + 272631301503072468 n^{2} + 89876772716256332 n + 5411146610376015, 24893568 n^{2} (n + 1) (2 n - 1) (4 n - 1) (4 n + 1) (1424799848 n^{2} - 1316093194 n - 20955) (1424799848 n^{2} + 4383106198 n + 3066992049))$
        & $\frac{1047212167162854000}{-341446000 + 108685699 \pi}$
        \\

        \midrule

        $\delta = -0.49$
        & 3
        & \makecell[l]{$\sum_{k=0}^{\infty} (-82944)^{- k} {\binom{2 k}{k + 1}}^{2} {\binom{4 k}{2 k}}\cdot$ \\ $\cdot (2475740800 k^{2} + 4950772932 k +$ \\ $ 2475031103)$}
        & $- \frac{2238840}{\pi}$
        & $PCF(25589256908800 n^{6} + 307024146260352 n^{5} + 1483653045985888 n^{4} + 3682571629531512 n^{3} + 4933895619830194 n^{2} + 3372826099775973 n + 919254248208450, 10368 n^{2} (n + 2)^{3} (2 n + 3) (4 n + 5) (4 n + 7) (2475740800 n^{2} + 4950772932 n + 2475031103) (2475740800 n^{2} + 14853736132 n + 22279540167))$
        & $\frac{9653638716124064686080}{-2063314944 + 660102989 \pi}$
        \\

        & 4
        & \makecell[l]{$\sum_{k=0}^{\infty} (-82944)^{- k} k^{3}{\binom{2 k}{k}}^{2} {\binom{4 k}{2 k}}$ \\ $\cdot (2428400 k^{2} - 5044368 k +$ \\ $ 2584321)$}
        & $\frac{243}{5 \pi}$
        & $PCF(25099942400 n^{5} + 73089143552 n^{4} + 68107981504 n^{3} + 16786867552 n^{2} - 3892088262 n - 560080521, 10368 n^{3} (2 n + 3) (4 n + 5) (4 n + 7) (2428400 n^{2} - 5044368 n + 2584321) (2428400 n^{2} + 4669232 n + 2209185))$
        & $\frac{64935193276800}{-279936 + 52745 \pi}$
        \\

        \bottomrule
   \end{tabular}
    % \end{sc}
    % \end{small}
    \end{center}
\end{table}

\begin{table}[th!]
    \label{tab:collected-formulas2}
    \vskip 0.15in
    \centering
    \begin{center}
    % \begin{small}
    % \begin{sc}
    \begin{tabular}{ccP{5cm}cP{5cm}c}
        $\delta = -0.50$
        & 5
        & \makecell[l]{$\sum_{k=0}^{\infty} \frac{4096^{- k} (42 k^{2} + 81 k + 38) {\binom{2 k}{k}}^{3}}{(k + 1)^{3}}$}
        & $-512 + \frac{1728}{\pi}$
        & $PCF(21840 n^{5} + 63120 n^{4} + 58976 n^{3} + 15348 n^{2} - 2925 n - 550, - 512 n^{3} (2 n - 1)^{3} (42 n^{2} - 87 n + 44) (42 n^{2} + 81 n + 38))$
        & $\frac{1026}{-27 + 8 \pi}$
        \\

        & 6
        & \makecell[l]{$\sum_{k=0}^{\infty} \frac{4096^{- k} {\binom{2 k}{k}}^{3}}{(k + 1)^{2} (2 k - 1)} \cdot$ \\ $\cdot (420 k^{2} + 992 k + 551)$}
        & $- \frac{1728}{\pi}$
        & $PCF(218400 n^{5} + 1599440 n^{4} + 4564144 n^{3} + 6337720 n^{2} + 4282754 n + 1126485, - 512 n (n + 1)^{2} (2 n - 1) (2 n + 1)^{2} (420 n^{2} + 152 n - 21) (420 n^{2} + 1832 n + 1963))$
        & $\frac{3392064}{-1728 + 551 \pi}$
        \\

        & 7
        & \makecell[l]{$\sum_{k=0}^{\infty} \frac{4096^{- k} {\binom{2 k}{k}}^{3}}{(k + 1)^{2}} \cdot$ \\ $\cdot (56 k^{2} + 118 k + 61)$}
        & $\frac{192}{\pi}$
        & $PCF(29120 n^{5} + 206288 n^{4} + 567664 n^{3} + 758432 n^{2} + 493160 n + 125163, - 512 n (n + 1)^{2} (2 n + 1)^{3} (56 n^{2} + 6 n - 1) (56 n^{2} + 230 n + 235))$
        & $ - \frac{45120}{-192 + 61 \pi}$
        \\

        & 8
        & \makecell[l]{$\sum_{n=0}^{\infty} \frac{4096^{- n} {\binom{2 n}{n}}^{3}}{(2 n - 1)^{2}} \cdot$ \\ $(2 n + 1) (6 n + 1) (14 n - 3)$}
        & $-\frac{8}{\pi}$
        & $PCF(43680 n^{5} + 128624 n^{4} + 121776 n^{3} + 31176 n^{2} - 6934 n - 1305, - 512 n^{3} (2 n - 1)^{2} (2 n + 3) (6 n - 5) (6 n + 7) (14 n - 17) (14 n + 11))$
        & $ - \frac{1848}{-8 + 3 \pi}$
        \\

        & 9 % a ton of sources.
        & \makecell[l]{$\sum_{n=0}^{\infty} \frac{2^{- 12 n} (42 n + 5) (2 n)!^{3}}{n!^{6}}$}
        & $\frac{16}{\pi}$
        & $PCF(21840 n^{4} + 67952 n^{3} + 73008 n^{2} + 29508 n + 2607, - 512 n^{3} (2 n + 1)^{3} (42 n - 37) (42 n + 47))$
        & $- \frac{752}{-16 + 5 \pi}$
        \\

        & 10
        & \makecell[c]{$\sum_{k=0}^{\infty} \frac{4096^{- k} k^{2} {\binom{2 k}{k}}^{3}}{(k + 1)^{3}} \cdot $ \\ $\cdot (78162 k^{2} + 145175 k +$ \\ $ 64431)$}
        & \makecell[c]{$-1321984 +$ \\ $ \frac{4153360}{\pi}$}
        & $PCF(40644240 n^{7} + 603553472 n^{6} + 3740699664 n^{5} + 12504251940 n^{4} + 24268130979 n^{3} + 27259698627 n^{2} + 16367928552 n + 4050187164, - 512 n^{2} (n + 2)^{5} (2 n + 3)^{3} (78162 n^{2} + 145175 n + 64431) (78162 n^{2} + 457823 n + 667429))$
        & \textbf{*}
        \\

        &
        &
        &
        & \textbf{*} $(-153284543710986240 + 48789345068384256 \pi) / (-2126520320 + 676891779 \pi)$
        &
        \\
        
        & 11
        & $\sum_{k=0}^{\infty} 4096^{- k} k (210 k^{2} - 5 k + 1) {\binom{2 k}{k}}^{3}$
        & $\frac{4}{3 \pi}$
        & $PCF(109200 n^{5} + 764320 n^{4} + 2076080 n^{3} + 2737956 n^{2} + 1761115 n + 444325, - 512 n (n + 1)^{2} (2 n + 3)^{3} (210 n^{2} - 5 n + 1) (210 n^{2} + 835 n + 831))$
        & \makecell[c]{$ - \frac{22975488}{-1024 + 309 \pi}$}
        \\

        & 12
        & $\sum_{k=0}^{\infty} 4096^{- k} k^{2} (504 k^{2} - 314 k - 11) {\binom{2 k}{k}}^{3}$
        & $\frac{4}{3 \pi}$
        & $PCF(262080 n^{5} + 1419280 n^{4} + 2902704 n^{3} + 2814416 n^{2} + 1289208 n + 220475, - 512 n^{2} (n + 1) (2 n + 3)^{3} (504 n^{2} - 314 n - 11) (504 n^{2} + 1702 n + 1377))$
        & \makecell[c]{$- \frac{76142592}{-2048 + 537 \pi}$}
        \\

        & 13
        & \makecell[l]{
            $\sum_{k=0}^{\infty} 4096^{- k} k^{3} {\binom{2 k}{k}}^{3} \cdot $ \\ $\cdot (198 k^{2} - 425 k + 210)$
        \\
            $\sum_{k=0}^{\infty} 4096^{- k} k^{2} {\binom{2 k}{k}}^{3} \cdot $ \\ $\cdot (5544 k^{3} - 11900 k^{2} + 5880 k)$
        }
        & \makecell[c]{
            $- \frac{1}{21 \pi}$
        \\
            $-\frac{4}{3 \pi}$
        }
        & $PCF(102960 n^{5} + 299344 n^{4} + 276000 n^{3} + 61356 n^{2} - 22843 n - 4600, - 512 n^{3} (2 n + 3)^{3} (198 n^{2} - 425 n + 210) (198 n^{2} + 367 n + 152))$
        & \makecell[c]{$- \frac{2101248}{-512 + 357 \pi}$}
        \\
        % \bottomrule5
   \end{tabular}
    % \end{sc}
    % \end{small}
    \end{center}
\end{table}

\begin{table}[th!]
    \label{tab:collected-formulas3}
    \vskip 0.15in
    \centering
    \begin{center}
    % \begin{small}
    % \begin{sc}
    \begin{tabular}{ccP{5cm}cP{5cm}c}
        \midrule
        $\delta = -0.64$
        & 14
        & \makecell[l]{$\sum_{k=0}^{\infty} \frac{1458^{- k} {\binom{2 k}{k}}^{2} {\binom{3 k}{k}}}{(k + 1) (2 k - 1) (3 k - 1)} \cdot$ \\ $(1524 k^{2} + 899 k + 263)$}
        & $\frac{3375}{4\pi}$
        & $PCF(397764 n^{5} + 1757115 n^{4} + 2799174 n^{3} + 2058041 n^{2} + 755034 n + 130504, - 243 n^{2} (n + 1) (2 n - 1) (3 n - 1) (3 n + 1) (1524 n^{2} - 2149 n + 888) (1524 n^{2} + 3947 n + 2686))$
        & $- \frac{9065250}{-3375 + 1052 \pi}$
        \\

        & 15
        & \makecell[l]{$\sum_{n=0}^{\infty} \frac{1458^{- n} }{n!^{5}} \cdot$ \\ $(15 n + 2) (2 n)! (3 n)!$}
        & $\frac{27}{4 \pi}$
        & $PCF(3915 n^{4} + 12132 n^{3} + 13047 n^{2} + 5354 n + 520, - 243 n^{3} (2 n + 1) (3 n + 1) (3 n + 2) (15 n - 13) (15 n + 17))$
        & $- \frac{918}{-27 + 8 \pi}$
        \\

        \midrule
        $\delta = -0.71$        
        & 16
        & \makecell[l]{$\sum_{q=0}^{\infty} \frac{3 \cdot 2^{- 4 q} {\binom{2 q}{q}}}{2 q + 1}$}
        & $\pi$
        & $PCF(20 n^{2} + 44 n + 25, - 8 n (2 n + 1)^{3})$
        & $\frac{\pi}{-3 + \pi}$
        \\

        & 17
        & \makecell[l]{$\sum_{n=0}^{\infty} - \frac{2 \cdot 16^{- n} (6 n + 5) {\binom{2 n}{n}}}{(2 n - 1) (2 n + 1) (2 n + 3)}$}
        & $\frac{1}{\pi}$
        & $PCF(120 n^{3} + 460 n^{2} + 514 n + 189, - 8 n (2 n - 1) (2 n + 1) (2 n + 3) (6 n - 1) (6 n + 11))$
        & $- \frac{33 \pi}{-10 + 3 \pi}$
        \\

        \midrule
        $\delta = -0.74$
        & 18
        & \makecell[l]{$\sum_{n=0}^{\infty} (-1)^{n} 4^{- n} (\frac{1}{4 n + 3} + $ \\ $\frac{2}{4 n + 2} + \frac{2}{4 n + 1})$}
        & $\pi$
        & $PCF(1920 n^{5} + 1696 n^{4} - 968 n^{3} - 826 n^{2} + 77 n + 63, 4 (2 n - 1)^{2} (4 n - 3)^{2} (4 n - 1)^{2} (20 n^{2} - 59 n + 43) (20 n^{2} + 21 n + 5))$
        & $\frac{15 \pi + 160}{\pi}$
        \\

        & 19
        & \makecell[l]{$\sum_{k=0}^{\infty} 16^{- k} (- \frac{1}{4 (8 k + 7)} - \frac{1}{2 (8 k + 6)} - $ \\ $ \frac{1}{2 (8 k + 5)} + \frac{1}{8 k + 3} + $ \\ $\frac{2}{8 k + 2} + \frac{2}{8 k + 1})$}
        & $\pi$
        & $PCF(34225520640 n^{11} + 30215766016 n^{10} - 41244688384 n^{9} - 37765251072 n^{8} + 13730807808 n^{7} + 13978398720 n^{6} - 922759680 n^{5} - 1528522048 n^{4} - 35302928 n^{3} + 39542964 n^{2} + 1694619 n + 67095, - 16 (4 n - 3)^{2} (4 n - 1)^{2} (8 n - 7)^{2} (8 n - 5)^{2} (8 n - 3)^{2} (8 n - 1)^{2} (30720 n^{5} - 216832 n^{4} + 605920 n^{3} - 837860 n^{2} + 573329 n - 155325) (30720 n^{5} + 90368 n^{4} + 100064 n^{3} + 51292 n^{2} + 11905 n + 981))$
        & $\frac{-753408 + 309015 \pi}{\pi}$
        \\

        \midrule
        
        $\delta = -0.75$
        & 20
        & \makecell[l]{$\sum_{k=0}^{\infty} \frac{256^{- k} k (6 k - 1) {\binom{2 k}{k}}^{3}}{(2 k - 1)^{3}}$}
        & $\frac{1}{2\pi}$
        & $PCF((2 n + 3) (120 n^{3} + 460 n^{2} + 562 n + 217), - 32 n (n + 1)^{2} (2 n + 1)^{3} (6 n - 1) (6 n + 11))$
        & $- \frac{176}{-16 + 5 \pi}$
        \\

        & 21
        & \makecell[l]{$\sum_{k=0}^{\infty} \frac{256^{- k} (12 k^{2} - 1) {\binom{2 k}{k}}^{3}}{(2 k - 1)^{2}}$}
        & $\frac{2}{\pi}$
        & $PCF(480 n^{5} + 1296 n^{4} + 1088 n^{3} + 208 n^{2} - 94 n - 21, - 32 n^{3} (2 n - 1)^{2} (2 n + 1) (12 n^{2} - 24 n + 11) (12 n^{2} + 24 n + 11))$
        & $ - \frac{22}{-2 + \pi}$
        \\

        & 22
        & \makecell[l]{$\sum_{n=0}^{\infty} \frac{256^{- n} n^{2} {\binom{2 n}{n}}^{3}}{(2 n - 3) (2 n - 1)}$}
        & $\pi$
        & $PCF(40 n^{3} + 148 n^{2} + 166 n + 55, - 32 n^{2} (n + 1) (2 n - 1) (2 n + 3)^{2})$
        & $\frac{72}{-8 + 3 \pi}$
        \\
        % \bottomrule5
   \end{tabular}
    % \end{sc}
    % \end{small}
    \end{center}
\end{table}

\begin{table}[th!]
    \label{tab:collected-formulas4}
    \vskip 0.15in
    \centering
    \begin{center}
    % \begin{small}
    % \begin{sc}
    \begin{tabular}{ccP{5cm}cP{5cm}c}
        & 23
        & \makecell[l]{$\sum_{n=0}^{\infty} 2^{- 8 n} (6 n + 1) {\binom{2 n}{n}}^{3}$}
        & $\frac{4}{\pi}$
        & $PCF(240 n^{4} + 736 n^{3} + 792 n^{2} + 336 n + 39, - 32 n^{3} (2 n + 1)^{3} (6 n - 5) (6 n + 7))$
        & $- \frac{28}{-4 + \pi}$
        \\

        & 24
        & \makecell[l]{$\sum_{k=0}^{\infty} 256^{- k} (k^{2} + 1) {\binom{2 k}{k}}^{3} \cdot$ \\ $(192 k^{2} - 626 k - 103)$}
        & $- \frac{1373}{3 \pi}$
        & $PCF(7680 n^{7} + 1840 n^{6} - 41056 n^{5} - 84504 n^{4} - 105848 n^{3} - 90009 n^{2} - 37922 n - 4370, - 32 n^{3} (2 n + 1)^{3} (n^{2} - 2 n + 2) (n^{2} + 2 n + 2) (192 n^{2} - 1010 n + 715) (192 n^{2} - 242 n - 537))$
        & $\frac{1474602}{-1373 + 309 \pi}$
        \\

        \midrule
        $\delta = -0.80$        
        & 25
        & \makecell[l]{$\sum_{k=0}^{\infty} \frac{(-1024)^{- k} (8 k^{2} - 2 k - 1) {\binom{2 k}{k}}^{2} {\binom{4 k}{2 k}}}{(k + 1) (2 k - 1) (4 k - 1)}$}
        & $-\frac{16}{\pi}$
        & $PCF(96 n^{3} + 464 n^{2} + 634 n + 261, 128 n^{2} (n + 1) (2 n + 1) (4 n - 1) (4 n + 5))$
        & $- \frac{80}{-16 + 5 \pi}$
        \\

        & 26
        & \makecell[l]{$\sum_{k=0}^{\infty} \frac{(-1024)^{- k} (40 k^{2} - 2 k - 1) {\binom{2 k}{k}}^{2} {\binom{4 k}{2 k}}}{(2 k - 1) (4 k - 1)}$}
        & $-\frac{4}{\pi}$
        & $PCF(3840 n^{5} + 13248 n^{4} + 14608 n^{3} + 4676 n^{2} - 644 n - 165, 128 n^{3} (2 n - 1) (4 n - 1) (4 n + 1) (40 n^{2} - 82 n + 41) (40 n^{2} + 78 n + 37))$
        & $\frac{148}{-4 + \pi}$
        \\

        & 27
        & \makecell[l]{$\sum_{k=0}^{\infty} (-1024)^{- k} (20 k + 3) {\binom{2 k}{k}}^{2} {\binom{4 k}{2 k}}$}
        & $\frac{8}{\pi}$
        & $PCF(1920 n^{4} + 6368 n^{3} + 7288 n^{2} + 3146 n + 315, 128 n^{3} (2 n + 1) (4 n + 1) (4 n + 3) (20 n - 17) (20 n + 23))$
        & $\frac{552}{-8 + 3 \pi}$
        \\

        \midrule
            
        $\delta = -0.86$
        & 28
        & \makecell[l]{$\sum_{k=0}^{\infty} \frac{648^{- k} (1903 k^{2} + 114 k + 41) {\binom{2 k}{k}}^{2} {\binom{4 k}{2 k}}}{(k + 1) (2 k - 1) (4 k - 1)}$}
        & $\frac{343}{2\pi}$
        & $PCF(215039 n^{5} + 720798 n^{4} + 810302 n^{3} + 328875 n^{2} + 34877 n + 8700, - 81 n^{2} (n + 1) (2 n - 1) (4 n - 1) (4 n + 1) (1903 n^{2} - 3692 n + 1830) (1903 n^{2} + 3920 n + 2058))$
        & $- \frac{705894}{-343 + 82 \pi}$
        \\
        
        & 29
        & \makecell[l]{$\sum_{k=0}^{\infty} 648^{- k} (7 k + 1) {\binom{2 k}{k}}^{2} {\binom{4 k}{2 k}}$}
        & $\frac{9}{2 \pi}$
        & $PCF(791 n^{4} + 2374 n^{3} + 2482 n^{2} + 1007 n + 105, - 81 n^{3} (2 n + 1) (4 n + 1) (4 n + 3) (7 n - 6) (7 n + 8))$
        & $- \frac{216}{-9 + 2 \pi}$
        \\

        \midrule
        $\delta = -1.00$
        & 30
        & \makecell[l]{$\sum_{k=0}^{\infty} \frac{(-64)^{- k} (4 k - 1) {\binom{2 k}{k}}^{3}}{(2 k - 1)^{3}}$}
        & $\frac{2}{\pi}$
        & $PCF((4 n + 1) (28 n^{2} + 14 n - 5), 8 n^{3} (2 n - 1)^{3} (4 n - 5) (4 n + 3))$
        & $- \frac{6}{-2 + \pi}$
        \\

        & 31
        & \makecell[l]{$\sum_{k=0}^{\infty} \frac{(-64)^{- k} (4 k + 1) {\binom{2 k}{k}}^{3}}{(k + 1) (2 k - 1)}$}
        & $- \frac{4}{\pi}$
        & $PCF((4 n + 3) (20 n^{2} + 30 n + 7), 8 n^{2} (n + 1) (2 n - 1) (2 n + 1)^{2} (4 n - 3) (4 n + 5))$
        & $- \frac{20}{-4 + \pi}$
        \\

        \midrule
        $\delta = -1.00$
        & 32
        & \makecell[l]{$\sum_{n=1}^{\infty} \frac{1}{(2 n - 1) (2 n + 1) (4 n - 1) (4 n + 1)}$}
        & $- \frac{1}{2} + \frac{\pi}{6}$
        & $PCF(2 (16 n^{2} + 16 n + 23), - (4 n - 1)^{2} (4 n + 1)^{2})$
        & $\frac{3 + \pi}{-3 + \pi}$
        \\

        & 33
        & \makecell[l]{$\sum_{n=1}^{\infty} \frac{3}{n (n + 1) (4 n + 1) (4 n + 3)}$}
        & $\frac{19}{3} - 2\pi$
        & $PCF(2 (16 n^{2} + 32 n + 35), - (4 n + 1)^{2} (4 n + 3)^{2})$
        & $- \frac{9}{-19 + 6 \pi}$
        \\

        \bottomrule
   \end{tabular}
    % \end{sc}
    % \end{small}
    \end{center}
\end{table}

\begin{table}[th!]
    \label{tab:collected-formulas-singletons}
    \vskip 0.15in
    \centering
    \begin{center}
    % \begin{small}
    % \begin{sc}
    \begin{tabular}{ccP{5cm}cP{5cm}c}
        \midrule
        $\delta = -0.30$        
        & 34
        & \makecell[l]{$\sum_{k=0}^{\infty} \frac{2^{- k} (25 k - 3)}{{\binom{3 k}{k}}}$}
        & $\frac{\pi}{2}$
        & $PCF(725 n^{3} - 1462 n^{2} + 909 n - 168, - 3 n (2 n - 1) (3 n - 5) (3 n - 4) (25 n - 53) (25 n - 3))$
        & $\frac{1008}{\pi}$
        \\

        \midrule
        $\delta = -0.34$        
        & 35
        & \makecell[l]{$\sum_{n=1}^{\infty} \frac{(2 n + 1) (3 n + 1) (14 n + 11) {\binom{2 n}{n}}}{(2 n - 1) (4 n + 1)^{2} (4 n + 3) {\binom{4 n}{2 n}}^{2}}$}
        & $\frac{1}{\pi}$
        & $PCF(21840 n^{6} + 133472 n^{5} + 326016 n^{4} + 405404 n^{3} + 268671 n^{2} + 88937 n + 11250, - 2 (n + 1) (2 n - 1) (2 n + 1) (2 n + 3) (3 n - 2) (3 n + 4) (4 n - 1) (4 n + 1)^{2} (4 n + 3) (14 n - 3) (14 n + 25))$
        & $\frac{-3600 - 900 \pi}{-10 + 3 \pi}$
        \\

        \midrule % base 10 decimal BBP
        $\delta = -0.59$        
        & 36
        & \makecell[l]{$\sum_{k=0}^{\infty} (-1)^{k} 2^{- 10 k} (\frac{1}{64 (10 k + 9)} $ \\ $ - \frac{1}{16 (10 k + 7)} - \frac{1}{16 (10 k + 5)} $ \\ $ - \frac{1}{10 k + 3} + \frac{4}{10 k + 1} $ \\ $ - \frac{1}{64 (4 k + 3)} - \frac{1}{2 (4 k + 1)})$}
        & $\pi$
        & $PCF(2684352000000000 n^{13} + 3194384000000000 n^{12} - 2711998880000000 n^{11} - 3852244008000000 n^{10} + 824570409120000 n^{9} + 1620913668000000 n^{8} - 71379351936000 n^{7} - 292825355691200 n^{6} - 2477111745952 n^{5} + 22491764548320 n^{4} + 289095761160 n^{3} - 678390090570 n^{2} + 1624760937 n + 5512490235, 1024 (2 n - 1)^{2} (4 n - 3)^{2} (4 n - 1)^{2} (10 n - 9)^{2} (10 n - 7)^{2} (10 n - 3)^{2} (10 n - 1)^{2} (8200000 n^{6} - 68150000 n^{5} + 234882000 n^{4} - 429729800 n^{3} + 440200812 n^{2} - 239398031 n + 54004235) (8200000 n^{6} + 30250000 n^{5} + 45382000 n^{4} + 35326200 n^{3} + 14990012 n^{2} + 3271617 n + 285021))$
        & $\frac{808034535 \pi + 84056113152}{5 \pi}$
        \\

        \midrule
        $\delta = -0.61$        
        & 37
        & \makecell[l]{$\sum_{k=0}^{\infty} \frac{(-64)^{- k} (28 k^{2} + 10 k + 1) {\binom{2 k}{k}}^{5}}{(6 k + 1) {\binom{3 k}{k}} {\binom{6 k}{3 k}}}$}
        & $\frac{3}{\pi}$
        & $PCF(23296 n^{7} + 125248 n^{6} + 276544 n^{5} + 320864 n^{4} + 207200 n^{3} + 72320 n^{2} + 12192 n + 801, 24 n^{3} (2 n + 1)^{5} (6 n - 1) (6 n + 1) (28 n^{2} - 46 n + 19) (28 n^{2} + 66 n + 39))$
        & $\frac{117}{-3 + \pi}$
        \\

        \midrule % base 16 hex hexadecimal BBP
        $\delta = -0.68$        
        & 38
        & \makecell[l]{$\sum_{i=0}^{\infty} 16^{- i} (- \frac{1}{8 i + 6} - \frac{1}{8 i + 5} - $ \\ $ \frac{2}{8 i + 4} + \frac{4}{8 i + 1})$}
        & $\pi$
        & $PCF(1044480 n^{6} + 1191424 n^{5} - 5184 n^{4} - 344600 n^{3} - 60306 n^{2} + 21793 n + 4827, - 16 (2 n - 1)^{2} (4 n - 1)^{2} (8 n - 7)^{2} (8 n - 3)^{2} (120 n^{2} - 329 n + 225) (120 n^{2} + 151 n + 47))$
        & $\frac{987 \pi + 12032}{\pi}$
        \\

    \bottomrule
    \end{tabular}
    % \end{sc}
    % \end{small}
    \end{center}
\end{table}

\begin{table}[th!]
    \label{tab:collected-formulas6}
    \vskip 0.15in
    \centering
    \begin{center}
    % \begin{small}
    % \begin{sc}
    \begin{tabular}{ccP{5cm}cP{5cm}c}
        \midrule
        $\delta = -0.68$        
        & 39
        & \makecell[l]{$\sum_{k=0}^{\infty} 16^{- k} (- \frac{1}{8 k + 7} + \frac{4}{8 k + 4} $ \\ $ + \frac{4}{8 k + 3} + \frac{8}{8 k + 2})$}
        & $2\pi$
        & $PCF(4177920 n^{7} + 2643968 n^{6} - 3223296 n^{5} - 1855840 n^{4} + 515184 n^{3} + 339842 n^{2} + 17793 n - 4401, - 16 (2 n - 1)^{2} (4 n - 3)^{2} (8 n - 5)^{2} (8 n - 1)^{2} (480 n^{3} - 2068 n^{2} + 2925 n - 1353) (480 n^{3} + 812 n^{2} + 413 n + 65))$
        & $\frac{-16640}{\pi} + 975$
        \\
        
        \midrule
        $\delta = -0.72$        
        & 40
        & \makecell[l]{$\sum_{n=0}^{\infty} \frac{(\frac{5}{3})^{-6n} (\frac{1}{6})_{(n)} (\frac{1}{3})_{(n)} (\frac{2}{3})_{(n)} (\frac{5}{6})_{(n)}}{(2n + 1) (\frac{1}{2})_{(n)} n!^{3}} \cdot$ \\ $\cdot (133 n^{2} + 79 n + 6)$}
        & $\frac{625}{32\pi}$
        & $PCF(8700328 n^{6} + 44125426 n^{5} + 88113105 n^{4} + 87218230 n^{3} + 43677997 n^{2} + 9729354 n + 582120, - 281250 n^{3} (2 n + 1) (3 n + 1) (3 n + 2) (6 n + 1) (6 n + 5) (133 n^{2} - 187 n + 60) (133 n^{2} + 345 n + 218))$
        & $- \frac{12262500}{-625 + 192 \pi}$
        \\

        \midrule
        $\delta = -0.75$        
        & 41
        & \makecell[l]{$\sum_{n=0}^{\infty} \frac{2^{-6n} (\frac{1}{10})_{(n)} (\frac{3}{10})_{(n)} (\frac{7}{10})_{(n)} (\frac{9}{10})_{(n)}}{(2n + 1) (\frac{1}{2})_{(n)} (1)_{(n)}^{3}}\cdot$ \\ $\cdot (2100 n^{2} + 1160 n + 63)$}
        & $\frac{200}{\pi}$
        & $PCF(1365000000 n^{6} + 6886000000 n^{5} + 13628850000 n^{4} + 13283880000 n^{3} + 6461275900 n^{2} + 1346342040 n + 61108047, - 320000 n^{3} (2 n + 1) (10 n + 1) (10 n + 3) (10 n + 7) (10 n + 9) (2100 n^{2} - 3040 n + 1003) (2100 n^{2} + 5360 n + 3323))$
        & $- \frac{125609400}{-200 + 63 \pi}$
        \\

        \midrule
        $\delta = -0.78$
        & 42
        & \makecell[l]{$\sum_{j=1}^{\infty} \frac{(-16)^{j} (40 j^{2} - 12 j - 1) {\binom{2 j}{j}}}{j (2 j - 1)^{2} (4 j + 1) {\binom{4 j}{2 j}}^{2}}$}
        & $8 - 4\pi$
        & $PCF(3 (2560 n^{6} + 8192 n^{5} + 7872 n^{4} + 1568 n^{3} - 1048 n^{2} - 364 n - 15), 8 n (2 n - 1)^{2} (2 n + 1) (4 n - 3) (4 n - 1)^{2} (4 n + 1) (40 n^{2} - 92 n + 51) (40 n^{2} + 68 n + 27))$
        & $- \frac{54}{-2 + \pi}$
        \\

        \midrule
        $\delta = -0.89$        
        & 43
        & \makecell[l]{$\sum_{n=0}^{\infty} \frac{2^{- 8 n} 3^{- 6 n} {\binom{4 n}{2 n}} {\binom{6 n}{3 n}} {\binom{6 n}{4 n}}}{(6 n - 5)^{2}} \cdot$ \\ $\cdot (25 - 108 n^{2})$}
        & $\frac{3}{5 \pi}$
        & $PCF(699840 n^{6} + 2239488 n^{5} + 2302992 n^{4} + 557280 n^{3} - 385164 n^{2} - 189000 n - 12925, - 2592 n^{3} (2 n - 1) (6 n - 5)^{2} (6 n + 5)^{2} (108 n^{2} - 216 n + 83) (108 n^{2} + 216 n + 83))$
        & $- \frac{155625}{-3 + 5 \pi}$
        \\

        \midrule
        $\delta = -1.00$        
        & 44
        & \makecell[l]{$\sum_{n=1}^{\infty} \frac{(-1)^{n - 1}}{(2 n - 1) (2 n + 1) (2 n + 3) (2 n + 5) (2 n + 7)}$}
        & $\frac{\pi}{96} - \frac{2}{63}$
        & $PCF(10, (2 n - 1) (2 n + 7))$
        & $\frac{224 - 105\pi}{320 - 105\pi}$
        \\

        \midrule
        $\delta = -1.00$        
        & 45
        & \makecell[l]{$\sum_{n=1}^{\infty} \frac{(-1)^{n - 1}}{(2 n - 1) (2 n + 1) (2 n + 3) (2 n + 5)}$}
        & $\frac{\pi}{24} - \frac{11}{90}$
        & $PCF(8, (2 n - 1) (2 n + 5))$
        & $\frac{20 - 15\pi}{44 - 15\pi}$
        \\

        \midrule
        $\delta = -1.00$        
        & 46
        & \makecell[l]{$\sum_{n=1}^{\infty} \frac{1}{n (2 n - 1) (4 n - 3)}$}
        & $\frac{\pi}{3}$
        & $ PCF(16 n^{3} + 4 n^{2} + 10 n + 1, - n^{2} (2 n - 1)^{2} (4 n - 3)^{2})$
        & $\frac{3}{\pi}$
        \\

        \bottomrule
   \end{tabular}
    % \end{sc}
    % \end{small}
    \end{center}
\end{table}

\begin{table*}[th!]
    \caption{arXiv sources of formulas from \cref{tab:collected-formulas}.}
    \label{tab:collected-formulas-sources}
    \vskip 0.15in
    \centering
    \begin{center}
    % \begin{small}
    % \begin{sc}
    \begin{tabular}{cC{5cm}|cC{5cm}}
        \toprule
        & arXiv sources & & arXiv sources \\
        \midrule

        1
        & \cite{1610.04839},\cite{1611.02217}
        &
        17
        & \cite{0807.0872},\cite{0911.2415},\cite{1103.3893},\cite{1804.00394}
        \\

        2
        &\cite{1911.05456}
        &
        18
        & \cite{0807.0872},\cite{1603.08540}
        \\

        3
        &\cite{2110.03651}
        &
        19
        & \cite{1906.09629}
        \\

        4
        &\cite{2110.03651}
        &
        20
        & \cite{2203.16047}
        \\

        5
        &\cite{1911.05456},\cite{2110.03651}
        &
        21
        & \cite{1808.04717},\cite{1911.05456},\cite{2203.16047}
        \\

        6
        &\cite{1911.05456}
        &
        22
        & \cite{2212.09965}
        \\

        7
        &\cite{1911.05456}
        &
        23
        & \cite{0708.3307},\cite{0807.0872},\cite{0906.5560},\cite{1104.0392},\cite{1302.5984},\cite{1303.6228},\cite{1510.02575},\cite{1512.04608},\cite{1604.01106},\cite{1802.01506},\cite{1808.03213},\cite{1909.10294},\cite{2109.09877},\cite{2110.03651},\cite{2203.16047}
        \\

        8
        &\cite{2212.09965}
        &
        24
        & \cite{2110.03651}
        \\

        9
        &\cite{0807.0872},\cite{1103.6022},\cite{1104.0392},\cite{1410.5514},\cite{1512.04608},\cite{1604.01106},\cite{1609.07276},\cite{2104.12412},\cite{2110.03651},\cite{2210.07238},\cite{2307.03086},\cite{2410.19289},\cite{math/0503345}
        &
        25
        & \cite{1911.05456}
        \\

        10
        &\cite{2110.03651}
        &
        26
        & \cite{1911.05456}
        \\

        11
        &\cite{2110.03651}
        &
        27
        & \cite{0704.2438},\cite{0805.2788},\cite{1004.4623},\cite{1104.0392},\cite{2001.08104},\cite{2210.07238},\cite{2305.00498},\cite{2307.03086},\cite{math/0503345}
        \\

        12
        &\cite{2110.03651}
        &
        28
        & \cite{1911.05456}
        \\

        13
        &\cite{2110.03651}
        &
        29
        & \cite{1104.3856},\cite{1302.5984},\cite{1309.1140},\cite{1804.02695},\cite{2001.08104}
        \\

        14
        &\cite{1911.05456}
        &
        30
        & \cite{1808.04717},\cite{1911.05456},\cite{2105.05567},\cite{2203.16047}
        \\

        15
        &\cite{0807.0872},\cite{1512.04608},\cite{2001.08104}
        &
        31
        & \cite{2105.05567},\cite{2203.16047}
        \\

        16
        &\cite{0807.0872},\cite{0911.2415},\cite{1103.3893},\cite{1804.00394}
        &
        32
        & \cite{2203.09465},\cite{2405.11134}
        \\


    \bottomrule
    \end{tabular}
    % \end{sc}
    % \end{small}
    \end{center}
\end{table*}

\begin{table*}[th!]
    \label{tab:collected-formulas-sources2}
    \vskip 0.15in
    \centering
    \begin{center}
    % \begin{small}
    % \begin{sc}
    \begin{tabular}{cC{5cm}|cC{5cm}}

            33
        & \cite{2206.07174}
        &
        40
        & \cite{2206.07174}
        \\

        34
        & \cite{2206.07174}
        &
        41
        & \cite{2206.07174}
        \\

        35
        & \cite{2206.07174}
        &
        42
        & \cite{2206.07174}
        \\

        36
        & \cite{2206.07174}
        &
        43
        & \cite{2206.07174}
        \\

        37
        & \cite{2206.07174}
        &
        44
        & \cite{2206.07174}
        \\

        38
        & \cite{2206.07174}
        &
        45
        & \cite{2206.07174}
        \\

        39
        & \cite{2206.07174}
        &
        46
        & \cite{2206.07174}
        \\
    
    \bottomrule
    \end{tabular}
    % \end{sc}
    % \end{small}
    \end{center}
\end{table*}

\end{document}
