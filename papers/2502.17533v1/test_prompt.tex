\documentclass{article}
\usepackage{amsmath}
\usepackage{listings}
\usepackage{xcolor}
\usepackage{tcolorbox}

\tcbset{
  colframe=black!75,    % Box border color
  colback=gray!10,      % Box background color
  sharp corners=all,    % Use sharp corners for alignment
  boxrule=0.5mm,        % Border thickness
  left=4mm,             % Left padding
  right=4mm,            % Right padding
  top=2mm,              % Top padding
  bottom=2mm,           % Bottom padding
}

\begin{document}

\section*{Prompt}

\begin{tcolorbox}
You are a model that classifies whether a LaTeX string is a formula that can be rearranged to calculate the constant {constant}. 
Specifically, we are interested in continued fractions, series, or products. Keep all answers concise and accurate.
\end{tcolorbox}

\vspace{4mm} % Space between sections

\noindent \textbf{User Message:}
\vspace{2mm} % Small space before the box

\begin{tcolorbox}
Is this a continued fraction, series, or product that can be rearranged to calculate the constant {constant}? {latex_string}
\end{tcolorbox}

\end{document}


% \documentclass{article}
% \usepackage{amsmath}
% \usepackage{listings}
% \usepackage{xcolor}

% \lstset{
%   basicstyle=\ttfamily\small,
%   backgroundcolor=\color{gray!10},
%   frame=single,
%   breaklines=true,
%   tabsize=2,
% }

% \begin{document}

% \section*{Prompt}

% \begin{lstlisting}
% You are a model that classifies whether a LaTeX string is a formula that can be rearranged to calculate the constant {constant}. 
% Specifically, we are interested in continued fractions, series, or products. Keep all answers concise and accurate.
% \end{lstlisting}

% \noindent User Message: \\
% \begin{lstlisting}
% Is this a continued fraction, series, or product that can be rearranged to calculate the constant {constant}? {latex_string}
% \end{lstlisting}

% \end{document}
