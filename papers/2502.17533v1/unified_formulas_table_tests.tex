\documentclass{article}
\usepackage{tabularx}
\usepackage{array}
\usepackage{makecell}

\begin{document}


\documentclass{article}
\usepackage{array}

\begin{document}

% Set a column type for automatic wrapping
\newcolumntype{P}[1]{>{\raggedright\arraybackslash}p{#1}} % Left-aligned with wrapping

\begin{tabular}{|P{5cm}|P{2cm}|P{2cm}|} % All columns have a width of 5cm
\hline
This is some long text that will wrap automatically if it exceeds the column width. & Another long text that also wraps automatically. & A third column with wrapped text. and an equation $x = y= z = dd= f= f= f= f= f=f f$ \\
\hline
Short text & Another short text & Some other short text \\
\hline
\end{tabular}

\end{document}



% \begin{tabularx}{\textwidth}{|X|X|X|}
% \hline
% Cell 1 & Cell 2 & Cell 3 \\
% \hline
% This is some text. \\ New line within cell 1. & Here is text \\ for cell 2. & Another cell. \\
% \hline
% \end{tabularx}



% \begin{tabular}{c|c}
% \hline
% \makecell{Line 1 \\ Line 2} & Another cell \\
% \hline

% \makecell{ 1 \\ 2 \\ 3} & 5 \\
% \end{tabular}



% \renewcommand{\arraystretch}{1.5} % Increase row height for multiline cells

% \begin{tabular}{|m{3cm}|m{3cm}|m{3cm}|}
% \hline
% Cell 1 & Cell 2 & Cell 3 \\
% \hline
% This is some text. \\ New line here\\ and another new line \\ and another one. wow a lot of new lines!! & Multiline cell. \\ Works well. & Another example. \\
% \hline
% \end{tabular}



\end{document}
