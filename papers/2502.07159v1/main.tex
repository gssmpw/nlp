\documentclass[11pt]{article}
\usepackage[utf8]{inputenc}

\usepackage{style}

\newcommand{\RO}[1]{\textcolor{blue}{{ RO: #1}}}
\newcommand{\des}[1]{\mathrm{DES}[#1]}
\DeclareMathOperator{\polylog}{polylog}


\title{Pseudorandomness Properties of Random Reversible Circuits}
 \author{
William Gay${}^*$
\and
 William He${}^*$
 \and
 Nicholas Kocurek${}^*$
 \and 
Ryan O'Donnell\thanks{Computer Science Department, Carnegie Mellon University. \texttt{\{wrhe,odonnell\}@cs.cmu.edu}. Supported in part by ARO grant W911NF2110001. \texttt{\{wgay,nkocurek\}@andrew.cmu.edu}}.}
\date{\small\today}

\begin{document}
\maketitle
\allowdisplaybreaks
\begin{abstract}
    Motivated by practical concerns in cryptography, we study pseudorandomness properties of permutations on $\{0,1\}^n$ computed by random circuits made from reversible $3$-bit gates (permutations on $\{0,1\}^3$). Our main result is that a random circuit of depth $\sqrt{n} \cdot \wt{O}(k^3)$, with each layer consisting of $\Theta(n)$ random gates in a fixed two-dimensional nearest-neighbor architecture, yields approximate $k$-wise independent permutations. 

    Our result can be seen as a particularly simple/practical block cipher construction that gives provable statistical security against attackers with access to $k$~input-output pairs within few rounds. 
    
    The main technical component of our proof consists of two parts:
    \begin{enumerate}
        \item We show that the Markov chain on $k$-tuples of $n$-bit strings induced by a single random $3$-bit one-dimensional nearest-neighbor gate has spectral gap at least $1/n \cdot \wt{O}(k)$. Then we infer that a random circuit with layers of random gates in a fixed \textit{one-dimensional} gate architecture yields approximate $k$-wise independent permutations of $\{0,1\}^n$ in depth $n\cdot \wt{O}(k^2)$
        \item We show that if the $n$ wires are layed out on a \textit{two-dimensional} lattice of bits, then repeatedly alternating applications of approximate $k$-wise independent permutations of $\{0,1\}^{\sqrt n}$ to the rows and columns of the lattice yields an approximate $k$-wise independent permutation of $\{0,1\}^n$ in small depth.
    \end{enumerate}
    Our work improves on the original work of Gowers~\cite{gowers1996almost}, who showed a gap of $1/\poly(n,k)$ for one random gate (with non-neighboring inputs); and, on subsequent work~\cite{hoory2005simple,brodsky2008simple} improving the gap to $\Omega(1/n^2k)$ in the same setting.
\end{abstract}

\newpage

\tableofcontents
\section{Introduction}
\label{sec:introduction}
The business processes of organizations are experiencing ever-increasing complexity due to the large amount of data, high number of users, and high-tech devices involved \cite{martin2021pmopportunitieschallenges, beerepoot2023biggestbpmproblems}. This complexity may cause business processes to deviate from normal control flow due to unforeseen and disruptive anomalies \cite{adams2023proceddsriftdetection}. These control-flow anomalies manifest as unknown, skipped, and wrongly-ordered activities in the traces of event logs monitored from the execution of business processes \cite{ko2023adsystematicreview}. For the sake of clarity, let us consider an illustrative example of such anomalies. Figure \ref{FP_ANOMALIES} shows a so-called event log footprint, which captures the control flow relations of four activities of a hypothetical event log. In particular, this footprint captures the control-flow relations between activities \texttt{a}, \texttt{b}, \texttt{c} and \texttt{d}. These are the causal ($\rightarrow$) relation, concurrent ($\parallel$) relation, and other ($\#$) relations such as exclusivity or non-local dependency \cite{aalst2022pmhandbook}. In addition, on the right are six traces, of which five exhibit skipped, wrongly-ordered and unknown control-flow anomalies. For example, $\langle$\texttt{a b d}$\rangle$ has a skipped activity, which is \texttt{c}. Because of this skipped activity, the control-flow relation \texttt{b}$\,\#\,$\texttt{d} is violated, since \texttt{d} directly follows \texttt{b} in the anomalous trace.
\begin{figure}[!t]
\centering
\includegraphics[width=0.9\columnwidth]{images/FP_ANOMALIES.png}
\caption{An example event log footprint with six traces, of which five exhibit control-flow anomalies.}
\label{FP_ANOMALIES}
\end{figure}

\subsection{Control-flow anomaly detection}
Control-flow anomaly detection techniques aim to characterize the normal control flow from event logs and verify whether these deviations occur in new event logs \cite{ko2023adsystematicreview}. To develop control-flow anomaly detection techniques, \revision{process mining} has seen widespread adoption owing to process discovery and \revision{conformance checking}. On the one hand, process discovery is a set of algorithms that encode control-flow relations as a set of model elements and constraints according to a given modeling formalism \cite{aalst2022pmhandbook}; hereafter, we refer to the Petri net, a widespread modeling formalism. On the other hand, \revision{conformance checking} is an explainable set of algorithms that allows linking any deviations with the reference Petri net and providing the fitness measure, namely a measure of how much the Petri net fits the new event log \cite{aalst2022pmhandbook}. Many control-flow anomaly detection techniques based on \revision{conformance checking} (hereafter, \revision{conformance checking}-based techniques) use the fitness measure to determine whether an event log is anomalous \cite{bezerra2009pmad, bezerra2013adlogspais, myers2018icsadpm, pecchia2020applicationfailuresanalysispm}. 

The scientific literature also includes many \revision{conformance checking}-independent techniques for control-flow anomaly detection that combine specific types of trace encodings with machine/deep learning \cite{ko2023adsystematicreview, tavares2023pmtraceencoding}. Whereas these techniques are very effective, their explainability is challenging due to both the type of trace encoding employed and the machine/deep learning model used \cite{rawal2022trustworthyaiadvances,li2023explainablead}. Hence, in the following, we focus on the shortcomings of \revision{conformance checking}-based techniques to investigate whether it is possible to support the development of competitive control-flow anomaly detection techniques while maintaining the explainable nature of \revision{conformance checking}.
\begin{figure}[!t]
\centering
\includegraphics[width=\columnwidth]{images/HIGH_LEVEL_VIEW.png}
\caption{A high-level view of the proposed framework for combining \revision{process mining}-based feature extraction with dimensionality reduction for control-flow anomaly detection.}
\label{HIGH_LEVEL_VIEW}
\end{figure}

\subsection{Shortcomings of \revision{conformance checking}-based techniques}
Unfortunately, the detection effectiveness of \revision{conformance checking}-based techniques is affected by noisy data and low-quality Petri nets, which may be due to human errors in the modeling process or representational bias of process discovery algorithms \cite{bezerra2013adlogspais, pecchia2020applicationfailuresanalysispm, aalst2016pm}. Specifically, on the one hand, noisy data may introduce infrequent and deceptive control-flow relations that may result in inconsistent fitness measures, whereas, on the other hand, checking event logs against a low-quality Petri net could lead to an unreliable distribution of fitness measures. Nonetheless, such Petri nets can still be used as references to obtain insightful information for \revision{process mining}-based feature extraction, supporting the development of competitive and explainable \revision{conformance checking}-based techniques for control-flow anomaly detection despite the problems above. For example, a few works outline that token-based \revision{conformance checking} can be used for \revision{process mining}-based feature extraction to build tabular data and develop effective \revision{conformance checking}-based techniques for control-flow anomaly detection \cite{singh2022lapmsh, debenedictis2023dtadiiot}. However, to the best of our knowledge, the scientific literature lacks a structured proposal for \revision{process mining}-based feature extraction using the state-of-the-art \revision{conformance checking} variant, namely alignment-based \revision{conformance checking}.

\subsection{Contributions}
We propose a novel \revision{process mining}-based feature extraction approach with alignment-based \revision{conformance checking}. This variant aligns the deviating control flow with a reference Petri net; the resulting alignment can be inspected to extract additional statistics such as the number of times a given activity caused mismatches \cite{aalst2022pmhandbook}. We integrate this approach into a flexible and explainable framework for developing techniques for control-flow anomaly detection. The framework combines \revision{process mining}-based feature extraction and dimensionality reduction to handle high-dimensional feature sets, achieve detection effectiveness, and support explainability. Notably, in addition to our proposed \revision{process mining}-based feature extraction approach, the framework allows employing other approaches, enabling a fair comparison of multiple \revision{conformance checking}-based and \revision{conformance checking}-independent techniques for control-flow anomaly detection. Figure \ref{HIGH_LEVEL_VIEW} shows a high-level view of the framework. Business processes are monitored, and event logs obtained from the database of information systems. Subsequently, \revision{process mining}-based feature extraction is applied to these event logs and tabular data input to dimensionality reduction to identify control-flow anomalies. We apply several \revision{conformance checking}-based and \revision{conformance checking}-independent framework techniques to publicly available datasets, simulated data of a case study from railways, and real-world data of a case study from healthcare. We show that the framework techniques implementing our approach outperform the baseline \revision{conformance checking}-based techniques while maintaining the explainable nature of \revision{conformance checking}.

In summary, the contributions of this paper are as follows.
\begin{itemize}
    \item{
        A novel \revision{process mining}-based feature extraction approach to support the development of competitive and explainable \revision{conformance checking}-based techniques for control-flow anomaly detection.
    }
    \item{
        A flexible and explainable framework for developing techniques for control-flow anomaly detection using \revision{process mining}-based feature extraction and dimensionality reduction.
    }
    \item{
        Application to synthetic and real-world datasets of several \revision{conformance checking}-based and \revision{conformance checking}-independent framework techniques, evaluating their detection effectiveness and explainability.
    }
\end{itemize}

The rest of the paper is organized as follows.
\begin{itemize}
    \item Section \ref{sec:related_work} reviews the existing techniques for control-flow anomaly detection, categorizing them into \revision{conformance checking}-based and \revision{conformance checking}-independent techniques.
    \item Section \ref{sec:abccfe} provides the preliminaries of \revision{process mining} to establish the notation used throughout the paper, and delves into the details of the proposed \revision{process mining}-based feature extraction approach with alignment-based \revision{conformance checking}.
    \item Section \ref{sec:framework} describes the framework for developing \revision{conformance checking}-based and \revision{conformance checking}-independent techniques for control-flow anomaly detection that combine \revision{process mining}-based feature extraction and dimensionality reduction.
    \item Section \ref{sec:evaluation} presents the experiments conducted with multiple framework and baseline techniques using data from publicly available datasets and case studies.
    \item Section \ref{sec:conclusions} draws the conclusions and presents future work.
\end{itemize}
\section{Overview}

\revision{In this section, we first explain the foundational concept of Hausdorff distance-based penetration depth algorithms, which are essential for understanding our method (Sec.~\ref{sec:preliminary}).
We then provide a brief overview of our proposed RT-based penetration depth algorithm (Sec.~\ref{subsec:algo_overview}).}



\section{Preliminaries }
\label{sec:Preliminaries}

% Before we introduce our method, we first overview the important basics of 3D dynamic human modeling with Gaussian splatting. Then, we discuss the diffusion-based 3d generation techniques, and how they can be applied to human modeling.
% \ZY{I stopp here. TBC.}
% \subsection{Dynamic human modeling with Gaussian splatting}
\subsection{3D Gaussian Splatting}
3D Gaussian splatting~\cite{kerbl3Dgaussians} is an explicit scene representation that allows high-quality real-time rendering. The given scene is represented by a set of static 3D Gaussians, which are parameterized as follows: Gaussian center $x\in {\mathbb{R}^3}$, color $c\in {\mathbb{R}^3}$, opacity $\alpha\in {\mathbb{R}}$, spatial rotation in the form of quaternion $q\in {\mathbb{R}^4}$, and scaling factor $s\in {\mathbb{R}^3}$. Given these properties, the rendering process is represented as:
\begin{equation}
  I = Splatting(x, c, s, \alpha, q, r),
  \label{eq:splattingGA}
\end{equation}
where $I$ is the rendered image, $r$ is a set of query rays crossing the scene, and $Splatting(\cdot)$ is a differentiable rendering process. We refer readers to Kerbl et al.'s paper~\cite{kerbl3Dgaussians} for the details of Gaussian splatting. 



% \ZY{I would suggest move this part to the method part.}
% GaissianAvatar is a dynamic human generation model based on Gaussian splitting. Given a sequence of RGB images, this method utilizes fitted SMPLs and sampled points on its surface to obtain a pose-dependent feature map by a pose encoder. The pose-dependent features and a geometry feature are fed in a Gaussian decoder, which is employed to establish a functional mapping from the underlying geometry of the human form to diverse attributes of 3D Gaussians on the canonical surfaces. The parameter prediction process is articulated as follows:
% \begin{equation}
%   (\Delta x,c,s)=G_{\theta}(S+P),
%   \label{eq:gaussiandecoder}
% \end{equation}
%  where $G_{\theta}$ represents the Gaussian decoder, and $(S+P)$ is the multiplication of geometry feature S and pose feature P. Instead of optimizing all attributes of Gaussian, this decoder predicts 3D positional offset $\Delta{x} \in {\mathbb{R}^3}$, color $c\in\mathbb{R}^3$, and 3D scaling factor $ s\in\mathbb{R}^3$. To enhance geometry reconstruction accuracy, the opacity $\alpha$ and 3D rotation $q$ are set to fixed values of $1$ and $(1,0,0,0)$ respectively.
 
%  To render the canonical avatar in observation space, we seamlessly combine the Linear Blend Skinning function with the Gaussian Splatting~\cite{kerbl3Dgaussians} rendering process: 
% \begin{equation}
%   I_{\theta}=Splatting(x_o,Q,d),
%   \label{eq:splatting}
% \end{equation}
% \begin{equation}
%   x_o = T_{lbs}(x_c,p,w),
%   \label{eq:LBS}
% \end{equation}
% where $I_{\theta}$ represents the final rendered image, and the canonical Gaussian position $x_c$ is the sum of the initial position $x$ and the predicted offset $\Delta x$. The LBS function $T_{lbs}$ applies the SMPL skeleton pose $p$ and blending weights $w$ to deform $x_c$ into observation space as $x_o$. $Q$ denotes the remaining attributes of the Gaussians. With the rendering process, they can now reposition these canonical 3D Gaussians into the observation space.



\subsection{Score Distillation Sampling}
Score Distillation Sampling (SDS)~\cite{poole2022dreamfusion} builds a bridge between diffusion models and 3D representations. In SDS, the noised input is denoised in one time-step, and the difference between added noise and predicted noise is considered SDS loss, expressed as:

% \begin{equation}
%   \mathcal{L}_{SDS}(I_{\Phi}) \triangleq E_{t,\epsilon}[w(t)(\epsilon_{\phi}(z_t,y,t)-\epsilon)\frac{\partial I_{\Phi}}{\partial\Phi}],
%   \label{eq:SDSObserv}
% \end{equation}
\begin{equation}
    \mathcal{L}_{\text{SDS}}(I_{\Phi}) \triangleq \mathbb{E}_{t,\epsilon} \left[ w(t) \left( \epsilon_{\phi}(z_t, y, t) - \epsilon \right) \frac{\partial I_{\Phi}}{\partial \Phi} \right],
  \label{eq:SDSObservGA}
\end{equation}
where the input $I_{\Phi}$ represents a rendered image from a 3D representation, such as 3D Gaussians, with optimizable parameters $\Phi$. $\epsilon_{\phi}$ corresponds to the predicted noise of diffusion networks, which is produced by incorporating the noise image $z_t$ as input and conditioning it with a text or image $y$ at timestep $t$. The noise image $z_t$ is derived by introducing noise $\epsilon$ into $I_{\Phi}$ at timestep $t$. The loss is weighted by the diffusion scheduler $w(t)$. 
% \vspace{-3mm}

\subsection{Overview of the RTPD Algorithm}\label{subsec:algo_overview}
Fig.~\ref{fig:Overview} presents an overview of our RTPD algorithm.
It is grounded in the Hausdorff distance-based penetration depth calculation method (Sec.~\ref{sec:preliminary}).
%, similar to that of Tang et al.~\shortcite{SIG09HIST}.
The process consists of two primary phases: penetration surface extraction and Hausdorff distance calculation.
We leverage the RTX platform's capabilities to accelerate both of these steps.

\begin{figure*}[t]
    \centering
    \includegraphics[width=0.8\textwidth]{Image/overview.pdf}
    \caption{The overview of RT-based penetration depth calculation algorithm overview}
    \label{fig:Overview}
\end{figure*}

The penetration surface extraction phase focuses on identifying the overlapped region between two objects.
\revision{The penetration surface is defined as a set of polygons from one object, where at least one of its vertices lies within the other object. 
Note that in our work, we focus on triangles rather than general polygons, as they are processed most efficiently on the RTX platform.}
To facilitate this extraction, we introduce a ray-tracing-based \revision{Point-in-Polyhedron} test (RT-PIP), significantly accelerated through the use of RT cores (Sec.~\ref{sec:RT-PIP}).
This test capitalizes on the ray-surface intersection capabilities of the RTX platform.
%
Initially, a Geometry Acceleration Structure (GAS) is generated for each object, as required by the RTX platform.
The RT-PIP module takes the GAS of one object (e.g., $GAS_{A}$) and the point set of the other object (e.g., $P_{B}$).
It outputs a set of points (e.g., $P_{\partial B}$) representing the penetration region, indicating their location inside the opposing object.
Subsequently, a penetration surface (e.g., $\partial B$) is constructed using this point set (e.g., $P_{\partial B}$) (Sec.~\ref{subsec:surfaceGen}).
%
The generated penetration surfaces (e.g., $\partial A$ and $\partial B$) are then forwarded to the next step. 

The Hausdorff distance calculation phase utilizes the ray-surface intersection test of the RTX platform (Sec.~\ref{sec:RT-Hausdorff}) to compute the Hausdorff distance between two objects.
We introduce a novel Ray-Tracing-based Hausdorff DISTance algorithm, RT-HDIST.
It begins by generating GAS for the two penetration surfaces, $P_{\partial A}$ and $P_{\partial B}$, derived from the preceding step.
RT-HDIST processes the GAS of a penetration surface (e.g., $GAS_{\partial A}$) alongside the point set of the other penetration surface (e.g., $P_{\partial B}$) to compute the penetration depth between them.
The algorithm operates bidirectionally, considering both directions ($\partial A \to \partial B$ and $\partial B \to \partial A$).
The final penetration depth between the two objects, A and B, is determined by selecting the larger value from these two directional computations.

%In the Hausdorff distance calculation step, we compute the Hausdorff distance between given two objects using a ray-surface-intersection test. (Sec.~\ref{sec:RT-Hausdorff}) Initially, we construct the GAS for both $\partial A$ and $\partial B$ to utilize the RT-core effectively. The RT-based Hausdorff distance algorithms then determine the Hausdorff distance by processing the GAS of one object (e.g. $GAS_{\partial A}$) and set of the vertices of the other (e.g. $P_{\partial B}$). Following the Hausdorff distance definition (Eq.~\ref{equation:hausdorff_definition}), we compute the Hausdorff distance to both directions ($\partial A \to \partial B$) and ($\partial B \to \partial A$). As a result, the bigger one is the final Hausdorff distance, and also it is the penetration depth between input object $A$ and $B$.


%the proposed RT-based penetration depth calculation pipeline.
%Our proposed methods adopt Tang's Hausdorff-based penetration depth methods~\cite{SIG09HIST}. The pipeline is divided into the penetration surface extraction step and the Hausdorff distance calculation between the penetration surface steps. However, since Tang's approach is not suitable for the RT platform in detail, we modified and applied it with appropriate methods.

%The penetration surface extraction step is extracting overlapped surfaces on other objects. To utilize the RT core, we use the ray-intersection-based PIP(Point-In-Polygon) algorithms instead of collision detection between two objects which Tang et al.~\cite{SIG09HIST} used. (Sec.~\ref{sec:RT-PIP})
%RT core-based PIP test uses a ray-surface intersection test. For purpose this, we generate the GAS(Geometry Acceleration Structure) for each object. RT core-based PIP test takes the GAS of one object (e.g. $GAS_{A}$) and a set of vertex of another one (e.g. $P_{B}$). Then this computes the penetrated vertex set of another one (e.g. $P_{\partial B}$). To calculate the Hausdorff distance, these vertex sets change to objects constructed by penetrated surface (e.g. $\partial B$). Finally, the two generated overlapped surface objects $\partial A$ and $\partial B$ are used in the Hausdorff distance calculation step.
% !TEX root =  ../main.tex
\section{Background on causality and abstraction}\label{sec:preliminaries}

This section provides the notation and key concepts related to causal modeling and abstraction theory.

\spara{Notation.} The set of integers from $1$ to $n$ is $[n]$.
The vectors of zeros and ones of size $n$ are $\zeros_n$ and $\ones_n$.
The identity matrix of size $n \times n$ is $\identity_n$. The Frobenius norm is $\frob{\mathbf{A}}$.
The set of positive definite matrices over $\reall^{n\times n}$ is $\pd^n$. The Hadamard product is $\odot$.
Function composition is $\circ$.
The domain of a function is $\dom{\cdot}$ and its kernel $\ker$.
Let $\mathcal{M}(\mathcal{X}^n)$ be the set of Borel measures over $\mathcal{X}^n \subseteq \reall^n$. Given a measure $\mu^n \in \mathcal{M}(\mathcal{X}^n)$ and a measurable map $\varphi^{\V}$, $\mathcal{X}^n \ni \mathbf{x} \overset{\varphi^{\V}}{\longmapsto} \V^\top \mathbf{x} \in \mathcal{X}^m$, we denote by $\varphi^{\V}_{\#}(\mu^n) \coloneqq \mu^n(\varphi^{\V^{-1}}(\mathbf{x}))$ the pushforward measure $\mu^m \in \mathcal{M}(\mathcal{X}^m)$. 


We now present the standard definition of SCM.

\begin{definition}[SCM, \citealp{pearl2009causality}]\label{def:SCM}
A (Markovian) structural causal model (SCM) $\scm^n$ is a tuple $\langle \myendogenous, \myexogenous, \myfunctional, \zeta^\myexogenous \rangle$, where \emph{(i)} $\myendogenous = \{X_1, \ldots, X_n\}$ is a set of $n$ endogenous random variables; \emph{(ii)} $\myexogenous =\{Z_1,\ldots,Z_n\}$ is a set of $n$ exogenous variables; \emph{(iii)} $\myfunctional$ is a set of $n$ functional assignments such that $X_i=f_i(\parents_i, Z_i)$, $\forall \; i \in [n]$, with $ \parents_i \subseteq \myendogenous \setminus \{ X_i\}$; \emph{(iv)} $\zeta^\myexogenous$ is a product probability measure over independent exogenous variables $\zeta^\myexogenous=\prod_{i \in [n]} \zeta^i$, where $\zeta^i=P(Z_i)$. 
\end{definition}
A Markovian SCM induces a directed acyclic graph (DAG) $\mathcal{G}_{\scm^n}$ where the nodes represent the variables $\myendogenous$ and the edges are determined by the structural functions $\myfunctional$; $ \parents_i$ constitutes then the parent set for $X_i$. Furthermore, we can recursively rewrite the set of structural function $\myfunctional$ as a set of mixing functions $\mymixing$ dependent only on the exogenous variables (cf. \cref{app:CA}). A key feature for studying causality is the possibility of defining interventions on the model:
\begin{definition}[Hard intervention, \citealp{pearl2009causality}]\label{def:intervention}
Given SCM $\scm^n = \langle \myendogenous, \myexogenous, \myfunctional, \zeta^\myexogenous \rangle$, a (hard) intervention $\iota = \operatorname{do}(\myendogenous^{\iota} = \mathbf{x}^{\iota})$, $\myendogenous^{\iota}\subseteq \myendogenous$,
is an operator that generates a new post-intervention SCM $\scm^n_\iota = \langle \myendogenous, \myexogenous, \myfunctional_\iota, \zeta^\myexogenous \rangle$ by replacing each function $f_i$ for $X_i\in\myendogenous^{\iota}$ with the constant $x_i^\iota\in \mathbf{x}^\iota$. 
Graphically, an intervention mutilates $\mathcal{G}_{\mathsf{M}^n}$ by removing all the incoming edges of the variables in $\myendogenous^{\iota}$.
\end{definition}

Given multiple SCMs describing the same system at different levels of granularity, CA provides the definition of an $\alpha$-abstraction map to relate these SCMs:
\begin{definition}[$\abst$-abstraction, \citealp{rischel2020category}]\label{def:abstraction}
Given low-level $\mathsf{M}^\ell$ and high-level $\mathsf{M}^h$ SCMs, an $\abst$-abstraction is a triple $\abst = \langle \Rset, \amap, \alphamap{} \rangle$, where \emph{(i)} $\Rset \subseteq \datalow$ is a subset of relevant variables in $\mathsf{M}^\ell$; \emph{(ii)} $\amap: \Rset \rightarrow \datahigh$ is a surjective function between the relevant variables of $\mathsf{M}^\ell$ and the endogenous variables of $\mathsf{M}^h$; \emph{(iii)} $\alphamap{}: \dom{\Rset} \rightarrow \dom{\datahigh}$ is a modular function $\alphamap{} = \bigotimes_{i\in[n]} \alphamap{X^h_i}$ made up by surjective functions $\alphamap{X^h_i}: \dom{\amap^{-1}(X^h_i)} \rightarrow \dom{X^h_i}$ from the outcome of low-level variables $\amap^{-1}(X^h_i) \in \datalow$ onto outcomes of the high-level variables $X^h_i \in \datahigh$.
\end{definition}
Notice that an $\abst$-abstraction simultaneously maps variables via the function $\amap$ and values through the function $\alphamap{}$. The definition itself does not place any constraint on these functions, although a common requirement in the literature is for the abstraction to satisfy \emph{interventional consistency} \cite{rubenstein2017causal,rischel2020category,beckers2019abstracting}. An important class of such well-behaved abstractions is \emph{constructive linear abstraction}, for which the following properties hold. By constructivity, \emph{(i)} $\abst$ is interventionally consistent; \emph{(ii)} all low-level variables are relevant $\Rset=\datalow$; \emph{(iii)} in addition to the map $\alphamap{}$ between endogenous variables, there exists a map ${\alphamap{}}_U$ between exogenous variables satisfying interventional consistency \cite{beckers2019abstracting,schooltink2024aligning}. By linearity, $\alphamap{} = \V^\top \in \reall^{h \times \ell}$ \cite{massidda2024learningcausalabstractionslinear}. \cref{app:CA} provides formal definitions for interventional consistency, linear and constructive abstraction.


\section{Spectral Gaps}\label{sec:spectral gaps}



\subsection{Fully Random Gates}
Throughout this section let $n$ be a fixed positive integer. Recalling \Cref{def:the random walk operator}, the operators in focus in this section are operators of the following form for $m\leq n$:
\begin{align*}
    R_{n,m,k}=\frac1{\binom{n}{m}}\sum_{S\in \binom{[n]}{m}}R_{n,S,k}.
\end{align*}
Also define the Laplacian $L_{n,m,k}=L(R_{n,m,k})$.

In this section we prove \Cref{thm:one-random-nonlocal} by building on the previously-mentioned following result of Brodsky and Hoory:

\begin{theorem}[\cite{brodsky2008simple}, Theorem 2]\label{Brodsky-Hoory}
    For any $m$ and $f:\{\pm1\}^{mk}\to\R$ and $k\leq 2^m-2$, we have that
    \begin{align*}
       \abs{ \left\langle f,(R_{m,3,k}-R_{m,m,k})f\right\rangle }\leq 1-\Omega\pbra{\frac1{m^2k}}\langle f,f\rangle.\footnotemark
    \end{align*}
\end{theorem}
\footnotetext{\cite{brodsky2008simple} actually proves this inequality for the operator $R_{m,3,k}^{\des{2}}$ in place of $R_{m,3,k}$, which is the random walk operator induced by placing a random width-2 permutation (which acts on 3 bits). However, a standard comparison of Markov chains shows that our statement of the result easily follows. See \Cref{sec:DES gate set}.}
Our main contribution is a finer analysis in the case when $k$ is small relative to $m$, which results in the following theorem.

\begin{theorem}\label{thm:small k}
    Assume that $m\geq100$ and $k\leq 2^{m/10}$. Given $f:\{\pm1\}^{mk}\to\R$, we have 
    \begin{align*}
       \abs{ \left\langle f,(R_{m,m-1,k}-R_{m,m,k})f\right\rangle }\leq  \pbra{\frac1m+\frac{k^2}{2^{m/4}}}\left\langle f,f\right\rangle.
    \end{align*}
\end{theorem}

\Cref{thm:small k} is proven as \Cref{thm:small k internal} in \Cref{sec:small k}.


The following lemma allows us to combine \Cref{thm:small k} with the previously-known \Cref{Brodsky-Hoory} to bring the quadratic dependence on the number of wires ($n$) to linear.

\begin{lemma}[\cite{o2023explicit}, Lemma 3.2]\label{lem:induction}
    Fix a positive integer $n_0\geq 4$. For each $k$ and $m_1\geq m_2$ let $\tau_{m_1,m_2,k}$ be some real number such that
    \begin{align*}
        L_{m_1,m_2,k}\geq \tau_{m_1,m_2,k}L_{m_1,m_1,k}.\footnotemark
    \end{align*}
    \footnotetext{All inequalities between operators in this paper are in the PSD order.}
    Then for any sequence $n_0=m_0\leq m_1\leq \dots\leq m_{t-1}\leq m_t = n$ we have
    \begin{align*}
        L_{n,n_0,k}\geq \pbra{\prod_{i\in[t-1]}\tau_{m_i,m_{i-1},k}}L_{n,n,k}.
    \end{align*}
\end{lemma}

Together, \Cref{Brodsky-Hoory}, \Cref{thm:small k}, and \Cref{lem:induction} yield the following initial spectral gap. 

\begin{corollary}\label{cor:initial spectral gap}
    For any $n$ and $k\leq 2^{n/3}$, we have
    \begin{align*}
        L_{n,3,k}\geq \Omega\pbra{\frac1{nk \cdot\log k}}L_{n,n,k}.
    \end{align*}
\end{corollary}
\begin{proof}
    \Cref{thm:small k} shows that for all $m\geq 20\log k$ we have $\tau_{m,m-1,k}\geq 1-\frac1m-\frac{k^2}{2^{m/3}}$. \Cref{Brodsky-Hoory} shows that 
    \begin{align*}
        L_{20\log k,3,k}\geq  \Omega\pbra{\frac1{k\cdot\log^2k }}L_{20\log k,20\log k,k},
    \end{align*}
    or equivalently $\tau_{20\log k,3,k}\geq \Omega\pbra{\frac1{k\log^2 k}}$.

    Combining these using \Cref{lem:induction} we have for large enough $k$ that
    \begin{align*}
        L_{n,3,k}\geq &\Omega\pbra{\frac1{k\log^2 k}}\prod_{m=20\log k}^{n}\pbra{1-\frac1m-\frac{k^2}{2^{m/4}}}L_{n,n,k}\\
        \geq &\Omega\pbra{\frac1{k\log^2 k}}\frac{20\log k}{n} \prod_{m=20\log k}^{n}\pbra{1-\frac{k^2}{2^{m/4-1}}}\cdot L_{n,n,k}\\
        \geq&\Omega\pbra{\frac1{nk\cdot\log k}}\cdot\pbra{1-\sum_{m=20\log k}^{n}\frac{k^2}{2^{m/4-1}}}\cdot L_{n,n,k} \geq \Omega\pbra{\frac1{nk\cdot\log k}}\cdot L_{n,n,k}.\qedhere
    \end{align*}
\end{proof}


We leverage the initial spectral gap from \Cref{cor:initial spectral gap} to produce our designs by sequentially composing many copies of this pseudorandom permutation. This is akin to showing that the second largest eigenvalue of the square of a graph is quadratically smaller than that of the original graph.

\Cref{thm:one-random-nonlocal} follows directly from \Cref{sec:DES gate set} with standard spectral-gap to mixing time bounds.


\subsection{Nearest-Neighbor Random Gates}
\subsubsection{Reduction to the Large $k$ Case}

One step in a random reversible circuit with 1D-nearest-neighbor gates is described by the operator $R_{n,[n-2],k}^{3\text{-NN}}$. Note that we can write 
\begin{align*}
    R_{n,[n-2],k}^{3\text{-NN}}=& \Ex_{\mathbf{a}\in[n-2]}\sbra{R_{n,\{\mathbf{a},\mathbf{a}+1,\mathbf{a}+2\},k}}.\footnotemark
\end{align*}
Define the corresponding Laplacian $L_{n,I,k}^{\gamma\text{-NN}}=L(R_{n,I,k}^{\gamma\text{-NN}})$. 

Because the local terms $R_{n,\{\mathbf{a},\mathbf{a}+1,\mathbf{a}+2\},k}$ are projectors, to analyze such an operator we can use the following theorem of Nachtergaele.
\footnotetext{More generally, we write $R_{n,S,k}^{\gamma\text{-NN}}$ to denote $\Ex_{a\in S}\sbra{R_{n,\{\mathbf{a},\dots,\mathbf{a}+\gamma-1\},k}}$.}
\begin{theorem}[\cite{nachtergaele1996spectral}, Theorem 3]\label{thm:nachtergaele}
    Let $\{h_{a,a+1,a+2}\}_{a\in[n-2]}$ be projectors acting on $(\R^2)^{\otimes d}$ such that each $h_{\{a,a+1,a+2\}}$ only acts on the $a,a+1,a+2$th tensor factor. For $I=[a,b]\subseteq[n]$ define the subspace
    \begin{align*}
        \mathcal{G}_I=& \cbra{f\in (\R^2)^{\otimes d}:\sum_{a'\in [a,b-2]}h_{a',a'+1,a'+2}f = 0}.
    \end{align*}
    Let $G_I$ be the projector to $\mathcal{G}_I$.

    Now suppose there exists $\ell$ and $n_\ell$ and $\epsilon_\ell\leq \frac1{\sqrt\ell}$ such that for all $n_\ell\leq m\leq n$,
    \begin{align*}
        \norm{G_{[m-\ell-1,m]}\pbra{G_{[m-1]}-G_{[m]}}}_{\mathrm{op}}\leq \epsilon_\ell.
    \end{align*}
    Then
    \begin{align*}
        \lambda_{2}\pbra{\sum_{a\in[n-2]}h_{\{{a},{a}+1,{a}+2\}}}&\geq \frac{\pbra{1-\epsilon_\ell\sqrt\ell}^2}{\ell-3}\lambda_{2}\pbra{
        \sum_{a\in[\ell]}h_{\{{a},{a}+1,{a}+2\}}}.
    \end{align*}
    Recall $\lambda_2(h)$ denotes the second-smallest distinct eigenvalue of the operator $h$.
\end{theorem}

\begin{theorem}\label{thm:nachtergaele hypothesis}
    Fix any $m\geq 100$ and $k\leq 2^m-2$ and set $\ell=10\log k$. Then we have
    \begin{align*}
        \norm{R_{m,[m-\ell-1,m],k}\pbra{R_{m,[m-1],k}-R_{m,[m],k}}}_{\mathrm{op}}\leq \frac1\ell.
    \end{align*}
\end{theorem}
\Cref{thm:nachtergaele hypothesis} is established in \Cref{sec:nachtergaele hypothesis} as \Cref{thm:nachtergaele hypothesis internal}.

\begin{corollary}\label{cor:reduce to logk}
    We have for $k\geq 3$ that 
    \begin{align*}
        \lambda_2\pbra{L_{n,[n-2],k}^{3\text{-NN}}}\geq \frac{1}{2n}\lambda_2\pbra{L_{10\log k+2,[10\log k-2],k}^{3\text{-NN}}}
    \end{align*}
\end{corollary}
\begin{proof}
    Setting $h_{a,a+1,a+2}=\Id-R_{n,\{a,a+1,a+2\},k}$ for each $a\in [n]$, we see that the projections $G$ (as in the statement of \Cref{thm:nachtergaele}) are given by $G_{[a]}=R_{n,[\min\{n,a+2\}],k}$ for any $a\in[n]$. To see this, note first that $R_{n,[\min\{n,a+2\}],k}$ is indeed a projection. Now let $f$ be such that $R_{n,[\min\{n,a+2\}],k}f=f$. For every $\sigma\in \mathfrak{S}_{\{\pm1\}^{n}}$ define $f^\sigma$ by $f^\sigma(X)=f(\sigma X)$ for all $X\in\{\pm1\}^{nk}$. Then by invariance of $R_{n,[\min\{n,a+2\}],k}$ under a permutation applied to bits in $[\min\{n,a+2\}]$ we have
    \begin{align*}
        f^\sigma = R_{n,[\min\{n,a+2\}],k}f^\sigma =R_{n,[\min\{n,a+2\}],k}f =f.
    \end{align*}
    for any $\sigma \in \mathfrak{S}_{\{\pm1\}^{[\min\{n,a+2\}]}}$. The converse of this holds as well by a similar argument. Therefore, $f^{\sigma^{\{a',a'+1,a'+2\}}} = f$ for any $\sigma\in \mathfrak{S}_{\{0,1\}^8}$ for $a'\leq a$, proving that such an $f$ is truly in the ground space of $R_{n,\{a',a'+1,a'+2\},k}$ for any $a'\leq a$. The converse of this holds as well, because the permutations of the form $\sigma^{\{a',a'+1,a'+2\}}$ with $a'\leq a$ generate the group of permutations of the form $\rho^{\{a,b,c\}}$ for $\{a,b,c\}\in\binom{[\min\{n,a+2\}]}{3}$. Such an argument also proves the converse, so the $R$ operators are serve as the projections from \Cref{thm:nachtergaele}.
    
    Therefore, by \Cref{thm:nachtergaele hypothesis} we have that the hypotheses of \Cref{thm:nachtergaele} are satisfied with $\ell=10\log k$ and $\epsilon_\ell = \frac1\ell$. That is, for any $m\leq n$ we have
    \begin{align*}
    &\norm{G_{[m-\ell-1,m]}\pbra{G_{[m-1]}-G_{[m]}}}_{\mathrm{op}}
        =\norm{R_{n,[m-\ell-1,m],k}\pbra{R_{n,[m-1],k}-R_{n,[m],k}}}_{\mathrm{op}}\\
        =&\norm{\pbra{R_{m,[m-\ell-1,m],k}\pbra{R_{m,[m-1],k}-R_{m,[m],k}}}\otimes \Id_{[m+1,n]}}_{\mathrm{op}}
        \leq \frac1\ell.
    \end{align*}
    Therefore the conclusion of \Cref{thm:nachtergaele} is that 
    \begin{align*}
        &\lambda_2\pbra{(n-2)L_{n,[n-2],k}^{3\text{-NN}}}
        \geq \frac{\pbra{1-\frac1{\sqrt\ell}}^2}{\ell-2}\lambda_2\pbra{\ell L_{\ell+2,[\ell-2],k}^{3\text{-NN}}}
        \geq \frac{1}{2}\lambda_2\pbra{L_{\ell+2,[\ell-2],k}^{3\text{-NN}}}.
    \end{align*}
    Recalling our setting of $\ell$ completes the proof. 
\end{proof}



\subsubsection{Comparison Method for the Large $k$ Case}
We can use the spectral gap proved in \Cref{Brodsky-Hoory} for the random walk induced by completely random 3-bit gates to show a spectral gap for the random walk induced by random 3-bit gates, where the three bits on which the gate acts on are $a,a+1,a+2$ for some $a\in[n]$. Our proof is a simple application of the comparison method applied to random walks on multigraphs.

We take the following definition of (multi)graphs. A graph is a pair of sets $(V,E)$ such that there is a partition $E=\bigcup_{(x,y)\in V^2}E_{x,y}$. If $e\in E_{x,y}$ then we say that $e$ \textit{connects} the vertex $x$ to the vertex $y$, and we define $u(e)=x$ and $v(e)=y$. The degree $\deg(x)$ of a vertex $x\in V$ is the number of edges originating at $e$, or $\sum_{y\in V}|E_{x,y}|$. We say that a graph is regular if $\deg(x)=\deg(y)$ for all $x,y\in V$.

The \textit{random walk} on a graph $(V,E)$ beginning at a vertex $x\in V$ consists of the Markov chain $\{\bm{x}_i\}_{i\geq0}$ on state space $V$ such that $\bm{x}_0=x$ with probability 1, and to draw $\bm{x}_{i+1}$ given $\bm{x}_i$ we sample a uniform random edge $\bm{e}$ from $\bigcup_{y\in V}E_{x,y}$. We set $\bm{x}_{i+1}$ equal to the unique $y\in V$ such that $\bm{e}\in E_{x,y}$.

A \textit{Schreier graph} is a graph with vertices $V$ such that some group $\mathfrak{G}$ acts on $V$. Let $S\subseteq \mathfrak{G}$ be some subset of group elements. The edge set consists of elements of the form $(X,\sigma)$ ($\sigma\in S$), so that $(X,\sigma)\in E_{X,\sigma X}$. We call the resulting graph $\mathrm{Sch}(V,S)$.

\begin{definition}\label{def:congestion}
    Let $S$ and $\widetilde{S}$ be subsets of a group acting on a set $V$. For each ${\sigma}\in {S}$ let $\Gamma({\sigma})$ be a sequence $(\wt{\sigma}_1,\dots,{\wt\sigma}_t)$ of elements of $\wt S$ such that $\sigma v=\wt{\sigma}_t\dots \wt{\sigma}_1 v$ for all $v\in V$, so we regard $\Gamma$ as a map from $S$ to sets of paths using edges in $\wt{S}$. Define the \textit{congestion ratio} of $\Gamma$ to be
    \begin{align*}
        B({\Gamma})=&\max_{\wt\sigma\in \wt{S}}\cbra{|\wt{S}|\Ex_{\sigma\in S}\sbra{{N(\wt{\sigma},\Gamma(\sigma))|\Gamma({\sigma})|}}},
    \end{align*}
    where $N(\wt{\sigma},\Gamma(\sigma))$ is the number of times $\wt \sigma$ appears in the sequence $\Gamma(\sigma)$.
\end{definition}

\begin{lemma}\label{lem:comparison schreier}
    Let $G=\mathrm{Sch}(V,S)$ and $\wt{G}=\mathrm{Sch}(V,\wt{S})$ be the connected Schreier graphs of the action of a group. Let $L$ and $\widetilde{L}$ be the Laplacian operators for the non-lazy random walks on $G$ and $\wt{G}$, respectively. Suppose there exists a $\Gamma$ as in \Cref{def:congestion}. Then 
    \begin{align*}
        \lambda_2(L)\geq \pbra{\max_{v\in V}\frac{\pi(v)}{\widetilde{\pi}(v)}}B({\Gamma})\lambda_2(\wt{L}).
    \end{align*}
    Here $\pi$ and $\widetilde{\pi}$ are the stationary distributions for $G$ and $\widetilde{G}$, respectively.
\end{lemma}

We prove this in \Cref{appendix:comparison}. It is essentially a reformulation of a standard result about comparisons on general Markov chains in \cite{wilmer2009markov}. 

\begin{lemma}\label{lem:initial spectral gap local random gates large k}
    For any $n,k$ we have
    \begin{align*}
        \lambda_2\pbra{L_{n,[n-2],k}^{3\text{-NN}}}\geq \frac1{100000n^3}\lambda_2\pbra{L_{n,3,k}}.
    \end{align*}
\end{lemma}
\begin{proof}
    Note that $L_{n,3,k}$ and $L_{n,[n-2],k}^{3\text{-NN}}$ are simply the Laplacians of random walks on Schreier graphs with $\mathfrak{S}_{\{\pm1\}^{n}}$ acting on $\{\pm1\}^{nk}$ by $e(X^1,\dots,X^k)=(eX^1,\dots,eX^k)$. In the case of $L_{n,3,k}$ the edges are given by elements of the form $h^{\{a,b,c\}}$ for $h\in\mathfrak{S}_{\{\pm1\}^3}$ and $\{a,b,c\}\in\binom{[n]}{3}$. In the case of $L_{n,[n-3],k}^{3\text{-NN}}$ the edges are given by elements of the form $g^{\{a,a+1,a+2\}}$ for $g\in\mathfrak{S}_{\{\pm1\}^3}$ and $a\in[n-2]$. We deal with each connected component separately. Note that every connected component is isomorphic to $\{(X^1,\dots,X^{k'}):X^i\neq X^j\iff i\neq j\}$ for $k'\leq k$, so we bound the spectral gap for the walk on $\{(X^1,\dots,X^{k}):X^i\neq X^j\iff i\neq j\}$.

    We provide a map $\Gamma$ from $\{h^{\{a,b,c\}}:h\in\mathfrak{S}_{\{0,1\}^3},a,b,c\in[n]\}$ to sequences of elements of the form $g^{\{a,a+1,a+2\}}$ for $a\in[n-2]$ such that for any $h^{\{a,b,c\}}$ the sequence $\Gamma(h^{\{a,b,c\}})=(g_1^{\{a_1,a_1+1,a_1+2\}},\dots ,g_t^{\{a_t,a_t+1,a_t+2\}})$ satisfies $h^{\{a,b,c\}}=g_t^{\{a_t,a_t+1,a_t+2\}}\dots g_1^{\{a_1,a_1+1,a_1+2\}}$.

    The hope is to construct $\Gamma$ such that $B(\Gamma)$ is small and then to apply \Cref{lem:comparison schreier}. To this end we define $\Gamma$ as follows. Assume $a<b<c$. Fix $h\in\mathfrak{S}_8,a\in[n-2]$. Let $d\in[n-2]$ be arbitrary. Then write
    \begin{align*}
        h^{\{a,b,c\}}=\mathsf{Sort}^{-1} \cdot g^{\{d,d+1,d+2\}}\cdot\mathsf{Sort},
    \end{align*}
    Here $\mathsf{Sort}$ sends the $a$th coordinate to the $d$th coordinate, the $b$th to the $d+1$th, and the $c$th to the $d+2$th. The permutations $\mathsf{Sort}$ and $\mathsf{Sort}^{-1}$ can each be implemented using at most $3n$ gates of the form $g^{\{a'-1,a',a'+1\}}$ and $g^{\{a'+1,a'+2,a'+3\}}$ where each $g$ swaps coordinates; this is just by a standard partial sorting algorithm. Write $\mathsf{Sort}=g_{3n}^{\{a_{3n},a_{3n}+1,a_{3n}+2\}}\dots g_1^{\{a_1,a_1+1,a_1+2\}}$. Then set 
    \begin{align*}
        \Gamma(h^{\{a,b,c\}})=\pbra{g_1^{\{a_1,a_1+1,a_1+2\}},\dots,g_{3n}^{\{a_{3n},a_{3n}+1,a_{3n}+2\}},h^{(d,d+1,d+2)},\dots,\pbra{g_1^{\{a_1,a_1+1,a_1+2\}}}^{-1}}.
    \end{align*}
    We have $B(\Gamma)\leq 100000n^3$ trivially, so we have proved the result by applying \Cref{lem:comparison schreier} and the fact that the stationary distributions for both chains are uniform.
\end{proof}


\begin{corollary}\label{cor:initial spectral gap local}
For any $n,k$ we have
    \begin{align*}
        \lambda_2\pbra{L_{n,[n-2],k}^{3\text{-NN}}}\geq \Omega\pbra{\frac1{nk\cdot\log^5(k)}}.
    \end{align*}
\end{corollary}
\begin{proof}
    \Cref{lem:initial spectral gap local random gates large k} shows that $\lambda_2\pbra{L_{\ell+2,[\ell],k'}^{3\text{-NN}}}\geq \frac{1}{\ell^3}\lambda_2\pbra{L_{\ell+2,3,k'}}$ for all $\ell$ and $k'$. \Cref{Brodsky-Hoory} states that $\lambda_2\pbra{L_{\ell+2,3,k'}}\geq \frac{1}{(\ell+2)^2k'}$ for all $\ell$ and $k'$. If we set $\ell=10\log k$ and use \Cref{cor:reduce to logk} then we get
    \begin{align*}
         &\lambda_2\pbra{L_{n,[n-2],k}^{3\text{-NN}}} 
         \geq \frac{1}{2n}\cdot\lambda_2\pbra{L_{10\log k+2,[10\log k],k}^{3\text{-NN}}}
         \geq  \frac{1}{2n}\cdot\frac1{100000k\log^5(k)}.
    \end{align*}
    This implies the result.
\end{proof}

As in the case for fully random gates, \Cref{thm:one-random-local} follows from \Cref{cor:initial spectral gap local} and \Cref{sec:DES gate set}.

\subsection{Restricting the Gate Set}\label{sec:DES gate set}

So far all of our results have dealt with random circuits with arbitrary gates acting on 3 bits. However, for practical applications we are often in further restricting the type of 3-bit gates. However, as long as the arbitrary gate set is universal on 3 bits, we lose just a constant factor in the mixing time when we restrict our random circuits to use that gate set, by a standard application of the comparison method. We prove that we can perform this conversion before proving our results about brickwork circuits in \Cref{sec:brickwork} because it is somewhat easier to prove for the case of single gates acting at a time.

\begin{lemma}\label{lem:any gate set comparison}
    We have
    \begin{align*}
        \lambda_2\pbra{L_{n,[n-2],k}^{3\text{-NN},\des{2}}}\geq \Omega\pbra{\lambda_2\pbra{L_{n,[n-2],k}^{3\text{-NN}}}}
    \end{align*}
\end{lemma}
\begin{proof}
    We compare the Markov chains given by these two Laplacians by providing a way to write edges in the one induced by arbitrary 3-bit nearest-neighbor gates as paths in the one induced by 3-bit nearest-neighbor gates with generators $\mathcal{G}$. Again we focus on the connected component $\{(X^1,\dots,X^{k}):X^i\neq X^j\iff i\neq j\}$.

    For each $g^S$ for $g\in\mathfrak{S}_8$ and $S\in \binom{[n]}{3}$ let $\Gamma(g^S)=\pbra{g_{i_1}^S,\dots,g_{i_{8!}}^S}$ where we have fixed an arbitrary expansion of $g=g_{i_1}\dots g_{i_{8!}}$, and each $g_{i_j}$ is of type $\des{2}$. Then in the notation of \Cref{lem:comparison schreier} we have that
    \begin{align*}
        B(\Gamma)=&\max_{g\in \mathcal{G},a\in[n-2]}\cbra{\abs{\mathcal{G}}(n-2)\Ex_{\mathbf{g}\in \mathfrak{S}_{8},\mathbf{a}\in[n-2]}\sbra{N\pbra{g^{\{a,a+1,a+2\}},\Gamma(\mathbf{g}^{\{\mathbf a,\mathbf a+1,\mathbf a+2\}})}\abs{\Gamma(\mathbf{g}^{\{\mathbf a,\mathbf a+1,\mathbf a+2\}})}}}\\
        \leq &\max_{g\in \mathcal{G},a\in [n-2]}\cbra{8!n\Ex_{\mathbf{g}\in \mathfrak{S}_{8},\mathbf{a}\in[n-2]}\sbra{8!N\pbra{g^{\{a,a+1,a+2\}},\Gamma(\mathbf{g}^{\{\mathbf a,\mathbf a+1,\mathbf a+2\}})}}}\\
        \leq &\max_{g\in \mathcal{G},a\in [n-2]}\cbra{(8!)^3n\Pr_{\mathbf{g}\in \mathfrak{S}_{8},\mathbf{a}\in[n-2]}\sbra{g^{\{a,a+1,a+2\}}\in \Gamma(\mathbf{g}^{\{\mathbf a,\mathbf a+1,\mathbf a+2\}})}}\\
        \leq &\max_{g\in \mathcal{G},a\in [n-2]}\cbra{(8!)^3n\Pr_{\mathbf{g}\in \mathfrak{S}_{8},\mathbf{a}\in[n-2]}\sbra{a=\mathbf{a}}}\\
        \leq &(8!)^3.
    \end{align*}
    Applying \Cref{lem:comparison schreier} completes the proof.
\end{proof}


This shows that the random walk given by applying random gates on 3 bits of type $\des{2}$ has spectral gap $\wt{\Omega}\pbra{1/nk}$, by combining with \Cref{cor:initial spectral gap local}. A similar proof essentially shows the same result for the random circuit models where gates on arbitrary sets of 3 bits.



\subsection{Brickwork Circuits}\label{sec:brickwork}
The spectral gap for brickwork circuits follows almost directly from the spectral gap for circuits with nearest-neighbor gates (\Cref{cor:initial spectral gap local}), as in~\cite{brandao2016local}. First we show that the random walk induced by 3-bit nearest neighbor $\des{2}$ gates, where the 3 bits on which gates act on are of the form $\{a,a+1,a+2\}$ for any $a\in[n-2]$, has approximately the same spectral gap as that in which the random gates are of the form $\{a,a+1,a+2\}$ for $a\in[n-2]$ but with the restriction that $a\neq 0$ mod 3. Use the notation $L_{n,[n-2],k}^{3\text{-NN},\des{2}}$ and $L_{n,\{a\in[n-2],a=1,2\text{ mod } 3\},k}^{3\text{-NN},\des{2}}$ for the Laplacians of these random walks. Assume that $n=0$ mod 3; the other cases follow similarly. 

\begin{lemma}\label{lem:012 to 01}
    For any $n,k$ we have
    \begin{align*}
       \lambda_2\pbra{L_{n,\{a\in[n-2],a=1,2\text{ mod } 3\},k}^{3\text{-NN},\des{2}}}\geq  \Omega\pbra{\lambda_2\pbra{L_{n,[n-2],k}^{3\text{-NN}}}}.
    \end{align*}
\end{lemma}
\begin{proof}
    By \Cref{lem:any gate set comparison}, it suffices to show 
    \begin{align*}
       \lambda_2\pbra{L_{n,\{a\in[n-2],a=1,2\text{ mod } 3\},k}^{3\text{-NN},\des{2}}}\geq  \Omega\pbra{\lambda_2\pbra{L_{n,[n-2],k}^{3\text{-NN},\des{2}}}}.
    \end{align*}
    We use the comparison method. Again we focus on the connected component $\{(X^1,\dots,X^{k}):X^i\neq X^j\iff i\neq j\}$. For each $g\in\mathfrak{S}_{\{0,1\}^3}\cong \mathfrak{S}_8$ of type $\des{2}$ and $a\in[n-2]$ we provide a sequence $\Gamma(g^{\{a,a+1,a+2\}})$ of permutations multiplying to $g^{\{a,a+1,a+2\}}$ using only permutations of the form $h^{\{b,b+1,b+2\}}$ with $b\neq 0$ mod 3 such that the resulting congestion $B(\Gamma)$ is small. 

    We define $\Gamma$ as follows. Fix $g\in\mathfrak{S}_8,a\in[n-2]$ where $g$ is of type $\des{2}$. If $a\neq 0$ mod 3 then simply set $\Gamma(g^{\{a,a+1,a+2\}})=(g^{\{a,a+1,a+2\}})$. Otherwise $a=0$ mod 3. Then there exists a sequence of 64! permutations of the form $g_i^{\{b_i,b_i+1,b_i+2\}}$ with each $b_i\in \{a-1,a+1\}$ such that $g=g_1^{\{b_1,b_1+1,b_{1}+2\}}\cdots g_1^{\{b_{64!},\dots,b_{64!}+1,\dots,b_{64!}+2\}}$. This is because we can implement the gate $g^{\{a,a+1,a+2\}}$ as 
    \begin{align*}
        g^{\{a,a+1,a+2\}}=\mathsf{Sort}^{-1} \cdot g^{\{a-1,a,a+1\}}\cdot\mathsf{Sort},
    \end{align*}
    where $\mathsf{Sort}$ sends $(x_1,\dots,x_{a-1},x_a,x_{a+1},x_{a+2},\dots,x_n)\to (x_1,\dots,x_a,x_{a+1},x_{a+2},x_{a-1},\dots,x_n)$. The permutations $\mathsf{Sort}$ and $\mathsf{Sort}^{-1}$ can each be implemented as the product of at most $32!$ permutations of the form $h^{\{a-1,a,a+1\}}$ and $h^{\{a+1,a+2,a+3\}}$ where each $h$ is of type $\des{2}$. This gives the implementation of $g^{\{a,a+1,a+2\}}$ as the product of at most $64!$ elements of the form $g^{\{a-1,a,a+1\}}$ or $g^{\{a,a+1,a+2\}}$, and this defines $\Gamma(g^{\{a,a+1,a+2\}})$ for such $g$ and $a$. 
    
    In the notation of \Cref{lem:comparison schreier}, the congestion of $\Gamma$ is bounded:
    \begin{align*}
        B(\Gamma)\leq &\max_{g\in \mathcal{G},a=1,2\text{ mod 3}}\cbra{\frac{8!\cdot2n}{3}\Ex_{\mathbf{g}\in \mathfrak{S}_{8},\mathbf{a}\in[n-2]}\sbra{N\pbra{g^{\{a,a+1,a+2\}},\Gamma(\mathbf{g}^{\{\mathbf a,\mathbf a+1,\mathbf a+2\}})}\abs{\Gamma(\mathbf{g}^{\{\mathbf a,\mathbf a+1,\mathbf a+2\}})}}}\\
        \leq & 70!\max_{g\in \mathcal{G},a=1,2\text{ mod 3}}\cbra{n\Ex_{\mathbf{g}\in \mathfrak{S}_{8},\mathbf{a}\in[n-2]}\sbra{N\pbra{g^{\{a,a+1,a+2\}},\Gamma(\mathbf{g}^{\{\mathbf a,\mathbf a+1,\mathbf a+2\}})}}}\\
        \leq &70!\max_{g\in \mathcal{G},a=1,2\text{ mod 3}}\cbra{n\Pr_{\mathbf{g}\in \mathfrak{S}_{8},\mathbf{a}\in[n-2]}\sbra{g^{\{a,a+1,a+2\}}\in \Gamma(\mathbf{g}^{\{\mathbf a,\mathbf a+1,\mathbf a+2\}})}}\\
        \leq &70!\max_{g\in \mathcal{G},a=1,2\text{ mod 3}}\cbra{n\Pr_{\mathbf{g}\in \mathfrak{S}_{8},\mathbf{a}\in[n-2]}\sbra{a\in\{\mathbf{a}-1,\mathbf{a},\mathbf{a}+1\}}}\\
        \leq & 71!.
    \end{align*}
    Here we used that $\abs{\Gamma(g^{\{a,a+1,a+2\}})}\leq 64!$ always. Applying \Cref{lem:comparison schreier} completes the proof.
\end{proof}


Given this restriction to $\des{2}$ gates, the idea to prove the spectral gap for the random walk corresponding to a random brickwork circuit is to write the transition operator corresponding to one layer of brickwork gates as
\begin{align*}
    &R_{n,k}^{\text{brickwork},\des{2}}\\
    =&R_{n,\{1,2,3\},k}^{\des{2}}R_{n,\{4,5,6\},k}^{\des{2}}\cdots R_{n,\{n-2,n-1,n\},k}^{\des{2}}R_{n,\{2,3,4\},k}^{\des{2}}R_{n,\{5,6,7\},k}^{\des{2}}\cdots R_{n,\{n-4,n-3,n-2\},k}^{\des{2}}.
\end{align*}
Here the operator $R_{n,S,k}^{\des{2}}$ is the transition operator for the Markov chain that is similar to $R_{n,S,k}$ but with the restriction that each step is induced by a gate of type $\des{2}$. Then we note that each individual factor in each of the two products all commute, and every factor commutes with all but at most 7 other factors. The detectability lemma~\cite{aharonov2009detectability} bounds the spectral gap of this operator.


\begin{lemma}[\cite{brandao2016local}, Section 4.A]\label{lem:local to brickwork}
    For any $n,k$ we have
    \begin{align*}
        \lambda_2\pbra{L_{n,k}^{\mathrm{brickwork},\des{2}}}\geq  n\Omega\pbra{\lambda_2\pbra{L_{n,\{a\in[n-2],a=1,2\text{ mod } 3\},k}^{3\text{-NN},\des{2}}}}.
    \end{align*}
\end{lemma}

\begin{corollary}\label{cor:initial spectral gap brickwork}
    For any $n,k$ we have
    \begin{align*}
        \lambda_2\pbra{L_{n,k}^{\mathrm{brickwork},\des{2}}}\geq  \Omega\pbra{\frac1{k\cdot\polylog(k)}}.
    \end{align*}
\end{corollary}

As in the case for fully random gates and nearest-neighbor random gates, \Cref{thm:one-layer-brickwork} follows from \Cref{cor:initial spectral gap brickwork}.


\section{Proof of \Cref{thm:small k}}\label{sec:small k}
Throughout this section fix $m\geq 3$. Our goal in this section is to establish \Cref{thm:small k}, which states that $R_{m,m-1,k}-R_{m,m,k}$ has small spectral norm. Informally, we show that completely randomizing $m-1$ out of $m$ wires in a reversible circuit is very similar to randomizing all $m$ wires. Recall that $R_{m,m-1,k}$ is defined via the distributions $\mathcal{D}_X^{m,m-1,k}$ (the equal mixture of $\mcD_X^{m,S,k}$ for all $S$ with $|S|=m-1$) for $X\in\{\pm1\}^{mk}$, from which one samples by sampling a random set $\mathbf{S}\subseteq [m]$ with $|\mathbf{S}|=1$ (so really $\mathbf{S}=\{\mathbf{a}\}$, where $\mathbf{a}$ is a random element of $[m]$), ``fixes" the entries $X^i_{\mathbf{a}}$, and applies a random permutation to the coordinates not equal to $\mathbf{a}$. We use this notation and terminology of a ``fixed" coordinate $\mathbf{a}$ throughout this section.

As alluded to in \Cref{sec:overview}, our proof will decompose the space $\R^{\{\pm1\}^{mk}}$ on which these operators act into three orthogonal components, to be defined in \Cref{sec:decomposition}. Then \Cref{sec:decomposition}, \Cref{sec:B=1}, and \Cref{sec:B>=2} will bound the contributions from vectors lying in these orthogonal components and their cross terms.



\subsection{An Orthogonal Decomposition}\label{sec:decomposition}
\begin{definition}
    Regard elements of $\{\pm1\}^{mk}$ as $k$-by-$m$ matrices, so that the $i$th row of $X$ is $X^i$, and the $a$th column of $X$ is $X_a$. Define 
    \begin{align*}
        B_{\geq 2}&=\cbra{X\in\{\pm1\}^{mk}:\forall i\neq j\in[k], d\pbra{X^i, X^j}\geq 2},\\
        B_{=1}&=\cbra{X\in\{\pm1\}^{mk}:\forall i\neq j\in[k], d\pbra{X^i, X^j}\geq 1}\setminus B_{\geq 2},\\
        B_{=0}&=\cbra{X\in\{\pm1\}^{mk}:\exists i\neq j\in[k], d\pbra{X^i, X^j}=0}.
    \end{align*}
\end{definition}

Our proof that $R_{m,m-1,k}-R_{m,m,k}$ has small spectral norm will go by induction on $k$. \Cref{lemma:f supported on B0} helps to connect the cases of $k-1$ and $k$ in the proof. In particular, it shows that we can pass those functions supported on $B_{=0}$ into the induction.

\begin{lemma}\label{lemma:f supported on B0}
    Let $f:\{\pm1\}^{mk}\to \R$ be supported on $B_{=0}$. Then for any $S_1,\dots,S_t\subseteq[m]$ and $c_1,\dots,c_t\in\R$ we have
    \begin{align*}
        \abs{\left\langle f,\sum_{s=1}^t c_s \pbra{R_{m,S_s,k}-R_{m,m,k}}f\right\rangle }\leq \norm{\sum_{s=1}^t c_s \pbra{R_{m,S_s,k-1}-R_{m,m,k-1}}}_{\mathrm{op}}\cdot\norm{f}_2^2.
    \end{align*}
\end{lemma}
\begin{proof}
    For any map $\phi:[k]\to[k-1]$ (viewed as a coloring of $[k]$ with $k-1$ colors) define the set 
    \begin{align*}
        \mathcal{J}_\phi = \cbra{\pbra{X^1,\dots,X^k}:X^i=X^j\iff \phi(i)=\phi(j)}.
    \end{align*}
    These sets $\mathcal{J}_\phi$ partition $B_{=0}$. Thus, for every $f$ supported on $B_{=0}$ we have a decomposition $f=\sum_{\phi}f_\phi$, where each $f_\phi$ is supported on $\mathcal{J}_\phi$. 

    Now, for each $\phi$ define a map $\mathrm{Res}_\phi:\cbra{f:{\{\pm1\}^{mk}}\to \R:f\text{ supported on }\mathcal{J}_\phi}\to \R^{\{\pm1\}^{m(k-1)}}$ by arbitrarily choosing $i,j\in[k]$ such that $\phi(i)=\phi(j)$ and defining for $f_\phi:\{\pm1\}^{mk}\to \R$ supported on $\mathcal{J}_{\phi}$ the new function $\mathrm{Res}_\phi f_\phi:\{\pm1\}^{m(k-1)}\to \R$ by defining for $X'\in \{\pm1\}^{m(k-1)}$
    \begin{align*}
        \mathrm{Res}_\phi f_\phi(X')=& f(\mathrm{Res}_\phi^*(X')).
    \end{align*}
    where $\mathrm{Res}_\phi^*(X')$ is the unique element of $\mathcal{J}_\phi$ such that $(\mathrm{Res}_\phi^*(X'))^{[k]\setminus\{j\}}=X'$. Note this is well-defined because $f_\phi$ is supported on $\mathcal{J}_\phi$.

    \begin{claim}\label{claim:Res norm}
        For any $f_\phi:{\{\pm1\}^{mk}}\to \R$ supported on $\mathcal{J}_\phi$ we have $\norm{f_\phi}_2=\norm{\mathrm{Res}_\phi f_\phi}_2$ and for any $S\subseteq[m]$,
        \begin{align*}
        \left\langle f_\phi,R_{m,S,k}f_\phi\right\rangle = \left\langle \mathrm{Res}_\phi f_\phi,R_{m,S,k}\mathrm{Res}_\phi f_\phi\right\rangle.
    \end{align*}
    \end{claim}
    We prove the claim later, and for now use it to compute
    \begin{align*}
        &\abs{\left\langle f,\sum_{s=1}^t c_s \pbra{R_{m,S_s,k}-R_{m,m,k}}f\right\rangle} \\
        =&\abs{\sum_{\phi,\phi'}\left\langle f_\phi , \sum_{s=1}^t c_s \pbra{R_{m,S_s,k}-R_{m,m,k}}f_{\phi'}\right\rangle} \\
        =&\abs{\sum_{\phi}\left\langle f_\phi , \sum_{s=1}^t c_s \pbra{R_{m,S_s,k}-R_{m,m,k}}f_{\phi}\right\rangle + \sum_{\phi\neq \phi'}\left\langle f_\phi , \sum_{s=1}^t c_s \pbra{R_{m,S_s,k}-R_{m,m,k}}f_{\phi'}\right\rangle} \\
        =&\abs{\sum_{\phi}\left\langle \mathrm{Res}_\phi f_\phi,\sum_{s=1}^t c_s \pbra{R_{m,S_s,k}-R_{m,m,k}}\mathrm{Res}_\phi f_\phi \right\rangle} \tag{\Cref{claim:Res norm}, \Cref{eq:different colorings} below}\\
        \leq & \sum_{\phi}\norm{\sum_{s=1}^t c_s \pbra{R_{m,S_s,k-1}-R_{m,m,k-1}}}_{\mathrm{op}}\norm{\mathrm{Res}_\phi f_\phi}_2^2 \\
        =&\norm{\sum_{s=1}^t c_s \pbra{R_{m,S_s,k-1}-R_{m,m,k-1}}}_{\mathrm{op}}\sum_{\phi}\norm{f_\phi}_2^2\\
        =& \norm{\sum_{s=1}^t c_s \pbra{R_{m,S_s,k-1}-R_{m,m,k-1}}}_{\mathrm{op}}\norm{f}_2^2.
    \end{align*}
    The last equality follows from orthogonality of the $f_\phi$.
    
    To take care of the cross terms, we observe that for any $X\in\mathcal{J}_\phi$ and $S\subseteq[n]$, we have $\Pr\sbra{X\to_{R_{m,S,k}} \mathcal{J}_{\phi'}}=0$ for any $\phi'\neq \phi$. Then by \Cref{lem:escape probs} we have
    \begin{align}\label{eq:different colorings}
        &\sum_{\phi\neq \phi'}\left\langle f_\phi , \sum_{s=1}^t c_s R_{m,S_s,k}f_{\phi'}\right\rangle =0. 
    \end{align}
    This completes the proof.
\end{proof}

\begin{proof}[Proof of \Cref{claim:Res norm}]
    Without loss of generality assume that $\phi(k-1)=\phi(k)$ so we can regard $\mathrm{Res}_\phi f_\phi$ as a real function on $\{\pm1\}^{m(k-1)}$. Then
    \begin{align*}
        &\left\langle f_\phi,f_\phi\right\rangle 
        =\sum_{X\in\{\pm1\}^{mk}}f_\phi(X)^2
        = \sum_{X\in \mathcal{J}_\phi}\pbra{{f_\phi(X)}}^2\\
        =& \sum_{X'\in \{\pm1\}^{m(k-1)}}\left(\mathrm{Res}_\phi f(X')\right)^2
        = \left\langle \mathrm{Res}_\phi f_\phi , \mathrm{Res}_\phi f_\phi \right\rangle.
    \end{align*}
    To prove the second statement, we compute
    \begin{align*}
        &\left\langle f_\phi,R_{m,S,k}f_\phi\right\rangle\\
        =&\sum_{X\in\{\pm1\}^{mk}}f_\phi(X)\sum_{Y\in\{\pm1\}^{mk}}f_\phi(Y)\Pr\sbra{X\to_{R_{m,S,k}} Y}\\
        =&\sum_{X\in\mathcal{J}_\phi}f_\phi(X)\sum_{Y\in\mathcal{J}_\phi}f_\phi(Y)\Pr\sbra{X\to_{R_{m,S,k}} Y}\\
        =&\sum_{X'\in\{\pm1\}^{m(k-1)}}\mathrm{Res}_\phi f_\phi(X')\sum_{Y'\in\{\pm1\}^{m(k-1)}}\mathrm{Res}_\phi f_\phi(Y')\Pr\sbra{X'\to_{R_{m,S,k-1}} Y'}\\
        =&\left\langle \mathrm{Res}_\phi f_\phi ,R_{m,S,k-1}\mathrm{Res}_\phi f_\phi  \right\rangle.
    \end{align*}
    The second-to-last equality follows because the corresponding $X,Y$ are in $\mathcal{J}_\phi$.
\end{proof}

We now prove \Cref{thm:small k}, deferring proofs of the remaining needed auxiliary results to \Cref{sec:B=1}, \Cref{sec:cross terms}, and \Cref{sec:B>=2}.
\begin{theorem}[\Cref{thm:small k} restated]\label{thm:small k internal}
    Let $m\geq 100$ and let $2\leq k\leq 2^{m/10}$. Given any $f:\{\pm1\}^{mk}\to\R$, we have
    \begin{align*}
       \abs{ \left\langle f,(R_{m,m-1,k}-R_{m,m,k})f\right\rangle }\leq  \pbra{\frac1m+\frac{k^2}{2^{m/4}}}\left\langle f,f\right\rangle.
    \end{align*}
\end{theorem}
\begin{proof}
    We prove by induction on $k$. In the base case $k=1$ and the result holds by the following argument when we write $f=f_2$. Now assume that the result holds for real functions on $\{\pm1\}^{m(k-1)}$. 
    
    Let $f:\{\pm1\}^{mk}\to\R$. Write $f=f_0+f_1+f_2$ where $f_0$ is supported on $B_{=0}$, $f_1$ is supported on $B_{=1}$, and $f_2$ is supported on $B_{\geq 2}$. By \Cref{lem:cross terms B0} applied with both $R_{m,m-1,k}$ and $R_{m,m,k}$ and $B_{=0}$ and $\{\pm1\}^{mk}\setminus B_{=0}=B_{=1}\cup B_{\geq 2}$, the other cross terms vanish, and we have
    \begin{align*}
        &\abs{ \left\langle f,(R_{m,m-1,k}-R_{m,m,k})f\right\rangle }\\
        \leq & \abs{\left\langle f_0,(R_{m,m-1,k}-R_{m,m,k})f_0\right\rangle}+\abs{\left\langle f_1,(R_{m,m-1,k}-R_{m,m,k})f_1\right\rangle}\\
        &\;\;\;\;\;\;+\abs{\left\langle f_2,(R_{m,m-1,k}-R_{m,m,k})f_2\right\rangle}+\abs{2\left\langle f_1,(R_{m,m-1,k}-R_{m,m,k})f_2\right\rangle }\tag{Self-adjointness (\Cref{fact:self-adjoint})}\\
        \leq & \pbra{\frac1m+\frac{(k-1)^2}{2^{m/4}}}\left\langle f_0,f_0\right\rangle +\pbra{\frac1m+\frac{m^5k}{2^{m/2-2}}}\left\langle f_1,f_1\right\rangle + \pbra{\frac1m+\frac{k^2}{2^{m/2}}}\left\langle f_2,f_2\right\rangle + \frac{\sqrt{m}k}{2^{m/2-2}}\norm{f_1}_2\norm{f_2}_2\tag{\Cref{lemma:f supported on B0} + induction, \Cref{cor:f1-f1}, \Cref{prop:B>=2 square term}, \Cref{lem:cross terms B=1 to B=2}, in that order}\\
        \leq & \pbra{\frac1m+\frac{k^2-k}{2^{m/4}}}\left\langle f,f\right\rangle+\frac{m^5k}{2^{m/2-3}}\norm{f_1}_2\norm{f_2}_2\\
        \leq & \pbra{\frac1m+\frac{k^2-k}{2^{m/4}}}\left\langle f,f\right\rangle+\frac{k}{2^{m/3}}\left\langle f,f\right\rangle\\
        \leq&\pbra{\frac1m+\frac{k^2}{2^{m/4}}}\left\langle f,f\right\rangle.\qedhere
    \end{align*}
\end{proof}


\subsection{$f$ Supported on $B_{=1}$}\label{sec:B=1}

We now use \Cref{lem:escape probs} to bound $\abs{\left\langle f, (R_{m,m-1,k}-R_{m,m,k})f\right\rangle}\leq \abs{\left\langle f, R_{m,m-1,k}f\right\rangle}$ when $f$ is supported only on $B_{= 1}$ and the cross terms contributed. We first define a partition of $B_{=1}$, and apply \Cref{lem:escape probs} to these different parts. One of our main observations to bound $f$ supported on $B_{=1}$ is the observation that when $k$ is small, it is highly unlikely that two vectors out of any $k$ are close to each other. Thus, a random walk beginning in $B_{=1}$ and obeying the transition probabilities given by $B_{m,m-1,k}$ will rarely remain in $B_{=1}$. Formally, \Cref{lem:escape probs} bounds the contributions to the spectral norm by these transition probabilities.
\begin{definition}
    For $S\subseteq[m]$ define the set $\mathcal{I}_S\subseteq\{\pm1\}^{mk}$ by
    \begin{align*}
        \mathcal{I}_S=\cbra{X\in B_{=1}: \forall a\in S ,\exists i,j\in[k]:\Delta\pbra{X^i,X^j}=\{a\}}.
    \end{align*}
\end{definition}
Unfortunately, the sets $\mathcal{I}_S$ for different $S\subseteq[m]$ do not form a partition of $B_{=1}$, since there is overlap between $\mathcal{I}_S$ and $\mathcal{I}_T$ for $S\neq T$. However, we can artificially make this into a partition.
\begin{definition}
    For $S\subseteq[m]$ with $|S|=1$ (so $S=\{a\}$ for some $a\in[m]$) define the set $\widetilde{\mathcal{I}}_S\subseteq\{\pm1\}^{nk}$ by
    \begin{align*}
        \widetilde{\mathcal{I}}_S=\mathcal{I}_S\setminus \bigcup_{a'\neq a}\mathcal{I}_{\{a'\}}.
    \end{align*}
    Now place an arbitrary ordering $\preceq $ on the set $\{S\subseteq[m]:|S|=2\}$ and define
    \begin{align*}
        \widetilde{\mathcal{I}}_S = \mathcal{I}_S\setminus \pbra{ \bigcup_{S'\subseteq [m]:|S'| = 2,S'\preceq S}\mathcal{I}_{S'}}.
    \end{align*}
\end{definition}

\begin{observation}\label{obs:Is partition}
    The collection of sets $\{\widetilde{\mathcal{I}}_S:S\subseteq[m],|S|\leq 2\}$ forms a partition of $B_{=1}$.
\end{observation}

\begin{lemma}\label{lem:square terms B=1}
    Let $S\subseteq[m]$ be such that $|S|\leq 2$. If $k\geq 2$ and $f:\{\pm1\}^{mk}\to\R$ is supported on $\widetilde{\mathcal{I}}_S$ then 
    \begin{align*}
        \abs{\left\langle f,(R_{m,m-1,k}-R_{m,m,k})f\right\rangle }\leq\pbra{\frac1m+\frac{k}{2^{m/2-2}}}\left\langle f,f\right\rangle.
    \end{align*}
\end{lemma}
\begin{proof}
    We bound $\abs{ \left\langle f,R_{m,m-1,k}f\right\rangle }\geq \abs{ \left\langle f,(R_{m,m-1,k}-R_{m,m,k})f\right\rangle }$. This inequality is true because $R_{m,m,k}$ and $R_{m,m-1,k}-R_{m,m,k}$ are PSD by. Suppose first that $|S|=1$ so that $S=\{a\}$ for some $a\in[m]$. Let $X\in\widetilde{\mathcal{I}}_S$. Then for every $W\subseteq[m]$ with $|W|=m-1$ and $a\in W$ we have
    \begin{align*}
        \Pr\sbra{X\to_{R_{m,W,k}} \widetilde{\mathcal{I}}_S}
        \leq& \sum_{i,j\in[k]}\Pr_{\mathbf{Y}\sim \mathcal{D}_X^{m,W,k}}\sbra{\Delta\pbra{\mathbf{Y}^i,\mathbf{Y}^j}=\{a\}}
        \leq  \frac{k^2}{2^{m-1}}.
    \end{align*}
    This is because $a\in W$, and so it must be the case that for all $i\neq j\in[k]$ we have $X^i_{[m]\setminus\{b\}}\neq X^j_{[m]\setminus\{b\}}$. Otherwise we would have $\widetilde{\mathcal I}_{\{a,b\}}$, contradicting that $X\in \widetilde{\mathcal I}_{\{a\}}$. Using this bound, we have that
    \begin{align*}
        &\abs{\left\langle f,(R_{m,m-1,k}-R_{m,m,k})f\right\rangle }\\
        \leq&\frac1m\sum_{W\subseteq [m],|W|=m-1}\abs{\left\langle f,R_{m,W,k}f\right\rangle }+\abs{\left\langle f,R_{m,m,k}f\right\rangle }\\
        \leq&\frac1m\abs{\left\langle f,R_{m,[m]\setminus\{a\},k}f\right\rangle }+ \frac1m\sum_{W\subseteq [m],|W|=m-1,a\in W}\abs{\left\langle f,R_{m,W,k}f\right\rangle }+\abs{\left\langle f,R_{m,m,k}f\right\rangle }\\
        \leq&\frac1m\langle f,f\rangle+  \frac{k}{2^{m/2-1}}\langle f,f\rangle+ \frac{k}{2^{m/2}}\langle f,f\rangle.
    \end{align*}
    The last inequality follows from the \Cref{lem:escape probs} and our previous calculation, while noticing that any $R_{m,W,k}$ has the uniform distribution over $\{\pm1\}^{mk}$ as a stationary distribution. A similar calculation shows that $\Pr\sbra{X\to_{R_{m,m,k}} \widetilde{\mathcal{I}}_S}\leq \frac{k^2}{2^{m}}$, which is used to bound the third term $\abs{\left\langle f,R_{m,m,k}f\right\rangle }$ by \Cref{lem:escape probs}. This completes the proof for the case $S=\{a\}$.

    Now assume that $|S|=2$ so that $S=\{a,b\}$ and $X\in\widetilde{\mathcal{I}}_S$. Fix $W\subseteq[m]$ with $|W|=m-1$. Assume that $a\not\in W$. Then
    \begin{align*}
        \Pr\sbra{X\to_{R_{m,W,k}} \widetilde{\mathcal{I}}_S}\leq&\sum_{i,j\in[k]}\Pr_{\mathbf{Y}\sim\mathcal{D}^{m,W,k}_X}\sbra{\Delta\pbra{\mathbf{Y}^i,\mathbf{Y}^j}=b}\leq\sum_{i,j\in[k]}\frac{1}{2^{m-1}}\leq \frac{k^2}{2^{m-1}}.
    \end{align*}
    This is because if $\Delta(X^i,X^j)=\{a\}$ then $\Delta(\mathbf{Y}^i,\mathbf{Y}^j)=\{a\}\neq \{b\}$ and otherwise $X^i_{[m]\setminus\{a\}}=X^i_W \neq X^j_W= X^j_{[m]\setminus\{a\}}$. Using this we find that
    \begin{align*}
        \Pr\sbra{X\to_{R_{m,m-1,k}} \widetilde{\mathcal{I}}_S}=&\frac1m\sum_{W\subseteq[m],|S|=m-1}\Pr\sbra{X\to_{R_{m,W,k}} \widetilde{\mathcal{I}}_S}\leq \frac{k^2}{2^{m-1}}.
    \end{align*}
    If $b\in W$ or $a,b\not\in W$ a similar proof shows the same bound. Using the bound $\Pr\sbra{X\to_{R_{m,m,k}} \widetilde{\mathcal{I}}_T}\leq \frac{k^2}{2^{m-1}}$ completes the proof:
    \begin{align*}
        &\abs{\left\langle f,(R_{m,m-1,k}-R_{m,m,k})f\right\rangle}
        \leq\abs{\left\langle f,R_{m,m-1,k}f\right\rangle}+\abs{\left\langle f,R_{m,m,k}f\right\rangle}
        \leq\frac{k}{2^{m/2-1}}\langle f,f\rangle+ \frac{k}{2^{m/2-1}}\langle f,f\rangle.
    \end{align*}
    The second inequality is an application of \Cref{lem:escape probs}.
\end{proof}

\begin{lemma}\label{lem:cross terms B=1 to B=1}
    Let $S\neq T\subseteq[m]$ be such that $|S|,|T|\leq 2$. If $k\geq 2$ and $f:\{\pm1\}^{mk}\to\R$ is supported on $\widetilde{\mathcal{I}}_S$ and $g:\{\pm1\}^{mk}\to\R$ is supported on $\widetilde{\mathcal{I}}_T$ then 
    \begin{align*}
       \abs{ \left\langle f,(R_{m,m-1,k}-R_{m,m,k})g\right\rangle }\leq\frac{k}{2^{m/2-2}}\norm{f}_2\norm{g}_2.
    \end{align*}
\end{lemma}
\begin{proof}
    Let $X\in\widetilde{\mathcal{I}}_S$. Let $a\in T\setminus S$ be such that there does not exist $i,j\in[k]$ such that $X\in \mathcal{I}_{\{a\}}$. Then 
    \begin{align*}
        &\Pr\sbra{X\to_{R_{m,m-1,k}} \widetilde{\mathcal{I}}_T}
        \leq \Pr\sbra{X\to_{R_{m,m-1,k}} \widetilde{\mathcal{I}}_{\{a\}}}
        \leq \Pr\sbra{X\to_{R_{m,m-1,k}} {\mathcal{I}}_{\{a\}}}\\
        \leq & \sum_{i,j\in[k]}\Pr_{\mathbf{Y}\sim \mathcal{D}_X^{m,m-1,k}}\sbra{\Delta\pbra{\mathbf{Y}^i,\mathbf{Y}^j}=\{a\}}
        \leq \frac{k^2}{2^{m-1}}.\tag{$X\not\in\mathcal{I}_{\{a\}}$}
    \end{align*}
    Then applying \Cref{lem:escape probs} gives
    \begin{align*}
        \abs{ \left\langle f,R_{m,m-1,k}g\right\rangle }\leq\frac{k}{2^{m/2-1}}\norm{f}_2\norm{g}_2.
    \end{align*}
    A similar calculation shows that $\Pr\sbra{X\to_{R_{m,m,k}} \widetilde{\mathcal{I}}}\leq \frac{mk^2}{2^{m-1}}$. Then \Cref{lem:escape probs} gives
    \begin{align*}
        \abs{ \left\langle f,R_{m,m,k}g\right\rangle }\leq\frac{k}{2^{m/2-1}}\norm{f}_2\norm{g}_2.
    \end{align*}
    Applying the triangle inequality completes the proof.
\end{proof}



\begin{corollary}\label{cor:f1-f1}
    Assume $k\geq 2$. Let $f:\{\pm1\}^{mk}\to\R$ be supported on $B_{=1}$. Then 
    \begin{align*}
        \abs{\left\langle f,(R_{m,m-1,k}-R_{m,m,k})f\right\rangle }\leq \pbra{\frac1m+\frac{m^5k}{2^{m/2-2}}}\left\langle f,f\right\rangle.
    \end{align*}
\end{corollary}
\begin{proof}
    Write $f=\sum_{S\subseteq[m]:|S|\leq 2}f_{S}$ where each $f_S$ is supported on $\mathcal{I}_S$. Then
    \begin{align*}
        & \abs{\left\langle f,(R_{m,m-1,k}-R_{m,m,k})f\right\rangle}\\
        \leq &\abs{\left\langle f,R_{m,m-1,k}f\right\rangle} \tag{$R_{m,m,k}$ and $R_{m,m-1,k}-R_{m,m,k}$ both PSD}\\
        = &\sum_{S,T\subseteq [m]:|S|,|T|\leq 2}\abs{\left\langle f_S,R_{m,m-1,k}f_T\right\rangle}\\
        \leq &\sum_{S\subseteq [m]:|S|\leq 2}\abs{\left\langle f_S,R_{m,m-1,k}f_S\right\rangle}+ \sum_{S\neq T\subseteq [m]:|S|,|T|\leq 2}\abs{\left\langle f_S,R_{m,m-1,k}f_T\right\rangle} \\
        \leq & \pbra{\frac1m+\frac{k}{2^{m/2-2}}}\sum_{S\subseteq [m]:|S|\leq 2}\left\langle f_S,f_S\right\rangle + \frac{k}{2^{m/2-2}}\sum_{S\neq T\subseteq[m]:|S|,|T|\leq 2}\norm{f_S}_2\norm{f_T}_2\tag{\Cref{lem:square terms B=1}, \Cref{lem:cross terms B=1 to B=1}}\\
        = & \pbra{\frac1m+\frac{k}{2^{m/2-2}}}\left\langle f,f\right\rangle + \frac{k}{2^{m/2-2}}\sum_{S\neq T\subseteq[m]:|S|,|T|\leq 2}\norm{f_S}_2\norm{f_T}_2 \\
        \leq & \pbra{\frac1m+\frac{k}{2^{m/2-2}}}\left\langle f,f\right\rangle + \frac{k}{2^{m/2-2}}\sum_{S\neq T\subseteq[m]:|S|,|T|\leq 2}\left\langle f,f\right\rangle\\
        \leq & \pbra{\frac1m+\frac{k}{2^{m/2-2}}}\left\langle f,f\right\rangle + \frac{m^5k^2}{2^{m/2-2}}\left\langle f,f\right\rangle\\
        \leq & \pbra{\frac1m+\frac{m^5k}{2^{m/2-2}}}\left\langle f,f\right\rangle .
    \end{align*}
    Note that we can apply \Cref{lem:square terms B=1} and \Cref{lem:cross terms B=1 to B=1} because $k\geq 2$.
\end{proof}
\subsection{Cross Terms}\label{sec:cross terms}
We can use the same idea to bound the contributions from the cross terms.
\begin{lemma}\label{lem:cross terms B=1 to B=2}
    Let $f_1:\{\pm1\}^{mk}\to\R$ be supported on $B_{=1}$ and let $f_2:\{\pm1\}^{mk}\to\R$ be supported on $B_{\geq 2}$. Then 
    \begin{align*}
        \abs{\left\langle f_1,(R_{m,m-1,k}-R_{m,m,k})f_2\right\rangle} \leq \frac{\sqrt{m}k}{2^{m/2-2}}\norm{f_1}_2\norm{f_2}_2.
    \end{align*}
\end{lemma}
\begin{proof}
    For $X\in B_{\geq 2}$ we have $\Pr\sbra{X\to_{R_{m,m-1,k}} B_{=1}}\leq \frac{k^2m}{2^{m-1}}$. Apply \Cref{lem:escape probs} to find that
    \begin{align*}
        \abs{\left\langle f_1,R_{m,m-1,k}f_2\right\rangle} \leq \frac{\sqrt{m}k}{2^{m/2-1}}\norm{f_1}_2\norm{f_2}_2.
    \end{align*}
    For $X\in B_{\geq 2}$ we have $\Pr\sbra{X\to_{R_{m,m,k}} B_{=1}}\leq \frac{k^2m}{2^{m-1}}$. Apply \Cref{lem:escape probs} to find that
    \begin{align*}
        \abs{\left\langle f_1,R_{m,m,k}f_2\right\rangle} \leq \frac{\sqrt{m}k}{2^{m/2-1}}\norm{f_1}_2\norm{f_2}_2.
    \end{align*}
    Applying the triangle inequality completes the proof.
\end{proof}

\begin{lemma}\label{lem:cross terms B0}
    Let $f_0:\{\pm1\}^{mk}\to\R$ be supported on $B_{=0}$ and let $f_1:\{\pm1\}^{mk}\to\R$ be supported on $B_{=1}\cup B_{\geq 2}$. Then 
    \begin{align*}
        \abs{\left\langle f_0,(R_{m,m-1,k}-R_{m,m,k})f_1\right\rangle} = 0.
    \end{align*}
\end{lemma}
\begin{proof}
    For $X\in B_{\geq 2}\cup B_{=1}$ we have $\Pr\sbra{X\to_{R_{m,m-1,k}} B_{=0}}=\Pr\sbra{X\to_{R_{m,m,k}} B_{=0}}=0$. Apply \Cref{lem:escape probs} to bound 
    \begin{align*}
        &\abs{\left\langle f_0,(R_{m,m-1,k}-R_{m,m,k})f_1\right\rangle} \leq \abs{\left\langle f_0,R_{m,m-1,k}f_1\right\rangle} + \abs{\left\langle f_0,R_{m,m,k}f_1\right\rangle} \leq0.\qedhere
    \end{align*}
\end{proof}

\subsection{A Hybrid Argument for $f$ Supported on $B_{\geq 2}$}\label{sec:B>=2}


In this section we bound the square terms $\left\langle f_2,(R_{m,m-1,k}-R_{m,m,k})f_2\right\rangle$ for $f_2$ supported on $B_{\geq 2}$. As mentioned in \Cref{sec:overview}, our key idea is that when $k$ is small compared to $m$, the fraction of $X\in\{\pm1\}^{mk}$ with two identical columns is so small that applying the noise by randomly permuting the rows is almost the same as randomly replacing the rows with completely random rows. That is, sampling without replacement resembles sampling with replacement closely.

This observation allows us to pass from the random walk described by $R_{m,m-1,k}$ to a different random walk described by a nicer noise model described by operators we will call $Q_{m,m-1,k}$. The key tool we use is the bound given by \Cref{lem:TV distance bound} for relating total-variation distances between Markov chain transition probabilities to a more linear-algebraic notion of closeness, stated in terms of their transition matrices. Fourier-analytic techniques will then be useful to bound the spectral norm of these $Q$-operators.


\begin{definition}
    We define four random walk operators\footnotemark\footnotetext{$R_{m,m-1,k}$ and $R_{m,m,k}$ have already been defined, but we define them again here for ease of comparison.} $R_{m,m-1,k}$, $Q_{m,m-1,k}$, $R_{m,m,k}$, and $Q_{m,m,k}$ on $\R^{\{\pm1\}^{mk}}$. 
    \begin{itemize}     
        \item Define
        \begin{align*}
            (R_{m,m-1,k}f)(X)&=\underset{\mathbf{Y} \sim\mathcal{D}^{m,m-1,k}_{X}}{\E}\sbra{f(\mathbf{Y})}.
        \end{align*}
           
        \item To define $Q_{m,m-1,k}$, for any $X\in \cbra{\pm1}^{mk}$ we define the distribution $\mathcal{C}^{m,m-1,k}_{X}$ as follows. To sample $\mathbf{Y}$ from $\mathcal{C}^{m,m-1,k}_X$, we sample $\mathbf{a}\in[m]$ uniformly randomly and set $\mathbf{Y}^i_{\mathbf{a}}=X^i_{\mathbf{a}}$ for all $i\in[k]$. Then set $\mathbf{Y}^i_{a}$ uniformly randomly for $a\neq \mathbf{a}$. Then
        \begin{align*}
            (Q_{m,m-1,k}f)(X)&=\underset{\mathbf{Y}\sim\mathcal{C}^{m,m-1,k}_X}{\E}\sbra{f(\mathbf{Y})}.
        \end{align*}
        
        \item Define
        \begin{align*}
            (R_{m,m,k}f)(X)&=\Ex_{\mathbf{Y}\sim\mathcal{D}^{m,m,k}_X}\sbra{f(\mathbf Y)}.
        \end{align*}

        \item The operator $Q_{m,m,k}$ is defined by setting for each $f:\cbra{\pm1}^{mk}\to\R$
        \begin{align*}
            (Q_{m,m,k}f)(X)&=\Ex_{\mathbf{Y}\sim \mathrm{Unif}\pbra{\cbra{\pm1}^{mk}}}\sbra{f(\mathbf{Y})}.
        \end{align*}
    \end{itemize}
\end{definition}
\begin{fact}\label{fact:Q self-adjoint}
    The matrices $Q_{m,m-1,k}$ and $Q_{m,m,k}$ are self-adjoint and PSD for any $m,k$.
\end{fact}

We intend to show that $\abs{\left\langle f,(R_{m,m-1,k}-R_{m,m,k})f\right\rangle}$ is small for $f$ supported on $B_{\geq 2}$. We do so by a hybrid argument. In the following inequality, the first (and last) term on the RHS will be bounded by a simple bound on the total variation distance between the distribution $\mathcal{C}_X^{m,m-1,k}$ and $\mathcal{D}_X^{m,m-1,k}$ ($\mathcal{C}_X^{m,m,k}$ and $\mathcal{D}_X^{m,m,k}$). As mentioned before, the second term on the RHS will be bounded using Fourier analysis.
\begin{align*}\label{eq:hybrid}
    &\abs{\left\langle f,(R_{m,m-1,k}-R_{m,m,k})f\right\rangle} \\
        \leq& \abs{\left\langle f,\pbra{R_{m,m-1,k}-Q_{m,m-1,k}}f\right\rangle}+ \abs{\left\langle f,\pbra{Q_{m,m,k}-Q_{m,m-1,k}}f\right\rangle}+\abs{\left\langle f,\pbra{R_{m,m,k}-Q_{m,m,k}}f\right\rangle}.
\end{align*}


\subsubsection{The First Hybrid: $R_{m,m-1,k}$ to $Q_{m,m-1,k}$}

\begin{lemma}\label{lem:hybrid 1}
    Assume that $k\leq 2^{m/3}$. For any $f:\{\pm1\}^{mk}\to\R$ supported on $B_{\geq 2}$ and we have that
    \begin{align*}
        \abs{\left\langle f,(R_{m,m-1,k}-Q_{m,m-1,k})f\right\rangle}\leq \frac{k^2}{2^{m-1}}\left\langle f,f\right\rangle.
    \end{align*}
\end{lemma}
\begin{proof}
We directly compute (using the appropriate definitions of $p_0$ and $p_1$ given in \Cref{lem:TV distance bound}):
\begin{align*}
        &\abs{\left\langle f,(R_{m,m-1,k}-Q_{m,m-1,k})f\right\rangle}\\
        \leq & \sum_{X\in B_{\geq 2}}f(X)^2\sum_{Y\in B_{\geq 2}}\abs{\Pr\sbra{X\to_{R_{m,m-1,k}} Y}-\Pr\sbra{X\to_{Q_{m,m-1,k}}Y}}\tag{\Cref{lem:TV distance bound} + self-adjointness (\Cref{fact:self-adjoint}, \Cref{fact:Q self-adjoint})}\\
        \leq & \frac{k^2}{2^{m-1}}\sum_{X\in B_{\geq 2}}f(X)^2\tag{\Cref{eq:D0 D1 TV} below}\\
        \leq & \frac{k^2}{2^{m-1}}\left\langle f,f\right\rangle.
    \end{align*}
    It suffices to establish \Cref{eq:D0 D1 TV}. Assume that $X\in B_{\geq 2}$. Then
    \begin{align*}
        &\sum_{Y\in B_{\geq 2}}\abs{\Pr\sbra{X\to_{R_{m,m-1,k}} Y}-\Pr\sbra{X\to_{Q_{m,m-1,k}} Y}}\\
        =&\frac1m\sum_{a\in[m]}\sum_{Y\in B_{\geq 2}}\abs{\Pr_{\mathbf{Y}\sim\mathcal{D}^{m,m-1,k}_X}\sbra{\mathbf{Y}=Y|a\text{ fixed}}-\Pr_{\mathbf{Y}\sim\mathcal{C}^{m,m-1,k}_X}\sbra{\mathbf{Y}=Y|a\text{ fixed}}}\\
        =&\frac1m\sum_{a\in[m]}\sum_{Y\in B_{\geq 2},Y_a=X_a}\abs{\Pr_{\mathbf{Y}\sim\mathcal{D}^{m,m-1,k}_X}\sbra{\mathbf{Y}=Y|a\text{ fixed}}-\Pr_{\mathbf{Y}\sim\mathcal{C}^{m,m-1,k}_X}\sbra{\mathbf{Y}=Y|a\text{ fixed}}}\\
        =&\frac1{m}\sum_{a\in[m]}\sum_{Y\in B_{\geq 2},Y_a=X_a}\abs{ \prod_{i=0}^{k-1}\frac{1}{2^{m-1}-i}- \frac1{2^{(m-1)k}}}\tag{$k\leq 2^{m/3}\leq 2^m-2$ and $X\in B_{\geq 2}$}\\
        \leq &\frac1{m}\sum_{a\in[m]}\sum_{Y\in B_{\geq 2},Y_a=X_a}\abs{ \frac1{2^{(m-1)k}}\pbra{\prod_{i=0}^{k-1}\frac{2^{m-1}}{2^{m-1}-i}- 1}}\\
        \leq &\frac1{m}\sum_{a\in[m]}\sum_{Y\in B_{\geq 2},Y_a=X_a}\abs{ \frac1{2^{(m-1)k}}\pbra{1+\frac{k^2}{2^{m-1}}- 1}}\tag{$k\leq 2^{m/3}$, \Cref{fact:k^2}}\\
        = &\frac1{m}\sum_{a\in[m]}\sum_{Y\in B_{\geq 2},Y_a=X_a} \frac{k^2}{2^{m-1}2^{(m-1)k}}\\
        = &\sum_{Y\in B_{\geq 2},Y_1=X_1} \frac{k^2}{2^{m-1}2^{(m-1)k}}\\
        \leq & \frac{2^{mk}}{2^k}\cdot \frac{k^2}{2^{m-1}2^{(m-1)k}}\\
        =& \frac{k^2}{2^{m-1}}.\numberthis\label{eq:D0 D1 TV}
    \end{align*}
    Note that our computations for $\Pr_{\mathbf{Y}\sim\mathcal{D}^{m,m-1,k}_X}\sbra{\mathbf{Y}=Y|a\text{ fixed}}$ and $\Pr_{\mathbf{Y}\sim\mathcal{C}^{m,m-1,k}_X}\sbra{\mathbf{Y}=Y|a\text{ fixed}}$ relied on the fact that $Y\in B_{\geq 2}$.
\end{proof}




\subsubsection{The Second Hybrid: $Q_{m,m-1,k}$ to $Q_{m,m,k}$}
We use Fourier analysis to analyze the spectrum of the operator $Q_{m,m-1,k}-Q_{m,m,k}$, which will prove that $\norm{Q_{m,m-1,k}-Q_{m,m,k}}_{\mathrm{op}}$ is small. We use the Fourier characters as an eigenbasis for $Q_{m,m-1,k}-Q_{m,m,k}$.
\begin{fact}\label{fact:characters under A1-R1}
    Fix $S_1,\dots,S_k\subseteq[n]$. Then
    \begin{align*}
        \pbra{(Q_{m,m-1,k}-Q_{m,m,k})\chi_{S_1,\dots,S_k}}=&
        \begin{cases}
            \frac1m\chi_{S_1,\dots,S_k} &\text{ if }S_1\cup \dots\cup S_k=\{a\} \text{ for some $a\in [m]$}.\\
            0 &\text{ otherwise.}
        \end{cases}
    \end{align*}
\end{fact}
\begin{proof}
    If $\abs{\bigcup_i S_i}=1$ then $S_1=\dots=S_k=\emptyset$ and it is clear that 
    \begin{align*}
        Q_{m,m-1,k}\chi_{S_1,\dots,S_k}=1=Q_{m,m,k}\chi_{S_1,\dots,S_k}.
    \end{align*}
    Now assume $\abs{\bigcup_i S_i}\geq 2$. Then for any $X=(X^1,\dots,X^k)\in \{\pm1\}^{mk}$, 
    \begin{align*}
        &Q_{m,m-1,k}\chi_{S_1,\dots,S_k}(X^1,\dots,X^k)\\
        =&\Ex_{(\mathbf{Y}^1,\dots,\mathbf{Y}^k)\sim\mathcal{C}^{m,m-1,k}_{X}}\sbra{\chi_{S_1,\dots,S_k}(\mathbf{Y}^1,\dots,\mathbf{Y}^k)}\\
        =&\frac1m\sum_{a'\in[m]}\Ex_{\substack{(\mathbf{Y}^1,\dots,\mathbf{Y}^k)\sim\mathcal{C}^{m,m-1,k}_{X}\\\mathbf{a}=a'}}\sbra{\chi_{S_1,\dots,S_k}(\mathbf{Y}^1,\dots,\mathbf{Y}^k)}\\
        =&\frac1m\sum_{a'\in[m]}\Ex_{\substack{(\mathbf{Y}^1,\dots,\mathbf{Y}^k)\sim\mathcal{C}^{m,m-1,k}_{X}\\\mathbf{a}=a'}}\sbra{\prod_{i\in[k]}\prod_{a\in S_i} \mathbf{Y}^i_a}\\
        =&\frac1m\sum_{a'\in[m]}\Ex_{\substack{(\mathbf{Y}^1,\dots,\mathbf{Y}^k)\sim\mathcal{C}^{m,m-1,k}_{X}\\\mathbf{a}=a'}}\sbra{\pbra{\prod_{\substack{i\in[k]\\a'\in S_i}}\mathbf{Y}^i_{a'}}\pbra{\prod_{i\in[k]}\prod_{a\in S_i\setminus\{a'\}} \mathbf{Y}^i_a }}\\
        =&\frac1m\sum_{a'\in[m]}\pbra{\prod_{\substack{i\in[k]\\a'\in S_i}}\mathbf{Y}^i_{a'}}\Ex_{\substack{(\mathbf{Y}^1,\dots,\mathbf{Y}^k)\sim\mathcal{C}^{m,m-1,k}_{X}\\\mathbf{a}=a'}}\sbra{\prod_{i\in[k]}\prod_{a\in S_i\setminus\{a'\}} \mathbf{Y}^i_a }\\
        =&\frac1m\sum_{a'\in[m]}\pbra{\prod_{\substack{i\in[k]\\a'\in S_i}}\mathbf{Y}^i_{a'}}\Ex_{(\mathbf{Y}^1,\dots,\mathbf{Y}^k)\in\pbra{\{\pm1\}^n}^{[k]\setminus\{a'\}}}\sbra{\chi_{S_1\setminus\{a'\},\dots,S_k\setminus\{a'\}}(\mathbf{Y}^1,\dots,\mathbf{Y}^k) }\\
        =&0.\tag{since at least one of the $S_i\setminus\{a'\}$ is nonempty}
    \end{align*}
    It is easy to see that $Q_{m,m,k}\chi_{S_1,\dots,S_k}=0$, since $S_1\cup\dots\cup S_k\neq\emptyset$.

    Now assume that $\bigcup_i S_i=\{a'\}$ for some $a'\in[m]$. Then it is clear that $Q_{m,m,k}\chi_{S_1,\dots,S_k}=0$, since $S_1\cup\dots\cup S_k\neq\emptyset$. We now compute
    \begin{align*}
        &Q_{m,m-1,k}\chi_{S_1,\dots,S_k}(X^1,\dots,X^k)\\
        =&\Ex_{(\mathbf{Y}^1,\dots,\mathbf{Y}^k)\sim\mathcal{C}^{m,m-1,k}_{X}}\sbra{\chi_{S_1,\dots,S_k}(\mathbf{Y}^1,\dots,\mathbf{Y}^k)}\\
        =&\frac1m\sum_{a\in[m]}\Ex_{\substack{(\mathbf{Y}^1,\dots,\mathbf{Y}^k)\sim\mathcal{C}^{m,m-1,k}_{X}\\\mathbf{a}=a}}\sbra{\prod_{\substack{i\in[k]\\S_i=\{a'\}}}\mathbf{Y}^k_{a'}}\\
        =&\frac1m\Ex_{\substack{(\mathbf{Y}^1,\dots,\mathbf{Y}^k)\sim\mathcal{C}^{m,m-1,k}_{X}\\\mathbf{a}=a'}}\sbra{\prod_{\substack{i\in[k]\\S_i=\{a'\}}}\mathbf{Y}^k_{a'}}\tag{If $\mathbf{a}\neq a'$ then the $\mathbf{Y}^i_{a'}$ are uniformly random.}\\
        =&\frac1m\Ex_{\substack{(\mathbf{Y}^1,\dots,\mathbf{Y}^k)\sim\mathcal{C}^{m,m-1,k}_{X}\\\mathbf{a}=a'}}\sbra{\prod_{\substack{i\in[k]\\S_i=\{a'\}}}X^k_{a'}}\\
        =&\frac1m\chi_{S_1,\dots,S_k}(X^1,\dots,X^k).
    \end{align*}
    Therefore $\pbra{Q_{m,m-1,k}-Q_{m,m,k}}\chi_{S_1,\dots,S_k}=\frac1m\chi_{S_1,\dots,S_k}(X^1,\dots,X^k)-0=\frac1m\chi_{S_1,\dots,S_k}(X^1,\dots,X^k)$.
\end{proof}

\begin{corollary}\label{cor:hybrid 2}
    Let $f:\{\pm1\}^{mk}\to\R$. Then
    \begin{align*}
        \abs{\left\langle f,(Q_{m,m-1,k}-Q_{m,m,k})f\right\rangle} \leq \frac1m \left\langle f,f\right\rangle.
    \end{align*}
\end{corollary}
\begin{proof}
    The $\chi_{S_1,\dots,S_k}$ form an orthonormal eigenbasis (\Cref{fact:fourier characters}) for $Q_{m,m-1,k}-Q_{m,m,k}$, and by \Cref{fact:characters under A1-R1} each basis element has eigenvalue with absolute value at most $\frac1m$.
\end{proof}

\subsubsection{The Third Hybrid: $Q_{m,m,k}$ to $R_{m,m,k}$}

\begin{lemma}\label{lem:hybrid 3}
    Assume that $k\leq 2^{m/3}$. For any $f:\{\pm1\}^{mk}\to\R$ supported on $B_{\geq 2}$ we have
    \begin{align*}
        \abs{\left\langle f,(R_{m,m,k}-Q_{m,m,k})f\right\rangle} \leq \frac{k^2}{2^{m}}\norm{f}_2^2 .
    \end{align*}
\end{lemma}
\begin{proof}
    As in the proof of \Cref{lem:hybrid 1} we find that (with the appropriate definitions of $p_0$ and $p_1$ for use of \Cref{lem:TV distance bound},
    \begin{align*}
       &\abs{\left\langle f,(R_{m,m,k}-Q_{m,m,k})f\right\rangle}\\
       \leq & \sum_{X\in B_{\geq 2}}f(X)^2\sum_{Y\in B_{\geq 2}}\abs{\Pr\sbra{X\to_{R_{m,m,k}} Y}-\Pr\sbra{X\to_{Q_{m,m,k}} Y}}\tag{\Cref{lem:TV distance bound} + self-adjointness (\Cref{fact:self-adjoint}, \Cref{fact:Q self-adjoint})}\\
        \leq & \frac{k^2}{2^m}\sum_{X\in B_{\geq 2}}f(X)^2\tag{\Cref{eq:TV for hybrid 3} below}\\
        \leq & \frac{k^2}{2^m}\left\langle f,f\right\rangle.
    \end{align*}
    It remains to prove \Cref{eq:TV for hybrid 3}:
    \begin{align*}
        &\sum_{Y\in B_{\geq 2}}\abs{\Pr\sbra{X\to_{R_{m,m,k}} Y}-\Pr\sbra{X\to_{Q_{m,m,k}} Y}}\\
        =&\sum_{Y\in B_{\geq 2}}\abs{\Pr_{\mathbf{Y}\sim\mathcal{D}^{m,m,k}_X}\sbra{\mathbf{Y}=Y}-\Pr_{\mathbf{Y}\sim\mathcal{C}^{m,m,k}_X}\sbra{\mathbf{Y}=Y}}\\
        =&\sum_{Y\in B_{\geq 2}}\abs{ \prod_{i=0}^{k-1}\frac{1}{2^{m}-i}- \frac1{2^{mk}}}\tag{$k\leq 2^{m/3}\leq 2^m-2$ and $X\in B_{\geq 2}$}\\
        \leq &\sum_{Y\in B_{\geq 2}}\abs{ \frac1{2^{mk}}\pbra{\prod_{i=0}^{k-1}\frac{2^{m}}{2^{m}-i}- 1}}\\
        \leq &\sum_{Y\in B_{\geq 2}}\abs{ \frac1{2^{mk}}\pbra{1+\frac{k^2}{2^m}- 1}}\tag{$k\leq 2^{m/3}$, \Cref{fact:k^2}}\\
        = &\sum_{Y\in B_{\geq 2}} \frac{k^2}{2^m2^{mk}}\\
        \leq &{2^{mk}}\cdot \frac{k^2}{2^m2^{mk}}\\
        =& \frac{k^2}{2^m}.\numberthis\label{eq:TV for hybrid 3}
    \end{align*}
    Note that our computations relied on the fact that $Y\in B_{\geq 2}$.
\end{proof}




\subsubsection{Putting Hybrids Together}
\begin{proposition}\label{prop:B>=2 square term}
    Assume that $k\leq 2^{m/3}$. For any $f:\{\pm1\}^{mk}\to\R$ supported on $B_{\geq 2}$, we have
    \begin{align*}
        \abs{\left\langle f,(R_{m,m-1,k}-R_{m,m,k})f\right\rangle} \leq \pbra{\frac1m+\frac{k^2}{2^{m/2}}}\left\langle f,f\right\rangle
    \end{align*}
\end{proposition}
\begin{proof}
    By the triangle inequality,
    \begin{align*}
        &\abs{\left\langle f,(R_{m,m-1,k}-R_{m,m,k})f\right\rangle} \\
        \leq& \abs{\left\langle f,\pbra{R_{m,m-1,k}-Q_{m,m-1,k}}f\right\rangle}+ \abs{\left\langle f,\pbra{Q_{m,m,k}-Q_{m,m-1,k}}f\right\rangle}+\abs{\left\langle f,\pbra{R_{m,m,k}-Q_{m,m,k}}f\right\rangle}\\
        \leq & \frac{k^2}{2^{m-1}}\left\langle f,f\right\rangle+\frac1m\left\langle f,f\right\rangle+\frac{k^2}{2^{m}}\left\langle f,f\right\rangle \tag{\Cref{lem:hybrid 1}, \Cref{cor:hybrid 2}, \Cref{lem:hybrid 3}}\\
        \leq  &\pbra{\frac1m+\frac{k^2}{2^{m-2}}}\left\langle f,f\right\rangle .
    \end{align*}
    Note we can apply \Cref{lem:hybrid 1} and \Cref{lem:hybrid 3} because $f$ is supported on $B_{\geq 2}$.
\end{proof}

\section{Proof of \Cref{thm:nachtergaele hypothesis}}\label{sec:nachtergaele hypothesis}


In this section let $m$ and $k\leq 2^m-2$ be fixed positive integers. As in the proof of \Cref{thm:small k} (via its restatement \Cref{thm:small k internal}) we will break $\R^{\{\pm1\}^{mk}}$ into orthogonal components and bound the contributions of each of these cross terms to evaluation on the quadratic form given by $R_{m,[m-\ell-1,m],k}\pbra{R_{m,[m-1],k}-R_{m,[m],k}}$. 

Define the following subsets of $\{\pm1\}^{mk}$:
\begin{itemize}
    \item $B_{=0}=\cbra{X\in\{\pm1\}^{mk}:\exists i\neq j\in[k], X^i=X^j}$.\footnotemark\footnotetext{$B_{=0}$ was already defined in \Cref{sec:small k} but we define it again here for convenience.} 
    \item $B_{\geq1}^{\mathrm{coll}}=\cbra{X\in\{\pm1\}^{mk}\setminus B_{=0}:\exists i\neq j\in[k], X^i_{[m-\ell-1,m-1]}=X^j_{[m-\ell-1,m-1]}}$.
    \item $B_{\geq1}^{\mathrm{safe}}=\cbra{X\in\{\pm1\}^{mk}:\forall i\neq j\in[k], X^i_{[m-\ell-1,m-1]}\neq X^j_{[m-\ell-1,m-1]}}$.
\end{itemize}
Note that these sets form a partition of $\{\pm1\}^{mk}$.

As in the proof of \Cref{thm:small k}, the contribution from the parts of functions supported on $B_{=0}$ will be bounded by induction. The component $B_{\geq1}^{\mathrm{safe}}$ will play the role that $B_{\geq2}$ did: the part of the domain on which the noise model induced by random permutations (sampling without replacement) to the bits in $[m-\ell-1,m-1]$ is close to the noise model induced by completely random replacement of bits (sampling with replacement). $B_{\geq1}^{\mathrm{coll}}$ is the component on which these two noise models are not similar, but as in the case of $B_{=1}$, this set will already show good expansion.


The following is equivalent to \Cref{thm:nachtergaele hypothesis} because for any $f:\{\pm1\}^{mk}\to \R$ we have that
\begin{align*}
    \norm{R_{m,[m-\ell-1,m],k}\pbra{R_{m,[m-1],k}-R_{m,[m],k}}}_{\mathrm{op}}=&
    \norm{\pbra{R_{m,[m-\ell-1,m],k}\pbra{R_{m,[m-1],k}-R_{m,[m],k}}}^*}_{\mathrm{op}}\\
    =&\norm{\pbra{R_{m,[m-1],k}-R_{m,[m],k}}R_{m,[m-\ell-1,m],k}}_{\mathrm{op}}.
\end{align*}
To bound this quantity, we can use that
\begin{align*}
    &\norm{\pbra{R_{m,[m-1],k}-R_{m,[m],k}}R_{m,[m-\ell-1,m],k}f}^2_2\\
    =&\left\langle \pbra{R_{m,[m-1],k}-R_{m,[m],k}}R_{m,[m-\ell-1,m],k}f,\pbra{R_{m,[m-1],k}-R_{m,[m],k}}R_{m,[m-\ell-1,m],k}f\right\rangle\\
    =&\left\langle R_{m,[m-\ell-1,m],k}\pbra{R_{m,[m-1],k}-R_{m,[m],k}}\pbra{R_{m,[m-\ell-1,m],k}\pbra{R_{m,[m-1],k}-R_{m,[m],k}}}^*f,f\right\rangle\\
    =&\left\langle R_{m,[m-\ell-1,m],k}\pbra{R_{m,[m-1],k}-R_{m,[m],k}}^2R_{m,[m-\ell-1,m],k}f,f\right\rangle\\
    =&\left\langle R_{m,[m-\ell-1,m],k}\pbra{R_{m,[m-1],k}-R_{m,[m],k}-R_{m,[m],k}+R_{m,[m],k}}R_{m,[m-\ell-1,m],k}f,f\right\rangle\\
    =&\left\langle f,R_{m,[m-\ell-1,m],k}\pbra{R_{m,[m-1],k}-R_{m,[m],k}}R_{m,[m-\ell-1,m],k}f\right\rangle.
\end{align*}

\begin{theorem}[\Cref{thm:nachtergaele hypothesis} restated]\label{thm:nachtergaele hypothesis internal}
    Fix any $m\geq100$ and set $100\leq\ell\leq m$. Suppose $k\leq 2^{\ell/10}$. Then we have for any $f:\{\pm1\}^{mk}\to\R$ that
    \begin{align*}
        \abs{\left\langle f,R_{m,[m-\ell-1,m],k}\pbra{R_{m,[m-1],k}-R_{m,[m],k}}R_{m,[m-\ell-1,m],k}f\right\rangle}\leq \frac{k^3}{2^{\ell/4-60}}\left\langle f,f\right\rangle\leq \frac1{\ell^2} \left\langle f,f\right\rangle.
    \end{align*}
\end{theorem}
\begin{proof}
    We prove by induction on $k$. In the base case $k=1$ and the result holds by the following argument when we write $f=f_2$. Now assume that the result holds for real functions on $\{\pm1\}^{m(k-1)}$. 
    
    Let $f:\{\pm1\}^{mk}\to\R$. Write $f=f_0+f_1+f_2$ where $f_0$ is supported on $B_{=0}$, $f_1$ is supported on $B_{\geq1}^{\mathrm{coll}}$, and $f_2$ is supported on $B_{\geq1}^{\mathrm{safe}}$. By \Cref{lem:cross terms B0} applied with $R_{m,[m-\ell-1,m],k}$, $R_{m,[m-1],k}$ and $R_{m,[m],k}$ and $B_{=0}$ and $\{\pm1\}^{mk}\setminus B_{=0}=B_{\geq1}^{\mathrm{coll}}\cup B_{\geq1}^{\mathrm{safe}}$, the other cross terms vanish, and we have
    \begin{align*}
        &\abs{ \left\langle f,R_{m,[m-\ell-1,m],k}\pbra{R_{m,[m-1],k}-R_{m,[m],k}}R_{m,[m-\ell-1,m],k}f\right\rangle }\\
        =&\abs{ \left\langle f,Af\right\rangle }\tag{Defining $A$ for convenience of notation}\\
        \leq&\abs{ \left\langle f_0,Af_0\right\rangle }+\abs{ \left\langle f_2,Af_2\right\rangle }+\abs{ \left\langle f_1,A(f_1+f_2)\right\rangle }+\abs{ \left\langle f_2,Af_1\right\rangle }\\
        =&\abs{ \left\langle f_0,Af_0\right\rangle }+\abs{ \left\langle f_2,Af_2\right\rangle }+\abs{ \left\langle f_1,A(f_1+f_2)\right\rangle }+\abs{ \left\langle f_1,A^*f_2\right\rangle }\\
        \leq&\frac{(k-1)^3}{2^{\ell/4-60}}\left\langle f_0,f_0 \right\rangle+\frac{k^2}{2^{\ell/4-20}}\langle f_2,f_2\rangle  +\frac{k^2}{2^{\ell/2-4}}\norm{f_1+f_2}_2\norm{f_2}_2+\frac{k^2}{2^{\ell/2-4}}\norm{f_1}_2\norm{f_2}_2\tag{\Cref{lemma:f supported on B0} + Induction, \Cref{cor:square terms safe to safe}, \Cref{cor:cross terms safe to coll}, \Cref{cor:cross terms safe to coll}, in that order}\\
        \leq & \frac{k^3}{2^{\ell/4-60}}\langle f,f\rangle.
    \end{align*}
    In the last step we used that $k\leq 2^{\ell/10}$.
\end{proof}




Similar to \Cref{sec:B>=2} we will again argue that because $\ell$ is large, applying a random permutation to the same indices of all elements of a tuple of binary strings is similar to replacing those indices with random binary strings in all elements of a tuple. To model this situation, we similarly use the matrix $Q_{m,S,k}$ to denote the random walk matrix induced by the following distribution $\mathcal{C}_X^{m,S,k}$ for each $X\in\{\pm1\}^{mk}$. To draw $\mathbf{Y}\sim\mathcal{C}_X^{m,S,k}$, set $\mathbf{Y}^i_{[m]\setminus S}=X^i_{[m]\setminus S}$ and set $\mathbf{Y}^i_S$ uniformly at random for all $i\in[k]$.

\subsection{$f$ Supported on $B_{\geq 1}^{\mathrm{coll}}$ and Cross Terms}\label{sec:coll and cross terms}

The first step in comparing the noise model generated by the application of a random permutation to a substring of each string in a $k$-tuple of $n$-bit strings is showing that the set of strings to which this comparison fails already exhibits good expansion.

In this section we leverage that $B_{\geq1}^{\mathrm{coll}}$ is a set in $\{\pm1\}^{mk}$ with very good expansion. For intuition, $B_{\geq1}^{\mathrm{coll}}$ plays the role that $B_{=1}$ played in \Cref{sec:small k}. The following lemma formalizes the expansion property that we need.

\begin{lemma}\label{lem:anything transition into coll}
    For any $X\in B_{\geq 1}^{\mathrm{coll}}\cup B_{\geq 1}^{\mathrm{safe}}$ we have that for any $I,J\subseteq[m]$ such that $I\cup J=[m]$ and $I,J\supseteq [m-\ell-1,m-1]$,
    \begin{align*}
        \Pr\sbra{X\to_{R_{m,I,k}R_{m,J,k}} B_{\geq 1}^{\mathrm{coll}}}
        \leq\frac{k^2}{2^{\ell-3}}.
    \end{align*}
    Moreover, we also have
    \begin{align*}
        \Pr\sbra{X\to_{R_{m,I,k}R_{m,J,k}R_{m,I,k}} B_{\geq 1}^{\mathrm{coll}}}
        \leq\frac{k^2}{2^{\ell-3}}.
    \end{align*}
    In fact, this holds with $Q$ replacing $R$ anywhere in this inequality.
\end{lemma}
\begin{proof}
    We first prove the first statement (with two steps) for the case of drawing from the distribution $\mathcal{D}$ (corresponding to the random walk operator $R$), but the proof translates easily to the case for $\mathcal{C}$ (corresponding to the random wak operator $Q$). Model the process as first drawing $\bm{Y}\sim \mcD^{m,I,k}_X$ and then $\bm{Z}\sim \mcD^{m,J,k}_{\bm{Y}}$. Fix any $X\in B_{\geq1}^{\mathrm{coll}}\cup B_{\geq1}^{\mathrm{safe}}$. Define the following subset of pairs of indices in $\binom{[k]}{2}$:
    \begin{align*}
        P_1=& \cbra{\{i,j\}\in\binom{[k]}{2}:X^i_{I}=X^j_{I}}.
    \end{align*}
    Assume that $\{i,j\}\in P_1$. Then $\mathbf{Y}^i_I=\mathbf{Y}^j_I$ with probability 1. But now note that $X^i_{J}\neq X^j_{J}$ since otherwise $X^i=X^j$, which contradicts that $X\not\in B_{=0}$. Therefore, 
    \begin{align*}
        &\Pr_{\substack{\mathbf{Y}\sim \mathcal{D}_{X}^{m,I,k}\\\mathbf{Z}\sim\mathcal{D}_{\mathbf{Y}}^{m,J,k}}}\sbra{\mathbf{Z}^i_{[m-\ell-1,m-1]}=\mathbf{Z}^j_{[m-\ell-1,m-1]}}\\
        =&\Pr_{\substack{\mathbf{Z}\sim \mathcal{D}_{X}^{m,J,k}}}\sbra{\mathbf{Z}^i_{[m-\ell-1,m-1]}=\mathbf{Z}^j_{[m-\ell-1,m-1]}}\\
        =& \frac1{2^{\ell}-1}.\numberthis\label{eq:P1 contribution}
    \end{align*}
    Now assume that $\{i,j\}\not\in P_1$. Let $E$ be the event that $\mathbf{Y}^i_{[m-\ell-1,m-1]}=\mathbf{Y}^j_{[m-\ell-1,m-1]}$. Then 
    \begin{align*}
        &\Pr_{\substack{\mathbf{Y}\sim \mathcal{D}_{X}^{m,I,k}\\\mathbf{Z}\sim\mathcal{D}_{\mathbf{Y}}^{m,J,k}}}\sbra{\mathbf{Z}^i_{[m-\ell-1,m-1]}=\mathbf{Z}^j_{[m-\ell-1,m-1]}}\\
        =&\Pr_{\mathbf{Y}\sim \mathcal{D}_{X}^{m,I,k}}\sbra{E}\Pr_{\substack{\mathbf{Y}\sim \mathcal{D}_{X}^{m,I,k}\\\mathbf{Z}\sim\mathcal{D}_{\mathbf{Y}}^{m,J,k}}}\sbra{\mathbf{Z}^i_{[m-\ell-1,m-1]}=\mathbf{Z}^j_{[m-\ell-1,m-1]}|E}\\
        &+\Pr_{\mathbf{Y}\sim \mathcal{D}_{X}^{m,I,k}}\sbra{\overline{E}}\Pr_{\substack{\mathbf{Y}\sim \mathcal{D}_{X}^{m,I,k}\\\mathbf{Z}\sim\mathcal{D}_{\mathbf{Y}}^{m,J,k}}}\sbra{\mathbf{Z}^i_{[m-\ell-1,m-1]}=\mathbf{Z}^j_{[m-\ell-1,m-1]}|\overline{E}}\\
        \leq& \frac1{2^\ell} + \Pr_{\substack{\mathbf{Y}\sim \mathcal{D}_{X}^{m,I,k}\\\mathbf{Z}\sim\mathcal{D}_{\mathbf{Y}}^{m,J,k}}}\sbra{\mathbf{Z}^i_{[m-\ell-1,m-1]}=\mathbf{Z}^j_{[m-\ell-1,m-1]}|\overline{E}}\\
        =& \frac1{2^\ell}+\frac1{2^\ell-1}\\
        \leq&\frac1{2^{\ell-2}}.\numberthis\label{eq:not P1 contribution}
    \end{align*}
    To complete the proof, we use a union bound and find
    \begin{align*}
        &\Pr_{\substack{\mathbf{Y}\sim \mathcal{D}_{X}^{m,I,k}\\\mathbf{Z}\sim\mathcal{D}_{\mathbf{Y}}^{m,J,k}}}\sbra{\mathbf{Z}\in B_{\geq1}^{\mathrm{coll}}}\\
        \leq& \sum_{\{i,j\}\in\binom{[k]}{2}}\Pr_{\substack{\mathbf{Y}\sim \mathcal{D}_{X}^{m,I,k}\\\mathbf{Z}\sim\mathcal{D}_{\mathbf{Y}}^{m,J,k}}}\sbra{\mathbf{Z}^i_{[m-\ell-1,m-1]}=\mathbf{Z}^j_{[m-\ell-1,m-1]}}\\
        \leq&\sum_{\{i,j\}\in P_1}\Pr_{\substack{\mathbf{Y}\sim \mathcal{D}_{X}^{m,I,k}\\\mathbf{Z}\sim\mathcal{D}_{\mathbf{Y}}^{m,J,k}}}\sbra{\mathbf{Z}^i_{[m-\ell-1,m-1]}=\mathbf{Z}^j_{[m-\ell-1,m-1]}}\\
        &+\sum_{\{i,j\}\not\in P_1}\Pr_{\substack{\mathbf{Y}\sim \mathcal{D}_{X}^{m,I,k}\\\mathbf{Z}\sim\mathcal{D}_{\mathbf{Y}}^{m,J,k}}}\sbra{\mathbf{Z}^i_{[m-\ell-1,m-1]}=\mathbf{Z}^j_{[m-\ell-1,m-1]}}\\
        \leq& \sum_{\{i,j\}\in P_1}\frac1{2^{\ell}}+\sum_{\{i,j\}\not\in P_1}\frac1{2^{\ell-1}}\\
        \leq&\sum_{\{i,j\}\in \binom{[k]}{2}}\frac1{2^{\ell-2}}\\
        \leq& \frac{k^2}{2^{\ell-2}}\numberthis\label{eq:two step bound}.
    \end{align*}
    To prove the second statement (with three steps taken), note that the probability that a random permutation applied to the bits in $I$ sends an element of $B_{\geq1}^{\mathrm{safe}}$ to an element of $B_{\geq1}^{\mathrm{coll}}$ is at most $\frac{k^2}{2^{\ell}-1}$. Applying a union bound and the bound \Cref{eq:two step bound} completes the proof.
\end{proof}


The expansion property just proved results in the following linear-algebraic statements, which show that the contributions from $B_{\geq1}^{\mathrm{coll}}$ to our operator norm bound is extremely small.


\begin{lemma}\label{lem:randomize everything}
    Let $k\geq 2$ and $f:\{\pm1\}^{mk}$ be supported on $B_{\geq 1}^{\mathrm{coll}}$ and $g:\{\pm1\}^{mk}$ be supported on $B_{\geq 1}^{\mathrm{safe}}\cup B_{\geq 1}^{\mathrm{coll}}$. Let $I$ and $J$ be such that $I\cup J=[m]$. Then
    \begin{align*}
        &\abs{\left\langle f,R_{m,J,k}R_{m,I,k}R_{m,J,k}g\right\rangle}      \leq\sqrt{\frac{k^2}{2^{\ell-3}}}\norm{f}_2\norm{g}_2,\\
        &\abs{\left\langle f,R_{m,J,k}R_{m,I,k}g\right\rangle}\leq\sqrt{\frac{k^2}{2^{\ell-3}}}\norm{f}_2\norm{g}_2.
    \end{align*}
    Moreover, this holds with $Q$ in place of $R$ anywhere.
\end{lemma}
\begin{proof}
    The inequality directly follows from \Cref{lem:escape probs} and \Cref{lem:anything transition into coll}. We can apply \Cref{lem:escape probs} because the uniform distribution on $\{\pm1\}^{mk}$ is indeed a stationary distribution under $R_{m,S,k}$ for any $S$ (\Cref{fact:uniform is stationary}). The $Q$ case follows from the fact that \Cref{lem:anything transition into coll} applies for the distributions $\mathcal{C}$ too.
\end{proof}

\begin{corollary}\label{cor:cross terms safe to coll}
    Let $k\geq 2$ and $f:\{\pm1\}^{mk}$ be supported on $B_{\geq 1}^{\mathrm{coll}}$ and $g:\{\pm1\}^{mk}$ be supported on $B_{\geq 1}^{\mathrm{safe}}\cup B_{\geq 1}^{\mathrm{coll}}$. Then 
    \begin{align*}
        \abs{\left\langle f,R_{m,[m-\ell-1,m],k}\pbra{R_{m,[m-1],k}-R_{m,[m],k}}R_{m,[m-\ell-1,m],k}g\right\rangle }\leq \frac{k}{2^{\ell/2-4}}\norm{f}_2\norm{g}_2,\\
        \abs{\left\langle f,R_{m,[m-\ell-1,m],k}\pbra{R_{m,[m-1],k}-R_{m,[m],k}}g\right\rangle }\leq \frac{k}{2^{\ell/2-4}}\norm{f}_2\norm{g}_2.
    \end{align*}
\end{corollary}
\begin{proof}
    We compute
    \begin{align*}
        &\abs{\left\langle f,R_{m,[m-\ell-1,m],k}\pbra{R_{m,[m-1],k}-R_{m,[m],k}}R_{m,[m-\ell-1,m],k}g\right\rangle }\\
        \leq&\abs{\left\langle f,{R_{m,[m-\ell-1,m],k}R_{m,[m-1],k}}R_{m,[m-\ell-1,m],k}g\right\rangle}+\abs{\left\langle f,{R_{m,[m-\ell-1,m],k}R_{m,[m],k}}R_{m,[m-\ell-1,m],k}g\right\rangle }\\
        =&\abs{\left\langle f,{R_{m,[m-\ell-1,m],k}R_{m,[m-1],k}}R_{m,[m-\ell-1,m],k}g\right\rangle}+\abs{\left\langle f,{R_{m,[m],k}}g\right\rangle }\\
        \leq&\frac{k}{2^{\ell/2-3}}\norm{f}_2\norm{g}_2+\frac{k}{2^{\ell/2}}\norm{f}_2\norm{g}_2\tag{\Cref{lem:randomize everything}}\\
        \leq& \frac{k}{2^{\ell/2-4}}\norm{f}_2\norm{g}_2.
    \end{align*}
    The proof of the second statement is similar.
\end{proof}






\subsection{A Hybrid Argument for $f$ Supported on $B_{\geq 1}^{\mathrm{safe}}$}

The role that $B_{\geq 1}^{\mathrm{safe}}$ plays in this section is similar to the role played by $B_{\geq2}$ in \Cref{sec:small k}. It is the region of $\{\pm1\}^{mk}$ that is ``well-behaved" in the sense that the nicer noise model given by the $Q$ operators is similar to the noise model given by the $R$ operators on this region of $\{\pm1\}^{mk}$. 

In particular, because these noise models behave similarly when restricted to these $k$-tuples, we can bound the difference between the corresponding transition matrices as follows:

\begin{lemma}\label{lem:R to Q hybrid local}
    Assume that $k\leq 2^{\ell/10}$ and $f,g:\{\pm1\}^{mk}$ be supported on $B_{\geq 1}^{\mathrm{safe}}$. Then for any $S\supseteq [m-\ell-1,m-1]$, we have
    \begin{align*}
        \abs{\left\langle f,\pbra{Q_{m,S,k}-R_{m,S,k}}g\right\rangle}\leq \frac{k^2}{2^{\ell-1}}\norm{f}_2\norm{g}_2.
    \end{align*}
\end{lemma}
\begin{proof}
    By assumption $f$ and $g$ are supported on $B_{\geq1}^{\mathsf{safe}}$. Therefore, by \Cref{lem:TV distance bound} and self-adjointness of $Q_{m,S,k}$ and $R_{m,S,k}$ (\Cref{fact:self-adjoint}, \Cref{fact:Q self-adjoint}) we have
    \begin{align*}
        &\abs{\left\langle f,(R_{m,S,k}-Q_{m,S,k})g\right\rangle}\\
        \leq & \sqrt{\sum_{X\in B_{\geq 1}^{\mathrm{safe}}}f(X)^2\sum_{Y\in B_{\geq 1}^{\mathrm{safe}}}\abs{\Pr\sbra{X\to_{R_{m,S,k}} Y}-\Pr\sbra{X\to_{Q_{m,S,k}} Y}}}\\
        &\;\;\;\;\;\cdot\sqrt{\sum_{X\in B_{\geq 1}^{\mathrm{safe}}}g(X)^2\sum_{Y\in B_{\geq 1}^{\mathrm{safe}}}\abs{\Pr\sbra{X\to_{R_{m,S,k}} Y}-\Pr\sbra{X\to_{Q_{m,S,k}} Y}}}\\
        \leq & \frac{k^2}{2^{\ell-1}}\sqrt{\sum_{X\in B_{\geq 1}^{\mathrm{safe}}}f(X)^2}\sqrt{\sum_{X\in B_{\geq 1}^{\mathrm{safe}}}g(X)^2}\tag{\Cref{eq:tv for safe} below}\\
        =& \frac{k^2}{2^{\ell-1}}\norm{f}_2\norm{g}_2.
    \end{align*}
Here $p_0$ and $p_1$ are as used below. Now it suffices to establish \Cref{eq:tv for safe}. Assume $X\in B_{\geq 1}^{\mathrm{safe}}$. Then because $S\supseteq [m-\ell-1,m-1]$, we know that for all $i\neq j$ we have $X^i_{S}\neq X^j_S$.
\begin{align*}
    &\sum_{Y\in B_{\geq 1}^{\mathrm{safe}}}\abs{\Pr\sbra{X\to_{R_{m,S,k}} Y}-\Pr\sbra{X\to_{Q_{m,S,k}} Y}}\\
    =&\sum_{\substack{Y\in B_{\geq 1}^{\mathrm{safe}}\\\forall i\in[k], a\in [m]\setminus S, Y^i_a=X^i_a}}\abs{\Pr_{\mathbf{Y}\sim\mathcal{D}^{m,S,k}_X}\sbra{\mathbf{Y}=Y}- \Pr_{\mathbf{Y}\sim\mathcal{C}^{m,S,k}_X}\sbra{\mathbf{Y}=Y}}\\
    =&\sum_{\substack{Y\in B_{\geq 1}^{\mathrm{safe}}\\\forall i\in[k], a\in [m]\setminus S, Y^i_a=X^i_a}}\abs{ \prod_{j=0}^{k-1}\frac{1}{2^{|S|}-j}- \frac1{2^{|S|k}}}\tag{$k\leq 2^{m/3}\leq 2^m-2$ and $X,Y\in B_{\geq 1}^{\mathrm{safe}}$}\\
    \leq &\sum_{\substack{Y\in B_{\geq 1}^{\mathrm{safe}}\\\forall i\in[k], a\in [m]\setminus S, Y^i_a=X^i_a}}\abs{ \frac1{2^{|S|k}}\pbra{\prod_{j=0}^{k-1}\frac{2^{|S|}}{2^{|S|}-j}- 1}}\\
    \leq &\sum_{\substack{Y\in B_{\geq 1}^{\mathrm{safe}}\\\forall i\in[k], a\in [m]\setminus S, Y^i_a=X^i_a}}\abs{ \frac1{2^{|S|k}}\pbra{1+\frac{k^2}{2^{|S|}}- 1}}\tag{$k^2\leq 2^{\ell}\leq 2^{|S|}$, \Cref{fact:k^2}}\\
    = &\sum_{\substack{Y\in B_{\geq 1}^{\mathrm{safe}}\\\forall i\in[k], a\in [m]\setminus S, Y^i_a=X^i_a}}\frac{k^2}{2^{|S|}2^{|S|k}}\\
    = &\cdot 2^{|S|k}\cdot \frac{k^2}{2^{|S|}2^{|S|k}}\\
    =& \frac{k^2}{2^{|S|}}\\
    \leq& \frac{k^2}{2^{\ell-1}}.\numberthis\label{eq:tv for safe}
\end{align*}
The last inequality follows because $|S|\geq\abs{[m-\ell-1,m-1]}=\ell-1$. Having established \Cref{eq:tv for safe}, we have completed the proof.
\end{proof}





At this point it may seem like we are essentially finished with the proof, since we should just replace the $R$ operators in the expression $\left\langle f,{R_{m,[m-\ell-1,m],k}\pbra{R_{m,[m-1],k}-R_{m,[m],k}}}R_{m,[m-\ell-1,m],k}f\right\rangle$ with the corresponding $Q$ operators and finish the proof. However, we don't quite show an upper bound on the \textit{operator norm} of $R-Q$. Rather, we simply show that they are close on the well-behaved region. A priori, this gives us no information about how products of these operators may behave, since the first term in the product may ``rotate" vectors into the badly-behaved region. So a straightforward application of the triangle inequality fails.

However, we observe that we have already shown in \Cref{sec:coll and cross terms} that random walks starting outside the badly-behaved region rarely transition into it, so that we almost can pretend as if all of these operators are operators on $\R^{B_{\geq1}^{\mathrm{safe}}}$. We formalize this by inserting projections to the space of functions supported on $B_{\geq1}^{\mathrm{safe}}$, and showing that this move does very little quantitatively.


\begin{lemma}\label{lem:product of operators quadratic form}
    We have for any $f,g:\{\pm1\}^{mk}\to\R$ supported on $B_{\geq1}^{\mathrm{safe}}$ that
    \begin{align*}
        \abs{\left\langle f,\pbra{R_{m,[m-\ell-1,m],k}-Q_{m,[m-\ell-1,m],k}}\pbra{R_{m,[m-1],k}-R_{m,[m],k}}g\right\rangle }\leq \frac{k^2}{2^{\ell/4-20}}\norm{f}_2\norm{g}_2.
    \end{align*}
    Moreover, we have for such $f$ and $g$ that
    \begin{align*}
        \abs{\left\langle f,Q_{m,[m-\ell-1,m],k}\pbra{R_{m,[m-1],k}-Q_{m,[m-1],k}-R_{m,[m],k}+Q_{m,[m],k}}g\right\rangle }\leq \frac{k^2}{2^{\ell/4-20}}\norm{f}_2\norm{g}_2.
    \end{align*}
\end{lemma}
\begin{proof}
    Let $\Pi_{\mathrm{safe}}$ be the projection to $\{h:\{\pm1\}^{mk}\to\R:h\text{ supported on }B_{\geq1}^{\mathrm{safe}}\}$. We directly compute
    \begin{align*}
        &\abs{\left\langle f,\pbra{R_{m,[m-\ell-1,m],k}-Q_{m,[m-\ell-1,m],k}}R_{m,[m-1],k}g\right\rangle }\\
        \leq&\abs{\left\langle f,\pbra{R_{m,[m-\ell-1,m],k}-Q_{m,[m-\ell-1,m],k}}\pbra{\Id-\Pi_{\mathrm{safe}}}R_{m,[m-1],k}g\right\rangle }\\
        &\;\;+\abs{\left\langle f,\pbra{R_{m,[m-\ell-1,m],k}-Q_{m,[m-\ell-1,m],k}}\Pi_{\mathrm{safe}}R_{m,[m-1],k}g\right\rangle}\\
        \leq&\abs{\left\langle f,\pbra{R_{m,[m-\ell-1,m],k}-Q_{m,[m-\ell-1,m],k}}\pbra{\Id-\Pi_{\mathrm{safe}}}R_{m,[m-1],k}g\right\rangle }+\frac{k^2}{2^{\ell-1}}\norm{f}_2\norm{\Pi_{\mathrm{safe}}R_{m,[m-1],k}g}_2\tag{\Cref{lem:R to Q hybrid local}}\\
        \leq&\abs{\left\langle f,R_{m,[m-\ell-1,m],k}\pbra{\Id-\Pi_{\mathrm{safe}}}R_{m,[m-1],k}g\right\rangle}+\abs{\left\langle f,Q_{m,[m-\ell-1,m],k}\pbra{\Id-\Pi_{\mathrm{safe}}}R_{m,[m-1],k}g\right\rangle}+\frac{k^2}{2^{\ell-1}}\norm{f}_2\norm{g}_2\\
        \leq&2\norm{f}_2\norm{\pbra{\Id-\Pi_{\mathrm{safe}}}R_{m,[m-1],k}g}_2+\frac{k^2}{2^{\ell-1}}\norm{f}_2\norm{g}_2\\
        \leq&\frac{k^2}{2^{\ell/4-10}}\norm{f}_2\norm{g}_2+\frac{k^2}{2^{\ell-1}}\norm{f}_2\norm{g}_2.\tag{\Cref{eq:safe to not safe R} below}
    \end{align*}
    Here our application of the \Cref{lem:R to Q hybrid local} depended on the fact that $\mathrm{Supp}(\Pi_{\mathrm{safe}}R_{m,[m-1],k}f)\subseteq B_{\geq1}^{\mathrm{safe}}$ and $\mathrm{Supp}((\Id-\Pi_{\mathrm{safe}})R_{m,[m-1],k}f)\subseteq \{\pm1\}^{mk}\setminus B_{\geq1}^{\mathrm{safe}}$.
    
    To establish \Cref{eq:safe to not safe R} we compute 
    \begin{align*}
        &\norm{\pbra{\Id-\Pi_{\mathrm{safe}}}R_{m,[m-1],k}g}_2^2\\
        =&\left\langle \pbra{\Id-\Pi_{\mathrm{safe}}}R_{m,[m-1],k}g,\pbra{\Id-\Pi_{\mathrm{safe}}}R_{m,[m-1],k}g\right\rangle\\
        =&\left\langle R_{m,[m-1],k}g,\pbra{\Id-\Pi_{\mathrm{safe}}}^2R_{m,[m-1],k}g\right\rangle\\
        =&\left\langle f,R_{m,[m-1],k}\pbra{\Id-\Pi_{\mathrm{safe}}}R_{m,[m-1],k}g\right\rangle\tag{self-adjointness (\Cref{fact:self-adjoint})}\\
        \leq &\frac{k}{2^{\ell/2-1}}\norm{g}_2\norm{\pbra{\Id-\Pi_{\mathrm{safe}}}R_{m,[m-1],k}g}_2\\
        \leq&\frac{k}{2^{\ell/2-1}}\norm{g}_2^2\numberthis\label{eq:safe to not safe R}.
    \end{align*}
    The first inequality follows from \Cref{lem:escape probs} the fact that $\mathrm{Supp}(f)\subseteq B_{\geq1}^{\mathrm{safe}}$ and for any $X\in B_{\geq1}^{\mathrm{safe}}$, we have $\Pr\sbra{X\to_{R_{m,[m-1],k}}B_{\geq1}^{\mathrm{coll}}}\leq \frac{k^2}{2^{\ell-1}}$ (using the same proof as \Cref{lem:anything transition into coll}), and that $\mathrm{Supp}\pbra{\pbra{\Id-\Pi_{\mathrm{safe}}}R_{m,[m-1],k}f}\subseteq \{\pm1\}^{mk}\setminus B_{\geq1}^{\mathrm{safe}}$. Bounding the similar quantity but with $R_{m,[m],k}$ instead of $R_{m,[m-1],k}$ is the same and the first part of the lemma statement follows from the triangle inequality. 
    
    The second part of the lemma statement follows from the same argument and an application of the triangle inequality:
    \begin{align*}
        &\abs{\left\langle f,Q_{m,[m-\ell-1,m],k}\pbra{R_{m,[m-1],k}-Q_{m,[m-1],k}-R_{m,[m],k}+Q_{m,[m],k}}g\right\rangle}\\
        =&\abs{\left\langle f,\pbra{R_{m,[m-1],k}-Q_{m,[m-1],k}-R_{m,[m],k}+Q_{m,[m],k}}Q_{m,[m-\ell-1,m],k}g\right\rangle}\tag{\Cref{fact:self-adjoint}, \Cref{fact:Q self-adjoint}}\\
        \leq&\abs{\left\langle f,\pbra{R_{m,[m-1],k}-Q_{m,[m-1],k}}Q_{m,[m-\ell-1,m],k}g\right\rangle}+\abs{\left\langle f,\pbra{R_{m,[m],k}-Q_{m,[m],k}}Q_{m,[m-\ell-1,m],k}g\right\rangle}.\numberthis\label{eq:hybrid the difference}
    \end{align*}
    We show how to bound the first term in this sum:
    \begin{align*}
        &\abs{\left\langle f,\pbra{R_{m,[m-1],k}-Q_{m,[m-1],k}}Q_{m,[m-\ell-1,m],k}g\right\rangle}\\
        \leq&\abs{\left\langle f,\pbra{R_{m,[m-1],k}-Q_{m,[m-1],k}}\pbra{\Id-\Pi_{\mathrm{safe}}}Q_{m,[m-\ell-1,m],k}g\right\rangle }\\
        &\;\;+\abs{\left\langle f,\pbra{R_{m,[m-1],k}-Q_{m,[m-1],k}}\Pi_{\mathrm{safe}}Q_{m,[m-\ell-1,m],k}g\right\rangle}\\
        \leq&\abs{\left\langle f,\pbra{R_{m,[m-1],k}-Q_{m,[m-1],k}}\pbra{\Id-\Pi_{\mathrm{safe}}}Q_{m,[m-\ell-1,m],k}g\right\rangle }+\frac{k^2}{2^{\ell-1}}\norm{f}_2\norm{\Pi_{\mathrm{safe}}Q_{m,[m-\ell-1,m],k}g}_2\tag{\Cref{lem:R to Q hybrid local}}\\
        \leq&\abs{\left\langle f,R_{m,[m-1],k}\pbra{\Id-\Pi_{\mathrm{safe}}}Q_{m,[m-\ell-1,m],k}g\right\rangle }\\
        &+\abs{\left\langle f,Q_{m,[m-1],k}\pbra{\Id-\Pi_{\mathrm{safe}}}Q_{m,[m-\ell-1,m],k}g\right\rangle }+\frac{k^2}{2^{\ell-1}}\norm{f}_2\norm{g}_2\\
        \leq&2\norm{f}_2\norm{\pbra{\Id-\Pi_{\mathrm{safe}}}Q_{m,[m-\ell-1,m],k}g}_2+\frac{k^2}{2^{\ell-1}}\norm{f}_2\norm{g}_2\\
        \leq&\frac{k}{2^{\ell/4-10}}\norm{f}_2\norm{g}_2+\frac{k^2}{2^{\ell-1}}\norm{f}_2\norm{g}_2.\tag{\Cref{eq:safe to not safe Q} below}
    \end{align*}
    To establish \Cref{eq:safe to not safe Q} we compute 
    \begin{align*}
        &\norm{\pbra{\Id-\Pi_{\mathrm{safe}}}Q_{m,[m-\ell-1,m],k}g}_2^2\\
        =&\left\langle g,Q_{m,[m-\ell-1,m],k}\pbra{\Id-\Pi_{\mathrm{safe}}}Q_{m,[m-\ell-1,m],k}g\right\rangle\tag{self-adjointness (\Cref{fact:Q self-adjoint}}\\
        \leq &\frac{k}{2^{\ell/2-1}}\norm{g}_2\norm{\pbra{\Id-\Pi_{\mathrm{safe}}}Q_{m,[m-\ell-1,m],k}g}_2\\
        \leq&\frac{k}{2^{\ell/2-1}}\norm{g}_2^2\numberthis\label{eq:safe to not safe Q}.
    \end{align*}
    The first inequality follows from \Cref{lem:escape probs} the fact that $\mathrm{Supp}(f)\subseteq B_{\geq1}^{\mathrm{safe}}$ and for any $X\in B_{\geq1}^{\mathrm{safe}}$, we have $\Pr\sbra{X\to_{Q_{m,[m-1],k}}B_{\geq1}^{\mathrm{coll}}}\leq \frac{k^2}{2^{\ell-1}}$ (using the same proof as \Cref{lem:anything transition into coll}). The bound on the second term in the sum \Cref{eq:hybrid the difference} follows from the same argument.
\end{proof}




\begin{corollary}\label{cor:square terms safe to safe}
    Let $k\geq 2$ and $f:\{\pm1\}^{mk}\to\R$ be supported on $B_{\geq 1}^{\mathrm{safe}}$. Then 
    \begin{align*}
        \abs{\left\langle f,R_{m,[m-\ell-1,m],k}\pbra{R_{m,[m-1],k}-R_{m,[m],k}}R_{m,[m-\ell-1,m],k}f\right\rangle }\leq \frac{k^2}{2^{\ell/4-50}}\langle f,f\rangle.
    \end{align*}
\end{corollary}
\begin{proof}
    Let $g=R_{m,[m-\ell-1,m],k}f$ and we can write $g=g_1+g_2$ where $g_1=\Pi_{\mathrm{safe}}g$ and $g_2=\pbra{\mathrm{Id}-\Pi_{\mathrm{safe}}}g$. Then
    \begin{align*}
        &\abs{\left\langle f,R_{m,[m-\ell-1,m],k}\pbra{R_{m,[m-1],k}-R_{m,[m],k}}R_{m,[m-\ell-1,m],k}f\right\rangle }\\
        \leq&\abs{\left\langle f,R_{m,[m-\ell-1,m],k}\pbra{R_{m,[m-1],k}-R_{m,[m],k}}g_1\right\rangle }+\abs{\left\langle f,R_{m,[m-\ell-1,m],k}\pbra{R_{m,[m-1],k}-R_{m,[m],k}}g_2\right\rangle }\\
        \leq&\abs{\left\langle f,R_{m,[m-\ell-1,m],k}\pbra{R_{m,[m-1],k}-R_{m,[m],k}}g_1\right\rangle }+\frac{k^2}{2^{\ell/2-3}}\norm{f}_2\norm{g_2}_2.\tag{\Cref{cor:cross terms safe to coll}}
    \end{align*}
    To bound the first term, we note that $g_1$ is supported on $B_{\geq1}^{\mathrm{safe}}$ and directly compute
    \begin{align*}
        &\abs{\left\langle f,R_{m,[m-\ell-1,m],k}\pbra{R_{m,[m-1],k}-R_{m,[m],k}}g_1\right\rangle }\\
        \leq&\abs{\left\langle f,Q_{m,[m-\ell-1,m],k}\pbra{R_{m,[m-1],k}-R_{m,[m],k}}g_1\right\rangle }+\\
        &\;\;\abs{\left\langle f,\pbra{R_{m,[m-\ell-1,m],k}-Q_{m,[m-\ell-1,m],k}}\pbra{R_{m,[m-1],k}-R_{m,[m],k}}g_1\right\rangle }\\
        \leq &\abs{\left\langle f,Q_{m,[m-\ell-1,m],k}\pbra{R_{m,[m-1],k}-R_{m,[m],k}}g_1\right\rangle }+\frac{k^2}{2^{\ell/4-20}}\norm{f}_2\norm{g_1}_2\tag{\Cref{lem:product of operators quadratic form}, first part}\\
        \leq &\frac{k^2}{2^{\ell/2-20}}\norm{f}_2\norm{g_1}_2+\abs{\left\langle f,Q_{m,[m-\ell-1,m],k}\pbra{Q_{m,[m-1],k}-Q_{m,[m],k}}f\right\rangle }\\
        &\;\;\;+\abs{\left\langle f,Q_{m,[m-\ell-1,m],k}\pbra{R_{m,[m-1],k}-Q_{m,[m-1],k}-R_{m,[m],k}+Q_{m,[m],k}}g_1\right\rangle }\\
        = &\frac{k^2}{2^{\ell/4-20}}\norm{f}_2\norm{g_1}_2+\abs{\left\langle f,Q_{m,[m-\ell-1,m],k}\pbra{R_{m,[m-1],k}-Q_{m,[m-1],k}-R_{m,[m],k}+Q_{m,[m],k}}g_1\right\rangle}\\
        =&\frac{k^2}{2^{\ell/4-20}}\norm{f}_2\norm{g_1}_2+\frac{k^2}{2^{\ell/4-20}}\norm{f}_2\norm{g_1}_2\tag{\Cref{lem:product of operators quadratic form}, second part}.
    \end{align*}
    For the second-to-last equality we used that $Q_{m,[m-\ell-1,m],k}\pbra{Q_{m,[m-1],k}-Q_{m,[m],k}}=0$ because $Q_{m,S,k}Q_{m,T,k}=Q_{m,S\cup T,k}$ for any $S,T\subseteq[m]$.

    To complete the proof we use that $\norm{g_1}_2,\norm{g_2}_2\leq \norm{f}_2$.
\end{proof}



\section{Reduction from Two-Dimensional to One-Dimensional Constructions: Proof of \Cref{thm:2D to 1D reduction}}\label{sec:proof of 2D to 1D reduction}

\subsection{Definitions}\label{sec:definitions}
To prove \Cref{thm:2D to 1D reduction} we introduce some new notation required for dealing with bit arrays on a higher-dimensional lattice architecture. 
\subsubsection{Bit Arrays}
We regard an element $x\in \{\pm1\}^{n}$ as a function $x : \sbra{\sqrt{n}\,} \times \sbra{\sqrt{n}\,} \to \{\pm1\}$. 
Similarly, we regard an element $X \in \{\pm1\}^{nk}$ as a function $X : \sbra{\sqrt{n}\,} \times \sbra{\sqrt{n}\,} \times \sbra{k} \to \{\pm1\}$. 
For $X \in \{\pm1\}^{nk}$, and $i,j \in \sbra{\sqrt{n}\,}$, and $\ell \in [k]$, we use the notation:
\begin{itemize}
    \item $X^\ell_{i, j} = X(i, j, \ell) \in \{\pm1\}$
    \item $X^\ell = X \mid_{\sbra{\sqrt{n}\,} \times \sbra{\sqrt{n}\,} \times \{\ell\}} \in \{\pm1\}^{n}$
    \item $X^\ell_{i, \cdot} = X \mid_{\{i\} \times \sbra{\sqrt{n}\,} \times \{\ell\}} \in \{\pm1\}^{\sqrt{n}}$
    \item $X^\ell_{\cdot, j} = X \mid_{\sbra{\sqrt{n}\,} \times \{j\} \times \{\ell\}} \in \{\pm1\}^{\sqrt{n}}$
    \item $X_{i, \cdot} = X \mid_{\{i\} \times \sbra{\sqrt{n}\,} \times [k]} \in \{\pm1\}^{\sqrt{n}k}$
    \item $X_{\cdot, j} = X \mid_{\sbra{\sqrt{n}\,} \times \{j\} \times [k]} \in \{\pm1\}^{\sqrt{n}k}$
\end{itemize}
We will use $\sD_n^{(k)}$ to denote the set of all $X \in \{\pm1\}^{nk}$ such that $X^i \ne X^j$ when $i \ne j$. Unless otherwise specified, $\sD$ refers to $\sD_{n}^{(k)}$.

\subsubsection{Color Classes}
\label{sec:colorclasses}
We partition $\{\pm1\}^{nk}$ into ``color classes'' via the following relation. Let $R^{(\sqrt{n})}$ be a tuple of $\sqrt{n}$ equivalence relations on $[k]$. That is, there is one equivalence relation for each row in $[\sqrt{n}]$. Then we define:
\begin{equation*}
        B_{R^{(\sqrt{n})}} = \left\{ X \in \{\pm1\}^{nk} : \forall i \in \sqrt{n}, X_{i, \cdot}^\ell = X_{i, \cdot}^m \text{ if and only if } \ell \,R_i\, m \right\}.
\end{equation*}

Informally, $X^{\ell}_{i, \cdot}$ and $X^m_{i, \cdot}$ share a color if $X^{\ell}_{i, \cdot} = X^m_{i, \cdot}$. This relation induces a coloring on the rows of $X$. We then say that $X$ and $Y$ are colored the same if all of their rows are colored the same. Since $R^{(\sqrt{n})}$ is an equivalence relation itself, the sets $\{B_{R^{(\sqrt{n})}}\}_{R^{(\sqrt{n})} \in \sR^{\otimes \sqrt{n}}}$ partition $\{\pm1\}^{nk}$ for $\sR$ the set of equivalence relations on $[k]$. Additionally, each $X$ has a unique color class we will denote as $B(X)$. We will also use $\sB$ to denote the set of color classes. This partition is useful in part due to its size.
\begin{fact}
    \label{fact:numofcolorclasses}
    There are $\leq k^{k\sqrt{n}}$ color classes, that is, $\abs{\sB} \leq k^{k\sqrt{n}}$.
\end{fact}

\begin{proof}
    We can count each color class by identifying the partition of each row, of which there are $\sqrt{n}$. Each row consists of $k$ elements, so we can overcount the number as putting the $k$ elements into $k$ partitions, $k^k$.
\end{proof}

We will define a simpler partition that will facilitate much of our analysis:
\begin{align*}
    &B_\text{safe} := \cbra{X \in \sD : \forall \ell \neq  m \in [k], i \in [\sqrt{n}], X^\ell_{i, \cdot} \neq X^m_{i, \cdot}},\\
    &B_\text{coll} := \sD \setminus B_\text{safe},\\
    &B_{=0} := \{\pm1\}^{nk} \setminus \sD.
\end{align*}
That is, $B_\text{safe}$ is the color class determined by the $\sqrt{n}$-wise product of the identity relation. $B_\text{coll}$ then consists of all other color classes within $\sD$, whereas $B_{=0}$ consists of all elements outside of $\sD$. We will often use the following result on the size of $B_\text{coll}$:
\begin{fact}
\label[fact]{fact:colorclasssizes}
    $\frac{\abs{B_{\mathrm{coll}}}}{\abs{\sD}} \leq \frac{2\sqrt{n}k^2}{2^{\sqrt{n}}}$.
\end{fact}

\begin{proof}
We may write:
\begin{equation*}
    \frac{\abs{B_{\mathrm{coll}}}}{\abs{\sD}} = \frac{\abs{B_{\mathrm{coll}}}}{\abs{\{\pm1\}^{nk}}} \cdot \frac{\abs{\{\pm1\}^{nk}}}{\abs{\sD}}.
\end{equation*}
The first can be viewed as the probability of sampling an element of $B_{\mathrm{coll}}$ when sampling from $\{\pm1\}^{nk}$. The process of sampling from $\{\pm1\}^{nk}$ can be seen as sampling $\sqrt{n}k$ rows from $\{\pm1\}^{\sqrt{n}}$. Under this view, a simple union bound tells us that there are at most $\sqrt{n}k^2$ possible ``collisions'' that would induce a non-distinct color class, allowing us to bound the probability by $\frac{\sqrt{n}k^2}{2^{\sqrt{n}}}$.

For the other term, we will prove simply that $\frac{\abs{B_{=0}}}{\abs{\{\pm1\}^{nk}}} \leq \frac{1}{2}$. Note that our analysis above actually bounds the probability $X \sim \{\pm1\}^{nk}$ is not in $B_{\mathrm{safe}}$, which is more than sufficient for this. \end{proof}


\subsubsection{Distributions}
\label{subsec:distributions}
Because in this section we are dealing with permutations of $n$-bit strings where the bits lie on a higher-dimensional lattice, it will be convenient to introduce completely new notation that will supplant the random walk operators $R_{n,S,k}$ from before.

If $\pi \in \mfS_{\{\pm1\}^{n}}$, then let $\pi^{\otimes k} \in \mfS_{\{\pm1\}^{nk}}$ be such that $\pi^{\otimes k}(X)^\ell = \pi(X^\ell)$ for all $X \in \{\pm1\}^{nk}$ and $\ell \in [k]$. 

\begin{itemize}
    \item Let $\mcB$ be a distribution on $\mfS_{\{\pm1\}^{\sqrt{n}}}$.
    
    \item Let $\mcP_R$ be a distribution on $\mfS_{\{\pm1\}^{n}}$ such that $\pi \sim \mcP_R$ is sampled as follows: 
          Sample $\sigma_i \sim \mcB$ independently for each $i \in \sbra{\sqrt{n}\,}$ and define $\pi$ such that $\pi(x)_{i, \cdot} = \sigma_i(x_{i, \cdot})$ for all $x \in \{\pm1\}^{n}$ and all $i \in \sbra{\sqrt{n}\,}$. 
    \item Let $\mcP_C$ be a distribution on $\mfS_{\{\pm1\}^{n}}$ such that $\pi \sim \mcP_C$ is sampled as follows: 
          Sample $\sigma_i \sim \mcB$ independently for each $i \in \sbra{\sqrt{n}\,}$ and define $\pi$ such that $\pi(x)_{\cdot, i} = \sigma_i(x_{\cdot, i})$ for all $x \in \{\pm1\}^{n}$ and all $i \in \sbra{\sqrt{n}\,}$. 
    \item Let $\mcP^0 = \mcP_R$. For all $t \ge 1$, let $\mcP^t$ be a distribution on $\mfS_{\{\pm1\}^{n}}$ such that $\pi \sim \mcP^t$ is sampled as follows:
          Sample $\sigma_1 \sim \mcP^{t-1}$, $\sigma_2 \sim \mcP_C$, and $\sigma_3 \sim \mcP_R$ and define $\pi$ such that $\pi(x) = (\sigma_3 \circ \sigma_2 \circ \sigma_1)(x)$ for all $x \in \{\pm1\}^{n}$. It is worth noting that this construction is exactly that of our circuit model above.
    \item Let $\mcG_R$ be a distribution on $\mfS_{\{\pm1\}^{n}}$ such that $\pi \sim \mcG_R$ is sampled as follows:
          Sample $\sigma_i \sim \mcU(\mfS_{\{\pm1\}^{\sqrt{n}}})$ independently for each $i \in \sbra{\sqrt{n}\,}$ and define $\pi$ such that $\pi(x)_{i, \cdot} = \sigma_i(x_{i,\cdot})$ for all $x \in \{\pm1\}^{n}$. 
    \item Let $\mcG_C$ be a distribution on $\mfS_{\{\pm1\}^{n}}$ such that $\pi \sim \mcG_C$ is sampled as follows:
          Sample $\sigma_i \sim \mcU(\mfS_{\{\pm1\}^{\sqrt{n}}})$ independently for each $i \in \sbra{\sqrt{n}\,}$ and define $\pi$ such that $\pi(x)_{\cdot, i} = \sigma_i(x_{\cdot, i})$ for all $x \in \{\pm1\}^{n}$. 
    \item Let $\mcG^0 = \mcG_R$. For all $t \ge 1$, let $\mcG^t$ be a distribution on $\mfS_{\{\pm1\}^{n}}$ such that $\pi \sim \mcG^t$ is sampled as follows:
          Sample $\sigma_1 \sim \mcG^{t-1}$, $\sigma_2 \sim \mcG_C$, and $\sigma_3 \sim \mcG_R$ and define $\pi$ such that $\pi(x) = (\sigma_3 \circ \sigma_2 \circ \sigma_1)(x)$ for all $x \in \{\pm1\}^{n}$.
    \item Let $\mcG$ be $\mcU(\mfS_{\{\pm1\}^{n}})$, i.e. the uniform distribution on $\mfS_{\{\pm1\}^{n}}$.
    \item If $\mcD$ is a distribution on $\mfS_{\{\pm1\}^{n}}$ and $X \in \{\pm1\}^{nk}$, let $\mcD^{(k)}_X$ be a distribution on $\{\pm1\}^{nk}$ such that $Y \sim \mcD^{(k)}_X$ is sampled as follows:
          Sample $\pi \sim \mcD$ and define $Y$ such that $Y = \pi^{\otimes k}(X)$. 
          Note that if $X \in \sD$, then $\mcG^{(k)}_X$ is $\mcU\pbra{\sD}$. If the superscript is understood from the context of $X$ to be $k$, we will often drop it and just write $\mcD_X$.
\end{itemize}


It will be helpful to think of the distribution $\mcD$ as defining a Markov chain on $\{\pm1\}^{nk}$ for any choice $k \geq 1$. More specifically, $\mcD^{(k)}_X$ can be thought of as specifying the transition probabilities out of $X$ in the corresponding Markov chain on $\{\pm1\}^{nk}$. The transition operators corresponding to these Markov chains are given by:
\begin{align*}
    (T_{\mcD}f)(X)=\Ex_{\mathbf{Y} \sim \mathcal{D}_X^{(k)}}\sbra{f(\mathbf{Y})} = \sum_{Y \in \{\pm1\}^{nk}} \Pr_{\bm{Y}\sim\mcD_{X}^{(k)} }[\bm{Y}=Y] \cdot f(Y).
\end{align*}
When the value of $k$ is not clear from context, we will write $T_{\mcD^{(k)}}$ for this operator.



\subsection{Linear Algebra}
Recall that $\mcD_X$ can be thought of as the distribution after a 1-step random walk from $X$ according to a permutation drawn from $\mcD$.
\begin{fact}
    \label[fact]{fact:selfadjoint}
    $T_{\mcG_R}$, $T_{\mcG_C}$, and $T_{\mcG}$ are self-adjoint w.r.t our inner product.
\end{fact}
\begin{proof}
We will state the proof for $T_{\mcG_R}$, the reasoning for the others being symmetric. Let $f, g : \{\pm1\}^{nk} \to \R$.
\begin{align*}
    \langle f, T_{\mcG_R} g \rangle &= \sum_{X \in \{\pm1\}^{nk}} f(X)(T_{\mcG_R}g)(X)\\
    &= \sum_{X \in \{\pm1\}^{nk}} f(X) \sum_{Y \in \{\pm1\}^{nk}} \Pr[X \to_{T_{\mcG_R}} Y] \cdot g(Y)\\
    &= \sum_{X \in \{\pm1\}^{nk}}\sum_{Y \in \{\pm1\}^{nk}} f(X) \cdot g(Y) \cdot \Pr[X \to_{T_{\mcG_R}} Y].
\end{align*}
Note the symmetry in above: we are done if we prove that $\Pr[X \to_{T_{\mcG_R}} Y] = \Pr[Y \to_{T_{\mcG_R}} X]$. This fact is quite observable from the definition of $\mcG_R$, since for any $\sigma \in \mfS_{\{\pm1\}^{n}}$ the probability it is drawn from $\mcG_R$ is the same as the probability of drawing $\sigma^{-1}$. 
\end{proof}

\begin{fact}
    \label[fact]{fact:TGabsorbs}
    Let $\mcD$ be a distribution on $\mfS_{\{\pm1\}^{n}}$. Then:
    \begin{equation*}
        T_\mcD T_\mcG = T_\mcG = T_\mcG T_\mcD.
    \end{equation*}
\end{fact}

\begin{fact}
    \label{fact:differenceofproducts}
    Let $U_1, ..., U_s$ and $W_1,..., W_s$ be operators. Then we have:
    \begin{equation*}
        \prod_{i= 1}^s U_i - \prod_{i=1}^s W_i = \sum_{i=1}^s \prod_{j=1}^{i-1} U_j \cdot \pbra{U_i - W_i} \cdot \prod_{j = i+1}^s W_j.
    \end{equation*}
\end{fact}
\begin{proof}
We prove this by induction on $s$. The base case $s=1$ is immediate. For the inductive hypothesis note that:
\begin{align*}
    \prod_{i= 1}^{s+1} U_i - \prod_{i=1}^{s+1} W_i &= \pbra{\prod_{i= 1}^s U_i - \prod_{i=1}^s W_i}W_{s+1} + \prod_{i=1}^s U_i \pbra{U_{s+1} - W_{s+1}}\\
    &= \sum_{i=1}^s \prod_{j=1}^{i-1} U_j \cdot \pbra{U_i - W_i} \cdot \prod_{j = i+1}^{s+1} W_j + \prod_{i=1}^s U_i \pbra{U_{s+1} - W_{s+1}} \tag{Induction}\\
    &= \sum_{i=1}^{s+1} \prod_{j=1}^{i-1} U_j \cdot \pbra{U_i - W_i} \cdot \prod_{j = i+1}^{s+1} W_j.
\end{align*}
This completes the induction and the proof.
\end{proof}



\subsection{Proof of \Cref{thm:2D to 1D reduction}}
\label{sec:2D to 1D proof}

In this section we will prove \Cref{thm:2D to 1D reduction}, reducing the result to a spectral norm bound to be proved in \Cref{sec:spectralproof}. Our key insight will be that given the distribution and operator framework outlined in the previous section, we can now restate \Cref{thm:2D to 1D reduction} in a more ``palatable'' way by translating statements about the TV distance between distributions as quantities of their corresponding operators. More concretely we have the following:
\begin{claim}
    \label{tvdistancetolinearform}
    For $X \in \sD$, $d_{\mathrm{TV}}((\mcD_1)_X, (\mcD_2)_X) = \frac{1}{2} \sum_{Y \in \sD} \abs{\ip{e_X}{(T_{\mcD_1}-T_{\mcD_2}) e_Y}}$
\end{claim}

\begin{proof}We directly compute:
    \begin{align*}
        d_{\textrm{TV}}((\mcD_1)_X, (\mcD_2)_X) 
        &= \frac{1}{2} \sum_{Y \in \{\pm1\}^{nk}} \abs{\Pr[X \to_{T_{\mcD_1}} Y] - \Pr[X \to_{T_{\mcD_2}} Y]}\\
        &= \frac{1}{2} \sum_{Y \in \{\pm1\}^{nk}} \abs{\ip{e_X}{T_{\mcD_1} e_Y} - \ip{e_X}{T_{\mcD_2} e_Y}}\\
        &= \frac{1}{2} \sum_{Y \in \sD} \abs{\ip{e_X}{(T_{\mcD_1}-T_{\mcD_2}) e_Y}}.
    \end{align*}
    Here we used \Cref{fact:colorclasssizes} and that under a permutation elements in $\sD$ only map to $\sD$. 
\end{proof}

We apply this to prove \Cref{thm:2D to 1D reduction}:
\begin{theorem}[\Cref{thm:2D to 1D reduction} restated]
\label{thm:2D to 1D reduction technical}
    Given $k \leq 2^{\sqrt{n}/500}$, then for any $t \geq 500\pbra{k \log k + \frac{\log(1/\varepsilon)}{\sqrt{n}}}$, the following holds. Let $\mcP^t$ be the distribution on $\mathcal{S}_{\{\pm1\}^{n}}$ defined from our circuit model with a circuit family computing an $\frac{\varepsilon}{2\sqrt{n} \cdot (2t+1)}$-approximate $k$-wise independent distribution on $\mfS_{\{\pm1\}^{\sqrt{n}}}$ as the base. Then $\mcP^t$ is $\varepsilon$-approximate $k$-wise independent. That is, for all $X \in \sD$ we have $d_{\textrm{TV}}\pbra{\mcP^t_X, \mcG_X} \leq \varepsilon$ when $n$ is large enough.
\end{theorem}


\begin{proof}
We will bound the two quantities arising from the following application of the triangle inequality:
\begin{equation*}
    d_{\textrm{TV}}\pbra{\mcP^t_X, \mcG_X} \leq d_{\textrm{TV}}\pbra{\mcP^t_X, \mcG^t_X} + d_{\textrm{TV}}\pbra{\mcG^t_X, \mcG_X}.
\end{equation*}
Here we introduce the intermediate distribution $\mcG^t$ (defined in \Cref{subsec:distributions}) which will facilitate our analysis. The two steps are then to bound each of the latter terms separately by $\frac{\varepsilon}{2}$.

\begin{lemma}
    \label{lem:reduction}
    Assume the hypotheses of \Cref{thm:2D to 1D reduction}. Then for any $X \in \sD$:
\begin{equation*}
    \sum_{Y \in \sD} \abs{\ip{e_X}{(T_{\mcP^t}-T_{\mcG^t}) e_Y}} \leq  \max_{f \in \mcF} \abs{\ip{e_X}{(T_{\mcP^t}-T_{\mcG^t}) f}} \leq \frac\varepsilon2
\end{equation*}
where $\mcF$ is the set of functions $f : \{\pm1\}^{nk} \to [-1, 1]$ with $\mathrm{Supp}(f) \subseteq \sD$.
\end{lemma}

This lemma is a just a quantitative result capturing the intuition that if we replace all of the black-box $1$D pseudorandom permutation circuits in our design with truly random permutations, the TV distance does not change much.

\begin{lemma}
    \label{lem:maintrick}
    Assume the hypotheses of \Cref{thm:2D to 1D reduction}. Then for any $X \in \sD$:
\begin{equation*}
    \sum_{Y \in \sD} \abs{\ip{e_X}{(T_{\mcG^t}-T_{\mcG}) e_Y}} \leq \frac\varepsilon2.
\end{equation*}
\end{lemma}

This lemma claims that random columns and random row permutations approximate truly random permutations in low depth $t = \poly(k)$. In light of \Cref{tvdistancetolinearform}, we have shown that
\begin{equation*}
    d_{\textrm{TV}}\pbra{\mcP^t_X, \mcG^t_X} + d_{\textrm{TV}}\pbra{\mcG^t_X, \mcG_X} \leq \varepsilon,
\end{equation*}
which finishes the proof of \Cref{thm:2D to 1D reduction technical}, given the two lemmas.
\end{proof}


\subsection{Proof of \Cref{lem:reduction}}
\label{subsec:reduction}

In this proof, we roughly want to decompose the operators of our circuit as 1) sequential pieces corresponding to each layer of the circuit and 2) for each layer, since the pieces act independently on either the columns or rows, we can represent them as tensor products of simpler operators acting only on one row or column. Since these individual pieces are assumed to be approximately $k$-wise independent, we just show that this is preserved through tensorization and sequential application. 

Recall that $T_{\mcP^t} = T_{\mcP_R}\pbra{T_{\mcP_C}T_{\mcP_R}}^t$ and $T_{\mcG^t} = T_{\mcG_R}\pbra{T_{\mcG_C}T_{\mcG_R}}^t$, that is, they are products of $2t-1$ operators corresponding to the sequential pieces in the circuit. We can then rewrite the difference of the two operators as a telescoping sum following \Cref{fact:differenceofproducts}. For clarity let $T_{\mcP^{(i)}}$ denote the $i$th operator in the product $T_{\mcP^t}$ and likewise for $T_{\mcG^t}$.
\begin{equation*}
    T_{\mcP_R}\pbra{T_{\mcP_C}T_{\mcP_R}}^t - T_{\mcG_R}\pbra{T_{\mcG_C}T_{\mcG_R}}^t = \sum_{i=1}^{2t+1} \prod_{j=1}^{i-1} T_{\mcP^{(j)}} \cdot \pbra{T_{\mcP^{(i)}} - T_{\mcG^{(i)}}} \cdot \prod_{j = i+1}^s T_{\mcG^{(j)}}
\end{equation*}
To simplify this sum, we will make use of the following claim:

\begin{claim}\label{claim:sandwiching}
    Let $\mcD_1, \mcD_2$ be distributions on $\mfS_{\{\pm1\}^{n}}$ and $A$ an operator on the space $\R^{\{\pm1\}^{nk}}$. Let $X \in \sD$ and $f \in \mcF$. Then there exists some $X^* \in \sD$ and $f^* \in \mcF$ s.t.:
    \begin{equation*}
        \abs{\ip{e_X}{T_{\mcD_1}AT_{\mcD_2} f}} \leq \abs{\ip{e_{X^*}}{Af^*}}.
    \end{equation*}
\end{claim}
\begin{proof} Observe:
\begin{align*}
    \abs{\ip{e_X}{T_{\mcD_1}AT_{\mcD_2} f}} &= \abs{T_{\mcD_1}AT_{\mcD_2} f(X)}\\
    &= \abs{\sum_{Y \in \{\pm1\}^{nk}} \Pr[X \to_{T_{\mcD_1}} Y] \cdot AT_{\mcD_2} f(Y)}\\
    &= \abs{\sum_{Y \in \sD} \Pr[X \to_{T_{\mcD_1}} Y] \cdot \ip{e_Y}{AT_{\mcD_2} f}}\\
    &\leq \max_{X^* \in \sD} \abs{\ip{e_{X^*}}{AT_{\mcD_2} f}}.
\end{align*}
In the last line we are using triangle inequality. For the second part of the proof, it suffices to claim $T_{\mcD_2}f \in \mcF$. To see this just note:
\begin{equation*}
    T_{\mcD_2}f(Z) = \sum_{Y \in \mathrm{Supp}(\mcD_2(Z)))} \Pr[Z \to_{T_{\mcD_2}} Y] \cdot f(Y)
\end{equation*}
Since $f$ maps to $[-1, 1]$ which is a convex set, $T_{\mcD_2}f$ maps to $[-1, 1]$ as well. Additionally, for any $Z \notin \sD$, the support of $(\mcD_2)_Z$ cannot intersect $\sD$ so the term is 0, thus $T_{\mcD_2}f$ is supported on a subset of $\sD$ so is in $\mcF$. \end{proof}


\begin{claim}
    \begin{equation*}
        \max_{X \in \sD, f \in \mcF} \abs{\ip{e_X}{\pbra{T_{\mcP^t} - T_{\mcG^t}} f}} \leq \pbra{2t+1} \cdot \max_{X \in \sD, f \in \mcF} \abs{\ip{e_X}{\pbra{T_{\mcP_R} - T_{\mcG_R}} f}}.
    \end{equation*}
    Noting importantly that the latter term is interchangeable for $T_{\mcP_R} - T_{\mcG_R}$ and $T_{\mcP_C} - T_{\mcG_C}$. 
\end{claim}

\begin{proof}We directly compute:
    \begin{align*}
        \max_{X \in \sD, f \in \mcF} \abs{\ip{e_X}{\pbra{T_{\mcP^t} - T_{\mcG^t}} f}}
        &= \max_{X \in \sD, f \in \mcF} \abs{\ip{e_X}{\pbra{\sum_{i=1}^{2t+1} \prod_{j=1}^{i-1} T_{\mcP^{(j)}} \cdot \pbra{T_{\mcP^{(i)}} - T_{\mcG^{(i)}}} \cdot \prod_{j = i+1}^s T_{\mcG^{(j)}}} f}} \tag{\Cref{fact:differenceofproducts}}\\
        &\leq \max_{X \in \sD, f \in \mcF} \sum_{i=1}^{2t+1} \abs{\ip{e_X}{\pbra{ \prod_{j=1}^{i-1} T_{\mcP^{(j)}} \cdot \pbra{T_{\mcP^{(i)}} - T_{\mcG^{(i)}}} \cdot \prod_{j = i+1}^s T_{\mcG^{(j)}}} f}}\\
        &\leq \max_{X \in \sD, f \in \mcF} \sum_{i=1}^{2t+1} \abs{\ip{e_X}{\pbra{T_{\mcP^{(i)}} - T_{\mcG^{(i)}}} f}} \tag{\Cref{claim:sandwiching}}\\
        &\leq (2t+1) \cdot \max_{X \in \sD, f \in \mcF} \abs{\ip{e_X}{\pbra{T_{\mcP_R} - T_{\mcG_R}} f}}.
    \end{align*}
    In the last line we are using that $T_{\mcP_R}$ and $T_{\mcP_C}$ are symmetric, and likewise for $T_{\mcG_R}$ and $T_{\mcG_C}$.
\end{proof}

It is worth pointing out at this point we could convert back to the total variation distance making the above statement:
\begin{equation*}
    d_{\textrm{TV}}\pbra{\mcP^t_X, \mcG^t_X} \leq (2t+1) \cdot d_{\textrm{TV}}\pbra{(\mcP_R)_X, (\mcG_R)_X}.
\end{equation*}
We have in essence reduced the TV distance bound on our sequential circuit to just a single layer. Our next move will be to reduce the distance further to the individual parallel gates making up each layer, which is what we assumed black box is $\varepsilon'$-approximate $k$-wise independent. Towards this end we write:
\begin{equation*}
    \sum_{Y \in \sD} \abs{\Pr[X \to_{T_{\mcP_R}} Y] - \Pr[X \to_{T_{\mcG_R}} Y]} = \sum_{Y \in \sD} \abs{\prod_{i=1}^{\sqrt{n}}\Pr[X_{i, \cdot} \to_{T_\mcB} Y_{i ,\cdot}] - \prod_{i=1}^{\sqrt{n}}\Pr[X_{i, \cdot} \to_{T_{\mcG_{\sqrt{n}}}} Y_{i ,\cdot}]}.
\end{equation*}
We denote here $\mcG_{\sqrt{n}} = \mcU(\mfS_{\{\pm1\}^{\sqrt{n}}})$. The key fact here is that the operators correspond to product distributions on individual rows. We can again utilize \Cref{fact:differenceofproducts} to simplify the difference of products:
\begin{align*}
    &\sum_{Y \in \sD} \abs{\sum_{j = 1}^{\sqrt{n}} \prod_{i=1}^{j-1}\Pr[X_{i, \cdot} \to_{T_\mcB} Y_{i ,\cdot}] \cdot \pbra{\Pr[X_{j, \cdot} \to_{T_\mcB} Y_{j ,\cdot}] - \Pr[X_{j, \cdot} \to_{T_{\mcG_{\sqrt{n}}}} Y_{j ,\cdot}]} \cdot \prod_{i=j+1}^{\sqrt{n}}\Pr[X_{i, \cdot} \to_{T_{\mcG_{\sqrt{n}}}} Y_{i ,\cdot}]}\\
    \leq&  \sum_{j = 1}^{\sqrt{n}} \sum_{Y \in \sD} \abs{\prod_{i=1}^{j-1}\Pr[X_{i, \cdot} \to_{T_\mcB} Y_{i ,\cdot}] \cdot \pbra{\Pr[X_{j, \cdot} \to_{T_\mcB} Y_{j ,\cdot}] - \Pr[X_{j, \cdot} \to_{T_{\mcG_{\sqrt{n}}}} Y_{j ,\cdot}]} \cdot \prod_{i=j+1}^{\sqrt{n}}\Pr[X_{i, \cdot} \to_{T_{\mcG_{\sqrt{n}}}} Y_{i ,\cdot}]}\\
    =& \sum_{j = 1}^{\sqrt{n}} \sum_{y \in \{\pm1\}^{\sqrt{n}k}} \abs{\Pr[X_{j, \cdot} \to_{T_\mcB} y] - \Pr[X_{j, \cdot} \to_{T_{\mcG_{\sqrt{n}}}} y]} \sum_{\substack{Y \in \sD\\ Y_{j, \cdot} = y}} \prod_{i=1}^{j-1}\Pr[X_{i, \cdot} \to_{T_\mcB} Y_{i ,\cdot}] \prod_{i=j+1}^{\sqrt{n}}\Pr[X_{i, \cdot} \to_{T_{\mcG_{\sqrt{n}}}} Y_{i ,\cdot}]\\
    \leq& \sum_{j = 1}^{\sqrt{n}} \sum_{y \in \{\pm1\}^{\sqrt{n}k}} \abs{\Pr[X_{j, \cdot} \to_{T_\mcB} y] - \Pr[X_{j, \cdot} \to_{T_{\mcG_{\sqrt{n}}}} y]} \sum_{\substack{Y \in \{\pm1\}^{nk}\\ Y_{j, \cdot} = y}} \prod_{i=1}^{j-1}\Pr[X_{i, \cdot} \to_{T_\mcB} Y_{i ,\cdot}] \prod_{i=j+1}^{\sqrt{n}}\Pr[X_{i, \cdot} \to_{T_{\mcG_{\sqrt{n}}}} Y_{i ,\cdot}]\\
    =& \sum_{j = 1}^{\sqrt{n}} \sum_{y \in \{\pm1\}^{\sqrt{n}k}} \abs{\Pr[X_{j, \cdot} \to_{T_\mcB} y] - \Pr[X_{j, \cdot} \to_{T_{\mcG_{\sqrt{n}}}} y]}.
\end{align*}
Here we are partitioning the sum over $Y$ into its fixed row $Y_{j, \cdot}$. The large sum of products we get is just the probability the ``free'' rows map to different elements of $\{\pm1\}^{\sqrt{n}k}$, which marginalizes to 1 when we sum over the entire region. We are nearly done, as the term now looks very close to that which shows up in the definition of $\varepsilon$-approximate $k$-wise independent. The only difference is that we sum over the entire set of rows $\{\pm1\}^{\sqrt{n}k}$, whereas in the definition of $\varepsilon$-approximate $k$-wise independence, the sum is over ``distinct'' $k$-tuples. Distinct $k$-tuples of ``grids'' $X$ may share rows $X_{j, \cdot}$ that are not distinctly colored. Nonetheless, our result still follows from the definition of $\varepsilon$-approximate $k$-wise independence.

Using \Cref{kwiseimplies} below, we compute
\begin{equation*}
    \max_{f \in \mcF} \abs{\ip{e_X}{(T_{\mcP^t}-T_{\mcG^t}) f}} \leq (2t+1) \cdot \sum_{j=1}^{\sqrt{n}} \frac{\varepsilon}{2\sqrt{n} \cdot (2t+1)} \leq \frac\varepsilon2.
\end{equation*}
This completes the proof of \Cref{lem:reduction}.

\begin{lemma}
    \label{kwiseimplies}
    For every $x \in \{\pm1\}^{\sqrt{n}k}$ we have:
    \begin{equation*}
        \sum_{y \in \{\pm1\}^{\sqrt{n}k}} \abs{\Pr[x \to_{T_\mcB} y] - \Pr[x \to_{T_{\mcG_{\sqrt{n}}}} y]} \leq \frac{2\varepsilon}{\sqrt{n} \cdot (2t+1)}.
    \end{equation*}
\end{lemma}


\begin{proof}[Proof of \Cref{kwiseimplies}]
For $x$ corresponding to a distinct $k$-tuple, this reduces to the term in $\varepsilon'$-approximate $k$-wise independence, for which $\mcB$ is assumed to fulfill. The proof is then a matter of showing that ``$\varepsilon$-approximate $k$-wise independence'' implies ``$\varepsilon$-approximate $\tau$-wise independence'' for $\tau < k$. For this, we will need a notion of color class for elements in $\{\pm1\}^{\sqrt{n}k}$ analogous to the one defined in \Cref{sec:colorclasses}. We will define for an equivalence relation $R$ on $[k]$:
\begin{equation*}
    B_R = \left\{x \in \{\pm1\}^{\sqrt{n}k} \mid x^i = x^j \iff i R j\right\}.
\end{equation*}
That is, we think of $x$ as a $k$-tuple of rows and take the corresponding coloring.

First note that we only need to consider terms such that $y \in B(x)$, as if they are not colored the same then the transition probability under any permutation becomes 0. Now assume $x$ is $\tau$-colored for $\tau < k$ and $B = B(x)$. Let $T$ be the set of indices corresponding to the first instances of a color appearing in $x$. For example, if $x$ was colored with $k-1$ colors, with the first and last elements of the $k$-tuple colored the same, then $T$ would be $[k-1]$. Importantly, $T$ is the same across the color class $B(x)$ and $\abs{T} = \tau$. We will then create a function $\varphi_B : \{\pm1\}^{\sqrt{n}k} \to \{\pm1\}^{\sqrt{n}\tau}$ that projects out the indices outside of $T$. As a result we have for all $y \in B$, $\varphi_B(y) \in \sD_{\sqrt{n}}^{(\tau)}$, the set of distinct tuples in $\{\pm1\}^{\sqrt{n}\tau}$ and moreover the image of $\varphi_B$ under $B$ is entirely $\sD_{\sqrt{n}}^{(\tau)}$. The key observation is then that $[k] \setminus T$, the indices not in $T$, can be ignored across transitions since they are completely fixed:
\begin{align*}
    \sum_{y \in B(x)} \abs{\Pr[x \to_{T_\mcB} y] - \Pr[x \to_{T_{\mcG_{\sqrt{n}}}} y]} =& \sum_{y \in B} \abs{\Pr[\varphi_B(x) \to_{T_\mcB} \varphi_B(y)] - \Pr[\varphi_B(x) \to_{T_{\mcG_{\sqrt{n}}}} \varphi_B(y)]}\\
    =& \sum_{\varphi_B(y) \in \sD_{\sqrt{n}}^{(\tau)}} \abs{\Pr[\varphi_B(x) \to_{T_\mcB} \varphi_B(y)] - \Pr[\varphi_B(x) \to_{T_{\mcG_{\sqrt{n}}}} \varphi_B(y)]}.
\end{align*}

Note the end formula above has no dependence on the fixed indices $[k] \setminus T$. This allows us to pretend they are distinct, writing the above sum over elements of $\sD$ instead:
\begin{align*}
    &\sum_{y \in B(x)} \abs{\Pr[x \to_{T_\mcB} y] - \Pr[x \to_{T_{\mcG_{\sqrt{n}}}} y]}\\
    =& \sum_{\varphi_B(y) \in \sD_{\sqrt{n}}^{(\tau)}} \abs{\sum_{y_{[k] \setminus T} \in \sD_{\sqrt{n}}^{(k-\tau)}}\Pr[(\varphi_B(x), \cdot) \to_{T_\mcB} (\varphi_B(y), y_{[k] \setminus T})] - \Pr[\varphi_B(x) \to_{T_{\mcG_{\sqrt{n}}}} (\varphi_B(y), y_{[k] \setminus T})]}\\
    \leq& \sum_{\substack{\varphi_B(y) \in \sD_{\sqrt{n}}^{(\tau)} \\ y_{[k] \setminus T} \in \sD_{\sqrt{n}}^{(k-\tau)}}} \abs{\Pr[(\varphi_B(x), \cdot) \to_{T_\mcB} (\varphi_B(y), y_{[k] \setminus T})] - \Pr[\varphi_B(x) \to_{T_{\mcG_{\sqrt{n}}}} (\varphi_B(y), y_{[k] \setminus T})]}\\
    =& \sum_{y \in \sD^{(k)}_{\sqrt{n}}} \abs{\Pr[(\varphi_B(x), \cdot) \to_{T_\mcB} (\varphi_B(y), y_{[k] \setminus T})] - \Pr[\varphi_B(x) \to_{T_{\mcG_{\sqrt{n}}}} (\varphi_B(y), y_{[k] \setminus T})]}.
\end{align*}

In this last line we can choose $(\varphi_B(x), \cdot)$ to be from $\sD^{(k)}_{\sqrt{n}}$ so any $y$ outside of this class contributes nothing to the sum. Appealing to the approximate $k$-wise independence of $\mcB$ finishes the proof.
\end{proof}

\subsection{Proof of \Cref{lem:maintrick}}
\label{subsec:inductiontrick}

The following lemma will help us achieve the bound in \Cref{lem:maintrick}.
\begin{lemma}
\label[lemma]{lem:offdiagonalmoment}
    Assume the hypothesis of \Cref{lem:maintrick}. Then for any $Y \in\sD$, we have $\abs{\ip{e_X}{(T_{\mcG^t}-T_{\mcG}) e_Y}} \leq \frac{t+1}{2^{\sqrt{n}(t-1)/128}} \cdot \frac{1}{\abs{B(Y)}}$.
\end{lemma}

To see why the lemma is sufficient, observe:
\begin{equation*}
    \sum_{Y \in \sD} \abs{\ip{e_X}{(T_{\mcG^t}-T_{\mcG}) e_Y}} 
    \leq \frac{t+1}{2^{\sqrt{n}(t-1)/128}} \sum_{Y \in \sD} \frac{1}{\abs{B(Y)}} 
    \leq \frac{t+1}{2^{\sqrt{n}(t-1)/128}}\sum_{B \in \sB} \sum_{Y \in B} \frac{1}{\abs{B}} \leq \frac{\abs{\sB} \cdot (t+1)}{2^{\sqrt{n}(t-1)/128}}.
\end{equation*}
Here we partition the sum based on color classes, and note that each color class contributes a total of 1 to the sum. We can use \Cref{fact:numofcolorclasses} bounding the number of color classes and the fact that $\frac{t+1}{2^{t-1}}$ very quickly to write:
\begin{equation*}
    \sum_{Y \in \sD} \abs{\ip{e_X}{(T_{\mcG^t}-T_{\mcG}) e_Y}} 
    \leq \frac{k^{k\sqrt{n}} \cdot (t+1)}{2^{\sqrt{n}(t-1)/128}} \leq \frac{k^{k\sqrt{n}}}{2^{\pbra{\sqrt{n}/128-1} (t-1)}} \leq \frac{\varepsilon}{2}.
\end{equation*}

This is then bounded by $\frac\varepsilon2$ for $t \geq  \frac{\sqrt{n}k\log_2 k + \log_2 2/\varepsilon}{\sqrt{n}/128-1}+1$. When $n$ is large enough, the bound holds when $t\geq 500\pbra{k \log_2 k + \frac{\log_2 1/\varepsilon}{\sqrt{n}}}$.

\begin{proof}[Proof of \Cref{lem:offdiagonalmoment}.]
Recall that $T_{\mcG^t} = T_{\mcG_R}(T_{\mcG_C}T_{\mcG_R})^t$. We can then write $T_{\mcG^t}-T_{\mcG} = (T_{\mcG_R}T_{\mcG_C})^t(T_{\mcG_R} - T_\mcG)$ and prove the claim by induction on $t$. Consider first when $t = 0$.
\begin{equation*}
    \abs{\ip{e_X}{(T_{\mcG_R}-T_{\mcG}) e_Y}} = \abs{\Pr[X \to_{T_{\mcG_R}} Y] - \Pr[X \to_{T_{\mcG}} Y]} = \abs{\Pr[Y \to_{T_{\mcG_R}} X] - \Pr[Y \to_{T_{\mcG}} X]}.
\end{equation*}
We use the self-adjointness of the two operators here. Observe that under the action of $T_{\mcG_R}$, $Y$ goes to a uniform element of $B(Y)$, and under $T_\mcG$ goes to a uniform element of $\sD$. Thus, the quantity is either $\frac{1}{\abs{B(Y)}} - \frac{1}{\abs{\sD}}$ or $\frac{1}{\abs{\sD}}$. Either way it is below $\frac{1}{\abs{B(Y)}}$. For the induction step, we assume the lemma for fixed $t \geq 0$. Then we compute
\begin{align*}
    \abs{\ip{e_X}{(T_{\mcG_R}T_{\mcG_C})^{t+1}(T_{\mcG_R}-T_{\mcG}) e_Y}} &= \abs{\ip{e_X}{(T_{\mcG_R}T_{\mcG_C})(T_{\mcG_R}T_{\mcG_C})^t(T_{\mcG_R}-T_{\mcG}) e_Y}}\\
    &= \abs{T_{\mcG_R}\pbra{T_{\mcG_C}(T_{\mcG_R}T_{\mcG_C})^t(T_{\mcG_R}-T_{\mcG}) e_Y}(X)}\\
    &= \abs{\sum_{Z \in \sD} \Pr[X \to_{T_{\mcG_R}} Z]\pbra{T_{\mcG_C}(T_{\mcG_R}T_{\mcG_C})^t(T_{\mcG_R}-T_{\mcG}) e_Y}(Z)}\\
    &= \abs{\sum_{Z \in \sD} \Pr[X \to_{T_{\mcG_C}T_{\mcG_R}} Z]\pbra{(T_{\mcG_R}T_{\mcG_C})^t(T_{\mcG_R}-T_{\mcG}) e_Y}(Z)}\\
    &= \abs{\sum_{Z \in \sD} \Pr[X \to_{T_{\mcG_C}T_{\mcG_R}} Z] \ip{e_Z}{(T_{\mcG_R}T_{\mcG_C})^t(T_{\mcG_R}-T_{\mcG}) e_Y}}.
\end{align*}
We have managed to write the $(t+1)$ case inner product as a convex combination of the case with $t$. However, if we try to apply the induction hypothesis here we will make no progress. Instead, we will break the sum up and handle only one half with induction. The other half we will bound ``from scratch'', and it is here we will make progress. Recall that we may partition $\sD$ into two regions, $B_\text{safe}$ and $B_\text{coll}$:
\begin{align*}
    \abs{\ip{e_X}{(T_{\mcG_R}T_{\mcG_C})^{t+1}(T_{\mcG_R}-T_{\mcG}) e_Y}} \leq& \abs{\sum_{Z \in B_\text{safe}} \Pr[X \to_{T_{\mcG_C}T_{\mcG_R}} Z] \ip{e_Z}{(T_{\mcG_R}T_{\mcG_C})^t(T_{\mcG_R}-T_{\mcG}) e_Y}}\\
    &\;\;+ \abs{\sum_{Z \in B_\text{coll}} \Pr[X \to_{T_{\mcG_C}T_{\mcG_R}} Z] \ip{e_Z}{(T_{\mcG_R}T_{\mcG_C})^t(T_{\mcG_R}-T_{\mcG}) e_Y}}\\
    \leq& \abs{\sum_{Z \in B_\text{safe}} \Pr[X \to_{T_{\mcG_C}T_{\mcG_R}} Z] \ip{e_Z}{(T_{\mcG_R}T_{\mcG_C})^t(T_{\mcG_R}-T_{\mcG}) e_Y}}\\
    &\;\;+ \Pr[X \to_{T_{\mcG_C}T_{\mcG_R}} B_{\text{coll}}] \cdot \frac{t+1}{2^{\sqrt{n}(t-1)/128}} \cdot \frac{1}{\abs{B(Y)}}.
\end{align*}
In the last line we used the inductive hypothesis. We will then show that the first term is smaller than is demanded by the induction due to a straightforward spectral norm argument. The second term is small because the probability of ``collision'', or that a walk transitions to $B_{\mathrm{coll}}$, is small. More specifically we will need the following two lemmas which we will prove in \Cref{sec:spectralproof}.
\begin{lemma}\label{lem:spectralnorm}
    Assuming $k \leq 2^{\sqrt{n}/500}$ and $n$ large enough, $\norm{T_{\mcG_R}T_{\mcG_C}T_{\mcG_R}-T_{\mcG}}_{\mathrm{op}} \leq \frac{1}{2^{\sqrt{n}/128}}$.
\end{lemma}

\begin{lemma}
\label{lem:lowcollprob}
    Assuming $k \leq 2^{\sqrt{n}/500}$, for all $X \in \sD$, $\Pr[X \to_{T_{\mcG_C}T_{\mcG_R}} B_\text{coll}] \leq \frac{1}{2^{\sqrt{n}/128}}$.
\end{lemma}
To use \Cref{lem:spectralnorm} we write for $Z \in B_{\mathrm{safe}}$:
\begin{align*}
    \abs{\ip{e_Z}{(T_{\mcG_R}T_{\mcG_C})^t(T_{\mcG_R}-T_{\mcG}) e_Y}} =& \abs{\ip{T_{\mcG_R}e_Z}{(T_{\mcG_R}T_{\mcG_C}T_{\mcG_R}-T_{\mcG})^t T_{\mcG_R} e_Y}} \\
    \leq& \norm{T_{\mcG_R}T_{\mcG_C}T_{\mcG_R}-T_{\mcG}}{2}^t \norm{T_{\mcG_R}e_Z}_{2} \norm{T_{\mcG_R}e_Y}_{2}\\
    \leq& \frac{1}{2^{\sqrt{n}t/128}} \cdot \frac{1}{\abs{B(Y)}^{1/2}\abs{B_\text{safe}}^{1/2}}\\
    \leq& \frac{1}{2^{\sqrt{n}t/128}} \cdot \frac{1}{\abs{B(Y)}}.
\end{align*}
The first step uses the self-adjointness of $T_{\mcG_R}$, the fact that $T_{\mcG_R}^2 = T_{\mcG_R}$, and \Cref{fact:TGabsorbs}. The inequality is an application of Cauchy-Schwarz and submultiplicativity of the operator norm. The second to last step uses \Cref{lem:spectralnorm} and \Cref{TGR eU 2 norm} below, and the last step uses \Cref{fact:colorclasssizes}, namely that $B_{\mathrm{safe}}$ is larger than every other color class given our choice of $k$ and large enough $n$.

\begin{claim}\label{TGR eU 2 norm}
    For arbitrary $U \in \{\pm1\}^{nk}$:
\begin{equation*}
    \norm{T_{\mcG_R}e_U}_{2} = \frac{1}{\abs{B(U)}^{1/2}}.
\end{equation*}
\end{claim}

Continuing from the equation above we have:
\begin{align*}
    \abs{\ip{e_X}{(T_{\mcG_R}T_{\mcG_C})^{t+1}(T_{\mcG_R}-T_{\mcG}) e_Y}} \leq& \abs{\sum_{Z \in B_\text{safe}} \Pr[X \to_{T_{\mcG_C}T_{\mcG_R}} Z] \ip{e_Z}{(T_{\mcG_R}T_{\mcG_C})^t(T_{\mcG_R}-T_{\mcG}) e_Y}}\\
    &\;\;+ \Pr[X \to_{T_{\mcG_C}T_{\mcG_R}} B_{\text{coll}}] \cdot \frac{t+1}{2^{\sqrt{n}(t-1)/128}} \cdot \frac{1}{\abs{B(Y)}}\\
    \leq& \frac{1}{2^{\sqrt{n}t/128}} \cdot \frac{1}{\abs{B(Y)}} \tag{\Cref{lem:spectralnorm}}\\
    &\;\;+ \Pr[X \to_{T_{\mcG_C}T_{\mcG_R}} B_{\text{coll}}] \cdot \frac{t+1}{2^{\sqrt{n}(t-1)/128}} \cdot \frac{1}{\abs{B(Y)}}\\
    \leq& \frac{1}{2^{\sqrt{n}t/128}} \cdot \frac{1}{\abs{B(Y)}} + \frac{t+1}{2^{\sqrt{n}t/128}} \cdot \frac{1}{\abs{B(Y)}} \tag{\Cref{lem:lowcollprob}}\\
    \leq& \frac{t+2}{2^{\sqrt{n}t/128}} \cdot \frac{1}{\abs{B(Y)}}.
\end{align*}
This completes the induction. We finish by proving the claim above:

\begin{proof}[Proof of \Cref{TGR eU 2 norm}]
    Observe:
    \begin{align*}
        &\norm{T_{\mcG_R}e_U}_{2} = \sqrt{\sum_{W \in B(U)} \pbra{T_{\mcG_R}e_U(W)}^2} = \sqrt{\sum_{W \in B(U)} \Pr[W \to_{\mcG_R} U]^2} \\
        =& \sqrt{\sum_{W \in B(U)} \pbra{\frac{1}{\abs{B(U)}}}^2} = \frac{1}{\abs{B(U)}^{1/2}}.\qedhere
    \end{align*}
\end{proof}
This completes the proof of \Cref{lem:offdiagonalmoment}.
\end{proof} 

\section{Proof of Spectral Properties of $T_{\mcG_R}$ and $T_{\mcG_C}$}
\label{sec:spectralproof}

In this section we prove \Cref{lem:spectralnorm}, which is a spectral norm bound on the difference between two operators related to our constructions above. As an intermediate result in the proof we will show \Cref{lem:lowcollprob} as well. This will then conclude the proof of \Cref{thm:2D to 1D reduction}.

\begin{lemma}[\Cref{lem:spectralnorm} restated]
    Assuming $k \leq 2^{\sqrt{n}/500}$, we have for large enough $n$,
    \begin{equation*}
        \norm{T_{\mcG_R}T_{\mcG_C}T_{\mcG_R} - T_\mcG}_{\mathrm{op}} \leq \frac{1}{2^{\sqrt{n}/128}},
    \end{equation*}
    or rather for $f : \{\pm 1 \}^{nk} \to \R$:
    \begin{equation*}
        \ip{f}{(T_{\mcG_R}T_{\mcG_C}T_{\mcG_R} - T_\mcG) f} \leq \frac{1}{2^{\sqrt{n}/128}}\cdot \ip{f}{f}.
    \end{equation*}
\end{lemma}

Note that it suffices to prove maximization across symmetric linear forms because the operator is self-adjoint. We will proceed by decomposing $f = f_{B_{\mathrm{safe}}} + f_{B_{\mathrm{coll}}} + f_{B_{=0}}$ where $f_S$ is supported on $S \subseteq \{\pm1\}^{nk}$. Note that these regions form a partition of $\{\pm1\}^{nk}$, so these functions are orthogonal to one another.
\begin{align*}
    \abs{\ip{f}{(T_{\mcG_R}T_{\mcG_C}T_{\mcG_R} - T_\mcG) f}} \leq& \abs{\ip{f_{B_{\mathrm{safe}}}}{(T_{\mcG_R}T_{\mcG_C}T_{\mcG_R} - T_\mcG) f_{\sD}}}\\
    &+ \abs{\ip{f_{B_{\mathrm{coll}}}}{(T_{\mcG_R}T_{\mcG_C}T_{\mcG_R} - T_\mcG) f_\sD}}\\
    &+ \abs{\ip{f_{B_{=0}}}{(T_{\mcG_R}T_{\mcG_C}T_{\mcG_R} - T_\mcG) f_{B_{=0}}}}.
\end{align*}
Note that the cross terms involving $B_{=0}$ are all zero, as a permutation will not cross between these regions. Our proof will bound each of these terms separately.

\subsection{$f$ Supported on $B_{\textrm{safe}}$}

\begin{lemma}
    $\abs{\ip{f_{B_{\mathrm{safe}}}}{(T_{\mcG_R}T_{\mcG_C}T_{\mcG_R} - T_\mcG) f_{\sD}}} \leq \frac{4\sqrt{n}k^2}{2^{\sqrt{n}}} \cdot \ip{f}{f}$.
\end{lemma}
\begin{proof}
Let $X \in B_{\mathrm{safe}}$, $g : \{\pm1\}^{nk} \to \R$.
\begin{align*}
    (T_{\mcG_R} - T_\mcG)g(X ) &= \sum_{Y \in \{\pm1\}^{nk}} \Pr[X \to_{T_{\mcG_R}} Y] \cdot g(Y) - \sum_{Y \in \{\pm1\}^{nk}} \Pr[X \to_{T_{\mcG}} Y] \cdot g(Y)\\
    &= \frac{1}{\abs{B_{\mathrm{safe}}}}\sum_{Y \in B_{\mathrm{safe}}} g(Y) - \frac{1}{\abs{\sD}}\sum_{Y \in \sD} g(Y)\\
    &= \pbra{\frac{1}{\abs{B_{\mathrm{safe}}}} - \frac{1}{\abs{\sD}}}\sum_{Y \in B_{\mathrm{safe}}} g(Y) - \frac{1}{\abs{\sD}}\sum_{Y \in B_{\mathrm{coll}}} g(Y)\\
    &= \pbra{1 - \frac{\abs{B_{\mathrm{safe}}}}{\abs{\sD}}} \cdot \frac{1}{\abs{B_{\mathrm{safe}}}}\sum_{Y \in B_{\mathrm{safe}}} g(Y) - \frac{\abs{B_{\mathrm{coll}}}}{\abs{\sD}} \cdot \frac{1}{\abs{B_{\mathrm{coll}}}}\sum_{Y \in B_{\mathrm{coll}}} g(Y)\\
    &= \frac{\abs{B_{\mathrm{coll}}}}{\abs{\sD}} \pbra{T_{\mcG_R} - \mcH}g(X).
\end{align*}

Our definition of $\mcH g(X) = \frac{1}{\abs{B_{\mathrm{coll}}}}\sum_{Y \in B_{\mathrm{coll}}} g(Y)$ corresponds to the random walk operator that puts all probability weight into $B_{\mathrm{coll}}$ uniformly. Note that $\mcH$ does not correspond to any random walk induced by a distribution on $\mfS_{\{\pm1\}^{nk}}$ (so it cannot be written as $T_{\mcH}$), but is still a random walk operator. With this in hand we may write:
\begin{align*}
    \abs{\ip{f_{B_{\mathrm{safe}}}}{(T_{\mcG_R}T_{\mcG_C}T_{\mcG_R} - T_\mcG) f_{\sD}}} &= \sum_{X \in \{\pm1\}^{nk}} f_{B_{\mathrm{safe}}}(X) \cdot (T_{\mcG_R}-T_\mcG)(T_{\mcG_C}T_{\mcG_R}f_{\sD})(X)\\
    &= \sum_{X \in B_{\mathrm{safe}}} f_{B_{\mathrm{safe}}}(X) \cdot (T_{\mcG_R}-T_\mcG)(T_{\mcG_C}T_{\mcG_R}f_{\sD})(X)\\
    &= \frac{\abs{B_{\mathrm{coll}}}}{\abs{\sD}} \sum_{X \in B_{\mathrm{safe}}} f_{B_{\mathrm{safe}}}(X) \cdot (T_{\mcG_R}-\mcH)(T_{\mcG_C}T_{\mcG_R}f_{\sD})(X)\\
    &= \frac{\abs{B_{\mathrm{coll}}}}{\abs{\sD}} \abs{\ip{f_{B_{\mathrm{safe}}}}{(T_{\mcG_R}T_{\mcG_C}T_{\mcG_R} - \mcH T_{\mcG_C}T_{\mcG_R}) f_{\sD}}}.
\end{align*}

The only important fact about $\mcH$ is that it is a valid random walk operator, which allows us to use \Cref{lem:escape probs} to bound this final inner product crudely as:
\begin{align*}
    \abs{\ip{f_{B_{\mathrm{safe}}}}{(T_{\mcG_R}T_{\mcG_C}T_{\mcG_R} - T_\mcG) f_{\sD}}} &\leq \frac{\abs{B_{\mathrm{coll}}}}{\abs{\sD}} \pbra{\abs{\ip{f_{B_{\mathrm{safe}}}}{(T_{\mcG_R}T_{\mcG_C}T_{\mcG_R} f_{\sD}}} + \abs{\ip{f_{B_{\mathrm{safe}}}}{\mcH T_{\mcG_C}T_{\mcG_R}) f_{\sD}}}}\\
    &\leq \frac{2\abs{B_{\mathrm{coll}}}}{\abs{\sD}} \norm{f_{B_{\mathrm{safe}}}}_{2} \norm{f_\sD}_{2} \tag{\Cref{lem:escape probs}}\\ 
    &\leq \frac{2\abs{B_{\mathrm{coll}}}}{\abs{\sD}} \langle f, f \rangle.
\end{align*}
\Cref{fact:colorclasssizes} then suffices to prove the claim. \end{proof}

\subsection{$f$ Supported on $B_{\textrm{coll}}$}

\begin{lemma}
    \label{lem:collcase}
    $\abs{\ip{f_{B_{\mathrm{coll}}}}{(T_{\mcG_R}T_{\mcG_C}T_{\mcG_R} - T_\mcG) f_\sD}} \leq \frac{8\sqrt{n}k^2}{2^{\sqrt{n}/32}} \ip{f}{f}$.
\end{lemma}


\begin{proof}
First, we can decompose $f_\sD = f_{B_{\mathrm{safe}}} + f_{B_{\mathrm{coll}}}$ and bound:
\begin{align*}
    &\abs{\ip{f_{B_{\mathrm{coll}}}}{(T_{\mcG_R}T_{\mcG_C}T_{\mcG_R} - T_\mcG) f_\sD}} \\
    \leq &\abs{\ip{f_{B_{\mathrm{coll}}}}{(T_{\mcG_R}T_{\mcG_C}T_{\mcG_R} - T_\mcG) f_{B_{\mathrm{safe}}}}} + \abs{\ip{f_{B_{\mathrm{coll}}}}{(T_{\mcG_R}T_{\mcG_C}T_{\mcG_R} - T_\mcG) f_{B_{\mathrm{coll}}}}}.
\end{align*}
By the self-adjointness of the operator, the first term is bounded by the case above, so it suffices to bound the latter. For this term, we can appeal directly to \Cref{lem:escape probs} and the triangle inequality to get:
\begin{align*}
    &\abs{\ip{f_{B_{\mathrm{coll}}}}{(T_{\mcG_R}T_{\mcG_C}T_{\mcG_R} - T_\mcG) f_{B_{\mathrm{coll}}}}} \\
    \leq &\sqrt{\max_{X \in B_{\mathrm{coll}}} \Pr[X \to_{T_{\mcG_R}T_{\mcG_C}T_{\mcG_R}} B_{\mathrm{coll}}] + \max_{X \in B_{\mathrm{coll}}} \Pr[X \to_{T_\mcG} B_{\mathrm{coll}}]} \ip{f_{B_{\mathrm{coll}}}}{f_{B_{\mathrm{coll}}}}.
\end{align*}
Note that regardless of choice of $X$, the latter probability $\Pr[X \to_{T_\mcG} B_{\mathrm{coll}}] = \frac{\abs{B_{\mathrm{coll}}}}{\abs{\sD}}$ which is less than $\frac{2\sqrt{n}k^2}{2^{\sqrt{n}}}$ by \Cref{fact:colorclasssizes}. For the former, we will need a slightly more detailed analysis which also serves as the proof of \Cref{lem:lowcollprob} in the previous section:

\begin{lemma}[Restatement of \Cref{lem:lowcollprob}]
    \label{lem:relowcollprob}
    For all $X \in \sD$, $\Pr[X \to_{T_{\mcG_C}T_{\mcG_R}} B_{\mathrm{coll}}] \leq \frac{2\sqrt{n}k^2}{2^{\sqrt{n}/16}}$.
\end{lemma}

The lemma is immediately sufficient to achieve the bound in \Cref{lem:collcase}.
\end{proof}

\begin{proof}[Proof of \Cref{lem:relowcollprob}]
We will apply a union bound over the probability of any pair of rows ``colliding'', which would put them in $B_{\mathrm{coll}}$. Let $X \in \sD$. We will model our process as:
\begin{equation*}
    X \to_{T_{\mcG_R}} \bm{Y} \to_{T_{\mcG_C}} \bm{Z}.
\end{equation*}
We then fix $\bm{Z}^\ell_{i, \cdot}$ and $\bm{Z}^m_{i, \cdot}$ for $i \in [\sqrt{n}]$, $\ell \neq m \in [k]$. The only fact we will use about $\bm{Y}$ is that for some $j \in [k]$ (potentially equal to $i$), we have $(\bm{Y}^\ell_{j, \cdot}, \bm{Y}^m_{j, \cdot})$ are uniform from ${\{\pm1\}^{\sqrt{n}} \choose 2}$. To see this note that there must exist some $j$ s.t. $X^\ell_{j, \cdot} \neq X^m_{j,\cdot}$, otherwise $X \notin \sD$. Since the permutation applied to these two rows is uniform from $\mfS_{\{\pm1\}^{\sqrt{n}}}$, the resulting rows in $Y$ look like a uniform unequal pair.

With this in mind, we will now condition on the event that $d(\bm{Y}^\ell_{j, \cdot}, \bm{Y}^m_{j, \cdot}) \geq \sqrt{n}/4$ (distance here is Hamming distance), allowing us to split our analysis into two cases:
\begin{align*}
    \Pr[\bm{Z}^\ell_{i, \cdot} = \bm{Z}_{i, \cdot}^m]  \leq&\Pr[\bm{Z}^\ell_{i, \cdot} = \bm{Z}^m_{i, \cdot} \mid d(\bm{Y}^\ell_{j, \cdot}, \bm{Y}^m_{j, \cdot}) > \sqrt{n}/4] + \Pr[d(\bm{Y}^\ell_{j, \cdot}, \bm{Y}^m_{j, \cdot}) \leq \sqrt{n}/4]\\
    \leq&\frac1{2^{\sqrt{n}/4}}+\frac1{e^{\sqrt{n}/16}}\tag{\Cref{lem:given Y far 2D}, \Cref{lem:Y probably far 2D}}\\
    \leq& \frac2{2^{\sqrt{n}/16}}.
\end{align*}
Applying a union bound over $\leq \sqrt{n}k^2$ rows completes the proof.
\end{proof}


\begin{lemma}\label{lem:given Y far 2D}
    $\Pr[\bm{Z}^\ell_{i, \cdot} = \bm{Z}^m_{i, \cdot} \mid d(\bm{Y}^\ell_{j, \cdot}, \bm{Y}^m_{j, \cdot}) > \sqrt{n}/4] \leq \frac{1}{2^{\sqrt{n}/4}}$.
\end{lemma}
\begin{proof}
The probability that $\bm{Z}^\ell_{i, \cdot}$ and $\bm{Z}^m_{i, \cdot}$ are equal can be viewed as the probability that all of their individual bits are equal, and they are all independent since they come from independently sampled column permutations. Since $\bm{Y}^\ell_{j, \cdot}$ and $\bm{Y}^m_{j, \cdot}$ differ in at least $\sqrt{n}/4$ places, $\bm{Y}^\ell$ and $\bm{Y}^m$ must differ in at least that many columns. In these columns, it can be seen that the corresponding bits in $\bm{Z}^\ell_{i, \cdot}$ and $Z^m_{i, \cdot}$ are the same with probability $\leq \frac{1}{2}$ (the probability is exactly one half when the columns are sampled uniformly independently, conditioning that they are unequal only lowers this probability). By independence the probability is less than $\frac{1}{2^{\sqrt{n}/4}}$. \end{proof}

\begin{lemma}\label{lem:Y probably far 2D}
    $\Pr[d(\bm{Y}^\ell_{j, \cdot}, \bm{Y}^m_{j, \cdot}) \leq \sqrt{n}/4] \leq \frac{1}{e^{\sqrt{n}/16}}$.
\end{lemma}
\begin{proof}
Note that $\Pr[d(\bm{Y}^\ell_{j, \cdot},\bm{Y}^m_{j, \cdot}) \leq \sqrt{n}/4] \leq \Pr_{\bx, \by \sim \{\pm1\}^{\sqrt{n}}}[d(\bx, \by) \leq \sqrt{n}/4]$. For uniform $x,y$, the random variable $d(x,y)$ is the sum of $\sqrt{n}$ independent Bernoulli$(1/2)$ random variables. This has expectation $\sqrt{n}/2$ and thus by Hoeffding's Inequality:
\begin{equation*}
    \Pr_{\bx,\by \sim \{\pm1\}^{\sqrt{n}}}[d(\bx,\by) \leq \sqrt{n}/4] \leq e^{-\sqrt{n}/16}.\qedhere
\end{equation*}
\end{proof}

\subsection{The Induction Case}

\begin{lemma} \label{lemmaInductionCase}
Let $f : \{\pm1\}^{nk} \to \R$ be supported on $B_{=0}$ for $k \geq 2$. Then, we have
\[ 
\abs{\ip{f}{\pbra{T_{\mcG_R}^{(k)}T_{\mcG_C}^{(k)}T_{\mcG_R}^{(k)} - T_\mcG^{(k)}}f}} \le \norm{T_{\mcG_R}^{(k-1)}T_{\mcG_C}^{(k-1)}T_{\mcG_R}^{(k-1)} - T_{\mcG}^{(k-1)}}_{\mathrm{op}} \ip{f}{f}. 
\]
\end{lemma}

\begin{proof}
    The proof follows from the same proof as the proof of \Cref{lemma:f supported on B0}.
\end{proof}

\subsection{Wrapping Up}

Putting together all three cases we have:
\begin{equation*}
    \abs{\ip{f}{(T_{\mcG_R}T_{\mcG_C}T_{\mcG_R} - T_\mcG) f}} \leq \frac{4\sqrt{n}k^2}{2^{\sqrt{n}}}\ip{f}{f} + \frac{8\sqrt{n}k^2}{2^{\sqrt{n}/32}} \ip{f}{f} + \norm{T_{\mcG_R}^{(k-1)}T_{\mcG_C}^{(k-1)}T_{\mcG_R}^{(k-1)} - T_\mcG^{(k-1)}}_{\mathrm{op}}\ip{f}{f}.
\end{equation*}

Since $\norm{T_{\mcG_R}^{(1)}T_{\mcG_C}^{(1)}T_{\mcG_R}^{(1)} - T_\mcG^{(1)}}_{\mathrm{op}} = 0$ and by assumption $k \leq 2^{\sqrt{n}/500}$, it follows by induction that:
\[ \norm{T_{\mcG_R}^{(k)}T_{\mcG_C}^{(k)}T_{\mcG_R}^{(k)} - T_\mcG^{(k)}}_{\mathrm{op}} \le \sum_{\ell = 2}^k \pbra{\frac{4\sqrt{n}\ell^2}{2^{\sqrt{n}}} + \frac{8\sqrt{n}\ell^2}{2^{\sqrt{n}/32}}} \leq \frac{k^3}{2^{\sqrt{n}/64}} \leq \frac{1}{2^{\sqrt{n}/128}}. \]





\section*{Acknowledgments}
W.H. thanks Angelos Pelecanos and Lucas Gretta for helpful discussions on the comparison method, and for allowing him to reuse results from their discussion and \cite{gretta2024more} in \Cref{appendix:comparison}.

\bibliographystyle{alpha}
\bibliography{references}


\appendix
\section{Comparison Method}\label{appendix:comparison}


\begin{theorem}[\cite{wilmer2009markov}, Theorem 13.23]
    \label{thm:wilmer comparison}
    Let $\wt{P}$ and $P$ be transition matrices for two ergodic Markov chains on the same state space $V$. Assume that for each $(x,y)\in V^2$ there exists a random path 
    \begin{align*}
        \bm{\Delta}(x,y)=\pbra{(x,\bm{u}_1),(\bm{u}_1,\bm{u}_2),(\bm{u}_2,\bm{u}_3),\dots,(\bm{u}_{\ell},y)}.
    \end{align*}
    Then we have that
    \begin{align*}
        \lambda_2(L)\geq \pbra{\max_{v\in V}\frac{\pi(v)}{\widetilde{\pi}(v)}}A(\bm\Delta)\lambda_2(\wt{L}).
    \end{align*}
    where the comparison constant of $\bm{\Delta}$ is defined to be
    \begin{align*}
        A(\bm{\Delta}) \coloneq \max_{\substack{(a,b)\in V^2 \\\wt{P}(a,b)>0}} \cbra{\frac{1}{\wt{\pi}(x)\wt{P}(a,b)}\sum_{(x,y)\in V^2}\Ex_{\bm{\Delta}}\sbra{\mathbf{1}_{(a, b) \in \bm{\Delta}(x,y)}\cdot |\bm{\Delta}(x,y)|}\cdot  \pi(x)\cdot P(x, y)}.
    \end{align*}
    Here $\pi$ and $\wt{\pi}$ are the (unique) stationary distributions for $P$ and $\wt{P}$, respectively, and $\mathbf{1}_{(a, b) \in \calP}$ is the indicator variable which captures whether $(a,b)$ appears in the sequence $\calP$.
\end{theorem}

\Cref{lem:comparison schreier} is a direct consequence of the following \Cref{cor:comparison random walks}, with symmetry applied.

\begin{corollary}
\label{cor:comparison random walks}
    Let $\wt{P}$ and $P$ be transition matrices for random walks on undirected (multi)graphs $\wt{G}=(V,\wt{E})$ and $G=(V,E)$, respectively. The graph $G$ is $d$-regular, and $\wt{G}$ is $\wt{d}$-regular. Assume that for each $e\in E$ there exists a random path
    \begin{align*}
        \bm{\Gamma}(e)=(\wt{\bm{e}}_1,\dots,\wt{\bm{e}}_{T_e}),
    \end{align*}
    where $(\wt{\bm{e}}_1,\dots,\wt{\bm{e}}_{T_e})$\footnote{Here $T_e$ is a (deterministic) quantity determined by the edge $e$.} is drawn from a distribution on sequences of edges connecting the endpoints of $E$. Then we have for any $f:V\to\R$ that
    \begin{align*}
        \lambda_2(L)\geq \pbra{\max_{v\in V}\frac{\pi(v)}{\widetilde{\pi}(v)}}B(\bm{\Gamma})\lambda_2(\wt{L}).
    \end{align*}
    where the \emph{multigraph comparison constant} is defined as the maximum congestion over all edges $\tilde{e} \in \wt{E}$ connecting vertices $a,b\in V$,
    \begin{align*}
        B(\bm{\Gamma}) \coloneq \max_{\wt{e}\in \wt{E}} \cbra{\frac{\wt{d}}{d}\sum_{e\in E}\Ex_{\bm{\Gamma}}\sbra{\mathbf{1}_{\wt{e} \in \bm{\Gamma}(e)}\cdot |\bm{\Gamma}(e)|}}.\footnotemark
    \end{align*}
    Here $\pi$ and $\wt\pi$ are the (unique) stationary distributions for $P$ and $\wt{P}$, respectively.
    \footnotetext{If $\calP$ is a sequence then $|\calP|$ is its length.}
\end{corollary}
\begin{proof}
    We construct a (randomized) map $\bm\Delta$ as in \Cref{lem:comparison schreier} from the map $\bm\Gamma$ as follows. For each $(x,y)\in V^2$, if $x$ and $y$ are not connected by an edge in $E$ then set $\bm\Delta(x,y)=()$ (the sequence of length 0). Otherwise select a random edge $\bm{e}$ from $(E)_{x,y}$ and let $\pbra{\bm{\wt{e}}_1,\dots,\bm{\wt{e}}_{{T_e}}}$ be the random path $\bm\Gamma(\bm{e})$. For each $i\in [T_e]$ let $(\bm{u}_i,\bm{v}_i)$ be the vertices connected by $\wt{\bm{e}}_i$. Then set 
    \begin{align*}
        \bm\Delta(x,y)=\pbra{(\bm{u}_1,\bm{v}_1),\dots, (\bm{u}_{\ell},\bm{v}_{T_e})}=\pbra{(x,\bm{v}_1),\dots, (\bm{u}_{T_e},y)}.
    \end{align*}
    The comparison constant of $\bm{\Delta}$ is
    $$A(\bm\Delta)=\max_{(a,b)\in V^2} \cbra{\frac{1}{\wt{\pi}(x)\wt{P}(a,b)}\sum_{(x,y)\in V^2}\Ex_{\bm{\Delta}}\sbra{\mathbf{1}_{(a, b) \in \bm{\Delta}(x,y)}\cdot |\bm{\Delta}(x,y)|}\cdot  \pi_{\mathsf{ref}}(x)\cdot P(x, y)}.$$
    Since both Markov chains have the same stationary distribution over the same state space, the $\pi$ terms cancel, so the above is equal to
    \begin{align*}
        &\max_{(a,b)\in V^2} \cbra{\frac{1}{\wt{P}(a,b)}\sum_{(x,y)\in V^2}\Ex_{\bm{\Delta}}\sbra{\mathbf{1}_{(a, b) \in \bm{\Delta}(x,y)}\cdot |\bm{\Delta}(x,y)|}\cdot  P(x, y)}\\
        =&\max_{(a,b)\in V^2} \cbra{\frac{\wt{d}}{\abs{\wt{E}_{a,b}}}\sum_{(x,y)\in V^2}\Ex_{\bm{\Delta}}\sbra{\mathbf{1}_{(a, b) \in \bm{\Delta}(x,y)}\cdot |\bm{\Delta}(x,y)|}\cdot  \frac{\abs{E_{x,y}}}{d}}.
    \end{align*}
    Let us now start translating from pairs of vertices to edges of the multigraph. Each pair of vertices $(x, y)$ has $\left|E_{x,y}\right|$ edges connecting them, we can change the summation from pairs of vertices to edges $e \in E$. Recall that $u(e), v(e)$ are the endpoints of edge $e$. Continuing the calculation, the above is equal to 
    \begin{align*}
        &\max_{(a,b)\in V^2} \cbra{\frac{\wt{d}}{d\abs{\wt{E}_{a,b}}}\sum_{e\in E}\Ex_{\bm{\Delta}}\sbra{\mathbf{1}_{(a, b) \in \bm{\Delta}(u(e),v(e))}\cdot |\bm{\Delta}(u(e),v(e))|} }\\
        =&\max_{(a,b)\in V^2} \cbra{\frac{\wt{d}}{d}\sum_{e\in E} \frac{1}{\abs{\wt{E}_{a,b}}}\Ex_{\bm{\Delta}}\sbra{\mathbf{1}_{(a, b) \in \bm{\Delta}(u(e),v(e))}\cdot |\bm{\Delta}(u(e),v(e))|}}\\
        =&\max_{(a,b)\in V^2} \cbra{\frac{\wt{d}}{d}\sum_{e\in E} \frac{1}{\abs{\wt{E}_{a,b}}}|\bm{\Gamma}(e)|\Pr_{\bm{\Delta}}\sbra{\mathbf{1}_{(a, b) \in \bm{\Delta}(u(e),v(e))}}}.
    \end{align*}
    The last equality is because $|\bm\Delta(u(e),v(e))|=|\bm\Gamma(e)|$ with certainty, and $|\bm\Gamma(e)|$ is a deterministic quantity that only depends on $e$.

    The sum of probabilities that $\wt{e}\in (\wt{E})_{a,b}$ appears in the sequence $\bm\Gamma(e)$, over all such $\wt{e}$, is equal to the probability that $(a,b)$ appears in $\bm\Delta(u(e),v(e))$. By averaging, we have that the probability that $\wt{e}\in \wt{E}_{a,b}$ appears in $\bm\Gamma(e)$, where $\wt{e}$ maximizes this quantity, is at least $\frac1{|\wt{E}_{a,b}|}$ times the appearance probability of $(a,b)$. This results in
    \begin{align*}
        A(\bm\Delta)\leq&\max_{\wt{e}\in \wt{E}} \cbra{\frac{\wt{d}}{d}\sum_{e\in E}\Ex_{\bm{\Gamma}}\sbra{\mathbf{1}_{\wt{e} \in \bm{\Gamma}(e)}\cdot |\bm{\Gamma}(e)|} }=B(\bm\Gamma).
    \end{align*}
    Applying \Cref{thm:wilmer comparison} with this new map $\bm\Delta$ completes the proof.
\end{proof}


\section{Extension to $D$-dimensional Lattices}
\label{sec:generallattices}

\subsection{More Bit Arrays and Color Classes}

For $1\leq D' \leq D$, we regard an element $x \in \{\pm1\}^{n^{D'/D}}$ as a function $x : \sbra{n^{1/D}\,}^{\otimes D'} \to \{\pm1\}$. 
Similarly, we regard an element $X \in \{\pm1\}^{n^{D'/D}k}$ as a function $X : \sbra{n^{1/D}\,}^{\otimes D'} \times \sbra{k} \to \{\pm1\}$. 
For $X \in \{\pm1\}^{n^{D'/D}k}$, $i \in \sbra{n^{1/D}\,}$, $\tau \in \sbra{n^{1/D}\,}^{\otimes D'-1}$ and $\ell \in [k]$, we use the notation:
\begin{itemize}
    \item $X^\ell_{i, \tau} = X(i, \tau, \ell) \in \{\pm1\}$
    \item $X^\ell = X \mid_{\sbra{n^{1/D}\,}^{\otimes D'} \times \{\ell\}} \in \{\pm1\}^{n^{D'/D}}$
    \item $X_{i, \cdot} = X \mid_{\{i\} \times \sbra{n^{1/D}\,}^{\otimes D'-1} \times [k]} \in \{\pm1\}^{n^{(D'-1)/D}k}$
    \item $X_{\cdot, \tau} = X \mid_{\sbra{n^{1/D}\,} \times \{\tau\} \times [k]} \in \{\pm1\}^{n^{1/D}k}$
\end{itemize}

Our definition for coloring will remain the same, namely for $X \in \{\pm1\}^{n^{D'/D}k}$ we will say $X^\ell_{i, \cdot}$ and $X^m_{i, \cdot}$ are colored the same if they are equal, but it is worth noting that these objects are $(D'-1)$-dimensional sublattices and the underlying relations are then $n^{1/D}$ tuples of equivalence relations. Note that in the case $D = 2$ these do in fact correspond to rows. Since the number of such sublattices is $n^{1/D}$ in general, the number of color classes is at most $k^{kn^{1/D}}$.

Our partition into $B_{\mathrm{safe}}$, $B_{\mathrm{coll}}$, and $B_{=0}$ remains mostly the same but based on the generalized notion of color class defined above:
\begin{align*}
    &B_\text{safe} := \cbra{X \in \sD : \forall \ell \neq m \in [k], i \in [n^{1/D}], X^\ell_{i, \cdot} \neq X^m_{i, \cdot}},\\
    &B_\text{coll} := \sD_{n^{D'/D}} \setminus B_\text{safe},\\
    &B_{=0} := \{\pm1\}^{n^{D'/D}k} \setminus \sD_{n^{D'/D}}.
\end{align*}
Throughout this section the value of $D'$ will be clear from context.


\begin{fact}
\label{fact:gencolorclasssizes}
    $\frac{\abs{B_{\mathrm{coll}}}}{\abs{\sD_{n^{D'/D}}}} \leq \frac{2n^{1/D}k^2}{2^{n^{(D'-1)/D}}}$.
\end{fact}


\begin{proof}
We write:
\begin{equation*}
    \frac{\abs{B_{\mathrm{coll}}}}{\abs{\sD_{n^{D'/D}}}} = \frac{\abs{B_{\mathrm{coll}}}}{\abs{\{\pm1\}^{n^{D'/D}k}}} \cdot \frac{\abs{\{\pm1\}^{n^{D'/D}k}}}{\abs{\sD_{n^{D'/D}}}}.
\end{equation*}
The process of sampling from $\{\pm1\}^{n^{D'/D}k}$ can now be seen as sampling $n^{1/D}k$ sublattices from $\{\pm1\}^{n^{(D'-1)/D}}$. Under this view, a simple union bound tells us that there are at most $n^{1/D}k^2$ possible collisions, allowing us to bound the probability by $\frac{n^{1/D}k^2}{2^{n^{(D'-1)/D}}}$. Again, this bounds the size of $\abs{B_{=0}}$ as well, allowing us to crudely claim $\frac{\abs{\sD_{n^{D'/D}}}}{\abs{\{\pm1\}^{n^{D'/D}k}}} \geq \frac{1}{2}$ using our assumption on $k$.
\end{proof}

\subsection{Inductively Defined Random Permutations}\label{sec:defs higher dim}
Fix $n$, $k$, and $D\geq 2$. Let $\mathcal{P}_1$ be a random permutation of $\{\pm1\}^{n^{1/D}}$. We will inductively define for all $2 \leq D'\leq D$ a random permutation $\mcP_{D'}$ on $\{\pm1\}^{n^{D'/D}}$.
\begin{itemize}
    \item Let $\mcP_{D'-1}$ be a distribution on $\mfS{\{\pm1\}^{n^{(D'-1)/D}}}$. 
    \item Let $\mcP_C$ be a distribution on $\mfS{\{\pm1\}^{n^{D'/D}}}$ such that $\pi \sim \mcP_C$ is sampled as follows: 
          Sample $\sigma_\tau \sim \mcP_1$ independently for each $\tau \in \sbra{n^{1/D}\,}^{\otimes D'-1}$ and define $\pi$ such that $\pi(x)_{\cdot, \tau} = \sigma_\tau(x_{\cdot, \tau})$ for all $x \in \{\pm1\}^{n^{D'/D}}$ and all $\tau \in \sbra{n^{1/D}}^{\otimes D'-1}$. 
    \item Let $\mcP_{L,D'-1}$ be a distribution on $\mfS{\{\pm1\}^{n^{D'/D}}}$ such that $\pi \sim \mcP_{L, D'-1}$ is sampled as follows: 
          Sample $\sigma_i \sim \mcP_{D'-1}$ independently for each $i \in \sbra{n^{1/D}}$ and define $\pi$ such that $\pi(x)_{i, \cdot} = \sigma_i(x_{i, \cdot})$ for all $x \in \{\pm1\}^{n^{D'/D}}$ and all $i \in \sbra{n^{1/D'}}$. 
    \item Let $\mcP^0_{D'} = \mcP_{L,D'-1}$. For all $s \ge 1$, let $\mcP_{D'}^{s}$ be the distribution on $\mfS{\{\pm1\}^{n^{D'/D}}}$ such that $\pi \sim \mcP^{s}_{D'}$ is sampled as follows:
          Sample $\sigma_1 \sim \mcP_{D'}^{s-1}$, $\sigma_2 \sim \mcP_{C}$, and $\sigma_3 \sim \mcP_{L,D'-1}$ and define $\pi=\sigma_3 \circ \sigma_2 \circ \sigma_1$.
    \item Set $\mcP_{D'} = \mcP_{D'}^t$, where $t$ is the constant from \Cref{lem:genreduction} below if $D'\geq 3$. Otherwise if $D'=2$ then set $t=\Theta(k\log k)$, where the constant is chosen from the statement of \Cref{thm:2D to 1D reduction technical}.
\end{itemize}
For ease of analyzing the above random permutations, we define the idealized versions of the above distributions based on the following pieces.
\begin{itemize}
    \item Let $\mcG_C$ be a distribution on $\mfS{\{\pm1\}^{n^{D'/D}}}$ such that $\pi \sim \mcG_C$ is sampled as follows: 
          Sample $\sigma_\tau \sim \mcU\pbra{\mfS{\{\pm1\}^{n^{1/D}}}}$ independently for each $\tau \in \sbra{n^{1/D}\,}^{\otimes D'-1}$ and define $\pi$ such that $\pi(x)_{\cdot, \tau} = \sigma_\tau(x_{\cdot, \tau})$ for all $x \in \{\pm1\}^{n}$ and all $\tau \in \sbra{n^{1/D}\,}^{\otimes D'-1}$. 
    \item Let $\mcG_{L, D'-1}$ be a distribution on $\mfS{\{\pm1\}^{n^{D'/D}}}$ such that $\pi \sim \mcG_{L, D'-1}$ is sampled as follows: 
          Sample $\sigma_i \sim \mcU\pbra{\mfS{\{\pm1\}^{n^{(D'-1)/D}}}}$ independently for each $i \in \sbra{n^{1/D}\,}$ and define $\pi$ such that $\pi(x)_{i, \cdot} = \sigma_i(x_{i, \cdot})$ for all $x \in \{\pm1\}^{n}$ and all $i \in \sbra{n^{1/D}\,}$. 
\end{itemize}

\subsection{Generalization of Main Theorem}

Our proof will largely follow the blueprint of the $D = 2$ case, our main result. For $X\in\sD$,
\begin{equation*}
    d_{\textrm{TV}}\pbra{\mcP^t_{D, X}, \mcG_X} \leq d_{\textrm{TV}}\pbra{\mcP^t_{D,X}, \mcG^t_{D,X}} + d_{\textrm{TV}}\pbra{\mcG^t_{D, X}, \mcG_{D, X}}.
\end{equation*}

We prove analogues of \Cref{lem:reduction} and \Cref{lem:maintrick}. 
\begin{lemma}
    \label{lem:genreduction}
    Assume the hypotheses of \Cref{thm:genresult}. Fix any $D'\geq 3$. Suppose that $\mathcal{P}_{D'-1}$ is a $\frac{1}{(4(t+1)n)^{D-D'+1}} \cdot \frac1{2^{n^{1/D}}}$-approximate $k$-wise independent permutation of $\{\pm1\}^{n^{(D'-1)/D}}$ and $\mathcal{P}_1$ is a $\frac1{(4(t+1)n)^{D}}\cdot\frac1{2^{n^{1/D}}}$-approximate $k$-wise independent permutation of $\{\pm1\}^{n^{1/D}}$. Then with the above definitions, for any $X \in \{\pm1\}^{n^{D'/D}k}$,
    \begin{equation*}
        \sum_{Y \in \{\pm1\}^{n^{D'/D}k}} \abs{\ip{e_X}{(T_{\mcP_{D'}^t}-T_{\mcG^t_{D'}}) e_Y}} \leq  \frac12\cdot \frac1{(4(t+1)n)^{D-D'}}\cdot\frac1{2^{n^{1/D}}}.
    \end{equation*}
\end{lemma}

\begin{lemma}
    \label{lem:genmaintrick}
    Assume that $k\log k\leq n^{1/3}$, that $n$ is large enough, and fix $D$. Then for all $t \geq 2500$, any $3\leq D'\leq D$, and any $X \in \{\pm1\}^{n^{D'/D}k}$,
    \begin{equation*}
        \sum_{Y \in \{\pm1\}^{n^{D'/D}k}} \abs{\ip{e_X}{(T_{\mcG^t_{D'}}-T_{\mcG_{D'}}) e_Y}} \leq \frac{1}{(4(t+1)n)^{D-D'+1}}\cdot\frac1{2^{n^{1/D}}}.
    \end{equation*}
\end{lemma}

We apply these two lemmas along with \Cref{tvdistancetolinearform} to obtain the generalization of our main result to higher-dimensional lattices. 

\begin{proof}[Proof of \Cref{thm:genresult}]
    Fix $D$ and set $t\geq 2500$ as in \Cref{lem:genmaintrick}. Let $\mathcal{P}_1$ be a $\frac1{(4(t+1)n)^D}\cdot \frac1{2^{n^{1/D}}}$-approximate $k$-wise independent permutation of $\{\pm1\}^{n^{1/D}}$. Let $\mathcal{P}_{D'}$ be constructed from $\mathcal{P}_1$ as in \Cref{sec:defs higher dim} for all $2\leq D'\leq D$.

    We prove by induction on $D'$ that for all $D'\leq D$, the random permutation $\mathcal{P}_{D'}$ is a $\frac1{(4(t+1)n)^{D-D'}}\cdot\frac1{2^{n^{1/D}}}$-approximate $k$-wise independent permutation of $\{\pm1\}^{n^{D'/D}}$. In the base case $D'=1$, this follows by assumption on $\mcP_1$. In the other base case $D'=2$, this follows from \Cref{thm:2D to 1D reduction technical}. 

    Now fix $3\leq D'\leq D$. Because $k\log k\leq n^{1/3}$ so that the hypothesis of \Cref{lem:genmaintrick} is satisfied. Assume that $\mathcal{P}_{D'-1}$ is a $\frac1{(4(t+1)n)^{D-D'+1}}\cdot\frac1{2^{n^{1/D}}}$-approximate $k$-wise independent permutation of $\{\pm1\}^{n^{(D'-1)/D}}$. By \Cref{lem:genreduction} and \Cref{lem:genmaintrick}, we have that $\mathcal{P}_{D'}^t$ is a $\frac1{(4(t+1)n)^{D-D'}}\cdot\frac1{2^{n^{1/D}}}$-approximate $k$-wise independent permutation of $\{\pm1\}^{n^{D'/D}}$:
    \begin{align*}
        d_{\textrm{TV}}\pbra{\mcP^t_{D', X}, \mcG_X} &\leq d_{\textrm{TV}}\pbra{\mcP^t_{D',X}, \mcG^t_{D',X}} + d_{\textrm{TV}}\pbra{\mcG^t_{D', X}, \mcG_{D',X}}\\
        &\leq \frac12\cdot\frac1{(4(t+1)n)^{D-D'}}\cdot\frac1{2^{n^{1/D}}}+\frac{1}{(4(t+1)n)^{D-D'+1}}\cdot\frac1{2^{n^{1/D}}}\\
        &\leq \frac12\cdot\frac1{(4(t+1)n)^{D-D'}}\cdot\frac1{2^{n^{1/D}}}+\frac12\cdot\frac{1}{(4(t+1)n)^{D-D'}}\cdot\frac1{2^{n^{1/D}}}\\
        &\leq\frac1{(4(t+1)n)^{D-D'}}\cdot\frac1{2^{n^{1/D}}}.
    \end{align*}
    This completes the induction on $D'$. As a result of the induction, we find that $\mathcal{P}_D$ is a $\frac1{2^{n^{1/D}}}$-approximate $k$-wise independent permutation of $\{\pm1\}^{n^{D/D}}=\{\pm1\}^{n}$.


    To instantiate our construction, we take $\mathcal{P}_1$ to be the depth $\widetilde{O}(k)\cdot (n^{1/D}k + n^{1/D}D\log n)=\widetilde{O}(n^{1/D}Dk^2)$ random one-dimensional brickwork circuit from \Cref{thm:1D main}. By \Cref{thm:2D main}, the random permutation $\mathcal{P}_2$ is implemented by a random two-dimensional brickwork circuit of depth $\widetilde{O}(n^{1/D}Dk^3)$. By the construction, if $\mathcal{P}_{D'-1}$ can be implemented by a random $D'-1$-dimensional brickwork circuit of depth $\leq d$ and $\mcP_1$ can be implemented by a random one-dimensional brickwork of depth $\leq d$ then $\mathcal{P}_{D'}$ can be implemented by a random $D'$-dimensional brickwork circuit of depth $d\cdot(2t+1)$. This implies that $\mathcal{P}_D$ can be implemented by a $D$-dimensional brickwork circuit of depth $(2t+1)^{D-2}\cdot \widetilde{O}(n^{1/D}Dk^3)=\mathrm{exp}(D)\cdot \widetilde{O}(n^{1/D}k^3)$. 
\end{proof}


\subsubsection{Proof of \Cref{lem:genreduction}}
Following the proof of \Cref{lem:reduction} in the $D = 2$ case, we use \Cref{fact:differenceofproducts} to bound:
\begin{align*}
     &\sum_{Y \in \{\pm1\}^{n^{D'/D}k}} \abs{\ip{e_X}{(T_{\mcP_{D'}^t}-T_{\mcG^t_{D'}}) e_Y}}\\
     &\leq (t+1) \cdot \sum_{Y \in \{\pm1\}^{n^{D'/D}k}} \abs{\ip{e_X}{(T_{\mcP_{L, D'-1}}-T_{\mcG_{L, D'-1}}) e_Y}} + t \cdot \sum_{Y \in \{\pm1\}^{nk}} \abs{\ip{e_X}{(T_{\mcP_C}-T_{\mcG_C}) e_Y}}.
\end{align*}
To bound each of the two terms, we will establish the following two lemmas.
\begin{lemma}\label{lem:genlatticereduced}
    Assume the hypothesis of \Cref{lem:genreduction}. Then,
    \begin{align*}
        \sum_{Y \in \sD} \abs{\ip{e_X}{(T_{\mcP_{L, D'-1}}-T_{\mcG_{L, D'-1}}) e_Y}} \leq n^{1/D}\cdot \frac{1}{(4(t+1)n)^{D-D'+1}} \cdot \frac1{2^{n^{1/D}}}.
    \end{align*}
\end{lemma}

\begin{lemma}
    \label{lem:genrowreduced}
    Assume the hypothesis of \Cref{lem:genreduction}. Then,
    \begin{align*}
        \sum_{Y \in \sD} \abs{\ip{e_X}{(T_{\mcP_C}-T_{\mcG_C}) e_Y}} \leq n^{(D'-1)/D}\cdot \frac1{(4(t+1)n)^D}\cdot \frac1{2^{n^{1/D}}}.
    \end{align*}
\end{lemma}
Plugging directly into the equation above finishes the proof of \Cref{lem:genreduction}.
\begin{align*}
     &\sum_{Y \in \{\pm1\}^{n^{D'/D}k}} \abs{\ip{e_X}{(T_{\mcP_{D'}^t}-T_{\mcG^t_{D'}}) e_Y}}\\
     &\leq(t+1)\cdot n^{1/D}\cdot \frac{1}{(4(t+1)n)^{D-D'+1}}\cdot \frac1{2^{n^{1/D}}}+ t\cdot n^{(D'-1)/D}\cdot \frac1{(4(t+1)n)^D}\cdot \frac1{2^{n^{1/D}}}\\
     &\leq \frac14\cdot\frac1{(4(t+1)n)^{D-D'}}\cdot\frac1{2^{n^{1/D}}}+\frac14\cdot\frac1{(4(t+1)n)^{D-1}}\cdot\frac1{2^{n^{1/D}}}\\
     &\leq \frac12\cdot\frac1{(4(t+1)n)^{D-D'}}\cdot\frac1{2^{n^{1/D}}}.
\end{align*}
Note that we used the definitions of $\mcP_{D'}^t$ and $\mcG_{D'}^t$ from \Cref{sec:defs higher dim}. This concludes the proof of \Cref{lem:genreduction}.

\begin{proof}[Proof of \Cref{lem:genlatticereduced}]
Recall that $X, Y \in \{\pm1\}^{n^{D'/D}k}$ and we write $X_{i, \cdot}$ for $i \in \sbra{n^{1/D}}$ to denote one of $n^{1/D}$ $(D'-1)$-dimensional slices. The operator $T_{\mcP_{L, D'-1}}$ can be seen as a $n^{1/D}$-wise tensorization of $T_{\mcP_{D-1}}$ acting individually on each slice. As such, we compute:
\begin{align*}
    &\sum_{Y \in \{\pm1\}^{n^{D'/D}k}} \abs{\ip{e_X}{(T_{\mcP_{L, D'-1}}-T_{\mcG_{L, D'-1}}) e_Y}}\\
    &=\sum_Y \abs{ \prod_{i=1}^{n^{1/D}}\Pr[X_{i, \cdot} \to_{T_{\mcP_{D'-1}}} Y_{i ,\cdot}] - \prod_{i=1}^{n^{1/D}}\Pr[X_{i, \cdot} \to_{T_{\mcG_{n^{D'/D}}}} Y_{i ,\cdot}]}\\
    &=\sum_Y \abs{\sum_{j = 1}^{n^{1/D}} \prod_{i=1}^{j-1}\Pr[X_{i, \cdot} \to_{T_{\mcP_{D'-1}}} Y_{i ,\cdot}] \pbra{\Pr[X_{j, \cdot} \to_{T_{\mcP_{D'-1}}} Y_{j ,\cdot}] - \Pr[X_{j, \cdot} \to_{T_{\mcG_{n^{D'/D}}}} Y_{j ,\cdot}]} \prod_{i=j+1}^{n^{1/D}}\Pr[X_{i, \cdot} \to_{T_{\mcG_{n^{D'/D}}}} Y_{i ,\cdot}]}\\
    &\leq  \sum_{j = 1}^{n^{1/D}} \sum_Y \abs{\prod_{i=1}^{j-1}\Pr[X_{i, \cdot} \to_{T_{\mcP_{D'-1}}} Y_{i ,\cdot}] \pbra{\Pr[X_{j, \cdot} \to_{T_{\mcP_{D'-1}}} Y_{j ,\cdot}] - \Pr[X_{j, \cdot} \to_{T_{\mcG_{n^{D'/D}}}} Y_{j ,\cdot}]} \prod_{i=j+1}^{n^{1/D}}\Pr[X_{i, \cdot} \to_{T_{\mcG_{n^{D'/D}}}} Y_{i ,\cdot}]}\\
    &= \sum_{j = 1}^{n^{1/D}} \sum_y \abs{\Pr[X_{j, \cdot} \to_{T_{\mcP_{D'-1}}} y] - \Pr[X_{j, \cdot} \to_{T_{\mcG_{n^{D'/D}}}} y]} \sum_{\substack{Y\\ Y_{j, \cdot} = y}} \prod_{i=1}^{j-1}\Pr[X_{i, \cdot} \to_{T_{\mcP_{D'-1}}} Y_{i ,\cdot}] \prod_{i=j+1}^{n^{1/D}}\Pr[X_{i, \cdot} \to_{T_{\mcG_{n^{D'/D}}}} Y_{i ,\cdot}]\\
    &\leq \sum_{j = 1}^{n^{1/D}} \sum_{y \in \{\pm1\}^{n^{(D'-1)/D}k}} \abs{\Pr[X_{j, \cdot} \to_{T_{\mcP_{D'-1}}} y] - \Pr[X_{j, \cdot} \to_{T_{\mcG_{n^{D'/D}}}} y]}\\
    &\leq n^{1/D}\cdot \frac{1}{(4(t+1)n)^{D-D'+1}} \cdot \frac1{2^{n^{1/D}}}.
\end{align*}
The last line follows from \Cref{genkwiseimplies}.
\end{proof}


\begin{lemma}
    \label{genkwiseimplies}
    For every $x \in \{\pm1\}^{n^{(D'-1)/D}k}$ we have:
    \begin{equation*}
        \sum_{y \in \{\pm1\}^{n^{(D'-1)/D}k}} \abs{\Pr[x \to_{T_{\mcP_{D'-1}}} y] - \Pr[x \to_{T_{\mcG_{n^{(D'-1)/D}}}} y]} \leq \frac{1}{(4(t+1)n)^{D-D'+1}} \cdot \frac1{2^{n^{1/D}}}.
    \end{equation*}
\end{lemma}

\begin{proof}[Proof of \Cref{genkwiseimplies}]
We view $x$ as a $k$-tuple of $(D'-1)$-dimensional grids. We denote by $B$ the ``tuple-wise'' color class of $x$ (if two grids are equal they are colored the same). We create a projection function $\varphi_B$ defined analogously to that in \Cref{kwiseimplies}, taking $x$ to a corresponding $\tau$-tuple with distinct elements.
    \begin{align*}
    &\sum_{y \in \{\pm1\}^{n^{(D'-1)/D}k}} \abs{\Pr[x \to_{T_{\mcP_{D'-1}}} y] - \Pr[x \to_{T_{\mcG_{n^{(D'-1)/D}}}} y]}\\
    =&\sum_{y \in B(x)} \abs{\Pr[x \to_{T_{\mcP_{D'-1}}} y] - \Pr[x \to_{T_{\mcG_{n^{(D'-1)/D}}}} y]}\\
    =& \sum_{y \in B(x)} \abs{\Pr[\varphi_B(x) \to_{T_{\mcP_{D'-1}}} \varphi_B(y)] - \Pr[\varphi_B(x) \to_{T_{\mcG_{n^{(D'-1)/D}}}} \varphi_B(y)]}\\
    =& \sum_{\varphi_B(y) \in \sD_{n^{(D'-1)/D}}^{(\tau)}} \abs{\Pr[\varphi_B(x) \to_{T_{\mcP_{D'-1}}} \varphi_B(y)] - \Pr[\varphi_B(x) \to_{T_{\mcG_{n^{(D'-1)/D}}}} \varphi_B(y)]}\\
    =& \sum_{\varphi_B(y) \in \sD_{n^{(D'-1)/D}}^{(\tau)}} \abs{\sum_{z \in \sD_{n^{(D'-1)/D}}^{(k-\tau)}}\Pr[(\varphi_B(x), \cdot) \to_{T_{\mcP_{D'-1}}} (\varphi_B(y), z)] - \Pr[(\varphi_B(x), \cdot) \to_{T_{\mcG_{n^{(D'-1)/D}}}} (\varphi_B(y), z)]}\\
    \leq& \sum_{\substack{\varphi_B(y) \in \sD_{n^{(D'-1)/D}}^{(\tau)} \\ z \in \sD_{n^{(D'-1)/D}}^{(k-\tau)}}} \abs{\Pr[(\varphi_B(x), \cdot) \to_{T_{\mcP_{D'-1}}} (\varphi_B(y), z)] - \Pr[(\varphi_B(x), \cdot) \to_{T_{\mcG_{n^{(D'-1)/D}}}} (\varphi_B(y), z)]}\\
    =& \sum_{y \in \sD^{(k)}_{n^{(D'-1)/D}}} \abs{\Pr[(\varphi_B(x), \cdot) \to_{T_{\mcP_{D'-1}}} (\varphi_B(y), y_{[k] \setminus T})] - \Pr[(\varphi_B(x), \cdot) \to_{T_{\mcG_{n^{(D'-1)/D}}}} (\varphi_B(y), y_{[k] \setminus T})]}.
\end{align*}
The last step assumes $(\varphi_B(x), \cdot) \in \sD_{n^{(D'-1)/D}}$, that is, it is a distinct $k$-tuple. We then appeal to the fact that ${\mcP_{D'-1}}$ is assumed to be $\frac{1}{(4(t+1)n)^{D-D'+1}} \cdot \frac1{2^{n^{1/D}}}$-approximate $k$-wise independent to finish.
\end{proof}

The proof of \Cref{lem:genrowreduced} is nearly identical to that of \Cref{lem:genlatticereduced}, but partitioning $\{\pm1\}^{n^{D'/D}}$ over one-dimensional columns yields a tensor product of order $n^{(D'-1)/D}$, which becomes a factor in the result, and additionally we appeal to the error in $\mcP_1$ at the end.

\subsubsection{Proof of \Cref{lem:genmaintrick}}
\label{sec:genmaintrick}

This proof follows near identically to \Cref{subsec:inductiontrick}. Throughout this section we assume the hypothesis of \Cref{lem:genmaintrick}, namely that $k\log k\leq n^{1/3}$. It suffices to prove for any $X \in \sD_{n^{D'/D}}$ via \Cref{lem:genoffdiagonalmoment}:
\begin{equation*}
    \sum_{Y \in \sD_{n^{D'/D}}} \abs{\ip{e_X}{(T_{\mcG^t_{D'}}-T_{\mcG_{D'}}) e_Y}} \leq \frac{1}{(4(t+1)n)^{D-D'+1}}\cdot\frac1{2^{n^{1/D}}}.
\end{equation*}

For clarity, we will assume all operators and distributions from this point on are implicitly parameterized by $D'$ and drop the subscript.

\begin{lemma}
\label[lemma]{lem:genoffdiagonalmoment}
    Assume the hypotheses of \Cref{lem:genmaintrick}. Then $\abs{\ip{e_X}{(T_{\mcG^t}-T_{\mcG}) e_Y}} \leq \frac{t+1}{2^{n^{(D'-1)/D}(t-1)/128}} \cdot \frac{1}{\abs{B(Y)}}$.
\end{lemma}
The lemma is used in the following calculation:
\begin{equation*}
    \sum_{Y \in \sD_{n^{D'/D}}} \abs{\ip{e_X}{(T_{\mcG^t}-T_{\mcG}) e_Y}} \leq \frac{t+1}{2^{n^{(D'-1)/D}(t-1)/128}} \sum_{Y \in \sD_{n^{(D'-1)/D}}} \frac{1}{\abs{B(Y)}} \leq \frac{k^{kn^{1/D}} \cdot (t+1)}{2^{n^{(D'-1)/D}(t-1)/128}}.
\end{equation*}
We use that the number of color classes is less than $k^{kn^{1/D}}$. Since $k \log k \leq n^{1/3} \leq n^{(D'-2)/D}$ for $D, D' \geq 3$, we have that:
\begin{equation*}
    \sum_{Y \in \sD_{n^{D'/D}}} \abs{\ip{e_X}{(T_{\mcG^t}-T_{\mcG}) e_Y}} \leq \frac{2^{n^{(D'-1)/D}} \cdot (t+1)}{2^{n^{(D'-1)/D}(t-1)/128}} \leq \frac{1}{2^{(n^{(D'-1)/D}/128-1)(t-1)-1}}.
\end{equation*}

If we set $t = \frac{n^{1/D}D\log_2 (4(t+1)n)+1}{n^{(D'-1)/D}/128-1}+1$ we achieve the desired bound. Note that for large enough $n$ we have $t \leq \frac{2500 D \log_2 n}{n^{1/D}}$. Further if $D \leq \frac{1}{2} \cdot\frac{\log_2 n}{\log_2 \log_2 n}$ then we have $n^{1/D} \geq (\log_2 n)^2$, which is enough to conclude $t \leq 2500$. This concludes the proof of \Cref{lem:genmaintrick}.

\begin{proof}[Proof of \Cref{lem:genoffdiagonalmoment}.]
Recall $T_{\mcG^t} = T_{\mcG_L}(T_{\mcG_C}T_{\mcG_L})^t$ so $T_{\mcG^t} - T_\mcG = (T_{\mcG_L}T_{\mcG_C})^t(T_{\mcG_L} - T_\mcG)$. We induct on $t$. Consider first when $t = 0$.
\begin{equation*}
    \abs{\ip{e_X}{(T_{\mcG_L}-T_{\mcG}) e_Y}} = \abs{\Pr[X \to_{T_{\mcG_L}} Y] - \Pr[X \to_{T_{\mcG}} Y]} = \abs{\Pr[Y \to_{T_{\mcG_L}} X] - \Pr[Y \to_{T_{\mcG}} X]}.
\end{equation*}
Note that we guarantee inductively that $\mcG_L$ is self-adjoint. This quantity is bounded by $\frac{1}{\abs{B(Y)}}$ as before.

For the induction step, assume the lemma for $t \geq 0$ and compute
\begin{align*}
    \abs{\ip{e_X}{(T_{\mcG_L}T_{\mcG_C})^{t+1}(T_{\mcG_L}-T_{\mcG}) e_Y}} 
    =& \abs{\ip{e_X}{(T_{\mcG_L}T_{\mcG_C})(T_{\mcG_L}T_{\mcG_C})^t(T_{\mcG_L}-T_{\mcG}) e_Y}}\\
    =& \abs{T_{\mcG_L}\pbra{T_{\mcG_C}(T_{\mcG_L}T_{\mcG_C})^t(T_{\mcG_L}-T_{\mcG}) e_Y}(X)}\\
    =& \abs{\sum_{Z \in \sD} \Pr[X \to_{T_{\mcG_C}T_{\mcG_L}} Z] \ip{e_Z}{(T_{\mcG_L}T_{\mcG_C})^t(T_{\mcG_L}-T_{\mcG}) e_Y}}\\
    \leq &\abs{\sum_{Z \in B_\text{safe}} \Pr[X \to_{T_{\mcG_L}T_{\mcG_C}} Z] \ip{e_Z}{(T_{\mcG_L}T_{\mcG_C})^t(T_{\mcG_L}-T_{\mcG}) e_Y}}\\
    &\;\;\;\;+ \abs{\sum_{Z \in B_\text{coll}} \Pr[X \to_{T_{\mcG_C}T_{\mcG_C}} Z] \ip{e_Z}{(T_{\mcG_L}T_{\mcG_C})^t(T_{\mcG_L}-T_{\mcG}) e_Y}}\\
    \leq& \max_{Z \in B_{\mathrm{safe}}} \abs{\ip{e_Z}{(T_{\mcG_L}T_{\mcG_C})^t(T_{\mcG_L}-T_{\mcG}) e_Y}}\\
    &\;\;\;\;+ \frac{1}{2^{n^{(D'-1)/D}/128}} \cdot \max_{Z \in B_{\mathrm{coll}}} \abs{\ip{e_Z}{(T_{\mcG_L}T_{\mcG_C})^t(T_{\mcG_L}-T_{\mcG}) e_Y}}\\
    \leq& \max_{Z \in B_{\mathrm{safe}}} \abs{\ip{e_Z}{(T_{\mcG_L}T_{\mcG_C})^t(T_{\mcG_L}-T_{\mcG}) e_Y}} + \frac{t+1}{2^{n^{(D'-1)/D}t/128}} \cdot \frac{1}{\abs{B(Y)}}.
\end{align*}
We apply the induction in the last line and \Cref{lem:genlowcollprob} as stated below in the previous line, in order to bound the probability $X$ lands in the collision region.

\begin{lemma}\label{lem:genspectralnorm}
    Assume the hypotheses of \Cref{lem:genmaintrick}. Then $\norm{T_{\mcG_L}T_{\mcG_C}T_{\mcG_L}-T_{\mcG}}_{2} \leq \frac1{2^{n^{(D'-1)/D}/128}}$.
\end{lemma}

\begin{lemma}
\label{lem:genlowcollprob}
    Assume the hypotheses of \Cref{lem:genmaintrick}. Then for all $X \in \sD$, $\Pr[X \to_{T_{\mcG_C}T_{\mcG_L}} B_\text{coll}] \leq \frac1{2^{n^{(D'-1)/D}/128}}$.
\end{lemma}
We prove these two lemmas in \Cref{sec:genspectral}. The use of \Cref{lem:genlowcollprob} is in bounding the latter term above. To use \Cref{lem:genspectralnorm} we write for $Z \in B_{\mathrm{safe}}$:
\begin{align*}
    \abs{\ip{e_Z}{(T_{\mcG_L}T_{\mcG_C})^t(T_{\mcG_L}-T_{\mcG}) e_Y}} =& \abs{\ip{T_{\mcG_L}e_Z}{(T_{\mcG_L}T_{\mcG_C}T_{\mcG_L}-T_{\mcG})^t T_{\mcG_L} e_Y}} \\
    \leq& \norm{T_{\mcG_L}T_{\mcG_C}T_{\mcG_L}-T_{\mcG}}_{2}^t \norm{T_{\mcG_L}e_Z}_{2} \norm{T_{\mcG_L}e_Y}_{2}\\
    \leq& \frac{1}{2^{n^{(D'-1)/D}t/128}} \cdot \frac{1}{\abs{B(Y)}^{1/2}\abs{B_\text{safe}}^{1/2}}\\
    \leq& \frac{1}{2^{n^{(D'-1)/D}t/128}} \cdot \frac{1}{\abs{B(Y)}}.
\end{align*}
The first step uses the self-adjointness of $T_{\mcG_C}$, the fact that $T_{\mcG_C}^2 = T_{\mcG_C}$, and \Cref{fact:TGabsorbs}. The inequality is an application of Cauchy-Schwarz and submultiplicativity of the operator norm. The second to last step uses \Cref{lem:genspectralnorm} and \Cref{TGL eU 2 norm} below, and the last step uses \Cref{fact:gencolorclasssizes}, namely that $B_{\mathrm{safe}}$ is larger than every other color class for our choice of $k$ and large enough $n$.

\begin{claim}\label{TGL eU 2 norm}
    For arbitrary $U \in \{\pm1\}^{nk}$:
\begin{equation*}
    \norm{T_{\mcG_L}e_U}_{2} = \frac{1}{\abs{B(U)}^{1/2}}.
\end{equation*}
\end{claim}

\begin{proof}
    Observe:
    \begin{equation*}
        \norm{T_{\mcG_L}e_U}_{2} = \sqrt{\sum_{W \in B(U)} \pbra{T_{\mcG_L}e_U(W)}^2} = \sqrt{\sum_{W \in B(U)} \Pr[W \to_{\mcG_L} U]^2} = \sqrt{\sum_{W \in B(U)} \pbra{\frac{1}{\abs{B(U)}}}^2} = \frac{1}{\abs{B(U)}^{1/2}}.\qedhere
    \end{equation*}
\end{proof}
Putting the two together then gives us:
\begin{equation*}
    \abs{\ip{e_X}{(T_{\mcG_L}T_{\mcG_C})^{t+1}(T_{\mcG_L}-T_{\mcG}) e_Y}} \leq \frac{1}{2^{n^{(D'-1)/D}t/128}} + \frac{t+1}{2^{n^{(D'-1)/D}t/128}} \leq \frac{t+2}{2^{n^{(D'-1)/D}t/128}}.
\end{equation*}
This concludes the proof of \Cref{lem:genoffdiagonalmoment}.
\end{proof}

\subsection{Proof of Spectral Properties}\label{sec:genspectral}

In this section we prove \Cref{lem:genspectralnorm} and \Cref{lem:genlowcollprob}. We will proceed by decomposing $f = f_{B_{\mathrm{safe}}} + f_{B_{\mathrm{coll}}} + f_{B_{=0}}$ where $f_B$ is supported on $B \subseteq \{\pm1\}^{n^{D'/D}k}$.
\begin{align*}
    \abs{\ip{f}{(T_{\mcG_L}T_{\mcG_C}T_{\mcG_L} - T_\mcG) f}} \leq &\abs{\ip{f_{B_{\mathrm{safe}}}}{(T_{\mcG_L}T_{\mcG_C}T_{\mcG_L} - T_\mcG) f_{\sD_{n^{D'/D}}}}}
    + \abs{\ip{f_{B_{\mathrm{coll}}}}{(T_{\mcG_L}T_{\mcG_C}T_{\mcG_L} - T_\mcG) f_{\sD_{n^{D'/D}}}}}\\
    &\;\;+ \abs{\ip{f_{B_{=0}}}{(T_{\mcG_L}T_{\mcG_C}T_{\mcG_L} - T_\mcG) f_{B_{=0}}}}.
\end{align*}

\subsubsection{$f$ Supported on $B_{\mathrm{safe}}$}

\begin{lemma}
    $\abs{\ip{f_{B_{\mathrm{safe}}}}{(T_{\mcG_L}T_{\mcG_C}T_{\mcG_L} - T_\mcG) f_{\sD}}} \leq \frac{4n^{1/D}k^2}{2^{n^{(D'-1)/D}}} \cdot \ip{f}{f}$.
\end{lemma}


\begin{proof}
Let $X \in B_{\mathrm{safe}}$, $g : \{\pm1\}^{n^{D'/D}k} \to \R$.
\begin{align*}
    (T_{\mcG_L} - T_\mcG)g(X) &= \sum_{Y \in \{\pm1\}^{n^{D'/D}k}} \Pr[X \to_{T_{\mcG_L}} Y] \cdot g(Y) - \sum_{Y \in \{\pm1\}^{n^{D'/D}k}} \Pr[X \to_{T_{\mcG}} Y] \cdot g(Y)\\
    &= \frac{1}{\abs{B_{\mathrm{safe}}}}\sum_{Y \in B_{\mathrm{safe}}} g(Y) - \frac{1}{\abs{\sD_{n^{D'/D}}}}\sum_{Y \in \sD_{n^{D'/D}}} g(Y)\\
    &= \pbra{\frac{1}{\abs{B_{\mathrm{safe}}}} - \frac{1}{\abs{\sD_{n^{D'/D}}}}}\sum_{Y \in B_{\mathrm{safe}}} g(Y) - \frac{1}{\abs{\sD_{n^{D'/D}}}}\sum_{Y \in B_{\mathrm{coll}}} g(Y)\\
    &= \pbra{1 - \frac{\abs{B_{\mathrm{safe}}}}{\abs{\sD_{n^{D'/D}}}}} \cdot \frac{1}{\abs{B_{\mathrm{safe}}}}\sum_{Y \in B_{\mathrm{safe}}} g(Y) - \frac{\abs{B_{\mathrm{coll}}}}{\abs{\sD_{n^{D'/D}}}} \cdot \frac{1}{\abs{B_{\mathrm{coll}}}}\sum_{Y \in B_{\mathrm{coll}}} g(Y)\\
    &= \frac{\abs{B_{\mathrm{coll}}}}{\abs{\sD_{n^{D'/D}}}} \pbra{T_{\mcG_L} - \mcH}g(X),
\end{align*}
where $\mcH g(X) = \frac{1}{\abs{B_{\mathrm{coll}}}}\sum_{Y \in B_{\mathrm{coll}}} g(Y)$. We write:
\begin{align*}
    \abs{\ip{f_{B_{\mathrm{safe}}}}{(T_{\mcG_L}T_{\mcG_C}T_{\mcG_L} - T_\mcG) f_{\sD_{n^{D'/D}}}}} &= \sum_{X \in \{\pm1\}^{n^{D'/D}k}} f_{B_{\mathrm{safe}}}(X) \cdot (T_{\mcG_L}-T_\mcG)(T_{\mcG_C}T_{\mcG_L}f_{\sD_{n^{D'/D}}})(X)\\
    &= \sum_{X \in B_{\mathrm{safe}}} f_{B_{\mathrm{safe}}}(X) \cdot (T_{\mcG_L}-T_\mcG)(T_{\mcG_C}T_{\mcG_L}f_{\sD_{n^{D'/D}}})(X)\\
    &= \frac{\abs{B_{\mathrm{coll}}}}{\abs{\sD_{n^{D'/D}}}} \sum_{X \in B_{\mathrm{safe}}} f_{B_{\mathrm{safe}}}(X) \cdot (T_{\mcG_L}-\mcH)(T_{\mcG_C}T_{\mcG_L}f_{\sD_{n^{D'/D}}})(X)\\
    &= \frac{\abs{B_{\mathrm{coll}}}}{\abs{\sD_{n^{D'/D}}}} \abs{\ip{f_{B_{\mathrm{safe}}}}{(T_{\mcG_L}T_{\mcG_C}T_{\mcG_L} - \mcH T_{\mcG_C}T_{\mcG_L}) f_{\sD_{n^{D'/D}}}}}.
\end{align*}
The fact that $\mcH$ is a random walk operator once again establishes:
\begin{align*}
    &\abs{\ip{f_{B_{\mathrm{safe}}}}{(T_{\mcG_L}T_{\mcG_C}T_{\mcG_L} - T_\mcG) f_{\sD_{n^{D'/D}}} }} \\
    \leq &\frac{\abs{B_{\mathrm{coll}}}}{\abs{\sD_{n^{D'/D}}}} \pbra{\abs{\ip{f_{B_{\mathrm{safe}}}}{T_{\mcG_L}T_{\mcG_C}T_{\mcG_L} f_{\sD_{n^{D'/D}}}}} + \abs{\ip{f_{B_{\mathrm{safe}}}}{\mcH T_{\mcG_C}T_{\mcG_L} f_{\sD_{n^{D'/D}}}}}}\\
    \leq& \frac{2\abs{B_{\mathrm{coll}}}}{\abs{\sD_{n^{D'/D}}}} \norm{f_{B_{\mathrm{safe}}}}_{2} \norm{f_{\sD_{n^{D'/D}}}}_{2} \tag{\Cref{lem:escape probs}}\\
    \leq& \frac{2\abs{B_{\mathrm{coll}}}}{\abs{\sD_{n^{D'/D}}}} \langle f, f \rangle.
\end{align*}
\Cref{fact:gencolorclasssizes} then suffices to prove the claim. \end{proof}

\subsubsection{$f$ Supported on $B_{\mathrm{coll}}$}

\begin{lemma}
    $\abs{\ip{f_{B_{\mathrm{coll}}}}{(T_{\mcG_L}T_{\mcG_C}T_{\mcG_L} - T_\mcG) f_{\sD_{n^{(D'-1)/D}}}}} \leq \frac{8n^{1/D}k^2}{2^{n^{D'/D}/32}} \ip{f}{f}$.
\end{lemma}


\begin{proof}
First, we can decompose $f_{\sD_{n^{D'/D}}} = f_{B_{\mathrm{safe}}} + f_{B_{\mathrm{coll}}}$ and bound:
\begin{align*}
    \abs{\ip{f_{B_{\mathrm{coll}}}}{(T_{\mcG_L}T_{\mcG_C}T_{\mcG_L} - T_\mcG) f_{\sD_{n^{D'/D}}}}} 
    &\leq \abs{\ip{f_{B_{\mathrm{coll}}}}{(T_{\mcG_L}T_{\mcG_C}T_{\mcG_L} - T_\mcG) f_{B_{\mathrm{safe}}}}} \\
    &+ \abs{\ip{f_{B_{\mathrm{coll}}}}{(T_{\mcG_L}T_{\mcG_C}T_{\mcG_L} - T_\mcG) f_{B_{\mathrm{coll}}}}}.
\end{align*}
By the self-adjointness of the operator, the first term is bounded by the case above, so it suffices to bound the latter. For this term, we can appeal directly to \Cref{lem:escape probs} and the triangle inequality to get:
\begin{align*}
    &\abs{\ip{f_{B_{\mathrm{coll}}}}{(T_{\mcG_L}T_{\mcG_C}T_{\mcG_L} - T_\mcG) f_{B_{\mathrm{coll}}}}} \\
    &\leq\sqrt{\max_{X \in B_{\mathrm{coll}}} \Pr[X \to_{T_{\mcG_L}T_{\mcG_C}T_{\mcG_L}} B_{\mathrm{coll}}] + \max_{X \in B_{\mathrm{coll}}} \Pr[X \to_{T_\mcG} B_{\mathrm{coll}}]} \ip{f_{B_{\mathrm{coll}}}}{f_{B_{\mathrm{coll}}}}.
\end{align*}
Note that regardless of choice of $X$, the latter probability $\Pr[X \to_{T_\mcG} B_{\mathrm{coll}}] = \frac{\abs{B_{\mathrm{coll}}}}{\abs{\sD_{n^{D'/D}}}} \leq \frac{2n^{1/D}k^2}{2^{n^{(D'-1)/D}}}$ by \Cref{fact:gencolorclasssizes}. We finish by proving \Cref{lem:genlowcollprob} from the previous section below.
\end{proof}
\begin{lemma}[Restatement of \Cref{lem:genlowcollprob}]
    For all $X \in \sD_{n^{D'/D}}$, $\Pr[X \to_{T_{\mcG_C}T_{\mcG_L}} B_{\mathrm{coll}}] \leq \frac{2n^{1/D}k^2}{2^{n^{(D'-1)/D}/16}}$.
\end{lemma}
\begin{proof}
    Our goal is to union bound over the probability of any pair of sublattices colliding. There are at most $n^{1/D}k^2$ pairs of sublattices. 
    
    Let $X \in \sD_{n^{D'/D}}$. We will model our process as:
    \begin{equation*}
        X \to_{T_{\mcG_L}} Y \to_{T_{\mcG_C}} Z.
    \end{equation*}
    We then fix $Z^\ell_{i, \cdot}$ and $Z^m_{i, \cdot}$ (which recall are $(D'-1)$-dimensional slices, in $\{\pm1\}^{n^{(D'-1)/D}}$) for $i \in [n^{1/D}]$, $\ell \neq m \in [k]$. We use that for some $j \in [k]$, we have $(Y^\ell_{j, \cdot}, Y^m_{j, \cdot})$ are uniform from ${\{\pm1\}^{n^{(D'-1)/D}} \choose 2}$. To see this note that there must exist some $j$ s.t. $X^\ell_{j, \cdot} \neq X^m_{j,\cdot}$, otherwise $X \notin \sD_{n^{D'/D}}$. Since the permutation applied to these two grids is uniform from $\mfS{\{\pm1\}^{n^{(D'-1)/D}}}$, the resulting rows in $Y$ look like a uniform distinct pair.
    
    With this in mind, we will now condition on the event that $d(Y^\ell_{j, \cdot}, Y^m_{j, \cdot}) \geq n^{(D'-1)/D}/4$ and compute for $n$ large enough that
    \begin{align*}
        \Pr[Z^\ell_{i, \cdot} = Z^m_{i, \cdot}] =& \Pr[Z^\ell_{i, \cdot} = Z^m_{i, \cdot} \mid d(Y^\ell_{j, \cdot}, Y^m_{j, \cdot}) > n^{(D'-1)/D}/4] \\
        +& \Pr[Z^\ell_{i, \cdot} = Z^m_{i, \cdot} \mid d(Y^\ell_{j, \cdot}, Y^m_{j, \cdot}) \leq n^{(D'-1)/D}/4]\Pr[d(Y^\ell_{j, \cdot}, Y^m_{j, \cdot}) \leq n^{(D'-1)/D}/4]\\
        \leq&\Pr[Z^\ell_{i, \cdot} = Z^m_{i, \cdot} \mid d(Y^\ell_{j, \cdot}, Y^m_{j, \cdot}) > n^{(D'-1)/D}/4] +\Pr[d(Y^\ell_{j, \cdot}, Y^m_{j, \cdot}) \leq n^{(D'-1)/D}/4]\\
        \leq&\frac{1}{2^{n^{(D'-1)/D}/16}}+\frac1{e^{n^{(D'-1)/D}/16}}\tag{\Cref{lem:given Y far}, \Cref{lem:Y probably far}}\\
        \leq&\frac{1}{2^{n^{(D'-1)/D}/32}}.
    \end{align*}
    Applying a union bound over all $n^{1/D}k^2$ pairs of sublattices completes the proof.
\end{proof}

\begin{lemma}\label{lem:given Y far}
    $\Pr[Z^\ell_{i, \cdot} = Z^m_{i, \cdot} \mid d(Y^\ell_{j, \cdot}, Y^m_{j, \cdot}) > n^{(D'-1)/D}/4] \leq \frac{1}{2^{n^{(D'-1)/D}/4}}$
\end{lemma}
\begin{proof}
    The probability that $Z^\ell_{i, \cdot}$ and $Z^m_{i, \cdot}$ are equal can be viewed as the probability that all of their individual bits are equal, and they are all independent since they come from independently sampled rows. Since $Y^\ell_{j, \cdot}$ and $Y^m_{j, \cdot}$ differ in at least $n^{(D'-1)/D}/4$ places, $Y^\ell$ and $Y^m$ must differ in at least that many rows. In these rows, it can be seen that the corresponding bits in $Z^\ell_{i, \cdot}$ and $Z^m_{i, \cdot}$ are the same with probability $\leq \frac{1}{2}$. By independence the probability is less than $\frac{1}{2^{n^{(D'-1)/D}/4}}$. 
\end{proof}

\begin{lemma}\label{lem:Y probably far}
    $\Pr[d(Y^\ell_{j, \cdot}, Y^m_{j, \cdot}) \leq n^{(D'-1)/D}/4] \leq \frac{1}{e^{n^{(D'-1)/D}/16}}$
\end{lemma}
\begin{proof}
    This can be seen by a simple Chernoff bound. Note that $\Pr[d(Y^\ell_{j, \cdot}, Y^m_{j, \cdot}) \leq n^{(D'-1)/D}/4] \leq \Pr_{x, y \sim \{\pm1\}^{n^{D'/D}}}[d(x, y) \leq n^{(D'-1)/D}/4]$, as if they are equal the distance is minimized. For uniform $x,y$, $d(x,y)$ can be seen as the sum of $n^{(D'-1)/D}$ independent Bernoulli$(1/2)$ r.v.s. By Hoeffding's Inequality:
    \begin{equation*}
        \Pr_{x,y \sim \{\pm1\}^{n^{(D'-1)/D}}}[d(x,y) \leq n^{(D'-1)/D}/4] \leq e^{-n^{(D'-1)/D}/16}.\qedhere
    \end{equation*}
\end{proof}

\subsubsection{The Induction Case}

\begin{lemma}
Let $f : \{\pm1\}^{n^{D'/D}k} \to \R$ be supported on $B_{=0}$ and $k \geq 2$. Then, we have
\[ 
\abs{\ip{f}{\pbra{T_{\mcG_L}^{(k)}T_{\mcG_C}^{(k)}T_{\mcG_L}^{(k)} - T_\mcG^{(k)}}f}} \le \norm{T_{\mcG_L}^{(k-1)}T_{\mcG_C}^{(k-1)}T_{\mcG_L}^{(k-1)} - T_{\mcG}^{(k-1)}}_{2} \ip{f}{f}. 
\]
\end{lemma}

\begin{proof}
    The proof is nearly notationally identical to \Cref{lemma:f supported on B0} as the notion of color class developed in that section is on the tuple so is not dependent on the choice of sublattice, so we will refer back for brevity.
\end{proof}





\end{document}