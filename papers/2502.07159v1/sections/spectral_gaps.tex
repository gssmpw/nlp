

\section{Spectral Gaps}\label{sec:spectral gaps}



\subsection{Fully Random Gates}
Throughout this section let $n$ be a fixed positive integer. Recalling \Cref{def:the random walk operator}, the operators in focus in this section are operators of the following form for $m\leq n$:
\begin{align*}
    R_{n,m,k}=\frac1{\binom{n}{m}}\sum_{S\in \binom{[n]}{m}}R_{n,S,k}.
\end{align*}
Also define the Laplacian $L_{n,m,k}=L(R_{n,m,k})$.

In this section we prove \Cref{thm:one-random-nonlocal} by building on the previously-mentioned following result of Brodsky and Hoory:

\begin{theorem}[\cite{brodsky2008simple}, Theorem 2]\label{Brodsky-Hoory}
    For any $m$ and $f:\{\pm1\}^{mk}\to\R$ and $k\leq 2^m-2$, we have that
    \begin{align*}
       \abs{ \left\langle f,(R_{m,3,k}-R_{m,m,k})f\right\rangle }\leq 1-\Omega\pbra{\frac1{m^2k}}\langle f,f\rangle.\footnotemark
    \end{align*}
\end{theorem}
\footnotetext{\cite{brodsky2008simple} actually proves this inequality for the operator $R_{m,3,k}^{\des{2}}$ in place of $R_{m,3,k}$, which is the random walk operator induced by placing a random width-2 permutation (which acts on 3 bits). However, a standard comparison of Markov chains shows that our statement of the result easily follows. See \Cref{sec:DES gate set}.}
Our main contribution is a finer analysis in the case when $k$ is small relative to $m$, which results in the following theorem.

\begin{theorem}\label{thm:small k}
    Assume that $m\geq100$ and $k\leq 2^{m/10}$. Given $f:\{\pm1\}^{mk}\to\R$, we have 
    \begin{align*}
       \abs{ \left\langle f,(R_{m,m-1,k}-R_{m,m,k})f\right\rangle }\leq  \pbra{\frac1m+\frac{k^2}{2^{m/4}}}\left\langle f,f\right\rangle.
    \end{align*}
\end{theorem}

\Cref{thm:small k} is proven as \Cref{thm:small k internal} in \Cref{sec:small k}.


The following lemma allows us to combine \Cref{thm:small k} with the previously-known \Cref{Brodsky-Hoory} to bring the quadratic dependence on the number of wires ($n$) to linear.

\begin{lemma}[\cite{o2023explicit}, Lemma 3.2]\label{lem:induction}
    Fix a positive integer $n_0\geq 4$. For each $k$ and $m_1\geq m_2$ let $\tau_{m_1,m_2,k}$ be some real number such that
    \begin{align*}
        L_{m_1,m_2,k}\geq \tau_{m_1,m_2,k}L_{m_1,m_1,k}.\footnotemark
    \end{align*}
    \footnotetext{All inequalities between operators in this paper are in the PSD order.}
    Then for any sequence $n_0=m_0\leq m_1\leq \dots\leq m_{t-1}\leq m_t = n$ we have
    \begin{align*}
        L_{n,n_0,k}\geq \pbra{\prod_{i\in[t-1]}\tau_{m_i,m_{i-1},k}}L_{n,n,k}.
    \end{align*}
\end{lemma}

Together, \Cref{Brodsky-Hoory}, \Cref{thm:small k}, and \Cref{lem:induction} yield the following initial spectral gap. 

\begin{corollary}\label{cor:initial spectral gap}
    For any $n$ and $k\leq 2^{n/3}$, we have
    \begin{align*}
        L_{n,3,k}\geq \Omega\pbra{\frac1{nk \cdot\log k}}L_{n,n,k}.
    \end{align*}
\end{corollary}
\begin{proof}
    \Cref{thm:small k} shows that for all $m\geq 20\log k$ we have $\tau_{m,m-1,k}\geq 1-\frac1m-\frac{k^2}{2^{m/3}}$. \Cref{Brodsky-Hoory} shows that 
    \begin{align*}
        L_{20\log k,3,k}\geq  \Omega\pbra{\frac1{k\cdot\log^2k }}L_{20\log k,20\log k,k},
    \end{align*}
    or equivalently $\tau_{20\log k,3,k}\geq \Omega\pbra{\frac1{k\log^2 k}}$.

    Combining these using \Cref{lem:induction} we have for large enough $k$ that
    \begin{align*}
        L_{n,3,k}\geq &\Omega\pbra{\frac1{k\log^2 k}}\prod_{m=20\log k}^{n}\pbra{1-\frac1m-\frac{k^2}{2^{m/4}}}L_{n,n,k}\\
        \geq &\Omega\pbra{\frac1{k\log^2 k}}\frac{20\log k}{n} \prod_{m=20\log k}^{n}\pbra{1-\frac{k^2}{2^{m/4-1}}}\cdot L_{n,n,k}\\
        \geq&\Omega\pbra{\frac1{nk\cdot\log k}}\cdot\pbra{1-\sum_{m=20\log k}^{n}\frac{k^2}{2^{m/4-1}}}\cdot L_{n,n,k} \geq \Omega\pbra{\frac1{nk\cdot\log k}}\cdot L_{n,n,k}.\qedhere
    \end{align*}
\end{proof}


We leverage the initial spectral gap from \Cref{cor:initial spectral gap} to produce our designs by sequentially composing many copies of this pseudorandom permutation. This is akin to showing that the second largest eigenvalue of the square of a graph is quadratically smaller than that of the original graph.

\Cref{thm:one-random-nonlocal} follows directly from \Cref{sec:DES gate set} with standard spectral-gap to mixing time bounds.


\subsection{Nearest-Neighbor Random Gates}
\subsubsection{Reduction to the Large $k$ Case}

One step in a random reversible circuit with 1D-nearest-neighbor gates is described by the operator $R_{n,[n-2],k}^{3\text{-NN}}$. Note that we can write 
\begin{align*}
    R_{n,[n-2],k}^{3\text{-NN}}=& \Ex_{\mathbf{a}\in[n-2]}\sbra{R_{n,\{\mathbf{a},\mathbf{a}+1,\mathbf{a}+2\},k}}.\footnotemark
\end{align*}
Define the corresponding Laplacian $L_{n,I,k}^{\gamma\text{-NN}}=L(R_{n,I,k}^{\gamma\text{-NN}})$. 

Because the local terms $R_{n,\{\mathbf{a},\mathbf{a}+1,\mathbf{a}+2\},k}$ are projectors, to analyze such an operator we can use the following theorem of Nachtergaele.
\footnotetext{More generally, we write $R_{n,S,k}^{\gamma\text{-NN}}$ to denote $\Ex_{a\in S}\sbra{R_{n,\{\mathbf{a},\dots,\mathbf{a}+\gamma-1\},k}}$.}
\begin{theorem}[\cite{nachtergaele1996spectral}, Theorem 3]\label{thm:nachtergaele}
    Let $\{h_{a,a+1,a+2}\}_{a\in[n-2]}$ be projectors acting on $(\R^2)^{\otimes d}$ such that each $h_{\{a,a+1,a+2\}}$ only acts on the $a,a+1,a+2$th tensor factor. For $I=[a,b]\subseteq[n]$ define the subspace
    \begin{align*}
        \mathcal{G}_I=& \cbra{f\in (\R^2)^{\otimes d}:\sum_{a'\in [a,b-2]}h_{a',a'+1,a'+2}f = 0}.
    \end{align*}
    Let $G_I$ be the projector to $\mathcal{G}_I$.

    Now suppose there exists $\ell$ and $n_\ell$ and $\epsilon_\ell\leq \frac1{\sqrt\ell}$ such that for all $n_\ell\leq m\leq n$,
    \begin{align*}
        \norm{G_{[m-\ell-1,m]}\pbra{G_{[m-1]}-G_{[m]}}}_{\mathrm{op}}\leq \epsilon_\ell.
    \end{align*}
    Then
    \begin{align*}
        \lambda_{2}\pbra{\sum_{a\in[n-2]}h_{\{{a},{a}+1,{a}+2\}}}&\geq \frac{\pbra{1-\epsilon_\ell\sqrt\ell}^2}{\ell-3}\lambda_{2}\pbra{
        \sum_{a\in[\ell]}h_{\{{a},{a}+1,{a}+2\}}}.
    \end{align*}
    Recall $\lambda_2(h)$ denotes the second-smallest distinct eigenvalue of the operator $h$.
\end{theorem}

\begin{theorem}\label{thm:nachtergaele hypothesis}
    Fix any $m\geq 100$ and $k\leq 2^m-2$ and set $\ell=10\log k$. Then we have
    \begin{align*}
        \norm{R_{m,[m-\ell-1,m],k}\pbra{R_{m,[m-1],k}-R_{m,[m],k}}}_{\mathrm{op}}\leq \frac1\ell.
    \end{align*}
\end{theorem}
\Cref{thm:nachtergaele hypothesis} is established in \Cref{sec:nachtergaele hypothesis} as \Cref{thm:nachtergaele hypothesis internal}.

\begin{corollary}\label{cor:reduce to logk}
    We have for $k\geq 3$ that 
    \begin{align*}
        \lambda_2\pbra{L_{n,[n-2],k}^{3\text{-NN}}}\geq \frac{1}{2n}\lambda_2\pbra{L_{10\log k+2,[10\log k-2],k}^{3\text{-NN}}}
    \end{align*}
\end{corollary}
\begin{proof}
    Setting $h_{a,a+1,a+2}=\Id-R_{n,\{a,a+1,a+2\},k}$ for each $a\in [n]$, we see that the projections $G$ (as in the statement of \Cref{thm:nachtergaele}) are given by $G_{[a]}=R_{n,[\min\{n,a+2\}],k}$ for any $a\in[n]$. To see this, note first that $R_{n,[\min\{n,a+2\}],k}$ is indeed a projection. Now let $f$ be such that $R_{n,[\min\{n,a+2\}],k}f=f$. For every $\sigma\in \mathfrak{S}_{\{\pm1\}^{n}}$ define $f^\sigma$ by $f^\sigma(X)=f(\sigma X)$ for all $X\in\{\pm1\}^{nk}$. Then by invariance of $R_{n,[\min\{n,a+2\}],k}$ under a permutation applied to bits in $[\min\{n,a+2\}]$ we have
    \begin{align*}
        f^\sigma = R_{n,[\min\{n,a+2\}],k}f^\sigma =R_{n,[\min\{n,a+2\}],k}f =f.
    \end{align*}
    for any $\sigma \in \mathfrak{S}_{\{\pm1\}^{[\min\{n,a+2\}]}}$. The converse of this holds as well by a similar argument. Therefore, $f^{\sigma^{\{a',a'+1,a'+2\}}} = f$ for any $\sigma\in \mathfrak{S}_{\{0,1\}^8}$ for $a'\leq a$, proving that such an $f$ is truly in the ground space of $R_{n,\{a',a'+1,a'+2\},k}$ for any $a'\leq a$. The converse of this holds as well, because the permutations of the form $\sigma^{\{a',a'+1,a'+2\}}$ with $a'\leq a$ generate the group of permutations of the form $\rho^{\{a,b,c\}}$ for $\{a,b,c\}\in\binom{[\min\{n,a+2\}]}{3}$. Such an argument also proves the converse, so the $R$ operators are serve as the projections from \Cref{thm:nachtergaele}.
    
    Therefore, by \Cref{thm:nachtergaele hypothesis} we have that the hypotheses of \Cref{thm:nachtergaele} are satisfied with $\ell=10\log k$ and $\epsilon_\ell = \frac1\ell$. That is, for any $m\leq n$ we have
    \begin{align*}
    &\norm{G_{[m-\ell-1,m]}\pbra{G_{[m-1]}-G_{[m]}}}_{\mathrm{op}}
        =\norm{R_{n,[m-\ell-1,m],k}\pbra{R_{n,[m-1],k}-R_{n,[m],k}}}_{\mathrm{op}}\\
        =&\norm{\pbra{R_{m,[m-\ell-1,m],k}\pbra{R_{m,[m-1],k}-R_{m,[m],k}}}\otimes \Id_{[m+1,n]}}_{\mathrm{op}}
        \leq \frac1\ell.
    \end{align*}
    Therefore the conclusion of \Cref{thm:nachtergaele} is that 
    \begin{align*}
        &\lambda_2\pbra{(n-2)L_{n,[n-2],k}^{3\text{-NN}}}
        \geq \frac{\pbra{1-\frac1{\sqrt\ell}}^2}{\ell-2}\lambda_2\pbra{\ell L_{\ell+2,[\ell-2],k}^{3\text{-NN}}}
        \geq \frac{1}{2}\lambda_2\pbra{L_{\ell+2,[\ell-2],k}^{3\text{-NN}}}.
    \end{align*}
    Recalling our setting of $\ell$ completes the proof. 
\end{proof}



\subsubsection{Comparison Method for the Large $k$ Case}
We can use the spectral gap proved in \Cref{Brodsky-Hoory} for the random walk induced by completely random 3-bit gates to show a spectral gap for the random walk induced by random 3-bit gates, where the three bits on which the gate acts on are $a,a+1,a+2$ for some $a\in[n]$. Our proof is a simple application of the comparison method applied to random walks on multigraphs.

We take the following definition of (multi)graphs. A graph is a pair of sets $(V,E)$ such that there is a partition $E=\bigcup_{(x,y)\in V^2}E_{x,y}$. If $e\in E_{x,y}$ then we say that $e$ \textit{connects} the vertex $x$ to the vertex $y$, and we define $u(e)=x$ and $v(e)=y$. The degree $\deg(x)$ of a vertex $x\in V$ is the number of edges originating at $e$, or $\sum_{y\in V}|E_{x,y}|$. We say that a graph is regular if $\deg(x)=\deg(y)$ for all $x,y\in V$.

The \textit{random walk} on a graph $(V,E)$ beginning at a vertex $x\in V$ consists of the Markov chain $\{\bm{x}_i\}_{i\geq0}$ on state space $V$ such that $\bm{x}_0=x$ with probability 1, and to draw $\bm{x}_{i+1}$ given $\bm{x}_i$ we sample a uniform random edge $\bm{e}$ from $\bigcup_{y\in V}E_{x,y}$. We set $\bm{x}_{i+1}$ equal to the unique $y\in V$ such that $\bm{e}\in E_{x,y}$.

A \textit{Schreier graph} is a graph with vertices $V$ such that some group $\mathfrak{G}$ acts on $V$. Let $S\subseteq \mathfrak{G}$ be some subset of group elements. The edge set consists of elements of the form $(X,\sigma)$ ($\sigma\in S$), so that $(X,\sigma)\in E_{X,\sigma X}$. We call the resulting graph $\mathrm{Sch}(V,S)$.

\begin{definition}\label{def:congestion}
    Let $S$ and $\widetilde{S}$ be subsets of a group acting on a set $V$. For each ${\sigma}\in {S}$ let $\Gamma({\sigma})$ be a sequence $(\wt{\sigma}_1,\dots,{\wt\sigma}_t)$ of elements of $\wt S$ such that $\sigma v=\wt{\sigma}_t\dots \wt{\sigma}_1 v$ for all $v\in V$, so we regard $\Gamma$ as a map from $S$ to sets of paths using edges in $\wt{S}$. Define the \textit{congestion ratio} of $\Gamma$ to be
    \begin{align*}
        B({\Gamma})=&\max_{\wt\sigma\in \wt{S}}\cbra{|\wt{S}|\Ex_{\sigma\in S}\sbra{{N(\wt{\sigma},\Gamma(\sigma))|\Gamma({\sigma})|}}},
    \end{align*}
    where $N(\wt{\sigma},\Gamma(\sigma))$ is the number of times $\wt \sigma$ appears in the sequence $\Gamma(\sigma)$.
\end{definition}

\begin{lemma}\label{lem:comparison schreier}
    Let $G=\mathrm{Sch}(V,S)$ and $\wt{G}=\mathrm{Sch}(V,\wt{S})$ be the connected Schreier graphs of the action of a group. Let $L$ and $\widetilde{L}$ be the Laplacian operators for the non-lazy random walks on $G$ and $\wt{G}$, respectively. Suppose there exists a $\Gamma$ as in \Cref{def:congestion}. Then 
    \begin{align*}
        \lambda_2(L)\geq \pbra{\max_{v\in V}\frac{\pi(v)}{\widetilde{\pi}(v)}}B({\Gamma})\lambda_2(\wt{L}).
    \end{align*}
    Here $\pi$ and $\widetilde{\pi}$ are the stationary distributions for $G$ and $\widetilde{G}$, respectively.
\end{lemma}

We prove this in \Cref{appendix:comparison}. It is essentially a reformulation of a standard result about comparisons on general Markov chains in \cite{wilmer2009markov}. 

\begin{lemma}\label{lem:initial spectral gap local random gates large k}
    For any $n,k$ we have
    \begin{align*}
        \lambda_2\pbra{L_{n,[n-2],k}^{3\text{-NN}}}\geq \frac1{100000n^3}\lambda_2\pbra{L_{n,3,k}}.
    \end{align*}
\end{lemma}
\begin{proof}
    Note that $L_{n,3,k}$ and $L_{n,[n-2],k}^{3\text{-NN}}$ are simply the Laplacians of random walks on Schreier graphs with $\mathfrak{S}_{\{\pm1\}^{n}}$ acting on $\{\pm1\}^{nk}$ by $e(X^1,\dots,X^k)=(eX^1,\dots,eX^k)$. In the case of $L_{n,3,k}$ the edges are given by elements of the form $h^{\{a,b,c\}}$ for $h\in\mathfrak{S}_{\{\pm1\}^3}$ and $\{a,b,c\}\in\binom{[n]}{3}$. In the case of $L_{n,[n-3],k}^{3\text{-NN}}$ the edges are given by elements of the form $g^{\{a,a+1,a+2\}}$ for $g\in\mathfrak{S}_{\{\pm1\}^3}$ and $a\in[n-2]$. We deal with each connected component separately. Note that every connected component is isomorphic to $\{(X^1,\dots,X^{k'}):X^i\neq X^j\iff i\neq j\}$ for $k'\leq k$, so we bound the spectral gap for the walk on $\{(X^1,\dots,X^{k}):X^i\neq X^j\iff i\neq j\}$.

    We provide a map $\Gamma$ from $\{h^{\{a,b,c\}}:h\in\mathfrak{S}_{\{0,1\}^3},a,b,c\in[n]\}$ to sequences of elements of the form $g^{\{a,a+1,a+2\}}$ for $a\in[n-2]$ such that for any $h^{\{a,b,c\}}$ the sequence $\Gamma(h^{\{a,b,c\}})=(g_1^{\{a_1,a_1+1,a_1+2\}},\dots ,g_t^{\{a_t,a_t+1,a_t+2\}})$ satisfies $h^{\{a,b,c\}}=g_t^{\{a_t,a_t+1,a_t+2\}}\dots g_1^{\{a_1,a_1+1,a_1+2\}}$.

    The hope is to construct $\Gamma$ such that $B(\Gamma)$ is small and then to apply \Cref{lem:comparison schreier}. To this end we define $\Gamma$ as follows. Assume $a<b<c$. Fix $h\in\mathfrak{S}_8,a\in[n-2]$. Let $d\in[n-2]$ be arbitrary. Then write
    \begin{align*}
        h^{\{a,b,c\}}=\mathsf{Sort}^{-1} \cdot g^{\{d,d+1,d+2\}}\cdot\mathsf{Sort},
    \end{align*}
    Here $\mathsf{Sort}$ sends the $a$th coordinate to the $d$th coordinate, the $b$th to the $d+1$th, and the $c$th to the $d+2$th. The permutations $\mathsf{Sort}$ and $\mathsf{Sort}^{-1}$ can each be implemented using at most $3n$ gates of the form $g^{\{a'-1,a',a'+1\}}$ and $g^{\{a'+1,a'+2,a'+3\}}$ where each $g$ swaps coordinates; this is just by a standard partial sorting algorithm. Write $\mathsf{Sort}=g_{3n}^{\{a_{3n},a_{3n}+1,a_{3n}+2\}}\dots g_1^{\{a_1,a_1+1,a_1+2\}}$. Then set 
    \begin{align*}
        \Gamma(h^{\{a,b,c\}})=\pbra{g_1^{\{a_1,a_1+1,a_1+2\}},\dots,g_{3n}^{\{a_{3n},a_{3n}+1,a_{3n}+2\}},h^{(d,d+1,d+2)},\dots,\pbra{g_1^{\{a_1,a_1+1,a_1+2\}}}^{-1}}.
    \end{align*}
    We have $B(\Gamma)\leq 100000n^3$ trivially, so we have proved the result by applying \Cref{lem:comparison schreier} and the fact that the stationary distributions for both chains are uniform.
\end{proof}


\begin{corollary}\label{cor:initial spectral gap local}
For any $n,k$ we have
    \begin{align*}
        \lambda_2\pbra{L_{n,[n-2],k}^{3\text{-NN}}}\geq \Omega\pbra{\frac1{nk\cdot\log^5(k)}}.
    \end{align*}
\end{corollary}
\begin{proof}
    \Cref{lem:initial spectral gap local random gates large k} shows that $\lambda_2\pbra{L_{\ell+2,[\ell],k'}^{3\text{-NN}}}\geq \frac{1}{\ell^3}\lambda_2\pbra{L_{\ell+2,3,k'}}$ for all $\ell$ and $k'$. \Cref{Brodsky-Hoory} states that $\lambda_2\pbra{L_{\ell+2,3,k'}}\geq \frac{1}{(\ell+2)^2k'}$ for all $\ell$ and $k'$. If we set $\ell=10\log k$ and use \Cref{cor:reduce to logk} then we get
    \begin{align*}
         &\lambda_2\pbra{L_{n,[n-2],k}^{3\text{-NN}}} 
         \geq \frac{1}{2n}\cdot\lambda_2\pbra{L_{10\log k+2,[10\log k],k}^{3\text{-NN}}}
         \geq  \frac{1}{2n}\cdot\frac1{100000k\log^5(k)}.
    \end{align*}
    This implies the result.
\end{proof}

As in the case for fully random gates, \Cref{thm:one-random-local} follows from \Cref{cor:initial spectral gap local} and \Cref{sec:DES gate set}.

\subsection{Restricting the Gate Set}\label{sec:DES gate set}

So far all of our results have dealt with random circuits with arbitrary gates acting on 3 bits. However, for practical applications we are often in further restricting the type of 3-bit gates. However, as long as the arbitrary gate set is universal on 3 bits, we lose just a constant factor in the mixing time when we restrict our random circuits to use that gate set, by a standard application of the comparison method. We prove that we can perform this conversion before proving our results about brickwork circuits in \Cref{sec:brickwork} because it is somewhat easier to prove for the case of single gates acting at a time.

\begin{lemma}\label{lem:any gate set comparison}
    We have
    \begin{align*}
        \lambda_2\pbra{L_{n,[n-2],k}^{3\text{-NN},\des{2}}}\geq \Omega\pbra{\lambda_2\pbra{L_{n,[n-2],k}^{3\text{-NN}}}}
    \end{align*}
\end{lemma}
\begin{proof}
    We compare the Markov chains given by these two Laplacians by providing a way to write edges in the one induced by arbitrary 3-bit nearest-neighbor gates as paths in the one induced by 3-bit nearest-neighbor gates with generators $\mathcal{G}$. Again we focus on the connected component $\{(X^1,\dots,X^{k}):X^i\neq X^j\iff i\neq j\}$.

    For each $g^S$ for $g\in\mathfrak{S}_8$ and $S\in \binom{[n]}{3}$ let $\Gamma(g^S)=\pbra{g_{i_1}^S,\dots,g_{i_{8!}}^S}$ where we have fixed an arbitrary expansion of $g=g_{i_1}\dots g_{i_{8!}}$, and each $g_{i_j}$ is of type $\des{2}$. Then in the notation of \Cref{lem:comparison schreier} we have that
    \begin{align*}
        B(\Gamma)=&\max_{g\in \mathcal{G},a\in[n-2]}\cbra{\abs{\mathcal{G}}(n-2)\Ex_{\mathbf{g}\in \mathfrak{S}_{8},\mathbf{a}\in[n-2]}\sbra{N\pbra{g^{\{a,a+1,a+2\}},\Gamma(\mathbf{g}^{\{\mathbf a,\mathbf a+1,\mathbf a+2\}})}\abs{\Gamma(\mathbf{g}^{\{\mathbf a,\mathbf a+1,\mathbf a+2\}})}}}\\
        \leq &\max_{g\in \mathcal{G},a\in [n-2]}\cbra{8!n\Ex_{\mathbf{g}\in \mathfrak{S}_{8},\mathbf{a}\in[n-2]}\sbra{8!N\pbra{g^{\{a,a+1,a+2\}},\Gamma(\mathbf{g}^{\{\mathbf a,\mathbf a+1,\mathbf a+2\}})}}}\\
        \leq &\max_{g\in \mathcal{G},a\in [n-2]}\cbra{(8!)^3n\Pr_{\mathbf{g}\in \mathfrak{S}_{8},\mathbf{a}\in[n-2]}\sbra{g^{\{a,a+1,a+2\}}\in \Gamma(\mathbf{g}^{\{\mathbf a,\mathbf a+1,\mathbf a+2\}})}}\\
        \leq &\max_{g\in \mathcal{G},a\in [n-2]}\cbra{(8!)^3n\Pr_{\mathbf{g}\in \mathfrak{S}_{8},\mathbf{a}\in[n-2]}\sbra{a=\mathbf{a}}}\\
        \leq &(8!)^3.
    \end{align*}
    Applying \Cref{lem:comparison schreier} completes the proof.
\end{proof}


This shows that the random walk given by applying random gates on 3 bits of type $\des{2}$ has spectral gap $\wt{\Omega}\pbra{1/nk}$, by combining with \Cref{cor:initial spectral gap local}. A similar proof essentially shows the same result for the random circuit models where gates on arbitrary sets of 3 bits.



\subsection{Brickwork Circuits}\label{sec:brickwork}
The spectral gap for brickwork circuits follows almost directly from the spectral gap for circuits with nearest-neighbor gates (\Cref{cor:initial spectral gap local}), as in~\cite{brandao2016local}. First we show that the random walk induced by 3-bit nearest neighbor $\des{2}$ gates, where the 3 bits on which gates act on are of the form $\{a,a+1,a+2\}$ for any $a\in[n-2]$, has approximately the same spectral gap as that in which the random gates are of the form $\{a,a+1,a+2\}$ for $a\in[n-2]$ but with the restriction that $a\neq 0$ mod 3. Use the notation $L_{n,[n-2],k}^{3\text{-NN},\des{2}}$ and $L_{n,\{a\in[n-2],a=1,2\text{ mod } 3\},k}^{3\text{-NN},\des{2}}$ for the Laplacians of these random walks. Assume that $n=0$ mod 3; the other cases follow similarly. 

\begin{lemma}\label{lem:012 to 01}
    For any $n,k$ we have
    \begin{align*}
       \lambda_2\pbra{L_{n,\{a\in[n-2],a=1,2\text{ mod } 3\},k}^{3\text{-NN},\des{2}}}\geq  \Omega\pbra{\lambda_2\pbra{L_{n,[n-2],k}^{3\text{-NN}}}}.
    \end{align*}
\end{lemma}
\begin{proof}
    By \Cref{lem:any gate set comparison}, it suffices to show 
    \begin{align*}
       \lambda_2\pbra{L_{n,\{a\in[n-2],a=1,2\text{ mod } 3\},k}^{3\text{-NN},\des{2}}}\geq  \Omega\pbra{\lambda_2\pbra{L_{n,[n-2],k}^{3\text{-NN},\des{2}}}}.
    \end{align*}
    We use the comparison method. Again we focus on the connected component $\{(X^1,\dots,X^{k}):X^i\neq X^j\iff i\neq j\}$. For each $g\in\mathfrak{S}_{\{0,1\}^3}\cong \mathfrak{S}_8$ of type $\des{2}$ and $a\in[n-2]$ we provide a sequence $\Gamma(g^{\{a,a+1,a+2\}})$ of permutations multiplying to $g^{\{a,a+1,a+2\}}$ using only permutations of the form $h^{\{b,b+1,b+2\}}$ with $b\neq 0$ mod 3 such that the resulting congestion $B(\Gamma)$ is small. 

    We define $\Gamma$ as follows. Fix $g\in\mathfrak{S}_8,a\in[n-2]$ where $g$ is of type $\des{2}$. If $a\neq 0$ mod 3 then simply set $\Gamma(g^{\{a,a+1,a+2\}})=(g^{\{a,a+1,a+2\}})$. Otherwise $a=0$ mod 3. Then there exists a sequence of 64! permutations of the form $g_i^{\{b_i,b_i+1,b_i+2\}}$ with each $b_i\in \{a-1,a+1\}$ such that $g=g_1^{\{b_1,b_1+1,b_{1}+2\}}\cdots g_1^{\{b_{64!},\dots,b_{64!}+1,\dots,b_{64!}+2\}}$. This is because we can implement the gate $g^{\{a,a+1,a+2\}}$ as 
    \begin{align*}
        g^{\{a,a+1,a+2\}}=\mathsf{Sort}^{-1} \cdot g^{\{a-1,a,a+1\}}\cdot\mathsf{Sort},
    \end{align*}
    where $\mathsf{Sort}$ sends $(x_1,\dots,x_{a-1},x_a,x_{a+1},x_{a+2},\dots,x_n)\to (x_1,\dots,x_a,x_{a+1},x_{a+2},x_{a-1},\dots,x_n)$. The permutations $\mathsf{Sort}$ and $\mathsf{Sort}^{-1}$ can each be implemented as the product of at most $32!$ permutations of the form $h^{\{a-1,a,a+1\}}$ and $h^{\{a+1,a+2,a+3\}}$ where each $h$ is of type $\des{2}$. This gives the implementation of $g^{\{a,a+1,a+2\}}$ as the product of at most $64!$ elements of the form $g^{\{a-1,a,a+1\}}$ or $g^{\{a,a+1,a+2\}}$, and this defines $\Gamma(g^{\{a,a+1,a+2\}})$ for such $g$ and $a$. 
    
    In the notation of \Cref{lem:comparison schreier}, the congestion of $\Gamma$ is bounded:
    \begin{align*}
        B(\Gamma)\leq &\max_{g\in \mathcal{G},a=1,2\text{ mod 3}}\cbra{\frac{8!\cdot2n}{3}\Ex_{\mathbf{g}\in \mathfrak{S}_{8},\mathbf{a}\in[n-2]}\sbra{N\pbra{g^{\{a,a+1,a+2\}},\Gamma(\mathbf{g}^{\{\mathbf a,\mathbf a+1,\mathbf a+2\}})}\abs{\Gamma(\mathbf{g}^{\{\mathbf a,\mathbf a+1,\mathbf a+2\}})}}}\\
        \leq & 70!\max_{g\in \mathcal{G},a=1,2\text{ mod 3}}\cbra{n\Ex_{\mathbf{g}\in \mathfrak{S}_{8},\mathbf{a}\in[n-2]}\sbra{N\pbra{g^{\{a,a+1,a+2\}},\Gamma(\mathbf{g}^{\{\mathbf a,\mathbf a+1,\mathbf a+2\}})}}}\\
        \leq &70!\max_{g\in \mathcal{G},a=1,2\text{ mod 3}}\cbra{n\Pr_{\mathbf{g}\in \mathfrak{S}_{8},\mathbf{a}\in[n-2]}\sbra{g^{\{a,a+1,a+2\}}\in \Gamma(\mathbf{g}^{\{\mathbf a,\mathbf a+1,\mathbf a+2\}})}}\\
        \leq &70!\max_{g\in \mathcal{G},a=1,2\text{ mod 3}}\cbra{n\Pr_{\mathbf{g}\in \mathfrak{S}_{8},\mathbf{a}\in[n-2]}\sbra{a\in\{\mathbf{a}-1,\mathbf{a},\mathbf{a}+1\}}}\\
        \leq & 71!.
    \end{align*}
    Here we used that $\abs{\Gamma(g^{\{a,a+1,a+2\}})}\leq 64!$ always. Applying \Cref{lem:comparison schreier} completes the proof.
\end{proof}


Given this restriction to $\des{2}$ gates, the idea to prove the spectral gap for the random walk corresponding to a random brickwork circuit is to write the transition operator corresponding to one layer of brickwork gates as
\begin{align*}
    &R_{n,k}^{\text{brickwork},\des{2}}\\
    =&R_{n,\{1,2,3\},k}^{\des{2}}R_{n,\{4,5,6\},k}^{\des{2}}\cdots R_{n,\{n-2,n-1,n\},k}^{\des{2}}R_{n,\{2,3,4\},k}^{\des{2}}R_{n,\{5,6,7\},k}^{\des{2}}\cdots R_{n,\{n-4,n-3,n-2\},k}^{\des{2}}.
\end{align*}
Here the operator $R_{n,S,k}^{\des{2}}$ is the transition operator for the Markov chain that is similar to $R_{n,S,k}$ but with the restriction that each step is induced by a gate of type $\des{2}$. Then we note that each individual factor in each of the two products all commute, and every factor commutes with all but at most 7 other factors. The detectability lemma~\cite{aharonov2009detectability} bounds the spectral gap of this operator.


\begin{lemma}[\cite{brandao2016local}, Section 4.A]\label{lem:local to brickwork}
    For any $n,k$ we have
    \begin{align*}
        \lambda_2\pbra{L_{n,k}^{\mathrm{brickwork},\des{2}}}\geq  n\Omega\pbra{\lambda_2\pbra{L_{n,\{a\in[n-2],a=1,2\text{ mod } 3\},k}^{3\text{-NN},\des{2}}}}.
    \end{align*}
\end{lemma}

\begin{corollary}\label{cor:initial spectral gap brickwork}
    For any $n,k$ we have
    \begin{align*}
        \lambda_2\pbra{L_{n,k}^{\mathrm{brickwork},\des{2}}}\geq  \Omega\pbra{\frac1{k\cdot\polylog(k)}}.
    \end{align*}
\end{corollary}

As in the case for fully random gates and nearest-neighbor random gates, \Cref{thm:one-layer-brickwork} follows from \Cref{cor:initial spectral gap brickwork}.

