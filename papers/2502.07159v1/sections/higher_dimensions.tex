
\section{Extension to $D$-dimensional Lattices}
\label{sec:generallattices}

\subsection{More Bit Arrays and Color Classes}

For $1\leq D' \leq D$, we regard an element $x \in \{\pm1\}^{n^{D'/D}}$ as a function $x : \sbra{n^{1/D}\,}^{\otimes D'} \to \{\pm1\}$. 
Similarly, we regard an element $X \in \{\pm1\}^{n^{D'/D}k}$ as a function $X : \sbra{n^{1/D}\,}^{\otimes D'} \times \sbra{k} \to \{\pm1\}$. 
For $X \in \{\pm1\}^{n^{D'/D}k}$, $i \in \sbra{n^{1/D}\,}$, $\tau \in \sbra{n^{1/D}\,}^{\otimes D'-1}$ and $\ell \in [k]$, we use the notation:
\begin{itemize}
    \item $X^\ell_{i, \tau} = X(i, \tau, \ell) \in \{\pm1\}$
    \item $X^\ell = X \mid_{\sbra{n^{1/D}\,}^{\otimes D'} \times \{\ell\}} \in \{\pm1\}^{n^{D'/D}}$
    \item $X_{i, \cdot} = X \mid_{\{i\} \times \sbra{n^{1/D}\,}^{\otimes D'-1} \times [k]} \in \{\pm1\}^{n^{(D'-1)/D}k}$
    \item $X_{\cdot, \tau} = X \mid_{\sbra{n^{1/D}\,} \times \{\tau\} \times [k]} \in \{\pm1\}^{n^{1/D}k}$
\end{itemize}

Our definition for coloring will remain the same, namely for $X \in \{\pm1\}^{n^{D'/D}k}$ we will say $X^\ell_{i, \cdot}$ and $X^m_{i, \cdot}$ are colored the same if they are equal, but it is worth noting that these objects are $(D'-1)$-dimensional sublattices and the underlying relations are then $n^{1/D}$ tuples of equivalence relations. Note that in the case $D = 2$ these do in fact correspond to rows. Since the number of such sublattices is $n^{1/D}$ in general, the number of color classes is at most $k^{kn^{1/D}}$.

Our partition into $B_{\mathrm{safe}}$, $B_{\mathrm{coll}}$, and $B_{=0}$ remains mostly the same but based on the generalized notion of color class defined above:
\begin{align*}
    &B_\text{safe} := \cbra{X \in \sD : \forall \ell \neq m \in [k], i \in [n^{1/D}], X^\ell_{i, \cdot} \neq X^m_{i, \cdot}},\\
    &B_\text{coll} := \sD_{n^{D'/D}} \setminus B_\text{safe},\\
    &B_{=0} := \{\pm1\}^{n^{D'/D}k} \setminus \sD_{n^{D'/D}}.
\end{align*}
Throughout this section the value of $D'$ will be clear from context.


\begin{fact}
\label{fact:gencolorclasssizes}
    $\frac{\abs{B_{\mathrm{coll}}}}{\abs{\sD_{n^{D'/D}}}} \leq \frac{2n^{1/D}k^2}{2^{n^{(D'-1)/D}}}$.
\end{fact}


\begin{proof}
We write:
\begin{equation*}
    \frac{\abs{B_{\mathrm{coll}}}}{\abs{\sD_{n^{D'/D}}}} = \frac{\abs{B_{\mathrm{coll}}}}{\abs{\{\pm1\}^{n^{D'/D}k}}} \cdot \frac{\abs{\{\pm1\}^{n^{D'/D}k}}}{\abs{\sD_{n^{D'/D}}}}.
\end{equation*}
The process of sampling from $\{\pm1\}^{n^{D'/D}k}$ can now be seen as sampling $n^{1/D}k$ sublattices from $\{\pm1\}^{n^{(D'-1)/D}}$. Under this view, a simple union bound tells us that there are at most $n^{1/D}k^2$ possible collisions, allowing us to bound the probability by $\frac{n^{1/D}k^2}{2^{n^{(D'-1)/D}}}$. Again, this bounds the size of $\abs{B_{=0}}$ as well, allowing us to crudely claim $\frac{\abs{\sD_{n^{D'/D}}}}{\abs{\{\pm1\}^{n^{D'/D}k}}} \geq \frac{1}{2}$ using our assumption on $k$.
\end{proof}

\subsection{Inductively Defined Random Permutations}\label{sec:defs higher dim}
Fix $n$, $k$, and $D\geq 2$. Let $\mathcal{P}_1$ be a random permutation of $\{\pm1\}^{n^{1/D}}$. We will inductively define for all $2 \leq D'\leq D$ a random permutation $\mcP_{D'}$ on $\{\pm1\}^{n^{D'/D}}$.
\begin{itemize}
    \item Let $\mcP_{D'-1}$ be a distribution on $\mfS{\{\pm1\}^{n^{(D'-1)/D}}}$. 
    \item Let $\mcP_C$ be a distribution on $\mfS{\{\pm1\}^{n^{D'/D}}}$ such that $\pi \sim \mcP_C$ is sampled as follows: 
          Sample $\sigma_\tau \sim \mcP_1$ independently for each $\tau \in \sbra{n^{1/D}\,}^{\otimes D'-1}$ and define $\pi$ such that $\pi(x)_{\cdot, \tau} = \sigma_\tau(x_{\cdot, \tau})$ for all $x \in \{\pm1\}^{n^{D'/D}}$ and all $\tau \in \sbra{n^{1/D}}^{\otimes D'-1}$. 
    \item Let $\mcP_{L,D'-1}$ be a distribution on $\mfS{\{\pm1\}^{n^{D'/D}}}$ such that $\pi \sim \mcP_{L, D'-1}$ is sampled as follows: 
          Sample $\sigma_i \sim \mcP_{D'-1}$ independently for each $i \in \sbra{n^{1/D}}$ and define $\pi$ such that $\pi(x)_{i, \cdot} = \sigma_i(x_{i, \cdot})$ for all $x \in \{\pm1\}^{n^{D'/D}}$ and all $i \in \sbra{n^{1/D'}}$. 
    \item Let $\mcP^0_{D'} = \mcP_{L,D'-1}$. For all $s \ge 1$, let $\mcP_{D'}^{s}$ be the distribution on $\mfS{\{\pm1\}^{n^{D'/D}}}$ such that $\pi \sim \mcP^{s}_{D'}$ is sampled as follows:
          Sample $\sigma_1 \sim \mcP_{D'}^{s-1}$, $\sigma_2 \sim \mcP_{C}$, and $\sigma_3 \sim \mcP_{L,D'-1}$ and define $\pi=\sigma_3 \circ \sigma_2 \circ \sigma_1$.
    \item Set $\mcP_{D'} = \mcP_{D'}^t$, where $t$ is the constant from \Cref{lem:genreduction} below if $D'\geq 3$. Otherwise if $D'=2$ then set $t=\Theta(k\log k)$, where the constant is chosen from the statement of \Cref{thm:2D to 1D reduction technical}.
\end{itemize}
For ease of analyzing the above random permutations, we define the idealized versions of the above distributions based on the following pieces.
\begin{itemize}
    \item Let $\mcG_C$ be a distribution on $\mfS{\{\pm1\}^{n^{D'/D}}}$ such that $\pi \sim \mcG_C$ is sampled as follows: 
          Sample $\sigma_\tau \sim \mcU\pbra{\mfS{\{\pm1\}^{n^{1/D}}}}$ independently for each $\tau \in \sbra{n^{1/D}\,}^{\otimes D'-1}$ and define $\pi$ such that $\pi(x)_{\cdot, \tau} = \sigma_\tau(x_{\cdot, \tau})$ for all $x \in \{\pm1\}^{n}$ and all $\tau \in \sbra{n^{1/D}\,}^{\otimes D'-1}$. 
    \item Let $\mcG_{L, D'-1}$ be a distribution on $\mfS{\{\pm1\}^{n^{D'/D}}}$ such that $\pi \sim \mcG_{L, D'-1}$ is sampled as follows: 
          Sample $\sigma_i \sim \mcU\pbra{\mfS{\{\pm1\}^{n^{(D'-1)/D}}}}$ independently for each $i \in \sbra{n^{1/D}\,}$ and define $\pi$ such that $\pi(x)_{i, \cdot} = \sigma_i(x_{i, \cdot})$ for all $x \in \{\pm1\}^{n}$ and all $i \in \sbra{n^{1/D}\,}$. 
\end{itemize}

\subsection{Generalization of Main Theorem}

Our proof will largely follow the blueprint of the $D = 2$ case, our main result. For $X\in\sD$,
\begin{equation*}
    d_{\textrm{TV}}\pbra{\mcP^t_{D, X}, \mcG_X} \leq d_{\textrm{TV}}\pbra{\mcP^t_{D,X}, \mcG^t_{D,X}} + d_{\textrm{TV}}\pbra{\mcG^t_{D, X}, \mcG_{D, X}}.
\end{equation*}

We prove analogues of \Cref{lem:reduction} and \Cref{lem:maintrick}. 
\begin{lemma}
    \label{lem:genreduction}
    Assume the hypotheses of \Cref{thm:genresult}. Fix any $D'\geq 3$. Suppose that $\mathcal{P}_{D'-1}$ is a $\frac{1}{(4(t+1)n)^{D-D'+1}} \cdot \frac1{2^{n^{1/D}}}$-approximate $k$-wise independent permutation of $\{\pm1\}^{n^{(D'-1)/D}}$ and $\mathcal{P}_1$ is a $\frac1{(4(t+1)n)^{D}}\cdot\frac1{2^{n^{1/D}}}$-approximate $k$-wise independent permutation of $\{\pm1\}^{n^{1/D}}$. Then with the above definitions, for any $X \in \{\pm1\}^{n^{D'/D}k}$,
    \begin{equation*}
        \sum_{Y \in \{\pm1\}^{n^{D'/D}k}} \abs{\ip{e_X}{(T_{\mcP_{D'}^t}-T_{\mcG^t_{D'}}) e_Y}} \leq  \frac12\cdot \frac1{(4(t+1)n)^{D-D'}}\cdot\frac1{2^{n^{1/D}}}.
    \end{equation*}
\end{lemma}

\begin{lemma}
    \label{lem:genmaintrick}
    Assume that $k\log k\leq n^{1/3}$, that $n$ is large enough, and fix $D$. Then for all $t \geq 2500$, any $3\leq D'\leq D$, and any $X \in \{\pm1\}^{n^{D'/D}k}$,
    \begin{equation*}
        \sum_{Y \in \{\pm1\}^{n^{D'/D}k}} \abs{\ip{e_X}{(T_{\mcG^t_{D'}}-T_{\mcG_{D'}}) e_Y}} \leq \frac{1}{(4(t+1)n)^{D-D'+1}}\cdot\frac1{2^{n^{1/D}}}.
    \end{equation*}
\end{lemma}

We apply these two lemmas along with \Cref{tvdistancetolinearform} to obtain the generalization of our main result to higher-dimensional lattices. 

\begin{proof}[Proof of \Cref{thm:genresult}]
    Fix $D$ and set $t\geq 2500$ as in \Cref{lem:genmaintrick}. Let $\mathcal{P}_1$ be a $\frac1{(4(t+1)n)^D}\cdot \frac1{2^{n^{1/D}}}$-approximate $k$-wise independent permutation of $\{\pm1\}^{n^{1/D}}$. Let $\mathcal{P}_{D'}$ be constructed from $\mathcal{P}_1$ as in \Cref{sec:defs higher dim} for all $2\leq D'\leq D$.

    We prove by induction on $D'$ that for all $D'\leq D$, the random permutation $\mathcal{P}_{D'}$ is a $\frac1{(4(t+1)n)^{D-D'}}\cdot\frac1{2^{n^{1/D}}}$-approximate $k$-wise independent permutation of $\{\pm1\}^{n^{D'/D}}$. In the base case $D'=1$, this follows by assumption on $\mcP_1$. In the other base case $D'=2$, this follows from \Cref{thm:2D to 1D reduction technical}. 

    Now fix $3\leq D'\leq D$. Because $k\log k\leq n^{1/3}$ so that the hypothesis of \Cref{lem:genmaintrick} is satisfied. Assume that $\mathcal{P}_{D'-1}$ is a $\frac1{(4(t+1)n)^{D-D'+1}}\cdot\frac1{2^{n^{1/D}}}$-approximate $k$-wise independent permutation of $\{\pm1\}^{n^{(D'-1)/D}}$. By \Cref{lem:genreduction} and \Cref{lem:genmaintrick}, we have that $\mathcal{P}_{D'}^t$ is a $\frac1{(4(t+1)n)^{D-D'}}\cdot\frac1{2^{n^{1/D}}}$-approximate $k$-wise independent permutation of $\{\pm1\}^{n^{D'/D}}$:
    \begin{align*}
        d_{\textrm{TV}}\pbra{\mcP^t_{D', X}, \mcG_X} &\leq d_{\textrm{TV}}\pbra{\mcP^t_{D',X}, \mcG^t_{D',X}} + d_{\textrm{TV}}\pbra{\mcG^t_{D', X}, \mcG_{D',X}}\\
        &\leq \frac12\cdot\frac1{(4(t+1)n)^{D-D'}}\cdot\frac1{2^{n^{1/D}}}+\frac{1}{(4(t+1)n)^{D-D'+1}}\cdot\frac1{2^{n^{1/D}}}\\
        &\leq \frac12\cdot\frac1{(4(t+1)n)^{D-D'}}\cdot\frac1{2^{n^{1/D}}}+\frac12\cdot\frac{1}{(4(t+1)n)^{D-D'}}\cdot\frac1{2^{n^{1/D}}}\\
        &\leq\frac1{(4(t+1)n)^{D-D'}}\cdot\frac1{2^{n^{1/D}}}.
    \end{align*}
    This completes the induction on $D'$. As a result of the induction, we find that $\mathcal{P}_D$ is a $\frac1{2^{n^{1/D}}}$-approximate $k$-wise independent permutation of $\{\pm1\}^{n^{D/D}}=\{\pm1\}^{n}$.


    To instantiate our construction, we take $\mathcal{P}_1$ to be the depth $\widetilde{O}(k)\cdot (n^{1/D}k + n^{1/D}D\log n)=\widetilde{O}(n^{1/D}Dk^2)$ random one-dimensional brickwork circuit from \Cref{thm:1D main}. By \Cref{thm:2D main}, the random permutation $\mathcal{P}_2$ is implemented by a random two-dimensional brickwork circuit of depth $\widetilde{O}(n^{1/D}Dk^3)$. By the construction, if $\mathcal{P}_{D'-1}$ can be implemented by a random $D'-1$-dimensional brickwork circuit of depth $\leq d$ and $\mcP_1$ can be implemented by a random one-dimensional brickwork of depth $\leq d$ then $\mathcal{P}_{D'}$ can be implemented by a random $D'$-dimensional brickwork circuit of depth $d\cdot(2t+1)$. This implies that $\mathcal{P}_D$ can be implemented by a $D$-dimensional brickwork circuit of depth $(2t+1)^{D-2}\cdot \widetilde{O}(n^{1/D}Dk^3)=\mathrm{exp}(D)\cdot \widetilde{O}(n^{1/D}k^3)$. 
\end{proof}


\subsubsection{Proof of \Cref{lem:genreduction}}
Following the proof of \Cref{lem:reduction} in the $D = 2$ case, we use \Cref{fact:differenceofproducts} to bound:
\begin{align*}
     &\sum_{Y \in \{\pm1\}^{n^{D'/D}k}} \abs{\ip{e_X}{(T_{\mcP_{D'}^t}-T_{\mcG^t_{D'}}) e_Y}}\\
     &\leq (t+1) \cdot \sum_{Y \in \{\pm1\}^{n^{D'/D}k}} \abs{\ip{e_X}{(T_{\mcP_{L, D'-1}}-T_{\mcG_{L, D'-1}}) e_Y}} + t \cdot \sum_{Y \in \{\pm1\}^{nk}} \abs{\ip{e_X}{(T_{\mcP_C}-T_{\mcG_C}) e_Y}}.
\end{align*}
To bound each of the two terms, we will establish the following two lemmas.
\begin{lemma}\label{lem:genlatticereduced}
    Assume the hypothesis of \Cref{lem:genreduction}. Then,
    \begin{align*}
        \sum_{Y \in \sD} \abs{\ip{e_X}{(T_{\mcP_{L, D'-1}}-T_{\mcG_{L, D'-1}}) e_Y}} \leq n^{1/D}\cdot \frac{1}{(4(t+1)n)^{D-D'+1}} \cdot \frac1{2^{n^{1/D}}}.
    \end{align*}
\end{lemma}

\begin{lemma}
    \label{lem:genrowreduced}
    Assume the hypothesis of \Cref{lem:genreduction}. Then,
    \begin{align*}
        \sum_{Y \in \sD} \abs{\ip{e_X}{(T_{\mcP_C}-T_{\mcG_C}) e_Y}} \leq n^{(D'-1)/D}\cdot \frac1{(4(t+1)n)^D}\cdot \frac1{2^{n^{1/D}}}.
    \end{align*}
\end{lemma}
Plugging directly into the equation above finishes the proof of \Cref{lem:genreduction}.
\begin{align*}
     &\sum_{Y \in \{\pm1\}^{n^{D'/D}k}} \abs{\ip{e_X}{(T_{\mcP_{D'}^t}-T_{\mcG^t_{D'}}) e_Y}}\\
     &\leq(t+1)\cdot n^{1/D}\cdot \frac{1}{(4(t+1)n)^{D-D'+1}}\cdot \frac1{2^{n^{1/D}}}+ t\cdot n^{(D'-1)/D}\cdot \frac1{(4(t+1)n)^D}\cdot \frac1{2^{n^{1/D}}}\\
     &\leq \frac14\cdot\frac1{(4(t+1)n)^{D-D'}}\cdot\frac1{2^{n^{1/D}}}+\frac14\cdot\frac1{(4(t+1)n)^{D-1}}\cdot\frac1{2^{n^{1/D}}}\\
     &\leq \frac12\cdot\frac1{(4(t+1)n)^{D-D'}}\cdot\frac1{2^{n^{1/D}}}.
\end{align*}
Note that we used the definitions of $\mcP_{D'}^t$ and $\mcG_{D'}^t$ from \Cref{sec:defs higher dim}. This concludes the proof of \Cref{lem:genreduction}.

\begin{proof}[Proof of \Cref{lem:genlatticereduced}]
Recall that $X, Y \in \{\pm1\}^{n^{D'/D}k}$ and we write $X_{i, \cdot}$ for $i \in \sbra{n^{1/D}}$ to denote one of $n^{1/D}$ $(D'-1)$-dimensional slices. The operator $T_{\mcP_{L, D'-1}}$ can be seen as a $n^{1/D}$-wise tensorization of $T_{\mcP_{D-1}}$ acting individually on each slice. As such, we compute:
\begin{align*}
    &\sum_{Y \in \{\pm1\}^{n^{D'/D}k}} \abs{\ip{e_X}{(T_{\mcP_{L, D'-1}}-T_{\mcG_{L, D'-1}}) e_Y}}\\
    &=\sum_Y \abs{ \prod_{i=1}^{n^{1/D}}\Pr[X_{i, \cdot} \to_{T_{\mcP_{D'-1}}} Y_{i ,\cdot}] - \prod_{i=1}^{n^{1/D}}\Pr[X_{i, \cdot} \to_{T_{\mcG_{n^{D'/D}}}} Y_{i ,\cdot}]}\\
    &=\sum_Y \abs{\sum_{j = 1}^{n^{1/D}} \prod_{i=1}^{j-1}\Pr[X_{i, \cdot} \to_{T_{\mcP_{D'-1}}} Y_{i ,\cdot}] \pbra{\Pr[X_{j, \cdot} \to_{T_{\mcP_{D'-1}}} Y_{j ,\cdot}] - \Pr[X_{j, \cdot} \to_{T_{\mcG_{n^{D'/D}}}} Y_{j ,\cdot}]} \prod_{i=j+1}^{n^{1/D}}\Pr[X_{i, \cdot} \to_{T_{\mcG_{n^{D'/D}}}} Y_{i ,\cdot}]}\\
    &\leq  \sum_{j = 1}^{n^{1/D}} \sum_Y \abs{\prod_{i=1}^{j-1}\Pr[X_{i, \cdot} \to_{T_{\mcP_{D'-1}}} Y_{i ,\cdot}] \pbra{\Pr[X_{j, \cdot} \to_{T_{\mcP_{D'-1}}} Y_{j ,\cdot}] - \Pr[X_{j, \cdot} \to_{T_{\mcG_{n^{D'/D}}}} Y_{j ,\cdot}]} \prod_{i=j+1}^{n^{1/D}}\Pr[X_{i, \cdot} \to_{T_{\mcG_{n^{D'/D}}}} Y_{i ,\cdot}]}\\
    &= \sum_{j = 1}^{n^{1/D}} \sum_y \abs{\Pr[X_{j, \cdot} \to_{T_{\mcP_{D'-1}}} y] - \Pr[X_{j, \cdot} \to_{T_{\mcG_{n^{D'/D}}}} y]} \sum_{\substack{Y\\ Y_{j, \cdot} = y}} \prod_{i=1}^{j-1}\Pr[X_{i, \cdot} \to_{T_{\mcP_{D'-1}}} Y_{i ,\cdot}] \prod_{i=j+1}^{n^{1/D}}\Pr[X_{i, \cdot} \to_{T_{\mcG_{n^{D'/D}}}} Y_{i ,\cdot}]\\
    &\leq \sum_{j = 1}^{n^{1/D}} \sum_{y \in \{\pm1\}^{n^{(D'-1)/D}k}} \abs{\Pr[X_{j, \cdot} \to_{T_{\mcP_{D'-1}}} y] - \Pr[X_{j, \cdot} \to_{T_{\mcG_{n^{D'/D}}}} y]}\\
    &\leq n^{1/D}\cdot \frac{1}{(4(t+1)n)^{D-D'+1}} \cdot \frac1{2^{n^{1/D}}}.
\end{align*}
The last line follows from \Cref{genkwiseimplies}.
\end{proof}


\begin{lemma}
    \label{genkwiseimplies}
    For every $x \in \{\pm1\}^{n^{(D'-1)/D}k}$ we have:
    \begin{equation*}
        \sum_{y \in \{\pm1\}^{n^{(D'-1)/D}k}} \abs{\Pr[x \to_{T_{\mcP_{D'-1}}} y] - \Pr[x \to_{T_{\mcG_{n^{(D'-1)/D}}}} y]} \leq \frac{1}{(4(t+1)n)^{D-D'+1}} \cdot \frac1{2^{n^{1/D}}}.
    \end{equation*}
\end{lemma}

\begin{proof}[Proof of \Cref{genkwiseimplies}]
We view $x$ as a $k$-tuple of $(D'-1)$-dimensional grids. We denote by $B$ the ``tuple-wise'' color class of $x$ (if two grids are equal they are colored the same). We create a projection function $\varphi_B$ defined analogously to that in \Cref{kwiseimplies}, taking $x$ to a corresponding $\tau$-tuple with distinct elements.
    \begin{align*}
    &\sum_{y \in \{\pm1\}^{n^{(D'-1)/D}k}} \abs{\Pr[x \to_{T_{\mcP_{D'-1}}} y] - \Pr[x \to_{T_{\mcG_{n^{(D'-1)/D}}}} y]}\\
    =&\sum_{y \in B(x)} \abs{\Pr[x \to_{T_{\mcP_{D'-1}}} y] - \Pr[x \to_{T_{\mcG_{n^{(D'-1)/D}}}} y]}\\
    =& \sum_{y \in B(x)} \abs{\Pr[\varphi_B(x) \to_{T_{\mcP_{D'-1}}} \varphi_B(y)] - \Pr[\varphi_B(x) \to_{T_{\mcG_{n^{(D'-1)/D}}}} \varphi_B(y)]}\\
    =& \sum_{\varphi_B(y) \in \sD_{n^{(D'-1)/D}}^{(\tau)}} \abs{\Pr[\varphi_B(x) \to_{T_{\mcP_{D'-1}}} \varphi_B(y)] - \Pr[\varphi_B(x) \to_{T_{\mcG_{n^{(D'-1)/D}}}} \varphi_B(y)]}\\
    =& \sum_{\varphi_B(y) \in \sD_{n^{(D'-1)/D}}^{(\tau)}} \abs{\sum_{z \in \sD_{n^{(D'-1)/D}}^{(k-\tau)}}\Pr[(\varphi_B(x), \cdot) \to_{T_{\mcP_{D'-1}}} (\varphi_B(y), z)] - \Pr[(\varphi_B(x), \cdot) \to_{T_{\mcG_{n^{(D'-1)/D}}}} (\varphi_B(y), z)]}\\
    \leq& \sum_{\substack{\varphi_B(y) \in \sD_{n^{(D'-1)/D}}^{(\tau)} \\ z \in \sD_{n^{(D'-1)/D}}^{(k-\tau)}}} \abs{\Pr[(\varphi_B(x), \cdot) \to_{T_{\mcP_{D'-1}}} (\varphi_B(y), z)] - \Pr[(\varphi_B(x), \cdot) \to_{T_{\mcG_{n^{(D'-1)/D}}}} (\varphi_B(y), z)]}\\
    =& \sum_{y \in \sD^{(k)}_{n^{(D'-1)/D}}} \abs{\Pr[(\varphi_B(x), \cdot) \to_{T_{\mcP_{D'-1}}} (\varphi_B(y), y_{[k] \setminus T})] - \Pr[(\varphi_B(x), \cdot) \to_{T_{\mcG_{n^{(D'-1)/D}}}} (\varphi_B(y), y_{[k] \setminus T})]}.
\end{align*}
The last step assumes $(\varphi_B(x), \cdot) \in \sD_{n^{(D'-1)/D}}$, that is, it is a distinct $k$-tuple. We then appeal to the fact that ${\mcP_{D'-1}}$ is assumed to be $\frac{1}{(4(t+1)n)^{D-D'+1}} \cdot \frac1{2^{n^{1/D}}}$-approximate $k$-wise independent to finish.
\end{proof}

The proof of \Cref{lem:genrowreduced} is nearly identical to that of \Cref{lem:genlatticereduced}, but partitioning $\{\pm1\}^{n^{D'/D}}$ over one-dimensional columns yields a tensor product of order $n^{(D'-1)/D}$, which becomes a factor in the result, and additionally we appeal to the error in $\mcP_1$ at the end.

\subsubsection{Proof of \Cref{lem:genmaintrick}}
\label{sec:genmaintrick}

This proof follows near identically to \Cref{subsec:inductiontrick}. Throughout this section we assume the hypothesis of \Cref{lem:genmaintrick}, namely that $k\log k\leq n^{1/3}$. It suffices to prove for any $X \in \sD_{n^{D'/D}}$ via \Cref{lem:genoffdiagonalmoment}:
\begin{equation*}
    \sum_{Y \in \sD_{n^{D'/D}}} \abs{\ip{e_X}{(T_{\mcG^t_{D'}}-T_{\mcG_{D'}}) e_Y}} \leq \frac{1}{(4(t+1)n)^{D-D'+1}}\cdot\frac1{2^{n^{1/D}}}.
\end{equation*}

For clarity, we will assume all operators and distributions from this point on are implicitly parameterized by $D'$ and drop the subscript.

\begin{lemma}
\label[lemma]{lem:genoffdiagonalmoment}
    Assume the hypotheses of \Cref{lem:genmaintrick}. Then $\abs{\ip{e_X}{(T_{\mcG^t}-T_{\mcG}) e_Y}} \leq \frac{t+1}{2^{n^{(D'-1)/D}(t-1)/128}} \cdot \frac{1}{\abs{B(Y)}}$.
\end{lemma}
The lemma is used in the following calculation:
\begin{equation*}
    \sum_{Y \in \sD_{n^{D'/D}}} \abs{\ip{e_X}{(T_{\mcG^t}-T_{\mcG}) e_Y}} \leq \frac{t+1}{2^{n^{(D'-1)/D}(t-1)/128}} \sum_{Y \in \sD_{n^{(D'-1)/D}}} \frac{1}{\abs{B(Y)}} \leq \frac{k^{kn^{1/D}} \cdot (t+1)}{2^{n^{(D'-1)/D}(t-1)/128}}.
\end{equation*}
We use that the number of color classes is less than $k^{kn^{1/D}}$. Since $k \log k \leq n^{1/3} \leq n^{(D'-2)/D}$ for $D, D' \geq 3$, we have that:
\begin{equation*}
    \sum_{Y \in \sD_{n^{D'/D}}} \abs{\ip{e_X}{(T_{\mcG^t}-T_{\mcG}) e_Y}} \leq \frac{2^{n^{(D'-1)/D}} \cdot (t+1)}{2^{n^{(D'-1)/D}(t-1)/128}} \leq \frac{1}{2^{(n^{(D'-1)/D}/128-1)(t-1)-1}}.
\end{equation*}

If we set $t = \frac{n^{1/D}D\log_2 (4(t+1)n)+1}{n^{(D'-1)/D}/128-1}+1$ we achieve the desired bound. Note that for large enough $n$ we have $t \leq \frac{2500 D \log_2 n}{n^{1/D}}$. Further if $D \leq \frac{1}{2} \cdot\frac{\log_2 n}{\log_2 \log_2 n}$ then we have $n^{1/D} \geq (\log_2 n)^2$, which is enough to conclude $t \leq 2500$. This concludes the proof of \Cref{lem:genmaintrick}.

\begin{proof}[Proof of \Cref{lem:genoffdiagonalmoment}.]
Recall $T_{\mcG^t} = T_{\mcG_L}(T_{\mcG_C}T_{\mcG_L})^t$ so $T_{\mcG^t} - T_\mcG = (T_{\mcG_L}T_{\mcG_C})^t(T_{\mcG_L} - T_\mcG)$. We induct on $t$. Consider first when $t = 0$.
\begin{equation*}
    \abs{\ip{e_X}{(T_{\mcG_L}-T_{\mcG}) e_Y}} = \abs{\Pr[X \to_{T_{\mcG_L}} Y] - \Pr[X \to_{T_{\mcG}} Y]} = \abs{\Pr[Y \to_{T_{\mcG_L}} X] - \Pr[Y \to_{T_{\mcG}} X]}.
\end{equation*}
Note that we guarantee inductively that $\mcG_L$ is self-adjoint. This quantity is bounded by $\frac{1}{\abs{B(Y)}}$ as before.

For the induction step, assume the lemma for $t \geq 0$ and compute
\begin{align*}
    \abs{\ip{e_X}{(T_{\mcG_L}T_{\mcG_C})^{t+1}(T_{\mcG_L}-T_{\mcG}) e_Y}} 
    =& \abs{\ip{e_X}{(T_{\mcG_L}T_{\mcG_C})(T_{\mcG_L}T_{\mcG_C})^t(T_{\mcG_L}-T_{\mcG}) e_Y}}\\
    =& \abs{T_{\mcG_L}\pbra{T_{\mcG_C}(T_{\mcG_L}T_{\mcG_C})^t(T_{\mcG_L}-T_{\mcG}) e_Y}(X)}\\
    =& \abs{\sum_{Z \in \sD} \Pr[X \to_{T_{\mcG_C}T_{\mcG_L}} Z] \ip{e_Z}{(T_{\mcG_L}T_{\mcG_C})^t(T_{\mcG_L}-T_{\mcG}) e_Y}}\\
    \leq &\abs{\sum_{Z \in B_\text{safe}} \Pr[X \to_{T_{\mcG_L}T_{\mcG_C}} Z] \ip{e_Z}{(T_{\mcG_L}T_{\mcG_C})^t(T_{\mcG_L}-T_{\mcG}) e_Y}}\\
    &\;\;\;\;+ \abs{\sum_{Z \in B_\text{coll}} \Pr[X \to_{T_{\mcG_C}T_{\mcG_C}} Z] \ip{e_Z}{(T_{\mcG_L}T_{\mcG_C})^t(T_{\mcG_L}-T_{\mcG}) e_Y}}\\
    \leq& \max_{Z \in B_{\mathrm{safe}}} \abs{\ip{e_Z}{(T_{\mcG_L}T_{\mcG_C})^t(T_{\mcG_L}-T_{\mcG}) e_Y}}\\
    &\;\;\;\;+ \frac{1}{2^{n^{(D'-1)/D}/128}} \cdot \max_{Z \in B_{\mathrm{coll}}} \abs{\ip{e_Z}{(T_{\mcG_L}T_{\mcG_C})^t(T_{\mcG_L}-T_{\mcG}) e_Y}}\\
    \leq& \max_{Z \in B_{\mathrm{safe}}} \abs{\ip{e_Z}{(T_{\mcG_L}T_{\mcG_C})^t(T_{\mcG_L}-T_{\mcG}) e_Y}} + \frac{t+1}{2^{n^{(D'-1)/D}t/128}} \cdot \frac{1}{\abs{B(Y)}}.
\end{align*}
We apply the induction in the last line and \Cref{lem:genlowcollprob} as stated below in the previous line, in order to bound the probability $X$ lands in the collision region.

\begin{lemma}\label{lem:genspectralnorm}
    Assume the hypotheses of \Cref{lem:genmaintrick}. Then $\norm{T_{\mcG_L}T_{\mcG_C}T_{\mcG_L}-T_{\mcG}}_{2} \leq \frac1{2^{n^{(D'-1)/D}/128}}$.
\end{lemma}

\begin{lemma}
\label{lem:genlowcollprob}
    Assume the hypotheses of \Cref{lem:genmaintrick}. Then for all $X \in \sD$, $\Pr[X \to_{T_{\mcG_C}T_{\mcG_L}} B_\text{coll}] \leq \frac1{2^{n^{(D'-1)/D}/128}}$.
\end{lemma}
We prove these two lemmas in \Cref{sec:genspectral}. The use of \Cref{lem:genlowcollprob} is in bounding the latter term above. To use \Cref{lem:genspectralnorm} we write for $Z \in B_{\mathrm{safe}}$:
\begin{align*}
    \abs{\ip{e_Z}{(T_{\mcG_L}T_{\mcG_C})^t(T_{\mcG_L}-T_{\mcG}) e_Y}} =& \abs{\ip{T_{\mcG_L}e_Z}{(T_{\mcG_L}T_{\mcG_C}T_{\mcG_L}-T_{\mcG})^t T_{\mcG_L} e_Y}} \\
    \leq& \norm{T_{\mcG_L}T_{\mcG_C}T_{\mcG_L}-T_{\mcG}}_{2}^t \norm{T_{\mcG_L}e_Z}_{2} \norm{T_{\mcG_L}e_Y}_{2}\\
    \leq& \frac{1}{2^{n^{(D'-1)/D}t/128}} \cdot \frac{1}{\abs{B(Y)}^{1/2}\abs{B_\text{safe}}^{1/2}}\\
    \leq& \frac{1}{2^{n^{(D'-1)/D}t/128}} \cdot \frac{1}{\abs{B(Y)}}.
\end{align*}
The first step uses the self-adjointness of $T_{\mcG_C}$, the fact that $T_{\mcG_C}^2 = T_{\mcG_C}$, and \Cref{fact:TGabsorbs}. The inequality is an application of Cauchy-Schwarz and submultiplicativity of the operator norm. The second to last step uses \Cref{lem:genspectralnorm} and \Cref{TGL eU 2 norm} below, and the last step uses \Cref{fact:gencolorclasssizes}, namely that $B_{\mathrm{safe}}$ is larger than every other color class for our choice of $k$ and large enough $n$.

\begin{claim}\label{TGL eU 2 norm}
    For arbitrary $U \in \{\pm1\}^{nk}$:
\begin{equation*}
    \norm{T_{\mcG_L}e_U}_{2} = \frac{1}{\abs{B(U)}^{1/2}}.
\end{equation*}
\end{claim}

\begin{proof}
    Observe:
    \begin{equation*}
        \norm{T_{\mcG_L}e_U}_{2} = \sqrt{\sum_{W \in B(U)} \pbra{T_{\mcG_L}e_U(W)}^2} = \sqrt{\sum_{W \in B(U)} \Pr[W \to_{\mcG_L} U]^2} = \sqrt{\sum_{W \in B(U)} \pbra{\frac{1}{\abs{B(U)}}}^2} = \frac{1}{\abs{B(U)}^{1/2}}.\qedhere
    \end{equation*}
\end{proof}
Putting the two together then gives us:
\begin{equation*}
    \abs{\ip{e_X}{(T_{\mcG_L}T_{\mcG_C})^{t+1}(T_{\mcG_L}-T_{\mcG}) e_Y}} \leq \frac{1}{2^{n^{(D'-1)/D}t/128}} + \frac{t+1}{2^{n^{(D'-1)/D}t/128}} \leq \frac{t+2}{2^{n^{(D'-1)/D}t/128}}.
\end{equation*}
This concludes the proof of \Cref{lem:genoffdiagonalmoment}.
\end{proof}

\subsection{Proof of Spectral Properties}\label{sec:genspectral}

In this section we prove \Cref{lem:genspectralnorm} and \Cref{lem:genlowcollprob}. We will proceed by decomposing $f = f_{B_{\mathrm{safe}}} + f_{B_{\mathrm{coll}}} + f_{B_{=0}}$ where $f_B$ is supported on $B \subseteq \{\pm1\}^{n^{D'/D}k}$.
\begin{align*}
    \abs{\ip{f}{(T_{\mcG_L}T_{\mcG_C}T_{\mcG_L} - T_\mcG) f}} \leq &\abs{\ip{f_{B_{\mathrm{safe}}}}{(T_{\mcG_L}T_{\mcG_C}T_{\mcG_L} - T_\mcG) f_{\sD_{n^{D'/D}}}}}
    + \abs{\ip{f_{B_{\mathrm{coll}}}}{(T_{\mcG_L}T_{\mcG_C}T_{\mcG_L} - T_\mcG) f_{\sD_{n^{D'/D}}}}}\\
    &\;\;+ \abs{\ip{f_{B_{=0}}}{(T_{\mcG_L}T_{\mcG_C}T_{\mcG_L} - T_\mcG) f_{B_{=0}}}}.
\end{align*}

\subsubsection{$f$ Supported on $B_{\mathrm{safe}}$}

\begin{lemma}
    $\abs{\ip{f_{B_{\mathrm{safe}}}}{(T_{\mcG_L}T_{\mcG_C}T_{\mcG_L} - T_\mcG) f_{\sD}}} \leq \frac{4n^{1/D}k^2}{2^{n^{(D'-1)/D}}} \cdot \ip{f}{f}$.
\end{lemma}


\begin{proof}
Let $X \in B_{\mathrm{safe}}$, $g : \{\pm1\}^{n^{D'/D}k} \to \R$.
\begin{align*}
    (T_{\mcG_L} - T_\mcG)g(X) &= \sum_{Y \in \{\pm1\}^{n^{D'/D}k}} \Pr[X \to_{T_{\mcG_L}} Y] \cdot g(Y) - \sum_{Y \in \{\pm1\}^{n^{D'/D}k}} \Pr[X \to_{T_{\mcG}} Y] \cdot g(Y)\\
    &= \frac{1}{\abs{B_{\mathrm{safe}}}}\sum_{Y \in B_{\mathrm{safe}}} g(Y) - \frac{1}{\abs{\sD_{n^{D'/D}}}}\sum_{Y \in \sD_{n^{D'/D}}} g(Y)\\
    &= \pbra{\frac{1}{\abs{B_{\mathrm{safe}}}} - \frac{1}{\abs{\sD_{n^{D'/D}}}}}\sum_{Y \in B_{\mathrm{safe}}} g(Y) - \frac{1}{\abs{\sD_{n^{D'/D}}}}\sum_{Y \in B_{\mathrm{coll}}} g(Y)\\
    &= \pbra{1 - \frac{\abs{B_{\mathrm{safe}}}}{\abs{\sD_{n^{D'/D}}}}} \cdot \frac{1}{\abs{B_{\mathrm{safe}}}}\sum_{Y \in B_{\mathrm{safe}}} g(Y) - \frac{\abs{B_{\mathrm{coll}}}}{\abs{\sD_{n^{D'/D}}}} \cdot \frac{1}{\abs{B_{\mathrm{coll}}}}\sum_{Y \in B_{\mathrm{coll}}} g(Y)\\
    &= \frac{\abs{B_{\mathrm{coll}}}}{\abs{\sD_{n^{D'/D}}}} \pbra{T_{\mcG_L} - \mcH}g(X),
\end{align*}
where $\mcH g(X) = \frac{1}{\abs{B_{\mathrm{coll}}}}\sum_{Y \in B_{\mathrm{coll}}} g(Y)$. We write:
\begin{align*}
    \abs{\ip{f_{B_{\mathrm{safe}}}}{(T_{\mcG_L}T_{\mcG_C}T_{\mcG_L} - T_\mcG) f_{\sD_{n^{D'/D}}}}} &= \sum_{X \in \{\pm1\}^{n^{D'/D}k}} f_{B_{\mathrm{safe}}}(X) \cdot (T_{\mcG_L}-T_\mcG)(T_{\mcG_C}T_{\mcG_L}f_{\sD_{n^{D'/D}}})(X)\\
    &= \sum_{X \in B_{\mathrm{safe}}} f_{B_{\mathrm{safe}}}(X) \cdot (T_{\mcG_L}-T_\mcG)(T_{\mcG_C}T_{\mcG_L}f_{\sD_{n^{D'/D}}})(X)\\
    &= \frac{\abs{B_{\mathrm{coll}}}}{\abs{\sD_{n^{D'/D}}}} \sum_{X \in B_{\mathrm{safe}}} f_{B_{\mathrm{safe}}}(X) \cdot (T_{\mcG_L}-\mcH)(T_{\mcG_C}T_{\mcG_L}f_{\sD_{n^{D'/D}}})(X)\\
    &= \frac{\abs{B_{\mathrm{coll}}}}{\abs{\sD_{n^{D'/D}}}} \abs{\ip{f_{B_{\mathrm{safe}}}}{(T_{\mcG_L}T_{\mcG_C}T_{\mcG_L} - \mcH T_{\mcG_C}T_{\mcG_L}) f_{\sD_{n^{D'/D}}}}}.
\end{align*}
The fact that $\mcH$ is a random walk operator once again establishes:
\begin{align*}
    &\abs{\ip{f_{B_{\mathrm{safe}}}}{(T_{\mcG_L}T_{\mcG_C}T_{\mcG_L} - T_\mcG) f_{\sD_{n^{D'/D}}} }} \\
    \leq &\frac{\abs{B_{\mathrm{coll}}}}{\abs{\sD_{n^{D'/D}}}} \pbra{\abs{\ip{f_{B_{\mathrm{safe}}}}{T_{\mcG_L}T_{\mcG_C}T_{\mcG_L} f_{\sD_{n^{D'/D}}}}} + \abs{\ip{f_{B_{\mathrm{safe}}}}{\mcH T_{\mcG_C}T_{\mcG_L} f_{\sD_{n^{D'/D}}}}}}\\
    \leq& \frac{2\abs{B_{\mathrm{coll}}}}{\abs{\sD_{n^{D'/D}}}} \norm{f_{B_{\mathrm{safe}}}}_{2} \norm{f_{\sD_{n^{D'/D}}}}_{2} \tag{\Cref{lem:escape probs}}\\
    \leq& \frac{2\abs{B_{\mathrm{coll}}}}{\abs{\sD_{n^{D'/D}}}} \langle f, f \rangle.
\end{align*}
\Cref{fact:gencolorclasssizes} then suffices to prove the claim. \end{proof}

\subsubsection{$f$ Supported on $B_{\mathrm{coll}}$}

\begin{lemma}
    $\abs{\ip{f_{B_{\mathrm{coll}}}}{(T_{\mcG_L}T_{\mcG_C}T_{\mcG_L} - T_\mcG) f_{\sD_{n^{(D'-1)/D}}}}} \leq \frac{8n^{1/D}k^2}{2^{n^{D'/D}/32}} \ip{f}{f}$.
\end{lemma}


\begin{proof}
First, we can decompose $f_{\sD_{n^{D'/D}}} = f_{B_{\mathrm{safe}}} + f_{B_{\mathrm{coll}}}$ and bound:
\begin{align*}
    \abs{\ip{f_{B_{\mathrm{coll}}}}{(T_{\mcG_L}T_{\mcG_C}T_{\mcG_L} - T_\mcG) f_{\sD_{n^{D'/D}}}}} 
    &\leq \abs{\ip{f_{B_{\mathrm{coll}}}}{(T_{\mcG_L}T_{\mcG_C}T_{\mcG_L} - T_\mcG) f_{B_{\mathrm{safe}}}}} \\
    &+ \abs{\ip{f_{B_{\mathrm{coll}}}}{(T_{\mcG_L}T_{\mcG_C}T_{\mcG_L} - T_\mcG) f_{B_{\mathrm{coll}}}}}.
\end{align*}
By the self-adjointness of the operator, the first term is bounded by the case above, so it suffices to bound the latter. For this term, we can appeal directly to \Cref{lem:escape probs} and the triangle inequality to get:
\begin{align*}
    &\abs{\ip{f_{B_{\mathrm{coll}}}}{(T_{\mcG_L}T_{\mcG_C}T_{\mcG_L} - T_\mcG) f_{B_{\mathrm{coll}}}}} \\
    &\leq\sqrt{\max_{X \in B_{\mathrm{coll}}} \Pr[X \to_{T_{\mcG_L}T_{\mcG_C}T_{\mcG_L}} B_{\mathrm{coll}}] + \max_{X \in B_{\mathrm{coll}}} \Pr[X \to_{T_\mcG} B_{\mathrm{coll}}]} \ip{f_{B_{\mathrm{coll}}}}{f_{B_{\mathrm{coll}}}}.
\end{align*}
Note that regardless of choice of $X$, the latter probability $\Pr[X \to_{T_\mcG} B_{\mathrm{coll}}] = \frac{\abs{B_{\mathrm{coll}}}}{\abs{\sD_{n^{D'/D}}}} \leq \frac{2n^{1/D}k^2}{2^{n^{(D'-1)/D}}}$ by \Cref{fact:gencolorclasssizes}. We finish by proving \Cref{lem:genlowcollprob} from the previous section below.
\end{proof}
\begin{lemma}[Restatement of \Cref{lem:genlowcollprob}]
    For all $X \in \sD_{n^{D'/D}}$, $\Pr[X \to_{T_{\mcG_C}T_{\mcG_L}} B_{\mathrm{coll}}] \leq \frac{2n^{1/D}k^2}{2^{n^{(D'-1)/D}/16}}$.
\end{lemma}
\begin{proof}
    Our goal is to union bound over the probability of any pair of sublattices colliding. There are at most $n^{1/D}k^2$ pairs of sublattices. 
    
    Let $X \in \sD_{n^{D'/D}}$. We will model our process as:
    \begin{equation*}
        X \to_{T_{\mcG_L}} Y \to_{T_{\mcG_C}} Z.
    \end{equation*}
    We then fix $Z^\ell_{i, \cdot}$ and $Z^m_{i, \cdot}$ (which recall are $(D'-1)$-dimensional slices, in $\{\pm1\}^{n^{(D'-1)/D}}$) for $i \in [n^{1/D}]$, $\ell \neq m \in [k]$. We use that for some $j \in [k]$, we have $(Y^\ell_{j, \cdot}, Y^m_{j, \cdot})$ are uniform from ${\{\pm1\}^{n^{(D'-1)/D}} \choose 2}$. To see this note that there must exist some $j$ s.t. $X^\ell_{j, \cdot} \neq X^m_{j,\cdot}$, otherwise $X \notin \sD_{n^{D'/D}}$. Since the permutation applied to these two grids is uniform from $\mfS{\{\pm1\}^{n^{(D'-1)/D}}}$, the resulting rows in $Y$ look like a uniform distinct pair.
    
    With this in mind, we will now condition on the event that $d(Y^\ell_{j, \cdot}, Y^m_{j, \cdot}) \geq n^{(D'-1)/D}/4$ and compute for $n$ large enough that
    \begin{align*}
        \Pr[Z^\ell_{i, \cdot} = Z^m_{i, \cdot}] =& \Pr[Z^\ell_{i, \cdot} = Z^m_{i, \cdot} \mid d(Y^\ell_{j, \cdot}, Y^m_{j, \cdot}) > n^{(D'-1)/D}/4] \\
        +& \Pr[Z^\ell_{i, \cdot} = Z^m_{i, \cdot} \mid d(Y^\ell_{j, \cdot}, Y^m_{j, \cdot}) \leq n^{(D'-1)/D}/4]\Pr[d(Y^\ell_{j, \cdot}, Y^m_{j, \cdot}) \leq n^{(D'-1)/D}/4]\\
        \leq&\Pr[Z^\ell_{i, \cdot} = Z^m_{i, \cdot} \mid d(Y^\ell_{j, \cdot}, Y^m_{j, \cdot}) > n^{(D'-1)/D}/4] +\Pr[d(Y^\ell_{j, \cdot}, Y^m_{j, \cdot}) \leq n^{(D'-1)/D}/4]\\
        \leq&\frac{1}{2^{n^{(D'-1)/D}/16}}+\frac1{e^{n^{(D'-1)/D}/16}}\tag{\Cref{lem:given Y far}, \Cref{lem:Y probably far}}\\
        \leq&\frac{1}{2^{n^{(D'-1)/D}/32}}.
    \end{align*}
    Applying a union bound over all $n^{1/D}k^2$ pairs of sublattices completes the proof.
\end{proof}

\begin{lemma}\label{lem:given Y far}
    $\Pr[Z^\ell_{i, \cdot} = Z^m_{i, \cdot} \mid d(Y^\ell_{j, \cdot}, Y^m_{j, \cdot}) > n^{(D'-1)/D}/4] \leq \frac{1}{2^{n^{(D'-1)/D}/4}}$
\end{lemma}
\begin{proof}
    The probability that $Z^\ell_{i, \cdot}$ and $Z^m_{i, \cdot}$ are equal can be viewed as the probability that all of their individual bits are equal, and they are all independent since they come from independently sampled rows. Since $Y^\ell_{j, \cdot}$ and $Y^m_{j, \cdot}$ differ in at least $n^{(D'-1)/D}/4$ places, $Y^\ell$ and $Y^m$ must differ in at least that many rows. In these rows, it can be seen that the corresponding bits in $Z^\ell_{i, \cdot}$ and $Z^m_{i, \cdot}$ are the same with probability $\leq \frac{1}{2}$. By independence the probability is less than $\frac{1}{2^{n^{(D'-1)/D}/4}}$. 
\end{proof}

\begin{lemma}\label{lem:Y probably far}
    $\Pr[d(Y^\ell_{j, \cdot}, Y^m_{j, \cdot}) \leq n^{(D'-1)/D}/4] \leq \frac{1}{e^{n^{(D'-1)/D}/16}}$
\end{lemma}
\begin{proof}
    This can be seen by a simple Chernoff bound. Note that $\Pr[d(Y^\ell_{j, \cdot}, Y^m_{j, \cdot}) \leq n^{(D'-1)/D}/4] \leq \Pr_{x, y \sim \{\pm1\}^{n^{D'/D}}}[d(x, y) \leq n^{(D'-1)/D}/4]$, as if they are equal the distance is minimized. For uniform $x,y$, $d(x,y)$ can be seen as the sum of $n^{(D'-1)/D}$ independent Bernoulli$(1/2)$ r.v.s. By Hoeffding's Inequality:
    \begin{equation*}
        \Pr_{x,y \sim \{\pm1\}^{n^{(D'-1)/D}}}[d(x,y) \leq n^{(D'-1)/D}/4] \leq e^{-n^{(D'-1)/D}/16}.\qedhere
    \end{equation*}
\end{proof}

\subsubsection{The Induction Case}

\begin{lemma}
Let $f : \{\pm1\}^{n^{D'/D}k} \to \R$ be supported on $B_{=0}$ and $k \geq 2$. Then, we have
\[ 
\abs{\ip{f}{\pbra{T_{\mcG_L}^{(k)}T_{\mcG_C}^{(k)}T_{\mcG_L}^{(k)} - T_\mcG^{(k)}}f}} \le \norm{T_{\mcG_L}^{(k-1)}T_{\mcG_C}^{(k-1)}T_{\mcG_L}^{(k-1)} - T_{\mcG}^{(k-1)}}_{2} \ip{f}{f}. 
\]
\end{lemma}

\begin{proof}
    The proof is nearly notationally identical to \Cref{lemma:f supported on B0} as the notion of color class developed in that section is on the tuple so is not dependent on the choice of sublattice, so we will refer back for brevity.
\end{proof}