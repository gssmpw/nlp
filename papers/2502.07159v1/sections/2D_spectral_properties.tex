\section{Proof of Spectral Properties of $T_{\mcG_R}$ and $T_{\mcG_C}$}
\label{sec:spectralproof}

In this section we prove \Cref{lem:spectralnorm}, which is a spectral norm bound on the difference between two operators related to our constructions above. As an intermediate result in the proof we will show \Cref{lem:lowcollprob} as well. This will then conclude the proof of \Cref{thm:2D to 1D reduction}.

\begin{lemma}[\Cref{lem:spectralnorm} restated]
    Assuming $k \leq 2^{\sqrt{n}/500}$, we have for large enough $n$,
    \begin{equation*}
        \norm{T_{\mcG_R}T_{\mcG_C}T_{\mcG_R} - T_\mcG}_{\mathrm{op}} \leq \frac{1}{2^{\sqrt{n}/128}},
    \end{equation*}
    or rather for $f : \{\pm 1 \}^{nk} \to \R$:
    \begin{equation*}
        \ip{f}{(T_{\mcG_R}T_{\mcG_C}T_{\mcG_R} - T_\mcG) f} \leq \frac{1}{2^{\sqrt{n}/128}}\cdot \ip{f}{f}.
    \end{equation*}
\end{lemma}

Note that it suffices to prove maximization across symmetric linear forms because the operator is self-adjoint. We will proceed by decomposing $f = f_{B_{\mathrm{safe}}} + f_{B_{\mathrm{coll}}} + f_{B_{=0}}$ where $f_S$ is supported on $S \subseteq \{\pm1\}^{nk}$. Note that these regions form a partition of $\{\pm1\}^{nk}$, so these functions are orthogonal to one another.
\begin{align*}
    \abs{\ip{f}{(T_{\mcG_R}T_{\mcG_C}T_{\mcG_R} - T_\mcG) f}} \leq& \abs{\ip{f_{B_{\mathrm{safe}}}}{(T_{\mcG_R}T_{\mcG_C}T_{\mcG_R} - T_\mcG) f_{\sD}}}\\
    &+ \abs{\ip{f_{B_{\mathrm{coll}}}}{(T_{\mcG_R}T_{\mcG_C}T_{\mcG_R} - T_\mcG) f_\sD}}\\
    &+ \abs{\ip{f_{B_{=0}}}{(T_{\mcG_R}T_{\mcG_C}T_{\mcG_R} - T_\mcG) f_{B_{=0}}}}.
\end{align*}
Note that the cross terms involving $B_{=0}$ are all zero, as a permutation will not cross between these regions. Our proof will bound each of these terms separately.

\subsection{$f$ Supported on $B_{\textrm{safe}}$}

\begin{lemma}
    $\abs{\ip{f_{B_{\mathrm{safe}}}}{(T_{\mcG_R}T_{\mcG_C}T_{\mcG_R} - T_\mcG) f_{\sD}}} \leq \frac{4\sqrt{n}k^2}{2^{\sqrt{n}}} \cdot \ip{f}{f}$.
\end{lemma}
\begin{proof}
Let $X \in B_{\mathrm{safe}}$, $g : \{\pm1\}^{nk} \to \R$.
\begin{align*}
    (T_{\mcG_R} - T_\mcG)g(X ) &= \sum_{Y \in \{\pm1\}^{nk}} \Pr[X \to_{T_{\mcG_R}} Y] \cdot g(Y) - \sum_{Y \in \{\pm1\}^{nk}} \Pr[X \to_{T_{\mcG}} Y] \cdot g(Y)\\
    &= \frac{1}{\abs{B_{\mathrm{safe}}}}\sum_{Y \in B_{\mathrm{safe}}} g(Y) - \frac{1}{\abs{\sD}}\sum_{Y \in \sD} g(Y)\\
    &= \pbra{\frac{1}{\abs{B_{\mathrm{safe}}}} - \frac{1}{\abs{\sD}}}\sum_{Y \in B_{\mathrm{safe}}} g(Y) - \frac{1}{\abs{\sD}}\sum_{Y \in B_{\mathrm{coll}}} g(Y)\\
    &= \pbra{1 - \frac{\abs{B_{\mathrm{safe}}}}{\abs{\sD}}} \cdot \frac{1}{\abs{B_{\mathrm{safe}}}}\sum_{Y \in B_{\mathrm{safe}}} g(Y) - \frac{\abs{B_{\mathrm{coll}}}}{\abs{\sD}} \cdot \frac{1}{\abs{B_{\mathrm{coll}}}}\sum_{Y \in B_{\mathrm{coll}}} g(Y)\\
    &= \frac{\abs{B_{\mathrm{coll}}}}{\abs{\sD}} \pbra{T_{\mcG_R} - \mcH}g(X).
\end{align*}

Our definition of $\mcH g(X) = \frac{1}{\abs{B_{\mathrm{coll}}}}\sum_{Y \in B_{\mathrm{coll}}} g(Y)$ corresponds to the random walk operator that puts all probability weight into $B_{\mathrm{coll}}$ uniformly. Note that $\mcH$ does not correspond to any random walk induced by a distribution on $\mfS_{\{\pm1\}^{nk}}$ (so it cannot be written as $T_{\mcH}$), but is still a random walk operator. With this in hand we may write:
\begin{align*}
    \abs{\ip{f_{B_{\mathrm{safe}}}}{(T_{\mcG_R}T_{\mcG_C}T_{\mcG_R} - T_\mcG) f_{\sD}}} &= \sum_{X \in \{\pm1\}^{nk}} f_{B_{\mathrm{safe}}}(X) \cdot (T_{\mcG_R}-T_\mcG)(T_{\mcG_C}T_{\mcG_R}f_{\sD})(X)\\
    &= \sum_{X \in B_{\mathrm{safe}}} f_{B_{\mathrm{safe}}}(X) \cdot (T_{\mcG_R}-T_\mcG)(T_{\mcG_C}T_{\mcG_R}f_{\sD})(X)\\
    &= \frac{\abs{B_{\mathrm{coll}}}}{\abs{\sD}} \sum_{X \in B_{\mathrm{safe}}} f_{B_{\mathrm{safe}}}(X) \cdot (T_{\mcG_R}-\mcH)(T_{\mcG_C}T_{\mcG_R}f_{\sD})(X)\\
    &= \frac{\abs{B_{\mathrm{coll}}}}{\abs{\sD}} \abs{\ip{f_{B_{\mathrm{safe}}}}{(T_{\mcG_R}T_{\mcG_C}T_{\mcG_R} - \mcH T_{\mcG_C}T_{\mcG_R}) f_{\sD}}}.
\end{align*}

The only important fact about $\mcH$ is that it is a valid random walk operator, which allows us to use \Cref{lem:escape probs} to bound this final inner product crudely as:
\begin{align*}
    \abs{\ip{f_{B_{\mathrm{safe}}}}{(T_{\mcG_R}T_{\mcG_C}T_{\mcG_R} - T_\mcG) f_{\sD}}} &\leq \frac{\abs{B_{\mathrm{coll}}}}{\abs{\sD}} \pbra{\abs{\ip{f_{B_{\mathrm{safe}}}}{(T_{\mcG_R}T_{\mcG_C}T_{\mcG_R} f_{\sD}}} + \abs{\ip{f_{B_{\mathrm{safe}}}}{\mcH T_{\mcG_C}T_{\mcG_R}) f_{\sD}}}}\\
    &\leq \frac{2\abs{B_{\mathrm{coll}}}}{\abs{\sD}} \norm{f_{B_{\mathrm{safe}}}}_{2} \norm{f_\sD}_{2} \tag{\Cref{lem:escape probs}}\\ 
    &\leq \frac{2\abs{B_{\mathrm{coll}}}}{\abs{\sD}} \langle f, f \rangle.
\end{align*}
\Cref{fact:colorclasssizes} then suffices to prove the claim. \end{proof}

\subsection{$f$ Supported on $B_{\textrm{coll}}$}

\begin{lemma}
    \label{lem:collcase}
    $\abs{\ip{f_{B_{\mathrm{coll}}}}{(T_{\mcG_R}T_{\mcG_C}T_{\mcG_R} - T_\mcG) f_\sD}} \leq \frac{8\sqrt{n}k^2}{2^{\sqrt{n}/32}} \ip{f}{f}$.
\end{lemma}


\begin{proof}
First, we can decompose $f_\sD = f_{B_{\mathrm{safe}}} + f_{B_{\mathrm{coll}}}$ and bound:
\begin{align*}
    &\abs{\ip{f_{B_{\mathrm{coll}}}}{(T_{\mcG_R}T_{\mcG_C}T_{\mcG_R} - T_\mcG) f_\sD}} \\
    \leq &\abs{\ip{f_{B_{\mathrm{coll}}}}{(T_{\mcG_R}T_{\mcG_C}T_{\mcG_R} - T_\mcG) f_{B_{\mathrm{safe}}}}} + \abs{\ip{f_{B_{\mathrm{coll}}}}{(T_{\mcG_R}T_{\mcG_C}T_{\mcG_R} - T_\mcG) f_{B_{\mathrm{coll}}}}}.
\end{align*}
By the self-adjointness of the operator, the first term is bounded by the case above, so it suffices to bound the latter. For this term, we can appeal directly to \Cref{lem:escape probs} and the triangle inequality to get:
\begin{align*}
    &\abs{\ip{f_{B_{\mathrm{coll}}}}{(T_{\mcG_R}T_{\mcG_C}T_{\mcG_R} - T_\mcG) f_{B_{\mathrm{coll}}}}} \\
    \leq &\sqrt{\max_{X \in B_{\mathrm{coll}}} \Pr[X \to_{T_{\mcG_R}T_{\mcG_C}T_{\mcG_R}} B_{\mathrm{coll}}] + \max_{X \in B_{\mathrm{coll}}} \Pr[X \to_{T_\mcG} B_{\mathrm{coll}}]} \ip{f_{B_{\mathrm{coll}}}}{f_{B_{\mathrm{coll}}}}.
\end{align*}
Note that regardless of choice of $X$, the latter probability $\Pr[X \to_{T_\mcG} B_{\mathrm{coll}}] = \frac{\abs{B_{\mathrm{coll}}}}{\abs{\sD}}$ which is less than $\frac{2\sqrt{n}k^2}{2^{\sqrt{n}}}$ by \Cref{fact:colorclasssizes}. For the former, we will need a slightly more detailed analysis which also serves as the proof of \Cref{lem:lowcollprob} in the previous section:

\begin{lemma}[Restatement of \Cref{lem:lowcollprob}]
    \label{lem:relowcollprob}
    For all $X \in \sD$, $\Pr[X \to_{T_{\mcG_C}T_{\mcG_R}} B_{\mathrm{coll}}] \leq \frac{2\sqrt{n}k^2}{2^{\sqrt{n}/16}}$.
\end{lemma}

The lemma is immediately sufficient to achieve the bound in \Cref{lem:collcase}.
\end{proof}

\begin{proof}[Proof of \Cref{lem:relowcollprob}]
We will apply a union bound over the probability of any pair of rows ``colliding'', which would put them in $B_{\mathrm{coll}}$. Let $X \in \sD$. We will model our process as:
\begin{equation*}
    X \to_{T_{\mcG_R}} \bm{Y} \to_{T_{\mcG_C}} \bm{Z}.
\end{equation*}
We then fix $\bm{Z}^\ell_{i, \cdot}$ and $\bm{Z}^m_{i, \cdot}$ for $i \in [\sqrt{n}]$, $\ell \neq m \in [k]$. The only fact we will use about $\bm{Y}$ is that for some $j \in [k]$ (potentially equal to $i$), we have $(\bm{Y}^\ell_{j, \cdot}, \bm{Y}^m_{j, \cdot})$ are uniform from ${\{\pm1\}^{\sqrt{n}} \choose 2}$. To see this note that there must exist some $j$ s.t. $X^\ell_{j, \cdot} \neq X^m_{j,\cdot}$, otherwise $X \notin \sD$. Since the permutation applied to these two rows is uniform from $\mfS_{\{\pm1\}^{\sqrt{n}}}$, the resulting rows in $Y$ look like a uniform unequal pair.

With this in mind, we will now condition on the event that $d(\bm{Y}^\ell_{j, \cdot}, \bm{Y}^m_{j, \cdot}) \geq \sqrt{n}/4$ (distance here is Hamming distance), allowing us to split our analysis into two cases:
\begin{align*}
    \Pr[\bm{Z}^\ell_{i, \cdot} = \bm{Z}_{i, \cdot}^m]  \leq&\Pr[\bm{Z}^\ell_{i, \cdot} = \bm{Z}^m_{i, \cdot} \mid d(\bm{Y}^\ell_{j, \cdot}, \bm{Y}^m_{j, \cdot}) > \sqrt{n}/4] + \Pr[d(\bm{Y}^\ell_{j, \cdot}, \bm{Y}^m_{j, \cdot}) \leq \sqrt{n}/4]\\
    \leq&\frac1{2^{\sqrt{n}/4}}+\frac1{e^{\sqrt{n}/16}}\tag{\Cref{lem:given Y far 2D}, \Cref{lem:Y probably far 2D}}\\
    \leq& \frac2{2^{\sqrt{n}/16}}.
\end{align*}
Applying a union bound over $\leq \sqrt{n}k^2$ rows completes the proof.
\end{proof}


\begin{lemma}\label{lem:given Y far 2D}
    $\Pr[\bm{Z}^\ell_{i, \cdot} = \bm{Z}^m_{i, \cdot} \mid d(\bm{Y}^\ell_{j, \cdot}, \bm{Y}^m_{j, \cdot}) > \sqrt{n}/4] \leq \frac{1}{2^{\sqrt{n}/4}}$.
\end{lemma}
\begin{proof}
The probability that $\bm{Z}^\ell_{i, \cdot}$ and $\bm{Z}^m_{i, \cdot}$ are equal can be viewed as the probability that all of their individual bits are equal, and they are all independent since they come from independently sampled column permutations. Since $\bm{Y}^\ell_{j, \cdot}$ and $\bm{Y}^m_{j, \cdot}$ differ in at least $\sqrt{n}/4$ places, $\bm{Y}^\ell$ and $\bm{Y}^m$ must differ in at least that many columns. In these columns, it can be seen that the corresponding bits in $\bm{Z}^\ell_{i, \cdot}$ and $Z^m_{i, \cdot}$ are the same with probability $\leq \frac{1}{2}$ (the probability is exactly one half when the columns are sampled uniformly independently, conditioning that they are unequal only lowers this probability). By independence the probability is less than $\frac{1}{2^{\sqrt{n}/4}}$. \end{proof}

\begin{lemma}\label{lem:Y probably far 2D}
    $\Pr[d(\bm{Y}^\ell_{j, \cdot}, \bm{Y}^m_{j, \cdot}) \leq \sqrt{n}/4] \leq \frac{1}{e^{\sqrt{n}/16}}$.
\end{lemma}
\begin{proof}
Note that $\Pr[d(\bm{Y}^\ell_{j, \cdot},\bm{Y}^m_{j, \cdot}) \leq \sqrt{n}/4] \leq \Pr_{\bx, \by \sim \{\pm1\}^{\sqrt{n}}}[d(\bx, \by) \leq \sqrt{n}/4]$. For uniform $x,y$, the random variable $d(x,y)$ is the sum of $\sqrt{n}$ independent Bernoulli$(1/2)$ random variables. This has expectation $\sqrt{n}/2$ and thus by Hoeffding's Inequality:
\begin{equation*}
    \Pr_{\bx,\by \sim \{\pm1\}^{\sqrt{n}}}[d(\bx,\by) \leq \sqrt{n}/4] \leq e^{-\sqrt{n}/16}.\qedhere
\end{equation*}
\end{proof}

\subsection{The Induction Case}

\begin{lemma} \label{lemmaInductionCase}
Let $f : \{\pm1\}^{nk} \to \R$ be supported on $B_{=0}$ for $k \geq 2$. Then, we have
\[ 
\abs{\ip{f}{\pbra{T_{\mcG_R}^{(k)}T_{\mcG_C}^{(k)}T_{\mcG_R}^{(k)} - T_\mcG^{(k)}}f}} \le \norm{T_{\mcG_R}^{(k-1)}T_{\mcG_C}^{(k-1)}T_{\mcG_R}^{(k-1)} - T_{\mcG}^{(k-1)}}_{\mathrm{op}} \ip{f}{f}. 
\]
\end{lemma}

\begin{proof}
    The proof follows from the same proof as the proof of \Cref{lemma:f supported on B0}.
\end{proof}

\subsection{Wrapping Up}

Putting together all three cases we have:
\begin{equation*}
    \abs{\ip{f}{(T_{\mcG_R}T_{\mcG_C}T_{\mcG_R} - T_\mcG) f}} \leq \frac{4\sqrt{n}k^2}{2^{\sqrt{n}}}\ip{f}{f} + \frac{8\sqrt{n}k^2}{2^{\sqrt{n}/32}} \ip{f}{f} + \norm{T_{\mcG_R}^{(k-1)}T_{\mcG_C}^{(k-1)}T_{\mcG_R}^{(k-1)} - T_\mcG^{(k-1)}}_{\mathrm{op}}\ip{f}{f}.
\end{equation*}

Since $\norm{T_{\mcG_R}^{(1)}T_{\mcG_C}^{(1)}T_{\mcG_R}^{(1)} - T_\mcG^{(1)}}_{\mathrm{op}} = 0$ and by assumption $k \leq 2^{\sqrt{n}/500}$, it follows by induction that:
\[ \norm{T_{\mcG_R}^{(k)}T_{\mcG_C}^{(k)}T_{\mcG_R}^{(k)} - T_\mcG^{(k)}}_{\mathrm{op}} \le \sum_{\ell = 2}^k \pbra{\frac{4\sqrt{n}\ell^2}{2^{\sqrt{n}}} + \frac{8\sqrt{n}\ell^2}{2^{\sqrt{n}/32}}} \leq \frac{k^3}{2^{\sqrt{n}/64}} \leq \frac{1}{2^{\sqrt{n}/128}}. \]


