\section{Proof of \Cref{thm:small k}}\label{sec:small k}
Throughout this section fix $m\geq 3$. Our goal in this section is to establish \Cref{thm:small k}, which states that $R_{m,m-1,k}-R_{m,m,k}$ has small spectral norm. Informally, we show that completely randomizing $m-1$ out of $m$ wires in a reversible circuit is very similar to randomizing all $m$ wires. Recall that $R_{m,m-1,k}$ is defined via the distributions $\mathcal{D}_X^{m,m-1,k}$ (the equal mixture of $\mcD_X^{m,S,k}$ for all $S$ with $|S|=m-1$) for $X\in\{\pm1\}^{mk}$, from which one samples by sampling a random set $\mathbf{S}\subseteq [m]$ with $|\mathbf{S}|=1$ (so really $\mathbf{S}=\{\mathbf{a}\}$, where $\mathbf{a}$ is a random element of $[m]$), ``fixes" the entries $X^i_{\mathbf{a}}$, and applies a random permutation to the coordinates not equal to $\mathbf{a}$. We use this notation and terminology of a ``fixed" coordinate $\mathbf{a}$ throughout this section.

As alluded to in \Cref{sec:overview}, our proof will decompose the space $\R^{\{\pm1\}^{mk}}$ on which these operators act into three orthogonal components, to be defined in \Cref{sec:decomposition}. Then \Cref{sec:decomposition}, \Cref{sec:B=1}, and \Cref{sec:B>=2} will bound the contributions from vectors lying in these orthogonal components and their cross terms.



\subsection{An Orthogonal Decomposition}\label{sec:decomposition}
\begin{definition}
    Regard elements of $\{\pm1\}^{mk}$ as $k$-by-$m$ matrices, so that the $i$th row of $X$ is $X^i$, and the $a$th column of $X$ is $X_a$. Define 
    \begin{align*}
        B_{\geq 2}&=\cbra{X\in\{\pm1\}^{mk}:\forall i\neq j\in[k], d\pbra{X^i, X^j}\geq 2},\\
        B_{=1}&=\cbra{X\in\{\pm1\}^{mk}:\forall i\neq j\in[k], d\pbra{X^i, X^j}\geq 1}\setminus B_{\geq 2},\\
        B_{=0}&=\cbra{X\in\{\pm1\}^{mk}:\exists i\neq j\in[k], d\pbra{X^i, X^j}=0}.
    \end{align*}
\end{definition}

Our proof that $R_{m,m-1,k}-R_{m,m,k}$ has small spectral norm will go by induction on $k$. \Cref{lemma:f supported on B0} helps to connect the cases of $k-1$ and $k$ in the proof. In particular, it shows that we can pass those functions supported on $B_{=0}$ into the induction.

\begin{lemma}\label{lemma:f supported on B0}
    Let $f:\{\pm1\}^{mk}\to \R$ be supported on $B_{=0}$. Then for any $S_1,\dots,S_t\subseteq[m]$ and $c_1,\dots,c_t\in\R$ we have
    \begin{align*}
        \abs{\left\langle f,\sum_{s=1}^t c_s \pbra{R_{m,S_s,k}-R_{m,m,k}}f\right\rangle }\leq \norm{\sum_{s=1}^t c_s \pbra{R_{m,S_s,k-1}-R_{m,m,k-1}}}_{\mathrm{op}}\cdot\norm{f}_2^2.
    \end{align*}
\end{lemma}
\begin{proof}
    For any map $\phi:[k]\to[k-1]$ (viewed as a coloring of $[k]$ with $k-1$ colors) define the set 
    \begin{align*}
        \mathcal{J}_\phi = \cbra{\pbra{X^1,\dots,X^k}:X^i=X^j\iff \phi(i)=\phi(j)}.
    \end{align*}
    These sets $\mathcal{J}_\phi$ partition $B_{=0}$. Thus, for every $f$ supported on $B_{=0}$ we have a decomposition $f=\sum_{\phi}f_\phi$, where each $f_\phi$ is supported on $\mathcal{J}_\phi$. 

    Now, for each $\phi$ define a map $\mathrm{Res}_\phi:\cbra{f:{\{\pm1\}^{mk}}\to \R:f\text{ supported on }\mathcal{J}_\phi}\to \R^{\{\pm1\}^{m(k-1)}}$ by arbitrarily choosing $i,j\in[k]$ such that $\phi(i)=\phi(j)$ and defining for $f_\phi:\{\pm1\}^{mk}\to \R$ supported on $\mathcal{J}_{\phi}$ the new function $\mathrm{Res}_\phi f_\phi:\{\pm1\}^{m(k-1)}\to \R$ by defining for $X'\in \{\pm1\}^{m(k-1)}$
    \begin{align*}
        \mathrm{Res}_\phi f_\phi(X')=& f(\mathrm{Res}_\phi^*(X')).
    \end{align*}
    where $\mathrm{Res}_\phi^*(X')$ is the unique element of $\mathcal{J}_\phi$ such that $(\mathrm{Res}_\phi^*(X'))^{[k]\setminus\{j\}}=X'$. Note this is well-defined because $f_\phi$ is supported on $\mathcal{J}_\phi$.

    \begin{claim}\label{claim:Res norm}
        For any $f_\phi:{\{\pm1\}^{mk}}\to \R$ supported on $\mathcal{J}_\phi$ we have $\norm{f_\phi}_2=\norm{\mathrm{Res}_\phi f_\phi}_2$ and for any $S\subseteq[m]$,
        \begin{align*}
        \left\langle f_\phi,R_{m,S,k}f_\phi\right\rangle = \left\langle \mathrm{Res}_\phi f_\phi,R_{m,S,k}\mathrm{Res}_\phi f_\phi\right\rangle.
    \end{align*}
    \end{claim}
    We prove the claim later, and for now use it to compute
    \begin{align*}
        &\abs{\left\langle f,\sum_{s=1}^t c_s \pbra{R_{m,S_s,k}-R_{m,m,k}}f\right\rangle} \\
        =&\abs{\sum_{\phi,\phi'}\left\langle f_\phi , \sum_{s=1}^t c_s \pbra{R_{m,S_s,k}-R_{m,m,k}}f_{\phi'}\right\rangle} \\
        =&\abs{\sum_{\phi}\left\langle f_\phi , \sum_{s=1}^t c_s \pbra{R_{m,S_s,k}-R_{m,m,k}}f_{\phi}\right\rangle + \sum_{\phi\neq \phi'}\left\langle f_\phi , \sum_{s=1}^t c_s \pbra{R_{m,S_s,k}-R_{m,m,k}}f_{\phi'}\right\rangle} \\
        =&\abs{\sum_{\phi}\left\langle \mathrm{Res}_\phi f_\phi,\sum_{s=1}^t c_s \pbra{R_{m,S_s,k}-R_{m,m,k}}\mathrm{Res}_\phi f_\phi \right\rangle} \tag{\Cref{claim:Res norm}, \Cref{eq:different colorings} below}\\
        \leq & \sum_{\phi}\norm{\sum_{s=1}^t c_s \pbra{R_{m,S_s,k-1}-R_{m,m,k-1}}}_{\mathrm{op}}\norm{\mathrm{Res}_\phi f_\phi}_2^2 \\
        =&\norm{\sum_{s=1}^t c_s \pbra{R_{m,S_s,k-1}-R_{m,m,k-1}}}_{\mathrm{op}}\sum_{\phi}\norm{f_\phi}_2^2\\
        =& \norm{\sum_{s=1}^t c_s \pbra{R_{m,S_s,k-1}-R_{m,m,k-1}}}_{\mathrm{op}}\norm{f}_2^2.
    \end{align*}
    The last equality follows from orthogonality of the $f_\phi$.
    
    To take care of the cross terms, we observe that for any $X\in\mathcal{J}_\phi$ and $S\subseteq[n]$, we have $\Pr\sbra{X\to_{R_{m,S,k}} \mathcal{J}_{\phi'}}=0$ for any $\phi'\neq \phi$. Then by \Cref{lem:escape probs} we have
    \begin{align}\label{eq:different colorings}
        &\sum_{\phi\neq \phi'}\left\langle f_\phi , \sum_{s=1}^t c_s R_{m,S_s,k}f_{\phi'}\right\rangle =0. 
    \end{align}
    This completes the proof.
\end{proof}

\begin{proof}[Proof of \Cref{claim:Res norm}]
    Without loss of generality assume that $\phi(k-1)=\phi(k)$ so we can regard $\mathrm{Res}_\phi f_\phi$ as a real function on $\{\pm1\}^{m(k-1)}$. Then
    \begin{align*}
        &\left\langle f_\phi,f_\phi\right\rangle 
        =\sum_{X\in\{\pm1\}^{mk}}f_\phi(X)^2
        = \sum_{X\in \mathcal{J}_\phi}\pbra{{f_\phi(X)}}^2\\
        =& \sum_{X'\in \{\pm1\}^{m(k-1)}}\left(\mathrm{Res}_\phi f(X')\right)^2
        = \left\langle \mathrm{Res}_\phi f_\phi , \mathrm{Res}_\phi f_\phi \right\rangle.
    \end{align*}
    To prove the second statement, we compute
    \begin{align*}
        &\left\langle f_\phi,R_{m,S,k}f_\phi\right\rangle\\
        =&\sum_{X\in\{\pm1\}^{mk}}f_\phi(X)\sum_{Y\in\{\pm1\}^{mk}}f_\phi(Y)\Pr\sbra{X\to_{R_{m,S,k}} Y}\\
        =&\sum_{X\in\mathcal{J}_\phi}f_\phi(X)\sum_{Y\in\mathcal{J}_\phi}f_\phi(Y)\Pr\sbra{X\to_{R_{m,S,k}} Y}\\
        =&\sum_{X'\in\{\pm1\}^{m(k-1)}}\mathrm{Res}_\phi f_\phi(X')\sum_{Y'\in\{\pm1\}^{m(k-1)}}\mathrm{Res}_\phi f_\phi(Y')\Pr\sbra{X'\to_{R_{m,S,k-1}} Y'}\\
        =&\left\langle \mathrm{Res}_\phi f_\phi ,R_{m,S,k-1}\mathrm{Res}_\phi f_\phi  \right\rangle.
    \end{align*}
    The second-to-last equality follows because the corresponding $X,Y$ are in $\mathcal{J}_\phi$.
\end{proof}

We now prove \Cref{thm:small k}, deferring proofs of the remaining needed auxiliary results to \Cref{sec:B=1}, \Cref{sec:cross terms}, and \Cref{sec:B>=2}.
\begin{theorem}[\Cref{thm:small k} restated]\label{thm:small k internal}
    Let $m\geq 100$ and let $2\leq k\leq 2^{m/10}$. Given any $f:\{\pm1\}^{mk}\to\R$, we have
    \begin{align*}
       \abs{ \left\langle f,(R_{m,m-1,k}-R_{m,m,k})f\right\rangle }\leq  \pbra{\frac1m+\frac{k^2}{2^{m/4}}}\left\langle f,f\right\rangle.
    \end{align*}
\end{theorem}
\begin{proof}
    We prove by induction on $k$. In the base case $k=1$ and the result holds by the following argument when we write $f=f_2$. Now assume that the result holds for real functions on $\{\pm1\}^{m(k-1)}$. 
    
    Let $f:\{\pm1\}^{mk}\to\R$. Write $f=f_0+f_1+f_2$ where $f_0$ is supported on $B_{=0}$, $f_1$ is supported on $B_{=1}$, and $f_2$ is supported on $B_{\geq 2}$. By \Cref{lem:cross terms B0} applied with both $R_{m,m-1,k}$ and $R_{m,m,k}$ and $B_{=0}$ and $\{\pm1\}^{mk}\setminus B_{=0}=B_{=1}\cup B_{\geq 2}$, the other cross terms vanish, and we have
    \begin{align*}
        &\abs{ \left\langle f,(R_{m,m-1,k}-R_{m,m,k})f\right\rangle }\\
        \leq & \abs{\left\langle f_0,(R_{m,m-1,k}-R_{m,m,k})f_0\right\rangle}+\abs{\left\langle f_1,(R_{m,m-1,k}-R_{m,m,k})f_1\right\rangle}\\
        &\;\;\;\;\;\;+\abs{\left\langle f_2,(R_{m,m-1,k}-R_{m,m,k})f_2\right\rangle}+\abs{2\left\langle f_1,(R_{m,m-1,k}-R_{m,m,k})f_2\right\rangle }\tag{Self-adjointness (\Cref{fact:self-adjoint})}\\
        \leq & \pbra{\frac1m+\frac{(k-1)^2}{2^{m/4}}}\left\langle f_0,f_0\right\rangle +\pbra{\frac1m+\frac{m^5k}{2^{m/2-2}}}\left\langle f_1,f_1\right\rangle + \pbra{\frac1m+\frac{k^2}{2^{m/2}}}\left\langle f_2,f_2\right\rangle + \frac{\sqrt{m}k}{2^{m/2-2}}\norm{f_1}_2\norm{f_2}_2\tag{\Cref{lemma:f supported on B0} + induction, \Cref{cor:f1-f1}, \Cref{prop:B>=2 square term}, \Cref{lem:cross terms B=1 to B=2}, in that order}\\
        \leq & \pbra{\frac1m+\frac{k^2-k}{2^{m/4}}}\left\langle f,f\right\rangle+\frac{m^5k}{2^{m/2-3}}\norm{f_1}_2\norm{f_2}_2\\
        \leq & \pbra{\frac1m+\frac{k^2-k}{2^{m/4}}}\left\langle f,f\right\rangle+\frac{k}{2^{m/3}}\left\langle f,f\right\rangle\\
        \leq&\pbra{\frac1m+\frac{k^2}{2^{m/4}}}\left\langle f,f\right\rangle.\qedhere
    \end{align*}
\end{proof}


\subsection{$f$ Supported on $B_{=1}$}\label{sec:B=1}

We now use \Cref{lem:escape probs} to bound $\abs{\left\langle f, (R_{m,m-1,k}-R_{m,m,k})f\right\rangle}\leq \abs{\left\langle f, R_{m,m-1,k}f\right\rangle}$ when $f$ is supported only on $B_{= 1}$ and the cross terms contributed. We first define a partition of $B_{=1}$, and apply \Cref{lem:escape probs} to these different parts. One of our main observations to bound $f$ supported on $B_{=1}$ is the observation that when $k$ is small, it is highly unlikely that two vectors out of any $k$ are close to each other. Thus, a random walk beginning in $B_{=1}$ and obeying the transition probabilities given by $B_{m,m-1,k}$ will rarely remain in $B_{=1}$. Formally, \Cref{lem:escape probs} bounds the contributions to the spectral norm by these transition probabilities.
\begin{definition}
    For $S\subseteq[m]$ define the set $\mathcal{I}_S\subseteq\{\pm1\}^{mk}$ by
    \begin{align*}
        \mathcal{I}_S=\cbra{X\in B_{=1}: \forall a\in S ,\exists i,j\in[k]:\Delta\pbra{X^i,X^j}=\{a\}}.
    \end{align*}
\end{definition}
Unfortunately, the sets $\mathcal{I}_S$ for different $S\subseteq[m]$ do not form a partition of $B_{=1}$, since there is overlap between $\mathcal{I}_S$ and $\mathcal{I}_T$ for $S\neq T$. However, we can artificially make this into a partition.
\begin{definition}
    For $S\subseteq[m]$ with $|S|=1$ (so $S=\{a\}$ for some $a\in[m]$) define the set $\widetilde{\mathcal{I}}_S\subseteq\{\pm1\}^{nk}$ by
    \begin{align*}
        \widetilde{\mathcal{I}}_S=\mathcal{I}_S\setminus \bigcup_{a'\neq a}\mathcal{I}_{\{a'\}}.
    \end{align*}
    Now place an arbitrary ordering $\preceq $ on the set $\{S\subseteq[m]:|S|=2\}$ and define
    \begin{align*}
        \widetilde{\mathcal{I}}_S = \mathcal{I}_S\setminus \pbra{ \bigcup_{S'\subseteq [m]:|S'| = 2,S'\preceq S}\mathcal{I}_{S'}}.
    \end{align*}
\end{definition}

\begin{observation}\label{obs:Is partition}
    The collection of sets $\{\widetilde{\mathcal{I}}_S:S\subseteq[m],|S|\leq 2\}$ forms a partition of $B_{=1}$.
\end{observation}

\begin{lemma}\label{lem:square terms B=1}
    Let $S\subseteq[m]$ be such that $|S|\leq 2$. If $k\geq 2$ and $f:\{\pm1\}^{mk}\to\R$ is supported on $\widetilde{\mathcal{I}}_S$ then 
    \begin{align*}
        \abs{\left\langle f,(R_{m,m-1,k}-R_{m,m,k})f\right\rangle }\leq\pbra{\frac1m+\frac{k}{2^{m/2-2}}}\left\langle f,f\right\rangle.
    \end{align*}
\end{lemma}
\begin{proof}
    We bound $\abs{ \left\langle f,R_{m,m-1,k}f\right\rangle }\geq \abs{ \left\langle f,(R_{m,m-1,k}-R_{m,m,k})f\right\rangle }$. This inequality is true because $R_{m,m,k}$ and $R_{m,m-1,k}-R_{m,m,k}$ are PSD by. Suppose first that $|S|=1$ so that $S=\{a\}$ for some $a\in[m]$. Let $X\in\widetilde{\mathcal{I}}_S$. Then for every $W\subseteq[m]$ with $|W|=m-1$ and $a\in W$ we have
    \begin{align*}
        \Pr\sbra{X\to_{R_{m,W,k}} \widetilde{\mathcal{I}}_S}
        \leq& \sum_{i,j\in[k]}\Pr_{\mathbf{Y}\sim \mathcal{D}_X^{m,W,k}}\sbra{\Delta\pbra{\mathbf{Y}^i,\mathbf{Y}^j}=\{a\}}
        \leq  \frac{k^2}{2^{m-1}}.
    \end{align*}
    This is because $a\in W$, and so it must be the case that for all $i\neq j\in[k]$ we have $X^i_{[m]\setminus\{b\}}\neq X^j_{[m]\setminus\{b\}}$. Otherwise we would have $\widetilde{\mathcal I}_{\{a,b\}}$, contradicting that $X\in \widetilde{\mathcal I}_{\{a\}}$. Using this bound, we have that
    \begin{align*}
        &\abs{\left\langle f,(R_{m,m-1,k}-R_{m,m,k})f\right\rangle }\\
        \leq&\frac1m\sum_{W\subseteq [m],|W|=m-1}\abs{\left\langle f,R_{m,W,k}f\right\rangle }+\abs{\left\langle f,R_{m,m,k}f\right\rangle }\\
        \leq&\frac1m\abs{\left\langle f,R_{m,[m]\setminus\{a\},k}f\right\rangle }+ \frac1m\sum_{W\subseteq [m],|W|=m-1,a\in W}\abs{\left\langle f,R_{m,W,k}f\right\rangle }+\abs{\left\langle f,R_{m,m,k}f\right\rangle }\\
        \leq&\frac1m\langle f,f\rangle+  \frac{k}{2^{m/2-1}}\langle f,f\rangle+ \frac{k}{2^{m/2}}\langle f,f\rangle.
    \end{align*}
    The last inequality follows from the \Cref{lem:escape probs} and our previous calculation, while noticing that any $R_{m,W,k}$ has the uniform distribution over $\{\pm1\}^{mk}$ as a stationary distribution. A similar calculation shows that $\Pr\sbra{X\to_{R_{m,m,k}} \widetilde{\mathcal{I}}_S}\leq \frac{k^2}{2^{m}}$, which is used to bound the third term $\abs{\left\langle f,R_{m,m,k}f\right\rangle }$ by \Cref{lem:escape probs}. This completes the proof for the case $S=\{a\}$.

    Now assume that $|S|=2$ so that $S=\{a,b\}$ and $X\in\widetilde{\mathcal{I}}_S$. Fix $W\subseteq[m]$ with $|W|=m-1$. Assume that $a\not\in W$. Then
    \begin{align*}
        \Pr\sbra{X\to_{R_{m,W,k}} \widetilde{\mathcal{I}}_S}\leq&\sum_{i,j\in[k]}\Pr_{\mathbf{Y}\sim\mathcal{D}^{m,W,k}_X}\sbra{\Delta\pbra{\mathbf{Y}^i,\mathbf{Y}^j}=b}\leq\sum_{i,j\in[k]}\frac{1}{2^{m-1}}\leq \frac{k^2}{2^{m-1}}.
    \end{align*}
    This is because if $\Delta(X^i,X^j)=\{a\}$ then $\Delta(\mathbf{Y}^i,\mathbf{Y}^j)=\{a\}\neq \{b\}$ and otherwise $X^i_{[m]\setminus\{a\}}=X^i_W \neq X^j_W= X^j_{[m]\setminus\{a\}}$. Using this we find that
    \begin{align*}
        \Pr\sbra{X\to_{R_{m,m-1,k}} \widetilde{\mathcal{I}}_S}=&\frac1m\sum_{W\subseteq[m],|S|=m-1}\Pr\sbra{X\to_{R_{m,W,k}} \widetilde{\mathcal{I}}_S}\leq \frac{k^2}{2^{m-1}}.
    \end{align*}
    If $b\in W$ or $a,b\not\in W$ a similar proof shows the same bound. Using the bound $\Pr\sbra{X\to_{R_{m,m,k}} \widetilde{\mathcal{I}}_T}\leq \frac{k^2}{2^{m-1}}$ completes the proof:
    \begin{align*}
        &\abs{\left\langle f,(R_{m,m-1,k}-R_{m,m,k})f\right\rangle}
        \leq\abs{\left\langle f,R_{m,m-1,k}f\right\rangle}+\abs{\left\langle f,R_{m,m,k}f\right\rangle}
        \leq\frac{k}{2^{m/2-1}}\langle f,f\rangle+ \frac{k}{2^{m/2-1}}\langle f,f\rangle.
    \end{align*}
    The second inequality is an application of \Cref{lem:escape probs}.
\end{proof}

\begin{lemma}\label{lem:cross terms B=1 to B=1}
    Let $S\neq T\subseteq[m]$ be such that $|S|,|T|\leq 2$. If $k\geq 2$ and $f:\{\pm1\}^{mk}\to\R$ is supported on $\widetilde{\mathcal{I}}_S$ and $g:\{\pm1\}^{mk}\to\R$ is supported on $\widetilde{\mathcal{I}}_T$ then 
    \begin{align*}
       \abs{ \left\langle f,(R_{m,m-1,k}-R_{m,m,k})g\right\rangle }\leq\frac{k}{2^{m/2-2}}\norm{f}_2\norm{g}_2.
    \end{align*}
\end{lemma}
\begin{proof}
    Let $X\in\widetilde{\mathcal{I}}_S$. Let $a\in T\setminus S$ be such that there does not exist $i,j\in[k]$ such that $X\in \mathcal{I}_{\{a\}}$. Then 
    \begin{align*}
        &\Pr\sbra{X\to_{R_{m,m-1,k}} \widetilde{\mathcal{I}}_T}
        \leq \Pr\sbra{X\to_{R_{m,m-1,k}} \widetilde{\mathcal{I}}_{\{a\}}}
        \leq \Pr\sbra{X\to_{R_{m,m-1,k}} {\mathcal{I}}_{\{a\}}}\\
        \leq & \sum_{i,j\in[k]}\Pr_{\mathbf{Y}\sim \mathcal{D}_X^{m,m-1,k}}\sbra{\Delta\pbra{\mathbf{Y}^i,\mathbf{Y}^j}=\{a\}}
        \leq \frac{k^2}{2^{m-1}}.\tag{$X\not\in\mathcal{I}_{\{a\}}$}
    \end{align*}
    Then applying \Cref{lem:escape probs} gives
    \begin{align*}
        \abs{ \left\langle f,R_{m,m-1,k}g\right\rangle }\leq\frac{k}{2^{m/2-1}}\norm{f}_2\norm{g}_2.
    \end{align*}
    A similar calculation shows that $\Pr\sbra{X\to_{R_{m,m,k}} \widetilde{\mathcal{I}}}\leq \frac{mk^2}{2^{m-1}}$. Then \Cref{lem:escape probs} gives
    \begin{align*}
        \abs{ \left\langle f,R_{m,m,k}g\right\rangle }\leq\frac{k}{2^{m/2-1}}\norm{f}_2\norm{g}_2.
    \end{align*}
    Applying the triangle inequality completes the proof.
\end{proof}



\begin{corollary}\label{cor:f1-f1}
    Assume $k\geq 2$. Let $f:\{\pm1\}^{mk}\to\R$ be supported on $B_{=1}$. Then 
    \begin{align*}
        \abs{\left\langle f,(R_{m,m-1,k}-R_{m,m,k})f\right\rangle }\leq \pbra{\frac1m+\frac{m^5k}{2^{m/2-2}}}\left\langle f,f\right\rangle.
    \end{align*}
\end{corollary}
\begin{proof}
    Write $f=\sum_{S\subseteq[m]:|S|\leq 2}f_{S}$ where each $f_S$ is supported on $\mathcal{I}_S$. Then
    \begin{align*}
        & \abs{\left\langle f,(R_{m,m-1,k}-R_{m,m,k})f\right\rangle}\\
        \leq &\abs{\left\langle f,R_{m,m-1,k}f\right\rangle} \tag{$R_{m,m,k}$ and $R_{m,m-1,k}-R_{m,m,k}$ both PSD}\\
        = &\sum_{S,T\subseteq [m]:|S|,|T|\leq 2}\abs{\left\langle f_S,R_{m,m-1,k}f_T\right\rangle}\\
        \leq &\sum_{S\subseteq [m]:|S|\leq 2}\abs{\left\langle f_S,R_{m,m-1,k}f_S\right\rangle}+ \sum_{S\neq T\subseteq [m]:|S|,|T|\leq 2}\abs{\left\langle f_S,R_{m,m-1,k}f_T\right\rangle} \\
        \leq & \pbra{\frac1m+\frac{k}{2^{m/2-2}}}\sum_{S\subseteq [m]:|S|\leq 2}\left\langle f_S,f_S\right\rangle + \frac{k}{2^{m/2-2}}\sum_{S\neq T\subseteq[m]:|S|,|T|\leq 2}\norm{f_S}_2\norm{f_T}_2\tag{\Cref{lem:square terms B=1}, \Cref{lem:cross terms B=1 to B=1}}\\
        = & \pbra{\frac1m+\frac{k}{2^{m/2-2}}}\left\langle f,f\right\rangle + \frac{k}{2^{m/2-2}}\sum_{S\neq T\subseteq[m]:|S|,|T|\leq 2}\norm{f_S}_2\norm{f_T}_2 \\
        \leq & \pbra{\frac1m+\frac{k}{2^{m/2-2}}}\left\langle f,f\right\rangle + \frac{k}{2^{m/2-2}}\sum_{S\neq T\subseteq[m]:|S|,|T|\leq 2}\left\langle f,f\right\rangle\\
        \leq & \pbra{\frac1m+\frac{k}{2^{m/2-2}}}\left\langle f,f\right\rangle + \frac{m^5k^2}{2^{m/2-2}}\left\langle f,f\right\rangle\\
        \leq & \pbra{\frac1m+\frac{m^5k}{2^{m/2-2}}}\left\langle f,f\right\rangle .
    \end{align*}
    Note that we can apply \Cref{lem:square terms B=1} and \Cref{lem:cross terms B=1 to B=1} because $k\geq 2$.
\end{proof}
\subsection{Cross Terms}\label{sec:cross terms}
We can use the same idea to bound the contributions from the cross terms.
\begin{lemma}\label{lem:cross terms B=1 to B=2}
    Let $f_1:\{\pm1\}^{mk}\to\R$ be supported on $B_{=1}$ and let $f_2:\{\pm1\}^{mk}\to\R$ be supported on $B_{\geq 2}$. Then 
    \begin{align*}
        \abs{\left\langle f_1,(R_{m,m-1,k}-R_{m,m,k})f_2\right\rangle} \leq \frac{\sqrt{m}k}{2^{m/2-2}}\norm{f_1}_2\norm{f_2}_2.
    \end{align*}
\end{lemma}
\begin{proof}
    For $X\in B_{\geq 2}$ we have $\Pr\sbra{X\to_{R_{m,m-1,k}} B_{=1}}\leq \frac{k^2m}{2^{m-1}}$. Apply \Cref{lem:escape probs} to find that
    \begin{align*}
        \abs{\left\langle f_1,R_{m,m-1,k}f_2\right\rangle} \leq \frac{\sqrt{m}k}{2^{m/2-1}}\norm{f_1}_2\norm{f_2}_2.
    \end{align*}
    For $X\in B_{\geq 2}$ we have $\Pr\sbra{X\to_{R_{m,m,k}} B_{=1}}\leq \frac{k^2m}{2^{m-1}}$. Apply \Cref{lem:escape probs} to find that
    \begin{align*}
        \abs{\left\langle f_1,R_{m,m,k}f_2\right\rangle} \leq \frac{\sqrt{m}k}{2^{m/2-1}}\norm{f_1}_2\norm{f_2}_2.
    \end{align*}
    Applying the triangle inequality completes the proof.
\end{proof}

\begin{lemma}\label{lem:cross terms B0}
    Let $f_0:\{\pm1\}^{mk}\to\R$ be supported on $B_{=0}$ and let $f_1:\{\pm1\}^{mk}\to\R$ be supported on $B_{=1}\cup B_{\geq 2}$. Then 
    \begin{align*}
        \abs{\left\langle f_0,(R_{m,m-1,k}-R_{m,m,k})f_1\right\rangle} = 0.
    \end{align*}
\end{lemma}
\begin{proof}
    For $X\in B_{\geq 2}\cup B_{=1}$ we have $\Pr\sbra{X\to_{R_{m,m-1,k}} B_{=0}}=\Pr\sbra{X\to_{R_{m,m,k}} B_{=0}}=0$. Apply \Cref{lem:escape probs} to bound 
    \begin{align*}
        &\abs{\left\langle f_0,(R_{m,m-1,k}-R_{m,m,k})f_1\right\rangle} \leq \abs{\left\langle f_0,R_{m,m-1,k}f_1\right\rangle} + \abs{\left\langle f_0,R_{m,m,k}f_1\right\rangle} \leq0.\qedhere
    \end{align*}
\end{proof}

\subsection{A Hybrid Argument for $f$ Supported on $B_{\geq 2}$}\label{sec:B>=2}


In this section we bound the square terms $\left\langle f_2,(R_{m,m-1,k}-R_{m,m,k})f_2\right\rangle$ for $f_2$ supported on $B_{\geq 2}$. As mentioned in \Cref{sec:overview}, our key idea is that when $k$ is small compared to $m$, the fraction of $X\in\{\pm1\}^{mk}$ with two identical columns is so small that applying the noise by randomly permuting the rows is almost the same as randomly replacing the rows with completely random rows. That is, sampling without replacement resembles sampling with replacement closely.

This observation allows us to pass from the random walk described by $R_{m,m-1,k}$ to a different random walk described by a nicer noise model described by operators we will call $Q_{m,m-1,k}$. The key tool we use is the bound given by \Cref{lem:TV distance bound} for relating total-variation distances between Markov chain transition probabilities to a more linear-algebraic notion of closeness, stated in terms of their transition matrices. Fourier-analytic techniques will then be useful to bound the spectral norm of these $Q$-operators.


\begin{definition}
    We define four random walk operators\footnotemark\footnotetext{$R_{m,m-1,k}$ and $R_{m,m,k}$ have already been defined, but we define them again here for ease of comparison.} $R_{m,m-1,k}$, $Q_{m,m-1,k}$, $R_{m,m,k}$, and $Q_{m,m,k}$ on $\R^{\{\pm1\}^{mk}}$. 
    \begin{itemize}     
        \item Define
        \begin{align*}
            (R_{m,m-1,k}f)(X)&=\underset{\mathbf{Y} \sim\mathcal{D}^{m,m-1,k}_{X}}{\E}\sbra{f(\mathbf{Y})}.
        \end{align*}
           
        \item To define $Q_{m,m-1,k}$, for any $X\in \cbra{\pm1}^{mk}$ we define the distribution $\mathcal{C}^{m,m-1,k}_{X}$ as follows. To sample $\mathbf{Y}$ from $\mathcal{C}^{m,m-1,k}_X$, we sample $\mathbf{a}\in[m]$ uniformly randomly and set $\mathbf{Y}^i_{\mathbf{a}}=X^i_{\mathbf{a}}$ for all $i\in[k]$. Then set $\mathbf{Y}^i_{a}$ uniformly randomly for $a\neq \mathbf{a}$. Then
        \begin{align*}
            (Q_{m,m-1,k}f)(X)&=\underset{\mathbf{Y}\sim\mathcal{C}^{m,m-1,k}_X}{\E}\sbra{f(\mathbf{Y})}.
        \end{align*}
        
        \item Define
        \begin{align*}
            (R_{m,m,k}f)(X)&=\Ex_{\mathbf{Y}\sim\mathcal{D}^{m,m,k}_X}\sbra{f(\mathbf Y)}.
        \end{align*}

        \item The operator $Q_{m,m,k}$ is defined by setting for each $f:\cbra{\pm1}^{mk}\to\R$
        \begin{align*}
            (Q_{m,m,k}f)(X)&=\Ex_{\mathbf{Y}\sim \mathrm{Unif}\pbra{\cbra{\pm1}^{mk}}}\sbra{f(\mathbf{Y})}.
        \end{align*}
    \end{itemize}
\end{definition}
\begin{fact}\label{fact:Q self-adjoint}
    The matrices $Q_{m,m-1,k}$ and $Q_{m,m,k}$ are self-adjoint and PSD for any $m,k$.
\end{fact}

We intend to show that $\abs{\left\langle f,(R_{m,m-1,k}-R_{m,m,k})f\right\rangle}$ is small for $f$ supported on $B_{\geq 2}$. We do so by a hybrid argument. In the following inequality, the first (and last) term on the RHS will be bounded by a simple bound on the total variation distance between the distribution $\mathcal{C}_X^{m,m-1,k}$ and $\mathcal{D}_X^{m,m-1,k}$ ($\mathcal{C}_X^{m,m,k}$ and $\mathcal{D}_X^{m,m,k}$). As mentioned before, the second term on the RHS will be bounded using Fourier analysis.
\begin{align*}\label{eq:hybrid}
    &\abs{\left\langle f,(R_{m,m-1,k}-R_{m,m,k})f\right\rangle} \\
        \leq& \abs{\left\langle f,\pbra{R_{m,m-1,k}-Q_{m,m-1,k}}f\right\rangle}+ \abs{\left\langle f,\pbra{Q_{m,m,k}-Q_{m,m-1,k}}f\right\rangle}+\abs{\left\langle f,\pbra{R_{m,m,k}-Q_{m,m,k}}f\right\rangle}.
\end{align*}


\subsubsection{The First Hybrid: $R_{m,m-1,k}$ to $Q_{m,m-1,k}$}

\begin{lemma}\label{lem:hybrid 1}
    Assume that $k\leq 2^{m/3}$. For any $f:\{\pm1\}^{mk}\to\R$ supported on $B_{\geq 2}$ and we have that
    \begin{align*}
        \abs{\left\langle f,(R_{m,m-1,k}-Q_{m,m-1,k})f\right\rangle}\leq \frac{k^2}{2^{m-1}}\left\langle f,f\right\rangle.
    \end{align*}
\end{lemma}
\begin{proof}
We directly compute (using the appropriate definitions of $p_0$ and $p_1$ given in \Cref{lem:TV distance bound}):
\begin{align*}
        &\abs{\left\langle f,(R_{m,m-1,k}-Q_{m,m-1,k})f\right\rangle}\\
        \leq & \sum_{X\in B_{\geq 2}}f(X)^2\sum_{Y\in B_{\geq 2}}\abs{\Pr\sbra{X\to_{R_{m,m-1,k}} Y}-\Pr\sbra{X\to_{Q_{m,m-1,k}}Y}}\tag{\Cref{lem:TV distance bound} + self-adjointness (\Cref{fact:self-adjoint}, \Cref{fact:Q self-adjoint})}\\
        \leq & \frac{k^2}{2^{m-1}}\sum_{X\in B_{\geq 2}}f(X)^2\tag{\Cref{eq:D0 D1 TV} below}\\
        \leq & \frac{k^2}{2^{m-1}}\left\langle f,f\right\rangle.
    \end{align*}
    It suffices to establish \Cref{eq:D0 D1 TV}. Assume that $X\in B_{\geq 2}$. Then
    \begin{align*}
        &\sum_{Y\in B_{\geq 2}}\abs{\Pr\sbra{X\to_{R_{m,m-1,k}} Y}-\Pr\sbra{X\to_{Q_{m,m-1,k}} Y}}\\
        =&\frac1m\sum_{a\in[m]}\sum_{Y\in B_{\geq 2}}\abs{\Pr_{\mathbf{Y}\sim\mathcal{D}^{m,m-1,k}_X}\sbra{\mathbf{Y}=Y|a\text{ fixed}}-\Pr_{\mathbf{Y}\sim\mathcal{C}^{m,m-1,k}_X}\sbra{\mathbf{Y}=Y|a\text{ fixed}}}\\
        =&\frac1m\sum_{a\in[m]}\sum_{Y\in B_{\geq 2},Y_a=X_a}\abs{\Pr_{\mathbf{Y}\sim\mathcal{D}^{m,m-1,k}_X}\sbra{\mathbf{Y}=Y|a\text{ fixed}}-\Pr_{\mathbf{Y}\sim\mathcal{C}^{m,m-1,k}_X}\sbra{\mathbf{Y}=Y|a\text{ fixed}}}\\
        =&\frac1{m}\sum_{a\in[m]}\sum_{Y\in B_{\geq 2},Y_a=X_a}\abs{ \prod_{i=0}^{k-1}\frac{1}{2^{m-1}-i}- \frac1{2^{(m-1)k}}}\tag{$k\leq 2^{m/3}\leq 2^m-2$ and $X\in B_{\geq 2}$}\\
        \leq &\frac1{m}\sum_{a\in[m]}\sum_{Y\in B_{\geq 2},Y_a=X_a}\abs{ \frac1{2^{(m-1)k}}\pbra{\prod_{i=0}^{k-1}\frac{2^{m-1}}{2^{m-1}-i}- 1}}\\
        \leq &\frac1{m}\sum_{a\in[m]}\sum_{Y\in B_{\geq 2},Y_a=X_a}\abs{ \frac1{2^{(m-1)k}}\pbra{1+\frac{k^2}{2^{m-1}}- 1}}\tag{$k\leq 2^{m/3}$, \Cref{fact:k^2}}\\
        = &\frac1{m}\sum_{a\in[m]}\sum_{Y\in B_{\geq 2},Y_a=X_a} \frac{k^2}{2^{m-1}2^{(m-1)k}}\\
        = &\sum_{Y\in B_{\geq 2},Y_1=X_1} \frac{k^2}{2^{m-1}2^{(m-1)k}}\\
        \leq & \frac{2^{mk}}{2^k}\cdot \frac{k^2}{2^{m-1}2^{(m-1)k}}\\
        =& \frac{k^2}{2^{m-1}}.\numberthis\label{eq:D0 D1 TV}
    \end{align*}
    Note that our computations for $\Pr_{\mathbf{Y}\sim\mathcal{D}^{m,m-1,k}_X}\sbra{\mathbf{Y}=Y|a\text{ fixed}}$ and $\Pr_{\mathbf{Y}\sim\mathcal{C}^{m,m-1,k}_X}\sbra{\mathbf{Y}=Y|a\text{ fixed}}$ relied on the fact that $Y\in B_{\geq 2}$.
\end{proof}




\subsubsection{The Second Hybrid: $Q_{m,m-1,k}$ to $Q_{m,m,k}$}
We use Fourier analysis to analyze the spectrum of the operator $Q_{m,m-1,k}-Q_{m,m,k}$, which will prove that $\norm{Q_{m,m-1,k}-Q_{m,m,k}}_{\mathrm{op}}$ is small. We use the Fourier characters as an eigenbasis for $Q_{m,m-1,k}-Q_{m,m,k}$.
\begin{fact}\label{fact:characters under A1-R1}
    Fix $S_1,\dots,S_k\subseteq[n]$. Then
    \begin{align*}
        \pbra{(Q_{m,m-1,k}-Q_{m,m,k})\chi_{S_1,\dots,S_k}}=&
        \begin{cases}
            \frac1m\chi_{S_1,\dots,S_k} &\text{ if }S_1\cup \dots\cup S_k=\{a\} \text{ for some $a\in [m]$}.\\
            0 &\text{ otherwise.}
        \end{cases}
    \end{align*}
\end{fact}
\begin{proof}
    If $\abs{\bigcup_i S_i}=1$ then $S_1=\dots=S_k=\emptyset$ and it is clear that 
    \begin{align*}
        Q_{m,m-1,k}\chi_{S_1,\dots,S_k}=1=Q_{m,m,k}\chi_{S_1,\dots,S_k}.
    \end{align*}
    Now assume $\abs{\bigcup_i S_i}\geq 2$. Then for any $X=(X^1,\dots,X^k)\in \{\pm1\}^{mk}$, 
    \begin{align*}
        &Q_{m,m-1,k}\chi_{S_1,\dots,S_k}(X^1,\dots,X^k)\\
        =&\Ex_{(\mathbf{Y}^1,\dots,\mathbf{Y}^k)\sim\mathcal{C}^{m,m-1,k}_{X}}\sbra{\chi_{S_1,\dots,S_k}(\mathbf{Y}^1,\dots,\mathbf{Y}^k)}\\
        =&\frac1m\sum_{a'\in[m]}\Ex_{\substack{(\mathbf{Y}^1,\dots,\mathbf{Y}^k)\sim\mathcal{C}^{m,m-1,k}_{X}\\\mathbf{a}=a'}}\sbra{\chi_{S_1,\dots,S_k}(\mathbf{Y}^1,\dots,\mathbf{Y}^k)}\\
        =&\frac1m\sum_{a'\in[m]}\Ex_{\substack{(\mathbf{Y}^1,\dots,\mathbf{Y}^k)\sim\mathcal{C}^{m,m-1,k}_{X}\\\mathbf{a}=a'}}\sbra{\prod_{i\in[k]}\prod_{a\in S_i} \mathbf{Y}^i_a}\\
        =&\frac1m\sum_{a'\in[m]}\Ex_{\substack{(\mathbf{Y}^1,\dots,\mathbf{Y}^k)\sim\mathcal{C}^{m,m-1,k}_{X}\\\mathbf{a}=a'}}\sbra{\pbra{\prod_{\substack{i\in[k]\\a'\in S_i}}\mathbf{Y}^i_{a'}}\pbra{\prod_{i\in[k]}\prod_{a\in S_i\setminus\{a'\}} \mathbf{Y}^i_a }}\\
        =&\frac1m\sum_{a'\in[m]}\pbra{\prod_{\substack{i\in[k]\\a'\in S_i}}\mathbf{Y}^i_{a'}}\Ex_{\substack{(\mathbf{Y}^1,\dots,\mathbf{Y}^k)\sim\mathcal{C}^{m,m-1,k}_{X}\\\mathbf{a}=a'}}\sbra{\prod_{i\in[k]}\prod_{a\in S_i\setminus\{a'\}} \mathbf{Y}^i_a }\\
        =&\frac1m\sum_{a'\in[m]}\pbra{\prod_{\substack{i\in[k]\\a'\in S_i}}\mathbf{Y}^i_{a'}}\Ex_{(\mathbf{Y}^1,\dots,\mathbf{Y}^k)\in\pbra{\{\pm1\}^n}^{[k]\setminus\{a'\}}}\sbra{\chi_{S_1\setminus\{a'\},\dots,S_k\setminus\{a'\}}(\mathbf{Y}^1,\dots,\mathbf{Y}^k) }\\
        =&0.\tag{since at least one of the $S_i\setminus\{a'\}$ is nonempty}
    \end{align*}
    It is easy to see that $Q_{m,m,k}\chi_{S_1,\dots,S_k}=0$, since $S_1\cup\dots\cup S_k\neq\emptyset$.

    Now assume that $\bigcup_i S_i=\{a'\}$ for some $a'\in[m]$. Then it is clear that $Q_{m,m,k}\chi_{S_1,\dots,S_k}=0$, since $S_1\cup\dots\cup S_k\neq\emptyset$. We now compute
    \begin{align*}
        &Q_{m,m-1,k}\chi_{S_1,\dots,S_k}(X^1,\dots,X^k)\\
        =&\Ex_{(\mathbf{Y}^1,\dots,\mathbf{Y}^k)\sim\mathcal{C}^{m,m-1,k}_{X}}\sbra{\chi_{S_1,\dots,S_k}(\mathbf{Y}^1,\dots,\mathbf{Y}^k)}\\
        =&\frac1m\sum_{a\in[m]}\Ex_{\substack{(\mathbf{Y}^1,\dots,\mathbf{Y}^k)\sim\mathcal{C}^{m,m-1,k}_{X}\\\mathbf{a}=a}}\sbra{\prod_{\substack{i\in[k]\\S_i=\{a'\}}}\mathbf{Y}^k_{a'}}\\
        =&\frac1m\Ex_{\substack{(\mathbf{Y}^1,\dots,\mathbf{Y}^k)\sim\mathcal{C}^{m,m-1,k}_{X}\\\mathbf{a}=a'}}\sbra{\prod_{\substack{i\in[k]\\S_i=\{a'\}}}\mathbf{Y}^k_{a'}}\tag{If $\mathbf{a}\neq a'$ then the $\mathbf{Y}^i_{a'}$ are uniformly random.}\\
        =&\frac1m\Ex_{\substack{(\mathbf{Y}^1,\dots,\mathbf{Y}^k)\sim\mathcal{C}^{m,m-1,k}_{X}\\\mathbf{a}=a'}}\sbra{\prod_{\substack{i\in[k]\\S_i=\{a'\}}}X^k_{a'}}\\
        =&\frac1m\chi_{S_1,\dots,S_k}(X^1,\dots,X^k).
    \end{align*}
    Therefore $\pbra{Q_{m,m-1,k}-Q_{m,m,k}}\chi_{S_1,\dots,S_k}=\frac1m\chi_{S_1,\dots,S_k}(X^1,\dots,X^k)-0=\frac1m\chi_{S_1,\dots,S_k}(X^1,\dots,X^k)$.
\end{proof}

\begin{corollary}\label{cor:hybrid 2}
    Let $f:\{\pm1\}^{mk}\to\R$. Then
    \begin{align*}
        \abs{\left\langle f,(Q_{m,m-1,k}-Q_{m,m,k})f\right\rangle} \leq \frac1m \left\langle f,f\right\rangle.
    \end{align*}
\end{corollary}
\begin{proof}
    The $\chi_{S_1,\dots,S_k}$ form an orthonormal eigenbasis (\Cref{fact:fourier characters}) for $Q_{m,m-1,k}-Q_{m,m,k}$, and by \Cref{fact:characters under A1-R1} each basis element has eigenvalue with absolute value at most $\frac1m$.
\end{proof}

\subsubsection{The Third Hybrid: $Q_{m,m,k}$ to $R_{m,m,k}$}

\begin{lemma}\label{lem:hybrid 3}
    Assume that $k\leq 2^{m/3}$. For any $f:\{\pm1\}^{mk}\to\R$ supported on $B_{\geq 2}$ we have
    \begin{align*}
        \abs{\left\langle f,(R_{m,m,k}-Q_{m,m,k})f\right\rangle} \leq \frac{k^2}{2^{m}}\norm{f}_2^2 .
    \end{align*}
\end{lemma}
\begin{proof}
    As in the proof of \Cref{lem:hybrid 1} we find that (with the appropriate definitions of $p_0$ and $p_1$ for use of \Cref{lem:TV distance bound},
    \begin{align*}
       &\abs{\left\langle f,(R_{m,m,k}-Q_{m,m,k})f\right\rangle}\\
       \leq & \sum_{X\in B_{\geq 2}}f(X)^2\sum_{Y\in B_{\geq 2}}\abs{\Pr\sbra{X\to_{R_{m,m,k}} Y}-\Pr\sbra{X\to_{Q_{m,m,k}} Y}}\tag{\Cref{lem:TV distance bound} + self-adjointness (\Cref{fact:self-adjoint}, \Cref{fact:Q self-adjoint})}\\
        \leq & \frac{k^2}{2^m}\sum_{X\in B_{\geq 2}}f(X)^2\tag{\Cref{eq:TV for hybrid 3} below}\\
        \leq & \frac{k^2}{2^m}\left\langle f,f\right\rangle.
    \end{align*}
    It remains to prove \Cref{eq:TV for hybrid 3}:
    \begin{align*}
        &\sum_{Y\in B_{\geq 2}}\abs{\Pr\sbra{X\to_{R_{m,m,k}} Y}-\Pr\sbra{X\to_{Q_{m,m,k}} Y}}\\
        =&\sum_{Y\in B_{\geq 2}}\abs{\Pr_{\mathbf{Y}\sim\mathcal{D}^{m,m,k}_X}\sbra{\mathbf{Y}=Y}-\Pr_{\mathbf{Y}\sim\mathcal{C}^{m,m,k}_X}\sbra{\mathbf{Y}=Y}}\\
        =&\sum_{Y\in B_{\geq 2}}\abs{ \prod_{i=0}^{k-1}\frac{1}{2^{m}-i}- \frac1{2^{mk}}}\tag{$k\leq 2^{m/3}\leq 2^m-2$ and $X\in B_{\geq 2}$}\\
        \leq &\sum_{Y\in B_{\geq 2}}\abs{ \frac1{2^{mk}}\pbra{\prod_{i=0}^{k-1}\frac{2^{m}}{2^{m}-i}- 1}}\\
        \leq &\sum_{Y\in B_{\geq 2}}\abs{ \frac1{2^{mk}}\pbra{1+\frac{k^2}{2^m}- 1}}\tag{$k\leq 2^{m/3}$, \Cref{fact:k^2}}\\
        = &\sum_{Y\in B_{\geq 2}} \frac{k^2}{2^m2^{mk}}\\
        \leq &{2^{mk}}\cdot \frac{k^2}{2^m2^{mk}}\\
        =& \frac{k^2}{2^m}.\numberthis\label{eq:TV for hybrid 3}
    \end{align*}
    Note that our computations relied on the fact that $Y\in B_{\geq 2}$.
\end{proof}




\subsubsection{Putting Hybrids Together}
\begin{proposition}\label{prop:B>=2 square term}
    Assume that $k\leq 2^{m/3}$. For any $f:\{\pm1\}^{mk}\to\R$ supported on $B_{\geq 2}$, we have
    \begin{align*}
        \abs{\left\langle f,(R_{m,m-1,k}-R_{m,m,k})f\right\rangle} \leq \pbra{\frac1m+\frac{k^2}{2^{m/2}}}\left\langle f,f\right\rangle
    \end{align*}
\end{proposition}
\begin{proof}
    By the triangle inequality,
    \begin{align*}
        &\abs{\left\langle f,(R_{m,m-1,k}-R_{m,m,k})f\right\rangle} \\
        \leq& \abs{\left\langle f,\pbra{R_{m,m-1,k}-Q_{m,m-1,k}}f\right\rangle}+ \abs{\left\langle f,\pbra{Q_{m,m,k}-Q_{m,m-1,k}}f\right\rangle}+\abs{\left\langle f,\pbra{R_{m,m,k}-Q_{m,m,k}}f\right\rangle}\\
        \leq & \frac{k^2}{2^{m-1}}\left\langle f,f\right\rangle+\frac1m\left\langle f,f\right\rangle+\frac{k^2}{2^{m}}\left\langle f,f\right\rangle \tag{\Cref{lem:hybrid 1}, \Cref{cor:hybrid 2}, \Cref{lem:hybrid 3}}\\
        \leq  &\pbra{\frac1m+\frac{k^2}{2^{m-2}}}\left\langle f,f\right\rangle .
    \end{align*}
    Note we can apply \Cref{lem:hybrid 1} and \Cref{lem:hybrid 3} because $f$ is supported on $B_{\geq 2}$.
\end{proof}
