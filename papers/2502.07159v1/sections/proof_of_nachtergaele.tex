\section{Proof of \Cref{thm:nachtergaele hypothesis}}\label{sec:nachtergaele hypothesis}


In this section let $m$ and $k\leq 2^m-2$ be fixed positive integers. As in the proof of \Cref{thm:small k} (via its restatement \Cref{thm:small k internal}) we will break $\R^{\{\pm1\}^{mk}}$ into orthogonal components and bound the contributions of each of these cross terms to evaluation on the quadratic form given by $R_{m,[m-\ell-1,m],k}\pbra{R_{m,[m-1],k}-R_{m,[m],k}}$. 

Define the following subsets of $\{\pm1\}^{mk}$:
\begin{itemize}
    \item $B_{=0}=\cbra{X\in\{\pm1\}^{mk}:\exists i\neq j\in[k], X^i=X^j}$.\footnotemark\footnotetext{$B_{=0}$ was already defined in \Cref{sec:small k} but we define it again here for convenience.} 
    \item $B_{\geq1}^{\mathrm{coll}}=\cbra{X\in\{\pm1\}^{mk}\setminus B_{=0}:\exists i\neq j\in[k], X^i_{[m-\ell-1,m-1]}=X^j_{[m-\ell-1,m-1]}}$.
    \item $B_{\geq1}^{\mathrm{safe}}=\cbra{X\in\{\pm1\}^{mk}:\forall i\neq j\in[k], X^i_{[m-\ell-1,m-1]}\neq X^j_{[m-\ell-1,m-1]}}$.
\end{itemize}
Note that these sets form a partition of $\{\pm1\}^{mk}$.

As in the proof of \Cref{thm:small k}, the contribution from the parts of functions supported on $B_{=0}$ will be bounded by induction. The component $B_{\geq1}^{\mathrm{safe}}$ will play the role that $B_{\geq2}$ did: the part of the domain on which the noise model induced by random permutations (sampling without replacement) to the bits in $[m-\ell-1,m-1]$ is close to the noise model induced by completely random replacement of bits (sampling with replacement). $B_{\geq1}^{\mathrm{coll}}$ is the component on which these two noise models are not similar, but as in the case of $B_{=1}$, this set will already show good expansion.


The following is equivalent to \Cref{thm:nachtergaele hypothesis} because for any $f:\{\pm1\}^{mk}\to \R$ we have that
\begin{align*}
    \norm{R_{m,[m-\ell-1,m],k}\pbra{R_{m,[m-1],k}-R_{m,[m],k}}}_{\mathrm{op}}=&
    \norm{\pbra{R_{m,[m-\ell-1,m],k}\pbra{R_{m,[m-1],k}-R_{m,[m],k}}}^*}_{\mathrm{op}}\\
    =&\norm{\pbra{R_{m,[m-1],k}-R_{m,[m],k}}R_{m,[m-\ell-1,m],k}}_{\mathrm{op}}.
\end{align*}
To bound this quantity, we can use that
\begin{align*}
    &\norm{\pbra{R_{m,[m-1],k}-R_{m,[m],k}}R_{m,[m-\ell-1,m],k}f}^2_2\\
    =&\left\langle \pbra{R_{m,[m-1],k}-R_{m,[m],k}}R_{m,[m-\ell-1,m],k}f,\pbra{R_{m,[m-1],k}-R_{m,[m],k}}R_{m,[m-\ell-1,m],k}f\right\rangle\\
    =&\left\langle R_{m,[m-\ell-1,m],k}\pbra{R_{m,[m-1],k}-R_{m,[m],k}}\pbra{R_{m,[m-\ell-1,m],k}\pbra{R_{m,[m-1],k}-R_{m,[m],k}}}^*f,f\right\rangle\\
    =&\left\langle R_{m,[m-\ell-1,m],k}\pbra{R_{m,[m-1],k}-R_{m,[m],k}}^2R_{m,[m-\ell-1,m],k}f,f\right\rangle\\
    =&\left\langle R_{m,[m-\ell-1,m],k}\pbra{R_{m,[m-1],k}-R_{m,[m],k}-R_{m,[m],k}+R_{m,[m],k}}R_{m,[m-\ell-1,m],k}f,f\right\rangle\\
    =&\left\langle f,R_{m,[m-\ell-1,m],k}\pbra{R_{m,[m-1],k}-R_{m,[m],k}}R_{m,[m-\ell-1,m],k}f\right\rangle.
\end{align*}

\begin{theorem}[\Cref{thm:nachtergaele hypothesis} restated]\label{thm:nachtergaele hypothesis internal}
    Fix any $m\geq100$ and set $100\leq\ell\leq m$. Suppose $k\leq 2^{\ell/10}$. Then we have for any $f:\{\pm1\}^{mk}\to\R$ that
    \begin{align*}
        \abs{\left\langle f,R_{m,[m-\ell-1,m],k}\pbra{R_{m,[m-1],k}-R_{m,[m],k}}R_{m,[m-\ell-1,m],k}f\right\rangle}\leq \frac{k^3}{2^{\ell/4-60}}\left\langle f,f\right\rangle\leq \frac1{\ell^2} \left\langle f,f\right\rangle.
    \end{align*}
\end{theorem}
\begin{proof}
    We prove by induction on $k$. In the base case $k=1$ and the result holds by the following argument when we write $f=f_2$. Now assume that the result holds for real functions on $\{\pm1\}^{m(k-1)}$. 
    
    Let $f:\{\pm1\}^{mk}\to\R$. Write $f=f_0+f_1+f_2$ where $f_0$ is supported on $B_{=0}$, $f_1$ is supported on $B_{\geq1}^{\mathrm{coll}}$, and $f_2$ is supported on $B_{\geq1}^{\mathrm{safe}}$. By \Cref{lem:cross terms B0} applied with $R_{m,[m-\ell-1,m],k}$, $R_{m,[m-1],k}$ and $R_{m,[m],k}$ and $B_{=0}$ and $\{\pm1\}^{mk}\setminus B_{=0}=B_{\geq1}^{\mathrm{coll}}\cup B_{\geq1}^{\mathrm{safe}}$, the other cross terms vanish, and we have
    \begin{align*}
        &\abs{ \left\langle f,R_{m,[m-\ell-1,m],k}\pbra{R_{m,[m-1],k}-R_{m,[m],k}}R_{m,[m-\ell-1,m],k}f\right\rangle }\\
        =&\abs{ \left\langle f,Af\right\rangle }\tag{Defining $A$ for convenience of notation}\\
        \leq&\abs{ \left\langle f_0,Af_0\right\rangle }+\abs{ \left\langle f_2,Af_2\right\rangle }+\abs{ \left\langle f_1,A(f_1+f_2)\right\rangle }+\abs{ \left\langle f_2,Af_1\right\rangle }\\
        =&\abs{ \left\langle f_0,Af_0\right\rangle }+\abs{ \left\langle f_2,Af_2\right\rangle }+\abs{ \left\langle f_1,A(f_1+f_2)\right\rangle }+\abs{ \left\langle f_1,A^*f_2\right\rangle }\\
        \leq&\frac{(k-1)^3}{2^{\ell/4-60}}\left\langle f_0,f_0 \right\rangle+\frac{k^2}{2^{\ell/4-20}}\langle f_2,f_2\rangle  +\frac{k^2}{2^{\ell/2-4}}\norm{f_1+f_2}_2\norm{f_2}_2+\frac{k^2}{2^{\ell/2-4}}\norm{f_1}_2\norm{f_2}_2\tag{\Cref{lemma:f supported on B0} + Induction, \Cref{cor:square terms safe to safe}, \Cref{cor:cross terms safe to coll}, \Cref{cor:cross terms safe to coll}, in that order}\\
        \leq & \frac{k^3}{2^{\ell/4-60}}\langle f,f\rangle.
    \end{align*}
    In the last step we used that $k\leq 2^{\ell/10}$.
\end{proof}




Similar to \Cref{sec:B>=2} we will again argue that because $\ell$ is large, applying a random permutation to the same indices of all elements of a tuple of binary strings is similar to replacing those indices with random binary strings in all elements of a tuple. To model this situation, we similarly use the matrix $Q_{m,S,k}$ to denote the random walk matrix induced by the following distribution $\mathcal{C}_X^{m,S,k}$ for each $X\in\{\pm1\}^{mk}$. To draw $\mathbf{Y}\sim\mathcal{C}_X^{m,S,k}$, set $\mathbf{Y}^i_{[m]\setminus S}=X^i_{[m]\setminus S}$ and set $\mathbf{Y}^i_S$ uniformly at random for all $i\in[k]$.

\subsection{$f$ Supported on $B_{\geq 1}^{\mathrm{coll}}$ and Cross Terms}\label{sec:coll and cross terms}

The first step in comparing the noise model generated by the application of a random permutation to a substring of each string in a $k$-tuple of $n$-bit strings is showing that the set of strings to which this comparison fails already exhibits good expansion.

In this section we leverage that $B_{\geq1}^{\mathrm{coll}}$ is a set in $\{\pm1\}^{mk}$ with very good expansion. For intuition, $B_{\geq1}^{\mathrm{coll}}$ plays the role that $B_{=1}$ played in \Cref{sec:small k}. The following lemma formalizes the expansion property that we need.

\begin{lemma}\label{lem:anything transition into coll}
    For any $X\in B_{\geq 1}^{\mathrm{coll}}\cup B_{\geq 1}^{\mathrm{safe}}$ we have that for any $I,J\subseteq[m]$ such that $I\cup J=[m]$ and $I,J\supseteq [m-\ell-1,m-1]$,
    \begin{align*}
        \Pr\sbra{X\to_{R_{m,I,k}R_{m,J,k}} B_{\geq 1}^{\mathrm{coll}}}
        \leq\frac{k^2}{2^{\ell-3}}.
    \end{align*}
    Moreover, we also have
    \begin{align*}
        \Pr\sbra{X\to_{R_{m,I,k}R_{m,J,k}R_{m,I,k}} B_{\geq 1}^{\mathrm{coll}}}
        \leq\frac{k^2}{2^{\ell-3}}.
    \end{align*}
    In fact, this holds with $Q$ replacing $R$ anywhere in this inequality.
\end{lemma}
\begin{proof}
    We first prove the first statement (with two steps) for the case of drawing from the distribution $\mathcal{D}$ (corresponding to the random walk operator $R$), but the proof translates easily to the case for $\mathcal{C}$ (corresponding to the random wak operator $Q$). Model the process as first drawing $\bm{Y}\sim \mcD^{m,I,k}_X$ and then $\bm{Z}\sim \mcD^{m,J,k}_{\bm{Y}}$. Fix any $X\in B_{\geq1}^{\mathrm{coll}}\cup B_{\geq1}^{\mathrm{safe}}$. Define the following subset of pairs of indices in $\binom{[k]}{2}$:
    \begin{align*}
        P_1=& \cbra{\{i,j\}\in\binom{[k]}{2}:X^i_{I}=X^j_{I}}.
    \end{align*}
    Assume that $\{i,j\}\in P_1$. Then $\mathbf{Y}^i_I=\mathbf{Y}^j_I$ with probability 1. But now note that $X^i_{J}\neq X^j_{J}$ since otherwise $X^i=X^j$, which contradicts that $X\not\in B_{=0}$. Therefore, 
    \begin{align*}
        &\Pr_{\substack{\mathbf{Y}\sim \mathcal{D}_{X}^{m,I,k}\\\mathbf{Z}\sim\mathcal{D}_{\mathbf{Y}}^{m,J,k}}}\sbra{\mathbf{Z}^i_{[m-\ell-1,m-1]}=\mathbf{Z}^j_{[m-\ell-1,m-1]}}\\
        =&\Pr_{\substack{\mathbf{Z}\sim \mathcal{D}_{X}^{m,J,k}}}\sbra{\mathbf{Z}^i_{[m-\ell-1,m-1]}=\mathbf{Z}^j_{[m-\ell-1,m-1]}}\\
        =& \frac1{2^{\ell}-1}.\numberthis\label{eq:P1 contribution}
    \end{align*}
    Now assume that $\{i,j\}\not\in P_1$. Let $E$ be the event that $\mathbf{Y}^i_{[m-\ell-1,m-1]}=\mathbf{Y}^j_{[m-\ell-1,m-1]}$. Then 
    \begin{align*}
        &\Pr_{\substack{\mathbf{Y}\sim \mathcal{D}_{X}^{m,I,k}\\\mathbf{Z}\sim\mathcal{D}_{\mathbf{Y}}^{m,J,k}}}\sbra{\mathbf{Z}^i_{[m-\ell-1,m-1]}=\mathbf{Z}^j_{[m-\ell-1,m-1]}}\\
        =&\Pr_{\mathbf{Y}\sim \mathcal{D}_{X}^{m,I,k}}\sbra{E}\Pr_{\substack{\mathbf{Y}\sim \mathcal{D}_{X}^{m,I,k}\\\mathbf{Z}\sim\mathcal{D}_{\mathbf{Y}}^{m,J,k}}}\sbra{\mathbf{Z}^i_{[m-\ell-1,m-1]}=\mathbf{Z}^j_{[m-\ell-1,m-1]}|E}\\
        &+\Pr_{\mathbf{Y}\sim \mathcal{D}_{X}^{m,I,k}}\sbra{\overline{E}}\Pr_{\substack{\mathbf{Y}\sim \mathcal{D}_{X}^{m,I,k}\\\mathbf{Z}\sim\mathcal{D}_{\mathbf{Y}}^{m,J,k}}}\sbra{\mathbf{Z}^i_{[m-\ell-1,m-1]}=\mathbf{Z}^j_{[m-\ell-1,m-1]}|\overline{E}}\\
        \leq& \frac1{2^\ell} + \Pr_{\substack{\mathbf{Y}\sim \mathcal{D}_{X}^{m,I,k}\\\mathbf{Z}\sim\mathcal{D}_{\mathbf{Y}}^{m,J,k}}}\sbra{\mathbf{Z}^i_{[m-\ell-1,m-1]}=\mathbf{Z}^j_{[m-\ell-1,m-1]}|\overline{E}}\\
        =& \frac1{2^\ell}+\frac1{2^\ell-1}\\
        \leq&\frac1{2^{\ell-2}}.\numberthis\label{eq:not P1 contribution}
    \end{align*}
    To complete the proof, we use a union bound and find
    \begin{align*}
        &\Pr_{\substack{\mathbf{Y}\sim \mathcal{D}_{X}^{m,I,k}\\\mathbf{Z}\sim\mathcal{D}_{\mathbf{Y}}^{m,J,k}}}\sbra{\mathbf{Z}\in B_{\geq1}^{\mathrm{coll}}}\\
        \leq& \sum_{\{i,j\}\in\binom{[k]}{2}}\Pr_{\substack{\mathbf{Y}\sim \mathcal{D}_{X}^{m,I,k}\\\mathbf{Z}\sim\mathcal{D}_{\mathbf{Y}}^{m,J,k}}}\sbra{\mathbf{Z}^i_{[m-\ell-1,m-1]}=\mathbf{Z}^j_{[m-\ell-1,m-1]}}\\
        \leq&\sum_{\{i,j\}\in P_1}\Pr_{\substack{\mathbf{Y}\sim \mathcal{D}_{X}^{m,I,k}\\\mathbf{Z}\sim\mathcal{D}_{\mathbf{Y}}^{m,J,k}}}\sbra{\mathbf{Z}^i_{[m-\ell-1,m-1]}=\mathbf{Z}^j_{[m-\ell-1,m-1]}}\\
        &+\sum_{\{i,j\}\not\in P_1}\Pr_{\substack{\mathbf{Y}\sim \mathcal{D}_{X}^{m,I,k}\\\mathbf{Z}\sim\mathcal{D}_{\mathbf{Y}}^{m,J,k}}}\sbra{\mathbf{Z}^i_{[m-\ell-1,m-1]}=\mathbf{Z}^j_{[m-\ell-1,m-1]}}\\
        \leq& \sum_{\{i,j\}\in P_1}\frac1{2^{\ell}}+\sum_{\{i,j\}\not\in P_1}\frac1{2^{\ell-1}}\\
        \leq&\sum_{\{i,j\}\in \binom{[k]}{2}}\frac1{2^{\ell-2}}\\
        \leq& \frac{k^2}{2^{\ell-2}}\numberthis\label{eq:two step bound}.
    \end{align*}
    To prove the second statement (with three steps taken), note that the probability that a random permutation applied to the bits in $I$ sends an element of $B_{\geq1}^{\mathrm{safe}}$ to an element of $B_{\geq1}^{\mathrm{coll}}$ is at most $\frac{k^2}{2^{\ell}-1}$. Applying a union bound and the bound \Cref{eq:two step bound} completes the proof.
\end{proof}


The expansion property just proved results in the following linear-algebraic statements, which show that the contributions from $B_{\geq1}^{\mathrm{coll}}$ to our operator norm bound is extremely small.


\begin{lemma}\label{lem:randomize everything}
    Let $k\geq 2$ and $f:\{\pm1\}^{mk}$ be supported on $B_{\geq 1}^{\mathrm{coll}}$ and $g:\{\pm1\}^{mk}$ be supported on $B_{\geq 1}^{\mathrm{safe}}\cup B_{\geq 1}^{\mathrm{coll}}$. Let $I$ and $J$ be such that $I\cup J=[m]$. Then
    \begin{align*}
        &\abs{\left\langle f,R_{m,J,k}R_{m,I,k}R_{m,J,k}g\right\rangle}      \leq\sqrt{\frac{k^2}{2^{\ell-3}}}\norm{f}_2\norm{g}_2,\\
        &\abs{\left\langle f,R_{m,J,k}R_{m,I,k}g\right\rangle}\leq\sqrt{\frac{k^2}{2^{\ell-3}}}\norm{f}_2\norm{g}_2.
    \end{align*}
    Moreover, this holds with $Q$ in place of $R$ anywhere.
\end{lemma}
\begin{proof}
    The inequality directly follows from \Cref{lem:escape probs} and \Cref{lem:anything transition into coll}. We can apply \Cref{lem:escape probs} because the uniform distribution on $\{\pm1\}^{mk}$ is indeed a stationary distribution under $R_{m,S,k}$ for any $S$ (\Cref{fact:uniform is stationary}). The $Q$ case follows from the fact that \Cref{lem:anything transition into coll} applies for the distributions $\mathcal{C}$ too.
\end{proof}

\begin{corollary}\label{cor:cross terms safe to coll}
    Let $k\geq 2$ and $f:\{\pm1\}^{mk}$ be supported on $B_{\geq 1}^{\mathrm{coll}}$ and $g:\{\pm1\}^{mk}$ be supported on $B_{\geq 1}^{\mathrm{safe}}\cup B_{\geq 1}^{\mathrm{coll}}$. Then 
    \begin{align*}
        \abs{\left\langle f,R_{m,[m-\ell-1,m],k}\pbra{R_{m,[m-1],k}-R_{m,[m],k}}R_{m,[m-\ell-1,m],k}g\right\rangle }\leq \frac{k}{2^{\ell/2-4}}\norm{f}_2\norm{g}_2,\\
        \abs{\left\langle f,R_{m,[m-\ell-1,m],k}\pbra{R_{m,[m-1],k}-R_{m,[m],k}}g\right\rangle }\leq \frac{k}{2^{\ell/2-4}}\norm{f}_2\norm{g}_2.
    \end{align*}
\end{corollary}
\begin{proof}
    We compute
    \begin{align*}
        &\abs{\left\langle f,R_{m,[m-\ell-1,m],k}\pbra{R_{m,[m-1],k}-R_{m,[m],k}}R_{m,[m-\ell-1,m],k}g\right\rangle }\\
        \leq&\abs{\left\langle f,{R_{m,[m-\ell-1,m],k}R_{m,[m-1],k}}R_{m,[m-\ell-1,m],k}g\right\rangle}+\abs{\left\langle f,{R_{m,[m-\ell-1,m],k}R_{m,[m],k}}R_{m,[m-\ell-1,m],k}g\right\rangle }\\
        =&\abs{\left\langle f,{R_{m,[m-\ell-1,m],k}R_{m,[m-1],k}}R_{m,[m-\ell-1,m],k}g\right\rangle}+\abs{\left\langle f,{R_{m,[m],k}}g\right\rangle }\\
        \leq&\frac{k}{2^{\ell/2-3}}\norm{f}_2\norm{g}_2+\frac{k}{2^{\ell/2}}\norm{f}_2\norm{g}_2\tag{\Cref{lem:randomize everything}}\\
        \leq& \frac{k}{2^{\ell/2-4}}\norm{f}_2\norm{g}_2.
    \end{align*}
    The proof of the second statement is similar.
\end{proof}






\subsection{A Hybrid Argument for $f$ Supported on $B_{\geq 1}^{\mathrm{safe}}$}

The role that $B_{\geq 1}^{\mathrm{safe}}$ plays in this section is similar to the role played by $B_{\geq2}$ in \Cref{sec:small k}. It is the region of $\{\pm1\}^{mk}$ that is ``well-behaved" in the sense that the nicer noise model given by the $Q$ operators is similar to the noise model given by the $R$ operators on this region of $\{\pm1\}^{mk}$. 

In particular, because these noise models behave similarly when restricted to these $k$-tuples, we can bound the difference between the corresponding transition matrices as follows:

\begin{lemma}\label{lem:R to Q hybrid local}
    Assume that $k\leq 2^{\ell/10}$ and $f,g:\{\pm1\}^{mk}$ be supported on $B_{\geq 1}^{\mathrm{safe}}$. Then for any $S\supseteq [m-\ell-1,m-1]$, we have
    \begin{align*}
        \abs{\left\langle f,\pbra{Q_{m,S,k}-R_{m,S,k}}g\right\rangle}\leq \frac{k^2}{2^{\ell-1}}\norm{f}_2\norm{g}_2.
    \end{align*}
\end{lemma}
\begin{proof}
    By assumption $f$ and $g$ are supported on $B_{\geq1}^{\mathsf{safe}}$. Therefore, by \Cref{lem:TV distance bound} and self-adjointness of $Q_{m,S,k}$ and $R_{m,S,k}$ (\Cref{fact:self-adjoint}, \Cref{fact:Q self-adjoint}) we have
    \begin{align*}
        &\abs{\left\langle f,(R_{m,S,k}-Q_{m,S,k})g\right\rangle}\\
        \leq & \sqrt{\sum_{X\in B_{\geq 1}^{\mathrm{safe}}}f(X)^2\sum_{Y\in B_{\geq 1}^{\mathrm{safe}}}\abs{\Pr\sbra{X\to_{R_{m,S,k}} Y}-\Pr\sbra{X\to_{Q_{m,S,k}} Y}}}\\
        &\;\;\;\;\;\cdot\sqrt{\sum_{X\in B_{\geq 1}^{\mathrm{safe}}}g(X)^2\sum_{Y\in B_{\geq 1}^{\mathrm{safe}}}\abs{\Pr\sbra{X\to_{R_{m,S,k}} Y}-\Pr\sbra{X\to_{Q_{m,S,k}} Y}}}\\
        \leq & \frac{k^2}{2^{\ell-1}}\sqrt{\sum_{X\in B_{\geq 1}^{\mathrm{safe}}}f(X)^2}\sqrt{\sum_{X\in B_{\geq 1}^{\mathrm{safe}}}g(X)^2}\tag{\Cref{eq:tv for safe} below}\\
        =& \frac{k^2}{2^{\ell-1}}\norm{f}_2\norm{g}_2.
    \end{align*}
Here $p_0$ and $p_1$ are as used below. Now it suffices to establish \Cref{eq:tv for safe}. Assume $X\in B_{\geq 1}^{\mathrm{safe}}$. Then because $S\supseteq [m-\ell-1,m-1]$, we know that for all $i\neq j$ we have $X^i_{S}\neq X^j_S$.
\begin{align*}
    &\sum_{Y\in B_{\geq 1}^{\mathrm{safe}}}\abs{\Pr\sbra{X\to_{R_{m,S,k}} Y}-\Pr\sbra{X\to_{Q_{m,S,k}} Y}}\\
    =&\sum_{\substack{Y\in B_{\geq 1}^{\mathrm{safe}}\\\forall i\in[k], a\in [m]\setminus S, Y^i_a=X^i_a}}\abs{\Pr_{\mathbf{Y}\sim\mathcal{D}^{m,S,k}_X}\sbra{\mathbf{Y}=Y}- \Pr_{\mathbf{Y}\sim\mathcal{C}^{m,S,k}_X}\sbra{\mathbf{Y}=Y}}\\
    =&\sum_{\substack{Y\in B_{\geq 1}^{\mathrm{safe}}\\\forall i\in[k], a\in [m]\setminus S, Y^i_a=X^i_a}}\abs{ \prod_{j=0}^{k-1}\frac{1}{2^{|S|}-j}- \frac1{2^{|S|k}}}\tag{$k\leq 2^{m/3}\leq 2^m-2$ and $X,Y\in B_{\geq 1}^{\mathrm{safe}}$}\\
    \leq &\sum_{\substack{Y\in B_{\geq 1}^{\mathrm{safe}}\\\forall i\in[k], a\in [m]\setminus S, Y^i_a=X^i_a}}\abs{ \frac1{2^{|S|k}}\pbra{\prod_{j=0}^{k-1}\frac{2^{|S|}}{2^{|S|}-j}- 1}}\\
    \leq &\sum_{\substack{Y\in B_{\geq 1}^{\mathrm{safe}}\\\forall i\in[k], a\in [m]\setminus S, Y^i_a=X^i_a}}\abs{ \frac1{2^{|S|k}}\pbra{1+\frac{k^2}{2^{|S|}}- 1}}\tag{$k^2\leq 2^{\ell}\leq 2^{|S|}$, \Cref{fact:k^2}}\\
    = &\sum_{\substack{Y\in B_{\geq 1}^{\mathrm{safe}}\\\forall i\in[k], a\in [m]\setminus S, Y^i_a=X^i_a}}\frac{k^2}{2^{|S|}2^{|S|k}}\\
    = &\cdot 2^{|S|k}\cdot \frac{k^2}{2^{|S|}2^{|S|k}}\\
    =& \frac{k^2}{2^{|S|}}\\
    \leq& \frac{k^2}{2^{\ell-1}}.\numberthis\label{eq:tv for safe}
\end{align*}
The last inequality follows because $|S|\geq\abs{[m-\ell-1,m-1]}=\ell-1$. Having established \Cref{eq:tv for safe}, we have completed the proof.
\end{proof}





At this point it may seem like we are essentially finished with the proof, since we should just replace the $R$ operators in the expression $\left\langle f,{R_{m,[m-\ell-1,m],k}\pbra{R_{m,[m-1],k}-R_{m,[m],k}}}R_{m,[m-\ell-1,m],k}f\right\rangle$ with the corresponding $Q$ operators and finish the proof. However, we don't quite show an upper bound on the \textit{operator norm} of $R-Q$. Rather, we simply show that they are close on the well-behaved region. A priori, this gives us no information about how products of these operators may behave, since the first term in the product may ``rotate" vectors into the badly-behaved region. So a straightforward application of the triangle inequality fails.

However, we observe that we have already shown in \Cref{sec:coll and cross terms} that random walks starting outside the badly-behaved region rarely transition into it, so that we almost can pretend as if all of these operators are operators on $\R^{B_{\geq1}^{\mathrm{safe}}}$. We formalize this by inserting projections to the space of functions supported on $B_{\geq1}^{\mathrm{safe}}$, and showing that this move does very little quantitatively.


\begin{lemma}\label{lem:product of operators quadratic form}
    We have for any $f,g:\{\pm1\}^{mk}\to\R$ supported on $B_{\geq1}^{\mathrm{safe}}$ that
    \begin{align*}
        \abs{\left\langle f,\pbra{R_{m,[m-\ell-1,m],k}-Q_{m,[m-\ell-1,m],k}}\pbra{R_{m,[m-1],k}-R_{m,[m],k}}g\right\rangle }\leq \frac{k^2}{2^{\ell/4-20}}\norm{f}_2\norm{g}_2.
    \end{align*}
    Moreover, we have for such $f$ and $g$ that
    \begin{align*}
        \abs{\left\langle f,Q_{m,[m-\ell-1,m],k}\pbra{R_{m,[m-1],k}-Q_{m,[m-1],k}-R_{m,[m],k}+Q_{m,[m],k}}g\right\rangle }\leq \frac{k^2}{2^{\ell/4-20}}\norm{f}_2\norm{g}_2.
    \end{align*}
\end{lemma}
\begin{proof}
    Let $\Pi_{\mathrm{safe}}$ be the projection to $\{h:\{\pm1\}^{mk}\to\R:h\text{ supported on }B_{\geq1}^{\mathrm{safe}}\}$. We directly compute
    \begin{align*}
        &\abs{\left\langle f,\pbra{R_{m,[m-\ell-1,m],k}-Q_{m,[m-\ell-1,m],k}}R_{m,[m-1],k}g\right\rangle }\\
        \leq&\abs{\left\langle f,\pbra{R_{m,[m-\ell-1,m],k}-Q_{m,[m-\ell-1,m],k}}\pbra{\Id-\Pi_{\mathrm{safe}}}R_{m,[m-1],k}g\right\rangle }\\
        &\;\;+\abs{\left\langle f,\pbra{R_{m,[m-\ell-1,m],k}-Q_{m,[m-\ell-1,m],k}}\Pi_{\mathrm{safe}}R_{m,[m-1],k}g\right\rangle}\\
        \leq&\abs{\left\langle f,\pbra{R_{m,[m-\ell-1,m],k}-Q_{m,[m-\ell-1,m],k}}\pbra{\Id-\Pi_{\mathrm{safe}}}R_{m,[m-1],k}g\right\rangle }+\frac{k^2}{2^{\ell-1}}\norm{f}_2\norm{\Pi_{\mathrm{safe}}R_{m,[m-1],k}g}_2\tag{\Cref{lem:R to Q hybrid local}}\\
        \leq&\abs{\left\langle f,R_{m,[m-\ell-1,m],k}\pbra{\Id-\Pi_{\mathrm{safe}}}R_{m,[m-1],k}g\right\rangle}+\abs{\left\langle f,Q_{m,[m-\ell-1,m],k}\pbra{\Id-\Pi_{\mathrm{safe}}}R_{m,[m-1],k}g\right\rangle}+\frac{k^2}{2^{\ell-1}}\norm{f}_2\norm{g}_2\\
        \leq&2\norm{f}_2\norm{\pbra{\Id-\Pi_{\mathrm{safe}}}R_{m,[m-1],k}g}_2+\frac{k^2}{2^{\ell-1}}\norm{f}_2\norm{g}_2\\
        \leq&\frac{k^2}{2^{\ell/4-10}}\norm{f}_2\norm{g}_2+\frac{k^2}{2^{\ell-1}}\norm{f}_2\norm{g}_2.\tag{\Cref{eq:safe to not safe R} below}
    \end{align*}
    Here our application of the \Cref{lem:R to Q hybrid local} depended on the fact that $\mathrm{Supp}(\Pi_{\mathrm{safe}}R_{m,[m-1],k}f)\subseteq B_{\geq1}^{\mathrm{safe}}$ and $\mathrm{Supp}((\Id-\Pi_{\mathrm{safe}})R_{m,[m-1],k}f)\subseteq \{\pm1\}^{mk}\setminus B_{\geq1}^{\mathrm{safe}}$.
    
    To establish \Cref{eq:safe to not safe R} we compute 
    \begin{align*}
        &\norm{\pbra{\Id-\Pi_{\mathrm{safe}}}R_{m,[m-1],k}g}_2^2\\
        =&\left\langle \pbra{\Id-\Pi_{\mathrm{safe}}}R_{m,[m-1],k}g,\pbra{\Id-\Pi_{\mathrm{safe}}}R_{m,[m-1],k}g\right\rangle\\
        =&\left\langle R_{m,[m-1],k}g,\pbra{\Id-\Pi_{\mathrm{safe}}}^2R_{m,[m-1],k}g\right\rangle\\
        =&\left\langle f,R_{m,[m-1],k}\pbra{\Id-\Pi_{\mathrm{safe}}}R_{m,[m-1],k}g\right\rangle\tag{self-adjointness (\Cref{fact:self-adjoint})}\\
        \leq &\frac{k}{2^{\ell/2-1}}\norm{g}_2\norm{\pbra{\Id-\Pi_{\mathrm{safe}}}R_{m,[m-1],k}g}_2\\
        \leq&\frac{k}{2^{\ell/2-1}}\norm{g}_2^2\numberthis\label{eq:safe to not safe R}.
    \end{align*}
    The first inequality follows from \Cref{lem:escape probs} the fact that $\mathrm{Supp}(f)\subseteq B_{\geq1}^{\mathrm{safe}}$ and for any $X\in B_{\geq1}^{\mathrm{safe}}$, we have $\Pr\sbra{X\to_{R_{m,[m-1],k}}B_{\geq1}^{\mathrm{coll}}}\leq \frac{k^2}{2^{\ell-1}}$ (using the same proof as \Cref{lem:anything transition into coll}), and that $\mathrm{Supp}\pbra{\pbra{\Id-\Pi_{\mathrm{safe}}}R_{m,[m-1],k}f}\subseteq \{\pm1\}^{mk}\setminus B_{\geq1}^{\mathrm{safe}}$. Bounding the similar quantity but with $R_{m,[m],k}$ instead of $R_{m,[m-1],k}$ is the same and the first part of the lemma statement follows from the triangle inequality. 
    
    The second part of the lemma statement follows from the same argument and an application of the triangle inequality:
    \begin{align*}
        &\abs{\left\langle f,Q_{m,[m-\ell-1,m],k}\pbra{R_{m,[m-1],k}-Q_{m,[m-1],k}-R_{m,[m],k}+Q_{m,[m],k}}g\right\rangle}\\
        =&\abs{\left\langle f,\pbra{R_{m,[m-1],k}-Q_{m,[m-1],k}-R_{m,[m],k}+Q_{m,[m],k}}Q_{m,[m-\ell-1,m],k}g\right\rangle}\tag{\Cref{fact:self-adjoint}, \Cref{fact:Q self-adjoint}}\\
        \leq&\abs{\left\langle f,\pbra{R_{m,[m-1],k}-Q_{m,[m-1],k}}Q_{m,[m-\ell-1,m],k}g\right\rangle}+\abs{\left\langle f,\pbra{R_{m,[m],k}-Q_{m,[m],k}}Q_{m,[m-\ell-1,m],k}g\right\rangle}.\numberthis\label{eq:hybrid the difference}
    \end{align*}
    We show how to bound the first term in this sum:
    \begin{align*}
        &\abs{\left\langle f,\pbra{R_{m,[m-1],k}-Q_{m,[m-1],k}}Q_{m,[m-\ell-1,m],k}g\right\rangle}\\
        \leq&\abs{\left\langle f,\pbra{R_{m,[m-1],k}-Q_{m,[m-1],k}}\pbra{\Id-\Pi_{\mathrm{safe}}}Q_{m,[m-\ell-1,m],k}g\right\rangle }\\
        &\;\;+\abs{\left\langle f,\pbra{R_{m,[m-1],k}-Q_{m,[m-1],k}}\Pi_{\mathrm{safe}}Q_{m,[m-\ell-1,m],k}g\right\rangle}\\
        \leq&\abs{\left\langle f,\pbra{R_{m,[m-1],k}-Q_{m,[m-1],k}}\pbra{\Id-\Pi_{\mathrm{safe}}}Q_{m,[m-\ell-1,m],k}g\right\rangle }+\frac{k^2}{2^{\ell-1}}\norm{f}_2\norm{\Pi_{\mathrm{safe}}Q_{m,[m-\ell-1,m],k}g}_2\tag{\Cref{lem:R to Q hybrid local}}\\
        \leq&\abs{\left\langle f,R_{m,[m-1],k}\pbra{\Id-\Pi_{\mathrm{safe}}}Q_{m,[m-\ell-1,m],k}g\right\rangle }\\
        &+\abs{\left\langle f,Q_{m,[m-1],k}\pbra{\Id-\Pi_{\mathrm{safe}}}Q_{m,[m-\ell-1,m],k}g\right\rangle }+\frac{k^2}{2^{\ell-1}}\norm{f}_2\norm{g}_2\\
        \leq&2\norm{f}_2\norm{\pbra{\Id-\Pi_{\mathrm{safe}}}Q_{m,[m-\ell-1,m],k}g}_2+\frac{k^2}{2^{\ell-1}}\norm{f}_2\norm{g}_2\\
        \leq&\frac{k}{2^{\ell/4-10}}\norm{f}_2\norm{g}_2+\frac{k^2}{2^{\ell-1}}\norm{f}_2\norm{g}_2.\tag{\Cref{eq:safe to not safe Q} below}
    \end{align*}
    To establish \Cref{eq:safe to not safe Q} we compute 
    \begin{align*}
        &\norm{\pbra{\Id-\Pi_{\mathrm{safe}}}Q_{m,[m-\ell-1,m],k}g}_2^2\\
        =&\left\langle g,Q_{m,[m-\ell-1,m],k}\pbra{\Id-\Pi_{\mathrm{safe}}}Q_{m,[m-\ell-1,m],k}g\right\rangle\tag{self-adjointness (\Cref{fact:Q self-adjoint}}\\
        \leq &\frac{k}{2^{\ell/2-1}}\norm{g}_2\norm{\pbra{\Id-\Pi_{\mathrm{safe}}}Q_{m,[m-\ell-1,m],k}g}_2\\
        \leq&\frac{k}{2^{\ell/2-1}}\norm{g}_2^2\numberthis\label{eq:safe to not safe Q}.
    \end{align*}
    The first inequality follows from \Cref{lem:escape probs} the fact that $\mathrm{Supp}(f)\subseteq B_{\geq1}^{\mathrm{safe}}$ and for any $X\in B_{\geq1}^{\mathrm{safe}}$, we have $\Pr\sbra{X\to_{Q_{m,[m-1],k}}B_{\geq1}^{\mathrm{coll}}}\leq \frac{k^2}{2^{\ell-1}}$ (using the same proof as \Cref{lem:anything transition into coll}). The bound on the second term in the sum \Cref{eq:hybrid the difference} follows from the same argument.
\end{proof}




\begin{corollary}\label{cor:square terms safe to safe}
    Let $k\geq 2$ and $f:\{\pm1\}^{mk}\to\R$ be supported on $B_{\geq 1}^{\mathrm{safe}}$. Then 
    \begin{align*}
        \abs{\left\langle f,R_{m,[m-\ell-1,m],k}\pbra{R_{m,[m-1],k}-R_{m,[m],k}}R_{m,[m-\ell-1,m],k}f\right\rangle }\leq \frac{k^2}{2^{\ell/4-50}}\langle f,f\rangle.
    \end{align*}
\end{corollary}
\begin{proof}
    Let $g=R_{m,[m-\ell-1,m],k}f$ and we can write $g=g_1+g_2$ where $g_1=\Pi_{\mathrm{safe}}g$ and $g_2=\pbra{\mathrm{Id}-\Pi_{\mathrm{safe}}}g$. Then
    \begin{align*}
        &\abs{\left\langle f,R_{m,[m-\ell-1,m],k}\pbra{R_{m,[m-1],k}-R_{m,[m],k}}R_{m,[m-\ell-1,m],k}f\right\rangle }\\
        \leq&\abs{\left\langle f,R_{m,[m-\ell-1,m],k}\pbra{R_{m,[m-1],k}-R_{m,[m],k}}g_1\right\rangle }+\abs{\left\langle f,R_{m,[m-\ell-1,m],k}\pbra{R_{m,[m-1],k}-R_{m,[m],k}}g_2\right\rangle }\\
        \leq&\abs{\left\langle f,R_{m,[m-\ell-1,m],k}\pbra{R_{m,[m-1],k}-R_{m,[m],k}}g_1\right\rangle }+\frac{k^2}{2^{\ell/2-3}}\norm{f}_2\norm{g_2}_2.\tag{\Cref{cor:cross terms safe to coll}}
    \end{align*}
    To bound the first term, we note that $g_1$ is supported on $B_{\geq1}^{\mathrm{safe}}$ and directly compute
    \begin{align*}
        &\abs{\left\langle f,R_{m,[m-\ell-1,m],k}\pbra{R_{m,[m-1],k}-R_{m,[m],k}}g_1\right\rangle }\\
        \leq&\abs{\left\langle f,Q_{m,[m-\ell-1,m],k}\pbra{R_{m,[m-1],k}-R_{m,[m],k}}g_1\right\rangle }+\\
        &\;\;\abs{\left\langle f,\pbra{R_{m,[m-\ell-1,m],k}-Q_{m,[m-\ell-1,m],k}}\pbra{R_{m,[m-1],k}-R_{m,[m],k}}g_1\right\rangle }\\
        \leq &\abs{\left\langle f,Q_{m,[m-\ell-1,m],k}\pbra{R_{m,[m-1],k}-R_{m,[m],k}}g_1\right\rangle }+\frac{k^2}{2^{\ell/4-20}}\norm{f}_2\norm{g_1}_2\tag{\Cref{lem:product of operators quadratic form}, first part}\\
        \leq &\frac{k^2}{2^{\ell/2-20}}\norm{f}_2\norm{g_1}_2+\abs{\left\langle f,Q_{m,[m-\ell-1,m],k}\pbra{Q_{m,[m-1],k}-Q_{m,[m],k}}f\right\rangle }\\
        &\;\;\;+\abs{\left\langle f,Q_{m,[m-\ell-1,m],k}\pbra{R_{m,[m-1],k}-Q_{m,[m-1],k}-R_{m,[m],k}+Q_{m,[m],k}}g_1\right\rangle }\\
        = &\frac{k^2}{2^{\ell/4-20}}\norm{f}_2\norm{g_1}_2+\abs{\left\langle f,Q_{m,[m-\ell-1,m],k}\pbra{R_{m,[m-1],k}-Q_{m,[m-1],k}-R_{m,[m],k}+Q_{m,[m],k}}g_1\right\rangle}\\
        =&\frac{k^2}{2^{\ell/4-20}}\norm{f}_2\norm{g_1}_2+\frac{k^2}{2^{\ell/4-20}}\norm{f}_2\norm{g_1}_2\tag{\Cref{lem:product of operators quadratic form}, second part}.
    \end{align*}
    For the second-to-last equality we used that $Q_{m,[m-\ell-1,m],k}\pbra{Q_{m,[m-1],k}-Q_{m,[m],k}}=0$ because $Q_{m,S,k}Q_{m,T,k}=Q_{m,S\cup T,k}$ for any $S,T\subseteq[m]$.

    To complete the proof we use that $\norm{g_1}_2,\norm{g_2}_2\leq \norm{f}_2$.
\end{proof}


