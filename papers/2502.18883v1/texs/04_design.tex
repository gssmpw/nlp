\section{Empirical Evaluation Design} \label{sec:design}
To evaluate the performance of the proposed approaches in four scenarios, we investigate the following research questions:

\begin{enumerate}[label=\textbf{RQ$_\arabic*$:}, ref=\textbf{RQ$_\arabic*$}, wide, labelindent=5pt]\setlength{\itemsep}{0.2em}
    \item \label{rq:baseline}{\textit{How effective is our unsupervised COOD when compared to unsupervised baselines?}}
    \item \label{rq:athena}{\textit{How effective is our weakly-supervised COOD+ when compared to weakly-supervised baselines?}}
    \item \label{rq:athena}{\textit{How effective is our weakly-supervised COOD+ when using different modules or encode backbone?}}
    \item \label{rq:impact}{\textit{Is the main task (Code Search) performance affected by our COOD/COOD+ auxiliary, and to what extent?}}
\end{enumerate}

\subsection{Datasets}
In our experiments, we rely on two benchmark datasets: CodeSearchNet (CSN)~\cite{guo2020graphcodebert} and TLCS~\cite{salza2022effectiveness}. CSN contains bimodal data points consisting of code paired with function-level NL descriptions (\ie first lines of documentation comments) in six PLs (\eg Python, Java) collected from GitHub repositories. While CSN was originally created for a specific downstream task (\ie code search), it has since been widely adopted by large (NL, PL) models~\cite{guo2020graphcodebert, guo2022unixcoder} for pre-training due to the informative nature of bimodal instances. Large NL-PL models are first pre-trained across \textit{all} six languages, and then further fine-tuned for a \textit{specific} PL for some downstream task to enhance performance. For code search, the goal is to retrieve the most relevant code given a NL query, where CSN is widely used to further fine-tune a PL-specific code search model~\cite{feng2020codebert}. 

% Consistent with prior research \cite{guo2020graphcodebert, liu2023contrabert}, we utilize the filtered version of CSN, which removes low-quality samples using handcrafted rules \cite{lu2021codexglue}.

Salza \etal \cite{salza2022effectiveness} used training samples from CSN for pre-training, and created a new dataset sourced from StackOverflow (SO) for fine-tuning the code search model, involving only \textit{three} PLs: Java, Python, and JavaScript. Specifically, they leverage SO user questions as search queries and accepted answers as retrieved code snippets, which differ from GitHub comments and the corresponding code in CSN. We refer to this new dataset as TLCS. Existing work~\cite{Yang2017msr, lotter2018so, Abdalkareem2017OnCR} investigated code clones between SO and GitHub, demonstrating there exists \textit{only} 1-3\% code reuse. Besides code, user questions in SO are typically formulated before code answers, without concrete knowledge of what code answers will be, and are mostly written by end-users. Conversely, in GitHub, method docstrings (\ie comments) are often written following code snippets, and are mostly written by developers. These distinctions cause performance shortfall when directly applying models trained on CSN to TLCS without further fine-tuning or transfer learning~\cite{guo2022unixcoder, huang2021cosqa, salza2022effectiveness}. %In line with CSN where each code snippet is a single function, we further preprocess the TLCS dataset by localizing functions within accepted answers and extracting the longest one as the final corresponding code snippet. Consequently, TLCS contains 20,531 and 27,879 question-answer pairs for Java and Python, respectively. \yanfu{javascript?}

\subsection{OOD Scenarios}

% \section{Dataset Generation}
\label{sec:dataset}
\revise{
To train the proposed GNN, we constructed a dataset of building structures and a subset of these structures were subjected to fire simulations using FEA. The dataset generation process is illustrated in \figref{fig:dataset_generation_procedure}. Initially, a total of 33,000 building structures with geometrical details, material properties, and gravity loads were created. Due to randomness in generating these structures, a filter is applied to remove unreasonable data after gravity load simulation, which included 15,377 structures. A trade-off between computational feasibility and model performance is made among the remaining 17,623 structures. As further labeling structures with MIDR requires resource-intensive fire simulations via OpenSeesRT, a large proportion of 16,050 structures is selected as unlabeled dataset. On the other hand, each of the other 1,573 structures was further subjected to 30 different fire simulations, forming the labeled dataset containing $1,573\times 30 = 47,190$ fire cases.} This section details the step-by-step process for generating the dataset, including geometry creation, material property assignment, and simulations due to gravity loads and fire scenarios. 
% To train the proposed neural network, we constructed a dataset comprising building structure data and a subset of fire scenario data. The dataset generation process is illustrated in \figref{fig:dataset_generation_procedure}. 
% A total of 33,000 building structures with geometric details, material properties, and gravity loads were initially created. Out of these, 3,000 structures were selected as labeled data, and the remaining 30,000 were designated as unlabeled data. Further, about half of them filtered out due to instability under gravity loads only. 
\begin{figure*}[h!]
    \centering
    \includegraphics[width=0.8\linewidth]{figures/dataset_filter_procedure.pdf}
    \caption{Workflow for dataset generation (geometry, material property, gravity loads, and fire scenarios).}
    \label{fig:dataset_generation_procedure}
\end{figure*}

\subsection{Geometry Generation}
\label{subsec:geometry_generation}
The geometry of the building structures forms the foundation of the dataset. Regular 
\revise{3D structures} resembling multi-story parking structures or shopping malls were generated, with parameters such as building floor dimensions and story heights selected randomly. Each building structure is composed of multiple rooms, which serve as the basic unit in this study. A room herein is a cuboid space defined by specific length, width, and height. Within a structure, rooms of the same dimensions are uniformly arranged along the length, width, and height, corresponding to the $x$-, $y$-, and $z$-axes, respectively. Structures vary in room size and number of rooms along each axis. Specifically, the room length, width, and height are independently sampled from a uniform distribution within the interval $[2, 5]$ meters along the three directions of the structure. Similarly, the room number along each axis is uniformly sampled independently as an integer within the interval $[2, 7]$, i.e., the maximum number of stories of the buildings simulated in this study is 7.

To introduce variability and simulate real-world scenarios, approximately $8\%$ of structural elements (beams or columns) are randomly removed after initial geometry creation. 
\revise{Such removal is not fire-induced damage, but reflects functional diversity often observed in real buildings, such as open spaces designed for activities in shopping malls, e.g., ice skating rinks. Examples of the generated geometries are illustrated in \figref{fig:example_generated_geometry}, showcasing the diversity and realism of the dataset. This element removal does not affect the definition of room's geometry in the structure and nor does it affect the number of considered fire scenarios.} 

\revise{A range of coefficient of variation values ($3.3\%$ to $17.5\%$) was derived from prior studies that investigated the statistics of geometrical and material properties of structural components of buildings (e.g., \cite{mirza1979variations, lee2004probabilistic}). These studies provide empirical data on the natural variability in parameters such as Young's modulus, yield strength, and dimensions of structural elements due to manufacturing tolerances and material inconsistencies. By selecting $8\%$ for the removal of structural elements in our database, we aimed to maintain a level of variability that is representative of real-world uncertainties while ensuring computational feasibility. This choice ensures that the database captures realistic deviations without introducing extreme cases that may not be commonly encountered in practice.}

\begin{figure*}[h!]
    \centering
    \includegraphics[width=\linewidth]{figures/example_generated_geometry.pdf}
    \caption{Examples of generated structural geometry of different sizes (all dimensions in meters).}
    \label{fig:example_generated_geometry} 
\end{figure*}

{\blockRevise

In this study, we opted for a deterministic square, dimension of $0.1$ m, solid cross-sectional steel elements due to their simplicity in modeling and analysis. Square sections exhibit uniform geometrical properties in all directions, simplifying the computation of structural responses and avoiding complications associated with more complex shapes, such as wide-flange sections, facilitating the computational efficiency and scalability to generate a large dataset. This choice also helps to mitigate issues related to stress concentrations and facilitates a more straightforward representation of structural behavior under thermal loads. 

\textit{Remark:} The selected cross-section provides a comparable flexural rigidity to a $W 130 \times 130 \times 28.1$ wide-flange section (metric units), albeit with significantly higher axial rigidity. This cross-section is acceptable for gravity-load-designed frames under service loading conditions where the models assume fully rigid, moment-resisting beam-column connections for the evaluation of the IDR under thermal loading. This assumption is reasonable in this computational study where the primary interest is to understand the global deformation response of frames under fire conditions. The selection of uniform square cross-sections for both beams and columns, rather than adherence to standard capacity design principles, was made here primarily for computational efficiency and to reduce design parameters in the database generation process. This choice allows for simplified and scalable approach to analyze the fire-induced response of generic steel frames without the need for large section variations, where this study mainly focuses on the fire vulnerability assessment using ML-based predictions. However, if additional loading conditions, e.g., seismic or wind loads, were to be considered, larger sections, strong-column/weak-beam principle, and ductile detailing would be required in the generated buildings for realistic structural behavior under combined loading conditions. Future studies may also consider investigating the influence of variable cross-sectional dimensions and semi-rigid connections on the structural performance under fire conditions. 
} % blockRevise

\subsection{Material Properties}
Steel is chosen as the material for the structures. To reflect real-world variations, we randomly assign one of five slightly different steel material types to each structural element. \revise{
The ranges of material properties are provided in \tabref{tab:material_property_ranges} and the properties are sampled from uniform distributions of the corresponding ranges. These variations simulate differences arising from manufacturing batches or regional material properties. That these properties are at ambient temperature and change when the temperature rises due to a fire. The selection of materials with varying properties is aimed at increasing the diversity of the data. Our goal is to represent as wide a range of data as possible with a limited amount of building structure data, thereby enhancing the generalization ability of the GNN. Our assumed material property ranges are expected to be wider than the real-world conditions based on findings in \cite{mirza1979variations, lee2004probabilistic}. Therefore, we are essentially tackling a more challenging and general task. If we can solve this problem, we are confident that our method will perform equally well or even better in real-world scenarios.
}
\begin{table}[h!]
    \centering
    \caption{Material properties ranges for considered steel structures.}
    \begin{tabular}{lc}
        \toprule
        Property & Range \\
        \midrule
        Young's modulus & [168, 252] GPa \\
        Yield strength & [220, 330] MPa \\
        Strain-hardening ratio & [0.8, 1.2] \% \\
        \bottomrule
    \end{tabular}
    \label{tab:material_property_ranges}
\end{table}

\subsection{Gravity Loads}
Gravity loads are applied to columns and beams based on their \revise{influence (tributary) areas as typically conducted in structural analysis. The considered ``service'' load conditions include the column self-weight and the additional loads directly supported on the beams from their self-weight and weights of the reinforced concrete slabs, people as live load, and building content. An edge beam typically carries approximately half the gravity load supported by a parallel interior beam}. The ranges of gravity loads are listed in \tabref{tab:gravity_load_ranges}. \revise{The loads are sampled from uniform distributions of the corresponding ranges.} Structures that failed to meet an MIDR threshold of $1\%$ under gravity loads were deemed unacceptable designs and filtered out, as such configurations of randomly chosen geometry, material, and gravity load combinations were considered unrealistic from a regulatory and practicality points of view.
\begin{table}[h!]
    \centering
    \caption{Gravity load ranges for considered beams and columns.}
    \begin{tabular}{lc}
        \toprule
        Element & Range (kN/m)  \\
        \midrule
        Column & [0.5, 1.0]  \\
        Edge beam & [1.5, 4.5]  \\
        Interior beam & [3.0, 7.5]  \\
        \bottomrule
    \end{tabular}
    \label{tab:gravity_load_ranges}
\end{table} 

\subsection{Rule-based Thermal Load Generation}
\label{subsec:thermal_load_generation}
To evaluate a building's structural response during a fire event, we employed a simplified rule-based approach for thermal load generation. 
% Previous studies \cite{nan_structuralfire_2023} have demonstrated that steel structures rapidly equilibrate with surrounding gases temperatures due to efficient heat exchange. Consequently, gas temperatures can be directly used as inputs for FEA tools, e.g., OpenSees, simplifying the process of modeling thermal loads. 
% Accurately simulating temperature fields in fire scenarios poses significant challenges. Advanced thermodynamic simulations, such as those performed using Fire Dynamics Simulator (FDS) \cite{mcgrattan_fire_2000}, provide precise temperature predictions. However, these methods are hindered by high computational costs, prolonging execution times, and limited scalability, making them impractical for generating large datasets. Additionally, real-world fire loads often display substantial spatial variability across different rooms \cite{dundar_fire_2023}, resulting in scenario-specific temperature fields with limited generalizability. For example, studies on bridge fires \cite{he_study_2024} have demonstrated that environmental factors, such as wind speeds, can significantly influence temperature distributions. Furthermore, even within identical scenarios, variations in fire modeling methodologies can produce distinctly different temperature fields \cite{zhang_temperature_2020, du_new_2012}. These challenges emphasize the need for efficient and adaptable methods to generate fire temperature data.
% To address these issues, we adopted a rule-based approach to model temperature variations. 
According to \cite{spearpoint_fire_2008}, a typical fire development follows a predictable pattern. During the {\em{growth stage}}, the temperature rises slowly and approximately linearly after ignition. This is followed by the {\em{flashover stage}}, where temperatures increase rapidly to peak values. After reaching the peak, the temperature either stabilizes or continues to rise slowly until the {\em{decay stage}} begins. Inspired by this fire development pattern, we describe the temperature evolution in time, $t$, prior to the decay stage in two distinct stages:
\begin{enumerate}
    \item {\bf{Initial linear increase stage}}: For $t \in [0, t_1)$, temperature increases gradually and linearly as the fire spreads through the building. This stage represents the time before the fire directly affects a structural element.  
    \item {\bf{ISO 834 fire curve stage}}: For $t \in [t_1, t_{\thre}]$, temperature rises rapidly following the ISO 834 curve \cite{ISO834}, modeling the direct impact of the fire on the structural element. 
\end{enumerate}
The slope of the linear temperature increase, $c$, and the transition time, $t_1$, are influenced by the spatial relationship between the fire source and the structural element. For the second stage of temperature evolution, we utilize the ISO 834 curve, a widely accepted standard for fire resistance testing. This standardized fire curve describes the temperature rise over time, enabling rapid and consistent thermal fields across various scenarios. The duration of fire simulation in this study is set to $t_{\thre}=60$ minutes. This value represents the upper limit for the temperature evolution of each structural element, providing a consistent basis for analyzing the structural response to fire.

Let $(x, y, z)$ represents the midpoint of a structural element and $(x_{\subfire}, y_{\subfire}, z_{\subfire})$ the fire source point. \revise{Integer parameters $h$ and $h_{\subfire}$ correspond to the respective floor levels of the element and the fire source}. The temperature evolution for each element is expressed as follows:
\begin{enumerate}
    \item Linear increase stage ($0 < t < t_1$):
    \begin{equation}
    T(t) = c \cdot t,
    \end{equation}
    where $c$, the rate of temperature increase ($^\circ\mathrm{C}/\mathrm{min}$), depends on the height difference between the element, $h$, and the fire source, $h_{\subfire}$:
    \begin{equation}
        c = 
        \begin{cases} 
        5\left/\left(h - h_{\subfire} + 1\right)\right., & h \geq h_{\subfire}, \\
        2\left/\left(h_{\subfire} - h\right)\right., & h < h_{\subfire}.
        \end{cases}
    \end{equation}
     \item ISO 834 stage ($t \geq t_1$):
\begin{equation}
    T(t) = c \cdot t_1 + 345 \log_{10} \left(8 \left(t - t_1\right) + 1\right).
\end{equation}
\end{enumerate}

The transition (arrival) time $t_1$, marking the end of the linear stage, depends on the spatial distance between the fire source and the element. We define the following two Euclidean distances $L_p$ in the $xy$ plane and $L_s$ in the $xyz$ space:
\begin{eqnarray}
L_p & \triangleq & \sqrt{(x - x_{\subfire})^2 + (y - y_{\subfire})^2}, \\
\label{eq:Lp}
L_s & \triangleq & \sqrt{(x - x_{\subfire})^2 + (y - y_{\subfire})^2 + (z - z_{\subfire})^2}.
\label{eq:Ls}
\end{eqnarray}
Accordingly, the transition time, $t_1$, is expressed as follows:
\begin{equation}
    t_1 = 
    \begin{cases}
    \beta_{1} \cdot \left(1 - \exp\left\{- L_s\left/\alpha_{1}\right.\right\}\right), & h > h_{\subfire}, \\
    \beta_{2} \cdot \left(1 - \exp\left\{- L_p\left/\alpha_{2}\right.\right\}\right), & h = h_{\subfire}, \\
    \beta_{3} \cdot \left(1 - \exp\left\{- L_s\left/\alpha_{3}\right.\right\}\right), & h < h_{\subfire} .
    \end{cases}
    \label{eq:t1}
\end{equation}
The parameters $\beta_i$ and $\alpha_i$ for determining $t_1$ are summarized in Table~\ref{tab:fire_spread_parameters}. In this study, we take $r_{\mathrm{up}}=0.95$ and $r_{\mathrm{down}}=0.97$.
\begin{table}[ht]
    \centering
    \caption{Fire spread parameters for $t_1$ calculations.}
    \begin{tabular}{lcc}
        \toprule
        Case  & $\beta_i$ & $\alpha_i$  \\
        \midrule
        $i=1$, Upward spread & $16 \left.\left(1-r_{\mathrm{up}}^{\left|h-h_{\subfire}\right|}\right)\right/\left(1-r_{\mathrm{up}}\right)$ & $10$  \\
        $i=2$, Horizontal spread & $18$ & $18$  \\
        $i=3$, Downward spread & $30 \left.\left(1-r_{\mathrm{down}}^{\left|h-h_{\subfire}\right|}\right)\right/\left(1-r_{\mathrm{down}}\right)$ & $5$  \\
        \bottomrule
    \end{tabular}
    \label{tab:fire_spread_parameters}
\end{table}

\figref{fig:t1_curve} illustrates the $t_1$ curves for various fire scenarios: (1) fire originating on the lower floor, $h-h_{\subfire}=1$ with rapid upward spread, (2) fire on the same floor, $h=h_{\subfire}$ with the fastest spread, and (3) fire on the upper floor, $h_{\subfire}-h=1$ with slow downward spread. The exponential decay in $t_1$ reflects the accelerating fire propagation speed as the distance increases. \figref{fig:t1_curve} also indicates that the employed simplified model is consistent with the Markov chain-based dynamic model given by \cite{cheng_dynamic_2011}, where the rooms at the same floor of the fire point start flashover slightly before the corresponding upper floors. Additionally, $\beta_{1}$ and $\beta_{3}$ are the summation of a geometric sequence, where story level $h$ is the index. The common ratios $r_{\mathrm{up}}<1$ in $\beta_{1}$ and $r_{\mathrm{down}}<1$ in $\beta_{3}$ indicate that the fire speeds up to spread through the next story, which is consistent with the real-world fire spread mechanism given in \cite{hokugo_mechanism_2000}. The temperature profile within the range $t \in [0, t_{\thre}]$ is subsequently used as the thermal load in OpenSeesRT simulations to compute displacements at each structural node at time $t_{\thre}$.
\begin{figure}[h!]
    \centering
    \includegraphics[width=0.8\linewidth]{figures/m204_t1_curve.pdf}
    \caption{Three examples for the $t_1$ curve.}
    \label{fig:t1_curve}
\end{figure}

\revise{
\textit{Remark:} The effects of structural elements, such as concrete floor slabs and partitions, are not explicitly modeled in our approach. Instead, their influence is implicitly captured through the careful selection of the parameters $ \alpha, \beta, r_\mathrm{up} $, and $ r_\mathrm{down} $. This parameterization provides a unified framework for generating temperature fields. Indeed, fire propagation is governed by a multitude of factors and remains an open research question. For instance, if the fire resistance of a floor slab is enhanced by fire protective coating, the corresponding model can account for this by decreasing $\alpha_1$ \& $\alpha_3$, increasing $\beta_1$ \& $\beta_3$, and adopting larger values for $r_\mathrm{up}$ \& $r_\mathrm{down}$, which collectively slow down the vertical spread of fire. Conversely, scenarios involving higher amounts of combustible materials would warrant the opposite adjustments. This flexible and integrated approach avoids the need to design separate models for different fire propagation scenarios while still capturing the essential effects.
}

\revise{
In conclusion, our rule-based approach is a computationally efficient method for approximating fire temperature fields, enabling large-scale dataset generation to train predictive models. By combining ISO 834 fire curves with spatial considerations and embedding structural effects through parameter calibration, the method achieves a balanced trade-off between accuracy and scalability, making it a practical solution for thermal load modeling in fire scenarios. After generating the temperature of each beam or column according to the middle point, the temperature is applied as uniform thermal load to the elements of the structure in question using OpenSeesRT. 
}

% In conclusion, this rule-based approach is a computationally efficient method to approximate fire temperature fields, enabling large-scale dataset generation to train predictive models. By combining ISO 834 fire curves with spatial considerations, the method balances accuracy and scalability, making it a practical solution for thermal load modeling in fire scenarios.

% \subsection{Interstory Drift Ratio}
\subsection{OpenSeesRT Simulation}
\label{subsec:opensees_simulation}

The thermal and mechanical responses of 3D frame structures under combined fire and gravity loads are simulated using OpenSeesRT \cite{perez2024openseesrt}. \revise{In the simulation, the IDR of each node at $t_{\thre}$ is computed using the computed nodal displacements. Each structural model features six degrees of freedom per node (3 translational  and 3 rotational), with linear geometrical transformations (\texttt{geomTransf: Linear}) defining how the element local coordinate systems are mapped to the global coordinate system and assuming small displacements and rotations. Although OpenSeesRT allows a variety of options for modeling finite deformations, in the present simulations and mainly for simplicity, we did not consider large deformations. All bottom nodes (nodes on the ground) are fully constrained in all six degrees of freedom, while degrees of freedom os all other nodes are free.} Material behavior is temperature-dependent and modeled with \texttt{Steel01Thermal}, while fiber-based sections (\texttt{FiberThermal}) capture nonlinear interactions between thermal and mechanical responses at the cross-section level. \revise{Structural elements are represented as displacement-based Euler-Bernoulli beam-columns (\texttt{dispBeamColumnThermal}). This element  formulation accounts for thermal strains (temperature gradients) in the section, which is discretized into fibers. Numerical integration is used along the length of each element using three integration (Gauss) points, one at each end and the third in the middle of the element.}

{\revise{Thermal expansion of steel members plays a crucial role in IDR development. In reality, reinforced concrete floor slabs heat at a different rate than steel members due to their higher thermal mass and lower thermal conductivity. This differential heating can lead to restrained thermal expansion, introducing axial compression in beams and affecting the overall structural response. In this study, explicit {\em{composite action}} between steel members and concrete slabs is not modeled. Instead, our approach focuses on isolating the response of the steel structural frame, which is often the critical load-bearing component in fire scenarios. This assumption aligns with prior studies \cite{Possidente_2024} demonstrating that steel structures reach thermal equilibrium with surrounding gases quickly, allowing the use of uniform thermal loading in fire analysis. Future work could enhance this framework by incorporating slab-beam interaction effects, through a refined FEA for an extended dataset where constraints imposed by floor slabs are explicitly considered.}

The analysis begins with the application of gravity loads, followed by incremental thermal loads simulating the fire exposure. A static nonlinear solver using  \texttt{ExpressNewton} algorithm ensures convergence, while the \texttt{NormDispIncr} test maintains accuracy. An incremental \texttt{LoadControl} scheme with small step sizes is employed to guarantee numerical stability, using 10\% for gravity loads and 1\% for thermal loads. 

\revise{
In the thermal load analysis, uniform thermal load is applied to each beam or column, i.e., the temperature of each element is set to be that at the middle point, according to \secref{subsec:thermal_load_generation}. The \texttt{Steel01Thermal} material allows the properties (e.g., Young's modulus and yield strength) to be adjusted at increasing temperatures according to \cite{EN1993} using its Table 3.1: Reduction factors for the stress-strain relationship of carbon steel at elevated temperatures. For example, if the Young’s modulus at ambient temperature is $E_0$, then as the temperature ($T$) increases, the modulus changes as $E(T) = \eta (T) \times E_0$. \cite{EN1993} directly provides the values of $\eta(T) \in \left[0,1\right] $ at every $100 ^\circ\mathrm{C}$ interval and recommends using linear interpolation to obtain $\eta(T)$ for intermediate values of $T$.
} OpenSeesRT documentation \cite{OpenSeesThermalExamples} provides several examples of thermal analyses.

This modeling framework accommodates variations in material properties, cross-sectional geometries, and temperature profiles, providing robust simulations of structural behavior under fire conditions. The primary settings and configurations for the OpenSeesRT simulations are summarized in \tabref{tab:ops_detail}.
\begin{table}[h!]
    \centering
        \caption{Key settings of OpenSeesRT simulations.}
    \begin{tabular}{l|>{\raggedright\arraybackslash}p{0.6\linewidth}} %
    \toprule
    Modeling Aspect     & Details \\
    \midrule
    Geometry            & 3D models; 6 degrees of freedom per node \\
    Transformation      & geomTransf: Linear \\ 
    Material            & Steel01Thermal \\
    Section             & FiberThermal; Cross-section: $0.1$ m $\times$ $0.1$ m \\ 
    Element type        & {dispBeamColumnThermal} \\ 
    Loading             & Gravity loads: {beamUniform}; Thermal loads: {beamThermal} \\
    Integration scheme  & Incremental {LoadControl}; Step size: $10\%$ (gravity analysis), $1\%$ (thermal analysis) \\
    Nonlinear solver    & {ExpressNewton} algorithm; {UmfPack} solver; Convergence test: {NormDispIncr} tolerance: $10^{-8}$; Maximum \# iterations per step: $1000$. \\ 
    \bottomrule
    \end{tabular}
    \label{tab:ops_detail}
\end{table}

For each structure in the labeled dataset, 30 fire points are selected using a dual-granularity approach, \revise{i.e., two-stage sampling strategy,} to ensure they are well-distributed. Specifically, rooms are sequentially selected, with one fire point randomly chosen within each selected room. If a building is large and contains more than 30 rooms, we randomly select 30 rooms without replacement, i.e., ensuring that no more than one fire point is located in the same room. Conversely, if the building is small and has fewer than 30 rooms, all rooms are initially selected, with one fire point randomly assigned to each room. Additionally, rooms are then selected with replacement until a total of 30 fire points are assigned. \revise{The room-level sampling prioritizes selecting distinct rooms to avoid spatial clustering of fire points, while the point-level sampling ensures intra-room variability. This approach aligns with stratified sampling principles commonly used for efficient spatial representation, where multi-stage sampling strategies optimize coverage and variability, e.g., \cite{arunachalam_generalized_2023}, and enables a more comprehensive characterizing of how the structures respond under fire conditions.}
% This selection method prevents fire points from clustering too closely while maintaining an element of randomness. By distributing fire points in this manner, the 30 fire scenarios are effectively utilized, enabling a more comprehensive characterizing of how the structures respond under fire conditions.

\subsection{Summary of the Dataset Generation}
As discussed in this section and related to  \figref{fig:dataset_generation_procedure}, three key steps were considered in the development of the dataset: 
\begin{enumerate}
    \item {\bf{Filtering process}}: Structures with MIDR exceeding $1\%$ under gravity loads were excluded,  resulting in $1,573$ labeled structures retained for fire simulation and $16,050$ unlabeled structures for training the MFSP predictor.
    \item {\bf{Fire simulations}}: For each retained labeled structure, 30 fire scenarios were simulated using OpenSeesRT, yielding $47,190$ fire cases.
    \item {\bf{Data distribution check}}: MIDR distributions for labeled and unlabeled data under gravity loads were highly similar, because both datasets were generated using the same method. Under fire conditions, the MIDR distribution shifted, reflecting significant structural deformation with values reaching a maximum of about 6\%, an average of 1.70\%, and a standard deviation of 1.12\%. This step ensured a diverse and comprehensive dataset for the proposed predictive framework.
\end{enumerate}
The statistical distribution histograms for MIDR (after applying the $1\%$ filtering threshold \revise{for gravity load responses}) under different loading conditions are plotted in \figref{fig:histogram_mdr}. Figures \ref{fig:histogram_mdr}(a) and \ref{fig:histogram_mdr}(b) show the MIDR distributions of the labeled and unlabeled data, respectively, under gravity loads only. \figref{fig:histogram_mdr}(c) shows the MIDR distribution of the labeled data under the combined effects of gravity and fire loads. Fire load causes the structures to significantly deform, leading to a noticeably \revise{right-skewed} MIDR distribution.

\begin{figure*}[h!]
    \centering
    \includegraphics[width=\linewidth]{figures/histogram_mdr.pdf}
    \caption{Histograms of MIDR for labeled and unlabeled structures with gravity loads and fire cases.}
    \label{fig:histogram_mdr}
\end{figure*}

\revise{
This dataset provides the basis for training and testing the performance of the GNN-based framework. Although we employed a simplified rule-based thermal load generation method compared with conventional CFD-based simulations, the temperature field, the changes of the material properties, and the response of the structures, are all still highly nonlinear and complex. Therefore, it is still a challenging task for the NN to predict the MIDRs based on this dataset.
}

%%%%-------examples---------------
% \begin{figure*}[!t]
% \caption{Examples of four OOD scenarios from CodeSearchNet-Java dataset, with descriptions for each scenario and "defective" modality highlighted in \textcolor{red}{red} boxes.}
% \begin{center}
% % \fbox{\rule{0pt}{2in} \rule{0.9\linewidth}{0pt}}
%    %\includegraphics[width=0.8\linewidth]{egfigure.eps}
% \begin{subfigure}[b]{0.23\textwidth}
%     \centering
%     \includegraphics[width=\textwidth]{figs/scenario1.pdf}
%     \caption{Out-domain: Code-comment pair from another dataset (Python).}
%     \label{fig:scen1}
% \end{subfigure}
% \hspace{0.5em}
% \begin{subfigure}[b]{0.23\textwidth}
%     \centering
%     \includegraphics[width=\textwidth]{figs/scenario2.pdf}
%     \caption{Misaligned: ID comment belongs to another ID code.}
%     \label{fig:scen2}
% \end{subfigure}
% \hspace{0.5em}
% \begin{subfigure}[b]{0.23\textwidth}
%     \centering
%     \includegraphics[width=\textwidth]{figs/scenario3.pdf}
%     \caption{Shuffled-text: Different meaning comment due to shuffled tokens. \yanfu{Viet, please update the figure based on the new scenario setting}}
%     \label{fig:scen3}
% \end{subfigure}
% \hspace{0.5em}
% \begin{subfigure}[b]{0.23\textwidth}
%     \centering
%     \includegraphics[width=\textwidth]{figs/scenario4.pdf}
%     \caption{AST-defective: Code that cannot be parsed into AST.}
%     \label{fig:scen4}
% \end{subfigure}
% \end{center}
% \label{fig:example}
% \end{figure*}


We design four distinct OOD scenarios using the datasets described above, with CSN-Java and CSN-Python as inliers due to their common use for the pre-training of code models. %We excluded CSN-JavaScript because the buggy code generation algorithm we use in scenario 4 doesn't support JavaScript.

\noindent\textbf{Scenario 1: Out-domain.} Following existing ML work~\cite{hendrycks2020pretrained, podolskiy2021revisiting, zhou2021contrastive}, we create an out-of-domain setting by choosing OOD samples from a different dataset than the training data. Thus, samples from TLCS-Java or TLCS-Python are treated as outliers accordingly. Inliers and their corresponding outliers belong to the same PL to ensure approaches don't identify OODs based on syntax differences between PLs but on data domains: GitHub vs. SO. Prior studies~\cite{wang2022enriching} show that CSN queries are longer than SO questions on average, so we sampled TLCS questions and answers to match the length distribution of CSN comments and code, to avoid OOD approaches exploiting spurious cues of query length differences. % By doing this, the query and code length differences between inliers and outliers are within two tokens.	
We didn't consider other code search datasets~\cite{yao2018staqc, li2020hierarchical, rao2021search4code} because they either contain only one of the PLs (Python or Java) or have a smaller dataset size. 

\noindent\textbf{Scenario 2: Misaligned.} In this scenario, we shuffle normal NL-PL pairs so that each code doesn't match its NL description. Although the NL modality sourced from attached comments in code are typically aligned with the PL modality, documentation errors may still occur and not effectively filtered by handcrafted rules~\cite{guo2020graphcodebert}. %Therefore, in order to avoid laborious human annotation and verification, we construct this OOD scenario to leverage DL for automatically detecting documentation errors, which is also expected to benefit bimodal pre-training and the downstream code search task.


\noindent\textbf{Scenario 3: Shuffled-comment.} For (comment, code) pairs, we modify the syntactic information in each comment by shuffling 20\% of selected tokens using a seeded random algorithm~\cite{sinha-etal-2021-masked} with positions of stopwords and punctuations unchanged. No changes are made to the code for this scenario. This scenario is inspired by~\cite{sinha2020unnatural, mai2022self}. \cite{sinha2020unnatural} discovered that NL pre-trained models are insensitive to permuted sentences, which contrasts with human behavior as humans struggle to understand ungrammatical sentences, or interpret a completely different meaning from a few changes in word order. \cite{mai2022self} further introduces syntactic (shuffling) outliers into NL pre-training corpora to enhance OOD robustness and NL understanding performance.

\noindent\textbf{Scenario 4: Buggy-code.} We create buggy code using a \textit{semantic} mutation algorithm which injects more natural and realistic bugs into code than other traditional loose/strict mutators~\cite{richter2022learning}. This simulates buggy programs that the model may encounter during testing, typically absent from the training dataset, and should be taken into account by OOD code detectors according to the OOD definition~\cite{he2022distribution}. We avoid using real bug/vulnerability datasets~\cite{just2014defects4j, zhou2019devign, widyasari2020bugsinpy} due to limitations like the absence of paired comments, lack of support for Python or Java, introduction to a new dataset domain \etc. We generate buggy code for each code in CSN-Java and CSN-Python using~\cite{richter2022learning} to serve as outliers, ensuring the inliers and outliers are from the same dataset domain with the only difference being normal vs. buggy code. We focus on variable-misuse bugs, as only this mutation algorithm is available for both Python and Java in ~\cite{richter2022learning}. Variable-misuses occur when a variable name is used but another was meant in scope, and often remain undetected after compilation and regarded as hard-to-detect by recent bug detection techniques~\cite{he2022distribution, Richter2023how}. Comments remain unchanged for this scenario. %Since the mutation algorithm requires the code snippets to be successfully parsed into Concrete Syntax Trees, not all normal code can be mutated. We finally obtain 90,200/5,935 and 14,920/1,693 buggy code snippets for the original CSN-Java and CSN-Python training/testing sets.  


% We considered four OOD scenarios as comprehensive for code-related tasks involving both NL and PL modalities, or only one of them. While prior works in OOD detection typically focused only on out-domain and shuffled-text OOD scenarios, we included two additional ones: misalignment and buggy code. Other OOD instances that might occur in the PL modality, such as unusual data/control flow, are typically syntactically infeasible and can mostly be detected by leveraging static analysis tools~\cite{tree_sitter} or compilers, and thus are not included in our current design. Note that the buggy code in scenario 4 is semantically incorrect but syntactically feasible, which cannot be detected by static parsers/compilers, necessitating the use of DL models to extract code semantics.

%we initially also considered syntactically-incorrect code which can not parse into AST, but such code can be directly handled by a parser/compiler. Similarly, unusual data/control flow issues are likely to be detected by leveraging static analysis tools \REF. Therefore, we didn’t include these scenarios in our current design. 

\subsection{Model Configurations} 
For the weakly-supervised COOD+, we experiment with either the contrastive learning module (COOD+\_CL) or the binary OOD rejection module (COOD+\_BC) to compare against the combined model.  All models are trained using the Adam optimizer with a learning rate of $1e-5$, and a linear schedule with 10\% warmup steps. The batch size is set to 64, and the number of training epochs is 10. For the COOD+\_CL and COOD+, the margins in the margin-based loss are set to $0.2$. The balancing value $\lambda$ is set to $0.2$ after a grid search. The hidden layer size in the binary OOD rejection module for COOD+ is 384 ($768/2$). We also explore the robustness
and agnosticism of our COOD+ approach to different NL-PL models by replacing the GraphCodeBERT encoder with CodeBERT~\cite{feng2020codebert}, UniXcoder~\cite{guo2022unixcoder}, and ContraBERT~\cite{liu2023contrabert}.



\subsection{OOD detection model training and measurement}
%For evaluation, we focus on measuring their effectiveness using two standard metrics. In real-world deployment scenarios of ML-guided models, retaining as many ID samples as possible is crucial, as these are safely applicable to the main task training and inference, while OOD instances are rejected or handled separately. Since detection models typically output and rank OOD scores to identify OOD examples, setting a threshold is necessary to ensure a high proportion of ID data (e.g., 95\%). 

For unsupervised COOD, we use only ID data for model training, thus involving all training data from CSN-Python and CSN-Java, with 10\% randomly sampled for validation. We avoid using the CSN development dataset for validation due to its smaller size. For weakly-supervised COOD+, we randomly select 1\% of the training data and replaced them with OOD samples generated for each scenario (following~\cite{tian2020few}), resulting in a total of 4\% OOD samples and 96\% ID samples for training. During inference, both COOD and COOD+ utilize the same ratio (20\%) for inliers and outliers from each scenario, which is more convincing than using an imbalanced dataset (\ie tiny number of OOD data). Detailed dataset statistics are provided in our online appendix~\cite{cood-tool}. Since all outliers are randomly selected, we report average OOD detection results across \textit{five} random seeds of the test dataset to ensure evaluation reliability and reproducibility.  
 
Following prior work in ML~\cite{hendrycks2019using, liznerski2022exposing}, we use two standard metrics to measure the effectiveness of our COOD/COOD+ models: the area under the receiver operating characteristic curve (AUROC) and the false positive rate at 95\% (FPR95). AUROC is threshold-independent, calculating the area under the ROC curve over a range of threshold values, representing the trade-off between true positive rate and false positive rate. It quantifies the probability that a positive example (ID sample) receives a higher score than a negative one (OOD sample). Higher AUROC indicates better performance. Additionally, FPR95 corresponds to the false positive rate (FPR) when the true positive rate of ID samples is 95\%. FPR95 is threshold-dependent, where OODs are identified by setting a threshold $\sigma$ with $ P^{OOD}<1-\sigma$ ($P^{ID}>\sigma$) so that a high fraction (95\%) of ID data is above the threshold. It measures the proportion of OOD samples that are mistakenly classified when 95\% of ID samples are correctly recalled based on the threshold. Lower FPR95 indicates better performance.


\subsection{Baselines}
We compare our COOD/COOD+ against various OOD detection baselines, including adaptations of existing unsupervised NLP OOD approaches on NL-PL encoders (1-2), weakly-supervised approaches based on outlier exposure (3), and neural bug detection techniques (4-5). Since unsupervised approaches (1-2) rely on classification outputs for OOD scoring, we reformulate code search as binary classification to fine-tune the encoders similarly to~\cite{feng2020codebert}. (1-2) is supervised for code search, but unsupervised for OOD detection. For weakly-supervised baselines (3-5), we use the same number of OOD samples as COOD+ for a fair comparison. Note that the encoder backbone of (1-3) is also GraphCodeBERT. (4-5) are specifically designed for neural bug detection, thus not requiring other encoder backbone for OOD detection. 

\begin{enumerate}[leftmargin=1.5em] 
    \item \textbf{Supervised Contrastive Learning For Classification (SCL)}~\cite{khosla2020supervised}. This method fine-tunes transformer-based classification models by maximizing similarity of input pairs if they are from the same class and minimize it otherwise. Following \cite{zhou2021contrastive}, we adopts MSP, Energy, and Mahalanobis OOD scoring algorithms for OOD detection.
    
    \item \textbf{Margin-based Contrastive Learning for Classification (MCL)}~\cite{zhou2021contrastive}. This approach fine-tunes transformer-based classification models by minimizing the L2 distances between instances from the same class, and encouraging the L2 distances between instances of different classes to exceed a margin. We also detect OODs by applying MSP, Energy, and Mahalanobis OOD scoring algorithms.
    
    \item \textbf{Energy-based Outlier Exposure (EOE)}~\cite{liu2020energy}. This approach uses a few auxiliary OOD data to fine-tune the classification model with an energy-based margin loss~\cite{liu2020energy}, and then utilize Energy scores for OOD detection. 
    
    \item \textbf{CuBERT}~\cite{kanade2020learning}. CuBERT is pre-trained on a large code corpus using masked language modeling, then fine-tuned for bug detection and repair. We adapt CuBERT for OOD classification by alternatively fine-tuning it on our datasets with comments appended to their corresponding code, as CuBERT only accepts single instance inputs.

    \item \textbf{2P-CuBERT}~\cite{he2022distribution}. This method enhances CuBERT's bug detection accuracy with a two-phase fine-tuning approach. The first phase utilizes contrastive learning on generated synthetic buggy code~\cite{allamanis2021self}. For the second phase, we alternatively fine-tune CuBERT to detect OOD using our datasets. Results are reported only for CSN-Python due to the lack of Java bug generation algorithms in~\cite{he2022distribution}.

\end{enumerate}
%\yanfu{please also include the neural bug detection baseline}

\subsection{Main Task Performance Analysis}
An effective OOD detector, serving as an auxiliary component, should identify and reject OOD samples without negatively impacting the original model's performance on the main downstream task with ID data~\cite{zhou2021contrastive}. Consequently, we validate the effectiveness of our COOD/COOD+ auxiliary on the code search task using the official evaluation benchmark~\cite{guo2020graphcodebert, liu2023contrabert} by calculating the mean reciprocal rank (mRR) for each pair of comment-code data over distractor codes in the testing code corpus. Specifically, we first measure the performance of original GraphCodeBERT code search model on both ID and OOD data, whose performance is expected to be negatively affected with the presence of OOD samples. Then, we utilize our COOD/COOD+ auxiliary to filter the testing dataset by setting a threshold to retain 95\% of ID instances with higher scores (following existing ML work~\cite{ming2022delving} and the FPR95 definition), as real-world deployment typically involves few OODs. Finally, we directly use the fine-tuned encoder in COOD/COOD+ to perform code search but on the retained ID instances, and compare this performance with that on the ground-truth ID instances. If the performance loss is recovered by using COOD/COOD+, we actually enhance the trustworthiness and robustness of the original code search model (as shown in Sec. VI-D). Here trustworthiness and robustness mean that predictions of code models become more reliable when encountering OOD data in real-world deployment. Note that the dataset used for COOD/COOD+ training is the same as that used for PL-specific training of existing SOTA code search models. 


% Therefore, we leveraged our COOD+ to first identify and filter out OOD instances, and then directly used the fine-tuned encoder (\ie GraphCodeBERT) of COOD+ to perform code search on ID instances. Specifically, we filtered the corrupted testing dataset by setting a threshold to retain 95\% of ID instances with higher scores[1], given few OODs present in real-world deployment. For code search evaluation, we followed the official evaluation metric to calculate the mean reciprocal rank (mRR) for each pair of comment-code data over distractor codes in the testing code corpus. Current SOTA models for code search are currently trained and evaluated on the CSN dataset per PL, and our COOD+ model is trained exactly on CSN-Java or CSN-Python data, treating data from different distributions than the training data as outliers. Subsequently, we deploy our COOD+ model to filter out outliers and retain at least 95\%~\cite{ming2022delving} of normal samples. This allows us to assess the extent to which our models can safeguard and support pretrained NL-PL models in OOD deployment settings (\textbf{RQ4}).  \yanfu {highlight robustness and trustworthiness}

% Our evaluation results on code search (RQ4) can support the claim of enhancing trustworthiness and robustness of existing code models in real-world applications. Here trustworthiness and robustness imply that predictions of code models become more reliable. For RQ4 (as detailed in Section V-F), we leveraged our COOD+ to first identify and filter out OOD instances, and then directly used the fine-tuned encoder of COOD+ to perform code search on ID instances. Specifically, we filtered the corrupted testing dataset by setting a threshold to retain 95\% of ID instances with higher scores[1], given few OODs present in real-world deployment. For code search evaluation, we followed the official evaluation metric to calculate the mean reciprocal rank (mRR) for each pair of comment-code data over distractor codes in the testing code corpus. Our experimental results in Table V show that if we didn't perform OOD detection, the performance of the GraphCodeBERT-based code search model will drop by ~10\%, indicating a lack of robustness to OODs. In contrast, if it adopts our COOD+ detector, the performance loss will be recovered.

% Since OOD instances can be seen when models are deployed in real-world, we apply our approach for the code search task under various OOD scenarios to support this claim. Using our COOD+ model, we can distinguish OOD instances from ID ones, filter out the OOD instances, and directly leverage the encoder of our COOD+ model to perform code search. As shown in Table V, this process significantly enhances code search performance. The evaluation metric we use for code search involves taking (NL, PL) pairs as input and calculating the mean reciprocal rank (mRR) for each (comment, code) pair over distractor codes from the entire testing code corpus. The mRR metric is widely used by the existing code search models for evaluation. Table V demonstrates that accepting outliers during inference leads to a performance decline of ~10\% mRR across both datasets. Our COOD+ detector addresses this issue by identifying and filtering out OOD instances, which recover the performance loss and also improve the code search performance by ~4\%. Given that code search is a real world task, the OOD scenario design is realistic, and the datasets (CodeSearchNet and TLCS) used are also sourced from real-world, we claim that our model can enhance the trustworthiness of existing code models in real-world.

% Since OOD instances are common when models are deployed in the real-world, we apply our COOD+ detector to the code search task (RQ4) under various OOD scenarios to support this claim.  Although Sec. V-F (Main Task Analysis) describes how to use our COOD+ model for code search with the evaluation metric, we agree on the need for a clearer explanation of RQ4 and the metric involved. For RQ4, we leverage our COOD+ model to first identify and filter out OOD instances, and then directly use the fine-tuned encoder of our COOD+ to perform code search on ID instances. We follow the official evaluation metric to calculate the mean reciprocal rank (mRR) for each pair of comment-code data over distractor codes in the testing code corpus. By corrupting 15\% of the testing dataset with OOD samples, we show that the performance of a (original) fine-tuned GraphCodeBERT code search model declines by ~10\% (72.20\%-62.32\% on CSN-Java and 73.37\%-63.23\% on CSN-Python) in Table V, indicating a lack of robustness to OODs. Using COOD+ detector, we filter the corrupted testing dataset by setting a threshold to retain 95\% of ID instances with higher scores (following existing work in NLP [1]), given that the real-world deployment usually involves few OODs. We then use COOD+’s encoder to perform code search on the filtered dataset. Table V shows that our model can recover this performance loss and also improve the code search performance by ~3% (75.57%-72.20% on CSN-Java and 76.99%-73.37% on CSN Python).

% Our evaluation results on the code search task (RQ4) can support the claim of trustworthiness and robustness of existing code models in real-world applications. Please note that here trustworthiness and robustness means the predictions of code models are more reliable. For RQ4, we leveraged our COOD+ model to first identify and filter out OOD instances, and then directly used the fine-tuned encoder of our COOD+ to perform code search on ID instances. We followed the official evaluation metric to calculate the mean reciprocal rank (mRR) for each pair of comment-code data over distractor codes in the testing code corpus. Our experimental results in Table V shows that if we did not perform OOD detection, the performance of the GraphCodeBERT-based code search model will drop by ~10%. In contrast, if it adopts our OOD detector, the performance will ==(XXXX, please write the remaining sentences, then remove the following paragraph)= By corrupting 15% of the testing dataset with OOD samples, we showed that the performance of a (original) fine-tuned GraphCodeBERT code search model declines by ~10% (72.20%-62.32% on CSN-Java and 73.37%-63.23% on CSN-Python) in Table V, indicating a lack of robustness to OODs. Using COOD+ detector, we filtered the corrupted testing dataset by setting a threshold to retain 95% of ID instances with higher scores (following existing work in OOD detection [1]), given that the real-world deployment usually involves few OODs. We then utilized COOD+’s encoder to perform code search on the filtered dataset. Table V shows that our model can recover this performance loss and also improve the code search performance by ~3% (75.57%-72.20% on CSN-Java and 76.99%-73.37% on CSN Python).

% Our evaluation results on the code search task (RQ4) can support the claim of trustworthiness and robustness of existing code models in real-world applications. Please note that here trustworthiness and robustness means the predictions of code models are more reliable. For RQ4 as described in Section V-F, we leveraged our COOD+ model to first identify and filter out OOD instances, and then directly used the fine-tuned encoder of our COOD+ to perform code search on ID instances. Specifically, we filtered the corrupted testing dataset by setting a threshold to retain 95% of ID instances with higher scores[1], given that the real-world deployment usually involves few OODs. For code search evaluation, we followed the official evaluation metric to calculate the mean reciprocal rank (mRR) for each pair of comment-code data over distractor codes in the testing code corpus. Our experimental results in Table V shows that if we did not perform OOD detection, the performance of the GraphCodeBERT-based code search model will drop by ~10%, indicating a lack of robustness to OODs. In contrast, if it adopts our COOD+ detector, the performance loss will be recovered.