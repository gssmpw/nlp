\begin{table}[ht!]
\centering
\caption{\textbf{Super Resolution Performance Results.} Our proposed WGAN EEG Spatial Upsampling method significantly outperforms a baseline of Bicubic Interpolation commonly used in EEG upsampling pipelines.}
\label{tab:results}
\resizebox{0.8\linewidth}{!}{%
\begin{tabular}{@{}cccccc@{}}
\toprule
\multirow{2}{*}{\textbf{Dataset}} & \multirow{2}{*}{\textbf{Scale}} & \multicolumn{2}{c}{\textbf{Bicubic}} & \multicolumn{2}{c}{\textbf{WGAN}} \\ \cmidrule(l){3-6} 
                      &   & \textbf{MSE} & \textbf{MAE} & \textbf{MSE}    & \textbf{MAE}   \\
\toprule
\multirow{2}{*}{Val}  & 2 & 3.71E7       & 3.89E3       & \textbf{2.01E3} & \textbf{24.38} \\
                      & 4 & 7.23E7       & 6.42E3       & \textbf{8.53E3} & \textbf{63.83} \\
\midrule
\multirow{2}{*}{Test} & 2 & 3.75E7       & 3.91E3       & \textbf{2.06E3} & \textbf{24.66} \\
                      & 4 & 7.30E7       & 6.45E3       & \textbf{8.68E3} & \textbf{64.39} \\
\bottomrule
\end{tabular}%
}
\end{table}

\section{Experimental Results}\label{sec:results}
\subsection{RQ1: Unsupervised COOD Performance}
In this subsection, we analyze the experimental results to assess the detection performance of our unsupervised COOD model compared with the unsupervised baselines. According to Table~\ref{tab:results-python} and~\ref{tab:results-java}, we can observe that COOD outperforms all unsupervised baselines on both CSN-Python and CSN-Java. Notably, COOD effectively detect \textit{out-domain} and \textit{misaligned OOD} testing samples, while other unsupervised approaches only work for the \textit{out-domain} scenario. This is because COOD effectively captures alignment information within (comment, code) pairs through a multi-modal contrastive learning objective with InfoNCE loss and uses similarity scores between comments and code to detect OODs. Specifically, COOD outputs low similarity scores for the out-domain data from TLCS by additionally considering the knowledge gap difference in (comment, code) pairs between ID and out-domain data. Also, as the misaligned scenario involves misaligned (comment, code) pairs, their similarity scores are naturally low. In contrast, the unsupervised baselines aggregate misaligned information into classification logits and rely on the confidence of the "aligned" class to detect OODs. As previously discussed in Sec. IV-C, the contrastive losses~\cite{khosla2020supervised, zhou2021contrastive} used by them are not as effective for learning alignment information, leading to inferior performance. Additionally, detecting token-level OOD in \textit{shuffled-comment} and \textit{buggy-code} scenarios proves challenging without seeing OOD samples during training, as all unsupervised methods fail to detect these OODs. 

% Some of the results and analysis could have been explained \begin{table*}[t]
    \centering
    \resizebox{\textwidth}{!}{
\begin{tabular}{l|rrllrrll}
\toprule
\textbf{Dataset} & \multicolumn{4}{c}{\textbf{GSM8K}} & \multicolumn{4}{c}{\textbf{MATH}} \\
\cmidrule(lr){1-1} \cmidrule(lr){2-5} \cmidrule(lr){6-9}
\textbf{Method} & Acc & Len & Rel. Acc & Rel. Len & Acc & Len & Rel. Acc & Rel. Len \\
\midrule
\multicolumn{9}{l}{\textit{Zero-Shot Prompting}} \\
\midrule
\hspace{12pt}Baseline & 78.06 & 241.87 & 100.00 \small{(0.00)} & 100.00 \small{(0.00)} & 46.40 & 480.37 & 100.00 \small{(0.00)} & 100.00 \small{(0.00)} \\
\hspace{12pt}Be Concise & 77.98 & 214.87 & 99.85 \small{(1.18)} & 88.46 \small{(10.37)} & 47.76 & 446.09 & 102.71 \small{(7.59)} & 92.66 \small{(7.46)} \\
\hspace{12pt}Hand Crafted 2 (ours) & 76.72 & 184.13 & 98.27 \small{(3.67)} & 77.10 \small{(22.27)} & 46.84 & 404.85 & 101.62 \small{(4.79)} & 85.26 \small{(15.97)} \\
\midrule
\multicolumn{9}{l}{\textit{FT - External Data}} \\
\midrule
\hspace{12pt}Direct Answer & 19.70 & 3.17 & 24.88 \small{(5.03)} & 1.36 \small{(0.40)} & 15.08 & 6.98 & 35.16 \small{(10.34)} & 1.44 \small{(0.73)} \\
\hspace{12pt}Human CoT & 65.73 & 127.85 & 83.82 \small{(7.28)} & 54.95 \small{(13.17)} & 33.88 & 243.54 & 75.61 \small{(13.56)} & 53.14 \small{(13.87)} \\
\hspace{12pt}GPT4o CoT & 76.36 & 156.24 & 97.65 \small{(3.63)} & 67.60 \small{(16.70)} & 40.44 & 399.80 & 90.52 \small{(15.07)} & 87.21 \small{(22.22)} \\
\midrule
\multicolumn{9}{l}{\textit{FT - Best-of-N Self-Generation}} \\
\midrule
\hspace{12pt}Naive BoN & 77.12 & 214.22 & 98.79 \small{(1.64)} & 87.17 \small{(8.79)} & 47.64 & 433.26 & 101.74 \small{(7.04)} & 89.89 \small{(3.99)} \\
\hspace{12pt}Rational Metareasoning & 76.15 & 207.49 & 97.21 \small{(5.74)} & 84.93 \small{(5.09)} & 47.56 & 432.56 & 103.02 \small{(6.56)} & 90.56 \small{(5.25)} \\
\midrule
\multicolumn{9}{l}{\textit{FT - Few-Shot Conditioned Self-Generation (ours)}} \\
\midrule
\hspace{12pt}FS-Human & 76.66 & 161.72 & 98.06 \small{(3.28)} & 67.96 \small{(16.62)} & 46.44 & 421.54 & 99.69 \small{(6.97)} & 87.78 \small{(5.98)} \\
\hspace{12pt}FS-GPT4o & 78.07 & 175.54 & 99.94 \small{(1.69)} & 73.15 \small{(13.49)} & 47.36 & 421.21 & 101.87 \small{(5.33)} & 87.58 \small{(6.60)} \\
\hspace{12pt}FS-Self & 77.27 & 190.03 & 98.86 \small{(2.51)} & 77.51 \small{(9.18)} & 48.00 & 426.67 & 102.67 \small{(5.24)} & 88.50 \small{(4.49)} \\
\midrule
\multicolumn{9}{l}{\textit{FT - Few-Shot Conditioned Best-of-N Self-Generation (ours)}} \\
\midrule
% GPT4o Best-of-16 (Naive) & 75.48 & 153.51 & 96.56 \small{(3.79)} & 64.12 \small{(16.35)} & 47.28 & 367.49 & 101.50 \small{(9.81)} & 76.96 \small{(11.42)} \\
\hspace{12pt}FS-GPT4o-BoN & 75.88 & 153.38 & 97.00 \small{(4.11)} & 64.25 \small{(16.66)} & 47.36 & 364.33 & 102.56 \small{(6.24)} & 76.30 \small{(10.56)} \\
\hspace{24pt}\raisebox{0.5ex}{$\llcorner$}\hspace{4pt}\textit{Budget-Matched} & 76.24 & 160.59 & 97.44 \small{(3.67)} & 67.15 \small{(16.41)} & 47.52 & 384.43 & 101.58 \small{(9.53)} & 80.43 \small{(9.04)} \\

\bottomrule
\end{tabular}
    }
    \caption{
        Evaluation of zero-shot prompting, fine-tuning (FT), and our proposed methods on GSM8K and MATH datasets, averaged across five main models.
        We report absolute accuracy (\%) and length (tokens), along with relative accuracy (\%) and length (\%) compared to baseline prompting. Values in parentheses show standard deviations.
    }
    \label{tab:main_results}
\end{table*}

\section{Results}
\label{results}

% \begin{table*}[t]
% \resizebox{\textwidth}{!}{
% \centering
% \small
% \begin{tabular}{m{2.5cm}|m{15cm}}
% \toprule
% \textbf{Method} & \textbf{Content} \\ %& \# \textbf{Tokens}\\
% \midrule
% \midrule[0.05em]
% \multicolumn{2}{c}{Question} \\ 
% \midrule
% - & A robe takes 2 bolts of blue fiber and half that much white fiber.  How many bolts in total does it take?  \\
% \midrule
% \midrule[0.05em]
% \multicolumn{2}{c}{Reasoning Paths} \\
% \midrule
% Zero-Shot & To find the total number of bolts needed, we need to calculate the amount of white fiber first, since it's half the amount of blue fiber. Step 1: Determine the amount of blue fiber needed. The robe takes 2 bolts of blue fiber. Step 2: Calculate the amount of white fiber needed. The robe takes half the amount of white fiber as blue fiber. Half of 2 bolts is $2/2$ = 1 bolt. Step 3: Add the amount of blue fiber and white fiber to find the total. Total bolts = blue fiber + white fiber. Total bolts = 2 bolts + 1 bolt. Total bolts = 3 bolts. The answer is \textbf{\textit{3}}. \\ %& 213 \\
% \midrule
% FS-GPT4o-BoN & The robe takes 2 bolts of blue fiber. It takes half that much white fiber, which is $2 / 2$ = 1 bolt. Add the blue and white fiber together: 2 + 1 = 3 bolts. The answer is \textbf{\textit{3}}. \\ %57 \\
% \bottomrule
% \end{tabular}
% }
% \caption{Example rationales generated from the original Llama-3.1-8B model and the model fine-tuned using our method. The question is randomly sampled from GSM8K. For the same question, one can observe that our rationale is much more concise, removing redundant parts that explain the question.}
% \label{tab:samples}
% \vspace{-0.1in}
% \end{table*}


\begin{table*}[t]
\resizebox{\textwidth}{!}{
\centering
\normalsize
\begin{tabular}{m{17.5cm}}
\toprule
\multicolumn{1}{c}{\textbf{Question}} \\ 
\midrule
A robe takes 2 bolts of blue fiber and half that much white fiber.  How many bolts in total does it take? \\
\midrule
\multicolumn{1}{c}{\textbf{Reasoning Paths}} \\
\midrule
\textbf{[Zero-Shot]} 
To find the total number of bolts needed, we need to calculate the amount of white fiber first, since it's half the amount of blue fiber. Step 1: Determine the amount of blue fiber needed. The robe takes 2 bolts of blue fiber. Step 2: Calculate the amount of white fiber needed. The robe takes half the amount of white fiber as blue fiber. Half of 2 bolts is $2/2$ = 1 bolt. Step 3: Add the amount of blue fiber and white fiber to find the total. Total bolts = blue fiber + white fiber. Total bolts = 2 bolts + 1 bolt. Total bolts = 3 bolts. The answer is 3. \\
\midrule
\textbf{[FS-GPT4o-BoN]} 
The robe takes 2 bolts of blue fiber. It takes half that much white fiber, which is $2 / 2$ = 1 bolt. Add the blue and white fiber together: 2 + 1 = 3 bolts. The answer is 3. \\
\bottomrule
\end{tabular}
}
\caption{Example rationales generated from the original Llama-3.1-8B model (\textbf{Zero-Shot}) and the model fine-tuned using our method (\textbf{FS-GPT4o-BoN}). The question is randomly sampled from GSM8K. For the same question, one can observe that our rationale is much more concise, removing redundant parts that explain the question.}
\label{tab:samples}
\vspace{-0.1in}
\end{table*}


\subsection{Main results}

Our main results, presented in \autoref{tab:main_results} and \autoref{fig:main_methods_comparison}, demonstrate the performance of our self-training methods against baseline approaches.
% We highlight key observations from these results below.

\paragraph{Naive BoN fine-tuning is effective but sample inefficient.}
Naive BoN fine-tuning effectively reduces output length without significantly degrading model performance. 
This also holds true for Qwen2.5-Math-1.5B and DeepSeekMath-7B (\autoref{tab:main_results_full_gsm8k} and \autoref{tab:main_results_full_math}), which failed to achieve length reduction through zero-shot prompting.
% However, while naive BoN does reduce output length, the reduction is limited to 12\%.
However, the length reduction from naive BoN with $N=16$ is limited to 12\% on average.
Furthermore, as illustrated in Figure~\ref{fig:bon_sample_efficiency}, achieving more compression with BoN becomes progressively less efficient.

\paragraph{Iterative baseline yields similar results as naive BoN fine-tuning.}
% We compare our single-step naive BoN approach with Rational Metareasoning \cite{de2024rational}, an iterative approach using expert iteration \cite{zelikman2022star}  which incorporates an additional \textit{utility reward} to balance efficiency and accuracy in BoN sampling.
Rational Metareasoning, an iterative baseline, yields similar relative length reduction and relative accuracy to BoN fine-tuning. 
This suggests that the utility reward proposed by \citet{de2024rational} may not effectively achieve both accuracy gains and token length reduction.

\begin{figure}[t] % "h" places the figure roughly here
    \centering
    \includegraphics[width=\columnwidth]{figures/main_methods_comparison.pdf} % Adjust width as needed
    \caption{Tradeoff between relative accuracy and length reduction for main methods. Results are averaged over GSM8K and MATH across five main models. Matching colors and shapes indicate the same FS prompt. FS conditioning without augmentation (†) are marked with lighter colors. 
    Relative length reduction refers to 100 - relative length (\%).}
    \label{fig:main_methods_comparison} % Label for referencing in text
\end{figure}
% \red{TODO - shorten this}

\paragraph{Few-shot conditioning outperforms BoN in length reduction.}
The results demonstrate that few-shot conditioning achieves a greater relative length reduction compared to naive BoN, including math-specialized models (\autoref{tab:main_results_full_gsm8k} and \autoref{tab:main_results_full_math}).
% This reduction is attributed to the fact that the fine-tuning datasets generated through few-shot conditioning contain shorter reasoning paths compared to those generated by naive BoN, as illustrated in \autoref{fig:bon_sample_efficiency}.  % too long
This is in line with the superior length reduction of few-shot conditioning, compared to naive BoN as shown in \autoref{fig:bon_sample_efficiency}.
Notably, self-training on generations conditioned on human-annotated examples (FS-Human) achieves an average relative length of 67.96\% on GSM8K, compared to 87.17\% with naive BoN.  % good to have some specific numbers in the text
% We further analyze the effect of fine-tuning on length reduction in \autoref{analysis}.



\paragraph{Self-training better preserves accuracy than training with external data.} 
\autoref{tab:main_results} shows fine-tuning with external data (\textit{FT-External Data}) leads to a significant reduction in relative length but causes a severe drop in relative accuracy. 
% \autoref{fig:main_methods_comparison} further highlights that while fine-tuning with GPT-4o CoT (FT-GPT4o) achieves slightly greater reduction in relative length than fine-tuning with self-generated data using few-shots from GPT-4o (FS-GPT4o), it results in substantially lower relative accuracy.  % a bit complicated / not concrete (conrete evidence = one where we beat external FT in both accuracy and reduction)
\autoref{fig:main_methods_comparison} further highlights the accuracy preservation of self-training: fine-tuning with external concise reasoning supervision from GPT-4o (FT-GPT4o) lies below the Pareto-curve of relative accuracy and relative length reduction, established by our self-training methods.
% NAMGYU - TODO add some commentary

\paragraph{Few-shot conditioned BoN achieves best length reduction while maintaining accuracy.}
% Few-shot conditioned BoN enables substantial length reduction compared to all other BoN and few-shot methods while maintaining relative accuracy.
FS-BoN elicits the largest length reduction among our self-training methods, while maintaining relative accuracy, on average.
Notably, for math-specialized models, FS-GPT4o-BoN achieves the greatest reduction among all methods, except those fine-tuned on external data which greatly sacrifice the accuracy (\autoref{tab:main_results_full_gsm8k} and \autoref{tab:main_results_full_math}). 
% This result reflects how applying BoN to few-shot conditioning further reduces the relative length of the training data while also increasing the proportion of correct samples.  % unnecessary

\paragraph{Augmentation boosts accuracy for few-shot conditioning.}
\autoref{fig:main_methods_comparison} compares few-shot conditioning, i.e., FS and FS-BoN, with and without augmentation (†). 
Augmentation improves accuracy by providing solutions for previously unsolvable hard questions as discussed in \autoref{sample_augmentation}. 
While augmentation may slightly affect reduction rates, they remain superior to naive BoN and RM.
% Similar effect is observed for augmentation in FS-BoN.
% Even when matching the budget (\textit{Budget-Matched}) with other fine-tuning methods using self-generated data in \autoref{tab:main_results}, it achieves the greatest length reduction among them with minimal accuracy degradation.
Even when matching the budget (\textit{Budget-Matched}) with other self-training methods in \autoref{tab:main_results}, it achieves the greatest length reduction among them with minimal accuracy degradation.
The effect of augmentation on training data length is analyzed in \autoref{appx_augmentation_length}.
% Furthermore, as shown in Figure \ref{fig:main_methods_comparison}, augmentation on few-shot conditioned BoN enhances accuracy similar to naive BoN and Meta-Reasoning while achieving greater length reduction.

\begin{figure}[t]
    \centering
    \includegraphics[width=\columnwidth]{figures/length_by_difficulty.pdf} % Adjust width as needed
    \caption{\textbf{Tokens are reduced adaptively according to question difficulty.} 
    Token reduction rate for each difficulty level on MATH, for 4 models individually and averaged.
    % Higher difficulty levels show lower reduction rates.
    Relative length reduction refers to 100 - relative length (\%).
    }
    \label{fig:length_difficulty} % Label for referencing in text
\end{figure}

\subsection{Analysis}
\label{analysis}
% This section analyzes length reduction: transfer from generation to fine-tuning, reduction by question difficulty, qualitative analysis, and consistency across model sizes. DeepSeekMath-7B is excluded from quantitative analysis due to cost.
% let's keep this short
In this section, we analyze the length reduction effects in depth.
We exclude DeepSeekMath-7B from quantiative analysis due to cost.


% \paragraph{Analysis on sample efficiency}
% As shown in \autoref{fig:bon_sample_efficiency}, best-of-n (BoN) sampling requires a substantial number of samples to be generated to achieve a level of reasoning length reduction comparable to that achievable through few-shot conditioning.
% In other words, it is infeasible to reach the reasoning length reduction performance of few-shot conditioning using BoN alone, without generating a prohibitively large number of samples.
% However, our experiments consistently demonstrate that combining few-shot conditioning with BoN sampling is more effective in reducing reasoning length than using either technique in isolation.
% Specifically, few-shot conditioning helps to guide the model towards generating more concise reasoning paths, while BoN sampling allows us to select the shortest and most accurate path from a diverse set of candidates.
% This synergistic effect results in a more efficient and effective approach to concise reasoning.


% \paragraph{FT can reduce generation length effectively.}
% As shown in \autoref{fig:ft_length_scatter}, after fine-tuning, the models tend to follow the length of the training data, suggesting that reasoning length reduction can be achieved through simple supervised fine-tuning on short reasoning samples.
% Note that test generation length is relatively longer than the training data length, as the models can generate lengthy incorrect answers, while the training data consists of correct answers.
% Correctly generated answers align more closely with training data length as shown in (Appendix~\ref{appx_length_scatter_correct}).

% \paragraph{Length reduction through generation and fine-tuning}
% Our method reduces reasoning length in two stages: generation and fine-tuning.
% First, as shown in \autoref{fig:ft_length_scatter}, 
% % generation length for training data varies depending on the method. 
% few-shot conditioning methods produce shorter outputs than naive BoN, with few-shot conditioned BoN achieving the shortest. 
% Second, fine-tuning with shorter rationales results in shorter model outputs, showing a strong correlation between test and training lengths\footnote{Test generation lengths are generally longer than training data lengths due to the possibility of lengthy incorrect answers during testing. Test outputs that are correct align more closely with training data lengths, as shown in Appendix~\ref{appx_length_scatter_correct}.}.
% Overall, FS-GPT4o-BoN effectively generates and trains for shorter reasoning paths.
% Additionally, unlike zero-shot methods, our approach significantly reduces token length in math-tuned models like Qwen2.5-Math-1.5B with FS-GPT4o-BoN, achieving 54.7\% relative length after fine-tuning. (See \autoref{tab:main_results_full_gsm8k} and \autoref{tab:main_results_full_math}).

\paragraph{Tokens are reduced adaptively according to question complexity.} 
The MATH dataset's difficulty levels range from 1 (basic algebra) to 5 (advanced calculus and complex mathematical reasoning).
As shown in \autoref{fig:length_difficulty}, our method adaptively reduces tokens based on question difficulty, with higher difficulty leading to less reduction.
% Most models achieve their peak reduction (around 20\%--40\%) at difficulty levels 1-2, where simple concepts allow for more concise explanations.
% The reduction rate gradually declines at levels 3-5, indicating our method's ability to preserve necessary details for complex problems automatically.
%  -> not precise. simple concepts allow for more concise explanations *in absolute terms*, but this does not necessarily mean that length reduction *relative to the default* should be high. E.g., if the model already uses very few tokens for easy questions, then relative reduction would be low.
The higher reduction (20\%--40\%) at easier difficulty levels (1--2) suggests that the original model outputs for these easier questions contained unnecessary tokens.
This reveals a gap in current models' ability to tailor their inference budget to problem complexity.
Our method effectively closes this gap by reducing redundancy, allowing for more precise token allocation based on question difficulty.

\begin{figure}[t] % "h" places the figure roughly here
    \centering
    \includegraphics[width=\columnwidth]{figures/scaling_methods_comparison.pdf} % Adjust width as needed
    \caption{Scaling study on baseline and few-shot conditioned self-training methods. Results are averaged over GSM8K and MATH for Llama 1B, 3B, and 8B.
    % Accuracy tends to be maintained, with greater length reduction using our FS-GPT4o(-BoN) method.
    Relative length reduction refers to 100 - relative length (\%).
    }
    \label{fig:scaling_methods_comparison} % Label for referencing in text
\end{figure}

\paragraph{Self-training maintains consistency across model scales.}
We conduct a scaling study on Llama-3.2-1B, 3B, and Llama-3.1-8B to examine consistency across different model sizes (\autoref{fig:scaling_methods_comparison}). 
Overall, token reduction increases as the model size increases, while the maintenance of accuracy does not show a strong correlation with model size. 
RM exhibits lower reduction rates compared to our few-shot conditioned self-training methods across all models and shows a decrease in accuracy with increasing model size. 
% The few-shot method also shows a similar trend in length reduction, but it achieves the best relative accuracy in the 3B model.
Our standalone few-shot conditioning method (FS-GPT4o) also shows a similar trend in length reduction, but better preserves accuracy.
Our joint FS-GPT4o-BoN method achieves the greatest reduction across all models, maintaining relative accuracy across different model sizes, especially in the largest 8B model.



\paragraph{Sample study}
\autoref{tab:samples} presents qualitative examples of reasoning paths generated by the model before and after fine-tuning with our method. 
The original reasoning exhibits verbosity, containing redundant processes such as question confirmation and repeated instructions. 
In contrast, the reasoning generated by our method includes only the necessary steps, significantly reducing the number of tokens while still arriving at the correct answer. 
% These examples demonstrate the effectiveness of our method in reducing token count. 
More examples are provided in the \autoref{appx_sample_studies}.

\begin{figure}[t]
    \centering
    \includegraphics[width=\columnwidth]{figures/both_length_scatter.pdf} % Adjust width as needed
    \caption{\textbf{Fine-tuning effectively transfers the length reduction to the model.} Correlation between the relative length of train data and test output averaged over GSM8K and MATH across 4 models. Training length includes only correct solutions. Solid points represent test lengths including all generated outputs, while lighter points indicate test lengths of correct solutions only.}
    \label{fig:ft_length_scatter} % Label for referencing in text
\end{figure}

\paragraph{Length reduction is transferred through fine-tuning.}
As shown in \autoref{fig:ft_length_scatter}, fine-tuning with shorter rationales results in shorter model outputs, showing a strong correlation between test and training lengths.
% Test generation lengths (solid datapoints) are generally longer than training data lengths due to the possibility of lengthy incorrect answers during testing.
% However, when comparing with test generation lengths that are correct (lighter datapoints), they align more closely with training data lengths.
We note that the length of test outputs (incorrect and correct) are longer than the length of training samples (only correct) on average.
This is mainly because incorrect paths are generally longer than correct ones.
We find a closer correspondence between train length and test length of correct samples only, indicated by the lighter datapoints.
This discrepancy suggests the need to terminate incorrect paths early to minimize redundant inference overhead.
We consider this for future work.
in more detail. For instance, in Section VI-A, it would have been better to explain why the proposed COOD method significantly outperforms the unsupervised baselines on the out-domain and misaligned scenarios individually and overall, while illustrating Table III. 

% Explain the evaluation setup and results more clearly. e.g., the analysis of the results in Section VI-A

\subsection{RQ2: Weakly-supervised COOD+ Performance}
We further investigate the performance of our weakly-supervised COOD+ method against several weakly-supervised baselines on CSN-Python and CSN-Java. Table~\ref{tab:results-python} shows that weak supervision on a tiny amount of OOD data enables COOD+ (and EOE) to not only address unsupervised COOD's shortcomings in detecting finer-grained \textit{shuffled-comment} and \textit{buggy-code} OODs, but also enhance performance for the \textit{out-domain} scenario for CSN-Python. This improvement aligns with previous research ~\cite{hendrycks2018deep, liu2020energy, kim-etal-2023-pseudo} which enhances OOD detection by complementing the downstream task objective with an complementary discriminator operating to distinguish IDs from external OODs. While EOE slightly outperforms COOD+ for the \textit{out-domain} and \textit{shuffled-comment} scenarios by utilizing the prediction probabilities from one classification module, our COOD+, which combines the BC and CL modules, delivers consistently high performance across all four scenarios, resulting in superior overall performance. In addition, the BC module can be directly adapted to the overall COOD+ framework without modifying the underlying learning objective, but the outlier exposure-based methods (\eg EOE) typically require additional engineering (\eg determining class-probabilistic distributions~\cite{hendrycks2018deep}, boundaries for energy scores~\cite{liu2020energy}) to equip ML models with OOD detection abilities. Besides, the bug detection method 2P-CuBERT can reasonably detect OODs, but its performance for the \textit{buggy-code} scenario is negatively impacted by the limited amount of training OOD examples.

On the CSN-Java dataset, our COOD+ also achieves the best overall performance compared to all baselines, despite trailing slightly behind EOE for \textit{out-domain} and \textit{shuffled-comment} OODs. While EOE has higher AUROC score than that of COOD+ for the \textit{buggy-code} scenario, it suffers from a high FPR95, indicating a higher margin of error for OOD inference using a threshold of 95\% ID recall. Moreover, similar to CSN-Python, CuBERT fails to detect OODs effectively on CSN-Java either, likely due to the lack of training examples. In summary, the superior performance of our COOD+ model results from the interplay between the CL and BL modules, where contrastive learning captures high-level alignment between NL-PL input pairs that is naturally suitable for \textit{out-domain} and \textit{misaligned} OODs, while the OOD rejection classifier targets lower-level OOD information from \textit{shuffled-comment} and \textit{buggy-code} samples. Furthermore, by utilizing a weakly-supervised contrastive learning objective that jointly optimizes for OOD detection and the code search task, our method enables effective deployment of the code search model in OOD environments, which will be further studied in Sec. VI-D.


\subsection{RQ3: Weakly-Supervised COOD+ Performance with Different Model Components and Encoder Backbone}
In this subsection, we evaluate the effect of using only the CL (COOD+\_CL) or the BC module (COOD+\_BC) against the proposed combined COOD+ model to illustrate how COOD+ generalizes in four OOD scenarios. As shown in Table~\ref{tab:results-python} and~\ref{tab:results-java}, COOD+\_CL performs well in the \textit{out-domain} and \textit{misaligned} scenarios, which is due to its ability to effectively capture high-level (comment, code) alignment information. COOD+\_BC excels in the \textit{out-domain}, \textit{shuffled-comment}, and \textit{buggy-code} scenarios, since it can learn lower-level features from these types of OOD samples. While COOD+\_BC maintains acceptable OOD detection performance with high AUROC ($>$90\%) and low FPR95 ($<$25\%), the CL module remains crucial for overall performance, since without it the overall performance of COOD+ will drop below the EOE baseline. Moreover, removing the BC module has a more negative impact on the OOD detection as COOD+ loses the ability to capture the necessary lower-level OOD information for detecting \textit{shuffled-comment} and \textit{buggy-code} OODs. Note that the standalone CL module performs better than the unsupervised COOD overall, demonstrating that our proposed modification to the original CL objective enhance OOD detection by leveraging the margin-based loss. Thus, the combined model's superior performance validates our design choices. That is, the combined scoring function (cosine similarities from CL and the prediction probabilities from BC) is thoughtfully designed to leverage the advantage of each module for high detection accuracy.

% How do unsupervised model and weakly-supervised models generalize in four OOD scenarios?

\begin{table*}
  [t]
  \centering
  \resizebox{\textwidth}{!}{%
  \begin{tabular}{cccccccccccc}
    \toprule \multicolumn{2}{c}{Components}                                                             & \multicolumn{5}{c}{Re-executability Rate (\%)} & \multicolumn{5}{c}{Readability (\#)} \\
    \cmidrule(lr){1-2} \cmidrule(lr){3-7} \cmidrule(lr){8-12}        \hspace{8pt}\labelemoji\hspace{8pt}                                                                & \hspace{8pt}\toolemoji\hspace{8pt}                                      & O0                                 & O1             & O2             & O3             & AVG            & O0             & O1             & O2             & O3             & AVG            \\
    \hline
    \rowcolor[rgb]{0.93,0.93,0.93}\multicolumn{12}{c}{\textbf{Initialize with LLM4Decompile-End-6.7B~\citep{llm4decompile}}}   \\
    \xmark                                                                                              & \xmark                                    & 69.51                              & 46.95          & 50.61          & 46.34          & 53.35          & 3.98 & 3.41 & 3.44 & 3.38 & 3.55 \\
    \cmark                                                                                              & \xmark                                    & 75.61                              & 50.61          & 50.00          & 50.00          & 56.55          & 4.01 & 3.44 & 3.39 & \textbf{3.49} & 3.58 \\
    \xmark                                                                                              & \cmark                                    & 83.54                     & \textbf{56.10}          & 51.22          & 50.61 & 60.37 & 4.05 & 3.51 & 3.51 & 3.42 & 3.62 \\
    \cmark                                                                                              & \cmark                                    & \textbf{85.37}                            & \textbf{56.10}                     & \textbf{51.83} & \textbf{52.43}          & \textbf{61.43} & \textbf{4.13} & \textbf{3.60} & \textbf{3.54} & \textbf{3.49} & \textbf{3.69} \\

    \rowcolor[rgb]{0.93,0.93,0.93}\multicolumn{12}{c}{\textbf{Initialize with Deepseek-Coder-6.7B-base~\citep{deepseekcoder}}} \\
    \xmark                                                                                              & \xmark                                    & 59.15                              & 35.98          & 39.02          & 37.80          & 42.99          & 3.71 & 3.05 & 3.16 & 3.05 & 3.24 \\
    \cmark                                                                                              & \xmark                                    & 66.46                              & 41.46          & 38.41          & 36.59          & 45.73          & 3.76 & 3.17 & \textbf{3.21} & 3.08 & 3.31 \\
    \xmark                                                                                              & \cmark                                    & 70.73                              & 39.63          & 39.02          & 40.24          & 47.41          & 3.90 & 3.17 & 3.08 & 3.11 & 3.31 \\
    \cmark                                                                                              & \cmark                                    & \textbf{79.88}                     & \textbf{45.73} & \textbf{43.90} & \textbf{42.68} & \textbf{53.05} & \textbf{3.96} & \textbf{3.21} & 3.18 & \textbf{3.19} & \textbf{3.38} \\
    \bottomrule
  \end{tabular}%
  }
  \caption{The ablation study of different methods across four optimization levels
  (O0, O1, O2, O3), as well as their average scores (AVG). The results in bold represent the optimal performance. The ~\labelemoji~ and ~\toolemoji~ means Relabedling and Function Call. \textbf{Bold} denotes the best performance.}
  \label{tab:ablation}
\end{table*}
Moreover, we compare the detection performance of our COOD+ with various underlying NL-PL pre-trained encoder. Specifically, we compare our choice of GraphCodeBERT~\cite{guo2020graphcodebert} against other NL-PL encoders from the literature including its predecessor, CodeBERT~\cite{feng2020codebert}, and more recent ones such as UniXcoder~\cite{guo2022unixcoder} and ContraBERT~\cite{liu2023contrabert}. As shown in Table~\ref{tab:ablation}, all encoders perform within a 1-2\% difference, indicating that our COOD+ framework is robust across different encoders. This demonstrates our framework's flexibility and effectiveness in detecting OODs when deploying various NL-PL encoders for code-related tasks. Furthermore, we investigate key hyperparameters in COOD+, such as $m$ for margin-based contrastive loss and  $\lambda$ in the overall loss function. The detailed results are available in our online appendix~\cite{cood-tool}.


%%%---margin-based loss----
% \noindent
% \textbf{Effect of Margin-based Loss and its Hyperparameter.} We explore the impact of the margin $m$ in Eq.~\ref{eq:contrast_id} and Eq.~\ref{eq:contrast_ood} on the detection performance. Table ~\ref{tab:ablation} shows the detection performance under different margin parameters. When $m=0$, the proposed model does not use the margin, meaning that it arbitrarily encourages the OOD cosine similarities to be lower than that of ID samples. We can observe from Table~\ref{tab:ablation} that our method has the best performance when $m=0.4$. In addition, our detection method without the margin-based loss (when $m=0$) indeed performs worse, although the decrease in performance is offset by the rejection classifier. Therefore, we can conclude that the proposed margin-based loss plays an important role in multi-modal OOD detection since it enables OOD-aware finetuning for pretrained NL-PL transformer-based models.\\

% \noindent
% \textbf{Effect of Weight in the Objective.} We also evaluate the performance of the proposed model for different $\lambda$ values in the overall objective function (\ref{eq:overall_obj}) \shao{typos here} as it increases from $0.2$, to $0.5$, and to $1$. As observed in Table~\ref{tab:ablation}, our model performs best when $\lambda=0.2$ on two datasets. One possible reason is that the NL-PL encoders are pretrained on multiple PLs, which requires more weight in the contrastive learning module for language-specific representation learning during optimization. In addition, as representations learned by transformer-based encoders transfer well to fully-connected classification components~\cite{feng2020codebert, guo2020graphcodebert}, the binary OOD rejection classifier needs smaller weight updates during back-propagation to perform effectively.\\

% \noindent \textbf{Effect of Different NL-PL Encoders.}


% \noindent 


\subsection{RQ4: Main Task Performance}
We present the code search performance under the impact of OOD instances by using GraphCodeBERT (GCB), COOD/COOD+, and the closest competitor EOE in Table~\ref{tab:maintask}. As described in Sec. V-F, we use the official metric mRR  and follow the same testing scheme as the original GraphCodeBERT code search model for evaluation. From Table~\ref{tab:maintask}, we first observe that our COOD/COOD+ achieves performance comparable to GraphCodeBERT, while the EOE suffers from a significant reduction in performance, as it reformulates code search as binary classification to gain OOD detection ability. This reveals a critical trade-off between OOD detection and downstream task performance. To further validate the importance of OOD detection for code search, we construct outliers based on the CSN-Java and -Python testing dataset, respectively. Given that code search aims to retrieve the most aligned code from a code corpus given an NL query, the outliers are only sampled from three OOD scenarios: \textit{out-domain}, \textit{shuffled-comment} and \textit{buggy-code}, each replacing 5\% ID data of the original testing set. We then show the results when the dataset contains 15\% OOD samples (\ie 15\% outliers), discard OOD samples by filtering the testing set by ground-truth labels (\ie Filtered-GT) or using various OOD detection models (\ie Filtered-OOD-model). Note that the Filtered-GT dataset is the original CSN's subset with 15\% of ID samples removed.

\begin{table}[!tb]
\centering
\caption{Code search performance under the impact of OOD detection. Higher numbers represent better performance}
\label{tab:maintask}
% \setlength{\tabcolsep}{3pt}
\begin{adjustbox}{width=0.48\textwidth}\small
\begin{tabular}{c|c|c|c|c|c}
\toprule
\textbf{Dataset}            & \textbf{Testing Subset} & \textbf{GCB} & \textbf{EOE} & \textbf{COOD} & \textbf{COOD+} \\ \hline
\multirow{4}{*}{CSN-Python} & Origin                  & 69.20        & 50.11        & 68.47         & 69.69          \\
                            & 15\% outliers            & 65.85        & 43.68        & 64.67         & 65.67          \\
                            & Filtered-GT  & 70.24        & 44.85        & 68.95         & 70.24          \\
                            & Filtered-OOD-model     & --           & 46.82        & 70.30         & \textbf{73.10} \\ \hline
\multirow{4}{*}{CSN-Java}   & Origin                  & 69.10        & 46.29        & 68.85         & 69.46          \\
                            & 15\% outliers            & 64.99        & 37.77        & 64.86         & 64.54          \\
                            & Filtered-GT  & 69.12        & 38.94        & 69.36         & 69.93          \\
                            & Filtered-OOD-model     & --           & 39.33        & 71.02         & \textbf{73.18} \\ \bottomrule
\end{tabular}
\end{adjustbox}
\end{table}

According to Table~\ref{tab:maintask}, the performance of the original GraphCodeBERT code search model drops by 4.84\% and 5.95\% ((69.10-64.99)/69.10) mRR when outliers are present in CSN-Python and -Java, respectively. As a solution to this issue, our COOD/COOD+ detector recover the performance losses by identifying and filtering out the OOD samples without negatively impacting the model's code understanding ability in code search. Specifically, the code search performance of COOD/COOD+ on the Filtered-COOD/COOD+ dataset (70.30\%/73.10\% and 71.02\%/73.18\% on CSN-Python and -Java, respectively) is comparable to or even better than GraphCodeBERT on the Filtered-GT dataset (70.24\% and 69.12\% on CSN-Python and -Java, respectively). This slight improvement is probably because our detectors filter out additional lower-quality testing samples that resemble outliers. Thus, our COOD/COOD+ enhance the trustworthiness and robustness of the GraphCodeBERT, since the model's predictions become more reliable when encountering OOD data. % Also, although both COOD and COOD+ can recover the GraphCodeBERT's performance losses, COOD+ still perform better than COOD on each possible testing subset, especially the Filtered-OOD-model dataset. %Compared to only using 15\% outliers in this experiment, when more OOD instances from diverse OOD scenarios are encountered (\eg 75\% outliers in the OOD detection experiments), we expect the performance difference between COOD+ and COOD to be more significant. 
Note that the original GraphCodeBERT is not equipped with the OOD detection ability, so its corresponding cells for the Filtered-OOD-model in Table~\ref{tab:maintask} are left blank.

% It is not clear how RQ4 (Main Task Performance) is answered using Table V since we do not know what those numbers mean for CSN-Java and CSN-Python. It would have been better if the rationale of RQ4 is explained clearly along with the metrics that have been used to answer this RQ.  it is not clear why RQ4 is addressed and why only COOD+ is considered rather than both COOD and COOD+

% \subsection{RQ5: Overconfident Case}
% \viet{Yanfu: Please move these into where you want} Given an OOD testing sample, DNN models (pre)trained on ID data are prone to predict a higher MSP confidence score than the threshold and wrongly identify it as an
% ID sample~\cite{lee2017training, liang2017enhancing}. This overconfidence issue limits the effectiveness of OOD detection. For NL data, this is caused by the the spurious correlation between OOD and ID features such as entities and syntactic structures~\cite{zheng2020out, wu2022revisit}. Such correlation also occurs in PL data. For example, an OOD PL input with the syntactic structure ``def ... if ... return ... else ... return ..." may receive an ID score if this pattern is commonly used in other ID inputs. To overcome overconfident predictions, previous works explored techniques such as temperature scaling~\cite{liang2017enhancing}, confidence calibration using adversarial samples~\cite{lee2017training, bitterwolf2020certifiably}, or adaptive class-dependent threshold~\cite{wu2022revisit}. In contrast, our proposed COOD+ utilizes a weakly-supervised contrastive learning objective to take advantage of a small number of OOD samples during training and prevent the alignment between OOD pairs. Moreover, we adopt the binary OOD rejection module to discriminate the fused OOD and ID representations. We further verify whether COOD+ overcome the overconfidence issue through the lens of Conformal Prediction~\cite{angelopoulos2023conformal}.

% Conformal Prediction (CP) involves post-processing uncertainty quantification techniques that are model-agnostic, and provide statistical guarantees on the predictions of a trained model~\cite{angelopoulos2023conformal}. The commonly used split CP technique first computes the nonconformity scores, which are OOD scores in our case, on a calibration set independent of the training data. Then, it builds a prediction set for each testing sample $\mathcal{C}_\alpha(t^{test}_l, c^{test}_l)$ satisfying the condition $P(y_l^{test}\in \mathcal{C}_\alpha(t^{test}_l, c^{test}_l))\geq 1-\alpha$, where $\alpha$ is a small error rate (e.g., 0.05) that the user is willing to tolerate. Here, this condition guarantees that the true outcome is covered by the prediction set with probability $1-\alpha$, which is also known as the CP coverage. When CP is applied to the OOD detection scores, all scores have the same statistical guarantee, but better OOD scores will give tighter prediction sets. Conversely, worse scores will give large and uninformative prediction sets, which corresponds to ineffective OOD detection caused by overconfident predictions.

% In our experiment, we apply split CP by reserving 20\% of the testing samples from each testing dataset (CSN-Java and CSN-Python) for CP calibration, and construct the prediction sets with tolerable error rate $\alpha=0.05$ on the remaining testing samples. To assess how effectively COOD+ addresses the overconfidence issue, we compare its average prediction set (P-Set) size (between 1 and 2 for binary predictions) with that of selected baselines including MCL+MSP, the best performing approach using MSP OOD scores, and our proposed COOD. As observed in Table~\ref{tab:conformal}, the proposed COOD+ achieves the smallest prediction sets on both datasets. Specifically, the vast majority of prediction sets obtained by COOD+ contain one value that is 95\% statistically guaranteed to be the true OOD label according to the CP condition, indicating the minimal overconfidence of OOD scores. In contrast, the MCL+MSP method is prone to overconfidence, because it produces large prediction sets (i.e., size 2) including both IDs and overconfident OODs. Additionally, without utilizing OOD samples during training, COOD cannot effectively prevent overconfident predictions. Note that in the CP context, although higher coverage is desired, the main goal is to build the smallest prediction sets given the user-specified error rate of 0.05. Therefore, COOD+ is the most effective at overcoming the overconfidence barrier despite its slightly lower coverage than that of MCL+MSP and COOD.

% \begin{table}[!tb]
% \centering
% \caption{Effectiveness of COOD+ compared to selected methods for overcoming overconfident OOD predictions.}
% \small
% \label{tab:conformal}
% \begin{adjustbox}{width=\linewidth,center}
% \begin{tabular}{c|cc|cc}
% \toprule
% \multirow{2}{*}{\textbf{Methods}} & \multicolumn{2}{c|}{\textbf{CSN-Java}} & \multicolumn{2}{c}{\textbf{CSN-Python}} \\ \cline{2-5} 
% & \textbf{Coverage↑}    & \textbf{ P-Set Size↓}   & \textbf{Coverage↑}    & \textbf{P-Set Size↓}    \\ \hline
% MCL+MSP                      &  \textbf{97.03}    &   1.891   &   95.00   &      1.834          \\
% COOD                           &   95.66            &      1.593         &   \textbf{95.81}   &  1.576     \\
% COOD+                          & 95.31              & \textbf{1.077}              &       95.35        &      \textbf{1.010}          \\
% \bottomrule
% \end{tabular}
% \end{adjustbox}
% \end{table}