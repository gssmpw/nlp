
\begin{table*}[t]
\centering
\fontsize{11pt}{11pt}\selectfont
\begin{tabular}{lllllllllllll}
\toprule
\multicolumn{1}{c}{\textbf{task}} & \multicolumn{2}{c}{\textbf{Mir}} & \multicolumn{2}{c}{\textbf{Lai}} & \multicolumn{2}{c}{\textbf{Ziegen.}} & \multicolumn{2}{c}{\textbf{Cao}} & \multicolumn{2}{c}{\textbf{Alva-Man.}} & \multicolumn{1}{c}{\textbf{avg.}} & \textbf{\begin{tabular}[c]{@{}l@{}}avg.\\ rank\end{tabular}} \\
\multicolumn{1}{c}{\textbf{metrics}} & \multicolumn{1}{c}{\textbf{cor.}} & \multicolumn{1}{c}{\textbf{p-v.}} & \multicolumn{1}{c}{\textbf{cor.}} & \multicolumn{1}{c}{\textbf{p-v.}} & \multicolumn{1}{c}{\textbf{cor.}} & \multicolumn{1}{c}{\textbf{p-v.}} & \multicolumn{1}{c}{\textbf{cor.}} & \multicolumn{1}{c}{\textbf{p-v.}} & \multicolumn{1}{c}{\textbf{cor.}} & \multicolumn{1}{c}{\textbf{p-v.}} &  &  \\ \midrule
\textbf{S-Bleu} & 0.50 & 0.0 & 0.47 & 0.0 & 0.59 & 0.0 & 0.58 & 0.0 & 0.68 & 0.0 & 0.57 & 5.8 \\
\textbf{R-Bleu} & -- & -- & 0.27 & 0.0 & 0.30 & 0.0 & -- & -- & -- & -- & - &  \\
\textbf{S-Meteor} & 0.49 & 0.0 & 0.48 & 0.0 & 0.61 & 0.0 & 0.57 & 0.0 & 0.64 & 0.0 & 0.56 & 6.1 \\
\textbf{R-Meteor} & -- & -- & 0.34 & 0.0 & 0.26 & 0.0 & -- & -- & -- & -- & - &  \\
\textbf{S-Bertscore} & \textbf{0.53} & 0.0 & {\ul 0.80} & 0.0 & \textbf{0.70} & 0.0 & {\ul 0.66} & 0.0 & {\ul0.78} & 0.0 & \textbf{0.69} & \textbf{1.7} \\
\textbf{R-Bertscore} & -- & -- & 0.51 & 0.0 & 0.38 & 0.0 & -- & -- & -- & -- & - &  \\
\textbf{S-Bleurt} & {\ul 0.52} & 0.0 & {\ul 0.80} & 0.0 & 0.60 & 0.0 & \textbf{0.70} & 0.0 & \textbf{0.80} & 0.0 & {\ul 0.68} & {\ul 2.3} \\
\textbf{R-Bleurt} & -- & -- & 0.59 & 0.0 & -0.05 & 0.13 & -- & -- & -- & -- & - &  \\
\textbf{S-Cosine} & 0.51 & 0.0 & 0.69 & 0.0 & {\ul 0.62} & 0.0 & 0.61 & 0.0 & 0.65 & 0.0 & 0.62 & 4.4 \\
\textbf{R-Cosine} & -- & -- & 0.40 & 0.0 & 0.29 & 0.0 & -- & -- & -- & -- & - & \\ \midrule
\textbf{QuestEval} & 0.23 & 0.0 & 0.25 & 0.0 & 0.49 & 0.0 & 0.47 & 0.0 & 0.62 & 0.0 & 0.41 & 9.0 \\
\textbf{LLaMa3} & 0.36 & 0.0 & \textbf{0.84} & 0.0 & {\ul{0.62}} & 0.0 & 0.61 & 0.0 &  0.76 & 0.0 & 0.64 & 3.6 \\
\textbf{our (3b)} & 0.49 & 0.0 & 0.73 & 0.0 & 0.54 & 0.0 & 0.53 & 0.0 & 0.7 & 0.0 & 0.60 & 5.8 \\
\textbf{our (8b)} & 0.48 & 0.0 & 0.73 & 0.0 & 0.52 & 0.0 & 0.53 & 0.0 & 0.7 & 0.0 & 0.59 & 6.3 \\  \bottomrule
\end{tabular}
\caption{Pearson correlation on human evaluation on system output. `R-': reference-based. `S-': source-based.}
\label{tab:sys}
\end{table*}



\begin{table}%[]
\centering
\fontsize{11pt}{11pt}\selectfont
\begin{tabular}{llllll}
\toprule
\multicolumn{1}{c}{\textbf{task}} & \multicolumn{1}{c}{\textbf{Lai}} & \multicolumn{1}{c}{\textbf{Zei.}} & \multicolumn{1}{c}{\textbf{Scia.}} & \textbf{} & \textbf{} \\ 
\multicolumn{1}{c}{\textbf{metrics}} & \multicolumn{1}{c}{\textbf{cor.}} & \multicolumn{1}{c}{\textbf{cor.}} & \multicolumn{1}{c}{\textbf{cor.}} & \textbf{avg.} & \textbf{\begin{tabular}[c]{@{}l@{}}avg.\\ rank\end{tabular}} \\ \midrule
\textbf{S-Bleu} & 0.40 & 0.40 & 0.19* & 0.33 & 7.67 \\
\textbf{S-Meteor} & 0.41 & 0.42 & 0.16* & 0.33 & 7.33 \\
\textbf{S-BertS.} & {\ul0.58} & 0.47 & 0.31 & 0.45 & 3.67 \\
\textbf{S-Bleurt} & 0.45 & {\ul 0.54} & {\ul 0.37} & 0.45 & {\ul 3.33} \\
\textbf{S-Cosine} & 0.56 & 0.52 & 0.3 & {\ul 0.46} & {\ul 3.33} \\ \midrule
\textbf{QuestE.} & 0.27 & 0.35 & 0.06* & 0.23 & 9.00 \\
\textbf{LlaMA3} & \textbf{0.6} & \textbf{0.67} & \textbf{0.51} & \textbf{0.59} & \textbf{1.0} \\
\textbf{Our (3b)} & 0.51 & 0.49 & 0.23* & 0.39 & 4.83 \\
\textbf{Our (8b)} & 0.52 & 0.49 & 0.22* & 0.43 & 4.83 \\ \bottomrule
\end{tabular}
\caption{Pearson correlation on human ratings on reference output. *not significant; we cannot reject the null hypothesis of zero correlation}
\label{tab:ref}
\end{table}


\begin{table*}%[]
\centering
\fontsize{11pt}{11pt}\selectfont
\begin{tabular}{lllllllll}
\toprule
\textbf{task} & \multicolumn{1}{c}{\textbf{ALL}} & \multicolumn{1}{c}{\textbf{sentiment}} & \multicolumn{1}{c}{\textbf{detoxify}} & \multicolumn{1}{c}{\textbf{catchy}} & \multicolumn{1}{c}{\textbf{polite}} & \multicolumn{1}{c}{\textbf{persuasive}} & \multicolumn{1}{c}{\textbf{formal}} & \textbf{\begin{tabular}[c]{@{}l@{}}avg. \\ rank\end{tabular}} \\
\textbf{metrics} & \multicolumn{1}{c}{\textbf{cor.}} & \multicolumn{1}{c}{\textbf{cor.}} & \multicolumn{1}{c}{\textbf{cor.}} & \multicolumn{1}{c}{\textbf{cor.}} & \multicolumn{1}{c}{\textbf{cor.}} & \multicolumn{1}{c}{\textbf{cor.}} & \multicolumn{1}{c}{\textbf{cor.}} &  \\ \midrule
\textbf{S-Bleu} & -0.17 & -0.82 & -0.45 & -0.12* & -0.1* & -0.05 & -0.21 & 8.42 \\
\textbf{R-Bleu} & - & -0.5 & -0.45 &  &  &  &  &  \\
\textbf{S-Meteor} & -0.07* & -0.55 & -0.4 & -0.01* & 0.1* & -0.16 & -0.04* & 7.67 \\
\textbf{R-Meteor} & - & -0.17* & -0.39 & - & - & - & - & - \\
\textbf{S-BertScore} & 0.11 & -0.38 & -0.07* & -0.17* & 0.28 & 0.12 & 0.25 & 6.0 \\
\textbf{R-BertScore} & - & -0.02* & -0.21* & - & - & - & - & - \\
\textbf{S-Bleurt} & 0.29 & 0.05* & 0.45 & 0.06* & 0.29 & 0.23 & 0.46 & 4.2 \\
\textbf{R-Bleurt} & - &  0.21 & 0.38 & - & - & - & - & - \\
\textbf{S-Cosine} & 0.01* & -0.5 & -0.13* & -0.19* & 0.05* & -0.05* & 0.15* & 7.42 \\
\textbf{R-Cosine} & - & -0.11* & -0.16* & - & - & - & - & - \\ \midrule
\textbf{QuestEval} & 0.21 & {\ul{0.29}} & 0.23 & 0.37 & 0.19* & 0.35 & 0.14* & 4.67 \\
\textbf{LlaMA3} & \textbf{0.82} & \textbf{0.80} & \textbf{0.72} & \textbf{0.84} & \textbf{0.84} & \textbf{0.90} & \textbf{0.88} & \textbf{1.00} \\
\textbf{Our (3b)} & 0.47 & -0.11* & 0.37 & 0.61 & 0.53 & 0.54 & 0.66 & 3.5 \\
\textbf{Our (8b)} & {\ul{0.57}} & 0.09* & {\ul 0.49} & {\ul 0.72} & {\ul 0.64} & {\ul 0.62} & {\ul 0.67} & {\ul 2.17} \\ \bottomrule
\end{tabular}
\caption{Pearson correlation on human ratings on our constructed test set. 'R-': reference-based. 'S-': source-based. *not significant; we cannot reject the null hypothesis of zero correlation}
\label{tab:con}
\end{table*}

\section{Results}
We benchmark the different metrics on the different datasets using correlation to human judgement. For content preservation, we show results split on data with system output, reference output and our constructed test set: we show that the data source for evaluation leads to different conclusions on the metrics. In addition, we examine whether the metrics can rank style transfer systems similar to humans. On style strength, we likewise show correlations between human judgment and zero-shot evaluation approaches. When applicable, we summarize results by reporting the average correlation. And the average ranking of the metric per dataset (by ranking which metric obtains the highest correlation to human judgement per dataset). 

\subsection{Content preservation}
\paragraph{How do data sources affect the conclusion on best metric?}
The conclusions about the metrics' performance change radically depending on whether we use system output data, reference output, or our constructed test set. Ideally, a good metric correlates highly with humans on any data source. Ideally, for meta-evaluation, a metric should correlate consistently across all data sources, but the following shows that the correlations indicate different things, and the conclusion on the best metric should be drawn carefully.

Looking at the metrics correlations with humans on the data source with system output (Table~\ref{tab:sys}), we see a relatively high correlation for many of the metrics on many tasks. The overall best metrics are S-BertScore and S-BLEURT (avg+avg rank). We see no notable difference in our method of using the 3B or 8B model as the backbone.

Examining the average correlations based on data with reference output (Table~\ref{tab:ref}), now the zero-shoot prompting with LlaMA3 70B is the best-performing approach ($0.59$ avg). Tied for second place are source-based cosine embedding ($0.46$ avg), BLEURT ($0.45$ avg) and BertScore ($0.45$ avg). Our method follows on a 5. place: here, the 8b version (($0.43$ avg)) shows a bit stronger results than 3b ($0.39$ avg). The fact that the conclusions change, whether looking at reference or system output, confirms the observations made by \citet{scialom-etal-2021-questeval} on simplicity transfer.   

Now consider the results on our test set (Table~\ref{tab:con}): Several metrics show low or no correlation; we even see a significantly negative correlation for some metrics on ALL (BLEU) and for specific subparts of our test set for BLEU, Meteor, BertScore, Cosine. On the other end, LlaMA3 70B is again performing best, showing strong results ($0.82$ in ALL). The runner-up is now our 8B method, with a gap to the 3B version ($0.57$ vs $0.47$ in ALL). Note our method still shows zero correlation for the sentiment task. After, ranks BLEURT ($0.29$), QuestEval ($0.21$), BertScore ($0.11$), Cosine ($0.01$).  

On our test set, we find that some metrics that correlate relatively well on the other datasets, now exhibit low correlation. Hence, with our test set, we can now support the logical reasoning with data evidence: Evaluation of content preservation for style transfer needs to take the style shift into account. This conclusion could not be drawn using the existing data sources: We hypothesise that for the data with system-based output, successful output happens to be very similar to the source sentence and vice versa, and reference-based output might not contain server mistakes as they are gold references. Thus, none of the existing data sources tests the limits of the metrics.  


\paragraph{How do reference-based metrics compare to source-based ones?} Reference-based metrics show a lower correlation than the source-based counterpart for all metrics on both datasets with ratings on references (Table~\ref{tab:sys}). As discussed previously, reference-based metrics for style transfer have the drawback that many different good solutions on a rewrite might exist and not only one similar to a reference.


\paragraph{How well can the metrics rank the performance of style transfer methods?}
We compare the metrics' ability to judge the best style transfer methods w.r.t. the human annotations: Several of the data sources contain samples from different style transfer systems. In order to use metrics to assess the quality of the style transfer system, metrics should correctly find the best-performing system. Hence, we evaluate whether the metrics for content preservation provide the same system ranking as human evaluators. We take the mean of the score for every output on each system and the mean of the human annotations; we compare the systems using the Kendall's Tau correlation. 

We find only the evaluation using the dataset Mir, Lai, and Ziegen to result in significant correlations, probably because of sparsity in a number of system tests (App.~\ref{app:dataset}). Our method (8b) is the only metric providing a perfect ranking of the style transfer system on the Lai data, and Llama3 70B the only one on the Ziegen data. Results in App.~\ref{app:results}. 


\subsection{Style strength results}
%Evaluating style strengths is a challenging task. 
Llama3 70B shows better overall results than our method. However, our method scores higher than Llama3 70B on 2 out of 6 datasets, but it also exhibits zero correlation on one task (Table~\ref{tab:styleresults}).%More work i s needed on evaluating style strengths. 
 
\begin{table}%[]
\fontsize{11pt}{11pt}\selectfont
\begin{tabular}{lccc}
\toprule
\multicolumn{1}{c}{\textbf{}} & \textbf{LlaMA3} & \textbf{Our (3b)} & \textbf{Our (8b)} \\ \midrule
\textbf{Mir} & 0.46 & 0.54 & \textbf{0.57} \\
\textbf{Lai} & \textbf{0.57} & 0.18 & 0.19 \\
\textbf{Ziegen.} & 0.25 & 0.27 & \textbf{0.32} \\
\textbf{Alva-M.} & \textbf{0.59} & 0.03* & 0.02* \\
\textbf{Scialom} & \textbf{0.62} & 0.45 & 0.44 \\
\textbf{\begin{tabular}[c]{@{}l@{}}Our Test\end{tabular}} & \textbf{0.63} & 0.46 & 0.48 \\ \bottomrule
\end{tabular}
\caption{Style strength: Pearson correlation to human ratings. *not significant; we cannot reject the null hypothesis of zero corelation}
\label{tab:styleresults}
\end{table}

\subsection{Ablation}
We conduct several runs of the methods using LLMs with variations in instructions/prompts (App.~\ref{app:method}). We observe that the lower the correlation on a task, the higher the variation between the different runs. For our method, we only observe low variance between the runs.
None of the variations leads to different conclusions of the meta-evaluation. Results in App.~\ref{app:results}.

\section{Experimental Results}\label{sec:results}
\subsection{RQ1: Unsupervised COOD Performance}
In this subsection, we analyze the experimental results to assess the detection performance of our unsupervised COOD model compared with the unsupervised baselines. According to Table~\ref{tab:results-python} and~\ref{tab:results-java}, we can observe that COOD outperforms all unsupervised baselines on both CSN-Python and CSN-Java. Notably, COOD effectively detect \textit{out-domain} and \textit{misaligned OOD} testing samples, while other unsupervised approaches only work for the \textit{out-domain} scenario. This is because COOD effectively captures alignment information within (comment, code) pairs through a multi-modal contrastive learning objective with InfoNCE loss and uses similarity scores between comments and code to detect OODs. Specifically, COOD outputs low similarity scores for the out-domain data from TLCS by additionally considering the knowledge gap difference in (comment, code) pairs between ID and out-domain data. Also, as the misaligned scenario involves misaligned (comment, code) pairs, their similarity scores are naturally low. In contrast, the unsupervised baselines aggregate misaligned information into classification logits and rely on the confidence of the "aligned" class to detect OODs. As previously discussed in Sec. IV-C, the contrastive losses~\cite{khosla2020supervised, zhou2021contrastive} used by them are not as effective for learning alignment information, leading to inferior performance. Additionally, detecting token-level OOD in \textit{shuffled-comment} and \textit{buggy-code} scenarios proves challenging without seeing OOD samples during training, as all unsupervised methods fail to detect these OODs. 

% Some of the results and analysis could have been explained \section{Results}
We first describe communication patterns within the full chronological context of the game in \textit{League of Legends (LoL)}, separated into four sections based on changing coordination dynamics. Based on this context, we identify core factors players assess to decide when to participate in communication with other teammates. Afterward, we discuss how communication shapes player perceptions toward their teammates, showing player's wariness towards players actively engaging in communication. 

\subsection{Communication Patterns in Context}

We discuss the communication patterns among teammates within the game. We organize the data into chronological phases of the game for a structured analysis of how the context shapes communication patterns. 

\subsubsection{Pre-game stage}
Before gameplay begins, team communication opens with \textit{team drafting}, where players are assigned roles (Top, Mid, Bot, Support, or Jungle) and take turns picking or banning champions. In Solo Ranked mode, roles are pre-assigned based on player preferences selected before queueing. Once teams are set, all players enter \textit{champion select} stage, alternating champion picks and banning up to five champions per team. During this stage, communication is limited to text chat. The usernames are anonymized (i.e., replacing the name with aliases) to prevent queue dodging by checking third-party stats sites such as OP.GG\footnote{https://www.op.gg/}, leaving the chat as the only option to inform individual strengths and preferences. 

Team composition in \textit{LoL} is crucial to the strategy and outcome of the game~\cite{ong2015player}, setting the basis for future interactions. Most participants acknowledged the importance of balanced and synergistic team composition, especially as players move into higher ranks where team coordination outweighs individual excellence. Yet, we observed a distinct lack of verbal communication between the members during this period across all ranks. Participants attributed the lack of willingness to initiate a conversation on the dangers of starting the game on a bad footing. They prioritized ``not creating friction'' during this stage as negative impressions can propagate throughout the game. Some participants attempted communication to reduce such friction, such as P14, who stated,``\textit{If I had the time, I wanted to say that I will be banning [this Champion], just in case a player on my team wanted to play them.}'' However, several participants viewed any communication during the pre-game phase with wariness, as dissatisfaction or conflict at this step portended negative interactions between players in the game (P3, P9, P15). Thus, even when participants expressed doubt about other teammates' unconventional or non-meta champion picks, they refrained from entering into discourse. This contrasts with findings by Kou and Gui~\cite{kou2014}, which showed players attempt to maintain a harmonious and constructive atmosphere through greetings and introductions.

Another emergent code of the reason for not engaging in communication in the pre-game stage stems from different purposes of playing the game (P1, P5, P13, P16, P17). Despite being in ranked mode, which is more prone to increased competitiveness and effort, participants showed differing goals and levels of interest in winning the game. Several players stated that they had previously exerted great mental load in coordinating synergistic plays, but stopped as they gave less importance to winning at all costs (``\textit{I don't really play to win. I play \textit{LoL} to relieve stress, so I don't engage in chat.}'', P5). These players saw verbal communication with the goal of coordination as an unnecessary or even cumbersome component of the pre-game stage.


\subsubsection{Structured phase}
In many MOBAs, including \textit{LoL}, the early stages of the game play out in a formulaic manner: players join their lanes (Top, Mid, and Bot/Support), defeat minions to gain gold, buy items towards certain ``builds'', kill or assist in early objectives (Jungle), and battle counterparts in their respective lanes. Participants at this stage expressed that most players possessed tacit knowledge of what must be done, such as knowing when to aid their Jungle to capture a jungle monster, choosing the opportune moments to leave their lanes, or positioning wards (i.e., a deployable unit which provides a vision of the surrounding area) at the ideal placements. The participants assumed each player knew their ``role'' to fulfill, often comparing it to ``doing their share'' (P1, P3, P7, P19). In line with this belief, players rarely initiated preemptive or proactive verbal communication for strategic or social purposes at the early stage. 

Pings, on the other hand, constantly permeated the game. At this stage, players used ping to provide information relevant to others from their position, such as letting others know if an enemy went missing from their lane. As the players are largely separated and independent from one another, pings (coupled with the minimap and scoreboard) served as the primary channel for maintaining context over the game rather than as warnings or direct guidance to the players. For other non-verbal gestures, while objective votes would occasionally appear, they were rarely answered. Instead, relevant players near the objective would place pings or move toward it to help out their teammates.

Participants viewed the structured phase as a routine, but uncertain period of the game where the pendulum could swing in either team's favor. Players --- especially Jungles who roam the board looking for opportunities to ambush the enemy team in lanes (``gank'') --- sometimes felt hesitant to make calls and demands at this stage since ``\textit{[they] could make a call, but if I fail, they'll start blaming my decisions down the line.}'' (P7) But at this stage, participants believed that they held personal agency over the final game outcome. P1 and P6 stated that they entered the game with the mindset that only they had to succeed regardless of the performance of their teammates. This belief was reflected in their chatting behavior, where players prioritized focusing on their circumstances over the team's (``\textit{I mute the chat so that I don't get swayed by the team, as I can win the game if I do well.}'', P9).


\subsubsection{Group engagement phase}
As the game enters its middle phase, it provides opportunities for more diverse decision-making. Players may swap lanes, seize or trade crucial objectives, and fight in large battles involving multiple champions. At this point, teams typically have a clear outlook on which players and team have the advantage, requiring more team-driven decisions to maintain or overcome their current standing. Thus, players used verbal communication to discuss more complicated tactics that could not be effectively conveyed through pings.

But more often than not, chat messages became judgment-based. As enemy engagement with larger groups occurred more frequently, the availability for chatting would come after death, which led to comments on past actions rather than future choices. Additionally, the respawn timer for deaths becomes longer as the game progresses, providing more time to observe other players than in earlier phases. This gave players more opportunities to express dissatisfaction specifically towards certain plays, such as placing Enemy Missing pings on the map where other teammates are located to bring attention to their questionable play.

This stage also gave much more exposure of each other to the allies as the team would gather at a single point, giving way to greater scrutiny by their teammates. Repeated or critical mistakes put participants on edge, as they braced for criticism from their teammates. They expressed relief or surprise when the chat remained silent or civil, with P8 stating ``\textit{I messed up there. No one is saying anything, thankfully.}''


\subsubsection{Point of no return}
Meanwhile, verbal communication flowed out when the game had a clear trajectory to the end. Previous research has shown that both toxic and non-toxic communication skyrockets near the end of the game~\cite{kwak2015linguistic} when the players have determined the game outcome with certainty. We saw that this phase opened up both positive and negative sides of communication for guaranteed win and loss, respectively. The winning team would compliment and cheer each other through chat messages and emotes, while the losing side devolved into arguments and calling out. The communication at this stage was driven by emotion, showing excitement or venting frustration.


\subsection{Communication Assessment Process}

We describe the factors that users mainly focus on to assess when or when not to involve themselves in communication with their teammates. 

\subsubsection{Calculating communication cost}
One of the most proximate factors behind when communication is performed is the limited action economy of the game. In \textit{LoL} and other MOBAs, players can rarely afford time to type out messages due to the fast-paced nature of the game. In time-sensitive scenarios, the time pressure makes communication particularly costly. It is therefore unsurprising that much of the communication occurs after major events (e.g., battles and objective hunting), as players are given more downtime while waiting for teammates or enemies to respawn or regroup.

For periods where players were still actively involved in gameplay, the players made conscious decisions on choosing which communication media to use based on the perceived action availability and the importance of communicating the message. Players relied on pings for non-critical indications, believing that the mutual understanding of the game would get the message across. However, many players recognized that pings were prone to be missed, ignored, or misinterpreted by their allies (P2, P9, P16, P17, P20). Subsequently, participants typed out information considered to be too important to the situation to be misunderstood or missed by other players even if it caused delays in their gameplay (P10, P11, P14). Simultaneously, the priority of importance constantly shifted --- we observed multiple times participants start to type, but stop to react to an ongoing play, only to never send out their message.

\subsubsection{Evaluating relevance and responsiveness}
When the brief window of communication opportunity is missed, players are unlikely to ever send out that information. In \textit{LoL}, situations can change within seconds and certain communication media cannot keep up with the changing state of the game. For example, almost all study participants did not participate in votes for objectives. Among the tens of objective votes initiated among all the games in this study, no objective vote saw more than three votes, frequently being left with no vote beyond the player who initiated the vote. Some players, when asked why they did not participate, stated that the votes they made often became irrelevant as the game state had changed during the time it took to vote (P2, P11). Other players also discussed how information conveyed through communication can get outdated fast (P1, P8, P9). 

\begin{quote}
I can't always follow through with what I say [in the chat] since the game is really dynamic. My teammates don't understand such situations, so I tend to not chat proactively. - P9
\end{quote}

Thus, some players instead preferred to react through direct action (P8, P10, P11, P16, P20). P10 stated, ``\textit{I think it's enough to show through action rather than [using objective voting]. I can look out for how the player reacts when I request something from them.}''

On the other hand, such action-based responses left the player to assess whether and how the communication was received. P10 stated that they tried to predict whether a player understood their ping direction by how they moved, but it was hard to interpret their intent: ``\textit{members sometimes seem to move towards me but then turn around, and sometimes they even ping back but don't come.}''. P16 discussed how they weren't sure whether the ping was received, but performed it anyway since it felt helpful.

Similarly, participation in surrender votes (or lack thereof) carried different intent by the player. During most of the games that ended in a loss, one or more surrender votes were called by the participant's team. However, only two surrender votes achieved four or more players' participation. However, the reasons why a player chose to not participate varied. Some had decided to wait and see how other teammates voted, which may have paradoxically led many members to not participate in the vote (P4, P9). Meanwhile, others didn't reply as they didn't think the vote was actually calling for a response: P13 stated, ``\textit{I didn't vote because they were just showing their anger. It's just a member venting through a surrender vote that they're not doing well.}''

\subsubsection{Balancing information access and psychological safety}
While recognizing that communication would be useful or even necessary in certain situations, participants also put their psychological safety first over information access. Some players, worn down by the normalization of toxic communication such as flaming, muted the chat (P1, P9).

Many participants expressed the sentiment of ``protecting [their] mentality'', describing how certain communication harmed their psychological well-being. This communication did not always refer to negative communication; P9 often muted players who gave commands as they did not want to be ``swept up'' by others' play-related judgments. This separation even extended to other more widely considered essential communication forms, such as pings. Even after acknowledging that pings were vital and useful to the game, P9 went as far as muting the ping of the support player in the same lane after they sent a barrage of Enemy Missing pings that signified aggression and criticism. 

Additionally, the abundance and high frequency of communication also strained the limited mental capacity of the players. Many players, when asked why they had not replied to an objective vote or other chat messages, stated that they simply did not notice them among other events happening (P1, P2, P3, P9, P12, P15, P18, P20, P21). The information overload caused stress and became distracting to players.

\subsubsection{Reducing potential friction}
As demonstrated in the pre-game stage of the game, players sometimes used communication to minimize friction between their teammates. Some participants sacrificed time to apologize to other players when they believed themselves to be at fault. When asked why, P12 replied, ``\textit{There are too many people who don't come to help gank if I don't apologize.}''. Similarly, P5 sacrificed time typing in an apology after a teammate had died despite still being in the middle of a fight as they didn't wish to give the other player a reason to start an attack.

However, some noted that silence is sometimes the best answer to a negative situation. P4, after dying to the enemy, put into chat ``Fighting!'' (roughly meaning, ``We can do it!''). They stated ``\textit{I don't know why I do it... it probably angers [my teammates] more.
}'' They also stated that ``\textit{for certain people, talking in the chat only spurs them more. You just have to let them be.}'' Other players shared similar sentiments that being quiet and dedicating focus to the game was a better choice (P1, P11, P14).

For female players, the fear of gender-based harassment shaped their communication patterns. While \textit{LoL} does not provide any demographic information of a player to other players, almost all female participants noted experiences of receiving derogatory remarks or doubts about their abilities based on other players' assumptions of their gender, a trend frequently seen in male-dominated online gaming cultures~\cite{fox2016women, norris2004gender, mclean2019female}. They noted that the players were able to correctly guess their gender when the participant's role and champion fit into the preconceived notions of what women ``tended to play'' (i.e., female-identifying support champions, such as Lulu) or their username ``seemed feminine'' (P18, P19, P20, P21, P22). This led to certain players adopting tactics that signaled male-like behavior, such as changing their speech style to be more gender-neutral or male-like (P19, P21) and changing their username to sound more gender-neutral. Cote describes similar instances of ``camouflaging gender'' as one of five main strategies for women coping with harassment~\cite{cote2017coping}. However, some players opted to keep playing their preferred character or maintaining their username even if it signaled their gender, such as P21 who expressed, ``\textit{I cherish and feel attached to my username, so I don’t want to change it just because of [harassment and inappropriate comments].}'' These players valued self-expression and identity even at the risk of increased risk to unpleasant communication experiences.


\subsubsection{Forming performance-based hierachy}
Naturally formed leadership has often been observed in other works on \textit{LoL} teams~\cite{kou2014}. Kim et al. showed that more hierarchy in in-game decision-making led to higher collective intelligence~\cite{kim2017}. While they used ``hierarchy'' to mean varying amounts of communication throughout the game, we observed that the hierarchy extends further to performance-based hierarchy, where teammates in more advantageous positions are given greater weight when communicating with other players. Players actively chose to refrain from suggesting strategic plans when they were ``holding down the team'', recognizing that they held less power and trust among the team members (P8, P10, P12, P14, P22). The player who was losing against the enemy team was viewed as having no ``right'' to lead the team, which was reserved for well-performing players.


\subsubsection{Enforcing norms and habits}
One of the most common answers to why players performed certain communication actions, especially non-verbal actions such as pings and emotes, was ``a force of habit'' (P6, P7, P8, P9, P10, P12, P17). Players formed learned practices of using communication channels at certain points by observing other players exhibit the same behaviors. This promoted, for example, replying to an emote sent by the teammate with their own or pinging readied skills and items to emphasize relevant information for other players throughout the game. 

On the other hand, this meant that players were averse to communication patterns outside of the norm --- participants stated that they had a hard time adapting to new forms of communication, seeing no immediate benefit or impact from using them (P1, P8, P14, P13, P15, P17). Most egregiously, the recently introduced objective pings were largely viewed to be awkward to use and unnecessary (P1, P4, P8, P12).


\subsection{Impact of Communication Assessment}
We describe how the communication patterns and assessment of the players impact the individual players' perspectives on team dynamics.

\subsubsection{Relationship between trust and communication frequency}
Most participants saw value in constant and well-informed communication but with an important distinction: verbal communication with strangers rarely ended well. Players largely recognized frequent verbal communication to burgeon conflict, regardless of the message within. Even when players understood the helpful intent behind positive messages from the players, they compared actively talking players to be possible bad actors who were likely to exhibit toxic behaviors when the game turned against them. (P1, P4, P8, P12, P14)

\begin{quote}
I need to make sure to not disturb Twisted Fate. I saw him start to flame. It's not because I don't want to hear more criticism. I know these types. The more I react and chat with them, the more deviant they will become. - P4  
\end{quote}

Similarly, P19 lamented that players used to socialize more in the chat during the pre-game phase to build a fun and prosocial environment, noting a memorable example of encouraging each other to do well on their academic exams, but noted that such prosocial behavior has become much rarer during the recent seasons. They noted that there are inevitably players ``who take it negatively'' and thus stopped proactively typing non-game related messages in the chat.

Ultimately, players desired assurance and trust of player commitment. The participants trusted actions more than words to prove that the player remained dedicated to the game. Both P10 and P17 pointed out that it was easy to tell who was still ``in the game'' and motivated to try their best and that ``staying on the keyboard'' likely meant that they weren't invested or focused on the game. Players viewed such commitment to be the most important aspect of a ``good'' teammate, sometimes even more than their skill or performance (P9, P14). It is interesting to note that unlike what previous literature may suggest~\cite{marlow2018}, players' averseness to talkative teammates had less to do with the cognitive overload or distraction caused by the frequent communication, but rather due to the threats of future team breakdown. This view in turn also affected how players decided to communicate or not, as they believed that players would not take their suggestions or comments in a positive light. 


\subsubsection{Perception of player commitment and fortitude}

Communication also acted as a mirror of their teammates' mental fortitude. A number of players mentioned how they valued a resilient mindset in their teammates playing the game, referring to players who remained committed to the game until the very end. They saw players who provoked or complained to teammates as ``having a weak mentality'' who had been altered by the bad outcomes of the game to act in an unhelpful manner towards the team through their communication. The communication actions of the teammate informed the participants of how steadfast their teammate remained in disadvantageous situations.  

\begin{quote}
It's not like I constantly reply in the chat or anything, but I pay attention [to the chat] to grasp the overall atmosphere of the team. If the team doesn't collaborate well then we lose, so I try to have a rough understanding of the mentality of the other players. - P13
\end{quote}

There were also instances of communication that helped players maintain a positive view of their teammates. For example, P11 mentioned near the beginning of the game, ``\textit{Looking at the chat, Varus player has strong mentality [for being so positive]. There were lots of points [in his support's] plays that he could have criticized.}'' Unfortunately, this view quickly soured when the Varus player devolved into criticism later in the late game phase where the Varus player started criticizing the support and other players. P11 then noted that the Varus player seemed to merely be ``bearing through the game''.in more detail. For instance, in Section VI-A, it would have been better to explain why the proposed COOD method significantly outperforms the unsupervised baselines on the out-domain and misaligned scenarios individually and overall, while illustrating Table III. 

% Explain the evaluation setup and results more clearly. e.g., the analysis of the results in Section VI-A

\subsection{RQ2: Weakly-supervised COOD+ Performance}
We further investigate the performance of our weakly-supervised COOD+ method against several weakly-supervised baselines on CSN-Python and CSN-Java. Table~\ref{tab:results-python} shows that weak supervision on a tiny amount of OOD data enables COOD+ (and EOE) to not only address unsupervised COOD's shortcomings in detecting finer-grained \textit{shuffled-comment} and \textit{buggy-code} OODs, but also enhance performance for the \textit{out-domain} scenario for CSN-Python. This improvement aligns with previous research ~\cite{hendrycks2018deep, liu2020energy, kim-etal-2023-pseudo} which enhances OOD detection by complementing the downstream task objective with an complementary discriminator operating to distinguish IDs from external OODs. While EOE slightly outperforms COOD+ for the \textit{out-domain} and \textit{shuffled-comment} scenarios by utilizing the prediction probabilities from one classification module, our COOD+, which combines the BC and CL modules, delivers consistently high performance across all four scenarios, resulting in superior overall performance. In addition, the BC module can be directly adapted to the overall COOD+ framework without modifying the underlying learning objective, but the outlier exposure-based methods (\eg EOE) typically require additional engineering (\eg determining class-probabilistic distributions~\cite{hendrycks2018deep}, boundaries for energy scores~\cite{liu2020energy}) to equip ML models with OOD detection abilities. Besides, the bug detection method 2P-CuBERT can reasonably detect OODs, but its performance for the \textit{buggy-code} scenario is negatively impacted by the limited amount of training OOD examples.

On the CSN-Java dataset, our COOD+ also achieves the best overall performance compared to all baselines, despite trailing slightly behind EOE for \textit{out-domain} and \textit{shuffled-comment} OODs. While EOE has higher AUROC score than that of COOD+ for the \textit{buggy-code} scenario, it suffers from a high FPR95, indicating a higher margin of error for OOD inference using a threshold of 95\% ID recall. Moreover, similar to CSN-Python, CuBERT fails to detect OODs effectively on CSN-Java either, likely due to the lack of training examples. In summary, the superior performance of our COOD+ model results from the interplay between the CL and BL modules, where contrastive learning captures high-level alignment between NL-PL input pairs that is naturally suitable for \textit{out-domain} and \textit{misaligned} OODs, while the OOD rejection classifier targets lower-level OOD information from \textit{shuffled-comment} and \textit{buggy-code} samples. Furthermore, by utilizing a weakly-supervised contrastive learning objective that jointly optimizes for OOD detection and the code search task, our method enables effective deployment of the code search model in OOD environments, which will be further studied in Sec. VI-D.


\subsection{RQ3: Weakly-Supervised COOD+ Performance with Different Model Components and Encoder Backbone}
In this subsection, we evaluate the effect of using only the CL (COOD+\_CL) or the BC module (COOD+\_BC) against the proposed combined COOD+ model to illustrate how COOD+ generalizes in four OOD scenarios. As shown in Table~\ref{tab:results-python} and~\ref{tab:results-java}, COOD+\_CL performs well in the \textit{out-domain} and \textit{misaligned} scenarios, which is due to its ability to effectively capture high-level (comment, code) alignment information. COOD+\_BC excels in the \textit{out-domain}, \textit{shuffled-comment}, and \textit{buggy-code} scenarios, since it can learn lower-level features from these types of OOD samples. While COOD+\_BC maintains acceptable OOD detection performance with high AUROC ($>$90\%) and low FPR95 ($<$25\%), the CL module remains crucial for overall performance, since without it the overall performance of COOD+ will drop below the EOE baseline. Moreover, removing the BC module has a more negative impact on the OOD detection as COOD+ loses the ability to capture the necessary lower-level OOD information for detecting \textit{shuffled-comment} and \textit{buggy-code} OODs. Note that the standalone CL module performs better than the unsupervised COOD overall, demonstrating that our proposed modification to the original CL objective enhance OOD detection by leveraging the margin-based loss. Thus, the combined model's superior performance validates our design choices. That is, the combined scoring function (cosine similarities from CL and the prediction probabilities from BC) is thoughtfully designed to leverage the advantage of each module for high detection accuracy.

% How do unsupervised model and weakly-supervised models generalize in four OOD scenarios?

% \begin{table}[!t]
% \centering
% \scalebox{0.68}{
%     \begin{tabular}{ll cccc}
%       \toprule
%       & \multicolumn{4}{c}{\textbf{Intellipro Dataset}}\\
%       & \multicolumn{2}{c}{Rank Resume} & \multicolumn{2}{c}{Rank Job} \\
%       \cmidrule(lr){2-3} \cmidrule(lr){4-5} 
%       \textbf{Method}
%       &  Recall@100 & nDCG@100 & Recall@10 & nDCG@10 \\
%       \midrule
%       \confitold{}
%       & 71.28 &34.79 &76.50 &52.57 
%       \\
%       \cmidrule{2-5}
%       \confitsimple{}
%     & 82.53 &48.17
%        & 85.58 &64.91
     
%        \\
%        +\RunnerUpMiningShort{}
%     &85.43 &50.99 &91.38 &71.34 
%       \\
%       +\HyReShort
%         &- & -
%        &-&-\\
       
%       \bottomrule

%     \end{tabular}
%   }
% \caption{Ablation studies using Jina-v2-base as the encoder. ``\confitsimple{}'' refers using a simplified encoder architecture. \framework{} trains \confitsimple{} with \RunnerUpMiningShort{} and \HyReShort{}.}
% \label{tbl:ablation}
% \end{table}
\begin{table*}[!t]
\centering
\scalebox{0.75}{
    \begin{tabular}{l cccc cccc}
      \toprule
      & \multicolumn{4}{c}{\textbf{Recruiting Dataset}}
      & \multicolumn{4}{c}{\textbf{AliYun Dataset}}\\
      & \multicolumn{2}{c}{Rank Resume} & \multicolumn{2}{c}{Rank Job} 
      & \multicolumn{2}{c}{Rank Resume} & \multicolumn{2}{c}{Rank Job}\\
      \cmidrule(lr){2-3} \cmidrule(lr){4-5} 
      \cmidrule(lr){6-7} \cmidrule(lr){8-9} 
      \textbf{Method}
      & Recall@100 & nDCG@100 & Recall@10 & nDCG@10
      & Recall@100 & nDCG@100 & Recall@10 & nDCG@10\\
      \midrule
      \confitold{}
      & 71.28 & 34.79 & 76.50 & 52.57 
      & 87.81 & 65.06 & 72.39 & 56.12
      \\
      \cmidrule{2-9}
      \confitsimple{}
      & 82.53 & 48.17 & 85.58 & 64.91
      & 94.90&78.40 & 78.70& 65.45
       \\
      +\HyReShort{}
       &85.28 & 49.50
       &90.25 & 70.22
       & 96.62&81.99 & \textbf{81.16}& 67.63
       \\
      +\RunnerUpMiningShort{}
       % & 85.14& 49.82
       % &90.75&72.51
       & \textbf{86.13}&\textbf{51.90} & \textbf{94.25}&\textbf{73.32}
       & \textbf{97.07}&\textbf{83.11} & 80.49& \textbf{68.02}
       \\
   %     +\RunnerUpMiningShort{}
   %    & 85.43 & 50.99 & 91.38 & 71.34 
   %    & 96.24 & 82.95 & 80.12 & 66.96
   %    \\
   %    +\HyReShort{} old
   %     &85.28 & 49.50
   %     &90.25 & 70.22
   %     & 96.62&81.99 & 81.16& 67.63
   %     \\
   % +\HyReShort{} 
   %     % & 85.14& 49.82
   %     % &90.75&72.51
   %     & 86.83&51.77 &92.00 &72.04
   %     & 97.07&83.11 & 80.49& 68.02
   %     \\
      \bottomrule

    \end{tabular}
  }
\caption{\framework{} ablation studies. ``\confitsimple{}'' refers using a simplified encoder architecture. \framework{} trains \confitsimple{} with \RunnerUpMiningShort{} and \HyReShort{}. We use Jina-v2-base as the encoder due to its better performance.
}
\label{tbl:ablation}
\end{table*}
Moreover, we compare the detection performance of our COOD+ with various underlying NL-PL pre-trained encoder. Specifically, we compare our choice of GraphCodeBERT~\cite{guo2020graphcodebert} against other NL-PL encoders from the literature including its predecessor, CodeBERT~\cite{feng2020codebert}, and more recent ones such as UniXcoder~\cite{guo2022unixcoder} and ContraBERT~\cite{liu2023contrabert}. As shown in Table~\ref{tab:ablation}, all encoders perform within a 1-2\% difference, indicating that our COOD+ framework is robust across different encoders. This demonstrates our framework's flexibility and effectiveness in detecting OODs when deploying various NL-PL encoders for code-related tasks. Furthermore, we investigate key hyperparameters in COOD+, such as $m$ for margin-based contrastive loss and  $\lambda$ in the overall loss function. The detailed results are available in our online appendix~\cite{cood-tool}.


%%%---margin-based loss----
% \noindent
% \textbf{Effect of Margin-based Loss and its Hyperparameter.} We explore the impact of the margin $m$ in Eq.~\ref{eq:contrast_id} and Eq.~\ref{eq:contrast_ood} on the detection performance. Table ~\ref{tab:ablation} shows the detection performance under different margin parameters. When $m=0$, the proposed model does not use the margin, meaning that it arbitrarily encourages the OOD cosine similarities to be lower than that of ID samples. We can observe from Table~\ref{tab:ablation} that our method has the best performance when $m=0.4$. In addition, our detection method without the margin-based loss (when $m=0$) indeed performs worse, although the decrease in performance is offset by the rejection classifier. Therefore, we can conclude that the proposed margin-based loss plays an important role in multi-modal OOD detection since it enables OOD-aware finetuning for pretrained NL-PL transformer-based models.\\

% \noindent
% \textbf{Effect of Weight in the Objective.} We also evaluate the performance of the proposed model for different $\lambda$ values in the overall objective function (\ref{eq:overall_obj}) \shao{typos here} as it increases from $0.2$, to $0.5$, and to $1$. As observed in Table~\ref{tab:ablation}, our model performs best when $\lambda=0.2$ on two datasets. One possible reason is that the NL-PL encoders are pretrained on multiple PLs, which requires more weight in the contrastive learning module for language-specific representation learning during optimization. In addition, as representations learned by transformer-based encoders transfer well to fully-connected classification components~\cite{feng2020codebert, guo2020graphcodebert}, the binary OOD rejection classifier needs smaller weight updates during back-propagation to perform effectively.\\

% \noindent \textbf{Effect of Different NL-PL Encoders.}


% \noindent 


\subsection{RQ4: Main Task Performance}
We present the code search performance under the impact of OOD instances by using GraphCodeBERT (GCB), COOD/COOD+, and the closest competitor EOE in Table~\ref{tab:maintask}. As described in Sec. V-F, we use the official metric mRR  and follow the same testing scheme as the original GraphCodeBERT code search model for evaluation. From Table~\ref{tab:maintask}, we first observe that our COOD/COOD+ achieves performance comparable to GraphCodeBERT, while the EOE suffers from a significant reduction in performance, as it reformulates code search as binary classification to gain OOD detection ability. This reveals a critical trade-off between OOD detection and downstream task performance. To further validate the importance of OOD detection for code search, we construct outliers based on the CSN-Java and -Python testing dataset, respectively. Given that code search aims to retrieve the most aligned code from a code corpus given an NL query, the outliers are only sampled from three OOD scenarios: \textit{out-domain}, \textit{shuffled-comment} and \textit{buggy-code}, each replacing 5\% ID data of the original testing set. We then show the results when the dataset contains 15\% OOD samples (\ie 15\% outliers), discard OOD samples by filtering the testing set by ground-truth labels (\ie Filtered-GT) or using various OOD detection models (\ie Filtered-OOD-model). Note that the Filtered-GT dataset is the original CSN's subset with 15\% of ID samples removed.

\begin{table}[!tb]
\centering
\caption{Code search performance under the impact of OOD detection. Higher numbers represent better performance}
\label{tab:maintask}
% \setlength{\tabcolsep}{3pt}
\begin{adjustbox}{width=0.48\textwidth}\small
\begin{tabular}{c|c|c|c|c|c}
\toprule
\textbf{Dataset}            & \textbf{Testing Subset} & \textbf{GCB} & \textbf{EOE} & \textbf{COOD} & \textbf{COOD+} \\ \hline
\multirow{4}{*}{CSN-Python} & Origin                  & 69.20        & 50.11        & 68.47         & 69.69          \\
                            & 15\% outliers            & 65.85        & 43.68        & 64.67         & 65.67          \\
                            & Filtered-GT  & 70.24        & 44.85        & 68.95         & 70.24          \\
                            & Filtered-OOD-model     & --           & 46.82        & 70.30         & \textbf{73.10} \\ \hline
\multirow{4}{*}{CSN-Java}   & Origin                  & 69.10        & 46.29        & 68.85         & 69.46          \\
                            & 15\% outliers            & 64.99        & 37.77        & 64.86         & 64.54          \\
                            & Filtered-GT  & 69.12        & 38.94        & 69.36         & 69.93          \\
                            & Filtered-OOD-model     & --           & 39.33        & 71.02         & \textbf{73.18} \\ \bottomrule
\end{tabular}
\end{adjustbox}
\end{table}

According to Table~\ref{tab:maintask}, the performance of the original GraphCodeBERT code search model drops by 4.84\% and 5.95\% ((69.10-64.99)/69.10) mRR when outliers are present in CSN-Python and -Java, respectively. As a solution to this issue, our COOD/COOD+ detector recover the performance losses by identifying and filtering out the OOD samples without negatively impacting the model's code understanding ability in code search. Specifically, the code search performance of COOD/COOD+ on the Filtered-COOD/COOD+ dataset (70.30\%/73.10\% and 71.02\%/73.18\% on CSN-Python and -Java, respectively) is comparable to or even better than GraphCodeBERT on the Filtered-GT dataset (70.24\% and 69.12\% on CSN-Python and -Java, respectively). This slight improvement is probably because our detectors filter out additional lower-quality testing samples that resemble outliers. Thus, our COOD/COOD+ enhance the trustworthiness and robustness of the GraphCodeBERT, since the model's predictions become more reliable when encountering OOD data. % Also, although both COOD and COOD+ can recover the GraphCodeBERT's performance losses, COOD+ still perform better than COOD on each possible testing subset, especially the Filtered-OOD-model dataset. %Compared to only using 15\% outliers in this experiment, when more OOD instances from diverse OOD scenarios are encountered (\eg 75\% outliers in the OOD detection experiments), we expect the performance difference between COOD+ and COOD to be more significant. 
Note that the original GraphCodeBERT is not equipped with the OOD detection ability, so its corresponding cells for the Filtered-OOD-model in Table~\ref{tab:maintask} are left blank.

% It is not clear how RQ4 (Main Task Performance) is answered using Table V since we do not know what those numbers mean for CSN-Java and CSN-Python. It would have been better if the rationale of RQ4 is explained clearly along with the metrics that have been used to answer this RQ.  it is not clear why RQ4 is addressed and why only COOD+ is considered rather than both COOD and COOD+

% \subsection{RQ5: Overconfident Case}
% \viet{Yanfu: Please move these into where you want} Given an OOD testing sample, DNN models (pre)trained on ID data are prone to predict a higher MSP confidence score than the threshold and wrongly identify it as an
% ID sample~\cite{lee2017training, liang2017enhancing}. This overconfidence issue limits the effectiveness of OOD detection. For NL data, this is caused by the the spurious correlation between OOD and ID features such as entities and syntactic structures~\cite{zheng2020out, wu2022revisit}. Such correlation also occurs in PL data. For example, an OOD PL input with the syntactic structure ``def ... if ... return ... else ... return ..." may receive an ID score if this pattern is commonly used in other ID inputs. To overcome overconfident predictions, previous works explored techniques such as temperature scaling~\cite{liang2017enhancing}, confidence calibration using adversarial samples~\cite{lee2017training, bitterwolf2020certifiably}, or adaptive class-dependent threshold~\cite{wu2022revisit}. In contrast, our proposed COOD+ utilizes a weakly-supervised contrastive learning objective to take advantage of a small number of OOD samples during training and prevent the alignment between OOD pairs. Moreover, we adopt the binary OOD rejection module to discriminate the fused OOD and ID representations. We further verify whether COOD+ overcome the overconfidence issue through the lens of Conformal Prediction~\cite{angelopoulos2023conformal}.

% Conformal Prediction (CP) involves post-processing uncertainty quantification techniques that are model-agnostic, and provide statistical guarantees on the predictions of a trained model~\cite{angelopoulos2023conformal}. The commonly used split CP technique first computes the nonconformity scores, which are OOD scores in our case, on a calibration set independent of the training data. Then, it builds a prediction set for each testing sample $\mathcal{C}_\alpha(t^{test}_l, c^{test}_l)$ satisfying the condition $P(y_l^{test}\in \mathcal{C}_\alpha(t^{test}_l, c^{test}_l))\geq 1-\alpha$, where $\alpha$ is a small error rate (e.g., 0.05) that the user is willing to tolerate. Here, this condition guarantees that the true outcome is covered by the prediction set with probability $1-\alpha$, which is also known as the CP coverage. When CP is applied to the OOD detection scores, all scores have the same statistical guarantee, but better OOD scores will give tighter prediction sets. Conversely, worse scores will give large and uninformative prediction sets, which corresponds to ineffective OOD detection caused by overconfident predictions.

% In our experiment, we apply split CP by reserving 20\% of the testing samples from each testing dataset (CSN-Java and CSN-Python) for CP calibration, and construct the prediction sets with tolerable error rate $\alpha=0.05$ on the remaining testing samples. To assess how effectively COOD+ addresses the overconfidence issue, we compare its average prediction set (P-Set) size (between 1 and 2 for binary predictions) with that of selected baselines including MCL+MSP, the best performing approach using MSP OOD scores, and our proposed COOD. As observed in Table~\ref{tab:conformal}, the proposed COOD+ achieves the smallest prediction sets on both datasets. Specifically, the vast majority of prediction sets obtained by COOD+ contain one value that is 95\% statistically guaranteed to be the true OOD label according to the CP condition, indicating the minimal overconfidence of OOD scores. In contrast, the MCL+MSP method is prone to overconfidence, because it produces large prediction sets (i.e., size 2) including both IDs and overconfident OODs. Additionally, without utilizing OOD samples during training, COOD cannot effectively prevent overconfident predictions. Note that in the CP context, although higher coverage is desired, the main goal is to build the smallest prediction sets given the user-specified error rate of 0.05. Therefore, COOD+ is the most effective at overcoming the overconfidence barrier despite its slightly lower coverage than that of MCL+MSP and COOD.

% \begin{table}[!tb]
% \centering
% \caption{Effectiveness of COOD+ compared to selected methods for overcoming overconfident OOD predictions.}
% \small
% \label{tab:conformal}
% \begin{adjustbox}{width=\linewidth,center}
% \begin{tabular}{c|cc|cc}
% \toprule
% \multirow{2}{*}{\textbf{Methods}} & \multicolumn{2}{c|}{\textbf{CSN-Java}} & \multicolumn{2}{c}{\textbf{CSN-Python}} \\ \cline{2-5} 
% & \textbf{Coverage↑}    & \textbf{ P-Set Size↓}   & \textbf{Coverage↑}    & \textbf{P-Set Size↓}    \\ \hline
% MCL+MSP                      &  \textbf{97.03}    &   1.891   &   95.00   &      1.834          \\
% COOD                           &   95.66            &      1.593         &   \textbf{95.81}   &  1.576     \\
% COOD+                          & 95.31              & \textbf{1.077}              &       95.35        &      \textbf{1.010}          \\
% \bottomrule
% \end{tabular}
% \end{adjustbox}
% \end{table}