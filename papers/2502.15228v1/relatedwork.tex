\section{RELATED WORK}
Existing research on gesture and motion data analysis spans task-specific models~\cite{liang2024mask}, AutoMR frameworks~\cite{damdoo2020adaptive}, and hyperparameter tuning tools~\cite{nayak2021hyper}. While each of these areas has made significant contributions, they also present certain limitations when applied to multimodal datasets. 

\subsection{Task-Specific Models}
% Task-specific models have been widely adopted for gesture and motion data recognition tasks. For instance, DeepConvLSTM integrates convolutional layers with LSTMs to capture spatiotemporal patterns, achieving strong results on datasets such as UCI-HAR~\cite{khatun2022deep}. 
Task-specific models are widely used in gesture and motion recognition. For instance, DeepConvLSTM combines convolutional layers and LSTMs to capture spatiotemporal patterns, showing strong performance on UCI-HAR~\cite{khatun2022deep}.
Similarly, Temporal Convolutional Networks (TCNs) use dilated convolutions to model long-range dependencies in time-series data~\cite{airlangga2024performance}, demonstrating success on OPPORTUNITY~\cite{al2024tcn} and other multimodal datasets. However, these methods are typically tailored to specific modalities (e.g., IMU or skeletal motion) and often lack scalability to new data types such as sEMG signals or unsegmented motion capture data.
% Similarly, Temporal Convolutional Networks (TCN) utilize dilated convolutions to model long-range dependencies in time-series data~\cite{airlangga2024performance} and have shown success in OPPORTUNITY~\cite{al2024tcn} and other multimodal datasets. However, these methods are typically tailored to specific datasets or modalities, such as inertial measurement unit (IMU) data or skeletal motion data, and often lack scalability to new datasets or modalities like sEMG signals or unsegmented motion capture data.

\subsection{Auto Training Frameworks for Time-Series Data}
Auto training frameworks, such as Auto-sklearn and H2O AutoGR~\cite{9534091}, offer automation in model selection and hyperparameter tuning. These tools reduce the expertise required to build machine learning pipelines and have demonstrated potential in time-series classification.
However, these frameworks are not tailored for multimodal datasets and often require extensive customization for preprocessing and handling heterogeneous sensor data. For example, TPOT’s limited modality-specific pipelines restrict its applicability to datasets like LMDHG, where skeletal motion data demands complex preprocessing and feature extraction~\cite{gijsbers2018layered}.
% However, most frameworks are not tailored for multimodal datasets, requiring substantial customization for preprocessing and handling heterogeneous sensor data. 
% For example, while TPOT has been used for time-series applications, its limited support for modality-specific pipelines restricts its applicability to datasets like LMDHG, where skeletal motion data requires complex preprocessing and feature extraction~\cite{gijsbers2018layered}.

\subsection{Hyperparameter Tuning Tools}
% Frameworks such as Optuna and Hyperopt~\cite{shekhar2021comparative} leverage Bayesian optimization to efficiently explore hyperparameter spaces and have been applied on datasets like MHEALTH~\cite{talaat2022effective} and OPPORTUNITY~\cite{ozcan2020human} for tuning parameters such as learning rates and network depths. 

Frameworks such as Optuna and Hyperopt are commonly used for optimizing hyperparameters~\cite{shekhar2021comparative}. These tools use algorithms like Bayesian optimization to efficiently explore the search space, and they have been applied to datasets like MHEALTH~\cite{talaat2022effective} and OPPORTUNITY~\cite{ozcan2020human} for tuning hyperparameters such as learning rates and network depths. 
However, they are not integrated into end-to-end workflows and often require significant manual setup for data preprocessing, model training, and evaluation.
% However, these tools lack integration into end-to-end workflows, often requiring significant manual intervention to set up pipelines for data preprocessing, model training, and evaluation.

% Our work addresses these limitations by introducing AutoGR, an end-to-end AutoMR framework designed specifically for gesture and motion data recognition. AutoGR integrates preprocessing, model selection, and hyperparameter optimization, supporting diverse modalities such as sEMG, skeletal motion, and IMU data. Unlike task-specific models, AutoGR is adaptable across datasets without requiring significant manual adjustments. It also incorporates automated hyperparameter tuning, fully integrated into the pipeline, to simplify deployment and reduce the technical barriers for users.

\addtolength{\textheight}{-3cm}   % This command serves to balance the column lengths
                                  % on the last page of the document manually. It shortens
                                  % the textheight of the last page by a suitable amount.
                                  % This command does not take effect until the next page
                                  % so it should come on the page before the last. Make
                                  % sure that you do not shorten the textheight too much.

%%%%%%%%%%%%%%%%%%%%%%%%%%%%%%%%%%%%%%%%%%%%%%%%%%%%%%%%%%%%%%%%%%%%%%%%%%%%%%%%