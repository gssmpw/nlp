\begin{abstract}
The content moderation systems used by social media sites are a topic of widespread interest and research, but less is known about the use of similar systems by web search engines. For example, Google Search attempts to help its users navigate three distinct types of data voids---when the available search results are deemed low-quality, low-relevance, or rapidly-changing---by placing one of three corresponding warning banners at the top of those search results. Here we collected 1.4M unique search queries shared on social media to surface Google's warning banners, examine when and why those banners were applied, and train deep learning models to identify data voids beyond Google's classifications. Across three data collection waves (Oct 2023, Mar 2024, Sept 2024), we found that Google returned a warning banner for about 1\% of our search queries, with substantial churn in the set of queries that received a banner across waves. The low-quality banners, which warn users that their results ``may not have reliable information on this topic,'' were especially rare, and their presence was associated with low-quality domains in the search results and conspiracy-related keywords in the search query. Low-quality banner presence was also inconsistent over short time spans, even when returning highly similar search results. In August 2024, low-quality banners stopped appearing on the SERPs we collected, but average search result quality remained largely unchanged, suggesting they may have been discontinued by Google. %
Using our deep learning models to analyze both queries and search results in context, we identify 29 to 58 times more low-quality data voids than there were low-quality banners, and find a similar number after the banners had disappeared. Our findings point to the need for greater transparency on search engines' content moderation practices, especially around important events like elections.
\end{abstract}

 

