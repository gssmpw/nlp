\section{Discussion}
\label{sec:discussion}

In this study we developed and deployed methods for surfacing, evaluating, and predicting warning banners and data voids on web search engines.
Using data collected over several waves of varying size and inter-collection latency, we found that Google's low-quality banners were rare, governed by inconsistent and evadable rules, 
and appear to have been discontinued around August 2024.
We do not find that the disappearance of these banners can be explained by improvements in average domain quality, or by replacement with a new banner type, suggesting that the system for placing those banners has either become exceedingly conservative or turned off entirely.
These findings raise questions about the reliability and transparency
of Google's content moderation systems, their capacity and willingness warn users about unreliable information in their search results, and the timing of the changes, which occurred in the months preceding the 2024 US Presidential election.

When Google was using its low-quality warning banners, our results show that it rarely displayed them for the search queries in our sample (0.02\%), and the rules governing their presence on a page of search results were inconsistent and unstable over short time spans.
Our findings also show that these rules can be easily evaded through the use of advanced operators in the search query---as Google never displayed a low-quality banner for such queries---and highlight examples of search queries shared on social media that abuse this loophole to guide people into data voids (e.g., the query ``covid vaccine detox site:naturalnews.com'').
The absence of low-quality banners in such scenarios could be framed as respect for user choice: Google has stated that ``users may decide to seek and select content that our signals determine to be of low-quality ... we believe it is of fundamental importance to respect their choices''~\citep{google2024information}.
While this decision may be reasonable for users formulating their own queries, it ignores the social sharing of search queries with advanced operators, where not all users will understand their impact on the search results.
Alternatively, the non-committal language consistently used to describe the conditions under which low-quality banners are shown may also help explain this loophole.
This includes announcement of the low-quality banner (``This doesn't mean that no helpful information is available, or that a particular result is low-quality''~\cite{nayak2022new}), the text on the banner itself (``some of [the results] may not have reliable information on this topic,'' Figure~\ref{fig:banner_ex}), and the lack of a ``there aren't \textit{any} great results'' banner variant as we found in the low-relevance banners (``there aren't \textit{many/any} great matches,'' see Methods~\ref{sec:methods-queries-operators}).
If a low-quality banner were returned for a search query including a single ``site:'' operator, limiting all results to a single domain, the ambiguity of which website triggered the warning would be removed.

One potential explanation for the absence of low-quality banners since August 2024 is that Google's recent ``core updates''~\citep{aug2024update}, which went into effect around the same time, have improved the quality of its search results such that the low-quality banners are no longer needed.
However, comparing across data collection waves, we find little evidence of substantive increases in the domain quality of the SERPs returned for queries that had previously received a low-quality banner.
Instead, many of those queries continued to return low-quality domains, but no longer receive any type of warning banner.
More importantly, when considering the rate at which Google applied its low-quality banners to different definitions of data voids, including a simple average domain quality threshold and a heterogeneous GNN, we identified up to 50x more SERPs as data voids than Google applied its low-quality banners to. 
If we conservatively extrapolate from Google's self-reported 3.3 billion searches per day in 2012~\citep{google2013google} to 5 billion per day in 2024, and apply the data void rate estimated by our model (0.77\%, averaged across crawls)---which is not representative of the queries real users search but does provide a broad and diverse sample---then the estimated number of data voids encountered per day on Google would be 38.5M. 
Using the percentage of these data voids that received one of Google'’'s low-quality banners in our study (0.3\%, averaged across crawls), we can further estimate that only 115.5K of those 38.5M would have received a low-quality banner.

Our use of search directives to identify a set of search queries is both an advantage and limitation of our study. 
Compared to prior algorithm audits of web search, where researchers often select queries on their own or solicit them from survey participants (\cite{ballatore2015google}; \cite{norocel2023google}; \cite{lurie2021searching}; \cite{vanhoof2022searching}), our use of search directives allowed us to develop one of the largest query sets ever used in an algorithm audit, which was crucial to our goal of surfacing a sufficient sample of data voids and Google's warning banners. 
However, the search directives we used for this purpose come from a long time window (2006 to 2023), so the search results we retrieved for each query do not reflect the results one might have seen had they conducted that search at the time the search directive was posted. 
Similarly, we used queries from search directives that were intended to be used on search engines other than Google, but Google is the largest search engine in the world, and developing tools for collecting and parsing search results from each search engine would have taken a massive infrastructure investment for what is often a moving target.

In addition, we also only examine search directives that were shared as links---which provide a shortcut to search results with pre-specified query---and did not include queries shared in text or images~\citep{robertson2023identifying}. 
Beyond explicit search directives that provide a specific search query to use, users may also be compelled to conduct searches after being exposed to more implicit forms, such as keywords or phrases that are repeatedly promoted by influencers or political elites~\citep{tripodi2019devin}, 
and the recently proposed concept of ``dredge words''~\citep{williams2024dredge}---queries for which unreliable websites rank highly---may offer a promising and scalable avenue for identifying queries likely to produce data voids.
Future work that develops methods for surfacing explicit and implicit search directive queries in real time may help researchers examine the rapidly-changing warning banners that were vanishingly rare in our dataset.

Given Google's public efforts to understand and improve the reliability of their rankings~\citep{google2023search}, our results may be viewed as a lower bound for the prevalence of data voids in web search engines more broadly. 
Other search engines, especially those that are reportedly favored by conspiracy theorists~\citep{thompson2022fed}, may produce a higher rate of data voids,
and should be investigated in future work. 
Similarly, while the queries we searched do not provide a representative sample of real users' queries, only 0.1\% contained a conspiracy-related keyword from a relatively conservative and non-exhaustive list. 
While it is unclear what that proportion would be for real users' queries, their relative scarcity in our dataset also suggests that our results may provide a lower bound estimate for the prevalence of data voids in web search.

The harmful nature of data voids, combined with the potential of search directives to guide people into them, both suggest the prevalence rates we found should provide cause for concern.
Our study sheds light on both the use of warning banners and the prevalence of data voids in web search, but without any official reports on the use or effectiveness of warning banners in search, it remains unclear how harmful the unlabeled data voids that we discovered may be to real users.
Future work should explore independent approaches to collecting exposure and engagement with warning banners among real search engine users~\citep{feal2024introduction}, collaborations with industry that could facilitate that research, and examine a broader set of search engines, languages, and locations~\citep{borge2021how}.

As search evolves and continues to incorporate Large Language Models (LLMs) and agent-based approaches to task completion~\citep{white2024advancing}, studies like ours may become both increasingly important and difficult to conduct. 
The methods we developed here are platform agnostic, but LLMs add greater stochasticity to the outputs of search engines, may change how people write their search queries, and introduce a conversational interface that could change fundamental aspects of how search engines were previously used. 
While recent research suggests that LLMs may be able to reduce conspiracy beliefs when specifically prompted to do so~\citep{costello2024durably}, it's unclear whether popular LLMs will incorporate the prompts needed to elicit that behavior, especially in the context of emerging conspiracies and the incorporation of LLMs into search engines. 
Indeed, with respect to data voids, LLMs may make them easier to create, harder to identify, and more likely to be presented as a valid response to a user's query. 
As such, the intersection of search directives, data voids, and the content moderation practices of both search engines and LLM providers, especially as two become more intertwined, presents an important and pressing area for future research.

\clearpage
\newpage
