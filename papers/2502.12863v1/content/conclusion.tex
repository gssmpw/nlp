\section{Conclusion}
\label{sec:conclusion}
\iffalse
We created a large corpora of API calls from very recent malware and benign software based on the ntdll Windows library.
Our experiments show that having simply the function call count itself can help us differentiate between malware and benign software to a great degree.
The tests we ran could detect malware with a high degree of certainty.
We also show that for a malware detection system to be effective to a certain degree, we need at least one hundred API calls to the ntdll.dll library.
\fi
We created a substantial corpus of API calls from recent malware and benign software, with a specific focus on the ntdll Windows library, underscoring its significance in our research. 
Our dataset, the largest of its kind available publicly, is a unique and invaluable resource for the research community in malware detection. 
Our experiments demonstrate that even the basic metric of function call counts can significantly distinguish between malicious and benign software. 
The created models show that by using good feature engineering techniques, we can detect malware precisely with negligible performance overhead.
Our method has been proven to detect malware with a small number of API calls, demonstrating its efficiency and practicality. 
Furthermore, we found that for a malware detection system to be effective, it must analyze at least two hundred and fifty API calls to the ntdll.dll library. 
This threshold ensures a reliable level of detection accuracy, reinforcing the importance of comprehensive data collection in developing robust malware detection systems. 
Our findings also highlight the potential of simple yet effective features in developing efficient and scalable malware detection solutions, paving the way for future research and advancements in cybersecurity.