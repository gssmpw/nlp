\section{Introduction}
\label{sec:introduction}
Malware attacks are increasing all over the world year over year\cite{av_test_statistics}. 
Zero-day exploits, a term used to describe vulnerabilities in software that are unknown to the software vendor and therefore unpatched, also help malicious attackers hide their presence on hijacked machines. 
These exploits pose a significant challenge for malware detection as they can be used to launch attacks that are difficult to detect and defend against. 
%With polymorphic malware, which changes its code multiple times to evade detection, the task of detecting malware is becoming more complex. 
Polymorphic malware presents a significant challenge for traditional code-based checks, as the code itself gets encrypted, making it difficult to identify and analyze.
With the introduction of polymorphic malware into the wild, regular code-based checks do not work, as the code itself gets encrypted, along with garbage values inserted into the malware payload files.

Machine learning shows promise in identifying malware but requires large amounts of data to generalize well. 
Although many cybersecurity companies use machine-learning-based methods, most datasets are proprietary. 
Existing datasets are often proprietary, narrow in scope, or insufficiently comprehensive, posing challenges for researchers and developers aiming to create and refine machine learning models for malware detection. 
While some publicly available datasets exist, they tend to be outdated, small in scale, or lacking in diversity, limiting their utility in addressing modern malware's sophisticated and rapidly evolving nature. 
This highlights a critical need for the development and dissemination of large-scale, diverse datasets that encompass a wide range of benign and malicious software behaviors. 
We address this gap by creating a dataset from data collected with the help of~\gdata~and making it publicly available.

Another aspect is the practical side of malware detection. 
System software running on machines should have a manageable overhead and not impede the performance of essential business software.
Therefore, having very complex models for real-time analysis that take up valuable resources is not feasible, as the memory and performance requirements will hinder other software.
We could have more complex offline models for forensic analysis. However, the best option on a running machine would be a lightweight model that uses simple feature engineering but offers a very high degree of accuracy.

Our primary contribution is the creation of the largest publicly available dataset of API calls, with over 300,000 malware samples and 10,000 benign samples of API call instances sourced from recent malware and benign software samples. 
The uncompressed size of the dataset exceeds 550GB and is available on Zenodo.
The second contribution is lightweight models that could be used in real-time malware detection based on their API calls.
We empirically show that we could detect malware reliably with as few as 250 API calls.
Our findings show that the mere frequency of API calls is a powerful indicator of malicious intent, enabling the detection of malware with high certainty.
Moreover, our research underscores the importance of comprehensive data collection in developing robust malware detection systems, providing a valuable resource for the cybersecurity community.
For our third contribution, we conduct a comprehensive study to determine the number of API calls required for the best detection performance at different maximum API call counts.

We introduce the premise and research gap in section~\autoref{sec:introduction}.
We discuss the current landscape of malware detection methods in~\autoref{sec:related_work}.
Our dataset creation and malware identification method is present in~\autoref{sec:method}. We present the results of our experiments in~\autoref{sec:results}~and our conclusions in~\autoref{sec:conclusion}.