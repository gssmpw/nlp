\section{Method}
\label{sec:method}
We structure this section into two parts. The first describes the dataset, its creation, and corresponding statistics. 
The second explains the order-invariant method we use to generate the results.
\section{Dataset}
\label{sec:dataset}

\subsection{Data Collection}

To analyze political discussions on Discord, we followed the methodology in \cite{singh2024Cross-Platform}, collecting messages from politically-oriented public servers in compliance with Discord's platform policies.

Using Discord's Discovery feature, we employed a web scraper to extract server invitation links, names, and descriptions, focusing on public servers accessible without participation. Invitation links were used to access data via the Discord API. To ensure relevance, we filtered servers using keywords related to the 2024 U.S. elections (e.g., Trump, Kamala, MAGA), as outlined in \cite{balasubramanian2024publicdatasettrackingsocial}. This resulted in 302 server links, further narrowed to 81 English-speaking, politics-focused servers based on their names and descriptions.

Public messages were retrieved from these servers using the Discord API, collecting metadata such as \textit{content}, \textit{user ID}, \textit{username}, \textit{timestamp}, \textit{bot flag}, \textit{mentions}, and \textit{interactions}. Through this process, we gathered \textbf{33,373,229 messages} from \textbf{82,109 users} across \textbf{81 servers}, including \textbf{1,912,750 messages} from \textbf{633 bots}. Data collection occurred between November 13th and 15th, covering messages sent from January 1st to November 12th, just after the 2024 U.S. election.

\subsection{Characterizing the Political Spectrum}
\label{sec:timeline}

A key aspect of our research is distinguishing between Republican- and Democratic-aligned Discord servers. To categorize their political alignment, we relied on server names and self-descriptions, which often include rules, community guidelines, and references to key ideologies or figures. Each server's name and description were manually reviewed based on predefined, objective criteria, focusing on explicit political themes or mentions of prominent figures. This process allowed us to classify servers into three categories, ensuring a systematic and unbiased alignment determination.

\begin{itemize}
    \item \textbf{Republican-aligned}: Servers referencing Republican and right-wing and ideologies, movements, or figures (e.g., MAGA, Conservative, Traditional, Trump).  
    \item \textbf{Democratic-aligned}: Servers mentioning Democratic and left-wing ideologies, movements, or figures (e.g., Progressive, Liberal, Socialist, Biden, Kamala).  
    \item \textbf{Unaligned}: Servers with no defined spectrum and ideologies or opened to general political debate from all orientations.
\end{itemize}

To ensure the reliability and consistency of our classification, three independent reviewers assessed the classification following the specified set of criteria. The inter-rater agreement of their classifications was evaluated using Fleiss' Kappa \cite{fleiss1971measuring}, with a resulting Kappa value of \( 0.8191 \), indicating an almost perfect agreement among the reviewers. Disagreements were resolved by adopting the majority classification, as there were no instances where a server received different classifications from all three reviewers. This process guaranteed the consistency and accuracy of the final categorization.

Through this process, we identified \textbf{7 Republican-aligned servers}, \textbf{9 Democratic-aligned servers}, and \textbf{65 unaligned servers}.

Table \ref{tab:statistics} shows the statistics of the collected data. Notably, while Democratic- and Republican-aligned servers had a comparable number of user messages, users in the latter servers were significantly more active, posting more than double the number of messages per user compared to their Democratic counterparts. 
This suggests that, in our sample, Democratic-aligned servers attract more users, but these users were less engaged in text-based discussions. Additionally, around 10\% of the messages across all server categories were posted by bots. 

\subsection{Temporal Data} 

Throughout this paper, we refer to the election candidates using the names adopted by their respective campaigns: \textit{Kamala}, \textit{Biden}, and \textit{Trump}. To examine how the content of text messages evolves based on the political alignment of servers, we divided the 2024 election year into three periods: \textbf{Biden vs Trump} (January 1 to July 21), \textbf{Kamala vs Trump} (July 21 to September 20), and the \textbf{Voting Period} (after September 20). These periods reflect key phases of the election: the early campaign dominated by Biden and Trump, the shift in dynamics with Kamala Harris replacing Joe Biden as the Democratic candidate, and the final voting stage focused on electoral outcomes and their implications. This segmentation enables an analysis of how discourse responds to pivotal electoral moments.

Figure \ref{fig:line-plot} illustrates the distribution of messages over time, highlighting trends in total messages volume and mentions of each candidate. Prior to Biden's withdrawal on July 21, mentions of Biden and Trump were relatively balanced. However, following Kamala's entry into the race, mentions of Trump surged significantly, a trend further amplified by an assassination attempt on him, solidifying his dominance in the discourse. The only instance where Trump’s mentions were exceeded occurred during the first debate, as concerns about Biden’s age and cognitive abilities temporarily shifted the focus. In the final stages of the election, mentions of all three candidates rose, with Trump’s mentions peaking as he emerged as the victor.
\subsection{Order invariant Method}
We aim to develop a malware detection method independent of the temporal constraints and ordering of API calls. This approach ensures that the detection system remains effective even when the sequence of API calls is altered, which is a common evasion technique used by malware. To achieve this, it is crucial to thoroughly investigate the impact on performance as we progressively increase the sequence length of successive API calls under consideration.

Our proposed solution involves mapping each function call directly to a feature in the feature vector, with the value in each position representing the number of times the sample invoked that particular function. This method allows us to create a robust feature representation that is not influenced by the order of API calls, focusing instead on the frequency of each function's invocation.

We structure our experiments into four distinct parts to evaluate this approach comprehensively and ensure the validity of our findings. In the first part, we examine each API call individually, disregarding the context provided by previous and subsequent API calls. We refer to this as the Unigram model, a concept borrowed from natural language processing (NLP). In NLP, a Unigram model analyzes text by considering each word independently, without accounting for the sequence in which words appear. Similarly, in our Unigram model for API calls, we treat each function call as an independent event, counting its occurrences without considering its position in the sequence.

This initial experiment establishes a baseline understanding of how well individual API call frequencies can distinguish between benign and malicious software. By focusing solely on the count of each function, we can determine the effectiveness of this simple yet powerful feature representation in detecting malware. Subsequent parts of our experiments will build upon this foundation, progressively incorporating more contextual information to explore how the performance varies following different lengths of sequences.

By employing this methodical approach, we ensure a comprehensive analysis of the relationship between API call sequences and malware detection accuracy. Our ultimate goal is to identify the optimal balance between feature complexity and detection performance, ultimately developing a robust and efficient malware detection system that is resilient to common evasion tactics.

For the Unigram approach, we simply map each function call to an array index in the feature vector.
We do this by creating a vector, $V$ of length as in~\autoref{eqn:feature_vector}.
\begin{equation}
    |V| = |{Call_1, Call_2, ..., Call_i, ..., Call_n}|
    \label{eqn:feature_vector}
\end{equation}
Each dimension in $V$ corresponds to a specific API Call, and the value of the dimension is the number of API calls belonging to the specific function in a particular sample as shown in~\autoref{api_eqn:call}.
\begin{equation}
    V_i = |Call_i| 
    \label{api_eqn:call}
\end{equation}

When we consider the Bigram model, we also look at the immediately preceding API call for the second part. 
Furthermore, we do this using a sliding window approach over the entire API call sequence.
The length of the feature vector in theory would be the total number of combinations of two API calls, which is $|V|^2$~($59^2$).

Like the Unigram model, we create a feature vector for every combination of API calls.
The vector would comprise two consecutive calls we concatenate, like $Call_1Call_2$.
The total count will be the value for the feature at index $i$.
\begin{equation}
    V_i = |\theta(Call_{i-1},Call_{i})|
\end{equation}
where,
$\theta$ is a mapping from two consecutive API calls to an index.
For the third part, we consider two previous API calls for the sequence. 
%The feature vector is quite large compared to the feature vectors of the unigram and bigram models.
We follow the same procedure as the Unigram model with the only difference in the length of the feature vector and the number of API calls considered.

And for the Trigram model, we consider three consecutive calls as given in~\autoref{eqn:trigram}. In this case, the size of the feature vector expands to $|V|^3$~($=59^3$)~which is quite large compared to the feature vectors of the Unigram and Bigram models.
\begin{equation}
    V_i = |\theta(Call_{i-2},Call_{i-1},Call_{i})|
    \label{eqn:trigram}
\end{equation}
where,
$\theta$ is a mapping from three consecutive API calls to an index.
The index mapping the unique function sequences to an index is provided as a JSON file in our online code repository.
The final model, which we call the Combined model, consists of creating a feature vector that concatenates the feature vectors of the Unigram, Bigram, and Trigram models. 
We do this in the hope that the~\textit{Combined Model}~exploits the positive aspects of Unigram, Bigram, and Trigram models independently to obtain discriminating information from any of the inputs of the models.
%We observe that there are 59 unique API calls to the ntdll.dll library corresponding to the fifty-nine functions we mapped.
%Using these fifty-nine unique function calls, we find that there are 2540 unique~(where we consider two consecutive API calls)~API calls and 5483 unique trigrams ~(where we consider three consecutive API calls)~API calls using a sliding window approach.
The feature vector length for the Unigram model is $59$ corresponding to the number of traced functions to the ntdll.dll, which is manageable, but the Bigram and Trigram models have theoretical lengths of $3481$ and $205379$. 
Although the Bigram feature vector is manageable, it is still quite large, and a Trigram-based feature vector is only possible for more than $330k$ samples on machines with substantial amounts of memory.
Moreover, such a feature vector could be sparse since most values would be zero. 
Therefore, we efficiently identify all the unique bigram and trigram function calls and create feature vectors using only those present in the dataset.
There are~$2540$~unique bigram calls and~$5483$~unique trigram combinations in the dataset.
Therefore, for practicality and to save memory, we limit the Bigram model and Trigram model feature vectors to a length of~$2540$ and $ 5483$~respectively.
We then train a random forest on these feature vectors and predict whether a given sample from the test set is malware or benign.
The dataset is unbalanced, containing many more malicious samples than benign ones.

