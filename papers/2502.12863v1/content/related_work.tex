\section{Related Work}
\label{sec:related_work}
Aboaoja et al. published a survey on the various issues and challenges of Malware Detection~\cite{aboaoja2022malware}. The authors outline that detecting evasive malware is still one of the biggest challenges. Although there are several approaches to detecting evasive malware, such as using multiple execution environments to identify evasive behavior, the time and resource complexity lead to each approach having its weakness.
Shukla et al. developed a method that uses RNN to detect the so-called stealthy malware~\cite{shukla2019stealthy}. The authors define stealthy malware as "malware created by embedding the malware into a benign application through advanced obfuscation strategies to thwart the detection." The authors "translate the application binaries into images, further convert it into sequences, and extract local features for stealthy malware detection."
Feng et al. proposed the method DawnGNN using Graph Attention Networks~(GAT)~\cite{feng2024dawngnn}. The authors proposed a novel documentation-augmented Windows Malware Detection Framework. The method works by converting the API sequences into API graphs to extract contextual information. The authors encode the functionality descriptions using BERT and finally use Graph attention for classification.
Li et al. proposed a method to detect dynamic malware based on API calls~\cite{li2022novel}. The authors used \textit{intrinsic features} of the API sequence. The authors claim that this allows the models to capture and combine more meaningful features. 
The authors then use the category, action, and operation object of the API to represent the semantic information of each API call. 
The authors do the classification using a Bidirectional LSTM module and their results outperform the baselines.
Cui et al. proposed a graph-based approach to detect malware from API-based call sequences~\cite{cui2023api2vec}. 
The proposed method works by creating two graphs, a Temporal Process Graph~(TPG)~and a Temporal API Graph~(TAG)~to model intra-process behavior. 
A heuristic random walk algorithm then generates several paths that can capture the malware behavior.
The authors generate the embeddings using the paths pre-trained by the Doc2Vec model.
Chen et al. proposed a parameter-augmented approach for the Microsoft Windows platform called~\textit{CruParamer}~\cite{chen2022cruparamer}. The method employs rule-based and clustering-based classification to compute the sensitivity of an API-call parameter to malicious behavior.
The classification is done by concatenating the API embedding to the sensitive embedding of the labeled APIs so that their relationship is captured in the input vector.
The authors then train a binary classifier to identify malware, and according to the authors, their model outperforms naive models.
Almousa et al. proposed a method to identify ransomware attacks based on API calls~\cite {almousa2021api}. The authors initially studied the lifecycle of ransomware on the Microsoft Windows platform. The next step was to extract malicious code patterns. The authors used data from publicly available repositories and sampled the malicious code in a sandbox. Machine learning models were built based on this analysis and yielded a high detection rate.

There are also multiple public malware datasets based on API calls.
Catak et al. published a malware dataset that contained different malware types~\cite{catak2019benchmark}. The dataset contains eight malware types, namely, Trojan, Backdoor, Downloader, Worms, Spyware Adware, Dropper, and Virus, for 7107 samples. 
The authors created the dataset using the Cuckoo Sandbox, available on GitHub.
Zhang published a dataset of 10,654 samples with sample labels~\cite{zhang22}.
The author divides the dataset into normal, ransomware, miner, DDoS Trojan, worm, infective virus, backdoor, and Trojan.
This dataset is from the Alibaba Security Algorithm Challenge.
Trinh created a much larger dataset of 1.55 million samples, of which 800,000 malware and 750,000 "goodware" samples are present~\cite{trinh21}.
Another dataset is from Oliveira with "42,797 malware API call sequences and 1,079 goodware API call sequences"~\cite{de2023behavioral}.
The dataset comprises API call sequences of the "first 100 non-repeated consecutive API calls associated with the parent process, extracted from the 'calls' elements of Cuckoo Sandbox reports".
A few other open datasets from challenges, such as the Aliyun Malware Detection Dataset~\cite{tianchi2016} exist, but most of the datasets are either unavailable publicly or have a too narrow scope.
More recent work is by Maniriho et al., where the authors created their dataset~\cite{maniriho2023api}. The dataset consists of 1285 malicious and 1285 benign samples. The dataset is publicly available and hosted on GitHub.