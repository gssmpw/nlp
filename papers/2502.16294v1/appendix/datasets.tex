\section{Datasets}
As datasets, we used 9 benchmark datasets which are commonly used in multivariate time-series forecasting. These consist of four ETT datasets \cite{haoyietal-informer-2021} (ETTh1, ETTh2, ETTm1, ETTm2), ECL (Electricity Consuming Load), Exchange, Traffic, Weather and Solar Energy.   Except for the Solar Energy datasets, the others are benchmarked by \cite{wu2021autoformer}, while the Solar is introduced by \cite{LSTNet}. We splitted the training, validation and test sets in a chronological way deterministically, following \cite{Yuqietal-2023-PatchTST, liu2023itransformer}.  We will briefly explain the features of these datasets in this section. 

\textbf{ETT Datasets.} The abbreviation ETT refers to Electricity Transformer Temperature \cite{haoyietal-informer-2021}. All ETT datasets consist of seven variates. The ETTh1 and ETTh2 datasets are sampled hourly, while the ETTm1 and ETTm2 datasets are sampled every 15 minutes. Specifically, the ETTh datasets contain 8545, 2881, and 2881 data points in the training, validation, and test sets, respectively. In contrast, the ETTm datasets comprise 34465, 11521, and 11521 data points in the training, validation, and test sets, respectively \cite{liu2023itransformer}.

\textbf{ECL Dataset.} The abbreviation ECL refers to the electricity consumption load of 321 users \cite{wu2021autoformer}. It is recorded in hourly intervals, resulting in a dataset with 321 variates. The ECL dataset contains 18317, 2633, and 5261 data points in the training, validation, and test sets, respectively \cite{liu2023itransformer}.

\textbf{Exchange Dataset.} This dataset provides daily exchange rates for eight countries \cite{wu2021autoformer}, comprising eight variates. The Exchange dataset includes 5120, 665, and 1422 data points in the training, validation, and test sets, respectively \cite{liu2023itransformer}. Some works, such as PatchTST \cite{Yuqietal-2023-PatchTST}, avoid using this dataset as a benchmark because simple naive predictions (using the last observed value) often outperform more complex methods. However, for completeness, we have included it in our analysis.

\textbf{Traffic Dataset.} This dataset includes hourly road occupancy rates from 862 locations \cite{wu2021autoformer}, resulting in 862 variates. The traffic dataset contains 12185, 1757, and 3509 data points in the training, validation, and test sets, respectively \cite{liu2023itransformer}. It is by far the most high-dimensional dataset in our evaluation. 

\textbf{Weather Dataset.} This dataset includes 21 meteorological factors collected every 10 minutes \cite{wu2021autoformer}, resulting in 21 variates. The weather dataset contains 36792, 5271, and 10540 data points in the training, validation, and test sets, respectively \cite{liu2023itransformer}.

\textbf{Solar-Energy Dataset.} This dataset includes power production values from 137 solar power plants, sampled every 10 minutes \cite{LSTNet}, resulting in 137 variables. The solar energy dataset contains 36601, 5161, and 10417 data points in the training, validation, and test sets, respectively \cite{liu2023itransformer}.  