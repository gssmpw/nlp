\begin{figure}[t!] 
   \centering    \includegraphics[width= 0.47\textwidth]{figures/timePFN.pdf}
   \caption{Illustration of the architecture of TimePFN. Variates are filtered with 1D convolutions, to be patched with overlapping strides, following \cite{Yuqietal-2023-PatchTST}. They are then fed into transformer encoder with channel mixing, with the final forecast coming from the final feedforward network.}
    %in the example - the one above is an acyclic graph while the other graph contains a cycle. For each sub-graph, we calculate the strongly connected components. According to Definition \ref{cyc sub}, we derive $\bdata$ based on $\data$ as shown in the right figure. In other words, the graphs are reduced to non-singleton SCCs. As a result, $\Gc^{(1)}$ is removed from the dataset.   \label{fig:main_figure}
\end{figure}