\def\isarxiv{1} %%% for icml submission version, we comment this line

\ifdefined\isarxiv
\documentclass[11pt]{article}

\usepackage[numbers]{natbib}

\else
\documentclass[twoside]{article}

% \usepackage{aistats2025}
% If your paper is accepted, change the options for the package
% aistats2025 as follows:
%
\usepackage[accepted]{aistats2025}
%
% This option will print headings for the title of your paper and
% headings for the authors names, plus a copyright note at the end of
% the first column of the first page.

% If you set papersize explicitly, activate the following three lines:
%\special{papersize = 8.5in, 11in}
%\setlength{\pdfpageheight}{11in}
%\setlength{\pdfpagewidth}{8.5in}

% If you use natbib package, activate the following three lines:
\usepackage[round]{natbib}
\renewcommand{\bibname}{References}
\renewcommand{\bibsection}{\subsubsection*{\bibname}}

% If you use BibTeX in apalike style, activate the following line:
%\bibliographystyle{apalike}

\fi


\usepackage{amsmath}
\usepackage{amsthm}
\usepackage{amssymb}
\usepackage{algorithm}
\usepackage{subfig}
\usepackage{algpseudocode}
\usepackage{graphicx}
\usepackage{grffile}
\usepackage{wrapfig,epsfig}
\usepackage{url}
\usepackage{xcolor}
\usepackage{epstopdf}


\usepackage{bbm}
\usepackage{dsfont}

 %%% print refs in table of contents
%\displaybreak
\allowdisplaybreaks

%\usepackage[lmargin=1in,rmargin=1in,tmargin=0.8in,bmargin=0.8in]{geometry}

\ifdefined\isarxiv

\let\C\relax
\usepackage{tikz}
\usepackage{hyperref}  %%% arxiv don't allow this.
\hypersetup{colorlinks=true,citecolor=blue,linkcolor=blue} %%% Zhao : maybe we should comment this in submission.
\usetikzlibrary{arrows}
\usepackage[margin=1in]{geometry}

\else

% \usepackage[utf8]{inputenc} % allow utf-8 input
% \usepackage[T1]{fontenc}    % use 8-bit T1 fonts

% \usepackage{booktabs}       % professional-quality tables
% \usepackage{amsfonts}       % blackboard math symbols
% \usepackage{nicefrac}       % compact symbols for 1/2, etc.
% \usepackage{microtype}      % microtypography
\usepackage{hyperref}       % hyperlinks
\definecolor{mydarkblue}{rgb}{0,0.08,0.45}
\hypersetup{colorlinks=true, citecolor=mydarkblue,linkcolor=mydarkblue}
%\usepackage[capitalize,noabbrev]{cleveref}
%\usepackage{colortbl}

\fi
%\linespread{1}
%\newcommand{\QED}{\hfill$\qed$}
%\graphicspath{{./figs/}}

\theoremstyle{plain}
\newtheorem{theorem}{Theorem}[section]
\newtheorem{lemma}[theorem]{Lemma}
\newtheorem{definition}[theorem]{Definition}
\newtheorem{notation}[theorem]{Notation}
%\newtheorem{proof}[theorem]{Proof}
\newtheorem{proposition}[theorem]{Proposition}
\newtheorem{corollary}[theorem]{Corollary}
\newtheorem{conjecture}[theorem]{Conjecture}
\newtheorem{assumption}[theorem]{Assumption}
\newtheorem{observation}[theorem]{Observation}
\newtheorem{fact}[theorem]{Fact}
\newtheorem{remark}[theorem]{Remark}
\newtheorem{claim}[theorem]{Claim}
\newtheorem{example}[theorem]{Example}
\newtheorem{problem}[theorem]{Problem}
\newtheorem{open}[theorem]{Open Problem}
\newtheorem{property}[theorem]{Property}
\newtheorem{hypothesis}[theorem]{Hypothesis}

\newcommand{\wh}{\widehat}
\newcommand{\wt}{\widetilde}
\newcommand{\ov}{\overline}
\newcommand{\N}{\mathcal{N}}
\newcommand{\R}{\mathbb{R}}
\newcommand{\RHS}{\mathrm{RHS}}
\newcommand{\LHS}{\mathrm{LHS}}
%\renewcommand{\d}{\mathrm{d}}
\renewcommand{\i}{\mathbf{i}}
\renewcommand{\tilde}{\wt}
\renewcommand{\hat}{\wh}
\newcommand{\Tmat}{{\cal T}_{\mathrm{mat}}}
\renewcommand{\S}{\mathsf{S}}


%%%Zhao: Below comments are only novel for this paper.
\renewcommand{\v}{\mathsf{v}}
\newcommand{\m}{\mathsf{m}}
\renewcommand{\d}{\mathsf{d}}
\renewcommand{\H}{\mathsf{H}}
%%%Zhao: Above comments are only novel for this paper.

\DeclareMathOperator*{\E}{{\mathbb{E}}}
\DeclareMathOperator*{\var}{\mathrm{Var}}
\DeclareMathOperator*{\Z}{\mathbb{Z}}
\DeclareMathOperator*{\C}{\mathbb{C}}
\DeclareMathOperator*{\D}{\mathcal{D}}
\DeclareMathOperator*{\median}{median}
\DeclareMathOperator*{\mean}{mean}
\DeclareMathOperator{\OPT}{OPT}
\DeclareMathOperator{\supp}{supp}
\DeclareMathOperator{\poly}{poly}

\DeclareMathOperator{\nnz}{nnz}
\DeclareMathOperator{\sparsity}{sparsity}
\DeclareMathOperator{\rank}{rank}
\DeclareMathOperator{\diag}{diag}
\DeclareMathOperator{\dist}{dist}
\DeclareMathOperator{\cost}{cost}
\DeclareMathOperator{\vect}{vec}
\DeclareMathOperator{\tr}{tr}
\DeclareMathOperator{\dis}{dis}
\DeclareMathOperator{\cts}{cts}



\makeatletter
\newcommand*{\RN}[1]{\expandafter\@slowromancap\romannumeral #1@}
\makeatother
% \newcommand{\Zhao}[1]{{\color{red}[Zhao: #1]}}
% \newcommand{\Zhenmei}[1]{{\color{purple}[Zhenmei: #1]}}  %%%Change to intern name
% \newcommand{\Chenyang}[1]{{\color{blue}[Chenyang: #1]}} %%%Change to intern name



\usepackage{lineno}
\def\linenumberfont{\normalfont\small}





\begin{document}

\ifdefined\isarxiv

\date{}


\title{When Can We Solve the Weighted Low Rank Approximation Problem in Truly Subquadratic Time?}
\author{ 
Chenyang Li\thanks{\texttt{
lchenyang550@gmail.com}. Fuzhou University.}
\and
Yingyu Liang\thanks{\texttt{
yingyul@hku.hk}. The University of Hong Kong. \texttt{
yliang@cs.wisc.edu}. University of Wisconsin-Madison.} 
\and
Zhenmei Shi\thanks{\texttt{
zhmeishi@cs.wisc.edu}. University of Wisconsin-Madison.}
\and 
Zhao Song\thanks{\texttt{ magic.linuxkde@gmail.com}. The Simons Institute for the Theory of Computing at the UC, Berkeley.}
}

\else

% If your paper is accepted and the title of your paper is very long,
% the style will print as headings an error message. Use the following
% command to supply a shorter title of your paper so that it can be
% used as headings.
%
\runningtitle{When Can We Solve the Weighted Low Rank Approximation Problem in Truly Subquadratic Time?}

% If your paper is accepted and the number of authors is large, the
% style will print as headings an error message. Use the following
% command to supply a shorter version of the authors names so that
% they can be used as headings (for example, use only the surnames)
%
%\runningauthor{Surname 1, Surname 2, Surname 3, ...., Surname n}

\twocolumn[

\aistatstitle{When Can We Solve the Weighted Low Rank Approximation Problem in Truly Subquadratic Time?}

\aistatsauthor{ 
Chenyang Li$^{1}$
% \thanks{{\tt \small yingyul@hku.hk}. {\tt\small yliang@cs.wisc.edu}.}
\And 
Yingyu Liang$^{2,3}$
% \thanks{{\tt \small yingyul@hku.hk}. {\tt\small yliang@cs.wisc.edu}.} 
\And  
Zhenmei Shi$^{3}$
% \thanks{{\tt \small zhmeishi@cs.wisc.edu}.}
\And 
Zhao Song$^{4}$
% \thanks{{\tt \small magic.linuxkde@gmail.com}.}
}


\aistatsaddress{ 
$^1$Fuzhou University. \qquad
$^2$The University of Hong Kong. \qquad
$^3$University of Wisconsin-Madison. 
\qquad
\\
$^4$The Simons Institute for the Theory of Computing at the University of California, Berkeley. 
} 
]

% $^4$Adobe Research, USA. 
%   \qquad 
%   $^2$University of Wisconsin-Madison, USA. 
%   \\
%   $^\varheart$The University of Hong Kong, HongKong. 
%   \qquad
%   $^1$Tsinghua University, China. 
%   \\
%   $^3$The Simons Institute for the Theory of Computing at the University of California, Berkeley, USA. 

\fi





\ifdefined\isarxiv
\begin{titlepage}
  \maketitle
  \begin{abstract}
\begin{abstract}


The choice of representation for geographic location significantly impacts the accuracy of models for a broad range of geospatial tasks, including fine-grained species classification, population density estimation, and biome classification. Recent works like SatCLIP and GeoCLIP learn such representations by contrastively aligning geolocation with co-located images. While these methods work exceptionally well, in this paper, we posit that the current training strategies fail to fully capture the important visual features. We provide an information theoretic perspective on why the resulting embeddings from these methods discard crucial visual information that is important for many downstream tasks. To solve this problem, we propose a novel retrieval-augmented strategy called RANGE. We build our method on the intuition that the visual features of a location can be estimated by combining the visual features from multiple similar-looking locations. We evaluate our method across a wide variety of tasks. Our results show that RANGE outperforms the existing state-of-the-art models with significant margins in most tasks. We show gains of up to 13.1\% on classification tasks and 0.145 $R^2$ on regression tasks. All our code and models will be made available at: \href{https://github.com/mvrl/RANGE}{https://github.com/mvrl/RANGE}.

\end{abstract}



  \end{abstract}
  \thispagestyle{empty}
\end{titlepage}

{\hypersetup{linkcolor=black}
%\tableofcontents
}
\newpage

\else

\begin{abstract}
\begin{abstract}


The choice of representation for geographic location significantly impacts the accuracy of models for a broad range of geospatial tasks, including fine-grained species classification, population density estimation, and biome classification. Recent works like SatCLIP and GeoCLIP learn such representations by contrastively aligning geolocation with co-located images. While these methods work exceptionally well, in this paper, we posit that the current training strategies fail to fully capture the important visual features. We provide an information theoretic perspective on why the resulting embeddings from these methods discard crucial visual information that is important for many downstream tasks. To solve this problem, we propose a novel retrieval-augmented strategy called RANGE. We build our method on the intuition that the visual features of a location can be estimated by combining the visual features from multiple similar-looking locations. We evaluate our method across a wide variety of tasks. Our results show that RANGE outperforms the existing state-of-the-art models with significant margins in most tasks. We show gains of up to 13.1\% on classification tasks and 0.145 $R^2$ on regression tasks. All our code and models will be made available at: \href{https://github.com/mvrl/RANGE}{https://github.com/mvrl/RANGE}.

\end{abstract}


\end{abstract}

\fi


\section{Introduction}
Backdoor attacks pose a concealed yet profound security risk to machine learning (ML) models, for which the adversaries can inject a stealth backdoor into the model during training, enabling them to illicitly control the model's output upon encountering predefined inputs. These attacks can even occur without the knowledge of developers or end-users, thereby undermining the trust in ML systems. As ML becomes more deeply embedded in critical sectors like finance, healthcare, and autonomous driving \citep{he2016deep, liu2020computing, tournier2019mrtrix3, adjabi2020past}, the potential damage from backdoor attacks grows, underscoring the emergency for developing robust defense mechanisms against backdoor attacks.

To address the threat of backdoor attacks, researchers have developed a variety of strategies \cite{liu2018fine,wu2021adversarial,wang2019neural,zeng2022adversarial,zhu2023neural,Zhu_2023_ICCV, wei2024shared,wei2024d3}, aimed at purifying backdoors within victim models. These methods are designed to integrate with current deployment workflows seamlessly and have demonstrated significant success in mitigating the effects of backdoor triggers \cite{wubackdoorbench, wu2023defenses, wu2024backdoorbench,dunnett2024countering}.  However, most state-of-the-art (SOTA) backdoor purification methods operate under the assumption that a small clean dataset, often referred to as \textbf{auxiliary dataset}, is available for purification. Such an assumption poses practical challenges, especially in scenarios where data is scarce. To tackle this challenge, efforts have been made to reduce the size of the required auxiliary dataset~\cite{chai2022oneshot,li2023reconstructive, Zhu_2023_ICCV} and even explore dataset-free purification techniques~\cite{zheng2022data,hong2023revisiting,lin2024fusing}. Although these approaches offer some improvements, recent evaluations \cite{dunnett2024countering, wu2024backdoorbench} continue to highlight the importance of sufficient auxiliary data for achieving robust defenses against backdoor attacks.

While significant progress has been made in reducing the size of auxiliary datasets, an equally critical yet underexplored question remains: \emph{how does the nature of the auxiliary dataset affect purification effectiveness?} In  real-world  applications, auxiliary datasets can vary widely, encompassing in-distribution data, synthetic data, or external data from different sources. Understanding how each type of auxiliary dataset influences the purification effectiveness is vital for selecting or constructing the most suitable auxiliary dataset and the corresponding technique. For instance, when multiple datasets are available, understanding how different datasets contribute to purification can guide defenders in selecting or crafting the most appropriate dataset. Conversely, when only limited auxiliary data is accessible, knowing which purification technique works best under those constraints is critical. Therefore, there is an urgent need for a thorough investigation into the impact of auxiliary datasets on purification effectiveness to guide defenders in  enhancing the security of ML systems. 

In this paper, we systematically investigate the critical role of auxiliary datasets in backdoor purification, aiming to bridge the gap between idealized and practical purification scenarios.  Specifically, we first construct a diverse set of auxiliary datasets to emulate real-world conditions, as summarized in Table~\ref{overall}. These datasets include in-distribution data, synthetic data, and external data from other sources. Through an evaluation of SOTA backdoor purification methods across these datasets, we uncover several critical insights: \textbf{1)} In-distribution datasets, particularly those carefully filtered from the original training data of the victim model, effectively preserve the model’s utility for its intended tasks but may fall short in eliminating backdoors. \textbf{2)} Incorporating OOD datasets can help the model forget backdoors but also bring the risk of forgetting critical learned knowledge, significantly degrading its overall performance. Building on these findings, we propose Guided Input Calibration (GIC), a novel technique that enhances backdoor purification by adaptively transforming auxiliary data to better align with the victim model’s learned representations. By leveraging the victim model itself to guide this transformation, GIC optimizes the purification process, striking a balance between preserving model utility and mitigating backdoor threats. Extensive experiments demonstrate that GIC significantly improves the effectiveness of backdoor purification across diverse auxiliary datasets, providing a practical and robust defense solution.

Our main contributions are threefold:
\textbf{1) Impact analysis of auxiliary datasets:} We take the \textbf{first step}  in systematically investigating how different types of auxiliary datasets influence backdoor purification effectiveness. Our findings provide novel insights and serve as a foundation for future research on optimizing dataset selection and construction for enhanced backdoor defense.
%
\textbf{2) Compilation and evaluation of diverse auxiliary datasets:}  We have compiled and rigorously evaluated a diverse set of auxiliary datasets using SOTA purification methods, making our datasets and code publicly available to facilitate and support future research on practical backdoor defense strategies.
%
\textbf{3) Introduction of GIC:} We introduce GIC, the \textbf{first} dedicated solution designed to align auxiliary datasets with the model’s learned representations, significantly enhancing backdoor mitigation across various dataset types. Our approach sets a new benchmark for practical and effective backdoor defense.


 %%% Section 1. Introduction
\section{Related Work}
\label{sec:related-works}
\subsection{Novel View Synthesis}
Novel view synthesis is a foundational task in the computer vision and graphics, which aims to generate unseen views of a scene from a given set of images.
% Many methods have been designed to solve this problem by posing it as 3D geometry based rendering, where point clouds~\cite{point_differentiable,point_nfs}, mesh~\cite{worldsheet,FVS,SVS}, planes~\cite{automatci_photo_pop_up,tour_into_the_picture} and multi-plane images~\cite{MINE,single_view_mpi,stereo_magnification}, \etal
Numerous methods have been developed to address this problem by approaching it as 3D geometry-based rendering, such as using meshes~\cite{worldsheet,FVS,SVS}, MPI~\cite{MINE,single_view_mpi,stereo_magnification}, point clouds~\cite{point_differentiable,point_nfs}, etc.
% planes~\cite{automatci_photo_pop_up,tour_into_the_picture}, 


\begin{figure*}[!t]
    \centering
    \includegraphics[width=1.0\linewidth]{figures/overview-v7.png}
    %\caption{\textbf{Overview.} Given a set of images, our method obtains both camera intrinsics and extrinsics, as well as a 3DGS model. First, we obtain the initial camera parameters, global track points from image correspondences and monodepth with reprojection loss. Then we incorporate the global track information and select Gaussian kernels associated with track points. We jointly optimize the parameters $K$, $T_{cw}$, 3DGS through multi-view geometric consistency $L_{t2d}$, $L_{t3d}$, $L_{scale}$ and photometric consistency $L_1$, $L_{D-SSIM}$.}
    \caption{\textbf{Overview.} Given a set of images, our method obtains both camera intrinsics and extrinsics, as well as a 3DGS model. During the initialization, we extract the global tracks, and initialize camera parameters and Gaussians from image correspondences and monodepth with reprojection loss. We determine Gaussian kernels with recovered 3D track points, and then jointly optimize the parameters $K$, $T_{cw}$, 3DGS through the proposed global track constraints (i.e., $L_{t2d}$, $L_{t3d}$, and $L_{scale}$) and original photometric losses (i.e., $L_1$ and $L_{D-SSIM}$).}
    \label{fig:overview}
\end{figure*}

Recently, Neural Radiance Fields (NeRF)~\cite{2020NeRF} provide a novel solution to this problem by representing scenes as implicit radiance fields using neural networks, achieving photo-realistic rendering quality. Although having some works in improving efficiency~\cite{instant_nerf2022, lin2022enerf}, the time-consuming training and rendering still limit its practicality.
Alternatively, 3D Gaussian Splatting (3DGS)~\cite{3DGS2023} models the scene as explicit Gaussian kernels, with differentiable splatting for rendering. Its improved real-time rendering performance, lower storage and efficiency, quickly attract more attentions.
% Different from NeRF-based methods which need MLPs to model the scene and huge computational cost for rendering, 3DGS has stronger real-time performance, higher storage and computational efficiency, benefits from its explicit representation and gradient backpropagation.

\subsection{Optimizing Camera Poses in NeRFs and 3DGS}
Although NeRF and 3DGS can provide impressive scene representation, these methods all need accurate camera parameters (both intrinsic and extrinsic) as additional inputs, which are mostly obtained by COLMAP~\cite{colmap2016}.
% This strong reliance on COLMAP significantly limits their use in real-world applications, so optimizing the camera parameters during the scene training becomes crucial.
When the prior is inaccurate or unknown, accurately estimating camera parameters and scene representations becomes crucial.

% In early works, only photometric constraints are used for scene training and camera pose estimation. 
% iNeRF~\cite{iNerf2021} optimizes the camera poses based on a pre-trained NeRF model.
% NeRFmm~\cite{wang2021nerfmm} introduce a joint optimization process, which estimates the camera poses and trains NeRF model jointly.
% BARF~\cite{barf2021} and GARF~\cite{2022GARF} provide new positional encoding strategy to handle with the gradient inconsistency issue of positional embedding and yield promising results.
% However, they achieve satisfactory optimization results when only the pose initialization is quite closed to the ground-truth, as the photometric constrains can only improve the quality of camera estimation within a small range.
% Later, more prior information of geometry and correspondence, \ie monocular depth and feature matching, are introduced into joint optimisation to enhance the capability of camera poses estimation.
% SC-NeRF~\cite{SCNeRF2021} minimizes a projected ray distance loss based on correspondence of adjacent frames.
% NoPe-NeRF~\cite{bian2022nopenerf} chooses monocular depth maps as geometric priors, and defines undistorted depth loss and relative pose constraints for joint optimization.
In earlier studies, scene training and camera pose estimation relied solely on photometric constraints. iNeRF~\cite{iNerf2021} refines the camera poses using a pre-trained NeRF model. NeRFmm~\cite{wang2021nerfmm} introduces a joint optimization approach that simultaneously estimates camera poses and trains the NeRF model. BARF~\cite{barf2021} and GARF~\cite{2022GARF} propose a new positional encoding strategy to address the gradient inconsistency issues in positional embedding, achieving promising results. However, these methods only yield satisfactory optimization when the initial pose is very close to the ground truth, as photometric constraints alone can only enhance camera estimation quality within a limited range. Subsequently, 
% additional prior information on geometry and correspondence, such as monocular depth and feature matching, has been incorporated into joint optimization to improve the accuracy of camera pose estimation. 
SC-NeRF~\cite{SCNeRF2021} minimizes a projected ray distance loss based on correspondence between adjacent frames. NoPe-NeRF~\cite{bian2022nopenerf} utilizes monocular depth maps as geometric priors and defines undistorted depth loss and relative pose constraints.

% With regard to 3D Gaussian Splatting, CF-3DGS~\cite{CF-3DGS-2024} also leverages mono-depth information to constrain the optimization of local 3DGS for relative pose estimation and later learn a global 3DGS progressively in a sequential manner.
% InstantSplat~\cite{fan2024instantsplat} focus on sparse view scenes, first use DUSt3R~\cite{dust3r2024cvpr} to generate a set of densely covered and pixel-aligned points for 3D Gaussian initialization, then introduce a parallel grid partitioning strategy in joint optimization to speed up.
% % Jiang et al.~\cite{Jiang_2024sig} proposed to build the scene continuously and progressively, to next unregistered frame, they use registration and adjustment to adjust the previous registered camera poses and align unregistered monocular depths, later refine the joint model by matching detected correspondences in screen-space coordinates.
% \gjh{Jiang et al.~\cite{Jiang_2024sig} also implemented an incremental approach for reconstructing camera poses and scenes. Initially, they perform feature matching between the current image and the image rendered by a differentiable surface renderer. They then construct matching point errors, depth errors, and photometric errors to achieve the registration and adjustment of the current image. Finally, based on the depth map, the pixels of the current image are projected as new 3D Gaussians. However, this method still exhibits limitations when dealing with complex scenes and unordered images.}
% % CG-3DGS~\cite{sun2024correspondenceguidedsfmfree3dgaussian} follows CF-3DGS, first construct a coarse point cloud from mono-depth maps to train a 3DGS model, then progressively estimate camera poses based on this pre-trained model by constraining the correspondences between rendering view and ground-truth.
% \gjh{Similarly, CG-3DGS~\cite{sun2024correspondenceguidedsfmfree3dgaussian} first utilizes monocular depth estimation and the camera parameters from the first frame to initialize a set of 3D Gaussians. It then progressively estimates camera poses based on this pre-trained model by constraining the correspondences between the rendered views and the ground truth.}
% % Free-SurGS~\cite{freesurgs2024} matches the projection flow derived from 3D Gaussians with optical flow to estimate the poses, to compensate for the limitations of photometric loss.
% \gjh{Free-SurGS~\cite{freesurgs2024} introduces the first SfM-free 3DGS approach for surgical scene reconstruction. Due to the challenges posed by weak textures and photometric inconsistencies in surgical scenes, Free-SurGS achieves pose estimation by minimizing the flow loss between the projection flow and the optical flow. Subsequently, it keeps the camera pose fixed and optimizes the scene representation by minimizing the photometric loss, depth loss and flow loss.}
% \gjh{However, most current works assume camera intrinsics are known and primarily focus on optimizing camera poses. Additionally, these methods typically rely on sequentially ordered image inputs and incrementally optimize camera parameters and scene representation. This inevitably leads to drift errors, preventing the achievement of globally consistent results. Our work aims to address these issues.}

Regarding 3D Gaussian Splatting, CF-3DGS~\cite{CF-3DGS-2024} utilizes mono-depth information to refine the optimization of local 3DGS for relative pose estimation and subsequently learns a global 3DGS in a sequential manner. InstantSplat~\cite{fan2024instantsplat} targets sparse view scenes, initially employing DUSt3R~\cite{dust3r2024cvpr} to create a densely covered, pixel-aligned point set for initializing 3D Gaussian models, and then implements a parallel grid partitioning strategy to accelerate joint optimization. Jiang \etal~\cite{Jiang_2024sig} develops an incremental method for reconstructing camera poses and scenes, but it struggles with complex scenes and unordered images. 
% Similarly, CG-3DGS~\cite{sun2024correspondenceguidedsfmfree3dgaussian} progressively estimates camera poses using a pre-trained model by aligning the correspondences between rendered views and actual scenes. Free-SurGS~\cite{freesurgs2024} pioneers an SfM-free 3DGS method for reconstructing surgical scenes, overcoming challenges such as weak textures and photometric inconsistencies by minimizing the discrepancy between projection flow and optical flow.
%\pb{SF-3DGS-HT~\cite{ji2024sfmfree3dgaussiansplatting} introduced VFI into training as additional photometric constraints. They separated the whole scene into several local 3DGS models and then merged them hierarchically, which leads to a significant improvement on simple and dense view scenes.}
HT-3DGS~\cite{ji2024sfmfree3dgaussiansplatting} interpolates frames for training and splits the scene into local clips, using a hierarchical strategy to build 3DGS model. It works well for simple scenes, but fails with dramatic motions due to unstable interpolation and low efficiency.
% {While effective for simple scenes, it struggles with dramatic motion due to unstable view interpolation and suffers from low computational efficiency.}

However, most existing methods generally depend on sequentially ordered image inputs and incrementally optimize camera parameters and 3DGS, which often leads to drift errors and hinders achieving globally consistent results. Our work seeks to overcome these limitations.



\section{Preliminary} \label{sec:preliminary}
In this section, we first introduce the notations in Section~\ref{sec:notations}. Then, we present the general problem formulation in Section~\ref{sec:problem_formulation}.

\subsection{Notations}\label{sec:notations}
For any positive integer $n$, we use $[n]$ to denote the set $\{1, 2, \ldots, n\}$. We use $\mathbb{N}_+$ to represent the set of all positive integers. For two sets $\mathcal{B}$ and $\mathcal{C}$, we denote the set difference as $\mathcal{B} \setminus \mathcal{C}:=\{x\in \mathcal{B}:x\notin\mathcal{C}\}$. For a vector $x \in \mathbb{R}^d$, $\Diag(d)$ denotes a diagonal matrix $X \in \mathbb{R}^{d \times d}$, where the diagonal entries satisfy $X_{i,i} = x_i$ for all $i \in [d]$, and all off-diagonal entries are zero. We use $\mathbf{1}_n$ to denote an $n$-dimensional column vector with all entries equal to one.


\begin{figure}[!ht]
    \centering
    \includegraphics[width=0.95\linewidth]{our_research.pdf}
    \caption{
    Our research objective. This figure presents the goal of our study: creating a more equitable desk-rejection system. Consider Professor A, who has carelessly submitted numerous papers exceeding the submission limit, collaborating with another senior researcher (Professor B) with many submissions, and a young student with only one paper. Our proposed system prioritizes desk-rejecting papers from authors with a large number of submissions first, thereby increasing the student’s chances of having their paper accepted. This approach aims to mitigate the disparity in the impact of desk rejections and promote fairness.
    }
    \label{fig:our_research}
\end{figure}

\subsection{Problem Formulation} \label{sec:problem_formulation}

In this section, we further introduce the actual problem we will investigate in this paper, where we begin with introducing the definition for three kinds of authors that will appear later in our discussion. 


\begin{definition}[Submission Limit Problem]\label{def:submit_limit_problem}
    Let $\mathcal{A} = \{a_1, a_2, \dots, a_n\}$ denote the set of $n$ authors, and let $\mathcal{P} = \{p_1, p_2, \dots, p_m\}$ denote the set of $m$ papers. Each author $a_i \in \mathcal{A}$ has a subset of papers $P_i \subseteq \mathcal{P}$, and each paper $p_j \in \mathcal{P}$ is authored by a subset of authors $A_j \subseteq \mathcal{A}$. For each author, $a_i \in \mathcal{A}$, let $C_i$ denote the set of all coauthors of $a_i$ and let $x \in \mathbb{N}_+$ denote the maximum number of papers each author can submit. 

    The goal is to find a subset $S \subseteq \mathcal{P}$ of papers (to keep) such that for every $a_i\in\mathcal{A},$
    \begin{align*}
        \underbrace{|\{p_j \in  S : a_i \in A_j\}|}_{\#\mathrm{remained~papers~of~author}~a_i} \leq x.  
    \end{align*}
    or equivalently find a subset $\ov{S} \subseteq \mathcal{P}$ of papers (to reject) such that for every $a_i \in \mathcal{A}$,
        \begin{align*}
        |P_i| - \underbrace{|\{j \in  \ov{S} : i \in A_j\}|}_{\#{\mathrm{rejected~papers~of~author}~}a_i} \leq x.  
    \end{align*}
\end{definition}

We now present several fundamental facts related to Definition~\ref{def:submit_limit_problem}, which can be easily verified through basic set theory. 
\begin{fact}
    For any author $a_i \in \mathcal{A}$ and paper $p_j \in \mathcal{P}$, $a_i \in A_j$ if and only if $p_j \in P_i$.
\end{fact}

\begin{fact}
    For each author $a_i \in \mathcal{A}$, the number of papers submitted by the author can be formulated as:
    \begin{align*}
        |P_i| = |\{p_j \in \mathcal{P} : a_i \in A_j\}|.
    \end{align*}
\end{fact}

\begin{fact}
    For each paper $j \in [m]$, the number of authors of this paper can be formulated as:
    \begin{align*}
        |A_j| = |\{a_i \in \mathcal{A} : p_j \in P_i\}|.
    \end{align*}
\end{fact}

\begin{fact}
    For each author $a_i \in \mathcal{A}$, the set of coauthors for author $a_i$ can be formulated as:
    \begin{align*}
        C_i = (\bigcup_{p_j \in P_i} A_j) \setminus \{ a_i \}.
    \end{align*}
\end{fact}


\section{WARMUP}\label{sec:additive}
 

In this section, to demonstrate the new technique, we prove the following theorem.

\begin{lemma}\label{lem:warmup}
Given two $n \times n$ size matrices $A$ and $W$, $1 \leq k \leq n$  such that: Let each entry of $A,W$ can be represented by $n^{\gamma}$ bits, with $\gamma \in (0,1)$; Let $\OPT$ be defined as Definition~\ref{def:opt}.  
Then, we can show
\begin{itemize}
\item {\bf Part 1.} There is an algorithm that runs in $2^{O(nk\log n)}$ time, and outputs a number $\Lambda$  such that $\OPT \leq \Lambda \leq 2 \OPT $. 
\item {\bf Part 2.} There is an algorithm that runs in $2^{O(n k \log n)}$ time and returns $U \in \R^{n \times k}$ and $V \in \R^{n \times k}$ such that $\| (UV^\top - A) \circ W \|_F^2 \leq 2 \OPT$.
\end{itemize}
\end{lemma}



\begin{proof}

{\bf Proof of Part 1.}

We can create $2nk$ variables to explicitly represent each entry of $U$ and $V$. Let $g(x) =  \| W \circ ( U V^\top - A ) \|_F^2$. Let $L = 2^{n^{\gamma}}$. Then, we can write down a polynomial system (the decision problem defined in Theorem~\ref{thm:decision_problem})
\begin{align*}
   \min & ~  g(x) \\
    \mathrm{~s.t.~} & ~  U_{i,j} \in [-L,L], \forall i,j \\
    & ~ V_{i,j} \in [-L,L], \forall i,j
\end{align*}
Using Theorem~\ref{thm:jpt13}, we know the above system has
\begin{align*}
    \m = 2nk, \v = 2nk, \d = 4, \H = n^{\gamma}, \wt{\H} = n^{\gamma} + O(nk).
\end{align*}
The lower bound on $g(x)$ (if $g(x)$ is not zero) is going to be  
\begin{align*}
    c_{\mathrm{lower}} = & ~ 2^{-2^{3\v \log (\d)} \log (\wt{H}) } \\
    \geq & ~  ~ 2^{-2^{ O( nk )} \log(n^{\gamma} + nk) } \\
    \geq & ~  ~ 2^{-2^{O(nk \log n)}}.
\end{align*}

By Lemma~\ref{lem:opt}, we know the upper bound is $C_{\mathrm{upper}} = \poly(n) \cdot 2^{n^{\gamma}}$.
After knowing the lower bound and upper bound on cost, the number of binary search iterations is upper-bounded by
\begin{align*}
\log( \frac{C_{\mathrm{upper}}}{ c_{\mathrm{lower}} } ) 
= ~ \log( \frac{ \poly(n) 2^{n^{\gamma}} }{ 2^{-2^{ O(nk \log n)} } } ) 
\leq  ~ 2^{O(nk \log n)}.
\end{align*}


In each of the above iterations, we need to run Theorem~\ref{thm:decision_problem} with the system 
\begin{align*}
    \mathrm{~s.t.~} 
    & ~  g(x) \in [\Gamma_t, 2 \Gamma_t] , ~ U_{i,j} \in [-L, L] , ~ V_{i,j} \in [-L, L]
\end{align*}
with parameters,
\begin{align*}
\m = 2nk+1,  \v = 2nk, \d =4, \H = n^{\gamma}.
\end{align*}
Then, running time complexity is
\begin{align*}
 (\m \d)^{O(\v)} \cdot \poly(\H) 
 =  & ~ (10nk)^{O(nk)} \cdot \poly(n^{\gamma}) \\
 =  & ~ 2^{O( n k \log n) }.
\end{align*}

Thus, combining the number of iterations and time for each iteration, we can find the number $\Gamma \in [\OPT, 2\OPT]$.

{\bf Proof of Part 2.}

Next, similar to {\bf Part 1}, we need to repeat the binary search for $2nk$ times for each variable in $U$ and $V$, and each time, the number of total binary search steps is $n^{\gamma}$. Thus, we can output the $U, V$ in the same running time as finding $\Gamma$.
\end{proof}

In the next few sections, we will explain how to reduce the number of variables and how to reduce the number of constraints.
\section{LOWER BOUND ON OPT}\label{sec:relative}



We assume that $W$ has $r$ distinct rows and $r$ distinct columns. Then, we get rid of the dependence on $n$ in the degree.

\begin{theorem}[Implicitly in \cite{rsw16}]\label{thm:lower_bound_on_cost}
Assuming that $W$ has $r$ distinct rows and $r$ distinct columns, each entry of $A$ and $W$ needs $n^{\gamma}$ bits to represent. Assume $\OPT >0$. Then we know that, with high probability,  
\begin{align*}
    \OPT \geq 2^{-n^{\gamma} 2^{\wt{O}(rk^2/\epsilon)} }.
\end{align*}
\end{theorem}


\begin{proof}


We use $A_{i,*} \in \R^n$ denote the $i$-th row of $A$. We use $W_{i,*} \in \R^n$ to denote the $i$-th row of $W$. Let $(U_1)_{i,*}$ denote the $i$-th row of $U_1$.  
For any $n \times k$ matrix $U_1$ and for any $k \times n$ matrix $Z_1$, we have
\begin{align*}
 & ~ \| (U_1 Z_1 - A) \circ W \|_F^2 \\
 = & ~ \sum_{i=1}^n \| (U_1)_{i,*} Z_1 \diag( W_{i,*} ) - A_{i,*} \diag( W_{i,*} ) \|_2^2.
\end{align*}

Based on the observation that $W$ has $r$ distinct rows, we use group $g_{1,1}, g_{1,2}, \cdots , g_{1,r}$ to denote $r$ disjoint sets such that
\begin{align*}
    \cup_{i=1}^r g_{1,i} = [n]
\end{align*}
For any $i \in [r]$, for any $j_1, j_2 \in g_{1,i}$, we have $W_{j_1,*} = W_{j_2,*}$.


Thus, we can have 
\begin{align*}
    & ~ \sum_{i=1}^n \| (U_1)_{i,*} Z_1 \diag( W_{i,*} ) - A_{i,*} \diag( W_{i,*} ) \|_2^2 \\
    = & ~ \sum_{i=1}^r  \sum_{\ell \in g_{1,i}} \| (U_1)_{\ell,*} Z_1 \diag( W_{\ell,*} ) - A_{\ell,*} \diag( W_{\ell,*} ) \|_2^2  .
\end{align*}
We can sketch the objective function by choosing Gaussian matrices $S_1 \in \R^{n \times s_1}$ with $s_1 = O(k/\epsilon)$.
\begin{align*}
\sum_{i=1}^n \| (U_1)_{i,*} Z_1 \diag( W_{i,*} ) S_1 - A_{i,*} \diag( W_{i,*} ) S_1 \|_2^2 .
\end{align*}
Let $\wh{U}_1$ denote the optimal solution of the sketch problem,

\begin{align*}
    & ~ \wh{U}_1 =  \arg\min_{U_1 \in \R^{n \times k}} \\
    & ~ \sum_{i=1}^n \| (U_1)_{i,*} Z_1 \diag( W_{i,*} ) S_1 - A_{i,*} \diag( W_{i,*} ) S_1 \|_2^2 .
\end{align*}
By properties of $S_1$, plugging $\wh{U}_1$ into the original problem, we obtain
\begin{align*}
& ~ \sum_{i=1}^r  \sum_{\ell \in g_{1,i}} \| (\wh{U}_1)_{\ell,*} Z_1 \diag( W_{\ell,*} ) - A_{\ell,*} \diag( W_{\ell,*} ) \|_2^2  \\
\leq & ~  (1+\epsilon) \cdot \OPT.
\end{align*}

Let $R$ denote the set of all $\S(g_{1,i})$ (for all $i \in [r]$ and $|R| = r$)
 

Note that $\wh{U}_1$ also has the following form, for each $\ell \in L \subset [n]$ (Note that $|L| = r p$.)
\begin{align*}
(\wh{U}_1)_{\ell, *} 
= & ~ A_{\ell,*} \diag(W_{\ell,*}) S_1 \cdot ( Z_1 \diag( W_{\ell,*} ) S_1 )^\dagger \\
= & ~  A_{\ell,*} \diag(W_{\ell,*}) S_1 \cdot ( Z_1 \diag( W_{\ell,*} ) S_1 )^\top \\
& ~ \cdot ( ( Z_1 \diag( W_{\ell,*} ) S_1 ) ( Z_1 \diag( W_{\ell,*} ) S_1 )^\top )^{-1}.
\end{align*}

Recall the number of different $ \diag(W_{\ell,*})$ is at most $r$.

For each $k \times s_1$ matrix $Z_1 \diag( W_{\ell,*}) S_1$, we create $k \times s_1$ variables to represent it. Thus, we create $r$ matrices,
\begin{align*}
\{ Z_1 D_{W_{i,*}} S_1 \}_{i \in R}.
\end{align*}
For simplicity, let $P_{1,i} \in \R^{k \times s_1}$ denote $Z_1 \diag(W_{i,*}) S_1$. Then we can rewrite $\wh{U}^i$ as follows
\begin{align*}
    \wh{U}_1^i = A_{i,*} \diag(W_{i,*}) S_1  \cdot P_{1,i}^\top (P_{1,i} P_{1,i}^\top )^{-1}.
\end{align*}
If $P_{1,i} P_{1,i}^\top \in \R^{k \times k}$ has rank-$k$, then we can use Cramer's rule to write down the inverse of $P_{1,i} P_{1,i}^\top$.
For the situation, it is not full rank. We can guess the rank. Let $t_i \leq k$ denote the rank of $P_{1,i}$. Then, we need to figure out a maximal linearly independent subset of columns of $P_{1,i}$. We can also guess all the possibilities, which is at most $2^{O(k)}$. Because we have $r$ different $P_{1,i}$, the total number of guesses we have is at most $2^{O(rk)}$. Thus, we can write down $(P_{1,i} P_{1,i}^\top)^{-1}$ according to Cramer's rule. Note that $(P_{1,i} P_{1,i}^\top)^{-1}$ can be view as $P_a/P_b$ where $P_a$ is a polynomial and $P_b$ is another polynomial which is essentially $\det(P_{1,i} P_{1,i}^\top)$.

 
After $\wh{U}_1$ is obtained, we fix $\wh{U}_1$. We consider 
\begin{align*}
    & ~ \wh{U}_2 =  \arg\min_{U_2 \in \R^{n \times k}}  \| (\wh{U}_1 U_2^\top - A) \circ W \|_F^2 .
\end{align*}
In a similar way, we can get and write $\wh{U}_2$. 

Overall, by creating $l = O(rk^2/\epsilon)$ variables, we have rational polynomials $\wh{U}_1(x)$ and $\wh{U}_2(x)$. Note that $\wh{U}_1(x)$ only has $rp$ different rows, and same for $\wh{U}_2(x)$.

 Indeed, now we have only $2 r$ distinct denominators (w.l.o.g., assume the first $r$ columns are distinct and the first $r$ rows are distinct),
\begin{align*}
h_{1,i}(x) & = \det ( P_{1,i} P_{1,i}^\top ), \forall i \in[r] \\
h_{2,i}(x) & = \det ( P_{2,i} P_{2,i}^\top ), \forall i \in[r].
\end{align*}



Then, we can write down the following optimization problem,
\begin{align*}
\min _{x \in \R^l} & ~ p(x) / q(x) \\
\mathrm{s.t. } & ~ h_{1,i}^2(x) \neq 0, h_{2,i}^2(x) \neq 0, \forall i \in[r], \\
& ~ q(x)=\prod_{i=1}^r h_{1,i}^2(x) h_{2,i}^2(x),
\end{align*}
where $q(x)$ has degree $O(r k)$, the maximum coefficient in absolute value is 
\begin{align*}
( 2^{n^{\gamma}} )^ {O (r k )}
\end{align*}
 and the number of variables $O (r k^2 / \epsilon )$. However, that formulation $p(x)/q(x)$ is not a polynomial, to further make it a polynomial, we introduce variable $y$:
 \begin{align*}
\min _{x \in \R^l} & ~ p(x) y \\
\mathrm{s.t. } & ~ h_{1,i}^2(x) \neq 0, h_{2,i}^2(x) \neq 0, \forall i \in[r], \\
& ~ q(x)=\prod_{i=1}^r h_{1,i}^2(x) h_{2,i}^2(x) \\
& ~ q(x)y - 1 =0.
\end{align*}

Applying Theorem~\ref{thm:jpt13}, with parameters
\begin{align*}
& \m= O(r), \v = O(rk^2/\epsilon),  \d = O(r) , \\
& \H = 2^{O( n^{\gamma} rk ) } ,  \wt{H} = O(\H) .
\end{align*}

we can achieve the following minimum nonzero cost: 
\begin{align*}
\geq  ~ 2^{-2^{3 \v \log (\d)} \log( \wt{\H}) } 
\geq  ~ 
2^{-n^\gamma 2^{\wt{O} (r k^2 / \epsilon )}  } .
\end{align*}
 Thus, we complete the proofs.
\end{proof}

\section{FEW DISTINCT COLUMNS}\label{sec:few_distinct_columns}

In this section, we try to estimate the $\OPT$ value. 

\begin{theorem}\label{thm:few_distinct_columns}
Given a matrix $A$ and $W$, each entry can be written using $O(n^{\gamma})$ bits for $\gamma >0$.
Given $A \in \R^{n \times n}, W \in \R^{n \times n}, 1 \leq k \leq n$.  Assume that $W$ has $r$ distinct columns and rows.
   
Then, with high probability,  
one can output a number $\Lambda$ in time $n^{1+\gamma} \cdot p \cdot 2^{O (k^2 r / \epsilon )} $ such that 
\begin{align*}
    \OPT \leq \Lambda \leq(1+\epsilon) \OPT .
\end{align*}
\end{theorem}
\begin{proof}

Let $U_1^*, U_2^* \in \R^{n \times k}$ denote the matrices satisfying 
\begin{align*}
    \| W \circ (U_1^* (U_2^*)^\top - A) \|_F^2 = \OPT.
\end{align*}


We use $A_{i,*} \in \R^n$ denote the $i$-th row of $A$. We use $W_{i,*} \in \R^n$ to denote the $i$-th row of $W$. Let $(U_1)_{i,*}$ denote the $i$-th row of $U_1$. Let $Z_1 = (U_2^*)^\top$.  
For any $n \times k$ matrix $U_1$,
\begin{align*}
 & ~ \| (U_1 Z_1 - A) \circ W \|_F^2 \\
 = & ~ \sum_{i=1}^n \| (U_1)_{i,*} Z_1 \diag( W_{i,*} ) - A_{i,*} \diag( W_{i,*} ) \|_2^2.
\end{align*}

Based on the observation that $W$ has $r$ distinct rows, we use group $g_{1,1}, g_{1,2}, \cdots , g_{1,r}$ to denote $r$ disjoint sets such that
\begin{align*}
    \cup_{i=1}^r g_{1,i} = [n]
\end{align*}
For any $i \in [r]$, for any $j_1, j_2 \in g_{1,i}$, we have $W_{j_1,*} = W_{j_2,*}$.

Next, based on assumptions on $W$ and $A$, we use $g_{1,i,1}$, $g_{1,i,2}$, $g_{1,i,p}$ to denote $p$ groups such that 
\begin{align*}
    \cup_{j=1}^p g_{1,i,j} = g_{1,i} .
\end{align*}
For any $i \in [r]$, for any $j \in [p]$, for any $\ell_1, \ell_2 \in g_{1,i,j}$, we have  $(W_{\ell_1,*} \circ A_{\ell_1,*}) = (W_{\ell_2,*} \circ A_{\ell_2,*})$.

Let $\S(g_{1,i,j})$ denote the smallest index from set $g_{1,i,j}$

Thus, we can have 
\begin{align*}
    & ~ \sum_{i=1}^n \| (U_1)_{i,*} Z_1 \diag( W_{i,*} ) - A_{i,*} \diag( W_{i,*} ) \|_2^2 \\
    = & ~ \sum_{i=1}^r \sum_{j =1}^p \sum_{\ell \in g_{1,i,j}} \| (U_1)_{\ell,*} Z_1 \diag( W_{\ell,*} ) \\
     & ~ \quad\quad\quad\quad\quad\quad - A_{\ell,*} \diag( W_{\ell,*} ) \|_2^2 \\
    = & ~\sum_{i=1}^r \sum_{j =1}^p |g_{1,i,j}| \cdot \| (U_1)_{\S(g_{1,i,j}),*} Z_1 \diag( W_{ \S(g_{1,i,j}),*} ) \\
    & ~ \quad\quad\quad\quad\quad\quad - A_{\S(g_{1,i,j}),*} \diag( W_{\S(g_{1,i,j}),*} ) \|_2^2 .
\end{align*}
We can sketch the objective function by choosing Gaussian matrices $S_1 \in \R^{n \times s_1}$ with $s_1 = O(k/\epsilon)$.
\begin{align*}
& ~ \sum_{i=1}^r \sum_{j =1}^p |g_{1,i,j}| \cdot \| (U_1)_{\S(g_{1,i,j}),*} Z_1 \diag( W_{ \S(g_{1,i,j}),*} ) S_1 \\
& ~ \quad\quad\quad\quad\quad\quad - A_{\S(g_{1,i,j}),*} \diag( W_{\S(g_{1,i,j}),*} ) S_1 \|_2^2.
\end{align*}
Let $\wh{U}_1$ denote the optimal solution of the sketch problem,
\begin{align*}
    \wh{U}_1 = & ~ \arg\min_{U_1} \sum_{i=1}^r \sum_{j =1}^p |g_{1,i,j}| \\
    & ~ \cdot \| (U_1)_{\S(g_{1,i,j}),*} Z_1 \diag( W_{ \S(g_{1,i,j}),*} ) S_1 \\
    & ~ \quad - A_{\S(g_{1,i,j}),*} \diag( W_{\S(g_{1,i,j}),*} ) S_1 \|_2^2.
\end{align*}
By properties of $S_1$, plugging $\wh{U}_1$ into the original problem, we obtain
\begin{align*}
& ~ \sum_{i=1}^r \sum_{j =1}^p |g_{1,i,j}| \cdot \| (U_1)_{\S(g_{1,i,j}),*} Z_1 \diag( W_{ \S(g_{1,i,j}),*} ) \\
& ~ \quad\quad - A_{\S(g_{1,i,j}),*} \diag( W_{\S(g_{1,i,j}),*} ) \|_2^2 \leq (1+\epsilon) \cdot \OPT.
\end{align*}

Let $R$ denote the set of all $\S(g_{1,i})$ (for all $i \in [r]$ and $|R| = r$).

Let $L$ denote the set of all $\S(g_{1,i,j})$ (for all $i \in [r]$, $j \in [p]$ and $|L| = rp$).

Note that $\wh{U}_1$ also has the following form, for each $\ell \in L \subset [n]$ (Note that $|L| = r p$.)
\begin{align*}
(U_1)_{\ell, *} 
= & ~ A_{\ell,*} \diag(W_{\ell,*}) S_1 \cdot ( Z_1 \diag( W_{\ell,*} ) S_1 )^\dagger \\
= & ~  A_{\ell,*} \diag(W_{\ell,*}) S_1 \cdot ( Z_1 \diag( W_{\ell,*} ) S_1 )^\top \\
& ~ \cdot ( ( Z_1 \diag( W_{\ell,*} ) S_1 ) ( Z_1 \diag( W_{\ell,*} ) S_1 )^\top )^{-1}.
\end{align*}

Recall the number of different $A_{\ell,*} \diag(W_{\ell,*})$ is at most $rp$, and the number of different $ \diag(W_{\ell,*})$ is at most $r$. For each $k \times s_1$ matrix $Z_1 \diag( W_{\ell,*}) S_1$, we create $k \times s_1$ variables to represent it. Thus, we create $r$ matrices,
\begin{align*}
\{ Z_1 \diag( W_{i,*} ) S_1 \}_{i \in R}. 
\end{align*}
For simplicity, let $P_{1,i} \in \R^{k \times s_1}$ denote $Z_1 \diag(W_{i,*}) S_1$. Then we can rewrite $(\wh{U}_1)_{i,*}$ as follows
\begin{align*}
    (\wh{U}_1)_{i,*} = A_{i,*} \diag(W_{i,*}) S_1  \cdot P_{1,i}^\top (P_{1,i} P_{1,i}^\top )^{-1}.
\end{align*}
If $P_{1,i} P_{1,i}^\top \in \R^{k \times k}$ has rank-$k$, then we can use Cramer's rule to write down the inverse of $P_{1,i} P_{1,i}^\top$.
For the situation, it is not full rank. We can guess the rank. Let $t_i \leq k$ denote the rank of $P_{1,i}$. Then, we need to figure out a maximal linearly independent subset of columns of $P_{1,i}$. We can also guess all the possibilities, which is at most $2^{O(k)}$. Because we have $r$ different $P_{1,i}$, the total number of guesses we have is at most $2^{O(rk)}$. Thus, we can write down $(P_{1,i} P_{1,i}^\top)^{-1}$ according to Cramer's rule. Note that $(P_{1,i} P_{1,i}^\top)^{-1}$ can be view as $P_a/P_b$ where $P_a$ is a polynomial and $P_b$ is another polynomial which is essentially $\det(P_{1,i} P_{1,i})$.


After $\wh{U}_1$ is obtained, we will fix $\wh{U}_1$ in the next round.

For any $n \times k$ matrix $U_2$, we can rewrite 
\begin{align*}
 & ~ \| (\wh{U}_1 U_2^\top  - A ) \circ \|_F^2 \\
 = & ~ \sum_{i=1}^n \| \diag(  W_{*,i} ) \wh{U}_1   (U_2^\top)_{*,i}   - \diag(W_{*,i}) A_{*,i} \|_2^2 .
\end{align*}



Based on the observation that $W$ has $r$ columns rows, we use group $g_{2,1}, g_{2,2}, \cdots , g_{2,r}$ to denote $r$ disjoint sets such that
\begin{align*}
    \cup_{i=1}^r g_{2,i} = [n].
\end{align*}
For any $i \in [r]$, for any $j_1, j_2 \in g_{2,i}$, we have $W_{*,j_1} = W_{*,j_2}$.

Next, based on assumptions on $W$ and $A$, we use $g_{2,i,1}$, $g_{2,i,2}$, $g_{2,i,p}$ to denote $p$ groups such that 
\begin{align*}
    \cup_{j=1}^p g_{2,i,j} = g_{2,i}.
\end{align*}
For any $i \in [r]$, for any $j \in [p]$, for any $\ell_1, \ell_2 \in g_{2,i,j}$, we have  $(W_{*,\ell_1} \circ A_{*,\ell_1}) = (W_{*,\ell_2} \circ A_{*,\ell_2})$.

Let $\S(g_{2,i,j})$ denote the smallest index from set $g_{2,i,j}$

Thus, we can have 
\begin{align*}
    & ~ \sum_{i=1}^n \| \diag( W_{*,i} ) \wh{U}_1   (U_2^\top )_{*,i}  -  \diag( W_{*,i} ) A_{*,i} \|_2^2 \\
    = & ~ \sum_{i=1}^r \sum_{j =1}^p \sum_{\ell \in g_{1,i,j}} \|  \diag( W_{*,\ell} ) \wh{U}_1 (U_2^\top )_{*,\ell} \\
    & ~ \quad\quad\quad\quad\quad\quad - \diag( W_{*,\ell} ) A_{*,\ell}  \|_2^2 \\
    = & ~\sum_{i=1}^r \sum_{j =1}^p |g_{2,i,j}| \cdot \|  \diag( W_{ *, \S(g_{2,i,j}) } ) \wh{U}_1 (U_2^\top )_{*, \S(g_{2,i,j})} \\
    & ~ \quad\quad\quad\quad\quad\quad -  \diag( W_{*, \S(g_{1,i,j})} ) A_{*,\S(g_{1,i,j})} \|_2^2 .
\end{align*}
We can sketch the objective function by choosing Gaussian matrices $S_2 \in \R^{n \times s_1}$ with $s_2 = O(k/\epsilon)$.
\begin{align*}
& ~ \sum_{i=1}^r \sum_{j =1}^p |g_{2,i,j}| \cdot \| S_2  \diag( W_{ \S(g_{2,i,j}),*} ) \wh{U}_1  (U_2^\top )_{*, \S(g_{2,i,j}) } \\
& ~ \quad\quad\quad\quad - S_2 \diag( W_{*, \S(g_{2,i,j})} )  A_{*, \S(g_{2,i,j}) }  \|_2^2.
\end{align*}
Let $\wh{U}_1$ denote the optimal solution of the sketch problem,
\begin{align*}
    \wh{U}_1 = & ~ \arg\min_{U_1} \sum_{i=1}^r \sum_{j =1}^p |g_{2,i,j}| \\
    & ~  \cdot \| S_2  \diag( W_{ \S(g_{2,i,j}),*} ) \wh{U}_1  (U_2^\top )_{*, \S(g_{2,i,j}) } \\
    & ~ - S_2 \diag( W_{*, \S(g_{2,i,j})} )  A_{*, \S(g_{2,i,j}) }  \|_2^2.
\end{align*}
By properties of $S_1$, plugging $\wh{U}_2$ into the original problem, we obtain
\begin{align*}
& ~ \sum_{i=1}^r \sum_{j =1}^p |g_{2,i,j}| \cdot \|  \diag( W_{ \S(g_{2,i,j}),*} ) \wh{U}_1  ( \wh{U}_2^\top )_{*, \S(g_{2,i,j}) } \\
& ~ -  \diag( W_{*, \S(g_{2,i,j})} )  A_{*, \S(g_{2,i,j}) }  \|_2^2\leq (1+\epsilon) \cdot \OPT.
\end{align*}

Let $R$ denote the set of all $\S(g_{1,i})$ (for all $i \in [r]$ and $|R| = r$).

Let $L$ denote the set of all $\S(g_{1,i,j})$ (for all $i \in [r]$, $j \in [p]$ and $|L| = rp$).

Note that $\wh{U}_1$ also has the following form, for each $\ell \in L \subset [n]$ (Note that $|L| = r p$.)
\begin{align*}
(U_1)_{\ell, *} 
= & ~  ( S_2 \diag( W_{*,\ell} ) \wh{U}_1 )^\dagger \cdot S_2 \diag(W_{*,\ell})  A_{*,\ell} \\
= & ~ (  ( S_2 \diag( W_{*,\ell} ) \wh{U}_1 ) ( S_2 \diag( W_{*,\ell} ) \wh{U}_1)^\top )^{-1} \\
& ~ \cdot S_2 \diag( W_{*,\ell} ) \wh{U}_1 \cdot S_2 \diag(W_{*,\ell})  A_{*,\ell}.
\end{align*}

Recall the number of different $A_{*,\ell} \diag(W_{*,\ell})$ is at most $rp$, and the number of different $ \diag(W_{*,\ell})$ is at most $r$.

For each $s_2 \times k$ matrix $S_2 \diag( W_{*,i} ) \wh{U}_1$, we create $s_2 \times k$ variables to represent it. Thus, we create $r$ matrices,
\begin{align*}
\{ S_2 \diag( W_{*,i} ) \wh{U}_1 \}_{i \in R}.
\end{align*}
For simplicity, let $P_{2,i} \in \R^{k \times s_2}$ denote $S_2 \diag( W_{*,i} ) \wh{U}_1$. Then we can rewrite $\wh{U}_2$ as follows
\begin{align*}
    (\wh{U}_2^\top)_{*,i} = (P_{2,i} P_{2,i}^\top)^{-1}  P_{2,i} S_2 \diag(W_{*,i})  A_{*,i}.
\end{align*}

In a similar way, we can write $\wh{U}_2$. Overall, by creating $l = O(rk^2/\epsilon)$ variables, we have rational polynomials $\wh{U}_1(x)$ and $\wh{U}_2(x)$. Note that $\wh{U}_1(x)$ only has $rp$ different rows, and same for $\wh{U}_2(x)$.

Putting it all together, we can write the objective function,
\begin{align*}
\min_{x \in \R^l} & ~ \| ( \wh{U}_1(x) \wh{U}_2(x)^\top - A ) \circ W \|_F^2 \\
\mathrm{~s.t.~}& ~ h_{1,i}(x) \neq 0 ,  \forall i \in [r] \\
& ~ h_{2,i}(x) \neq 0, \forall i \in [r].
\end{align*}

Note that $\wh{U}_1(x) \wh{U}_{2}(x) \circ W$ only has $rp$ distinct rows. Also, $A \circ W$ only has $rp$ distinct rows. Writing down the objective function $\| ( \wh{U}_1(x) \wh{U}_2(x)^\top - A ) \circ W \|_F^2$ only requires $n (rp) \cdot \poly(kr/\epsilon)$ time.

Combining the binary search explained in Lemma~\ref{lem:warmup} 
with the lower bound on cost (Theorem~\ref{thm:lower_bound_on_cost}) we obtained, we can find the solution for the original problem in time,
\begin{align*}
(np \cdot \poly(kr/\epsilon) + n 2^{\wt{O} (rk^2 / \epsilon) }  ) \cdot n^{\gamma} = n^{1+\gamma} p \cdot 2^{\wt{O}(rk^2/\epsilon)}.
\end{align*}

\end{proof}
\section{RECOVER A SOLUTION}\label{sec:recover_solution}

We state our results and proof of recovering a solution.
\begin{theorem}\label{thm:recover_solution}
Given a matrix $A$ and $W$, each entry can be written using $O(n^{\gamma})$ bits for $\gamma >0$.
Given $A \in \R^{n \times n}, W \in \R^{n \times n}, 1 \leq k \leq n$ and $ \epsilon \in (0,0.1)$. 
Assume $W$ has $r$ distinct columns and rows. 
Assume $A \circ W$ has at most $r \cdot p$ distinct columns and at most $r \cdot p$ distinct rows. 
Let $\OPT$ be defined as Definition~\ref{def:opt}. 
Then, with high probability,  
one can output two matrices $U,V \in \R^{n \times k}$ in time $n^{1+\gamma} \cdot 2^{O (k^2 r / \epsilon )} $ such that 
\begin{align*}
    \| (UV^\top- A) \circ W \|_F^2 \leq(1+\epsilon) \OPT .
\end{align*}

Further, if we choose (1) $k^2 r = O(\log n /\log\log n)$, (2) $\epsilon \in (0,0.1)$ to be a small constant, (3) $p = n^{o(1)}$ and (4) $\gamma = o(1)$, then the running time becomes $n^{1+o(1)}$.
\end{theorem}


\begin{proof}

Here, we show how to recover an approximate solution, not only the value of $\OPT$.

The idea is to recover the entries of $U$ and $V$ one by one and use the algorithm from the previous section for the corresponding decision problem. We initialize the semialgebraic set to be
\begin{align*}
    S= \{x \in \R^l \mid q(x) \neq 0, p(x) \leq \Lambda q(x) \}.
\end{align*}
We start by recovering the first entry of $U$. We perform the binary search to localize the entry, which takes $ \log  ( 2^{n^{\gamma}} )$ invocations of the decision algorithm. For each step of binary search, we use Theorem~\ref{thm:jpt13} to determine whether the following semi-algebraic set $S$ is empty or not,
\begin{align*}
S \cap \{U_{1,1}(x) \geq \wh{U}_{1,1}^{-}, U_{1,1}(x) \leq \wh{U}_{1,1}^{+} \}.
\end{align*}
After that, we declare the first entry of $U$ to be any point in this interval.  Then, we add an equality constraint that fixes the entry of $\wh{U}$ to this value and add a new constraint into $S$ permanently, e.g., $S  \leftarrow S \cap \{U_{1,1}(x)=\wh{U}_{1,1} \}$. Next, we repeat the same with the second entry of $U$ and so on.

This allows us to recover a solution of cost at most $(1+ \epsilon) \OPT$ in time
\begin{align*}
n^{1+\gamma} \cdot p \cdot 2^{O (k^{2} r / \epsilon )} .
\end{align*}
If we choose $\gamma = o(1)$, $\epsilon = \Theta(1)$, $p =n^{o(1)}$ and $k^2 r = O(\log n / \log\log n )$, then the running time becomes $n^{1+o(1)}$.
Thus, we complete the proof.
\end{proof}


\section{Conclusion \& Future Work}\label{conclusion}
This work presents XAMBA, the first framework optimizing SSMs on COTS NPUs, removing the need for specialized accelerators. XAMBA mitigates key bottlenecks in SSMs like CumSum, ReduceSum, and activations using ActiBA, CumBA, and ReduBA, transforming sequential operations into parallel computations. These optimizations improve latency, throughput (Tokens/s), and memory efficiency. Future work will extend XAMBA to other models, explore compression, and develop dynamic optimizations for broader hardware platforms.



% This work introduces XAMBA, the first framework to optimize SSMs on COTS NPUs, eliminating the need for specialized hardware accelerators. XAMBA addresses key bottlenecks in SSM execution, including CumSum, ReduceSum, and activation functions, through techniques like ActiBA, CumBA, and ReduBA, which restructure sequential operations into parallel matrix computations. These optimizations reduce latency, enhance throughput, and improve memory efficiency. 
% Experimental results show up to 2.6$\times$ performance improvement on Intel\textregistered\ Core\texttrademark\ Ultra Series 2 AI PC. 
% Future work will extend XAMBA to other models, incorporate compression techniques, and explore dynamic optimization strategies for broader hardware platforms.


% This work presents XAMBA, an optimization framework that enhances the performance of SSMs on NPUs. Unlike transformers, SSMs rely on structured state transitions and implicit recurrence, which introduce sequential dependencies that challenge efficient hardware execution. XAMBA addresses these inefficiencies by introducing CumBA, ReduBA, and ActiBA, which optimize cumulative summation, ReduceSum, and activation functions, respectively, significantly reducing latency and improving throughput. By restructuring sequential computations into parallelizable matrix operations and leveraging specialized hardware acceleration, XAMBA enables efficient execution of SSMs on NPUs. Future work will extend XAMBA to other state-space models, integrate advanced compression techniques like pruning and quantization, and explore dynamic optimization strategies to further enhance performance across various hardware platforms and frameworks.
% This work presents XAMBA, an optimization framework that enhances the performance of SSMs on NPUs. Key techniques, including CumBA, ReduBA, and ActiBA, achieve significant latency reductions by optimizing operations like cumulative summation, ReduceSum, and activation functions. Future work will focus on extending XAMBA to other state-space models, integrating advanced compression techniques, and exploring dynamic optimization strategies to further improve performance across various hardware platforms and frameworks.

% This work introduces XAMBA, an optimization framework for improving the performance of Mamba-2 and Mamba models on NPUs. XAMBA includes three key techniques: CumBA, ReduBA, and ActiBA. CumBA reduces latency by transforming cumulative summation operations into matrix multiplication using precomputed masks. ReduBA optimizes the ReduceSum operation through matrix-vector multiplication, reducing execution time. ActiBA accelerates activation functions like Swish and Softplus by mapping them to specialized hardware during the DPU’s drain phase, avoiding sequential execution bottlenecks. Additionally, XAMBA enhances memory efficiency by reducing SRAM access, increasing data reuse, and utilizing Zero Value Compression (ZVC) for masks. The framework provides significant latency reductions, with CumBA, ReduBA, and ActiBA achieving up to 1.8X, 1.1X, and 2.6X reductions, respectively, compared to the baseline.
% Future work includes extending XAMBA to other state-space models (SSMs) and exploring further hardware optimizations for emerging NPUs. Additionally, integrating advanced compression techniques like pruning and quantization, and developing adaptive strategies for dynamic optimization, could enhance performance. Expanding XAMBA's compatibility with other frameworks and deployment environments will ensure broader adoption across various hardware platforms.

\ifdefined\isarxiv
% \section*{Acknowledgement}
% Research is partially supported by the National Science Foundation (NSF) Grants 2023239-DMS, CCF-2046710, and Air Force Grant FA9550-18-1-0166.
\else
\bibliography{ref}
\bibliographystyle{plainnat}
\input{20_checklist}
\fi



% \newpage
% \onecolumn
% \appendix


% \ifdefined\isarxiv
% \begin{center}
% 	\textbf{\LARGE Appendix }
% \end{center}
% \else
% \aistatstitle{When Can We Solve the Weighted Low Rank Approximation Problem in Truly Subquadratic Time?: \\
% Supplementary Materials}
% {\hypersetup{linkcolor=black}
% %\tableofcontents
% \bigbreak
% \bigbreak
% \bigbreak
% }
% \fi
% \input{app_preli}
% \input{app_recover_solution}




\ifdefined\isarxiv
%\section*{Acknowledgments}
\bibliographystyle{alpha}
\bibliography{ref}

\else




\fi

%%%% Cut-line between first 10 pages and appendix







%%% some writing rules

%% Writing rule for creating tags.
%% Tags :
%% Theorem    \ref{thm:bla_bla}
%% Lemma      \ref{lem:bla_bla}
%% Claim      \ref{cla:bla_bla}
%% Corollary  \ref{cor:bla_bla}
%% Fact       \ref{fac:bla_bla}
%% Definition \ref{def:bla_bla}
%% Section    \ref{sec:bla_bla}
%% Subsection \ref{sub:bla_bla}
%% Equation   \ref{eq:bla_bla}



\end{document}



%%%%%%%%%%%%%%%%%%%%%%%%%%%%%%%%%%%%%%%%%%%%%%%%%%%%%%%%%%%%%%%%%%%%%%%%%%%%%%%%%%%%%%%%%%%%%%%%%%%%%%%%%%%%%%%%%%%%%%%%%%%%%%%%%%%%%%%%%%%%%%%%%%%%%%%%%%%%%%%%%%%%%%%%%%%%%%%%%%%%%%%%%%%%%%%%%%%%%%%%%%%%%%%%%%%%%%%%%%%%%%%%%%%%%%%%%%%%%%%%%%%%%%%%%%%%%%%%%%%%%%%%%%%%%%%%%%%%%%%%%%%%%%%%%%%%%%%%%%%%%%%%%%%%%%%%%%%%%%%%%%%%%%%%%%%%%%%%%%%%%%%%%%%%%%%%%%%%%%%%%%%%%%%%%%%%%%%%%%%%%%%%%%%%%%%%%%%%%%%%%%%%%%%%%%%%%%%%%%%%%%%%%%%%%%%%%%%%%%%%%%%%%%%%%%%%%%%%%%%%%%
