\def\isarxiv{1} %%% for icml submission version, we comment this line

\ifdefined\isarxiv
\documentclass[11pt]{article}

\usepackage[numbers]{natbib}

\else
\documentclass[twoside]{article}

% \usepackage{aistats2025}
% If your paper is accepted, change the options for the package
% aistats2025 as follows:
%
\usepackage[accepted]{aistats2025}
%
% This option will print headings for the title of your paper and
% headings for the authors names, plus a copyright note at the end of
% the first column of the first page.

% If you set papersize explicitly, activate the following three lines:
%\special{papersize = 8.5in, 11in}
%\setlength{\pdfpageheight}{11in}
%\setlength{\pdfpagewidth}{8.5in}

% If you use natbib package, activate the following three lines:
\usepackage[round]{natbib}
\renewcommand{\bibname}{References}
\renewcommand{\bibsection}{\subsubsection*{\bibname}}

% If you use BibTeX in apalike style, activate the following line:
%\bibliographystyle{apalike}

\fi


\usepackage{amsmath}
\usepackage{amsthm}
\usepackage{amssymb}
\usepackage{algorithm}
\usepackage{subfig}
\usepackage{algpseudocode}
\usepackage{graphicx}
\usepackage{grffile}
\usepackage{wrapfig,epsfig}
\usepackage{url}
\usepackage{xcolor}
\usepackage{epstopdf}


\usepackage{bbm}
\usepackage{dsfont}

 %%% print refs in table of contents
%\displaybreak
\allowdisplaybreaks

%\usepackage[lmargin=1in,rmargin=1in,tmargin=0.8in,bmargin=0.8in]{geometry}

\ifdefined\isarxiv

\let\C\relax
\usepackage{tikz}
\usepackage{hyperref}  %%% arxiv don't allow this.
\hypersetup{colorlinks=true,citecolor=blue,linkcolor=blue} %%% Zhao : maybe we should comment this in submission.
\usetikzlibrary{arrows}
\usepackage[margin=1in]{geometry}

\else

% \usepackage[utf8]{inputenc} % allow utf-8 input
% \usepackage[T1]{fontenc}    % use 8-bit T1 fonts

% \usepackage{booktabs}       % professional-quality tables
% \usepackage{amsfonts}       % blackboard math symbols
% \usepackage{nicefrac}       % compact symbols for 1/2, etc.
% \usepackage{microtype}      % microtypography
\usepackage{hyperref}       % hyperlinks
\definecolor{mydarkblue}{rgb}{0,0.08,0.45}
\hypersetup{colorlinks=true, citecolor=mydarkblue,linkcolor=mydarkblue}
%\usepackage[capitalize,noabbrev]{cleveref}
%\usepackage{colortbl}

\fi
%\linespread{1}
%\newcommand{\QED}{\hfill$\qed$}
%\graphicspath{{./figs/}}

\theoremstyle{plain}
\newtheorem{theorem}{Theorem}[section]
\newtheorem{lemma}[theorem]{Lemma}
\newtheorem{definition}[theorem]{Definition}
\newtheorem{notation}[theorem]{Notation}
%\newtheorem{proof}[theorem]{Proof}
\newtheorem{proposition}[theorem]{Proposition}
\newtheorem{corollary}[theorem]{Corollary}
\newtheorem{conjecture}[theorem]{Conjecture}
\newtheorem{assumption}[theorem]{Assumption}
\newtheorem{observation}[theorem]{Observation}
\newtheorem{fact}[theorem]{Fact}
\newtheorem{remark}[theorem]{Remark}
\newtheorem{claim}[theorem]{Claim}
\newtheorem{example}[theorem]{Example}
\newtheorem{problem}[theorem]{Problem}
\newtheorem{open}[theorem]{Open Problem}
\newtheorem{property}[theorem]{Property}
\newtheorem{hypothesis}[theorem]{Hypothesis}

\newcommand{\wh}{\widehat}
\newcommand{\wt}{\widetilde}
\newcommand{\ov}{\overline}
\newcommand{\N}{\mathcal{N}}
\newcommand{\R}{\mathbb{R}}
\newcommand{\RHS}{\mathrm{RHS}}
\newcommand{\LHS}{\mathrm{LHS}}
%\renewcommand{\d}{\mathrm{d}}
\renewcommand{\i}{\mathbf{i}}
\renewcommand{\tilde}{\wt}
\renewcommand{\hat}{\wh}
\newcommand{\Tmat}{{\cal T}_{\mathrm{mat}}}
\renewcommand{\S}{\mathsf{S}}


%%%Zhao: Below comments are only novel for this paper.
\renewcommand{\v}{\mathsf{v}}
\newcommand{\m}{\mathsf{m}}
\renewcommand{\d}{\mathsf{d}}
\renewcommand{\H}{\mathsf{H}}
%%%Zhao: Above comments are only novel for this paper.

\DeclareMathOperator*{\E}{{\mathbb{E}}}
\DeclareMathOperator*{\var}{\mathrm{Var}}
\DeclareMathOperator*{\Z}{\mathbb{Z}}
\DeclareMathOperator*{\C}{\mathbb{C}}
\DeclareMathOperator*{\D}{\mathcal{D}}
\DeclareMathOperator*{\median}{median}
\DeclareMathOperator*{\mean}{mean}
\DeclareMathOperator{\OPT}{OPT}
\DeclareMathOperator{\supp}{supp}
\DeclareMathOperator{\poly}{poly}

\DeclareMathOperator{\nnz}{nnz}
\DeclareMathOperator{\sparsity}{sparsity}
\DeclareMathOperator{\rank}{rank}
\DeclareMathOperator{\diag}{diag}
\DeclareMathOperator{\dist}{dist}
\DeclareMathOperator{\cost}{cost}
\DeclareMathOperator{\vect}{vec}
\DeclareMathOperator{\tr}{tr}
\DeclareMathOperator{\dis}{dis}
\DeclareMathOperator{\cts}{cts}



\makeatletter
\newcommand*{\RN}[1]{\expandafter\@slowromancap\romannumeral #1@}
\makeatother
% \newcommand{\Zhao}[1]{{\color{red}[Zhao: #1]}}
% \newcommand{\Zhenmei}[1]{{\color{purple}[Zhenmei: #1]}}  %%%Change to intern name
% \newcommand{\Chenyang}[1]{{\color{blue}[Chenyang: #1]}} %%%Change to intern name



\usepackage{lineno}
\def\linenumberfont{\normalfont\small}





\begin{document}

\ifdefined\isarxiv

\date{}


\title{When Can We Solve the Weighted Low Rank Approximation Problem in Truly Subquadratic Time?}
\author{ 
Chenyang Li\thanks{\texttt{
lchenyang550@gmail.com}. Fuzhou University.}
\and
Yingyu Liang\thanks{\texttt{
yingyul@hku.hk}. The University of Hong Kong. \texttt{
yliang@cs.wisc.edu}. University of Wisconsin-Madison.} 
\and
Zhenmei Shi\thanks{\texttt{
zhmeishi@cs.wisc.edu}. University of Wisconsin-Madison.}
\and 
Zhao Song\thanks{\texttt{ magic.linuxkde@gmail.com}. The Simons Institute for the Theory of Computing at the UC, Berkeley.}
}

\else

% If your paper is accepted and the title of your paper is very long,
% the style will print as headings an error message. Use the following
% command to supply a shorter title of your paper so that it can be
% used as headings.
%
\runningtitle{When Can We Solve the Weighted Low Rank Approximation Problem in Truly Subquadratic Time?}

% If your paper is accepted and the number of authors is large, the
% style will print as headings an error message. Use the following
% command to supply a shorter version of the authors names so that
% they can be used as headings (for example, use only the surnames)
%
%\runningauthor{Surname 1, Surname 2, Surname 3, ...., Surname n}

\twocolumn[

\aistatstitle{When Can We Solve the Weighted Low Rank Approximation Problem in Truly Subquadratic Time?}

\aistatsauthor{ 
Chenyang Li$^{1}$
% \thanks{{\tt \small yingyul@hku.hk}. {\tt\small yliang@cs.wisc.edu}.}
\And 
Yingyu Liang$^{2,3}$
% \thanks{{\tt \small yingyul@hku.hk}. {\tt\small yliang@cs.wisc.edu}.} 
\And  
Zhenmei Shi$^{3}$
% \thanks{{\tt \small zhmeishi@cs.wisc.edu}.}
\And 
Zhao Song$^{4}$
% \thanks{{\tt \small magic.linuxkde@gmail.com}.}
}


\aistatsaddress{ 
$^1$Fuzhou University. \qquad
$^2$The University of Hong Kong. \qquad
$^3$University of Wisconsin-Madison. 
\qquad
\\
$^4$The Simons Institute for the Theory of Computing at the University of California, Berkeley. 
} 
]

% $^4$Adobe Research, USA. 
%   \qquad 
%   $^2$University of Wisconsin-Madison, USA. 
%   \\
%   $^\varheart$The University of Hong Kong, HongKong. 
%   \qquad
%   $^1$Tsinghua University, China. 
%   \\
%   $^3$The Simons Institute for the Theory of Computing at the University of California, Berkeley, USA. 

\fi





\ifdefined\isarxiv
\begin{titlepage}
  \maketitle
  \begin{abstract}
\begin{abstract}

% Recent works to jointly reconstruct 3D human and object from a single RGB image, are mostly model-based, that fail to capture the fine details of the clothed human body and object surface. In this paper, we introduce ReCHOR, a novel, model-free, first-method to produce realistic clothed human-object reconstructions from a monocular view. This is extremely challenging due to human-object occlusions, diverse interactions and depth ambiguity, as it needs to infer both 3D spatial awareness and high resolution details. Our core idea is based on estimating neural implicit representations for human and object respectively by an attention-based neural implicit model that attends to pixel-aligned features from both the global human-object image for spatial awareness and  the local separate view of human and object images for high quality details. Additionally, the network is conditioned on semantic features from an initial estimated human-object pose prior and a generative diffusion model that inpaints occluded regions, thus enabling the retrieval of details from them.
% We also propose a synthetic dataset with rendered scenes of diverse, inter-occluded 3D human and object scans, to train our network. We evaluate our method on the synthetic and real world BEHAVE dataset. Our experiments show that our method outperforms the SOTA in achieving realistic clothed human-object reconstructions.
Recent approaches to jointly reconstruct 3D humans and objects from a single RGB image represent 3D shapes with template-based or coarse models, which fail to capture details of loose clothing on human bodies. In this paper, we introduce a novel implicit approach for jointly reconstructing realistic 3D clothed humans and objects from a monocular view. For the first time, we model both the human and the object with an implicit representation, allowing to capture more realistic details such as clothing. This task is extremely challenging due to human-object occlusions and the lack of 3D information in 2D images, often leading to poor detail reconstruction and depth ambiguity. To address these problems, we propose a novel attention-based neural implicit model that leverages image pixel alignment from both the input human-object image for a global understanding of the human-object scene and from local separate views of the human and object images to improve realism with, for example, clothing details. Additionally, the network is conditioned on semantic features derived from an estimated human-object pose prior, which provides 3D spatial information about the shared space of humans and objects. To handle human occlusion caused by objects, we use a generative diffusion model that inpaints the occluded regions, recovering otherwise lost details. For training and evaluation, we introduce a synthetic dataset featuring rendered scenes of inter-occluded 3D human scans and diverse objects. Extensive evaluation on both synthetic and real-world datasets demonstrates the superior quality of the proposed human-object reconstructions over competitive methods.
\end{abstract}

  \end{abstract}
  \thispagestyle{empty}
\end{titlepage}

{\hypersetup{linkcolor=black}
%\tableofcontents
}
\newpage

\else

\begin{abstract}
\begin{abstract}

% Recent works to jointly reconstruct 3D human and object from a single RGB image, are mostly model-based, that fail to capture the fine details of the clothed human body and object surface. In this paper, we introduce ReCHOR, a novel, model-free, first-method to produce realistic clothed human-object reconstructions from a monocular view. This is extremely challenging due to human-object occlusions, diverse interactions and depth ambiguity, as it needs to infer both 3D spatial awareness and high resolution details. Our core idea is based on estimating neural implicit representations for human and object respectively by an attention-based neural implicit model that attends to pixel-aligned features from both the global human-object image for spatial awareness and  the local separate view of human and object images for high quality details. Additionally, the network is conditioned on semantic features from an initial estimated human-object pose prior and a generative diffusion model that inpaints occluded regions, thus enabling the retrieval of details from them.
% We also propose a synthetic dataset with rendered scenes of diverse, inter-occluded 3D human and object scans, to train our network. We evaluate our method on the synthetic and real world BEHAVE dataset. Our experiments show that our method outperforms the SOTA in achieving realistic clothed human-object reconstructions.
Recent approaches to jointly reconstruct 3D humans and objects from a single RGB image represent 3D shapes with template-based or coarse models, which fail to capture details of loose clothing on human bodies. In this paper, we introduce a novel implicit approach for jointly reconstructing realistic 3D clothed humans and objects from a monocular view. For the first time, we model both the human and the object with an implicit representation, allowing to capture more realistic details such as clothing. This task is extremely challenging due to human-object occlusions and the lack of 3D information in 2D images, often leading to poor detail reconstruction and depth ambiguity. To address these problems, we propose a novel attention-based neural implicit model that leverages image pixel alignment from both the input human-object image for a global understanding of the human-object scene and from local separate views of the human and object images to improve realism with, for example, clothing details. Additionally, the network is conditioned on semantic features derived from an estimated human-object pose prior, which provides 3D spatial information about the shared space of humans and objects. To handle human occlusion caused by objects, we use a generative diffusion model that inpaints the occluded regions, recovering otherwise lost details. For training and evaluation, we introduce a synthetic dataset featuring rendered scenes of inter-occluded 3D human scans and diverse objects. Extensive evaluation on both synthetic and real-world datasets demonstrates the superior quality of the proposed human-object reconstructions over competitive methods.
\end{abstract}
\end{abstract}

\fi


\section{Introduction}
\label{sec:intro}
% Image editing methods in diffusion models depend on user-defined control directions - users can unlock their creativity using these methods by specifying the desired manipulation through prompts~\cite{gandikota2023concept}, reference images~\cite{ruiz2022dreambooth, kumari2022customdiffusion, gal2022image, chen2024trainingfreeregionalpromptingdiffusion}, or attribute vectors~\cite{parmar2023zero,hertz2022prompt}. In this work, we ask a fundamentally different question: \emph{Can we automatically discover the underlying visual structure of a concept within diffusion model's knowledge?} %Rather than requiring user-specified controls, we aim to decompose the model's internal knowledge into meaningful directions.

% This question touches on a fundamental limitation in how we interact with diffusion models. Current control methods ~\cite{zhang2023addingconditionalcontroltexttoimage, gandikota2023concept, ye2023ipadaptertextcompatibleimage,ye2023ipadaptertextcompatibleimage, hertz2024stylealignedimagegeneration, li2023photomaker, shi2024instantbooth, chen2024trainingfreeregionalpromptingdiffusion} require users to specify their desired manipulations in advance, limiting interactive creativity. This contrasts with natural human artistic workflows, where creators dynamically explore creative ideas while jointly refining them toward meaningful artistic outcomes~\cite{hoffmann2016modeling}. This synergy between specification and exploration is not new to generative models. Early GAN architectures naturally developed disentangled latent spaces that enabled continuous\cite{harkonen2020ganspace,radford2015unsupervised, wu2021stylespace, shen2020interfacegan}, compositional control over generated images. Users could explore these spaces to discover interesting variations that would be difficult to describe in words~\cite{wu2021stylespace}, then combine them to achieve their creative goals~\cite{grabe2022towards}. 


% While diffusion models have largely superseded GANs in conditional image synthesis~\cite{dhariwal2021diffusion},  their underlying structure remains less understood. Diffusion models achieve remarkable diversity through high-dimensional latents, unlike GANs' compact latent spaces.  With a single prompt, diffusion models can generate radically different variations through different random initializations of input noise. We ask - Is it possible to discover interpretable structure within this vast space of variations?

Text-to-image diffusion models are capable of generating remarkable visual variations from a single prompt through different random initializations. However, this vast creative potential remains largely opaque to users---while we can generate diverse images, we lack understanding of the underlying structure of these variations. This presents a fundamental challenge: how can we discover and expose the latent visual capabilities encoded within these models?

\let\thefootnote\relax \footnote{$^{*}$Correspondence to \texttt{gandikota.ro@northeastern.edu}}

The challenge touches on a key limitation in how we interact with diffusion models today. Current control methods require users to explicitly specify their desired edits in advance through prompts~\cite{gandikota2023concept}, reference images~\cite{zhang2023addingconditionalcontroltexttoimage, chen2024trainingfreeregionalpromptingdiffusion, ruiz2022dreambooth,kumari2022customdiffusion, Ryu_lora, hu2021lora}, or attribute vectors~\cite{ye2023ipadaptertextcompatibleimage, hertz2024stylealignedimagegeneration, li2023photomaker, shi2024instantbooth,parmar2023zero,hertz2022prompt}. That contrasts sharply with natural human creative workflows, where artists dynamically explore creative ideas and jointly refine them toward meaningful artistic outcomes~\cite{hoffmann2016modeling}. The need for pre-specified controls creates a barrier between users and the full creative potential of these models.

Interestingly, earlier generative models like GANs~\cite{gans,karras2019style,brock2018large} naturally developed more interpretable internal structures. Their compact latent spaces often exhibited emergent disentanglement~\cite{harkonen2020ganspace,radford2015unsupervised, wu2021stylespace, shen2020interfacegan}, enabling continuous and compositional control over generated images. Users could explore these spaces to discover interesting variations that would be difficult to describe in words~\cite{wu2021stylespace}, then combine them to achieve their creative goals~\cite{grabe2022towards}.

Diffusion models have largely superseded GANs in conditional image synthesis~\cite{dhariwal2021diffusion}, achieving greater diversity through much higher-dimensional latents. And yet an understanding of the underlying structure of these larger latent spaces has remained elusive. In this work, we ask a fundamental question: \emph{Can we automatically discover the visual structure within a diffusion model's knowledge of a concept?} Rather than requiring user-specified controls, we aim to decompose the model's internal representations into expressive directions that users can explore and combine.

To address these needs, we present \textbf{SliderSpace}, a framework that brings systematic explorability to diffusion models. Given just a text prompt, SliderSpace discovers a canonical set of meaningful, diverse, and controllable directions within the model's knowledge of that concept. Each direction is implemented as a low-rank adapter~\cite{hu2021lora} that can be scaled and composed with others, allowing users to explore and smoothly combine different aspects of variation, as shown in Figure~\ref{fig:intro}.

We ground SliderSpace discovery in three key requirements for meaningful decomposition of a diffusion model's visual manifold: 
\begin{enumerate}
    \item \textbf{Unsupervised Discovery:} The decomposition process should emerge from the intrinsic structure of the model's learned representation, rather than being guided by predefined attributes. This ensures we capture the true topology of the model's knowledge space rather than projecting our assumptions onto it.
    
    \item \textbf{Semantic Orthogonality:} Each discovered control must represent a distinct semantic direction. This is enforced in a semantic feature space, like CLIP, where every slider has an orthogonal effect in embeddings. This prevents discovering multiple controls that create similar semantic effects, making the system more efficient and easier.
    
    \item \textbf{Distribution Consistency:} Directions must induce consistent transformations across both random seeds and prompt variations. 
\end{enumerate}

These requirements naturally lead to our proposed framework, which we formalize in Section~\ref{sec:method}. As we show in our experiments, SliderSpace is architecture-agnostic, working with both conventional U-Net based models like Stable Diffusion~\cite{rombach2022high, rombach2022sd20, podell2023sdxl, turbo, dmd} and recent transformer-based architectures like Flux~\cite{flux}.

We demonstrate the expressiveness of SliderSpace through three applications: First, we show how SliderSpace can decompose high-level concepts into diverse and expressive components, revealing the natural axes of variation in the model's understanding. Second, we explore artistic style variation, where SliderSpace discovers directions that match or exceed the diversity of manually curated artist lists while being judged more useful by human evaluators. Finally, we show how SliderSpace can help reverse the mode collapse commonly observed in distilled diffusion models, restoring diversity while maintaining generation speed.

Beyond providing practical creative control, SliderSpace opens new avenues for understanding and utilizing the latent capabilities of diffusion models. By mapping these models' visual potential into intuitive, composable directions, we take a step toward making their creative possibilities more accessible and interpretable to users.

% Image editing methods in diffusion models unlock the creativity of users. In this work we ask an alternate question: \emph{Can we organize and expose what of the diffusion model is already capable of?}.
% Existing methods for controlling image generation typically require users to manually specify edit directions for desired changes. This process is time-consuming, requires technical expertise, and limits the spontaneity of the creative process. For instance, if a user wants to adjust the smile of a generated person, they must explicitly request this edit, often through imprecise prompt engineering or model fine-tuning. This approach of predefined controls or manual specifications restricts users from fully exploring the latent capabilities of the model. There may be interesting stylistic variations or attributes that the model can generate, but users have no easy way to discover or utilize these.

% Natural visual disentanglement was an emergent property in the latent space of Generative Adversarial Models (GANs) \cite{harkonen2020ganspace,radford2015unsupervised, wu2021stylespace, shen2020interfacegan}. In particular, it has been observed that StyleGAN~\cite{karras2019style} stylespace neurons offer detailed control over many meaningful aspects of images that would be difficult to describe in words~\cite{wu2021stylespace}. However, diffusion models do not share such a compact latent space~\cite{park2023unsupervised}; and efforts to uncover such a space in the semantic embeddings of the text conditioning have met with limited success \nik{Nick - is there a specific citation you were thinking about?}.

% In this work we introduce \textbf{SliderSpace}, which takes a step towards uncovering an analogous low dimensional representation of diffusion models' visual breadth; in essence treating the diffusion model as many generators sharing parameters, where a particular generator is defined by a specific prompt. For a given prompt we sample many random seeds (and optionally prompt expansions using an LLM), generate the corresponding images, and apply an off the shelf feature extractor (in this work CLIP, but our method can be applied to any differentiable feature extractor). We use PCA to analyze these features, and for each of the leading $k$ principal components we train a LoRA \cite{} which causes the diffusion model to produces images which increase the feature magnitude along that component when passed back through the same feature extractor. This leads to a 'Slider' for each principal component, because each LoRA can be scaled and applied to the original diffusion model, continuously varying those visual features in the generated results (as measured, in our case, by CLIP).

% There are many other works that enhance the controllability of diffusion models. One common approach is enabling users to add spatial constraints to a generation either manually, or via a reference image \cite{zhang2023addingconditionalcontroltexttoimage, chen2024trainingfreeregionalpromptingdiffusion}, a second is leveraging more abstract embeddings (e.g. identity, style) extracted from a reference image \cite{ye2023ipadaptertextcompatibleimage, hertz2024stylealignedimagegeneration, li2023photomaker, shi2024instantbooth}, a third is finetuning a foundation model to better generate a concept important to the user \cite{ruiz2022dreambooth, kumari2022customdiffusion, Ryu_lora, hu2021lora}, and a fourth (most relevant to this work) is finding low-rank adaptors of the model based on a prompt or small training set which can be scaled to provide continous control over one aspect of generated image (e.g. night vs day, basic vs luxury, etc.) \cite{gandikota2023concept}. SliderSpace is complementary to all of these methods and offers something distinct. All of the other methods we are aware require the user (and / or model designer) to know in advance what type of control they want. In contrast SliderSpace assists users in discovering and controlling hidden capabilities present in the diffusion model's distribution of possible generations.

%We propose that truly intuitive creative control in a text-to-image model should meet three key criteria: \emph{discoverability}, \emph{intuitiveness}, and \emph{specificity}. The model should reveal controllable attributes that may not be immediately obvious, offer controls that are easy to understand and manipulate, and ensure each control affects a distinct attribute of the generated image.

% We demonstrate the utility and power of SliderSpace using three applications built on top of SDXL-DMD \cite{dmd}, because its fast generation speed lends itself well to the continuous control offered by SliderSpace.

% First, we study concept decomposition (Section \ref{sec:concept_exp}), where we learn sliders for a specific concept (e.g. 'monster', 'waterfall', 'car'). Through quantitative metrics of diversity and text alignment we demonstrate that the learned sliders dramatically boost the diversity of generations when randomly applied without harming text alignment; we also ask humans to qualitatively judge these results in a user study where they find the SliderSpace results to be more 'Diverse', 'Useful', and 'Creative' than our baselines.

% Second, we attempt to compare the automatic discoveries of SliderSpace to a large scale manual study of artistic styles (Section \ref{sec:art_exp}), open-sourced by ParrotZone \cite{parrotzone}. In this study SDXL was prompted with over 4300 artist names,  and based on visual inspection the cases of successful stylistic mimicry recorded. Quantitatively SliderSpace more closely matches the distribution of artistic variation discovered by ParrotZone than other baselines, and in our user studies was judged to be significantly more 'Diverse' and 'Useful' than the baselines. To our surprise humans even judged SliderSpace results to be slightly more 'Diverse' than the results generated by the manually discovered artist names of \cite{parrotzone}.

% Third, we attempt to use SliderSpace to reverse the mode collapse commonly observed in distilled few-step diffusion models relative to the original teacher model (Section \ref{sec:diverse_exp}). We quantitatively demonstrate that applying SliderSpace to SDXL-DMD leads to more closely matching the distribution of images by the original teacher, SDXL.

%Through extensive experiments on various state-of-the-art text-to-image models, we demonstrate that SliderSpace significantly enhances user control and creative expression in AI-assisted image generation tasks. Our method enables a range of applications, including concept decomposition and control, diversity improvement in generated images, customization dissection and edits, and the exploration of artistic styles inherent in the model.

% SliderSpace goes beyond providing a practical tool for enhanced creative control. By mapping the visual potential of diffusion models it can open new avenues for generative creativity and deepens our understanding of each model's hidden potential. %%% Section 1. Introduction
\section{Related Work}

\paragraph{LLMs for Agent tasks.}

Our research is related to deploying large language models (LLMs) as agents for decision-making tasks in interactive environments~\citep{liu2023agentbench,zhou2023webarena,shridhar2020alfred,toyama2021androidenv}. Earlier works, such as~\citep{yao2023webshopscalablerealworldweb}, fine-tuned models like BERT~\citep{devlin2019bertpretrainingdeepbidirectional} for decision-making in simplified environments, such as online shopping or mobile phone manipulation. With the advent of large language models~\citep{brown2020languagemodelsfewshotlearners,openai2024gpt4technicalreport}, it became feasible to perform decision-making tasks through zero-shot or few-shot in-context learning. To better assess the capabilities of LLMs as agents, several models have been developed~\citep{deng2024mind2web,xiong2024watch,hong2023cogagent,yan2023gpt}. Most approaches~\citep{zheng2024seeact,deng2024mind2web} provide the agent with observation and action history, and the language model predicts the next action via in-context learning. Additionally, some methods~\citep{zhang2023building,li2023camel,song2024trial} attempt to distill trajectories from state-of-the-art language models to train more effective policy models. In contrast, our paper introduces a novel framework that automatically learns a reward model from LLM agent navigation, using it to guide the agents in making more effective plans.

\textbf{LLM Planning.} Our paper is also related to planning with large language models. Early researchers~\citep{brown2020languagemodelsfewshotlearners} often prompted large language models to directly perform agent tasks. Later, \citet{yao2022react} proposed ReAct, which combined LLMs for action prediction with chain-of-thought prompting~\citep{wei2022chain}. Several other works~\citep{yao2023treethoughtsdeliberateproblem,hao2023reasoning,zhao2023large,qiao2024agentplanningworldknowledge} have focused on enhancing multi-step reasoning capabilities by integrating LLMs with tree search methods. Our model differs from these previous studies in several significant ways. First, rather than solely focusing on text generation tasks, our pipeline addresses multi-step action planning tasks in interactive environments, where we must consider not only historical input but also multimodal feedback from the environment. Additionally, our pipeline involves automatic learning of the reward model from the environment without relying on human-annotated data, whereas previous works rely on prompting-based frameworks that require large commercial LLMs like GPT-4~\citep{openai2024gpt4technicalreport} to learn action prediction. Furthermore, \Model supports a variety of planning algorithms beyond tree search.

\textbf{Learning from AI Feedback.} In contrast to prior work on LLM planning, our approach also draws on recent advances in learning from AI feedback~\citep{bai2022constitutional,lee2023rlaif,yuan2024self,sharma2024critical,pan2024autonomous,koh2024tree}. These studies initially prompt state-of-the-art large language models to generate text responses that adhere to predefined principles and then potentially fine-tune the LLMs with reinforcement learning. Like previous studies, we also prompt large language models to generate synthetic data. However, unlike them, we focus not on fine-tuning a better generative model but on developing a classification model that evaluates how well action trajectories fulfill the intended instructions. This approach is simpler, requires no reliance on state-of-the-art LLMs, and is more efficient. We also demonstrate that our learned reward model can integrate with various LLMs and planning algorithms, consistently improving their performance.

\textbf{Inference-Time Scaling.} ~\citet{snell2024scaling} validates the efficacy of inference-time scaling for language models. Based on inference-time scaling, various methods have been proposed, such as random sampling~\citep{wang2022self} and tree-search methods~\citep{hao2023reasoning, zhang2024accessing, guan2025rstar}. Concurrently, several works have also leveraged inference-time scaling to improve the performance of agentic tasks. ~\citet{koh2024tree} adopts a training-free approach, employing MCTS to enhance policy model performance during inference and prompting the LLM to return the reward. ~\citet{gu2024your} introduces a novel speculative reasoning approach to bypass irreversible actions by leveraging LLMs or VLMs. It also employs tree search to improve performance and prompts an LLM to output rewards. ~\citet{yu2024exact} proposes Reflective-MCTS to perform tree search and fine-tune the GPT model, leading to improvements in ~\citet{koh2024visualwebarena}. ~\citet{putta2024agent} also utilizes MCTS to enhance performance on web-based tasks such as ~\citet{yao2023webshopscalablerealworldweb} and real-world booking environments. ~\cite{lin2025qlass} utilizes the stepwise reward to give effective intermediate guidance across different agentic tasks. Our work differs from previous efforts in two key aspects: (1) Broader Application Domain. Unlike prior studies that primarily focus on tasks from a single domain, our method demonstrates strong generalizability across web agents, mathematical reasoning, and scientific discovery domains, further proving its effectiveness. (2) Flexible and Effective Reward Modeling. Instead of simply prompting an LLM as a reward model, we finetune a small scale VLM~\citep{lin2023vila} to evaluate input trajectories. %Our reward scores range continuously between 0 and 1, in contrast to existing methods that rely on discrete scoring (e.g., 0 and 1, or 0, 0.5, and 1) through direct LLM prompting.

% Concurrently, several works have also leveraged inference-time scaling to improve the performance of agentic tasks. ~\citet{pan2024autonomous} demonstrates that LLMs and VLMs, such as the GPT series, can function as evaluators or reward models to provide guidance for fine-tuning or reflection, thereby enhancing digital agents. This lays the groundwork for subsequent studies that directly prompt LLMs as reward models. ~\citet{koh2024tree} adopts a training-free approach, employing MCTS to enhance policy model performance during inference. However, it is limited to web environments~\citep{koh2024visualwebarena}. Moreover, its value function relies on prompting an LLM, which is less effective than our proposed method. We validate our approach through ablation studies, demonstrating that our fine-tuned reward model is more effective. ~\citet{gu2024your} introduces a novel speculative reasoning approach to bypass irreversible actions, such as purchasing a product, by leveraging LLMs or VLMs. It also employs tree search to improve performance, but it remains restricted to the web domain~\citep{koh2024visualwebarena, deng2024mind2web}. Additionally, it lacks reward modeling and instead prompts an LLM to output rewards. ~\citet{yu2024exact} proposes Reflective-MCTS to perform tree search and fine-tune the GPT model, leading to improvements in ~\citep{koh2024visualwebarena}. However, this work focuses solely on a single web agent task, and its reward modeling is derived from multi-agent debate, differing from our more effective and efficient reward modeling approach. ~\citet{putta2024agent} also utilizes MCTS to enhance performance, but it is limited to web-based tasks such as ~\citep{yao2023webshopscalablerealworldweb} and real-world booking environments.
\section{Preliminary} \label{sec:preli}

In Section~\ref{sub:notation}, we introduce all the notations we used in our paper. Then, in Section~\ref{sub:flow_matching}, we show the basic facts about flow matching. In Section~\ref{sub:special_relativity}, we present the basic background of special relativity and define the relativistic force.

\subsection{Notations} \label{sub:notation}

For any positive integer $n$, we use $[n]$ to denote set $\{1,2,\cdots, n\}$. 
For two vectors $x \in \R^n$ and $y \in \R^n$, we use $\langle x, y \rangle$ to denote the inner product between $x,y$.
For a vector $v \in \R^n$, we use $\|v\|_2$ to denote the $\ell_2$-norm of $v$.
We use ${\bf 1}_n$ to denote a length-$n$ vector where all the entries are ones.
We use the symbol $ \perp $ to represent a component that is perpendicular to the direction of velocity, as exemplified by $ a_{\perp t} $, which denotes the perpendicular acceleration. Similarly, the symbol $ \parallel $ is employed to indicate a component parallel to the direction of velocity, such as $ f_{\parallel t} $, which represents the parallel force. We use $\dot{x}_t$ to denote $\frac{\d x_t}{\d t}$, and $\ddot{x}_t$ to denote $\frac{\d^2 x_t}{\d t^2}$.


\subsection{Flow Matching} \label{sub:flow_matching}

Flow Matching (FM) \cite{lcb+22,lgl22} is a generative modeling technique that constructs a smooth, invertible (i.e., diffeomorphic) mapping from a simple prior distribution to a complex target distribution. In FM, a time-dependent mapping $Z_t$ is defined to evolve according to an ordinary differential equation (ODE) driven by a vector field:
\begin{align*}
    \frac{\d x_t}{\d t} = V_t(x_t), \quad t \in [0, T].
\end{align*}
The goal is to ensure that, at the terminal time $T$, the ODE transforms a sample $x_0$ from a simple distribution (e.g., a Gaussian) into a sample $x_T$ from the target data distribution $\mathcal{D}$.

To achieve this, Flow Matching (FM) constructs a stochastic interpolation between a sample $x_1 \sim \mathcal{D}$ and a sample $x_0$ drawn from a known prior distribution, typically $\N(0,I)$. The interpolation is defined as
\begin{align*}
    x_t := \alpha_t x_1 + \sigma_t x_0, \quad t\in [0,T],
\end{align*}
where the time-dependent coefficients $\alpha_t$ and $\sigma_t$ are chosen so that
\begin{align*}
    \alpha_0 = 0,\quad \sigma_0 = 1,\quad \alpha_T = 1,\quad \sigma_T = 0.
\end{align*}
Thus, at $t=0$ the interpolated sample is purely the prior ($x_0$), and at $t=T$ it becomes a data sample ($x_1$).

The instantaneous change of $x$ is obtained by differentiating the interpolation:
\begin{align*}
    \frac{\d x_t}{\d t} = \frac{\d \alpha_t}{\d t} x_1 + \frac{\d \sigma_t}{\d t} x_0.
\end{align*}

The vector field is approximated by a neural network $V_t(x_t)$ with learnable parameters $\theta$. The FM training objective is then given by
\begin{align*}
    \mathcal{L}_\mathrm{FM}(\theta) := \E_{t\sim {\sf Uniform}[0,T], x_1 \sim \mathcal{D}} [\| V_t(x_t) - v_t(x_t) \|_2^2 ].
\end{align*}
This loss ensures that the learned velocity field $V_t(x_t)$ closely tracks the conditional dynamics $v_t(x_t)$ along the interpolation path.

After training, samples are generated by solving the ODE
\begin{align*}
    \frac{\d x_t}{\d t} = V_t(x_t),
\end{align*}
starting from an initial sample $x_0 \sim \N(0,I)$. Integrating this ODE from $t=0$ to $t=T$ yields a sample $x_T$ that approximates a draw from the target distribution. This ODE-based formulation offers a flexible and powerful framework for modeling complex data distributions while naturally incorporating conditional sampling.

\subsection{Background on Special Relativity} \label{sub:special_relativity}

We first introduce several essential ideas of special relativity \cite{e+05}.

\begin{definition}[Lorentz Factor]
\label{def:LorentzFactor}
According to special relativity~\cite{e+05}, the Lorentz factor at lab time $t$ is given by
\begin{align*}
\gamma_t := \frac{1}{\sqrt{1 - {\|v_t^{\rm lab}\|_2^2}/{c^2}}},
\end{align*}
where $v_t^{\rm lab}$ is the velocity at lab frame of reference, $c = 3 \times 10^8$ is the speed of light in vacuum.
\end{definition}

Then, we introduce the proper time of special relativity.

\begin{definition}[Proper Time]
\label{def:ProperTime}
The proper time is defined as the time interval measured in the rest frame of a moving object according to special relativity~\cite{e+05}. The differential form of the proper time is given by
\begin{align*}
    \d \tau = \frac{\d t}{\gamma_t},
\end{align*}
where $\d t$ is the time interval in the laboratory frame of reference, and $\gamma_t$ is the Lorentz factor at time lab time $t$ as defined in Definition~\ref{def:LorentzFactor}.
\end{definition}

Next, we define the force under special relativity here.

\begin{definition}[Relativistic Force]
\label{def:RelativisticForce}
In the framework of special relativity, the \emph{local force} (i.e., the force measured in the instantaneous rest frame of the particle) denoted as $f^{\rm local}$ has
\begin{align}
    f^{\rm local} := \frac{\d p^{\rm lab}}{\d \tau}, \label{eq:f_local}
\end{align}
where $p^{\rm lab}$ is the momentum at lab frame of reference, $\tau$ denotes the proper time defined in Definition~\ref{def:ProperTime}.

The momentum in the lab frame is defined as
\begin{align}
    p^{\rm lab} := m^{\rm lab} v_t^{\rm lab}, \label{eq:p}
\end{align}
where $m^{\rm lab}$ is the mass at lab frame of reference, and $v_t^{\rm lab}$ is the velocity at lab frame of reference.
\end{definition}

We state an equivalence lemma. Due to the space limitation, we delayed the proofs into the appendix.
\begin{lemma}[Equivalent Form of Relativistic Force, informal version of Lemma~\ref{lem:equiv_relativistic_force:formal}]\label{lem:equiv_relativistic_force:informal}
Let $p^{\rm lab}$ be the momentum defined in Eq.~\eqref{eq:p}, $\gamma_t$ be the Lorentz factor at lab time $t$ defined in Definition~\ref{def:LorentzFactor}, $\tau$ denotes the proper time, $v_t^{\rm lab} = \dot{x}_t$ denotes the velocity, 
$a_t^{\rm lab} = \ddot{x}_t$ denotes the acceleration.
The relativistic force, defined as the time derivative of the momentum in the lab frame, can be written as
\begin{align*}
f^{\rm local} =  m^{\rm lab}  (\gamma_t a_t^{\rm lab} + \gamma_t^3 \frac{ \langle v_t^{\rm lab}, a_t^{\rm lab} \rangle}{c^2} v_t^{\rm lab}).
\end{align*}

\end{lemma} 
\section{WARMUP}\label{sec:additive}
 

In this section, to demonstrate the new technique, we prove the following theorem.

\begin{lemma}\label{lem:warmup}
Given two $n \times n$ size matrices $A$ and $W$, $1 \leq k \leq n$  such that: Let each entry of $A,W$ can be represented by $n^{\gamma}$ bits, with $\gamma \in (0,1)$; Let $\OPT$ be defined as Definition~\ref{def:opt}.  
Then, we can show
\begin{itemize}
\item {\bf Part 1.} There is an algorithm that runs in $2^{O(nk\log n)}$ time, and outputs a number $\Lambda$  such that $\OPT \leq \Lambda \leq 2 \OPT $. 
\item {\bf Part 2.} There is an algorithm that runs in $2^{O(n k \log n)}$ time and returns $U \in \R^{n \times k}$ and $V \in \R^{n \times k}$ such that $\| (UV^\top - A) \circ W \|_F^2 \leq 2 \OPT$.
\end{itemize}
\end{lemma}



\begin{proof}

{\bf Proof of Part 1.}

We can create $2nk$ variables to explicitly represent each entry of $U$ and $V$. Let $g(x) =  \| W \circ ( U V^\top - A ) \|_F^2$. Let $L = 2^{n^{\gamma}}$. Then, we can write down a polynomial system (the decision problem defined in Theorem~\ref{thm:decision_problem})
\begin{align*}
   \min & ~  g(x) \\
    \mathrm{~s.t.~} & ~  U_{i,j} \in [-L,L], \forall i,j \\
    & ~ V_{i,j} \in [-L,L], \forall i,j
\end{align*}
Using Theorem~\ref{thm:jpt13}, we know the above system has
\begin{align*}
    \m = 2nk, \v = 2nk, \d = 4, \H = n^{\gamma}, \wt{\H} = n^{\gamma} + O(nk).
\end{align*}
The lower bound on $g(x)$ (if $g(x)$ is not zero) is going to be  
\begin{align*}
    c_{\mathrm{lower}} = & ~ 2^{-2^{3\v \log (\d)} \log (\wt{H}) } \\
    \geq & ~  ~ 2^{-2^{ O( nk )} \log(n^{\gamma} + nk) } \\
    \geq & ~  ~ 2^{-2^{O(nk \log n)}}.
\end{align*}

By Lemma~\ref{lem:opt}, we know the upper bound is $C_{\mathrm{upper}} = \poly(n) \cdot 2^{n^{\gamma}}$.
After knowing the lower bound and upper bound on cost, the number of binary search iterations is upper-bounded by
\begin{align*}
\log( \frac{C_{\mathrm{upper}}}{ c_{\mathrm{lower}} } ) 
= ~ \log( \frac{ \poly(n) 2^{n^{\gamma}} }{ 2^{-2^{ O(nk \log n)} } } ) 
\leq  ~ 2^{O(nk \log n)}.
\end{align*}


In each of the above iterations, we need to run Theorem~\ref{thm:decision_problem} with the system 
\begin{align*}
    \mathrm{~s.t.~} 
    & ~  g(x) \in [\Gamma_t, 2 \Gamma_t] , ~ U_{i,j} \in [-L, L] , ~ V_{i,j} \in [-L, L]
\end{align*}
with parameters,
\begin{align*}
\m = 2nk+1,  \v = 2nk, \d =4, \H = n^{\gamma}.
\end{align*}
Then, running time complexity is
\begin{align*}
 (\m \d)^{O(\v)} \cdot \poly(\H) 
 =  & ~ (10nk)^{O(nk)} \cdot \poly(n^{\gamma}) \\
 =  & ~ 2^{O( n k \log n) }.
\end{align*}

Thus, combining the number of iterations and time for each iteration, we can find the number $\Gamma \in [\OPT, 2\OPT]$.

{\bf Proof of Part 2.}

Next, similar to {\bf Part 1}, we need to repeat the binary search for $2nk$ times for each variable in $U$ and $V$, and each time, the number of total binary search steps is $n^{\gamma}$. Thus, we can output the $U, V$ in the same running time as finding $\Gamma$.
\end{proof}

In the next few sections, we will explain how to reduce the number of variables and how to reduce the number of constraints.
\section{LOWER BOUND ON OPT}\label{sec:relative}



We assume that $W$ has $r$ distinct rows and $r$ distinct columns. Then, we get rid of the dependence on $n$ in the degree.

\begin{theorem}[Implicitly in \cite{rsw16}]\label{thm:lower_bound_on_cost}
Assuming that $W$ has $r$ distinct rows and $r$ distinct columns, each entry of $A$ and $W$ needs $n^{\gamma}$ bits to represent. Assume $\OPT >0$. Then we know that, with high probability,  
\begin{align*}
    \OPT \geq 2^{-n^{\gamma} 2^{\wt{O}(rk^2/\epsilon)} }.
\end{align*}
\end{theorem}


\begin{proof}


We use $A_{i,*} \in \R^n$ denote the $i$-th row of $A$. We use $W_{i,*} \in \R^n$ to denote the $i$-th row of $W$. Let $(U_1)_{i,*}$ denote the $i$-th row of $U_1$.  
For any $n \times k$ matrix $U_1$ and for any $k \times n$ matrix $Z_1$, we have
\begin{align*}
 & ~ \| (U_1 Z_1 - A) \circ W \|_F^2 \\
 = & ~ \sum_{i=1}^n \| (U_1)_{i,*} Z_1 \diag( W_{i,*} ) - A_{i,*} \diag( W_{i,*} ) \|_2^2.
\end{align*}

Based on the observation that $W$ has $r$ distinct rows, we use group $g_{1,1}, g_{1,2}, \cdots , g_{1,r}$ to denote $r$ disjoint sets such that
\begin{align*}
    \cup_{i=1}^r g_{1,i} = [n]
\end{align*}
For any $i \in [r]$, for any $j_1, j_2 \in g_{1,i}$, we have $W_{j_1,*} = W_{j_2,*}$.


Thus, we can have 
\begin{align*}
    & ~ \sum_{i=1}^n \| (U_1)_{i,*} Z_1 \diag( W_{i,*} ) - A_{i,*} \diag( W_{i,*} ) \|_2^2 \\
    = & ~ \sum_{i=1}^r  \sum_{\ell \in g_{1,i}} \| (U_1)_{\ell,*} Z_1 \diag( W_{\ell,*} ) - A_{\ell,*} \diag( W_{\ell,*} ) \|_2^2  .
\end{align*}
We can sketch the objective function by choosing Gaussian matrices $S_1 \in \R^{n \times s_1}$ with $s_1 = O(k/\epsilon)$.
\begin{align*}
\sum_{i=1}^n \| (U_1)_{i,*} Z_1 \diag( W_{i,*} ) S_1 - A_{i,*} \diag( W_{i,*} ) S_1 \|_2^2 .
\end{align*}
Let $\wh{U}_1$ denote the optimal solution of the sketch problem,

\begin{align*}
    & ~ \wh{U}_1 =  \arg\min_{U_1 \in \R^{n \times k}} \\
    & ~ \sum_{i=1}^n \| (U_1)_{i,*} Z_1 \diag( W_{i,*} ) S_1 - A_{i,*} \diag( W_{i,*} ) S_1 \|_2^2 .
\end{align*}
By properties of $S_1$, plugging $\wh{U}_1$ into the original problem, we obtain
\begin{align*}
& ~ \sum_{i=1}^r  \sum_{\ell \in g_{1,i}} \| (\wh{U}_1)_{\ell,*} Z_1 \diag( W_{\ell,*} ) - A_{\ell,*} \diag( W_{\ell,*} ) \|_2^2  \\
\leq & ~  (1+\epsilon) \cdot \OPT.
\end{align*}

Let $R$ denote the set of all $\S(g_{1,i})$ (for all $i \in [r]$ and $|R| = r$)
 

Note that $\wh{U}_1$ also has the following form, for each $\ell \in L \subset [n]$ (Note that $|L| = r p$.)
\begin{align*}
(\wh{U}_1)_{\ell, *} 
= & ~ A_{\ell,*} \diag(W_{\ell,*}) S_1 \cdot ( Z_1 \diag( W_{\ell,*} ) S_1 )^\dagger \\
= & ~  A_{\ell,*} \diag(W_{\ell,*}) S_1 \cdot ( Z_1 \diag( W_{\ell,*} ) S_1 )^\top \\
& ~ \cdot ( ( Z_1 \diag( W_{\ell,*} ) S_1 ) ( Z_1 \diag( W_{\ell,*} ) S_1 )^\top )^{-1}.
\end{align*}

Recall the number of different $ \diag(W_{\ell,*})$ is at most $r$.

For each $k \times s_1$ matrix $Z_1 \diag( W_{\ell,*}) S_1$, we create $k \times s_1$ variables to represent it. Thus, we create $r$ matrices,
\begin{align*}
\{ Z_1 D_{W_{i,*}} S_1 \}_{i \in R}.
\end{align*}
For simplicity, let $P_{1,i} \in \R^{k \times s_1}$ denote $Z_1 \diag(W_{i,*}) S_1$. Then we can rewrite $\wh{U}^i$ as follows
\begin{align*}
    \wh{U}_1^i = A_{i,*} \diag(W_{i,*}) S_1  \cdot P_{1,i}^\top (P_{1,i} P_{1,i}^\top )^{-1}.
\end{align*}
If $P_{1,i} P_{1,i}^\top \in \R^{k \times k}$ has rank-$k$, then we can use Cramer's rule to write down the inverse of $P_{1,i} P_{1,i}^\top$.
For the situation, it is not full rank. We can guess the rank. Let $t_i \leq k$ denote the rank of $P_{1,i}$. Then, we need to figure out a maximal linearly independent subset of columns of $P_{1,i}$. We can also guess all the possibilities, which is at most $2^{O(k)}$. Because we have $r$ different $P_{1,i}$, the total number of guesses we have is at most $2^{O(rk)}$. Thus, we can write down $(P_{1,i} P_{1,i}^\top)^{-1}$ according to Cramer's rule. Note that $(P_{1,i} P_{1,i}^\top)^{-1}$ can be view as $P_a/P_b$ where $P_a$ is a polynomial and $P_b$ is another polynomial which is essentially $\det(P_{1,i} P_{1,i}^\top)$.

 
After $\wh{U}_1$ is obtained, we fix $\wh{U}_1$. We consider 
\begin{align*}
    & ~ \wh{U}_2 =  \arg\min_{U_2 \in \R^{n \times k}}  \| (\wh{U}_1 U_2^\top - A) \circ W \|_F^2 .
\end{align*}
In a similar way, we can get and write $\wh{U}_2$. 

Overall, by creating $l = O(rk^2/\epsilon)$ variables, we have rational polynomials $\wh{U}_1(x)$ and $\wh{U}_2(x)$. Note that $\wh{U}_1(x)$ only has $rp$ different rows, and same for $\wh{U}_2(x)$.

 Indeed, now we have only $2 r$ distinct denominators (w.l.o.g., assume the first $r$ columns are distinct and the first $r$ rows are distinct),
\begin{align*}
h_{1,i}(x) & = \det ( P_{1,i} P_{1,i}^\top ), \forall i \in[r] \\
h_{2,i}(x) & = \det ( P_{2,i} P_{2,i}^\top ), \forall i \in[r].
\end{align*}



Then, we can write down the following optimization problem,
\begin{align*}
\min _{x \in \R^l} & ~ p(x) / q(x) \\
\mathrm{s.t. } & ~ h_{1,i}^2(x) \neq 0, h_{2,i}^2(x) \neq 0, \forall i \in[r], \\
& ~ q(x)=\prod_{i=1}^r h_{1,i}^2(x) h_{2,i}^2(x),
\end{align*}
where $q(x)$ has degree $O(r k)$, the maximum coefficient in absolute value is 
\begin{align*}
( 2^{n^{\gamma}} )^ {O (r k )}
\end{align*}
 and the number of variables $O (r k^2 / \epsilon )$. However, that formulation $p(x)/q(x)$ is not a polynomial, to further make it a polynomial, we introduce variable $y$:
 \begin{align*}
\min _{x \in \R^l} & ~ p(x) y \\
\mathrm{s.t. } & ~ h_{1,i}^2(x) \neq 0, h_{2,i}^2(x) \neq 0, \forall i \in[r], \\
& ~ q(x)=\prod_{i=1}^r h_{1,i}^2(x) h_{2,i}^2(x) \\
& ~ q(x)y - 1 =0.
\end{align*}

Applying Theorem~\ref{thm:jpt13}, with parameters
\begin{align*}
& \m= O(r), \v = O(rk^2/\epsilon),  \d = O(r) , \\
& \H = 2^{O( n^{\gamma} rk ) } ,  \wt{H} = O(\H) .
\end{align*}

we can achieve the following minimum nonzero cost: 
\begin{align*}
\geq  ~ 2^{-2^{3 \v \log (\d)} \log( \wt{\H}) } 
\geq  ~ 
2^{-n^\gamma 2^{\wt{O} (r k^2 / \epsilon )}  } .
\end{align*}
 Thus, we complete the proofs.
\end{proof}

\section{FEW DISTINCT COLUMNS}\label{sec:few_distinct_columns}

In this section, we try to estimate the $\OPT$ value. 

\begin{theorem}\label{thm:few_distinct_columns}
Given a matrix $A$ and $W$, each entry can be written using $O(n^{\gamma})$ bits for $\gamma >0$.
Given $A \in \R^{n \times n}, W \in \R^{n \times n}, 1 \leq k \leq n$.  Assume that $W$ has $r$ distinct columns and rows.
   
Then, with high probability,  
one can output a number $\Lambda$ in time $n^{1+\gamma} \cdot p \cdot 2^{O (k^2 r / \epsilon )} $ such that 
\begin{align*}
    \OPT \leq \Lambda \leq(1+\epsilon) \OPT .
\end{align*}
\end{theorem}
\begin{proof}

Let $U_1^*, U_2^* \in \R^{n \times k}$ denote the matrices satisfying 
\begin{align*}
    \| W \circ (U_1^* (U_2^*)^\top - A) \|_F^2 = \OPT.
\end{align*}


We use $A_{i,*} \in \R^n$ denote the $i$-th row of $A$. We use $W_{i,*} \in \R^n$ to denote the $i$-th row of $W$. Let $(U_1)_{i,*}$ denote the $i$-th row of $U_1$. Let $Z_1 = (U_2^*)^\top$.  
For any $n \times k$ matrix $U_1$,
\begin{align*}
 & ~ \| (U_1 Z_1 - A) \circ W \|_F^2 \\
 = & ~ \sum_{i=1}^n \| (U_1)_{i,*} Z_1 \diag( W_{i,*} ) - A_{i,*} \diag( W_{i,*} ) \|_2^2.
\end{align*}

Based on the observation that $W$ has $r$ distinct rows, we use group $g_{1,1}, g_{1,2}, \cdots , g_{1,r}$ to denote $r$ disjoint sets such that
\begin{align*}
    \cup_{i=1}^r g_{1,i} = [n]
\end{align*}
For any $i \in [r]$, for any $j_1, j_2 \in g_{1,i}$, we have $W_{j_1,*} = W_{j_2,*}$.

Next, based on assumptions on $W$ and $A$, we use $g_{1,i,1}$, $g_{1,i,2}$, $g_{1,i,p}$ to denote $p$ groups such that 
\begin{align*}
    \cup_{j=1}^p g_{1,i,j} = g_{1,i} .
\end{align*}
For any $i \in [r]$, for any $j \in [p]$, for any $\ell_1, \ell_2 \in g_{1,i,j}$, we have  $(W_{\ell_1,*} \circ A_{\ell_1,*}) = (W_{\ell_2,*} \circ A_{\ell_2,*})$.

Let $\S(g_{1,i,j})$ denote the smallest index from set $g_{1,i,j}$

Thus, we can have 
\begin{align*}
    & ~ \sum_{i=1}^n \| (U_1)_{i,*} Z_1 \diag( W_{i,*} ) - A_{i,*} \diag( W_{i,*} ) \|_2^2 \\
    = & ~ \sum_{i=1}^r \sum_{j =1}^p \sum_{\ell \in g_{1,i,j}} \| (U_1)_{\ell,*} Z_1 \diag( W_{\ell,*} ) \\
     & ~ \quad\quad\quad\quad\quad\quad - A_{\ell,*} \diag( W_{\ell,*} ) \|_2^2 \\
    = & ~\sum_{i=1}^r \sum_{j =1}^p |g_{1,i,j}| \cdot \| (U_1)_{\S(g_{1,i,j}),*} Z_1 \diag( W_{ \S(g_{1,i,j}),*} ) \\
    & ~ \quad\quad\quad\quad\quad\quad - A_{\S(g_{1,i,j}),*} \diag( W_{\S(g_{1,i,j}),*} ) \|_2^2 .
\end{align*}
We can sketch the objective function by choosing Gaussian matrices $S_1 \in \R^{n \times s_1}$ with $s_1 = O(k/\epsilon)$.
\begin{align*}
& ~ \sum_{i=1}^r \sum_{j =1}^p |g_{1,i,j}| \cdot \| (U_1)_{\S(g_{1,i,j}),*} Z_1 \diag( W_{ \S(g_{1,i,j}),*} ) S_1 \\
& ~ \quad\quad\quad\quad\quad\quad - A_{\S(g_{1,i,j}),*} \diag( W_{\S(g_{1,i,j}),*} ) S_1 \|_2^2.
\end{align*}
Let $\wh{U}_1$ denote the optimal solution of the sketch problem,
\begin{align*}
    \wh{U}_1 = & ~ \arg\min_{U_1} \sum_{i=1}^r \sum_{j =1}^p |g_{1,i,j}| \\
    & ~ \cdot \| (U_1)_{\S(g_{1,i,j}),*} Z_1 \diag( W_{ \S(g_{1,i,j}),*} ) S_1 \\
    & ~ \quad - A_{\S(g_{1,i,j}),*} \diag( W_{\S(g_{1,i,j}),*} ) S_1 \|_2^2.
\end{align*}
By properties of $S_1$, plugging $\wh{U}_1$ into the original problem, we obtain
\begin{align*}
& ~ \sum_{i=1}^r \sum_{j =1}^p |g_{1,i,j}| \cdot \| (U_1)_{\S(g_{1,i,j}),*} Z_1 \diag( W_{ \S(g_{1,i,j}),*} ) \\
& ~ \quad\quad - A_{\S(g_{1,i,j}),*} \diag( W_{\S(g_{1,i,j}),*} ) \|_2^2 \leq (1+\epsilon) \cdot \OPT.
\end{align*}

Let $R$ denote the set of all $\S(g_{1,i})$ (for all $i \in [r]$ and $|R| = r$).

Let $L$ denote the set of all $\S(g_{1,i,j})$ (for all $i \in [r]$, $j \in [p]$ and $|L| = rp$).

Note that $\wh{U}_1$ also has the following form, for each $\ell \in L \subset [n]$ (Note that $|L| = r p$.)
\begin{align*}
(U_1)_{\ell, *} 
= & ~ A_{\ell,*} \diag(W_{\ell,*}) S_1 \cdot ( Z_1 \diag( W_{\ell,*} ) S_1 )^\dagger \\
= & ~  A_{\ell,*} \diag(W_{\ell,*}) S_1 \cdot ( Z_1 \diag( W_{\ell,*} ) S_1 )^\top \\
& ~ \cdot ( ( Z_1 \diag( W_{\ell,*} ) S_1 ) ( Z_1 \diag( W_{\ell,*} ) S_1 )^\top )^{-1}.
\end{align*}

Recall the number of different $A_{\ell,*} \diag(W_{\ell,*})$ is at most $rp$, and the number of different $ \diag(W_{\ell,*})$ is at most $r$. For each $k \times s_1$ matrix $Z_1 \diag( W_{\ell,*}) S_1$, we create $k \times s_1$ variables to represent it. Thus, we create $r$ matrices,
\begin{align*}
\{ Z_1 \diag( W_{i,*} ) S_1 \}_{i \in R}. 
\end{align*}
For simplicity, let $P_{1,i} \in \R^{k \times s_1}$ denote $Z_1 \diag(W_{i,*}) S_1$. Then we can rewrite $(\wh{U}_1)_{i,*}$ as follows
\begin{align*}
    (\wh{U}_1)_{i,*} = A_{i,*} \diag(W_{i,*}) S_1  \cdot P_{1,i}^\top (P_{1,i} P_{1,i}^\top )^{-1}.
\end{align*}
If $P_{1,i} P_{1,i}^\top \in \R^{k \times k}$ has rank-$k$, then we can use Cramer's rule to write down the inverse of $P_{1,i} P_{1,i}^\top$.
For the situation, it is not full rank. We can guess the rank. Let $t_i \leq k$ denote the rank of $P_{1,i}$. Then, we need to figure out a maximal linearly independent subset of columns of $P_{1,i}$. We can also guess all the possibilities, which is at most $2^{O(k)}$. Because we have $r$ different $P_{1,i}$, the total number of guesses we have is at most $2^{O(rk)}$. Thus, we can write down $(P_{1,i} P_{1,i}^\top)^{-1}$ according to Cramer's rule. Note that $(P_{1,i} P_{1,i}^\top)^{-1}$ can be view as $P_a/P_b$ where $P_a$ is a polynomial and $P_b$ is another polynomial which is essentially $\det(P_{1,i} P_{1,i})$.


After $\wh{U}_1$ is obtained, we will fix $\wh{U}_1$ in the next round.

For any $n \times k$ matrix $U_2$, we can rewrite 
\begin{align*}
 & ~ \| (\wh{U}_1 U_2^\top  - A ) \circ \|_F^2 \\
 = & ~ \sum_{i=1}^n \| \diag(  W_{*,i} ) \wh{U}_1   (U_2^\top)_{*,i}   - \diag(W_{*,i}) A_{*,i} \|_2^2 .
\end{align*}



Based on the observation that $W$ has $r$ columns rows, we use group $g_{2,1}, g_{2,2}, \cdots , g_{2,r}$ to denote $r$ disjoint sets such that
\begin{align*}
    \cup_{i=1}^r g_{2,i} = [n].
\end{align*}
For any $i \in [r]$, for any $j_1, j_2 \in g_{2,i}$, we have $W_{*,j_1} = W_{*,j_2}$.

Next, based on assumptions on $W$ and $A$, we use $g_{2,i,1}$, $g_{2,i,2}$, $g_{2,i,p}$ to denote $p$ groups such that 
\begin{align*}
    \cup_{j=1}^p g_{2,i,j} = g_{2,i}.
\end{align*}
For any $i \in [r]$, for any $j \in [p]$, for any $\ell_1, \ell_2 \in g_{2,i,j}$, we have  $(W_{*,\ell_1} \circ A_{*,\ell_1}) = (W_{*,\ell_2} \circ A_{*,\ell_2})$.

Let $\S(g_{2,i,j})$ denote the smallest index from set $g_{2,i,j}$

Thus, we can have 
\begin{align*}
    & ~ \sum_{i=1}^n \| \diag( W_{*,i} ) \wh{U}_1   (U_2^\top )_{*,i}  -  \diag( W_{*,i} ) A_{*,i} \|_2^2 \\
    = & ~ \sum_{i=1}^r \sum_{j =1}^p \sum_{\ell \in g_{1,i,j}} \|  \diag( W_{*,\ell} ) \wh{U}_1 (U_2^\top )_{*,\ell} \\
    & ~ \quad\quad\quad\quad\quad\quad - \diag( W_{*,\ell} ) A_{*,\ell}  \|_2^2 \\
    = & ~\sum_{i=1}^r \sum_{j =1}^p |g_{2,i,j}| \cdot \|  \diag( W_{ *, \S(g_{2,i,j}) } ) \wh{U}_1 (U_2^\top )_{*, \S(g_{2,i,j})} \\
    & ~ \quad\quad\quad\quad\quad\quad -  \diag( W_{*, \S(g_{1,i,j})} ) A_{*,\S(g_{1,i,j})} \|_2^2 .
\end{align*}
We can sketch the objective function by choosing Gaussian matrices $S_2 \in \R^{n \times s_1}$ with $s_2 = O(k/\epsilon)$.
\begin{align*}
& ~ \sum_{i=1}^r \sum_{j =1}^p |g_{2,i,j}| \cdot \| S_2  \diag( W_{ \S(g_{2,i,j}),*} ) \wh{U}_1  (U_2^\top )_{*, \S(g_{2,i,j}) } \\
& ~ \quad\quad\quad\quad - S_2 \diag( W_{*, \S(g_{2,i,j})} )  A_{*, \S(g_{2,i,j}) }  \|_2^2.
\end{align*}
Let $\wh{U}_1$ denote the optimal solution of the sketch problem,
\begin{align*}
    \wh{U}_1 = & ~ \arg\min_{U_1} \sum_{i=1}^r \sum_{j =1}^p |g_{2,i,j}| \\
    & ~  \cdot \| S_2  \diag( W_{ \S(g_{2,i,j}),*} ) \wh{U}_1  (U_2^\top )_{*, \S(g_{2,i,j}) } \\
    & ~ - S_2 \diag( W_{*, \S(g_{2,i,j})} )  A_{*, \S(g_{2,i,j}) }  \|_2^2.
\end{align*}
By properties of $S_1$, plugging $\wh{U}_2$ into the original problem, we obtain
\begin{align*}
& ~ \sum_{i=1}^r \sum_{j =1}^p |g_{2,i,j}| \cdot \|  \diag( W_{ \S(g_{2,i,j}),*} ) \wh{U}_1  ( \wh{U}_2^\top )_{*, \S(g_{2,i,j}) } \\
& ~ -  \diag( W_{*, \S(g_{2,i,j})} )  A_{*, \S(g_{2,i,j}) }  \|_2^2\leq (1+\epsilon) \cdot \OPT.
\end{align*}

Let $R$ denote the set of all $\S(g_{1,i})$ (for all $i \in [r]$ and $|R| = r$).

Let $L$ denote the set of all $\S(g_{1,i,j})$ (for all $i \in [r]$, $j \in [p]$ and $|L| = rp$).

Note that $\wh{U}_1$ also has the following form, for each $\ell \in L \subset [n]$ (Note that $|L| = r p$.)
\begin{align*}
(U_1)_{\ell, *} 
= & ~  ( S_2 \diag( W_{*,\ell} ) \wh{U}_1 )^\dagger \cdot S_2 \diag(W_{*,\ell})  A_{*,\ell} \\
= & ~ (  ( S_2 \diag( W_{*,\ell} ) \wh{U}_1 ) ( S_2 \diag( W_{*,\ell} ) \wh{U}_1)^\top )^{-1} \\
& ~ \cdot S_2 \diag( W_{*,\ell} ) \wh{U}_1 \cdot S_2 \diag(W_{*,\ell})  A_{*,\ell}.
\end{align*}

Recall the number of different $A_{*,\ell} \diag(W_{*,\ell})$ is at most $rp$, and the number of different $ \diag(W_{*,\ell})$ is at most $r$.

For each $s_2 \times k$ matrix $S_2 \diag( W_{*,i} ) \wh{U}_1$, we create $s_2 \times k$ variables to represent it. Thus, we create $r$ matrices,
\begin{align*}
\{ S_2 \diag( W_{*,i} ) \wh{U}_1 \}_{i \in R}.
\end{align*}
For simplicity, let $P_{2,i} \in \R^{k \times s_2}$ denote $S_2 \diag( W_{*,i} ) \wh{U}_1$. Then we can rewrite $\wh{U}_2$ as follows
\begin{align*}
    (\wh{U}_2^\top)_{*,i} = (P_{2,i} P_{2,i}^\top)^{-1}  P_{2,i} S_2 \diag(W_{*,i})  A_{*,i}.
\end{align*}

In a similar way, we can write $\wh{U}_2$. Overall, by creating $l = O(rk^2/\epsilon)$ variables, we have rational polynomials $\wh{U}_1(x)$ and $\wh{U}_2(x)$. Note that $\wh{U}_1(x)$ only has $rp$ different rows, and same for $\wh{U}_2(x)$.

Putting it all together, we can write the objective function,
\begin{align*}
\min_{x \in \R^l} & ~ \| ( \wh{U}_1(x) \wh{U}_2(x)^\top - A ) \circ W \|_F^2 \\
\mathrm{~s.t.~}& ~ h_{1,i}(x) \neq 0 ,  \forall i \in [r] \\
& ~ h_{2,i}(x) \neq 0, \forall i \in [r].
\end{align*}

Note that $\wh{U}_1(x) \wh{U}_{2}(x) \circ W$ only has $rp$ distinct rows. Also, $A \circ W$ only has $rp$ distinct rows. Writing down the objective function $\| ( \wh{U}_1(x) \wh{U}_2(x)^\top - A ) \circ W \|_F^2$ only requires $n (rp) \cdot \poly(kr/\epsilon)$ time.

Combining the binary search explained in Lemma~\ref{lem:warmup} 
with the lower bound on cost (Theorem~\ref{thm:lower_bound_on_cost}) we obtained, we can find the solution for the original problem in time,
\begin{align*}
(np \cdot \poly(kr/\epsilon) + n 2^{\wt{O} (rk^2 / \epsilon) }  ) \cdot n^{\gamma} = n^{1+\gamma} p \cdot 2^{\wt{O}(rk^2/\epsilon)}.
\end{align*}

\end{proof}
\section{RECOVER A SOLUTION}\label{sec:recover_solution}

We state our results and proof of recovering a solution.
\begin{theorem}\label{thm:recover_solution}
Given a matrix $A$ and $W$, each entry can be written using $O(n^{\gamma})$ bits for $\gamma >0$.
Given $A \in \R^{n \times n}, W \in \R^{n \times n}, 1 \leq k \leq n$ and $ \epsilon \in (0,0.1)$. 
Assume $W$ has $r$ distinct columns and rows. 
Assume $A \circ W$ has at most $r \cdot p$ distinct columns and at most $r \cdot p$ distinct rows. 
Let $\OPT$ be defined as Definition~\ref{def:opt}. 
Then, with high probability,  
one can output two matrices $U,V \in \R^{n \times k}$ in time $n^{1+\gamma} \cdot 2^{O (k^2 r / \epsilon )} $ such that 
\begin{align*}
    \| (UV^\top- A) \circ W \|_F^2 \leq(1+\epsilon) \OPT .
\end{align*}

Further, if we choose (1) $k^2 r = O(\log n /\log\log n)$, (2) $\epsilon \in (0,0.1)$ to be a small constant, (3) $p = n^{o(1)}$ and (4) $\gamma = o(1)$, then the running time becomes $n^{1+o(1)}$.
\end{theorem}


\begin{proof}

Here, we show how to recover an approximate solution, not only the value of $\OPT$.

The idea is to recover the entries of $U$ and $V$ one by one and use the algorithm from the previous section for the corresponding decision problem. We initialize the semialgebraic set to be
\begin{align*}
    S= \{x \in \R^l \mid q(x) \neq 0, p(x) \leq \Lambda q(x) \}.
\end{align*}
We start by recovering the first entry of $U$. We perform the binary search to localize the entry, which takes $ \log  ( 2^{n^{\gamma}} )$ invocations of the decision algorithm. For each step of binary search, we use Theorem~\ref{thm:jpt13} to determine whether the following semi-algebraic set $S$ is empty or not,
\begin{align*}
S \cap \{U_{1,1}(x) \geq \wh{U}_{1,1}^{-}, U_{1,1}(x) \leq \wh{U}_{1,1}^{+} \}.
\end{align*}
After that, we declare the first entry of $U$ to be any point in this interval.  Then, we add an equality constraint that fixes the entry of $\wh{U}$ to this value and add a new constraint into $S$ permanently, e.g., $S  \leftarrow S \cap \{U_{1,1}(x)=\wh{U}_{1,1} \}$. Next, we repeat the same with the second entry of $U$ and so on.

This allows us to recover a solution of cost at most $(1+ \epsilon) \OPT$ in time
\begin{align*}
n^{1+\gamma} \cdot p \cdot 2^{O (k^{2} r / \epsilon )} .
\end{align*}
If we choose $\gamma = o(1)$, $\epsilon = \Theta(1)$, $p =n^{o(1)}$ and $k^2 r = O(\log n / \log\log n )$, then the running time becomes $n^{1+o(1)}$.
Thus, we complete the proof.
\end{proof}


\section{Summary and Conclusion}
\label{sec:conclusion}


In this paper, we introduced \ToolName{}, a method for discovering fine-grained \emph{sub-activities} from unlabeled smart home sensor data without relying on pre-segmentation. Our pipeline is organized into two core steps: Clustering and Labeling. 
The \textbf{Clustering step} consists of:

\begin{itemize}
    \item \textbf{Encoder Pre-Training:} We leverage a pre-trained BERT model adapted with sensor-specific tokens and train it using a masked language modeling (MLM) objective to generate context-rich embeddings for raw sensor sequences.
    
    \item \textbf{Clustering Model Fine-Tuning:} Using the SCAN loss function, we fine-tune these embeddings to form more homogeneous and distinct clusters of sensor sequences.
\end{itemize}

The \textbf{Labeling step} comprises:

\begin{itemize}
    \item \textbf{Cluster Centroid Annotation:} Representative sequences from each cluster are visualized with a custom tool, enabling expert annotators to assign meaningful sub-activity labels to the centroids.
    
    \item \textbf{Label Propagation:} The centroid labels are propagated to all sequences within their respective clusters, resulting in a fully labeled dataset with minimal manual effort.
    
    \item \textbf{Re-annotation of Original Time-Series Data:} 
    Finally, these propagated labels are mapped back onto the original time-series data, preserving temporal continuity and facilitating the analysis of longitudinal activity patterns.
\end{itemize}


Our approach addresses important challenges in HAR, including the high cost and effort of manual data annotation, the limitations of coarse activity labels, and the need for scalable and generalizable models. \ToolName{} offers an open source tool that facilitates the HAR annotation and re-annotation process and enables the dynamic discovery and validation of sub-activities, thus capturing a broader spectrum of behaviors observed in real homes.

\ifdefined\isarxiv
% \section*{Acknowledgement}
% Research is partially supported by the National Science Foundation (NSF) Grants 2023239-DMS, CCF-2046710, and Air Force Grant FA9550-18-1-0166.
\else
\bibliography{ref}
\bibliographystyle{plainnat}
\input{20_checklist}
\fi



% \newpage
% \onecolumn
% \appendix


% \ifdefined\isarxiv
% \begin{center}
% 	\textbf{\LARGE Appendix }
% \end{center}
% \else
% \aistatstitle{When Can We Solve the Weighted Low Rank Approximation Problem in Truly Subquadratic Time?: \\
% Supplementary Materials}
% {\hypersetup{linkcolor=black}
% %\tableofcontents
% \bigbreak
% \bigbreak
% \bigbreak
% }
% \fi
% \input{app_preli}
% \input{app_recover_solution}




\ifdefined\isarxiv
%\section*{Acknowledgments}
\bibliographystyle{alpha}
\bibliography{ref}

\else




\fi

%%%% Cut-line between first 10 pages and appendix







%%% some writing rules

%% Writing rule for creating tags.
%% Tags :
%% Theorem    \ref{thm:bla_bla}
%% Lemma      \ref{lem:bla_bla}
%% Claim      \ref{cla:bla_bla}
%% Corollary  \ref{cor:bla_bla}
%% Fact       \ref{fac:bla_bla}
%% Definition \ref{def:bla_bla}
%% Section    \ref{sec:bla_bla}
%% Subsection \ref{sub:bla_bla}
%% Equation   \ref{eq:bla_bla}



\end{document}



%%%%%%%%%%%%%%%%%%%%%%%%%%%%%%%%%%%%%%%%%%%%%%%%%%%%%%%%%%%%%%%%%%%%%%%%%%%%%%%%%%%%%%%%%%%%%%%%%%%%%%%%%%%%%%%%%%%%%%%%%%%%%%%%%%%%%%%%%%%%%%%%%%%%%%%%%%%%%%%%%%%%%%%%%%%%%%%%%%%%%%%%%%%%%%%%%%%%%%%%%%%%%%%%%%%%%%%%%%%%%%%%%%%%%%%%%%%%%%%%%%%%%%%%%%%%%%%%%%%%%%%%%%%%%%%%%%%%%%%%%%%%%%%%%%%%%%%%%%%%%%%%%%%%%%%%%%%%%%%%%%%%%%%%%%%%%%%%%%%%%%%%%%%%%%%%%%%%%%%%%%%%%%%%%%%%%%%%%%%%%%%%%%%%%%%%%%%%%%%%%%%%%%%%%%%%%%%%%%%%%%%%%%%%%%%%%%%%%%%%%%%%%%%%%%%%%%%%%%%%%%
