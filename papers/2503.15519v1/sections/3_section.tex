\section{Methods}
We create a workflow to maximize efficiency in letting human and AI work together. 

\begin{figure}[h]
    \centering
    \includegraphics[width=\textwidth]{./flowchart.png} \caption{Workflow graph}
\end{figure}

The code generates a GUI for interacting with multiple models at once, including models from OpenAI, Anthropic, and Gemini. By default, it uses GPT 4o, Claude 3.5 Sonnet, and Gemini 1.5 Flash. Due to monetary constraints, OpenAI's O1 model is not available. A GUI is created with Tkinter to sort information and streamline usage. Models are instructed to code in C++, since olympiad programming almost always guarantees a C++ solution but not necessarily in other languages.

Since the models' providers have different APIs, a unified client normalizes the format within the code. Since models run at different speed, it utilizes the async capabilities of these APIs to update the GUI as information comes in. This client also stores message history to allow for chats between the user and the models. A new instance of the client is initialized for each model. It will also limit token usage to a constant set for each model. 

The upper half consists of human input and the bottom half is split into columns, each for a different model's output. 

To initiate the chats, the human provides three types of information:
1. The problem text. A button takes the text from the clipboard and stores it, allowing for quick input. The program requires the problem text before allowing any generation. 
2. The algorithm description. The user freely types into the main input text box for this section. The program does not require an algorithm to start.
3. The reference material. The human lists chapters of the CP-Algorithm website to feed into the model as reference material. The user inputs simplified file paths in a small text box. The program does not require reference material to start.

\begin{figure}[h]
    \centering
    \includegraphics[width=\textwidth, trim=0 500 0 0, clip]{./corpus.png} 
    \caption{Corpus loaded}
\end{figure}

\begin{figure}[h]
    \centering
    \includegraphics[width=\textwidth, trim=0 500 0 0, clip]{./clipboard.png} 
    \caption{Problem text loaded}
\end{figure}


The figures are trimmed for enlarged size. Upon inputting the problem text or corpus, a status message confirms success or reports an error. Once everything is loaded, the information is compiled into a prompt and sent to each model. 

A button sends the starting prompt to each of the models and updates the GUI as each API returns text. 

\newpage

Once the chats are all initiated, the main input text box is no longer used for the algorithm description, but becomes a place to write messages to the models. There is a button for each individual model and one to send to all of them at once. 

\begin{figure}[h]
    \centering
    \includegraphics[width=\textwidth, trim=0 100 0 250, clip]{./usage.png} 
    \caption{Outputs asynchronously updating}
\end{figure}
