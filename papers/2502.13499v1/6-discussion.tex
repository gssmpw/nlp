\section{Discussion}
Our findings show that a significant percentage of web component designs produced by LLMs contain at least one dark pattern. If users directly implement this code on their websites, it may lead to unintended ethical issues that manipulate end-users into making choices against their best interests. Although there are variations in the frequency of dark patterns generated by the four models tested in this study, all produced significant numbers of dark patterns. As the generation of dark patterns is currently unavoidable, human intervention is crucial, and the responsibility extends beyond designers to include developers and other LLM users. This highlights the need for educational opportunities to raise awareness and equip these users to address dark patterns effectively.

\subsection{Prioritize Interventions based on Component Types}
Systematic interventions are needed when using LLMs for website generation with code assistance. Our research findings show that dark patterns are more likely to be generated in certain types of web components, such as newsletter and membership sign-ups, as well as promotional components like discount popovers and sales banners. Therefore, when using LLMs for code generation in front-end development, it is important to pay extra attention and prioritize interventions to components tied to company interests, such as revenue generation, customer retention, user data collection, and user engagement. 


\subsection{Building Mental Models as Interventions}
Gray et al. \cite{a:42} found that users often struggle to identify dark patterns that use deceptive strategies, such as hiding information, forcing actions, or using trick questions. This presents a significant challenge for LLM users who try to address dark patterns, as our study shows that a large portion of the LLM-generated dark patterns --- 57\% of the identified components --- fall under the category of information hiding. Without a clear mental model of what information should be provided, these dark patterns can be very challenging to spot. The examples provided in this study for the six dark pattern attributes can serve as a reference and a starting point for users to understand the types of tricks that LLM-generated designs might use under each attribute.

There is ongoing discussion on how to scale up efforts to expose users to a wider variety of dark patterns for educational purposes. Lu et al. \cite{a:56} proposed several approaches to increase scalability, including using machine learning for automated dark pattern recognition and using community contributions to crowdsource examples of dark patterns, gather meta-information about designers' intentions behind the designs, and support developers in creating new detection tools. Early efforts have also explored automated dark pattern detection tools, such as Aid UI \cite{a:66}. In addition to these approaches, it is equally important to help users build comprehensive mental models users can apply this knowledge across different attributes and contexts, making it transferable and not limited to specific instances. Such an approach ensures that users are equipped to identify and address dark patterns more effectively, even in unfamiliar scenarios.

\subsection{Proposing Safe Meta Prompts for Ethical Design Generation with LLMs}
In this study, we also tested conditions that reflected different stakeholder interests. However, these conditions did not yield statistically significant differences. The results suggest that providing broad stakeholder interests has a minimal impact on the generation of dark patterns by LLMs. % as they neither reduce nor increase their frequency. 
This indicates that containing high-level guidance in prompts is insufficient to address the issue of dark patterns in LLM-generated designs. %Given these findings, we are considering whether providing 
However, providing more detailed user-centered design principles in the form of meta-prompts could be more effective. Such prompts could offer specific instructions on prioritizing user needs and ethical design principles, potentially resulting in designs that better reflect user interests and reduce the presence of dark patterns.

\subsection{Extending Design Responsibility and Expanding the Need for Education}
Reducing dark patterns has been viewed as a primary responsibility of designers, as they have power and influence over interface design \cite{a:43, a:49}. A large portion of past efforts to combat unethical designs has focused on providing ethical education to designers and UX professionals to increase their awareness and sensitivity to user values \cite{a:53, a:54, a:55}. 

However, as LLMs become increasingly capable in front-end development, handling both design and coding tasks, dark pattern designs can bypass designers and be directly presented to end users. In one way, this shifts the power dynamic, as LLM users can redesign and modify the UI generated by LLMs, reducing the traditional control held by designers \cite{a:56}. On the other hand, this also makes it more challenging for users as they must critically evaluate the quality and ethics of the generated work, as they must assess it from both a design and a code quality perspective. This added complexity highlights the need for LLMs users to gain a deeper understanding of what are ethical designs. Without this knowledge, users may unintentionally integrate unethical designs into their applications. 

This emphasizes the importance of expanding ethical design education to a broader audience, beyond just designers. It is also necessary for LLM products to integrate user-centered design principles into the training and testing while increasing user awareness of potential risks in generated designs.