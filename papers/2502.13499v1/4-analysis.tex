\section{Analysis}

Our analysis hand-annotated all LLM-generated code for the presence/absence of dark patterns and used those counts to calculate statistical measures of difference. The original response for each prompt pair was a single file using HTML and CSS to create a single component of an ecommerce website. Rather than evaluate the code directly, we developed an automated pipeline to compile the code and screenshot the design. While these visual representations can include minor issues (the most common being that LLMs were prompted to use placeholder image URLs which do not compile), they were generally much easier to assess for the presence of dark patterns than the original code.%is built it is . It is this visual representation of the design 

Three independent, trained designers labeled each output for the presence of dark patterns. In addition, drawing on a taxonomy developed in prior work~\cite{a:44}, we labeled six attributes defined in Table~\ref{tab:darkpattern-definitions} for each LLM-generated component design: asymmetric, covert, deceptive, information hiding, restrictive, and disparate treatment. After an initial 30 components were labeled, the designers met to review any points of disagreement or uncertainty. On the basis of this, the schema was slightly updated, and the designers were able to produce labels more consistently. Nevertheless, there continued to be some opportunities for disagreement. For example, in one instance, there was a debate about whether disparate treatment dark patterns could occur in the LLM-generated designs. One designer initially believed that disparate treatment was unlikely because the LLMs generate only one component at a time, lacking distinct groups of users for comparison. However, another designer pointed out examples like discounts offered only to canceling users or first-time customers, which inherently treat different user groups unequally. This example reflected that the interpretation of these attributes can sometimes depend on personal understanding and tolerance. However, we made every effort to maintain a consistent schema %by keeping 
through real-time communication about controversial attributes, %when a component attribute seemed controversial and d
discussing each collectively as a group. The final label for each component was assigned by majority vote. 

Our analysis included both the presence/absence of dark patterns as well as the mechanisms of those dark patterns. To compare the frequency of producing dark patterns across different models and across different stakeholder interests, we used Chi-squared tests for statistical significance. 