%%
%% This is file `sample-manuscript.tex',
%% generated with the docstrip utility.
%%
%% The original source files were:
%%
%% samples.dtx  (with options: `all,proceedings,bibtex,manuscript')
%% 
%% IMPORTANT NOTICE:
%% 
%% For the copyright see the source file.
%% 
%% Any modified versions of this file must be renamed
%% with new filenames distinct from sample-manuscript.tex.
%% 
%% For distribution of the original source see the terms
%% for copying and modification in the file samples.dtx.
%% 
%% This generated file may be distributed as long as the
%% original source files, as listed above, are part of the
%% same distribution. (The sources need not necessarily be
%% in the same archive or directory.)
%%
%%
%% Commands for TeXCount
%TC:macro \cite [option:text,text]
%TC:macro \citep [option:text,text]
%TC:macro \citet [option:text,text]
%TC:envir table 0 1
%TC:envir table* 0 1
%TC:envir tabular [ignore] word
%TC:envir displaymath 0 word
%TC:envir math 0 word
%TC:envir comment 0 0
%%
%% The first command in your LaTeX source must be the \documentclass
%% command.
%%
%% For submission and review of your manuscript please change the
%% command to \documentclass[manuscript, screen, review]{acmart}.
%%
%% When submitting camera ready or to TAPS, please change the command
%% to \documentclass[sigconf]{acmart} or whichever template is required
%% for your publication.
%%
%%
\documentclass[manuscript,screen]{acmart}
%%
%% \BibTeX command to typeset BibTeX logo in the docs
\AtBeginDocument{%
  \providecommand\BibTeX{{%
    Bib\TeX}}}

\usepackage{tikz}
\usepackage{multirow}
%% Rights management information.  This information is sent to you
%% when you complete the rights form.  These commands have SAMPLE
%% values in them; it is your responsibility as an author to replace
%% the commands and values with those provided to you when you
%% complete the rights form.
\setcopyright{acmlicensed}
\copyrightyear{2025}
\acmYear{2025}
\acmDOI{XXXXXXX.XXXXXXX}
%% These commands are for a PROCEEDINGS abstract or paper.
% \acmConference[DIS 25]{Designing Interactive Systems}{June 03--05,
%   2018}{Woodstock, NY}
%%
%%  Uncomment \acmBooktitle if the title of the proceedings is different
%%  from ``Proceedings of ...''!
%%
%%\acmBooktitle{Woodstock '18: ACM Symposium on Neural Gaze Detection,
%%  June 03--05, 2018, Woodstock, NY}
% \acmISBN{978-1-4503-XXXX-X/2018/06}


%%
%% Submission ID.
%% Use this when submitting an article to a sponsored event. You'll
%% receive a unique submission ID from the organizers
%% of the event, and this ID should be used as the parameter to this command.
%%\acmSubmissionID{123-A56-BU3}

%%
%% For managing citations, it is recommended to use bibliography
%% files in BibTeX format.
%%
%% You can then either use BibTeX with the ACM-Reference-Format style,
%% or BibLaTeX with the acmnumeric or acmauthoryear sytles, that include
%% support for advanced citation of software artefact from the
%% biblatex-software package, also separately available on CTAN.
%%
%% Look at the sample-*-biblatex.tex files for templates showcasing
%% the biblatex styles.
%%

%%
%% The majority of ACM publications use numbered citations and
%% references.  The command \citestyle{authoryear} switches to the
%% "author year" style.
%%
%% If you are preparing content for an event
%% sponsored by ACM SIGGRAPH, you must use the "author year" style of
%% citations and references.
%% Uncommenting
%% the next command will enable that style.
%%\citestyle{acmauthoryear}


\usepackage{xcolor}
\usepackage{booktabs}
\usepackage{wrapfig}
\usepackage{enumitem}
\usepackage{makecell}
\usepackage{tikz}
\renewcommand{\cellalign}{l}
\renewcommand{\theadalign}{l}
\newcommand{\filledcircle}[2]{% #1: radius, #2: percentage
    \begin{tikzpicture}[scale=0.8]
        \draw[thick] (0,0) circle (#1); % Outer circle
        \fill[black] (0,0) -- (90:#1) arc (90:{90-(#2*3.6)}:#1) -- cycle; % Start at 90° and fill clockwise
    \end{tikzpicture}
}


%%
%% end of the preamble, start of the body of the document source.
\begin{document}

%%
%% The "title" command has an optional parameter,
%% allowing the author to define a "short title" to be used in page headers.
\title[Hidden Darkness in LLM-Generated Designs]{Hidden Darkness in LLM-Generated Designs: Exploring Dark Patterns in Ecommerce Web Components Generated by LLMs}
%%
%% The "author" command and its associated commands are used to define
%% the authors and their affiliations.
%% Of note is the shared affiliation of the first two authors, and the
%% "authornote" and "authornotemark" commands
%% used to denote shared contribution to the research.
\author{Ziwei Chen}
\email{zich@umich.edu}
\orcid{0009-0007-6137-5390}
\affiliation{%
  \institution{University of Michigan}
  \country{USA}
}

\author{Jiawen Shen}
\affiliation{%
  \institution{University of California San Diego}
  \city{San Diego}
  \state{California}
  \country{USA}}
% \email{jis033@ucsd.edu}

\author{Luna}
\affiliation{%
  \institution{University of California San Diego}
  \city{San Diego}
  \state{California}
  \country{USA}}
% \email{luna@ucsd.edu}

\author{Kristen Vaccaro}
\affiliation{%
  \institution{University of California San Diego}
  \city{San Diego}
  \state{California}
  \country{USA}}
% \email{kv@ucsd.edu}
%%
%% By default, the full list of authors will be used in the page
%% headers. Often, this list is too long, and will overlap
%% other information printed in the page headers. This command allows
%% the author to define a more concise list
%% of authors' names for this purpose.
\renewcommand{\shortauthors}{Chen et al.}

%%
%% The abstract is a short summary of the work to be presented in the
%% article.
\begin{abstract}
Recent work has highlighted the risks of LLM-generated content for a wide range of harmful behaviors, including incorrect and harmful code. In this work, we extend this by studying whether LLM-generated web design contains \textit{dark patterns}. This work evaluated designs of ecommerce web components generated by four popular LLMs: Claude, GPT, Gemini, and Llama. We tested 13 commonly used ecommerce components (e.g., search, product reviews) and used them as prompts to generate a total of 312 components across all models. Over one-third of generated components contain at least one dark pattern. The majority of dark pattern strategies involve hiding crucial information, limiting users' actions, and manipulating them into making decisions through a sense of urgency. Dark patterns are also more frequently produced in components that are related to company interests. These findings highlight the need for interventions to prevent dark patterns during front-end code generation with LLMs and emphasize the importance of expanding ethical design education to a broader audience.
\end{abstract}

% Recent work has highlighted the risks of LLM-generated content for a wide range of harmful behaviors, including incorrect and harmful code. In this work, we extend this by studying whether LLM-generated web design contains problematic design choices, often termed \textit{dark patterns}. This work evaluates designs of ecommerce web components generated by four popular LLMs: Claude, GPT, Gemini, and Llama. Specifically, we manually curate descriptions for 13 commonly used ecommerce components (e.g., search, product reviews, shipping tracking) and use them as prompts to generate a total of 312 components across all models. Over one-third of generated components contain at least one dark pattern, together representing six high-level dark pattern attributes. The majority of dark pattern strategies involve hiding crucial information, limiting users' actions, and manipulating them into making decisions through a sense of urgency. Dark patterns are also more frequently produced in components that are related to service benefits. These findings highlight the need for interventions to prevent dark patterns during front-end code generation with LLMs and emphasize the importance of expanding ethical design education to a broader audience.



%%
%% The code below is generated by the tool at http://dl.acm.org/ccs.cfm.
%% Please copy and paste the code instead of the example below.
%%

\begin{CCSXML}
<ccs2012>
   <concept>
       <concept_id>10003120.10003121.10003129.10011756</concept_id>
       <concept_desc>Human-centered computing~User interface programming</concept_desc>
       <concept_significance>500</concept_significance>
       </concept>
   <concept>
       <concept_id>10003120.10003123.10010860.10010859</concept_id>
       <concept_desc>Human-centered computing~User centered design</concept_desc>
       <concept_significance>500</concept_significance>
       </concept>
   <concept>
       <concept_id>10003456.10003462.10003544.10011709</concept_id>
       <concept_desc>Social and professional topics~Consumer products policy</concept_desc>
       <concept_significance>300</concept_significance>
       </concept>
 </ccs2012>
\end{CCSXML}

\ccsdesc[500]{Human-centered computing~User interface programming}
\ccsdesc[500]{Human-centered computing~User centered design}
\ccsdesc[300]{Social and professional topics~Consumer products policy}

%%
%% Keywords. The author(s) should pick words that accurately describe
%% the work being presented. Separate the keywords with commas.

\keywords{LLM Audit, Dark Patterns, Large Language Models, Design Ethics}

% \received{20 February 2007}
% \received[revised]{12 March 2009}
% \received[accepted]{5 June 2009}

%%
%% This command processes the author and affiliation and title
%% information and builds the first part of the formatted document.
\maketitle

\section{Introduction}

% State of the world (robots for creative activites)
The term ``robot,'' originally signifying `forced labor,' has long been associated with labor and work. Robots have demonstrated their utility in various automated productive and social contexts, where the primary goals are improving productivity, safety, and fostering social interactions with humans~\cite{simoes2022designing, weidemann2021role, honig2018understanding}. However, an increasing number of cases feature using of robots in creative settings. Unlike productive contexts, where the focus is on efficiency and task completion~\cite{arents2022smart}, or social contexts, where communication and trust are prioritized~\cite{nam2020trust, saunderson2019robots}, creative environments prioritize artistic innovation and expression~\cite{hsueh2024counts}. This shift fundamentally alters the dynamics of human-robot interaction, redefining the roles and expectations for both humans and robots.

For instance, robots’ social behaviors are leveraged to support the generation and expression of creative ideas~\cite{hu2021exploring, sandoval2022human, alves2020creativity}, and programmable robotic movements and trajectories are employed to inspire artistic activities such as sketching~\cite{lin2020your}. These studies often engage participants from creative fields who possess limited prior experience with robotics, and are typically conducted in short-term, experimental settings. Consequently, the findings from these studies remain constrained since much can be learned from professional practitioners' experiences to inform system design such as digital fabrication~\cite{hirsch2023nothing}. There is a notable gap in research examining the long-term, active, and practical experience of integrating robotic systems into the creative processes. As a result, the deeper insights into how robots facilitate and shape creative processes, beyond simply augmenting human creativity, remain underexplored. In this study, we aim to better understand the impacts of robots on creative processes and outcomes.

As early as Leonardo da Vinci's 16th century ``Automaton,'' artists have explored the creative affordances of robotic systems~\cite{shanken2002cybernetics, pagliarini2009development, jeon2017robotic}. The artistic creation process typically encompasses various stages, including the exploration of materials and techniques, ongoing experimentation and iteration, and the continual refinement of the artists' insights into their creative subjects~\cite{lewis2023art, sturdee2022state}. Therefore, investigating the artistic process involving robots offers an opportunity to gain deeper insights into robots' creative potential. Robotic art, in particular, provides a compelling case for this exploration.

We define robotic art as artworks that utilize robotic or automated machines to create artistic experiences and tangible artifacts. One example is robotic installation art, in which robots are programmed to follow specific rules that embody the artist’s expression (\autoref{fig:teaser} (a)). Another example is responsive art, in which robots react to their environment, with behaviors that change over time or in response to spectators (\autoref{fig:teaser} (b)). Additionally, there are robotic creators, which possess a degree of agency, allowing them to collaborate with human artists and produce works that extend beyond mere replication of human-created art (\autoref{fig:teaser} (c) and (d)). As such, robotic art becomes a rich case for exploring human-machine interactions in creative contexts. Gaining a deeper understanding of how robots facilitate artistic expression can provide insights for designing computing systems to support creative activities~\cite{gomez2021robot}.

% Therefore, we did...
We draw on semi-structured, in-depth interviews with renowned professional robotic artists to investigate the use of robots in artistic practice. Specifically, our goal is to understand how artistic exploration of robotic systems challenges conventional assumptions about the functions of robots, such as their roles in automating repetitive tasks or serving human needs. We also explore the implications of robots in the artistic process and examine how creativity may emerge within robotic art. To address these interrelated inquiries, our study focuses on the practice of robotic art, posing the research question: \textit{How do robotic artists utilize robots in their artistic practice?} We approach this inquiry through the perspectives and experiences of robotic artists, who creatively design, modify, and repurpose robotic systems for artistic expression and exploration.

% The key findings are...
Our findings highlight the social, material, and temporal dimensions of artists' practices that shape their creativity and artistic outcomes. The creation of robotic art is largely a social process, as artists receive both explicit and implicit feedback through the audience's reactions and reception of their work. Simultaneously, the embodiment and malfunctions inherent to robotic systems drive artistic experimentation. The temporal processes of creation and exhibition, beyond just the final product, further enhance the creative value. Our empirical analysis presents how creativity emerges through the interplay of social, material, and temporal interactions among artists, robots, audiences, and the environment.

% The contributions of this work are...
We make two main contributions to HCI in this study. 
First, we elucidate the interactive mechanisms among key actors---human creators, machines, audiences, and environments---within the practice of robotic art, a topic that remains underexplored in HCI. Our findings reveal the significance of sociality (e.g., interactions between artists and audiences), materiality (e.g., the embodiment and malfunctions of robots), and temporality (e.g., the processes of creation and exhibition) in shaping creative values. We propose that these three facets are central to the creative process and facilitate the emergence of creativity in robotic art.
Second, drawing from the findings, we offer implications for \textit{socially informed}, \textit{material-attentive}, and \textit{process-oriented} creation with computing systems. We suggest leveraging these three aspects to enhance creativity and the creative experience. Specifically, we discuss the value of incorporating implicit audience feedback, designing with technical malfunctions, and focusing on the post-creation process to foster alternative creative experiences with machines~\cite{alter2010designing, juarez2022glitch}.



\input{2-relwork}


% \subsection{Notations}  %% commented out as we do not use them

% The notations used throughout this paper are summarized in Table ~\ref{t:notations}.

% \begin{table}
%     \centering
%     \small % Reduce font size for the table (optional)
%     \begin{tabular}{|l|c|}
%         \hline
%         \textbf{Notation} & \textbf{Description}  \\
%         \hline
%         $X_{\text{tr}}$ & Training set inputs (messages) 
%         \\\hline
%         $y_{\text{tr}}^{\text{gold}}$ & Gold labels for $X_{\text{tr}}$\\
%         \hline
%         $y_{\text{tr}}^{\text{llm}}$ & Synthetic labels for $X_{\text{tr}}$ \\ \hline
%          $X_{\text{val}}$ & Validation set inputs \\
%          \hline
%         $y_{\text{val}}^{\text{gold}}$ & Gold labels for $X_{\text{val}}$ \\
%         \hline
%         $y_{\text{val}}^{\text{llm}}$ & Synthetic labels for $X_{\text{val}}$ \\ \hline
%         $X_{\text{test}}$ & Test set inputs \\
%         \hline
%         $y_{\text{test}}^{\text{gold}}$ & Gold labels for $X_{\text{test}}$ \\
%         \hline
%         $y_{\text{test}}^{\text{llm}}$ & Synthetic labels for $X_{\text{test}}$ \\ \hline
%         $(X, y)_{\text{tr}}^{\text{llm}}$ & Synthetic training data \\
%         \hline
%          $(X, y)_{\text{val}}^{\text{llm}}$ & Synthetic validation data \\  
%          \hline
%     \end{tabular}
%     % ACL style has the caption below the table or figure
%     \caption{Summary of notations used in the paper}
%     \label{t:notations} 
% \end{table}


\subsection{Overview of Scenarios}

%In this study, we
We investigate the role of LLMs in CB detection, focusing on their utility under varying data availability conditions
and under the assumption that direct use of LLMs as a classifier is too expensive due to the high volume of messages to be checked.
%To establish
As a baseline for comparison, we %first
evaluate a scenario in which a
lightweight, BERT-based
classifier is trained exclusively on gold-standard, manually labeled authentic data without %any
LLM involvement.
We then define three additional scenarios with different data availability
and that use LLMs in different ways.
%, each illustrating how LLMs can aid in CB detection depending on the availability and quantity of authentic data.
%The scenarios are as follows.

%To establish a baseline for comparison, in the first scenario, we evaluate a setup that relies exclusively on training a classifier using gold-standard, manually labeled authentic data with no LLM involvement. We then define three other distinct scenarios, each corresponding to
% %% JW: The following is unneccesary vague as the scenarios are more specifically
% %% about the way the synthetic data is used, apart from the zero-shot LLM.
% a unique way LLMs can be integrated into the detection pipeline.
% These scenarios range from directly serving as classifiers to generating synthetic data or labels for training. 

\paragraph{Scenario 1: Baseline}

This scenario represents the ideal situation where sufficient
%manually labeled (
gold-standard data is available for fine-tuning %a classic encoder such as
BERT.
It serves as the benchmark for evaluating the effectiveness of other approaches.
In this setup, no synthetic data or LLMs are involved.
%The system relies entirely on human annotations.
This scenario is feasible if resources such as time, budget and expert annotators are abundant. However, it often proves impractical due to the
%high costs and scalability
challenges of manual labeling.



\paragraph{Scenario 2: LLM as Classifier}  \label{s:m:sc2}

This scenario applies when labeled authentic data is unavailable, and there is no intention to train a separate classifier for CB detection. Instead, an instruction-tuned
LLM is used directly as a classifier, leveraging its pre-trained knowledge and its ability to follow instructions
to identify CB instances.
%This approach is particularly useful in contexts that require rapid deployment or when computational or time resources are limited for training a new model. 
The primary advantage of this method is its elimination of the need for labeled data and training time. However, there are trade-offs. While an LLM can handle nuanced language patterns, it may be less efficient and incur higher computational costs
compared to simpler BERT-based classifiers with a classification head and fine-
tuned on a labeled dataset.
%% JW: add reference to large zero-shot study in NLP
We explore two prompting strategies for generating synthetic labels:
\textit{(a)} guideline-enhanced (GE) prompts, guiding the LLM with detailed labeling instructions and
\textit{(b)} guidelne-free (GF) prompts, allowing the LLM to generate labels without such guidelines.

\paragraph{Scenario 3: Fully Synthetic Data}

In this scenario, only a small set of manually labeled gold data is available for testing, with no access to authentic data for training or validation.
%To address this, we
We
use an LLM to generate a fully synthetic dataset, consisting of both synthetic messages and corresponding labels, for training and validation.
This approach is particularly valuable in low-resource domains or emerging tasks where authentic data is scarce or difficult to collect.
It is especially useful in situations where creating authentic datasets is costly, time-consuming, or ethically challenging, such as annotating harmful or sensitive content or working with vulnerable populations.
The effective

%Commented for indusrty track \subsubsection{Scenario 4: Data Augmentation with Synthetic Data}
%This scenario assumes the availability of a moderate amount of gold-labeled data for training and validation, which may be insufficient to achieve optimal performance. To augment the dataset, we use an LLM to generate additional synthetic data, which is then combined with the gold-labeled data during training and validation. The experiment systematically varies the ratio of synthetic-to-gold data to evaluate its impact on model performance. This scenario explores how LLMs can supplement authentic data, striking a balance between scalability and accuracy.


\paragraph{Scenario 4: Synthetic Labels for Unlabeled Data} \label{s:m:sc4}

This scenario addresses the common situation where resources for manual annotation are limited. Here, gold-standard labeled data is available only for the test set, while a significant amount of unlabeled authentic data is available for training and validation.
%This scenario demonstrates the utility of LLMs in resource-constrained settings, enabling cost-effective dataset creation from unannotated corpora.
To utilize the unlabeled data, we label it using the best prompting strategy (GE or GF) from scenario~2.

% \subsubsection{Summary of Scenarios}
% Table~\ref{t:scenario-summary} presents an overview of the data used in the baseline system and each scenario, specifying the datasets utilized for training, validation, and testing. For Scenario 2, where no classifier is trained and the LLM is used directly as a classifier, only the test set is included.
% \begin{table}
%     \centering
%     \small % Reduce font size for the table (optional)
%     \begin{tabularx}{\columnwidth}{|X|X|X|X|}
%         \hline
%         \textbf{Scenario} & \textbf{Train} & \textbf{Validation} & \textbf{Test} \\
%         \hline
%          1 & $X_{\text{tr}}, y_{\text{tr}}^{\text{gold}}$ & $X_{\text{val}}, y_{\text{val}}^{\text{gold}}$ &  $X_{\text{test}}, y_{\text{test}}^{\text{gold}}$ \\
%         \hline
%            2 & - & - & $X_{\text{test}}, y_{\text{test}}^{\text{gold}}$ \\
%         \hline
%           3 & $(X, y)_{\text{tr}}^{\text{llm}}$ & $(X, y)_{\text{val}}^{\text{llm}}$ & $X_{\text{test}}, y_{\text{test}}^{\text{gold}}$ \\
%         \hline
%          4 & $X_{\text{tr}}, y_{\text{tr}}^{\text{llm}}$ & $X_{\text{val}}, y_{\text{val}}^{\text{llm}}$ &  $X_{\text{test}}, y_{\text{test}}^{\text{gold}}$ \\ \hline
       
%         % 4 & $X_{\text{tr}}, y_{\text{tr}}^{\text{llm}} + X_{\text{tr}}, y_{\text{tr}}^{\text{gold}}$ & $X_{\text{val}}, y_{\text{val}}^{\text{llm}}+ X_{\text{val}}, y_{\text{val}}^{\text{gold}}$ & $X_{\text{test}}, y_{\text{test}}^{\text{gold}}$ \\ \hline
%         % 4 & $(X, y)_{\text{tr}}^{\text{llm}} + X_{\text{tr}}, y_{\text{tr}}^{\text{gold}}$ & $(X, y)_{\text{val}}^{\text{llm}}+ X_{\text{val}}, y_{\text{val}}^{\text{gold}}$ & $X_{\text{test}}, y_{\text{test}}^{\text{gold}}$ \\ \hline
%     \end{tabularx}
%     % ACL style has the caption below the table or figure
%     \caption{Overview of data used in each scenario}
% \label{t:scenario-summary} 
% \end{table}





% \subsection{Intrinsic Evaluation Metrics}

% Intrinsic evaluation examines the inherent qualities of datasets, enabling the assessment of linguistic diversity, emotional tone, and conversational structure independently from task-specific performance. For our CB detection task, we utilize \textbf{four} categories of intrinsic metrics to compare the authentic dataset with LLM-generated synthetic data. These categories are: 1) lexical and linguistic characteristics, %including metrics such as Mean Words per Message, Mean Word Length, and Type-Token Ratio; 
% 2) content and CB indicators, 
% %such as rate of Harmful Messages, Bully Messages, Victim Messages, and Toxicity; 
% 3) sentiment and emotional tone, 
% %which classifies messages into negative, positive, or neutral; 
% and 4) dialogue act distribution.
% %categorizing messages into types such as Question, Statement, Greeting, Accept/Reject, and Other. 
% These categories are critical for understanding the fundamental differences between authentic and synthetic data in the context of CB detection, as they provide insight into how well the synthetic data replicates the linguistic, emotional, and conversational behaviors that are typically present in real-world online interactions.

% To ensure a fair comparison between the authentic and synthetic datasets, we first normalize both dataset by employing pre-processing techniques such as tokenization using NLTK \cite{loper-bird-2002-nltk} and punctuation handling. Additionally, data is segmented into equal-sized token slices to account for metrics that are influenced by corpus size.

% Sentiment scores are measured using VADER \cite{hutto2014vader}, a sentiment analysis tool optimized for short social media texts. Dialogue acts are classified using a Naive Bayes model trained on the NLTK \texttt{nps-chat} corpus,
% following \newcite[Chp.~6, Sec.~2.2]{bird2009natural}.\footnote{
%     While no citation is provided by \newcite{bird2009natural}, the source
%     of this corpus seems to be
%     \newcite{forsyth-martell-2007-lexical,forsyth-etal-2010-nps}.
% }
%



% Natural Language Processing with Python, by Steven Bird, Ewan Klein and Edward Loper
% Chapter 6, section 2.2 "Identifying Dialogue Act Types"
% refers to Chapter 2, section 1.2 "Web and Chat Text", for the
% NPS Chat Corpus but provides no source or citation.
%  
% An unrelated 2011 paper cites an "NPS Chat Corpus of North American English chat
% conversations (Forsyth and Martell 2007)".
%   * Forsyth, Eric. M. and Craig H. Martell (2007), Lexical and discourse analysis
%     of online chat dialog, Proceedings of the First IEEE International Conference
%     on Semantic Computing (ICSC) 2007, pp. 19–26.
%   * data collected in 2006
%   * approximately 500,000 chat posts gathered from various online services
%   * 10,567 posts tagged in Release 1.0
%   * available on http://faculty.nps.edu/cmartell/NPSChat.htm (page no longer
%     exists but is archived, e.g. on
%     http://web.archive.org/web/20190510121556/http://faculty.nps.edu/cmartell/NPSChat.htm
%        - "If you want just the data, you can get it through the Linguistic Data
%          Consortium.  It is catalog number LDC2010T05."
%        - This page asked for the 2007 paper above to be cited "when referring to
%          the NPS Chat Corpus".
%
% There is a 2010 thesis from Naval Postgraduate School, Monterey, California, by
% J. R. Hitt entitled "Implementation and Performance exploration of a cross-genre
% part of speech tagging methodology to determine dialog act tags in the chat
% domain".
%   * credits Lin and Forsyth
%


% Type-Token Ratio (TTR), which is calculated by dividing the number of unique words by the total tokens in fixed-size slices, serves as a normalized measure of vocabulary diversity. Toxicity scores, which represent the ratio of messages containing profanity, are derived using a publicly available profanity list \cite{surge2023profanity}.


\subsection{Evaluation Metrics}

We choose accuracy of label prediction for development decisions and reporting since the labels are reasonably balanced in the authentic test data with 30.3\% items labeled with the minority
label.\footnote{In the appendix, we further report macro average F1 scores that are also widely used in the area of harmful content detection.}
In scenarios 1, 3 and 4,
we train BERT\_base\_uncased \cite{devlin-etal-2019-bert}, a 110M parameter transformer model, with a linear classification head
% using
% the HuggingFace transformers library \cite{wolf-etal-2020-huggingface}
to detect harm, assigning binary labels to text messages.
To address noise from randomness in training, we train at least 45 models for each setting and report average accuracy and standard deviation.
\section{Analysis of Ability Degradation}
\label{analysis}

ROME~\cite{DBLP:conf/nips/MengBAB22} and MEMIT~\cite{DBLP:conf/iclr/MengSABB23} are 
 currently popular model editing methods. Given that MEMIT builds upon the foundations of ROME by implementing residual distribution across multiple layers,  our analysis in the main text focuses primarily on ROME. 
This section presents a detailed analysis of how the model is affected during sequential editing using ROME. 
Statistical and visual analyses reveal that the degradation of general abilities is related to the unintentional introduction of the non-trivial noise that can make the parameter matrix after editing deviate from its original semantics space.

\subsection{Comparison with Fine-tuning Approach}
First, a statistical analysis is conducted by editing GPT2-XL~\cite{radford2019language} using the ZsRE~\cite{DBLP:conf/conll/LevySCZ17} dataset. 
Considering the L1 norm effectively quantifies the absolute changes in parameter values pre- and post-editing, while providing insights into feature weight distributions within the matrix, it is used to represent the degree of change in the parameter matrix.
As illustrated in Figure~\ref{fig-edit}, when using editing-based methods such as ROME and MEMIT, the L1 norm of the matrix at the edited layer increases significantly with the number of edits.
It can be seen that the norm increases by 317\% (ROME) and 61\% (MEMIT), respectively by the end of sequential editing.
This result highlights a significant deviation from the unedited model, emphasizing the impact of sequential edits on stability.

\begin{figure}[t]
  \subfigure[Editing-based methods]{
  \includegraphics[width=0.22\textwidth]{figures/L1-Norm-GPT2XL.pdf}
  \label{fig-edit}}
  \subfigure[Fine-tuning approach]{
  \includegraphics[width=0.22\textwidth]{figures/finetune-GPT2XL.pdf}
  \label{fig-finetune}}
\vspace{-2mm}
\caption{Illustration of the change of L1 norm 
(a) in sequential editing at the edited layer using editing-based methods and 
(b) in fine-tuning different batch steps for selected layers.
Here we uniformly selected the layers of GPT2-XL for clarity when fine-tuning.} 
\vspace{-2mm}
\end{figure}

A gradient-based fine-tuning approach can markedly enhance the performance of the model on specific tasks while preserving its general abilities across other downstream tasks~\cite{DBLP:journals/corr/abs-2312-12740, DBLP:journals/corr/abs-2310-10047}. 
As depicted in Figure~\ref{fig-finetune}, there are no significant changes in the norm of the parameter matrix for the given layers, with a maximum change of only 0.27\%, even as the amount of fine-tuning knowledge increases. This stability in the parameter matrix norms suggests that the fine-tuning approach does not introduce significant non-trivial noise during the editing process. Thus, fine-tuning maintains the integrity and stability of the model's parameters, which is crucial for preserving its general abilities and preventing unintended non-trivial noise.

This comparison indicates that each editing method not only updates the intended fact as expected but also unintentionally introduces non-trivial noise into the model. This noise manifests as deviations in the parameter matrix during the sequential editing process. With each additional edit, the noise accumulates, progressively increasing the deviation in the parameter matrix. Consequently, as the number of edits grows, there is a significant deviation in the parameter matrix observed before and after the editing. This accumulated noise highlights the challenge of maintaining the stability and integrity of the parameter matrix through multiple edits, which can ultimately impact the general abilities of the model.

\begin{figure}[t]%[!htb]
  \subfigure{
  \includegraphics[width=0.45\textwidth]{figures/legend_visible.pdf}}
  \subfigure{
  \includegraphics[width=0.22\textwidth]{figures/pca-gpt2-right.pdf}}
  \subfigure{
  \includegraphics[width=0.22\textwidth]{figures/pca-llama3-right.pdf}}
\vspace{-2mm}
\caption{Visialization of six sets of facts recalled by LLMs using 2-dimensional PCA. Note that this hidden state is also projected by a language modeling head (linear mapping) for next-token prediction, implying the linear structure in the corresponding representation space (the PCA assumption).} 
\vspace{-3mm}
\label{fig-pca}
\end{figure}

\subsection{Visualization Analysis} \label{sec_visual}
Following the ROME, the second layer of MLP \( W^{(l)}_{\text{proj}} \) is viewed as a linear associative memory~\cite{anderson1972simple, DBLP:journals/tc/Kohonen72}. This perspective observes that any linear operation \( W \) can operate as a key-value store for a set of vector keys \( K = [\mathbf{k}_1 | \mathbf{k}_2 | \dots] \) and corresponding vector values \( V = [\mathbf{v}_1 | \mathbf{v}_2 | \dots] \), by solving \( WK \approx V \)~\cite{DBLP:conf/nips/MengBAB22}.
The key-value pair \( (\mathbf{k}_i, \mathbf{v}_i) \) represents the representation of the input prompt, where $\mathbf{k}_i$ identifies patterns of the input and $\mathbf{v}_i$ is the fact recalled by the model, which is considered to gather all the information about how the model understands the prompt and how it will respond. By stacking $\mathbf{k}_i$ and $\mathbf{v}_i$ separately for each prompt, matrix \( K \) and \( V \) are obtained.
Based on this, 200 prompts of the same downstream task are collected to compute \( K \) and \( V \). 

On GPT2-XL and LLaMA-3 (8B), Principal Component Analysis is employed to visualize the hidden state of the facts of the downstream task recalled by the model. 
The first two principal components of six sets of facts, representing most features, are computed~\cite{zheng2024prompt}. Two of these are derived from recalling the model before editing and the model after editing without any constraint, respectively.
To explore the relationship between the deviation of the parameter matrix after editing and the resulting degradation of general abilities, four additional settings were tested by setting different percentages of the columns in the update matrix to zero, evenly distributed according to an arithmetic progression.

As illustrated in Figure~\ref{fig-pca}, the principal
components of facts recalled by the original model and the edited model without any constraint can be largely distinguished, whose boundaries (black dashed lines) can be easily fitted using logistic regression.
This indicates a significant semantic discrepancy between the facts recalled by the unconstrained edited model and the original model, explaining the decline in general abilities is related to matrix deviation.
Furthermore, when the deviation of the parameter matrix is constrained by reducing the norm of the update matrix, as shown by the black arrows, the principal components of the recalled facts by the edited model gradually align with those of the original model.
This shows that by reducing the norm of the update matrix, the deviation of the parameter matrix after editing can be constrained, making the semantic distribution of the model before and after editing similar, thereby preserving the general abilities of the edited model.
\section{Results and Discussion}
\label{sec05}

In this section, we present the results, discuss them, and make some conclusions about the experiments.

With a slightly realistic scenario, the experiments present some interesting results. Figures \ref{fig:hist_score} and \ref{fig:hist_gen} show respectively histograms of (a) the final score after the full training process and (b) the number of generations the process took. Notice that most runs just stopped at 20 generations (maximum) and could not improve further, as Figure \ref{fig:hist_gen} suggests. Despite that, as can be seen in Fig. \ref{fig:hist_score}, more than 80\% of the runs ended with a score of 2 or less, meaning at most two wrong device activations on 260 interactions. 

\begin{figure*}
        \centering
        \begin{subfigure}[b]{0.475\textwidth}
            \centering
            \includegraphics[width=0.8\textwidth]{imgs/results/results_2/histogram_of_score_.png}
            \caption[]%
            {{\small Histogram of \textit{score} achieved on experiment runs. Notice that the lesser, the better.}}    
            \label{fig:hist_score}
        \end{subfigure}
        \hfill
        \begin{subfigure}[b]{0.475\textwidth}  
            \centering 
            \includegraphics[width=0.8\textwidth]{imgs/results/results_2/histogram_of_generations_.png}
            \caption[]%
            {{\small Histogram of \textit{generations} needed to achieve the lower score. Here, most runs needed the maximum number of generations}}    
            \label{fig:hist_gen}
        \end{subfigure}
        \caption[]
        {\small Histograms of the lowest score and generations needed to achieve that on each experiment run.} 
        \label{fig:hist_metrics}
\end{figure*}

Figures \ref{fig:hist_beh} and \ref{fig:hist_per} reflect the number of Behavioral and Perceptual Codelets respectively to achieve the best result in each run. As we can see in Fig. \ref{fig:hist_beh}, most runs needed 13 Behavioral Codelets, one for each Motor Codelet/actuation device. Fig \ref{fig:corr_behavior} shows the correlation between the number of behavioral codelets and score. The system response is better (lower score) as more Behavioral Codelets are used.
The number of Perceptual Codelets, on the other hand, shows approximate normal distributions with a slight bias to the right, meaning that the embedding may vary and the output still be good. This bias is reflected in the slight negative correlation between the number of Perceptual Codelets and Score (smaller than Behavioral).

\begin{figure*}
        \centering
        \begin{subfigure}[b]{0.475\textwidth}
            \centering
            \includegraphics[width=0.8\textwidth]{imgs/results/results_2/histogram_of_behaviorals_x.png}
            \caption[]%
            {{\small Histogram of the number of Behavioral Codelets needed to achieve the lowest score. Most runs needed 13, the same number of Motor Codelets (and actuators).}}    
            \label{fig:hist_beh}
        \end{subfigure}
        \hfill
        \begin{subfigure}[b]{0.475\textwidth}  
            \centering 
            \includegraphics[width=0.8\textwidth]{imgs/results/results_2/histogram_of_perceptuals_x.png}
            \caption[]%
            {{\small Histogram of the number of Perceptual Codelets needed to achieve the lowest score.}}    
            \label{fig:hist_per}
        \end{subfigure}
        \caption[]
        {\small Histogram of the number of ''internal'' Codelets needed to achieve the lowest score on each run.} 
        \label{fig:needed_codelets_2}
\end{figure*}


\begin{figure*}
        \centering
        \begin{subfigure}[b]{0.475\textwidth}
            \centering
            \includegraphics[width=0.8\textwidth]{imgs/results/results_2/correlation_correlation_of_score_with_behaviorals.png}
            \caption[]%
            {{\small Correlation between number of Behavioral Codelets and Score}}    
            \label{fig:corr_behavior}
        \end{subfigure}
        \hfill
        \begin{subfigure}[b]{0.475\textwidth}  
            \centering 
            \includegraphics[width=0.8\textwidth]{imgs/results/results_2/correlation_correlation_of_score_with_perceptuals.png}
            \caption[]%
            {{\small Correlation between number of Perceptual Codelets and Score}}    
            \label{fig:corr_per}
        \end{subfigure}
        \caption[]
        {\small Correlation between the number of ''internal'' Codelets and Score} 
        \label{fig:corr}
\end{figure*}



Table \ref{table:exp2} shows some statistics taken from the experiment. Notice that, while the individual number of Perceptual and Behavioral Codelets may go as low as 2, the combined ''Internal'' Codelets need a higher number to present satisfactory results.

\begin{table}[]
\centering
\caption{metrics on Experiments}
\label{table:exp2}
\begin{tabular}{@{}llllll@{}}
\toprule
              & mean    & median & std   & min & max \\ \midrule
score         & 1.438   & 1.0    & 1.894 & 0   & 26  \\
generations   & 13.6784 & 20.0   & 8.905 & 0   & 20  \\
n perceptuals & 9.5326  & 10.0   & 2.213 & 2   & 15  \\
n behaviorals & 11.2674 & 13.0   & 2.478 & 2   & 13  \\
n internals   & 20.8    & 22.0   & 3.998 & 7   & 28  \\ \bottomrule
\end{tabular}
\end{table}

\subsection{Conclusion and Future Works}

This paper has presented a pioneering approach to creating a Cognitive Twin by leveraging a distributed cognitive system in conjunction with an evolution strategy. Our work stands as a significant contribution to the field of cognitive computing by demonstrating the feasibility of orchestrating a multitude of simple physical and virtual devices to mimic a person's interaction behaviors. This achievement not only offers a practical application of distributed cognitive systems but also introduces a novel methodology for cognitive twin development, emphasizing the role of evolution strategies in optimizing system topology for more accurate behavior emulation.

In revisiting the themes introduced at the outset, our research seamlessly integrates the foundational principles of cognitive systems, Cyber-Physical Systems (CPS), and Systems of Systems (SoS) with contemporary advancements in artificial intelligence. By doing so, we have illustrated a comprehensive framework that not only addresses the complexities of human behavior simulation but also opens new avenues for automation, human-like agent creation, and in-depth behavioral analysis.

Comparatively, our approach distinguishes itself from established cognitive architectures such as ACT-R and SOAR, and the Standard Model of Mind, by emphasizing distributed processing and adaptability. While ACT-R and SOAR offer rich insights into cognitive processes through detailed psychological models, our model excels in harnessing distributed, interconnected devices to capture the multifaceted nature of human cognition. Similarly, the Standard Model of Mind provides a foundational framework for understanding cognitive functions. Yet, our work extends this understanding into the practical domain of CPS and distributed systems, offering a unique perspective on cognitive replication and interaction dynamics.

In conclusion, our research not only underscores the potential of distributed cognitive systems in creating sophisticated cognitive twins but also highlights the importance of evolutionary strategies in refining these systems. By drawing parallels and distinguishing our work from established cognitive architectures like ACT-R, SOAR, and the Standard Model of Mind, we contribute a novel perspective to the ongoing discourse on cognitive modeling and simulation. 



Future work will focus on further refining the distributed cognitive system and exploring its integration with other AI paradigms and models. This research sets the stage for developing more sophisticated Cognitive Twins capable of performing complex tasks with minimal human intervention. By continuing to build on this foundation, future studies can enhance the fidelity and applicability of Cognitive Twins, making them tools in the field of cognitive computing.
    

\section{Discussion}
\label{sec:discussion}
In this work, we propose to leverage few-shot learning to enable users to self-define personal undesirable actions for personalized intervention on smartwatches.
We developed a three-stage pipeline that began with a self-supervised, pre-trained IMU model for robust feature extraction, then fine-tuned it for accurate human activity recognition, and finally enhanced it with data augmentation and synthesis that enabled rapid customization of new user-defined actions using only a small number of examples. 
We implemented this pipeline on a smartwatch as a real-time intervention system, \projectname, and demonstrated its effectiveness and advantages over the rule-based method through a multi-hour user study.
In this section, we discuss some interesting takeaways from our study, together with our vision of how \projectname can be generally applied to other health domains. We also briefly summarize the limitations of our work.


\subsection{Distorted Perception with AI-powered Intervention}
\label{sub:discussion:distorted}
During the study, we observed an interesting phenomenon where some participants developed a distorted perception towards their own actions or the intervention (see Sec.~\ref{sub:intervention_evaluation:qualitative_results}).
For instance, several participants felt \projectname's vibrations were stronger than the baseline (yet the actual strength of vibration remained constant), and some felt they did the target actions more frequently with \projectname (yet the objective data indicated otherwise).
There are several potential interpretations of such interesting observations.
The distorted perception might be caused by participants' heightened awareness of the AI-guided interventions: because \projectname more accurately and promptly caught the target actions, users started to pay extra and prolonged attention to any intervention. This could leave a stronger impression on them, and subsequently, they found it stronger or more frequent.
Another potential explanation is that the participants, often associating their personal and idiosyncratic undesirable actions with ``wrong-doing'' and thus responding with negative emotions, might have subconsciously perceived their undesirable actions as being more frequent due to the \projectname's more precise and timely feedback eliciting stronger negative emotions. This, combined with an emotional interpretation of being 'corrected', may have amplified their perception of the intervention's intensity (vibration strength) and created the mistaken impression of performing these actions excessively.

Meanwhile, it is an interesting open question of how long such perception will last from a longitudinal intervention perspective. Depending on the cases, the growing self-awareness and/or reliability of AI could potentially assist users in building a long-term habit to reduce the target action, or on the contrary, the effect may fade away due to the AI intervention method no longer being novel or enticing.
Future work can explore the lasting effect of the intervention, the corresponding perception, as well as user engagement in a long-term, field-based intervention study.~\cite{middleton2013long, short2018measuring, wei2020design}.


\subsection{Towards Human-AI Collaborative Interventions}
\label{sub:discussion:collaboration}
Users' mental models of \projectname varied significantly. Some viewed it as a passive watchdog, and some viewed it as a playful interactive system, while others sought to take greater agency in the moment of intervention delivery.
Our findings show the potential for and benefit of developing a collaborative relationship between humans and AI for behavioral intervention.
An AI system can provide appropriate support to users and help them achieve effective intervention outcomes.
Such collaboration is closely relevant to the vision of just-in-time adaptive interventions (JITAIs)~\cite{nahum-shani_translating_2021, nahum2018just}, where the delivery timing and methods of intervention are designed to be dynamically adapting to an individual's internal state and surrounding context.

For instance, for users who see the system as a passive monitor, the system can provide the option for them to configure the frequency and style of intervention (\eg higher/lower-intensity vibrations or consolidated notifications), ensuring the AI remains in the background but still provides supportive nudges.
Taking one step further, the AI system may analyze user behavior over time and suggest new setups or goals for users with transparency (\eg transitioning from nail-biting to managing stress). Users can accept, modify, or reject these suggestions, creating a dialogue where AI acts as a coach or collaborator rather than a rigid enforcer of predefined behaviors.
Meanwhile, for those who see AI as a proactive system, one promising avenue is to incorporate user feedback into the AI's learning process~\cite{orzikulova2024time2stop}. Users can label the AI's predictions as accurate or not, which could serve as input for the model to further adapt to the user and improve performance over time (\eg through reinforcement learning).
Combined with contextual information that can potentially be inferred from sensors~\cite{xu2023globem}, such feedback can enable more precise, context-sensitive and personalized JITIs.
In addition, the system would periodically prompt users to reassess their goals and update intervention targets, ensuring long-term relevance and efficacy.

It is noteworthy that such a human-AI collaboration paradigm needs to follow the principles of transparency and ethical design.
Other than the options mentioned above, namely custom configurations and continuous feedback, users should have visibility into the system's functionality and action logic regardless of the specific collaboration setup. This is important to provide users with agency and build their trust in the system.

\subsection{Beyond Smartwatch and Broader Customization}
In this work, our real-time intervention was implemented on a smartwatch. However, our proposed idea of empowering users to define any personal action and achieve a personalized intervention system can be more broadly applied to other domains.
Instead of relying solely on a watch-based IMU, we can explore other body-based sensor arrays (\eg headbands, rings, or joint sensors) to capture a more diverse range of behaviors in real time.
This would enable the system to accommodate a wide variety of undesirable actions or habits, such as posture corrections and fidgeting management.
In addition, beyond physical interventions, future customization can also delve into psychological or mental health support.
For instance, individuals dealing with obsessive-compulsive disorder (OCD) or other habitual thought/action patterns could define personal triggers (\eg a particular repetitive motion or behavioral cue) and receive timely AI-driven interventions.
Such holistic approaches highlight the flexibility and scalability of our pipeline.
By enabling user-defined actions, we open up possibilities for long-term and effective management of both physical and psychological well-being using a multitude of wearable and sensor-based platforms.

\subsection{Limitations}

Despite \projectname's positive outcome and the promising insights generated, we recognize some limitations in our study design.
As mentioned above, our current model relies solely on accelerometer data for action recognition, which may limit its ability to capture the full range of motion characteristics or other physiology. Future work can explore additional sensing modalities, such as gyroscope, photoplethysmography (PPG), joint locations, to enhance the accuracy and robustness of action recognition. 
Besides, the study was conducted with a relatively small number of participants and a limited set of actions, which may not fully capture the variability and diversity of human activities in real-world scenarios \cite{trapp2015individual, narayanan2013behavioral}.
Additionally, although we tried to simulate real-life scenarios, our intervention study was conducted over a limited duration and in controlled experimental settings, which may not fully reflect the complexities and dynamics of real-life environments. 
Real-world contexts introduce factors such as environmental noise, varying sensor placements, and user behavior changes over time \cite{trapp2015individual,truong2015deployment,mejia2023enhancing,mills2022development}, which were not thoroughly simulated in this study. Future research should conduct longitudinal field experiments with real-world deployment of the system.




\section{Conclusion}\label{sec:conclusion}

In this paper, we proposed a prototype ASL generation system aimed at improving the naturalness, comprehensiveness, and overall quality of generated signs, addressing key limitations in existing approaches. Our technical evaluations indicate that our proposed approaches improve these aspects, enhancing the quality of generated ASL content. Feedback from DHH participants was mixed; while there was general interest in the system, concerns regarding visual quality and naturalness were noted. Reflecting on our design process and study findings, we discuss key insights and identify key areas for future improvement. While further work is needed, our study takes an initial step toward developing sign language generation systems that better meet the needs of the DHH and signing communities, offering real-world value.



%%
%% The next two lines define the bibliography style to be used, and
%% the bibliography file.
\bibliographystyle{ACM-Reference-Format}
\bibliography{reference}


\end{document}
\endinput
%%
%% End of file `sample-manuscript.tex'.
