\begin{figure}[]
  \centering
  \includegraphics[width=\linewidth]{Figures/widget_example.png}
  \Description{This image shows six attributes of dark patterns discovered in LLM-generate designs with two examples for each: Information Hiding (e.g., concealing review reasons, hiding terms), Asymmetric (e.g., default costly memberships, visually imbalanced buttons), Restrictive (e.g., no close buttons, restricted shopping actions), Disparate Treatment (e.g., discounts only for new or canceling users), Covert (e.g., urgency through timers or limited-time offers), and Deceptive (e.g., unclear add-on prices, misleading price highlights). These examples showcase how interfaces can manipulate user behavior in subtle or overt ways.}
  \caption{Examples of components identified with six dark pattern attributes.}
  \label{fig:widget_example}
\end{figure}

\section{Results}

In the 312 components generated by all four LLMs, 115 (37\%) were found to have at least one dark pattern. We found at least one example of every dark pattern attribute proposed by Mathur et al.~\cite{a:44} and provided an example of each in Figure~\ref{fig:widget_example}. Further analysis revealed which models, interests, and components of an ecommerce pipeline are most likely to generate dark patterns. We also separated the analysis of specific dark pattern attributes~\cite{a:44}, identifying which dark patterns are most frequently produced by LLMs. 

\subsection{Comparing Models: CodeLlama Produced Fewer Dark Patterns Than Other LLMs}


\setlength{\intextsep}{0pt}
\begin{wraptable}[8]{r}{4.2cm}
\begin{tabular}{l l l}\toprule  
Model & \multicolumn{2}{l}{\# Dark Patterns} \\\midrule
CodeLlama& 22 & \textit{28.2\%}\\
Gemini & 30 & \textit{38.5\%}\\
GPT & 31 & \textit{39.7\%}\\
Claude & 32 & \textit{41.0\%}\\
\bottomrule
\end{tabular}
\caption{Frequency of Dark Patterns for Different LLMs.}\label{tab:results-models}
\end{wraptable}
When comparing the number of generated components with dark patterns across different models, we found some models generated fewer dark patterns than others. In particular, CodeLlama-34b-Instruct generated only 22 (28.2\%) components with dark patterns, while the other three models produced similar larger numbers of dark patterns: Gemini-2.0-flash-exp created 30 components with dark patterns (38.5\%), GPT-4o generated 31 (39.7\%), and Claude 3.5 Sonnet produced 32 (41\%). % All results are included in Table~\ref{tab:results-models}.

We conducted a full four-way Chi-square test to evaluate whether the differences in the number of components with dark patterns generated by the four models were statistically significant. However, the test result revealed that while there are variations in the frequency of dark patterns across the models, these differences are not statistically significant ($\chi^2=3.46$, $p=0.33$). 

% Overall, the models tend to generate more dark pattern designs in the attributes of information hiding, restrictive, and covert (see Table \ref{tab:attribute_model}). Among the four models, Claude 3.5 Sonnet produced the widest variety of attributes (5 types), while CodeLlama-34B-Instruct generated the fewest (3 types). Each model also has attributes they are less likely to generate. For example: (1) Claude 3.5 Sonnet was less likely to produce designs with the deceptive attribute; (2) GPT-4o performed well in generating asymmetric and disparate treatment attributes; (3) Gemini-2.0-flash-exp produced fewer designs with deceptive and restrictive attributes; and (4) CodeLlama-34B-Instruct excelled in generating asymmetric, covert, and deceptive attributes.

% \begin{table}[h]
% \caption{Distribution of Dark Pattern Attributes by Model}
%   \begin{tabular}{p{3.3cm} p{1.4cm} p{1.4cm} p{1.5cm} p{2cm} p{2cm} p{1.4cm}} 
%     \toprule
%     {Model Name}&Asymmetric & Covert & Deceptive & Disparate Treatment & Information Hiding & Restrictive\\
%     \midrule
%      Claude 3.5 Sonnet & 9 & 13 & 0 & 7 & 7 & 9\\  
%     GPT-4o & 0 & 7 & 3 & 0 & 22 & 17\\
%     Gemini-2.0-flash-exp & 6 & 11 & 0 & 13 & 19 & 2\\
%     CodeLlama-34b-Instruct & 0 & 0 & 0 & 1 & 12 & 15\\
%     Total & 15 & 31 & 3 & 20 & 60 & 43 \\
%   \bottomrule
% \end{tabular}
%   \label{tab:attribute_model}
% \end{table}

\subsection{Comparing Ecommerce Components: Sales, Discount, and Membership Components Were Most Likely to Include Dark Patterns}


% \begin{figure}[b]
%   \centering
%   \includegraphics[width=\linewidth]{Figures/by_widget_type.png}
%   \caption{The number of components with dark patterns identified across different ecommerce components.}
%   \Description{The figure shows the distribution of widgets with dark patterns across different widget types. From highest to lowest, the counts are as follows: newsletter (22), discount (21), membership (18), banner (16), unsubscribe (11), reviews (10), checkout (8), details (3), login (2), cart (2), featured (2), and both tracking and search with 0.}
%   \label{fig:results-components}
% \end{figure}

%By analyzing the counts of widgets with dark patterns across 13 widget types,
Comparing the frequency of dark patterns across the different ecommerce components showed that some components are much more likely to be generated with dark patterns than others, as shown in Figure~\ref{fig:results-components}. Across 24 generation attempts for each type of ecommerce component, newsletter signup components contained dark patterns 92\% of the time, followed by discount offers (88\%), membership signups (75\%), and sales banners (67\%). On the other hand, certain ecommerce components, such as product details, featured product, user logins, % or registration, 
shopping carts, order tracking, and product search, rarely or never contained dark patterns.

% \begin{wrapfigure}{r}{1\textwidth}
%   \begin{center}
%     \includegraphics[width=0.95\textwidth]{Figures/by_widget_type.png}
%   \end{center}
%   \caption{Birds}
% \end{wrapfigure}

\begin{figure}[]
  \centering
  \includegraphics[width=\linewidth]{Figures/by_widget_type.png}
  \Description{The figure shows the distribution of widgets with dark patterns across different widget types. From highest to lowest, the counts are as follows: newsletter (22), discount (21), membership (18), banner (16), unsubscribe (11), reviews (10), checkout (8), details (3), login (2), cart (2), featured (2), and both tracking and search with 0.}
  \setlength{\belowcaptionskip}{-10pt}
  \caption{The number of components with dark patterns identified across different ecommerce components.}
  \label{fig:results-components}
\end{figure}

\setlength{\intextsep}{0pt}
\begin{wraptable}{r}{8cm} % Wrapped table aligned to the right
\begin{tabular}{p{2cm} p{0.5cm} p{0.5cm} p{0.5cm} p{0.5cm} p{0.5cm} p{0.5cm}} 
\toprule
\textbf{Component} & 
\rotatebox{90}{\textbf{Asymmetric}} & 
\rotatebox{90}{\textbf{Covert}} & 
\rotatebox{90}{\textbf{Deceptive}} & 
\rotatebox{90}{\textbf{Info Hiding}} & 
\rotatebox{90}{\textbf{Restrictive}} & 
\rotatebox{90}{\textbf{Disparate}} \\ \hline \rule{0pt}{1.2em}
Newsletter & \filledcircle{0.15cm}{0} & \filledcircle{0.15cm}{0} & \filledcircle{0.15cm}{0} & \filledcircle{0.15cm}{82} & \filledcircle{0.15cm}{77} & \filledcircle{0.15cm}{27}\\
Discount & \filledcircle{0.15cm}{5} & \filledcircle{0.15cm}{67} & \filledcircle{0.15cm}{0} & \filledcircle{0.15cm}{38} & \filledcircle{0.15cm}{38} & \filledcircle{0.15cm}{5}\\
Membership & \filledcircle{0.15cm}{17} & \filledcircle{0.15cm}{0} & \filledcircle{0.15cm}{0} & \filledcircle{0.15cm}{72} & \filledcircle{0.15cm}{78} & \filledcircle{0.15cm}{0}\\
Banner & \filledcircle{0.15cm}{0} & \filledcircle{0.15cm}{88} & \filledcircle{0.15cm}{0} & \filledcircle{0.15cm}{69} & \filledcircle{0.15cm}{6} & \filledcircle{0.15cm}{0}\\
Unsubscribe & \filledcircle{0.15cm}{64} & \filledcircle{0.15cm}{27} & \filledcircle{0.15cm}{0} & \filledcircle{0.15cm}{0} & \filledcircle{0.15cm}{27} & \filledcircle{0.15cm}{27}\\
Reviews & \filledcircle{0.15cm}{0} & \filledcircle{0.15cm}{0} & \filledcircle{0.15cm}{0} & \filledcircle{0.15cm}{100} & \filledcircle{0.15cm}{0} & \filledcircle{0.15cm}{0}\\
Checkout & \filledcircle{0.15cm}{38} & \filledcircle{0.15cm}{0} & \filledcircle{0.15cm}{25} & \filledcircle{0.15cm}{63} & \filledcircle{0.15cm}{0} & \filledcircle{0.15cm}{0}\\
Details & \filledcircle{0.15cm}{67} & \filledcircle{0.15cm}{0} & \filledcircle{0.15cm}{33} & \filledcircle{0.15cm}{0} & \filledcircle{0.15cm}{0} & \filledcircle{0.15cm}{0}\\
Featured & \filledcircle{0.15cm}{0} & \filledcircle{0.15cm}{0} & \filledcircle{0.15cm}{100} & \filledcircle{0.15cm}{0} & \filledcircle{0.15cm}{0} & \filledcircle{0.15cm}{0}\\
Login & \filledcircle{0.15cm}{0} & \filledcircle{0.15cm}{0} & \filledcircle{0.15cm}{50} & \filledcircle{0.15cm}{0} & \filledcircle{0.15cm}{50} & \filledcircle{0.15cm}{27}\\
Cart & \filledcircle{0.15cm}{0} & \filledcircle{0.15cm}{50} & \filledcircle{0.15cm}{0} & \filledcircle{0.15cm}{0} & \filledcircle{0.15cm}{50} & \filledcircle{0.15cm}{0}\\
Tracking & \filledcircle{0.15cm}{0} & \filledcircle{0.15cm}{0} & \filledcircle{0.15cm}{0} & \filledcircle{0.15cm}{0} & \filledcircle{0.15cm}{0} & \filledcircle{0.15cm}{0}\\ 
Search & \filledcircle{0.15cm}{0} & \filledcircle{0.15cm}{0} & \filledcircle{0.15cm}{0} & \filledcircle{0.15cm}{0} & \filledcircle{0.15cm}{0} & \filledcircle{0.15cm}{0}\\
\bottomrule
\end{tabular}
\caption{The frequency percentage of attributes present in dark patterns by component type. 
\protect\filledcircle{0.12cm}{0} = attribute not present, 
\protect\filledcircle{0.12cm}{100} = attribute present in 100\% of components with dark patterns within the component type.}\label{tab:darkpattern-attributes}
\end{wraptable}


% \begin{table}
% \caption{Top 3 Dark Pattern Attributes by Number of Instances Identified for Each Widget Type}
%   \begin{tabular}{p{2cm} p{4cm} p{4cm} p{4cm}} 
%     \toprule
%     {Widget Name}&Top 1 Attribute (instances) & Top 2 Attribute (instances) & Top 3 Attribute (instances)\\
%     \midrule
%     Newsletter & Information Hiding (18) & Restrictive (17) & Disparate Treatment (16)\\
%     Discount & Covert (14) & Restrictive (7) & Information Hiding (5)\\
%     Membership & Restrictive (14) & Information Hiding (13) & Asymmetric (3)\\
%     Banner & Covert (13) & Information Hiding (8) & \textcolor{red}{Not Decided (2)}\\
%     Unsubscribe & Asymmetric (6) & Covert (3) / Disparate Treatment (3) / Restrictive (3) & --\\
%     Reviews & Information Hiding (10) & -- & --\\
%     Checkout & Information Hiding (5) & Asymmetric (2) / Deceptive (2)\\
%     Details & Asymmetric (2) & Deceptive (1) & --\\\
%     Login & Asymmetric (1) / Information Hiding (1) / Disparate Treatment (1) & -- & --\\
%     Cart & Covert (1) / Restrictive (1) & -- & --\\
%     Featured & \textcolor{red}{Not Sure (2)} & -- & --\\
%     Search & -- & -- & --\\
%     Tracking & -- & -- & --\\
%   \bottomrule
% \end{tabular}
%   \label{tab:attribute_in_widget}
% \end{table}

% From highest to lowest counts, the percentage of widgets identified with dark patterns is as follows: newsletter sign-ups \textcolor{red}{(20.8\%)}, discount offer popovers \textcolor{red}{(19.3\%)}, membership sign-ups \textcolor{red}{(16.5\%)}, sales banners \textcolor{red}{(14.7\%)}, membership unsubscriptions (10.1\%), customer reviews \textcolor{red}{(9.2\%)}, checkouts with shipping options and add-ons \textcolor{red}{(6.4\%)}, product details (2.8\%), user login or registration \textcolor{red}{(1.8\%)}, shopping carts \textcolor{red}{(1.8\%)}, featured products \textcolor{red}{(1.8\%)}, order tracking \textcolor{red}{(0\%)}, and product search \textcolor{red}{(0\%)}.


Components more likely to contain dark patterns are those that provide significant benefits to companies. For example, newsletters and memberships encourage regular user engagement, while discount offers and sales banners drive sales. Table \ref{tab:darkpattern-attributes} summarizes the distribution of dark pattern attributes present in dark patterns by component type, which revealed some interesting patterns. For components designed to encourage users to sign up and stay engaged, such as newsletters and memberships, companies often use information hiding and restrictive attributes. In sales-driven components like discount pop-ups and sales banners, covert attributes are commonly used to pressure users into making quick decisions. 

\subsection{Comparing Whose Interests Are Prioritized: User Interests Decreased While Company Interests Increased Dark Patterns}

To evaluate whether varying interest priorities influence dark pattern generation, we analyzed the data across baseline, user interests, and company interests. Under the baseline condition, 38 components (36.5\% of all components generated under this condition) were identified with dark patterns (Table~\ref{tab:results-interests}). In the user-interest condition, this decreased to 35 components (33.7\%), while in the company-interest condition, it increased to 42 components (40.4\%). %components were identified with dark patterns. 
This indicates that emphasizing user interests in LLMs could result in fewer components with dark patterns, whereas emphasizing company interests could lead to the generation of more components with dark patterns. However, pairwise Chi-square tests comparing the baseline with the user-interest condition ($\chi^2= 0.08$, $p = 0.77$) and the baseline with the company-interest condition ($\chi^2= 0.18$, $p = 0.67$) revealed no statistically significant differences. % as both $p\ge 0.025$ (Bonferroni-adjusted). 
This suggests that the emphasis on interests has %only little 
a minor impact on the number of dark patterns produced.

\setlength{\intextsep}{0pt}
\begin{wraptable}[7]{r}{4cm}
\begin{tabular}{l l l}\toprule  
Interest & \multicolumn{2}{l}{\# Dark Patterns} \\\midrule
Baseline & 38 & \textit{36.5\%}\\
User & 35 & \textit{33.7\%}\\
Company & 42 & \textit{40.4\%}\\
\bottomrule
\end{tabular}
\caption{Frequency of Dark Patterns When Different Interests Are Prioritized.}\label{tab:results-interests}
\end{wraptable}

\subsection{Comparing Dark Pattern Attributes: Hiding Information And Restricting Actions Are Most Common} % Ideally these titles will be changed to have the summary of the most important result

Over one-third (37\%) of the components generated were found to have at least one dark pattern. Among these, 60 components exhibited a single dark pattern attribute, 50 had two dark pattern attributes, and 5 displayed three dark pattern attributes, resulting in a total of 175 identified instances of attributes. Since the labeling schema allowed multiple dark pattern attributes %to be identified within 
in the same component, the percentages reported below in this section reflect the proportion of components containing each attribute relative to the total number of components with dark patterns. % View examples of components 
Figure~\ref{fig:widget_example} provides examples drawn from our dataset that illustrate the six dark pattern attributes (originally defined in Table~\ref{tab:darkpattern-definitions}).

\subsubsection{\textit{Information Hiding}} 
The ``information hiding'' attribute is the most commonly identified dark pattern, appearing in 66 components (57\% of components with dark patterns). Most “information hiding” dark pattern designs omit crucial information that users need to make informed decisions. Examples include: not presenting terms of service or privacy policies during newsletter or membership sign-ups (24), using urgency messages like ``limited offer'' or ``flash sales'' without providing a deadline to pressure users into making decisions (19), displaying positive customer reviews without explaining why they were selected (10), hiding the cost of product add-ons (4), and hiding details about future charges while only mentioning that the first month of membership is free (2). Another tactic used by the dark pattern is obscuring or delaying information. For example, terms of service or privacy policies may be made to appear unclickable through subtle color choices (6), or the option for free shipping is delayed (1).

\subsubsection{\textit{Restrictive}}
The ``restrictive'' attribute is the second most common, with 44 instances identified (38\%). These dark patterns restrict users' actions, forcing them to proceed in a way that aligns with the service's interests. For example, some components do not include the absence of a close button for membership, newsletter sign-up, or other popovers (41). Some do not provide options to remove products or return to shopping in the shopping cart (3).

\subsubsection{\textit{Covert}}
The ``covert'' attribute is the third most common, appearing in 33 instances (29\%). The most common example is limited-time messages that do not have a specified expiration date (19), followed by countdown timers (10). Other examples include offering lures during membership cancellation, such as discounts or the option to pause membership (3), and pressured selling, where products users did not add are recommended in the shopping cart (1).

\subsubsection{\textit{Asymmetric}}
The ``asymmetric'' attribute is found in 16 instances (14\%). The asymmetry is reflected in using brighter or green colors for buttons such as ``Keep Membership'' during cancellations or ``Add to Cart'' on product detail pages (11), pre-selecting costly default options during membership sign-up (4), and using tiny fonts to disclose that discounts only apply to selected items (1).

\subsubsection{\textit{Disparate Treatment}}
The ``disparate treatment'' attribute appears in 11 instances (10\%). These designs use the attribute to encourage actions that benefit the company %service 
or discourage actions that could disadvantage it. The benefits offered to users are often discounts on products or membership prices. For example, a 10\% discount is used to encourage users to sign up for memberships or newsletters (8). Similarly, discounts are provided during the membership cancellation process to discourage users from completing their cancellations (3).

\subsubsection{\textit{Deceptive}}
The least frequently observed attribute is ``deceptive'', appearing in only 5 instances (4\%). The deception includes designs that highlight discounts for featured products without providing the original price (2), creating false beliefs that the products are on sale. In the checkout process, add-ons such as gift wrapping and shipping insurance lack price information, leading users to mistakenly believe these services are free (2). Additionally, deception can involve using color to emphasize prices, making users think a product is discounted without offering further clarification (1).

