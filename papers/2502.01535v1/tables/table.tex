%% 
%% Copyright 2007-2025 Elsevier Ltd
%% 
%% This file is part of the 'Elsarticle Bundle'.
%% ---------------------------------------------
%% 
%% It may be distributed under the conditions of the LaTeX Project Public
%% License, either version 1.3 of this license or (at your option) any
%% later version.  The latest version of this license is in
%%    http://www.latex-project.org/lppl.txt
%% and version 1.3 or later is part of all distributions of LaTeX
%% version 1999/12/01 or later.
%% 
%% The list of all files belonging to the 'Elsarticle Bundle' is
%% given in the file `manifest.txt'.
%% 
%% Template article for Elsevier's document class `elsarticle'
%% with numbered style bibliographic references
%% SP 2008/03/01
%% $Id: elsarticle-template-num.tex 272 2025-01-09 17:36:26Z rishi $
%%
\documentclass[preprint,12pt]{elsarticle}

%% Use the option review to obtain double line spacing
%% \documentclass[authoryear,preprint,review,12pt]{elsarticle}

%% Use the options 1p,twocolumn; 3p; 3p,twocolumn; 5p; or 5p,twocolumn
%% for a journal layout:
%% \documentclass[final,1p,times]{elsarticle}
%% \documentclass[final,1p,times,twocolumn]{elsarticle}
%% \documentclass[final,3p,times]{elsarticle}
%% \documentclass[final,3p,times,twocolumn]{elsarticle}
%% \documentclass[final,5p,times]{elsarticle}
%% \documentclass[final,5p,times,twocolumn]{elsarticle}

%% For including figures, graphicx.sty has been loaded in
%% elsarticle.cls. If you prefer to use the old commands
%% please give \usepackage{epsfig}

%% The amssymb package provides various useful mathematical symbols
\usepackage{amssymb}
%% The amsmath package provides various useful equation environments.
\usepackage{amsmath}
%% The amsthm package provides extended theorem environments
% \usepackage{amsthm}

%% The lineno packages adds line numbers. Start line numbering with
%% \begin{linenumbers}, end it with \end{linenumbers}. Or switch it on
%% for the whole article with \linenumbers.
%% \usepackage{lineno}

% Additional package
% \usepackage{xcolor}
\usepackage{booktabs}
\usepackage{hyperref}
\biboptions{sort&compress}
\usepackage{graphicx}
\usepackage[table,xcdraw]{xcolor}
\usepackage{multirow}
\pagenumbering{gobble}

\DeclareMathOperator*{\argmin}{arg\,min}
\DeclareMathOperator*{\argmax}{arg\,max}

\journal{Journal of Biomedical Informatics}

\begin{document}

\begin{frontmatter}

%% Title, authors and addresses

%% use the tnoteref command within \title for footnotes;
%% use the tnotetext command for theassociated footnote;
%% use the fnref command within \author or \affiliation for footnotes;
%% use the fntext command for theassociated footnote;
%% use the corref command within \author for corresponding author footnotes;
%% use the cortext command for theassociated footnote;
%% use the ead command for the email address,
%% and the form \ead[url] for the home page:
%% \title{Title\tnoteref{label1}}
%% \tnotetext[label1]{}
%% \author{Name\corref{cor1}\fnref{label2}}
%% \ead{email address}
%% \ead[url]{home page}
%% \fntext[label2]{}
%% \cortext[cor1]{}
%% \affiliation{organization={},
%%             addressline={},
%%             city={},
%%             postcode={},
%%             state={},
%%             country={}}
%% \fntext[label3]{}

%\title{Reliable Evidence-Driven Explainable Diagnosis of Alzheimer's Disease using a Multimodal Language-Vision Model with Contrastive Learning}

% \title{Med-LVMH$^{\dag}$\footnote{$^{\dag}$Multimodal Language-Vision Model Harness with Contrastive Learning.} for Evidence-Driven, Reliable, and Explainable Diagnosis of Alzheimer's Disease}

%\title{$\mu$-LVMH: Multimodal Language-Vision Model Harness with Contrastive Learning for Evidence-Driven Reliable Explainable Diagnosis of Alzheimer's Disease}

% \title{Med-LVMH: Multimodal Language-Vision Model Harness with Contrastive Learning for Evidence-Driven Reliable Explainable Diagnosis of Alzheimer's Disease}

% \title{VisTA: Vision-Integrated Similarity-based Text-Augmented Model with Contrastive Learning for Evidence-Driven Reliable Explainable Diagnosis of Alzheimer's Disease}
%\title{VisTA: Vision-Integrated Similarity-based Text-Augmented Model with Contrastive Learning for Evidence-Driven Reliable Explainable Alzheimer's Disease Diagnosis}

% \title{VisTA: Vision-Similarity-Text Augmentation Model with Contrastive Learning for Evidence-Driven, Reliable, and Explainable Alzheimer's Disease Diagnosis}

% \title{VisTA: Vision-Text Alignment Model with Contrastive Learning for Evidence-Driven, Reliable, and Explainable Alzheimer's Disease Diagnosis}
\title{VisTA: Vision-Text Alignment Model with Contrastive Learning using Multimodal Data for Evidence-Driven, Reliable, and Explainable Alzheimer's Disease Diagnosis}
% \title{VisTA: Vision-Text Adaptation Model with Contrastive Learning for Evidence-Driven, Reliable, and Explainable Alzheimer's Disease Diagnosis}

%% use optional labels to link authors explicitly to addresses:
%% \author[label1,label2]{}
%% \affiliation[label1]{organization={},
%%             addressline={},
%%             city={},
%%             postcode={},
%%             state={},
%%             country={}}
%%
%% \affiliation[label2]{organization={},
%%             addressline={},
%%             city={},
%%             postcode={},
%%             state={},
%%             country={}}

\author[chuv,unil,uet]{Duy-Cat Can} %% Author name
\ead{duy-cat.can@chuv.ch}
\author[hcmus]{Linh D. Dang} %% Author name
\ead{dang.diemlinh0212@gmail.com}
\author[hcmus]{Quang-Huy Tang} %% Author name
\ead{tqhuy23@apcs.fitus.edu.vn}
\author[mh175]{Dang Minh Ly} %% Author name
\ead{drminhdang@outlook.com}
\author[hcmiu]{Huong Ha} %% Author name
\ead{htthuong@hcmiu.edu.vn}
\author[chuv]{Guillaume Blanc} %% Author name
\ead{guillaume.e.blanc@gmail.com}
\author[chuv,unil]{Oliver Y. Ch\'en\corref{cor1}} %% Author name
\ead{olivery.chen@chuv.ch}
\author[hcmus]{Binh T. Nguyen\corref{cor1}} %% Author name
\ead{ngtbinh@hcmus.edu.vn}
\cortext[cor1]{Corresponding authors.}

%% Author affiliation
\affiliation[chuv]{
    organization={Plateforme de bio-informatique, Centre hospitalier universitaire vaudois (CHUV)},%Department and Organization
    addressline={Rue du Bugnon 46}, 
    city={Lausanne},
    postcode={1005}, 
    state={Vaud},
    country={Switzerland}
}
\affiliation[unil]{
    organization={Faculté de biologie et de médecine, Université de Lausanne (UNIL)},%Department and Organization
    addressline={Quartier Centre}, 
    city={Lausanne},
    postcode={1015}, 
    state={Vaud},
    country={Switzerland}
}
\affiliation[uet]{
    organization={VNU University of Engineering and Technology, Vietnam National University},%Department and Organization
    addressline={144 Xuan Thuy, Cau Giay}, 
    city={Hanoi},
    % postcode={100000}, 
    % state={},
    country={Vietnam}
}
\affiliation[hcmus]{
    organization={Department of Computer Science, University of Science, Vietnam National University Ho Chi Minh City},%Department and Organization
    addressline={227 Nguyen Van Cu St., Ward 4, District 5}, 
    city={Ho Chi Minh City},
    % postcode={700000}, 
    % state={},
    country={Vietnam}
}
\affiliation[mh175]{
    organization={Department of Neurology, Department of Clinical Neurophysiology, Military Hospital 175},%Department and Organization
    addressline={786 Nguyen Kiem Str, Ward 3, Go Vap District}, 
    city={Ho Chi Minh City},
    % postcode={700000}, 
    % state={},
    country={Vietnam}
}
\affiliation[hcmiu]{
    organization={School of Biomedical Engineering, International University, Vietnam National University},%Department and Organization
    addressline={Quarter 6, Linh Trung Ward, Thu Duc City}, 
    city={Ho Chi Minh City},
    % postcode={700000}, 
    % state={},
    country={Vietnam}
}
% \affiliation[scc]{
%     organization={Service de chimie clinique, Centre hospitalier universitaire vaudois (CHUV)},%Department and Organization
%     addressline={Rue du Bugnon 46}, 
%     city={Lausanne},
%     postcode={1005}, 
%     state={Vaud},
%     country={Switzerland}
% }

%% Abstract
\begin{abstract}
% Abstract is strictly limited to 300 words (now 295)
% Text of abstract
\end{abstract}

%%Graphical abstract
\begin{graphicalabstract}
\end{graphicalabstract}


%%Research highlights
\begin{highlights}
    \item 
\end{highlights}

%% Keywords
\begin{keyword}
%% keywords here, in the form: keyword \sep keyword

%% PACS codes here, in the form: \PACS code \sep code

%% MSC codes here, in the form: \MSC code \sep code
%% or \MSC[2008] code \sep code (2000 is the default)

\end{keyword}

\end{frontmatter}

%% Add \usepackage{lineno} before \begin{document} and uncomment 
%% following line to enable line numbers
%% \linenumbers

%% main text
%%


\begin{table}[!ht]
\centering
\caption{\textbf{%Comparative Analysis of Model Performance.
Model Comparisons regarding Abnormality Retrieval and Disease Prediction Tasks.} %Metrics for abnormality retrieval and disease prediction %of diagnosis across 
% We evaluate the pre-trained and VisTA models using accuracy, AUC, sensitivity, specificity, precision, and F1-score.
}
\label{tab:comparative_analysis}
\resizebox{\textwidth}{!}{%
\begin{tabular}{lcccccc}
\toprule
\multicolumn{1}{c}{\textbf{Method}} & \textbf{\begin{tabular}[c]{@{}c@{}}Accuracy\\ (\%)\end{tabular}} & \textbf{AUC-ROC} & \textbf{\begin{tabular}[c]{@{}c@{}}Sensitivity\\ (\%)\end{tabular}} & \textbf{\begin{tabular}[c]{@{}c@{}}Specificity\\ (\%)\end{tabular}} & \textbf{\begin{tabular}[c]{@{}c@{}}Precision\\ (\%)\end{tabular}} & \textbf{\begin{tabular}[c]{@{}c@{}}F1-score\\ (\%)\end{tabular}} \\ \hline
\multicolumn{7}{l}{\cellcolor[HTML]{C0C0C0}\textit{\textbf{a. Abnormality Retrieval Performance$^\dagger$}}} \\ \hline
\multicolumn{7}{l}{\textit{Performance on MINDset}} \\ \hline
BiomedCLIP & 26.00 & 0.74 & 37.85 & 80.04 & 25.96 & 27.39 \\
MedCLIP & 24.00 & 0.52 & 22.96 & 75.51 & 36.14 & 18.23 \\
ConVIRT & 26.00 & 0.56 & 25.00 & 75.00 & 6.50 & 10.32 \\
\textbf{VisTA}$^\S$ &  &  &  &  &  &  \\
\quad- Description & 48.00 & 0.76 & 50.19 & 82.34 & 53.58 & 45.06 \\
\quad- Abnormality & \textbf{74.00} & \textbf{0.87} & \textbf{72.30} & \textbf{90.53} & \textbf{74.02} & \textbf{72.05} \\
\quad- Summary & 70.00 & 0.85 & 66.96 & 89.15 & 72.78 & 68.35 \\
\quad- All & 46.00 & 0.73 & 46.77 & 81.25 & 43.91 & 40.86 \\ \hline
\multicolumn{7}{l}{\cellcolor[HTML]{C0C0C0}\textit{\textbf{b. Dementia Prediction Performance$^\ddagger$}}} \\ \hline
\multicolumn{7}{l}{\textit{Performance on MINDset}} \\ \hline
BiomedCLIP & 30.00 & 0.57 & 17.50 & 80.00 & 77.77 & 28.57 \\
MedCLIP & 58.00 & 0.56 & 65.00 & 30.00 & 78.78 & 71.23 \\
ConVIRT & 80.00 & 0.62 & 100.00 & 0.00 & 80.00 & 88.88 \\
\textbf{VisTA}$^\S$ &  &  &  &  &  &  \\
\quad- Description & 84.00 & 0.63 & \textbf{97.50} & 30.00 & 84.78 & 90.70 \\
\quad- Abnormality & \textbf{88.00} & \textbf{0.82} & 95.00 & \textbf{60.00} & \textbf{90.48} & 92.68 \\
\quad- Summary & \textbf{88.00} & 0.77 & \textbf{97.50} & 50.00 & 88.63 & \textbf{92.86} \\
\quad- All & 78.00 & 0.53 & 92.50 & 20.00 & 82.22 & 87.06 \\ \hline
\multicolumn{7}{l}{\textit{Performance on Public Data}} \\ \hline
BiomedCLIP & 49.45 & 0.50 & 0.15 & \textbf{99.68} & \textbf{100.00} & 0.31 \\
MedCLIP & 50.70 & 0.51 & 43.81 & 57.73 & 51.36 & 47.28 \\
ConVIRT & 50.47 & 0.53 & 100.00 & 0.00 & 50.46 & 67.08 \\
\textbf{VisTA}$^\S$ & \multicolumn{1}{l}{} & \multicolumn{1}{l}{} & \multicolumn{1}{l}{} &  &  &  \\
\quad- Description & 56.80 & 0.59 & 66.10 & 43.84 & 55.89 & 60.56 \\
\quad- Abnormality & 51.64 & 0.54 & \textbf{100.00} & 2.40 & 51.11 & 67.64 \\
\quad- Summary & \textbf{58.28} & \textbf{0.64} & 96.13 & 19.24 & 54.96 & \textbf{69.93} \\
\quad- All & 54.61 & 0.56 & 89.16 & 17.67 & 53.14 & 66.59 \\
\hline
\multicolumn{7}{l}{\cellcolor[HTML]{C0C0C0}\textit{\textbf{c. AD Prediction Performance$^\ddagger$}}} \\ \hline
\multicolumn{7}{l}{\textit{Performance on MINDset}} \\ \hline
BiomedCLIP & 48.00 & 0.54 & 0.00 & \textbf{100.00} & 0.00 & 0.00 \\
MedCLIP & 48.00 & 0.47 & 0.00 & \textbf{100.00} & 0.00 & 0.00 \\
ConVIRT & 48.00 & 0.50 & 0.00 & \textbf{100.00} & 0.00 & 0.00 \\
\textbf{VisTA}$^\S$ &  &  &  &  &  &  \\
\quad- Description & 44.00 & 0.47 & 26.92 & 62.50 & 43.75 & 33.33 \\
\quad- Abnormality & \textbf{60.00} & \textbf{0.60} & \textbf{46.15} & 75.00 & \textbf{66.66} & \textbf{54.55} \\
\quad- Summary & 54.00 & 0.59 & \textbf{46.15} & 62.50 & 57.14 & 51.06 \\
\quad- All & 48.00 & 0.50 & 34.62 & 62.50 & 50.00 & 40.91 \\ \bottomrule
\multicolumn{7}{r}{\footnotesize\begin{tabular}[c]{@{}r@{}}
The best results in each column and experiment are in \textbf{bold} fonts.\\ 
%Acc: accuracy. AUC-ROC: the area under the ROC curve.\\ 
%Sen: sensitivity (recall). Spe: Specificity. P: precision. F1: F1 score.\\
$^\dagger$:Macro-averaged results over four (04) abnormality types.
$^\ddagger$:Binary classification results.\\
% $^\S$\textbf{\textit{VisTA text modality}}:\\
% - Description: using image description.\\
% - Abnormality: using abnormality type.\\ 
% - Summary: using abnormality type and dementia type.\\
% - All: using a combination of summary and description.
$^\S$\textbf{\textit{VisTA text modality}}:
Description: using image description;
Abnormality: using abnormality type; \\
Summary: using abnormality type and dementia type;
All: using a combination of summary and description.
\end{tabular}}
\end{tabular}%
}
\end{table}

\begin{table}[!ht]
\centering
\caption{\textbf{Error analysis across diagnostic stages of VisTA model using summary text modality.} Summary of false positive (FP) and false negative (FN) rates for each diagnostic stage, including abnormality retrieval, reference case retrieval, and prediction tasks. Representative errors highlight common misclassifications and potential causes.
}
\label{tab:error_analysis}
\resizebox{\textwidth}{!}{%
\begin{tabular}{llccp{.6\linewidth}}
\toprule
\multicolumn{1}{c}{\textbf{Stage}} & \multicolumn{1}{c}{\textbf{Type}} & \textbf{FP (\%)} & \textbf{FN (\%)} & \multicolumn{1}{c}{\textbf{Representative errors}} \\ \midrule
\multirow{4}{*}{\begin{tabular}[c]{@{}l@{}}\\\\\\\textbf{Abnormality}\\ \textbf{retrieval}\end{tabular}} & Normal & 50.00 & 16.67 & \begin{tabular}[c]{@{}p{\linewidth}@{}}- Misclassified as WMH due to image noise.\\ - Normal variations mistaken for subtle atrophy features.\end{tabular} \\ \cline{2-5} 
 & \begin{tabular}[c]{@{}l@{}}MTL\\ atrophy\end{tabular} & 36.00 & 27.27 & \begin{tabular}[c]{@{}p{\linewidth}@{}}- Mild atrophy misclassified as normal.\\ - Overlap with WMH features. 
 % causing confusion.
 \end{tabular} \\ \cline{2-5} 
 & WMH & 0.00 & 46.15 & \begin{tabular}[c]{@{}p{\linewidth}@{}}- Missed subtle WMH regions in low-contrast images.\\ - Misclassified as MTL atrophy due to similar intensity patterns.\end{tabular} \\ \cline{2-5} 
 & \begin{tabular}[c]{@{}l@{}}Other\\ atrophy\end{tabular} & 41.67 & 46.15 & \begin{tabular}[c]{@{}p{\linewidth}@{}}- Overlap with MTL atrophy features.\\ - Subtle features mistaken for normal.\end{tabular} \\ \hline
\multicolumn{2}{l}{\textbf{Reference cases retrieval}$^*$} & \multicolumn{2}{c}{14.71} & \begin{tabular}[c]{@{}p{\linewidth}@{}}- Low-confidence cases with ambiguous similarity scores.\\ - Unique cases lacking any matching reference images.\\ - Retrieval errors for overlapping features across abnormality types.\end{tabular} \\ \hline
\multirow{2}{*}{\begin{tabular}[c]{@{}l@{}}\\\textbf{Dementia}\\ \textbf{prediction}\end{tabular}} & Dementia & 16.67 & 50.00 & \begin{tabular}[c]{@{}p{\linewidth}@{}}- WMH misclassified as non-dementia due to mild features.\\ - Subtle MTL atrophy cases missed.\end{tabular} \\ \cline{2-5} 
 & \begin{tabular}[c]{@{}l@{}}Non-\\ dementia\end{tabular} & 11.36 & 2.50 & \begin{tabular}[c]{@{}p{\linewidth}@{}}- Mild cognitive cases flagged as dementia.\\ - Normal cases with noise in images labeled incorrectly.\end{tabular} \\ \hline
\multirow{2}{*}{\begin{tabular}[c]{@{}l@{}}\\\textbf{AD}\\ \textbf{prediction}\end{tabular}} & AD & 48.28 & 37.50 & \begin{tabular}[c]{@{}p{\linewidth}@{}}- Misclassified as other dementia due to overlapping features with WMH or mild atrophy..\\ - Overemphasis on normal features in borderline cases.\end{tabular} \\ \cline{2-5} 
 & Non-AD & 42.86 & 53.85 & \begin{tabular}[c]{@{}p{\linewidth}@{}}- WMH-related non-AD misclassified as AD.\\ - Other dementia cases misclassified as AD due to feature overlap with MTL atrophy.\end{tabular} \\ \bottomrule
\multicolumn{5}{r}{\footnotesize\begin{tabular}[c]{@{}r@{}}FP (\%): Percentage of the predicted labels. FN (\%): Percentage of the true labels.\\ $^*$Incorrect matches rate of top-1 retrieval cases.\end{tabular}}
\end{tabular}%
}
\end{table}


% The Appendices part is started with the command \appendix;
% appendix sections are then done as normal sections
% \appendix
% \section{Example Appendix Section}
% \label{app1}

% Appendix text.

% %% For citations use: 
% %%      ~\cite{<label>} ==> [1]

% %%
% Example citation, See~\cite{lamport94}.

%% If you have bib database file and want bibtex to generate the
%% bibitems, please use
%%
\bibliographystyle{elsarticle-num} 
\bibliography{refs}

%% else use the following coding to input the bibitems directly in the
%% TeX file.

%% Refer following link for more details about bibliography and citations.
%% https://en.wikibooks.org/wiki/LaTeX/Bibliography_Management

% \begin{thebibliography}{00}

% %% For numbered reference style
% %% \bibitem{label}
% %% Text of bibliographic item

% \bibitem{lamport94}
%   Leslie Lamport,
%   \textit{\LaTeX: a document preparation system},
%   Addison Wesley, Massachusetts,
%   2nd edition,
%   1994.

% \end{thebibliography}
\end{document}

\endinput
%%
%% End of file `elsarticle-template-num.tex'.
