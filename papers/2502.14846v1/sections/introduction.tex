Instruction-tuned vision-language models (VLMs) have shown strong performance across a range of multimodal tasks \citep{clip,gpt4,liu2023llava}.
However, these tasks typically focus on general image understanding over natural images rather than the specialized reasoning required for text-rich images such as charts, documents, diagrams, signs, labels, and screenshots.
Understanding and reasoning over text-rich images is crucial for many applications, including analyzing scientific literature and figures \cite{asai2024openscholar}, improving accessibility for users with visual impairments \cite{gurari2018vizwiz}, and enabling agentic workflows in real-world environments \cite{OSWorld}.
Effectively interpreting these structured visual formats requires both textual comprehension and spatial reasoning, which current models struggle with due to the limited availability of high-quality, realistic, and diverse vision-language datasets \citep{methani2020plotqa}.

To address these challenges and inspired by the fact that text-rich images are typically rendered from code, we develop \textbf{Co}de Guided \textbf{Syn}thetic data generation system (\textbf{CoSyn}), a flexible framework for generating diverse synthetic text-rich multimodal data for vision-language instruction tuning.
As illustrated in Figure \ref{fig: system}, CoSyn can generate multimodal data for various target domains from a short natural language query, such as \textit{book covers}.
CoSyn leverages text-only LLMs, which excel at code generation, to produce both data and code that render diverse text-rich images using 11 supported rendering tools (e.g., Python, HTML, LaTeX). 
Grounded in the underlying code representation of the images, textual instructions are also generated by the text-only LLM to create vision-language instruction-tuning datasets.

Using this framework, we construct the CoSyn-400K, as shown in Figure \ref{fig: dataset}, a large-scale and diverse synthetic vision-language instruction-tuning dataset tailored for text-rich image understanding.
We comprehensively evaluate the effectiveness of training on CoSyn-generated synthetic data across seven text-rich VQA benchmarks. 
Our model achieves state-of-the-art performance among competitive open-source models and surpasses proprietary models such as GPT-4V and Gemini 1.5.
Notably, training on CoSyn synthetic data enables sample-efficient learning, achieving stronger performance with less data.
In addition, CoSyn can synthesize chain-of-thought (CoT) reasoning data \citep{wei2022chain}, improving performance on tasks requiring multi-hop reasoning. 
A fine-grained analysis of question types in ChartQA \cite{masry-etal-2022-chartqa} reveals that training on CoSyn-400K results in stronger generalization to human-written questions. 
In contrast, models trained solely on existing academic datasets often overfit to biased training data, overperforming on templated or machine-generated questions but struggling with more realistic, human-asked queries.

We then identify a key limitation of open-source VLMs that they struggle to generalize to out-of-domain tasks they were not trained on.
As shown in Figure \ref{fig: teaser}, we introduce NutritionQA, a novel benchmark for understanding photos of nutrition labels, with practical applications like aiding users with visual impairments.
Open-source VLMs perform poorly on this novel task, even after training on millions of images. 
However, by training on CoSyn-400K, our model adapts strongly to this novel domain in a zero-shot setting with significantly less training data.
Remarkably, by generating just 7K in-domain synthetic nutrition label examples using CoSyn for fine-tuning, our model surpasses most open VLMs trained on millions of images. 
This highlights CoSyn’s data efficiency and ability to help VLMs adapt to new domains through targeted synthetic data generation.

Finally, beyond the standard VQA task, we use CoSyn to generate synthetic \textit{pointing} training data, which is particularly useful in agentic tasks. 
The pointing data enables VLMs to retrieve coordinates for specific elements in a screenshot given a query like ``Point to the Checkout button'' \citep{deitke2024molmo}.
Our model trained on synthetic pointing data achieves state-of-the-art performance on the ScreenSpot click prediction benchmark \citep{baechler2024screenai}.
Overall, our work demonstrates that synthetic data is a promising solution for advancing vision-language models in understanding text-rich images and unlocking their potential as multimodal digital assistants for real-world applications.