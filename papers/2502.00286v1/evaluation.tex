% \subsubsection*{\textbf{Evaluation}}\label{sec:evaluation}

\vspace{-3pt}
\section{Evaluation}\label{sec:evaluation}
\vspace{-5pt}
As a baseline for our design exploration, we use the NVDLA architecture paradigm, which include MAC arrays ranging from 64 to 2048 PEs in powers of $2$. The sizes of the local and global convolution buffers scale proportionally with the dimensions of the MAC arrays, as specified by NVIDIA \cite{nvdla2017}. To evaluate these configurations, we employ two specialized tools: the nn-dataflow~\cite{gao2019tangram}, to estimate DNN workload performance, and ApproxTrain~\cite{gong2023approxtrain}, to calculate the accuracy impact of the approximate multipliers.

Figure \ref{fig:combined} illustrates the trade-off between embodied carbon and performance for DNN accelerators running VGG16 at the 7nm technology node. The configurations for the exact (baseline) accelerator show exponential carbon increase as the architecture becomes more compute-intensive. By incorporating approximate units only, while keeping the architecture unchanged (same number of PEs and memory), we achieved a $5\%$ reduction in embodied carbon. In our experiments, we tested approximate units that resulted in accuracy losses of up to 0.5\%, 1.0\%, and 2.0\% (Appx in legend). Similar trends appeared at 14nm and 28nm, as shown in the corresponding table, with gains of up to $12.75\%$. 
However, the frames per second (FPS) achieved by large accelerators often far exceed the requirements for edge applications~\cite{gupta2022act}. To address this, we applied realistic performance thresholds of 30, 40, and 50 FPS and utilized our genetic algorithm with approximate multipliers. This approach significantly reduced the embodied carbon footprint, achieving reductions of up to 50\%. Each point on the plot (GA-CDP in legend) represents an accelerator design that optimizes the CDP while meeting the specified performance thresholds (30, 40, and 50 FPS).



\begin{figure}[t]
    \vspace{-10pt}
    \centering
    \begin{minipage}[t]{0.5\linewidth}
        \centering
        \vspace{-2pt}
        \resizebox{\linewidth}{!}{\includegraphics{output-gnuplottex-fig1.pdf}}
        % \resizebox{\linewidth}{!}{\begin{gnuplot}[terminal=pdfcairo, terminaloptions={size 4in,3in enhanced}]
set terminal pdfcairo size 4in,3in enhanced font "Times New Roman,13"
set margins 10,2,3,2
set title "7nm - VGG16" offset 0,-0.8 font ",15"
set xlabel "Performance (frames/second)" offset 0,0.3 font ",15"
set ylabel "Manufacturing Carbon (gCO_{2})" offset -0.2,0 font ",15"
set key font ",16"
set xtics font ",13"
set ytics font ",13"
set logscale x
set xrange [1.5:250]
set yrange [0:40]
set arrow from 30, graph 0 to 30, graph 1 nohead lt 1 lw 1 linecolor rgb "#666666"
set arrow from 40, graph 0 to 40, graph 1 nohead lt 1 lw 1 linecolor rgb "#666666"
set arrow from 50, graph 0 to 50, graph 1 nohead lt 1 lw 1 linecolor rgb "#666666"
set label "FPS 30" at 27, graph 0.80 rotate by 90 font ",13"
set label "FPS 40" at 37, graph 0.80 rotate by 90 font ",13"
set label "FPS 50" at 47, graph 0.80 rotate by 90 font ",13"
set label "64 PEs" at 1.5,5.4 offset 1,1 font ",13"
set label "128 PEs" at 3.8,5.5 offset 0,-1 font ",13"
set label "256 PEs" at 7,15.5 offset 0,-1 font ",13"
set label "2048 PEs" at 75,40 offset 0,-1 font ",13"
set key at graph 0.43,0.95 box spacing 0.9 width 1 maxcols 1 
set grid lt 1 lc rgb "#dddddd"
plot \
    'plots/7nm_drop00.dat' using 1:2 title 'Exact' with linespoints pt 7 lc rgb "#004C99" lw 2, \
    'plots/7nm_drop05.dat' using 1:2 title 'Appx 0.5%' with linespoints pt 6 lc rgb "#006400" lw 2, \
    'plots/7nm_drop10.dat' using 1:2 title 'Appx 1%' with linespoints pt 3 lc rgb "#7E1E9C" lw 2, \
    'plots/7nm_drop20.dat' using 1:2 title 'Appx 2%' with linespoints pt 5 lc rgb "#A3333D" lw 2,  \
    'plots/cdp_plot_data.dat' using 1:2 title 'GA-CDP' with points pt 5 lc rgb "#E65100" ps 0.8

\end{gnuplot}}
        \label{fig:trend}
    \end{minipage}
    \hfill
    \begin{minipage}[t]{0.45\linewidth}
    \centering
    \vspace{15pt}
    \resizebox{\linewidth}{!}{%
    \begin{tabular}{|c|c|c|c|c|}
    \hline
    \multicolumn{5}{|c|}{Carbon Footprint Reduction (\%)} \\
    \hline
    \multirow{2}{*}{Technology} & \multirow{2}{*}{Type} & \multicolumn{3}{c|}{Accuracy Drop } \\
    \cline{3-5}
    (nm) &  & 0.5\% & 1.0\% & 2.0\% \\
    \hline
    \multirow{2}{*}{7} & Avg & 2.83 & 4.49 & 5.17 \\
                       & Peak & 5.78 & 9.18 & 10.56 \\
    \hline
    \multirow{2}{*}{14} & Avg & 5.58 & 6.90 & 8.02 \\
                        & Peak & 8.87 & 10.98 & \textbf{12.75} \\
    \hline
    \multirow{2}{*}{28} & Avg & 3.33 & 5.71 & 8.44 \\
                        & Peak & 4.60 & 7.87 & 11.65 \\
    \hline
    \end{tabular}
    }
    \label{fig:carbon_table}
    \end{minipage}
    \vspace{-15pt}
    \caption{Embodied Carbon trends for VGG16 across different accuracy and performance levels.}
    \label{fig:combined}
\end{figure}



% 2nd reviewer comment: Not clear why the footprint for 7/14/28nm is the same (I would expect it to be different...).
\begin{figure}
    \vspace{-13pt}
    \centering
    \resizebox{0.48\textwidth}{!}{\includegraphics[clip]{plots/bar_plot.pdf}}
    \vspace{-5pt}
    \caption{Embodied carbon comparison across DNN models \\ (Normalized to exact implementation for each technology node)}
    %\caption{Embodied carbon across DNN models}
    \vspace{-15pt}
    \label{fig:overview2}
\end{figure}

To validate our methodology, we evaluated VGG16, VGG19, ResNet50, and ResNet152, on ImageNET subset, across $7$, $14$, and $28$ nm nodes. Figure \ref{fig:overview2} compares three designs: the exact baseline meeting a $30$ FPS threshold, an approximate version using area-efficient multipliers with up to $2.0\%$ accuracy drop, and our proposed solution (GA-CDP). While approximation alone reduces embodied carbon, our methodoogy further optimizes the design through minimal architecture and efficient multipliers. Results show significant reductions in embodied carbon across all networks and nodes, with up to $65\%$ savings for VGG16 and $30\%$–$70\%$ for others, proving that our approach creates sustainable, performance-compliant designs.