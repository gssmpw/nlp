% \subsubsection*{\textbf{Methodology}}\label{sec:methodology}
\vspace{-3pt}
\section{Methodology}\label{sec:methodology}
\vspace{-3pt}
\begin{figure}
    \centering
    \resizebox{0.35\textwidth}{!}{\includegraphics[clip]{figures/methodology-cropped.pdf}}
    % \includegraphics[clip, width=0.47\textwidth]{figures/methodology-cropped.pdf}
    \caption{Overview of the proposed methodology}
    \vspace{-10pt}
    \label{fig:overview}
\end{figure}



The primary objective of our methodology is to design an approximate DNN accelerator and determine its corresponding mapping to optimize the Carbon Delay Product (CDP). CDP is a comprehensive metric that integrates performance and the embodied carbon footprint, offering a holistic assessment of the trade-offs between sustainability and efficiency.
By focusing on minimizing the CDP, we aim to design hardware that achieves a balance between performance and carbon emissions, ensuring suitability for edge-based systems.

% 4th reviewer comment: The carbon footprint is proportional to the accelerator area. So, I don't see in the proposed approach what is different from existing approximate computing approaches targeting the minimization of the accelerator area

%The embodied carbon estimation of an accelerator is primarily influenced by its manufacturing process, which depends on several factors.
The embodied carbon estimation of an accelerator, while related to chip area, primarily depends on the die manufacturing process, which incorporates several technology and fab-specific factors beyond area alone. Key contributors include the fabrication facility's attributes, such as its power consumption and the carbon intensity of its electricity grid. Additionally, the technology node used in the fabrication process significantly impacts scaling trends and yield results. For a monolithic DNN accelerator die, the embodied carbon footprint is calculated based on emissions produced during the manufacturing of its logic chip area, using a specified technology node \cite{sudarshan2024eco}. The total embodied carbon of a chip comprises two main components: the product of the Carbon Footprint Per unit Area (CFPA) of the die and its area (A$_\text{die}$), and the product of the CFPA of Silicon (CFPA$_\text{Si}$) and the wasted area of the silicon wafer (A$_\text{wasted}$) during fabrication, as shown in Eq. \ref{eq:embodied}. The CFPA depends on factors such as the Carbon Intensity of the fabrication facility (CI$_\text{fab}$), the Energy consumed per unit Area during manufacturing (EPA), the greenhouse gas emissions (C$_\text{gas}$), the carbon impact of raw material procurement (C$_\text{material}$), and the yield (Y) of the fabrication process.

\vspace{-6pt}
\begin{equation}
    C_{\text{embodied}} = \text{CFPA} \times A_{\text{die}} + \text{CFPA}_{\text{Si}} \times A_{\text{wasted}}
     \label{eq:embodied}
\end{equation}
\vspace{-6pt}
\begin{equation}
    \text{CFPA} =  \frac{\text{CI}_\text{fab} \times \text{EPA} + \text{C}_\text{gas} + \text{C}_\text{material}}{\text{Y}}
     \label{eq:cfpa}
\end{equation}

To optimize embodied carbon, while maintaining computational efficiency, we start by generating area-aware approximate multipliers for the MAC units. To achieve this, we apply gate-level pruning and precision scaling approximation techniques to modify the netlist structure or the connections between its gates, effectively reducing the circuit area \cite{9804695}. These approximations are guided by a multi-objective optimization algorithm that explores the design space to identify near-Pareto-optimal solutions with minimal functional error. The resulting approximate multipliers not only lower the embodied carbon footprint by reducing the required hardware area but also maintain the computational accuracy needed for error-resilient DNN tasks.

In the second step, we integrate these approximate multipliers into exploring hardware configurations and mappings for the accelerator. This involves optimizing key characteristics such as the width and height of the accelerator (number of Processing Elements), local register file sizes, and global buffer capacity. Mapping parameters, including tiling strategies, execution order, and levels of parallelism, are also considered. 
%To navigate this vast design space, we employ a genetic algorithm that selects the Pareto-optimal approximate multipliers from step one and identifies the most efficient DNN topology.
To navigate this vast design space, we employ a genetic algorithm, with CDP metric as fitness function, to select the Pareto-optimal approximate multipliers from step one and identify the most efficient DNN topology. The optimization process is constrained by thresholds for accuracy drop and performance, measured in inferences per second, ensuring that the design meets the realistic requirements of edge systems. This approach addresses the overdesign issue observed in previous accelerators, resulting in a more sustainable and efficient solution.

% 1st reviewer comment: Why the CDP metric has not been used as the fitness function of the genetic algorithm? It seems that the Manufacturing Carbon (Fig. 2) was computed only after the generation of DNN accelerators. ---> Make clear that CDP is in the GA