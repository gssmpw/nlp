\section{Shapley-based Composition Analysis}
\label{sec:appendix-sv-composition-analysis}

To understand the relationships between the different prompting techniques of a composition, we conduct a game theoretic analysis based on the Shapley value (SV) and Shapley Interactions (SI).
We further use the results from this Shapley-based analysis to predict optimal compositions for each model and dataset.

\paragraph{Setup}
To gain further insights into the interplay of prompting techniques, we analyze the prompt composition games (cf. Equation~\ref{def-prompt-composition}) across three datasets and models, exploring all possible variants of in-context demonstrations (category, similarity, and random).
Specifically, the players in each game include the personas (per.), definitions (def.), the specified in-context demonstration variant (cat./sim./rand. dem.), reasoning step instructions (rea.), and directional stimulus (dir. stim.).

We evaluate the games on all $\vert 2^T \vert = 2^5 = 32$ compositions, measuring the macro F1 scores of the models on both the \textit{validation} and \textit{test} sets for each composition $S \subseteq T$.
Next, we compute exact SVs and pairwise SIs \cite{Bord.2023,Lundberg.2020} on the \emph{validation} set using the \texttt{shapiq}\footnote{\url{https://github.com/mmschlk/shapiq}} package \cite{muschalik2025}.

Similar to Section~\ref{sec:composition-prediction}, we use the SVs and SIs to predict an optimal composition for each setting based on the validation data.
We reconstruct all game values for each composition $S \subseteq T$ using the SVs and SIs to select the set of prompting techniques with the highest reconstructed macro F1 score.
Formally, we iterate over all $S \subseteq T$ to combine the individual SV or pairwise SI scores into an additive prediction of the game with
\begin{align*}
    \hat{\nu}^\text{SV}(S) := \sum_{i \in S} \phi_i^{\text{SV}}
    \text{\quad and\quad}
    \hat{\nu}^\text{SI}(S) := \sum_{\substack{L \subseteq S\\|L| \leq 2}} \phi_L^{\text{SI}}
\end{align*}
where $\phi^{\text{SV}}$ and $\phi^{\text{SI}}$ are the SV and SI scores, respectively.
We then compare the performance of this selected composition on the \emph{test} dataset against naive compositions (using all techniques or none) and the overall best-performing compositions.

\paragraph{Visualizing the SVs and SIs}
To visually investigate the SIs, we employ \emph{force} \cite{Lundberg.2017} and \emph{network} plots \cite{muschalik2024}.
Force plots, as presented in Figure~\ref{fig_force_plots_stereoset}, are commonly used to represent the SVs on a number line representing the prediction space.
On average, prompting techniques with a positive SV increase the performance of the models, and techniques with a negative value decrease the performance.
In the force plots this is represented by the positive techniques ``forcing'' the performance ``away'' from the performance of the empty composition $\nu(\emptyset)$ towards the performance of the full composition $\nu(T)$.
Additionally, the SIs indicate synergies (positive value) and redundancies (negative value) between prompting techniques \citep{fumagalli2024a}.
To illustrate second-order SIs among the individual prompting techniques, network plots, as depicted in Figure~\ref{network-stereoset}, Figure~\ref{network-sbic}, and Figure~\ref{network-cobra}, arrange the techniques in a circular layout and represent first-order and second-order interactions as nodes and edges, respectively.
The size of the nodes and edges represents the strength of the interactions, and the color denotes the direction (red increases performance, blue decreases performance).

\paragraph{Findings}
The results of the Shapley-based composition analysis are summarized in Table~\ref{tab:table-sv-composition-analysis} and Table~\ref{tab:table-si-selection}, as well as in Figure~\ref{fig_force_plots_stereoset}, Figure~\ref{network-stereoset}, Figure~\ref{network-sbic}, and Figure~\ref{network-cobra}.

Our results highlight a strong interaction between the different prompting techniques.
We present \emph{five} main findings.
\textbf{(1)} First, adding all possible prompting techniques to a composition does not consistently enhance performance compared to providing only a task description.
This is demonstrated in Table~\ref{tab:table-sv-composition-analysis}, where $\nu(T)$ (value of all compositions $T$) is not consistently higher than $\nu(\emptyset)$ (value of task description only) across all settings.
\textbf{(2)} Second, however, adding prompting techniques consistently improves performance, as all best-on-test compositions in Table~\ref{tab:table-sv-composition-analysis} consist of a non-empty set of techniques.
\textbf{(3)} Third, the selection of compositions requires empirical validation or optimization, as the best-on-test compositions \emph{never} contain all techniques but rather a \emph{heterogeneous} set.
The heterogeneity of the compositions suggests the need for a more stringent mechanism in selecting the best compositions, such as learning a \emph{meta-composition prediction model} or conducting a \emph{game-theoretic assessment}.
\textbf{(4)} Fourth, choosing the composition based on SVs improves performance compared to baseline conditions where no additional information is used, as SV compositions often outperform settings with either no prompting technique or all techniques.
\textbf{(5)} Fifth, modeling the selection problem with SIs, and thus with higher fidelity, substantially improves the performance of composition choices over SV-based selection for the StereoSet corpus, as summarized in Table~\ref{tab:table-si-selection}.

\begin{figure*}
    \centering

    \includegraphics[width=1\linewidth]{force_mistral_category.pdf}
    \\[-1em]
    \includegraphics[width=1\linewidth]{force_mistral_similar.pdf}
    \\[-1em]
    \includegraphics[width=1\linewidth]{force_mistral_random.pdf}

    \caption{Force plots of Shapley values for three variants (top: category in-context demonstrations, middle: similar in-context demonstrations, bottom: random in-context demonstrations) of the composition game for the \emph{Mistral} model on \emph{Stereoset}. In all three settings the \textbf{in-context demonstrations are most influential}. Red color denotes positive attribution (increasing the performance), and blue color denotes negative attribution (decreasing the performance).}
    \label{fig_force_plots_stereoset}
\end{figure*}

\hwfigure{network-sbic}{Network plots of the shapley interactions for the three evaluated LLMs on SBIC.}
\hwfigure{network-cobra}{Network plots of the shapley interactions for the three evaluated LLMs on CobraFrames.}

\begin{table*}
    \small
    \centering
    \setlength{\tabcolsep}{2.5pt}

    \begin{tabular}{@{}lllrrlrlrc@{}}
        \toprule
         &  &  &  &  & \multicolumn{2}{c}{\textbf{Best-on-Test Composition}} & \multicolumn{3}{c}{\textbf{SV Composition}} \\

        \cmidrule(l{2pt}r{2pt}){6-7}\cmidrule(l{2pt}){8-10}

        \textbf{Corpus} &  \textbf{Model} & \textbf{Variant} & \multicolumn{1}{l}{\textbf{$\nu(\emptyset)$}} & \multicolumn{1}{l}{\textbf{$\nu(T)$}}  & \textbf{Composition} & \multicolumn{1}{l}{\textbf{Score}} & \textbf{Composition} & \multicolumn{1}{l}{\textbf{Score}} & \multicolumn{1}{l}{\textbf{Best}} \\ \midrule

        \textbf{StereoSet\,\,} & \textbf{Mistral} & category & 0.711 & 0.682 & rea., def. & \textbf{0.722} & in-cont., rea. & 0.705 & \xmark \\
        &  & similar & 0.711 & 0.795 & in-cont., rea., def. & \textbf{0.800} & in-cont., rea., per. & 0.790 & \xmark \\
        &  & random & 0.711 & 0.705 & rea., def. & \textbf{0.722} & in-cont., rea. & 0.705 & \xmark \\

        & \textbf{Command-R} & category & 0.462 & 0.582 & in-cont., per. & \textbf{0.685} & in-cont., def., dir. sti., per. & 0.579 & \xmark \\
        &  & similar & 0.462 & 0.650 & in-cont., per. & \textbf{0.706} & in-cont., def., dir. sti., per. & 0.582 & \xmark \\
        &  & random & 0.462 & 0.652 & in-cont., per. & \textbf{0.677} & in-cont., def., dir. sti., per. & 0.588 & \xmark \\

        & \textbf{Llama~3} & category & 0.575 & 0.583 & in-cont., def. & \textbf{0.760} & in-cont., def. & \textbf{0.760} & \cmark \\
        &  & similar & 0.575 & 0.608 & in-cont., def., per. & \textbf{0.817} & in-cont., def., dir. sti. & \textbf{0.798} & \xmark \\
        &  & random & 0.575 & 0.495 & in-cont., rea. & \textbf{0.768} & in-cont., def. & \textbf{0.736} & \xmark

        \\\midrule


        \textbf{SBIC} & \textbf{Mistral} & category & 0.702 & 0.771 & in-cont., def., dir. sti., per. & \textbf{0.783} & in-cont., def., dir. sti., per. & \textbf{0.783} & \cmark \\
        &  & similar & 0.702 & 0.770 & in-cont., rea., def. & \textbf{0.772} & in-cont., def., dir. sti. & 0.758 & \xmark \\
        &  & random & 0.702 & 0.772 & in-cont., def., dir. sti. & \textbf{0.792} & in-cont., def., dir. sti. & \textbf{0.792} & \cmark \\

        & \textbf{Command-R} & category & 0.470 & 0.751 & in-cont., def. & \textbf{0.772} & in-cont., def., per. & \textbf{0.767} & \xmark \\
        &  & similar & 0.470 & 0.712 & in-cont., def., dir. sti. & \textbf{0.770} & in-cont., def. & \textbf{0.770} & \xmark \\
        &  & random & 0.470 & 0.724 & in-cont., def., per. & \textbf{0.788} & in-cont., def., per. & \textbf{0.788} & \cmark \\

        & \textbf{Llama~3} & category & 0.651 & 0.556 & in-cont., def., per. & \textbf{0.821} & in-cont., def. & \textbf{0.810} & \xmark \\
        &  & similar & 0.651 & 0.469 & in-cont., def., per. & \textbf{0.825} & def. & \textbf{0.788} & \xmark \\
        &  & random & 0.651 & 0.502 & in-cont., def., per. & \textbf{0.831} & in-cont., def. & \textbf{0.826} & \xmark

        \\\midrule


        \textbf{Cobra} & \textbf{Mistral} & category & 0.449 & 0.499 & rea., def. & \textbf{0.548} & in-cont., rea., def. & \textbf{0.522} & \xmark \\
        &  & similar & 0.449 & 0.532 & in-cont. & \textbf{0.604} & in-cont., def. & \textbf{0.604} & \xmark \\
        &  & random & 0.449 & 0.515 & rea., def. & \textbf{0.548} & in-cont., rea., def. & \textbf{0.544} & \xmark \\

        & \textbf{Command-R} & category & 0.535 & 0.633 & in-cont., rea., dir. sti., per. & \textbf{0.651} & in-cont., rea., def. & \textbf{0.633} & \xmark \\
        &  & similar & 0.535 & 0.641 & in-cont., rea., dir. sti. & \textbf{0.668} & in-cont., rea., def. & \textbf{0.639} & \xmark \\
        &  & random & 0.535 & 0.645 & in-cont., rea., def. & \textbf{0.654} & in-cont., rea., def., per. & \textbf{0.650} & \xmark \\

        & \textbf{Llama~3} & category & 0.461 & 0.536 & in-cont. & \textbf{0.599} & in-cont., def., dir. sti. & \textbf{0.599} & \xmark \\
        &  & similar & 0.461 & 0.376 & in-cont. & \textbf{0.605} & in-cont., def. & \textbf{0.594} & \xmark \\
        &  & random & 0.461 & 0.318 & in-cont., def. & \textbf{0.576} & in-cont., def. & \textbf{0.576} & \cmark \\

        \bottomrule
    \end{tabular}%

    \caption{Summary of Shapley Value-based composition selection on the test split for each corpus. For all three datasets, models and in-context demonstration \textit{variants} (category, similar, and random), the table depicts the F$_1$ scores (\textit{Score}) of the composition using no additional techniques ($\nu(\emptyset)$) and all remaining techniques ($\nu(T)$), the \textit{Best-on-Test Composition}, and the composition as determined by the Shapley Values (\textit{SV Composition}). Compositions improving over $\nu(\emptyset)$ and $\nu(T)$ are marked in bold.}
    \label{tab:table-sv-composition-analysis}
\end{table*}

\begin{table*}
    \footnotesize
    \renewcommand{\arraystretch}{.9}
    \centering
    \setlength{\tabcolsep}{1.4pt}

        \begin{tabular}{llllrclrcc}
            \toprule
            &  &  & \multicolumn{3}{c}{\textbf{SV Composition}} & \multicolumn{4}{c}{\textbf{SI Composition (2-SII)}} \\

            \cmidrule(r{2pt}){4-6}\cmidrule(l{2pt}){7-10}

            \textbf{Corpus} & \textbf{Model} & \textbf{Variant} & \textbf{Composition} & \textbf{Score} & \multicolumn{1}{c}{\textbf{Best}} & \textbf{Composition} & \textbf{Score} & \textbf{Best} & \textbf{SI > SV}

            \\\midrule


            \textbf{StereoSet} & \textbf{Mistral} & category & rea. & 0.705 & \xmark & -- \textcolor{gray}{task description only ($\emptyset$)} & \textbf{0.711} & \xmark & \cmark \\
            &  & similar & in-cont., rea., per. & 0.790 & \xmark & in-cont., rea., def.,   dir. sti., per. & \textbf{0.795} & \xmark & \cmark \\
            &  & random & in-cont., rea. & 0.705 & \xmark & in-cont., rea. & 0.705 & \xmark & \neutral \\

            & \textbf{Command-R} & category & def., dir. sti., per. & 0.579 & \xmark & def., per. & \textbf{0.669} & \xmark & \cmark \\
            &  & similar & in-cont., def., dir. sti., per. & 0.582 & \xmark & in-cont., def., per. & \textbf{0.671} & \xmark & \cmark \\
            &  & random & in-cont., def., dir. sti., per. & 0.588 & \xmark & in-cont., def., per. & \textbf{0.667} & \xmark & \cmark \\

            & \textbf{Llama~3} & category & def. & \textbf{0.760} & \cmark & def. & \textbf{0.760} & \cmark & \neutral \\
            &  & similar & in-cont., def., dir. sti. & \textbf{0.798} & \xmark & in-cont., def. & \textbf{0.800} & \xmark & \cmark \\
            &  & random & in-cont., def. & \textbf{0.736} & \xmark & in-cont., def. & \textbf{0.736} & \xmark & \neutral

            \\\midrule


            \textbf{SBIC} & \textbf{Mistral} & category & def., dir. sti., per. & \textbf{0.783} & \cmark & rea., def.,   dir. sti., per. & \textbf{0.771} & \xmark & \xmark \\
            &  & similar & in-cont., def., dir. sti. & 0.758 & \xmark & in-cont., rea., def.,   dir. sti., per. & \textbf{0.770} & \xmark & \cmark \\
            &  & random & in-cont., def., dir. sti. & \textbf{0.792} & \cmark & in-cont., def., dir.   sti. & \textbf{0.792} & \cmark & \neutral \\

            & \textbf{Command-R} & category & def., per. & \textbf{0.767} & \xmark & def., per. & \textbf{0.767} & \xmark & \neutral \\
            &  & similar & in-cont., def. & \textbf{0.770} & \xmark & in-cont., def., per. & \textbf{0.766} & \xmark & \xmark \\
            &  & random & in-cont., def., per. & \textbf{0.788} & \cmark & in-cont., def., per. & \textbf{0.788} & \cmark & \neutral \\

            & \textbf{Llama~3} & category & def. & \textbf{0.810} & \xmark & def. & \textbf{0.810} & \xmark & \neutral \\
            &  & similar & def. & \textbf{0.788} & \xmark & in-cont., def. & \textbf{0.821} & \xmark & \cmark \\
            &  & random & in-cont., def. & \textbf{0.826} & \xmark & in-cont., def. & \textbf{0.826} & \xmark & \neutral

            \\\midrule


            \textbf{Cobra} & \textbf{Mistral} & category & rea., def. & \textbf{0.522} & \xmark & rea., def. & \textbf{0.548} & \cmark & \cmark \\
            &  & similar & in-cont., def. & \textbf{0.604} & \xmark & rea., def. & \textbf{0.548} & \xmark & \xmark \\
            &  & random & in-cont., rea., def. & \textbf{0.544} & \xmark & in-cont., rea.,   def. & \textbf{0.544} & \xmark & \neutral \\

            & \textbf{Command-R} & category & rea., def. & \textbf{0.633} & \xmark & rea., per. & \textbf{0.648} & \xmark & \cmark \\
            &  & similar & in-cont., rea., def. & 0.639 & \xmark & in-cont., rea., dir.   sti., per. & \textbf{0.660} & \xmark & \cmark \\
            &  & random & in-cont., rea., def., per. & \textbf{0.650} & \xmark & in-cont., rea., dir.   sti., per. & \textbf{0.642} & \xmark & \xmark \\

            & \textbf{Llama~3} & category & def., dir. sti. & \textbf{0.599} & \xmark & def., dir.   sti., per. & \textbf{0.573} & \xmark & \xmark \\
            &  & similar & in-cont., def. & \textbf{0.594} & \xmark & in-cont., def., dir.   sti. & \textbf{0.574} & \xmark & \xmark \\
            &  & random & in-cont., def. & \textbf{0.576} & \cmark & in-cont., def. & \textbf{0.576} & \cmark & \neutral \\

            \bottomrule
        \end{tabular}%

    \caption{Summary of Shapley Interaction-based composition selection. For all three corpora, models and in-context demonstration \textit{variants} (category, similar, and random) games, the best composition and its F1 score (\textit{Score}) on the test split are shown for the composition selected with Shapley Values (\textit{SV Composition}) and Shapley Interactions (\textit{SI Composition}). Compositions improving over $\nu(\emptyset)$ and $\nu(T)$ (cf. Table~\ref{tab:table-sv-composition-analysis}) are marked in bold. Notably for StereoSet, compositions selected via Shapley interactions always improve or stay the same compared to compositions selected via the Shapley values.}
    \label{tab:table-si-selection}
\end{table*}

