
\title{Adaptive Prompting: Ad-hoc Prompt Composition for Social Bias Detection}

\author{
    Maximilian Spliethöver \textsuperscript{1},
    Tim Knebler \textsuperscript{1},
    Fabian Fumagalli \textsuperscript{2},
    Maximilian Muschalik \textsuperscript{3},
    \\
    \textbf{Barbara Hammer \textsuperscript{2},}
    \textbf{Eyke Hüllermeier \textsuperscript{3},}
    \textbf{Henning Wachsmuth \textsuperscript{1}}
    \\
    \textsuperscript{1}Leibniz University Hannover, Institute of Artificial Intelligence \\
    \textsuperscript{2}Bielefeld University, CITEC \\
    \textsuperscript{3}LMU Munich, MCML \\
    \tt \href{mailto:m.spliethoever@ai.uni-hannover.de}{m.spliethoever@ai.uni-hannover.de}
}


\maketitle

\begin{abstract}
Recent advances on instruction fine-tuning have led to the development of various prompting techniques for large language models, such as explicit reasoning steps.
However, the success of techniques depends on various parameters, such as the task, language model, and context provided. Finding an effective prompt is, therefore, often a trial-and-error process.
Most existing approaches to automatic prompting aim to optimize individual techniques instead of compositions of techniques and their dependence on the input.
To fill this gap, we propose an \emph{adaptive prompting} approach that predicts the optimal prompt composition ad-hoc for a given input.
We apply our approach to social bias detection, a highly context-dependent task that requires semantic understanding.
We evaluate it with three large language models on three datasets, comparing compositions to individual techniques and other baselines.
The results underline the importance of finding an effective prompt composition. Our approach robustly ensures high detection performance, and is best in several settings. Moreover, first experiments on other tasks support its generalizability.
\end{abstract}
