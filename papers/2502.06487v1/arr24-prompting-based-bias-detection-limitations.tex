\section*{Limitations}
\label{sec:limitations}

For a focused study, we have exclusively modeled the detection of social bias as a binary classification setting. While we have considered multiple facets and settings of social bias by evaluating three datasets that employ diverse settings and definitions, the decision of whether a text elicits social bias or not is often more sophisticated than a binary answer. Our approach might not be applicable to settings requiring more nuanced decisions, but it still can support debiasing decisions or output explanations within respective NLP systems. It should, therefore, serve as a stepping stone to better and more inclusive systems.

As already explained in the main part of the paper, the proposed experiments are exhaustive, and their computational requirements depend on the number of  prompting techniques included, with a near-exponential growth in the inference steps required during training. Future work may aim to abstract from the specific techniques and learn to predict compositions by approximating their importance. We hope that the publication of our experimental data and results pave the way for more efficient adaptive prompting approaches and serve as a training ground to evaluate their feasibility of lowering the computations required.

As our experiments focus on prompt compositions, we do not evaluate lexical variations of the techniques and use a single phrasing per dataset for each technique. While lexical variations can influence the predictions of the model, we think that the general benefit of adaptive prompting still holds for prompting techniques with different phrasing, as the method is independent of the lexical properties of the prompt and rather learns from its predictions.

Lastly, our experimental setting focuses on the task of social bias detection, and the insights presented should, therefore, be considered in this context. However, we think the results are transferable to other tasks in the sense that the benefit of compositions and automatic prediction holds across tasks. Such transfer of the presented approach may require adjustments though, as also discussed in Section~\ref{sec:results}. The task we have focused on further limits our selection of techniques. Including more target tasks in the evaluation could allow for a more diverse selection of techniques, but it also requires a technique-agnostic approach to selecting optimal compositions. We plan to address this aspect in the future.
