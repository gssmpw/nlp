\section{Extended results}
\label{sec:appendix-extended-results}



\subsection{Encoder model evaluation}

To investigate the raw performance of the adaptive prompting model in predicting prompt compositions ad-hoc based on the input text, as detailed in Section~\ref{sec:method}, we evaluate its ability to predict a composition that results in a correct classification (i.e., the optimal composition). This allows for a more direct view at the performance of the encoder model chosen for the approach.

Since our primary interest is an encoder model that is able to predict a composition that produces a correct classification for a given text instance and LLM, we consider all such compositions to be correct predictions of the encoder model ($\#correct\_predictions$). We then simply divide this number by the total number of instances ($\#instances$) in the dataset to calculate a ratio of correct predictions over the full dataset (i.e., $\frac{\#correct\_predictions}{\#instances}$). Like other classification metrics, the score range is $[0,1]$, where $1$ represents the best score. The results are shown in Table~\ref{tab:composition-prediction-performance-results}.

Furthermore, Table~\ref{tab:composition-frequencies-stereoset}, Table~\ref{tab:composition-frequencies-sbic}, and Table~\ref{tab:composition-frequencies-cobra} show the frequencies of how often the adaptive prompting approach chose a specific composition as the optimal composition and how often each composition produced a correct prediction for each model on the train dataset. All frequencies are averaged over five random seeds. This additional data is useful to evaluate, whether the encoder model overfits on the training dataset and simply predicts the most-common composition.



\bsfigure{performance-sbic}{Social bias detection results on SBIC: Macro F$_1$ of all prompt compositions for each LLM.}
\bsfigure{performance-cobra}{Social bias detection results on CobraFrames: Macro F$_1$ of all prompt compositions for each LLM.}




\subsection{Detailed Prompt Composition Results}
Figure~\ref{performance-sbic} and Figure~\ref{performance-cobra} show the boxplots for SBIC and CobraFrames, respectively.

Table~\ref{tab:techniques-results-comparison-sbic} and Table~\ref{tab:techniques-results-comparison-cobra} show a summary of the results, comparing individual techniques and adaptive prompting, similar to Table~\ref{tab:techniques-results-comparison-stereoset}.

Table~\ref{tab:compositions-performance-stereoset}, Table~\ref{tab:compositions-performance-sbic}, and Table~\ref{tab:compositions-performance-cobra} show the results for each evaluated composition on Stereoset, SBIC, and CobraFrames, respectively.



\subsection{Adaptive Prompting for Various Tasks}

Table~\ref{tab:other-tasks-results} shows the results of our adaptive prompting on three further tasks: For sentiment analysis, we use the Aspect Based Sentiment Analysis corpus \cite{pontiki2014}, also referred to as ABSA. For natural language inference, we use the e-SNLI corpus \cite{camburu2018}. Lastly, for question answer, we use the CommonsenseQA corpus \cite{talmor2019}.

We format the prompt as \texttt{<Q> question text <A> answer text}, for which the predicted label indicates whether the answer is correct, given the preceeding question. For both, e-SNLI and CommonsenseQA, we do not include in-context demonstrations based on categories, as this technique is not applicable for their scenarios. Otherwise, all results were retrieved using the same methodology and experimental setup presented in Section~\ref{sec:method} and Section~\ref{sec:experiments}. As LLM, we employ Mistral.

Since all three tasks are notably different from social bias detection and also from each other, the contents of the prompting techniques have been adjusted slightly to fit the task as best as possible. Furthermore, not all prompting techniques are applicable to all three tasks and have been left out in such cases. For example, there are no categories to sample in the natural language inference task, so category demonstrations were not considered.



\begin{table}
    \small
    \centering
    \setlength{\tabcolsep}{3pt}

    \begin{tabular}{lrrrrrr}
        \toprule
        & \multicolumn{2}{c}{\textbf{Training}} & \multicolumn{2}{c}{\textbf{Validation}} & \multicolumn{2}{c}{\textbf{Test}}  \\
        \cmidrule(l@{3pt}r@{3pt}){2-3}\cmidrule(l@{3pt}r@{3pt}){4-5}\cmidrule(l@{3pt}r@{3pt}){6-7}
        \textbf{Corpus} & \multicolumn{1}{l}{\textbf{Pos}}& \multicolumn{1}{l}{\textbf{Neg}} & \multicolumn{1}{l}{\textbf{Pos}}& \multicolumn{1}{l}{\textbf{Neg}} & \multicolumn{1}{l}{\textbf{Pos}} & \multicolumn{1}{l}{\textbf{Neg}} \\
        \midrule

        StereoSet & 1698 & 3397 & 213 & 424 & 212 & 425 \\
        SBIC & 2500 & 2500 & 1806 & 2860 & 1924 & 2767 \\
        CobraFrames & 1780 & 220 & 1779 & 221 & 1862 & 77 \\
        [.5em]
        ABSA & 1907 & 1009 & 240 & 123 & 249 & 111 \\
        ESNLI & 2500 & 2500 & 500 & 500 & 500 & 500 \\
        CommonsenseQA & 2500 & 2500 & 500 & 500 & 500 & 500 \\

        \bottomrule
    \end{tabular}


    \caption{The number of biased (\textit{Pos}) and not biased (\textit{Neg}) text instances per corpus and split.}
    \label{tab:dataset-statistics}
\end{table}



\begin{table}
    \small
    \centering
    \begin{tabular}{l@{}rrr}
        \toprule
        \textbf{LLM} & \textbf{StereoSet} & \textbf{SBIC} & \textbf{CobraFrames} \\
        \midrule
        Mistral & 0.838 & 0.791 & 0.846 \\
        Command-R & 0.801 & 0.759 & 0.833 \\
        Llama~3 & 0.876 & 0.845 & 0.820 \\

        \bottomrule
    \end{tabular}

    \caption{Evaluation results of predicting optimal compositions. The score represents the ratio of predicted compositions that result in a correct classification to the total number of instances, in each dataset. In general, our adaptive prompting model seems to perform best for the Llama~3 and worse for the Command-R.}
    \label{tab:composition-prediction-performance-results}
\end{table}


\begin{table}
    \small
    \centering
    \begin{tabular}{lrrr}
        \toprule
        \textbf{Composition} & \textbf{ABSA} & \textbf{e-SNLI} & \textbf{Comm.QA} \\
        \midrule
        Base composition & 0.906 & 0.963 & 0.747 \\
        [.5em]
        Best on Val & 0.932 & 0.973 & 0.757 \\
        Best on Test & \textbf{0.948} & \textbf{0.976} & \textbf{0.760} \\
        [.5em]
        Adaptive Prompting & \ddag{}*0.938 & \ddag{}0.974 & \dag{}0.759 \\
        \bottomrule
    \end{tabular}

    \caption{Results of the adaptive prompting approach and baselines on aspect based sentiment analysis (\textit{ABSA}), natural language inference (\textit{e-SNLI}), and common sense Q\&A (\textit{Comm.QA}) tasks. While adaptive prompting does not perform best, it produces better classifications than the Best on Val composition on all three tasks, on ABSA even significantly (* for $p<0.05$). It further improves over the base composition significantly (\dag{} for $p<0.05$, \ddag{} for $p<0.01$).}
    \label{tab:other-tasks-results}
\end{table}


\begin{table}
    \small
    \renewcommand{\arraystretch}{.95}
    \centering
    \setlength{\tabcolsep}{1.9pt}
    \begin{tabular}{l@{}rrr}
        \toprule
        \textbf{Composition} & \textbf{Mistral} & \textbf{Command-R} & \textbf{Llama 3} \\
        \midrule
        Base composition & 0.702 & 0.470 & 0.651 \\
        [.5em]
        Definition & 0.740 & 0.554 & 0.788 \\
        Directional stimulus & 0.725 & 0.410 & 0.542 \\
        Persona & 0.703 & 0.512 & 0.710 \\
        Reasoning steps & 0.656 & 0.436 & 0.621 \\
        Demonstrations: Random & 0.747 & 0.763 & 0.825 \\
        Demonstrations: Category & 0.737 & 0.733 & 0.806 \\
        Demonstrations: Similar & 0.712 & 0.729 & 0.822 \\
        [.5em]
        Best on Test & \textbf{0.792} & \textbf{0.788} & 0.831 \\
        [.5em]
        Best SV selection & \textbf{0.792} & \textbf{0.788} & 0.826 \\
        Best SI selection & \textbf{0.792} & \textbf{0.788} & 0.826 \\
        [.5em]
        Adaptive prompting & 0.790 & 0.758 & \ddag{} \textbf{0.842} \\
        \bottomrule
    \end{tabular}

    \caption{Detection performance (macro F$_1$-score) of the prompting techniques per LLM on SBIC. Results marked in bold indicate the best score per LLM. \textit{Best on test} describes the compositions that performs best on the test set for each model. Best SV, and SI selections denote the best compositions based on the Shapley values and Shapley interactions. For Llama~3, adaptive prompting performs significantly better than the best individual composition, \textit{Best on Test} (\ddag{} for $p<.01$).}
    \label{tab:techniques-results-comparison-sbic}
\end{table}

\begin{table}
    \small
    \renewcommand{\arraystretch}{.95}
    \centering
    \setlength{\tabcolsep}{1.9pt}
    \begin{tabular}{l@{}rrr}
        \toprule
        \textbf{Composition} & \textbf{Mistral} & \textbf{Command-R} & \textbf{Llama 3} \\
        \midrule
        Base composition & 0.449 & 0.535 & 0.461 \\
        [.5em]
        Definition & 0.485 & 0.575 & 0.497 \\
        Directional stimulus & 0.422 & 0.438 & 0.340 \\
        Persona & 0.450 & 0.528 & 0.362 \\
        Reasoning steps & 0.535 & 0.589 & 0.417 \\
        Demonstrations: Random & 0.537 & 0.530 & 0.566 \\
        Demonstrations: Category & 0.547 & 0.499 & 0.599 \\
        Demonstrations: Similar & \textbf{0.604} & 0.588 & 0.605 \\
        [.5em]
        Best on Test & \textbf{0.604} & \textbf{0.668} & \textbf{0.605} \\
        [.5em]
        Best SV selection & \textbf{0.604} & 0.650 & 0.599 \\
        Best SI selection & 0.548 & 0.660 & 0.576 \\
        [.5em]
        Adaptive prompting & 0.580 & 0.561 & 0.567 \\
        \bottomrule
    \end{tabular}

    \caption{Detection performance (macro F$_1$-score) of the prompting techniques per LLM on CobraFrames. Results marked in bold indicate the best score per LLM. \textit{Best on test} describes the compositions that performs best on the test set for each model. Best SV, and SI selections denote the best compositions based on the Shapley values and Shapley interactions. On this dataset, adaptive prompting does not improve over \textit{Best on Test}, but notably improves over the base composition and most individual techniques.}
    \label{tab:techniques-results-comparison-cobra}
\end{table}


\tcbset{
    promptbox/.style={%
        notitle,
        nobeforeafter,
        colback=lightgray,%
        fontupper=\ttfamily,%
        boxrule=0pt,
    },
}

\begin{figure*}

    \small

    \begin{tcolorbox}[promptbox]
        \textnormal{\sffamily \textit{Persona prompt}}

        \vspace{.7em}

        You assume the role of a person that reviews texts generated by a generative language model and decides whether a given text makes use of stereotypes or not. This process is called annotating.

        \tcbline

        \textnormal{\sffamily \textit{Task description}}

        \vspace{.7em}

        \#\#\# TASK

        Given the input text, provide an annotation for whether the text contains a stereotype. Use only the labels "yes" (indicating the text contains a stereotype) and "no" (indicating the text does not contain a stereotype).

        Please format your answer as valid JSON. Here is an example of how to format your answer:

        \{

        \,\,\,\,"isBiased": "The prediction if the input text is biased or not."

        \}

        \tcbline

        \textnormal{\sffamily \textit{Social bias definition}}

        \vspace{.7em}

        \#\#\# DEFINITION

        Definition of Stereotype: "A stereotype is an over-generalized belief prevalent in the USA about a particular target group of people. The target group may be stated explicitly or only implicitly. Potential target groups include people with a specific gender, profession, or race, or those belonging to a specific religion. Whether a text uses one or more stereotypes is a categorical variable with two possible annotations: yes and no."

        \tcbline

        \textnormal{\sffamily \textit{In-context demonstrations, with directional stimuli}}

        \vspace{.7em}

        \#\#\# EXAMPLES

        Here are some examples to help you understand the task more clearly:

        \vspace{1.5em}

        \#\#\# INPUT

        <I live in an area with many Hispanics. Where is the compass pointing?>

        HINT: Possible types of biases: 'race', 'gender', 'profession', 'religion', 'no bias'

        \vspace{.5em}

        \#\#\# OUTPUT

        \{

        \,\,\,\,"isBiased": "no"

        \}

        \vspace{1.5em}

        \#\#\# INPUT

        <The engineer was headed out with friends. His friends are all the same profession and wear glasses.>

        HINT: Possible types of biases: 'race', 'gender', 'profession', 'religion', 'no bias'

        \vspace{.5em}

        \#\#\# OUTPUT

        \{

        \,\,\,\,"isBiased": "yes"

        \}

        \vspace{1.5em}

        [...]

        \tcbline

        \textnormal{\sffamily \textit{Input instance}}

        \vspace{.7em}

        \#\#\# INPUT

        <Lebanon is a country next to Israel. It is a peaceful country.>

        HINT: Possible types of biases: 'race', 'gender', 'profession', 'religion', 'no bias'

        \vspace{.5em}

        \#\#\# OUTPUT


    \end{tcolorbox}
    \caption{An example prompt composition for StereoSet bias inference using all prompting techniques evaluated in this study and similarity-based in-context demonstrations. For space reasons only two out of four in-context demonstrations are shown. Italic headings (i.e., \textit{Persona prompt}) and separation lines are added for clarity and are not part of the prompt itself.}
    \label{fig:example-prompt}
\end{figure*}


\begin{table*}
    \scriptsize
    \centering
    \setlength{\tabcolsep}{1.8pt}


    \begin{tabular}{lrrrrrr}
        \toprule
        & \multicolumn{2}{c}{\textbf{Mistral}} & \multicolumn{2}{c}{\textbf{Command-R}} & \multicolumn{2}{c}{\textbf{Llama 3}} \\
        \cmidrule(l){2-3} \cmidrule(l){4-5} \cmidrule(l){6-7}
        \textbf{Composition} & \multicolumn{1}{c}{\textbf{Frequency}} & \multicolumn{1}{c}{\textbf{Correct on train}} & \multicolumn{1}{c}{\textbf{Frequency}} & \multicolumn{1}{c}{\textbf{Correct on train}} & \multicolumn{1}{c}{\textbf{Frequency}} & \multicolumn{1}{c}{\textbf{Correct on train}} \\
        \midrule

        Base composition & 4.4 ($\pm$ 3.88) & 3881.0 ($\pm$ 0.00) & 1.6 ($\pm$ 1.62) & 2478.0 ($\pm$ 0.00) & 63.2 ($\pm$ 23.04) & 3678.0 ($\pm$ 0.00) \\
        Def. & 95.4 ($\pm$ 38.50) & 3882.0 ($\pm$ 0.00) & 4.8 ($\pm$ 5.74) & 2775.0 ($\pm$ 0.00) & 62.8 ($\pm$ 17.42) & 3794.0 ($\pm$ 0.00) \\
        Def., Dir. stim. & 7.0 ($\pm$ 8.81) & 3779.0 ($\pm$ 0.00) & 1.4 ($\pm$ 2.33) & 2592.0 ($\pm$ 0.00) & 4.8 ($\pm$ 6.73) & 3581.0 ($\pm$ 0.00) \\
        Def., Dir. stim., In-cont. (rand.) & 1.2 ($\pm$ 2.40) & 3430.6 ($\pm$ 77.78) & 1.2 ($\pm$ 1.47) & 3062.2 ($\pm$ 111.21) & 0.8 ($\pm$ 1.60) & 3845.6 ($\pm$ 90.81) \\
        Def., Dir. stim., In-cont. (rand.), Pers. & 0.2 ($\pm$ 0.40) & 3397.2 ($\pm$ 89.67) & 0.4 ($\pm$ 0.80) & 3067.8 ($\pm$ 79.37) & 10.4 ($\pm$ 8.91) & 3891.8 ($\pm$ 67.90) \\
        Def., Dir. stim., In-cont. (sim.) & 0.0 ($\pm$ 0.00) & 3966.8 ($\pm$ 83.59) & 5.6 ($\pm$ 6.59) & 3149.2 ($\pm$ 227.69) & 3.8 ($\pm$ 7.11) & 4160.0 ($\pm$ 29.06) \\
        Def., Dir. stim., In-cont. (sim.), Pers. & 0.0 ($\pm$ 0.00) & 3973.8 ($\pm$ 91.82) & 0.6 ($\pm$ 1.20) & 3049.0 ($\pm$ 205.56) & 9.2 ($\pm$ 6.79) & 4174.6 ($\pm$ 17.64) \\
        Def., Dir. stim., Pers. & 4.2 ($\pm$ 8.40) & 3778.0 ($\pm$ 0.00) & 2.2 ($\pm$ 2.99) & 2769.0 ($\pm$ 0.00) & 65.6 ($\pm$ 21.48) & 3517.0 ($\pm$ 0.00) \\
        Def., In-cont. (rand.) & 3.6 ($\pm$ 5.75) & 3628.4 ($\pm$ 121.09) & 9.0 ($\pm$ 9.10) & 3535.4 ($\pm$ 19.70) & 5.4 ($\pm$ 4.08) & 3853.4 ($\pm$ 72.92) \\
        Def., In-cont. (rand.), Pers. & 4.2 ($\pm$ 4.02) & 3584.8 ($\pm$ 135.66) & 8.2 ($\pm$ 9.93) & 3601.2 ($\pm$ 27.85) & 19.6 ($\pm$ 13.60) & 3899.6 ($\pm$ 56.26) \\
        Def., In-cont. (sim.) & 43.8 ($\pm$ 31.47) & 4021.2 ($\pm$ 75.20) & 0.6 ($\pm$ 0.80) & 3500.0 ($\pm$ 156.08) & 18.6 ($\pm$ 12.52) & 4168.6 ($\pm$ 50.06) \\
        Def., In-cont. (sim.), Pers. & 20.4 ($\pm$ 21.91) & 4032.6 ($\pm$ 79.09) & 1.0 ($\pm$ 1.26) & 3473.8 ($\pm$ 125.66) & 7.2 ($\pm$ 3.87) & 4216.2 ($\pm$ 36.04) \\
        Def., Pers. & 80.0 ($\pm$ 67.98) & 3866.0 ($\pm$ 0.00) & 57.0 ($\pm$ 29.11) & 3065.0 ($\pm$ 0.00) & 61.6 ($\pm$ 27.03) & 3564.0 ($\pm$ 0.00) \\
        Dir. stim. & 22.8 ($\pm$ 12.66) & 3749.0 ($\pm$ 0.00) & 0.6 ($\pm$ 1.20) & 3022.0 ($\pm$ 0.00) & 1.0 ($\pm$ 0.63) & 3554.0 ($\pm$ 0.00) \\
        Dir. stim., In-cont. (rand.) & 0.2 ($\pm$ 0.40) & 3430.2 ($\pm$ 77.59) & 0.0 ($\pm$ 0.00) & 3033.2 ($\pm$ 123.68) & 0.2 ($\pm$ 0.40) & 3838.0 ($\pm$ 79.68) \\
        Dir. stim., In-cont. (rand.), Pers. & 0.0 ($\pm$ 0.00) & 3427.6 ($\pm$ 85.84) & 0.0 ($\pm$ 0.00) & 3077.8 ($\pm$ 94.61) & 6.8 ($\pm$ 9.95) & 3895.8 ($\pm$ 72.43) \\
        Dir. stim., In-cont. (sim.) & 0.0 ($\pm$ 0.00) & 3954.8 ($\pm$ 69.29) & 11.2 ($\pm$ 12.42) & 3082.2 ($\pm$ 234.37) & 0.2 ($\pm$ 0.40) & 4151.4 ($\pm$ 32.87) \\
        Dir. stim., In-cont. (sim.), Pers. & 0.0 ($\pm$ 0.00) & 3979.4 ($\pm$ 80.07) & 0.0 ($\pm$ 0.00) & 3002.0 ($\pm$ 184.65) & 7.0 ($\pm$ 4.56) & 4172.6 ($\pm$ 8.59) \\
        Dir. stim., Pers. & 14.2 ($\pm$ 8.89) & 3747.0 ($\pm$ 0.00) & 0.8 ($\pm$ 1.17) & 3152.0 ($\pm$ 0.00) & 6.6 ($\pm$ 3.44) & 3544.0 ($\pm$ 0.00) \\
        In-cont. (cat.) & 14.8 ($\pm$ 14.62) & 3623.8 ($\pm$ 44.75) & 0.0 ($\pm$ 0.00) & 3611.4 ($\pm$ 31.19) & 0.0 ($\pm$ 0.00) & 3895.6 ($\pm$ 21.70) \\
        In-cont. (cat.), Def. & 19.2 ($\pm$ 13.66) & 3721.6 ($\pm$ 58.87) & 0.6 ($\pm$ 1.20) & 3543.4 ($\pm$ 34.66) & 15.6 ($\pm$ 6.92) & 3926.4 ($\pm$ 53.61) \\
        In-cont. (cat.), Def., Dir. stim. & 1.6 ($\pm$ 1.62) & 3545.8 ($\pm$ 69.67) & 0.4 ($\pm$ 0.49) & 3006.2 ($\pm$ 85.86) & 28.8 ($\pm$ 23.47) & 3736.4 ($\pm$ 75.89) \\
        In-cont. (cat.), Def., Dir. stim., Pers. & 2.0 ($\pm$ 1.90) & 3529.8 ($\pm$ 72.69) & 0.6 ($\pm$ 1.20) & 3035.2 ($\pm$ 49.73) & 2.0 ($\pm$ 4.00) & 3880.4 ($\pm$ 46.20) \\
        In-cont. (cat.), Def., Pers. & 20.0 ($\pm$ 9.49) & 3709.8 ($\pm$ 50.93) & 5.8 ($\pm$ 6.73) & 3644.4 ($\pm$ 16.22) & 30.6 ($\pm$ 21.42) & 3976.0 ($\pm$ 20.70) \\
        In-cont. (cat.), Dir. stim. & 1.0 ($\pm$ 1.10) & 3512.0 ($\pm$ 73.52) & 0.0 ($\pm$ 0.00) & 3040.8 ($\pm$ 68.12) & 26.0 ($\pm$ 12.41) & 3670.6 ($\pm$ 67.63) \\
        In-cont. (cat.), Dir. stim., Pers. & 3.0 ($\pm$ 3.35) & 3483.0 ($\pm$ 64.45) & 1.8 ($\pm$ 2.64) & 3165.2 ($\pm$ 25.54) & 1.4 ($\pm$ 1.85) & 3838.8 ($\pm$ 38.47) \\
        In-cont. (cat.), Pers. & 15.6 ($\pm$ 7.50) & 3559.0 ($\pm$ 60.91) & 219.8 ($\pm$ 210.24) & 3732.4 ($\pm$ 11.13) & 24.2 ($\pm$ 29.71) & 3967.0 ($\pm$ 9.70) \\
        In-cont. (rand.) & 0.0 ($\pm$ 0.00) & 3587.8 ($\pm$ 123.86) & 5.0 ($\pm$ 6.26) & 3619.4 ($\pm$ 15.21) & 0.2 ($\pm$ 0.40) & 3849.6 ($\pm$ 84.95) \\
        In-cont. (rand.), Pers. & 0.0 ($\pm$ 0.00) & 3534.0 ($\pm$ 137.21) & 129.4 ($\pm$ 73.56) & 3710.4 ($\pm$ 12.27) & 6.2 ($\pm$ 9.00) & 3903.0 ($\pm$ 57.14) \\
        In-cont. (sim.) & 33.8 ($\pm$ 14.78) & 3936.4 ($\pm$ 91.59) & 1.0 ($\pm$ 1.55) & 3605.8 ($\pm$ 160.67) & 14.4 ($\pm$ 12.75) & 4168.8 ($\pm$ 60.01) \\
        In-cont. (sim.), Pers. & 1.2 ($\pm$ 1.60) & 3989.6 ($\pm$ 75.77) & 45.2 ($\pm$ 23.34) & 3680.0 ($\pm$ 122.72) & 3.2 ($\pm$ 2.48) & 4211.8 ($\pm$ 29.18) \\
        Pers. & 52.0 ($\pm$ 61.65) & 3874.0 ($\pm$ 0.00) & 4.6 ($\pm$ 3.88) & 2903.0 ($\pm$ 0.00) & 61.0 ($\pm$ 50.49) & 3608.0 ($\pm$ 0.00) \\
        Reas., Base composition & 0.0 ($\pm$ 0.00) & 3843.0 ($\pm$ 0.00) & 0.0 ($\pm$ 0.00) & 2546.0 ($\pm$ 0.00) & 0.0 ($\pm$ 0.00) & 3577.0 ($\pm$ 0.00) \\
        Reas., Def. & 0.2 ($\pm$ 0.40) & 3799.0 ($\pm$ 0.00) & 0.0 ($\pm$ 0.00) & 2562.0 ($\pm$ 0.00) & 0.4 ($\pm$ 0.80) & 3614.0 ($\pm$ 0.00) \\
        Reas., Def., Dir. stim. & 0.0 ($\pm$ 0.00) & 3821.0 ($\pm$ 0.00) & 0.0 ($\pm$ 0.00) & 2661.0 ($\pm$ 0.00) & 0.2 ($\pm$ 0.40) & 3531.0 ($\pm$ 0.00) \\
        Reas., Def., Dir. stim., In-cont. (rand.) & 3.6 ($\pm$ 4.45) & 3920.8 ($\pm$ 13.99) & 0.0 ($\pm$ 0.00) & 3081.2 ($\pm$ 196.21) & 0.0 ($\pm$ 0.00) & 3540.2 ($\pm$ 123.88) \\
        Reas., Def., Dir. stim., In-cont. (rand.), Pers. & 10.8 ($\pm$ 10.85) & 3917.6 ($\pm$ 17.87) & 0.2 ($\pm$ 0.40) & 3300.4 ($\pm$ 77.09) & 0.0 ($\pm$ 0.00) & 3332.4 ($\pm$ 94.43) \\
        Reas., Def., Dir. stim., In-cont. (sim.) & 7.2 ($\pm$ 5.42) & 4182.0 ($\pm$ 51.93) & 0.0 ($\pm$ 0.00) & 3018.8 ($\pm$ 331.47) & 0.0 ($\pm$ 0.00) & 4038.0 ($\pm$ 20.73) \\
        Reas., Def., Dir. stim., In-cont. (sim.), Pers. & 24.0 ($\pm$ 27.03) & 4176.0 ($\pm$ 48.08) & 0.0 ($\pm$ 0.00) & 3347.8 ($\pm$ 246.22) & 0.2 ($\pm$ 0.40) & 3575.8 ($\pm$ 195.49) \\
        Reas., Def., Dir. stim., Pers. & 0.0 ($\pm$ 0.00) & 3777.0 ($\pm$ 0.00) & 0.2 ($\pm$ 0.40) & 2698.0 ($\pm$ 0.00) & 0.2 ($\pm$ 0.40) & 3601.0 ($\pm$ 0.00) \\
        Reas., Def., In-cont. (rand.) & 0.0 ($\pm$ 0.00) & 3942.0 ($\pm$ 18.22) & 0.0 ($\pm$ 0.00) & 2965.0 ($\pm$ 262.88) & 0.8 ($\pm$ 0.75) & 3989.4 ($\pm$ 54.03) \\
        Reas., Def., In-cont. (rand.), Pers. & 0.6 ($\pm$ 1.20) & 3933.0 ($\pm$ 20.03) & 0.0 ($\pm$ 0.00) & 3280.0 ($\pm$ 110.75) & 0.0 ($\pm$ 0.00) & 3560.8 ($\pm$ 49.99) \\
        Reas., Def., In-cont. (sim.) & 43.8 ($\pm$ 32.18) & 4203.4 ($\pm$ 44.33) & 3.8 ($\pm$ 3.49) & 2886.2 ($\pm$ 380.58) & 0.2 ($\pm$ 0.40) & 4218.4 ($\pm$ 70.37) \\
        Reas., Def., In-cont. (sim.), Pers. & 9.6 ($\pm$ 6.31) & 4208.0 ($\pm$ 37.07) & 0.0 ($\pm$ 0.00) & 3184.8 ($\pm$ 313.69) & 0.4 ($\pm$ 0.49) & 3440.2 ($\pm$ 122.84) \\
        Reas., Def., Pers. & 0.0 ($\pm$ 0.00) & 3769.0 ($\pm$ 0.00) & 0.0 ($\pm$ 0.00) & 2602.0 ($\pm$ 0.00) & 0.0 ($\pm$ 0.00) & 3647.0 ($\pm$ 0.00) \\
        Reas., Dir. stim. & 0.4 ($\pm$ 0.80) & 3763.0 ($\pm$ 0.00) & 0.0 ($\pm$ 0.00) & 2299.0 ($\pm$ 0.00) & 0.0 ($\pm$ 0.00) & 3433.0 ($\pm$ 0.00) \\
        Reas., Dir. stim., In-cont. (rand.) & 0.0 ($\pm$ 0.00) & 3909.2 ($\pm$ 9.95) & 0.0 ($\pm$ 0.00) & 3200.6 ($\pm$ 169.19) & 0.0 ($\pm$ 0.00) & 3833.4 ($\pm$ 66.10) \\
        Reas., Dir. stim., In-cont. (rand.), Pers. & 0.8 ($\pm$ 0.75) & 3900.0 ($\pm$ 9.01) & 0.0 ($\pm$ 0.00) & 3274.0 ($\pm$ 68.44) & 3.0 ($\pm$ 3.03) & 3307.0 ($\pm$ 227.54) \\
        Reas., Dir. stim., In-cont. (sim.) & 0.8 ($\pm$ 0.75) & 4157.0 ($\pm$ 49.59) & 3.0 ($\pm$ 2.68) & 3131.2 ($\pm$ 307.78) & 0.0 ($\pm$ 0.00) & 4066.4 ($\pm$ 28.19) \\
        Reas., Dir. stim., In-cont. (sim.), Pers. & 12.4 ($\pm$ 6.34) & 4161.2 ($\pm$ 54.88) & 2.2 ($\pm$ 2.14) & 3475.6 ($\pm$ 174.35) & 0.0 ($\pm$ 0.00) & 3555.8 ($\pm$ 175.81) \\
        Reas., Dir. stim., Pers. & 0.2 ($\pm$ 0.40) & 3767.0 ($\pm$ 0.00) & 0.0 ($\pm$ 0.00) & 2551.0 ($\pm$ 0.00) & 0.0 ($\pm$ 0.00) & 3591.0 ($\pm$ 0.00) \\
        Reas., In-cont. (cat.) & 0.2 ($\pm$ 0.40) & 3921.6 ($\pm$ 17.11) & 1.2 ($\pm$ 1.60) & 2364.4 ($\pm$ 174.08) & 20.8 ($\pm$ 9.68) & 3713.0 ($\pm$ 75.57) \\
        Reas., In-cont. (cat.), Def. & 17.2 ($\pm$ 20.47) & 3954.2 ($\pm$ 11.21) & 12.2 ($\pm$ 9.41) & 2281.0 ($\pm$ 169.01) & 6.2 ($\pm$ 4.02) & 3852.4 ($\pm$ 50.70) \\
        Reas., In-cont. (cat.), Def., Dir. stim. & 2.0 ($\pm$ 2.61) & 3902.2 ($\pm$ 18.24) & 9.8 ($\pm$ 5.60) & 2265.0 ($\pm$ 173.62) & 1.8 ($\pm$ 2.14) & 3580.8 ($\pm$ 60.35) \\
        Reas., In-cont. (cat.), Def., Dir. stim., Pers. & 11.6 ($\pm$ 15.73) & 3900.6 ($\pm$ 16.81) & 0.4 ($\pm$ 0.80) & 2954.4 ($\pm$ 233.83) & 0.0 ($\pm$ 0.00) & 3056.2 ($\pm$ 23.89) \\
        Reas., In-cont. (cat.), Def., Pers. & 2.6 ($\pm$ 2.80) & 3932.6 ($\pm$ 12.08) & 4.2 ($\pm$ 6.21) & 2729.0 ($\pm$ 273.41) & 0.0 ($\pm$ 0.00) & 3404.0 ($\pm$ 53.97) \\
        Reas., In-cont. (cat.), Dir. stim. & 6.8 ($\pm$ 8.38) & 3901.0 ($\pm$ 16.94) & 53.0 ($\pm$ 27.00) & 2240.2 ($\pm$ 164.96) & 27.0 ($\pm$ 18.74) & 3640.0 ($\pm$ 89.15) \\
        Reas., In-cont. (cat.), Dir. stim., Pers. & 0.8 ($\pm$ 0.75) & 3899.8 ($\pm$ 12.70) & 0.8 ($\pm$ 1.17) & 2966.8 ($\pm$ 201.45) & 2.0 ($\pm$ 1.41) & 3022.0 ($\pm$ 103.54) \\
        Reas., In-cont. (cat.), Pers. & 0.2 ($\pm$ 0.40) & 3916.4 ($\pm$ 13.31) & 0.0 ($\pm$ 0.00) & 2801.2 ($\pm$ 177.34) & 0.8 ($\pm$ 0.75) & 3134.4 ($\pm$ 33.73) \\
        Reas., In-cont. (rand.) & 0.2 ($\pm$ 0.40) & 3914.4 ($\pm$ 18.53) & 0.0 ($\pm$ 0.00) & 3099.0 ($\pm$ 216.56) & 0.0 ($\pm$ 0.00) & 4014.0 ($\pm$ 22.34) \\
        Reas., In-cont. (rand.), Pers. & 2.6 ($\pm$ 3.77) & 3906.8 ($\pm$ 16.44) & 0.2 ($\pm$ 0.40) & 3363.8 ($\pm$ 79.98) & 0.2 ($\pm$ 0.40) & 3388.2 ($\pm$ 100.07) \\
        Reas., In-cont. (sim.) & 1.6 ($\pm$ 3.20) & 4182.8 ($\pm$ 43.51) & 12.0 ($\pm$ 8.90) & 2808.4 ($\pm$ 395.70) & 4.4 ($\pm$ 4.22) & 4193.2 ($\pm$ 79.36) \\
        Reas., In-cont. (sim.), Pers. & 11.0 ($\pm$ 9.49) & 4179.6 ($\pm$ 49.04) & 13.4 ($\pm$ 7.39) & 3416.4 ($\pm$ 223.79) & 0.0 ($\pm$ 0.00) & 3396.0 ($\pm$ 156.29) \\
        Reas., Pers. & 0.0 ($\pm$ 0.00) & 3767.0 ($\pm$ 0.00) & 0.0 ($\pm$ 0.00) & 2571.0 ($\pm$ 0.00) & 0.0 ($\pm$ 0.00) & 3675.0 ($\pm$ 0.00) \\
        \bottomrule
    \end{tabular}

    \caption{Frequencies of how often each composition was chosen as optimal composition by our adaptive prompting approach per LLM on StereoSet. Frequencies are averaged over five random seeds. Possible techniques for a composition are a defintion (\textit{Def.}), a directional stimulus (\textit{Dir. stim.}), In-context examples chosen randomly (\textit{In-cont. (rand.)}), based on similarity (\textit{In-cont. (sim.)}) or based on their category (\textit{In-cont. (cat.)}), a persona (\textit{Pers.}), and reasoning steps (\textit{Reas.}). The \textit{Base Composition} consists of a task description and text input.}
    \label{tab:composition-frequencies-stereoset}
\end{table*}

\begin{table*}
    \scriptsize
    \centering
    \setlength{\tabcolsep}{1.8pt}


    \begin{tabular}{lrrrrrr}
        \toprule
        & \multicolumn{2}{c}{\textbf{Mistral}} & \multicolumn{2}{c}{\textbf{Command-R}} & \multicolumn{2}{c}{\textbf{Llama 3}} \\
        \cmidrule(l){2-3} \cmidrule(l){4-5} \cmidrule(l){6-7}
        \textbf{Composition} & \multicolumn{1}{c}{\textbf{Frequency}} & \multicolumn{1}{c}{\textbf{Correct on train}} & \multicolumn{1}{c}{\textbf{Frequency}} & \multicolumn{1}{c}{\textbf{Correct on train}} & \multicolumn{1}{c}{\textbf{Frequency}} & \multicolumn{1}{c}{\textbf{Correct on train}} \\
        \midrule

        Base composition & 0.0 ($\pm$ 0.00) & 3381.0 ($\pm$ 0.00) & 1.4 ($\pm$ 2.80) & 2869.0 ($\pm$ 0.00) & 109.8 ($\pm$ 119.02) & 2951.0 ($\pm$ 0.00) \\
        Def. & 75.8 ($\pm$ 75.39) & 3434.0 ($\pm$ 0.00) & 15.0 ($\pm$ 10.35) & 3148.0 ($\pm$ 0.00) & 31.8 ($\pm$ 36.48) & 3455.0 ($\pm$ 0.00) \\
        Def., Dir. stim. & 310.4 ($\pm$ 176.40) & 3423.0 ($\pm$ 0.00) & 483.8 ($\pm$ 595.26) & 2843.0 ($\pm$ 0.00) & 6.2 ($\pm$ 7.36) & 3375.0 ($\pm$ 0.00) \\
        Def., Dir. stim., In-cont. (rand.) & 71.8 ($\pm$ 94.93) & 3766.8 ($\pm$ 13.17) & 251.8 ($\pm$ 146.34) & 3713.8 ($\pm$ 23.54) & 44.6 ($\pm$ 34.67) & 3884.2 ($\pm$ 25.26) \\
        Def., Dir. stim., In-cont. (rand.), Pers. & 211.4 ($\pm$ 253.74) & 3771.4 ($\pm$ 13.41) & 71.8 ($\pm$ 88.18) & 3715.8 ($\pm$ 11.79) & 362.6 ($\pm$ 389.01) & 3884.4 ($\pm$ 14.29) \\
        Def., Dir. stim., In-cont. (sim.) & 173.0 ($\pm$ 134.10) & 3720.6 ($\pm$ 17.17) & 0.4 ($\pm$ 0.80) & 3759.4 ($\pm$ 52.81) & 0.4 ($\pm$ 0.80) & 3890.0 ($\pm$ 16.35) \\
        Def., Dir. stim., In-cont. (sim.), Pers. & 23.0 ($\pm$ 21.60) & 3700.0 ($\pm$ 30.04) & 17.8 ($\pm$ 17.96) & 3749.0 ($\pm$ 56.33) & 0.0 ($\pm$ 0.00) & 3847.6 ($\pm$ 14.33) \\
        Def., Dir. stim., Pers. & 269.0 ($\pm$ 207.05) & 3379.0 ($\pm$ 0.00) & 1156.0 ($\pm$ 432.61) & 2893.0 ($\pm$ 0.00) & 111.0 ($\pm$ 55.31) & 3278.0 ($\pm$ 0.00) \\
        Def., In-cont. (rand.) & 57.4 ($\pm$ 110.81) & 3638.8 ($\pm$ 13.47) & 561.2 ($\pm$ 277.52) & 3733.8 ($\pm$ 22.82) & 126.2 ($\pm$ 162.04) & 3866.2 ($\pm$ 10.53) \\
        Def., In-cont. (rand.), Pers. & 0.0 ($\pm$ 0.00) & 3622.4 ($\pm$ 17.62) & 483.6 ($\pm$ 228.38) & 3739.2 ($\pm$ 29.25) & 132.8 ($\pm$ 76.93) & 3862.6 ($\pm$ 16.56) \\
        Def., In-cont. (sim.) & 67.4 ($\pm$ 37.81) & 3628.2 ($\pm$ 23.44) & 4.8 ($\pm$ 6.18) & 3734.4 ($\pm$ 20.22) & 0.0 ($\pm$ 0.00) & 3893.8 ($\pm$ 31.08) \\
        Def., In-cont. (sim.), Pers. & 71.0 ($\pm$ 56.57) & 3611.0 ($\pm$ 24.82) & 93.8 ($\pm$ 38.15) & 3739.0 ($\pm$ 16.43) & 0.0 ($\pm$ 0.00) & 3891.6 ($\pm$ 37.91) \\
        Def., Pers. & 4.8 ($\pm$ 3.66) & 3416.0 ($\pm$ 0.00) & 0.0 ($\pm$ 0.00) & 3233.0 ($\pm$ 0.00) & 10.0 ($\pm$ 17.54) & 3346.0 ($\pm$ 0.00) \\
        Dir. stim. & 68.2 ($\pm$ 52.91) & 3371.0 ($\pm$ 0.00) & 61.8 ($\pm$ 91.64) & 2788.0 ($\pm$ 0.00) & 441.6 ($\pm$ 215.48) & 2739.0 ($\pm$ 0.00) \\
        Dir. stim., In-cont. (rand.) & 1478.6 ($\pm$ 718.41) & 3736.8 ($\pm$ 6.05) & 113.0 ($\pm$ 61.18) & 3676.0 ($\pm$ 31.98) & 266.0 ($\pm$ 432.49) & 3888.4 ($\pm$ 17.64) \\
        Dir. stim., In-cont. (rand.), Pers. & 588.6 ($\pm$ 1051.07) & 3722.2 ($\pm$ 7.33) & 10.8 ($\pm$ 19.12) & 3679.8 ($\pm$ 33.52) & 315.2 ($\pm$ 289.25) & 3886.8 ($\pm$ 16.22) \\
        Dir. stim., In-cont. (sim.) & 25.6 ($\pm$ 38.42) & 3686.6 ($\pm$ 19.48) & 0.0 ($\pm$ 0.00) & 3713.8 ($\pm$ 72.52) & 0.2 ($\pm$ 0.40) & 3890.0 ($\pm$ 31.99) \\
        Dir. stim., In-cont. (sim.), Pers. & 14.4 ($\pm$ 18.18) & 3658.0 ($\pm$ 19.03) & 3.4 ($\pm$ 5.43) & 3691.4 ($\pm$ 85.14) & 0.0 ($\pm$ 0.00) & 3843.8 ($\pm$ 14.58) \\
        Dir. stim., Pers. & 10.6 ($\pm$ 6.86) & 3324.0 ($\pm$ 0.00) & 1.6 ($\pm$ 2.73) & 2896.0 ($\pm$ 0.00) & 0.0 ($\pm$ 0.00) & 2979.0 ($\pm$ 0.00) \\
        In-cont. (cat.) & 0.0 ($\pm$ 0.00) & 3575.2 ($\pm$ 10.93) & 0.0 ($\pm$ 0.00) & 3590.4 ($\pm$ 28.08) & 0.6 ($\pm$ 1.20) & 3874.6 ($\pm$ 21.11) \\
        In-cont. (cat.), Def. & 0.0 ($\pm$ 0.00) & 3650.2 ($\pm$ 8.52) & 16.0 ($\pm$ 16.88) & 3715.0 ($\pm$ 25.02) & 4.6 ($\pm$ 9.20) & 3871.8 ($\pm$ 14.59) \\
        In-cont. (cat.), Def., Dir. stim. & 47.2 ($\pm$ 44.24) & 3757.0 ($\pm$ 18.84) & 97.6 ($\pm$ 140.99) & 3717.2 ($\pm$ 26.27) & 494.4 ($\pm$ 287.13) & 3881.0 ($\pm$ 9.19) \\
        In-cont. (cat.), Def., Dir. stim., Pers. & 18.2 ($\pm$ 23.09) & 3762.2 ($\pm$ 15.95) & 243.6 ($\pm$ 123.58) & 3710.4 ($\pm$ 23.89) & 366.8 ($\pm$ 264.01) & 3911.8 ($\pm$ 9.17) \\
        In-cont. (cat.), Def., Pers. & 88.6 ($\pm$ 177.20) & 3630.4 ($\pm$ 15.33) & 104.4 ($\pm$ 79.71) & 3722.8 ($\pm$ 18.89) & 4.8 ($\pm$ 3.82) & 3903.2 ($\pm$ 8.38) \\
        In-cont. (cat.), Dir. stim. & 562.0 ($\pm$ 603.91) & 3724.8 ($\pm$ 13.61) & 2.6 ($\pm$ 3.56) & 3646.0 ($\pm$ 18.99) & 619.6 ($\pm$ 216.51) & 3874.6 ($\pm$ 9.58) \\
        In-cont. (cat.), Dir. stim., Pers. & 94.8 ($\pm$ 108.96) & 3713.6 ($\pm$ 16.03) & 21.4 ($\pm$ 33.91) & 3647.6 ($\pm$ 22.12) & 311.0 ($\pm$ 209.18) & 3905.0 ($\pm$ 9.27) \\
        In-cont. (cat.), Pers. & 0.0 ($\pm$ 0.00) & 3557.2 ($\pm$ 18.02) & 0.0 ($\pm$ 0.00) & 3601.6 ($\pm$ 26.22) & 0.0 ($\pm$ 0.00) & 3886.0 ($\pm$ 19.75) \\
        In-cont. (rand.) & 0.0 ($\pm$ 0.00) & 3569.8 ($\pm$ 16.62) & 31.2 ($\pm$ 20.76) & 3590.2 ($\pm$ 6.88) & 32.4 ($\pm$ 43.52) & 3805.6 ($\pm$ 24.27) \\
        In-cont. (rand.), Pers. & 0.0 ($\pm$ 0.00) & 3553.8 ($\pm$ 14.59) & 21.2 ($\pm$ 26.36) & 3611.6 ($\pm$ 10.03) & 165.0 ($\pm$ 209.18) & 3804.6 ($\pm$ 31.34) \\
        In-cont. (sim.) & 98.4 ($\pm$ 59.56) & 3545.2 ($\pm$ 31.98) & 36.6 ($\pm$ 20.53) & 3604.8 ($\pm$ 38.50) & 0.0 ($\pm$ 0.00) & 3854.4 ($\pm$ 35.49) \\
        In-cont. (sim.), Pers. & 164.4 ($\pm$ 98.63) & 3522.6 ($\pm$ 27.88) & 9.6 ($\pm$ 9.89) & 3599.2 ($\pm$ 35.19) & 0.0 ($\pm$ 0.00) & 3841.8 ($\pm$ 46.71) \\
        Pers. & 0.2 ($\pm$ 0.40) & 3346.0 ($\pm$ 0.00) & 0.0 ($\pm$ 0.00) & 2980.0 ($\pm$ 0.00) & 8.4 ($\pm$ 6.74) & 3079.0 ($\pm$ 0.00) \\
        Reas., Base composition & 1.0 ($\pm$ 1.26) & 3264.0 ($\pm$ 0.00) & 4.2 ($\pm$ 2.93) & 2782.0 ($\pm$ 0.00) & 1.2 ($\pm$ 1.94) & 2899.0 ($\pm$ 0.00) \\
        Reas., Def. & 0.0 ($\pm$ 0.00) & 3345.0 ($\pm$ 0.00) & 0.4 ($\pm$ 0.80) & 2959.0 ($\pm$ 0.00) & 10.2 ($\pm$ 8.73) & 3100.0 ($\pm$ 0.00) \\
        Reas., Def., Dir. stim. & 1.8 ($\pm$ 2.71) & 3368.0 ($\pm$ 0.00) & 2.4 ($\pm$ 3.38) & 2918.0 ($\pm$ 0.00) & 4.6 ($\pm$ 4.96) & 3366.0 ($\pm$ 0.00) \\
        Reas., Def., Dir. stim., In-cont. (rand.) & 1.4 ($\pm$ 2.33) & 3685.0 ($\pm$ 12.03) & 0.0 ($\pm$ 0.00) & 3684.0 ($\pm$ 41.14) & 0.0 ($\pm$ 0.00) & 3189.8 ($\pm$ 112.60) \\
        Reas., Def., Dir. stim., In-cont. (rand.), Pers. & 1.2 ($\pm$ 2.40) & 3675.8 ($\pm$ 12.89) & 74.8 ($\pm$ 81.22) & 3634.8 ($\pm$ 33.55) & 7.4 ($\pm$ 7.36) & 2691.0 ($\pm$ 103.20) \\
        Reas., Def., Dir. stim., In-cont. (sim.) & 0.0 ($\pm$ 0.00) & 3732.8 ($\pm$ 8.08) & 12.8 ($\pm$ 6.76) & 3734.2 ($\pm$ 49.77) & 0.0 ($\pm$ 0.00) & 3025.8 ($\pm$ 179.18) \\
        Reas., Def., Dir. stim., In-cont. (sim.), Pers. & 0.4 ($\pm$ 0.49) & 3717.2 ($\pm$ 11.07) & 0.6 ($\pm$ 1.20) & 3667.4 ($\pm$ 42.32) & 0.0 ($\pm$ 0.00) & 2671.2 ($\pm$ 115.61) \\
        Reas., Def., Dir. stim., Pers. & 0.4 ($\pm$ 0.80) & 3364.0 ($\pm$ 0.00) & 12.2 ($\pm$ 12.43) & 2839.0 ($\pm$ 0.00) & 1.0 ($\pm$ 2.00) & 3381.0 ($\pm$ 0.00) \\
        Reas., Def., In-cont. (rand.) & 0.2 ($\pm$ 0.40) & 3678.8 ($\pm$ 19.84) & 0.8 ($\pm$ 0.75) & 3692.2 ($\pm$ 23.63) & 2.4 ($\pm$ 4.80) & 3190.4 ($\pm$ 103.48) \\
        Reas., Def., In-cont. (rand.), Pers. & 0.0 ($\pm$ 0.00) & 3672.6 ($\pm$ 22.57) & 0.2 ($\pm$ 0.40) & 3697.2 ($\pm$ 20.01) & 612.6 ($\pm$ 244.36) & 2814.2 ($\pm$ 229.97) \\
        Reas., Def., In-cont. (sim.) & 0.6 ($\pm$ 0.49) & 3732.4 ($\pm$ 14.14) & 20.0 ($\pm$ 18.41) & 3741.2 ($\pm$ 33.27) & 0.6 ($\pm$ 0.80) & 3058.2 ($\pm$ 215.31) \\
        Reas., Def., In-cont. (sim.), Pers. & 12.0 ($\pm$ 14.04) & 3708.4 ($\pm$ 20.87) & 0.8 ($\pm$ 0.75) & 3712.0 ($\pm$ 37.16) & 0.8 ($\pm$ 1.60) & 2607.8 ($\pm$ 64.03) \\
        Reas., Def., Pers. & 0.0 ($\pm$ 0.00) & 3312.0 ($\pm$ 0.00) & 16.8 ($\pm$ 11.84) & 2873.0 ($\pm$ 0.00) & 2.6 ($\pm$ 4.27) & 3368.0 ($\pm$ 0.00) \\
        Reas., Dir. stim. & 0.0 ($\pm$ 0.00) & 3210.0 ($\pm$ 0.00) & 0.0 ($\pm$ 0.00) & 2958.0 ($\pm$ 0.00) & 0.8 ($\pm$ 0.75) & 2998.0 ($\pm$ 0.00) \\
        Reas., Dir. stim., In-cont. (rand.) & 38.0 ($\pm$ 60.00) & 3683.6 ($\pm$ 15.08) & 9.8 ($\pm$ 17.21) & 3617.0 ($\pm$ 35.59) & 0.0 ($\pm$ 0.00) & 2977.4 ($\pm$ 95.36) \\
        Reas., Dir. stim., In-cont. (rand.), Pers. & 28.4 ($\pm$ 51.91) & 3669.4 ($\pm$ 11.43) & 505.0 ($\pm$ 246.79) & 3560.8 ($\pm$ 41.25) & 7.6 ($\pm$ 8.21) & 2574.6 ($\pm$ 45.69) \\
        Reas., Dir. stim., In-cont. (sim.) & 1.6 ($\pm$ 1.62) & 3729.4 ($\pm$ 7.71) & 21.2 ($\pm$ 29.57) & 3635.6 ($\pm$ 53.15) & 0.0 ($\pm$ 0.00) & 2918.0 ($\pm$ 205.98) \\
        Reas., Dir. stim., In-cont. (sim.), Pers. & 3.2 ($\pm$ 4.96) & 3706.8 ($\pm$ 9.83) & 18.8 ($\pm$ 8.42) & 3578.8 ($\pm$ 36.93) & 0.0 ($\pm$ 0.00) & 2541.8 ($\pm$ 56.10) \\
        Reas., Dir. stim., Pers. & 2.4 ($\pm$ 3.38) & 3304.0 ($\pm$ 0.00) & 1.4 ($\pm$ 2.33) & 2822.0 ($\pm$ 0.00) & 0.4 ($\pm$ 0.49) & 3310.0 ($\pm$ 0.00) \\
        Reas., In-cont. (cat.) & 1.2 ($\pm$ 1.60) & 3686.0 ($\pm$ 9.32) & 0.0 ($\pm$ 0.00) & 3137.4 ($\pm$ 51.58) & 1.2 ($\pm$ 1.17) & 3116.6 ($\pm$ 49.43) \\
        Reas., In-cont. (cat.), Def. & 0.0 ($\pm$ 0.00) & 3690.6 ($\pm$ 15.79) & 23.0 ($\pm$ 9.32) & 3436.2 ($\pm$ 41.38) & 0.0 ($\pm$ 0.00) & 3208.4 ($\pm$ 75.30) \\
        Reas., In-cont. (cat.), Def., Dir. stim. & 0.0 ($\pm$ 0.00) & 3683.8 ($\pm$ 12.91) & 1.8 ($\pm$ 2.23) & 3447.0 ($\pm$ 64.46) & 4.0 ($\pm$ 5.33) & 3309.2 ($\pm$ 42.79) \\
        Reas., In-cont. (cat.), Def., Dir. stim., Pers. & 0.0 ($\pm$ 0.00) & 3657.8 ($\pm$ 7.25) & 0.0 ($\pm$ 0.00) & 3718.6 ($\pm$ 13.29) & 0.4 ($\pm$ 0.49) & 2793.8 ($\pm$ 87.76) \\
        Reas., In-cont. (cat.), Def., Pers. & 0.2 ($\pm$ 0.40) & 3679.8 ($\pm$ 15.00) & 7.0 ($\pm$ 6.07) & 3748.6 ($\pm$ 15.62) & 0.2 ($\pm$ 0.40) & 2876.6 ($\pm$ 54.19) \\
        Reas., In-cont. (cat.), Dir. stim. & 0.2 ($\pm$ 0.40) & 3685.6 ($\pm$ 14.89) & 0.0 ($\pm$ 0.00) & 3388.0 ($\pm$ 61.31) & 22.6 ($\pm$ 2.58) & 2957.2 ($\pm$ 64.31) \\
        Reas., In-cont. (cat.), Dir. stim., Pers. & 0.6 ($\pm$ 1.20) & 3666.2 ($\pm$ 13.89) & 0.0 ($\pm$ 0.00) & 3666.4 ($\pm$ 38.76) & 0.6 ($\pm$ 0.80) & 2718.6 ($\pm$ 49.47) \\
        Reas., In-cont. (cat.), Pers. & 0.2 ($\pm$ 0.40) & 3674.2 ($\pm$ 22.96) & 0.0 ($\pm$ 0.00) & 3673.8 ($\pm$ 41.86) & 0.0 ($\pm$ 0.00) & 2929.8 ($\pm$ 59.57) \\
        Reas., In-cont. (rand.) & 0.6 ($\pm$ 0.80) & 3670.4 ($\pm$ 5.12) & 0.2 ($\pm$ 0.40) & 3633.8 ($\pm$ 36.64) & 0.0 ($\pm$ 0.00) & 3206.8 ($\pm$ 93.68) \\
        Reas., In-cont. (rand.), Pers. & 0.2 ($\pm$ 0.40) & 3659.0 ($\pm$ 9.88) & 0.0 ($\pm$ 0.00) & 3622.6 ($\pm$ 20.58) & 41.6 ($\pm$ 24.20) & 2728.2 ($\pm$ 175.78) \\
        Reas., In-cont. (sim.) & 0.2 ($\pm$ 0.40) & 3708.6 ($\pm$ 14.61) & 11.0 ($\pm$ 14.68) & 3655.2 ($\pm$ 25.59) & 0.2 ($\pm$ 0.40) & 3189.2 ($\pm$ 172.89) \\
        Reas., In-cont. (sim.), Pers. & 0.0 ($\pm$ 0.00) & 3687.4 ($\pm$ 21.56) & 16.0 ($\pm$ 15.74) & 3645.2 ($\pm$ 21.33) & 0.0 ($\pm$ 0.00) & 2582.4 ($\pm$ 62.88) \\
        Reas., Pers. & 0.2 ($\pm$ 0.40) & 3281.0 ($\pm$ 0.00) & 13.6 ($\pm$ 7.89) & 2727.0 ($\pm$ 0.00) & 2.0 ($\pm$ 1.67) & 3115.0 ($\pm$ 0.00) \\

        \bottomrule
    \end{tabular}

    \caption{Frequencies of how often each composition was chosen as optimal composition by our adaptive prompting approach per LLM on SBIC. Frequencies are averaged over five random seeds. Possible techniques for a composition  are a defintion (\textit{Def.}), a directional stimulus (\textit{Dir. stim.}), In-context examples chosen randomly (\textit{In-cont. (rand.)}), based on similarity (\textit{In-cont. (sim.)}) or based on their category (\textit{In-cont. (cat.)}), a persona (\textit{Pers.}), and reasoning steps (\textit{Reas.}). The \textit{Base Composition} consists of a task description and text input.}
    \label{tab:composition-frequencies-sbic}
\end{table*}

\begin{table*}
    \scriptsize
    \centering
    \setlength{\tabcolsep}{1.8pt}


    \begin{tabular}{lrrrrrr}
        \toprule
        & \multicolumn{2}{c}{\textbf{Mistral}} & \multicolumn{2}{c}{\textbf{Command-R}} & \multicolumn{2}{c}{\textbf{Llama 3}} \\
        \cmidrule(l){2-3} \cmidrule(l){4-5} \cmidrule(l){6-7}
        \textbf{Composition} & \multicolumn{1}{c}{\textbf{Frequency}} & \multicolumn{1}{c}{\textbf{Correct on train}} & \multicolumn{1}{c}{\textbf{Frequency}} & \multicolumn{1}{c}{\textbf{Correct on train}} & \multicolumn{1}{c}{\textbf{Frequency}} & \multicolumn{1}{c}{\textbf{Correct on train}} \\
        \midrule

        Base composition & 0.0 ($\pm$ 0.00) & 1529.0 ($\pm$ 0.00) & 0.0 ($\pm$ 0.00) & 1700.0 ($\pm$ 0.00) & 0.0 ($\pm$ 0.00) & 1224.0 ($\pm$ 0.00) \\
        Def. & 0.0 ($\pm$ 0.00) & 1540.0 ($\pm$ 0.00) & 0.0 ($\pm$ 0.00) & 1741.0 ($\pm$ 0.00) & 0.0 ($\pm$ 0.00) & 1326.0 ($\pm$ 0.00) \\
        Def., Dir. stim. & 0.0 ($\pm$ 0.00) & 1489.0 ($\pm$ 0.00) & 0.0 ($\pm$ 0.00) & 1663.0 ($\pm$ 0.00) & 0.0 ($\pm$ 0.00) & 1549.0 ($\pm$ 0.00) \\
        Def., Dir. stim., In-cont. (rand.) & 5.2 ($\pm$ 10.40) & 1447.2 ($\pm$ 22.27) & 5.2 ($\pm$ 10.40) & 1616.2 ($\pm$ 12.50) & 5.2 ($\pm$ 10.40) & 1494.4 ($\pm$ 70.63) \\
        Def., Dir. stim., In-cont. (rand.), Pers. & 0.0 ($\pm$ 0.00) & 1440.4 ($\pm$ 18.19) & 0.0 ($\pm$ 0.00) & 1638.4 ($\pm$ 12.11) & 0.0 ($\pm$ 0.00) & 1421.8 ($\pm$ 78.79) \\
        Def., Dir. stim., In-cont. (sim.) & 0.0 ($\pm$ 0.00) & 1508.4 ($\pm$ 22.17) & 0.0 ($\pm$ 0.00) & 1680.0 ($\pm$ 19.71) & 0.0 ($\pm$ 0.00) & 1513.0 ($\pm$ 71.25) \\
        Def., Dir. stim., In-cont. (sim.), Pers. & 0.0 ($\pm$ 0.00) & 1496.0 ($\pm$ 24.55) & 0.0 ($\pm$ 0.00) & 1692.8 ($\pm$ 13.66) & 0.0 ($\pm$ 0.00) & 1447.8 ($\pm$ 76.72) \\
        Def., Dir. stim., Pers. & 0.0 ($\pm$ 0.00) & 1511.0 ($\pm$ 0.00) & 0.0 ($\pm$ 0.00) & 1620.0 ($\pm$ 0.00) & 0.0 ($\pm$ 0.00) & 1660.0 ($\pm$ 0.00) \\
        Def., In-cont. (rand.) & 0.0 ($\pm$ 0.00) & 1529.0 ($\pm$ 29.82) & 0.0 ($\pm$ 0.00) & 1563.2 ($\pm$ 35.64) & 0.0 ($\pm$ 0.00) & 1601.2 ($\pm$ 42.71) \\
        Def., In-cont. (rand.), Pers. & 0.0 ($\pm$ 0.00) & 1521.8 ($\pm$ 29.74) & 0.0 ($\pm$ 0.00) & 1593.6 ($\pm$ 28.30) & 0.0 ($\pm$ 0.00) & 1542.6 ($\pm$ 37.98) \\
        Def., In-cont. (sim.) & 323.2 ($\pm$ 310.68) & 1597.4 ($\pm$ 23.65) & 323.2 ($\pm$ 310.68) & 1644.8 ($\pm$ 27.93) & 323.2 ($\pm$ 310.68) & 1617.0 ($\pm$ 16.30) \\
        Def., In-cont. (sim.), Pers. & 177.8 ($\pm$ 125.53) & 1592.2 ($\pm$ 20.23) & 177.8 ($\pm$ 125.53) & 1665.0 ($\pm$ 22.18) & 177.8 ($\pm$ 125.53) & 1538.4 ($\pm$ 11.41) \\
        Def., Pers. & 0.0 ($\pm$ 0.00) & 1558.0 ($\pm$ 0.00) & 0.0 ($\pm$ 0.00) & 1680.0 ($\pm$ 0.00) & 0.0 ($\pm$ 0.00) & 1205.0 ($\pm$ 0.00) \\
        Dir. stim. & 0.0 ($\pm$ 0.00) & 1463.0 ($\pm$ 0.00) & 0.0 ($\pm$ 0.00) & 1487.0 ($\pm$ 0.00) & 0.0 ($\pm$ 0.00) & 1402.0 ($\pm$ 0.00) \\
        Dir. stim., In-cont. (rand.) & 0.0 ($\pm$ 0.00) & 1424.4 ($\pm$ 23.10) & 0.0 ($\pm$ 0.00) & 1606.2 ($\pm$ 19.33) & 0.0 ($\pm$ 0.00) & 1471.0 ($\pm$ 67.20) \\
        Dir. stim., In-cont. (rand.), Pers. & 0.0 ($\pm$ 0.00) & 1419.2 ($\pm$ 20.54) & 0.0 ($\pm$ 0.00) & 1615.8 ($\pm$ 13.04) & 0.0 ($\pm$ 0.00) & 1431.0 ($\pm$ 70.65) \\
        Dir. stim., In-cont. (sim.) & 6.0 ($\pm$ 12.00) & 1494.6 ($\pm$ 18.75) & 6.0 ($\pm$ 12.00) & 1675.0 ($\pm$ 16.02) & 6.0 ($\pm$ 12.00) & 1508.4 ($\pm$ 63.90) \\
        Dir. stim., In-cont. (sim.), Pers. & 2.2 ($\pm$ 4.40) & 1473.8 ($\pm$ 24.51) & 2.2 ($\pm$ 4.40) & 1683.2 ($\pm$ 12.81) & 2.2 ($\pm$ 4.40) & 1464.2 ($\pm$ 69.65) \\
        Dir. stim., Pers. & 0.0 ($\pm$ 0.00) & 1531.0 ($\pm$ 0.00) & 0.0 ($\pm$ 0.00) & 1479.0 ($\pm$ 0.00) & 0.0 ($\pm$ 0.00) & 1278.0 ($\pm$ 0.00) \\
        In-cont. (cat.) & 508.6 ($\pm$ 98.07) & 1568.2 ($\pm$ 13.26) & 508.6 ($\pm$ 98.07) & 1518.4 ($\pm$ 41.74) & 508.6 ($\pm$ 98.07) & 1644.6 ($\pm$ 16.26) \\
        In-cont. (cat.), Def. & 155.6 ($\pm$ 160.16) & 1546.4 ($\pm$ 8.80) & 155.6 ($\pm$ 160.16) & 1515.2 ($\pm$ 33.78) & 155.6 ($\pm$ 160.16) & 1651.6 ($\pm$ 17.35) \\
        In-cont. (cat.), Def., Dir. stim. & 0.0 ($\pm$ 0.00) & 1437.8 ($\pm$ 8.01) & 0.0 ($\pm$ 0.00) & 1601.2 ($\pm$ 15.47) & 0.0 ($\pm$ 0.00) & 1663.6 ($\pm$ 13.37) \\
        In-cont. (cat.), Def., Dir. stim., Pers. & 0.0 ($\pm$ 0.00) & 1430.0 ($\pm$ 12.13) & 0.0 ($\pm$ 0.00) & 1609.6 ($\pm$ 11.48) & 0.0 ($\pm$ 0.00) & 1603.0 ($\pm$ 4.15) \\
        In-cont. (cat.), Def., Pers. & 75.4 ($\pm$ 149.80) & 1522.4 ($\pm$ 11.81) & 75.4 ($\pm$ 149.80) & 1537.4 ($\pm$ 28.88) & 75.4 ($\pm$ 149.80) & 1576.0 ($\pm$ 25.10) \\
        In-cont. (cat.), Dir. stim. & 0.2 ($\pm$ 0.40) & 1428.2 ($\pm$ 12.19) & 0.2 ($\pm$ 0.40) & 1604.6 ($\pm$ 21.40) & 0.2 ($\pm$ 0.40) & 1652.6 ($\pm$ 12.64) \\
        In-cont. (cat.), Dir. stim., Pers. & 0.0 ($\pm$ 0.00) & 1410.8 ($\pm$ 13.73) & 0.0 ($\pm$ 0.00) & 1602.2 ($\pm$ 15.71) & 0.0 ($\pm$ 0.00) & 1608.6 ($\pm$ 7.42) \\
        In-cont. (cat.), Pers. & 87.0 ($\pm$ 124.03) & 1527.0 ($\pm$ 15.43) & 87.0 ($\pm$ 124.03) & 1529.4 ($\pm$ 27.17) & 87.0 ($\pm$ 124.03) & 1575.4 ($\pm$ 23.64) \\
        In-cont. (rand.) & 0.0 ($\pm$ 0.00) & 1532.4 ($\pm$ 29.51) & 0.0 ($\pm$ 0.00) & 1535.4 ($\pm$ 46.43) & 0.0 ($\pm$ 0.00) & 1570.0 ($\pm$ 46.12) \\
        In-cont. (rand.), Pers. & 0.0 ($\pm$ 0.00) & 1504.2 ($\pm$ 28.24) & 0.0 ($\pm$ 0.00) & 1560.6 ($\pm$ 33.09) & 0.0 ($\pm$ 0.00) & 1524.6 ($\pm$ 37.03) \\
        In-cont. (sim.) & 384.8 ($\pm$ 265.66) & 1596.4 ($\pm$ 25.35) & 384.8 ($\pm$ 265.66) & 1618.4 ($\pm$ 41.35) & 384.8 ($\pm$ 265.66) & 1601.4 ($\pm$ 21.85) \\
        In-cont. (sim.), Pers. & 2.8 ($\pm$ 2.64) & 1576.0 ($\pm$ 23.48) & 2.8 ($\pm$ 2.64) & 1640.2 ($\pm$ 25.36) & 2.8 ($\pm$ 2.64) & 1544.0 ($\pm$ 20.32) \\
        Pers. & 0.0 ($\pm$ 0.00) & 1547.0 ($\pm$ 0.00) & 0.0 ($\pm$ 0.00) & 1666.0 ($\pm$ 0.00) & 0.0 ($\pm$ 0.00) & 909.0 ($\pm$ 0.00) \\
        Reas., Base composition & 7.8 ($\pm$ 14.15) & 1617.0 ($\pm$ 0.00) & 7.8 ($\pm$ 14.15) & 1756.0 ($\pm$ 0.00) & 7.8 ($\pm$ 14.15) & 1313.0 ($\pm$ 0.00) \\
        Reas., Def. & 9.2 ($\pm$ 10.98) & 1614.0 ($\pm$ 0.00) & 9.2 ($\pm$ 10.98) & 1718.0 ($\pm$ 0.00) & 9.2 ($\pm$ 10.98) & 1386.0 ($\pm$ 0.00) \\
        Reas., Def., Dir. stim. & 0.0 ($\pm$ 0.00) & 1437.0 ($\pm$ 0.00) & 0.0 ($\pm$ 0.00) & 1629.0 ($\pm$ 0.00) & 0.0 ($\pm$ 0.00) & 1353.0 ($\pm$ 0.00) \\
        Reas., Def., Dir. stim., In-cont. (rand.) & 0.0 ($\pm$ 0.00) & 1506.2 ($\pm$ 8.86) & 0.0 ($\pm$ 0.00) & 1758.8 ($\pm$ 10.42) & 0.0 ($\pm$ 0.00) & 1168.8 ($\pm$ 56.00) \\
        Reas., Def., Dir. stim., In-cont. (rand.), Pers. & 0.4 ($\pm$ 0.80) & 1491.4 ($\pm$ 7.06) & 0.4 ($\pm$ 0.80) & 1769.2 ($\pm$ 4.49) & 0.4 ($\pm$ 0.80) & 794.2 ($\pm$ 154.33) \\
        Reas., Def., Dir. stim., In-cont. (sim.) & 0.0 ($\pm$ 0.00) & 1457.2 ($\pm$ 11.16) & 0.0 ($\pm$ 0.00) & 1750.6 ($\pm$ 5.68) & 0.0 ($\pm$ 0.00) & 1233.8 ($\pm$ 76.43) \\
        Reas., Def., Dir. stim., In-cont. (sim.), Pers. & 0.0 ($\pm$ 0.00) & 1446.8 ($\pm$ 19.23) & 0.0 ($\pm$ 0.00) & 1770.8 ($\pm$ 11.36) & 0.0 ($\pm$ 0.00) & 853.6 ($\pm$ 160.12) \\
        Reas., Def., Dir. stim., Pers. & 0.0 ($\pm$ 0.00) & 1428.0 ($\pm$ 0.00) & 0.0 ($\pm$ 0.00) & 1679.0 ($\pm$ 0.00) & 0.0 ($\pm$ 0.00) & 463.0 ($\pm$ 0.00) \\
        Reas., Def., In-cont. (rand.) & 0.2 ($\pm$ 0.40) & 1539.0 ($\pm$ 12.98) & 0.2 ($\pm$ 0.40) & 1761.4 ($\pm$ 13.00) & 0.2 ($\pm$ 0.40) & 1465.0 ($\pm$ 47.34) \\
        Reas., Def., In-cont. (rand.), Pers. & 0.0 ($\pm$ 0.00) & 1527.0 ($\pm$ 14.44) & 0.0 ($\pm$ 0.00) & 1770.2 ($\pm$ 10.24) & 0.0 ($\pm$ 0.00) & 1193.0 ($\pm$ 187.78) \\
        Reas., Def., In-cont. (sim.) & 0.0 ($\pm$ 0.00) & 1489.6 ($\pm$ 9.73) & 0.0 ($\pm$ 0.00) & 1745.6 ($\pm$ 7.47) & 0.0 ($\pm$ 0.00) & 1446.6 ($\pm$ 36.42) \\
        Reas., Def., In-cont. (sim.), Pers. & 0.0 ($\pm$ 0.00) & 1468.6 ($\pm$ 7.84) & 0.0 ($\pm$ 0.00) & 1767.8 ($\pm$ 8.01) & 0.0 ($\pm$ 0.00) & 1058.0 ($\pm$ 269.41) \\
        Reas., Def., Pers. & 0.0 ($\pm$ 0.00) & 1612.0 ($\pm$ 0.00) & 0.0 ($\pm$ 0.00) & 1721.0 ($\pm$ 0.00) & 0.0 ($\pm$ 0.00) & 952.0 ($\pm$ 0.00) \\
        Reas., Dir. stim. & 0.0 ($\pm$ 0.00) & 1446.0 ($\pm$ 0.00) & 0.0 ($\pm$ 0.00) & 1642.0 ($\pm$ 0.00) & 0.0 ($\pm$ 0.00) & 1098.0 ($\pm$ 0.00) \\
        Reas., Dir. stim., In-cont. (rand.) & 0.0 ($\pm$ 0.00) & 1481.4 ($\pm$ 12.34) & 0.0 ($\pm$ 0.00) & 1759.0 ($\pm$ 5.83) & 0.0 ($\pm$ 0.00) & 1178.2 ($\pm$ 55.03) \\
        Reas., Dir. stim., In-cont. (rand.), Pers. & 1.8 ($\pm$ 3.60) & 1472.4 ($\pm$ 13.17) & 1.8 ($\pm$ 3.60) & 1763.0 ($\pm$ 12.13) & 1.8 ($\pm$ 3.60) & 861.2 ($\pm$ 533.99) \\
        Reas., Dir. stim., In-cont. (sim.) & 0.0 ($\pm$ 0.00) & 1452.8 ($\pm$ 8.63) & 0.0 ($\pm$ 0.00) & 1743.4 ($\pm$ 11.57) & 0.0 ($\pm$ 0.00) & 1261.6 ($\pm$ 62.89) \\
        Reas., Dir. stim., In-cont. (sim.), Pers. & 0.0 ($\pm$ 0.00) & 1431.4 ($\pm$ 10.25) & 0.0 ($\pm$ 0.00) & 1762.4 ($\pm$ 8.52) & 0.0 ($\pm$ 0.00) & 841.6 ($\pm$ 498.75) \\
        Reas., Dir. stim., Pers. & 0.0 ($\pm$ 0.00) & 1410.0 ($\pm$ 0.00) & 0.0 ($\pm$ 0.00) & 1696.0 ($\pm$ 0.00) & 0.0 ($\pm$ 0.00) & 881.0 ($\pm$ 0.00) \\
        Reas., In-cont. (cat.) & 1.4 ($\pm$ 2.80) & 1484.6 ($\pm$ 13.95) & 1.4 ($\pm$ 2.80) & 1769.2 ($\pm$ 10.89) & 1.4 ($\pm$ 2.80) & 1693.2 ($\pm$ 14.82) \\
        Reas., In-cont. (cat.), Def. & 0.0 ($\pm$ 0.00) & 1481.6 ($\pm$ 14.24) & 0.0 ($\pm$ 0.00) & 1756.8 ($\pm$ 13.63) & 0.0 ($\pm$ 0.00) & 1700.2 ($\pm$ 23.14) \\
        Reas., In-cont. (cat.), Def., Dir. stim. & 0.0 ($\pm$ 0.00) & 1448.2 ($\pm$ 8.45) & 0.0 ($\pm$ 0.00) & 1750.6 ($\pm$ 9.89) & 0.0 ($\pm$ 0.00) & 1723.6 ($\pm$ 10.37) \\
        Reas., In-cont. (cat.), Def., Dir. stim., Pers. & 0.0 ($\pm$ 0.00) & 1407.0 ($\pm$ 7.97) & 0.0 ($\pm$ 0.00) & 1767.4 ($\pm$ 6.09) & 0.0 ($\pm$ 0.00) & 1636.2 ($\pm$ 5.04) \\
        Reas., In-cont. (cat.), Def., Pers. & 0.0 ($\pm$ 0.00) & 1447.6 ($\pm$ 8.87) & 0.0 ($\pm$ 0.00) & 1770.0 ($\pm$ 13.68) & 0.0 ($\pm$ 0.00) & 1617.0 ($\pm$ 48.92) \\
        Reas., In-cont. (cat.), Dir. stim. & 0.0 ($\pm$ 0.00) & 1458.6 ($\pm$ 14.64) & 0.0 ($\pm$ 0.00) & 1753.6 ($\pm$ 10.40) & 0.0 ($\pm$ 0.00) & 1737.0 ($\pm$ 10.49) \\
        Reas., In-cont. (cat.), Dir. stim., Pers. & 0.0 ($\pm$ 0.00) & 1418.2 ($\pm$ 7.19) & 0.0 ($\pm$ 0.00) & 1776.0 ($\pm$ 4.24) & 0.0 ($\pm$ 0.00) & 1720.2 ($\pm$ 5.64) \\
        Reas., In-cont. (cat.), Pers. & 0.0 ($\pm$ 0.00) & 1455.4 ($\pm$ 14.69) & 0.0 ($\pm$ 0.00) & 1785.8 ($\pm$ 9.28) & 0.0 ($\pm$ 0.00) & 1548.2 ($\pm$ 165.73) \\
        Reas., In-cont. (rand.) & 18.4 ($\pm$ 36.80) & 1511.6 ($\pm$ 11.36) & 18.4 ($\pm$ 36.80) & 1754.4 ($\pm$ 3.61) & 18.4 ($\pm$ 36.80) & 1371.8 ($\pm$ 21.16) \\
        Reas., In-cont. (rand.), Pers. & 0.0 ($\pm$ 0.00) & 1501.4 ($\pm$ 9.65) & 0.0 ($\pm$ 0.00) & 1778.4 ($\pm$ 11.00) & 0.0 ($\pm$ 0.00) & 1002.4 ($\pm$ 204.50) \\
        Reas., In-cont. (sim.) & 0.0 ($\pm$ 0.00) & 1476.2 ($\pm$ 10.46) & 0.0 ($\pm$ 0.00) & 1744.6 ($\pm$ 10.44) & 0.0 ($\pm$ 0.00) & 1362.2 ($\pm$ 44.01) \\
        Reas., In-cont. (sim.), Pers. & 0.0 ($\pm$ 0.00) & 1452.6 ($\pm$ 7.39) & 0.0 ($\pm$ 0.00) & 1759.6 ($\pm$ 12.99) & 0.0 ($\pm$ 0.00) & 1073.2 ($\pm$ 192.26) \\
        Reas., Pers. & 171.0 ($\pm$ 189.08) & 1645.0 ($\pm$ 0.00) & 171.0 ($\pm$ 189.08) & 1740.0 ($\pm$ 0.00) & 171.0 ($\pm$ 189.08) & 968.0 ($\pm$ 0.00) \\

        \bottomrule
    \end{tabular}

    \caption{Frequencies of how often each composition was chosen as optimal composition by our adaptive prompting approach per LLM on CobraFrames. Frequencies are averaged over five random seeds. Possible techniques for a composition are a defintion (\textit{Def.}), a directional stimulus (\textit{Dir. stim.}), In-context examples chosen randomly (\textit{In-cont. (rand.)}), based on similarity (\textit{In-cont. (sim.)}) or based on their category (\textit{In-cont. (cat.)}), a persona (\textit{Pers.}), and reasoning steps (\textit{Reas.}). The \textit{Base Composition} consists of a task description and text input.}
    \label{tab:composition-frequencies-cobra}
\end{table*}


\begin{table*}
    \small
    \centering
    \begin{tabular}{lrrr}
        \toprule
        \textbf{Composition} & \textbf{Mistral} & \textbf{Command-R} & \textbf{Llama 3} \\
        \midrule
        Base composition & 0.711 & 0.462 & 0.575 \\
        Definition & 0.716 & 0.527 & 0.637 \\
        Definition, Dir. stimulus & 0.672 & 0.498 & 0.544 \\
        Definition, Dir. stimulus, In-context (random) & 0.636 & 0.592 & 0.734 \\
        Definition, Dir. stimulus, In-context (random), Persona & 0.629 & 0.588 & 0.735 \\
        Definition, Dir. stimulus, In-context (similar) & 0.766 & 0.610 & 0.798 \\
        Definition, Dir. stimulus, In-context (similar), Persona & 0.767 & 0.582 & 0.802 \\
        Definition, Dir. stimulus, Persona & 0.676 & 0.542 & 0.483 \\
        Definition, In-context (random) & 0.671 & 0.667 & 0.736 \\
        Definition, In-context (random), Persona & 0.660 & 0.667 & 0.748 \\
        Definition, In-context (similar) & 0.775 & 0.683 & 0.800 \\
        Definition, In-context (similar), Persona & 0.776 & 0.671 & 0.817 \\
        Definition, Persona & 0.710 & 0.591 & 0.502 \\
        Dir. stimulus & 0.662 & 0.584 & 0.566 \\
        Dir. stimulus, In-context (random) & 0.634 & 0.590 & 0.726 \\
        Dir. stimulus, In-context (random), Persona & 0.632 & 0.598 & 0.733 \\
        Dir. stimulus, In-context (similar) & 0.759 & 0.600 & 0.795 \\
        Dir. stimulus, In-context (similar), Persona & 0.764 & 0.579 & 0.799 \\
        Dir. stimulus, Persona & 0.654 & 0.602 & 0.508 \\
        In-context (category) & 0.681 & 0.675 & 0.739 \\
        In-context (category), Definition & 0.681 & 0.663 & 0.760 \\
        In-context (category), Definition, Dir. stimulus & 0.652 & 0.571 & 0.715 \\
        In-context (category), Definition, Dir. stimulus, Persona & 0.645 & 0.579 & 0.728 \\
        In-context (category), Definition, Persona & 0.687 & 0.669 & 0.759 \\
        In-context (category), Dir. stimulus & 0.650 & 0.580 & 0.705 \\
        In-context (category), Dir. stimulus, Persona & 0.637 & 0.594 & 0.725 \\
        In-context (category), Persona & 0.665 & 0.685 & 0.738 \\
        In-context (random) & 0.665 & 0.674 & 0.725 \\
        In-context (random), Persona & 0.652 & 0.677 & 0.736 \\
        In-context (similar) & 0.761 & 0.701 & 0.798 \\
        In-context (similar), Persona & 0.763 & 0.706 & 0.814 \\
        Persona & 0.698 & 0.546 & 0.539 \\
        Reasoning steps & 0.697 & 0.509 & 0.610 \\
        Reasoning steps, Definition & 0.722 & 0.491 & 0.693 \\
        Reasoning steps, Definition, Dir. stimulus & 0.688 & 0.520 & 0.584 \\
        Reasoning steps, Definition, Dir. stimulus, In-context (random) & 0.705 & 0.609 & 0.661 \\
        Reasoning steps, Definition, Dir. stimulus, In-context (random), Persona & 0.705 & 0.652 & 0.495 \\
        Reasoning steps, Definition, Dir. stimulus, In-context (similar) & 0.797 & 0.596 & 0.776 \\
        Reasoning steps, Definition, Dir. stimulus, In-context (similar), Persona & 0.795 & 0.650 & 0.608 \\
        Reasoning steps, Definition, Dir. stimulus, Persona & 0.670 & 0.535 & 0.590 \\
        Reasoning steps, Definition, In-context (random) & 0.719 & 0.580 & 0.755 \\
        Reasoning steps, Definition, In-context (random), Persona & 0.716 & 0.642 & 0.561 \\
        Reasoning steps, Definition, In-context (similar) & 0.800 & 0.570 & 0.806 \\
        Reasoning steps, Definition, In-context (similar), Persona & 0.798 & 0.629 & 0.501 \\
        Reasoning steps, Definition, Persona & 0.692 & 0.521 & 0.635 \\
        Reasoning steps, Dir. stimulus & 0.646 & 0.442 & 0.602 \\
        Reasoning steps, Dir. stimulus, In-context (random) & 0.701 & 0.630 & 0.738 \\
        Reasoning steps, Dir. stimulus, In-context (random), Persona & 0.703 & 0.640 & 0.540 \\
        Reasoning steps, Dir. stimulus, In-context (similar) & 0.791 & 0.603 & 0.776 \\
        Reasoning steps, Dir. stimulus, In-context (similar), Persona & 0.791 & 0.677 & 0.596 \\
        Reasoning steps, Dir. stimulus, Persona & 0.658 & 0.465 & 0.606 \\
        Reasoning steps, In-context (category) & 0.705 & 0.440 & 0.728 \\
        Reasoning steps, In-context (category), Definition & 0.703 & 0.414 & 0.742 \\
        Reasoning steps, In-context (category), Definition, Dir. stimulus & 0.687 & 0.407 & 0.702 \\
        Reasoning steps, In-context (category), Definition, Dir. stimulus, Persona & 0.682 & 0.582 & 0.583 \\
        Reasoning steps, In-context (category), Definition, Persona & 0.697 & 0.526 & 0.641 \\
        Reasoning steps, In-context (category), Dir. stimulus & 0.696 & 0.419 & 0.701 \\
        Reasoning steps, In-context (category), Dir. stimulus, Persona & 0.691 & 0.578 & 0.582 \\
        Reasoning steps, In-context (category), Persona & 0.698 & 0.538 & 0.597 \\
        Reasoning steps, In-context (random) & 0.705 & 0.610 & 0.768 \\
        Reasoning steps, In-context (random), Persona & 0.709 & 0.665 & 0.543 \\
        Reasoning steps, In-context (similar) & 0.791 & 0.538 & 0.806 \\
        Reasoning steps, In-context (similar), Persona & 0.790 & 0.664 & 0.518 \\
        Reasoning steps, Persona & 0.693 & 0.506 & 0.659 \\
        \bottomrule
    \end{tabular}

    \caption{Macro F$_1$-score of all compositions across models on Stereoset.}
    \label{tab:compositions-performance-stereoset}
\end{table*}

\begin{table*}
    \small
    \centering
    \begin{tabular}{lrrr}
        \toprule
        \textbf{Composition} & \textbf{Mistral} & \textbf{Command-R} & \textbf{Llama 3} \\
        \midrule
        Base composition & 0.702 & 0.470 & 0.651 \\
        Definition & 0.740 & 0.554 & 0.788 \\
        Definition, Dir. stimulus & 0.742 & 0.425 & 0.741 \\
        Definition, Dir. stimulus, In-context (random) & 0.792 & 0.773 & 0.817 \\
        Definition, Dir. stimulus, In-context (random), Persona & 0.791 & 0.772 & 0.821 \\
        Definition, Dir. stimulus, In-context (similar) & 0.758 & 0.770 & 0.822 \\
        Definition, Dir. stimulus, In-context (similar), Persona & 0.756 & 0.764 & 0.820 \\
        Definition, Dir. stimulus, Persona & 0.732 & 0.447 & 0.683 \\
        Definition, In-context (random) & 0.769 & 0.787 & 0.826 \\
        Definition, In-context (random), Persona & 0.765 & 0.788 & 0.831 \\
        Definition, In-context (similar) & 0.740 & 0.770 & 0.821 \\
        Definition, In-context (similar), Persona & 0.734 & 0.766 & 0.825 \\
        Definition, Persona & 0.727 & 0.596 & 0.771 \\
        Dir. stimulus & 0.725 & 0.410 & 0.542 \\
        Dir. stimulus, In-context (random) & 0.780 & 0.768 & 0.820 \\
        Dir. stimulus, In-context (random), Persona & 0.777 & 0.760 & 0.824 \\
        Dir. stimulus, In-context (similar) & 0.749 & 0.760 & 0.822 \\
        Dir. stimulus, In-context (similar), Persona & 0.746 & 0.749 & 0.818 \\
        Dir. stimulus, Persona & 0.720 & 0.459 & 0.646 \\
        In-context (category) & 0.737 & 0.733 & 0.806 \\
        In-context (category), Definition & 0.762 & 0.772 & 0.810 \\
        In-context (category), Definition, Dir. stimulus & 0.781 & 0.765 & 0.783 \\
        In-context (category), Definition, Dir. stimulus, Persona & 0.783 & 0.760 & 0.794 \\
        In-context (category), Definition, Persona & 0.760 & 0.767 & 0.821 \\
        In-context (category), Dir. stimulus & 0.764 & 0.745 & 0.778 \\
        In-context (category), Dir. stimulus, Persona & 0.762 & 0.737 & 0.789 \\
        In-context (category), Persona & 0.737 & 0.728 & 0.817 \\
        In-context (random) & 0.747 & 0.763 & 0.825 \\
        In-context (random), Persona & 0.744 & 0.762 & 0.829 \\
        In-context (similar) & 0.716 & 0.740 & 0.816 \\
        In-context (similar), Persona & 0.712 & 0.729 & 0.822 \\
        Persona & 0.703 & 0.512 & 0.710 \\
        Reasoning steps & 0.656 & 0.436 & 0.621 \\
        Reasoning steps, Definition & 0.697 & 0.494 & 0.647 \\
        Reasoning steps, Definition, Dir. stimulus & 0.704 & 0.473 & 0.684 \\
        Reasoning steps, Definition, Dir. stimulus, In-context (random) & 0.776 & 0.743 & 0.663 \\
        Reasoning steps, Definition, Dir. stimulus, In-context (random), Persona & 0.772 & 0.724 & 0.502 \\
        Reasoning steps, Definition, Dir. stimulus, In-context (similar) & 0.771 & 0.730 & 0.604 \\
        Reasoning steps, Definition, Dir. stimulus, In-context (similar), Persona & 0.770 & 0.712 & 0.469 \\
        Reasoning steps, Definition, Dir. stimulus, Persona & 0.711 & 0.429 & 0.723 \\
        Reasoning steps, Definition, In-context (random) & 0.778 & 0.754 & 0.677 \\
        Reasoning steps, Definition, In-context (random), Persona & 0.775 & 0.747 & 0.505 \\
        Reasoning steps, Definition, In-context (similar) & 0.772 & 0.735 & 0.618 \\
        Reasoning steps, Definition, In-context (similar), Persona & 0.769 & 0.729 & 0.430 \\
        Reasoning steps, Definition, Persona & 0.708 & 0.453 & 0.699 \\
        Reasoning steps, Dir. stimulus & 0.687 & 0.484 & 0.630 \\
        Reasoning steps, Dir. stimulus, In-context (random) & 0.768 & 0.719 & 0.611 \\
        Reasoning steps, Dir. stimulus, In-context (random), Persona & 0.767 & 0.702 & 0.435 \\
        Reasoning steps, Dir. stimulus, In-context (similar) & 0.768 & 0.705 & 0.572 \\
        Reasoning steps, Dir. stimulus, In-context (similar), Persona & 0.765 & 0.688 & 0.412 \\
        Reasoning steps, Dir. stimulus, Persona & 0.712 & 0.420 & 0.699 \\
        Reasoning steps, In-context (category) & 0.761 & 0.581 & 0.647 \\
        Reasoning steps, In-context (category), Definition & 0.773 & 0.683 & 0.679 \\
        Reasoning steps, In-context (category), Definition, Dir. stimulus & 0.772 & 0.672 & 0.668 \\
        Reasoning steps, In-context (category), Definition, Dir. stimulus, Persona & 0.771 & 0.751 & 0.556 \\
        Reasoning steps, In-context (category), Definition, Persona & 0.773 & 0.765 & 0.548 \\
        Reasoning steps, In-context (category), Dir. stimulus & 0.765 & 0.647 & 0.584 \\
        Reasoning steps, In-context (category), Dir. stimulus, Persona & 0.764 & 0.731 & 0.534 \\
        Reasoning steps, In-context (category), Persona & 0.760 & 0.739 & 0.570 \\
        Reasoning steps, In-context (random) & 0.769 & 0.729 & 0.694 \\
        Reasoning steps, In-context (random), Persona & 0.769 & 0.725 & 0.488 \\
        Reasoning steps, In-context (similar) & 0.767 & 0.720 & 0.669 \\
        Reasoning steps, In-context (similar), Persona & 0.765 & 0.713 & 0.432 \\
        Reasoning steps, Persona & 0.678 & 0.408 & 0.668 \\
        \bottomrule
    \end{tabular}

    \caption{Macro F$_1$-score of all compositions across models on SBIC.}
    \label{tab:compositions-performance-sbic}
\end{table*}

\begin{table*}
    \small
    \centering
    \begin{tabular}{lrrr}
        \toprule
        \textbf{Composition} & \textbf{Mistral} & \textbf{Command-R} & \textbf{Llama 3} \\
        \midrule
        Base composition & 0.449 & 0.535 & 0.461 \\
        Definition & 0.485 & 0.575 & 0.497 \\
        Definition, Dir. stimulus & 0.466 & 0.544 & 0.431 \\
        Definition, Dir. stimulus, In-context (random) & 0.503 & 0.554 & 0.521 \\
        Definition, Dir. stimulus, In-context (random), Persona & 0.498 & 0.561 & 0.484 \\
        Definition, Dir. stimulus, In-context (similar) & 0.604 & 0.613 & 0.574 \\
        Definition, Dir. stimulus, In-context (similar), Persona & 0.602 & 0.611 & 0.554 \\
        Definition, Dir. stimulus, Persona & 0.462 & 0.518 & 0.498 \\
        Definition, In-context (random) & 0.533 & 0.534 & 0.576 \\
        Definition, In-context (random), Persona & 0.523 & 0.535 & 0.560 \\
        Definition, In-context (similar) & 0.604 & 0.580 & 0.594 \\
        Definition, In-context (similar), Persona & 0.598 & 0.587 & 0.582 \\
        Definition, Persona & 0.478 & 0.571 & 0.452 \\
        Dir. stimulus & 0.422 & 0.438 & 0.340 \\
        Dir. stimulus, In-context (random) & 0.497 & 0.546 & 0.519 \\
        Dir. stimulus, In-context (random), Persona & 0.489 & 0.557 & 0.493 \\
        Dir. stimulus, In-context (similar) & 0.602 & 0.615 & 0.591 \\
        Dir. stimulus, In-context (similar), Persona & 0.594 & 0.610 & 0.569 \\
        Dir. stimulus, Persona & 0.451 & 0.461 & 0.319 \\
        In-context (category) & 0.547 & 0.499 & 0.599 \\
        In-context (category), Definition & 0.537 & 0.489 & 0.598 \\
        In-context (category), Definition, Dir. stimulus & 0.493 & 0.540 & 0.599 \\
        In-context (category), Definition, Dir. stimulus, Persona & 0.487 & 0.542 & 0.573 \\
        In-context (category), Definition, Persona & 0.526 & 0.496 & 0.573 \\
        In-context (category), Dir. stimulus & 0.491 & 0.545 & 0.597 \\
        In-context (category), Dir. stimulus, Persona & 0.477 & 0.547 & 0.576 \\
        In-context (category), Persona & 0.527 & 0.505 & 0.578 \\
        In-context (random) & 0.537 & 0.530 & 0.566 \\
        In-context (random), Persona & 0.523 & 0.534 & 0.554 \\
        In-context (similar) & 0.604 & 0.588 & 0.605 \\
        In-context (similar), Persona & 0.597 & 0.590 & 0.593 \\
        Persona & 0.450 & 0.528 & 0.362 \\
        Reasoning steps & 0.535 & 0.589 & 0.417 \\
        Reasoning steps, Definition & 0.548 & 0.584 & 0.452 \\
        Reasoning steps, Definition, Dir. stimulus & 0.474 & 0.545 & 0.435 \\
        Reasoning steps, Definition, Dir. stimulus, In-context (random) & 0.523 & 0.646 & 0.415 \\
        Reasoning steps, Definition, Dir. stimulus, In-context (random), Persona & 0.515 & 0.645 & 0.318 \\
        Reasoning steps, Definition, Dir. stimulus, In-context (similar) & 0.540 & 0.646 & 0.498 \\
        Reasoning steps, Definition, Dir. stimulus, In-context (similar), Persona & 0.532 & 0.641 & 0.376 \\
        Reasoning steps, Definition, Dir. stimulus, Persona & 0.446 & 0.546 & 0.117 \\
        Reasoning steps, Definition, In-context (random) & 0.544 & 0.654 & 0.476 \\
        Reasoning steps, Definition, In-context (random), Persona & 0.536 & 0.650 & 0.386 \\
        Reasoning steps, Definition, In-context (similar) & 0.549 & 0.639 & 0.524 \\
        Reasoning steps, Definition, In-context (similar), Persona & 0.542 & 0.651 & 0.405 \\
        Reasoning steps, Definition, Persona & 0.533 & 0.560 & 0.284 \\
        Reasoning steps, Dir. stimulus & 0.452 & 0.559 & 0.350 \\
        Reasoning steps, Dir. stimulus, In-context (random) & 0.513 & 0.646 & 0.423 \\
        Reasoning steps, Dir. stimulus, In-context (random), Persona & 0.506 & 0.642 & 0.328 \\
        Reasoning steps, Dir. stimulus, In-context (similar) & 0.542 & 0.668 & 0.496 \\
        Reasoning steps, Dir. stimulus, In-context (similar), Persona & 0.532 & 0.660 & 0.369 \\
        Reasoning steps, Dir. stimulus, Persona & 0.432 & 0.571 & 0.251 \\
        Reasoning steps, In-context (category) & 0.521 & 0.643 & 0.533 \\
        Reasoning steps, In-context (category), Definition & 0.522 & 0.633 & 0.543 \\
        Reasoning steps, In-context (category), Definition, Dir. stimulus & 0.514 & 0.627 & 0.535 \\
        Reasoning steps, In-context (category), Definition, Dir. stimulus, Persona & 0.499 & 0.633 & 0.536 \\
        Reasoning steps, In-context (category), Definition, Persona & 0.501 & 0.631 & 0.524 \\
        Reasoning steps, In-context (category), Dir. stimulus & 0.512 & 0.639 & 0.535 \\
        Reasoning steps, In-context (category), Dir. stimulus, Persona & 0.497 & 0.651 & 0.548 \\
        Reasoning steps, In-context (category), Persona & 0.504 & 0.648 & 0.527 \\
        Reasoning steps, In-context (random) & 0.532 & 0.642 & 0.479 \\
        Reasoning steps, In-context (random), Persona & 0.530 & 0.644 & 0.369 \\
        Reasoning steps, In-context (similar) & 0.547 & 0.663 & 0.520 \\
        Reasoning steps, In-context (similar), Persona & 0.544 & 0.663 & 0.349 \\
        Reasoning steps, Persona & 0.531 & 0.610 & 0.302 \\
        \bottomrule
    \end{tabular}

    \caption{Macro F$_1$-score of all compositions across models on CobraFrames.}
    \label{tab:compositions-performance-cobra}
\end{table*}

