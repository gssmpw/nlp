
\subsection{\label{sec_cg_cube}Cayley graphs and Rubik's cubes}

Moves of Rubik's cube can be described by permutations (e.g., Chapter 5~\cite{mulholland2016permutation}, or  Kaggle notebook "Visualize allowed moves"\footnote{\url{https://www.kaggle.com/code/marksix/visualize-allowed-moves}}). Taking all the positions as nodes and connecting them by edges, which differ by single moves, one obtains a Cayley-type (Schreier) graph for Rubik's cube. Solving the puzzle is equivalent to finding a path on the graph between nodes representing the Rubik's cube's scramble initial and solved state. 
