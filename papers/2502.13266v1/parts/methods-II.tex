
\subsection{\label{sec_rw_ts}Random walks and train set generation}

The training set is generated by scrambling (i.e., applying random moves) the selected solved state and creating a set of pairs $(v,k)$, where $k$ is a number of scrambles, and  $v$ is a vector describing the node obtained after $k$ steps. In other words, we consider random walks on the graph. The main parameters are $K_{\text{max}}$ and $K$, where $K_{\text{max}}$ is a maximal number of scrambles (length of random walk trajectory), while $K\cdot K_{\text{max}}$ is a number of nodes to generate.

In the current research, we used so-called non-backtracking random walks \cite{alon2007non}, that forbid scrambling to the state of the previous step. A PyTorch-optimized implementation of train set generation can be found in \textit{trainer.py} in the code attached to this paper.

Current research does not investigate the influence of $K_{\text{max}}$ on the solver's performance. We used $K_{\text{max}}=26$ for solvers targeted on 3$\times$3$\times$3 cubes, $K_{\text{max}}=45$ -- for 4$\times$4$\times$4 cubes, and $K_{\text{max}}=65$ for 5$\times$5$\times$5 cubes.
