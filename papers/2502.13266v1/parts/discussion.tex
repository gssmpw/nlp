The paper proposes a machine learning-based approach to the pathfinding problem on large graphs. Experimental studies demonstrate that it is more efficient than state-of-the-art solutions in terms of average solution length, optimality, and computational performance.

The key parts of the approach are multi-agency, neural networks predicting diffusion distance and beam search. Deeper neural networks better approximate the graph of the large Rubik's cubes, though, for the 3$\times$3$\times$3 case, even a single-layer network provides excellent results. At the same time, the effect of enlarging the training set is limited: the trainset above 8196M examples for the tested models has no practical reason, which allowed us to avoid additional time spent during the training.
Conversely, raising the beam width effectively lowers the solution length and increases optimality.

The complete set of the proposed solutions allowed the creation of the multi-agent pathfinder, which managed to beat all the ML-based competitors: an agent equipped with single-layer MLP solved all the DeepCubeA dataset with 90.4\% of optimality significantly enhancing results of the most advanced state of the art solutions: DeepCubeA and EfficientCube. 26 agents equipped with 10-layer ResMLP models managed to solve all scrambles from DeepCubeA dataset with 97.6\% optimality, which is the best result ever achieved by any ML solution. Single-agent solutions implemented using our approach and 10-layer ResMLP beat all the best results corresponding to 3$\times$3$\times$3 and 4$\times$4$\times$4 Rubik's cubes submitted on the 2023 Santa Challenge. At the same time, six agents managed to solve all the 4$\times$4$\times$4 cube's scrambles from the 2023 Santa Challenge dataset with an average solution length below 48 (a 4$\times$4$\times$4 Rubik's cube diameter predicted in \cite{hirata2024probabilistic}). Finally, a composition of 69 agents beat all the best solutions for the 5$\times$5$\times$5 Rubik's cube submitted to the 2023 Santa Challenge, shortening the average solution rate among all the datasets on more than 4.4 units in QTM metrics.

The method's scope is quite wide and can be applied to various planning tasks in their graph pathfinding reformulation. In future works, we plan to explore its applications to mathematical, bioinformatic, and programming tasks. 