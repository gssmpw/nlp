
\subsection{\label{sec_search}Beam-search}
Beam search is a simple but effective search procedure used for various optimization tasks~\cite{hale2018finding, huang2019linearfold, scheidl2018word} as well as to improve outputs of the modern transformer-based language models~\cite{freitag2017beam, pryzant2023automatic, musolesi2024creative}. It has been used in EfficientCube~\cite{takano2023selfsupervision} and by many participants of the Kaggle Challenge~\cite{santa-2023}. We implemented a modified version of traditional beam search, which uses hash functions to remove duplicates, reducing the computation complexity of the pathfinder. Finally, in all the experiments, the scramble was considered unsolved if the path to the solved state was not found in 200 beam search steps. Additionally, the algorithm stops if the beam vector contains only already visited graph nodes. A PyTorch-optimized implementation of the beam search can be found in \textit{searcher.py} in the code attached to this paper.