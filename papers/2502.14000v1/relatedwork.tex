\section{Related Work}
\label{sec:relwork}
%%%%%%%%%%%%%%%%%%%%%%%%%%%%%%%%%%%%%%%
The implementation of agent interactions has been explored through various computational approaches. Basic interaction protocols have been formalized for distributed computation \cite{AspnesR09}, while more complex patterns have been developed for knowledge combination \cite{AndreoliBP94,BorghoffS96} and distributed problem-solving \cite{BorghoffPAF98}. Recent work has focused on higher-level interactions, particularly in intention recognition \cite{GreefDGL07,IngaRRNRKDLTNMH23}, which is becoming increasingly important as artificial agents grow more sophisticated in their ability to understand and respond to human goals.

\cite{GuamanDP21} presents an innovative unsupervised clustering theory using self-or\-ga\-niz\-ing maps to classify \textit{MVC} patterns based on software quality metrics in a continuous interaction process. Their goal is to identify quality features that determine the similarity of MVC applications without architectural bias.
\cite{LawlessS21} use interdependence theory to determine whether convergence has a positive or negative effect in a competition between human and artificial agents, as in system dynamics models. 

Mobile applications, as well as satellite and swarm robotics, have grown significantly in recent years, with architectural patterns playing a critical role in their success. To ensure the correct implementation of an MVC pattern, \cite{DobreanD22} propose an automated technique that analyzes and detects architectural issues using data from {\em Software Development Kits} (SDKs) specifically targeted at mobile code bases.
To enhance coordination and security in swarm robotic systems, \cite{carovilla2023integrating} introduce a semi-centralized framework that integrates blockchain technology, featuring a centralized control unit that coordinates the swarm, while blockchain technology ensures secure and decentralized data storage and communication.
%
\cite{KumarC23} take the opposite approach, extending pattern descriptions with a {\em System of Systems} (SoS) model to apply interactive dynamics to an MVC pattern.

Using the {\em Internet of Things} (IoT) paradigm as a complex distributed system that shares a 3-tier architecture consisting of embedded nodes, gateways that connect an embedded network to the wider internet, and data services in servers or the cloud, \cite{RiliskisHL15} propose a novel approach for programming applications across 3-tiers using a distributed extension of the MVC architecture. 
Also in the IoT context, with a focus on the gaming/sports arena, \cite{Chen25} sees 
%
%great 
%
potential for entertainment robots to recognize human posture through artificial intelligence. 
By detecting and monitoring users' movements in real time, these robots provide a personalized and interactive entertainment experience.
\cite{CaciD20} also discuss the interplay between games with personality and artificial intelligence. They use the term ``virtual human'' to describe a computer program that simulates a human in some aspects.


With the ubiquity of AI applications, HCI research is increasingly integrating these approaches, demonstrating that AI and HCI are mutually beneficial when they collaborate \cite{PandaR24}.
%
However, criticism and bad experiences have also been reported. 
For example, \cite{Choudhuri2024} state that there were no statistical differences in participants' productivity or self-efficacy when using ChatGPT compared to traditional resources.
Instead, they found significantly higher frustration levels, identifying five distinct errors resulting from violations of human-AI interaction guidelines, leading to various (negative) consequences for the participants.

LLMs are increasingly being used in studies to investigate the interaction between humans and artificial agents. 
\cite{BorghoffMM24} explore human-artificial interaction with generative AIs in a software engineering project course.
In the same domain, \cite{Nascimento2023} evaluates ChatGPT-generated code against developer-generated code to determine which tasks are better suited for engineers and which are better handled by AI. This could lead to more efficient interaction (e.g., AI as a tutor for SE developers) and provide new insights into innovative AI strategies that include the involvement of {\em humans-in-the-loop} to support the tasks of software engineering.
%
\cite{CapitanelliM24} demonstrate that LLMs can play an important role in planning actions in human-robot interactions.

\cite{GiudiciLTB24} present and evaluate a method for analyzing user reactions to AI using a live-streaming platform where human streamers conduct interviews that are transmitted to a specially developed GPT voice interface using a crowd-based approach.
\cite{KimI23} study the attribution of human characteristics to artificial intelligence. They developed a tool to measure how users form anthropomorphic reactions to interactions with AI chatbots in a banking service setting. 
For a survey of {\em Artificial Emotional Intelligence} (AEI) for cooperative social human-machine interactions see \cite{AhmadiH23}.

Message Passing Neural Networks (MPNNs) \cite{papillon2023} are a special class of neural networks that use the aforementioned message-passing paradigm to capture interactions between entities (often represented as nodes in a graph).
In MPNNs, nodes in a graph communicate with their neighbors through a series of message-passing steps. This interaction allows the network to learn representations by aggregating information from neighboring nodes; see also the \textsc{Rabbit} use case in Section~\ref{sec:usecase2}.

A formal model of the global activity of a system of agents is provided by \textit{concurrent game structures}, describing a situation where agents act according to individual \textit{strategies} by which to select individual actions based on knowledge of the global state of the system.
Reasoning on the possible evolutions of such systems leads to the definition of some variations of \textit{Alternating-Time Temporal Logic} (see e.g., \cite{AHK02,ALNR17,MMPV16}). In contrast, a logic that incorporates spatial constraints on a par with temporal ones has been recently proposed by \cite{BLP24}.
In general, these models abstract the actual mechanisms through which information about states can be exchanged among agents and assume that all agents act rationally regarding their strategies. 
The level of non-determinism inherent to considering human agents would, therefore, require some adaptation of these logics in the direction of identifying classes of agents that can follow several strategies at once.

Although beyond the scope of this paper, there exists a humanistic perspective \cite{KnellR24}, as well as an ethical dimension, to 
human-artificial interaction. 
For the latter, the reader is referred to \cite{KumarCx23,Karlan23} and \cite{Nieto24}.
In some ways, ``virtual superhuman''  AI technologies could undermine the value of human achievements \cite{SchaapSS24} or even may pose an existential threat \cite{KumarC23a}.

%%%%%%%%%%%%%%%%%%%%%%%%%%%%%%%%%%%%%%%