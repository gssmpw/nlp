\section{Experiment Setup} \label{app:setup}
\subsection{Model} \label{app:model}
\begin{table*}[!htbp]
    \small
    \centering
    \alternaterowcolors
    \begin{tabular}{lcccl}
        \toprule
        \bfseries Model Name & \bfseries  Scale & \bfseries Instruct? & \bfseries Open-Source? & \bfseries Checkpoint \\
        \midrule
        \llamaThree & 8B & \faCheck & \faCheck & \href{https://huggingface.co/meta-llama/Meta-Llama-3-8B-Instruct}{meta-llama/Meta-Llama-3-8B-Instruct} \\
        \llamaThreeOne & 8B & \faCheck & \faCheck & \href{https://huggingface.co/meta-llama/Meta-Llama-3-8B-Instruct}{meta-llama/Meta-Llama-3.1-8B-Instruct} \\
        \qwenTwo & 7B & \faCheck & \faCheck & \href{https://huggingface.co/Qwen/Qwen2-7B-Instruct}{Qwen/Qwen2-7B-Instruct} \\
        \qwenTwoFive & 7B & \faCheck & \faCheck & \href{https://huggingface.co/Qwen/Qwen2.5-7B-Instruct}{Qwen/Qwen2.5-7B-Instruct} \\
        \mistral & 12B & \faCheck & \faCheck & \href{https://huggingface.co/mistralai/Mistral-Nemo-Instruct-2407}{mistralai/Mistral-Nemo-Instruct-2407} \\
        \aya & 8B & \faCheck & \faCheck & \href{https://huggingface.co/CohereForAI/aya-expanse-8b}{CohereForAI/aya-expanse-8b} \\
        \gptThreeFive & NA & \faCheck & \faTimes & \href{https://platform.openai.com/docs/models/gpt-4o#gpt-3-5-turbo}{gpt-3.5-turbo-0125} \\
        % \gptFour & NA & \faCheck & \faTimes & \href{https://platform.openai.com/docs/models/gpt-4o#gpt-4-turbo-and-gpt-4}{gpt-4-turbo-2024-04-09} \\
        \gptFourO & NA & \faCheck & \faTimes & \href{https://platform.openai.com/docs/models/gpt-4o#gpt-4o-mini}{gpt-4o-mini-2024-07-18} \\
        \bottomrule
    \end{tabular}
    \caption{Model details. Checkpoints are either from Hugging Face or OpenAI API.}
    \label{tab:model_card}
\end{table*}
All model checkpoints we use and their properties are listed in \cref{tab:model_card}.


\subsection{Dataset} \label{app:dataset}
\begin{table*}[!htbp]
    %\resizebox{\columnwidth}{!}{
    \small
    \centering
    \alternaterowcolors
    \setlength{\tabcolsep}{6pt}
    \begin{tabular}{llllr}
        \toprule
        \bfseries Dataset & \bfseries HRL & \bfseries LRL & \bfseries Source \\
        \midrule
        \mgsm & de, en, es, fr, ja, ru, zh & bn, sw, te, th & \href{https://huggingface.co/datasets/juletxara/mgsm}{juletxara/mgsm} \\
        \xcopa & en, id, it, tr, zh & et, ht, 
        qu, sw, ta, th, vi & \href{https://autonlp.ai/datasets/choice-of-plausible-alternatives-(copa)}{English COPA} \& \href{https://huggingface.co/datasets/cambridgeltl/xcopa}{cambridgeltl/xcopa}   \\
        \xlwic & da, de, en, fr, it, ja, ko, nl, zh& bg, et, fa, hr& \href{https://pilehvar.github.io/xlwic/}{pilehvar.github.io/xlwic/} \\
        \bottomrule
    \end{tabular}
    %}
    \caption{Additional information for the three datasets we evaluate. The correspondence between language codes and names can be found in \cref{tab:lang25}. ``Source'' indicates where to download the dataset. ``En Avg Word Count'' represents the average word count and standard deviation of the demonstration questions of the English split.}
    \label{tab:dataset_additional_info}
\end{table*}
After preprocessing, datapoints for each language split are stored in a single JSON file. \cref{tab:dataset_additional_info} summarizes the supported languages for each dataset and the sources from which they are obtained.


\paragraph{\mgsm} The original dataset consists of parallel datapoints across all language splits and training/test splits of the same size. Example datapoint and the Chat Template for few-shot demonstrations can be found in \cref{fig:template:mgsm}.



\paragraph{\xcopa} The $100$ datapoints in the training split of XCOPA are parallel to the last 100 datapoints in the English COPA development split. Therefore, we exclude the first $400$ datapoints from the development split of English COPA. The test splits of both datasets are parallel and contain the same number of datapoints. We then merge both into our \xcopa dataset. In \xcopa, there are two types of questions: ``cause'' and ``effect'', each corresponding to a distinct template, as shown in \cref{fig:template:xcopa:cause,fig:template:xcopa:effect}.


\begin{figure}[!ht]
    \centering
    \includegraphics[width=\linewidth]{figures/template_mgsm.pdf}
    \caption{An example of an English datapoint from \mgsm training set. When calling Chat Template API, \user role message is the ``question'' value, while \assistant role message is the ``answer'' value. Note that in the test set, the answer is null without exemplar CoT response. The correct numerical answer is stored in ``answer\_number''.}
    \label{fig:template:mgsm}
\end{figure}

\begin{figure}[!t]
    \centering
    \begin{subfigure}[b]{\columnwidth}
        \centering
        \includegraphics[width=\linewidth]{figures/template_xcopa_cause.pdf}
        \captionsetup{skip=1pt}
        \caption{An example of ``cause'' datapoint.}
        \label{fig:template:xcopa:cause}
    \end{subfigure}

    \begin{subfigure}[b]{\columnwidth}
        \centering
        \includegraphics[width=\linewidth]{figures/template_xcopa_effect.pdf}
        \captionsetup{skip=1pt}
        \caption{An example of ``effect'' datapoint.}
        \label{fig:template:xcopa:effect}
    \end{subfigure}

    % \captionsetup{skip=1pt}
    \caption{Examples of English datapoints from \xcopa training set. First, based on whether ``question'' is  ``cause'' or ``effect'', we fill the ``premise'', ``choice1'', and ``choice2'' values into one of the two predefined templates. The template's language is changeable as per the language split of the datapoint. Then we call the Chat Template API, \user role message is the filled template, while \assistant role message is the ``label'' value.}
    \label{fig:template:xcopa}
\end{figure}




\paragraph{\xlwic} This benchmark is designed to determine whether a specific word in a given language has the same meaning in two different sentences. As a result, the dataset is inherently non-parallel. Among all language splits, Estonian (et) contains the fewest datapoints, with $98$ in the training split and $390$ in the test split. For all other languages, we randomly subsample to match the size of the Estonian split to satisfy the demonstration sampling requirements outlined in \cref{sec:icl:multilingual_mode}. To leverage the attention mechanism of transformers \cite{transformer}, we add asterisks around the target word in both sentences to indicate that the LLM needs to disambiguate the meaning of that specific word. An example datapoint is shown in \cref{fig:template:xlwic}.

\begin{figure}[!t]
    \centering
    \includegraphics[width=\linewidth]{figures/template_xlwic.pdf}
    \caption{An example of English datapoint from \xlwic training set. We first fill the ``example\_1'', ``example\_2'', and ``target\_word'' values into the predefined templates. Asterisks * are surrounded around ``target\_word'' to draw the LLM's attention. The template's language is changeable as per the language split of the datapoint. Then we call the Chat Template API, \user role message is the filled template, while \assistant role message is ``Yes'' or ``No'' (``label'' is $1$ or $0$).}
    \label{fig:template:xlwic}
\end{figure}



\paragraph{\combined} \label{app:dataset:combined} Identifying specific neurons depends on the nature of the input corpus. Since language is inherently conjugate with the task, and our focus is on language-specific neurons rather than task-specific neurons, it is necessary to input all three datasets into the LLM to eliminate the confounding factor of the task domain. To balance the three datasets, we subsample their test splits to only $250$ datapoints each. To balance the number of language splits across datasets, we excluded two HRL splits, Korean (ko) and Dutch (nl), from the \xlwic dataset. This ensures that all three (sub-)datasets have $11$ languages for combination. Additionally, for \mgsm, we limit the answers to only include the final numeric result without the CoT reasoning. This approach ensures that the total number of datapoints and the overall token count are roughly the same across the original three datasets.


\paragraph{\flores} \label{app:dataset:flores} This machine translation benchmark comprises $3,001$ sentences extracted from English Wikipedia, spanning diverse topics and domains. These sentences were translated into $101$ languages by professional translators via a carefully controlled process. The Wikipedia domain is largely unrelated to the domains of the three datasets we evaluate (math, commonsense reasoning, and word disambiguation). Therefore, we choose \flores as our source pool of irrelevant sentences. Note that in this benchmark, each datapoint carries the same semantic meaning across all $101$ language splits. We do not want irrelevant sentences to affect the LLM's understanding of the original task excessively, thus we select sentences with word counts between $10$ and $15$ in the English split, introducing limited noise. \cis are drawn form the filtered \flores dataset. An example is provided in \cref{fig:flores}.

\begin{figure}[!t]
    \centering
    \includegraphics[width=\linewidth]{figures/flores_all_high_lang.pdf}
    \captionsetup{skip=1pt}
    \caption{A datapoint example from \flores of semantic-equivalent context-irrelevant sentences in all high resource languages we study in this work.}
    \label{fig:flores}
\end{figure}




\subsection{Language} \label{app:lang}
\subsubsection{High-Resource Language List} \label{app:lang:hrl_list}
\begin{table}[!htbp]
    \small
    \centering
    \alternaterowcolors
    \begin{tabular}{llr}
        \toprule
            \bfseries Code & \bfseries Language & \bfseries Percentage \\
        \midrule
            en & English & $89.70 \%$ \\
            de & German & $0.17 \%$ \\ 
            fr & French & $0.16 \%$ \\
            sv & Swedish & $0.15 \%$ \\
            zh & Chinese & $0.13 \%$ \\
            es & Spanish & $0.13 \%$ \\
            ru & Russian & $0.13 \%$ \\
            nl & Dutch & $0.12 \%$ \\
            it & Italian & $0.11 \%$ \\
            ja & Japanese & $0.10 \%$ \\
            pl & Polish & $0.09 \%$ \\
            pt & Portuguese & $0.09 \%$ \\
            vi & Vietnamese & $0.08 \%$ \\
            uk & Ukrainian & $0.07 \%$ \\
            ko & Korean & $0.06 \%$ \\
            ca & Catalan & $0.04 \%$ \\
            sr & Serbian & $0.04 \%$ \\
            id & Indonesian & $0.03 \%$ \\
            cs & Czech & $0.03 \%$ \\
            fi & Finnish & $0.03 \%$ \\
        \bottomrule
    \end{tabular}
    \caption{Top 20 language distribution of the training data for \llamaTwo, excluding code and unknown data. Adopted from Table 10 in \citet{llama2}.}
    \vspace{-.2cm}
    \label{tab:llama2_top_20_lang}
\end{table}

\begin{table}[!htbp]
    \small
    \centering
    \alternaterowcolors
    \begin{tabular}{llrr}
        \toprule
        \bfseries Code & \bfseries Language & \bfseries Tokens (B) & \bfseries Percentage \\
        \midrule
            en & English & 578.064 & $77.984 \%$ \\
            de & German & 25.954 & $3.501 \%$ \\
            fr & French & 24.094 & $3.250 \%$ \\
            es & Spanish & 15.654 & $2.112 \%$ \\
            pl & Polish & 10.764 & $1.452 \%$ \\
            it & Italian & 9.699 & $1.308 \%$ \\
            nl & Dutch & 7.690 & $1.037 \%$ \\
            sv & Swedish & 5.218 & $0.704 \%$ \\
            tr & Turkish & 4.855 & $0.655 \%$ \\
            pt & Portuguese & 4.701 & $0.634 \%$ \\
            ru & Russian & 3.932 & $0.530 \%$ \\
            fi & Finnish & 3.101 & $0.418 \%$ \\
            cs & Czech & 2.991 & $0.404 \%$ \\
            zh & Chinese & 2.977 & $0.402 \%$ \\
            ja & Japanese & 2.832 & $0.382 \%$ \\
            no & Norwegian & 2.695 & $0.364 \%$ \\
            ko & Korean & 1.444 & $0.195 \%$ \\
            da & Danish & 1.387 & $0.187 \%$ \\
            id & Indonesian & 1.175 & $0.159 \%$ \\
            ar & Arabic & 1.091 & $0.147\%$ \\
        \bottomrule
    \end{tabular}
    \caption{Top 20 language distribution of the training data for \palm, excluding code and unknown data. Adopted from Table 28 in \citet{palm}.}
    % \vspace{-.2cm}
    \label{tab:palm_top_20_lang}
\end{table}



To the best of our knowledge, we find two multilingual LLMs—\llamaTwo \cite{llama2} and \palm \cite{palm}—that publicly report the language distribution used during pretraining. The top 20 languages and their percentages for each model are listed in \cref{tab:llama2_top_20_lang} and \cref{tab:palm_top_20_lang}, respectively. We take the union of these languages to form what we consider a \textit{high-resource language list} (in terms of the richness in the LLM pretraining dataset), including the following $24$ languages:
\begin{equation}
    \begin{aligned}
        \mathcal{L}_{H} = & \{ \text{ar, ca, cs, da, de, en, es, fi, fr, id, it, ja, } \\
                          & \text{ko, nl, no, pl, pt, ru, sr, sv, tr, uk, vi, zh} \}.
    \end{aligned}
    \label{eq:high_lang_list}
\end{equation}

The high overlap between the top languages of the two LLMs further supports the rationale for applying this list to other LLMs.


\subsubsection{Languages We Evaluate}
\cref{tab:lang25} presents the union of languages supported by the three datasets we evaluated (\cref{sec:setup}, datasets). Based on whether a language appears in the high-resource language list, we categorized the $24$ languages into  HRL and LRL groups, with $13$ classified as HRL and $11$ as LRL. Among them, only English and Chinese are present in all three datasets.

It is worth highlighting that the $11$ LRLs span $7$ distinct writing systems and $6$ language families. This diversity suggests that when tokenizing inputs in these LRLs, they are unlikely to share common tokens with HRLs ($9$ out of $13$ use the Latin script). Consequently, this limits the MLLMs' ability to leverage shared subwords or similar syntax structures for cross-lingual transfer across LRLs.
\begin{figure*}[t]
\centering
\begin{center} 
  \includegraphics[width=.63\textwidth]{figures/fidelity_legend.pdf}
\end{center}
\begin{subfigure}[b]{.3\textwidth}
  \centering
  \includegraphics[width=\linewidth]{figures/sentiment_fidelity3_small.pdf}
  \caption{\emph{Sentiment} $n\in [32,63]$}
  \label{fig:fidelity_sentiment}
\end{subfigure}%
\begin{subfigure}[b]{.3\textwidth}
  \centering
  \includegraphics[width=\linewidth]{figures/drop_fidelity_small.pdf}
  \caption{\emph{DROP} $n\in [32,63]$}
  \label{fig:fidelity_drop}
\end{subfigure}%
\begin{subfigure}[b]{.3\textwidth}
  \centering
  \includegraphics[width=\linewidth]{figures/hotpot_fidelity_small.pdf}
  \caption{\emph{HotpotQA} $n\in [32,63]$}
  \label{fig:fidelity_hotpot}
\end{subfigure}
\begin{subfigure}[b]{.3\textwidth}
  \centering
  \includegraphics[width=\linewidth]{figures/sentiment_fidelity3_large.pdf}
  \caption{\emph{Sentiment} $n\in [64,127]$}
  \label{fig:fidelity_sentiment2}
\end{subfigure}%
\begin{subfigure}[b]{.3\textwidth}
  \centering
  \includegraphics[width=\linewidth]{figures/drop_fidelity_large.pdf}
  \caption{\emph{DROP} $n\in [64,127]$}
  \label{fig:fidelity_drop2}
\end{subfigure}%
\begin{subfigure}[b]{.3\textwidth}
  \centering
  \includegraphics[width=\linewidth]{figures/hotpot_fidelity_large.pdf}
  \caption{\emph{HotpotQA} $n\in [64,127]$}
  \label{fig:fidelity_hotpot2}
\end{subfigure}
\vspace{-5pt}
\caption{On the removal task, \SpecExp{} performs competitively with 2\textsuperscript{nd} order methods on the \emph{Sentiment} dataset, and out-performs all approaches on \emph{DROP} and  \emph{HotpotQA} dataset for $n \in [32,63]$. When $n$ is too large to compute other interaction indices, we outperform marginal methods.}
\label{fig:fidelity}
\vspace{-12pt}
\end{figure*}

\vspace{-14pt}
\section{Experiments}
\label{sec:language}

\paragraph{Datasets} 
We use three popular datasets that require the LLM to understand interactions between features. 
\begin{enumerate}[ topsep=0pt, itemsep=0pt, leftmargin=*]
\item \emph{Sentiment} is primarily composed of the \emph{Large Movie Review Dataset} \cite{maas-EtAl:2011:ACL-HLT2011}, which contains both positive and negative IMDb movie reviews. The dataset is augmented with examples from the \emph{SST} dataset \cite{ socher2013recursive} to ensure coverage for small $n$. We treat the words of the reviews as the input features.
\item{\emph{HotpotQA} \cite{yang2018hotpotqa} is a question-answering dataset requiring multi-hop reasoning over multiple Wikipedia articles to answer complex questions. We use the sentences of the articles as the input features.}
\item{\emph{Discrete Reasoning Over Paragraphs} (DROP)} \cite{dua2019drop} is a comprehension benchmark requiring discrete reasoning operations like addition, counting, and sorting over paragraph-level content to answer questions. We use the words of the paragraphs as the input features. 
\end{enumerate}
%
%\emph{DROP} and \emph{HotpotQA} require , while \emph{Sentiment} is encoder-only. 
%
\vspace{-7pt}
\paragraph{Models} For \textit{DROP} and \textit{HotpotQA}, (generative question-answering tasks) we use \texttt{Llama-3.2-3B-Instruct} \cite{grattafiori2024llama3herdmodels} with $8$-bit quantization. For \emph{Sentiment} (classification), we use the encoder-only fine-tuned \texttt{DistilBERT} model \cite{Sanh2019DistilBERTAD,sentimentBert}.
%

\vspace{-7pt}
\paragraph{Baselines} We compare against popular marginal metrics LIME, SHAP, and the Banzhaf value. 
%
For interaction indices, we consider Faith-Shapley, Faith-Banzhaf, and the Shapley-Taylor Index. We compute all benchmarks where computationally feasible. That is, we always compute marginal attributions and interaction indices when $n$ is sufficiently small. In figures, we only show the best performing baselines. Results and implementation details for all baselines can be found in 
Appendix~\ref{apdx:experiments}.

\vspace{-6pt}
\paragraph{Hyperparameters} \SpecExp{} has several parameters to determine the number of model inferences (masks). We choose $C=3$, informed by \citet{li2015spright} under a simplified sparse Fourier setting. We fix $t = 5$, which is the error correction capability of \SpecExp{} and serves as an approximate bound on the maximum degree. 
%
We also set $b=8$; the total collected samples are $\approx C2^bt \log(n)$. 
%
For $\ell_1$ regression-based interaction indices, we choose the regularization parameter via $5$-fold cross-validation. 




\vspace{-3pt}
\subsection{Metrics}


We compare \SpecExp{} to other methods across a variety of well-established metrics to assess performance.
%\Efe{How about textbf rather than emph here?}

\textbf{Faithfulness}: To characterize how well the surrogate function $\hat{f}$ approximates the true function, we define \emph{faithfulness} \cite{zhang2023trade}:
\vspace{-3pt}
\begin{equation}
    R^2 = 1 -  \frac{\lVert \hat{f} - f \rVert^2}{\left\lVert f - \bar{f} \right\rVert^2},
\end{equation}
where $\left\lVert f  \right\rVert^2 = \sum_{\bbm \in \bbF_2^n}f(\bbm)^2$ and $\bar{f} = \frac{1}{2^n} \sum_{\bbm \in \bbF_2^n}f(\bbm)$.

Faithfulness measures the ability of different explanation methods to predict model output when masking random inputs. 
%
We measure faithfulness over 10,000 random \emph{test} masks per-sample, and report average $R^2$ across samples. 
%

\textbf{Top-$r$ Removal}: We measure the ability of methods to identify the top $r$ influential features to model output:
\vspace{-2pt}
\begin{align}
\begin{split}
    \mathrm{Rem}(r) = \frac{|f(\boldsymbol{1}) - f(\bbm^*)|}{|f(\boldsymbol{1})|}, \\
    \;\bbm^* = \argmax \limits_{\abs{\bbm} = n-r}|\hat{f}(\boldsymbol{1}) - \hat{f}(\bbm)|.
\end{split}
\end{align}
\vspace{-8pt}


\textbf{Recovery Rate@$r$:} 
%
Each question in \emph{HotpotQA} contains human-labeled annotations for the sentences required to correctly answer the question. 
%
We measure the ability of interaction indices to recover these human-labeled annotations. 
%
Let $S_{r^*} \subseteq [n]$ denote human-annotated sentence indices. %corresponding to the human-annotated sentences containing the answer. 
Let $S_{i}$ denote feature indices of the $i^{\text{th}}$ most important interaction for a given interaction index.
%
Define the recovery ability at $r$ for each method as follows
\vspace{-2pt}
\begin{equation}
\label{eq:recovery_k}
    \text{Recovery@}r = 
    \frac{1}{r}\sum^r_{i=1}\frac{\abs{S_r^*\cap S_i}}{|S_{i}|}.
\end{equation}
\vspace{-8pt}

Intuitively, \eqref{eq:recovery_k} measures how well interaction indices capture features that align with human-labels.   


\begin{figure*}[t]
\centering
\hfill
\begin{subfigure}[b]{.5\textwidth}
  \centering
    \hspace{0.82cm}\includegraphics[width=0.75\textwidth]{figures/recall_legend.pdf}
  \includegraphics[width=.9\linewidth]{figures/hotpot_recall.pdf}
  \caption{Recovery rate$@10$ for \emph{HotpotQA} }
  \label{fig:recovery_hotpot}
\end{subfigure}%
\hfill % To ensure space between the figures
\begin{subfigure}[b]{.46\textwidth}
  \centering
    \includegraphics[width=1\textwidth]{figures/hotpot.pdf}
  \caption{Human-labeled interaction identified by \SpecExp{}.}
  \label{fig:hotpot_additional}
\end{subfigure}
\hfill
\caption{(a) \SpecExp{} recovers more human-labeled features with significantly fewer training masks as compared to other methods. (b) For a long-context example ($n = 128$ sentences), \SpecExp{} identifies the three human-labeled sentences as the most important third order interaction while ignoring unimportant contextual information.}
\vspace{-8pt}
\end{figure*}

\vspace{-8pt}
\subsection{Faithfulness and Runtime}
\vspace{-3pt}

Fig.~\ref{fig:faith} shows the faithfulness of \SpecExp{} compared to other methods. We also plot the runtime of all approaches for the \emph{Sentiment} dataset for different values of $n$. 
%
All attribution methods are learned over a fixed number of training masks.
% 

\textbf{Comparison to Interaction Indices } \SpecExp{} maintains competitive performance with the best-performing interaction indices across datasets. 
%
Recall these indices enumerate \emph{all possible interactions}, whereas \SpecExp{} does not. 
%
This difference is reflected in the runtimes of Fig.~\ref{fig:faith}(a).
%
The runtime of other interaction indices explodes as $n$ increases while \SpecExp{} does not suffer any increase in runtime. 

\vspace{-2pt}
\textbf{Comparison to Marginal Attributions } For input lengths $n$ too large to run interaction indices, \SpecExp{} is significantly more faithful than marginal attribution approaches across all three datasets.

\vspace{-2pt}
\textbf{Varying number of training masks } Results in Appendix ~\ref{apdx:experiments} show that \SpecExp{} continues to out-perform other approaches as we vary the number of training masks. 

\vspace{-2pt}
\textbf{Sparsity of \SpecExp{} Surrogate Function} Results in Appendix ~\ref{apdx:experiments}, Table~\ref{tab:faith} show 
surrogate functions learned by \SpecExp{} have Fourier representations where only $\sim 10^{-100}$ percent of coefficients are non-zero. 


\vspace{-6pt}
\subsection{Removal}
\label{subsec:removal}

Fig.~\ref{fig:fidelity} plots the change in model output as we mask the top $r$ features for different regimes of $n$. 
%

\vspace{-2pt}
\textbf{Small $n$ } \SpecExp{} is competitive with other interaction indices for \textit{Sentiment}, and out-performs them for \textit{HotpotQA} and \textit{DROP}. 
%
Performance of \SpecExp{} in this task is particularly notable since Shapley-based methods are designed to identify a small set of influential features. 
%
On the other hand, \SpecExp{} does not optimize for this metric, but instead learns the function $f(\cdot)$ over all possible $2^n$ masks. 
%

\textbf{Large $n$ } \SpecExp{} out-performs all marginal approaches, indicating the utility of considering interactions.
%

\vspace{-10pt}
\subsection{Recovery Rate of Human-Labeled Interactions}

%
We compare the recovery rate \eqref{eq:recovery_k} for $r = 10$ of \SpecExp{} against third order Faith-Banzhaf and Faith-Shap interaction indices. 
%
We choose third order interaction indices because all examples 
are answerable with information from at most three sentences, i.e., maximum degree $d = 3$.
%
Recovery rate is measured as we vary the number of training masks. 

Results are shown in Fig.~\ref{fig:recovery_hotpot}, where \SpecExp{} has the highest recovery rate of all interaction indices across all sample sizes. 
%
Further, \SpecExp{} achieves close to its maximum performance with few samples, other approaches require many more samples to approach the recovery rate of \SpecExp{}. 

\textbf{Example of Learned Interaction by \SpecExp{}} Fig.~\ref{fig:hotpot_additional} displays a long-context example (128 sentences) from \emph{HotpotQA} whose answer is contained in the three highlighted sentences. 
%
\SpecExp{} identifies the three human-labeled sentences as the most important third order interaction while ignoring unimportant contextual information. 
%
Other third order methods are not computable at this length. 
%

\begin{figure*}[t]
    \centering
    \includegraphics[width=0.9\linewidth]{figures/case_studies.pdf}
    \caption{SHAP provides marginal feature attributions. Feature interaction attributions computed by SPEX provide a more comprehensive understanding of (above) words interactions that cause the model to answer incorrectly and (below) interactions between image patches that informed the model's output.}
    \label{fig:caseStudies}
\end{figure*}


