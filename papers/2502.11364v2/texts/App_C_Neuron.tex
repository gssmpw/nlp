\section{ICL Behavioral Analysis} \label{app:neuron}
\subsection{Specilized Neuron}
Inspired by the universal concept space \cite{llama_english_epfl, neuron_plnd, semantic_hub}, we hypothesize that MLLMs could activate more cross-lingual capabilities by aligning different linguistic representations. To validate the above hypothesis, we seek to find patterns in neuron behavior between ICL modes. Following \citet{neuron_specialization_translation,language_specific_neuron_msra,language_specific_neuron_tokyo,neuron_pim,neuron_plnd,neuron_sft}, we look at the activations of neurons in the multilayer perceptron (MLP, or \textit{Feedforward Network, FFN}) modules of the MLLMs.

\subsubsection{Identifying Top-Activated Neurons}
Each neuron in every MLP layer of the model is assigned a dedicated counter, initialized to $0$. During vanilla evaluation, we monitor the activation of every neuron in the forward pass. Since LLMs typically employ ReLU-like \cite{relu} activation functions (e.g., SwiGLU \cite{glu} for \textsf{Llama} series), a positive activation value can be interpreted as the neuron being ``activated''. If a neuron is ``activated'', we increment the corresponding counter by $1$, otherwise no action. To ensure balanced input across our three datasets, we curated a \combined dataset, see \cref{app:dataset} for details. After processing the inputs of a single ICL mode, each neuron accumulates an  ``activated'' count. The neurons with the highest counts are identified as specialized neurons attributed to this ICL mode.


We employ two methods for selecting the most activated neurons. Top-$k$ selects neurons in the top $k$ percentile \cite{language_specific_neuron_msra}, while top-$p$ selects neurons progressively until their cumulative activation counts reach $p\ (\%)$ of the sum of all activation values \cite{neuron_specialization_translation}. 



\subsection{\multilingual-specific Neurons Overlap Most with \native-specific Neurons}
\begin{table}[!htbp]
\setlength{\tabcolsep}{4pt}
    \small
    \centering
    \alternaterowcolors
\begin{tabular}{lccc}
\toprule
\midrule
\bfseries IoU $(\%)$&
\textbf{All Langs} &
\textbf{LRL} &
\textbf{HRL} \\
\midrule

\multicolumn{4}{l}{\textbf{\english -- \multilingual -- \native}} \\
\midrule

\multicolumn{4}{l}{\textbf{\llamaThreeOne}} \\
\native - \english & $\dynamicCellColor{61.82}$ & $\dynamicCellColor{56.81}$ & $\dynamicCellColor{68.50}$ \\
\native - \multilingual & $\dynamicCellColor{78.84}$ & $\dynamicCellColor{66.19}$ & $\dynamicCellColor{85.10}$ \\
\english - \multilingual & $\dynamicCellColor{65.84}$ & $\dynamicCellColor{66.89}$ & $\dynamicCellColor{64.60}$ \\

\multicolumn{4}{l}{\textbf{\qwenTwo}} \\
\native - \english & $\dynamicCellColor{70.61}$ & $\dynamicCellColor{66.05}$ & $\dynamicCellColor{81.22}$ \\
\native - \multilingual & $\dynamicCellColor{85.13}$ & $\dynamicCellColor{77.40}$ & $\dynamicCellColor{91.31}$ \\
\english - \multilingual & $\dynamicCellColor{78.24}$ & $\dynamicCellColor{79.83}$ & $\dynamicCellColor{77.46}$ \\

\midrule
\multicolumn{4}{l}{\textbf{\english -- \chinese -- \native}} \\
\midrule

\multicolumn{4}{l}{\textbf{\llamaThreeOne}} \\
\native - \english & $\dynamicCellColor{61.82}$ & $\dynamicCellColor{56.81}$ & $\dynamicCellColor{68.50}$ \\
\native - \chinese & $\dynamicCellColor{60.58}$ & $\dynamicCellColor{55.35}$ & $\dynamicCellColor{65.73}$ \\
\english - \chinese & $\dynamicCellColor{56.52}$ & $\dynamicCellColor{57.94}$ & $\dynamicCellColor{55.72}$ \\

\multicolumn{4}{l}{\textbf{\qwenTwo}} \\
\native - \english & $\dynamicCellColor{70.61}$ & $\dynamicCellColor{66.05}$ & $\dynamicCellColor{81.22}$ \\
\native - \chinese & $\dynamicCellColor{73.41}$ & $\dynamicCellColor{69.39}$ & $\dynamicCellColor{82.18}$ \\
\english - \chinese & $\dynamicCellColor{77.07}$ & $\dynamicCellColor{78.34}$ & $\dynamicCellColor{76.51}$ \\

\midrule
\bottomrule
\end{tabular}
    \caption{The IoU score of most-activated neurons between every pair of ICL modes in triplets \english -- \multilingual -- \native and \english -- \chinese -- \native. Neurons were selected by first filtering out neurons outside the first 8 and last 8 MLP layers, and applying top-$k$ method with $k = 0.7$.}
    \vspace{-.2cm}
    \label{tab:neuron_statistics}
\end{table}

We examined the overlaps among the most-activated neurons (\textit{specialized neurons}) across different ICL modes by calculating the Intersection over Union (IoU) scores. For ICL modes $M_1$ and $M_2$, with specialized neurons denoted as sets $S_1$ and $S_2$, their overlap is quantified by \cref{eq:iou}:

\begin{equation}
    \begin{aligned}
    \operatorname{IoU}\left(S_1, S_2\right)=\frac{\left|S_1 \cap S_2\right|}{\left|S_1 \cup S_2\right|}.
    \end{aligned}
    \label{eq:iou}
\end{equation}

Prior research \cite{language_specific_neuron_msra,neuron_plnd} has discovered that \textit{language-specific neurons} are located primarily in the models' top and bottom layers. Because we want to explain the multilingual reasoning capabilities of a model, we only consider neurons that belong to a certain prefix or suffix of the models' layers in an effort to restrict our selected neuron sets to be mainly language-specific neurons. 

The similarity in performance between \multilingual and \native can be explained by the high overlap in the sets of most-activated neurons. On the other hand, the poorer performance of \english can be explained by the low overlap between \english most-activated neurons and other ICL modes . 

The same experiment was also performed but with  \multilingual replaced by HRL \chinese. In this case, the patterns were less obvious and model-specific. This result aligns with our findings in vanilla evaluation that \english $\le$ Non-\english HRL \monolingual $\le$ \multilingual (\cref{sec:mono_vs_multilingual_results}).

We further split this neuron experiment to either use only LRL or HRL ICL modes when recording neuron activations. We observed that HRL tends to activate similar neurons between \native and \multilingual, whereas LRL tends to activate similar neurons between \english and \multilingual.



\begin{table*}[!htbp]
    \setlength{\tabcolsep}{3.7pt}
    \scriptsize
    \centering
    \alternaterowcolors
\begin{tabular}{l|ccccccccccc|lll}
\toprule
    \textbf{\mgsm} &
  \textbf{\underline{bn}} &
  \textbf{de} &
  \textbf{en} &
  \textbf{es} &
  \textbf{fr} &
  \textbf{ja} &
  \textbf{ru} &
  \textbf{\underline{sw}} &
  \textbf{\underline{te}} &
  \textbf{\underline{th}} &
  \textbf{zh} &
  \textbf{\underline{LRL Avg}} &
  \textbf{HRL Avg} &
  \textbf{ALL Avg} \\ 
  \midrule


\multicolumn{15}{l}{\textbf{\llamaThree}} \\
\english      & 66.40 & 76.40 & 86.40 & 78.80 & 77.60 & 66.40 & 77.20 & 59.20 & 60.00 & 71.20 & 75.60 & 64.20     & 76.91      & 72.29     \\
\french       & 67.60 & 77.60 & 86.00 & 77.60 & 75.20 & 70.00 & 79.60 & 56.40 & 59.60 & 68.00 & 70.40 & \decrease{62.90}{1.30} & \decrease{76.63}{0.28} & \decrease{71.64}{0.65} \\
\chinese      & 66.80 & 77.20 & 82.80 & 79.60 & 74.80 & 68.00 & 76.40 & 55.60 & 59.20 & 70.40 & 72.80 & \decrease{63.00}{1.20} & \decrease{75.94}{0.97} & \decrease{71.24}{1.05} \\
\japanese     & 66.80 & 78.80 & 83.20 & 76.00 & 74.80 & 70.80 & 76.40 & 56.40 & 56.80 & 68.80 & 68.00 & \decrease{62.20}{2.00} & \decrease{75.43}{1.48} & \decrease{70.62}{1.67} \\
\multilingual & 65.20 & 79.60 & 84.40 & 77.60 & 74.00 & 69.60 & 76.80 & 52.40 & 53.60 & 69.60 & 70.00 & \decrease{60.20}{4.00} & \decrease{76.00}{0.91} & \decrease{70.25}{2.04} \\
\native       & 64.80 & 76.80 & 86.40 & 80.00 & 75.20 & 70.80 & 76.80 & 58.40 & 54.80 & 70.40 & 72.80 & \decrease{62.10}{2.10} & \increase{76.97}{0.06}  & \decrease{71.56}{0.73} \\ 
\transEn & 63.60	 & 73.20	 &86.40	 &78.00	 &74.00	 &60.40	 &77.60	 &69.60	 &60.00	 &46.40	 &58.80	 &\decrease{59.90}{4.30}	 &\decrease{72.63}{4.28}	 &\decrease{68.00}{4.29} \\
\transSource &65.20	&78.80	&86.40	&79.60	&76.00	&68.40	&76.40	&58.80	&52.80	&69.20	&73.60	&\decrease{61.50}{2.70}	&\increase{77.03}{0.12}	&\decrease{71.38}{0.91}\\

\midrule


\multicolumn{15}{l}{\textbf{\llamaThreeOne}} \\
\english      & 52.40 & 69.20 & 87.20 & 63.60 & 76.80 & 68.00 & 78.80 & 67.20 & 39.60 & 69.20 & 75.20 & 57.10 & 74.11 & 67.93 \\
\french & 62.80 & 74.40 & 89.20 & 82.40 & 82.40 & 69.60 & 79.20 & 69.60 & 45.20 & 70.00 & 76.80 &  \increase{61.90}{4.80} &  \increase{79.14}{5.03} &  \increase{72.87}{4.94} \\
\chinese & 66.00 & 76.80 & 87.20 & 83.20 & 78.00 & 67.60 & 79.60 & 70.00 & 48.40 & 69.20 & 77.60 &  \increase{63.40}{6.30} &  \increase{78.57}{4.46} &  \increase{73.05}{5.12} \\
\japanese & 66.40 & 76.80 & 86.00 & 80.80 & 78.00 & 67.20 & 79.60 & 72.80 & 58.00 & 75.20 & 77.60 &  \increase{68.10}{11.00} &  \increase{78.00}{3.89} &  \increase{74.40}{6.47} \\
\multilingual & 68.00 & 76.40 & 88.00 & 84.80 & 80.40 & 69.20 & 80.40 & 68.80 & 55.60 & 71.60 & 75.20 &  \increase{66.00}{8.90} &  \increase{79.20}{5.09} &  \increase{74.40}{6.47} \\
\native & 67.20 & 80.40 & 87.20 & 81.20 & 82.40 & 67.20 & 80.00 & 72.40 & 58.00 & 76.40 & 77.60 &  \increase{68.50}{11.40} &  \increase{79.43}{5.32} &  \increase{75.45}{7.52} \\
\transEn & 62.80	&74.80	&87.20	&82.00	&77.60	&62.80	&81.60	&68.00	&62.00	&47.20	&57.60	&\increase{60.00}{2.90}	&\increase{74.80}{0.69}	&\increase{69.42}{1.49} \\
\transSource & 67.60	&80.00	&87.20	&82.80	&83.60	&70.80	&80.80	&70.80	&59.60	&76.40	&79.20	&\increase{68.60}{11.50}	&\increase{80.63}{6.52}	&\increase{76.25}{8.32} \\
\midrule

\multicolumn{15}{l}{\textbf{\qwenTwo}} \\
\english      & 57.20 & 74.00 & 90.40 & 82.00 & 80.00 & 67.60 & 80.40 & 20.80 & 23.20 & 73.60 & 79.20 & 43.70 & 79.09 & 66.22 \\
\french & 54.80 & 74.00 & 92.00 & 82.00 & 79.60 & 66.80 & 79.20 & 22.40 & 20.80 & 73.60 & 80.40 &  \decrease{42.90}{0.80} &  \increase{79.14}{0.05} &  \decrease{65.96}{0.26} \\
\chinese & 56.00 & 76.40 & 92.00 & 81.20 & 77.60 & 71.20 & 79.60 & 28.80 & 20.40 & 73.60 & 85.20 &  \increase{44.70}{1.00} &  \increase{80.46}{1.37} &  \increase{67.45}{1.23} \\
\japanese & 51.20 & 72.80 & 88.80 & 78.80 & 76.00 & 75.20 & 79.20 & 27.20 & 20.80 & 72.40 & 80.80 &  \decrease{42.90}{0.80} &  \decrease{78.80}{0.29} &  \decrease{65.75}{0.47} \\
\multilingual & 64.80 & 77.20 & 89.20 & 82.80 & 81.60 & 70.40 & 82.00 & 28.40 & 23.60 & 73.20 & 79.60 &  \increase{47.50}{3.80} &  \increase{80.40}{1.31} &  \increase{68.44}{2.22} \\
\native & 72.00 & 83.60 & 90.40 & 82.80 & 79.60 & 75.20 & 83.20 & 32.40 & 40.40 & 76.80 & 85.20 &  \increase{55.40}{11.70} &  \increase{82.86}{3.77} &  \increase{72.87}{6.65} \\
\transEn & 66.00	&76.00	&90.40	&79.60	&79.20	&64.00	&77.60	&72.40	&61.20	&47.60	&58.40	&\increase{61.80}{18.10}	&\decrease{75.03}{4.06}	&\increase{70.22}{4.00} \\
\transSource & 72.40	&81.20	&90.40	&82.40	&80.00	&69.20	&81.20	&32.80	&43.20	&74.80	&80.80	&\increase{55.80}{12.10}	&\increase{80.74}{1.65}	&\increase{71.67}{5.45} \\
\midrule

\multicolumn{15}{l}{\textbf{\qwenTwoFive}} \\
\english      & 75.20 & 82.40 & 94.40 & 88.80 & 87.60 & 76.80 & 88.00 & 30.80 & 48.80 & 82.80 & 86.80 & 59.40 & 86.40 & 76.58 \\
\french & 74.00 & 85.20 & 92.40 & 90.00 & 85.60 & 78.80 & 86.80 & 33.60 & 51.20 & 82.40 & 84.00 &  \increase{60.30}{0.90} &  \decrease{86.11}{0.29} &  \increase{76.73}{0.15} \\
\chinese & 77.20 & 85.20 & 94.00 & 88.80 & 86.00 & 81.60 & 88.00 & 31.20 & 50.40 & 81.20 & 86.40 &  \increase{60.00}{0.60} &  \increase{87.14}{0.74} &  \increase{77.27}{0.69} \\
\japanese & 74.80 & 84.80 & 94.80 & 90.80 & 88.00 & 81.20 & 88.40 & 32.80 & 50.80 & 83.20 & 85.60 &  \increase{60.40}{1.00} &  \increase{87.66}{1.26} &  \increase{77.75}{1.17} \\
\multilingual & 76.40 & 86.80 & 92.80 & 89.60 & 88.00 & 78.40 & 87.20 & 30.40 & 50.00 & 81.20 & 86.00 &  \increase{59.50}{0.10} &  \increase{86.97}{0.57} &  \increase{76.98}{0.40} \\
\native & 75.20 & 86.80 & 94.40 & 89.20 & 85.60 & 81.20 & 87.60 & 35.60 & 46.00 & 84.40 & 86.40 &  \increase{60.30}{0.90} &  \increase{87.31}{0.91} &  \increase{77.49}{0.91} \\
\transEn & 70.40	&79.60	&94.40	&83.60	&82.40	&64.00	&81.60	&72.80	&62.40	&48.00	&64.80	&\increase{63.40}{4.60}	&\decrease{78.63}{7.77}	&\decrease{73.09}{3.49} \\
\transSource & 74.80	&85.60	&94.40	&88.80	&84.40	&80.40	&87.20	&35.20	&46.80	&82.80	&88.00	&\increase{59.90}{0.50}	&\increase{86.97}{0.57}	&\increase{77.13}{0.55} \\
\midrule

\multicolumn{15}{l}{\textbf{\mistral}} \\
\english      & 66.80 & 75.60 & 90.80 & 81.20 & 80.00 & 64.40 & 83.60 & 42.00 & 54.00 & 65.20 & 76.80 & 57.00 & 78.91 & 70.95 \\
\french & 66.00 & 81.20 & 92.40 & 82.00 & 81.60 & 74.80 & 83.20 & 46.40 & 70.40 & 74.40 & 76.40 &  \increase{64.30}{7.30} &  \increase{81.66}{2.75} &  \increase{75.35}{4.40} \\
\chinese & 70.40 & 81.60 & 91.60 & 81.60 & 82.00 & 73.20 & 84.80 & 49.20 & 69.60 & 69.20 & 77.20 &  \increase{64.60}{7.60} &  \increase{81.71}{2.80} &  \increase{75.49}{4.54} \\
\japanese & 67.20 & 84.40 & 90.00 & 85.60 & 83.20 & 74.40 & 84.80 & 45.60 & 69.60 & 72.00 & 76.00 &  \increase{63.60}{6.60} &  \increase{82.63}{3.72} &  \increase{75.71}{4.76} \\
\multilingual & 67.20 & 82.00 & 90.40 & 82.80 & 77.60 & 71.60 & 83.20 & 50.80 & 66.40 & 68.80 & 80.00 &  \increase{63.30}{6.30} &  \increase{81.09}{2.18} &  \increase{74.62}{3.67} \\
\native & 66.40 & 81.20 & 90.80 & 82.00 & 81.60 & 74.40 & 81.60 & 58.00 & 67.60 & 71.20 & 77.20 &  \increase{65.80}{8.80} &  \increase{81.26}{2.35} &  \increase{75.64}{4.69} \\
\transEn & 65.20	&76.80	&90.80	&82.00	&78.80	&62.00	&80.00	&71.60	&62.40	&47.20	&61.60	&\increase{61.60}{4.60}	&\decrease{76.00}{2.91}	&\decrease{70.76}{0.19} \\
\transSource & 68.80	&83.60	&90.80	&80.40	&82.00	&74.00	&85.60	&57.60	&66.80	&73.20	&77.60	&\increase{66.60}{9.60}	&\increase{82.00}{3.09}	&\increase{76.40}{5.45} \\
\midrule

\multicolumn{15}{l}{\textbf{\aya}} \\
\english      & 39.60 & 75.60 & 82.40 & 81.60 & 72.40 & 67.20 & 77.20 & 20.00 & 17.20 & 40.80 & 71.20 & 29.40 & 75.37 & 58.65 \\
\french & 43.60 & 74.40 & 81.60 & 80.00 & 75.20 & 67.60 & 77.60 & 19.60 & 23.20 & 36.40 & 74.00 &  \increase{30.70}{1.30} &  \increase{75.77}{0.40} &  \increase{59.38}{0.73} \\
\chinese & 44.00 & 74.40 & 80.40 & 77.60 & 73.20 & 66.00 & 77.20 & 20.00 & 22.40 & 40.40 & 76.40 &  \increase{31.70}{2.30} &  \decrease{75.03}{0.34} &  \increase{59.27}{0.62} \\
\japanese & 42.80 & 72.40 & 83.20 & 79.20 & 72.80 & 66.80 & 77.20 & 20.40 & 20.00 & 38.40 & 72.00 &  \increase{30.40}{1.00} &  \decrease{74.80}{0.57} & $58.65_{0.00-}$ \\
\multilingual & 46.80 & 74.00 & 83.60 & 78.00 & 72.80 & 67.60 & 79.60 & 19.20 & 17.60 & 36.80 & 72.00 &  \increase{30.10}{0.70} & $75.37_{0.00-}$ &  \increase{58.91}{0.26} \\
\native & 44.80 & 77.60 & 82.40 & 79.60 & 75.20 & 66.80 & 79.20 & 19.60 & 22.00 & 44.00 & 76.40 &  \increase{32.60}{3.20} &  \increase{76.74}{1.37} &  \increase{60.69}{2.04} \\
\transEn & 64.40	&71.20	&82.40	&76.40	&74.40	&57.60	&69.20	&65.20	&62.80	&48.00	&59.60	&\increase{60.10}{30.70}	&\decrease{70.11}{5.26}	&\increase{66.47}{7.82}\\
\transSource & 46.40	&74.80	&82.40	&80.00	&74.80	&66.00	&78.80	&19.20	&20.80	&42.40	&76.00	&\increase{32.20}{2.80}	&\increase{76.11}{0.74}	&\increase{60.15}{1.50}\\
\midrule

\multicolumn{15}{l}{\textbf{\gptThreeFive}} \\
\english      & 39.60 & 75.20 & 86.00 & 76.80 & 62.80 & 59.20 & 66.40 & 63.60 & 12.80 & 60.80 & 67.20 & 44.20 & 70.51 & 60.95 \\
\french & 34.00 & 74.80 & 86.00 & 84.80 & 79.20 & 66.00 & 67.60 & 63.20 & 15.20 & 55.60 & 74.40 &  \decrease{42.00}{2.20} &  \increase{76.11}{5.60} &  \increase{63.71}{2.76} \\
\chinese & 30.40 & 78.80 & 83.20 & 78.40 & 77.20 & 69.20 & 77.60 & 67.20 & 15.60 & 62.40 & 73.20 &  \decrease{43.90}{0.30} &  \increase{76.80}{6.29} &  \increase{64.84}{3.89} \\
\japanese & 27.60 & 78.80 & 85.20 & 82.00 & 72.40 & 73.20 & 74.40 & 69.60 & 14.00 & 61.20 & 76.00 &  \decrease{43.10}{1.10} &  \increase{77.43}{6.92} &  \increase{64.95}{4.00} \\
\multilingual & 54.40 & 79.60 & 83.60 & 79.20 & 73.60 & 68.00 & 75.20 & 68.00 & 24.40 & 57.20 & 74.40 &  \increase{51.00}{6.80} &  \increase{76.23}{5.72} &  \increase{67.05}{6.10} \\
\native & 57.20 & 79.60 & 86.00 & 80.40 & 81.60 & 75.20 & 77.60 & 73.60 & 30.00 & 58.80 & 74.00 &  \increase{54.90}{10.70} &  \increase{79.26}{8.75} &  \increase{70.40}{9.45} \\
\midrule

\multicolumn{15}{l}{\textbf{\gptFourO}} \\
\english      & 87.20 & 90.80 & 94.80 & 92.00 & 87.20 & 84.80 & 92.00 & 83.20 & 84.00 & 88.80 & 90.00 & 85.80 & 90.23 & 88.62 \\
\french & 87.20 & 90.80 & 94.40 & 92.80 & 89.20 & 82.40 & 90.80 & 85.20 & 84.00 & 88.80 & 87.60 &  \increase{86.30}{0.50} &  \decrease{89.71}{0.52} &  \decrease{88.47}{0.15} \\
\chinese & 86.80 & 90.80 & 93.60 & 93.20 & 90.00 & 83.20 & 91.20 & 85.20 & 84.40 & 90.00 & 90.80 &  \increase{86.60}{0.80} &  \increase{90.40}{0.17} &  \increase{89.02}{0.40} \\
\japanese & 85.20 & 89.60 & 94.80 & 92.80 & 86.80 & 86.00 & 92.00 & 82.80 & 81.60 & 88.40 & 87.60 &  \decrease{84.50}{1.30} &  \decrease{89.94}{0.29} &  \decrease{87.96}{0.66} \\
\multilingual & 86.40 & 90.00 & 92.80 & 94.00 & 89.20 & 84.00 & 92.40 & 84.00 & 80.40 & 88.00 & 89.20 &  \decrease{84.70}{1.10} & $90.23_{0.00-}$ &  \decrease{88.22}{0.40} \\
\native & 85.20 & 88.00 & 94.80 & 91.20 & 89.20 & 85.20 & 90.40 & 85.60 & 81.60 & 88.40 & 90.00 &  \decrease{85.20}{0.60} &  \decrease{89.66}{0.57} &  \decrease{88.04}{0.58} \\
\bottomrule


    \end{tabular}
    \caption{Accuracies ($\%$) of \english, \multilingual, \native, all \monolingual ICL modes (\french, \chinese and \japanese) and two translation strategies (\transEn, \transSource) across $11$ languages of the \mgsm dataset. AVG represents the average accuracy of the language set (LRLs, HRLs or All languages). The \underline{underlined languages} in the table header are \underline{LRLs}, otherwise HRLs. The subscript indicates the performance \textcolor{ForestGreen}{increase$\uparrow$} (or \textcolor{OrangeRed}{decrease$\downarrow$}) of all other modes compared to the \english ICL mode.}
    % \vspace{-.2cm}
    \label{tab:vanilla_eval:mgsm}
\end{table*}




\begin{table*}[!htbp]
    \setlength{\tabcolsep}{3pt}
    \scriptsize
    \centering
    \alternaterowcolors
\begin{tabular}{l|cccccccccccc|lll}
\toprule
\textbf{\xcopa} &
  \multicolumn{1}{c}{\textbf{en}} &
  \multicolumn{1}{c}{{\textbf{\underline{et}}}} &
  \multicolumn{1}{c}{{\textbf{\underline{ht}}}} &
  \multicolumn{1}{c}{\textbf{id}} &
  \multicolumn{1}{c}{\textbf{it}} &
  \multicolumn{1}{c}{{\textbf{\underline{qu}}}} &
  \multicolumn{1}{c}{{\textbf{\underline{sw}}}} &
  \multicolumn{1}{c}{{\textbf{\underline{ta}}}} &
  \multicolumn{1}{c}{{\textbf{\underline{th}}}} &
  \multicolumn{1}{c}{\textbf{tr}} &
  \multicolumn{1}{c}{\textbf{\underline{vi}}} &
  \textbf{zh} &
  \textbf{\underline{LRL AVG}} &
  \textbf{HRL AVG} &
  \textbf{ALL AVG} \\
\midrule


\multicolumn{16}{l}{\textbf{\llamaThree}} \\
\english      & 95.20 & 55.80 & 10.20 & 79.60 & 86.60 & 7.60  & 40.00 & 59.40 & 69.60 & 72.00 & 81.00 & 87.00 & 46.23 & 84.08 & 62.00 \\
\italian & 93.80 & 59.80 & 46.80 & 81.00 & 89.60 & 41.20 & 57.40 & 58.80 & 73.40 & 76.00 & 80.20 & 88.20 &  \increase{59.66}{13.43} &  \increase{85.72}{1.64} &  \increase{70.52}{8.52} \\
\chinese & 93.80 & 59.80 & 52.80 & 82.40 & 85.60 & 47.60 & 55.40 & 59.00 & 79.00 & 74.60 & 79.60 & 90.80 &  \increase{61.89}{15.66} &  \increase{85.44}{1.36} &  \increase{71.70}{9.70} \\
\multilingual & 94.60 & 57.80 & 51.80 & 81.60 & 87.80 & 46.40 & 60.40 & 60.80 & 79.20 & 78.80 & 80.40 & 88.80 &  \increase{62.40}{16.17} &  \increase{86.32}{2.24} &  \increase{72.37}{10.37} \\
\native & 95.20 & 68.60 & 61.60 & 85.00 & 89.60 & 50.40 & 62.00 & 65.80 & 77.80 & 79.80 & 84.80 & 90.80 &  \increase{67.29}{21.06} &  \increase{88.08}{4.00} &  \increase{75.95}{13.95} \\
\transEn & 95.20	&84.00	&69.80	&83.00	&88.40	&61.00	&69.00	&70.40	&66.00	&86.20	&85.20	&85.20	&\increase{72.20}{25.97}	&\increase{87.60}{3.52}	&\increase{78.62}{16.62} \\
\transSource & 95.20	&69.60	&62.00	&87.40	&89.20	&52.00	&62.00	&63.80	&78.60	&79.80	&84.00	&89.20	&\increase{67.43}{21.20}	&\increase{88.16}{4.08}	&\increase{76.07}{14.07} \\
\midrule

\multicolumn{16}{l}{\textbf{\llamaThreeOne}} \\
\english      & 95.60 & 65.20 & 27.00 & 84.20 & 88.40 & 26.00 & 52.60 & 62.40 & 73.80 & 76.80 & 84.40 & 89.80 & 55.91 & 86.96 & 68.85 \\
\italian & 95.00 & 62.20 & 28.40 & 86.00 & 92.80 & 32.60 & 61.60 & 70.20 & 76.60 & 78.20 & 85.00 & 90.80 &  \increase{59.51}{3.60} &  \increase{88.56}{1.60} &  \increase{71.62}{2.77} \\
\chinese & 95.00 & 65.20 & 48.40 & 83.80 & 88.20 & 32.20 & 58.80 & 68.80 & 76.40 & 78.60 & 86.00 & 91.40 &  \increase{62.26}{6.35} &  \increase{87.40}{0.44} &  \increase{72.73}{3.88} \\
\multilingual & 96.00 & 60.40 & 57.20 & 87.60 & 90.80 & 46.80 & 61.60 & 70.80 & 78.60 & 83.00 & 87.40 & 90.80 &  \increase{66.11}{10.20} &  \increase{89.64}{2.68} &  \increase{75.92}{7.07} \\
\native & 95.60 & 72.40 & 66.20 & 89.80 & 92.80 & 52.60 & 66.60 & 75.20 & 80.80 & 84.40 & 87.60 & 91.40 &  \increase{71.63}{15.72} &  \increase{90.80}{3.84} &  \increase{79.62}{10.77} \\
\transEn & 95.60	&87.60	&74.60	&88.20	&89.60	&68.20	&73.20	&74.20	&72.60	&88.00	&86.80	&87.00	&\increase{76.74}{20.83}	&\increase{89.68}{2.72}	&\increase{82.13}{13.28} \\
\transSource & 95.60	&74.40	&63.40	&89.00	&90.00	&52.20	&65.20	&74.00	&78.40	&83.60	&85.80	&92.00	&\increase{70.49}{14.58}	&\increase{90.04}{3.08}	&\increase{78.63}{9.78} \\
\midrule

\multicolumn{16}{l}{\textbf{\qwenTwo}} \\
\english      & 97.00 & 61.80 & 50.60 & 88.00 & 90.20 & 49.80 & 53.20 & 58.40 & 77.80 & 75.40 & 84.40 & 91.00 & 62.29 & 88.32 & 73.13 \\
\italian & 92.20 & 66.40 & 51.60 & 88.40 & 95.40 & 52.20 & 54.40 & 59.80 & 79.80 & 78.20 & 83.80 & 87.20 &  \increase{64.00}{1.71} &  \decrease{88.28}{0.04} &  \increase{74.12}{0.99} \\
\chinese & 88.60 & 65.00 & 51.80 & 81.20 & 86.00 & 50.20 & 53.00 & 60.00 & 79.00 & 77.60 & 83.60 & 93.60 &  \increase{63.23}{0.94} &  \decrease{85.40}{2.92} &  \decrease{72.47}{0.66} \\
\multilingual & 96.00 & 63.20 & 53.80 & 90.60 & 94.00 & 53.00 & 53.40 & 59.60 & 80.40 & 77.60 & 83.40 & 93.00 &  \increase{63.83}{1.54} &  \increase{90.24}{1.92} &  \increase{74.83}{1.70} \\
\native & 97.00 & 71.20 & 54.80 & 93.00 & 95.40 & 51.40 & 60.20 & 63.20 & 83.60 & 81.20 & 89.00 & 93.60 &  \increase{67.63}{5.34} &  \increase{92.04}{3.72} &  \increase{77.80}{4.67} \\
\transEn & 97.00	&88.40	&80.00	&91.00	&92.00	&75.60	&79.80	&82.80	&76.20	&90.20	&88.40	&89.40	&\increase{81.60}{19.31}	&\increase{91.92}{3.60}	&\increase{85.90}{12.77} \\
\transSource & 97.00	&67.20	&54.60	&91.20	&94.80	&51.00	&56.80	&61.60	&83.00	&80.20	&86.40	&91.80	&\increase{65.80}{3.51}	&\increase{91.00}{2.68}	&\increase{76.30}{3.17} \\
\midrule

\multicolumn{16}{l}{\textbf{\qwenTwoFive}} \\
\english      & 97.40 & 62.20 & 56.40 & 89.40 & 93.20 & 50.80 & 53.40 & 58.20 & 83.40 & 80.20 & 88.40 & 93.60 & 64.69 & 90.76 & 75.55 \\
\italian & 96.40 & 65.20 & 57.80 & 89.00 & 95.40 & 49.00 & 52.20 & 58.40 & 82.00 & 83.60 & 87.20 & 92.80 &  \decrease{64.54}{0.15} &  \increase{91.44}{0.68} &  \increase{75.75}{0.20} \\
\chinese & 95.80 & 65.00 & 58.40 & 89.00 & 92.40 & 50.80 & 51.40 & 58.60 & 83.40 & 82.00 & 89.80 & 94.60 &  \increase{65.34}{0.65} & $90.76_{0.00-}$ &  \increase{75.93}{0.38} \\
\multilingual & 97.00 & 65.40 & 56.80 & 91.80 & 94.00 & 49.20 & 49.60 & 59.20 & 83.60 & 84.00 & 88.60 & 92.40 &  \decrease{64.63}{0.06} &  \increase{91.84}{1.08} &  \increase{75.97}{0.42} \\
\native & 97.40 & 69.60 & 62.80 & 91.40 & 95.40 & 50.60 & 54.00 & 61.40 & 85.40 & 84.40 & 90.00 & 94.60 &  \increase{67.69}{3.00} &  \increase{92.64}{1.88} &  \increase{78.08}{2.53} \\
\transEn & 97.40	&87.80	&80.60	&91.20	&91.20	&75.00	&77.40	&80.60	&74.20	&89.80	&87.80	&88.00	&\increase{80.49}{15.80}	&\increase{91.52}{0.76}	&\increase{85.08}{9.53} \\
\transSource & 97.40	&67.40	&61.20	&91.00	&96.00	&46.80	&53.60	&60.40	&85.40	&82.80	&88.80	&95.20	&\increase{66.23}{1.54}	&\increase{92.48}{1.72}	&\increase{77.17}{1.62} \\
\midrule

\multicolumn{16}{l}{\textbf{\mistral}} \\
\english      & 96.60 & 57.60 & 58.00 & 83.00 & 93.20 & 50.60 & 51.60 & 73.40 & 64.60 & 73.60 & 80.00 & 91.20 & 62.26 & 87.52 & 72.78 \\
\italian & 96.00 & 57.40 & 55.40 & 82.00 & 95.80 & 48.60 & 53.80 & 73.00 & 64.80 & 73.00 & 81.60 & 90.00 &  \decrease{62.09}{0.17} &  \decrease{87.36}{0.16} &  \decrease{72.62}{0.16} \\
\chinese & 96.40 & 58.20 & 52.00 & 83.00 & 92.60 & 49.00 & 53.80 & 70.20 & 66.80 & 71.00 & 81.00 & 91.60 &  \decrease{61.57}{0.69} &  \decrease{86.92}{0.60} &  \decrease{72.13}{0.65} \\
\multilingual & 95.80 & 60.20 & 54.00 & 85.80 & 94.40 & 48.80 & 56.00 & 76.20 & 62.80 & 72.60 & 82.60 & 92.20 &  \increase{62.94}{0.68} &  \increase{88.16}{0.64} &  \increase{73.45}{0.67} \\
\native & 96.60 & 74.00 & 64.00 & 87.40 & 95.80 & 48.80 & 62.20 & 82.20 & 79.00 & 80.00 & 86.20 & 91.60 &  \increase{70.91}{8.65} &  \increase{90.28}{2.76} &  \increase{78.98}{6.20} \\
\transEn & 96.60	&88.00	&75.40	&87.60	&89.40	&70.00	&74.20	&77.40	&74.00	&88.20	&85.80	&84.80	&\increase{77.83}{15.57}	&\increase{89.32}{1.80}	&\increase{82.62}{9.84} \\
\transSource & 96.60	&71.40	&63.00	&88.00	&94.40	&49.20	&60.60	&76.40	&73.20	&80.60	&85.40	&93.00	&\increase{68.46}{6.20}	&\increase{90.52}{3.00}	&\increase{77.65}{4.87} \\
\midrule

\multicolumn{16}{l}{\textbf{\aya}} \\
\english      & 95.40 & 28.20 & 17.60 & 83.80 & 84.20 & 0.00  & 11.20 & 45.20 & 10.00 & 80.40 & 73.60 & 77.60 & 26.54 & 84.28 & 50.60 \\
\italian & 92.00 & 15.40 & 18.40 & 86.40 & 91.40 & 0.20 & 11.20 & 45.80 & 34.60 & 82.20 & 79.60 & 85.60 &  \increase{29.31}{2.77} &  \increase{87.52}{3.24} &  \increase{53.57}{2.97} \\
\chinese & 90.60 & 39.40 & 34.80 & 81.00 & 82.80 & 0.00 & 17.60 & 38.60 & 23.00 & 79.00 & 78.40 & 92.20 &  \increase{33.11}{6.57} &  \increase{85.12}{0.84} &  \increase{54.78}{4.18} \\
\multilingual & 93.40 & 52.20 & 52.00 & 88.00 & 90.00 & 5.60 & 46.00 & 58.20 & 43.60 & 84.00 & 83.20 & 90.20 &  \increase{48.69}{22.15} &  \increase{89.12}{4.84} &  \increase{65.53}{14.93} \\
\native & 95.40 & 54.00 & 56.20 & 88.80 & 91.40 & 53.40 & 53.40 & 69.40 & 62.40 & 85.60 & 86.20 & 92.20 &  \increase{62.14}{35.60} &  \increase{90.68}{6.40} &  \increase{74.03}{23.43} \\
\transEn & 95.40	&84.40	&75.20	&86.60	&86.80	&67.40	&74.40	&74.60	&67.80	&86.40	&84.20	&85.60	&\increase{75.43}{48.89}	&\increase{88.16}{3.88}	&\increase{80.73}{30.13} \\
\transSource & 95.40	&54.20	&54.40	&89.40	&90.80	&45.80	&51.00	&65.80	&60.00	&83.80	&86.00	&90.60	&\increase{59.60}{33.06}	&\increase{90.00}{5.72}	&\increase{72.27}{21.67} \\
\midrule

\multicolumn{16}{l}{\textbf{\gptThreeFive}} \\
\english      & 96.00 & 77.20 & 56.80 & 83.00 & 88.80 & 48.40 & 70.80 & 52.00 & 64.80 & 76.20 & 74.00 & 83.80 & 63.43 & 85.56 & 72.65 \\
\italian & 95.00 & 78.60 & 57.20 & 83.20 & 91.20 & 49.20 & 71.60 & 49.40 & 60.40 & 79.40 & 77.40 & 85.60 &  \decrease{63.40}{0.03} &  \increase{86.88}{1.32} &  \increase{73.18}{0.53} \\
\chinese & 94.40 & 77.20 & 58.60 & 84.40 & 86.80 & 48.80 & 69.80 & 50.00 & 62.00 & 76.80 & 75.20 & 86.80 &  \decrease{63.09}{0.34} &  \increase{85.84}{0.28} &  \decrease{72.57}{0.08} \\
\multilingual & 93.80 & 75.00 & 56.80 & 87.40 & 89.60 & 49.20 & 71.40 & 48.00 & 62.80 & 85.80 & 75.80 & 86.80 &  \decrease{62.71}{0.72} &  \increase{88.68}{3.12} &  \increase{73.53}{0.88} \\
\native & 96.00 & 85.20 & 67.60 & 87.20 & 90.80 & 48.00 & 77.20 & 53.20 & 61.40 & 87.60 & 77.60 & 87.60 &  \increase{67.17}{3.74} &  \increase{89.80}{4.24} &  \increase{76.60}{3.95} \\
\midrule

\multicolumn{16}{l}{\textbf{\gptFourO}} \\
\english      & 98.60 & 93.20 & 80.00 & 94.20 & 97.60 & 49.80 & 84.20 & 83.40 & 88.20 & 95.20 & 92.80 & 95.60 & 81.66 & 96.24 & 87.73 \\
\italian & 98.80 & 91.60 & 74.40 & 95.00 & 98.20 & 49.20 & 82.60 & 84.40 & 89.00 & 95.60 & 92.80 & 96.40 &  \decrease{80.57}{1.09} &  \increase{96.80}{0.56} &  \decrease{87.33}{0.40} \\
\chinese & 98.00 & 93.40 & 81.00 & 94.20 & 97.80 & 49.80 & 84.00 & 84.40 & 88.80 & 94.80 & 93.40 & 95.00 &  \increase{82.11}{0.45} &  \decrease{95.96}{0.28} &  \increase{87.88}{0.15} \\
\multilingual & 98.60 & 93.80 & 80.00 & 96.00 & 98.20 & 50.20 & 85.00 & 86.20 & 92.40 & 96.20 & 94.40 & 95.20 &  \increase{83.14}{1.48} &  \increase{96.84}{0.60} &  \increase{88.85}{1.12} \\
\native & 98.60 & 94.80 & 89.60 & 95.20 & 98.20 & 52.20 & 87.60 & 88.80 & 93.80 & 95.20 & 95.00 & 95.20 &  \increase{85.97}{4.31} &  \increase{96.48}{0.24} &  \increase{90.35}{2.62} \\
\bottomrule

    \end{tabular}
    \caption{Accuracies ($\%$) of \english, \multilingual, \native, both \monolingual ICL modes (\italian and \chinese) and two translation strategies (\transEn, \transSource) across $12$ languages of the \xcopa dataset. AVG represents the average accuracy of the language set (LRLs, HRLs or All languages). The \underline{underlined languages} in the table header are \underline{LRLs}, otherwise HRLs. The subscript indicates the performance \textcolor{ForestGreen}{increase$\uparrow$} (or \textcolor{OrangeRed}{decrease$\downarrow$}) of all other modes compared to the \english ICL mode.}
    % \vspace{-.2cm}
    \label{tab:vanilla_eval:xcopa}
\end{table*}

\begin{table*}[!htbp]
    \setlength{\tabcolsep}{2.5pt}
    \scriptsize
    \centering
    \alternaterowcolors
\begin{tabular}{l|ccccccccccccc|lll}
\toprule
\textbf{\xlwic} &
  \multicolumn{1}{c}{\textbf{\underline{bg}}} &
  \multicolumn{1}{c}{\textbf{da}} &
  \multicolumn{1}{c}{\textbf{de}} &
  \multicolumn{1}{c}{\textbf{en}} &
  \multicolumn{1}{c}{\textbf{\underline{et}}} &
  \multicolumn{1}{c}{\textbf{\underline{fa}}} &
  \multicolumn{1}{c}{\textbf{fr}} &
  \multicolumn{1}{c}{\textbf{\underline{hr}}} &
  \multicolumn{1}{c}{\textbf{it}} &
  \multicolumn{1}{c}{\textbf{ja}} &
  \multicolumn{1}{c}{\textbf{ko}} &
  \multicolumn{1}{c}{\textbf{nl}} &
  \textbf{zh} &
  \textbf{\underline{LRL AVG}} &
  \textbf{HRL AVG} &
  \textbf{ALL AVG} \\
  \midrule



\multicolumn{17}{l}{\textbf{\llamaThree}} \\
\english      & 55.13 & 66.15 & 59.49 & 67.44 & 55.38 & 63.33 & 59.49 & 55.64 & 53.85 & 54.62 & 56.67 & 55.90 & 64.10 & 57.37 & 59.74 & 59.01 \\
\french & 57.95 & 63.08 & 66.67 & 64.62 & 52.82 & 68.21 & 66.15 & 57.18 & 56.15 & 55.13 & 51.79 & 62.82 & 65.38 &  \increase{59.04}{1.67} &  \increase{61.31}{1.57} &  \increase{60.61}{1.60} \\
\chinese & 56.67 & 64.10 & 64.87 & 63.59 & 55.38 & 63.08 & 58.97 & 54.87 & 62.31 & 57.18 & 48.21 & 63.85 & 63.33 &  \increase{57.50}{0.13} &  \increase{60.71}{0.97} &  \increase{59.72}{0.71} \\
\japanese & 59.74 & 61.79 & 66.15 & 65.38 & 52.56 & 70.77 & 60.00 & 56.67 & 59.23 & 58.97 & 55.13 & 64.87 & 63.33 &  \increase{59.94}{2.57} &  \increase{61.65}{1.91} &  \increase{61.12}{2.11} \\
\multilingual & 57.18 & 60.00 & 62.82 & 64.10 & 53.85 & 69.23 & 60.00 & 56.92 & 58.97 & 58.46 & 57.44 & 62.82 & 58.72 &  \increase{59.29}{1.92} &  \increase{60.37}{0.63} &  \increase{60.04}{1.03} \\
\native & 63.59 & 55.90 & 69.23 & 67.44 & 62.05 & 68.72 & 66.15 & 58.21 & 59.49 & 58.97 & 66.67 & 66.41 & 63.33 &  \increase{63.14}{5.77} &  \increase{63.73}{3.99} &  \increase{63.55}{4.54} \\
\midrule

\multicolumn{17}{l}{\textbf{\llamaThreeOne}} \\
\english      & 51.54 & 50.77 & 61.28 & 66.92 & 44.62 & 29.74 & 56.67 & 51.79 & 15.90 & 48.46 & 33.85 & 51.28 & 57.44 & 44.42 & 49.17 & 47.71 \\
\french & 54.62 & 60.00 & 66.41 & 64.62 & 49.74 & 22.56 & 62.31 & 57.44 & 52.56 & 49.49 & 53.33 & 65.13 & 54.10 &  \increase{46.09}{1.67} &  \increase{58.66}{9.49} &  \increase{54.79}{7.08} \\
\chinese & 54.87 & 62.82 & 62.31 & 61.79 & 53.08 & 55.64 & 56.15 & 57.95 & 56.67 & 49.74 & 48.21 & 62.31 & 62.05 &  \increase{55.38}{10.96} &  \increase{58.01}{8.84} &  \increase{57.20}{9.49} \\
\japanese & 54.36 & 60.00 & 63.59 & 60.00 & 47.69 & 63.33 & 58.72 & 52.05 & 58.21 & 53.59 & 48.97 & 58.21 & 56.41 &  \increase{54.36}{9.94} &  \increase{57.52}{8.35} &  \increase{56.55}{8.84} \\
\multilingual & 57.69 & 60.00 & 65.38 & 62.56 & 55.13 & 58.46 & 57.95 & 56.92 & 55.38 & 54.36 & 52.31 & 64.62 & 58.46 &  \increase{57.05}{12.63} &  \increase{59.00}{9.83} &  \increase{58.40}{10.69} \\
\native & 61.79 & 64.10 & 68.46 & 66.92 & 62.05 & 70.51 & 62.31 & 57.18 & 56.15 & 53.59 & 63.08 & 69.49 & 62.05 &  \increase{62.88}{18.46} &  \increase{62.91}{13.74} &  \increase{62.90}{15.19} \\
\midrule

\multicolumn{17}{l}{\textbf{\qwenTwo}} \\
\english      & 28.21 & 38.46 & 66.67 & 68.46 & 56.92 & 53.85 & 63.33 & 54.87 & 39.23 & 1.28  & 15.90 & 64.36 & 0.77  & 48.46 & 39.83 & 42.49 \\
\french & 53.59 & 51.54 & 64.87 & 66.92 & 53.85 & 58.21 & 63.08 & 56.41 & 49.74 & 25.90 & 33.08 & 62.82 & 14.10 &  \increase{55.51}{7.05} &  \increase{48.01}{8.18} &  \increase{50.32}{7.83} \\
\chinese & 47.18 & 27.44 & 68.46 & 67.44 & 57.44 & 58.97 & 61.79 & 55.13 & 40.51 & 56.41 & 37.69 & 60.00 & 68.72 &  \increase{54.68}{6.22} &  \increase{54.27}{14.44} &  \increase{54.40}{11.91} \\
\japanese & 56.15 & 30.26 & 66.41 & 66.92 & 56.15 & 60.26 & 61.79 & 56.67 & 47.95 & 60.77 & 59.49 & 66.41 & 40.26 &  \increase{57.31}{8.85} &  \increase{55.58}{15.75} &  \increase{56.11}{13.62} \\
\multilingual & 53.59 & 60.00 & 65.64 & 65.64 & 58.46 & 58.72 & 60.26 & 54.36 & 56.67 & 58.72 & 66.92 & 67.18 & 57.95 &  \increase{56.28}{7.82} &  \increase{62.11}{22.28} &  \increase{60.32}{17.83} \\
\native & 55.13 & 62.05 & 69.49 & 68.46 & 59.23 & 62.05 & 63.08 & 54.62 & 60.51 & 60.77 & 68.21 & 64.87 & 68.72 &  \increase{57.76}{9.30} &  \increase{65.13}{25.30} &  \increase{62.86}{20.37} \\
\midrule

\multicolumn{17}{l}{\textbf{\qwenTwoFive}} \\
\english      & 58.97 & 61.28 & 73.33 & 70.51 & 55.64 & 53.59 & 65.90 & 63.08 & 57.95 & 55.64 & 64.10 & 67.69 & 59.74 & 57.82 & 64.02 & 62.11 \\
\french & 55.13 & 57.18 & 68.97 & 65.13 & 55.38 & 54.10 & 64.87 & 57.18 & 62.05 & 57.95 & 62.82 & 65.64 & 62.56 &  \decrease{55.45}{2.37} &  \decrease{63.02}{1.00} &  \decrease{60.69}{1.42} \\
\chinese & 53.33 & 54.87 & 67.18 & 67.18 & 54.62 & 51.28 & 63.85 & 57.44 & 57.95 & 64.87 & 62.82 & 61.79 & 66.67 &  \decrease{54.17}{3.65} &  \decrease{63.02}{1.00} &  \decrease{60.30}{1.81} \\
\japanese & 53.59 & 54.62 & 67.44 & 67.18 & 54.10 & 49.74 & 63.08 & 57.18 & 56.41 & 64.10 & 63.33 & 65.13 & 60.00 &  \decrease{53.65}{4.17} &  \decrease{62.36}{1.66} &  \decrease{59.68}{2.43} \\
\multilingual & 57.44 & 60.51 & 69.23 & 68.21 & 54.62 & 55.38 & 61.28 & 56.41 & 57.18 & 64.36 & 66.67 & 67.69 & 64.62 &  \decrease{55.96}{1.86} &  \increase{64.42}{0.40} &  \decrease{61.81}{0.30} \\
\native & 60.00 & 63.33 & 72.56 & 70.51 & 55.90 & 63.33 & 64.87 & 55.64 & 62.31 & 64.10 & 70.77 & 72.82 & 66.67 &  \increase{58.72}{0.90} &  \increase{67.55}{3.53} &  \increase{64.83}{2.72} \\
\midrule

\multicolumn{17}{l}{\textbf{\mistral}} \\
\english      & 52.56 & 63.85 & 70.51 & 65.64 & 51.79 & 49.23 & 61.03 & 52.56 & 38.97 & 23.08 & 53.59 & 67.44 & 54.10 & 51.54 & 55.36 & 54.18 \\
\french & 52.05 & 57.69 & 66.41 & 64.36 & 51.54 & 48.46 & 60.77 & 57.18 & 57.18 & 56.67 & 55.13 & 59.74 & 54.10 &  \increase{52.31}{0.77} &  \increase{59.12}{3.76} &  \increase{57.02}{2.84} \\
\chinese & 47.95 & 57.95 & 63.33 & 63.85 & 50.51 & 47.18 & 59.49 & 52.31 & 56.41 & 57.18 & 54.62 & 61.28 & 62.82 &  \decrease{49.49}{2.05} &  \increase{59.66}{4.30} &  \increase{56.53}{2.35} \\
\japanese & 48.21 & 53.08 & 60.77 & 64.10 & 51.28 & 47.44 & 58.46 & 52.31 & 56.92 & 64.36 & 56.67 & 58.72 & 53.59 &  \decrease{49.81}{1.73} &  \increase{58.52}{3.16} &  \increase{55.84}{1.66} \\
\multilingual & 51.54 & 56.67 & 67.18 & 61.28 & 52.05 & 49.74 & 61.28 & 51.28 & 54.36 & 57.69 & 56.67 & 58.72 & 59.49 &  \decrease{51.15}{0.39} &  \increase{59.26}{3.90} &  \increase{56.77}{2.59} \\
\native & 57.44 & 57.69 & 70.00 & 65.64 & 57.69 & 66.92 & 60.77 & 58.72 & 55.13 & 64.36 & 64.87 & 65.90 & 62.82 &  \increase{60.19}{8.65} &  \increase{63.02}{7.66} &  \increase{62.15}{7.97} \\
\midrule

\multicolumn{17}{l}{\textbf{\aya}} \\
\english      & 53.08 & 57.44 & 60.51 & 66.41 & 58.46 & 66.41 & 57.44 & 57.18 & 26.41 & 44.62 & 59.74 & 61.79 & 60.51 & 58.78 & 54.99 & 56.15 \\
\french & 55.64 & 59.23 & 71.79 & 63.85 & 57.69 & 72.31 & 64.87 & 54.10 & 61.79 & 65.38 & 65.64 & 70.77 & 65.64 &  \increase{59.94}{1.16} &  \increase{65.44}{10.45} &  \increase{63.75}{7.60} \\
\chinese & 55.90 & 64.62 & 68.97 & 63.85 & 56.15 & 73.59 & 61.28 & 53.08 & 58.46 & 59.49 & 67.69 & 66.41 & 61.79 &  \increase{59.68}{0.90} &  \increase{63.62}{8.63} &  \increase{62.41}{6.26} \\
\japanese & 59.49 & 58.72 & 70.00 & 66.67 & 58.21 & 71.54 & 63.59 & 55.13 & 54.62 & 61.03 & 68.46 & 70.00 & 64.36 &  \increase{61.09}{2.31} &  \increase{64.16}{9.17} &  \increase{63.21}{7.06} \\
\multilingual & 58.46 & 58.46 & 69.74 & 63.33 & 56.41 & 69.23 & 66.15 & 55.38 & 61.28 & 65.64 & 68.21 & 71.28 & 63.08 &  \increase{59.87}{1.09} &  \increase{65.24}{10.25} &  \increase{63.59}{7.44} \\
\native & 51.54 & 63.08 & 71.54 & 66.41 & 56.92 & 78.97 & 64.87 & 59.49 & 61.03 & 61.03 & 67.95 & 70.51 & 61.79 &  \increase{61.73}{2.95} &  \increase{65.36}{10.37} &  \increase{64.24}{8.09} \\
\midrule

\multicolumn{17}{l}{\textbf{\gptThreeFive}} \\
\english      & 54.36 & 50.77 & 62.31 & 63.59 & 54.62 & 54.10 & 58.46 & 51.79 & 32.82 & 30.77 & 56.67 & 65.13 & 21.03 & 53.72 & 49.06 & 50.49 \\
\french & 55.13 & 56.67 & 64.62 & 62.56 & 58.97 & 52.05 & 58.72 & 56.41 & 56.41 & 58.46 & 59.23 & 61.79 & 58.97 &  \increase{55.64}{1.92} &  \increase{59.72}{10.66} &  \increase{58.46}{7.97} \\
\chinese & 53.59 & 61.28 & 62.56 & 59.74 & 55.13 & 54.36 & 56.41 & 56.92 & 55.90 & 55.64 & 56.41 & 59.23 & 53.85 &  \increase{55.00}{1.28} &  \increase{57.89}{8.83} &  \increase{57.00}{6.51} \\
\japanese & 53.33 & 55.38 & 65.90 & 63.08 & 58.97 & 53.08 & 58.97 & 55.90 & 56.15 & 56.41 & 55.90 & 64.36 & 54.87 &  \increase{55.32}{1.60} &  \increase{59.00}{9.94} &  \increase{57.87}{7.38} \\
\multilingual & 52.82 & 60.00 & 66.92 & 59.23 & 60.00 & 54.36 & 60.77 & 56.41 & 53.59 & 56.67 & 61.54 & 63.33 & 59.49 &  \increase{55.90}{2.18} &  \increase{60.17}{11.11} &  \increase{58.86}{8.37} \\
\native & 54.62 & 62.56 & 64.87 & 63.59 & 59.49 & 58.46 & 58.21 & 60.26 & 54.36 & 57.95 & 59.74 & 64.87 & 53.85 &  \increase{58.21}{4.49} &  \increase{60.11}{11.05} &  \increase{59.53}{9.04} \\
\midrule

\multicolumn{17}{l}{\textbf{\gptFourO}} \\
\english      & 68.72 & 27.95 & 74.36 & 73.33 & 62.56 & 28.46 & 71.79 & 65.64 & 38.21 & 4.10  & 53.59 & 75.64 & 5.38  & 56.35 & 47.15 & 49.98 \\
\french & 66.41 & 58.97 & 72.82 & 67.95 & 60.51 & 39.23 & 71.28 & 68.97 & 61.28 & 63.08 & 72.05 & 71.03 & 70.00 &  \increase{58.78}{2.43} &  \increase{67.61}{20.46} &  \increase{64.89}{14.91} \\
\chinese & 67.44 & 58.21 & 73.08 & 69.23 & 58.72 & 50.26 & 67.69 & 68.97 & 58.72 & 66.15 & 70.77 & 71.79 & 73.59 &  \increase{61.35}{5.00} &  \increase{67.69}{20.54} &  \increase{65.74}{15.76} \\
\japanese & 66.92 & 52.82 & 72.56 & 67.69 & 60.00 & 39.23 & 66.67 & 65.13 & 57.95 & 66.67 & 68.46 & 73.85 & 67.95 &  \increase{57.82}{1.47} &  \increase{66.07}{18.92} &  \increase{63.53}{13.55} \\
\multilingual & 66.41 & 65.13 & 72.31 & 68.72 & 62.05 & 64.62 & 68.97 & 67.95 & 65.64 & 64.10 & 71.28 & 73.85 & 67.69 &  \increase{65.26}{8.91} &  \increase{68.63}{21.48} &  \increase{67.59}{17.61} \\
\native & 71.54 & 70.26 & 76.15 & 73.33 & 65.13 & 82.82 & 71.79 & 67.95 & 67.95 & 66.15 & 76.41 & 75.64 & 72.05 &  \increase{71.86}{15.51} &  \increase{72.19}{25.04} &  \increase{72.09}{22.11} \\
\bottomrule


    \end{tabular}
    \caption{Accuracies ($\%$) of \english, \multilingual, \native all \monolingual ICL modes (\french, \chinese and \japanese) and two translation strategies (\transEn, \transSource) across $13$ languages of the \xlwic dataset. AVG represents the average accuracy of the language set (LRLs, HRLs or All languages). The \underline{underlined languages} in the table header are \underline{LRLs}, otherwise HRLs. The subscript indicates the performance \textcolor{ForestGreen}{increase$\uparrow$} (or \textcolor{OrangeRed}{decrease$\downarrow$}) of all other modes compared to the \english ICL mode.}
    % \vspace{-.2cm}
    \label{tab:vanilla_eval:xlwic}
\end{table*}

\begin{table*}[!htbp]
    \setlength{\tabcolsep}{0.5pt}
    \scriptsize
    \centering
    \alternaterowcolors[5]
\begin{tabular}{l|llccccc|llccccc}
\toprule
\multicolumn{1}{l|}{\textbf{McNemar's Test}} & \multicolumn{7}{c|}{\textbf{Low-Resource Languages}} & \multicolumn{7}{c}{\textbf{High-Resource   Languages}} \\
\multicolumn{1}{l|}{\textbf{\mgsm ICL Mode}} &
  \multicolumn{1}{c|}{\textbf{$\chi^2$}} &
  \multicolumn{1}{c|}{\textbf{$p$-value}} &
  \multicolumn{1}{c|}{\textbf{Sig.}} &
  \multicolumn{1}{c|}{\textbf{\#Both}} &
  \multicolumn{1}{c|}{\textbf{\#M1 Wrong}} &
  \multicolumn{1}{c|}{\textbf{\#M1 Correct}} &
  \multicolumn{1}{c|}{\textbf{\#Both}} &
  \multicolumn{1}{c|}{\textbf{$\chi^2$}} &
  \multicolumn{1}{c|}{\textbf{$p$-value}} &
  \multicolumn{1}{c|}{\textbf{Sig.}} &
  \multicolumn{1}{c|}{\textbf{\#Both}} &
  \multicolumn{1}{c|}{\textbf{\#M1 Wrong}} &
  \multicolumn{1}{c|}{\textbf{\#M1 Correct}} &
  \multicolumn{1}{c}{\textbf{\#Both}} \\
\multicolumn{1}{l|}{\textbf{M1 vs M2 Comparison}} &
  \multicolumn{1}{c|}{\textbf{}} &
  \multicolumn{1}{c|}{\textbf{}} &
  \multicolumn{1}{c|}{\textbf{Level}} &
  \multicolumn{1}{c|}{\textbf{Wrong}} &
  \multicolumn{1}{c|}{\textbf{M2 Correct}} &
  \multicolumn{1}{c|}{\textbf{M2 Wrong}} &
  \multicolumn{1}{c|}{\textbf{Correct}} &
  \multicolumn{1}{c|}{\textbf{}} &
  \multicolumn{1}{c|}{\textbf{}} &
  \multicolumn{1}{c|}{\textbf{Level}} &
  \multicolumn{1}{c|}{\textbf{False}} &
  \multicolumn{1}{c|}{\textbf{M2 Correct}} &
  \multicolumn{1}{c|}{\textbf{M2 Wrong}} &
  \multicolumn{1}{c}{\textbf{Correct}} \\
  \midrule


\multicolumn{15}{l}{\textbf{\llamaThree}}                                                                                                  \\
\english vs \french          & 0.85   & $3.56\times10^{-1}$  &      & 280  & 78   & 91  & 551 & 0.08   & $7.84\times10^{-1}$  &      & 300  & 104  & 109  & 1237  \\
\english vs \chinese         & 0.65   & $4.20\times10^{-1}$  &      & 271  & 87   & 99  & 543 & 1.09   & $2.97\times10^{-1}$  &      & 295  & 109  & 126  & 1220  \\
\english vs \japanese        & 2.01   & $1.57\times10^{-1}$  &      & 278  & 80   & 100 & 542 & 2.58   & $1.08\times10^{-1}$  &      & 296  & 108  & 134  & 1212  \\
\english vs \multilingual    & 7.53   & $6.07\times10^{-3}$  & **   & 277  & 81   & 121 & 521 & 1.02   & $3.12\times10^{-1}$  &      & 302  & 102  & 118  & 1228  \\
\english vs \native          & 2.16   & $1.41\times10^{-1}$  &      & 276  & 82   & 103 & 539 & 0.00   & $1.00\times10^{0}$   &      & 294  & 110  & 109  & 1237  \\
\midrule

\multicolumn{15}{l}{\textbf{\llamaThreeOne}}                                                                                               \\
\english vs \french          & 11.75  & $6.08\times10^{-4}$  & ***  & 311  & 118  & 70  & 501 & 25.57  & $4.26\times10^{-7}$  & ***  & 261  & 192  & 104  & 1193  \\
\english vs \chinese         & 17.39  & $3.04\times10^{-5}$  & ***  & 287  & 142  & 79  & 492 & 19.63  & $9.39\times10^{-6}$  & ***  & 263  & 190  & 112  & 1185  \\
\english vs \japanese        & 49.92  & $1.60\times10^{-12}$ & ***  & 255  & 174  & 64  & 507 & 14.77  & $1.22\times10^{-4}$  & ***  & 267  & 186  & 118  & 1179  \\
\english vs \multilingual    & 33.24  & $8.16\times10^{-9}$  & ***  & 268  & 161  & 72  & 499 & 24.12  & $9.03\times10^{-7}$  & ***  & 248  & 205  & 116  & 1181  \\
\english vs \native          & 50.67  & $1.09\times10^{-12}$ & ***  & 246  & 183  & 69  & 502 & 27.57  & $1.52\times10^{-7}$  & ***  & 253  & 200  & 107  & 1190  \\

\multicolumn{15}{l}{\textbf{\qwenTwo}}                                                                                                     \\
\english vs \french          & 0.40   & $5.30\times10^{-1}$  &      & 505  & 58   & 66  & 371 & 0.00   & $1.00\times10^{0}$   &      & 277  & 89   & 88   & 1296  \\
\english vs \chinese         & 0.53   & $4.68\times10^{-1}$  &      & 481  & 82   & 72  & 365 & 3.01   & $8.30\times10^{-2}$  &      & 266  & 100  & 76   & 1308  \\
\english vs \japanese        & 0.33   & $5.68\times10^{-1}$  &      & 492  & 71   & 79  & 358 & 0.08   & $7.82\times10^{-1}$  &      & 264  & 102  & 107  & 1277  \\
\english vs \multilingual    & 7.36   & $6.67\times10^{-3}$  & **   & 451  & 112  & 74  & 363 & 2.41   & $1.21\times10^{-1}$  &      & 254  & 112  & 89   & 1295  \\
\english vs \native          & 56.78  & $4.88\times10^{-14}$ & ***  & 386  & 177  & 60  & 377 & 20.92  & $4.80\times10^{-6}$  & ***  & 232  & 134  & 68   & 1316  \\
\midrule

\multicolumn{15}{l}{\textbf{\qwenTwoFive}}                                                                                                 \\
\english vs \french          & 0.56   & $4.56\times10^{-1}$  &      & 344  & 62   & 53  & 541 & 0.15   & $7.02\times10^{-1}$  &      & 186  & 52   & 57   & 1455  \\
\english vs \chinese         & 0.21   & $6.48\times10^{-1}$  &      & 343  & 63   & 57  & 537 & 1.07   & $3.02\times10^{-1}$  &      & 164  & 74   & 61   & 1451  \\
\english vs \japanese        & 0.69   & $4.07\times10^{-1}$  &      & 342  & 64   & 54  & 540 & 3.20   & $7.38\times10^{-2}$  &      & 158  & 80   & 58   & 1454  \\
\english vs \multilingual    & 0.00   & $1.00\times10^{0}$   &      & 342  & 64   & 63  & 531 & 0.71   & $3.99\times10^{-1}$  &      & 176  & 62   & 52   & 1460  \\
\english vs \native          & 0.38   & $5.36\times10^{-1}$  &      & 318  & 88   & 79  & 515 & 1.76   & $1.85\times10^{-1}$  &      & 166  & 72   & 56   & 1456  \\
\midrule

\multicolumn{15}{l}{\textbf{\mistral}}                                                                                                     \\
\english vs \french          & 22.44  & $2.17\times10^{-6}$  & ***  & 278  & 152  & 79  & 491 & 8.98   & $2.73\times10^{-3}$  & **   & 222  & 147  & 99   & 1282  \\
\english vs \chinese         & 23.24  & $1.43\times10^{-6}$  & ***  & 271  & 159  & 83  & 487 & 10.82  & $1.01\times10^{-3}$  & **   & 238  & 131  & 82   & 1299  \\
\english vs \japanese        & 17.17  & $3.41\times10^{-5}$  & ***  & 274  & 156  & 90  & 480 & 16.32  & $5.35\times10^{-5}$  & ***  & 211  & 158  & 93   & 1288  \\
\english vs \multilingual    & 16.93  & $3.87\times10^{-5}$  & ***  & 285  & 145  & 82  & 488 & 5.66   & $1.74\times10^{-2}$  & *    & 229  & 140  & 102  & 1279  \\
\english vs \native          & 29.80  & $4.79\times10^{-8}$  & ***  & 259  & 171  & 83  & 487 & 7.05   & $7.93\times10^{-3}$  & **   & 235  & 134  & 93   & 1288  \\
\midrule

\multicolumn{15}{l}{\textbf{\aya}}                                                                                                         \\
\english vs \french          & 0.81   & $3.67\times10^{-1}$  &      & 611  & 95   & 82  & 212 & 0.17   & $6.78\times10^{-1}$  &      & 323  & 108  & 101  & 1218  \\
\english vs \chinese         & 2.70   & $1.00\times10^{-1}$  &      & 605  & 101  & 78  & 216 & 0.11   & $7.44\times10^{-1}$  &      & 317  & 114  & 120  & 1199  \\
\english vs \japanese        & 0.47   & $4.95\times10^{-1}$  &      & 614  & 92   & 82  & 212 & 0.31   & $5.75\times10^{-1}$  &      & 307  & 124  & 134  & 1185  \\
\english vs \multilingual    & 0.20   & $6.52\times10^{-1}$  &      & 614  & 92   & 85  & 209 & 0.00   & $9.51\times10^{-1}$  &      & 296  & 135  & 135  & 1184  \\
\english vs \native          & 4.62   & $3.16\times10^{-2}$  & *    & 586  & 120  & 88  & 206 & 2.47   & $1.16\times10^{-1}$  &      & 312  & 119  & 95   & 1224  \\
\midrule

\multicolumn{15}{l}{\textbf{\gptThreeFive}}                                                                                                \\
\english vs \french          & 1.93   & $1.64\times10^{-1}$  &      & 455  & 103  & 125 & 317 & 25.29  & $4.92\times10^{-7}$  & ***  & 281  & 235  & 137  & 1097  \\
\english vs \chinese         & 0.02   & $8.98\times10^{-1}$  &      & 438  & 120  & 123 & 319 & 32.82  & $1.01\times10^{-8}$  & ***  & 280  & 236  & 126  & 1108  \\
\english vs \japanese        & 0.44   & $5.09\times10^{-1}$  &      & 449  & 109  & 120 & 322 & 40.56  & $1.90\times10^{-10}$ & ***  & 278  & 238  & 117  & 1117  \\
\english vs \multilingual    & 17.81  & $2.44\times10^{-5}$  & ***  & 398  & 160  & 92  & 350 & 27.69  & $1.43\times10^{-7}$  & ***  & 289  & 227  & 127  & 1107  \\
\english vs \native          & 43.05  & $5.34\times10^{-11}$ & ***  & 374  & 184  & 77  & 365 & 65.82  & $4.93\times10^{-16}$ & ***  & 264  & 252  & 99   & 1135  \\
\midrule

\multicolumn{15}{l}{\textbf{\gptFourO}}                                                                                                    \\
\english vs \french          & 0.24   & $6.25\times10^{-1}$  &      & 106  & 36   & 31  & 827 & 0.81   & $3.68\times10^{-1}$  &      & 136  & 35   & 44   & 1535  \\
\english vs \chinese         & 0.77   & $3.82\times10^{-1}$  &      & 106  & 36   & 28  & 830 & 0.05   & $8.17\times10^{-1}$  &      & 132  & 39   & 36   & 1543  \\
\english vs \japanese        & 1.69   & $1.93\times10^{-1}$  &      & 106  & 36   & 49  & 809 & 0.19   & $6.61\times10^{-1}$  &      & 132  & 39   & 44   & 1535  \\
\english vs \multilingual    & 1.27   & $2.61\times10^{-1}$  &      & 108  & 34   & 45  & 813 & 0.01   & $9.11\times10^{-1}$  &      & 131  & 40   & 40   & 1539  \\
\english vs \native          & 0.30   & $5.85\times10^{-1}$  &      & 103  & 39   & 45  & 813 & 1.09   & $2.95\times10^{-1}$  &      & 139  & 32   & 42   & 1537  \\



\bottomrule
\end{tabular}

    \caption{McNemar's test results of ICL modes on LRL and HRL splits of \mgsm dataset. Baseline is the \english mode, compared with other \monolingual, \multilingual, and \native modes.}
    % \vspace{-.2cm}
    \label{tab:hyp_test:vanilla_eval:mgsm}
\end{table*}

\begin{table*}[!htbp]
    \setlength{\tabcolsep}{0.5pt}
    \scriptsize
    \centering
    \alternaterowcolors[5]
\begin{tabular}{l|llccccc|llccccc}
\toprule
\multicolumn{1}{l|}{\textbf{McNemar's Test}} & \multicolumn{7}{c|}{\textbf{Low-Resource Languages}} & \multicolumn{7}{c}{\textbf{High-Resource   Languages}} \\
\multicolumn{1}{l|}{\textbf{\xcopa ICL Mode}} &
  \multicolumn{1}{c|}{\textbf{$\chi^2$ }} &
  \multicolumn{1}{c|}{\textbf{$p$-value}} &
  \multicolumn{1}{c|}{\textbf{Sig.}} &
  \multicolumn{1}{c|}{\textbf{\#Both}} &
  \multicolumn{1}{c|}{\textbf{\#M1 Wrong}} &
  \multicolumn{1}{c|}{\textbf{\#M1 Correct}} &
  \multicolumn{1}{c|}{\textbf{\#Both}} &
  \multicolumn{1}{c|}{\textbf{$\chi^2$ }} &
  \multicolumn{1}{c|}{\textbf{$p$-value}} &
  \multicolumn{1}{c|}{\textbf{Sig.}} &
  \multicolumn{1}{c|}{\textbf{\#Both}} &
  \multicolumn{1}{c|}{\textbf{\#M1 Wrong}} &
  \multicolumn{1}{c|}{\textbf{\#M1 Correct}} &
  \multicolumn{1}{c}{\textbf{\#Both}} \\
\multicolumn{1}{l|}{\textbf{M1 vs M2 Comparison}} &
  \multicolumn{1}{c|}{\textbf{}} &
  \multicolumn{1}{c|}{\textbf{}} &
  \multicolumn{1}{c|}{\textbf{Level}} &
  \multicolumn{1}{c|}{\textbf{Wrong}} &
  \multicolumn{1}{c|}{\textbf{M2 Correct}} &
  \multicolumn{1}{c|}{\textbf{M2 Wrong}} &
  \multicolumn{1}{c|}{\textbf{Correct}} &
  \multicolumn{1}{c|}{\textbf{}} &
  \multicolumn{1}{c|}{\textbf{}} &
  \multicolumn{1}{c|}{\textbf{Level}} &
  \multicolumn{1}{c|}{\textbf{Wrong}} &
  \multicolumn{1}{c|}{\textbf{M2 Correct}} &
  \multicolumn{1}{c|}{\textbf{M2 Wrong}} &
  \multicolumn{1}{c}{\textbf{Correct}} \\
  \midrule



\multicolumn{15}{l}{\textbf{\llamaThree}}                                                                                                   \\
\english vs \italian         & 274.27  & $1.33\times10^{-61}$  & *** & 1246 & 636  & 166 & 1452 & 8.84    & $2.95\times10^{-3}$  & **   & 287  & 111  & 70   & 2032 \\
\english vs \chinese         & 347.11  & $1.80\times10^{-77}$  & *** & 1177 & 705  & 157 & 1461 & 5.85    & $1.55\times10^{-2}$  & *    & 288  & 110  & 76   & 2026 \\
\english vs \multilingual    & 366.08  & $1.33\times10^{-81}$  & *** & 1163 & 719  & 153 & 1465 & 15.76   & $7.21\times10^{-5}$  & ***  & 274  & 124  & 68   & 2034 \\
\english vs \native          & 468.19  & $7.94\times10^{-104}$ & *** & 935  & 947  & 210 & 1408 & 45.38   & $1.63\times10^{-11}$ & ***  & 240  & 158  & 58   & 2044 \\
\midrule

\multicolumn{15}{l}{\textbf{\llamaThreeOne}}                                                                                                \\
\english vs \italian         & 27.32   & $1.73\times10^{-7}$   & *** & 1194 & 349  & 223 & 1734 & 9.27    & $2.32\times10^{-3}$  & **   & 224  & 102  & 62   & 2112 \\
\english vs \chinese         & 74.68   & $5.53\times10^{-18}$  & *** & 1105 & 438  & 216 & 1741 & 0.63    & $4.28\times10^{-1}$  &      & 241  & 85   & 74   & 2100 \\
\english vs \multilingual    & 157.05  & $5.00\times10^{-36}$  & *** & 961  & 582  & 225 & 1732 & 21.04   & $4.49\times10^{-6}$  & ***  & 189  & 137  & 70   & 2104 \\
\english vs \native          & 276.51  & $4.32\times10^{-62}$  & *** & 723  & 820  & 270 & 1687 & 46.05   & $1.16\times10^{-11}$ & ***  & 180  & 146  & 50   & 2124 \\
\midrule

\multicolumn{15}{l}{\textbf{\qwenTwo}}                                                                                                      \\
\english vs \italian         & 8.45    & $3.65\times10^{-3}$   & **  & 1084 & 236  & 176 & 2004 & 0.00    & $1.00\times10^{0}$   &      & 216  & 76   & 77   & 2131 \\
\english vs \chinese         & 2.59    & $1.07\times10^{-1}$   &     & 1106 & 214  & 181 & 1999 & 26.05   & $3.33\times10^{-7}$  & ***  & 229  & 63   & 136  & 2072 \\
\english vs \multilingual    & 6.11    & $1.35\times10^{-2}$   & *   & 1063 & 257  & 203 & 1977 & 14.73   & $1.24\times10^{-4}$  & ***  & 193  & 99   & 51   & 2157 \\
\english vs \native          & 40.18   & $2.31\times10^{-10}$  & *** & 796  & 524  & 337 & 1843 & 47.82   & $4.67\times10^{-12}$ & ***  & 157  & 135  & 42   & 2166 \\
\midrule

\multicolumn{15}{l}{\textbf{\qwenTwoFive}}                                                                                                  \\
\english vs \italian         & 0.03    & $8.71\times10^{-1}$   &     & 937  & 299  & 304 & 1960 & 1.82    & $1.78\times10^{-1}$  &      & 152  & 79   & 62   & 2207 \\
\english vs \chinese         & 1.03    & $3.10\times10^{-1}$   &     & 990  & 246  & 223 & 2041 & 0.01    & $9.30\times10^{-1}$  &      & 167  & 64   & 64   & 2205 \\
\english vs \multilingual    & 0.00    & $9.68\times10^{-1}$   &     & 931  & 305  & 307 & 1957 & 4.60    & $3.20\times10^{-2}$  & *    & 144  & 87   & 60   & 2209 \\
\english vs \native          & 11.49   & $6.98\times10^{-4}$   & *** & 713  & 523  & 418 & 1846 & 13.65   & $2.20\times10^{-4}$  & ***  & 130  & 101  & 54   & 2215 \\
\midrule

\multicolumn{15}{l}{\textbf{\mistral}}                                                                                                      \\
\english vs \italian         & 0.05    & $8.30\times10^{-1}$   &     & 1054 & 267  & 273 & 1906 & 0.06    & $8.13\times10^{-1}$  &      & 234  & 78   & 82   & 2106 \\
\english vs \chinese         & 0.93    & $3.35\times10^{-1}$   &     & 1048 & 273  & 297 & 1882 & 1.06    & $3.03\times10^{-1}$  &      & 227  & 85   & 100  & 2088 \\
\english vs \multilingual    & 0.72    & $3.95\times10^{-1}$   &     & 943  & 378  & 354 & 1825 & 1.22    & $2.69\times10^{-1}$  &      & 212  & 100  & 84   & 2104 \\
\english vs \native          & 88.46   & $5.18\times10^{-21}$  & *** & 654  & 667  & 364 & 1815 & 23.24   & $1.43\times10^{-6}$  & ***  & 178  & 134  & 65   & 2123 \\
\midrule

\multicolumn{15}{l}{\textbf{\aya}}                                                                                                          \\
\english vs \italian         & 18.11   & $2.09\times10^{-5}$   & *** & 2268 & 303  & 206 & 723  & 28.96   & $7.39\times10^{-8}$  & ***  & 242  & 151  & 70   & 2037 \\
\english vs \chinese         & 100.08  & $1.46\times10^{-23}$  & *** & 2194 & 377  & 147 & 782  & 1.51    & $2.19\times10^{-1}$  &      & 250  & 143  & 122  & 1985 \\
\english vs \multilingual    & 637.99  & $9.13\times10^{-141}$ & *** & 1714 & 857  & 82  & 847  & 63.44   & $1.66\times10^{-15}$ & ***  & 219  & 174  & 53   & 2054 \\
\english vs \native          & 1003.90 & $2.55\times10^{-220}$ & *** & 1176 & 1395 & 149 & 780  & 104.47  & $1.60\times10^{-24}$ & ***  & 192  & 201  & 41   & 2066 \\
\midrule

\multicolumn{15}{l}{\textbf{\gptThreeFive}}                                                                                                 \\
\english vs \italian         & 0.00    & $1.00\times10^{0}$    &     & 890  & 390  & 391 & 1829 & 3.89    & $4.85\times10^{-2}$  & *    & 213  & 148  & 115  & 2024 \\
\english vs \chinese         & 0.14    & $7.06\times10^{-1}$   &     & 860  & 420  & 432 & 1788 & 0.11    & $7.36\times10^{-1}$  &      & 199  & 162  & 155  & 1984 \\
\english vs \multilingual    & 0.67    & $4.12\times10^{-1}$   &     & 864  & 416  & 441 & 1779 & 22.46   & $2.15\times10^{-6}$  & ***  & 190  & 171  & 93   & 2046 \\
\english vs \native          & 15.10   & $1.02\times10^{-4}$   & *** & 655  & 625  & 494 & 1726 & 39.95   & $2.61\times10^{-10}$ & ***  & 170  & 191  & 85   & 2054 \\
\midrule

\multicolumn{15}{l}{\textbf{\gptFourO}}                                                                                                     \\
\english vs \italian         & 5.19    & $2.28\times10^{-2}$   & *   & 529  & 113  & 151 & 2707 & 2.82    & $9.33\times10^{-2}$  &      & 57   & 37   & 23   & 2383 \\
\english vs \chinese         & 1.00    & $3.18\times10^{-1}$   &     & 521  & 121  & 105 & 2753 & 0.61    & $4.35\times10^{-1}$  &      & 68   & 26   & 33   & 2373 \\
\english vs \multilingual    & 9.63    & $1.91\times10^{-3}$   & **  & 481  & 161  & 109 & 2749 & 3.32    & $6.84\times10^{-2}$  &      & 57   & 37   & 22   & 2384 \\
\english vs \native          & 51.49   & $7.21\times10^{-13}$  & *** & 348  & 294  & 143 & 2715 & 0.40    & $5.25\times10^{-1}$  &      & 60   & 34   & 28   & 2378 \\


\bottomrule
\end{tabular}

    \caption{McNemar's test results of ICL modes on LRL and HRL splits of \xcopa dataset. Baseline is the \english mode, compared with other \monolingual, \multilingual, and \native modes.}
    % \vspace{-.2cm}
    \label{tab:hyp_test:vanilla_eval:xcopa}
\end{table*}

\begin{table*}[!htbp]
    \setlength{\tabcolsep}{0.5pt}
    \scriptsize
    \centering
    \alternaterowcolors[5]
\begin{tabular}{l|llccccc|llccccc}
\toprule
\multicolumn{1}{l|}{\textbf{McNemar's Test}} & \multicolumn{7}{c|}{\textbf{Low-Resource Languages}} & \multicolumn{7}{c}{\textbf{High-Resource   Languages}} \\
\multicolumn{1}{l|}{\textbf{\xlwic ICL Mode}} &
  \multicolumn{1}{c|}{\textbf{$\chi^2$}} &
  \multicolumn{1}{c|}{\textbf{$p$-value}} &
  \multicolumn{1}{c|}{\textbf{Sig.}} &
  \multicolumn{1}{c|}{\textbf{\#Both}} &
  \multicolumn{1}{c|}{\textbf{\#M1 Wrong}} &
  \multicolumn{1}{c|}{\textbf{\#M1 Correct}} &
  \multicolumn{1}{c|}{\textbf{\#Both}} &
  \multicolumn{1}{c|}{\textbf{$\chi^2$ }} &
  \multicolumn{1}{c|}{\textbf{$p$-value}} &
  \multicolumn{1}{c|}{\textbf{Sig.}} &
  \multicolumn{1}{c|}{\textbf{\#Both}} &
  \multicolumn{1}{c|}{\textbf{\#M1 Wrong}} &
  \multicolumn{1}{c|}{\textbf{\#M1 Correct}} &
  \multicolumn{1}{c}{\textbf{\#Both}} \\
\multicolumn{1}{l|}{\textbf{M1 vs M2 Comparison}} &
  \multicolumn{1}{c|}{\textbf{}} &
  \multicolumn{1}{c|}{\textbf{}} &
  \multicolumn{1}{c|}{\textbf{Level}} &
  \multicolumn{1}{c|}{\textbf{Wrong}} &
  \multicolumn{1}{c|}{\textbf{M2 Correct}} &
  \multicolumn{1}{c|}{\textbf{M2 Wrong}} &
  \multicolumn{1}{c|}{\textbf{Correct}} &
  \multicolumn{1}{c|}{\textbf{}} &
  \multicolumn{1}{c|}{\textbf{}} &
  \multicolumn{1}{c|}{\textbf{Level}} &
  \multicolumn{1}{c|}{\textbf{Wrong}} &
  \multicolumn{1}{c|}{\textbf{M2 Correct}} &
  \multicolumn{1}{c|}{\textbf{M2 Wrong}} &
  \multicolumn{1}{c}{\textbf{Correct}} \\
  \midrule


  
\multicolumn{15}{l}{\textbf{\llamaThree}}                                                                                                  \\
\english vs \french          & 1.85    & $1.74\times10^{-1}$  &      & 483  & 182 & 156 & 739 & 3.70    & $5.45\times10^{-2}$   &      & 991  & 422  & 367 & 1730 \\
\english vs \chinese         & 0.00    & $9.60\times10^{-1}$  &      & 469  & 196 & 194 & 701 & 1.38    & $2.41\times10^{-1}$   &      & 1000 & 413  & 379 & 1718 \\
\english vs \japanese        & 3.67    & $5.53\times10^{-2}$  &      & 438  & 227 & 187 & 708 & 4.59    & $3.22\times10^{-2}$   & *    & 905  & 508  & 441 & 1656 \\
\english vs \multilingual    & 1.89    & $1.70\times10^{-1}$  &      & 427  & 238 & 208 & 687 & 0.45    & $5.00\times10^{-1}$   &      & 917  & 496  & 474 & 1623 \\
\english vs \native          & 13.85   & $1.98\times10^{-4}$  & ***  & 334  & 331 & 241 & 654 & 19.84   & $8.43\times10^{-6}$   & ***  & 856  & 557  & 417 & 1680 \\
\midrule

\multicolumn{15}{l}{\textbf{\llamaThreeOne}}                                                                                               \\
\english vs \french          & 1.47    & $2.26\times10^{-1}$  &      & 641  & 226 & 200 & 493 & 113.05  & $2.10\times10^{-26}$  & ***  & 1130 & 654  & 321 & 1405 \\
\english vs \chinese         & 66.13   & $4.22\times10^{-16}$ & ***  & 563  & 304 & 133 & 560 & 94.72   & $2.19\times10^{-22}$  & ***  & 1125 & 659  & 349 & 1377 \\
\english vs \japanese        & 50.57   & $1.15\times10^{-12}$ & ***  & 555  & 312 & 157 & 536 & 96.34   & $9.66\times10^{-23}$  & ***  & 1195 & 589  & 296 & 1430 \\
\english vs \multilingual    & 82.26   & $1.19\times10^{-19}$ & ***  & 535  & 332 & 135 & 558 & 94.44   & $2.52\times10^{-22}$  & ***  & 985  & 799  & 454 & 1272 \\
\english vs \native          & 134.15  & $5.06\times10^{-31}$ & ***  & 416  & 451 & 163 & 530 & 197.07  & $9.10\times10^{-45}$  & ***  & 956  & 828  & 346 & 1380 \\
\midrule

\multicolumn{15}{l}{\textbf{\qwenTwo}}                                                                                                     \\
\english vs \french          & 26.64   & $2.45\times10^{-7}$  & ***  & 526  & 278 & 168 & 588 & 99.87   & $1.62\times10^{-23}$  & ***  & 1559 & 553  & 266 & 1132 \\
\english vs \chinese         & 25.96   & $3.48\times10^{-7}$  & ***  & 578  & 226 & 129 & 627 & 274.42  & $1.23\times10^{-61}$  & ***  & 1392 & 720  & 213 & 1185 \\
\english vs \japanese        & 34.50   & $4.26\times10^{-9}$  & ***  & 463  & 341 & 203 & 553 & 311.88  & $8.52\times10^{-70}$  & ***  & 1347 & 765  & 212 & 1186 \\
\english vs \multilingual    & 20.45   & $6.13\times10^{-6}$  & ***  & 385  & 419 & 297 & 459 & 468.48  & $6.86\times10^{-104}$ & ***  & 1070 & 1042 & 260 & 1138 \\
\english vs \native          & 35.09   & $3.15\times10^{-9}$  & ***  & 436  & 368 & 223 & 533 & 645.95  & $1.70\times10^{-142}$ & ***  & 1059 & 1053 & 165 & 1233 \\
\midrule

\multicolumn{15}{l}{\textbf{\qwenTwoFive}}                                                                                                 \\
\english vs \french          & 5.71    & $1.69\times10^{-2}$  & *    & 563  & 95  & 132 & 770 & 2.29    & $1.30\times10^{-1}$   &      & 1028 & 235  & 270 & 1977 \\
\english vs \chinese         & 13.46   & $2.44\times10^{-4}$  & ***  & 570  & 88  & 145 & 757 & 1.80    & $1.80\times10^{-1}$   &      & 959  & 304  & 339 & 1908 \\
\english vs \japanese        & 17.89   & $2.34\times10^{-5}$  & ***  & 576  & 82  & 147 & 755 & 5.40    & $2.02\times10^{-2}$   & *    & 991  & 272  & 330 & 1917 \\
\english vs \multilingual    & 3.10    & $7.83\times10^{-2}$  &      & 546  & 112 & 141 & 761 & 0.30    & $5.85\times10^{-1}$   &      & 973  & 290  & 276 & 1971 \\
\english vs \native          & 0.45    & $5.04\times10^{-1}$  &      & 462  & 196 & 182 & 720 & 25.30   & $4.91\times10^{-7}$   & ***  & 902  & 361  & 237 & 2010 \\
\midrule

\multicolumn{15}{l}{\textbf{\mistral}}                                                                                                     \\
\english vs \french          & 0.95    & $3.31\times10^{-1}$  &      & 686  & 70  & 58  & 746 & 25.69   & $4.01\times10^{-7}$   & ***  & 1167 & 400  & 268 & 1675 \\
\english vs \chinese         & 7.17    & $7.41\times10^{-3}$  & **   & 705  & 51  & 83  & 721 & 28.59   & $8.95\times10^{-8}$   & ***  & 1098 & 469  & 318 & 1625 \\
\english vs \japanese        & 5.08    & $2.42\times10^{-2}$  & *    & 703  & 53  & 80  & 724 & 16.60   & $4.62\times10^{-5}$   & ***  & 1147 & 420  & 309 & 1634 \\
\english vs \multilingual    & 0.17    & $6.83\times10^{-1}$  &      & 684  & 72  & 78  & 726 & 28.68   & $8.56\times10^{-8}$   & ***  & 1176 & 391  & 254 & 1689 \\
\english vs \native          & 44.12   & $3.09\times10^{-11}$ & ***  & 485  & 271 & 136 & 668 & 87.70   & $7.63\times10^{-21}$  & ***  & 1023 & 544  & 275 & 1668 \\
\midrule

\multicolumn{15}{l}{\textbf{\aya}}                                                                                                         \\
\english vs \french          & 0.50    & $4.79\times10^{-1}$  &      & 346  & 297 & 279 & 638 & 118.02  & $1.71\times10^{-27}$  & ***  & 829  & 751  & 384 & 1546 \\
\english vs \chinese         & 0.31    & $5.79\times10^{-1}$  &      & 361  & 282 & 268 & 649 & 89.50   & $3.06\times10^{-21}$  & ***  & 919  & 661  & 358 & 1572 \\
\english vs \japanese        & 2.26    & $1.33\times10^{-1}$  &      & 354  & 289 & 253 & 664 & 91.35   & $1.20\times10^{-21}$  & ***  & 855  & 725  & 403 & 1527 \\
\english vs \multilingual    & 0.43    & $5.12\times10^{-1}$  &      & 337  & 306 & 289 & 628 & 109.78  & $1.10\times10^{-25}$  & ***  & 813  & 767  & 407 & 1523 \\
\english vs \native          & 3.20    & $7.35\times10^{-2}$  &      & 304  & 339 & 293 & 624 & 133.37  & $7.51\times10^{-31}$  & ***  & 904  & 676  & 312 & 1618 \\



\bottomrule
\end{tabular}

    \caption{McNemar's test results of ICL modes on LRL and HRL splits of \xlwic dataset. Baseline is the \english mode, compared with other \monolingual, \multilingual, and \native modes.}
    % \vspace{-.2cm}
    \label{tab:hyp_test:vanilla_eval:xlwic}
\end{table*}


\begin{table*}[!htbp]
    \setlength{\tabcolsep}{4pt}
    \scriptsize
    \centering
    \alternaterowcolors
\begin{tabular}{l|ccccccccccc|lll}
\toprule
    \textbf{\mgsm \cis} &
  \textbf{\underline{bn}} &
  \textbf{de} &
  \textbf{en} &
  \textbf{es} &
  \textbf{fr} &
  \textbf{ja} &
  \textbf{ru} &
  \textbf{\underline{sw}} &
  \textbf{\underline{te}} &
  \textbf{\underline{th}} &
  \textbf{zh} &
  \textbf{\underline{LRL AVG}} &
  \textbf{HRL AVG} &
  \textbf{ALL AVG} \\ 
  \midrule


\multicolumn{15}{l}{\textbf{\llamaThree}} \\
\english$+\ $\cisEn           & 64.40 & 75.20 & 84.80 & 78.40 & 74.00 & 69.20 & 74.00 & 56.80 & 49.20 & 73.60 & 72.40 & 61.00 & 75.43 & 70.18 \\
\english$+\ $\cisFr           & 66.00 & 77.20 & 85.20 & 80.40 & 76.40 & 67.20 & 75.60 & 54.00 & 58.40 & 68.00 & 70.80 & 61.60 & 76.11 & 70.84 \\
\english$+\ $\cisJa          & 66.40 & 75.60 & 83.20 & 78.80 & 74.80 & 66.80 & 77.20 & 53.20 & 53.60 & 72.00 & 70.80 & 61.30 & 75.31 & 70.22 \\
\english$+\ $\cisZh           & 63.60 & 76.80 & 83.20 & 77.20 & 76.40 & 69.20 & 77.60 & 52.40 & 52.40 & 68.80 & 71.60 & 59.30 & 76.00 & 69.93 \\
\english$+\ $\cisMulti & 65.60 & 75.20 & 83.20 & 78.40 & 77.20 & 66.40 & 76.40 & 51.60 & 55.60 & 69.20 & 70.40 & 60.50 & 75.31 & 69.93 \\
\midrule

\multicolumn{15}{l}{\textbf{\llamaThreeOne}}                                                                                          \\
\english$+\ $\cisEn           & 52.40 & 70.80 & 88.40 & 70.80 & 75.20 & 70.40 & 74.00 & 67.60 & 38.40 & 65.20 & 73.20 & 55.90 & 74.69 & 67.85 \\
\english$+\ $\cisFr           & 52.80 & 74.40 & 89.20 & 79.60 & 76.80 & 69.20 & 78.80 & 60.00 & 37.60 & 58.00 & 76.40 & 52.10 & 77.77 & 68.44 \\
\english$+\ $\cisJa           & 61.20 & 76.00 & 87.60 & 78.40 & 75.20 & 70.80 & 78.00 & 66.40 & 41.20 & 66.40 & 75.60 & 58.80 & 77.37 & 70.62 \\
\english$+\ $\cisZh           & 58.00 & 72.40 & 89.60 & 80.80 & 76.00 & 66.00 & 77.20 & 69.60 & 38.80 & 53.60 & 76.00 & 55.00 & 76.86 & 68.91 \\
\english$+\ $\cisMulti & 62.00 & 74.80 & 88.80 & 81.60 & 78.00 & 69.20 & 79.60 & 68.80 & 46.40 & 72.80 & 78.80 & 62.50 & 78.69 & 72.80 \\
\midrule

\multicolumn{15}{l}{\textbf{\qwenTwo}}                                                                                        \\
\english$+\ $\cisEn           & 54.40 & 74.00 & 89.60 & 82.00 & 78.80 & 69.60 & 79.20 & 24.80 & 19.20 & 73.60 & 81.20 & 43.00 & 79.20 & 66.04 \\
\english$+\ $\cisFr           & 55.20 & 75.20 & 90.00 & 84.00 & 72.80 & 68.00 & 80.80 & 26.40 & 18.00 & 74.40 & 82.00 & 43.50 & 78.97 & 66.07 \\
\english$+\ $\cisJa           & 56.00 & 75.20 & 90.40 & 78.00 & 76.80 & 69.20 & 79.20 & 24.80 & 20.40 & 74.00 & 80.00 & 43.80 & 78.40 & 65.82 \\
\english$+\ $\cisZh           & 55.60 & 74.80 & 90.80 & 79.20 & 76.80 & 67.60 & 80.80 & 21.60 & 18.40 & 75.20 & 80.40 & 42.70 & 78.63 & 65.56 \\
\english$+\ $\cisMulti & 54.80 & 76.40 & 89.60 & 82.00 & 76.80 & 69.20 & 80.00 & 24.40 & 18.00 & 73.20 & 81.20 & 42.60 & 79.31 & 65.96 \\
\midrule

\multicolumn{15}{l}{\textbf{\qwenTwoFive}}                                                                                          \\
\english$+\ $\cisEn          & 75.20 & 86.40 & 92.40 & 88.00 & 87.60 & 80.80 & 88.00 & 31.60 & 48.00 & 82.80 & 85.20 & 59.40 & 86.91 & 76.91 \\
\english$+\ $\cisFr           & 75.60 & 85.20 & 94.00 & 89.60 & 86.00 & 78.80 & 87.20 & 31.60 & 46.40 & 84.00 & 82.80 & 59.40 & 86.23 & 76.47 \\
\english$+\ $\cisJa           & 76.80 & 86.40 & 92.40 & 90.40 & 85.60 & 81.20 & 86.00 & 29.60 & 50.40 & 83.60 & 82.00 & 60.10 & 86.29 & 76.76 \\
\english$+\ $\cisZh          & 78.00 & 88.40 & 93.60 & 89.20 & 88.00 & 80.40 & 90.00 & 29.20 & 52.40 & 84.00 & 82.80 & 60.90 & 87.49 & 77.82 \\
\english$+\ $\cisMulti & 76.00 & 86.40 & 94.00 & 90.40 & 84.80 & 80.80 & 86.80 & 30.00 & 51.60 & 80.40 & 82.80 & 59.50 & 86.57 & 76.73 \\
\midrule

\multicolumn{15}{l}{\textbf{\mistral}}                                                                                    \\
\english$+\ $\cisEn           & 61.20 & 80.00 & 92.00 & 82.00 & 82.40 & 71.60 & 82.00 & 48.40 & 62.40 & 70.80 & 78.00 & 60.70 & 81.14 & 73.71 \\
\english$+\ $\cisFr           & 64.80 & 83.20 & 90.40 & 84.00 & 82.80 & 72.00 & 84.40 & 46.40 & 63.20 & 71.20 & 78.00 & 61.40 & 82.11 & 74.58 \\
\english$+\ $\cisJa           & 67.20 & 82.80 & 92.40 & 81.20 & 83.60 & 73.60 & 84.80 & 43.20 & 63.20 & 67.20 & 76.80 & 60.20 & 82.17 & 74.18 \\
\english$+\ $\cisZh           & 62.80 & 82.00 & 91.60 & 80.40 & 83.20 & 70.40 & 86.00 & 44.40 & 65.60 & 69.20 & 80.40 & 60.50 & 82.00 & 74.18 \\
\english$+\ $\cisMulti & 72.00 & 81.60 & 92.00 & 84.00 & 83.60 & 73.60 & 86.40 & 47.20 & 67.20 & 73.20 & 79.20 & 64.90 & 82.91 & 76.36 \\
\midrule

\multicolumn{15}{l}{\textbf{\aya}}                                                                                                    \\
\english$+\ $\cisEn           & 36.00 & 76.40 & 82.40 & 83.20 & 74.80 & 69.20 & 77.60 & 24.00 & 15.60 & 35.60 & 70.80 & 27.80 & 76.34 & 58.69 \\
\english$+\ $\cisFr           & 42.00 & 72.80 & 83.60 & 79.20 & 73.20 & 72.00 & 76.80 & 20.80 & 17.20 & 34.80 & 72.40 & 28.70 & 75.71 & 58.62 \\
\english$+\ $\cisJa           & 40.80 & 76.40 & 82.00 & 79.60 & 74.80 & 68.80 & 76.00 & 23.20 & 18.00 & 35.20 & 73.20 & 29.30 & 75.83 & 58.91 \\
\english$+\ $\cisZh           & 38.40 & 74.40 & 82.00 & 79.60 & 75.20 & 72.00 & 80.00 & 20.40 & 18.40 & 37.20 & 72.40 & 28.60 & 76.51 & 59.09 \\
\english$+\ $\cisMulti & 39.60 & 76.40 & 81.60 & 81.60 & 73.60 & 71.60 & 77.60 & 22.80 & 18.80 & 30.40 & 73.20 & 27.90 & 76.51 & 58.84 \\
\bottomrule


    \end{tabular}
    \caption{Accuracies ($\%$) of CIS modes across $11$ languages of the \mgsm dataset. AVG represents the average accuracy of the language set (LRLs, HRLs or All languages). The \underline{underlined languages} in the table header are \underline{LRLs}, otherwise HRLs. The subscript indicates the performance \textcolor{ForestGreen}{increase$\uparrow$} (or \textcolor{OrangeRed}{decrease$\downarrow$}) of all other modes compared to the \english$+\ $\cisEn  mode.}
    % \vspace{-.2cm}
    \label{tab:cis:mgsm}
\end{table*}




\begin{table*}[!htbp]
    \setlength{\tabcolsep}{3pt}
    \scriptsize
    \centering
    \alternaterowcolors
\begin{tabular}{l|cccccccccccc|lll}
\toprule
\textbf{\xcopa \cis} &
  \multicolumn{1}{c}{\textbf{en}} &
  \multicolumn{1}{c}{{\textbf{\underline{et}}}} &
  \multicolumn{1}{c}{{\textbf{\underline{ht}}}} &
  \multicolumn{1}{c}{\textbf{id}} &
  \multicolumn{1}{c}{\textbf{it}} &
  \multicolumn{1}{c}{{\textbf{\underline{qu}}}} &
  \multicolumn{1}{c}{{\textbf{\underline{sw}}}} &
  \multicolumn{1}{c}{{\textbf{\underline{ta}}}} &
  \multicolumn{1}{c}{{\textbf{\underline{th}}}} &
  \multicolumn{1}{c}{\textbf{tr}} &
  \multicolumn{1}{c}{\textbf{\underline{vi}}} &
  \textbf{zh} &
  \textbf{\underline{LRL AVG}} &
  \textbf{HRL AVG} &
  \textbf{ALL AVG} \\
\midrule

\multicolumn{16}{l}{\textbf{\llamaThree}}                                                                                                             \\
\english$+\ $\cisEn           & 95.40 & 56.60 & 16.00 & 80.20 & 84.80 & 18.60 & 39.00 & 58.40 & 69.80 & 70.80 & 80.20 & 87.40 & 48.37 & 83.72 & 63.10 \\
\english$+\ $\cisFr           & 95.40 & 57.00 & 47.00 & 80.40 & 85.80 & 42.20 & 52.40 & 57.80 & 72.40 & 74.20 & 80.60 & 86.60 & 58.49 & 84.48 & 69.32 \\
\english$+\ $\cisJa          & 95.00 & 57.20 & 39.60 & 82.20 & 86.60 & 46.40 & 56.00 & 57.00 & 73.20 & 74.00 & 81.40 & 85.60 & 58.69 & 84.68 & 69.52 \\
\english$+\ $\cisZh           & 95.00 & 58.40 & 35.80 & 82.00 & 86.00 & 43.00 & 56.80 & 57.00 & 73.80 & 73.60 & 81.40 & 87.60 & 58.03 & 84.84 & 69.20 \\
\english$+\ $\cisMulti & 95.60 & 58.40 & 51.00 & 81.60 & 86.80 & 49.00 & 59.20 & 56.40 & 73.60 & 74.60 & 80.40 & 88.00 & 61.14 & 85.32 & 71.22 \\
\midrule



\multicolumn{16}{l}{\textbf{\llamaThreeOne}}                                                                                                          \\
\english$+\ $\cisEn           & 96.00 & 64.00 & 25.00 & 84.80 & 88.60 & 27.60 & 51.60 & 62.20 & 72.80 & 76.80 & 85.00 & 90.00 & 55.46 & 87.24 & 68.70 \\
\english$+\ $\cisFr           & 95.20 & 65.80 & 23.00 & 85.00 & 88.20 & 42.80 & 53.20 & 69.80 & 76.00 & 78.00 & 85.20 & 89.20 & 59.40 & 87.12 & 70.95 \\
\english$+\ $\cisJa          & 96.40 & 65.80 & 53.60 & 86.60 & 89.00 & 46.60 & 61.40 & 68.60 & 77.00 & 79.20 & 86.60 & 88.40 & 65.66 & 87.92 & 74.93 \\
\english$+\ $\cisZh           & 95.80 & 65.40 & 52.00 & 86.80 & 88.80 & 47.40 & 58.40 & 68.40 & 76.00 & 78.40 & 85.00 & 89.00 & 64.66 & 87.76 & 74.28 \\
\english$+\ $\cisMulti & 96.40 & 63.20 & 51.20 & 86.80 & 89.80 & 48.80 & 59.80 & 69.20 & 75.60 & 80.80 & 85.40 & 89.40 & 64.74 & 88.64 & 74.70 \\
\midrule

\multicolumn{16}{l}{\textbf{\qwenTwo}}                                                                                                                \\
\english$+\ $\cisEn           & 97.00 & 62.00 & 51.40 & 87.80 & 90.20 & 51.00 & 52.40 & 56.40 & 79.00 & 76.20 & 84.00 & 90.20 & 62.31 & 88.28 & 73.13 \\
\english$+\ $\cisFr           & 97.00 & 62.60 & 49.80 & 87.20 & 90.20 & 50.20 & 54.00 & 57.20 & 79.20 & 74.40 & 83.20 & 90.60 & 62.31 & 87.88 & 72.97 \\
\english$+\ $\cisJa          & 97.00 & 62.60 & 51.40 & 88.40 & 89.60 & 50.80 & 54.40 & 58.40 & 80.20 & 75.00 & 83.80 & 91.80 & 63.09 & 88.36 & 73.62 \\
\english$+\ $\cisZh           & 97.20 & 60.80 & 51.00 & 86.80 & 88.80 & 50.60 & 53.60 & 57.80 & 80.20 & 76.20 & 83.40 & 92.80 & 62.49 & 88.36 & 73.27 \\
\english$+\ $\cisMulti & 97.20 & 62.20 & 52.20 & 87.80 & 90.80 & 51.80 & 52.20 & 56.80 & 78.80 & 74.60 & 83.60 & 90.40 & 62.51 & 88.16 & 73.20 \\
\midrule


\multicolumn{16}{l}{\textbf{\qwenTwoFive}}                                                                                                            \\
\english$+\ $\cisEn           & 97.00 & 63.00 & 56.60 & 89.40 & 91.00 & 50.20 & 52.40 & 56.00 & 83.40 & 78.20 & 87.80 & 94.20 & 64.20 & 89.96 & 74.93 \\
\english$+\ $\cisFr           & 96.40 & 63.60 & 58.80 & 90.00 & 93.00 & 49.00 & 51.00 & 56.80 & 82.20 & 79.20 & 87.40 & 94.00 & 64.11 & 90.52 & 75.12 \\
\english$+\ $\cisJa          & 96.60 & 63.60 & 60.00 & 90.00 & 92.60 & 50.00 & 48.80 & 56.00 & 82.80 & 78.80 & 88.20 & 93.80 & 64.20 & 90.36 & 75.10 \\
\english$+\ $\cisZh           & 97.20 & 64.00 & 59.80 & 89.80 & 92.60 & 50.20 & 52.00 & 58.00 & 84.00 & 79.20 & 88.40 & 93.60 & 65.20 & 90.48 & 75.73 \\
\english$+\ $\cisMulti & 96.60 & 64.60 & 61.20 & 90.20 & 93.00 & 52.00 & 48.40 & 59.20 & 82.20 & 80.00 & 87.80 & 93.60 & 65.06 & 90.68 & 75.73 \\
\midrule


\multicolumn{16}{l}{\textbf{\mistral}}                                                                                                                \\
\english$+\ $\cisEn           & 96.40 & 54.40 & 55.20 & 79.80 & 90.80 & 49.60 & 54.20 & 69.80 & 63.80 & 70.40 & 80.00 & 90.00 & 61.00 & 85.48 & 71.20 \\
\english$+\ $\cisFr           & 96.00 & 55.80 & 57.40 & 80.20 & 90.00 & 49.40 & 53.80 & 71.20 & 61.20 & 69.60 & 79.80 & 88.60 & 61.23 & 84.88 & 71.08 \\
\english$+\ $\cisJa          & 95.60 & 58.40 & 54.60 & 82.40 & 91.20 & 51.20 & 54.40 & 72.00 & 61.40 & 69.20 & 80.60 & 88.20 & 61.80 & 85.32 & 71.60 \\
\english$+\ $\cisZh           & 95.60 & 56.40 & 53.20 & 82.40 & 91.40 & 50.40 & 54.80 & 72.20 & 61.00 & 71.00 & 80.00 & 90.00 & 61.14 & 86.08 & 71.53 \\
\english$+\ $\cisMulti & 96.20 & 57.20 & 56.20 & 83.40 & 92.20 & 50.60 & 55.40 & 73.40 & 61.80 & 72.80 & 81.00 & 90.20 & 62.23 & 86.96 & 72.53 \\
\midrule


\multicolumn{16}{l}{\textbf{\aya}}                                                                                                                    \\
\english$+\ $\cisEn           & 93.80 & 16.40 & 8.60  & 82.80 & 85.20 & 1.80  & 3.20  & 40.00 & 17.60 & 81.40 & 74.80 & 74.80 & 23.20 & 83.60 & 48.37 \\
\english$+\ $\cisFr           & 94.20 & 27.80 & 13.00 & 84.00 & 87.80 & 0.60  & 5.00  & 47.00 & 28.20 & 81.60 & 79.40 & 79.40 & 28.71 & 85.40 & 52.33 \\
\english$+\ $\cisJa          & 94.00 & 31.60 & 12.80 & 85.00 & 87.40 & 0.80  & 5.40  & 62.00 & 40.60 & 81.40 & 78.20 & 83.60 & 33.06 & 86.28 & 55.23 \\
\english$+\ $\cisZh           & 94.80 & 28.60 & 12.00 & 84.80 & 87.20 & 0.40  & 5.40  & 43.60 & 21.80 & 81.60 & 76.80 & 83.60 & 26.94 & 86.40 & 51.72 \\
\english$+\ $\cisMulti & 94.60 & 36.40 & 18.80 & 86.40 & 89.00 & 0.80  & 12.80 & 49.80 & 37.00 & 82.80 & 79.60 & 83.60 & 33.60 & 87.28 & 55.97 \\
\bottomrule

    \end{tabular}
    \caption{Accuracies ($\%$) of CIS modes across $12$ languages of the \xcopa dataset. AVG represents the average accuracy of the language set (LRLs, HRLs or All languages). The \underline{underlined languages} in the table header are \underline{LRLs}, otherwise HRLs. The subscript indicates the performance \textcolor{ForestGreen}{increase$\uparrow$} (or \textcolor{OrangeRed}{decrease$\downarrow$}) of all other modes compared to the \english$+\ $\cisEn mode.}
    % \vspace{-.2cm}
    \label{tab:cis:xcopa}
\end{table*}

\begin{table*}[!htbp]
    \setlength{\tabcolsep}{2.5pt}
    \scriptsize
    \centering
    \alternaterowcolors
\begin{tabular}{l|ccccccccccccc|lll}
\toprule
\textbf{\xlwic \cis} &
  \multicolumn{1}{c}{\textbf{\underline{bg}}} &
  \multicolumn{1}{c}{\textbf{da}} &
  \multicolumn{1}{c}{\textbf{de}} &
  \multicolumn{1}{c}{\textbf{en}} &
  \multicolumn{1}{c}{\textbf{\underline{et}}} &
  \multicolumn{1}{c}{\textbf{\underline{fa}}} &
  \multicolumn{1}{c}{\textbf{fr}} &
  \multicolumn{1}{c}{\textbf{\underline{hr}}} &
  \multicolumn{1}{c}{\textbf{it}} &
  \multicolumn{1}{c}{\textbf{ja}} &
  \multicolumn{1}{c}{\textbf{ko}} &
  \multicolumn{1}{c}{\textbf{nl}} &
  \textbf{zh} &
  \textbf{\underline{LRL AVG}} &
  \textbf{HRL AVG} &
  \textbf{ALL AVG} \\
  \midrule

\multicolumn{17}{l}{\textbf{\llamaThree}}                                                                                                                             \\
\english$+\ $\cisEn           & 55.13 & 65.90 & 61.54 & 65.64 & 53.85 & 64.36 & 59.23 & 53.59 & 52.05 & 52.82 & 54.62 & 56.92 & 63.59 & 56.73 & 59.15 & 58.40 \\
\english$+\ $\cisFr           & 56.15 & 67.18 & 63.08 & 64.62 & 54.62 & 65.13 & 60.00 & 54.10 & 53.85 & 52.82 & 55.64 & 60.00 & 60.26 & 57.50 & 59.72 & 59.03 \\
\english$+\ $\cisJa          & 56.67 & 67.95 & 62.05 & 64.87 & 54.10 & 66.67 & 60.77 & 55.13 & 56.67 & 53.85 & 53.85 & 59.23 & 62.05 & 58.14 & 60.14 & 59.53 \\
\english$+\ $\cisZh           & 57.44 & 68.72 & 63.33 & 66.15 & 54.36 & 65.64 & 58.97 & 55.64 & 54.10 & 53.33 & 54.87 & 60.26 & 58.97 & 58.27 & 59.86 & 59.37 \\
\english$+\ $\cisMulti & 56.67 & 65.90 & 62.31 & 64.36 & 54.62 & 65.90 & 60.77 & 54.10 & 54.62 & 51.79 & 52.31 & 59.49 & 61.03 & 57.82 & 59.17 & 58.76 \\
\midrule


\multicolumn{17}{l}{\textbf{\llamaThreeOne}}                                                                                                                          \\
\english$+\ $\cisEn           & 53.08 & 52.05 & 62.05 & 64.62 & 41.79 & 44.36 & 54.36 & 52.31 & 31.03 & 48.46 & 52.56 & 54.36 & 57.95 & 47.88 & 53.05 & 51.46 \\
\english$+\ $\cisFr           & 55.13 & 58.97 & 63.08 & 64.87 & 53.08 & 47.95 & 60.77 & 55.13 & 53.59 & 50.77 & 55.38 & 60.51 & 58.97 & 52.82 & 58.55 & 56.79 \\
\english$+\ $\cisJa          & 55.13 & 60.77 & 64.87 & 65.13 & 53.33 & 61.79 & 59.74 & 53.59 & 54.62 & 50.77 & 53.85 & 62.05 & 54.87 & 55.96 & 58.52 & 57.73 \\
\english$+\ $\cisZh           & 54.62 & 58.97 & 64.36 & 65.64 & 53.59 & 58.46 & 61.28 & 52.05 & 55.13 & 51.28 & 54.36 & 61.28 & 57.95 & 54.68 & 58.92 & 57.61 \\
\english$+\ $\cisMulti & 55.38 & 62.56 & 62.56 & 64.62 & 52.31 & 61.03 & 61.28 & 55.38 & 52.82 & 51.79 & 51.03 & 64.10 & 58.46 & 56.03 & 58.80 & 57.95 \\
\midrule


\multicolumn{17}{l}{\textbf{\qwenTwo}}                                                                                                                                \\
\english$+\ $\cisEn           & 39.74 & 40.26 & 61.54 & 66.15 & 55.38 & 54.36 & 61.28 & 54.10 & 37.95 & 8.21  & 13.85 & 55.90 & 0.26  & 50.90 & 38.38 & 42.23 \\
\english$+\ $\cisFr           & 53.33 & 43.59 & 63.85 & 65.90 & 52.05 & 55.64 & 62.05 & 53.59 & 40.26 & 18.21 & 15.90 & 56.41 & 3.33  & 53.65 & 41.05 & 44.93 \\
\english$+\ $\cisJa          & 56.41 & 50.77 & 62.05 & 65.64 & 53.33 & 61.28 & 60.26 & 53.85 & 43.85 & 48.46 & 34.10 & 60.26 & 15.13 & 56.22 & 48.95 & 51.18 \\
\english$+\ $\cisZh           & 56.41 & 50.00 & 62.56 & 65.38 & 54.87 & 61.79 & 61.03 & 54.10 & 43.59 & 37.69 & 30.77 & 59.49 & 14.10 & 56.79 & 47.18 & 50.14 \\
\english$+\ $\cisMulti & 55.13 & 46.67 & 62.82 & 64.10 & 52.05 & 59.49 & 61.03 & 53.08 & 40.26 & 26.67 & 20.00 & 55.64 & 6.92  & 54.94 & 42.68 & 46.45 \\
\midrule


\multicolumn{17}{l}{\textbf{\qwenTwoFive}}                                                                                                                            \\
\english$+\ $\cisEn           & 59.49 & 61.79 & 72.05 & 73.33 & 57.44 & 60.51 & 66.67 & 58.72 & 60.77 & 52.56 & 64.87 & 74.62 & 63.59 & 59.04 & 65.58 & 63.57 \\
\english$+\ $\cisFr           & 55.90 & 62.56 & 71.03 & 72.05 & 57.18 & 62.31 & 64.36 & 60.77 & 59.49 & 57.95 & 66.67 & 71.28 & 64.10 & 59.04 & 65.50 & 63.51 \\
\english$+\ $\cisJa          & 57.95 & 62.31 & 72.05 & 71.54 & 58.46 & 61.28 & 64.10 & 59.74 & 58.72 & 65.13 & 64.62 & 73.59 & 65.64 & 59.36 & 66.41 & 64.24 \\
\english$+\ $\cisZh           & 56.92 & 61.54 & 72.56 & 71.28 & 57.44 & 61.54 & 64.36 & 61.03 & 58.72 & 65.13 & 65.13 & 71.79 & 65.64 & 59.23 & 66.24 & 64.08 \\
\english$+\ $\cisMulti & 57.18 & 62.31 & 71.03 & 72.82 & 57.95 & 63.59 & 65.13 & 58.72 & 57.95 & 64.10 & 64.87 & 73.08 & 65.90 & 59.36 & 66.35 & 64.20 \\
\midrule


\multicolumn{17}{l}{\textbf{\mistral}}                                                                                                                                \\
\english$+\ $\cisEn           & 48.97 & 62.56 & 69.74 & 66.67 & 51.79 & 47.95 & 58.21 & 49.74 & 34.10 & 38.46 & 52.56 & 67.44 & 53.85 & 49.62 & 55.95 & 54.00 \\
\english$+\ $\cisFr           & 48.72 & 62.31 & 69.74 & 65.90 & 51.79 & 48.97 & 59.23 & 52.82 & 42.56 & 52.82 & 51.79 & 62.82 & 55.13 & 50.58 & 58.03 & 55.74 \\
\english$+\ $\cisJa          & 48.21 & 62.05 & 71.28 & 66.15 & 51.03 & 47.95 & 58.72 & 52.31 & 37.18 & 57.69 & 53.85 & 65.13 & 54.62 & 49.87 & 58.52 & 55.86 \\
\english$+\ $\cisZh           & 48.46 & 60.00 & 69.49 & 65.64 & 51.54 & 49.74 & 58.46 & 50.51 & 37.18 & 54.36 & 53.85 & 65.13 & 57.18 & 50.06 & 57.92 & 55.50 \\
\english$+\ $\cisMulti & 48.46 & 58.97 & 69.74 & 66.15 & 52.31 & 47.95 & 61.28 & 52.05 & 39.23 & 54.10 & 54.36 & 65.90 & 56.15 & 50.19 & 58.43 & 55.90 \\
\midrule


\multicolumn{17}{l}{\textbf{\aya}}                                                                                                                                    \\
\english$+\ $\cisEn           & 52.31 & 55.64 & 57.44 & 65.13 & 56.92 & 64.62 & 54.10 & 57.44 & 19.49 & 49.74 & 54.62 & 55.38 & 57.95 & 57.82 & 52.17 & 53.91 \\
\english$+\ $\cisFr           & 56.92 & 60.00 & 64.10 & 65.13 & 59.49 & 73.59 & 59.74 & 56.15 & 38.97 & 63.33 & 64.62 & 63.59 & 59.74 & 61.54 & 59.91 & 60.41 \\
\english$+\ $\cisJa          & 55.90 & 58.21 & 62.82 & 65.38 & 61.28 & 71.54 & 54.62 & 55.38 & 28.21 & 64.36 & 64.10 & 61.03 & 58.46 & 61.03 & 57.46 & 58.56 \\
\english$+\ $\cisZh           & 56.15 & 57.95 & 63.08 & 65.13 & 60.51 & 70.77 & 54.36 & 54.87 & 29.23 & 63.08 & 64.62 & 61.03 & 60.51 & 60.58 & 57.66 & 58.56 \\
\english$+\ $\cisMulti & 57.95 & 60.77 & 64.36 & 65.64 & 59.49 & 74.36 & 56.41 & 57.18 & 39.49 & 62.31 & 65.64 & 64.87 & 58.21 & 62.24 & 59.74 & 60.51 \\

\bottomrule


    \end{tabular}
    \caption{Accuracies ($\%$) of CIS modes across $13$ languages of the \xlwic dataset. AVG represents the average accuracy of the language set (LRLs, HRLs or All languages). The \underline{underlined languages} in the table header are \underline{LRLs}, otherwise HRLs. The subscript indicates the performance \textcolor{ForestGreen}{increase$\uparrow$} (or \textcolor{OrangeRed}{decrease$\downarrow$}) of all other modes compared to the \english$+\ $\cisEn mode.}
    % \vspace{-.2cm}
    \label{tab:cis:xlwic}
\end{table*}

\begin{table*}[!htbp]
    \setlength{\tabcolsep}{0.5pt}
    \scriptsize
    \centering
    \alternaterowcolors[5]
\begin{tabular}{l|llccccc|llccccc}
\toprule
\multicolumn{1}{l|}{\textbf{McNemar's Test}} & \multicolumn{7}{c|}{\textbf{Low-Resource Languages}} & \multicolumn{7}{c}{\textbf{High-Resource Languages}} \\
\multicolumn{1}{l|}{\textbf{\mgsm \cis Mode}} &
  \multicolumn{1}{c|}{\textbf{$\chi^2$}} &
  \multicolumn{1}{c|}{\textbf{$p$-value}} &
  \multicolumn{1}{c|}{\textbf{Sig.}} &
  \multicolumn{1}{c|}{\textbf{\#Both}} &
  \multicolumn{1}{c|}{\textbf{\#M1 Wrong}} &
  \multicolumn{1}{c|}{\textbf{\#M1 Correct}} &
  \multicolumn{1}{c|}{\textbf{\#Both}} &
  \multicolumn{1}{c|}{\textbf{$\chi^2$}} &
  \multicolumn{1}{c|}{\textbf{$p$-value}} &
  \multicolumn{1}{c|}{\textbf{Sig.}} &
  \multicolumn{1}{c|}{\textbf{\#Both}} &
  \multicolumn{1}{c|}{\textbf{\#M1 Wrong}} &
  \multicolumn{1}{c|}{\textbf{\#M1 Correct}} &
  \multicolumn{1}{c}{\textbf{\#Both}} \\
\multicolumn{1}{l|}{\textbf{M1 vs M2 Comparison}} &
  \multicolumn{1}{c|}{\textbf{}} &
  \multicolumn{1}{c|}{\textbf{}} &
  \multicolumn{1}{c|}{\textbf{Level}} &
  \multicolumn{1}{c|}{\textbf{Wrong}} &
  \multicolumn{1}{c|}{\textbf{M2 Correct}} &
  \multicolumn{1}{c|}{\textbf{M2 Wrong}} &
  \multicolumn{1}{c|}{\textbf{Correct}} &
  \multicolumn{1}{c|}{\textbf{}} &
  \multicolumn{1}{c|}{\textbf{}} &
  \multicolumn{1}{c|}{\textbf{Level}} &
  \multicolumn{1}{c|}{\textbf{False}} &
  \multicolumn{1}{c|}{\textbf{M2 Correct}} &
  \multicolumn{1}{c|}{\textbf{M2 Wrong}} &
  \multicolumn{1}{c}{\textbf{Correct}} \\
  \midrule

\multicolumn{15}{l}{\textbf{\llamaThree}}                                                                                                                         \\
\tiny{\english$+\ $\cisEn vs   \english$+\ $\cisZh}            & 1.57  & 2.10E-01 &     & 317 & 73  & 90  & 520 & 0.47  & 4.93E-01 &     & 339 & 91  & 81  & 1239 \\
\tiny{\english$+\ $\cisEn vs   \english$+\ $\cisFr}            & 0.16  & 6.85E-01 &     & 311 & 79  & 73  & 537 & 0.60  & 4.39E-01 &     & 323 & 107 & 95  & 1225 \\
\tiny{\english$+\ $\cisEn vs   \english$+\ $\cisJa}            & 0.03  & 8.72E-01 &     & 311 & 79  & 76  & 534 & 0.01  & 9.40E-01 &     & 343 & 87  & 89  & 1231 \\
\tiny{\english$+\ $\cisEn vs   \english$+\ $\cisMulti}         & 0.08  & 7.72E-01 &     & 297 & 93  & 98  & 512 & 0.01  & 9.41E-01 &     & 341 & 89  & 91  & 1229 \\
\tiny{\multilingual$+\   $\cisMulti vs \english$+\ $\cisMulti} & 0.06  & 8.00E-01 &     & 327 & 72  & 68  & 533 & 0.16  & 6.93E-01 &     & 355 & 83  & 77  & 1235 \\
\midrule 

\multicolumn{15}{l}{\textbf{\llamaThreeOne}}                                                                                                                      \\
\tiny{\english$+\ $\cisEn vs   \english$+\ $\cisZh}            & 0.27  & 6.02E-01 &     & 328 & 113 & 122 & 437 & 5.35  & 2.08E-02 & *   & 296 & 147 & 109 & 1198 \\
\tiny{\english$+\ $\cisEn vs   \english$+\ $\cisFr}            & 5.90  & 1.51E-02 & *   & 344 & 97  & 135 & 424 & 10.48 & 1.21E-03 & **  & 282 & 161 & 107 & 1200 \\
\tiny{\english$+\ $\cisEn vs   \english$+\ $\cisJa}            & 3.79  & 5.16E-02 &     & 323 & 118 & 89  & 470 & 7.69  & 5.54E-03 & **  & 282 & 161 & 114 & 1193 \\
\tiny{\english$+\ $\cisEn vs   \english$+\ $\cisMulti}         & 19.20 & 1.17E-05 & *** & 298 & 143 & 77  & 482 & 18.31 & 1.88E-05 & *** & 278 & 165 & 95  & 1212 \\
\tiny{\multilingual$+\   $\cisMulti vs \english$+\ $\cisMulti} & 18.27 & 1.91E-05 & *** & 246 & 68 & 129 & 557 & 0.07 & 7.97E-01 &  & 250 & 118 & 123 & 1259 \\
\midrule 

\multicolumn{15}{l}{\textbf{\qwenTwo}}                                                                                                                            \\
\tiny{\english$+\ $\cisEn vs   \english$+\ $\cisZh}            & 0.03  & 8.69E-01 &     & 498 & 72  & 75  & 355 & 0.50  & 4.80E-01 &     & 288 & 76  & 86  & 1300 \\
\tiny{\english$+\ $\cisEn vs   \english$+\ $\cisFr}            & 0.12  & 7.27E-01 &     & 502 & 68  & 63  & 367 & 0.06  & 8.06E-01 &     & 291 & 73  & 77  & 1309 \\
\tiny{\english$+\ $\cisEn vs   \english$+\ $\cisJa}            & 0.37  & 5.42E-01 &     & 500 & 70  & 62  & 368 & 0.96  & 3.27E-01 &     & 283 & 81  & 95  & 1291 \\
\tiny{\english$+\ $\cisEn vs   \english$+\ $\cisMulti}         & 0.07  & 7.92E-01 &     & 507 & 63  & 67  & 363 & 0.01  & 9.37E-01 &     & 284 & 80  & 78  & 1308 \\
\tiny{\multilingual$+\   $\cisMulti vs \english$+\ $\cisMulti} & 12.23 & 4.70E-04 & *** & 464 & 63  & 110 & 363 & 0.35  & 5.54E-01 &     & 266 & 87  & 96  & 1301 \\
\midrule 

\multicolumn{15}{l}{\textbf{\qwenTwoFive}}                                                                                                                        \\
\tiny{\english$+\ $\cisEn vs   \english$+\ $\cisZh}            & 2.06  & 1.51E-01 &     & 351 & 55  & 40  & 554 & 0.86  & 3.53E-01 &     & 177 & 52  & 42  & 1479 \\
\tiny{\english$+\ $\cisEn vs   \english$+\ $\cisFr}            & 0.01  & 9.23E-01 &     & 352 & 54  & 54  & 540 & 1.41  & 2.36E-01 &     & 192 & 37  & 49  & 1472 \\
\tiny{\english$+\ $\cisEn vs   \english$+\ $\cisJa}            & 0.34  & 5.62E-01 &     & 349 & 57  & 50  & 544 & 0.97  & 3.24E-01 &     & 183 & 46  & 57  & 1464 \\
\tiny{\english$+\ $\cisEn vs   \english$+\ $\cisMulti}         & 0.00  & 1.00E+00 &     & 347 & 59  & 58  & 536 & 0.27  & 6.06E-01 &     & 185 & 44  & 50  & 1471 \\
\tiny{\multilingual$+\   $\cisMulti vs \english$+\ $\cisMulti} & 0.03  & 8.55E-01 &     & 347 & 61  & 58  & 534 & 0.01  & 9.28E-01 &     & 174 & 63  & 61  & 1452 \\
\midrule 

\multicolumn{15}{l}{\textbf{\mistral}}                                                                                                                            \\
\tiny{\english$+\ $\cisEn vs   \english$+\ $\cisZh}            & 0.01  & 9.42E-01 &     & 300 & 93  & 95  & 512 & 1.30  & 2.55E-01 &     & 247 & 83  & 68  & 1352 \\
\tiny{\english$+\ $\cisEn vs   \english$+\ $\cisFr}            & 0.20  & 6.57E-01 &     & 298 & 95  & 88  & 519 & 1.46  & 2.26E-01 &     & 234 & 96  & 79  & 1341 \\
\tiny{\english$+\ $\cisEn vs   \english$+\ $\cisJa}            & 0.08  & 7.80E-01 &     & 293 & 100 & 105 & 502 & 1.66  & 1.97E-01 &     & 234 & 96  & 78  & 1342 \\
\tiny{\english$+\ $\cisEn vs   \english$+\ $\cisMulti}         & 9.66  & 1.88E-03 & **  & 285 & 108 & 66  & 541 & 5.96  & 1.46E-02 & *   & 239 & 91  & 60  & 1360 \\
\tiny{\multilingual$+\   $\cisMulti vs \english$+\ $\cisMulti} & 2.31  & 1.28E-01 &     & 283 & 88  & 68  & 561 & 12.66 & 3.74E-04 & *** & 242 & 103 & 57  & 1348 \\
\midrule 

\multicolumn{15}{l}{\textbf{\aya}}                                                                                                                                \\
\tiny{\english$+\ $\cisEn vs   \english$+\ $\cisZh}            & 0.34  & 5.60E-01 &     & 646 & 76  & 68  & 210 & 0.02  & 8.78E-01 &     & 327 & 87  & 84  & 1252 \\
\tiny{\english$+\ $\cisEn vs   \english$+\ $\cisFr}            & 0.47  & 4.91E-01 &     & 650 & 72  & 63  & 215 & 0.56  & 4.55E-01 &     & 330 & 84  & 95  & 1241 \\
\tiny{\english$+\ $\cisEn vs   \english$+\ $\cisJa}            & 1.22  & 2.70E-01 &     & 634 & 88  & 73  & 205 & 0.35  & 5.52E-01 &     & 328 & 86  & 95  & 1241 \\
\tiny{\english$+\ $\cisEn vs   \english$+\ $\cisMulti}         & 0.00  & 1.00E+00 &     & 662 & 60  & 59  & 219 & 0.02  & 8.77E-01 &     & 329 & 85  & 82  & 1254 \\
\tiny{\multilingual$+\   $\cisMulti vs \english$+\ $\cisMulti} & 6.88  & 8.71E-03 & **  & 614 & 71  & 107 & 208 & 0.68  & 4.09E-01 &     & 294 & 131 & 117 & 1208 \\

\bottomrule
\end{tabular}

    \caption{McNemar's test results of ICL modes on LRL and HRL splits of \mgsm dataset across $6$ MLLMs we use.}
    % \vspace{-.2cm}
    \label{tab:hyp_test:cis:mgsm}
\end{table*}

\begin{table*}[!htbp]
    \setlength{\tabcolsep}{0.5pt}
    \scriptsize
    \centering
    \alternaterowcolors[5]
\begin{tabular}{l|llccccc|llccccc}
\toprule
\multicolumn{1}{l|}{\textbf{McNemar's Test}} & \multicolumn{7}{c|}{\textbf{Low-Resource Languages}} & \multicolumn{7}{c}{\textbf{High-Resource Languages}} \\
\multicolumn{1}{l|}{\textbf{\xcopa \cis Mode}} &
  \multicolumn{1}{c|}{\textbf{$\chi^2$}} &
  \multicolumn{1}{c|}{\textbf{$p$-value}} &
  \multicolumn{1}{c|}{\textbf{Sig.}} &
  \multicolumn{1}{c|}{\textbf{\#Both}} &
  \multicolumn{1}{c|}{\textbf{\#M1 Wrong}} &
  \multicolumn{1}{c|}{\textbf{\#M1 Correct}} &
  \multicolumn{1}{c|}{\textbf{\#Both}} &
  \multicolumn{1}{c|}{\textbf{$\chi^2$}} &
  \multicolumn{1}{c|}{\textbf{$p$-value}} &
  \multicolumn{1}{c|}{\textbf{Sig.}} &
  \multicolumn{1}{c|}{\textbf{\#Both}} &
  \multicolumn{1}{c|}{\textbf{\#M1 Wrong}} &
  \multicolumn{1}{c|}{\textbf{\#M1 Correct}} &
  \multicolumn{1}{c}{\textbf{\#Both}} \\
\multicolumn{1}{l|}{\textbf{M1 vs M2 Comparison}} &
  \multicolumn{1}{c|}{\textbf{}} &
  \multicolumn{1}{c|}{\textbf{}} &
  \multicolumn{1}{c|}{\textbf{Level}} &
  \multicolumn{1}{c|}{\textbf{Wrong}} &
  \multicolumn{1}{c|}{\textbf{M2 Correct}} &
  \multicolumn{1}{c|}{\textbf{M2 Wrong}} &
  \multicolumn{1}{c|}{\textbf{Correct}} &
  \multicolumn{1}{c|}{\textbf{}} &
  \multicolumn{1}{c|}{\textbf{}} &
  \multicolumn{1}{c|}{\textbf{Level}} &
  \multicolumn{1}{c|}{\textbf{False}} &
  \multicolumn{1}{c|}{\textbf{M2 Correct}} &
  \multicolumn{1}{c|}{\textbf{M2 Wrong}} &
  \multicolumn{1}{c}{\textbf{Correct}} \\
  \midrule

\multicolumn{15}{l}{\textbf{\llamaThree}}                                                                                                                           \\
\tiny{\english$+\ $\cisEn vs   \english$+\ $\cisZh}            & 199.24 & 3.05E-45 & *** & 1353 & 454 & 116 & 1577 & 8.28  & 4.00E-03 & **  & 349 & 58  & 30 & 2063 \\
\tiny{\english$+\ $\cisEn vs   \english$+\ $\cisFr}            & 214.84 & 1.21E-48 & *** & 1340 & 467 & 113 & 1580 & 3.34  & 6.76E-02 &     & 349 & 58  & 39 & 2054 \\
\tiny{\english$+\ $\cisEn vs   \english$+\ $\cisJa}            & 209.37 & 1.89E-47 & *** & 1317 & 490 & 129 & 1564 & 4.90  & 2.69E-02 & *   & 341 & 66  & 42 & 2051 \\
\tiny{\english$+\ $\cisEn vs   \english$+\ $\cisMulti}         & 278.98 & 1.25E-62 & *** & 1227 & 580 & 133 & 1560 & 13.34 & 2.60E-04 & *** & 330 & 77 & 37 & 2056 \\
\tiny{\multilingual$+\   $\cisMulti vs \english$+\ $\cisMulti} & 0.33   & 5.64E-01 &     & 1099 & 247 & 261 & 1893 & 0.65  & 4.19E-01 &     & 285 & 71  & 82 & 2062 \\
\midrule

\multicolumn{15}{l}{\textbf{\llamaThreeOne}}                                                                                                                        \\
\tiny{\english$+\ $\cisEn vs   \english$+\ $\cisZh}            & 194.42 & 3.45E-44 & *** & 1133 & 426 & 104 & 1837 & 1.55  & 2.13E-01 &     & 266 & 53  & 40 & 2141 \\
\tiny{\english$+\ $\cisEn vs   \english$+\ $\cisFr}            & 39.60  & 3.12E-10 & *** & 1253 & 306 & 168 & 1773 & 0.04  & 8.45E-01 &     & 268 & 51  & 54 & 2127 \\
\tiny{\english$+\ $\cisEn vs   \english$+\ $\cisJa}            & 206.75 & 7.04E-47 & *** & 1074 & 485 & 128 & 1813 & 2.44  & 1.18E-01 &     & 258 & 61  & 44 & 2137 \\
\tiny{\english$+\ $\cisEn vs   \english$+\ $\cisMulti}         & 164.28 & 1.31E-37 & *** & 1077 & 482 & 157 & 1784 & 10.41 & 1.25E-03 & **  & 246 & 73  & 38 & 2143 \\
\tiny{\multilingual$+\   $\cisMulti vs \english$+\ $\cisMulti} & 4.95   & 2.61E-02 & *   & 956  & 227 & 278 & 2039 & 0.06  & 8.13E-01 &     & 206 & 82  & 78 & 2134 \\
\midrule

\multicolumn{15}{l}{\textbf{\qwenTwo}}                                                                                                                              \\
\tiny{\english$+\ $\cisEn vs   \english$+\ $\cisZh}            & 0.13   & 7.15E-01 &     & 1222 & 97  & 91  & 2090 & 0.01  & 9.07E-01 &     & 255 & 38  & 36 & 2171 \\
\tiny{\english$+\ $\cisEn vs   \english$+\ $\cisFr}            & 0.00   & 9.48E-01 &     & 1200 & 119 & 119 & 2062 & 1.35  & 2.45E-01 &     & 268 & 25  & 35 & 2172 \\
\tiny{\english$+\ $\cisEn vs   \english$+\ $\cisJa}            & 2.78   & 9.53E-02 &     & 1184 & 135 & 108 & 2073 & 0.01  & 9.09E-01 &     & 254 & 39  & 37 & 2170 \\
\tiny{\english$+\ $\cisEn vs   \english$+\ $\cisMulti}         & 0.14   & 7.09E-01 &     & 1186 & 133 & 126 & 2055 & 0.05  & 8.15E-01 &     & 258 & 35  & 38 & 2169 \\
\tiny{\multilingual$+\   $\cisMulti vs \english$+\ $\cisMulti} & 2.96   & 8.55E-02 &     & 1099 & 178 & 213 & 2010 & 9.60  & 1.95E-03 & **  & 210 & 49 & 86 & 2155 \\
\midrule

\multicolumn{15}{l}{\textbf{\qwenTwoFive}}                                                                                                                          \\
\tiny{\english$+\ $\cisEn vs   \english$+\ $\cisZh}            & 4.03   & 4.48E-02 & *   & 1092 & 161 & 126 & 2121 & 2.53  & 1.12E-01 &     & 216 & 35  & 22 & 2227 \\
\tiny{\english$+\ $\cisEn vs   \english$+\ $\cisFr}            & 0.01   & 9.10E-01 &     & 1098 & 155 & 158 & 2089 & 1.84  & 1.75E-01 &     & 198 & 53  & 39 & 2210 \\
\tiny{\english$+\ $\cisEn vs   \english$+\ $\cisJa}            & 0.00   & 9.57E-01 &     & 1078 & 175 & 175 & 2072 & 0.86  & 3.53E-01 &     & 199 & 52  & 42 & 2207 \\
\tiny{\english$+\ $\cisEn vs   \english$+\ $\cisMulti}         & 2.06   & 1.51E-01 &     & 1034 & 219 & 189 & 2058 & 3.14  & 7.63E-02 &     & 196 & 55  & 37 & 2212 \\
\tiny{\multilingual$+\   $\cisMulti vs \english$+\ $\cisMulti} & 2.08   & 1.49E-01 &     & 1008 & 247 & 215 & 2030 & 0.92  & 3.38E-01 &     & 161 & 60  & 72 & 2207 \\
\midrule

\multicolumn{15}{l}{\textbf{\mistral}}                                                                                                                              \\
\tiny{\english$+\ $\cisEn vs   \english$+\ $\cisZh}            & 0.05   & 8.24E-01 &     & 1201 & 164 & 159 & 1976 & 1.87  & 1.72E-01 &     & 303 & 60  & 45 & 2092 \\
\tiny{\english$+\ $\cisEn vs   \english$+\ $\cisFr}            & 0.13   & 7.15E-01 &     & 1177 & 188 & 180 & 1955 & 1.98  & 1.59E-01 &     & 321 & 42  & 57 & 2080 \\
\tiny{\english$+\ $\cisEn vs   \english$+\ $\cisJa}            & 1.78   & 1.82E-01 &     & 1146 & 219 & 191 & 1944 & 0.08  & 7.84E-01 &     & 305 & 58  & 62 & 2075 \\
\tiny{\english$+\ $\cisEn vs   \english$+\ $\cisMulti}         & 4.27   & 3.88E-02 & *   & 1137 & 228 & 185 & 1950 & 10.54 & 1.17E-03 & **  & 283 & 80  & 43 & 2094 \\
\tiny{\multilingual$+\   $\cisMulti vs \english$+\ $\cisMulti} & 0.12   & 7.24E-01 &     & 1069 & 262 & 253 & 1916 & 0.88  & 3.47E-01 &     & 238 & 75  & 88 & 2099 \\
\midrule

\multicolumn{15}{l}{\textbf{\aya}}                                                                                                                                  \\
\tiny{\english$+\ $\cisEn vs   \english$+\ $\cisZh}            & 70.71  & 4.14E-17 & *** & 2503 & 185 & 54  & 758  & 36.62 & 1.43E-09 & *** & 310 & 100 & 30 & 2060 \\
\tiny{\english$+\ $\cisEn vs   \english$+\ $\cisFr}            & 124.96 & 5.19E-29 & *** & 2444 & 244 & 51  & 761  & 15.24 & 9.45E-05 & *** & 324 & 86  & 41 & 2049 \\
\tiny{\english$+\ $\cisEn vs   \english$+\ $\cisJa}            & 275.84 & 6.05E-62 & *** & 2301 & 387 & 42  & 770  & 34.30 & 4.73E-09 & *** & 313 & 97  & 30 & 2060 \\
\tiny{\english$+\ $\cisEn vs   \english$+\ $\cisMulti}         & 298.12 & 8.46E-67 & *** & 2285 & 403 & 39  & 773  & 57.51 & 3.37E-14 & *** & 292 & 118 & 26 & 2064 \\
\tiny{\multilingual$+\   $\cisMulti vs \english$+\ $\cisMulti} & 294.90 & 4.26E-66 & *** & 1795 & 98  & 529 & 1078 & 9.14  & 2.50E-03 & **  & 233 & 49 & 85 & 2133 \\

\bottomrule
\end{tabular}

    \caption{McNemar's test results of ICL modes on LRL and HRL splits of \xcopa dataset across $6$ MLLMs we use.}
    % \vspace{-.2cm}
    \label{tab:hyp_test:cis:xcopa}
\end{table*}

\begin{table*}[!htbp]
    \setlength{\tabcolsep}{0.5pt}
    \scriptsize
    \centering
    \alternaterowcolors[5]
\begin{tabular}{l|llccccc|llccccc}
\toprule
\multicolumn{1}{l|}{\textbf{McNemar's Test}} & \multicolumn{7}{c|}{\textbf{Low-Resource Languages}} & \multicolumn{7}{c}{\textbf{High-Resource Languages}} \\
\multicolumn{1}{l|}{\textbf{\xlwic \cis Mode}} &
  \multicolumn{1}{c|}{\textbf{$\chi^2$}} &
  \multicolumn{1}{c|}{\textbf{$p$-value}} &
  \multicolumn{1}{c|}{\textbf{Sig.}} &
  \multicolumn{1}{c|}{\textbf{\#Both}} &
  \multicolumn{1}{c|}{\textbf{\#M1 Wrong}} &
  \multicolumn{1}{c|}{\textbf{\#M1 Correct}} &
  \multicolumn{1}{c|}{\textbf{\#Both}} &
  \multicolumn{1}{c|}{\textbf{$\chi^2$}} &
  \multicolumn{1}{c|}{\textbf{$p$-value}} &
  \multicolumn{1}{c|}{\textbf{Sig.}} &
  \multicolumn{1}{c|}{\textbf{\#Both}} &
  \multicolumn{1}{c|}{\textbf{\#M1 Wrong}} &
  \multicolumn{1}{c|}{\textbf{\#M1 Correct}} &
  \multicolumn{1}{c}{\textbf{\#Both}} \\
\multicolumn{1}{l|}{\textbf{M1 vs M2 Comparison}} &
  \multicolumn{1}{c|}{\textbf{}} &
  \multicolumn{1}{c|}{\textbf{}} &
  \multicolumn{1}{c|}{\textbf{Level}} &
  \multicolumn{1}{c|}{\textbf{Wrong}} &
  \multicolumn{1}{c|}{\textbf{M2 Correct}} &
  \multicolumn{1}{c|}{\textbf{M2 Wrong}} &
  \multicolumn{1}{c|}{\textbf{Correct}} &
  \multicolumn{1}{c|}{\textbf{}} &
  \multicolumn{1}{c|}{\textbf{}} &
  \multicolumn{1}{c|}{\textbf{Level}} &
  \multicolumn{1}{c|}{\textbf{False}} &
  \multicolumn{1}{c|}{\textbf{M2 Correct}} &
  \multicolumn{1}{c|}{\textbf{M2 Wrong}} &
  \multicolumn{1}{c}{\textbf{Correct}} \\
  \midrule

  \multicolumn{15}{l}{\textbf{\llamaThree}}                                                                                                                           \\
\tiny{\english$+\ $\cisEn vs   \english$+\ $\cisZh}            & 3.31  & 6.90E-02 &     & 583 & 92  & 68  & 817 & 1.69   & 1.94E-01 &     & 1251 & 183 & 158 & 1918 \\
\tiny{\english$+\ $\cisEn vs   \english$+\ $\cisFr}            & 0.81  & 3.69E-01 &     & 594 & 81  & 69  & 816 & 1.16   & 2.81E-01 &     & 1269 & 165 & 145 & 1931 \\
\tiny{\english$+\ $\cisEn vs   \english$+\ $\cisJa}            & 2.25  & 1.34E-01 &     & 566 & 109 & 87  & 798 & 2.96   & 8.55E-02 &     & 1221 & 213 & 178 & 1898 \\
\tiny{\english$+\ $\cisEn vs   \english$+\ $\cisMulti}         & 1.29  & 2.57E-01 &     & 567 & 108 & 91  & 794 & 0.00   & 1.00E+00 &     & 1226 & 208 & 207 & 1869 \\
\tiny{\multilingual$+\   $\cisMulti vs \english$+\ $\cisMulti} & 0.13  & 7.20E-01 &     & 464 & 186 & 194 & 716 & 4.03   & 4.46E-02 & *   & 1044 & 334 & 389 & 1743 \\
  \midrule
  
\multicolumn{15}{l}{\textbf{\llamaThreeOne}}                                                                                                                        \\
\tiny{\english$+\ $\cisEn vs   \english$+\ $\cisZh}            & 34.03 & 5.43E-09 & *** & 598 & 215 & 109 & 638 & 67.56  & 2.04E-16 & *** & 1234 & 414 & 208 & 1654 \\
\tiny{\english$+\ $\cisEn vs   \english$+\ $\cisFr}            & 16.94 & 3.86E-05 & *** & 604 & 209 & 132 & 615 & 72.71  & 1.50E-17 & *** & 1298 & 350 & 157 & 1705 \\
\tiny{\english$+\ $\cisEn vs   \english$+\ $\cisJa}            & 47.64 & 5.13E-12 & *** & 586 & 227 & 101 & 646 & 52.87  & 3.56E-13 & *** & 1207 & 441 & 249 & 1613 \\
\tiny{\english$+\ $\cisEn vs   \english$+\ $\cisMulti}         & 45.23 & 1.75E-11 & *** & 574 & 239 & 112 & 635 & 53.72  & 2.31E-13 & *** & 1171 & 477 & 275 & 1587 \\
\tiny{\multilingual$+\   $\cisMulti vs \english$+\ $\cisMulti} & 0.97  & 3.25E-01 &     & 509 & 158 & 177 & 716 & 0.83   & 3.63E-01 &     & 992  & 426 & 454 & 1638 \\
  \midrule
  
\multicolumn{15}{l}{\textbf{\qwenTwo}}                                                                                                                              \\
\tiny{\english$+\ $\cisEn vs   \english$+\ $\cisZh}            & 26.89 & 2.16E-07 & *** & 566 & 200 & 108 & 686 & 200.56 & 1.58E-45 & *** & 1772 & 391 & 82  & 1265 \\
\tiny{\english$+\ $\cisEn vs   \english$+\ $\cisFr}            & 8.28  & 4.00E-03 & **  & 638 & 128 & 85  & 709 & 32.76  & 1.04E-08 & *** & 1984 & 179 & 85  & 1262 \\
\tiny{\english$+\ $\cisEn vs   \english$+\ $\cisJa}            & 19.38 & 1.07E-05 & *** & 551 & 215 & 132 & 662 & 242.30 & 1.24E-54 & *** & 1695 & 468 & 97  & 1250 \\
\tiny{\english$+\ $\cisEn vs   \english$+\ $\cisMulti}         & 14.08 & 1.75E-04 & *** & 598 & 168 & 105 & 689 & 61.98  & 3.46E-15 & *** & 1906 & 257 & 106 & 1241 \\
\tiny{\multilingual$+\   $\cisMulti vs \english$+\ $\cisMulti} & 0.34 & 5.61E-01 &  & 446 & 243 & 257 & 614 & 392.40 & 2.48E-87 & *** & 1089 & 245 & 923 & 1253 \\
  \midrule

  
\multicolumn{15}{l}{\textbf{\qwenTwoFive}}                                                                                                                          \\
\tiny{\english$+\ $\cisEn vs   \english$+\ $\cisZh}            & 0.03  & 8.64E-01 &     & 569 & 70  & 67  & 854 & 1.77   & 1.83E-01 &     & 1060 & 148 & 125 & 2177 \\
\tiny{\english$+\ $\cisEn vs   \english$+\ $\cisFr}            & 0.01  & 9.36E-01 &     & 561 & 78  & 78  & 843 & 0.01   & 9.07E-01 &     & 1063 & 145 & 148 & 2154 \\
\tiny{\english$+\ $\cisEn vs   \english$+\ $\cisJa}            & 0.10  & 7.48E-01 &     & 559 & 80  & 75  & 846 & 2.30   & 1.29E-01 &     & 1023 & 185 & 156 & 2146 \\
\tiny{\english$+\ $\cisEn vs   \english$+\ $\cisMulti}         & 0.09  & 7.70E-01 &     & 543 & 96  & 91  & 830 & 2.01   & 1.57E-01 &     & 1026 & 182 & 155 & 2147 \\
\tiny{\multilingual$+\   $\cisMulti vs \english$+\ $\cisMulti} & 4.75  & 2.92E-02 & *   & 494 & 180 & 140 & 746 & 0.24   & 6.27E-01 &     & 883  & 311 & 298 & 2018 \\
  \midrule

  
\multicolumn{15}{l}{\textbf{\mistral}}                                                                                                                              \\
\tiny{\english$+\ $\cisEn vs   \english$+\ $\cisZh}            & 0.71  & 4.01E-01 &     & 757 & 29  & 22  & 752 & 16.69  & 4.39E-05 & *** & 1373 & 173 & 104 & 1860 \\
\tiny{\english$+\ $\cisEn vs   \english$+\ $\cisFr}            & 4.17  & 4.11E-02 & *   & 755 & 31  & 16  & 758 & 16.67  & 4.45E-05 & *** & 1354 & 192 & 119 & 1845 \\
\tiny{\english$+\ $\cisEn vs   \english$+\ $\cisJa}            & 0.20  & 6.51E-01 &     & 762 & 24  & 20  & 754 & 26.40  & 2.77E-07 & *** & 1351 & 195 & 105 & 1859 \\
\tiny{\english$+\ $\cisEn vs   \english$+\ $\cisMulti}         & 0.93  & 3.36E-01 &     & 747 & 39  & 30  & 744 & 23.18  & 1.47E-06 & *** & 1343 & 203 & 116 & 1848 \\
\tiny{\multilingual$+\   $\cisMulti vs \english$+\ $\cisMulti} & 10.28 & 1.35E-03 & **  & 683 & 54  & 94  & 729 & 9.39   & 2.18E-03 & **  & 1130 & 254 & 329 & 1797 \\
  \midrule

  
\multicolumn{15}{l}{\textbf{\aya}}                                                                                                                                  \\
\tiny{\english$+\ $\cisEn vs   \english$+\ $\cisZh}            & 7.64  & 5.72E-03 & **  & 521 & 137 & 94  & 808 & 102.12 & 5.24E-24 & *** & 1402 & 277 & 84  & 1747 \\
\tiny{\english$+\ $\cisEn vs   \english$+\ $\cisFr}            & 10.90 & 9.60E-04 & *** & 480 & 178 & 120 & 782 & 146.30 & 1.12E-33 & *** & 1292 & 387 & 115 & 1716 \\
\tiny{\english$+\ $\cisEn vs   \english$+\ $\cisJa}            & 9.03  & 2.66E-03 & **  & 500 & 158 & 108 & 794 & 94.02  & 3.12E-22 & *** & 1404 & 275 & 89  & 1742 \\
\tiny{\english$+\ $\cisEn vs   \english$+\ $\cisMulti}         & 14.59 & 1.34E-04 & *** & 465 & 193 & 124 & 778 & 143.32 & 5.01E-33 & *** & 1301 & 378 & 112 & 1719 \\
\tiny{\multilingual$+\   $\cisMulti vs \english$+\ $\cisMulti} & 0.80  & 3.72E-01 &     & 417 & 190 & 172 & 781 & 43.63  & 3.96E-11 & *** & 965  & 270 & 448 & 1827 \\


\bottomrule
\end{tabular}

    \caption{McNemar's test results of ICL modes on LRL and HRL splits of \xlwic dataset across $6$ MLLMs we use.}
    % \vspace{-.2cm}
    \label{tab:hyp_test:cis:xlwic}
\end{table*}
