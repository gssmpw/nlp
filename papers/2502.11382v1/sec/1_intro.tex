
\section{Introduction}
\label{sec:intro}

Imaging systems drive broad applications across numerous fields. However, their practical performance is inherently constrained by spatially nonuniform aberrations. Accurately characterizing it is crucial for achieving high performance in digital photography ~\cite{zhang2019deep,yue2015blind,zamir2022restormer,delbracio2021polyblur,gong2024physics}, industrial inspection~\cite{wu2022integrated}, automotive driving~\cite{tseng2021differentiable}, astronomical observation~\cite{karabal2017deconvolution,guo2024direct} and microscopy~\cite{qiao2024deep,zhao2022sparse}. 

The point spread function (PSF) serves as a mathematical representation of blur. Although accurately modeling the PSF in imaging systems offers significant benefits, achieving both high accuracy and broad applicability remains a significant challenge. Despite the numerous methods proposed for PSF estimation ~\cite{jemec20172d,lin2023learning,liang2021mutual,liaudat2023rethinking,eboli2022fast,chen2021extreme,chen2022computational,qiao2024deep,shih2012image,mosleh2015camera,kee2011modeling}, accurately modeling the PSF requires a detailed characterization of the imaging system, which often involves simulating complex compound lenses~\cite{shih2012image,zhou2024optical,chen2021optical,chen2022computational} based on lens design files. Additionally, many of these models are tailored to specific imaging systems~\cite{zhou2024optical,chen2022computational}, limiting their generalization to other setups. This raises an important question: \emph{Is it possible to achieve universal and accurate PSF estimation through a simple calibration, similar to how camera noise calibration is handled in the industry?}

In this work, we propose a physics-informed PSF learning framework for imaging systems, which consists of a simple calibration step followed by a learning process. Our approach is designed to provide both broad applicability and high accuracy.

We propose a novel wavefront-based PSF model that effectively represents the PSF of imaging systems without prior knowledge of lens parameters, making it \emph{applicable to a wide range of imaging systems}. Additionally, we design a learning scheme targeting the spatial frequency response (SFR) measurement at the image plane. To improve estimation accuracy, we structure the basis of our PSF model so that each basis influences only a single SFR direction, allowing for a more accurate fit to diverse SFR measurements. Using curriculum learning~\cite{bengio2009curriculum}, we progressively learn the PSF outward from center to edge. Our learning scheme accelerates convergence with lower loss, resulting in \emph{high accuracy}.
% This integration of basis design and learning strategy 

Our PSF estimation framework achieves superior accuracy, outperforming existing methods, as demonstrated in \cref{fig:teaser}. To validate our approach, we compare it with recent PSF estimation methods (Degradation Transfer~\cite{chen2021extreme} and Fast Two-step~\cite{eboli2022fast}) through a deblurring task, where all estimated PSFs are used to train state-of-the-art deblurring algorithms. Quantitative comparisons on the Flickr2K dataset~\cite{lim2017enhanced} show significant improvements in image quality, as shown in \cref{tab:compare_deblur}. Additionally, the deblurred results on real captured images exhibit noticeable visual quality improvements, as shown in~\cref{fig:comparision}.








