\section{Related Work}
Various approaches were proposed for load forecasting in smart grids. The works in ____ proposed an LSTM-based model to estimate electrical load demands in smart grids; however, this technique requires data sharing, and hence sensitive user information such as energy consumption patterns may be exposed to third parties. In addition, they also require significant energy consumption ____. The authors in ____ demonstrated that using FL could further improve accuracy without compromising data privacy. Furthermore, it was also able to reduce significant networking load. \textcolor{black}{Another study in ____ introduced FedAVG, which performed better than other FL techniques like FedSGD. However, they struggle with training non-independent and identically distributed (IID) data where data distributions are heterogeneous across SMs}. Recently, PFL techniques have been considered to tackle the data heterogeneity issue in load forecasting. The study in ____ proposed a PFL technique for load forecasting where each SM customizes a federated prediction model. Another work in ____ introduced a Generative Adversarial Network (GAN) based differential privacy (DP) algorithm that included multi-task PFL. However, this solution increases computational complexity at the server in the load forecasting process. 

% \textit{However, this solution increases computational complexity at the server and reduces the scalability of the load-forecasting network.}
% But that creates an energy imbalance where the SM that will control the group will require significant data processing abilities than other SMs. However, this solution increases computational complexity at the server and reduces the scalability of the load-forecasting network.

\begin{table}
\footnotesize
\centering
\caption{Comparison of our approach with existing load forecasting methods.}
\begin{tabular}{|p{2cm}||p{0.6cm}|p{0.6cm}|p{0.6cm}|p{0.6cm}|p{0.6cm}|p{0.6cm}|}
 \hline
 Objectives & ____ &  ____ & ____ & ____ & ____ & Our Approach\\
 \hline
 Handle uncertain and non-iid data & & & \checkmark & \checkmark & \checkmark&\checkmark\\
 \hline
 Include diverse SMs & \checkmark & & \checkmark & \checkmark & \checkmark & \checkmark\\
 \hline
 Adaptability to user change & \checkmark & \checkmark &  & \checkmark & \checkmark& \checkmark\\
 \hline
 Handle large dataset & & \checkmark & \checkmark & \checkmark & \checkmark& \checkmark\\
 \hline
 Maintain server complexity & & \checkmark & \checkmark & & \checkmark & \checkmark\\
 \hline
 Keeping data secured & & \checkmark & \checkmark & \checkmark & \checkmark & \checkmark\\
 \hline
 Latency minimization & & & & & & \checkmark\\
 \hline
 Practical multi-hop settings & & & & & & \checkmark\\
 \hline
 Convergence Analysis & & & & & &\checkmark\\
 \hline
\end{tabular}
\label{table:related_work_table}
\vspace{-5mm}
\end{table}

Moreover, several studies have concentrated on communication in smart grids and smart metering networks. The authors in ____ and ____ explained a smart grid, introduced its components, and presented the communication methods used, highlighting their advantages and shortcomings. It also surveyed smart grid integration, classified communication technologies, and outlined hardware and software security requirements. The work in ____ proposed a smart metering infrastructure with DC and AC analog front ends and communication interfaces, and remote monitoring software for accurate and efficient measurement and transmission in microgrid and smart home applications. 
This work introduced a reconfigurable authenticated key exchange scheme using reconfigurable physical uncloneable functions (PUFs) for secure and efficient smart grid communication, offering advantages in computation and communication costs over current protocols.

Many studies have focused on improving the latency of  FL and addressed FL in multi-hop networks. For instance, ____ optimized model aggregation, routing, and spectrum allocation, while ____ introduced FedAir to mitigate communication impacts on FL performance. ____ used hierarchical FL with adaptive grouping, ____ aimed to reduce congestion by predicting future network topologies, and ____ examined jamming attacks on decentralized FL. Despite these efforts, latency minimization for  FL in multi-hop networks remains unaddressed. Single-hop networks often fail over large areas due to limited transmit power, whereas multi-hop networks provide better communication, coverage, and flexibility. Research on FL in multi-hop networks has focused on mesh networks. Still, it is crucial to consider scenarios with no direct links between non-consecutive nodes for worst-case analysis. \textcolor{black}{Our method is based on a joint design of a new PFL algorithm for collaborative load forecasting and a latency optimization solution for minimizing load forecasting delays in a multi-hop network setting}. We compare our approach with related works in Table~\ref{table:related_work_table}.
%\hfill \break