% CVPR 2025 Paper Template; see https://github.com/cvpr-org/author-kit

\documentclass[10pt,twocolumn,letterpaper]{article}

%%%%%%%%% PAPER TYPE  - PLEASE UPDATE FOR FINAL VERSION
\usepackage{cvpr}              % To produce the CAMERA-READY version
% \usepackage[review]{cvpr}      % To produce the REVIEW version
% \usepackage[pagenumbers]{cvpr} % To force page numbers, e.g. for an arXiv version



% Import additional packages in the preamble file, before hyperref
%
% --- inline annotations
%
\newcommand{\red}[1]{{\color{red}#1}}
\newcommand{\todo}[1]{{\color{red}#1}}
\newcommand{\TODO}[1]{\textbf{\color{red}[TODO: #1]}}
% --- disable by uncommenting  
% \renewcommand{\TODO}[1]{}
% \renewcommand{\todo}[1]{#1}



\newcommand{\VLM}{LVLM\xspace} 
\newcommand{\ours}{PeKit\xspace}
\newcommand{\yollava}{Yo’LLaVA\xspace}

\newcommand{\thisismy}{This-Is-My-Img\xspace}
\newcommand{\myparagraph}[1]{\noindent\textbf{#1}}
\newcommand{\vdoro}[1]{{\color[rgb]{0.4, 0.18, 0.78} {[V] #1}}}
% --- disable by uncommenting  
% \renewcommand{\TODO}[1]{}
% \renewcommand{\todo}[1]{#1}
\usepackage{slashbox}
% Vectors
\newcommand{\bB}{\mathcal{B}}
\newcommand{\bw}{\mathbf{w}}
\newcommand{\bs}{\mathbf{s}}
\newcommand{\bo}{\mathbf{o}}
\newcommand{\bn}{\mathbf{n}}
\newcommand{\bc}{\mathbf{c}}
\newcommand{\bp}{\mathbf{p}}
\newcommand{\bS}{\mathbf{S}}
\newcommand{\bk}{\mathbf{k}}
\newcommand{\bmu}{\boldsymbol{\mu}}
\newcommand{\bx}{\mathbf{x}}
\newcommand{\bg}{\mathbf{g}}
\newcommand{\be}{\mathbf{e}}
\newcommand{\bX}{\mathbf{X}}
\newcommand{\by}{\mathbf{y}}
\newcommand{\bv}{\mathbf{v}}
\newcommand{\bz}{\mathbf{z}}
\newcommand{\bq}{\mathbf{q}}
\newcommand{\bff}{\mathbf{f}}
\newcommand{\bu}{\mathbf{u}}
\newcommand{\bh}{\mathbf{h}}
\newcommand{\bb}{\mathbf{b}}

\newcommand{\rone}{\textcolor{green}{R1}}
\newcommand{\rtwo}{\textcolor{orange}{R2}}
\newcommand{\rthree}{\textcolor{red}{R3}}
\usepackage{amsmath}
%\usepackage{arydshln}
\DeclareMathOperator{\similarity}{sim}
\DeclareMathOperator{\AvgPool}{AvgPool}

\newcommand{\argmax}{\mathop{\mathrm{argmax}}}     



% It is strongly recommended to use hyperref, especially for the review version.
% hyperref with option pagebackref eases the reviewers' job.
% Please disable hyperref *only* if you encounter grave issues, 
% e.g. with the file validation for the camera-ready version.
%
% If you comment hyperref and then uncomment it, you should delete *.aux before re-running LaTeX.
% (Or just hit 'q' on the first LaTeX run, let it finish, and you should be clear).
\definecolor{cvprblue}{rgb}{0.21,0.49,0.74}
\usepackage[pagebackref,breaklinks,colorlinks,allcolors=cvprblue]{hyperref}

% ----- CVPR 官方推荐的一个包 ----- % 
\usepackage[accsupp]{axessibility}

%% ========== 导入其他的可用包 ========== % 
% 引用 URL 的包 %
\usepackage{url}
\usepackage{hyperref}

% 引入在表格中, 可以横跨多行的包
\usepackage{multirow} % 跨越多行后居中
\usepackage{makecell} % 在一个单元格中居中

\usepackage{adjustbox}


% 控制表格的颜色 
\usepackage{colortbl}
\usepackage{xcolor}
\usepackage{color}
\usepackage{array}   %对表列和表格线的设置需要用到array宏包

% 画图相关的包 
% \usepackage{subfig}
\usepackage{float}

% nicer tables
\usepackage{nicematrix}  
\usepackage{tabularx}

% 伪代码相关 
\usepackage{algorithmic}
\usepackage{algorithm}
\usepackage{caption}


% 导入特殊符号的包
\usepackage{pifont}


% ----- 定义颜色相关 ----- % 
% \definecolor{unit01green}{RGB}{82,208,83}
% \newcommand{\green}[1]{\textcolor{unit01green}{#1}}

% \definecolor{unit02red}{RGB}{211,41,15}
% \newcommand{\red}[1]{\textcolor{unit02red}{#1}}

% \definecolor{unit02blue}{RGB}{53,49,255}
% \newcommand{\blue}[1]{\textcolor{unit02blue}{#1}}


\newcommand{\highlight}[1]{{\cellcolor[rgb]{0.925,0.957,1}}{#1}}

% \definecolor{C1}{RGB}{220, 53, 34} %{182, 94, 42}
% \definecolor{C3}{RGB}{0, 128, 20} % {210, 105, 30} % {138, 43, 219} % {210, 105, 30}
% \definecolor{C2}{RGB}{139,69,19} % {0, 128, 9} % {0, 100, 0}
% \definecolor{C4}{RGB}{26, 111, 223}%{141, 82, 158}

\definecolor{green}{rgb}{0, 0.8, 0.2}
\newcommand{\cmark}{\textcolor{green}{\ding{51}}} %

\definecolor{red}{rgb}{1.0, 0.0, 0.0}
\newcommand{\xmark}{\textcolor{red}{\ding{55}}} %

\definecolor{grey}{rgb}{0.9, 0.9, 0.9}
\newcommand{\ccol}{\cellcolor{grey}}

\definecolor{lightblue}{rgb}{0.68, 0.85, 0.9}

\definecolor{softgray}{gray}{0.96}


\definecolor{Template}{RGB}{84, 130, 53}
\definecolor{Description}{RGB}{197, 90, 17}


\definecolor{removed}{RGB}{192, 0, 0}
\definecolor{retained}{RGB}{0, 112, 192}


%%%%%%%%% PAPER ID  - PLEASE UPDATE
\def\paperID{4381} % *** Enter the Paper ID here
\def\confName{CVPR}
\def\confYear{2025}

%%%%%%%%% TITLE - PLEASE UPDATE
\title{ProAPO: Progressively Automatic Prompt Optimization for Visual Classification}

\setcounter{footnote}{1}

%%%%%%%%% AUTHORS - PLEASE UPDATE
\author{%
  Xiangyan Qu$^{12}$ \quad Gaopeng Gou$^{12}$\thanks{Corresponding author} \quad Jiamin Zhuang$^{12}$ \quad Jing Yu$^{3}$  \\
   Kun Song$^{4}$ \quad Qihao Wang$^{12}$ \quad Yili Li$^{12}$ \quad Gang Xiong$^{12}$ \\ 
  \small $^1$Institute of Information Engineering, Chinese Academy of Sciences \quad 
  \small $^2$School of Cyber Security, Chinese Academy of Sciences \\ 
  \small $^3$School of Information Engineering, Minzu University of China \quad 
  \small $^4$University of Science and Technology Beijing \\
  \tt\small \{quxiangyan, gougaopeng, zhuangjiamin\}@iie.ac.cn, jing.yu@muc.edu.cn \\
  \tt\small songkun@xs.ustb.edu.cn, wangqihao22@mails.ucas.ac.cn, \{liyili, xionggang\}@iie.ac.cn
  \vspace{-4mm}
}

% \author{
%     Yaochen Ren\textsuperscript{\dag,\ddag}, 
%     Gaopeng Gou\textsuperscript{\dag,\ddag}, 
%     Chengshang Hou\textsuperscript{\dag,\ddag}, 
%     Tianyu Cui\textsuperscript{\S}, \\
%     Zhen Li\textsuperscript{\dag,\ddag}, 
%     Gang Xiong\textsuperscript{\dag,\ddag} and  
%     Chang Liu\textsuperscript{\dag,\ddag}(\Letter) \\
%     \textsuperscript{\dag}Institute of Information Engineering, Chinese Academy of Sciences, Beijing, China \\
%     \textsuperscript{\ddag}School of Cyber Security, University of Chinese Academy of Sciences, Beijing, China \\
%     \{renyaochen, gougaopeng, houchengshang, lizhen, xionggang, liuchang\}@iie.ac.cn\\
%     \textsuperscript{\S}Zhongguancun Laboratory, Beijing, China \{cuity\}@zgclab.edu.cn}


% \author{First Author\\
% Institution1\\
% Institution1 address\\
% {\tt\small firstauthor@i1.org}
% % For a paper whose authors are all at the same institution,
% % omit the following lines up until the closing ``}''.
% % Additional authors and addresses can be added with ``\and'',
% % just like the second author.
% % To save space, use either the email address or home page, not both
% \and
% Second Author\\
% Institution2\\
% First line of institution2 address\\
% {\tt\small secondauthor@i2.org}
% }

\begin{document}
\maketitle


\begin{abstract}


The choice of representation for geographic location significantly impacts the accuracy of models for a broad range of geospatial tasks, including fine-grained species classification, population density estimation, and biome classification. Recent works like SatCLIP and GeoCLIP learn such representations by contrastively aligning geolocation with co-located images. While these methods work exceptionally well, in this paper, we posit that the current training strategies fail to fully capture the important visual features. We provide an information theoretic perspective on why the resulting embeddings from these methods discard crucial visual information that is important for many downstream tasks. To solve this problem, we propose a novel retrieval-augmented strategy called RANGE. We build our method on the intuition that the visual features of a location can be estimated by combining the visual features from multiple similar-looking locations. We evaluate our method across a wide variety of tasks. Our results show that RANGE outperforms the existing state-of-the-art models with significant margins in most tasks. We show gains of up to 13.1\% on classification tasks and 0.145 $R^2$ on regression tasks. All our code and models will be made available at: \href{https://github.com/mvrl/RANGE}{https://github.com/mvrl/RANGE}.

\end{abstract}

    
\section{Introduction}
Backdoor attacks pose a concealed yet profound security risk to machine learning (ML) models, for which the adversaries can inject a stealth backdoor into the model during training, enabling them to illicitly control the model's output upon encountering predefined inputs. These attacks can even occur without the knowledge of developers or end-users, thereby undermining the trust in ML systems. As ML becomes more deeply embedded in critical sectors like finance, healthcare, and autonomous driving \citep{he2016deep, liu2020computing, tournier2019mrtrix3, adjabi2020past}, the potential damage from backdoor attacks grows, underscoring the emergency for developing robust defense mechanisms against backdoor attacks.

To address the threat of backdoor attacks, researchers have developed a variety of strategies \cite{liu2018fine,wu2021adversarial,wang2019neural,zeng2022adversarial,zhu2023neural,Zhu_2023_ICCV, wei2024shared,wei2024d3}, aimed at purifying backdoors within victim models. These methods are designed to integrate with current deployment workflows seamlessly and have demonstrated significant success in mitigating the effects of backdoor triggers \cite{wubackdoorbench, wu2023defenses, wu2024backdoorbench,dunnett2024countering}.  However, most state-of-the-art (SOTA) backdoor purification methods operate under the assumption that a small clean dataset, often referred to as \textbf{auxiliary dataset}, is available for purification. Such an assumption poses practical challenges, especially in scenarios where data is scarce. To tackle this challenge, efforts have been made to reduce the size of the required auxiliary dataset~\cite{chai2022oneshot,li2023reconstructive, Zhu_2023_ICCV} and even explore dataset-free purification techniques~\cite{zheng2022data,hong2023revisiting,lin2024fusing}. Although these approaches offer some improvements, recent evaluations \cite{dunnett2024countering, wu2024backdoorbench} continue to highlight the importance of sufficient auxiliary data for achieving robust defenses against backdoor attacks.

While significant progress has been made in reducing the size of auxiliary datasets, an equally critical yet underexplored question remains: \emph{how does the nature of the auxiliary dataset affect purification effectiveness?} In  real-world  applications, auxiliary datasets can vary widely, encompassing in-distribution data, synthetic data, or external data from different sources. Understanding how each type of auxiliary dataset influences the purification effectiveness is vital for selecting or constructing the most suitable auxiliary dataset and the corresponding technique. For instance, when multiple datasets are available, understanding how different datasets contribute to purification can guide defenders in selecting or crafting the most appropriate dataset. Conversely, when only limited auxiliary data is accessible, knowing which purification technique works best under those constraints is critical. Therefore, there is an urgent need for a thorough investigation into the impact of auxiliary datasets on purification effectiveness to guide defenders in  enhancing the security of ML systems. 

In this paper, we systematically investigate the critical role of auxiliary datasets in backdoor purification, aiming to bridge the gap between idealized and practical purification scenarios.  Specifically, we first construct a diverse set of auxiliary datasets to emulate real-world conditions, as summarized in Table~\ref{overall}. These datasets include in-distribution data, synthetic data, and external data from other sources. Through an evaluation of SOTA backdoor purification methods across these datasets, we uncover several critical insights: \textbf{1)} In-distribution datasets, particularly those carefully filtered from the original training data of the victim model, effectively preserve the model’s utility for its intended tasks but may fall short in eliminating backdoors. \textbf{2)} Incorporating OOD datasets can help the model forget backdoors but also bring the risk of forgetting critical learned knowledge, significantly degrading its overall performance. Building on these findings, we propose Guided Input Calibration (GIC), a novel technique that enhances backdoor purification by adaptively transforming auxiliary data to better align with the victim model’s learned representations. By leveraging the victim model itself to guide this transformation, GIC optimizes the purification process, striking a balance between preserving model utility and mitigating backdoor threats. Extensive experiments demonstrate that GIC significantly improves the effectiveness of backdoor purification across diverse auxiliary datasets, providing a practical and robust defense solution.

Our main contributions are threefold:
\textbf{1) Impact analysis of auxiliary datasets:} We take the \textbf{first step}  in systematically investigating how different types of auxiliary datasets influence backdoor purification effectiveness. Our findings provide novel insights and serve as a foundation for future research on optimizing dataset selection and construction for enhanced backdoor defense.
%
\textbf{2) Compilation and evaluation of diverse auxiliary datasets:}  We have compiled and rigorously evaluated a diverse set of auxiliary datasets using SOTA purification methods, making our datasets and code publicly available to facilitate and support future research on practical backdoor defense strategies.
%
\textbf{3) Introduction of GIC:} We introduce GIC, the \textbf{first} dedicated solution designed to align auxiliary datasets with the model’s learned representations, significantly enhancing backdoor mitigation across various dataset types. Our approach sets a new benchmark for practical and effective backdoor defense.



\section{Related Work}

\subsection{Large 3D Reconstruction Models}
Recently, generalized feed-forward models for 3D reconstruction from sparse input views have garnered considerable attention due to their applicability in heavily under-constrained scenarios. The Large Reconstruction Model (LRM)~\cite{hong2023lrm} uses a transformer-based encoder-decoder pipeline to infer a NeRF reconstruction from just a single image. Newer iterations have shifted the focus towards generating 3D Gaussian representations from four input images~\cite{tang2025lgm, xu2024grm, zhang2025gslrm, charatan2024pixelsplat, chen2025mvsplat, liu2025mvsgaussian}, showing remarkable novel view synthesis results. The paradigm of transformer-based sparse 3D reconstruction has also successfully been applied to lifting monocular videos to 4D~\cite{ren2024l4gm}. \\
Yet, none of the existing works in the domain have studied the use-case of inferring \textit{animatable} 3D representations from sparse input images, which is the focus of our work. To this end, we build on top of the Large Gaussian Reconstruction Model (GRM)~\cite{xu2024grm}.

\subsection{3D-aware Portrait Animation}
A different line of work focuses on animating portraits in a 3D-aware manner.
MegaPortraits~\cite{drobyshev2022megaportraits} builds a 3D Volume given a source and driving image, and renders the animated source actor via orthographic projection with subsequent 2D neural rendering.
3D morphable models (3DMMs)~\cite{blanz19993dmm} are extensively used to obtain more interpretable control over the portrait animation. For example, StyleRig~\cite{tewari2020stylerig} demonstrates how a 3DMM can be used to control the data generated from a pre-trained StyleGAN~\cite{karras2019stylegan} network. ROME~\cite{khakhulin2022rome} predicts vertex offsets and texture of a FLAME~\cite{li2017flame} mesh from the input image.
A TriPlane representation is inferred and animated via FLAME~\cite{li2017flame} in multiple methods like Portrait4D~\cite{deng2024portrait4d}, Portrait4D-v2~\cite{deng2024portrait4dv2}, and GPAvatar~\cite{chu2024gpavatar}.
Others, such as VOODOO 3D~\cite{tran2024voodoo3d} and VOODOO XP~\cite{tran2024voodooxp}, learn their own expression encoder to drive the source person in a more detailed manner. \\
All of the aforementioned methods require nothing more than a single image of a person to animate it. This allows them to train on large monocular video datasets to infer a very generic motion prior that even translates to paintings or cartoon characters. However, due to their task formulation, these methods mostly focus on image synthesis from a frontal camera, often trading 3D consistency for better image quality by using 2D screen-space neural renderers. In contrast, our work aims to produce a truthful and complete 3D avatar representation from the input images that can be viewed from any angle.  

\subsection{Photo-realistic 3D Face Models}
The increasing availability of large-scale multi-view face datasets~\cite{kirschstein2023nersemble, ava256, pan2024renderme360, yang2020facescape} has enabled building photo-realistic 3D face models that learn a detailed prior over both geometry and appearance of human faces. HeadNeRF~\cite{hong2022headnerf} conditions a Neural Radiance Field (NeRF)~\cite{mildenhall2021nerf} on identity, expression, albedo, and illumination codes. VRMM~\cite{yang2024vrmm} builds a high-quality and relightable 3D face model using volumetric primitives~\cite{lombardi2021mvp}. One2Avatar~\cite{yu2024one2avatar} extends a 3DMM by anchoring a radiance field to its surface. More recently, GPHM~\cite{xu2025gphm} and HeadGAP~\cite{zheng2024headgap} have adopted 3D Gaussians to build a photo-realistic 3D face model. \\
Photo-realistic 3D face models learn a powerful prior over human facial appearance and geometry, which can be fitted to a single or multiple images of a person, effectively inferring a 3D head avatar. However, the fitting procedure itself is non-trivial and often requires expensive test-time optimization, impeding casual use-cases on consumer-grade devices. While this limitation may be circumvented by learning a generalized encoder that maps images into the 3D face model's latent space, another fundamental limitation remains. Even with more multi-view face datasets being published, the number of available training subjects rarely exceeds the thousands, making it hard to truly learn the full distibution of human facial appearance. Instead, our approach avoids generalizing over the identity axis by conditioning on some images of a person, and only generalizes over the expression axis for which plenty of data is available. 

A similar motivation has inspired recent work on codec avatars where a generalized network infers an animatable 3D representation given a registered mesh of a person~\cite{cao2022authentic, li2024uravatar}.
The resulting avatars exhibit excellent quality at the cost of several minutes of video capture per subject and expensive test-time optimization.
For example, URAvatar~\cite{li2024uravatar} finetunes their network on the given video recording for 3 hours on 8 A100 GPUs, making inference on consumer-grade devices impossible. In contrast, our approach directly regresses the final 3D head avatar from just four input images without the need for expensive test-time fine-tuning.


\section{Methodology}

\subsection{Problem Definition}

Given a multivariate time series input $X \in \mathbb{R}^{C  \times T}$, multivariate time series forecasting tasks are designed to predict its future $F$ time steps $\hat{Y}\in \mathbb{R}^{C \times F}$ using past $T$ steps. $C $ is the number of variates or channels.

\subsection{Preliminary Analysis}

This section presents why RevIN~\citep{Kim_revin,liu2022non}, High-pass, and Low-pass filters fail to address the Mid-Frequency Spectrum Gap. Let the input univariate time series be $ x(t) $ with length $ T $ and target $ y(t) $ with length $ F $. 

\begin{definition}[Frequency Spectral Energy]\label{def:energy}
The Fourier transform of $x(t)$, $X(f)$, and its spectral energy $E_X(f)$ is given by:
\vspace{-0.2cm}
\begin{align}
X(f) = \sum_{t=0}^{T-1} x(t) e^{-i 2 \pi f t / {T-1}}, \quad &f = 0, 1, \dots, T-1\notag\\
E_X(f) = |X(f)|^2.
\end{align}
\vspace{-0.2cm}
\end{definition}

\textbf{Impact of RevIN on Frequency Spectrum \quad}
\begin{definition}[Reversible Instance Normalization]\label{def:RevIN}
Given a \textbf{forecast model} $ f: \mathbb{R}^T \rightarrow \mathbb{R}^F $ that generates a forecast $ \hat{y}(t) $ from a given input $x(t)$, RevIN is defined as:
\vspace{-0.2cm}
\begin{align}
&\hat{x}(t) = \frac{x(t) - \mu}{\sigma},\quad t = 0, 1, \dots, T-1\notag\\
&\hat{y}(t) = f(\hat{x}(t)), \quad \hat{y}(t)_{rev}= \hat{y}(t) \cdot \sigma + \mu,\notag\\
&\mu = \frac{1}{T} \sum_{t=0}^{T-1} x(t), \quad \sigma = \sqrt{\frac{1}{T} \sum_{t=0}^{T-1} (x(t) - \mu)^2}.
\end{align}
\vspace{-0.2cm}
\end{definition}

\begin{theorem} [Frequency Spectrum after RevIN] \label{theorem:RevIN}
\vspace{-0.2cm}
The spectral energy of $\hat{x}(t)$ (transformed using RevIN):
\begin{align}
E_{\hat{X}}(0)=0,& \quad f=0, \notag\\
E_{\hat{X}}(f) = \left( \frac{1}{\sigma} \right)^2 |X(f)|^2,&\quad f = 1,2,\dots, T-1 . 
\end{align}
\vspace{-0.2cm}
\end{theorem}
The proof is in Appendix~\ref{app:RevIN}. Theorem~\ref{theorem:RevIN} suggests that RevIN scales the absolute spectral energy by $ \sigma^2 $ but does not affect its relative distribution except $E_{\hat{X}}(0)=0$. Thus, RevIN preserves the relative spectral energy distribution and leaves the Mid-Frequency Spectrum Gap unresolved. \textit{However, our experiments still employ RevIN to ensure a fair comparison with other baselines.}
\begin{figure*}[h]
  \centering
  \includegraphics[width=1.\linewidth]{Faker/source/assets/jpg/ReFocus.jpg}
  \caption{General structure of \textbf{ReFocus}. `Adaptive Mid-Frequency Energy Optimizer (AMEO)' enhances mid-frequency components modeling, and `Energy-based Key-Frequency Picking Block' (EKPB) effectively captures shared Key-Frequency across channels}
  \label{fig:refocus}
\end{figure*}

\begin{figure*}[h]
  \centering
  \includegraphics[width=0.7\linewidth]{Faker/source/assets/jpg/ket.jpg}
  \caption{General process of the \textbf{Key-Frequency Enhanced Training strategy (KET)}, where spectral information from other channels is randomly introduced into each channel, to enhance the extraction of the shared Key-Frequency.}
  \label{fig:reshuffle}
\end{figure*}
\textbf{Impact of High- and Low-pass filter \quad}
We still define $\hat{x}(t)$ to be the filtered (processed) signal, obtained by applying a filter $H(f)$ (High/Low-pass filter). The filter $ H(f) $ is 1 in the passband (High/Low frequency) and 0 in the stopband (Middle frequency). So $E_{\hat{X}}(f)=0,\quad E_{\hat{X}}\leq E_X(f)$ for middle frequencies, which creates even larger gap.

\subsection{Overall Structure of The Proposed ReFocus}

In this section, we elucidate the overall architecture of \textbf{ReFocus}, depicted in Figure \ref{fig:refocus}. We define frequency domain projection as $D1\rightarrow D2$ representing a projection from dimension $D1$ to $D2$ in the frequency domain~\citep{xu2024fits}. Initially, we apply \textbf{AMEO} to the input $X \in \mathbb{R}^{C \times T}$, yielding the processed spectrum $ X_{am} \in \mathbb{R}^{C  \times T} $. Next, we use a projection $T\rightarrow D$ to transform $ X_{am}$ into the Variate Embedding $ X_{em} \in \mathbb{R}^{C  \times D}$~\citep{LiuiTransformer}. Then, $X_{em}$ go through $N$ \textbf{EKPB} to generate representation $H_{N+1}$, which is projected to obtain final prediction $\hat{Y}$. 

\textbf{Adaptive Mid-Frequency Energy Optimizer \quad}
Building upon the \textbf{Preliminary Analysis}, we propose a convolution- and residual learning-based solution to address the Mid-Frequency Spectrum Gap, which we denoted as AMEO. 
\begin{definition}[Adaptive Mid-Frequency Energy Optimizer]\label{def:AMEO}
AMEO is defined as:
\begin{align}
&\hat{x}(t) = x(t)-\frac{\beta}{K}\sum_{k=0}^{K-1} \tilde{x}(t+K-1-k),\notag\\
&\tilde{x}(t) =\notag\\
&\begin{cases}
x(t-(\frac{K}{2}+1)), \quad \text{if } \frac{K}{2}+1 \leq t < T+\frac{K}{2}+1, \\
0,  \quad\text{if } 0 \leq t < \frac{K}{2}+1 \text{ or } T+\frac{K}{2}+1 \leq t < T+K.
\end{cases}
\end{align}
\vspace{-0.2cm}
\end{definition}

It is equivalent to $x=x-\beta \cdot Conv(x)$. $Conv$ is a 1D convolution (Zero-padding at both ends, stride $s=1$, kernel size $K$, with values initialized as $ \frac{1}{K} $). $\beta \in \mathbb{R}^{1}$ is a hyperparameter.

\begin{theorem} [Frequency Spectrum after AMEO] \label{theorem:AMEO}
The spectral energy of $\hat{x}(t)$ obtained using AMEO:
\begin{align}
E_{\hat{X}}(f) =|X(f)|^2 \left\{1 - \beta \cdot \underbrace{\frac{1}{K} \sum_{k=0}^{K-1} e^{i 2 \pi f (\frac{3K}{2}-k -2) / {T-1}}}_{G(f)}\right\}^2
\end{align}
\vspace{-0.2cm}
\end{theorem}

The proof is in Appendix~\ref{app:AMEO}. We have $E_{\hat{X}}(f) =|X(f)|^2(1-\beta  \cdot G(f))^2$. Generally, $ G(f) $ behaves as a decay function, gradually reducing its value from \textbf{One} to \textbf{Zero}. Such \textbf{decay behavior} makes AMEO relatively enhances mid-frequency components, thus addressing the Mid-Frequency Spectrum Gap.

\textbf{Energy-based Key-Frequency Picking Block \quad} In each \textbf{EKPB}, the input $ H_i \in \mathbb{R}^{C  \times D} (H_1=X_{em}) $ is first processed through an MLP to generate $ H_i^k \in \mathbb{R}^{C  \times Q}$. Then, FFT is applied to get $ H_i^f \in \mathbb{R}^{C  \times (Q/2+1)}$. For $ H_i^f$, we calculate its energy, denoted as $ H_i^e \in \mathbb{R}^{C  \times (Q/2+1)}$. A cross-channel softmax is then applied to $ H_i^e$ per frequency to obtain a probability distribution $ H_i^{soft} \in \mathbb{R}^{C  \times (Q/2+1)}$. Using $H_i^{soft}$, we select values from $ H_i^f$ across channels for each frequency, resulting in $K^f_i \in \mathbb{R}^{1  \times (Q/2+1)}$, which represents the Shared Key-Frequency across all channels. Then iFFT is performed on $K^f_i$ to get $K_i\in \mathbb{R}^{1  \times Q}$, followed by projection $Q\rightarrow D$ and repeating (C times) to get $\hat{K}_i \in \mathbb{R}^{C  \times D}$. This $\hat{K}_i$ is point-wisely added to $\hat{H_i}\in \mathbb{R}^{C  \times D}$ , which is the projection of $ H_i$ using projection $D\rightarrow D$. Then, an MLP and $Add\&Norm$ is applied to the result $HK\in \mathbb{R}^{C  \times D}$ to fuse inter-series dependencies information, and another MLP and $Add\&Norm$ is used to capture intra-series variations~\citep{LiuiTransformer}. The output of each \textbf{EKPB} is $\hat{O_i} \in \mathbb{R}^{C  \times D}$, where $H_{i+1}=\hat{O_i}$.

\subsection{Key-Frequency Enhanced Training strategy}

In real-world time series, certain channels often exhibit spectral dependencies, which may not be fully captured in the training set, and the specific channels with such dependencies are also unknown~\citep{geweke1984freqchannel,Zhao2024freqchannel}. So this work borrows insight from recent advancement of mix-up in time series~\citep{zhou2023mixup,ansari2024mixup}, randomly introducing spectral information from other channels into each channel, to enhance the extraction of the shared Key-Frequency, as in Figure~\ref{fig:reshuffle}. Given a multivariate time series input $X \in \mathbb{R}^{C \times T}$ and its ground-truth $Y \in \mathbb{R}^{C \times F}$, we generate a pseudo sample pair: 

\begin{align}
X' = iFFT(FFT(X) +\alpha \cdot FFT(X[\text{perm},:]))&,  \notag\\ 
Y' = iFFT(FFT(Y) +\alpha \cdot FFT(Y[\text{perm},:]))&.
\end{align}

$\alpha \in \mathbb{R}^{C \times 1}$ is a weight vector sampled from a normal distribution, $\text{perm}$ is a reshuffled channel index. Since $FFT$ and $iFFT$ are linear operations, this mix-up process can be equivalently simplified in the \textbf{Time Domain}:
\begin{align}
X' = X +\alpha \cdot X[\text{perm},:]&,  \notag\\
Y' = Y +\alpha \cdot Y[\text{perm},:]&
 \end{align}
We alternate training between real and synthetic data to preserve the spectral dependencies in real samples. This combines the advantages of data augmentation, such as improved generalization, while mitigating potential drawbacks like over-smoothing and training instability~\citep{ryu2024tf,alkhalifah2022tf}.












\section{Experiments}

\subsection{Setups}
\subsubsection{Implementation Details}
We apply our FDS method to two types of 3DGS: 
the original 3DGS, and 2DGS~\citep{huang20242d}. 
%
The number of iterations in our optimization 
process is 35,000.
We follow the default training configuration 
and apply our FDS method after 15,000 iterations,
then we add normal consistency loss for both
3DGS and 2DGS after 25000 iterations.
%
The weight for FDS, $\lambda_{fds}$, is set to 0.015,
the $\sigma$ is set to 23,
and the weight for normal consistency is set to 0.15
for all experiments. 
We removed the depth distortion loss in 2DGS 
because we found that it degrades its results in indoor scenes.
%
The Gaussian point cloud is initialized using Colmap
for all datasets.
%
%
We tested the impact of 
using Sea Raft~\citep{wang2025sea} and 
Raft\citep{teed2020raft} on FDS performance.
%
Due to the blurriness of the ScanNet dataset, 
additional prior constraints are required.
Thus, we incorporate normal prior supervision 
on the rendered normals 
in ScanNet (V2) dataset by default.
The normal prior is predicted by the Stable Normal 
model~\citep{ye2024stablenormal}
across all types of 3DGS.
%
The entire framework is implemented in 
PyTorch~\citep{paszke2019pytorch}, 
and all experiments are conducted on 
a single NVIDIA 4090D GPU.

\begin{figure}[t] \centering
    \makebox[0.16\textwidth]{\scriptsize Input}
    \makebox[0.16\textwidth]{\scriptsize 3DGS}
    \makebox[0.16\textwidth]{\scriptsize 2DGS}
    \makebox[0.16\textwidth]{\scriptsize 3DGS + FDS}
    \makebox[0.16\textwidth]{\scriptsize 2DGS + FDS}
    \makebox[0.16\textwidth]{\scriptsize GT (Depth)}

    \includegraphics[width=0.16\textwidth]{figure/fig3_img/compare3/gt_rgb/frame_00522.jpg}
    \includegraphics[width=0.16\textwidth]{figure/fig3_img/compare3/3DGS/frame_00522.jpg}
    \includegraphics[width=0.16\textwidth]{figure/fig3_img/compare3/2DGS/frame_00522.jpg}
    \includegraphics[width=0.16\textwidth]{figure/fig3_img/compare3/3DGS+FDS/frame_00522.jpg}
    \includegraphics[width=0.16\textwidth]{figure/fig3_img/compare3/2DGS+FDS/frame_00522.jpg}
    \includegraphics[width=0.16\textwidth]{figure/fig3_img/compare3/gt_depth/frame_00522.jpg} \\

    % \includegraphics[width=0.16\textwidth]{figure/fig3_img/compare1/gt_rgb/frame_00137.jpg}
    % \includegraphics[width=0.16\textwidth]{figure/fig3_img/compare1/3DGS/frame_00137.jpg}
    % \includegraphics[width=0.16\textwidth]{figure/fig3_img/compare1/2DGS/frame_00137.jpg}
    % \includegraphics[width=0.16\textwidth]{figure/fig3_img/compare1/3DGS+FDS/frame_00137.jpg}
    % \includegraphics[width=0.16\textwidth]{figure/fig3_img/compare1/2DGS+FDS/frame_00137.jpg}
    % \includegraphics[width=0.16\textwidth]{figure/fig3_img/compare1/gt_depth/frame_00137.jpg} \\

     \includegraphics[width=0.16\textwidth]{figure/fig3_img/compare2/gt_rgb/frame_00262.jpg}
    \includegraphics[width=0.16\textwidth]{figure/fig3_img/compare2/3DGS/frame_00262.jpg}
    \includegraphics[width=0.16\textwidth]{figure/fig3_img/compare2/2DGS/frame_00262.jpg}
    \includegraphics[width=0.16\textwidth]{figure/fig3_img/compare2/3DGS+FDS/frame_00262.jpg}
    \includegraphics[width=0.16\textwidth]{figure/fig3_img/compare2/2DGS+FDS/frame_00262.jpg}
    \includegraphics[width=0.16\textwidth]{figure/fig3_img/compare2/gt_depth/frame_00262.jpg} \\

    \includegraphics[width=0.16\textwidth]{figure/fig3_img/compare4/gt_rgb/frame00000.png}
    \includegraphics[width=0.16\textwidth]{figure/fig3_img/compare4/3DGS/frame00000.png}
    \includegraphics[width=0.16\textwidth]{figure/fig3_img/compare4/2DGS/frame00000.png}
    \includegraphics[width=0.16\textwidth]{figure/fig3_img/compare4/3DGS+FDS/frame00000.png}
    \includegraphics[width=0.16\textwidth]{figure/fig3_img/compare4/2DGS+FDS/frame00000.png}
    \includegraphics[width=0.16\textwidth]{figure/fig3_img/compare4/gt_depth/frame00000.png} \\

    \includegraphics[width=0.16\textwidth]{figure/fig3_img/compare5/gt_rgb/frame00080.png}
    \includegraphics[width=0.16\textwidth]{figure/fig3_img/compare5/3DGS/frame00080.png}
    \includegraphics[width=0.16\textwidth]{figure/fig3_img/compare5/2DGS/frame00080.png}
    \includegraphics[width=0.16\textwidth]{figure/fig3_img/compare5/3DGS+FDS/frame00080.png}
    \includegraphics[width=0.16\textwidth]{figure/fig3_img/compare5/2DGS+FDS/frame00080.png}
    \includegraphics[width=0.16\textwidth]{figure/fig3_img/compare5/gt_depth/frame00080.png} \\



    \caption{\textbf{Comparison of depth reconstruction on Mushroom and ScanNet datasets.} The original
    3DGS or 2DGS model equipped with FDS can remove unwanted floaters and reconstruct
    geometry more preciously.}
    \label{fig:compare}
\end{figure}


\subsubsection{Datasets and Metrics}

We evaluate our method for 3D reconstruction 
and novel view synthesis tasks on
\textbf{Mushroom}~\citep{ren2024mushroom},
\textbf{ScanNet (v2)}~\citep{dai2017scannet}, and 
\textbf{Replica}~\citep{replica19arxiv}
datasets,
which feature challenging indoor scenes with both 
sparse and dense image sampling.
%
The Mushroom dataset is an indoor dataset 
with sparse image sampling and two distinct 
camera trajectories. 
%
We train our model on the training split of 
the long capture sequence and evaluate 
novel view synthesis on the test split 
of the long capture sequences.
%
Five scenes are selected to evaluate our FDS, 
including "coffee room", "honka", "kokko", 
"sauna", and "vr room". 
%
ScanNet(V2)~\citep{dai2017scannet}  consists of 1,613 indoor scenes
with annotated camera poses and depth maps. 
%
We select 5 scenes from the ScanNet (V2) dataset, 
uniformly sampling one-tenth of the views,
following the approach in ~\citep{guo2022manhattan}.
To further improve the geometry rendering quality of 3DGS, 
%
Replica~\citep{replica19arxiv} contains small-scale 
real-world indoor scans. 
We evaluate our FDS on five scenes from 
Replica: office0, office1, office2, office3 and office4,
selecting one-tenth of the views for training.
%
The results for Replica are provided in the 
supplementary materials.
To evaluate the rendering quality and geometry 
of 3DGS, we report PSNR, SSIM, and LPIPS for 
rendering quality, along with Absolute Relative Distance 
(Abs Rel) as a depth quality metrics.
%
Additionally, for mesh evaluation, 
we use metrics including Accuracy, Completion, 
Chamfer-L1 distance, Normal Consistency, 
and F-scores.




\subsection{Results}
\subsubsection{Depth rendering and novel view synthesis}
The comparison results on Mushroom and 
ScanNet are presented in \tabref{tab:mushroom} 
and \tabref{tab:scannet}, respectively. 
%
Due to the sparsity of sampling 
in the Mushroom dataset,
challenges are posed for both GOF~\citep{yu2024gaussian} 
and PGSR~\citep{chen2024pgsr}, 
leading to their relative poor performance 
on the Mushroom dataset.
%
Our approach achieves the best performance 
with the FDS method applied during the training process.
The FDS significantly enhances the 
geometric quality of 3DGS on the Mushroom dataset, 
improving the "abs rel" metric by more than 50\%.
%
We found that Sea Raft~\citep{wang2025sea}
outperforms Raft~\citep{teed2020raft} on FDS, 
indicating that a better optical flow model 
can lead to more significant improvements.
%
Additionally, the render quality of RGB 
images shows a slight improvement, 
by 0.58 in 3DGS and 0.50 in 2DGS, 
benefiting from the incorporation of cross-view consistency in FDS. 
%
On the Mushroom
dataset, adding the FDS loss increases 
the training time by half an hour, which maintains the same
level as baseline.
%
Similarly, our method shows a notable improvement on the ScanNet dataset as well using Sea Raft~\citep{wang2025sea} Model. The "abs rel" metric in 2DGS is improved nearly 50\%. This demonstrates the robustness and effectiveness of the FDS method across different datasets.
%


% \begin{wraptable}{r}{0.6\linewidth} \centering
% \caption{\textbf{Ablation study on geometry priors.}} 
%         \label{tab:analysis_prior}
%         \resizebox{\textwidth}{!}{
\begin{tabular}{c| c c c c c | c c c c}

    \hline
     Method &  Acc$\downarrow$ & Comp $\downarrow$ & C-L1 $\downarrow$ & NC $\uparrow$ & F-Score $\uparrow$ &  Abs Rel $\downarrow$ &  PSNR $\uparrow$  & SSIM  $\uparrow$ & LPIPS $\downarrow$ \\ \hline
    2DGS&   0.1078&  0.0850&  0.0964&  0.7835&  0.5170&  0.1002&  23.56&  0.8166& 0.2730\\
    2DGS+Depth&   0.0862&  0.0702&  0.0782&  0.8153&  0.5965&  0.0672&  23.92&  0.8227& 0.2619 \\
    2DGS+MVDepth&   0.2065&  0.0917&  0.1491&  0.7832&  0.3178&  0.0792&  23.74&  0.8193& 0.2692 \\
    2DGS+Normal&   0.0939&  0.0637&  0.0788&  \textbf{0.8359}&  0.5782&  0.0768&  23.78&  0.8197& 0.2676 \\
    2DGS+FDS &  \textbf{0.0615} & \textbf{ 0.0534}& \textbf{0.0574}& 0.8151& \textbf{0.6974}&  \textbf{0.0561}&  \textbf{24.06}&  \textbf{0.8271}&\textbf{0.2610} \\ \hline
    2DGS+Depth+FDS &  0.0561 &  0.0519& 0.0540& 0.8295& 0.7282&  0.0454&  \textbf{24.22}& \textbf{0.8291}&\textbf{0.2570} \\
    2DGS+Normal+FDS &  \textbf{0.0529} & \textbf{ 0.0450}& \textbf{0.0490}& \textbf{0.8477}& \textbf{0.7430}&  \textbf{0.0443}&  24.10&  0.8283& 0.2590 \\
    2DGS+Depth+Normal &  0.0695 & 0.0513& 0.0604& 0.8540&0.6723&  0.0523&  24.09&  0.8264&0.2575\\ \hline
    2DGS+Depth+Normal+FDS &  \textbf{0.0506} & \textbf{0.0423}& \textbf{0.0464}& \textbf{0.8598}&\textbf{0.7613}&  \textbf{0.0403}&  \textbf{24.22}& 
    \textbf{0.8300}&\textbf{0.0403}\\
    
\bottomrule
\end{tabular}
}
% \end{wraptable}



The qualitative comparisons on the Mushroom and ScanNet dataset 
are illustrated in \figref{fig:compare}. 
%
%
As seen in the first row of \figref{fig:compare}, 
both the original 3DGS and 2DGS suffer from overfitting, 
leading to corrupted geometry generation. 
%
Our FDS effectively mitigates this issue by 
supervising the matching relationship between 
the input and sampled views, 
helping to recover the geometry.
%
FDS also improves the refinement of geometric details, 
as shown in other rows. 
By incorporating the matching prior through FDS, 
the quality of the rendered depth is significantly improved.
%

\begin{table}[t] \centering
\begin{minipage}[t]{0.96\linewidth}
        \captionof{table}{\textbf{3D Reconstruction 
        and novel view synthesis results on Mushroom dataset. * 
        Represents that FDS uses the Raft model.
        }}
        \label{tab:mushroom}
        \resizebox{\textwidth}{!}{
\begin{tabular}{c| c c c c c | c c c c c}
    \hline
     Method &  Acc$\downarrow$ & Comp $\downarrow$ & C-L1 $\downarrow$ & NC $\uparrow$ & F-Score $\uparrow$ &  Abs Rel $\downarrow$ &  PSNR $\uparrow$  & SSIM  $\uparrow$ & LPIPS $\downarrow$ & Time  $\downarrow$ \\ \hline

    % DN-splatter &   &  &  &  &  &  &  &  & \\
    GOF &  0.1812 & 0.1093 & 0.1453 & 0.6292 & 0.3665 & 0.2380  & 21.37  &  0.7762  & 0.3132  & $\approx$1.4h\\ 
    PGSR &  0.0971 & 0.1420 & 0.1196 & 0.7193 & 0.5105 & 0.1723  & 22.13  & 0.7773  & 0.2918  & $\approx$1.2h \\ \hline
    3DGS &   0.1167 &  0.1033&  0.1100&  0.7954&  0.3739&  0.1214&  24.18&  0.8392& 0.2511 &$\approx$0.8h \\
    3DGS + FDS$^*$ & 0.0569  & 0.0676 & 0.0623 & 0.8105 & 0.6573 & 0.0603 & 24.72  & 0.8489 & 0.2379 &$\approx$1.3h \\
    3DGS + FDS & \textbf{0.0527}  & \textbf{0.0565} & \textbf{0.0546} & \textbf{0.8178} & \textbf{0.6958} & \textbf{0.0568} & \textbf{24.76}  & \textbf{0.8486} & \textbf{0.2381} &$\approx$1.3h \\ \hline
    2DGS&   0.1078&  0.0850&  0.0964&  0.7835&  0.5170&  0.1002&  23.56&  0.8166& 0.2730 &$\approx$0.8h\\
    2DGS + FDS$^*$ &  0.0689 &  0.0646& 0.0667& 0.8042& 0.6582& 0.0589& 23.98&  0.8255&0.2621 &$\approx$1.3h\\
    2DGS + FDS &  \textbf{0.0615} & \textbf{ 0.0534}& \textbf{0.0574}& \textbf{0.8151}& \textbf{0.6974}&  \textbf{0.0561}&  \textbf{24.06}&  \textbf{0.8271}&\textbf{0.2610} &$\approx$1.3h \\ \hline
\end{tabular}
}
\end{minipage}\hfill
\end{table}

\begin{table}[t] \centering
\begin{minipage}[t]{0.96\linewidth}
        \captionof{table}{\textbf{3D Reconstruction 
        and novel view synthesis results on ScanNet dataset.}}
        \label{tab:scannet}
        \resizebox{\textwidth}{!}{
\begin{tabular}{c| c c c c c | c c c c }
    \hline
     Method &  Acc $\downarrow$ & Comp $\downarrow$ & C-L1 $\downarrow$ & NC $\uparrow$ & F-Score $\uparrow$ &  Abs Rel $\downarrow$ &  PSNR $\uparrow$  & SSIM  $\uparrow$ & LPIPS $\downarrow$ \\ \hline
    GOF & 1.8671  & 0.0805 & 0.9738 & 0.5622 & 0.2526 & 0.1597  & 21.55  & 0.7575  & 0.3881 \\
    PGSR &  0.2928 & 0.5103 & 0.4015 & 0.5567 & 0.1926 & 0.1661  & 21.71 & 0.7699  & 0.3899 \\ \hline

    3DGS &  0.4867 & 0.1211 & 0.3039 & 0.7342& 0.3059 & 0.1227 & 22.19& 0.7837 & 0.3907\\
    3DGS + FDS &  \textbf{0.2458} & \textbf{0.0787} & \textbf{0.1622} & \textbf{0.7831} & 
    \textbf{0.4482} & \textbf{0.0573} & \textbf{22.83} & \textbf{0.7911} & \textbf{0.3826} \\ \hline
    2DGS &  0.2658 & 0.0845 & 0.1752 & 0.7504& 0.4464 & 0.0831 & 22.59& 0.7881 & 0.3854\\
    2DGS + FDS &  \textbf{0.1457} & \textbf{0.0679} & \textbf{0.1068} & \textbf{0.7883} & 
    \textbf{0.5459} & \textbf{0.0432} & \textbf{22.91} & \textbf{0.7928} & \textbf{0.3800} \\ \hline
\end{tabular}
}
\end{minipage}\hfill
\end{table}


\begin{table}[t] \centering
\begin{minipage}[t]{0.96\linewidth}
        \captionof{table}{\textbf{Ablation study on geometry priors.}}
        \label{tab:analysis_prior}
        \resizebox{\textwidth}{!}{
\begin{tabular}{c| c c c c c | c c c c}

    \hline
     Method &  Acc$\downarrow$ & Comp $\downarrow$ & C-L1 $\downarrow$ & NC $\uparrow$ & F-Score $\uparrow$ &  Abs Rel $\downarrow$ &  PSNR $\uparrow$  & SSIM  $\uparrow$ & LPIPS $\downarrow$ \\ \hline
    2DGS&   0.1078&  0.0850&  0.0964&  0.7835&  0.5170&  0.1002&  23.56&  0.8166& 0.2730\\
    2DGS+Depth&   0.0862&  0.0702&  0.0782&  0.8153&  0.5965&  0.0672&  23.92&  0.8227& 0.2619 \\
    2DGS+MVDepth&   0.2065&  0.0917&  0.1491&  0.7832&  0.3178&  0.0792&  23.74&  0.8193& 0.2692 \\
    2DGS+Normal&   0.0939&  0.0637&  0.0788&  \textbf{0.8359}&  0.5782&  0.0768&  23.78&  0.8197& 0.2676 \\
    2DGS+FDS &  \textbf{0.0615} & \textbf{ 0.0534}& \textbf{0.0574}& 0.8151& \textbf{0.6974}&  \textbf{0.0561}&  \textbf{24.06}&  \textbf{0.8271}&\textbf{0.2610} \\ \hline
    2DGS+Depth+FDS &  0.0561 &  0.0519& 0.0540& 0.8295& 0.7282&  0.0454&  \textbf{24.22}& \textbf{0.8291}&\textbf{0.2570} \\
    2DGS+Normal+FDS &  \textbf{0.0529} & \textbf{ 0.0450}& \textbf{0.0490}& \textbf{0.8477}& \textbf{0.7430}&  \textbf{0.0443}&  24.10&  0.8283& 0.2590 \\
    2DGS+Depth+Normal &  0.0695 & 0.0513& 0.0604& 0.8540&0.6723&  0.0523&  24.09&  0.8264&0.2575\\ \hline
    2DGS+Depth+Normal+FDS &  \textbf{0.0506} & \textbf{0.0423}& \textbf{0.0464}& \textbf{0.8598}&\textbf{0.7613}&  \textbf{0.0403}&  \textbf{24.22}& 
    \textbf{0.8300}&\textbf{0.0403}\\
    
\bottomrule
\end{tabular}
}
\end{minipage}\hfill
\end{table}




\subsubsection{Mesh extraction}
To further demonstrate the improvement in geometry quality, 
we applied methods used in ~\citep{turkulainen2024dnsplatter} 
to extract meshes from the input views of optimized 3DGS. 
The comparison results are presented  
in \tabref{tab:mushroom}. 
With the integration of FDS, the mesh quality is significantly enhanced compared to the baseline, featuring fewer floaters and more well-defined shapes.
 %
% Following the incorporation of FDS, the reconstruction 
% results exhibit fewer floaters and more well-defined 
% shapes in the meshes. 
% Visualized comparisons
% are provided in the supplementary material.

% \begin{figure}[t] \centering
%     \makebox[0.19\textwidth]{\scriptsize GT}
%     \makebox[0.19\textwidth]{\scriptsize 3DGS}
%     \makebox[0.19\textwidth]{\scriptsize 3DGS+FDS}
%     \makebox[0.19\textwidth]{\scriptsize 2DGS}
%     \makebox[0.19\textwidth]{\scriptsize 2DGS+FDS} \\

%     \includegraphics[width=0.19\textwidth]{figure/fig4_img/compare1/gt02.png}
%     \includegraphics[width=0.19\textwidth]{figure/fig4_img/compare1/baseline06.png}
%     \includegraphics[width=0.19\textwidth]{figure/fig4_img/compare1/baseline_fds05.png}
%     \includegraphics[width=0.19\textwidth]{figure/fig4_img/compare1/2dgs04.png}
%     \includegraphics[width=0.19\textwidth]{figure/fig4_img/compare1/2dgs_fds03.png} \\

%     \includegraphics[width=0.19\textwidth]{figure/fig4_img/compare2/gt00.png}
%     \includegraphics[width=0.19\textwidth]{figure/fig4_img/compare2/baseline02.png}
%     \includegraphics[width=0.19\textwidth]{figure/fig4_img/compare2/baseline_fds01.png}
%     \includegraphics[width=0.19\textwidth]{figure/fig4_img/compare2/2dgs04.png}
%     \includegraphics[width=0.19\textwidth]{figure/fig4_img/compare2/2dgs_fds03.png} \\
      
%     \includegraphics[width=0.19\textwidth]{figure/fig4_img/compare3/gt05.png}
%     \includegraphics[width=0.19\textwidth]{figure/fig4_img/compare3/3dgs03.png}
%     \includegraphics[width=0.19\textwidth]{figure/fig4_img/compare3/3dgs_fds04.png}
%     \includegraphics[width=0.19\textwidth]{figure/fig4_img/compare3/2dgs02.png}
%     \includegraphics[width=0.19\textwidth]{figure/fig4_img/compare3/2dgs_fds01.png} \\

%     \caption{\textbf{Qualitative comparison of extracted mesh 
%     on Mushroom and ScanNet datasets.}}
%     \label{fig:mesh}
% \end{figure}












\subsection{Ablation study}


\textbf{Ablation study on geometry priors:} 
To highlight the advantage of incorporating matching priors, 
we incorporated various types of priors generated by different 
models into 2DGS. These include a monocular depth estimation
model (Depth Anything v2)~\citep{yang2024depth}, a two-view depth estimation 
model (Unimatch)~\citep{xu2023unifying}, 
and a monocular normal estimation model (DSINE)~\citep{bae2024rethinking}.
We adapt the scale and shift-invariant loss in Midas~\citep{birkl2023midas} for
monocular depth supervision and L1 loss for two-view depth supervison.
%
We use Sea Raft~\citep{wang2025sea} as our default optical flow model.
%
The comparison results on Mushroom dataset 
are shown in ~\tabref{tab:analysis_prior}.
We observe that the normal prior provides accurate shape information, 
enhancing the geometric quality of the radiance field. 
%
% In contrast, the monocular depth prior slightly increases 
% the 'Abs Rel' due to its ambiguous scale and inaccurate depth ordering.
% Moreover, the performance of monocular depth estimation 
% in the sauna scene is particularly poor, 
% primarily due to the presence of numerous reflective 
% surfaces and textureless walls, which limits the accuracy of monocular depth estimation.
%
The multi-view depth prior, hindered by the limited feature overlap 
between input views, fails to offer reliable geometric 
information. We test average "Abs Rel" of multi-view depth prior
, and the result is 0.19, which performs worse than the "Abs Rel" results 
rendered by original 2DGS.
From the results, it can be seen that depth order information provided by monocular depth improves
reconstruction accuracy. Meanwhile, our FDS achieves the best performance among all the priors, 
and by integrating all
three components, we obtained the optimal results.
%
%
\begin{figure}[t] \centering
    \makebox[0.16\textwidth]{\scriptsize RF (16000 iters)}
    \makebox[0.16\textwidth]{\scriptsize RF* (20000 iters)}
    \makebox[0.16\textwidth]{\scriptsize RF (20000 iters)  }
    \makebox[0.16\textwidth]{\scriptsize PF (16000 iters)}
    \makebox[0.16\textwidth]{\scriptsize PF (20000 iters)}


    % \includegraphics[width=0.16\textwidth]{figure/fig5_img/compare1/16000.png}
    % \includegraphics[width=0.16\textwidth]{figure/fig5_img/compare1/20000_wo_flow_loss.png}
    % \includegraphics[width=0.16\textwidth]{figure/fig5_img/compare1/20000.png}
    % \includegraphics[width=0.16\textwidth]{figure/fig5_img/compare1/16000_prior.png}
    % \includegraphics[width=0.16\textwidth]{figure/fig5_img/compare1/20000_prior.png}\\

    % \includegraphics[width=0.16\textwidth]{figure/fig5_img/compare2/16000.png}
    % \includegraphics[width=0.16\textwidth]{figure/fig5_img/compare2/20000_wo_flow_loss.png}
    % \includegraphics[width=0.16\textwidth]{figure/fig5_img/compare2/20000.png}
    % \includegraphics[width=0.16\textwidth]{figure/fig5_img/compare2/16000_prior.png}
    % \includegraphics[width=0.16\textwidth]{figure/fig5_img/compare2/20000_prior.png}\\

    \includegraphics[width=0.16\textwidth]{figure/fig5_img/compare3/16000.png}
    \includegraphics[width=0.16\textwidth]{figure/fig5_img/compare3/20000_wo_flow_loss.png}
    \includegraphics[width=0.16\textwidth]{figure/fig5_img/compare3/20000.png}
    \includegraphics[width=0.16\textwidth]{figure/fig5_img/compare3/16000_prior.png}
    \includegraphics[width=0.16\textwidth]{figure/fig5_img/compare3/20000_prior.png}\\
    
    \includegraphics[width=0.16\textwidth]{figure/fig5_img/compare4/16000.png}
    \includegraphics[width=0.16\textwidth]{figure/fig5_img/compare4/20000_wo_flow_loss.png}
    \includegraphics[width=0.16\textwidth]{figure/fig5_img/compare4/20000.png}
    \includegraphics[width=0.16\textwidth]{figure/fig5_img/compare4/16000_prior.png}
    \includegraphics[width=0.16\textwidth]{figure/fig5_img/compare4/20000_prior.png}\\

    \includegraphics[width=0.30\textwidth]{figure/fig5_img/bar.png}

    \caption{\textbf{The error map of Radiance Flow and Prior Flow.} RF: Radiance Flow, PF: Prior Flow, * means that there is no FDS loss supervision during optimization.}
    \label{fig:error_map}
\end{figure}




\textbf{Ablation study on FDS: }
In this section, we present the design of our FDS 
method through an ablation study on the 
Mushroom dataset to validate its effectiveness.
%
The optional configurations of FDS are outlined in ~\tabref{tab:ablation_fds}.
Our base model is the 2DGS equipped with FDS,
and its results are shown 
in the first row. The goal of this analysis 
is to evaluate the impact 
of various strategies on FDS sampling and loss design.
%
We observe that when we 
replace $I_i$ in \eqref{equ:mflow} with $C_i$, 
as shown in the second row, the geometric quality 
of 2DGS deteriorates. Using $I_i$ instead of $C_i$ 
help us to remove the floaters in $\bm{C^s}$, which are also 
remained in $\bm{C^i}$.
We also experiment with modifying the FDS loss. For example, 
in the third row, we use the neighbor 
input view as the sampling view, and replace the 
render result of neighbor view with ground truth image of its input view.
%
Due to the significant movement between images, the Prior Flow fails to accurately 
match the pixel between them, leading to a further degradation in geometric quality.
%
Finally, we attempt to fix the sampling view 
and found that this severely damaged the geometric quality, 
indicating that random sampling is essential for the stability 
of the mean error in the Prior flow.



\begin{table}[t] \centering

\begin{minipage}[t]{1.0\linewidth}
        \captionof{table}{\textbf{Ablation study on FDS strategies.}}
        \label{tab:ablation_fds}
        \resizebox{\textwidth}{!}{
\begin{tabular}{c|c|c|c|c|c|c|c}
    \hline
    \multicolumn{2}{c|}{$\mathcal{M}_{\theta}(X, \bm{C^s})$} & \multicolumn{3}{c|}{Loss} & \multicolumn{3}{c}{Metric}  \\
    \hline
    $X=C^i$ & $X=I^i$  & Input view & Sampled view     & Fixed Sampled view        & Abs Rel $\downarrow$ & F-score $\uparrow$ & NC $\uparrow$ \\
    \hline
    & \ding{51} &     &\ding{51}    &    &    \textbf{0.0561}        &  \textbf{0.6974}         & \textbf{0.8151}\\
    \hline
     \ding{51} &           &     &\ding{51}    &    &    0.0839        &  0.6242         &0.8030\\
     &  \ding{51} &   \ding{51}  &    &    &    0.0877       & 0.6091        & 0.7614 \\
      &  \ding{51} &    &    & \ding{51}    &    0.0724           & 0.6312          & 0.8015 \\
\bottomrule
\end{tabular}
}
\end{minipage}
\end{table}




\begin{figure}[htbp] \centering
    \makebox[0.22\textwidth]{}
    \makebox[0.22\textwidth]{}
    \makebox[0.22\textwidth]{}
    \makebox[0.22\textwidth]{}
    \\

    \includegraphics[width=0.22\textwidth]{figure/fig6_img/l1/rgb/frame00096.png}
    \includegraphics[width=0.22\textwidth]{figure/fig6_img/l1/render_rgb/frame00096.png}
    \includegraphics[width=0.22\textwidth]{figure/fig6_img/l1/render_depth/frame00096.png}
    \includegraphics[width=0.22\textwidth]{figure/fig6_img/l1/depth/frame00096.png}

    % \includegraphics[width=0.22\textwidth]{figure/fig6_img/l2/rgb/frame00112.png}
    % \includegraphics[width=0.22\textwidth]{figure/fig6_img/l2/render_rgb/frame00112.png}
    % \includegraphics[width=0.22\textwidth]{figure/fig6_img/l2/render_depth/frame00112.png}
    % \includegraphics[width=0.22\textwidth]{figure/fig6_img/l2/depth/frame00112.png}

    \caption{\textbf{Limitation of FDS.} }
    \label{fig:limitation}
\end{figure}


% \begin{figure}[t] \centering
%     \makebox[0.48\textwidth]{}
%     \makebox[0.48\textwidth]{}
%     \\
%     \includegraphics[width=0.48\textwidth]{figure/loss_Ignatius.pdf}
%     \includegraphics[width=0.48\textwidth]{figure/loss_family.pdf}
%     \caption{\textbf{Comparison the photometric error of Radiance Flow and Prior Flow:} 
%     We add FDS method after 2k iteration during training.
%     The results show
%     that:  1) The Prior Flow is more precise and 
%     robust than Radiance Flow during the radiance 
%     optimization; 2) After adding the FDS loss 
%     which utilize Prior 
%     flow to supervise the Radiance Flow at 2k iterations, 
%     both flow are more accurate, which lead to
%     a mutually reinforcing effects.(TODO fix it)} 
%     \label{fig:flowcompare}
% \end{figure}






\textbf{Interpretive Experiments: }
To demonstrate the mutual refinement of two flows in our FDS, 
For each view, we sample the unobserved 
views multiple times to compute the mean error 
of both Radiance Flow and Prior Flow. 
We use Raft~\citep{teed2020raft} as our default optical flow model
for visualization.
The ground truth flow is calculated based on 
~\eref{equ:flow_pose} and ~\eref{equ:flow} 
utilizing ground truth depth in dataset.
We introduce our FDS loss after 16000 iterations during 
optimization of 2DGS.
The error maps are shown in ~\figref{fig:error_map}.
Our analysis reveals that Radiance Flow tends to 
exhibit significant geometric errors, 
whereas Prior Flow can more accurately estimate the geometry,
effectively disregarding errors introduced by floating Gaussian points. 

%





\subsection{Limitation and further work}

Firstly, our FDS faces challenges in scenes with 
significant lighting variations between different 
views, as shown in the lamp of first row in ~\figref{fig:limitation}. 
%
Incorporating exposure compensation into FDS could help address this issue. 
%
 Additionally, our method struggles with 
 reflective surfaces and motion blur,
 leading to incorrect matching. 
 %
 In the future, we plan to explore the potential 
 of FDS in monocular video reconstruction tasks, 
 using only a single input image at each time step.
 


\section{Conclusions}
In this paper, we propose Flow Distillation Sampling (FDS), which
leverages the matching prior between input views and 
sampled unobserved views from the pretrained optical flow model, to improve the geometry quality
of Gaussian radiance field. 
Our method can be applied to different approaches (3DGS and 2DGS) to enhance the geometric rendering quality of the corresponding neural radiance fields.
We apply our method to the 3DGS-based framework, 
and the geometry is enhanced on the Mushroom, ScanNet, and Replica datasets.

\section*{Acknowledgements} This work was supported by 
National Key R\&D Program of China (2023YFB3209702), 
the National Natural Science Foundation of 
China (62441204, 62472213), and Gusu 
Innovation \& Entrepreneurship Leading Talents Program (ZXL2024361)
\section{Conclusion}
We introduce a novel approach, \algo, to reduce human feedback requirements in preference-based reinforcement learning by leveraging vision-language models. While VLMs encode rich world knowledge, their direct application as reward models is hindered by alignment issues and noisy predictions. To address this, we develop a synergistic framework where limited human feedback is used to adapt VLMs, improving their reliability in preference labeling. Further, we incorporate a selective sampling strategy to mitigate noise and prioritize informative human annotations.

Our experiments demonstrate that this method significantly improves feedback efficiency, achieving comparable or superior task performance with up to 50\% fewer human annotations. Moreover, we show that an adapted VLM can generalize across similar tasks, further reducing the need for new human feedback by 75\%. These results highlight the potential of integrating VLMs into preference-based RL, offering a scalable solution to reducing human supervision while maintaining high task success rates. 

\section*{Impact Statement}
This work advances embodied AI by significantly reducing the human feedback required for training agents. This reduction is particularly valuable in robotic applications where obtaining human demonstrations and feedback is challenging or impractical, such as assistive robotic arms for individuals with mobility impairments. By minimizing the feedback requirements, our approach enables users to more efficiently customize and teach new skills to robotic agents based on their specific needs and preferences. The broader impact of this work extends to healthcare, assistive technology, and human-robot interaction. One possible risk is that the bias from human feedback can propagate to the VLM and subsequently to the policy. This can be mitigated by personalization of agents in case of household application or standardization of feedback for industrial applications. 

\section*{Acknowledgements}
This work was supported by the Central Guidance for Local Special Project (Grant No. Z231100005923044).



{
    \small
    \bibliographystyle{ieeenat_fullname}
    \bibliography{main}
}

\newpage

\appendix



We first describe detailed processes of building the prompt library and sampling strategies in our method:
\begin{itemize}
    \item \cref{sec_supp: build_prompt_library}: Details of building prompt library.
    \item \cref{sec_supp: prompt_sampling}: Details of prompt sampling strategy.
    \item \cref{sec_supp: group_sampling}: Details of group sampling strategy.
\end{itemize}

Then, we show more experiments to show the effectiveness of our ProAPO:
\begin{itemize}
    \item \cref{sec_supp: implement_details}: More implementation details.
    \item \cref{sec_supp: different_backbone_result}: Results on different backbones.
    \item \cref{supp_sec: more_comparison_with_sota_methods}: More comparisons with SOTA methods.
    \item \cref{supp_sec: ablation_template_and_description}: Ablation of progressive optimization.
    \item \cref{supp_sec: more_ablation_operator}: More ablation of operators.
    \item \cref{supp_sec: more_ablation_group_sampling}: More ablation of group sampling.
    \item \cref{supp_sec: ablation_of_cost_computation}: Ablation of cost computation.
    \item \cref{supp_sec: effect_shots}: Effect of shot numbers.
    \item \cref{supp_sec: effect_alpha}: Effect of scalar $\alpha$ in score function.
    \item \cref{supp_sec: effect_sampled_numbers}: Effect of sampled numbers in prompt sampling.
    \item \cref{supp_sec: effect_of_quality_of_prompt_library}: Effect of quality of prompt library.
    \item \cref{sec_supp: more_qualitative_result}: More qualitative results.
\end{itemize}


We also provide detailed results for experiments appearing in the main paper: 
\begin{itemize}
    \item \cref{sec_supp: transfer_to_adapter}: Results of transfering to adapter-based methods.
    \item \cref{sec_supp: transfer_to_backbones}: Results of transferring to different backbones.
    \item \cref{supp_sec: ensemble_vs_single}: Analysis of single vs ensemble prompts.
    \item \cref{sec_supp: improve_description_methods}: Improvement by iterative optimization.
    \item \cref{supp_sec: ablation_operator}: Ablation of edit and evolution operators.
    \item \cref{supp_sec: ablation_two_sampling}: Ablation of two sampling strategies.
    \item \cref{supp_sec: effect_score_func}: Ablation of different score functions.
\end{itemize}



\section{Details of Building Prompt Library}
\label{sec_supp: build_prompt_library}


% ---------------------------------------------------- % 
%             Template Library 的细节
% ---------------------------------------------------- %
\subsection{Details of Building Template Library}
The template library aims to collect a set of templates that provide task-specific contextual information, which can address issues of semantic ambiguity caused by class names. It contains processes for collecting templates, generating dataset domains, and adding dataset domains to templates.

% for subsequent prompt generation


% **************************** % 
% 直接借助 LLMs 生成 Template 的 Prompt
% **************************** %
\textbf{Collecting templates.} We utilize two ways to collect templates. First, pre-defined templates, such as Template-80~\cite{CLIP}, FILIP-8~\cite{FILIP}, and DEFILIP-6~\cite{DEFILIP} can be used. Second, similar to PN~\cite{P_N}, we query LLMs to create diverse templates by the following prompt:
\begin{quote}
    \makebox[\linewidth]{%
        \colorbox{lightblue}{%
            \hspace*{0mm} % Adjust left alignment
            \begin{minipage}{\dimexpr\linewidth+10\fboxsep\relax} % Adjust width
                % \fontsize{9pt}{10pt}\selectfont % Font settings
                ``Hi, ChatGPT! I would like your help to prompt for image classification using CLIP. As a human-level prompt engineer, your task is to create a set of Templates like the following for visual classification. For example: 
                \newline \newline
                a photo of a \{\}.''
            \end{minipage}%
        }%
    }
\end{quote}


% **************************** % 
%  生成 Dataset Type 的 Prompt
% **************************** %
\textbf{Generating dataset domain by LLMs.} Inspired by previous description-based methods~\cite{WaffleCLIP, VDT_2023_ICCV}, we query LLMs to generate dataset domain information to provide task-specific context. For this purpose, we use the prompt:
\begin{quote}
    \makebox[\linewidth]{%
        \colorbox{lightblue}{%
            \hspace*{0mm} % Adjust left alignment
            \begin{minipage}{\dimexpr\linewidth+10\fboxsep\relax} % Adjust width
                % \fontsize{9pt}{10pt}\selectfont % Font settings
                ``Hi, ChatGPT! I would like your help in generating dataset domain information for image classification based on the dataset paper. A few words are good. Please return directly without explanation. 
                \newline \newline
                \{\texttt{uploaded PDF}\}.''
            \end{minipage}%
        }%
    }
\end{quote}
Here, \{\texttt{uploaded PDF}\} represents the uploading of the paper of the dataset to LLMs. 
Generated dataset domain information is summarized in~\cref{supp_tab: domain_information}.


% **************************** % 
% 具体使用的 Dataset Domain
% **************************** %
{
\renewcommand{\arraystretch}{1.1} 
\begin{table}[htbp]
  \centering
  \resizebox{1.0\linewidth}{!}
    {
    \begin{tabularx}{0.56\textwidth}
        {l | X }  
        \toprule
        {\textbf{Dataset}}  & \textbf{Domain Information} \\
        \midrule
        IN-1K~\cite{Imagenet} & real scenario; natural scene \\
        Caltech~\cite{caltech101} & object; everyday objects; common items \\
        Cars~\cite{Cars} & car; vehicles; auto-mobile \\ 
        CUB~\cite{CUB} & bird; wildlife; ornithology  \\
        DTD~\cite{DTD} & textures; patterns; surface; material \\
        ESAT~\cite{EuroSAT} & land cover; remote sensing; satellite photo; satellite imagery; aerial or satellite images; centered satellite photo \\
        FGVC~\cite{FGVC} & aircraft; airplane; plane; airliner \\ 
        FLO~\cite{FLO} & flower; floral; botanical; bloom \\
        Food~\cite{Food101} & food; dishes; cuisine; nourishment \\ 
        Pets~\cite{oxford_pets} & pet; domestic animals; breed; dog or cat \\ 
        Places~\cite{Places365} & place; scene \\ 
        SUN~\cite{SUN} &  place; scene \\ 
        UCF~\cite{UCF101} & action; human action; human activities; person doing \\ 
        \bottomrule
    \end{tabularx}
}
\vspace{-6pt}
  \caption{\textbf{Generated dataset domain information.}}
% \vspace{-10pt}
  \label{supp_tab: domain_information}
\end{table}
}

% **************************** % 
% 将 Dataset Type 与 Template 融合的方式
% **************************** %
\textbf{Adding dataset domain to templates.} We supplement templates with dataset domain information in the following four ways: (1) Add ``a type of \{\texttt{domain}\}''. (2) Replace ``\{\texttt{class}\}'' with ``\{\texttt{domain}\}:\{\texttt{class}\}''. (3) Replace ``photo'' with ``\{\texttt{domain}\}''. (4) Replace ``photo'' with ``\{\texttt{domain}\} photo''. Taking ``a photo of a \{\texttt{class}\}'' as an example, we modify the templates with the above four ways to add dataset domain information as follows:
\begin{quote}
    \makebox[\linewidth]{%
        \colorbox{lightblue}{%
            \hspace*{0mm} % Adjust left alignment
            \begin{minipage}{\dimexpr\linewidth+10\fboxsep\relax} % Adjust width
                % \fontsize{9pt}{10pt}\selectfont % Font settings
                \begin{enumerate}
                    \item a photo of a \{\texttt{class}\}, a type of \{\texttt{domain}\}.
                    \item a photo of a \{\texttt{domain}\}: \{\texttt{class}\}.
                    \item a \{\texttt{domain}\} of a \{\texttt{class}\}.
                    \item a \{\texttt{domain}\} photo of a \{\texttt{class}\}.
                \end{enumerate}
            \end{minipage}%
        }%
    }
\end{quote}
Here, \{\texttt{class}\} and \{\texttt{domain}\} denote category name and dataset domain information, respectively.


% ---------------------------------------------------- % 
%             Description Library 的细节
% ---------------------------------------------------- %
\subsection{Details of Building Description Library}
\label{supp_sec: build_description_library}

Description Library aims to provide a set of visual descriptions for each category, enhancing visual semantics for fine-grained recognition in prompts. It contains processes for generating visual descriptions and category synonyms and integrating descriptions with the best templates.

% , including CuPL~\cite{CuPL}, DCLIP~\cite{DCLIP}, GPT4Vis~\cite{GPT4Vis}, and AdaptCLIP~\cite{AdaptCLIP}.


% **************************** % 
%      生成 Description 的 Prompt
% **************************** %
{
\renewcommand{\arraystretch}{1.1} 
\begin{table}[htbp]
  \centering
  \resizebox{1.0\linewidth}{!}
    {
    \begin{tabularx}{0.56\textwidth}
        {l | X }  
        \toprule
        {\textbf{Method}}  & \textbf{Prompts} \\
        \midrule
        DCLIP~\cite{DCLIP} & Q: What are useful visual features for distinguishing a \{\texttt{class}\} in a photo? \\
        & A: There are several useful visual features to tell there is a \{\texttt{class}\} in a photo: \\
        
        \midrule 
        CuPL-Base~\cite{CuPL} & Describe what a \{\texttt{class}\} looks like. \\ 
        & Describe a \{\texttt{class}\}. \\
        & What are the identifying characteristics of a \{\texttt{class}\}? \\
        
        \midrule

        CuPL-Full~\cite{CuPL} & Describe what a \{\texttt{class}\} looks like. \\ 
        & How can you identify a \{\texttt{class}\}? \\ 
        & What does a \{\texttt{class}\} look like? \\
        & Describe an image from the internet of a \{\texttt{class}\}\\
        & A caption of an image of a \{\texttt{class}\}: \\
        
        \midrule
        GPT4Vis~\cite{GPT4Vis} & I want you to act as an image description expert. I will give you a word and your task is to give me 20 sentences to describe the word. Your description must accurately revolve around this word and be as objective, detailed and diverse as possible. In addition, the subject of your description is a some kind of object photograph. Output the sentences in a json format which key is the the word and the value is a list composed of these sentences. Do not provide any explanations. The first word is ``\{\texttt{class}\}". \\ 

        \midrule 
        AdaptCLIP~\cite{AdaptCLIP} & What characteristics can be used to differentiate \{\texttt{class}\} from other \{\texttt{domain}\} based on just a photo? Provide an exhaustive list of all attributes that can be used to identify the \{\texttt{domain}\} uniquely. Texts should be of the form “\{\texttt{domain}\} with \{\texttt{characteristic}\}”. \\
        \bottomrule
    \end{tabularx}
}
% \vspace{-6pt}
  \caption{\textbf{Prompts for generating visual descriptions.}}
% \vspace{-10pt}
  \label{supp_tab: generate_description}
\end{table}
}


% **************************** % 
%      生成同义词的 Prompt
% **************************** %
\textbf{Generating category synonym}.
Except for descriptions, we also replace class names from the dataset with their synonyms to create diverse class-specific prompts. For this purpose, we use the following prompt to ask LLMs to generate category synonyms:
\begin{quote}
    \makebox[\linewidth]{%
        \colorbox{lightblue}{%
            \hspace*{0mm} % Adjust left alignment
            \begin{minipage}{\dimexpr\linewidth+10\fboxsep\relax} % Adjust width
                % \fontsize{9pt}{10pt}\selectfont % Font settings
                ``Hi, ChatGPT! I would like your help in generating category synonyms. As a \{\texttt{domain}\} expert, I will provide you with a category name. Your task is to provide synonyms for the current category. If it has subclasses, return them as well. Please return directly without explanation.
                \newline \newline 
                User: I want to give the synonyms of \{\texttt{class}\}. 
                \newline
                Assistant: ''
            \end{minipage}%
        }%
    }
\end{quote}

% **************************** % 
% 各个数据集使用的询问 LLMs 生成 Description 的 Prompt
% **************************** %
\textbf{Generating visual descriptions for each category}.
Similar to previous description methods~\cite{CuPL, DCLIP, GPT4Vis, AdaptCLIP}, we instruct LLM to generate visual descriptions for each category by several prompts, which are summarized in~\cref{supp_tab: generate_description}.


% **************************** % 
%  将 Description 和 Template 整合
% **************************** %
\textbf{Integrating descriptions with the best templates}.
We use the following prompt to integrate descriptions with templates: ``\{\texttt{template}\}. \{\texttt{description}.\}''. 


% **************************** % 
%  每个组迭代的描述库
% **************************** %
After the above processes, we collect diverse visual descriptions for each category $c$, denoted as $\text{VD}(c)$. For each group iteration, we select the descriptions for categories in the specific group as the description library. Moreover, the prompt sampling strategy also utilizes these descriptions for class-specific initialization.

% ---------------------------------------------------- % 
%              Prompt Sampling 策略的细节
% ---------------------------------------------------- %
\section{Details of Prompt Sampling Strategy}
\label{sec_supp: prompt_sampling} 
The detailed prompt sampling strategy is summarized in Alg.~\ref{supp_alg: prompt_strategy}. Visual descriptions of each class $\text{VD}(c)$ are collected by the above process (see~\cref{supp_sec: build_description_library}). We utilize the candidate prompt $P_t^*$ with the best templates as an initial point. The $\textsc{RandomSample}(\cdot)$ operator denotes randomly selecting a set of elements from a given set. We randomly sample descriptions for each category to create multiple candidate prompts (Lines 2-8). After $T_{sample}$-times steps, we select the candidate prompt $\hat{P}_0$ with the highest score for description initialization (Line 9). It ensures that subsequent optimization is around the optimal initial point. We set $T_{sample} = 32 $ for all datasets in the default setting.

% Notably, descriptions generated by LLMs in this strategy are also used as the specific-group description library.


% **************************** % 
%    Prompt Sampling 的具体算法
% **************************** %
\begin{algorithm}[htbp]
\caption{Prompt Sampling Strategy.}
\label{supp_alg: prompt_strategy}
\begin{algorithmic}[1]
\REQUIRE $\mathcal{D} \leftarrow \{{(x, y)}\}_n$: training samples, $F:  \mathcal{D} \times P \to \mathbb{R}$: score function, $\mathcal{C}$: class labels, $\text{VD}(c)$: visual descriptions of class $c$, $P_t^*$: the prompt candidate with the best template
\STATE $\mathcal{U} \leftarrow \{P_t^*\} $ 
\FOR{$i=1$ to $T_{sample}$}
    \STATE $P_i \leftarrow P_t^* $
    \FORALL{class $c \in \mathcal{C}$}
        \STATE $P_i \leftarrow P_i \cup \textsc{RandomSample}(\text{VD}(c))$
    \ENDFOR
    \STATE $\mathcal{U} \leftarrow \mathcal{U} \cup \{ P_i \} $ 
\ENDFOR
\STATE $\hat{P}_0 \leftarrow \arg\max_{{P} \in \mathcal{U}} F(\mathcal{D}, {P})$ 
\RETURN the candidate prompt with the highest score $\hat{P}_0$
\end{algorithmic}
\end{algorithm}


% $T_{repeat}$: repeated times, 


% **************************** % 
%    Group Sampling 的具体算法
% **************************** %
\begin{algorithm}[htbp]
\caption{Group Sampling Strategy.}
\label{supp_alg: group_strategy}
\begin{algorithmic}[1]
\REQUIRE $\mathcal{D} \leftarrow \{{(x, y)}\}_n$: training samples, $F:  \mathcal{D} \times P \to \mathbb{R}$: score function, $\mathcal{C}$: class labels, $\text{VD}(c)$: visual descriptions of class $c$, $P_t^*$: prompt candidate with the best template, $\text{pred}(x)$: prediction for image $x$
\FORALL{class $c \in \mathcal{C}$}
    \STATE $\textsc{MisClass}(c) \leftarrow \emptyset $
\ENDFOR
\FORALL{training sample $(x, y) \in \mathcal{D}$}
    \IF{$\text{pred}(x) \neq y $}
        \STATE $\textsc{MisClass}(y) \leftarrow \textsc{MisClass}(y) \cup \{\text{pred}(x)\} $
    \ENDIF
\ENDFOR

\FORALL{class $c \in \mathcal{C}$}
    \STATE \textbf{Select Class Images}: $\textsc{Data} (c) \leftarrow \{ (x, y) \; | \; y = c\}_{(x, y) \in \mathcal{D}}$
    \STATE \textbf{Compute Accuracy}: $\textsc{Acc} (c) \leftarrow F( \textsc{Data} (c), P^*_t )$
    \STATE \textbf{Add Descriptions}: $P_{c} \leftarrow P^*_t \cup \text{VD}(c) $
    \STATE \textbf{Compute Accuracy Gain}: $\textsc{AccGain} (c) \leftarrow F( \textsc{Data} (c), P_c ) -  \textsc{Acc} (c) $
\ENDFOR
\STATE \textbf{Sort Class by Accuracy}: $\mathcal{C}_{wst}$, retaining the classes with the lowest top-$n_{wst}$ accuracy
\STATE \textbf{Sort Class by Accuracy Gain}: $\mathcal{C}_{sln}$, retaining the classes with the top-$n_{sln}$ accuracy gain
\STATE \textbf{Initialize Group Set}: $\mathcal{G} \leftarrow \emptyset$
\FORALL{class $c \in \mathcal{C}_{wst}$}
    \STATE $\mathcal{G} \leftarrow \mathcal{G} \cup \{ \textsc{MisClass}(y) \}$
\ENDFOR
\FORALL{class $c \in \mathcal{C}_{sln}$}
    \STATE $\mathcal{G} \leftarrow \mathcal{G} \cup \{ \textsc{MisClass}(y) \}$
\ENDFOR
\RETURN sampled groups $\mathcal{G}$
\end{algorithmic}
\end{algorithm}



% ---------------------------------------------------- % 
%               Group Sampling 策略的细节
% ---------------------------------------------------- %
\section{Details of Group Sampling Strategy}
\label{sec_supp: group_sampling}
The detailed group sampling strategy is summarized in Alg.~\ref{supp_alg: group_strategy}.
It contains processes of obtaining misclassified categories and selecting the worst and salient groups.

% **************************** % 
% misclassified category 获取的算法
% **************************** %
\noindent \textbf{Obtaining misclassified categories}.
In Lines 1-8 of Alg.~\ref{supp_alg: group_strategy}, we collect misclassified set for each category by $\textsc{MisClass}(\cdot)$ operator. Given an image $x$, if the prediction $\text{pred}(x)$ is not its corresponding label $y$, we will add $\text{pred}(x)$ to the misclassified set for category $y$. In fact, we also ablate the K-means clustering algorithm to group categories (in~\cref{supp_sec: more_ablation_group_sampling}). Results show that the misclassified set achieves better performance than the K-means algorithm.

% **************************** % 
%       挑选最差组别的算法
% **************************** %
\noindent \textbf{Selecting the worst groups} aims to select categories with the lowest top-$n_{wst}$ accuracy and corresponding misclassified categories. We first compute the accuracy for each category in Line 11. Then, we sort the categories by accuracy and retain the top-$n_{wst}$ worst categories in Line 15. Finally, $n_{wst}$ groups are added to the set $\mathcal{G}$ in Lines 18-20.


% **************************** % 
%       挑选显著组别的算法
% **************************** %
\noindent \textbf{Selecting the salient groups} aims to select categories with the top-$n_{sln}$ performance gains and its misclassified categories after adding descriptions. In Line 13, we compute the accuracy gains after adding the descriptions. Then, we sort the categories by accuracy gain and retain the top-$n_{sln}$ accuracy gain categories in Line 16. At last, $n_{sln}$ groups are added to the set $\mathcal{G}$ in Lines 21-23.

Finally, we collect $S = n_{wst} + n_{sln}$ groups for subsequent description optimization.





\section{More Implementation Details}
\label{sec_supp: implement_details}


\subsection{Hyperparameter Settings}
In~\cref{supp_tab: exp_details}, we show the searched hyperparameter settings for thirteen datasets. All results are average with four seeds. Except for $1, 2, 3$ as seeds like CoOp~\cite{CoOp}, we add $42$ as our fourth seed to further evaluate the stability of our method. In the default setting, we use the same LLMs as the description methods, \textit{i.e.}, GPT-3~\cite{GPT3} for CuPL~\cite{CuPL} and DCLIP~\cite{DCLIP}, GPT-4~\cite{GPT4_Tech} for GPT4Vis~\cite{GPT4Vis} and AdaptCLIP~\cite{AdaptCLIP}. 

% Our codes are available at \href{https://anonymous.4open.science/r/ProAPO}{https://anonymous.4open.science/r/ProAPO}. We will release our code and optimized prompts after the review stage.


{
\renewcommand{\arraystretch}{1.1} 
% \setlength{\tabcolsep}{3.pt}
\begin{table}[htbp]
  \centering
  \resizebox{1.0\linewidth}{!}
    {
    \begin{tabular}
        {l | c | c | c | c | c | c | c  }  
        \toprule
        {\textbf{Dataset}}  & $T$ &  $M$ & $N$ & $\alpha$ & $n_{wst}$ & $n_{sln}$ & $T_{sample}$ \\
        \midrule
        IN-1K~\cite{Imagenet} & 4 & 8 & 8 & 1e3 & 4 & 4 & 32 \\
        Caltech~\cite{caltech101} & 2 & 8 & 8 & 1e2 & 2 & 2 & 32 \\
        Cars~\cite{Cars} & 4 & 8 & 8 & 1e4 & 4 & 4 & 32 \\ 
        CUB~\cite{CUB} & 4 & 8 & 8 & 1e2 & 4 & 4 & 32 \\
        DTD~\cite{DTD} & 4 &  8 & 8 & 1e3 & 4 & 4 & 32 \\
        ESAT~\cite{EuroSAT} & 4 &  8 &  8 & 1e3 & 3 & 3 & 32 \\
        FGVC~\cite{FGVC} & 4 & 8 & 8 & 1e3 & 4 & 4 & 32 \\ 
        FLO~\cite{FLO} & 4 & 8 & 8 & 1e3 & 4 & 4 & 32 \\
        Food~\cite{Food101} & 4 &  8 & 8 & 1e3 & 2 & 2 & 32 \\ 
        Pets~\cite{oxford_pets} & 2 & 8 &  8 & 1e4 & 2 & 2 & 32 \\ 
        Places~\cite{Places365} & 4  & 8 &  8 & 1e2 & 3 & 3 & 32 \\ 
        SUN~\cite{SUN} & 2 &  8 & 8 & 1e4 & 4 & 4 & 32 \\ 
        UCF~\cite{UCF101} & 4 &  8 & 8 & 1e3 & 3 & 3 & 32 \\ 
        \bottomrule
    \end{tabular}
}
\vspace{-6pt}
  \caption{\textbf{Hyperparameters settings for thirteen datasets.}}
\vspace{-10pt}
  \label{supp_tab: exp_details}
\end{table}
}


\subsection{More Related Work}
\noindent 
\textbf{Large-scale vision-language models}
like CLIP~\cite{CLIP} have shown promising performance on various tasks. They align visual and textual spaces to a joint space via training on millions of image-text pairs from the web. Other work~\cite{Align, DEFILIP, DeClip, FILIP, BLIP, Flamingo, SLIP, EVA-01, EVA-02} has furthered this paradigm to learn more accurate semantic alignment in joint space. 
In this work, we advance VLMs for downstream tasks by progressively learning optimal class-specific prompts with minimal supervision and no human intervention. 




% ---------------------------------------------------- % 
%                不同 Backbones 的实验
% ---------------------------------------------------- %
{
\renewcommand{\arraystretch}{1.1} 
\setlength{\tabcolsep}{3.8pt}

\begin{table*}[htbp]
  \centering
  \resizebox{0.98\linewidth}{!}
    {
    \begin{tabular}
        {l | ccccc ccccc ccc | c | c}
            
        \toprule
        \textbf{Module} & \rotatebox{90}{\textbf{IN-1K}} & \rotatebox{90}{\textbf{Caltech}} & \rotatebox{90}{\textbf{Cars}} & \rotatebox{90}{\textbf{CUB}} & \rotatebox{90}{\textbf{DTD}}  & \rotatebox{90}{\textbf{ESAT}} & \rotatebox{90}{\textbf{FGVC}} & \rotatebox{90}{\textbf{FLO}} & \rotatebox{90}{\textbf{Food}}  &  \rotatebox{90}{\textbf{Pets}} & \rotatebox{90}{\textbf{Places}} & \rotatebox{90}{\textbf{SUN}} & \rotatebox{90}{\textbf{UCF}} & \rotatebox{90}{\textbf{Avg (11)}} & \rotatebox{90}{\textbf{Avg (13)}} \\
        \midrule

        CLIP~\cite{CLIP} - ResNet50 &  57.9 & 84.5 & 53.9 & 44.7 & 38.8 & 28.6 & 15.9 & 60.2 & 74.0 & 83.2 & 38.2 & 58.0 & 56.9 & 55.6 & 53.4 \\ 
        CuPL~\cite{CuPL} &   61.2 & 88.3 & 55.3 & 48.7 & 49.5  & 38.2  & 18.9  & 67.0  & 80.1& 86.1& 41.2& 63.1  & 63.3   & 61.1  & 58.5 \\
        {\textbf{ProAPO} (ours)} & \textbf{61.5}  & \textbf{90.3} & \textbf{58.0} & \textbf{50.7} & \textbf{52.3} & \textbf{51.7} & \textbf{21.1} & \textbf{75.1} & \textbf{81.8} & \textbf{88.7} & \textbf{41.8} & \textbf{63.7} & \textbf{66.0} & \textbf{64.6}  & \textbf{61.8} \\

        $\Delta$ & \textcolor{retained}{+ 3.6} & \textcolor{retained}{+ 5.8} & \textcolor{retained}{+ 4.1} & \textcolor{retained}{+ 6.0} & \textcolor{retained}{+ 13.5} & \textcolor{retained}{+ 23.1} & \textcolor{retained}{+ 5.2} & \textcolor{retained}{+ 14.9} & \textcolor{retained}{+ 7.8} & \textcolor{retained}{+ 5.5} & \textcolor{retained}{+ 3.6} & \textcolor{retained}{+ 5.7} & \textcolor{retained}{+ 9.1} & \textcolor{retained}{+ 9.0} & \textcolor{retained}{+ 8.4}   \\

        \midrule

        CLIP~\cite{CLIP} - ResNet101 & 61.4  & 89.9  & 63.3  & 49.6  & 40.3  & 31.7  & 18.3  & 64.3  & 83.4  & 86.9  & 37.9  & 59.0  & 61.2  & 60.0  & 57.5  \\ 
        CuPL~\cite{CuPL} &  61.4  & 91.0  & 61.2  & 45.3  & 49.7  & 28.7  & 18.6  & 59.0  & 82.7  & 86.6  & \textbf{40.6}  & 62.3  & 56.4  & 59.8  & 57.2  \\
        {\textbf{ProAPO} (ours)} & \textbf{63.6} & \textbf{92.3} & \textbf{64.4} & \textbf{52.2} & \textbf{51.6} & \textbf{45.9} & \textbf{21.2} & \textbf{69.6} & \textbf{84.9} & \textbf{89.6} & \textbf{40.6} & \textbf{63.5} & \textbf{64.0} & \textbf{64.6}  & \textbf{61.8} \\
        $\Delta$ & \textcolor{retained}{+ 2.2} & \textcolor{retained}{+ 2.4} & \textcolor{retained}{+ 1.1} & \textcolor{retained}{+ 2.6} & \textcolor{retained}{+ 11.3} & \textcolor{retained}{+ 14.2} & \textcolor{retained}{+ 2.9} & \textcolor{retained}{+ 5.3} & \textcolor{retained}{+ 1.5} & \textcolor{retained}{+ 2.7} & \textcolor{retained}{+ 2.7} & \textcolor{retained}{+ 4.5} & \textcolor{retained}{+ 2.8} & \textcolor{retained}{+ 4.6} & \textcolor{retained}{+ 4.3} \\


        \midrule

        CLIP~\cite{CLIP} - ViT-B/32 & 62.1  & 91.2  & 60.4  & 51.7 & 42.9  & 43.9  & 20.2  & 66.0  & 83.2  & 86.8 & 39.9 & 62.1  & 60.9 & 61.8 & 59.3  \\
        CuPL~\cite{CuPL} &  {64.4}  & 92.9  & 60.7  & 53.3  & {50.6}  & 50.5  & 20.9  & 69.5  & 84.2  & 87.0  & {43.1}  & {66.3}  & 66.4  & 64.9  & 62.3  \\
        {\textbf{ProAPO} (ours)} & {\textbf{64.7}} & {\textbf{94.4}} & {\textbf{61.7}} & {\textbf{55.4}} & {\textbf{53.5}} & {\textbf{63.0}} & {\textbf{23.0}} & {\textbf{74.3}} & {\textbf{85.3}} & {\textbf{91.0}} & {\textbf{43.3}} & {\textbf{66.6}} & {\textbf{69.0}} & {\textbf{67.9}}  & {\textbf{65.0}} \\

        $\Delta$ & \textcolor{retained}{+ 2.6} & \textcolor{retained}{+ 3.2} & \textcolor{retained}{+ 1.3} & \textcolor{retained}{+ 3.7} & \textcolor{retained}{+ 10.6} & \textcolor{retained}{+ 19.1} & \textcolor{retained}{+ 2.8} & \textcolor{retained}{+ 8.3} & \textcolor{retained}{+ 2.1} & \textcolor{retained}{+ 4.2} & \textcolor{retained}{+ 3.4} & \textcolor{retained}{+ 4.5} & \textcolor{retained}{+ 8.1} & \textcolor{retained}{+ 6.1} & \textcolor{retained}{+ 5.7} \\

        \midrule

        CLIP~\cite{CLIP} - ViT-B/16 & 66.9  & 93.2  & 65.5  & 55.3  & 44.3  & 51.0  & 24.4  & 70.6  & 88.4  & 89.0  & 40.8  & 62.5  & 67.7  & 65.8  & 63.0  \\
        CuPL~\cite{CuPL} & 69.6  & 94.3  & 66.1  & 57.2  & 53.8  & 55.7  & 26.6  & 73.9  & 88.9  & 91.2  & 43.4  & \textbf{69.0}  & 70.3  & 69.0  & 66.1  \\
        {\textbf{ProAPO} (ours)} & \textbf{69.9} & \textbf{95.2} & \textbf{67.7} & \textbf{59.0} & \textbf{55.8} & \textbf{65.3} & \textbf{28.3} & \textbf{82.7} & \textbf{89.5} & \textbf{92.7} & \textbf{43.8} & {68.9} & \textbf{73.1} & \textbf{71.7}  & \textbf{68.6} \\
        $\Delta$ & \textcolor{retained}{+ 3.0} & \textcolor{retained}{+ 2.0} & \textcolor{retained}{+ 2.2} & \textcolor{retained}{+ 3.7} & \textcolor{retained}{+ 11.5} & \textcolor{retained}{+ 14.3} & \textcolor{retained}{+ 3.9} & \textcolor{retained}{+ 12.1} & \textcolor{retained}{+ 1.1} & \textcolor{retained}{+ 3.7} & \textcolor{retained}{+ 3.0} & \textcolor{retained}{+ 6.4} & \textcolor{retained}{+ 5.4} & \textcolor{retained}{+ 5.9} & \textcolor{retained}{+ 5.6} \\


        \midrule 


        CLIP~\cite{CLIP} - ViT-L/14 &  73.5  & 95.1  & 76.8  & 62.5  & 52.1  & 61.5  & 33.4  & 79.5  & 93.1  & 93.3  & 40.7  & 67.6  & 75.0   & 72.8  & 69.5 \\ 
        CuPL~\cite{CuPL} & 76.7  & 96.2 & 77.6 &  61.4 & 62.6 & 62.4 & 36.1  & 79.7  & 93.4 &  93.8 & 43.8 &  73.2  & 78.3  & 75.5  & 71.9   \\
        {\textbf{ProAPO} (ours)} & \textbf{76.8} & \textbf{97.1} & \textbf{78.8} & \textbf{65.1} & \textbf{64.8} & \textbf{74.3} & \textbf{38.3} & \textbf{87.3} & \textbf{93.9} & \textbf{94.6} & \textbf{44.4} & \textbf{73.4} & \textbf{80.1} & \textbf{78.1}  & \textbf{74.5} \\
        $\Delta$ & \textcolor{retained}{+ 3.3} & \textcolor{retained}{+ 2.0} & \textcolor{retained}{+ 2.0} & \textcolor{retained}{+ 2.6} & \textcolor{retained}{+ 12.7} & \textcolor{retained}{+ 12.8} & \textcolor{retained}{+ 4.9} & \textcolor{retained}{+ 7.8} & \textcolor{retained}{+ 0.8} & \textcolor{retained}{+ 1.3} & \textcolor{retained}{+ 3.7} & \textcolor{retained}{+ 5.3} & \textcolor{retained}{+ 5.1} & \textcolor{retained}{+ 5.8} & \textcolor{retained}{+ 5.0} \\

        \midrule 

        OpenCLIP~\cite{OpenCLIP} - ViT-B/32 &  66.2  & 94.7  & 88.2  & 65.6  & 51.3  & 49.4  & 23.0  & 71.2 & 82.4 & 90.7 & 41.5 & 68.1 & 65.0  & 68.2  & 65.9  \\ 
        CuPL~\cite{CuPL} &  66.7  & 94.4  & 86.6  & 65.9  & 62.4  & 50.1  & 25.5  & 69.5  & 81.7  & 90.8  & 43.3  & 69.1  & 65.8  & 69.3  & 67.1   \\
        {\textbf{ProAPO} (ours)} & \textbf{67.0} & \textbf{95.8} & \textbf{88.7} & \textbf{67.3} & \textbf{65.1} & \textbf{66.0} & \textbf{27.5} & \textbf{81.8} & \textbf{83.2} & \textbf{91.9} & \textbf{43.4} & \textbf{69.7} & \textbf{70.2} & \textbf{73.3}  & \textbf{70.6} \\
        $\Delta$ & \textcolor{retained}{+ 0.8} & \textcolor{retained}{+ 1.1} & \textcolor{retained}{+ 0.5} & \textcolor{retained}{+ 1.7} & \textcolor{retained}{+ 13.8} & \textcolor{retained}{+ 16.6} & \textcolor{retained}{+ 4.5} & \textcolor{retained}{+ 10.6} & \textcolor{retained}{+ 0.8} & \textcolor{retained}{+ 1.2} & \textcolor{retained}{+ 1.9} & \textcolor{retained}{+ 1.6} & \textcolor{retained}{+ 5.2} & \textcolor{retained}{+ 5.1} & \textcolor{retained}{+ 4.7} \\


        \midrule 

        EVA02~\cite{EVA-02} - ViT-B/16 & 74.6  & \textbf{97.2}  & 79.2  & 60.8  & 49.7  & 68.0  & 24.6  & 75.6  & 89.5  & 92.2  & 42.9  & 70.7 & 68.6  & 71.8  & 68.7  \\ 
        CuPL~\cite{CuPL} &  75.4  & 96.7  & 79.2  & 61.8  & 59.1  & 61.7  & 27.5  & 75.2  & 89.3  & 92.1  & 44.0  & 72.5  & 71.9  & 72.8  & 69.7 \\
        {\textbf{ProAPO} (ours)} & \textbf{75.5} & 97.0 & \textbf{80.0} & \textbf{62.8} & \textbf{61.3} & \textbf{74.2} & \textbf{29.7} & \textbf{89.1} & \textbf{89.6} & \textbf{93.5} & \textbf{44.5} & \textbf{72.5} & \textbf{75.2} & \textbf{76.2}  & \textbf{72.7}  \\
        $\Delta$ & \textcolor{retained}{+ 0.9} & -0.2 & \textcolor{retained}{+ 0.8} & \textcolor{retained}{+ 2.0} & \textcolor{retained}{+ 11.6} & \textcolor{retained}{+ 6.2} & \textcolor{retained}{+ 5.1} & \textcolor{retained}{+ 13.5} & \textcolor{retained}{+ 0.1} & \textcolor{retained}{+ 1.3} & \textcolor{retained}{+ 1.6} & \textcolor{retained}{+ 1.8} & \textcolor{retained}{+ 6.6} & \textcolor{retained}{+ 4.4} & \textcolor{retained}{+ 4.0}  \\

        \midrule 

        SigLIP~\cite{SigLIP} - ViT-B/16 & 75.8  & 97.3  & 90.5  & 62.3 & 62.8  & 44.6 & 43.6 & 85.5 & 91.5  & 94.1  & 41.6  & 69.5  & 74.9  & 75.5  & 71.8 \\ 
        CuPL~\cite{CuPL} &  76.0 & 98.0 & 90.5 & 63.0 & 64.9 & 42.8 & 45.1 & 87.0 & 90.7 & 94.5 & 43.5 & 69.9 & 73.4 & 75.7 & 72.3 \\
        {\textbf{ProAPO} (ours)} & \textbf{76.4} & \textbf{98.3} & \textbf{91.7} & \textbf{66.2} & \textbf{69.1} & \textbf{55.8} & \textbf{47.1} & \textbf{93.3} & \textbf{92.2} & \textbf{94.9} & \textbf{44.3} & \textbf{71.7} & \textbf{75.9} & \textbf{78.8}  & \textbf{75.2} \\
        $\Delta$ & \textcolor{retained}{+ 0.6} & \textcolor{retained}{+ 1.0} & \textcolor{retained}{+ 1.2} & \textcolor{retained}{+ 3.9} & \textcolor{retained}{+ 6.3} & \textcolor{retained}{+ 11.2} & \textcolor{retained}{+ 3.5} & \textcolor{retained}{+ 7.8} & \textcolor{retained}{+ 0.7} & \textcolor{retained}{+ 0.8} & \textcolor{retained}{+ 2.7} & \textcolor{retained}{+ 2.2} & \textcolor{retained}{+ 1.0} & \textcolor{retained}{+ 3.3} & \textcolor{retained}{+ 3.4} \\
        
        \bottomrule
    \end{tabular}
}
\vskip -0.04in
  \caption{\textbf{Results of our ProAPO on different backbones.} \textbf{Avg (11)} and \textbf{Avg (13)} denote average results across 11 datasets (excluding CUB~\cite{CUB} and Places~\cite{Places365}) and all 13 datasets, respectively. $\Delta$ denotes performance gains compared to vanilla VLMs.}
  \label{supp_tab: results_different_backbones}
  \vskip -0.15in
\end{table*}
}

\section{Results on Different Backbones}
\label{sec_supp: different_backbone_result}

% **************************** % 
%           实验设置
% **************************** %
\textbf{Settings}. In~\cref{supp_tab: results_different_backbones}, we show results of our ProAPO in different backbones, including ResNet50, ResNet101, ViT-B/32, ViT-B/16, ViT-L/14 for CLIP~\cite{CLIP}, ViT-B/32 for OpenCLIP~\cite{OpenCLIP}, ViT-B/16 for EVA02~\cite{EVA-02}, and ViT-B/16 for SigLIP~\cite{SigLIP}. We compare our ProAPO with vanilla VLMs and the SOTA description method CuPL~\cite{CuPL}.


% **************************** % 
%             结果
% **************************** %
\textbf{Results}. We see that our ProAPO consistently improves vanilla CLIP and CuPL in thirteen datasets across all backbones. Compared to vanilla VLMs, our ProAPO enhances them by at least 3.4\% average accuracy in thirteen datasets. Moreover, we see notable performance improvement on several fine-grained datasets, such as DTD~\cite{DTD}, ESAT~\cite{EuroSAT}, FLO~\cite{FLO}, and UCF~\cite{UCF101}. It further verifies that class-specific descriptions provide helpful knowledge for fine-grained recognition. Besides, iterative optimization by our ProAPO also enhances the description method CuPL. 


% **************************** % 
%         更多有趣的发现
% **************************** %
\textbf{More interesting findings}.
We find that as the backbones of VLMs become larger, the performance improvement by ProAPO gradually decreases. For example, from ViT-B/32 to ViT-B/16 to ViT-L/14, the gain for CLIP is from 5.7\% to 5.6\% to 5.0\%. Moreover, similar results appear in different models with the same backbone, \textit{i.e.}, the vanilla model with better results achieves a lower performance increase. For example, from CLIP~\cite{CLIP} to OpenCLIP~\cite{OpenCLIP} on ViT-B/32 backbone, the gain is from 5.7\% to 4.7\%, and from CLIP~\cite{CLIP} to EVA02~\cite{EVA-02} to SigLIP~\cite{SigLIP}, the gain is from 5.6\% to 4.0\% to 3.4\%. We argue that the model with the higher result has more knowledge, which may be affected less by prompt quality. Overall, our ProAPO continues to improve the performance of VLMs.

% ---------------------------------------------------- % 
%                    更多比较的实验
% ---------------------------------------------------- %
\section{More Comparisons with SOTA Methods}
\label{supp_sec: more_comparison_with_sota_methods}
In this section, we compare our ProAPO with more SOTA prompt tuning methods. These methods adapt VLMs from both visual and textual views.

{
\renewcommand{\arraystretch}{1.1} 
\setlength{\tabcolsep}{3.8pt}

\begin{table*}[htbp]
  \centering
  \resizebox{0.75\linewidth}{!}
    {
    \begin{tabular}
        {l | ccccc ccccc c | c}
            
        \toprule
        \textbf{Module} (ViT-B/16) & \rotatebox{90}{\textbf{IN-1K}} & \rotatebox{90}{\textbf{Caltech}} & \rotatebox{90}{\textbf{Cars}} & \rotatebox{90}{\textbf{DTD}}  & \rotatebox{90}{\textbf{ESAT}} & \rotatebox{90}{\textbf{FGVC}} & \rotatebox{90}{\textbf{FLO}} & \rotatebox{90}{\textbf{Food}}  &  \rotatebox{90}{\textbf{Pets}} & \rotatebox{90}{\textbf{SUN}} & \rotatebox{90}{\textbf{UCF}} & \rotatebox{90}{\textbf{Avg (11)}} \\
        \midrule

        Vanilla CLIP~\cite{CLIP} & 66.9  & 93.2  & 65.5 & 44.3  & 51.0  & 24.4  & 70.6  & 88.4  & 89.0   & 62.5  & 67.7  & 65.8  \\

        \midrule
        \multicolumn{13}{c}{\textit{\textbf{\ccol{Test-Time Prompt Tuning Methods}}}} \\
        \midrule 

        TPT~\cite{TPT} & 69.0 & 94.2 & 66.9 & 47.8 & 42.4 & 24.8 & 69.0 & 84.7 & 87.8 & 65.5 &  68.0 & 65.5   \\ 
        DiffTPT~\cite{DiffTPT} & 70.3 & 92.5 & 67.0 & 47.0 & 43.1 & 25.6 & 70.1 & 87.2 &  88.2 & 65.7 & 68.2 & 65.9    \\ 
        PromptAlign~\cite{PromptAlign} & 71.4 & 94.0 & 68.5 & 47.2 & 47.9 & 24.8 & 72.4 & 86.7 &  90.8 & 67.5 & 69.5 & 67.3  \\
        Self-TPT-v~\cite{Self_TPT_v} & \textbf{73.0} & 94.7 &  68.8 & 49.4 & 51.9 & 27.6 & 71.8 &  85.4 & 91.3 & 68.2 &  69.5 & 68.3  \\
        
        \midrule 
        \multicolumn{13}{c}{\textit{\textbf{\ccol{Vector-based Prompt Tuning Methods}}}} \\
        \midrule 
        UPT~\cite{prompt_tuning_UPT} & 69.6 & 93.7 & 67.6 & 45.0 & 66.5 & 28.4 & 75.0 & 84.2 & 82.9 & 68.8 & 72.0 & 68.5   \\
        CoCoOp~\cite{CoCoOp} & 69.4 & 93.8 & 67.2 & 48.5 & 55.3 & 12.7 & 72.1 & 85.7 & 91.3 & 68.3 & 70.3 & 66.8  \\
        MaPLe~\cite{MaPLe}  & 69.6 & 92.6 & 66.6 & 52.1 & 71.8 & 26.7 & 83.3 & 80.5 &  89.1 & 64.8 & 71.8 & 69.9   \\
        ALIGN~\cite{prompt_tuning_align} & 69.8 & 94.0 & 68.3 & 54.1 & 53.2 & 29.6 & 81.3 & 85.3 & 91.4 & 69.1 & 74.4 & 70.1   \\ 
        PromptSRC~\cite{PromptSRC} & 68.1 & 93.7 & 69.4 & 56.2 & \underline{73.1} & 27.7 & \underline{85.9} & 84.9 & 92.0 & 69.7 & 74.8 & 72.3    \\ 
        
        \midrule 
        \multicolumn{13}{c}{\textit{\textbf{\ccol{Description-Based Methods}}}} \\
        \midrule 
        
        \multicolumn{13}{l}{\textit{\textbf{w/o adapters}}} \\
        % ----- 同 Test-Time Adaptation 的比较 ----- %
        CuPL~\cite{CuPL} & 69.6  & 94.3  & 66.1   & 53.8  & 55.7  & 26.6  & 73.9  & 88.9  & 91.2  & 69.0  & 70.3  & 69.0  \\

        AWT-text~\cite{AWT} & 68.9  & 95.2  & 66.0   & 52.0  & 52.6  & 26.1  & 74.5  & 89.4  & 91.2   & 68.4  & 69.8 & 68.6  \\
        
        \highlight{\textbf{ProAPO} (ours)} & \highlight{69.9} & \highlight{95.2} & \highlight{67.7} & \highlight{55.8} & \highlight{65.3} & \highlight{28.3} & \highlight{82.7} & \highlight{89.5} & \highlight{92.7} & \highlight{68.9} & \highlight{73.1} & \highlight{71.7}  \\

        \highlight{\textbf{ProAPO} w/ AWT-text} & \highlight{69.4} & \highlight{\underline{95.3}} & \highlight{67.8} &  \highlight{54.3} & \highlight{67.1} & \highlight{27.4} & \highlight{82.1} & \highlight{\underline{89.6}} & \highlight{\underline{93.2}} & \highlight{68.5} & \highlight{73.1} & \highlight{71.6} \\

        \midrule 

         \multicolumn{13}{l}{\textit{\textbf{w/ adapters}}} \\

        AWT-Adapter~\cite{AWT} & \underline{72.1} & 95.1 & \textbf{73.4} & \underline{59.4} & \textbf{76.3} & \textbf{33.9} & 85.6 &  85.9 & 92.9 & \textbf{72.7} & \textbf{78.4} & \underline{75.1} \\
        
        \highlight{\textbf{ProAPO} w/ APE~\cite{APE}} & \highlight{71.3} & \highlight{\textbf{95.8}} & \highlight{\underline{70.9}} & \highlight{\textbf{60.6}} & \highlight{72.4} & \highlight{\underline{33.2}} & \highlight{\textbf{91.4}} & \highlight{\textbf{89.9}} & \highlight{\textbf{93.4}} & \highlight{\underline{71.0}} & \highlight{\underline{77.6}}  & \highlight{\textbf{75.2}} \\

        \bottomrule
    \end{tabular}
}
\vskip -0.04in
  \caption{\textbf{Comparison of our ProAPO with more SOTA methods under one-shot supervision.} \textbf{Avg (11)} denote average results across 11 datasets. }
  \label{supp_tab: comparison_with_more_SOTA}
  % \vskip -0.15in
\end{table*}
}


% **************************** % 
%      同 TPT 方法的对比
% **************************** %
% \noindent 
\textbf{Comparison of test-time prompt tuning methods}.
In~\cref{supp_tab: comparison_with_more_SOTA}, our ProAPO outperforms SOTA test-time prompt tuning methods on 11 datasets. Notably, we adapt VLMs solely from the textual view, while TPT methods introduce textual and visual views (\textit{i.e.}, augmented images), which further verifies the effectiveness of our method.


% **************************** % 
%   同更多 vector-based prompt tuning 方法的对比
% **************************** %
\textbf{Comparison of vector-based prompt-tuning methods}
 Since recent prompt-tuning methods adapt VLMs using both visual and textual views, we combine ProAPO with an adapter (\textit{i.e.}, APE~\cite{APE}) for a fair comparison. 
 \textbf{(1) Higher performance in low-shot.} In~\cref{supp_tab: comparison_with_more_SOTA}, ProAPO consistently outperforms these methods, which verifies that optimizing prompts in natural language is more effective in low-shot tasks. 
 \textbf{(2) Better transferability and interpretability}. Unlike vector-based prompt-tuning methods that search in a continuous space, ProAPO benefits from the discrete nature of natural language, leading to better interpretability and easily transfers across different backbones (shown in~\cref{tab: transfer_backbone}).
 \textbf{(3) Lower performance in high-shot}.  However, in~\cref{supp_tab: shots_influence}, ProAPO shows a sub-optimal result compared to CoOp~\cite{CoOp} in high-shot settings. This is due to the limited language search space and iteration steps.

% **************************** % 
%   同 AWT 方法的比较 
% **************************** %
\textbf{Comparison of AWT~\cite{AWT}}.
First, since AWT uses augmented visual and textual views to adapt VLMs, we compare ProAPO with AWT under the augmented textual view for a fair comparison. In~\cref{supp_tab: comparison_with_more_SOTA}, the result shows our ProAPO improves AWT-text by 6.1\% on average, verifying that our progressive optimization improves prompt quality. In addition, we introduce a common adapter-based method to our ProAPO and compare it with AWT-Adapter in the one-shot setting. We see that our ProAPO achieves comparable results. These results suggest that ProAPO and AWT are complementary.



% **************************** % 
%   同 iCM 方法的比较 
% **************************** %
\textbf{Comparison of iCM~\cite{iCM}}. iCM is somewhat similar to ours, optimizing class-specific prompts with chat-based LLMs. However, it uses the whole validation set as supervision. In~\cref{supp_tab: comparison_of_iCM}, we see that our ProAPO outperforms iCM significantly even under the one-shot supervision. 
This is because our ProAPO address challenges in class-specific prompt optimization by an offline generation algorithm to reduce LLM querying costs, an entropy-constrained fitness score to prevent overfitting, and two sampling strategies to find an optimal initial point and reduce iteration times.

% we solve high generation costs, long iteration times, and overfitting in class-specific optimization.



{
% \vspace{-10pt}
% \renewcommand{\arraystretch}{1.0} 
\begin{table*}[h]
  \centering
  \resizebox{0.8\linewidth}{!}
    {
    \begin{tabular}
        {l | ccccc ccc | c}  
        \toprule
        
        \textbf{Module} (ViT-B/32) & {\textbf{IN-1K}} & {\textbf{Caltech}} & {\textbf{CUB}} & {\textbf{DTD}}  & {\textbf{ESAT}} & {\textbf{FLO}} & {\textbf{SUN}} & {\textbf{UCF}} & \textbf{Avg (8)} \\
        \midrule
        Vanilla CLIP & 62.1 & 91.2 & 51.7 & 42.9 & 43.9 &  66.0 & 62.1 & 60.9 & 60.1 \\
        \midrule
        \multicolumn{10}{c}{\textit{\textbf{\ccol{Automatic Prompt Optimization Methods}}}} \\
        \midrule
        
        iCM~\cite{iCM} (w/ validation set) & 64.5 & 92.7 & \textbf{56.1} & 51.4 & 56.3 & 72.2 & 66.2 & 67.0 & 65.8  \\
        \highlight{\textbf{ProAPO} (w/ 1-shot)} &  \highlight{\textbf{64.7}} & \highlight{\textbf{94.4}} & \highlight{55.4} & \highlight{\textbf{53.5}} & \highlight{\textbf{63.0}} & \highlight{\textbf{74.3}} & \highlight{\textbf{66.6}} & \highlight{\textbf{69.0}} & \highlight{\textbf{67.6}} \\

        \bottomrule
    \end{tabular}
}
% \vspace{-5pt}
  \caption{\textbf{Comparison of our ProAPO with iCM~\cite{iCM}.} Avg (8) denotes average results across 8 datasets.}
  \label{supp_tab: comparison_of_iCM}
\end{table*}
}



% ---------------------------------------------------- % 
%           对 Template 和 Description 优化的消融
% ---------------------------------------------------- %
{
\renewcommand{\arraystretch}{1.1} 
% \setlength{\tabcolsep}{4pt}

\begin{table*}[htbp]
  \centering
  \resizebox{0.99\linewidth}{!}
    {
    \begin{tabular}
        {l | lllll lllll lll | l | l}

            
        \toprule
        \textbf{Module} (ResNet50) & \rotatebox{90}{\textbf{IN-1K}} & \rotatebox{90}{\textbf{Caltech}} & \rotatebox{90}{\textbf{Cars}} & \rotatebox{90}{\textbf{CUB}} & \rotatebox{90}{\textbf{DTD}}  & \rotatebox{90}{\textbf{ESAT}} & \rotatebox{90}{\textbf{FGVC}} & \rotatebox{90}{\textbf{FLO}} & \rotatebox{90}{\textbf{Food}}  &  \rotatebox{90}{\textbf{Pets}} & \rotatebox{90}{\textbf{Places}} & \rotatebox{90}{\textbf{SUN}} & \rotatebox{90}{\textbf{UCF}} & \rotatebox{90}{\textbf{Avg (11)}} & \rotatebox{90}{\textbf{Avg (13)}} \\
        \midrule

        % \multicolumn{16}{c}{\textit{\textbf{\ccol{ViT-B/32 Backbone}}}} \\
        % \midrule

        Vanilla CLIP & 57.9 & 84.5 & 53.9 & 44.7 & 38.8 & 28.6 & 15.9 & 60.2 & 74.0 & 83.2 & 38.2 & 58.0 & 56.9 & 55.6 & 53.4 \\ 
        
        \midrule
        \multicolumn{16}{c}{\textit{\textbf{\ccol{Template Optimization Methods}}}} \\

        \midrule 
        PN~\cite{P_N} & 59.6 & 89.1 & 56.2 & - & 44.8 & \underline{49.0} & 18.1 & 67.2 & 78.3 & 88.1 & - & 61.0 & 60.2 & 61.1 & -  \\

        \highlight{\textbf{ATO} (w/o dataset domain)} & \highlight{60.4} & \highlight{88.9} & \highlight{56.8} & \highlight{47.0} & \highlight{45.0} & \highlight{43.7} & \highlight{17.9} & \highlight{67.4} & \highlight{79.9} & \highlight{87.8} & \highlight{40.0} & \highlight{61.2} & \highlight{61.5} & \highlight{61.0} & \highlight{58.3} \\
        
        \highlight{\textbf{ATO}} & \highlight{\underline{61.3}} & \highlight{89.4} & \highlight{57.4} & \highlight{49.2} &  \highlight{45.4} & \highlight{46.4} & \highlight{18.4} & \highlight{68.1} & \highlight{80.5} & \highlight{88.5} & \highlight{40.2} &  \highlight{61.8} & \highlight{63.9} & \highlight{61.9}  & \highlight{59.3} \\

        \midrule

        \multicolumn{16}{c}{\textit{\textbf{\ccol{Description Optimization Methods}}}} \\

        \midrule
        \highlight{\textbf{ProAPO} (w/o synonyms)} & \highlight{\textbf{61.5}} & \highlight{\underline{89.7}} & \highlight{\textbf{58.3}} & \highlight{\underline{49.7}} & \highlight{\underline{46.6}} & \highlight{46.8} & \highlight{\underline{20.5}} & \highlight{\underline{74.6}} & \highlight{\underline{81.0}} & \highlight{\textbf{88.8}} & \highlight{\underline{40.9}} & \highlight{\underline{62.3}} & \highlight{\underline{64.8}} & \highlight{\underline{63.2}} & \highlight{\underline{60.4}} \\        
        
        \highlight{\textbf{ProAPO} (ours)} & \highlight{\textbf{61.5}} & \highlight{\textbf{90.3}} & \highlight{\underline{58.0}} & \highlight{\textbf{50.7}} & \highlight{\textbf{52.3}} & \highlight{\textbf{51.7}} & \highlight{\textbf{21.1}} & \highlight{\textbf{75.1}} & \highlight{\textbf{81.8}} & \highlight{\underline{88.7}} & \highlight{\textbf{41.8}} & \highlight{\textbf{63.7}} & \highlight{\textbf{66.0}} & \highlight{\textbf{64.6}}  & \highlight{\textbf{61.8}} \\
        
        \bottomrule
    \end{tabular}
}
  
  \caption{\textbf{Ablation of template and description optimization.} 
  Avg (11) and Avg (13) denote average results across 11 datasets (excluding CUB~\cite{CUB} and Places~\cite{Places365}) and all 13 datasets, respectively. ATO denotes our automatic template optimization algorithm.}
  \vspace{-4pt}
  \label{supp_tab: ablate_template_and_description}
\end{table*}
}

% ---------------------------------------------------- % 
%           更多的消融结果
% ---------------------------------------------------- %
\section{More Ablation Results}
\label{sec_supp: more_ablation_result}

% ---------------------------------------------------- % 
%           对 Template 和 Description 优化的消融
% ---------------------------------------------------- %
\subsection{Ablation of Template and Description Optimization}
\label{supp_sec: ablation_template_and_description}

In~\cref{supp_tab: ablate_template_and_description}, we ablate key components in template and description optimization on the ResNet50 backbone. 

\textbf{(1) Ablation of Template Optimization}. In the main paper (Sec. 4.3), we show that prompt ensembling is better than a single prompt. Moreover, dataset domain information also plays a significant role in template optimization. Without domain information, we see a performance drop in our ATO by an average of 1.0\% (from 58.3\% to 59.3 \%) on thirteen datasets. This is because domain information provides contextual information, which can mitigate issues of semantic ambiguity caused by class names.

\textbf{(2) Ablation of Description Optimization}. Without label synonyms to increase description diversity, a performance degradation appears by an average of 1.4\% (from 60.4\% to 61.8\%) on thirteen datasets. It verifies the effectiveness of optimization class names, which are usually ignored in previous description methods~\cite{CuPL, DCLIP, AdaptCLIP, GPT4Vis}.

\textbf{(3) Template VS Description Optimization}. Compared with template optimization, we see a notable performance improvement with description optimization, especially in CUB~\cite{CUB}, DTD~\cite{DTD}, ESAT~\cite{EuroSAT}, FLO~\cite{FLO}, and UCF~\cite{UCF101} datasets. It demonstrates that optimizing class-specific prompts can find discriminative information for fine-grained classification.


% ---------------------------------------------------- % 
%                   对每个 Operator 操作的消融
% ---------------------------------------------------- %
\subsection{More Ablation of Operators}
\label{supp_sec: more_ablation_operator}
To further explore whether each operator has a role in searching the optimal result, we show the number of each operator causing the new optimal score during the iterations in~\cref{supp_tab: more_ablation_operator}. We see that each operator in iterative optimization may generate a better prompt. It further demonstrates that each operator is helpful in ProAPO. Notably, the crossover operator has the highest times to update the optimal score, which demonstrates that it makes the model search for the optimal prompt faster with limited iterations. 

{
\renewcommand{\arraystretch}{1.1} 
% \setlength{\tabcolsep}{4.pt}
\begin{table}[t]
  \centering
  \resizebox{0.96\linewidth}{!}
    {
    \begin{tabular}
        {l | c  c  c  c  c | c  }  
        \toprule
        {\textbf{Dataset}} & \texttt{Add} & \texttt{Del} & \texttt{Rep} & \texttt{Cross} & \texttt{Mut} & \textbf{Total} \\
        \midrule
        IN-1K~\cite{Imagenet} & 3 & 4 & 5 & 5 & 2 &  19 \\
        Caltech~\cite{caltech101} & 5 & 5 & 6 & 12 & 3 & 31 \\
        Cars~\cite{Cars} & 7 & 8 & 5 & 8 & 3 & 31 \\ 
        CUB~\cite{CUB} & 9 & 4 & 10 & 6 & 2 & 31 \\
        DTD~\cite{DTD} & 5 & 3 & 8 & 8 & 2 & 26 \\
        ESAT~\cite{EuroSAT} & 2 & 4 & 6 & 8 & 1 & 21 \\
        FGVC~\cite{FGVC} & 6 & 2 & 6 & 5 & 3 & 22 \\ 
        FLO~\cite{FLO} & 5 & 3 & 11 & 5 & 4 & 28 \\
        Food~\cite{Food101} & 5 & 3 & 4 & 5 & 2 & 19 \\ 
        Pets~\cite{oxford_pets} & 4 & 2 & 5 & 6 & 2 & 19 \\ 
        Places~\cite{Places365} & 3 & 2 & 8 & 12 & 4 & 29 \\ 
        SUN~\cite{SUN} & 4 & 2 & 3 & 5 & 2 & 16 \\ 
        UCF~\cite{UCF101} & 5 & 6 & 8 & 6 & 2 & 27 \\ 
        \midrule
        \textbf{Sum} & 63 & 48 & 85 & 91 & 32 & 319 \\ 
        \bottomrule
    \end{tabular}
}
% \vspace{-6pt}
  \caption{\textbf{Number of times for each operator that update the optimal score.} \textbf{Total} denotes the total number of iterations when achieving the highest score.}
% \vspace{-8pt}
  \label{supp_tab: more_ablation_operator}
\end{table}
}

% ---------------------------------------------------- % 
%                   对 Group Sampling 的消融
% ---------------------------------------------------- %
{
\renewcommand{\arraystretch}{1.1} 
% \setlength{\tabcolsep}{4.pt}
\begin{table*}[t]
  \centering
  \resizebox{0.99\linewidth}{!}
    {
    \begin{tabular}
        {l | ccccc ccccc ccc | c | c | c }  
        \toprule
        {\textbf{Module} (ViT-B/32)}  & \rotatebox{90}{\textbf{IN-1K}} & \rotatebox{90}{\textbf{Caltech}} & \rotatebox{90}{\textbf{Cars}} & \rotatebox{90}{\textbf{CUB}} & \rotatebox{90}{\textbf{DTD}}  & \rotatebox{90}{\textbf{ESAT}} & \rotatebox{90}{\textbf{FGVC}} & \rotatebox{90}{\textbf{FLO}} & \rotatebox{90}{\textbf{Food}}  &  \rotatebox{90}{\textbf{Pets}} & \rotatebox{90}{\textbf{Places}} & \rotatebox{90}{\textbf{SUN}} & \rotatebox{90}{\textbf{UCF}} & \rotatebox{90}{\textbf{Avg (11)}} & \rotatebox{90}{\textbf{Avg (13)}} & \textbf{Times}  \\
        \midrule
        CuPL & 64.4  & 92.9  & 60.7  & 53.3  & {50.6}  & 50.5  & 20.9  & 69.5  & 84.2  & 87.0  & \underline{43.1}  & {66.3}  & 66.4  & 64.9  & 62.3 & - \\
        \midrule
        
        \texttt{a)} w/ all categories in one group & 64.5 & 93.3 & 60.9 & 53.5 & \underline{51.6} & 52.2 & 22.2 & 70.8 & 84.5 & 87.9 & 42.3 & \textbf{66.7} & \textbf{69.4} & 65.8  & 63.1 & \textbf{20 min} \\
        \texttt{b)} w/ random selected group & 64.3 & 93.7 & \textbf{61.8} & \underline{55.2} & 48.7 & 59.5 & 22.6 & 72.9 & \underline{85.2} & \underline{90.8} & 42.6 & 65.4 & 68.4 & 66.7  & 63.9 & 15 min  \\
        \texttt{c)} w/ performance best group & 64.1 & 93.0 & 61.2 & 54.4 & 47.4 & 56.8 & 20.7 & 68.2 & 85.1 & 88.6 & 42.4 & 65.0 & 65.4 & 65.0  & 62.5 & 15 min \\
        \texttt{d)} w/ K-Means algorithm & \underline{64.6} & \underline{93.8} & \textbf{61.8} & 55.1 & 49.4 & \underline{59.6} & \underline{22.8} & \underline{74.0} & \textbf{85.3} & 90.7 & {42.7} & 65.4 & \underline{69.0} & \underline{67.0}  & \underline{64.2} & \underline{17 min} \\
        
        \midrule

        
        \highlight{\textbf{ProAPO} (full model)} & \highlight{\textbf{64.7}}  & \highlight{\textbf{94.4}} & \highlight{\underline{61.7}} & \highlight{\textbf{55.4}} & \highlight{\textbf{53.5}} & \highlight{\textbf{63.0}} & \highlight{\textbf{23.0}} & \highlight{\textbf{74.3}} & \highlight{\textbf{85.3}} & \highlight{\textbf{91.0}} & \highlight{\textbf{43.3}} & \highlight{\underline{66.6}} & \highlight{\underline{69.0}} & \highlight{\textbf{67.9}}  & \highlight{\textbf{65.0}} & \highlight{15 min} \\
        \bottomrule
    \end{tabular}
}
    \vspace{-5pt}
  \caption{\textbf{More ablation of group sampling strategy.} We ablate the ways for selecting salient groups. \textbf{Times} denotes the time that ProAPO runs on ImageNet with the default setting.}
  \vspace{-7pt}
  \label{supp_tab: more_ablation_group_sampling}
\end{table*}

}


\subsection{More Ablation of Group Sampling}
\label{supp_sec: more_ablation_group_sampling}

In~\cref{supp_tab: more_ablation_group_sampling}, we ablate how to select categories in the group sampling strategy. We consider the settings for optimizing all categories in one group, selecting random categories and the best categories with their misclassified categories in groups. In rows a)-c) of~\cref{supp_tab: more_ablation_group_sampling}, we see notable performance degradation compared to full ProAPO. It further demonstrates that optimizing salient and worst groups can achieve comparable results with all categories and save iteration costs. Moreover, we also consider replacing misclassified categories with a K-Means clustering algorithm. A performance drop appears in row d), which verifies the effectiveness of selecting misclassified categories in groups.



% ---------------------------------------------------- % 
%                   对 Computation 的消融
% ---------------------------------------------------- %
\subsection{Ablation of Cost Computation}
\label{supp_sec: ablation_of_cost_computation}

In~\cref{supp_tab: extra_computation_cost}, we detail the time each process consumes on ImageNet. Compared to previous LLM-generated description methods, we similarly query LLMs one-time to generate descriptions (\textit{i.e.}, process of building prompt library). In addition, we introduce iterative processes to refine prompts and two sampling strategies to save costs. With a few additional costs (15 min v.s. 60 min), our ProAPO improves previous methods by at least 2.7\% on average. This further verifies the efficiency of our method.

{
% \vspace{-11pt}
\renewcommand{\arraystretch}{1.0} 
\setlength{\tabcolsep}{3.8pt}
\begin{table}[!h]
  \centering
  \resizebox{1.0\linewidth}{!}
    {
    \begin{tabular}
        {l | c | c c c c }  
        \toprule
        \textbf{Process} & \textbf{Build Library} & \highlight{\textbf{Sample Strategy}} & \highlight{\textbf{Template Optim.}} & \highlight{\textbf{Description Optim.}} \\ 
        \midrule 
        \textbf{Times} & 60 min &  \highlight{3 min}  & \highlight{1.6 min} & \highlight{10.4 min} \\
        \bottomrule
    \end{tabular}
}
\vspace{-5pt}
  \caption{\textbf{Computation cost analysis in the ImageNet dataset.}}
% \vspace{-18pt}
  \label{supp_tab: extra_computation_cost}
\end{table}
}




% ---------------------------------------------------- % 
%                    超参数分析
% ---------------------------------------------------- %
\section{More Hyperparameter Analysis}
\label{sec_supp: more_hyper_analysis}




{
\renewcommand{\arraystretch}{1.1} 
\setlength{\tabcolsep}{4.pt}
\begin{table}[tbp]
  \centering
  \resizebox{0.99\linewidth}{!}
    {
    \begin{tabular}
        {l | l | c | c c c c c | c }

            
        \toprule
        {\textbf{Dataset}} & \textbf{Module} & {\textbf{TF}} & \multicolumn{5}{c}{\textbf{Number of training samples}} & \textbf{UB} \\
        \cmidrule(lr){4-8}
         & (RN50) & &  1 & 2 & 4 & 8 & 16\\
        \midrule
        \multirow{2}{*}{\textbf{Avg (11)}} & CoOp~\cite{CoOp} & \xmark & 59.6 & 62.3 & \textbf{66.8} & \textbf{69.9} & \textbf{73.4} & -  \\
         & {\textbf{ProAPO}} & \cmark & \textbf{64.6} & \textbf{65.0} & 65.4 & 65.8 & 66.1 & 67.2 \\
        \midrule
        
        \multirow{2}{*}{\textbf{IN-1K}} & CoOp~\cite{CoOp} & \xmark  & 57.2 & 57.8 & 60.0 & \textbf{61.6} & \textbf{63.0} & -  \\
        & {\textbf{ProAPO}} & \cmark & \textbf{61.5} & \textbf{61.6} & \textbf{61.5} & \textbf{61.6} & 61.6 & 61.7 \\
        \midrule
        
        \multirow{2}{*}{\textbf{Caltech}} & CoOp~\cite{CoOp} & \xmark  & 87.5 & 87.9 & 89.6 & 90.2 & 91.8 & -  \\
        & {\textbf{ProAPO}} & \cmark & \textbf{90.3} & \textbf{90.4} & \textbf{90.6} & \textbf{90.7} & \textbf{91.0} & 91.1 \\        
        \midrule
        
        \multirow{2}{*}{\textbf{Cars}} & CoOp~\cite{CoOp} & \xmark & 55.6 & 58.3 & \textbf{62.6} & \textbf{68.4} & \textbf{73.4} & -  \\
        & {\textbf{ProAPO}} & \cmark & \textbf{58.0} & \textbf{58.5} & 58.8 & 58.9 & 59.1 & 60.8  \\
        \midrule
        
        \multirow{2}{*}{\textbf{DTD}} & CoOp~\cite{CoOp} & \xmark & 44.4 & 45.2 & \textbf{53.5} & \textbf{60.0} & \textbf{63.6} & -  \\
        & {\textbf{ProAPO}} & \cmark & \textbf{52.3} & \textbf{52.7} & 53.0 & 53.4 & 53.6 \\
        \midrule
        
        \multirow{2}{*}{\textbf{ESAT}} & CoOp~\cite{CoOp} & \xmark & 50.6 & \textbf{61.5} & \textbf{70.2} & \textbf{76.7} & \textbf{83.5} & -  \\
        & {\textbf{ProAPO}} & \cmark & \textbf{51.7} & 53.5 & 55.6 & 57.4 & 58.3 & 62.2 \\
        \midrule
        
        \multirow{2}{*}{\textbf{FGVC}} & CoOp~\cite{CoOp} & \xmark & 9.6 & 18.7 & \textbf{21.9} & \textbf{26.1} & \textbf{31.3} & -  \\
        & {\textbf{ProAPO}} & \cmark & \textbf{21.1} & \textbf{21.0} & 21.2 & 21.2 & 21.3 & 21.5 \\
        \midrule
        
        \multirow{2}{*}{\textbf{FLO}} & CoOp~\cite{CoOp} & \xmark & 68.1 & \textbf{77.5} & \textbf{86.2} & \textbf{91.2} & \textbf{94.5} & -  \\
        & {\textbf{ProAPO}} & \cmark & \textbf{75.1} & 75.6 & 76.4 & 76.7 & 77.8 & 79.1 \\
        \midrule
        
        \multirow{2}{*}{\textbf{Food}} & CoOp~\cite{CoOp} & \xmark  & 74.3 & 72.5 & 73.3 & 71.8 & 74.7 & -  \\
        & {\textbf{ProAPO}} & \cmark & \textbf{81.8} & \textbf{82.0} & \textbf{82.1} & \textbf{82.2} & \textbf{82.3} & {82.9} \\
        \midrule
        
        \multirow{2}{*}{\textbf{Pets}} & CoOp~\cite{CoOp} & \xmark  & 85.9 & 82.6 & 86.7 & 85.3 & 87.0 & -  \\
        & {\textbf{ProAPO}} & \cmark & \textbf{88.7} & \textbf{89.4} & \textbf{89.5} & \textbf{89.8} & \textbf{89.9} & {91.0} \\
        \midrule
        
        \multirow{2}{*}{\textbf{SUN}} & CoOp~\cite{CoOp} & \xmark  & 60.3 & 59.5 & 63.5 & \textbf{65.5} & \textbf{69.3} & -  \\
        & {\textbf{ProAPO}} & \cmark & \textbf{63.7} & \textbf{63.8} & \textbf{63.8} & 63.8 & 63.9 & 64.5 \\
        \midrule
        
        \multirow{2}{*}{\textbf{UCF}} & CoOp~\cite{CoOp} & \xmark  & 61.9 & 64.1 & 67.0 & \textbf{71.9} & \textbf{75.7} & -  \\
        & {\textbf{ProAPO}} & \cmark &  \textbf{66.0} & \textbf{66.8} & \textbf{67.1} & 68.1 & 68.9 & 71.4 \\
        \bottomrule
    \end{tabular}
}
  \vspace{-5pt}
  \caption{\textbf{Scaling up to more shots.} \textbf{Avg (11)} denotes average results across 11 datasets. \textbf{TF} denotes training-free approaches. \textbf{UB} denotes upper bound evaluated on the test set.}
  % \vspace{-5pt}
  \label{supp_tab: shots_influence}
\end{table}
}



% ---------------------------------------------------- % 
%                    Shots 的影响
% ---------------------------------------------------- %
\subsection{Effect of Shot Numbers}
\label{supp_sec: effect_shots}
In~\cref{supp_tab: shots_influence}, we show the effect of the number of training samples per category.
Specifically, we conduct experiments with 1, 2, 4, 8, and 16 shots.
Moreover, we introduce the performance of the optimal prompt searched in the test set as the upper bound of ProAPO.
Compared with CoOp~\cite{CoOp}, ProAPO achieves remarkable performance when shots $\leq 2$, which demonstrates the effectiveness of our method under low-shot settings. Since we only adapt VLMs in a training-free way, the performance increases finitely as the training samples increase. We attribute two key directions for further performance improvement in high-shot settings. First, our result is still far from the upper bound (66.1 \% in 16 shots VS 67.2 \% for the upper bound). We need to improve the prompt generation algorithm and the score function to find better candidate prompts within the limited iterations. Second, the upper bound of our ProAPO is much smaller than the prompt tuning method. We need to use a larger natural language search space (\textit{e.g.}, more diverse descriptions, or more query times of LLMs) to further increase the upper bound of the optimal result.



% ---------------------------------------------------- % 
%                    Alpha 值的影响
% ---------------------------------------------------- %
\subsection{Effect of Scalar in Score Function}
\label{supp_sec: effect_alpha}

In~\cref{supp_tab: ablation_alpha},  we show the effect of $\alpha$ in~\cref{eq: score_function}. We see that performance improves as the $\alpha$ increases. This is because the entropy constraint provides more information to select better candidate prompts. We see a stable result when $\alpha \in [5e2, 5e3]$, which means a better trade-off between accuracy and entropy constraint. However, a high $\alpha$ may be biased to the train set, thus harming the performance. 

{
% \vspace{-10pt}
% \renewcommand{\arraystretch}{1.0} 
\begin{table}[h]
  \centering
  \resizebox{1.0\linewidth}{!}
    {
    \begin{tabular}
        {l | c c c c c c c c}  
        \toprule
        $\alpha$ & $0$ & $1e1$ & $1e2$ & $5e2$ & $1e3$ & $5e3$ & $1e4$ & $1e5$ \\
        \midrule
        \textbf{Avg (13)} & 62.3 & 63.4 & 64.4 & 64.9 & \textbf{65.0} & 64.8 & 63.7 & 63.1 \\
        \bottomrule
    \end{tabular}
}
% \vspace{-5pt}
  \caption{\textbf{Effect of $\alpha$ value in Eq.6 across 13 datasets.}}
  % \vspace{-5pt}
  \label{supp_tab: ablation_alpha}
\end{table}
}


% ---------------------------------------------------- % 
%                   T_sample 的影响
% ---------------------------------------------------- %
\subsection{Effect of Sampled Numbers in Prompt Sampling Strategy}
\label{supp_sec: effect_sampled_numbers}
In~\cref{supp_fig: effect_sampled_numbers}, we show the effect of sampled numbers $T_{sample}$ of Alg.~\ref{supp_alg: prompt_strategy}. The $T_{sample} = 0$ means that the prompt sampling strategy is not used. As the number of $T_{sample}$ increases, we see a slight performance gain when $T_{sample} < 4$. After $T_{sample} \geq 4$, a consistent improvement appears because the initial search point achieves a higher score than the baseline. We achieve stable results when $T_{sample} \geq 32$.

\begin{figure}[htbp]
\centering
\includegraphics[width=0.8\linewidth]{Paper_Result_supp/hyper_result_Sampled_Numbers.pdf}
\vspace{-8pt}
\caption{\textbf{Effect of sampled numbers $T_{sample}$.} 
}
\label{supp_fig: effect_sampled_numbers}
\vspace{-6pt}
\end{figure}


% ---------------------------------------------------- % 
%                   Prompt Library 的影响
% ---------------------------------------------------- %
\subsection{Effect of Quality of Prompt Library}
\label{supp_sec: effect_of_quality_of_prompt_library}
In~\cref{supp_fig: effect_LLM_Query} and~\cref{supp_fig: effect_generated_descriptions}, we analyze two key factors affecting the prompt library: LLM-query prompts and generated descriptions. Our ProAPO improves prompt quality even under a small number of query prompts and descriptions, demonstrating its effectiveness in a limited prompt library. 




\begin{figure}[htbp]
\centering
\includegraphics[width=0.8\linewidth]{Paper_Result_supp/LLM-query.pdf}
\vspace{-8pt}
\caption{\textbf{Effect of Number of LLM-query Prompts.} 
}
\label{supp_fig: effect_LLM_Query}
% \vspace{-6pt}
\end{figure}



\begin{figure}[htbp]
\centering
\includegraphics[width=0.8\linewidth]{Paper_Result_supp/generated_descriptions.pdf}
\vspace{-8pt}
\caption{\textbf{Effect of Number of Generated Descriptions.} 
}
\label{supp_fig: effect_generated_descriptions}
% \vspace{-6pt}
\end{figure}




% ---------------------------------------------------- % 
%                     更多的可视化
% ---------------------------------------------------- %



\section{More Qualitative Results}
\label{sec_supp: more_qualitative_result}
In~\cref{supp_fig: qualitative_result}, we show more examples of the changes in descriptions with our ProAPO, including images of animals, flowers, and textures.
Similarly, we see that common descriptions are removed and discriminative ones are retained for fine-grained categories, which further verifies the effectiveness of our progressive optimization.


\begin{figure*}[htbp]
\centering
\includegraphics[width=0.8\linewidth]{Paper_Result_supp/qualitative_results_supp.pdf}
% \vspace{-8pt}
\caption{\textbf{Qualitative analysis of class-specific prompt optimization by ProAPO.} Shaded \textbf{\textcolor{removed}{red}} and \textbf{\textcolor{retained}{blue}} words denote common and discriminative descriptions in two confused categories.
}
\label{supp_fig: qualitative_result}
\vspace{20pt}
\end{figure*} 


% \newpage




\newpage

\section{Detailed Results of More Benefits by Optimal Prompts}
\label{sec_supp: detailed_results_of_main_paper}

% ---------------------------------------------------- % 
%               迁移到 Adapter 方法的结果
% ---------------------------------------------------- %
\subsection{Transfer to Adapter-based Methods}
\label{sec_supp: transfer_to_adapter}
In~\cref{supp_fig: improve_train_free_adapter} and ~\cref{supp_fig: improve_train_adapter}, we show the detailed results of popular training-free and training adapter-based methods~\cite{Tip, Tip-X, APE, CLIP_Adapter} with different prompt initialization, \textit{i.e.}, SOTA method CuPL~\cite{CuPL} and our ProAPO. Adapter-based methods with ProAPO (solid lines) consistently surpass those with CuPL (dotted lines). It reveals that high-quality prompts make adapters perform better. Even in low shots, training with ProAPO achieves notable performance gains, which further verifies its effectiveness. 


{
\begin{figure*}[ht]
\centering
\begin{adjustbox}{minipage=\textwidth,scale=0.88}
\begin{subfigure}{0.33\textwidth}
    \includegraphics[width=\textwidth]{Paper_Result_supp/transfer_adapter/Training-free_Results_on_ImageNet.pdf}
    \caption{ImageNet.}
    % \label{fig:first}
    % \vspace{-10pt}
\end{subfigure}
\hfill
\begin{subfigure}{0.33\textwidth}
    \includegraphics[width=\textwidth]{Paper_Result_supp/transfer_adapter/Training-free_Results_on_Caltech-101.pdf}
    \caption{Caltech.}
    % \label{fig:second}
    % \vspace{-10pt}
\end{subfigure}
\hfill
\begin{subfigure}{0.33\textwidth}
    \includegraphics[width=\textwidth]{Paper_Result_supp/transfer_adapter/Training-free_Results_on_StanfordCars.pdf}
    \caption{Cars.}
    % \label{fig:third}
    % \vspace{-10pt}
\end{subfigure}
% \vspace{-5pt}
\begin{subfigure}{0.33\textwidth}
    \includegraphics[width=\textwidth]{Paper_Result_supp/transfer_adapter/Training-free_Results_on_DTD.pdf}
    \caption{DTD.}
    % \label{fig:first}
    % \vspace{-10pt}
\end{subfigure}
\hfill
\begin{subfigure}{0.33\textwidth}
    \includegraphics[width=\textwidth]{Paper_Result_supp/transfer_adapter/Training-free_Results_on_FGVC.pdf}
    \caption{FGVC.}
    % \label{fig:second}
    % \vspace{-10pt}
\end{subfigure}
\hfill
\begin{subfigure}{0.33\textwidth}
    \includegraphics[width=\textwidth]{Paper_Result_supp/transfer_adapter/Training-free_Results_on_EuroSAT.pdf}
    \caption{ESAT.}
    % \label{fig:third}
    % \vspace{-10pt}
\end{subfigure}

\begin{subfigure}{0.33\textwidth}
    \includegraphics[width=\textwidth]{Paper_Result_supp/transfer_adapter/Training-free_Results_on_Flowers102.pdf}
    \caption{FLO.}
    % \label{fig:first}
    % \vspace{-10pt}
\end{subfigure}
\hfill
\begin{subfigure}{0.33\textwidth}
    \includegraphics[width=\textwidth]{Paper_Result_supp/transfer_adapter/Training-free_Results_on_Food101.pdf}
    \caption{Food.}
    % \label{fig:second}
    % \vspace{-10pt}
\end{subfigure}
\hfill
\begin{subfigure}{0.33\textwidth}
    \includegraphics[width=\textwidth]{Paper_Result_supp/transfer_adapter/Training-free_Results_on_OxfordPets.pdf}
    \caption{Pets.}
    % \label{fig:third}
    % \vspace{-10pt}
\end{subfigure}

\begin{subfigure}{0.33\textwidth}
    \includegraphics[width=\textwidth]{Paper_Result_supp/transfer_adapter/Training-free_Results_on_SUN397.pdf}
    \caption{SUN.}
    % \label{fig:first}
    % \vspace{-10pt}
\end{subfigure}
% \hfill
\begin{subfigure}{0.33\textwidth}
    \includegraphics[width=\textwidth]{Paper_Result_supp/transfer_adapter/Training-free_Results_on_UCF101.pdf}
    \caption{UCF.}
    % \label{fig:second}
    % \vspace{-10pt}
\end{subfigure}
\end{adjustbox}
\caption{\textbf{Results of training-free adapter-based methods with different initial prompts.} Solid and dotted lines denote prompt initialization with ProAPO and CuPL, respectively. We see that our ProAPO consistently improves adapter-based methods.}
\label{supp_fig: improve_train_free_adapter}
\vspace{60pt}
\end{figure*}
}


{
\begin{figure*}[ht]
\centering
\begin{adjustbox}{minipage=\textwidth,scale=0.88}
\begin{subfigure}{0.33\textwidth}
    \includegraphics[width=\textwidth]{Paper_Result_supp/transfer_adapter/Training_Results_on_ImageNet.pdf}
    \caption{ImageNet.}
    % \label{fig:first}
    % \vspace{-10pt}
\end{subfigure}
\hfill
\begin{subfigure}{0.33\textwidth}
    \includegraphics[width=\textwidth]{Paper_Result_supp/transfer_adapter/Training_Results_on_Caltech-101.pdf}
    \caption{Caltech.}
    % \label{fig:second}
    % \vspace{-10pt}
\end{subfigure}
\hfill
\begin{subfigure}{0.33\textwidth}
    \includegraphics[width=\textwidth]{Paper_Result_supp/transfer_adapter/Training_Results_on_StanfordCars.pdf}
    \caption{Cars.}
    % \label{fig:third}
    % \vspace{-10pt}
\end{subfigure}
% \vspace{-5pt}
\begin{subfigure}{0.33\textwidth}
    \includegraphics[width=\textwidth]{Paper_Result_supp/transfer_adapter/Training_Results_on_DTD.pdf}
    \caption{DTD.}
    % \label{fig:first}
    % \vspace{-10pt}
\end{subfigure}
\hfill
\begin{subfigure}{0.33\textwidth}
    \includegraphics[width=\textwidth]{Paper_Result_supp/transfer_adapter/Training_Results_on_FGVC.pdf}
    \caption{FGVC.}
    % \label{fig:second}
    % \vspace{-10pt}
\end{subfigure}
\hfill
\begin{subfigure}{0.33\textwidth}
    \includegraphics[width=\textwidth]{Paper_Result_supp/transfer_adapter/Training_Results_on_EuroSAT.pdf}
    \caption{ESAT.}
    % \label{fig:third}
    % \vspace{-10pt}
\end{subfigure}

\begin{subfigure}{0.33\textwidth}
    \includegraphics[width=\textwidth]{Paper_Result_supp/transfer_adapter/Training_Results_on_Flowers102.pdf}
    \caption{FLO.}
    % \label{fig:first}
    % \vspace{-10pt}
\end{subfigure}
\hfill
\begin{subfigure}{0.33\textwidth}
    \includegraphics[width=\textwidth]{Paper_Result_supp/transfer_adapter/Training_Results_on_Food101.pdf}
    \caption{Food.}
    % \label{fig:second}
    % \vspace{-10pt}
\end{subfigure}
\hfill
\begin{subfigure}{0.33\textwidth}
    \includegraphics[width=\textwidth]{Paper_Result_supp/transfer_adapter/Training_Results_on_OxfordPets.pdf}
    \caption{Pets.}
    % \label{fig:third}
    % \vspace{-10pt}
\end{subfigure}

\begin{subfigure}{0.33\textwidth}
    \includegraphics[width=\textwidth]{Paper_Result_supp/transfer_adapter/Training_Results_on_SUN397.pdf}
    \caption{SUN.}
    % \label{fig:first}
    % \vspace{-10pt}
\end{subfigure}
% \hfill
\begin{subfigure}{0.33\textwidth}
    \includegraphics[width=\textwidth]{Paper_Result_supp/transfer_adapter/Training_Results_on_UCF101.pdf}
    \caption{UCF.}
    % \label{fig:second}
    % \vspace{-10pt}
\end{subfigure}
\end{adjustbox}
\caption{\textbf{Results of training adapter-based methods with different initial prompts.} Solid and dotted lines denote prompt initialization with ProAPO and CuPL, respectively. We see that our ProAPO consistently improves adapter-based methods.}
\label{supp_fig: improve_train_adapter}
\vspace{60pt}
\end{figure*}
}


% ---------------------------------------------------- % 
%               迁移到不同 backbones 的结果
% ---------------------------------------------------- %
\subsection{Transfer to Different Backbones}
\label{sec_supp: transfer_to_backbones}

In~\cref{supp_fig: transfer_backbones}, we show detailed results of transferring prompts from source to target models in thirteen datasets. Our optimized prompts of ResNet50 and ViT-B/32 are reported.
We see that ProAPO achieves stable performance gains compared to CuPL~\cite{CuPL}, which verifies that ProAPO transfers easily across different backbones.




{
\begin{figure*}[htbp]
\centering
% \resizebox{0.9\linewidth}{!}
{
    \hfill
    \subfloat[ImageNet]{\includegraphics[width=0.32\textwidth]{Paper_Result_supp/transfer_backbones/transfer_to_backbones_imagenet.pdf}}
    \hfill
    \subfloat[Caltech]{\includegraphics[width=0.32\textwidth]{Paper_Result_supp/transfer_backbones/transfer_to_backbones_caltech101.pdf}}
    \hfill
    \subfloat[Cars]{\includegraphics[width=0.32\textwidth]{Paper_Result_supp/transfer_backbones/transfer_to_backbones_stanford_cars.pdf}}
    \hfill
    % \vspace{0.5em}
    \subfloat[CUB]{\includegraphics[width=0.32\textwidth]{Paper_Result_supp/transfer_backbones/transfer_to_backbones_cub.pdf}}
    \hfill
    \subfloat[DTD]{\includegraphics[width=0.32\textwidth]{Paper_Result_supp/transfer_backbones/transfer_to_backbones_dtd.pdf}}
    \hfill
    \subfloat[FGVC]{\includegraphics[width=0.32\textwidth]{Paper_Result_supp/transfer_backbones/transfer_to_backbones_fgvc_aircraft.pdf}}
    \hfill
    % \vspace{0.5em}
    \subfloat[ESAT]{\includegraphics[width=0.32\textwidth]{Paper_Result_supp/transfer_backbones/transfer_to_backbones_euro_sat.pdf}}
    \hfill
    \subfloat[FLO]{\includegraphics[width=0.32\textwidth]{Paper_Result_supp/transfer_backbones/transfer_to_backbones_flo.pdf}}
    \hfill
    \subfloat[Food]{\includegraphics[width=0.32\textwidth]{Paper_Result_supp/transfer_backbones/transfer_to_backbones_food101.pdf}}
    \hfill
    % \vspace{0.5em}
    \subfloat[Pets]{\includegraphics[width=0.32\textwidth]{Paper_Result_supp/transfer_backbones/transfer_to_backbones_oxford_pets.pdf}}
    \hfill
    \subfloat[Places]{\includegraphics[width=0.32\textwidth]{Paper_Result_supp/transfer_backbones/transfer_to_backbones_places365.pdf}}
    \hfill
    \subfloat[SUN]{\includegraphics[width=0.32\textwidth]{Paper_Result_supp/transfer_backbones/transfer_to_backbones_sun397.pdf}}
    \hfill
    % \vspace{0.5em}
    \subfloat[UCF]{\includegraphics[width=0.32\textwidth]{Paper_Result_supp/transfer_backbones/transfer_to_backbones_ucf101.pdf}}
    % \hfill
    \subfloat[Avg (11)]{\includegraphics[width=0.32\textwidth]{Paper_Result_supp/transfer_backbones/transfer_to_backbones_avg11.pdf}}
    % \hfill
}
\vspace{-7pt}
\caption{\textbf{Results of prompt transfer to different backbones.} The value denotes performance gains compared to vanilla VLMs. Our optimized prompts of ResNet50 and ViT-B/32 are reported. We see that we achieve stable performance gains compared to CuPL~\cite{CuPL}. 
}
% \Description{}
\label{supp_fig: transfer_backbones}
\vspace{-5pt}
\end{figure*}
}



% ---------------------------------------------------- % 
%               性能提升分析的详细实验
% ---------------------------------------------------- %
% \newpage
\section{Detailed Results of Performance Improvement Analysis}

% ---------------------------------------------------- % 
%               Single VS Ensemble Prompts
% ---------------------------------------------------- %

{
\renewcommand{\arraystretch}{1.1} 
\begin{table*}[htbp]
  \centering
  \resizebox{0.75\linewidth}{!}
    {
    \begin{tabular}
        {l |  ccccc ccccc c | c }
        \toprule
        \textbf{Module} (ResNet50) & \rotatebox{90}{\textbf{IN-1K}} & \rotatebox{90}{\textbf{Caltech}} & \rotatebox{90}{\textbf{Cars}}  & \rotatebox{90}{\textbf{DTD}}  & \rotatebox{90}{\textbf{ESAT}} & \rotatebox{90}{\textbf{FGVC}} & \rotatebox{90}{\textbf{FLO}} & \rotatebox{90}{\textbf{Food}}  &  \rotatebox{90}{\textbf{Pets}}  & \rotatebox{90}{\textbf{SUN}} & \rotatebox{90}{\textbf{UCF}} & \rotatebox{90}{\textbf{Avg (11)}}  \\
        \midrule

        CLIP (a photo of a \{\})  & 57.9 & 84.5 & 53.9 & 38.8 & 28.6 & 15.9 & 60.2 & 74.0 & 83.2 & 58.0 & 56.9 & 55.6 \\ 

        \midrule
        \multicolumn{13}{c}{\textit{\textbf{\ccol{Single Prompt}}}} \\
        \midrule
        
        PN~\cite{P_N} & {59.6} & 89.1 & 56.2  & 44.8 & \underline{49.0} & 18.1 & 67.2 & {78.3} & 88.1  & 61.0 & 60.2 & 61.1 \\
        
        Best Single* & 60.2 & \underline{89.2} & \underline{57.9}  & 45.0 & 46.0 & \underline{18.3} & \underline{68.1} & \underline{81.8} & 88.3 & 61.5 & 62.6 & 61.7 \\

        \midrule
        \multicolumn{13}{c}{\textit{\textbf{\ccol{Ensemble Prompt}}}} \\
        \midrule 
        
        \highlight{\textbf{ATO} (ours)} & \highlight{\underline{61.3}} & \highlight{\underline{89.2}} & \highlight{\underline{57.9}} & \highlight{\underline{45.4}} & \highlight{44.7} & \highlight{18.2} & \highlight{\underline{68.1}} & \highlight{\underline{81.8}} & \highlight{\underline{88.5}} & \highlight{\underline{61.8}} & \highlight{\underline{63.9}} & \highlight{\underline{61.9}}  \\
        
        Best Ensemble* & \textbf{61.5} & \textbf{90.0} & \textbf{58.4}  & \textbf{47.0} & \textbf{49.1} & \textbf{18.7} & \textbf{69.9} & \textbf{82.2} & \textbf{89.4} & \textbf{62.1} & \textbf{64.8} & \textbf{63.0}  \\

        % \midrule
        
        \bottomrule
    \end{tabular}
}
  \vspace{-6pt}
  \caption{\textbf{Analysis of the effect of single vs ensemble prompts.} * denotes results evaluated in the test set. ATO is our automatic template optimization algorithm. We see that our optimized templates achieve higher results than PN~\cite{P_N}, even better than the best single template.}
  \vspace{-6pt}
  \label{supp_tab: ensemble_vs_single}
\end{table*}
}



\subsection{Analysis of the Effect of Single VS Ensemble Prompts}
\label{supp_sec: ensemble_vs_single}
In~\cref{supp_tab: ensemble_vs_single}, we show detailed results of the effect of single vs ensemble prompts. Compared to PN~\cite{P_N}, we utilize prompt ensembling instead of a single prompt to optimize the template and description. We observe that ensemble templates have a higher upper bound than the single template. Similarly, our optimized templates achieve higher performance than PN~\cite{P_N}, even better than the best single template, further verifying the effectiveness of our method.


% ---------------------------------------------------- % 
%               提升 Description 方法的性能
% ---------------------------------------------------- %
\subsection{Performance Improvement of Description Methods by ProAPO}
\label{sec_supp: improve_description_methods}
In~\cref{tab:description_add_APO}, we show detailed results of description methods~\cite{DCLIP, CuPL, GPT4Vis, AdaptCLIP} with our ATO and ProAPO. We see a notable improvement in description methods by at least 2.7\% average in thirteen datasets. It further verifies the effectiveness of our progressive optimization. 

{
\renewcommand{\arraystretch}{1.1} 
% \setlength{\tabcolsep}{4pt}

\begin{table*}[tbp]
  \centering
  \resizebox{0.99\linewidth}{!}
    {
    \begin{tabular}
        {l | lllll lllll lll | l | l}

            
        \toprule
        \textbf{Module} (ViT-B/32) & \rotatebox{90}{\textbf{IN-1K}} & \rotatebox{90}{\textbf{Caltech}} & \rotatebox{90}{\textbf{Cars}} & \rotatebox{90}{\textbf{CUB}} & \rotatebox{90}{\textbf{DTD}}  & \rotatebox{90}{\textbf{ESAT}} & \rotatebox{90}{\textbf{FGVC}} & \rotatebox{90}{\textbf{FLO}} & \rotatebox{90}{\textbf{Food}}  &  \rotatebox{90}{\textbf{Pets}} & \rotatebox{90}{\textbf{Places}} & \rotatebox{90}{\textbf{SUN}} & \rotatebox{90}{\textbf{UCF}} & \rotatebox{90}{\textbf{Avg (11)}} & \rotatebox{90}{\textbf{Avg (13)}} \\
        \midrule

        % \multicolumn{16}{c}{\textit{\textbf{\ccol{ViT-B/32 Backbone}}}} \\
        % \midrule

        Vanilla CLIP & 62.1  & 91.2  & 60.4  & 51.7 & 42.9  & 43.9  & 20.2  & 66.0  & 83.2  & 86.8 & 39.9 & 62.1  & 60.9 & 61.8 & 59.3 \\ 
        
        \midrule
        DCLIP~\cite{DCLIP} & 63.3  & 92.7  & 59.4  & 52.7  & 44.1  & 38.4  & 19.4  & 66.1  & 83.9  & 88.1  & 41.2  & 65.0  & 65.8  & 62.4  & 60.0 \\
        + \textbf{ATO} & 63.8& 93.0& 60.3& 52.5& 46.5& 54.1& 21.8& 68.9& 84.0& 88.4& 41.5& 65.4& 66.0 & 64.7 & 62.0 \\
        + \textbf{ProAPO} & \textbf{64.1} & \textbf{93.2} & \textbf{60.6} & \textbf{53.6} & \textbf{48.2} & \textbf{59.4} & \textbf{22.6} & \textbf{71.5} & \textbf{84.2} & \textbf{88.7} & \textbf{42.7} & \textbf{66.0} & \textbf{68.0} & \textbf{66.0}  & \textbf{63.3}  \\
        $\Delta$ & \textcolor{retained}{+ 0.8} & \textcolor{retained}{+ 0.5} & \textcolor{retained}{+ 1.2} & \textcolor{retained}{+ 0.9} & \textcolor{retained}{+ 4.1} & \textcolor{retained}{+ 21.0} & \textcolor{retained}{+ 3.2} & \textcolor{retained}{+ 5.4} & \textcolor{retained}{+ 0.3} & \textcolor{retained}{+ 0.6} & \textcolor{retained}{+ 1.5} & \textcolor{retained}{+ 1.0} & \textcolor{retained}{+ 2.2} & \textcolor{retained}{+ 3.6} & \textcolor{retained}{+ 3.3} \\

        \midrule
        
        CuPL-base~\cite{CuPL} & 64.0  & 92.3  & 60.1  & 54.3  & 47.2  & 42.4  & 21.7  & 68.7  & 84.3  & 88.8  & 42.0  & \textbf{66.2}  & 66.7  & 63.8  & 61.4 \\
        + \textbf{ATO} & 64.2 & 93.3 & 60.9 & 54.8 & 47.8 & 53.1 & 22.2  & 70.4  & 84.9 & 89.2  & 42.3 & 65.5 & 67.4 & 65.3 & 62.8 \\
        + \textbf{ProAPO} & \textbf{64.4} & \textbf{94.2} & \textbf{61.8} & \textbf{55.9} & \textbf{48.1} & \textbf{62.1} & \textbf{23.2} & \textbf{74.4} & \textbf{85.4} & \textbf{91.0} & \textbf{42.7} & 65.6 & \textbf{68.6} & \textbf{67.2}  & \textbf{64.4} \\
        $\Delta$ & \textcolor{retained}{+ 0.4} & \textcolor{retained}{+ 1.9} & \textcolor{retained}{+ 1.7} & \textcolor{retained}{+ 1.6} & \textcolor{retained}{+ 0.9} & \textcolor{retained}{+ 19.7} & \textcolor{retained}{+ 1.5} & \textcolor{retained}{+ 5.7} & \textcolor{retained}{+ 1.1} & \textcolor{retained}{+ 2.2} & \textcolor{retained}{+ 0.7} & -0.6 & \textcolor{retained}{+ 1.9} & \textcolor{retained}{+ 3.4} & \textcolor{retained}{+ 3.0} \\

        \midrule
        CuPL-full~\cite{CuPL} & 64.4  & 92.9  & 60.7  & 53.3  & 50.6  & 50.5  & 20.9  & 69.5  & 84.2  & 87.0  & 43.1  & 66.3  & 66.4  & 64.9  & 62.3 \\
        + \textbf{ATO} & 64.5 & 93.7 & 61.0 & 54.0 & 52.0 & 58.7 & 22.1 & 70.5 & 84.6 & 89.2 & 43.2 & 66.4 & 67.5 & 66.4 &  63.6 \\
        + \textbf{ProAPO} &  \textbf{64.7} & \textbf{94.4} & \textbf{61.7} & \textbf{55.4} & \textbf{53.5} & \textbf{63.0} & \textbf{23.0} & \textbf{74.3} & \textbf{85.3} & \textbf{91.0} & \textbf{43.3} & \textbf{66.6} & \textbf{69.0} & \textbf{67.9}  & \textbf{65.0} \\
        $\Delta$ & \textcolor{retained}{+ 0.3} & \textcolor{retained}{+ 1.5} & \textcolor{retained}{+ 1.0} & \textcolor{retained}{+ 2.1} & \textcolor{retained}{+ 2.9} & \textcolor{retained}{+ 12.5} & \textcolor{retained}{+ 2.1} & \textcolor{retained}{+ 4.8} & \textcolor{retained}{+ 1.1} & \textcolor{retained}{+ 4.0} & \textcolor{retained}{+ 0.2} & \textcolor{retained}{+ 0.3} & \textcolor{retained}{+ 2.6} & \textcolor{retained}{+ 3.0} & \textcolor{retained}{+ 2.7} \\

        \midrule
        GPT4Vis~\cite{GPT4Vis} & 63.5  & 93.1  & 61.4  & 52.7  & 48.5  & 47.0  & 21.4  & 69.8  & 84.3  & 88.1  & 42.7  & 64.2  & 65.7  & 64.3  & 61.7 \\ 
        + \textbf{ATO} & 63.8 & 93.4 & 61.2 & 53.8 & 49.0 & 54.0  & 22.4 & 70.8  & 84.7  & 88.1  & 42.6 & 64.7 & 66.8 & 65.3 & 62.7 \\
        + \textbf{ProAPO} & \textbf{64.4} & \textbf{93.7} & \textbf{61.8} & \textbf{55.4} & \textbf{49.3} & \textbf{62.6} & \textbf{23.9} & \textbf{73.8} & \textbf{85.4} & \textbf{90.7} & \textbf{42.8} & \textbf{65.5} & \textbf{68.2} & \textbf{67.2}  & \textbf{64.4} \\
        $\Delta$ & \textcolor{retained}{+ 0.9} & \textcolor{retained}{+ 0.6} & \textcolor{retained}{+ 0.4} & \textcolor{retained}{+ 2.7} & \textcolor{retained}{+ 0.8} & \textcolor{retained}{+ 15.6} & \textcolor{retained}{+ 2.5} & \textcolor{retained}{+ 4.0} & \textcolor{retained}{+ 1.1} & \textcolor{retained}{+ 2.6} & \textcolor{retained}{+ 0.1} & \textcolor{retained}{+ 1.3} & \textcolor{retained}{+ 2.5} & \textcolor{retained}{+ 2.9} & \textcolor{retained}{+ 2.7} \\

        \midrule 
        AdaptCLIP~\cite{AdaptCLIP} & 63.3  & 92.7  & 59.7  & 53.6  & 47.4  & 51.3  & 20.8  & 67.2  & 84.2  & 87.6  & 41.9  & 66.1  & 66.5  & 64.2  & 61.7 \\
        + \textbf{ATO} & 63.9 & 93.2 & 60.4 & 54.2 & 47.9 & 55.5 & \textbf{22.4} & 69.1 & 84.7  & 88.8  & 42.3 & 66.3 & 67.6 & 65.4 & 62.8 \\
        + \textbf{ProAPO} & \textbf{64.4} & \textbf{93.7} & \textbf{61.8} & \textbf{55.5} & \textbf{49.6} & \textbf{61.6} & {23.3} & \textbf{73.8} & \textbf{85.4} & \textbf{91.0} & \textbf{42.6} & \textbf{66.5} & \textbf{68.6} & \textbf{67.2}  & \textbf{64.5} \\
        $\Delta$ & \textcolor{retained}{+ 1.1} & \textcolor{retained}{+ 1.0} & \textcolor{retained}{+ 2.1} & \textcolor{retained}{+ 1.9} & \textcolor{retained}{+ 2.2} & \textcolor{retained}{+ 10.3} & \textcolor{retained}{+ 2.5} & \textcolor{retained}{+ 6.6} & \textcolor{retained}{+ 1.2} & \textcolor{retained}{+ 3.4} & \textcolor{retained}{+ 0.7} & \textcolor{retained}{+ 0.4} & \textcolor{retained}{+ 2.1} & \textcolor{retained}{+ 3.0} & \textcolor{retained}{+ 2.8}  \\
        
        \bottomrule
    \end{tabular}
}
  % \vspace{-6pt}
  \caption{\textbf{Performance improvement of description methods by our ProAPO.} 
  Avg (11) and Avg (13) denote average results across 11 datasets (excluding CUB~\cite{CUB} and Places~\cite{Places365}) and all 13 datasets, respectively. $\Delta$ denotes performance gains compared to baseline.}
  \vspace{40pt}
  \label{tab:description_add_APO}
\end{table*}
}



{
\renewcommand{\arraystretch}{1.1} 
% \setlength{\tabcolsep}{4pt}
\begin{table*}[t]
  \centering
  \resizebox{0.99\linewidth}{!}
    {
    \begin{tabular}
        {l c c c c c | ccccc ccccc ccc | c | c }    
        \toprule
        \multicolumn{6}{c}{\textbf{Component}}  &   \\
        \cmidrule(lr){1-6} 
        & \texttt{Add} & \texttt{Del} & \texttt{Rep} & \texttt{Cross} & \texttt{Mut}  & \rotatebox{90}{\textbf{IN-1K}} & \rotatebox{90}{\textbf{Caltech}} & \rotatebox{90}{\textbf{Cars}} & \rotatebox{90}{\textbf{CUB}} & \rotatebox{90}{\textbf{DTD}}  & \rotatebox{90}{\textbf{ESAT}} & \rotatebox{90}{\textbf{FGVC}} & \rotatebox{90}{\textbf{FLO}} & \rotatebox{90}{\textbf{Food}}  &  \rotatebox{90}{\textbf{Pets}} & \rotatebox{90}{\textbf{Places}} & \rotatebox{90}{\textbf{SUN}} & \rotatebox{90}{\textbf{UCF}} & \rotatebox{90}{\textbf{Avg (11)}} & \rotatebox{90}{\textbf{Avg (13)}}  \\
        \midrule
       \multicolumn{5}{l}{Vanilla CLIP (ViT-B/32)} & & 62.1  & 91.2  & 60.4  & 51.7 & 42.9  & 43.9  & 20.2  & 66.0  & 83.2  & 86.8 & 39.9 & 62.1  & 60.9 & 61.8 & 59.3  \\ 
       \midrule
       \multicolumn{5}{l}{\textbf{\textit{edit-based generation}}} \\
       \texttt{a)} & \cmark & & & & & 63.8 & 93.6 & 60.0 & 54.6 & 51.8 & 59.0 & 21.8 & 74.0 & 82.2 & 86.7 & 43.0 & 65.7 & 66.8 & 66.0  & 63.3  \\ 
       \texttt{b)} & \cmark & \cmark & & & & \underline{64.6} & 94.0 & 60.9 & 55.0 & 52.6 & 59.3 & 21.8 & 72.0 & 83.2 & \underline{88.0} & 43.2 & 66.4 & 68.0 & 66.4  & 63.8 \\
       \texttt{c)} & \cmark &  & \cmark & & & 64.4 & 94.0 & 61.0 & \underline{55.2} & 52.3 & 59.7 & 22.4 & 71.9 & 84.0 & 87.7 & 43.2 & 66.4 & 67.8 & 66.5  & 63.8 \\
       \texttt{d)} & \cmark & \cmark & \cmark & & & \underline{64.6}  & 93.6 & 60.8 & 54.4 & 53.1 & 60.1 & 22.2 & \textbf{74.7} & 82.4 & 87.2 & \textbf{43.4} & 66.5 & \underline{68.6} & 66.7  & 64.0  \\ 
        \midrule
        \multicolumn{5}{l}{\textbf{\textit{evolution-based generation}}} \\
        \texttt{e)} & \cmark & \cmark & \cmark & \cmark & & \underline{64.6} & \underline{94.3} & 61.2 & 55.0 & \underline{53.2} & \underline{62.6} & \underline{22.9} & 73.9 & \underline{84.3} & \underline{88.0} & 43.1 & \textbf{66.8} & 68.5 & \underline{67.3}  & \underline{64.5} \\ 
        \texttt{f)} & \cmark & \cmark & \cmark & & \cmark &  \textbf{64.7} & \underline{94.3} & \underline{61.4} & 55.1 & 52.9 & 61.4 & 22.6 & 74.0 & 83.6 & 87.7 & \textbf{43.4} & \underline{66.7} & 68.3 & 67.1  & 64.3   \\ 
        \highlight{\texttt{g)}} & \highlight{\cmark} & \highlight{\cmark} & \highlight{\cmark} & \highlight{\cmark} & \highlight{\cmark} & \highlight{\textbf{64.7}}  & \highlight{\textbf{94.4}} & \highlight{\textbf{61.7}} & \highlight{\textbf{55.4}} & \highlight{\textbf{53.5}} & \highlight{\textbf{63.0}} & \highlight{\textbf{23.0}} & \highlight{\underline{74.3}} & \highlight{\textbf{85.3}} & \highlight{\textbf{91.0}} & \highlight{\underline{43.3}} & \highlight{{66.6}} & \highlight{\textbf{69.0}} & \highlight{\textbf{67.9}}  & \highlight{\textbf{65.0}}   \\
        \bottomrule
    \end{tabular}
}
% \vspace{-6pt}
\caption{\textbf{Ablation of edit- and evolution-based operators.}}
\vspace{50pt}
\label{supp_tab: ablation_generate}
\end{table*}
}




{
\renewcommand{\arraystretch}{1.1} 
\begin{table*}[tbp]
  \centering
  \resizebox{0.99\linewidth}{!}
    {
    \begin{tabular}
        {l | ccccc ccccc ccc | c | c | c}  
        \toprule
        {\textbf{Module} (ViT-B/32)}  & \rotatebox{90}{\textbf{IN-1K}} & \rotatebox{90}{\textbf{Caltech}} & \rotatebox{90}{\textbf{Cars}} & \rotatebox{90}{\textbf{CUB}} & \rotatebox{90}{\textbf{DTD}}  & \rotatebox{90}{\textbf{ESAT}} & \rotatebox{90}{\textbf{FGVC}} & \rotatebox{90}{\textbf{FLO}} & \rotatebox{90}{\textbf{Food}}  &  \rotatebox{90}{\textbf{Pets}} & \rotatebox{90}{\textbf{Places}} & \rotatebox{90}{\textbf{SUN}} & \rotatebox{90}{\textbf{UCF}} & \rotatebox{90}{\textbf{Avg (11)}} & \rotatebox{90}{\textbf{Avg (13)}} & \textbf{Times} \\
        \midrule
         
        \texttt{a)} w/o prompt sampling & 64.4 & 93.8 & \textbf{61.8} & \underline{55.4} & 51.8 & 60.0 & \underline{23.2} & 74.0 & 85.1 & 90.7 & \underline{43.0} & 66.0 & 69.3 & 67.3 & 64.5 & 12 min \\
        
        \texttt{b)} w/o group sampling & \textbf{64.8} & \textbf{94.5} & \underline{61.7} & \textbf{55.5} & \textbf{53.6} & \textbf{63.5} & \underline{23.2} & \underline{75.3} & \textbf{85.4} & 90.8 & \textbf{43.3} & \textbf{66.7} & \textbf{69.8} & \textbf{68.1}  & \textbf{65.2} & \textbf{306 min} \\ 
        
        \texttt{c)} w/o sampling strategies & 64.5 & 93.4 & 57.4 & 54.8 & \textbf{53.6} & \underline{63.2} & \textbf{23.4} & \textbf{76.8} & 83.8 & 86.9 & \textbf{43.3} & 66.1 & \underline{69.7} & 67.2  & 64.4 & \underline{302 min} \\

        \midrule
        
        \highlight{\textbf{ProAPO} (full model)} & \highlight{\underline{64.7}}  & \highlight{\underline{94.4}} & \highlight{\underline{61.7}} & \highlight{\underline{55.4}} & \highlight{\underline{53.5}} & \highlight{{63.0}} & \highlight{{23.0}} & \highlight{{74.3}} & \highlight{\underline{85.3}} & \highlight{\textbf{91.0}} & \highlight{\textbf{43.3}} & \highlight{\underline{66.6}} & \highlight{{69.0}} & \highlight{\underline{67.9}}  & \highlight{\underline{65.0}} & \highlight{15 min} \\
        \bottomrule
    \end{tabular}
}
% \vspace{-6pt}
  \caption{\textbf{Ablation of two sampling strategies.}}
% \vspace{-6pt}
  \label{supp_tab: ablation_sample}
\end{table*}
}


{
\renewcommand{\arraystretch}{1.1} 
\begin{table*}[tbp]
  \centering
  \resizebox{0.99\linewidth}{!}
    {
    \begin{tabular}
        {l | ccccc ccccc ccc | c | c }  
        \toprule
        {\textbf{Module} (ViT-B/32)}  & \rotatebox{90}{\textbf{IN-1K}} & \rotatebox{90}{\textbf{Caltech}} & \rotatebox{90}{\textbf{Cars}} & \rotatebox{90}{\textbf{CUB}} & \rotatebox{90}{\textbf{DTD}}  & \rotatebox{90}{\textbf{ESAT}} & \rotatebox{90}{\textbf{FGVC}} & \rotatebox{90}{\textbf{FLO}} & \rotatebox{90}{\textbf{Food}}  &  \rotatebox{90}{\textbf{Pets}} & \rotatebox{90}{\textbf{Places}} & \rotatebox{90}{\textbf{SUN}} & \rotatebox{90}{\textbf{UCF}} & \rotatebox{90}{\textbf{Avg (11)}} & \rotatebox{90}{\textbf{Avg (13)}} \\
        \midrule
         
        \texttt{a)} w/ only accuracy & 64.0 & 93.0 & 60.8 & 54.2 & 49.1 & 55.5 & 20.4 & 68.3 & 84.8 & 88.4 & 41.9 & 64.6 & 65.1 & 64.9 & 62.3 \\
        
        \texttt{b)} w/ only entropy constrain & 64.3 & 93.4 & 61.6 & 54.8 & 49.3 & 56.7 & 22.3 & 69.9 & 85.2 & 89.1 & 42.4 & 65.1 & 66.7 & 65.8 & 63.1 \\ 
        
        \midrule
        
        \highlight{\textbf{ProAPO} (full model)} & \highlight{\textbf{64.7}}  & \highlight{\textbf{94.4}} & \highlight{\textbf{61.7}} & \highlight{\textbf{55.4}} & \highlight{\textbf{53.5}} & \highlight{\textbf{63.0}} & \highlight{\textbf{23.0}} & \highlight{\textbf{74.3}} & \highlight{\textbf{85.3}} & \highlight{\textbf{91.0}} & \highlight{\textbf{43.3}} & \highlight{\textbf{66.6}} & \highlight{\textbf{69.0}} & \highlight{\textbf{67.9}}  & \highlight{\textbf{65.0}}  \\
        \bottomrule
    \end{tabular}
}
% \vspace{-6pt}
  \caption{\textbf{Ablation of different score functions.}}
% \vspace{-8pt}
  \label{supp_tab: ablation_score_function}
\end{table*}
}


% ---------------------------------------------------- % 
%               Ablation Study 的详细实验
% ---------------------------------------------------- %
% \newpage
\section{Detailed Results of Ablation Study}

% ---------------------------------------------------- % 
%             对 Edit- 和 Evolution- 操作的消融
% ---------------------------------------------------- %
\subsection{Ablation of Edit- and Evolution-based Operators}
\label{supp_sec: ablation_operator}
In~\cref{supp_tab: ablation_generate}, we show detailed ablation results of edit- and evolution-based operators. For edit-based operators, we observe that the model with add, delete, and replace operations achieves a higher result in row d). After introducing evolution-based operators, \textit{i.e.}, crossover operator to combine advantages of high-scoring candidates, and mutation operator to avoid locally optimal solutions, we see an increase in performance in rows e)-g). It confirms that evolution-based operators make the model search the optimal prompt faster with limited iterations.




% ---------------------------------------------------- % 
%                对 Sampling 策略的消融
% ---------------------------------------------------- %
\subsection{Ablation of Two Sampling Strategies}
\label{supp_sec: ablation_two_sampling}

In~\cref{supp_tab: ablation_sample}, we show detailed ablation results of two sampling strategies. Without the prompt sampling, we see a slight decrease in times while results drop in row a). It verifies the effectiveness of the prompt sampling. Without the group sampling to select salient categories for optimization, we observe a notable increase in time costs (from 15 min to 300+ min, 20 times) yet similar results in row b) and the full model. It reveals that group sampling simultaneously improves performance and efficiency.




% ---------------------------------------------------- % 
%               对 Score Function 的消融
% ---------------------------------------------------- %
\subsection{Ablation of Different Score Functions}
\label{supp_sec: effect_score_func}

In~\cref{supp_tab: ablation_score_function}, we show detailed ablation results of score functions. Accuracy obtains the worst result as the score function due to the overfitting problem. Our full model with accuracy and entropy constraints as the score function achieves the SOTA result. The score function with only accuracy or entropy constraint achieves suboptimal results, suggesting a trade-off process between them. 






% WARNING: do not forget to delete the supplementary pages from your submission 
% \clearpage
\pagenumbering{gobble}
\maketitlesupplementary

\section{Additional Results on Embodied Tasks}

To evaluate the broader applicability of our EgoAgent's learned representation beyond video-conditioned 3D human motion prediction, we test its ability to improve visual policy learning for embodiments other than the human skeleton.
Following the methodology in~\cite{majumdar2023we}, we conduct experiments on the TriFinger benchmark~\cite{wuthrich2020trifinger}, which involves a three-finger robot performing two tasks: reach cube and move cube. 
We freeze the pretrained representations and use a 3-layer MLP as the policy network, training each task with 100 demonstrations.

\begin{table}[h]
\centering
\caption{Success rate (\%) on the TriFinger benchmark, where each model's pretrained representation is fixed, and additional linear layers are trained as the policy network.}
\label{tab:trifinger}
\resizebox{\linewidth}{!}{%
\begin{tabular}{llcc}
\toprule
Methods       & Training Dataset & Reach Cube & Move Cube \\
\midrule
DINO~\cite{caron2021emerging}         & WT Venice        & 78.03     & 47.42     \\
DoRA~\cite{venkataramanan2023imagenet}          & WT Venice        & 81.62     & 53.76     \\
DoRA~\cite{venkataramanan2023imagenet}          & WT All           & 82.40     & 48.13     \\
\midrule
EgoAgent-300M & WT+Ego-Exo4D      & 82.61    & 54.21      \\
EgoAgent-1B   & WT+Ego-Exo4D      & \textbf{85.72}      & \textbf{57.66}   \\
\bottomrule
\end{tabular}%
}
\end{table}

As shown in Table~\ref{tab:trifinger}, EgoAgent achieves the highest success rates on both tasks, outperforming the best models from DoRA~\cite{venkataramanan2023imagenet} with increases of +3.32\% and +3.9\% respectively.
This result shows that by incorporating human action prediction into the learning process, EgoAgent demonstrates the ability to learn more effective representations that benefit both image classification and embodied manipulation tasks.
This highlights the potential of leveraging human-centric motion data to bridge the gap between visual understanding and actionable policy learning.



\section{Additional Results on Egocentric Future State Prediction}

In this section, we provide additional qualitative results on the egocentric future state prediction task. Additionally, we describe our approach to finetune video diffusion model on the Ego-Exo4D dataset~\cite{grauman2024ego} and generate future video frames conditioned on initial frames as shown in Figure~\ref{fig:opensora_finetune}.

\begin{figure}[b]
    \centering
    \includegraphics[width=\linewidth]{figures/opensora_finetune.pdf}
    \caption{Comparison of OpenSora V1.1 first-frame-conditioned video generation results before and after finetuning on Ego-Exo4D. Fine-tuning enhances temporal consistency, but the predicted pixel-space future states still exhibit errors, such as inaccuracies in the basketball's trajectory.}
    \label{fig:opensora_finetune}
\end{figure}

\subsection{Visualizations and Comparisons}

More visualizations of our method, DoRA, and OpenSora in different scenes (as shown in Figure~\ref{fig:supp pred}). For OpenSora, when predicting the states of $t_k$, we use all the ground truth frames from $t_{0}$ to $t_{k-1}$ as conditions. As OpenSora takes only past observations as input and neglects human motion, it performs well only when the human has relatively small motions (see top cases in Figure~\ref{fig:supp pred}), but can not adjust to large movements of the human body or quick viewpoint changes (see bottom cases in Figure~\ref{fig:supp pred}).

\begin{figure*}
    \centering
    \includegraphics[width=\linewidth]{figures/supp_pred.pdf}
    \caption{Retrieval and generation results for egocentric future state prediction. Correct and wrong retrieval images are marked with green and red boundaries, respectively.}
    \label{fig:supp pred}
\end{figure*}

\begin{figure*}[t]
    \centering
    \includegraphics[width=0.9\linewidth]{figures/motion_prediction.pdf}
    \vspace{-0.5mm}
    \caption{Motion prediction results in scenes with minor changes in observation.}
    \vspace{-1.5mm}
    \label{fig:motion_prediction}
\end{figure*}

\subsection{Finetuning OpenSora on Ego-Exo4D}

OpenSora V1.1~\cite{opensora}, initially trained on internet videos and images, produces severely inconsistent results when directly applied to infer future videos on the Ego-Exo4D dataset, as illustrated in Figure~\ref{fig:opensora_finetune}.
To address the gap between general internet content and egocentric video data, we fine-tune the official checkpoint on the Ego-Exo4D training set for 50 epochs.
OpenSora V1.1 proposed a random mask strategy during training to enable video generation by image and video conditioning. We adopted the default masking rate, which applies: 75\% with no masking, 2.5\% with random masking of 1 frame to 1/4 of the total frames, 2.5\% with masking at either the beginning or the end for 1 frame to 1/4 of the total frames, and 5\% with random masking spanning 1 frame to 1/4 of the total frames at both the beginning and the end.

As shown in Fig.~\ref{fig:opensora_finetune}, despite being trained on a large dataset, OpenSora struggles to generalize to the Ego-Exo4D dataset, producing future video frames with minimal consistency relative to the conditioning frame. While fine-tuning improves temporal consistency, the moving trajectories of objects like the basketball and soccer ball still deviate from realistic physical laws. Compared with our feature space prediction results, this suggests that training world models in a reconstructive latent space is more challenging than training them in a feature space.


\section{Additional Results on 3D Human Motion Prediction}

We present additional qualitative results for the 3D human motion prediction task, highlighting a particularly challenging scenario where egocentric observations exhibit minimal variation. This scenario poses significant difficulties for video-conditioned motion prediction, as the model must effectively capture and interpret subtle changes. As demonstrated in Fig.~\ref{fig:motion_prediction}, EgoAgent successfully generates accurate predictions that closely align with the ground truth motion, showcasing its ability to handle fine-grained temporal dynamics and nuanced contextual cues.

\section{OpenSora for Image Classification}

In this section, we detail the process of extracting features from OpenSora V1.1~\cite{opensora} (without fine-tuning) for an image classification task. Following the approach of~\cite{xiang2023denoising}, we leverage the insight that diffusion models can be interpreted as multi-level denoising autoencoders. These models inherently learn linearly separable representations within their intermediate layers, without relying on auxiliary encoders. The quality of the extracted features depends on both the layer depth and the noise level applied during extraction.


\begin{table}[h]
\centering
\caption{$k$-NN evaluation results of OpenSora V1.1 features from different layer depths and noising scales on ImageNet-100. Top1 and Top5 accuracy (\%) are reported.}
\label{tab:opensora-knn}
\resizebox{0.95\linewidth}{!}{%
\begin{tabular}{lcccccc}
\toprule
\multirow{2}{*}{Timesteps} & \multicolumn{2}{c}{First Layer} & \multicolumn{2}{c}{Middle Layer} & \multicolumn{2}{c}{Last Layer} \\
\cmidrule(r){2-3}   \cmidrule(r){4-5}  \cmidrule(r){6-7}  & Top1           & Top5           & Top1            & Top5           & Top1           & Top5          \\
\midrule
32        &  6.10           & 18.20             & 34.04               & 59.50             & 30.40             & 55.74             \\
64        & 6.12              & 18.48              & 36.04               & 61.84              & 31.80         & 57.06         \\
128       & 5.84             & 18.14             & 38.08               & 64.16              & 33.44       & 58.42 \\
256       & 5.60             & 16.58              & 30.34               & 56.38              &28.14          & 52.32        \\
512       & 3.66              & 11.70            & 6.24              & 17.62              & 7.24              & 19.44  \\ 
\bottomrule
\end{tabular}%
}
\end{table}

As shown in Table~\ref{tab:opensora-knn}, we first evaluate $k$-NN classification performance on the ImageNet-100 dataset using three intermediate layers and five different noise scales. We find that a noise timestep of 128 yields the best results, with the middle and last layers performing significantly better than the first layer.
We then test this optimal configuration on ImageNet-1K and find that the last layer with 128 noising timesteps achieves the best classification accuracy.

\section{Data Preprocess}
For egocentric video sequences, we utilize videos from the Ego-Exo4D~\cite{grauman2024ego} and WT~\cite{venkataramanan2023imagenet} datasets.
The original resolution of Ego-Exo4D videos is 1408×1408, captured at 30 fps. We sample one frame every five frames and use the original resolution to crop local views (224×224) for computing the self-supervised representation loss. For computing the prediction and action loss, the videos are downsampled to 224×224 resolution.
WT primarily consists of 4K videos (3840×2160) recorded at 60 or 30 fps. Similar to Ego-Exo4D, we use the original resolution and downsample the frame rate to 6 fps for representation loss computation.
As Ego-Exo4D employs fisheye cameras, we undistort the images to a pinhole camera model using the official Project Aria Tools to align them with the WT videos.

For motion sequences, the Ego-Exo4D dataset provides synchronized 3D motion annotations and camera extrinsic parameters for various tasks and scenes. While some annotations are manually labeled, others are automatically generated using 3D motion estimation algorithms from multiple exocentric views. To maximize data utility and maintain high-quality annotations, manual labels are prioritized wherever available, and automated annotations are used only when manual labels are absent.
Each pose is converted into the egocentric camera's coordinate system using transformation matrices derived from the camera extrinsics. These matrices also enable the computation of trajectory vectors for each frame in a sequence. Beyond the x, y, z coordinates, a visibility dimension is appended to account for keypoints invisible to all exocentric views. Finally, a sliding window approach segments sequences into fixed-size windows to serve as input for the model. Note that we do not downsample the frame rate of 3D motions.

\section{Training Details}
\subsection{Architecture Configurations}
In Table~\ref{tab:arch}, we provide detailed architecture configurations for EgoAgent following the scaling-up strategy of InternLM~\cite{team2023internlm}. To ensure the generalization, we do not modify the internal modules in InternML, \emph{i.e.}, we adopt the RMSNorm and 1D RoPE. We show that, without specific modules designed for vision tasks, EgoAgent can perform well on vision and action tasks.

\begin{table}[ht]
  \centering
  \caption{Architecture configurations of EgoAgent.}
  \resizebox{0.8\linewidth}{!}{%
    \begin{tabular}{lcc}
    \toprule
          & EgoAgent-300M & EgoAgent-1B \\
          \midrule
    Depth & 22    & 22 \\
    Embedding dim & 1024  & 2048 \\
    Number of heads & 8     & 16 \\
    MLP ratio &    8/3   & 8/3 \\
    $\#$param.  & 284M & 1.13B \\
    \bottomrule
    \end{tabular}%
    }
  \label{tab:arch}%
\end{table}%

Table~\ref{tab:io_structure} presents the detailed configuration of the embedding and prediction modules in EgoAgent, including the image projector ($\text{Proj}_i$), representation head/state prediction head ($\text{MLP}_i$), action projector ($\text{Proj}_a$) and action prediction head ($\text{MLP}_a$).
Note that the representation head and the state prediction head share the same architecture but have distinct weights.

\begin{table}[t]
\centering
\caption{Architecture of the embedding ($\text{Proj}_i$, $\text{Proj}_a$) and prediction ($\text{MLP}_i$, $\text{MLP}_a$) modules in EgoAgent. For details on module connections and functions, please refer to Fig.~2 in the main paper.}
\label{tab:io_structure}
\resizebox{\linewidth}{!}{%
\begin{tabular}{lcl}
\toprule
       & \multicolumn{1}{c}{Norm \& Activation} & \multicolumn{1}{c}{Output Shape}  \\
\midrule
\multicolumn{3}{l}{$\text{Proj}_i$ (\textit{Image projector})} \\
\midrule
Input image  & -          & 3$\times$224$\times$224 \\
Conv 2D (16$\times$16) & -       & Embedding dim$\times$14$\times$14    \\
\midrule
\multicolumn{3}{l}{$\text{MLP}_i$ (\textit{State prediction head} \& \textit{Representation head)}} \\
\midrule
Input embedding  & -          & Embedding dim \\
Linear & GELU       & 2048          \\
Linear & GELU       & 2048          \\
Linear & -          & 256           \\
Linear & -          & 65536     \\
\midrule
\multicolumn{3}{l}{$\text{Proj}_a$ (\textit{Action projector})} \\
\midrule
Input pose sequence  & -          & 4$\times$5$\times$17 \\
Conv 2D (5$\times$17) & LN, GELU   & Embedding dim$\times$1$\times$1    \\
\midrule
\multicolumn{3}{l}{$\text{MLP}_a$ (\textit{Action prediction head})} \\
\midrule
Input embedding  & -          & Embedding dim$\times$1$\times$1 \\
Linear & -          & 4$\times$5$\times$17     \\
\bottomrule
\end{tabular}%
}
\end{table}


\subsection{Training Configurations}
In Table~\ref{tab:training hyper}, we provide the detailed training hyper-parameters for experiments in the main manuscripts.

\begin{table}[ht]
  \centering
  \caption{Hyper-parameters for training EgoAgent.}
  \resizebox{0.86\linewidth}{!}{%
    \begin{tabular}{lc}
    \toprule
    Training Configuration & EgoAgent-300M/1B \\
    \midrule
    Training recipe: &  \\
    optimizer & AdamW~\cite{loshchilov2017decoupled} \\
    optimizer momentum & $\beta_1=0.9, \beta_2=0.999$ \\
    \midrule
    Learning hyper-parameters: &  \\
    base learning rate & 6.0E-04 \\
    learning rate schedule & cosine \\
    base weight decay & 0.04 \\
    end weight decay & 0.4 \\
    batch size & 1920 \\
    training iters & 72,000 \\
    lr warmup iters & 1,800 \\
    warmup schedule & linear \\
    gradient clip & 1.0 \\
    data type & float16 \\
    norm epsilon & 1.0E-06 \\
    \midrule
    EMA hyper-parameters: &  \\
    momentum & 0.996 \\
    \bottomrule
    \end{tabular}%
    }
  \label{tab:training hyper}%
\end{table}%

\clearpage


\end{document}
