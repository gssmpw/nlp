% CVPR 2025 Paper Template; see https://github.com/cvpr-org/author-kit

\documentclass[10pt,twocolumn,letterpaper]{article}

%%%%%%%%% PAPER TYPE  - PLEASE UPDATE FOR FINAL VERSION
\usepackage{cvpr}              % To produce the CAMERA-READY version
% \usepackage[review]{cvpr}      % To produce the REVIEW version
% \usepackage[pagenumbers]{cvpr} % To force page numbers, e.g. for an arXiv version



% Import additional packages in the preamble file, before hyperref
\newcommand{\CG}{\mathcal{G}\xspace}
\newcommand{\CV}{\mathcal{V}\xspace}
\newcommand{\CE}{\mathcal{E}\xspace}
\newcommand{\CA}{\mathcal{A}\xspace}
\newcommand{\CF}{\mathcal{F}\xspace}
\newcommand{\CR}{\mathcal{R}\xspace}
\newcommand{\CB}{\mathcal{B}\xspace}
\newcommand{\CX}{\mathcal{X}\xspace}
\newcommand{\CK}{\mathcal{K}\xspace}
\newcommand{\CM}{\mathcal{M}\xspace}
\newcommand{\CC}{\mathcal{C}\xspace}
\newcommand{\CL}{\mathcal{L}\xspace}
\newcommand{\CI}{\mathcal{I}\xspace}
\newcommand{\CQ}{\mathcal{Q}\xspace}
\newcommand{\CO}{\mathcal{O}\xspace}
\newcommand{\CP}{\mathcal{P}\xspace}
\newcommand{\CS}{\mathcal{S}\xspace}
\newcommand{\CT}{\mathcal{T}\xspace}
\newcommand{\CJ}{\mathcal{J}\xspace}
\usepackage[para]{footmisc}
\usepackage{subfig}
% \usepackage{subcaption}
% \usepackage{array}
% \usepackage{colortbl}



% It is strongly recommended to use hyperref, especially for the review version.
% hyperref with option pagebackref eases the reviewers' job.
% Please disable hyperref *only* if you encounter grave issues, 
% e.g. with the file validation for the camera-ready version.
%
% If you comment hyperref and then uncomment it, you should delete *.aux before re-running LaTeX.
% (Or just hit 'q' on the first LaTeX run, let it finish, and you should be clear).
\definecolor{cvprblue}{rgb}{0.21,0.49,0.74}
\usepackage[pagebackref,breaklinks,colorlinks,allcolors=cvprblue]{hyperref}

% ----- CVPR 官方推荐的一个包 ----- % 
\usepackage[accsupp]{axessibility}

%% ========== 导入其他的可用包 ========== % 
% 引用 URL 的包 %
\usepackage{url}
\usepackage{hyperref}

% 引入在表格中, 可以横跨多行的包
\usepackage{multirow} % 跨越多行后居中
\usepackage{makecell} % 在一个单元格中居中

\usepackage{adjustbox}


% 控制表格的颜色 
\usepackage{colortbl}
\usepackage{xcolor}
\usepackage{color}
\usepackage{array}   %对表列和表格线的设置需要用到array宏包

% 画图相关的包 
% \usepackage{subfig}
\usepackage{float}

% nicer tables
\usepackage{nicematrix}  
\usepackage{tabularx}

% 伪代码相关 
\usepackage{algorithmic}
\usepackage{algorithm}
\usepackage{caption}


% 导入特殊符号的包
\usepackage{pifont}


% ----- 定义颜色相关 ----- % 
% \definecolor{unit01green}{RGB}{82,208,83}
% \newcommand{\green}[1]{\textcolor{unit01green}{#1}}

% \definecolor{unit02red}{RGB}{211,41,15}
% \newcommand{\red}[1]{\textcolor{unit02red}{#1}}

% \definecolor{unit02blue}{RGB}{53,49,255}
% \newcommand{\blue}[1]{\textcolor{unit02blue}{#1}}


\newcommand{\highlight}[1]{{\cellcolor[rgb]{0.925,0.957,1}}{#1}}

% \definecolor{C1}{RGB}{220, 53, 34} %{182, 94, 42}
% \definecolor{C3}{RGB}{0, 128, 20} % {210, 105, 30} % {138, 43, 219} % {210, 105, 30}
% \definecolor{C2}{RGB}{139,69,19} % {0, 128, 9} % {0, 100, 0}
% \definecolor{C4}{RGB}{26, 111, 223}%{141, 82, 158}

\definecolor{green}{rgb}{0, 0.8, 0.2}
\newcommand{\cmark}{\textcolor{green}{\ding{51}}} %

\definecolor{red}{rgb}{1.0, 0.0, 0.0}
\newcommand{\xmark}{\textcolor{red}{\ding{55}}} %

\definecolor{grey}{rgb}{0.9, 0.9, 0.9}
\newcommand{\ccol}{\cellcolor{grey}}

\definecolor{lightblue}{rgb}{0.68, 0.85, 0.9}

\definecolor{softgray}{gray}{0.96}


\definecolor{Template}{RGB}{84, 130, 53}
\definecolor{Description}{RGB}{197, 90, 17}


\definecolor{removed}{RGB}{192, 0, 0}
\definecolor{retained}{RGB}{0, 112, 192}


%%%%%%%%% PAPER ID  - PLEASE UPDATE
\def\paperID{4381} % *** Enter the Paper ID here
\def\confName{CVPR}
\def\confYear{2025}

%%%%%%%%% TITLE - PLEASE UPDATE
\title{ProAPO: Progressively Automatic Prompt Optimization for Visual Classification}

\setcounter{footnote}{1}

%%%%%%%%% AUTHORS - PLEASE UPDATE
\author{%
  Xiangyan Qu$^{12}$ \quad Gaopeng Gou$^{12}$\thanks{Corresponding author} \quad Jiamin Zhuang$^{12}$ \quad Jing Yu$^{3}$  \\
   Kun Song$^{4}$ \quad Qihao Wang$^{12}$ \quad Yili Li$^{12}$ \quad Gang Xiong$^{12}$ \\ 
  \small $^1$Institute of Information Engineering, Chinese Academy of Sciences \quad 
  \small $^2$School of Cyber Security, Chinese Academy of Sciences \\ 
  \small $^3$School of Information Engineering, Minzu University of China \quad 
  \small $^4$University of Science and Technology Beijing \\
  \tt\small \{quxiangyan, gougaopeng, zhuangjiamin\}@iie.ac.cn, jing.yu@muc.edu.cn \\
  \tt\small songkun@xs.ustb.edu.cn, wangqihao22@mails.ucas.ac.cn, \{liyili, xionggang\}@iie.ac.cn
  \vspace{-4mm}
}

% \author{
%     Yaochen Ren\textsuperscript{\dag,\ddag}, 
%     Gaopeng Gou\textsuperscript{\dag,\ddag}, 
%     Chengshang Hou\textsuperscript{\dag,\ddag}, 
%     Tianyu Cui\textsuperscript{\S}, \\
%     Zhen Li\textsuperscript{\dag,\ddag}, 
%     Gang Xiong\textsuperscript{\dag,\ddag} and  
%     Chang Liu\textsuperscript{\dag,\ddag}(\Letter) \\
%     \textsuperscript{\dag}Institute of Information Engineering, Chinese Academy of Sciences, Beijing, China \\
%     \textsuperscript{\ddag}School of Cyber Security, University of Chinese Academy of Sciences, Beijing, China \\
%     \{renyaochen, gougaopeng, houchengshang, lizhen, xionggang, liuchang\}@iie.ac.cn\\
%     \textsuperscript{\S}Zhongguancun Laboratory, Beijing, China \{cuity\}@zgclab.edu.cn}


% \author{First Author\\
% Institution1\\
% Institution1 address\\
% {\tt\small firstauthor@i1.org}
% % For a paper whose authors are all at the same institution,
% % omit the following lines up until the closing ``}''.
% % Additional authors and addresses can be added with ``\and'',
% % just like the second author.
% % To save space, use either the email address or home page, not both
% \and
% Second Author\\
% Institution2\\
% First line of institution2 address\\
% {\tt\small secondauthor@i2.org}
% }

\begin{document}
\maketitle


\begin{abstract}

% Recent works to jointly reconstruct 3D human and object from a single RGB image, are mostly model-based, that fail to capture the fine details of the clothed human body and object surface. In this paper, we introduce ReCHOR, a novel, model-free, first-method to produce realistic clothed human-object reconstructions from a monocular view. This is extremely challenging due to human-object occlusions, diverse interactions and depth ambiguity, as it needs to infer both 3D spatial awareness and high resolution details. Our core idea is based on estimating neural implicit representations for human and object respectively by an attention-based neural implicit model that attends to pixel-aligned features from both the global human-object image for spatial awareness and  the local separate view of human and object images for high quality details. Additionally, the network is conditioned on semantic features from an initial estimated human-object pose prior and a generative diffusion model that inpaints occluded regions, thus enabling the retrieval of details from them.
% We also propose a synthetic dataset with rendered scenes of diverse, inter-occluded 3D human and object scans, to train our network. We evaluate our method on the synthetic and real world BEHAVE dataset. Our experiments show that our method outperforms the SOTA in achieving realistic clothed human-object reconstructions.
Recent approaches to jointly reconstruct 3D humans and objects from a single RGB image represent 3D shapes with template-based or coarse models, which fail to capture details of loose clothing on human bodies. In this paper, we introduce a novel implicit approach for jointly reconstructing realistic 3D clothed humans and objects from a monocular view. For the first time, we model both the human and the object with an implicit representation, allowing to capture more realistic details such as clothing. This task is extremely challenging due to human-object occlusions and the lack of 3D information in 2D images, often leading to poor detail reconstruction and depth ambiguity. To address these problems, we propose a novel attention-based neural implicit model that leverages image pixel alignment from both the input human-object image for a global understanding of the human-object scene and from local separate views of the human and object images to improve realism with, for example, clothing details. Additionally, the network is conditioned on semantic features derived from an estimated human-object pose prior, which provides 3D spatial information about the shared space of humans and objects. To handle human occlusion caused by objects, we use a generative diffusion model that inpaints the occluded regions, recovering otherwise lost details. For training and evaluation, we introduce a synthetic dataset featuring rendered scenes of inter-occluded 3D human scans and diverse objects. Extensive evaluation on both synthetic and real-world datasets demonstrates the superior quality of the proposed human-object reconstructions over competitive methods.
\end{abstract}    
\section{Introduction}
\label{sec:intro}
% Image editing methods in diffusion models depend on user-defined control directions - users can unlock their creativity using these methods by specifying the desired manipulation through prompts~\cite{gandikota2023concept}, reference images~\cite{ruiz2022dreambooth, kumari2022customdiffusion, gal2022image, chen2024trainingfreeregionalpromptingdiffusion}, or attribute vectors~\cite{parmar2023zero,hertz2022prompt}. In this work, we ask a fundamentally different question: \emph{Can we automatically discover the underlying visual structure of a concept within diffusion model's knowledge?} %Rather than requiring user-specified controls, we aim to decompose the model's internal knowledge into meaningful directions.

% This question touches on a fundamental limitation in how we interact with diffusion models. Current control methods ~\cite{zhang2023addingconditionalcontroltexttoimage, gandikota2023concept, ye2023ipadaptertextcompatibleimage,ye2023ipadaptertextcompatibleimage, hertz2024stylealignedimagegeneration, li2023photomaker, shi2024instantbooth, chen2024trainingfreeregionalpromptingdiffusion} require users to specify their desired manipulations in advance, limiting interactive creativity. This contrasts with natural human artistic workflows, where creators dynamically explore creative ideas while jointly refining them toward meaningful artistic outcomes~\cite{hoffmann2016modeling}. This synergy between specification and exploration is not new to generative models. Early GAN architectures naturally developed disentangled latent spaces that enabled continuous\cite{harkonen2020ganspace,radford2015unsupervised, wu2021stylespace, shen2020interfacegan}, compositional control over generated images. Users could explore these spaces to discover interesting variations that would be difficult to describe in words~\cite{wu2021stylespace}, then combine them to achieve their creative goals~\cite{grabe2022towards}. 


% While diffusion models have largely superseded GANs in conditional image synthesis~\cite{dhariwal2021diffusion},  their underlying structure remains less understood. Diffusion models achieve remarkable diversity through high-dimensional latents, unlike GANs' compact latent spaces.  With a single prompt, diffusion models can generate radically different variations through different random initializations of input noise. We ask - Is it possible to discover interpretable structure within this vast space of variations?

Text-to-image diffusion models are capable of generating remarkable visual variations from a single prompt through different random initializations. However, this vast creative potential remains largely opaque to users---while we can generate diverse images, we lack understanding of the underlying structure of these variations. This presents a fundamental challenge: how can we discover and expose the latent visual capabilities encoded within these models?

\let\thefootnote\relax \footnote{$^{*}$Correspondence to \texttt{gandikota.ro@northeastern.edu}}

The challenge touches on a key limitation in how we interact with diffusion models today. Current control methods require users to explicitly specify their desired edits in advance through prompts~\cite{gandikota2023concept}, reference images~\cite{zhang2023addingconditionalcontroltexttoimage, chen2024trainingfreeregionalpromptingdiffusion, ruiz2022dreambooth,kumari2022customdiffusion, Ryu_lora, hu2021lora}, or attribute vectors~\cite{ye2023ipadaptertextcompatibleimage, hertz2024stylealignedimagegeneration, li2023photomaker, shi2024instantbooth,parmar2023zero,hertz2022prompt}. That contrasts sharply with natural human creative workflows, where artists dynamically explore creative ideas and jointly refine them toward meaningful artistic outcomes~\cite{hoffmann2016modeling}. The need for pre-specified controls creates a barrier between users and the full creative potential of these models.

Interestingly, earlier generative models like GANs~\cite{gans,karras2019style,brock2018large} naturally developed more interpretable internal structures. Their compact latent spaces often exhibited emergent disentanglement~\cite{harkonen2020ganspace,radford2015unsupervised, wu2021stylespace, shen2020interfacegan}, enabling continuous and compositional control over generated images. Users could explore these spaces to discover interesting variations that would be difficult to describe in words~\cite{wu2021stylespace}, then combine them to achieve their creative goals~\cite{grabe2022towards}.

Diffusion models have largely superseded GANs in conditional image synthesis~\cite{dhariwal2021diffusion}, achieving greater diversity through much higher-dimensional latents. And yet an understanding of the underlying structure of these larger latent spaces has remained elusive. In this work, we ask a fundamental question: \emph{Can we automatically discover the visual structure within a diffusion model's knowledge of a concept?} Rather than requiring user-specified controls, we aim to decompose the model's internal representations into expressive directions that users can explore and combine.

To address these needs, we present \textbf{SliderSpace}, a framework that brings systematic explorability to diffusion models. Given just a text prompt, SliderSpace discovers a canonical set of meaningful, diverse, and controllable directions within the model's knowledge of that concept. Each direction is implemented as a low-rank adapter~\cite{hu2021lora} that can be scaled and composed with others, allowing users to explore and smoothly combine different aspects of variation, as shown in Figure~\ref{fig:intro}.

We ground SliderSpace discovery in three key requirements for meaningful decomposition of a diffusion model's visual manifold: 
\begin{enumerate}
    \item \textbf{Unsupervised Discovery:} The decomposition process should emerge from the intrinsic structure of the model's learned representation, rather than being guided by predefined attributes. This ensures we capture the true topology of the model's knowledge space rather than projecting our assumptions onto it.
    
    \item \textbf{Semantic Orthogonality:} Each discovered control must represent a distinct semantic direction. This is enforced in a semantic feature space, like CLIP, where every slider has an orthogonal effect in embeddings. This prevents discovering multiple controls that create similar semantic effects, making the system more efficient and easier.
    
    \item \textbf{Distribution Consistency:} Directions must induce consistent transformations across both random seeds and prompt variations. 
\end{enumerate}

These requirements naturally lead to our proposed framework, which we formalize in Section~\ref{sec:method}. As we show in our experiments, SliderSpace is architecture-agnostic, working with both conventional U-Net based models like Stable Diffusion~\cite{rombach2022high, rombach2022sd20, podell2023sdxl, turbo, dmd} and recent transformer-based architectures like Flux~\cite{flux}.

We demonstrate the expressiveness of SliderSpace through three applications: First, we show how SliderSpace can decompose high-level concepts into diverse and expressive components, revealing the natural axes of variation in the model's understanding. Second, we explore artistic style variation, where SliderSpace discovers directions that match or exceed the diversity of manually curated artist lists while being judged more useful by human evaluators. Finally, we show how SliderSpace can help reverse the mode collapse commonly observed in distilled diffusion models, restoring diversity while maintaining generation speed.

Beyond providing practical creative control, SliderSpace opens new avenues for understanding and utilizing the latent capabilities of diffusion models. By mapping these models' visual potential into intuitive, composable directions, we take a step toward making their creative possibilities more accessible and interpretable to users.

% Image editing methods in diffusion models unlock the creativity of users. In this work we ask an alternate question: \emph{Can we organize and expose what of the diffusion model is already capable of?}.
% Existing methods for controlling image generation typically require users to manually specify edit directions for desired changes. This process is time-consuming, requires technical expertise, and limits the spontaneity of the creative process. For instance, if a user wants to adjust the smile of a generated person, they must explicitly request this edit, often through imprecise prompt engineering or model fine-tuning. This approach of predefined controls or manual specifications restricts users from fully exploring the latent capabilities of the model. There may be interesting stylistic variations or attributes that the model can generate, but users have no easy way to discover or utilize these.

% Natural visual disentanglement was an emergent property in the latent space of Generative Adversarial Models (GANs) \cite{harkonen2020ganspace,radford2015unsupervised, wu2021stylespace, shen2020interfacegan}. In particular, it has been observed that StyleGAN~\cite{karras2019style} stylespace neurons offer detailed control over many meaningful aspects of images that would be difficult to describe in words~\cite{wu2021stylespace}. However, diffusion models do not share such a compact latent space~\cite{park2023unsupervised}; and efforts to uncover such a space in the semantic embeddings of the text conditioning have met with limited success \nik{Nick - is there a specific citation you were thinking about?}.

% In this work we introduce \textbf{SliderSpace}, which takes a step towards uncovering an analogous low dimensional representation of diffusion models' visual breadth; in essence treating the diffusion model as many generators sharing parameters, where a particular generator is defined by a specific prompt. For a given prompt we sample many random seeds (and optionally prompt expansions using an LLM), generate the corresponding images, and apply an off the shelf feature extractor (in this work CLIP, but our method can be applied to any differentiable feature extractor). We use PCA to analyze these features, and for each of the leading $k$ principal components we train a LoRA \cite{} which causes the diffusion model to produces images which increase the feature magnitude along that component when passed back through the same feature extractor. This leads to a 'Slider' for each principal component, because each LoRA can be scaled and applied to the original diffusion model, continuously varying those visual features in the generated results (as measured, in our case, by CLIP).

% There are many other works that enhance the controllability of diffusion models. One common approach is enabling users to add spatial constraints to a generation either manually, or via a reference image \cite{zhang2023addingconditionalcontroltexttoimage, chen2024trainingfreeregionalpromptingdiffusion}, a second is leveraging more abstract embeddings (e.g. identity, style) extracted from a reference image \cite{ye2023ipadaptertextcompatibleimage, hertz2024stylealignedimagegeneration, li2023photomaker, shi2024instantbooth}, a third is finetuning a foundation model to better generate a concept important to the user \cite{ruiz2022dreambooth, kumari2022customdiffusion, Ryu_lora, hu2021lora}, and a fourth (most relevant to this work) is finding low-rank adaptors of the model based on a prompt or small training set which can be scaled to provide continous control over one aspect of generated image (e.g. night vs day, basic vs luxury, etc.) \cite{gandikota2023concept}. SliderSpace is complementary to all of these methods and offers something distinct. All of the other methods we are aware require the user (and / or model designer) to know in advance what type of control they want. In contrast SliderSpace assists users in discovering and controlling hidden capabilities present in the diffusion model's distribution of possible generations.

%We propose that truly intuitive creative control in a text-to-image model should meet three key criteria: \emph{discoverability}, \emph{intuitiveness}, and \emph{specificity}. The model should reveal controllable attributes that may not be immediately obvious, offer controls that are easy to understand and manipulate, and ensure each control affects a distinct attribute of the generated image.

% We demonstrate the utility and power of SliderSpace using three applications built on top of SDXL-DMD \cite{dmd}, because its fast generation speed lends itself well to the continuous control offered by SliderSpace.

% First, we study concept decomposition (Section \ref{sec:concept_exp}), where we learn sliders for a specific concept (e.g. 'monster', 'waterfall', 'car'). Through quantitative metrics of diversity and text alignment we demonstrate that the learned sliders dramatically boost the diversity of generations when randomly applied without harming text alignment; we also ask humans to qualitatively judge these results in a user study where they find the SliderSpace results to be more 'Diverse', 'Useful', and 'Creative' than our baselines.

% Second, we attempt to compare the automatic discoveries of SliderSpace to a large scale manual study of artistic styles (Section \ref{sec:art_exp}), open-sourced by ParrotZone \cite{parrotzone}. In this study SDXL was prompted with over 4300 artist names,  and based on visual inspection the cases of successful stylistic mimicry recorded. Quantitatively SliderSpace more closely matches the distribution of artistic variation discovered by ParrotZone than other baselines, and in our user studies was judged to be significantly more 'Diverse' and 'Useful' than the baselines. To our surprise humans even judged SliderSpace results to be slightly more 'Diverse' than the results generated by the manually discovered artist names of \cite{parrotzone}.

% Third, we attempt to use SliderSpace to reverse the mode collapse commonly observed in distilled few-step diffusion models relative to the original teacher model (Section \ref{sec:diverse_exp}). We quantitatively demonstrate that applying SliderSpace to SDXL-DMD leads to more closely matching the distribution of images by the original teacher, SDXL.

%Through extensive experiments on various state-of-the-art text-to-image models, we demonstrate that SliderSpace significantly enhances user control and creative expression in AI-assisted image generation tasks. Our method enables a range of applications, including concept decomposition and control, diversity improvement in generated images, customization dissection and edits, and the exploration of artistic styles inherent in the model.

% SliderSpace goes beyond providing a practical tool for enhanced creative control. By mapping the visual potential of diffusion models it can open new avenues for generative creativity and deepens our understanding of each model's hidden potential.
\section{Related Work}
\label{sec:related_work}

The original investigation \cite{gibson1979ecological} on the relationship between visual perception and human action defines \emph{affordance} as the opportunities for interaction with the surrounding environment. Behavioral studies on regular and cognitively impaired persons have shown evidence that perception results in both visual and motor signals in the human brain. An extended study \cite{anderson2002attentional} shows that visual attention to the spatial characteristics of the perceived objects initiates automatic motor signals for different actions. In computer vision, human affordance learning involves novel pose prediction such that the estimated pose represents a valid human action within the scene context. The task is fundamental to many problems requiring robust semantic reasoning about the environment, such as human motion synthesis \cite{wang2021scene} and scene-aware human pose generation \cite{wang2017binge, roy2016multi, zhang2022inpaint, yao2023scene}.

Earlier methods of affordance learning have explored knowledge mining \cite{zhu2014reasoning} and multimodal feature cues \cite{roy2016multi} to address the problem. In \cite{zhu2014reasoning}, the authors use a Markov Logic Network for constructing a knowledge base by extracting several object attributes from different image and metadata sources, which can perform various downstream visual inference tasks without any additional classifier, including zero-shot affordance prediction. In \cite{roy2016multi}, the authors use depth map, surface normals, and segmentation map as multimodal cues to train a multi-scale convolutional neural network (CNN) for scene-level semantic label assignment associated with specific human actions. In \cite{do2018affordancenet}, the authors design a multi-branch end-to-end CNN with two separate pathways for object detection and affordance label assignment to achieve high real-time inference throughput. Researchers \cite{chuang2018learning} have also explored socially imposed constraints for affordance learning. In \cite{chuang2018learning}, the authors propose a graph neural network (GNN) to propagate contextual scene information from egocentric views for action-object affordance reasoning.

Probabilistic modeling of scene-aware human motion generation also involves semantic reasoning of human interaction with the environment. Initial works on human motion synthesis have taken different architectural approaches, such as sequence-to-sequence models \cite{barsoum2018hp}, generative adversarial networks (GAN) \cite{barsoum2018hp, cai2018deep, yang2018pose}, graph convolutional networks (GCN) \cite{yan2019convolutional}, and variational autoencoders (VAE) \cite{guo2020action2motion}. However, these methods have mostly ignored the role of environmental semantics. Due to potential uncertainty in human motion, in a recent approach \cite{wang2021scene}, the authors address such motion synthesis with a GAN conditioned on scene attributes and motion trajectory to predict probable body pose dynamics.

One key challenge of human affordance generation in 2D scenes is the lack of large-scale datasets with rich pose annotations. In \cite{wang2017binge}, the authors compile the only public dataset of annotated human body poses in complex 2D indoor scenes by extracting frames from sitcom videos. Aiming to generate a contextually valid human affordance at a user-defined location, the authors propose sampling the scale and deformation parameters for an existing human pose template using a VAE conditioned on the localized image patches as scene context. In \cite{zhang2022inpaint}, the authors introduce a two-stage GAN architecture for achieving a similar goal by estimating the affine bounding box parameters to localize a probable human in the scene and then generating a potential body pose at that location. The method uses the input scene, corresponding depth, and segmentation maps as semantic guidance. In \cite{yao2023scene}, the authors propose a transformer-based approach with knowledge distillation for generating human affordances in 2D indoor scenes.


% introduce PDDL domains
% why Gripper env as testing context
% motivation: comparing classical vs LLM planners
% - classical: PDDL solver fast-downward
% - LLM: gpt-4o
% explanation and refinement are two distinguishing features of LLM planners
% - how we demonstrate explanation and refinement in the study
We evaluate user trust in two planners over a set of planning problems and study the potential factors influencing user trust in the planners. In particular, we compare a language-model-based planner, denoted as an \emph{LLM Planner}, with a traditional graph-search-based planner, denoted as a \emph{PDDL Solver}. The PDDL Solver uses Fast Downwards \cite{fastdownward} as its underlying model, processing planning problems described in PDDL to generate an optimal solution. In comparison, the LLM Planner employs GPT-4o to interpret the planning problem and extract a solution generated by the language model. Unlike the PDDL Solver, the LLM Planner can reason through the planning problem, explain its proposed solution, and iteratively refine the solution based on external feedback. This study investigates how the correctness of solutions, the quality of explanations, and the refinement process influence user trust.

\subsection{Planning Problem}
% \begin{wrapfigure}{r}{0.4\textwidth}
% % \begin{figure}[t]
%     \centering
%     \includegraphics[width=\linewidth]{figures/problem-example.pdf}
%     \caption{A running example of a planning problem in our study.}
%     \Description{Planning Problem Example}
%     \label{fig: problem-example}
% % \end{figure}
% \end{wrapfigure}

We describe each planning problem in the \emph{Planning Domain Definition Language (PDDL)} and propose two planners to generate plans that solve the problem. We select the \emph{gripper} planning problems from the International Planning Competition \cite{IPC} for plan generation and evaluation. In a gripper planning problem, a robot moves balls between a set of rooms using two grippers. The objective is to create a plan for the robot to move the balls to the target rooms we defined. We present a few running examples of the gripper problem in Figure \ref{fig: correctness}.

A planning problem consists of a \emph{planning domain} and a \emph{problem description}, expressed in PDDL. 

\paragraph{Planning Domain}
A planning domain refers to the universal aspects of a problem that remains consistent across different instances of the problem. In particular, it defines the types of objects, predicates, and actions that exist in the planning problem. We present an example of the gripper problem in Appendix \ref{app: grippers}.

\paragraph{Problem Description} A problem description specifies the particular instance of a planning task within a given domain. It includes the planning domain to which it pertains, a set of objects, the initial state of these objects, and the goal state to be achieved.

\paragraph{Plan}
A plan is a sequence of actions with specific input parameters. Recall that an action corresponds to a state transition. If a plan (a sequence of actions) transits from the initial state to the goal state defined by a problem, then we consider the plan to be \emph{correct}. If a plan does not transit to the goal state or there exists an action violating its precondition, then the plan is \emph{wrong}.

\begin{figure}[t]
    \centering
    \includegraphics[width=0.8\linewidth]{figures/correct.jpeg}
    \caption{Examples where LLM Planner correctly generates a plan for the gripper planning problem.}
    \Description{Planning Problem Correctness}
    \label{fig: correct}
\end{figure}

\subsection{PDDL Solver}
The PDDL Solver takes the planning domain and the problem description as inputs and then generates a plan described in PDDL. 
% It generates a plan in the following format:
% \vspace{4pt}
% \begin{lstlisting}[language=completion]
% (move robot1 room1 room3)
% (pick robot1 ball2 room3 rgripper1)
% (move robot1 room3 room2) ......
% \end{lstlisting}
Next, we convert the generated plan into natural language for user studies following the procedure in \cite{seipp-et-al-zenodo2022} and display it to users. We present an example in Figure \ref{fig: correct}.

The PDDL Solver applies a graph search algorithm to find a path (i.e., a list of transitions) from the initial state to the goal state. It either generates a \emph{correct} plan---defined as the shortest path between the initial and goal states---or returns a signal indicating that no solution exists for the given problem.

\subsection{LLM Planner}

The LLM Planner addresses planning problems by querying a large language model. In particular, it transmits the planning domain and problem description to the language model using a structured prompt format. The planner then retrieves a natural language plan from the language model. We use GPT-4o as the language model for the planner. To ensure the output adheres to the desired format, we include a few in-context examples within the prompts.

A language model solves a planning problem by interpreting the domain and problem descriptions, simulating state transitions, and generating a sequence of actions to achieve the goal. While effective for reasoning and plan generation, language models may struggle with large state spaces. Unlike the PDDL Solver, the LLM Planner may generate \emph{incorrect} plans that violate the problem specifications (e.g., preconditions of actions) or fail to achieve the goal.

\subsection{Explanation and Refinement}
Alongside the generated plans, we offer detailed explanations of all the plans and revisions of any incorrect plans. This study examines how these explanations and refinements influence human trust in the two planners.

\paragraph{LLM Planner with Explanation (LLM+Expl)}
For each generated plan, we manually provide a natural language explanation. This explanation includes an assessment of the plan’s correctness, identification of any violations of action preconditions, and an analysis of inconsistencies between the final state achieved and the intended goal state. We present examples of explanations in Figure \ref{fig: explain} in Appendix.

In particular, if a plan is correct, the explanation is simply ``the plan successfully satisfies the goal conditions.'' 
If a plan is incorrect, we identify the underlying cause as either a violation of action preconditions or a failure to achieve the goal state. In cases involving precondition violations, we specify the action responsible for the issue. For example, consider the action ``robot moves from room 1 to room 2,'' but the robot is initially located in room 3. This scenario constitutes a violation of the precondition for the ``move'' action. In the latter case, we describe the differences between the final state achieved and the intended goal state, e.g., ``fail to move ball 2 to room 2.''

% \begin{wrapfigure}{r}{0.5\textwidth}
%     \centering
%     \includegraphics[width=0.98\linewidth]{figures/refine.jpeg}
%     \includegraphics[width=0.98\linewidth]{figures/refine-correct.jpeg}
%     \includegraphics[width=0.98\linewidth]{figures/refine-wrong.jpeg}
%     \caption{Plan refinement by the LLM Planner. The top row presents two choices of plan refinement (where the refinement starts). The second and third row shows the refinement outcomes of the two choices, where the second row shows a correctly refined plan and the third row shows an incorrect plan.}
%     \Description{Refinement}
%     \label{fig: refine}
% \end{wrapfigure}

\paragraph{LLM Planner with Refinement (LLM+Refine)}
Note that a plan generated by the LLM Planner could be incorrect. Therefore, we offer a prompting mechanism for the LLM Planner to refine the generated plan according to the user feedback. The mechanism works as follows:

1. Request the user to indicate the step number of the first action in the plan that is incorrect, such as the step where an action’s precondition is violated. We present a sample user interface on the left of Figure \ref{fig: refine} in Appendix.

2. Send the planning domain, problem description, and the original plan to the language model. Then, query the model to rewrite the subsequent steps starting from the user-specified step number. We present a sample input prompt in Figure \ref{fig: refine-prompt} in the Appendix.

3. Replace the original plan with the newly refined plan and display it to the user.

This mechanism allows users to interact with the language model to refine the plan. It enables the language model to focus on a subset of steps, facilitating a deeper interpretation of the incorrect component. However, the correctness of the refined plan is not guaranteed. Figure \ref{fig: refine} in the Appendix shows an example of a correctly refined plan and an incorrectly refined plan.

\section{Experiments}
\label{sec:experiments}

\begin{figure*}[t]
\vspace{-6mm}
    \centering
    \includegraphics[width=0.8\linewidth]{figs/compare.pdf}
    \vspace{-4mm}
    \caption{\textbf{Qualitative comparison} with the baseline for generating a sequence of novel view images.  
    The results demonstrate that our method synthesizes more consistent multi-view images compared to our baseline model (Zero123). In addition, compared to SyncDreamer, our method visually maintains better similarity to the conditioned image and appears more natural.}
    \label{fig:sota_compare}
\vspace{-5mm}
\end{figure*}

\subsection{Experimental Setups}
\textbf{Dataset.}
Following previous work~\cite{zero123, SyncDreamer}, we evaluate our work on the Google Scanned Object (GSO)~\cite{GSO} dataset to verify the zero-shot novel view image synthesis capability. 
We also provide results for additional datasets in the Supplementary Material.
Specifically, we randomly select 30 objects from the GSO dataset with various object categories. 
Unlike recent approaches~\cite{mvdream, SyncDreamer} that aim to enhance the consistency of novel view synthesis models by generating multiple fixed-view images, our method can generate images from any camera pose and any number of views. Therefore, we conduct experiments under different camera pose settings to validate our approach:
specifically, 
1) \textit{16-views with free camera pose}: for each object, we circularly render 16 views with the elevation angles ranging in $[-10\degree, 40\degree]$ and the azimuth angles are evenly distributed in $[0\degree, 360\degree]$. 
2) \textit{16-views with fixed camera pose}: We maintain a constant elevation angle of $30\degree$ and uniformly sample azimuth angles (same as SyncDreamer~\cite{SyncDreamer}).
3) \textit{32-views with free camera pose}: Similar to the first setting, but we sample 32 views.
It's important to note that our method does not require additional training or fine-tuning on any datasets.

\noindent\textbf{Metrics.}
To validate the effectiveness of our method, we mainly evaluate it based on three criteria:
1) \textit{Quality Score}. We evaluate the image quality of synthesized multi-view images by measuring their similarity with ground truth images. Following prior research~\cite{zero123, sparsefusion}, we report the similarity between the synthesized images and the ground truth images with standard metrics: PSNR, SSIM~\cite{ssim}, and LPIPS~\cite{lpips}.
2) \textit{Multi-view Consistency Score}. As the primary goal of our work is to improve the consistency of generated images, we also employ the 3D consistency score~\cite{3dim} to verify the consistency among the synthesized images. Specifically, we train an Instant-NGP~\cite{instant_ngp} with the input image and part of the synthesized novel view images of our model and evaluate the similarity between the remaining synthesized images and the rendered images of Instant-NGP. For the synthesized multi-view images of each object, we allocate $3/4$ for training and reserve the remaining $1/4$ for validation.
Intuitively, if the consistency of synthesized images is improved, the NeRF-like model will train a better object representation, and the re-rendered images will agree more with the validation images.
3) \textit{Input Consistency Score}. To assess the faithfulness of synthesized images in preserving the identity of the input condition image, we introduce the input consistency score. This score calculates the similarity of each synthesized image with the input condition image, utilizing the LPIPS metric.

In addition, we use synthesized multi-view images to train a neural 3D reconstruction model (NeuS~\cite{neus}) and report commonly used Chamfer Distances (CD) and Volume IoUs between the trained 3D model and the ground truth.

\noindent\textbf{Baselines.}
Given that our main goal is to improve the consistency of the trained baseline model without further fine-tuning, we mainly compare our approach with the used baseline model Zero123~\cite{zero123}. Additionally, we compare our method to the SOTA approaches such as PGD~\cite{tseng2023consistent} and SyncDreamer~\cite{SyncDreamer} using the same Zero123 base model.

\noindent\textbf{Implementation Details.}
We use the official checkpoint provided by Zero123~\cite{zero123}, which is trained on objaverse~\cite{objaverse} for 165,000 steps. We inject our epipolar attention layer after step $T=4$ and layer $L=10$ by default. We find that feature fusion weight $\alpha=0.5$, and the number of context views $M=2$ work better.

\begin{table}[t]
\centering
\caption{Comparison of multi-view consistency, image quality, and input consistency of synthesized multi-view images at the 16-view setting with free camera pose.}
\label{tab:view16_free_compare}
\vspace{-2mm}
\scalebox{0.6}{
\begin{tabular}{c ccc ccc c}
\toprule
              & \multicolumn{3}{c}{Multi-view Consistency} & \multicolumn{3}{c}{Quality Score} & \multicolumn{1}{c}{Input Consis.} \\
              \cmidrule(lr){2-4} \cmidrule(lr){5-7} \cmidrule(lr){8-8}
              & PSNR$\uparrow$  & SSIM$\uparrow$ & LPIPS$\downarrow$ 
              & PSNR$\uparrow$  & SSIM$\uparrow$ & LPIPS$\downarrow$ 
              & LPIPS$\downarrow$ 
              \\ \midrule

Zero123
& 15.225        & 0.645       & 0.408
& 14.255        & 0.747       &	0.208
& 0.303         
\\
SyncDreamer
& 14.830        & 0.626       & 0.434
& 12.650        & 0.713       &	0.254
& 0.317         
\\
Ours 
& \best{18.300}	& \best{0.734}	& \best{0.355}
& \best{14.947}	& \best{0.763}	& \best{0.191}
& \best{0.282}
\\

\bottomrule
\end{tabular}
}
\end{table}

\begin{table}[t]
\vspace{-1mm}
\centering
\caption{Comparison of multi-view consistency, image quality, and input consistency at the 16-view setting with fixed camera pose as SyncDreamer~\cite{SyncDreamer}.}
\label{tab:view16_fxied_compare}
\vspace{-3mm}
\scalebox{0.6}{
\begin{tabular}{c ccc ccc c}
\toprule
              & \multicolumn{3}{c}{Multi-view Consistency} & \multicolumn{3}{c}{Quality Score} & \multicolumn{1}{c}{Input Consis.} \\
              \cmidrule(lr){2-4} \cmidrule(lr){5-7} \cmidrule(lr){8-8}
              & PSNR$\uparrow$  & SSIM$\uparrow$ & LPIPS$\downarrow$ 
              & PSNR$\uparrow$  & SSIM$\uparrow$ & LPIPS$\downarrow$ 
              & LPIPS$\downarrow$ 
              \\ \midrule

Zero123
& 16.556        & 0.682       & 0.378
& 14.592        & 0.750       &	0.207
& 0.305         
\\
SyncDreamer
& \best{22.424}        & \best{0.812}       & \best{0.268}
& 15.269        & 0.749       &	0.196
& 0.300         
\\
Ours 
& 21.151	& 0.780	& 0.302
& \best{15.293}	& \best{0.764}	& \best{0.184}
& \best{0.287}
\\

\bottomrule
\end{tabular}
}
\vspace{-4mm}
\end{table}


\subsection{Comparison With Baseline Models}
The quantitative comparison on three settings are shown in Tab.~\ref{tab:view16_free_compare}, Tab.~\ref{tab:view16_fxied_compare}, and Tab.~\ref{tab:view32_free_compare}. The qualitative comparison is shown in Fig.~\ref{fig:sota_compare}.

\begin{table}[t]
\centering
\caption{Comparison of multi-view consistency and image quality scores of synthesized multi-view images at the 32-view setting with free camera pose.}
\vspace{-3mm}
\label{tab:view32_free_compare}
\scalebox{0.7}{
\begin{tabular}{c ccc ccc}
\toprule
              & \multicolumn{3}{c}{Multi-view Consistency} & \multicolumn{3}{c}{Quality Score} \\
              \cmidrule(lr){2-4} \cmidrule(lr){5-7}
              & PSNR$\uparrow$  & SSIM$\uparrow$ & LPIPS$\downarrow$ 
              & PSNR$\uparrow$  & SSIM$\uparrow$ & LPIPS$\downarrow$ 
              \\ \midrule

Zero123
& 16.515        & 0.694       & 0.378
& 15.142        & 0.733       &	0.211
\\
PGD~\cite{tseng2023consistent}
& 18.481        & 0.720       & 0.343
& 15.281        & 0.739       &	0.205
\\
Ours 
& \best{20.655}	& \best{0.792}	& \best{0.305}
& \best{15.268}	& \best{0.742}	& \best{0.203}
\\

\bottomrule
\end{tabular}
}
\vspace{-3mm}
\end{table}

\begin{table*}
  [t]
  \centering
  \resizebox{\textwidth}{!}{%
  \begin{tabular}{cccccccccccc}
    \toprule \multicolumn{2}{c}{Components}                                                             & \multicolumn{5}{c}{Re-executability Rate (\%)} & \multicolumn{5}{c}{Readability (\#)} \\
    \cmidrule(lr){1-2} \cmidrule(lr){3-7} \cmidrule(lr){8-12}        \hspace{8pt}\labelemoji\hspace{8pt}                                                                & \hspace{8pt}\toolemoji\hspace{8pt}                                      & O0                                 & O1             & O2             & O3             & AVG            & O0             & O1             & O2             & O3             & AVG            \\
    \hline
    \rowcolor[rgb]{0.93,0.93,0.93}\multicolumn{12}{c}{\textbf{Initialize with LLM4Decompile-End-6.7B~\citep{llm4decompile}}}   \\
    \xmark                                                                                              & \xmark                                    & 69.51                              & 46.95          & 50.61          & 46.34          & 53.35          & 3.98 & 3.41 & 3.44 & 3.38 & 3.55 \\
    \cmark                                                                                              & \xmark                                    & 75.61                              & 50.61          & 50.00          & 50.00          & 56.55          & 4.01 & 3.44 & 3.39 & \textbf{3.49} & 3.58 \\
    \xmark                                                                                              & \cmark                                    & 83.54                     & \textbf{56.10}          & 51.22          & 50.61 & 60.37 & 4.05 & 3.51 & 3.51 & 3.42 & 3.62 \\
    \cmark                                                                                              & \cmark                                    & \textbf{85.37}                            & \textbf{56.10}                     & \textbf{51.83} & \textbf{52.43}          & \textbf{61.43} & \textbf{4.13} & \textbf{3.60} & \textbf{3.54} & \textbf{3.49} & \textbf{3.69} \\

    \rowcolor[rgb]{0.93,0.93,0.93}\multicolumn{12}{c}{\textbf{Initialize with Deepseek-Coder-6.7B-base~\citep{deepseekcoder}}} \\
    \xmark                                                                                              & \xmark                                    & 59.15                              & 35.98          & 39.02          & 37.80          & 42.99          & 3.71 & 3.05 & 3.16 & 3.05 & 3.24 \\
    \cmark                                                                                              & \xmark                                    & 66.46                              & 41.46          & 38.41          & 36.59          & 45.73          & 3.76 & 3.17 & \textbf{3.21} & 3.08 & 3.31 \\
    \xmark                                                                                              & \cmark                                    & 70.73                              & 39.63          & 39.02          & 40.24          & 47.41          & 3.90 & 3.17 & 3.08 & 3.11 & 3.31 \\
    \cmark                                                                                              & \cmark                                    & \textbf{79.88}                     & \textbf{45.73} & \textbf{43.90} & \textbf{42.68} & \textbf{53.05} & \textbf{3.96} & \textbf{3.21} & 3.18 & \textbf{3.19} & \textbf{3.38} \\
    \bottomrule
  \end{tabular}%
  }
  \caption{The ablation study of different methods across four optimization levels
  (O0, O1, O2, O3), as well as their average scores (AVG). The results in bold represent the optimal performance. The ~\labelemoji~ and ~\toolemoji~ means Relabedling and Function Call. \textbf{Bold} denotes the best performance.}
  \label{tab:ablation}
\end{table*}



\begin{figure*}[ht]
    \centering
    \begin{minipage}{0.65\textwidth}
        \centering
        \includegraphics[width=0.95\linewidth]{figs/ablation.pdf}
        \vspace{-2mm}
        \captionof{figure}{Qualitative Comparison for different design choices. Our method, employing multi-view epipolar attention, demonstrates the best consistency.}
        \label{fig:ablation}
    \end{minipage}\hfill
    \begin{minipage}{0.33\textwidth}
        \centering
        \includegraphics[width=0.8\linewidth]{figs/neus_ver.pdf}
        \vspace{-3mm}
        \caption{Our method shows better direct 3D reconstruction~\cite{neus}.}
        \label{fig:neus}
    \end{minipage}
    \vspace{-5mm}
\end{figure*}

\noindent\textbf{Multi-view Consistency.}
Tab.~\ref{tab:view16_fxied_compare} presents the 3D consistency scores compared to our baseline model (Zero123) and SyncDreamer. The results indicate a significant improvement across all three metrics achieved by our method when compared with Zero123.
While our method exhibits a marginally lower numerical consistency score compared to SyncDreamer, it enables the synthesis of images with arbitrary camera poses.	
This capability is illustrated in Tab.~\ref{tab:view16_free_compare}, where our method consistently enhances consistency with changes in camera pose settings, whereas SyncDreamer fails to do so and exhibits inferior results compared to Zero123.
Furthermore, our method facilitates the synthesis of multi-view images with any number of camera views. This versatility is demonstrated in Tab.~\ref{tab:view32_free_compare}, where our method continues to achieve significant improvements in consistency scores, while SyncDreamer is unable to operate under such conditions.	

Meanwhile, Fig.~\ref{fig:sota_compare} provides a qualitative comparison with the baseline. While both our method and SyncDreamer enhance consistency, our method visually preserves better similarity to the input image, including color and texture details. The input consistency score further corroborates this.

\noindent\textbf{Image Quality.}
While our primary goal centers around enhancing the consistency of synthesized multi-view images, we also evaluate the image quality by comparing the similarity with the ground truth images. The results shown in Tab.~\ref{tab:view16_free_compare}, Tab.~\ref{tab:view16_fxied_compare}, and Tab.~\ref{tab:view32_free_compare} indicate that our method also enhances the image quality under different settings besides improving the consistency.
Moreover, our method shows better image quality compared with SyncDreamer even in the 16-view setting with fixed camera pose.

\noindent\textbf{Input Consistency.}
Input consistency terms whether the results align with the input image.
Fig.~\ref{fig:sota_compare} illustrates that both our method and SyncDreamer enhance multi-view consistency. However, the color and texture details of SyncDreamer's results diverge from the input image and appear visually unnatural.
This discrepancy is evident in the input consistency score presented in Tab.~\ref{tab:view16_fxied_compare}, indicating lower similarity with the condition image in the SyncDreamer results.	

\subsection{Ablation Study}
The overall quantitative results are shown in Tab.~\ref{tab:ablation}, and the qualitative comparisons are shown in Fig.~\ref{fig:ablation}.

\noindent \textbf{Full Attention \vs Epipolar Attention.}
The results presented in Tab.\ref{tab:ablation} and Fig.\ref{fig:ablation} demonstrate that our epipolar attention mechanism can synthesize more consistent multi-view images compared with full attention. Furthermore, our epipolar attention achieves a greater performance improvement compared to full attention when using multiple reference images. This could be attributed to the fact that our epipolar attention more effectively localizes target information, as depicted in Fig.~\ref{fig:full_attn_compare}, thereby reducing noise from the reference images. In the multi-view setting, where multiple reference images are utilized, this noise reduction becomes particularly crucial.
Moreover, it is noteworthy that the epipolar attention mechanism consumes less GPU memory compared to our baseline, as discussed in Sec.~\ref{sec:attn_analysis}.

\noindent \textbf{Attending Single-View \vs Multi-View.}
Applying the epipolar attention significantly improves the consistency between the input and target views. However, the consistency between different views in the unobserved regions of the input view is not well preserved.
After implementing our epipolar attention in the multi-view setting, the consistency across the generated multi-view images is further improved. The last row in Tab.~\ref{tab:ablation} shows that after applying our multi-view epipolar attention, the consistency score is further improved compared with the single-view setting. Besides, the qualitative result in Fig.~\ref{fig:ablation} also shows better consistency among different target views.



\begin{table}[t]
\centering
\vspace{-1mm}
\caption{Comparison of 3D reconstruction results. Our method significantly improves the reconstruction quality.}
\vspace{-3mm}
\label{tab:neus}
\scalebox{0.7}{
\begin{tabular}{c cc}
\toprule
              &  Chamfer Dist.$\downarrow$  & Volume IoU$\uparrow$
\\ \midrule

            Zero123         & 0.017         & 0.819    \\
            SyncDreamer     & \best{0.013}         & \best{0.847}    \\
            Ours            & 0.014	& 0.842 \\

\bottomrule
\end{tabular}
}
\vspace{-5mm}
\end{table}


\vspace{-2mm}
\subsection{Downstream Application}
\vspace{-2mm}
To demonstrate the effectiveness of our method, we also applied it to the downstream 3D reconstruction task. Specifically, we trained the NeuS model~\cite{neus} directly using images synthesized by our method, Zero123, and SyncDreamer, respectively.
The quantitative results in Tab.~\ref{tab:neus} show that the consistent multi-view images synthesized by our method can significantly improve the 3D reconstruction quality.
Additionally, our method exhibits similar performance to SyncDreamer which requires time-consuming re-training.
The qualitative results in Fig.~\ref{fig:neus} show that it is challenging to train the NeuS model directly due to the lack of consistency in the images generated by Zero123. In contrast, our method generates more consistent multi-view images and, therefore, better reconstructs the geometry and texture details.
We show improvements on other downstream applications such as image-to-3D in the Supplementary Material.


\section{Conclusion}

%In this paper, w
We propose a new PEFT method called DiffoRA, which enables efficient and adaptive LLM fine-tuning based on LoRA. 
Instead of adjusting every interior rank, 
%of the decomposition matrices 
%of all modules, 
we argue that adopting LoRA module-wisely is sufficient. 
To achieve this, we construct a DAM to select the modules that are most suitable and essential to fine-tune. We theoretically analyze how the DAM impacts the convergence rate and generalization capability.
%of the pre-trained model. 
Furthermore, we adopt continuous relaxation and discretization to establish DAM.
%for each task. 
To alleviate the issue of discretization discrepancy, we utilize the weight-sharing strategy for optimization. 
%We fully implement our method and t
The experimental results demonstrate that our DiffoRA works consistently better than the baselines across all benchmarks. 

\section*{Acknowledgements}
This work was supported by the Central Guidance for Local Special Project (Grant No. Z231100005923044).



{
    \small
    \bibliographystyle{ieeenat_fullname}
    \bibliography{main}
}

\newpage

\appendix



We first describe detailed processes of building the prompt library and sampling strategies in our method:
\begin{itemize}
    \item \cref{sec_supp: build_prompt_library}: Details of building prompt library.
    \item \cref{sec_supp: prompt_sampling}: Details of prompt sampling strategy.
    \item \cref{sec_supp: group_sampling}: Details of group sampling strategy.
\end{itemize}

Then, we show more experiments to show the effectiveness of our ProAPO:
\begin{itemize}
    \item \cref{sec_supp: implement_details}: More implementation details.
    \item \cref{sec_supp: different_backbone_result}: Results on different backbones.
    \item \cref{supp_sec: more_comparison_with_sota_methods}: More comparisons with SOTA methods.
    \item \cref{supp_sec: ablation_template_and_description}: Ablation of progressive optimization.
    \item \cref{supp_sec: more_ablation_operator}: More ablation of operators.
    \item \cref{supp_sec: more_ablation_group_sampling}: More ablation of group sampling.
    \item \cref{supp_sec: ablation_of_cost_computation}: Ablation of cost computation.
    \item \cref{supp_sec: effect_shots}: Effect of shot numbers.
    \item \cref{supp_sec: effect_alpha}: Effect of scalar $\alpha$ in score function.
    \item \cref{supp_sec: effect_sampled_numbers}: Effect of sampled numbers in prompt sampling.
    \item \cref{supp_sec: effect_of_quality_of_prompt_library}: Effect of quality of prompt library.
    \item \cref{sec_supp: more_qualitative_result}: More qualitative results.
\end{itemize}


We also provide detailed results for experiments appearing in the main paper: 
\begin{itemize}
    \item \cref{sec_supp: transfer_to_adapter}: Results of transfering to adapter-based methods.
    \item \cref{sec_supp: transfer_to_backbones}: Results of transferring to different backbones.
    \item \cref{supp_sec: ensemble_vs_single}: Analysis of single vs ensemble prompts.
    \item \cref{sec_supp: improve_description_methods}: Improvement by iterative optimization.
    \item \cref{supp_sec: ablation_operator}: Ablation of edit and evolution operators.
    \item \cref{supp_sec: ablation_two_sampling}: Ablation of two sampling strategies.
    \item \cref{supp_sec: effect_score_func}: Ablation of different score functions.
\end{itemize}



\section{Details of Building Prompt Library}
\label{sec_supp: build_prompt_library}


% ---------------------------------------------------- % 
%             Template Library 的细节
% ---------------------------------------------------- %
\subsection{Details of Building Template Library}
The template library aims to collect a set of templates that provide task-specific contextual information, which can address issues of semantic ambiguity caused by class names. It contains processes for collecting templates, generating dataset domains, and adding dataset domains to templates.

% for subsequent prompt generation


% **************************** % 
% 直接借助 LLMs 生成 Template 的 Prompt
% **************************** %
\textbf{Collecting templates.} We utilize two ways to collect templates. First, pre-defined templates, such as Template-80~\cite{CLIP}, FILIP-8~\cite{FILIP}, and DEFILIP-6~\cite{DEFILIP} can be used. Second, similar to PN~\cite{P_N}, we query LLMs to create diverse templates by the following prompt:
\begin{quote}
    \makebox[\linewidth]{%
        \colorbox{lightblue}{%
            \hspace*{0mm} % Adjust left alignment
            \begin{minipage}{\dimexpr\linewidth+10\fboxsep\relax} % Adjust width
                % \fontsize{9pt}{10pt}\selectfont % Font settings
                ``Hi, ChatGPT! I would like your help to prompt for image classification using CLIP. As a human-level prompt engineer, your task is to create a set of Templates like the following for visual classification. For example: 
                \newline \newline
                a photo of a \{\}.''
            \end{minipage}%
        }%
    }
\end{quote}


% **************************** % 
%  生成 Dataset Type 的 Prompt
% **************************** %
\textbf{Generating dataset domain by LLMs.} Inspired by previous description-based methods~\cite{WaffleCLIP, VDT_2023_ICCV}, we query LLMs to generate dataset domain information to provide task-specific context. For this purpose, we use the prompt:
\begin{quote}
    \makebox[\linewidth]{%
        \colorbox{lightblue}{%
            \hspace*{0mm} % Adjust left alignment
            \begin{minipage}{\dimexpr\linewidth+10\fboxsep\relax} % Adjust width
                % \fontsize{9pt}{10pt}\selectfont % Font settings
                ``Hi, ChatGPT! I would like your help in generating dataset domain information for image classification based on the dataset paper. A few words are good. Please return directly without explanation. 
                \newline \newline
                \{\texttt{uploaded PDF}\}.''
            \end{minipage}%
        }%
    }
\end{quote}
Here, \{\texttt{uploaded PDF}\} represents the uploading of the paper of the dataset to LLMs. 
Generated dataset domain information is summarized in~\cref{supp_tab: domain_information}.


% **************************** % 
% 具体使用的 Dataset Domain
% **************************** %
{
\renewcommand{\arraystretch}{1.1} 
\begin{table}[htbp]
  \centering
  \resizebox{1.0\linewidth}{!}
    {
    \begin{tabularx}{0.56\textwidth}
        {l | X }  
        \toprule
        {\textbf{Dataset}}  & \textbf{Domain Information} \\
        \midrule
        IN-1K~\cite{Imagenet} & real scenario; natural scene \\
        Caltech~\cite{caltech101} & object; everyday objects; common items \\
        Cars~\cite{Cars} & car; vehicles; auto-mobile \\ 
        CUB~\cite{CUB} & bird; wildlife; ornithology  \\
        DTD~\cite{DTD} & textures; patterns; surface; material \\
        ESAT~\cite{EuroSAT} & land cover; remote sensing; satellite photo; satellite imagery; aerial or satellite images; centered satellite photo \\
        FGVC~\cite{FGVC} & aircraft; airplane; plane; airliner \\ 
        FLO~\cite{FLO} & flower; floral; botanical; bloom \\
        Food~\cite{Food101} & food; dishes; cuisine; nourishment \\ 
        Pets~\cite{oxford_pets} & pet; domestic animals; breed; dog or cat \\ 
        Places~\cite{Places365} & place; scene \\ 
        SUN~\cite{SUN} &  place; scene \\ 
        UCF~\cite{UCF101} & action; human action; human activities; person doing \\ 
        \bottomrule
    \end{tabularx}
}
\vspace{-6pt}
  \caption{\textbf{Generated dataset domain information.}}
% \vspace{-10pt}
  \label{supp_tab: domain_information}
\end{table}
}

% **************************** % 
% 将 Dataset Type 与 Template 融合的方式
% **************************** %
\textbf{Adding dataset domain to templates.} We supplement templates with dataset domain information in the following four ways: (1) Add ``a type of \{\texttt{domain}\}''. (2) Replace ``\{\texttt{class}\}'' with ``\{\texttt{domain}\}:\{\texttt{class}\}''. (3) Replace ``photo'' with ``\{\texttt{domain}\}''. (4) Replace ``photo'' with ``\{\texttt{domain}\} photo''. Taking ``a photo of a \{\texttt{class}\}'' as an example, we modify the templates with the above four ways to add dataset domain information as follows:
\begin{quote}
    \makebox[\linewidth]{%
        \colorbox{lightblue}{%
            \hspace*{0mm} % Adjust left alignment
            \begin{minipage}{\dimexpr\linewidth+10\fboxsep\relax} % Adjust width
                % \fontsize{9pt}{10pt}\selectfont % Font settings
                \begin{enumerate}
                    \item a photo of a \{\texttt{class}\}, a type of \{\texttt{domain}\}.
                    \item a photo of a \{\texttt{domain}\}: \{\texttt{class}\}.
                    \item a \{\texttt{domain}\} of a \{\texttt{class}\}.
                    \item a \{\texttt{domain}\} photo of a \{\texttt{class}\}.
                \end{enumerate}
            \end{minipage}%
        }%
    }
\end{quote}
Here, \{\texttt{class}\} and \{\texttt{domain}\} denote category name and dataset domain information, respectively.


% ---------------------------------------------------- % 
%             Description Library 的细节
% ---------------------------------------------------- %
\subsection{Details of Building Description Library}
\label{supp_sec: build_description_library}

Description Library aims to provide a set of visual descriptions for each category, enhancing visual semantics for fine-grained recognition in prompts. It contains processes for generating visual descriptions and category synonyms and integrating descriptions with the best templates.

% , including CuPL~\cite{CuPL}, DCLIP~\cite{DCLIP}, GPT4Vis~\cite{GPT4Vis}, and AdaptCLIP~\cite{AdaptCLIP}.


% **************************** % 
%      生成 Description 的 Prompt
% **************************** %
{
\renewcommand{\arraystretch}{1.1} 
\begin{table}[htbp]
  \centering
  \resizebox{1.0\linewidth}{!}
    {
    \begin{tabularx}{0.56\textwidth}
        {l | X }  
        \toprule
        {\textbf{Method}}  & \textbf{Prompts} \\
        \midrule
        DCLIP~\cite{DCLIP} & Q: What are useful visual features for distinguishing a \{\texttt{class}\} in a photo? \\
        & A: There are several useful visual features to tell there is a \{\texttt{class}\} in a photo: \\
        
        \midrule 
        CuPL-Base~\cite{CuPL} & Describe what a \{\texttt{class}\} looks like. \\ 
        & Describe a \{\texttt{class}\}. \\
        & What are the identifying characteristics of a \{\texttt{class}\}? \\
        
        \midrule

        CuPL-Full~\cite{CuPL} & Describe what a \{\texttt{class}\} looks like. \\ 
        & How can you identify a \{\texttt{class}\}? \\ 
        & What does a \{\texttt{class}\} look like? \\
        & Describe an image from the internet of a \{\texttt{class}\}\\
        & A caption of an image of a \{\texttt{class}\}: \\
        
        \midrule
        GPT4Vis~\cite{GPT4Vis} & I want you to act as an image description expert. I will give you a word and your task is to give me 20 sentences to describe the word. Your description must accurately revolve around this word and be as objective, detailed and diverse as possible. In addition, the subject of your description is a some kind of object photograph. Output the sentences in a json format which key is the the word and the value is a list composed of these sentences. Do not provide any explanations. The first word is ``\{\texttt{class}\}". \\ 

        \midrule 
        AdaptCLIP~\cite{AdaptCLIP} & What characteristics can be used to differentiate \{\texttt{class}\} from other \{\texttt{domain}\} based on just a photo? Provide an exhaustive list of all attributes that can be used to identify the \{\texttt{domain}\} uniquely. Texts should be of the form “\{\texttt{domain}\} with \{\texttt{characteristic}\}”. \\
        \bottomrule
    \end{tabularx}
}
% \vspace{-6pt}
  \caption{\textbf{Prompts for generating visual descriptions.}}
% \vspace{-10pt}
  \label{supp_tab: generate_description}
\end{table}
}


% **************************** % 
%      生成同义词的 Prompt
% **************************** %
\textbf{Generating category synonym}.
Except for descriptions, we also replace class names from the dataset with their synonyms to create diverse class-specific prompts. For this purpose, we use the following prompt to ask LLMs to generate category synonyms:
\begin{quote}
    \makebox[\linewidth]{%
        \colorbox{lightblue}{%
            \hspace*{0mm} % Adjust left alignment
            \begin{minipage}{\dimexpr\linewidth+10\fboxsep\relax} % Adjust width
                % \fontsize{9pt}{10pt}\selectfont % Font settings
                ``Hi, ChatGPT! I would like your help in generating category synonyms. As a \{\texttt{domain}\} expert, I will provide you with a category name. Your task is to provide synonyms for the current category. If it has subclasses, return them as well. Please return directly without explanation.
                \newline \newline 
                User: I want to give the synonyms of \{\texttt{class}\}. 
                \newline
                Assistant: ''
            \end{minipage}%
        }%
    }
\end{quote}

% **************************** % 
% 各个数据集使用的询问 LLMs 生成 Description 的 Prompt
% **************************** %
\textbf{Generating visual descriptions for each category}.
Similar to previous description methods~\cite{CuPL, DCLIP, GPT4Vis, AdaptCLIP}, we instruct LLM to generate visual descriptions for each category by several prompts, which are summarized in~\cref{supp_tab: generate_description}.


% **************************** % 
%  将 Description 和 Template 整合
% **************************** %
\textbf{Integrating descriptions with the best templates}.
We use the following prompt to integrate descriptions with templates: ``\{\texttt{template}\}. \{\texttt{description}.\}''. 


% **************************** % 
%  每个组迭代的描述库
% **************************** %
After the above processes, we collect diverse visual descriptions for each category $c$, denoted as $\text{VD}(c)$. For each group iteration, we select the descriptions for categories in the specific group as the description library. Moreover, the prompt sampling strategy also utilizes these descriptions for class-specific initialization.

% ---------------------------------------------------- % 
%              Prompt Sampling 策略的细节
% ---------------------------------------------------- %
\section{Details of Prompt Sampling Strategy}
\label{sec_supp: prompt_sampling} 
The detailed prompt sampling strategy is summarized in Alg.~\ref{supp_alg: prompt_strategy}. Visual descriptions of each class $\text{VD}(c)$ are collected by the above process (see~\cref{supp_sec: build_description_library}). We utilize the candidate prompt $P_t^*$ with the best templates as an initial point. The $\textsc{RandomSample}(\cdot)$ operator denotes randomly selecting a set of elements from a given set. We randomly sample descriptions for each category to create multiple candidate prompts (Lines 2-8). After $T_{sample}$-times steps, we select the candidate prompt $\hat{P}_0$ with the highest score for description initialization (Line 9). It ensures that subsequent optimization is around the optimal initial point. We set $T_{sample} = 32 $ for all datasets in the default setting.

% Notably, descriptions generated by LLMs in this strategy are also used as the specific-group description library.


% **************************** % 
%    Prompt Sampling 的具体算法
% **************************** %
\begin{algorithm}[htbp]
\caption{Prompt Sampling Strategy.}
\label{supp_alg: prompt_strategy}
\begin{algorithmic}[1]
\REQUIRE $\mathcal{D} \leftarrow \{{(x, y)}\}_n$: training samples, $F:  \mathcal{D} \times P \to \mathbb{R}$: score function, $\mathcal{C}$: class labels, $\text{VD}(c)$: visual descriptions of class $c$, $P_t^*$: the prompt candidate with the best template
\STATE $\mathcal{U} \leftarrow \{P_t^*\} $ 
\FOR{$i=1$ to $T_{sample}$}
    \STATE $P_i \leftarrow P_t^* $
    \FORALL{class $c \in \mathcal{C}$}
        \STATE $P_i \leftarrow P_i \cup \textsc{RandomSample}(\text{VD}(c))$
    \ENDFOR
    \STATE $\mathcal{U} \leftarrow \mathcal{U} \cup \{ P_i \} $ 
\ENDFOR
\STATE $\hat{P}_0 \leftarrow \arg\max_{{P} \in \mathcal{U}} F(\mathcal{D}, {P})$ 
\RETURN the candidate prompt with the highest score $\hat{P}_0$
\end{algorithmic}
\end{algorithm}


% $T_{repeat}$: repeated times, 


% **************************** % 
%    Group Sampling 的具体算法
% **************************** %
\begin{algorithm}[htbp]
\caption{Group Sampling Strategy.}
\label{supp_alg: group_strategy}
\begin{algorithmic}[1]
\REQUIRE $\mathcal{D} \leftarrow \{{(x, y)}\}_n$: training samples, $F:  \mathcal{D} \times P \to \mathbb{R}$: score function, $\mathcal{C}$: class labels, $\text{VD}(c)$: visual descriptions of class $c$, $P_t^*$: prompt candidate with the best template, $\text{pred}(x)$: prediction for image $x$
\FORALL{class $c \in \mathcal{C}$}
    \STATE $\textsc{MisClass}(c) \leftarrow \emptyset $
\ENDFOR
\FORALL{training sample $(x, y) \in \mathcal{D}$}
    \IF{$\text{pred}(x) \neq y $}
        \STATE $\textsc{MisClass}(y) \leftarrow \textsc{MisClass}(y) \cup \{\text{pred}(x)\} $
    \ENDIF
\ENDFOR

\FORALL{class $c \in \mathcal{C}$}
    \STATE \textbf{Select Class Images}: $\textsc{Data} (c) \leftarrow \{ (x, y) \; | \; y = c\}_{(x, y) \in \mathcal{D}}$
    \STATE \textbf{Compute Accuracy}: $\textsc{Acc} (c) \leftarrow F( \textsc{Data} (c), P^*_t )$
    \STATE \textbf{Add Descriptions}: $P_{c} \leftarrow P^*_t \cup \text{VD}(c) $
    \STATE \textbf{Compute Accuracy Gain}: $\textsc{AccGain} (c) \leftarrow F( \textsc{Data} (c), P_c ) -  \textsc{Acc} (c) $
\ENDFOR
\STATE \textbf{Sort Class by Accuracy}: $\mathcal{C}_{wst}$, retaining the classes with the lowest top-$n_{wst}$ accuracy
\STATE \textbf{Sort Class by Accuracy Gain}: $\mathcal{C}_{sln}$, retaining the classes with the top-$n_{sln}$ accuracy gain
\STATE \textbf{Initialize Group Set}: $\mathcal{G} \leftarrow \emptyset$
\FORALL{class $c \in \mathcal{C}_{wst}$}
    \STATE $\mathcal{G} \leftarrow \mathcal{G} \cup \{ \textsc{MisClass}(y) \}$
\ENDFOR
\FORALL{class $c \in \mathcal{C}_{sln}$}
    \STATE $\mathcal{G} \leftarrow \mathcal{G} \cup \{ \textsc{MisClass}(y) \}$
\ENDFOR
\RETURN sampled groups $\mathcal{G}$
\end{algorithmic}
\end{algorithm}



% ---------------------------------------------------- % 
%               Group Sampling 策略的细节
% ---------------------------------------------------- %
\section{Details of Group Sampling Strategy}
\label{sec_supp: group_sampling}
The detailed group sampling strategy is summarized in Alg.~\ref{supp_alg: group_strategy}.
It contains processes of obtaining misclassified categories and selecting the worst and salient groups.

% **************************** % 
% misclassified category 获取的算法
% **************************** %
\noindent \textbf{Obtaining misclassified categories}.
In Lines 1-8 of Alg.~\ref{supp_alg: group_strategy}, we collect misclassified set for each category by $\textsc{MisClass}(\cdot)$ operator. Given an image $x$, if the prediction $\text{pred}(x)$ is not its corresponding label $y$, we will add $\text{pred}(x)$ to the misclassified set for category $y$. In fact, we also ablate the K-means clustering algorithm to group categories (in~\cref{supp_sec: more_ablation_group_sampling}). Results show that the misclassified set achieves better performance than the K-means algorithm.

% **************************** % 
%       挑选最差组别的算法
% **************************** %
\noindent \textbf{Selecting the worst groups} aims to select categories with the lowest top-$n_{wst}$ accuracy and corresponding misclassified categories. We first compute the accuracy for each category in Line 11. Then, we sort the categories by accuracy and retain the top-$n_{wst}$ worst categories in Line 15. Finally, $n_{wst}$ groups are added to the set $\mathcal{G}$ in Lines 18-20.


% **************************** % 
%       挑选显著组别的算法
% **************************** %
\noindent \textbf{Selecting the salient groups} aims to select categories with the top-$n_{sln}$ performance gains and its misclassified categories after adding descriptions. In Line 13, we compute the accuracy gains after adding the descriptions. Then, we sort the categories by accuracy gain and retain the top-$n_{sln}$ accuracy gain categories in Line 16. At last, $n_{sln}$ groups are added to the set $\mathcal{G}$ in Lines 21-23.

Finally, we collect $S = n_{wst} + n_{sln}$ groups for subsequent description optimization.





\section{More Implementation Details}
\label{sec_supp: implement_details}


\subsection{Hyperparameter Settings}
In~\cref{supp_tab: exp_details}, we show the searched hyperparameter settings for thirteen datasets. All results are average with four seeds. Except for $1, 2, 3$ as seeds like CoOp~\cite{CoOp}, we add $42$ as our fourth seed to further evaluate the stability of our method. In the default setting, we use the same LLMs as the description methods, \textit{i.e.}, GPT-3~\cite{GPT3} for CuPL~\cite{CuPL} and DCLIP~\cite{DCLIP}, GPT-4~\cite{GPT4_Tech} for GPT4Vis~\cite{GPT4Vis} and AdaptCLIP~\cite{AdaptCLIP}. 

% Our codes are available at \href{https://anonymous.4open.science/r/ProAPO}{https://anonymous.4open.science/r/ProAPO}. We will release our code and optimized prompts after the review stage.


{
\renewcommand{\arraystretch}{1.1} 
% \setlength{\tabcolsep}{3.pt}
\begin{table}[htbp]
  \centering
  \resizebox{1.0\linewidth}{!}
    {
    \begin{tabular}
        {l | c | c | c | c | c | c | c  }  
        \toprule
        {\textbf{Dataset}}  & $T$ &  $M$ & $N$ & $\alpha$ & $n_{wst}$ & $n_{sln}$ & $T_{sample}$ \\
        \midrule
        IN-1K~\cite{Imagenet} & 4 & 8 & 8 & 1e3 & 4 & 4 & 32 \\
        Caltech~\cite{caltech101} & 2 & 8 & 8 & 1e2 & 2 & 2 & 32 \\
        Cars~\cite{Cars} & 4 & 8 & 8 & 1e4 & 4 & 4 & 32 \\ 
        CUB~\cite{CUB} & 4 & 8 & 8 & 1e2 & 4 & 4 & 32 \\
        DTD~\cite{DTD} & 4 &  8 & 8 & 1e3 & 4 & 4 & 32 \\
        ESAT~\cite{EuroSAT} & 4 &  8 &  8 & 1e3 & 3 & 3 & 32 \\
        FGVC~\cite{FGVC} & 4 & 8 & 8 & 1e3 & 4 & 4 & 32 \\ 
        FLO~\cite{FLO} & 4 & 8 & 8 & 1e3 & 4 & 4 & 32 \\
        Food~\cite{Food101} & 4 &  8 & 8 & 1e3 & 2 & 2 & 32 \\ 
        Pets~\cite{oxford_pets} & 2 & 8 &  8 & 1e4 & 2 & 2 & 32 \\ 
        Places~\cite{Places365} & 4  & 8 &  8 & 1e2 & 3 & 3 & 32 \\ 
        SUN~\cite{SUN} & 2 &  8 & 8 & 1e4 & 4 & 4 & 32 \\ 
        UCF~\cite{UCF101} & 4 &  8 & 8 & 1e3 & 3 & 3 & 32 \\ 
        \bottomrule
    \end{tabular}
}
\vspace{-6pt}
  \caption{\textbf{Hyperparameters settings for thirteen datasets.}}
\vspace{-10pt}
  \label{supp_tab: exp_details}
\end{table}
}


\subsection{More Related Work}
\noindent 
\textbf{Large-scale vision-language models}
like CLIP~\cite{CLIP} have shown promising performance on various tasks. They align visual and textual spaces to a joint space via training on millions of image-text pairs from the web. Other work~\cite{Align, DEFILIP, DeClip, FILIP, BLIP, Flamingo, SLIP, EVA-01, EVA-02} has furthered this paradigm to learn more accurate semantic alignment in joint space. 
In this work, we advance VLMs for downstream tasks by progressively learning optimal class-specific prompts with minimal supervision and no human intervention. 




% ---------------------------------------------------- % 
%                不同 Backbones 的实验
% ---------------------------------------------------- %
{
\renewcommand{\arraystretch}{1.1} 
\setlength{\tabcolsep}{3.8pt}

\begin{table*}[htbp]
  \centering
  \resizebox{0.98\linewidth}{!}
    {
    \begin{tabular}
        {l | ccccc ccccc ccc | c | c}
            
        \toprule
        \textbf{Module} & \rotatebox{90}{\textbf{IN-1K}} & \rotatebox{90}{\textbf{Caltech}} & \rotatebox{90}{\textbf{Cars}} & \rotatebox{90}{\textbf{CUB}} & \rotatebox{90}{\textbf{DTD}}  & \rotatebox{90}{\textbf{ESAT}} & \rotatebox{90}{\textbf{FGVC}} & \rotatebox{90}{\textbf{FLO}} & \rotatebox{90}{\textbf{Food}}  &  \rotatebox{90}{\textbf{Pets}} & \rotatebox{90}{\textbf{Places}} & \rotatebox{90}{\textbf{SUN}} & \rotatebox{90}{\textbf{UCF}} & \rotatebox{90}{\textbf{Avg (11)}} & \rotatebox{90}{\textbf{Avg (13)}} \\
        \midrule

        CLIP~\cite{CLIP} - ResNet50 &  57.9 & 84.5 & 53.9 & 44.7 & 38.8 & 28.6 & 15.9 & 60.2 & 74.0 & 83.2 & 38.2 & 58.0 & 56.9 & 55.6 & 53.4 \\ 
        CuPL~\cite{CuPL} &   61.2 & 88.3 & 55.3 & 48.7 & 49.5  & 38.2  & 18.9  & 67.0  & 80.1& 86.1& 41.2& 63.1  & 63.3   & 61.1  & 58.5 \\
        {\textbf{ProAPO} (ours)} & \textbf{61.5}  & \textbf{90.3} & \textbf{58.0} & \textbf{50.7} & \textbf{52.3} & \textbf{51.7} & \textbf{21.1} & \textbf{75.1} & \textbf{81.8} & \textbf{88.7} & \textbf{41.8} & \textbf{63.7} & \textbf{66.0} & \textbf{64.6}  & \textbf{61.8} \\

        $\Delta$ & \textcolor{retained}{+ 3.6} & \textcolor{retained}{+ 5.8} & \textcolor{retained}{+ 4.1} & \textcolor{retained}{+ 6.0} & \textcolor{retained}{+ 13.5} & \textcolor{retained}{+ 23.1} & \textcolor{retained}{+ 5.2} & \textcolor{retained}{+ 14.9} & \textcolor{retained}{+ 7.8} & \textcolor{retained}{+ 5.5} & \textcolor{retained}{+ 3.6} & \textcolor{retained}{+ 5.7} & \textcolor{retained}{+ 9.1} & \textcolor{retained}{+ 9.0} & \textcolor{retained}{+ 8.4}   \\

        \midrule

        CLIP~\cite{CLIP} - ResNet101 & 61.4  & 89.9  & 63.3  & 49.6  & 40.3  & 31.7  & 18.3  & 64.3  & 83.4  & 86.9  & 37.9  & 59.0  & 61.2  & 60.0  & 57.5  \\ 
        CuPL~\cite{CuPL} &  61.4  & 91.0  & 61.2  & 45.3  & 49.7  & 28.7  & 18.6  & 59.0  & 82.7  & 86.6  & \textbf{40.6}  & 62.3  & 56.4  & 59.8  & 57.2  \\
        {\textbf{ProAPO} (ours)} & \textbf{63.6} & \textbf{92.3} & \textbf{64.4} & \textbf{52.2} & \textbf{51.6} & \textbf{45.9} & \textbf{21.2} & \textbf{69.6} & \textbf{84.9} & \textbf{89.6} & \textbf{40.6} & \textbf{63.5} & \textbf{64.0} & \textbf{64.6}  & \textbf{61.8} \\
        $\Delta$ & \textcolor{retained}{+ 2.2} & \textcolor{retained}{+ 2.4} & \textcolor{retained}{+ 1.1} & \textcolor{retained}{+ 2.6} & \textcolor{retained}{+ 11.3} & \textcolor{retained}{+ 14.2} & \textcolor{retained}{+ 2.9} & \textcolor{retained}{+ 5.3} & \textcolor{retained}{+ 1.5} & \textcolor{retained}{+ 2.7} & \textcolor{retained}{+ 2.7} & \textcolor{retained}{+ 4.5} & \textcolor{retained}{+ 2.8} & \textcolor{retained}{+ 4.6} & \textcolor{retained}{+ 4.3} \\


        \midrule

        CLIP~\cite{CLIP} - ViT-B/32 & 62.1  & 91.2  & 60.4  & 51.7 & 42.9  & 43.9  & 20.2  & 66.0  & 83.2  & 86.8 & 39.9 & 62.1  & 60.9 & 61.8 & 59.3  \\
        CuPL~\cite{CuPL} &  {64.4}  & 92.9  & 60.7  & 53.3  & {50.6}  & 50.5  & 20.9  & 69.5  & 84.2  & 87.0  & {43.1}  & {66.3}  & 66.4  & 64.9  & 62.3  \\
        {\textbf{ProAPO} (ours)} & {\textbf{64.7}} & {\textbf{94.4}} & {\textbf{61.7}} & {\textbf{55.4}} & {\textbf{53.5}} & {\textbf{63.0}} & {\textbf{23.0}} & {\textbf{74.3}} & {\textbf{85.3}} & {\textbf{91.0}} & {\textbf{43.3}} & {\textbf{66.6}} & {\textbf{69.0}} & {\textbf{67.9}}  & {\textbf{65.0}} \\

        $\Delta$ & \textcolor{retained}{+ 2.6} & \textcolor{retained}{+ 3.2} & \textcolor{retained}{+ 1.3} & \textcolor{retained}{+ 3.7} & \textcolor{retained}{+ 10.6} & \textcolor{retained}{+ 19.1} & \textcolor{retained}{+ 2.8} & \textcolor{retained}{+ 8.3} & \textcolor{retained}{+ 2.1} & \textcolor{retained}{+ 4.2} & \textcolor{retained}{+ 3.4} & \textcolor{retained}{+ 4.5} & \textcolor{retained}{+ 8.1} & \textcolor{retained}{+ 6.1} & \textcolor{retained}{+ 5.7} \\

        \midrule

        CLIP~\cite{CLIP} - ViT-B/16 & 66.9  & 93.2  & 65.5  & 55.3  & 44.3  & 51.0  & 24.4  & 70.6  & 88.4  & 89.0  & 40.8  & 62.5  & 67.7  & 65.8  & 63.0  \\
        CuPL~\cite{CuPL} & 69.6  & 94.3  & 66.1  & 57.2  & 53.8  & 55.7  & 26.6  & 73.9  & 88.9  & 91.2  & 43.4  & \textbf{69.0}  & 70.3  & 69.0  & 66.1  \\
        {\textbf{ProAPO} (ours)} & \textbf{69.9} & \textbf{95.2} & \textbf{67.7} & \textbf{59.0} & \textbf{55.8} & \textbf{65.3} & \textbf{28.3} & \textbf{82.7} & \textbf{89.5} & \textbf{92.7} & \textbf{43.8} & {68.9} & \textbf{73.1} & \textbf{71.7}  & \textbf{68.6} \\
        $\Delta$ & \textcolor{retained}{+ 3.0} & \textcolor{retained}{+ 2.0} & \textcolor{retained}{+ 2.2} & \textcolor{retained}{+ 3.7} & \textcolor{retained}{+ 11.5} & \textcolor{retained}{+ 14.3} & \textcolor{retained}{+ 3.9} & \textcolor{retained}{+ 12.1} & \textcolor{retained}{+ 1.1} & \textcolor{retained}{+ 3.7} & \textcolor{retained}{+ 3.0} & \textcolor{retained}{+ 6.4} & \textcolor{retained}{+ 5.4} & \textcolor{retained}{+ 5.9} & \textcolor{retained}{+ 5.6} \\


        \midrule 


        CLIP~\cite{CLIP} - ViT-L/14 &  73.5  & 95.1  & 76.8  & 62.5  & 52.1  & 61.5  & 33.4  & 79.5  & 93.1  & 93.3  & 40.7  & 67.6  & 75.0   & 72.8  & 69.5 \\ 
        CuPL~\cite{CuPL} & 76.7  & 96.2 & 77.6 &  61.4 & 62.6 & 62.4 & 36.1  & 79.7  & 93.4 &  93.8 & 43.8 &  73.2  & 78.3  & 75.5  & 71.9   \\
        {\textbf{ProAPO} (ours)} & \textbf{76.8} & \textbf{97.1} & \textbf{78.8} & \textbf{65.1} & \textbf{64.8} & \textbf{74.3} & \textbf{38.3} & \textbf{87.3} & \textbf{93.9} & \textbf{94.6} & \textbf{44.4} & \textbf{73.4} & \textbf{80.1} & \textbf{78.1}  & \textbf{74.5} \\
        $\Delta$ & \textcolor{retained}{+ 3.3} & \textcolor{retained}{+ 2.0} & \textcolor{retained}{+ 2.0} & \textcolor{retained}{+ 2.6} & \textcolor{retained}{+ 12.7} & \textcolor{retained}{+ 12.8} & \textcolor{retained}{+ 4.9} & \textcolor{retained}{+ 7.8} & \textcolor{retained}{+ 0.8} & \textcolor{retained}{+ 1.3} & \textcolor{retained}{+ 3.7} & \textcolor{retained}{+ 5.3} & \textcolor{retained}{+ 5.1} & \textcolor{retained}{+ 5.8} & \textcolor{retained}{+ 5.0} \\

        \midrule 

        OpenCLIP~\cite{OpenCLIP} - ViT-B/32 &  66.2  & 94.7  & 88.2  & 65.6  & 51.3  & 49.4  & 23.0  & 71.2 & 82.4 & 90.7 & 41.5 & 68.1 & 65.0  & 68.2  & 65.9  \\ 
        CuPL~\cite{CuPL} &  66.7  & 94.4  & 86.6  & 65.9  & 62.4  & 50.1  & 25.5  & 69.5  & 81.7  & 90.8  & 43.3  & 69.1  & 65.8  & 69.3  & 67.1   \\
        {\textbf{ProAPO} (ours)} & \textbf{67.0} & \textbf{95.8} & \textbf{88.7} & \textbf{67.3} & \textbf{65.1} & \textbf{66.0} & \textbf{27.5} & \textbf{81.8} & \textbf{83.2} & \textbf{91.9} & \textbf{43.4} & \textbf{69.7} & \textbf{70.2} & \textbf{73.3}  & \textbf{70.6} \\
        $\Delta$ & \textcolor{retained}{+ 0.8} & \textcolor{retained}{+ 1.1} & \textcolor{retained}{+ 0.5} & \textcolor{retained}{+ 1.7} & \textcolor{retained}{+ 13.8} & \textcolor{retained}{+ 16.6} & \textcolor{retained}{+ 4.5} & \textcolor{retained}{+ 10.6} & \textcolor{retained}{+ 0.8} & \textcolor{retained}{+ 1.2} & \textcolor{retained}{+ 1.9} & \textcolor{retained}{+ 1.6} & \textcolor{retained}{+ 5.2} & \textcolor{retained}{+ 5.1} & \textcolor{retained}{+ 4.7} \\


        \midrule 

        EVA02~\cite{EVA-02} - ViT-B/16 & 74.6  & \textbf{97.2}  & 79.2  & 60.8  & 49.7  & 68.0  & 24.6  & 75.6  & 89.5  & 92.2  & 42.9  & 70.7 & 68.6  & 71.8  & 68.7  \\ 
        CuPL~\cite{CuPL} &  75.4  & 96.7  & 79.2  & 61.8  & 59.1  & 61.7  & 27.5  & 75.2  & 89.3  & 92.1  & 44.0  & 72.5  & 71.9  & 72.8  & 69.7 \\
        {\textbf{ProAPO} (ours)} & \textbf{75.5} & 97.0 & \textbf{80.0} & \textbf{62.8} & \textbf{61.3} & \textbf{74.2} & \textbf{29.7} & \textbf{89.1} & \textbf{89.6} & \textbf{93.5} & \textbf{44.5} & \textbf{72.5} & \textbf{75.2} & \textbf{76.2}  & \textbf{72.7}  \\
        $\Delta$ & \textcolor{retained}{+ 0.9} & -0.2 & \textcolor{retained}{+ 0.8} & \textcolor{retained}{+ 2.0} & \textcolor{retained}{+ 11.6} & \textcolor{retained}{+ 6.2} & \textcolor{retained}{+ 5.1} & \textcolor{retained}{+ 13.5} & \textcolor{retained}{+ 0.1} & \textcolor{retained}{+ 1.3} & \textcolor{retained}{+ 1.6} & \textcolor{retained}{+ 1.8} & \textcolor{retained}{+ 6.6} & \textcolor{retained}{+ 4.4} & \textcolor{retained}{+ 4.0}  \\

        \midrule 

        SigLIP~\cite{SigLIP} - ViT-B/16 & 75.8  & 97.3  & 90.5  & 62.3 & 62.8  & 44.6 & 43.6 & 85.5 & 91.5  & 94.1  & 41.6  & 69.5  & 74.9  & 75.5  & 71.8 \\ 
        CuPL~\cite{CuPL} &  76.0 & 98.0 & 90.5 & 63.0 & 64.9 & 42.8 & 45.1 & 87.0 & 90.7 & 94.5 & 43.5 & 69.9 & 73.4 & 75.7 & 72.3 \\
        {\textbf{ProAPO} (ours)} & \textbf{76.4} & \textbf{98.3} & \textbf{91.7} & \textbf{66.2} & \textbf{69.1} & \textbf{55.8} & \textbf{47.1} & \textbf{93.3} & \textbf{92.2} & \textbf{94.9} & \textbf{44.3} & \textbf{71.7} & \textbf{75.9} & \textbf{78.8}  & \textbf{75.2} \\
        $\Delta$ & \textcolor{retained}{+ 0.6} & \textcolor{retained}{+ 1.0} & \textcolor{retained}{+ 1.2} & \textcolor{retained}{+ 3.9} & \textcolor{retained}{+ 6.3} & \textcolor{retained}{+ 11.2} & \textcolor{retained}{+ 3.5} & \textcolor{retained}{+ 7.8} & \textcolor{retained}{+ 0.7} & \textcolor{retained}{+ 0.8} & \textcolor{retained}{+ 2.7} & \textcolor{retained}{+ 2.2} & \textcolor{retained}{+ 1.0} & \textcolor{retained}{+ 3.3} & \textcolor{retained}{+ 3.4} \\
        
        \bottomrule
    \end{tabular}
}
\vskip -0.04in
  \caption{\textbf{Results of our ProAPO on different backbones.} \textbf{Avg (11)} and \textbf{Avg (13)} denote average results across 11 datasets (excluding CUB~\cite{CUB} and Places~\cite{Places365}) and all 13 datasets, respectively. $\Delta$ denotes performance gains compared to vanilla VLMs.}
  \label{supp_tab: results_different_backbones}
  \vskip -0.15in
\end{table*}
}

\section{Results on Different Backbones}
\label{sec_supp: different_backbone_result}

% **************************** % 
%           实验设置
% **************************** %
\textbf{Settings}. In~\cref{supp_tab: results_different_backbones}, we show results of our ProAPO in different backbones, including ResNet50, ResNet101, ViT-B/32, ViT-B/16, ViT-L/14 for CLIP~\cite{CLIP}, ViT-B/32 for OpenCLIP~\cite{OpenCLIP}, ViT-B/16 for EVA02~\cite{EVA-02}, and ViT-B/16 for SigLIP~\cite{SigLIP}. We compare our ProAPO with vanilla VLMs and the SOTA description method CuPL~\cite{CuPL}.


% **************************** % 
%             结果
% **************************** %
\textbf{Results}. We see that our ProAPO consistently improves vanilla CLIP and CuPL in thirteen datasets across all backbones. Compared to vanilla VLMs, our ProAPO enhances them by at least 3.4\% average accuracy in thirteen datasets. Moreover, we see notable performance improvement on several fine-grained datasets, such as DTD~\cite{DTD}, ESAT~\cite{EuroSAT}, FLO~\cite{FLO}, and UCF~\cite{UCF101}. It further verifies that class-specific descriptions provide helpful knowledge for fine-grained recognition. Besides, iterative optimization by our ProAPO also enhances the description method CuPL. 


% **************************** % 
%         更多有趣的发现
% **************************** %
\textbf{More interesting findings}.
We find that as the backbones of VLMs become larger, the performance improvement by ProAPO gradually decreases. For example, from ViT-B/32 to ViT-B/16 to ViT-L/14, the gain for CLIP is from 5.7\% to 5.6\% to 5.0\%. Moreover, similar results appear in different models with the same backbone, \textit{i.e.}, the vanilla model with better results achieves a lower performance increase. For example, from CLIP~\cite{CLIP} to OpenCLIP~\cite{OpenCLIP} on ViT-B/32 backbone, the gain is from 5.7\% to 4.7\%, and from CLIP~\cite{CLIP} to EVA02~\cite{EVA-02} to SigLIP~\cite{SigLIP}, the gain is from 5.6\% to 4.0\% to 3.4\%. We argue that the model with the higher result has more knowledge, which may be affected less by prompt quality. Overall, our ProAPO continues to improve the performance of VLMs.

% ---------------------------------------------------- % 
%                    更多比较的实验
% ---------------------------------------------------- %
\section{More Comparisons with SOTA Methods}
\label{supp_sec: more_comparison_with_sota_methods}
In this section, we compare our ProAPO with more SOTA prompt tuning methods. These methods adapt VLMs from both visual and textual views.

{
\renewcommand{\arraystretch}{1.1} 
\setlength{\tabcolsep}{3.8pt}

\begin{table*}[htbp]
  \centering
  \resizebox{0.75\linewidth}{!}
    {
    \begin{tabular}
        {l | ccccc ccccc c | c}
            
        \toprule
        \textbf{Module} (ViT-B/16) & \rotatebox{90}{\textbf{IN-1K}} & \rotatebox{90}{\textbf{Caltech}} & \rotatebox{90}{\textbf{Cars}} & \rotatebox{90}{\textbf{DTD}}  & \rotatebox{90}{\textbf{ESAT}} & \rotatebox{90}{\textbf{FGVC}} & \rotatebox{90}{\textbf{FLO}} & \rotatebox{90}{\textbf{Food}}  &  \rotatebox{90}{\textbf{Pets}} & \rotatebox{90}{\textbf{SUN}} & \rotatebox{90}{\textbf{UCF}} & \rotatebox{90}{\textbf{Avg (11)}} \\
        \midrule

        Vanilla CLIP~\cite{CLIP} & 66.9  & 93.2  & 65.5 & 44.3  & 51.0  & 24.4  & 70.6  & 88.4  & 89.0   & 62.5  & 67.7  & 65.8  \\

        \midrule
        \multicolumn{13}{c}{\textit{\textbf{\ccol{Test-Time Prompt Tuning Methods}}}} \\
        \midrule 

        TPT~\cite{TPT} & 69.0 & 94.2 & 66.9 & 47.8 & 42.4 & 24.8 & 69.0 & 84.7 & 87.8 & 65.5 &  68.0 & 65.5   \\ 
        DiffTPT~\cite{DiffTPT} & 70.3 & 92.5 & 67.0 & 47.0 & 43.1 & 25.6 & 70.1 & 87.2 &  88.2 & 65.7 & 68.2 & 65.9    \\ 
        PromptAlign~\cite{PromptAlign} & 71.4 & 94.0 & 68.5 & 47.2 & 47.9 & 24.8 & 72.4 & 86.7 &  90.8 & 67.5 & 69.5 & 67.3  \\
        Self-TPT-v~\cite{Self_TPT_v} & \textbf{73.0} & 94.7 &  68.8 & 49.4 & 51.9 & 27.6 & 71.8 &  85.4 & 91.3 & 68.2 &  69.5 & 68.3  \\
        
        \midrule 
        \multicolumn{13}{c}{\textit{\textbf{\ccol{Vector-based Prompt Tuning Methods}}}} \\
        \midrule 
        UPT~\cite{prompt_tuning_UPT} & 69.6 & 93.7 & 67.6 & 45.0 & 66.5 & 28.4 & 75.0 & 84.2 & 82.9 & 68.8 & 72.0 & 68.5   \\
        CoCoOp~\cite{CoCoOp} & 69.4 & 93.8 & 67.2 & 48.5 & 55.3 & 12.7 & 72.1 & 85.7 & 91.3 & 68.3 & 70.3 & 66.8  \\
        MaPLe~\cite{MaPLe}  & 69.6 & 92.6 & 66.6 & 52.1 & 71.8 & 26.7 & 83.3 & 80.5 &  89.1 & 64.8 & 71.8 & 69.9   \\
        ALIGN~\cite{prompt_tuning_align} & 69.8 & 94.0 & 68.3 & 54.1 & 53.2 & 29.6 & 81.3 & 85.3 & 91.4 & 69.1 & 74.4 & 70.1   \\ 
        PromptSRC~\cite{PromptSRC} & 68.1 & 93.7 & 69.4 & 56.2 & \underline{73.1} & 27.7 & \underline{85.9} & 84.9 & 92.0 & 69.7 & 74.8 & 72.3    \\ 
        
        \midrule 
        \multicolumn{13}{c}{\textit{\textbf{\ccol{Description-Based Methods}}}} \\
        \midrule 
        
        \multicolumn{13}{l}{\textit{\textbf{w/o adapters}}} \\
        % ----- 同 Test-Time Adaptation 的比较 ----- %
        CuPL~\cite{CuPL} & 69.6  & 94.3  & 66.1   & 53.8  & 55.7  & 26.6  & 73.9  & 88.9  & 91.2  & 69.0  & 70.3  & 69.0  \\

        AWT-text~\cite{AWT} & 68.9  & 95.2  & 66.0   & 52.0  & 52.6  & 26.1  & 74.5  & 89.4  & 91.2   & 68.4  & 69.8 & 68.6  \\
        
        \highlight{\textbf{ProAPO} (ours)} & \highlight{69.9} & \highlight{95.2} & \highlight{67.7} & \highlight{55.8} & \highlight{65.3} & \highlight{28.3} & \highlight{82.7} & \highlight{89.5} & \highlight{92.7} & \highlight{68.9} & \highlight{73.1} & \highlight{71.7}  \\

        \highlight{\textbf{ProAPO} w/ AWT-text} & \highlight{69.4} & \highlight{\underline{95.3}} & \highlight{67.8} &  \highlight{54.3} & \highlight{67.1} & \highlight{27.4} & \highlight{82.1} & \highlight{\underline{89.6}} & \highlight{\underline{93.2}} & \highlight{68.5} & \highlight{73.1} & \highlight{71.6} \\

        \midrule 

         \multicolumn{13}{l}{\textit{\textbf{w/ adapters}}} \\

        AWT-Adapter~\cite{AWT} & \underline{72.1} & 95.1 & \textbf{73.4} & \underline{59.4} & \textbf{76.3} & \textbf{33.9} & 85.6 &  85.9 & 92.9 & \textbf{72.7} & \textbf{78.4} & \underline{75.1} \\
        
        \highlight{\textbf{ProAPO} w/ APE~\cite{APE}} & \highlight{71.3} & \highlight{\textbf{95.8}} & \highlight{\underline{70.9}} & \highlight{\textbf{60.6}} & \highlight{72.4} & \highlight{\underline{33.2}} & \highlight{\textbf{91.4}} & \highlight{\textbf{89.9}} & \highlight{\textbf{93.4}} & \highlight{\underline{71.0}} & \highlight{\underline{77.6}}  & \highlight{\textbf{75.2}} \\

        \bottomrule
    \end{tabular}
}
\vskip -0.04in
  \caption{\textbf{Comparison of our ProAPO with more SOTA methods under one-shot supervision.} \textbf{Avg (11)} denote average results across 11 datasets. }
  \label{supp_tab: comparison_with_more_SOTA}
  % \vskip -0.15in
\end{table*}
}


% **************************** % 
%      同 TPT 方法的对比
% **************************** %
% \noindent 
\textbf{Comparison of test-time prompt tuning methods}.
In~\cref{supp_tab: comparison_with_more_SOTA}, our ProAPO outperforms SOTA test-time prompt tuning methods on 11 datasets. Notably, we adapt VLMs solely from the textual view, while TPT methods introduce textual and visual views (\textit{i.e.}, augmented images), which further verifies the effectiveness of our method.


% **************************** % 
%   同更多 vector-based prompt tuning 方法的对比
% **************************** %
\textbf{Comparison of vector-based prompt-tuning methods}
 Since recent prompt-tuning methods adapt VLMs using both visual and textual views, we combine ProAPO with an adapter (\textit{i.e.}, APE~\cite{APE}) for a fair comparison. 
 \textbf{(1) Higher performance in low-shot.} In~\cref{supp_tab: comparison_with_more_SOTA}, ProAPO consistently outperforms these methods, which verifies that optimizing prompts in natural language is more effective in low-shot tasks. 
 \textbf{(2) Better transferability and interpretability}. Unlike vector-based prompt-tuning methods that search in a continuous space, ProAPO benefits from the discrete nature of natural language, leading to better interpretability and easily transfers across different backbones (shown in~\cref{tab: transfer_backbone}).
 \textbf{(3) Lower performance in high-shot}.  However, in~\cref{supp_tab: shots_influence}, ProAPO shows a sub-optimal result compared to CoOp~\cite{CoOp} in high-shot settings. This is due to the limited language search space and iteration steps.

% **************************** % 
%   同 AWT 方法的比较 
% **************************** %
\textbf{Comparison of AWT~\cite{AWT}}.
First, since AWT uses augmented visual and textual views to adapt VLMs, we compare ProAPO with AWT under the augmented textual view for a fair comparison. In~\cref{supp_tab: comparison_with_more_SOTA}, the result shows our ProAPO improves AWT-text by 6.1\% on average, verifying that our progressive optimization improves prompt quality. In addition, we introduce a common adapter-based method to our ProAPO and compare it with AWT-Adapter in the one-shot setting. We see that our ProAPO achieves comparable results. These results suggest that ProAPO and AWT are complementary.



% **************************** % 
%   同 iCM 方法的比较 
% **************************** %
\textbf{Comparison of iCM~\cite{iCM}}. iCM is somewhat similar to ours, optimizing class-specific prompts with chat-based LLMs. However, it uses the whole validation set as supervision. In~\cref{supp_tab: comparison_of_iCM}, we see that our ProAPO outperforms iCM significantly even under the one-shot supervision. 
This is because our ProAPO address challenges in class-specific prompt optimization by an offline generation algorithm to reduce LLM querying costs, an entropy-constrained fitness score to prevent overfitting, and two sampling strategies to find an optimal initial point and reduce iteration times.

% we solve high generation costs, long iteration times, and overfitting in class-specific optimization.



{
% \vspace{-10pt}
% \renewcommand{\arraystretch}{1.0} 
\begin{table*}[h]
  \centering
  \resizebox{0.8\linewidth}{!}
    {
    \begin{tabular}
        {l | ccccc ccc | c}  
        \toprule
        
        \textbf{Module} (ViT-B/32) & {\textbf{IN-1K}} & {\textbf{Caltech}} & {\textbf{CUB}} & {\textbf{DTD}}  & {\textbf{ESAT}} & {\textbf{FLO}} & {\textbf{SUN}} & {\textbf{UCF}} & \textbf{Avg (8)} \\
        \midrule
        Vanilla CLIP & 62.1 & 91.2 & 51.7 & 42.9 & 43.9 &  66.0 & 62.1 & 60.9 & 60.1 \\
        \midrule
        \multicolumn{10}{c}{\textit{\textbf{\ccol{Automatic Prompt Optimization Methods}}}} \\
        \midrule
        
        iCM~\cite{iCM} (w/ validation set) & 64.5 & 92.7 & \textbf{56.1} & 51.4 & 56.3 & 72.2 & 66.2 & 67.0 & 65.8  \\
        \highlight{\textbf{ProAPO} (w/ 1-shot)} &  \highlight{\textbf{64.7}} & \highlight{\textbf{94.4}} & \highlight{55.4} & \highlight{\textbf{53.5}} & \highlight{\textbf{63.0}} & \highlight{\textbf{74.3}} & \highlight{\textbf{66.6}} & \highlight{\textbf{69.0}} & \highlight{\textbf{67.6}} \\

        \bottomrule
    \end{tabular}
}
% \vspace{-5pt}
  \caption{\textbf{Comparison of our ProAPO with iCM~\cite{iCM}.} Avg (8) denotes average results across 8 datasets.}
  \label{supp_tab: comparison_of_iCM}
\end{table*}
}



% ---------------------------------------------------- % 
%           对 Template 和 Description 优化的消融
% ---------------------------------------------------- %
{
\renewcommand{\arraystretch}{1.1} 
% \setlength{\tabcolsep}{4pt}

\begin{table*}[htbp]
  \centering
  \resizebox{0.99\linewidth}{!}
    {
    \begin{tabular}
        {l | lllll lllll lll | l | l}

            
        \toprule
        \textbf{Module} (ResNet50) & \rotatebox{90}{\textbf{IN-1K}} & \rotatebox{90}{\textbf{Caltech}} & \rotatebox{90}{\textbf{Cars}} & \rotatebox{90}{\textbf{CUB}} & \rotatebox{90}{\textbf{DTD}}  & \rotatebox{90}{\textbf{ESAT}} & \rotatebox{90}{\textbf{FGVC}} & \rotatebox{90}{\textbf{FLO}} & \rotatebox{90}{\textbf{Food}}  &  \rotatebox{90}{\textbf{Pets}} & \rotatebox{90}{\textbf{Places}} & \rotatebox{90}{\textbf{SUN}} & \rotatebox{90}{\textbf{UCF}} & \rotatebox{90}{\textbf{Avg (11)}} & \rotatebox{90}{\textbf{Avg (13)}} \\
        \midrule

        % \multicolumn{16}{c}{\textit{\textbf{\ccol{ViT-B/32 Backbone}}}} \\
        % \midrule

        Vanilla CLIP & 57.9 & 84.5 & 53.9 & 44.7 & 38.8 & 28.6 & 15.9 & 60.2 & 74.0 & 83.2 & 38.2 & 58.0 & 56.9 & 55.6 & 53.4 \\ 
        
        \midrule
        \multicolumn{16}{c}{\textit{\textbf{\ccol{Template Optimization Methods}}}} \\

        \midrule 
        PN~\cite{P_N} & 59.6 & 89.1 & 56.2 & - & 44.8 & \underline{49.0} & 18.1 & 67.2 & 78.3 & 88.1 & - & 61.0 & 60.2 & 61.1 & -  \\

        \highlight{\textbf{ATO} (w/o dataset domain)} & \highlight{60.4} & \highlight{88.9} & \highlight{56.8} & \highlight{47.0} & \highlight{45.0} & \highlight{43.7} & \highlight{17.9} & \highlight{67.4} & \highlight{79.9} & \highlight{87.8} & \highlight{40.0} & \highlight{61.2} & \highlight{61.5} & \highlight{61.0} & \highlight{58.3} \\
        
        \highlight{\textbf{ATO}} & \highlight{\underline{61.3}} & \highlight{89.4} & \highlight{57.4} & \highlight{49.2} &  \highlight{45.4} & \highlight{46.4} & \highlight{18.4} & \highlight{68.1} & \highlight{80.5} & \highlight{88.5} & \highlight{40.2} &  \highlight{61.8} & \highlight{63.9} & \highlight{61.9}  & \highlight{59.3} \\

        \midrule

        \multicolumn{16}{c}{\textit{\textbf{\ccol{Description Optimization Methods}}}} \\

        \midrule
        \highlight{\textbf{ProAPO} (w/o synonyms)} & \highlight{\textbf{61.5}} & \highlight{\underline{89.7}} & \highlight{\textbf{58.3}} & \highlight{\underline{49.7}} & \highlight{\underline{46.6}} & \highlight{46.8} & \highlight{\underline{20.5}} & \highlight{\underline{74.6}} & \highlight{\underline{81.0}} & \highlight{\textbf{88.8}} & \highlight{\underline{40.9}} & \highlight{\underline{62.3}} & \highlight{\underline{64.8}} & \highlight{\underline{63.2}} & \highlight{\underline{60.4}} \\        
        
        \highlight{\textbf{ProAPO} (ours)} & \highlight{\textbf{61.5}} & \highlight{\textbf{90.3}} & \highlight{\underline{58.0}} & \highlight{\textbf{50.7}} & \highlight{\textbf{52.3}} & \highlight{\textbf{51.7}} & \highlight{\textbf{21.1}} & \highlight{\textbf{75.1}} & \highlight{\textbf{81.8}} & \highlight{\underline{88.7}} & \highlight{\textbf{41.8}} & \highlight{\textbf{63.7}} & \highlight{\textbf{66.0}} & \highlight{\textbf{64.6}}  & \highlight{\textbf{61.8}} \\
        
        \bottomrule
    \end{tabular}
}
  
  \caption{\textbf{Ablation of template and description optimization.} 
  Avg (11) and Avg (13) denote average results across 11 datasets (excluding CUB~\cite{CUB} and Places~\cite{Places365}) and all 13 datasets, respectively. ATO denotes our automatic template optimization algorithm.}
  \vspace{-4pt}
  \label{supp_tab: ablate_template_and_description}
\end{table*}
}

% ---------------------------------------------------- % 
%           更多的消融结果
% ---------------------------------------------------- %
\section{More Ablation Results}
\label{sec_supp: more_ablation_result}

% ---------------------------------------------------- % 
%           对 Template 和 Description 优化的消融
% ---------------------------------------------------- %
\subsection{Ablation of Template and Description Optimization}
\label{supp_sec: ablation_template_and_description}

In~\cref{supp_tab: ablate_template_and_description}, we ablate key components in template and description optimization on the ResNet50 backbone. 

\textbf{(1) Ablation of Template Optimization}. In the main paper (Sec. 4.3), we show that prompt ensembling is better than a single prompt. Moreover, dataset domain information also plays a significant role in template optimization. Without domain information, we see a performance drop in our ATO by an average of 1.0\% (from 58.3\% to 59.3 \%) on thirteen datasets. This is because domain information provides contextual information, which can mitigate issues of semantic ambiguity caused by class names.

\textbf{(2) Ablation of Description Optimization}. Without label synonyms to increase description diversity, a performance degradation appears by an average of 1.4\% (from 60.4\% to 61.8\%) on thirteen datasets. It verifies the effectiveness of optimization class names, which are usually ignored in previous description methods~\cite{CuPL, DCLIP, AdaptCLIP, GPT4Vis}.

\textbf{(3) Template VS Description Optimization}. Compared with template optimization, we see a notable performance improvement with description optimization, especially in CUB~\cite{CUB}, DTD~\cite{DTD}, ESAT~\cite{EuroSAT}, FLO~\cite{FLO}, and UCF~\cite{UCF101} datasets. It demonstrates that optimizing class-specific prompts can find discriminative information for fine-grained classification.


% ---------------------------------------------------- % 
%                   对每个 Operator 操作的消融
% ---------------------------------------------------- %
\subsection{More Ablation of Operators}
\label{supp_sec: more_ablation_operator}
To further explore whether each operator has a role in searching the optimal result, we show the number of each operator causing the new optimal score during the iterations in~\cref{supp_tab: more_ablation_operator}. We see that each operator in iterative optimization may generate a better prompt. It further demonstrates that each operator is helpful in ProAPO. Notably, the crossover operator has the highest times to update the optimal score, which demonstrates that it makes the model search for the optimal prompt faster with limited iterations. 

{
\renewcommand{\arraystretch}{1.1} 
% \setlength{\tabcolsep}{4.pt}
\begin{table}[t]
  \centering
  \resizebox{0.96\linewidth}{!}
    {
    \begin{tabular}
        {l | c  c  c  c  c | c  }  
        \toprule
        {\textbf{Dataset}} & \texttt{Add} & \texttt{Del} & \texttt{Rep} & \texttt{Cross} & \texttt{Mut} & \textbf{Total} \\
        \midrule
        IN-1K~\cite{Imagenet} & 3 & 4 & 5 & 5 & 2 &  19 \\
        Caltech~\cite{caltech101} & 5 & 5 & 6 & 12 & 3 & 31 \\
        Cars~\cite{Cars} & 7 & 8 & 5 & 8 & 3 & 31 \\ 
        CUB~\cite{CUB} & 9 & 4 & 10 & 6 & 2 & 31 \\
        DTD~\cite{DTD} & 5 & 3 & 8 & 8 & 2 & 26 \\
        ESAT~\cite{EuroSAT} & 2 & 4 & 6 & 8 & 1 & 21 \\
        FGVC~\cite{FGVC} & 6 & 2 & 6 & 5 & 3 & 22 \\ 
        FLO~\cite{FLO} & 5 & 3 & 11 & 5 & 4 & 28 \\
        Food~\cite{Food101} & 5 & 3 & 4 & 5 & 2 & 19 \\ 
        Pets~\cite{oxford_pets} & 4 & 2 & 5 & 6 & 2 & 19 \\ 
        Places~\cite{Places365} & 3 & 2 & 8 & 12 & 4 & 29 \\ 
        SUN~\cite{SUN} & 4 & 2 & 3 & 5 & 2 & 16 \\ 
        UCF~\cite{UCF101} & 5 & 6 & 8 & 6 & 2 & 27 \\ 
        \midrule
        \textbf{Sum} & 63 & 48 & 85 & 91 & 32 & 319 \\ 
        \bottomrule
    \end{tabular}
}
% \vspace{-6pt}
  \caption{\textbf{Number of times for each operator that update the optimal score.} \textbf{Total} denotes the total number of iterations when achieving the highest score.}
% \vspace{-8pt}
  \label{supp_tab: more_ablation_operator}
\end{table}
}

% ---------------------------------------------------- % 
%                   对 Group Sampling 的消融
% ---------------------------------------------------- %
{
\renewcommand{\arraystretch}{1.1} 
% \setlength{\tabcolsep}{4.pt}
\begin{table*}[t]
  \centering
  \resizebox{0.99\linewidth}{!}
    {
    \begin{tabular}
        {l | ccccc ccccc ccc | c | c | c }  
        \toprule
        {\textbf{Module} (ViT-B/32)}  & \rotatebox{90}{\textbf{IN-1K}} & \rotatebox{90}{\textbf{Caltech}} & \rotatebox{90}{\textbf{Cars}} & \rotatebox{90}{\textbf{CUB}} & \rotatebox{90}{\textbf{DTD}}  & \rotatebox{90}{\textbf{ESAT}} & \rotatebox{90}{\textbf{FGVC}} & \rotatebox{90}{\textbf{FLO}} & \rotatebox{90}{\textbf{Food}}  &  \rotatebox{90}{\textbf{Pets}} & \rotatebox{90}{\textbf{Places}} & \rotatebox{90}{\textbf{SUN}} & \rotatebox{90}{\textbf{UCF}} & \rotatebox{90}{\textbf{Avg (11)}} & \rotatebox{90}{\textbf{Avg (13)}} & \textbf{Times}  \\
        \midrule
        CuPL & 64.4  & 92.9  & 60.7  & 53.3  & {50.6}  & 50.5  & 20.9  & 69.5  & 84.2  & 87.0  & \underline{43.1}  & {66.3}  & 66.4  & 64.9  & 62.3 & - \\
        \midrule
        
        \texttt{a)} w/ all categories in one group & 64.5 & 93.3 & 60.9 & 53.5 & \underline{51.6} & 52.2 & 22.2 & 70.8 & 84.5 & 87.9 & 42.3 & \textbf{66.7} & \textbf{69.4} & 65.8  & 63.1 & \textbf{20 min} \\
        \texttt{b)} w/ random selected group & 64.3 & 93.7 & \textbf{61.8} & \underline{55.2} & 48.7 & 59.5 & 22.6 & 72.9 & \underline{85.2} & \underline{90.8} & 42.6 & 65.4 & 68.4 & 66.7  & 63.9 & 15 min  \\
        \texttt{c)} w/ performance best group & 64.1 & 93.0 & 61.2 & 54.4 & 47.4 & 56.8 & 20.7 & 68.2 & 85.1 & 88.6 & 42.4 & 65.0 & 65.4 & 65.0  & 62.5 & 15 min \\
        \texttt{d)} w/ K-Means algorithm & \underline{64.6} & \underline{93.8} & \textbf{61.8} & 55.1 & 49.4 & \underline{59.6} & \underline{22.8} & \underline{74.0} & \textbf{85.3} & 90.7 & {42.7} & 65.4 & \underline{69.0} & \underline{67.0}  & \underline{64.2} & \underline{17 min} \\
        
        \midrule

        
        \highlight{\textbf{ProAPO} (full model)} & \highlight{\textbf{64.7}}  & \highlight{\textbf{94.4}} & \highlight{\underline{61.7}} & \highlight{\textbf{55.4}} & \highlight{\textbf{53.5}} & \highlight{\textbf{63.0}} & \highlight{\textbf{23.0}} & \highlight{\textbf{74.3}} & \highlight{\textbf{85.3}} & \highlight{\textbf{91.0}} & \highlight{\textbf{43.3}} & \highlight{\underline{66.6}} & \highlight{\underline{69.0}} & \highlight{\textbf{67.9}}  & \highlight{\textbf{65.0}} & \highlight{15 min} \\
        \bottomrule
    \end{tabular}
}
    \vspace{-5pt}
  \caption{\textbf{More ablation of group sampling strategy.} We ablate the ways for selecting salient groups. \textbf{Times} denotes the time that ProAPO runs on ImageNet with the default setting.}
  \vspace{-7pt}
  \label{supp_tab: more_ablation_group_sampling}
\end{table*}

}


\subsection{More Ablation of Group Sampling}
\label{supp_sec: more_ablation_group_sampling}

In~\cref{supp_tab: more_ablation_group_sampling}, we ablate how to select categories in the group sampling strategy. We consider the settings for optimizing all categories in one group, selecting random categories and the best categories with their misclassified categories in groups. In rows a)-c) of~\cref{supp_tab: more_ablation_group_sampling}, we see notable performance degradation compared to full ProAPO. It further demonstrates that optimizing salient and worst groups can achieve comparable results with all categories and save iteration costs. Moreover, we also consider replacing misclassified categories with a K-Means clustering algorithm. A performance drop appears in row d), which verifies the effectiveness of selecting misclassified categories in groups.



% ---------------------------------------------------- % 
%                   对 Computation 的消融
% ---------------------------------------------------- %
\subsection{Ablation of Cost Computation}
\label{supp_sec: ablation_of_cost_computation}

In~\cref{supp_tab: extra_computation_cost}, we detail the time each process consumes on ImageNet. Compared to previous LLM-generated description methods, we similarly query LLMs one-time to generate descriptions (\textit{i.e.}, process of building prompt library). In addition, we introduce iterative processes to refine prompts and two sampling strategies to save costs. With a few additional costs (15 min v.s. 60 min), our ProAPO improves previous methods by at least 2.7\% on average. This further verifies the efficiency of our method.

{
% \vspace{-11pt}
\renewcommand{\arraystretch}{1.0} 
\setlength{\tabcolsep}{3.8pt}
\begin{table}[!h]
  \centering
  \resizebox{1.0\linewidth}{!}
    {
    \begin{tabular}
        {l | c | c c c c }  
        \toprule
        \textbf{Process} & \textbf{Build Library} & \highlight{\textbf{Sample Strategy}} & \highlight{\textbf{Template Optim.}} & \highlight{\textbf{Description Optim.}} \\ 
        \midrule 
        \textbf{Times} & 60 min &  \highlight{3 min}  & \highlight{1.6 min} & \highlight{10.4 min} \\
        \bottomrule
    \end{tabular}
}
\vspace{-5pt}
  \caption{\textbf{Computation cost analysis in the ImageNet dataset.}}
% \vspace{-18pt}
  \label{supp_tab: extra_computation_cost}
\end{table}
}




% ---------------------------------------------------- % 
%                    超参数分析
% ---------------------------------------------------- %
\section{More Hyperparameter Analysis}
\label{sec_supp: more_hyper_analysis}




{
\renewcommand{\arraystretch}{1.1} 
\setlength{\tabcolsep}{4.pt}
\begin{table}[tbp]
  \centering
  \resizebox{0.99\linewidth}{!}
    {
    \begin{tabular}
        {l | l | c | c c c c c | c }

            
        \toprule
        {\textbf{Dataset}} & \textbf{Module} & {\textbf{TF}} & \multicolumn{5}{c}{\textbf{Number of training samples}} & \textbf{UB} \\
        \cmidrule(lr){4-8}
         & (RN50) & &  1 & 2 & 4 & 8 & 16\\
        \midrule
        \multirow{2}{*}{\textbf{Avg (11)}} & CoOp~\cite{CoOp} & \xmark & 59.6 & 62.3 & \textbf{66.8} & \textbf{69.9} & \textbf{73.4} & -  \\
         & {\textbf{ProAPO}} & \cmark & \textbf{64.6} & \textbf{65.0} & 65.4 & 65.8 & 66.1 & 67.2 \\
        \midrule
        
        \multirow{2}{*}{\textbf{IN-1K}} & CoOp~\cite{CoOp} & \xmark  & 57.2 & 57.8 & 60.0 & \textbf{61.6} & \textbf{63.0} & -  \\
        & {\textbf{ProAPO}} & \cmark & \textbf{61.5} & \textbf{61.6} & \textbf{61.5} & \textbf{61.6} & 61.6 & 61.7 \\
        \midrule
        
        \multirow{2}{*}{\textbf{Caltech}} & CoOp~\cite{CoOp} & \xmark  & 87.5 & 87.9 & 89.6 & 90.2 & 91.8 & -  \\
        & {\textbf{ProAPO}} & \cmark & \textbf{90.3} & \textbf{90.4} & \textbf{90.6} & \textbf{90.7} & \textbf{91.0} & 91.1 \\        
        \midrule
        
        \multirow{2}{*}{\textbf{Cars}} & CoOp~\cite{CoOp} & \xmark & 55.6 & 58.3 & \textbf{62.6} & \textbf{68.4} & \textbf{73.4} & -  \\
        & {\textbf{ProAPO}} & \cmark & \textbf{58.0} & \textbf{58.5} & 58.8 & 58.9 & 59.1 & 60.8  \\
        \midrule
        
        \multirow{2}{*}{\textbf{DTD}} & CoOp~\cite{CoOp} & \xmark & 44.4 & 45.2 & \textbf{53.5} & \textbf{60.0} & \textbf{63.6} & -  \\
        & {\textbf{ProAPO}} & \cmark & \textbf{52.3} & \textbf{52.7} & 53.0 & 53.4 & 53.6 \\
        \midrule
        
        \multirow{2}{*}{\textbf{ESAT}} & CoOp~\cite{CoOp} & \xmark & 50.6 & \textbf{61.5} & \textbf{70.2} & \textbf{76.7} & \textbf{83.5} & -  \\
        & {\textbf{ProAPO}} & \cmark & \textbf{51.7} & 53.5 & 55.6 & 57.4 & 58.3 & 62.2 \\
        \midrule
        
        \multirow{2}{*}{\textbf{FGVC}} & CoOp~\cite{CoOp} & \xmark & 9.6 & 18.7 & \textbf{21.9} & \textbf{26.1} & \textbf{31.3} & -  \\
        & {\textbf{ProAPO}} & \cmark & \textbf{21.1} & \textbf{21.0} & 21.2 & 21.2 & 21.3 & 21.5 \\
        \midrule
        
        \multirow{2}{*}{\textbf{FLO}} & CoOp~\cite{CoOp} & \xmark & 68.1 & \textbf{77.5} & \textbf{86.2} & \textbf{91.2} & \textbf{94.5} & -  \\
        & {\textbf{ProAPO}} & \cmark & \textbf{75.1} & 75.6 & 76.4 & 76.7 & 77.8 & 79.1 \\
        \midrule
        
        \multirow{2}{*}{\textbf{Food}} & CoOp~\cite{CoOp} & \xmark  & 74.3 & 72.5 & 73.3 & 71.8 & 74.7 & -  \\
        & {\textbf{ProAPO}} & \cmark & \textbf{81.8} & \textbf{82.0} & \textbf{82.1} & \textbf{82.2} & \textbf{82.3} & {82.9} \\
        \midrule
        
        \multirow{2}{*}{\textbf{Pets}} & CoOp~\cite{CoOp} & \xmark  & 85.9 & 82.6 & 86.7 & 85.3 & 87.0 & -  \\
        & {\textbf{ProAPO}} & \cmark & \textbf{88.7} & \textbf{89.4} & \textbf{89.5} & \textbf{89.8} & \textbf{89.9} & {91.0} \\
        \midrule
        
        \multirow{2}{*}{\textbf{SUN}} & CoOp~\cite{CoOp} & \xmark  & 60.3 & 59.5 & 63.5 & \textbf{65.5} & \textbf{69.3} & -  \\
        & {\textbf{ProAPO}} & \cmark & \textbf{63.7} & \textbf{63.8} & \textbf{63.8} & 63.8 & 63.9 & 64.5 \\
        \midrule
        
        \multirow{2}{*}{\textbf{UCF}} & CoOp~\cite{CoOp} & \xmark  & 61.9 & 64.1 & 67.0 & \textbf{71.9} & \textbf{75.7} & -  \\
        & {\textbf{ProAPO}} & \cmark &  \textbf{66.0} & \textbf{66.8} & \textbf{67.1} & 68.1 & 68.9 & 71.4 \\
        \bottomrule
    \end{tabular}
}
  \vspace{-5pt}
  \caption{\textbf{Scaling up to more shots.} \textbf{Avg (11)} denotes average results across 11 datasets. \textbf{TF} denotes training-free approaches. \textbf{UB} denotes upper bound evaluated on the test set.}
  % \vspace{-5pt}
  \label{supp_tab: shots_influence}
\end{table}
}



% ---------------------------------------------------- % 
%                    Shots 的影响
% ---------------------------------------------------- %
\subsection{Effect of Shot Numbers}
\label{supp_sec: effect_shots}
In~\cref{supp_tab: shots_influence}, we show the effect of the number of training samples per category.
Specifically, we conduct experiments with 1, 2, 4, 8, and 16 shots.
Moreover, we introduce the performance of the optimal prompt searched in the test set as the upper bound of ProAPO.
Compared with CoOp~\cite{CoOp}, ProAPO achieves remarkable performance when shots $\leq 2$, which demonstrates the effectiveness of our method under low-shot settings. Since we only adapt VLMs in a training-free way, the performance increases finitely as the training samples increase. We attribute two key directions for further performance improvement in high-shot settings. First, our result is still far from the upper bound (66.1 \% in 16 shots VS 67.2 \% for the upper bound). We need to improve the prompt generation algorithm and the score function to find better candidate prompts within the limited iterations. Second, the upper bound of our ProAPO is much smaller than the prompt tuning method. We need to use a larger natural language search space (\textit{e.g.}, more diverse descriptions, or more query times of LLMs) to further increase the upper bound of the optimal result.



% ---------------------------------------------------- % 
%                    Alpha 值的影响
% ---------------------------------------------------- %
\subsection{Effect of Scalar in Score Function}
\label{supp_sec: effect_alpha}

In~\cref{supp_tab: ablation_alpha},  we show the effect of $\alpha$ in~\cref{eq: score_function}. We see that performance improves as the $\alpha$ increases. This is because the entropy constraint provides more information to select better candidate prompts. We see a stable result when $\alpha \in [5e2, 5e3]$, which means a better trade-off between accuracy and entropy constraint. However, a high $\alpha$ may be biased to the train set, thus harming the performance. 

{
% \vspace{-10pt}
% \renewcommand{\arraystretch}{1.0} 
\begin{table}[h]
  \centering
  \resizebox{1.0\linewidth}{!}
    {
    \begin{tabular}
        {l | c c c c c c c c}  
        \toprule
        $\alpha$ & $0$ & $1e1$ & $1e2$ & $5e2$ & $1e3$ & $5e3$ & $1e4$ & $1e5$ \\
        \midrule
        \textbf{Avg (13)} & 62.3 & 63.4 & 64.4 & 64.9 & \textbf{65.0} & 64.8 & 63.7 & 63.1 \\
        \bottomrule
    \end{tabular}
}
% \vspace{-5pt}
  \caption{\textbf{Effect of $\alpha$ value in Eq.6 across 13 datasets.}}
  % \vspace{-5pt}
  \label{supp_tab: ablation_alpha}
\end{table}
}


% ---------------------------------------------------- % 
%                   T_sample 的影响
% ---------------------------------------------------- %
\subsection{Effect of Sampled Numbers in Prompt Sampling Strategy}
\label{supp_sec: effect_sampled_numbers}
In~\cref{supp_fig: effect_sampled_numbers}, we show the effect of sampled numbers $T_{sample}$ of Alg.~\ref{supp_alg: prompt_strategy}. The $T_{sample} = 0$ means that the prompt sampling strategy is not used. As the number of $T_{sample}$ increases, we see a slight performance gain when $T_{sample} < 4$. After $T_{sample} \geq 4$, a consistent improvement appears because the initial search point achieves a higher score than the baseline. We achieve stable results when $T_{sample} \geq 32$.

\begin{figure}[htbp]
\centering
\includegraphics[width=0.8\linewidth]{Paper_Result_supp/hyper_result_Sampled_Numbers.pdf}
\vspace{-8pt}
\caption{\textbf{Effect of sampled numbers $T_{sample}$.} 
}
\label{supp_fig: effect_sampled_numbers}
\vspace{-6pt}
\end{figure}


% ---------------------------------------------------- % 
%                   Prompt Library 的影响
% ---------------------------------------------------- %
\subsection{Effect of Quality of Prompt Library}
\label{supp_sec: effect_of_quality_of_prompt_library}
In~\cref{supp_fig: effect_LLM_Query} and~\cref{supp_fig: effect_generated_descriptions}, we analyze two key factors affecting the prompt library: LLM-query prompts and generated descriptions. Our ProAPO improves prompt quality even under a small number of query prompts and descriptions, demonstrating its effectiveness in a limited prompt library. 




\begin{figure}[htbp]
\centering
\includegraphics[width=0.8\linewidth]{Paper_Result_supp/LLM-query.pdf}
\vspace{-8pt}
\caption{\textbf{Effect of Number of LLM-query Prompts.} 
}
\label{supp_fig: effect_LLM_Query}
% \vspace{-6pt}
\end{figure}



\begin{figure}[htbp]
\centering
\includegraphics[width=0.8\linewidth]{Paper_Result_supp/generated_descriptions.pdf}
\vspace{-8pt}
\caption{\textbf{Effect of Number of Generated Descriptions.} 
}
\label{supp_fig: effect_generated_descriptions}
% \vspace{-6pt}
\end{figure}




% ---------------------------------------------------- % 
%                     更多的可视化
% ---------------------------------------------------- %



\section{More Qualitative Results}
\label{sec_supp: more_qualitative_result}
In~\cref{supp_fig: qualitative_result}, we show more examples of the changes in descriptions with our ProAPO, including images of animals, flowers, and textures.
Similarly, we see that common descriptions are removed and discriminative ones are retained for fine-grained categories, which further verifies the effectiveness of our progressive optimization.


\begin{figure*}[htbp]
\centering
\includegraphics[width=0.8\linewidth]{Paper_Result_supp/qualitative_results_supp.pdf}
% \vspace{-8pt}
\caption{\textbf{Qualitative analysis of class-specific prompt optimization by ProAPO.} Shaded \textbf{\textcolor{removed}{red}} and \textbf{\textcolor{retained}{blue}} words denote common and discriminative descriptions in two confused categories.
}
\label{supp_fig: qualitative_result}
\vspace{20pt}
\end{figure*} 


% \newpage




\newpage

\section{Detailed Results of More Benefits by Optimal Prompts}
\label{sec_supp: detailed_results_of_main_paper}

% ---------------------------------------------------- % 
%               迁移到 Adapter 方法的结果
% ---------------------------------------------------- %
\subsection{Transfer to Adapter-based Methods}
\label{sec_supp: transfer_to_adapter}
In~\cref{supp_fig: improve_train_free_adapter} and ~\cref{supp_fig: improve_train_adapter}, we show the detailed results of popular training-free and training adapter-based methods~\cite{Tip, Tip-X, APE, CLIP_Adapter} with different prompt initialization, \textit{i.e.}, SOTA method CuPL~\cite{CuPL} and our ProAPO. Adapter-based methods with ProAPO (solid lines) consistently surpass those with CuPL (dotted lines). It reveals that high-quality prompts make adapters perform better. Even in low shots, training with ProAPO achieves notable performance gains, which further verifies its effectiveness. 


{
\begin{figure*}[ht]
\centering
\begin{adjustbox}{minipage=\textwidth,scale=0.88}
\begin{subfigure}{0.33\textwidth}
    \includegraphics[width=\textwidth]{Paper_Result_supp/transfer_adapter/Training-free_Results_on_ImageNet.pdf}
    \caption{ImageNet.}
    % \label{fig:first}
    % \vspace{-10pt}
\end{subfigure}
\hfill
\begin{subfigure}{0.33\textwidth}
    \includegraphics[width=\textwidth]{Paper_Result_supp/transfer_adapter/Training-free_Results_on_Caltech-101.pdf}
    \caption{Caltech.}
    % \label{fig:second}
    % \vspace{-10pt}
\end{subfigure}
\hfill
\begin{subfigure}{0.33\textwidth}
    \includegraphics[width=\textwidth]{Paper_Result_supp/transfer_adapter/Training-free_Results_on_StanfordCars.pdf}
    \caption{Cars.}
    % \label{fig:third}
    % \vspace{-10pt}
\end{subfigure}
% \vspace{-5pt}
\begin{subfigure}{0.33\textwidth}
    \includegraphics[width=\textwidth]{Paper_Result_supp/transfer_adapter/Training-free_Results_on_DTD.pdf}
    \caption{DTD.}
    % \label{fig:first}
    % \vspace{-10pt}
\end{subfigure}
\hfill
\begin{subfigure}{0.33\textwidth}
    \includegraphics[width=\textwidth]{Paper_Result_supp/transfer_adapter/Training-free_Results_on_FGVC.pdf}
    \caption{FGVC.}
    % \label{fig:second}
    % \vspace{-10pt}
\end{subfigure}
\hfill
\begin{subfigure}{0.33\textwidth}
    \includegraphics[width=\textwidth]{Paper_Result_supp/transfer_adapter/Training-free_Results_on_EuroSAT.pdf}
    \caption{ESAT.}
    % \label{fig:third}
    % \vspace{-10pt}
\end{subfigure}

\begin{subfigure}{0.33\textwidth}
    \includegraphics[width=\textwidth]{Paper_Result_supp/transfer_adapter/Training-free_Results_on_Flowers102.pdf}
    \caption{FLO.}
    % \label{fig:first}
    % \vspace{-10pt}
\end{subfigure}
\hfill
\begin{subfigure}{0.33\textwidth}
    \includegraphics[width=\textwidth]{Paper_Result_supp/transfer_adapter/Training-free_Results_on_Food101.pdf}
    \caption{Food.}
    % \label{fig:second}
    % \vspace{-10pt}
\end{subfigure}
\hfill
\begin{subfigure}{0.33\textwidth}
    \includegraphics[width=\textwidth]{Paper_Result_supp/transfer_adapter/Training-free_Results_on_OxfordPets.pdf}
    \caption{Pets.}
    % \label{fig:third}
    % \vspace{-10pt}
\end{subfigure}

\begin{subfigure}{0.33\textwidth}
    \includegraphics[width=\textwidth]{Paper_Result_supp/transfer_adapter/Training-free_Results_on_SUN397.pdf}
    \caption{SUN.}
    % \label{fig:first}
    % \vspace{-10pt}
\end{subfigure}
% \hfill
\begin{subfigure}{0.33\textwidth}
    \includegraphics[width=\textwidth]{Paper_Result_supp/transfer_adapter/Training-free_Results_on_UCF101.pdf}
    \caption{UCF.}
    % \label{fig:second}
    % \vspace{-10pt}
\end{subfigure}
\end{adjustbox}
\caption{\textbf{Results of training-free adapter-based methods with different initial prompts.} Solid and dotted lines denote prompt initialization with ProAPO and CuPL, respectively. We see that our ProAPO consistently improves adapter-based methods.}
\label{supp_fig: improve_train_free_adapter}
\vspace{60pt}
\end{figure*}
}


{
\begin{figure*}[ht]
\centering
\begin{adjustbox}{minipage=\textwidth,scale=0.88}
\begin{subfigure}{0.33\textwidth}
    \includegraphics[width=\textwidth]{Paper_Result_supp/transfer_adapter/Training_Results_on_ImageNet.pdf}
    \caption{ImageNet.}
    % \label{fig:first}
    % \vspace{-10pt}
\end{subfigure}
\hfill
\begin{subfigure}{0.33\textwidth}
    \includegraphics[width=\textwidth]{Paper_Result_supp/transfer_adapter/Training_Results_on_Caltech-101.pdf}
    \caption{Caltech.}
    % \label{fig:second}
    % \vspace{-10pt}
\end{subfigure}
\hfill
\begin{subfigure}{0.33\textwidth}
    \includegraphics[width=\textwidth]{Paper_Result_supp/transfer_adapter/Training_Results_on_StanfordCars.pdf}
    \caption{Cars.}
    % \label{fig:third}
    % \vspace{-10pt}
\end{subfigure}
% \vspace{-5pt}
\begin{subfigure}{0.33\textwidth}
    \includegraphics[width=\textwidth]{Paper_Result_supp/transfer_adapter/Training_Results_on_DTD.pdf}
    \caption{DTD.}
    % \label{fig:first}
    % \vspace{-10pt}
\end{subfigure}
\hfill
\begin{subfigure}{0.33\textwidth}
    \includegraphics[width=\textwidth]{Paper_Result_supp/transfer_adapter/Training_Results_on_FGVC.pdf}
    \caption{FGVC.}
    % \label{fig:second}
    % \vspace{-10pt}
\end{subfigure}
\hfill
\begin{subfigure}{0.33\textwidth}
    \includegraphics[width=\textwidth]{Paper_Result_supp/transfer_adapter/Training_Results_on_EuroSAT.pdf}
    \caption{ESAT.}
    % \label{fig:third}
    % \vspace{-10pt}
\end{subfigure}

\begin{subfigure}{0.33\textwidth}
    \includegraphics[width=\textwidth]{Paper_Result_supp/transfer_adapter/Training_Results_on_Flowers102.pdf}
    \caption{FLO.}
    % \label{fig:first}
    % \vspace{-10pt}
\end{subfigure}
\hfill
\begin{subfigure}{0.33\textwidth}
    \includegraphics[width=\textwidth]{Paper_Result_supp/transfer_adapter/Training_Results_on_Food101.pdf}
    \caption{Food.}
    % \label{fig:second}
    % \vspace{-10pt}
\end{subfigure}
\hfill
\begin{subfigure}{0.33\textwidth}
    \includegraphics[width=\textwidth]{Paper_Result_supp/transfer_adapter/Training_Results_on_OxfordPets.pdf}
    \caption{Pets.}
    % \label{fig:third}
    % \vspace{-10pt}
\end{subfigure}

\begin{subfigure}{0.33\textwidth}
    \includegraphics[width=\textwidth]{Paper_Result_supp/transfer_adapter/Training_Results_on_SUN397.pdf}
    \caption{SUN.}
    % \label{fig:first}
    % \vspace{-10pt}
\end{subfigure}
% \hfill
\begin{subfigure}{0.33\textwidth}
    \includegraphics[width=\textwidth]{Paper_Result_supp/transfer_adapter/Training_Results_on_UCF101.pdf}
    \caption{UCF.}
    % \label{fig:second}
    % \vspace{-10pt}
\end{subfigure}
\end{adjustbox}
\caption{\textbf{Results of training adapter-based methods with different initial prompts.} Solid and dotted lines denote prompt initialization with ProAPO and CuPL, respectively. We see that our ProAPO consistently improves adapter-based methods.}
\label{supp_fig: improve_train_adapter}
\vspace{60pt}
\end{figure*}
}


% ---------------------------------------------------- % 
%               迁移到不同 backbones 的结果
% ---------------------------------------------------- %
\subsection{Transfer to Different Backbones}
\label{sec_supp: transfer_to_backbones}

In~\cref{supp_fig: transfer_backbones}, we show detailed results of transferring prompts from source to target models in thirteen datasets. Our optimized prompts of ResNet50 and ViT-B/32 are reported.
We see that ProAPO achieves stable performance gains compared to CuPL~\cite{CuPL}, which verifies that ProAPO transfers easily across different backbones.




{
\begin{figure*}[htbp]
\centering
% \resizebox{0.9\linewidth}{!}
{
    \hfill
    \subfloat[ImageNet]{\includegraphics[width=0.32\textwidth]{Paper_Result_supp/transfer_backbones/transfer_to_backbones_imagenet.pdf}}
    \hfill
    \subfloat[Caltech]{\includegraphics[width=0.32\textwidth]{Paper_Result_supp/transfer_backbones/transfer_to_backbones_caltech101.pdf}}
    \hfill
    \subfloat[Cars]{\includegraphics[width=0.32\textwidth]{Paper_Result_supp/transfer_backbones/transfer_to_backbones_stanford_cars.pdf}}
    \hfill
    % \vspace{0.5em}
    \subfloat[CUB]{\includegraphics[width=0.32\textwidth]{Paper_Result_supp/transfer_backbones/transfer_to_backbones_cub.pdf}}
    \hfill
    \subfloat[DTD]{\includegraphics[width=0.32\textwidth]{Paper_Result_supp/transfer_backbones/transfer_to_backbones_dtd.pdf}}
    \hfill
    \subfloat[FGVC]{\includegraphics[width=0.32\textwidth]{Paper_Result_supp/transfer_backbones/transfer_to_backbones_fgvc_aircraft.pdf}}
    \hfill
    % \vspace{0.5em}
    \subfloat[ESAT]{\includegraphics[width=0.32\textwidth]{Paper_Result_supp/transfer_backbones/transfer_to_backbones_euro_sat.pdf}}
    \hfill
    \subfloat[FLO]{\includegraphics[width=0.32\textwidth]{Paper_Result_supp/transfer_backbones/transfer_to_backbones_flo.pdf}}
    \hfill
    \subfloat[Food]{\includegraphics[width=0.32\textwidth]{Paper_Result_supp/transfer_backbones/transfer_to_backbones_food101.pdf}}
    \hfill
    % \vspace{0.5em}
    \subfloat[Pets]{\includegraphics[width=0.32\textwidth]{Paper_Result_supp/transfer_backbones/transfer_to_backbones_oxford_pets.pdf}}
    \hfill
    \subfloat[Places]{\includegraphics[width=0.32\textwidth]{Paper_Result_supp/transfer_backbones/transfer_to_backbones_places365.pdf}}
    \hfill
    \subfloat[SUN]{\includegraphics[width=0.32\textwidth]{Paper_Result_supp/transfer_backbones/transfer_to_backbones_sun397.pdf}}
    \hfill
    % \vspace{0.5em}
    \subfloat[UCF]{\includegraphics[width=0.32\textwidth]{Paper_Result_supp/transfer_backbones/transfer_to_backbones_ucf101.pdf}}
    % \hfill
    \subfloat[Avg (11)]{\includegraphics[width=0.32\textwidth]{Paper_Result_supp/transfer_backbones/transfer_to_backbones_avg11.pdf}}
    % \hfill
}
\vspace{-7pt}
\caption{\textbf{Results of prompt transfer to different backbones.} The value denotes performance gains compared to vanilla VLMs. Our optimized prompts of ResNet50 and ViT-B/32 are reported. We see that we achieve stable performance gains compared to CuPL~\cite{CuPL}. 
}
% \Description{}
\label{supp_fig: transfer_backbones}
\vspace{-5pt}
\end{figure*}
}



% ---------------------------------------------------- % 
%               性能提升分析的详细实验
% ---------------------------------------------------- %
% \newpage
\section{Detailed Results of Performance Improvement Analysis}

% ---------------------------------------------------- % 
%               Single VS Ensemble Prompts
% ---------------------------------------------------- %

{
\renewcommand{\arraystretch}{1.1} 
\begin{table*}[htbp]
  \centering
  \resizebox{0.75\linewidth}{!}
    {
    \begin{tabular}
        {l |  ccccc ccccc c | c }
        \toprule
        \textbf{Module} (ResNet50) & \rotatebox{90}{\textbf{IN-1K}} & \rotatebox{90}{\textbf{Caltech}} & \rotatebox{90}{\textbf{Cars}}  & \rotatebox{90}{\textbf{DTD}}  & \rotatebox{90}{\textbf{ESAT}} & \rotatebox{90}{\textbf{FGVC}} & \rotatebox{90}{\textbf{FLO}} & \rotatebox{90}{\textbf{Food}}  &  \rotatebox{90}{\textbf{Pets}}  & \rotatebox{90}{\textbf{SUN}} & \rotatebox{90}{\textbf{UCF}} & \rotatebox{90}{\textbf{Avg (11)}}  \\
        \midrule

        CLIP (a photo of a \{\})  & 57.9 & 84.5 & 53.9 & 38.8 & 28.6 & 15.9 & 60.2 & 74.0 & 83.2 & 58.0 & 56.9 & 55.6 \\ 

        \midrule
        \multicolumn{13}{c}{\textit{\textbf{\ccol{Single Prompt}}}} \\
        \midrule
        
        PN~\cite{P_N} & {59.6} & 89.1 & 56.2  & 44.8 & \underline{49.0} & 18.1 & 67.2 & {78.3} & 88.1  & 61.0 & 60.2 & 61.1 \\
        
        Best Single* & 60.2 & \underline{89.2} & \underline{57.9}  & 45.0 & 46.0 & \underline{18.3} & \underline{68.1} & \underline{81.8} & 88.3 & 61.5 & 62.6 & 61.7 \\

        \midrule
        \multicolumn{13}{c}{\textit{\textbf{\ccol{Ensemble Prompt}}}} \\
        \midrule 
        
        \highlight{\textbf{ATO} (ours)} & \highlight{\underline{61.3}} & \highlight{\underline{89.2}} & \highlight{\underline{57.9}} & \highlight{\underline{45.4}} & \highlight{44.7} & \highlight{18.2} & \highlight{\underline{68.1}} & \highlight{\underline{81.8}} & \highlight{\underline{88.5}} & \highlight{\underline{61.8}} & \highlight{\underline{63.9}} & \highlight{\underline{61.9}}  \\
        
        Best Ensemble* & \textbf{61.5} & \textbf{90.0} & \textbf{58.4}  & \textbf{47.0} & \textbf{49.1} & \textbf{18.7} & \textbf{69.9} & \textbf{82.2} & \textbf{89.4} & \textbf{62.1} & \textbf{64.8} & \textbf{63.0}  \\

        % \midrule
        
        \bottomrule
    \end{tabular}
}
  \vspace{-6pt}
  \caption{\textbf{Analysis of the effect of single vs ensemble prompts.} * denotes results evaluated in the test set. ATO is our automatic template optimization algorithm. We see that our optimized templates achieve higher results than PN~\cite{P_N}, even better than the best single template.}
  \vspace{-6pt}
  \label{supp_tab: ensemble_vs_single}
\end{table*}
}



\subsection{Analysis of the Effect of Single VS Ensemble Prompts}
\label{supp_sec: ensemble_vs_single}
In~\cref{supp_tab: ensemble_vs_single}, we show detailed results of the effect of single vs ensemble prompts. Compared to PN~\cite{P_N}, we utilize prompt ensembling instead of a single prompt to optimize the template and description. We observe that ensemble templates have a higher upper bound than the single template. Similarly, our optimized templates achieve higher performance than PN~\cite{P_N}, even better than the best single template, further verifying the effectiveness of our method.


% ---------------------------------------------------- % 
%               提升 Description 方法的性能
% ---------------------------------------------------- %
\subsection{Performance Improvement of Description Methods by ProAPO}
\label{sec_supp: improve_description_methods}
In~\cref{tab:description_add_APO}, we show detailed results of description methods~\cite{DCLIP, CuPL, GPT4Vis, AdaptCLIP} with our ATO and ProAPO. We see a notable improvement in description methods by at least 2.7\% average in thirteen datasets. It further verifies the effectiveness of our progressive optimization. 

{
\renewcommand{\arraystretch}{1.1} 
% \setlength{\tabcolsep}{4pt}

\begin{table*}[tbp]
  \centering
  \resizebox{0.99\linewidth}{!}
    {
    \begin{tabular}
        {l | lllll lllll lll | l | l}

            
        \toprule
        \textbf{Module} (ViT-B/32) & \rotatebox{90}{\textbf{IN-1K}} & \rotatebox{90}{\textbf{Caltech}} & \rotatebox{90}{\textbf{Cars}} & \rotatebox{90}{\textbf{CUB}} & \rotatebox{90}{\textbf{DTD}}  & \rotatebox{90}{\textbf{ESAT}} & \rotatebox{90}{\textbf{FGVC}} & \rotatebox{90}{\textbf{FLO}} & \rotatebox{90}{\textbf{Food}}  &  \rotatebox{90}{\textbf{Pets}} & \rotatebox{90}{\textbf{Places}} & \rotatebox{90}{\textbf{SUN}} & \rotatebox{90}{\textbf{UCF}} & \rotatebox{90}{\textbf{Avg (11)}} & \rotatebox{90}{\textbf{Avg (13)}} \\
        \midrule

        % \multicolumn{16}{c}{\textit{\textbf{\ccol{ViT-B/32 Backbone}}}} \\
        % \midrule

        Vanilla CLIP & 62.1  & 91.2  & 60.4  & 51.7 & 42.9  & 43.9  & 20.2  & 66.0  & 83.2  & 86.8 & 39.9 & 62.1  & 60.9 & 61.8 & 59.3 \\ 
        
        \midrule
        DCLIP~\cite{DCLIP} & 63.3  & 92.7  & 59.4  & 52.7  & 44.1  & 38.4  & 19.4  & 66.1  & 83.9  & 88.1  & 41.2  & 65.0  & 65.8  & 62.4  & 60.0 \\
        + \textbf{ATO} & 63.8& 93.0& 60.3& 52.5& 46.5& 54.1& 21.8& 68.9& 84.0& 88.4& 41.5& 65.4& 66.0 & 64.7 & 62.0 \\
        + \textbf{ProAPO} & \textbf{64.1} & \textbf{93.2} & \textbf{60.6} & \textbf{53.6} & \textbf{48.2} & \textbf{59.4} & \textbf{22.6} & \textbf{71.5} & \textbf{84.2} & \textbf{88.7} & \textbf{42.7} & \textbf{66.0} & \textbf{68.0} & \textbf{66.0}  & \textbf{63.3}  \\
        $\Delta$ & \textcolor{retained}{+ 0.8} & \textcolor{retained}{+ 0.5} & \textcolor{retained}{+ 1.2} & \textcolor{retained}{+ 0.9} & \textcolor{retained}{+ 4.1} & \textcolor{retained}{+ 21.0} & \textcolor{retained}{+ 3.2} & \textcolor{retained}{+ 5.4} & \textcolor{retained}{+ 0.3} & \textcolor{retained}{+ 0.6} & \textcolor{retained}{+ 1.5} & \textcolor{retained}{+ 1.0} & \textcolor{retained}{+ 2.2} & \textcolor{retained}{+ 3.6} & \textcolor{retained}{+ 3.3} \\

        \midrule
        
        CuPL-base~\cite{CuPL} & 64.0  & 92.3  & 60.1  & 54.3  & 47.2  & 42.4  & 21.7  & 68.7  & 84.3  & 88.8  & 42.0  & \textbf{66.2}  & 66.7  & 63.8  & 61.4 \\
        + \textbf{ATO} & 64.2 & 93.3 & 60.9 & 54.8 & 47.8 & 53.1 & 22.2  & 70.4  & 84.9 & 89.2  & 42.3 & 65.5 & 67.4 & 65.3 & 62.8 \\
        + \textbf{ProAPO} & \textbf{64.4} & \textbf{94.2} & \textbf{61.8} & \textbf{55.9} & \textbf{48.1} & \textbf{62.1} & \textbf{23.2} & \textbf{74.4} & \textbf{85.4} & \textbf{91.0} & \textbf{42.7} & 65.6 & \textbf{68.6} & \textbf{67.2}  & \textbf{64.4} \\
        $\Delta$ & \textcolor{retained}{+ 0.4} & \textcolor{retained}{+ 1.9} & \textcolor{retained}{+ 1.7} & \textcolor{retained}{+ 1.6} & \textcolor{retained}{+ 0.9} & \textcolor{retained}{+ 19.7} & \textcolor{retained}{+ 1.5} & \textcolor{retained}{+ 5.7} & \textcolor{retained}{+ 1.1} & \textcolor{retained}{+ 2.2} & \textcolor{retained}{+ 0.7} & -0.6 & \textcolor{retained}{+ 1.9} & \textcolor{retained}{+ 3.4} & \textcolor{retained}{+ 3.0} \\

        \midrule
        CuPL-full~\cite{CuPL} & 64.4  & 92.9  & 60.7  & 53.3  & 50.6  & 50.5  & 20.9  & 69.5  & 84.2  & 87.0  & 43.1  & 66.3  & 66.4  & 64.9  & 62.3 \\
        + \textbf{ATO} & 64.5 & 93.7 & 61.0 & 54.0 & 52.0 & 58.7 & 22.1 & 70.5 & 84.6 & 89.2 & 43.2 & 66.4 & 67.5 & 66.4 &  63.6 \\
        + \textbf{ProAPO} &  \textbf{64.7} & \textbf{94.4} & \textbf{61.7} & \textbf{55.4} & \textbf{53.5} & \textbf{63.0} & \textbf{23.0} & \textbf{74.3} & \textbf{85.3} & \textbf{91.0} & \textbf{43.3} & \textbf{66.6} & \textbf{69.0} & \textbf{67.9}  & \textbf{65.0} \\
        $\Delta$ & \textcolor{retained}{+ 0.3} & \textcolor{retained}{+ 1.5} & \textcolor{retained}{+ 1.0} & \textcolor{retained}{+ 2.1} & \textcolor{retained}{+ 2.9} & \textcolor{retained}{+ 12.5} & \textcolor{retained}{+ 2.1} & \textcolor{retained}{+ 4.8} & \textcolor{retained}{+ 1.1} & \textcolor{retained}{+ 4.0} & \textcolor{retained}{+ 0.2} & \textcolor{retained}{+ 0.3} & \textcolor{retained}{+ 2.6} & \textcolor{retained}{+ 3.0} & \textcolor{retained}{+ 2.7} \\

        \midrule
        GPT4Vis~\cite{GPT4Vis} & 63.5  & 93.1  & 61.4  & 52.7  & 48.5  & 47.0  & 21.4  & 69.8  & 84.3  & 88.1  & 42.7  & 64.2  & 65.7  & 64.3  & 61.7 \\ 
        + \textbf{ATO} & 63.8 & 93.4 & 61.2 & 53.8 & 49.0 & 54.0  & 22.4 & 70.8  & 84.7  & 88.1  & 42.6 & 64.7 & 66.8 & 65.3 & 62.7 \\
        + \textbf{ProAPO} & \textbf{64.4} & \textbf{93.7} & \textbf{61.8} & \textbf{55.4} & \textbf{49.3} & \textbf{62.6} & \textbf{23.9} & \textbf{73.8} & \textbf{85.4} & \textbf{90.7} & \textbf{42.8} & \textbf{65.5} & \textbf{68.2} & \textbf{67.2}  & \textbf{64.4} \\
        $\Delta$ & \textcolor{retained}{+ 0.9} & \textcolor{retained}{+ 0.6} & \textcolor{retained}{+ 0.4} & \textcolor{retained}{+ 2.7} & \textcolor{retained}{+ 0.8} & \textcolor{retained}{+ 15.6} & \textcolor{retained}{+ 2.5} & \textcolor{retained}{+ 4.0} & \textcolor{retained}{+ 1.1} & \textcolor{retained}{+ 2.6} & \textcolor{retained}{+ 0.1} & \textcolor{retained}{+ 1.3} & \textcolor{retained}{+ 2.5} & \textcolor{retained}{+ 2.9} & \textcolor{retained}{+ 2.7} \\

        \midrule 
        AdaptCLIP~\cite{AdaptCLIP} & 63.3  & 92.7  & 59.7  & 53.6  & 47.4  & 51.3  & 20.8  & 67.2  & 84.2  & 87.6  & 41.9  & 66.1  & 66.5  & 64.2  & 61.7 \\
        + \textbf{ATO} & 63.9 & 93.2 & 60.4 & 54.2 & 47.9 & 55.5 & \textbf{22.4} & 69.1 & 84.7  & 88.8  & 42.3 & 66.3 & 67.6 & 65.4 & 62.8 \\
        + \textbf{ProAPO} & \textbf{64.4} & \textbf{93.7} & \textbf{61.8} & \textbf{55.5} & \textbf{49.6} & \textbf{61.6} & {23.3} & \textbf{73.8} & \textbf{85.4} & \textbf{91.0} & \textbf{42.6} & \textbf{66.5} & \textbf{68.6} & \textbf{67.2}  & \textbf{64.5} \\
        $\Delta$ & \textcolor{retained}{+ 1.1} & \textcolor{retained}{+ 1.0} & \textcolor{retained}{+ 2.1} & \textcolor{retained}{+ 1.9} & \textcolor{retained}{+ 2.2} & \textcolor{retained}{+ 10.3} & \textcolor{retained}{+ 2.5} & \textcolor{retained}{+ 6.6} & \textcolor{retained}{+ 1.2} & \textcolor{retained}{+ 3.4} & \textcolor{retained}{+ 0.7} & \textcolor{retained}{+ 0.4} & \textcolor{retained}{+ 2.1} & \textcolor{retained}{+ 3.0} & \textcolor{retained}{+ 2.8}  \\
        
        \bottomrule
    \end{tabular}
}
  % \vspace{-6pt}
  \caption{\textbf{Performance improvement of description methods by our ProAPO.} 
  Avg (11) and Avg (13) denote average results across 11 datasets (excluding CUB~\cite{CUB} and Places~\cite{Places365}) and all 13 datasets, respectively. $\Delta$ denotes performance gains compared to baseline.}
  \vspace{40pt}
  \label{tab:description_add_APO}
\end{table*}
}



{
\renewcommand{\arraystretch}{1.1} 
% \setlength{\tabcolsep}{4pt}
\begin{table*}[t]
  \centering
  \resizebox{0.99\linewidth}{!}
    {
    \begin{tabular}
        {l c c c c c | ccccc ccccc ccc | c | c }    
        \toprule
        \multicolumn{6}{c}{\textbf{Component}}  &   \\
        \cmidrule(lr){1-6} 
        & \texttt{Add} & \texttt{Del} & \texttt{Rep} & \texttt{Cross} & \texttt{Mut}  & \rotatebox{90}{\textbf{IN-1K}} & \rotatebox{90}{\textbf{Caltech}} & \rotatebox{90}{\textbf{Cars}} & \rotatebox{90}{\textbf{CUB}} & \rotatebox{90}{\textbf{DTD}}  & \rotatebox{90}{\textbf{ESAT}} & \rotatebox{90}{\textbf{FGVC}} & \rotatebox{90}{\textbf{FLO}} & \rotatebox{90}{\textbf{Food}}  &  \rotatebox{90}{\textbf{Pets}} & \rotatebox{90}{\textbf{Places}} & \rotatebox{90}{\textbf{SUN}} & \rotatebox{90}{\textbf{UCF}} & \rotatebox{90}{\textbf{Avg (11)}} & \rotatebox{90}{\textbf{Avg (13)}}  \\
        \midrule
       \multicolumn{5}{l}{Vanilla CLIP (ViT-B/32)} & & 62.1  & 91.2  & 60.4  & 51.7 & 42.9  & 43.9  & 20.2  & 66.0  & 83.2  & 86.8 & 39.9 & 62.1  & 60.9 & 61.8 & 59.3  \\ 
       \midrule
       \multicolumn{5}{l}{\textbf{\textit{edit-based generation}}} \\
       \texttt{a)} & \cmark & & & & & 63.8 & 93.6 & 60.0 & 54.6 & 51.8 & 59.0 & 21.8 & 74.0 & 82.2 & 86.7 & 43.0 & 65.7 & 66.8 & 66.0  & 63.3  \\ 
       \texttt{b)} & \cmark & \cmark & & & & \underline{64.6} & 94.0 & 60.9 & 55.0 & 52.6 & 59.3 & 21.8 & 72.0 & 83.2 & \underline{88.0} & 43.2 & 66.4 & 68.0 & 66.4  & 63.8 \\
       \texttt{c)} & \cmark &  & \cmark & & & 64.4 & 94.0 & 61.0 & \underline{55.2} & 52.3 & 59.7 & 22.4 & 71.9 & 84.0 & 87.7 & 43.2 & 66.4 & 67.8 & 66.5  & 63.8 \\
       \texttt{d)} & \cmark & \cmark & \cmark & & & \underline{64.6}  & 93.6 & 60.8 & 54.4 & 53.1 & 60.1 & 22.2 & \textbf{74.7} & 82.4 & 87.2 & \textbf{43.4} & 66.5 & \underline{68.6} & 66.7  & 64.0  \\ 
        \midrule
        \multicolumn{5}{l}{\textbf{\textit{evolution-based generation}}} \\
        \texttt{e)} & \cmark & \cmark & \cmark & \cmark & & \underline{64.6} & \underline{94.3} & 61.2 & 55.0 & \underline{53.2} & \underline{62.6} & \underline{22.9} & 73.9 & \underline{84.3} & \underline{88.0} & 43.1 & \textbf{66.8} & 68.5 & \underline{67.3}  & \underline{64.5} \\ 
        \texttt{f)} & \cmark & \cmark & \cmark & & \cmark &  \textbf{64.7} & \underline{94.3} & \underline{61.4} & 55.1 & 52.9 & 61.4 & 22.6 & 74.0 & 83.6 & 87.7 & \textbf{43.4} & \underline{66.7} & 68.3 & 67.1  & 64.3   \\ 
        \highlight{\texttt{g)}} & \highlight{\cmark} & \highlight{\cmark} & \highlight{\cmark} & \highlight{\cmark} & \highlight{\cmark} & \highlight{\textbf{64.7}}  & \highlight{\textbf{94.4}} & \highlight{\textbf{61.7}} & \highlight{\textbf{55.4}} & \highlight{\textbf{53.5}} & \highlight{\textbf{63.0}} & \highlight{\textbf{23.0}} & \highlight{\underline{74.3}} & \highlight{\textbf{85.3}} & \highlight{\textbf{91.0}} & \highlight{\underline{43.3}} & \highlight{{66.6}} & \highlight{\textbf{69.0}} & \highlight{\textbf{67.9}}  & \highlight{\textbf{65.0}}   \\
        \bottomrule
    \end{tabular}
}
% \vspace{-6pt}
\caption{\textbf{Ablation of edit- and evolution-based operators.}}
\vspace{50pt}
\label{supp_tab: ablation_generate}
\end{table*}
}




{
\renewcommand{\arraystretch}{1.1} 
\begin{table*}[tbp]
  \centering
  \resizebox{0.99\linewidth}{!}
    {
    \begin{tabular}
        {l | ccccc ccccc ccc | c | c | c}  
        \toprule
        {\textbf{Module} (ViT-B/32)}  & \rotatebox{90}{\textbf{IN-1K}} & \rotatebox{90}{\textbf{Caltech}} & \rotatebox{90}{\textbf{Cars}} & \rotatebox{90}{\textbf{CUB}} & \rotatebox{90}{\textbf{DTD}}  & \rotatebox{90}{\textbf{ESAT}} & \rotatebox{90}{\textbf{FGVC}} & \rotatebox{90}{\textbf{FLO}} & \rotatebox{90}{\textbf{Food}}  &  \rotatebox{90}{\textbf{Pets}} & \rotatebox{90}{\textbf{Places}} & \rotatebox{90}{\textbf{SUN}} & \rotatebox{90}{\textbf{UCF}} & \rotatebox{90}{\textbf{Avg (11)}} & \rotatebox{90}{\textbf{Avg (13)}} & \textbf{Times} \\
        \midrule
         
        \texttt{a)} w/o prompt sampling & 64.4 & 93.8 & \textbf{61.8} & \underline{55.4} & 51.8 & 60.0 & \underline{23.2} & 74.0 & 85.1 & 90.7 & \underline{43.0} & 66.0 & 69.3 & 67.3 & 64.5 & 12 min \\
        
        \texttt{b)} w/o group sampling & \textbf{64.8} & \textbf{94.5} & \underline{61.7} & \textbf{55.5} & \textbf{53.6} & \textbf{63.5} & \underline{23.2} & \underline{75.3} & \textbf{85.4} & 90.8 & \textbf{43.3} & \textbf{66.7} & \textbf{69.8} & \textbf{68.1}  & \textbf{65.2} & \textbf{306 min} \\ 
        
        \texttt{c)} w/o sampling strategies & 64.5 & 93.4 & 57.4 & 54.8 & \textbf{53.6} & \underline{63.2} & \textbf{23.4} & \textbf{76.8} & 83.8 & 86.9 & \textbf{43.3} & 66.1 & \underline{69.7} & 67.2  & 64.4 & \underline{302 min} \\

        \midrule
        
        \highlight{\textbf{ProAPO} (full model)} & \highlight{\underline{64.7}}  & \highlight{\underline{94.4}} & \highlight{\underline{61.7}} & \highlight{\underline{55.4}} & \highlight{\underline{53.5}} & \highlight{{63.0}} & \highlight{{23.0}} & \highlight{{74.3}} & \highlight{\underline{85.3}} & \highlight{\textbf{91.0}} & \highlight{\textbf{43.3}} & \highlight{\underline{66.6}} & \highlight{{69.0}} & \highlight{\underline{67.9}}  & \highlight{\underline{65.0}} & \highlight{15 min} \\
        \bottomrule
    \end{tabular}
}
% \vspace{-6pt}
  \caption{\textbf{Ablation of two sampling strategies.}}
% \vspace{-6pt}
  \label{supp_tab: ablation_sample}
\end{table*}
}


{
\renewcommand{\arraystretch}{1.1} 
\begin{table*}[tbp]
  \centering
  \resizebox{0.99\linewidth}{!}
    {
    \begin{tabular}
        {l | ccccc ccccc ccc | c | c }  
        \toprule
        {\textbf{Module} (ViT-B/32)}  & \rotatebox{90}{\textbf{IN-1K}} & \rotatebox{90}{\textbf{Caltech}} & \rotatebox{90}{\textbf{Cars}} & \rotatebox{90}{\textbf{CUB}} & \rotatebox{90}{\textbf{DTD}}  & \rotatebox{90}{\textbf{ESAT}} & \rotatebox{90}{\textbf{FGVC}} & \rotatebox{90}{\textbf{FLO}} & \rotatebox{90}{\textbf{Food}}  &  \rotatebox{90}{\textbf{Pets}} & \rotatebox{90}{\textbf{Places}} & \rotatebox{90}{\textbf{SUN}} & \rotatebox{90}{\textbf{UCF}} & \rotatebox{90}{\textbf{Avg (11)}} & \rotatebox{90}{\textbf{Avg (13)}} \\
        \midrule
         
        \texttt{a)} w/ only accuracy & 64.0 & 93.0 & 60.8 & 54.2 & 49.1 & 55.5 & 20.4 & 68.3 & 84.8 & 88.4 & 41.9 & 64.6 & 65.1 & 64.9 & 62.3 \\
        
        \texttt{b)} w/ only entropy constrain & 64.3 & 93.4 & 61.6 & 54.8 & 49.3 & 56.7 & 22.3 & 69.9 & 85.2 & 89.1 & 42.4 & 65.1 & 66.7 & 65.8 & 63.1 \\ 
        
        \midrule
        
        \highlight{\textbf{ProAPO} (full model)} & \highlight{\textbf{64.7}}  & \highlight{\textbf{94.4}} & \highlight{\textbf{61.7}} & \highlight{\textbf{55.4}} & \highlight{\textbf{53.5}} & \highlight{\textbf{63.0}} & \highlight{\textbf{23.0}} & \highlight{\textbf{74.3}} & \highlight{\textbf{85.3}} & \highlight{\textbf{91.0}} & \highlight{\textbf{43.3}} & \highlight{\textbf{66.6}} & \highlight{\textbf{69.0}} & \highlight{\textbf{67.9}}  & \highlight{\textbf{65.0}}  \\
        \bottomrule
    \end{tabular}
}
% \vspace{-6pt}
  \caption{\textbf{Ablation of different score functions.}}
% \vspace{-8pt}
  \label{supp_tab: ablation_score_function}
\end{table*}
}


% ---------------------------------------------------- % 
%               Ablation Study 的详细实验
% ---------------------------------------------------- %
% \newpage
\section{Detailed Results of Ablation Study}

% ---------------------------------------------------- % 
%             对 Edit- 和 Evolution- 操作的消融
% ---------------------------------------------------- %
\subsection{Ablation of Edit- and Evolution-based Operators}
\label{supp_sec: ablation_operator}
In~\cref{supp_tab: ablation_generate}, we show detailed ablation results of edit- and evolution-based operators. For edit-based operators, we observe that the model with add, delete, and replace operations achieves a higher result in row d). After introducing evolution-based operators, \textit{i.e.}, crossover operator to combine advantages of high-scoring candidates, and mutation operator to avoid locally optimal solutions, we see an increase in performance in rows e)-g). It confirms that evolution-based operators make the model search the optimal prompt faster with limited iterations.




% ---------------------------------------------------- % 
%                对 Sampling 策略的消融
% ---------------------------------------------------- %
\subsection{Ablation of Two Sampling Strategies}
\label{supp_sec: ablation_two_sampling}

In~\cref{supp_tab: ablation_sample}, we show detailed ablation results of two sampling strategies. Without the prompt sampling, we see a slight decrease in times while results drop in row a). It verifies the effectiveness of the prompt sampling. Without the group sampling to select salient categories for optimization, we observe a notable increase in time costs (from 15 min to 300+ min, 20 times) yet similar results in row b) and the full model. It reveals that group sampling simultaneously improves performance and efficiency.




% ---------------------------------------------------- % 
%               对 Score Function 的消融
% ---------------------------------------------------- %
\subsection{Ablation of Different Score Functions}
\label{supp_sec: effect_score_func}

In~\cref{supp_tab: ablation_score_function}, we show detailed ablation results of score functions. Accuracy obtains the worst result as the score function due to the overfitting problem. Our full model with accuracy and entropy constraints as the score function achieves the SOTA result. The score function with only accuracy or entropy constraint achieves suboptimal results, suggesting a trade-off process between them. 






% WARNING: do not forget to delete the supplementary pages from your submission 
% 
\clearpage
% \setcounter{page}{1}
% \maketitlesupplementary
\begin{center}
Supplementary Material
\end{center}

% {
%     \onecolumn
%     \centering
%     \Large
%     \textbf{\thetitle}\\
%     \vspace{0.5em}Supplementary Material \\
%     \vspace{1.0em}
% }

\section{Proof of \cref{theorem:dr}}
We require some additional regularity assumptions:
\begin{assumption} 1) The number of classes $C$ is bounded w.r.t the number of samples $N$, 2) the missingness mechanism $P(A=1|Y,\theta)$, as well as its estimated counterpart $P(A=1|Y,\theta)$, are bounded below by some constant $\epsilon > 0$, 3) the quantities $P(Y|X,\theta)$ and $P(A|Y,\theta)$ are estimated using auxiliary samples independent of samples used for the sample averaging.
\label{assumption:extra}
\end{assumption}
Assumptions 1 and 2 are natural. For the missingness mechanism, the ground truth being bounded means that there is a non-vanishing proportion of samples for every class. The boundedness of the estimate can be enforced by clipping the estimate. Assumption 3 is called sample splitting in \cite{kennedy-dr}.

For convenience we use operator $\E_N$ to denote the average of $N$ samples i.e. $\frac{1}{N}\sum_{i=1}^N$. Note that this is by itself a random variable, in contrast to $\E$ which is a fixed number.

\begin{proof}[Proof of \cref{theorem:dr}] Because $C$ is bounded (assumption \ref{assumption:extra}), we can fix a class $c$ and prove the theorem.
Let us define the influence function $\phi$, parameterized by $\theta$, as
\begin{equation}
\phi(O | \theta)(c) = P(Y=c|X,\theta) + \frac{\one(A=1)}{P(A=1|Y,\theta)} (\one(Y=c) - P(Y=c|X,\theta)) - P(Y=c)
\end{equation}
As we have done in the main text, we use $\phi(O)$ to denote the same function but all estimated quantities are replaced with their truths. In other words, we use $\phi(O)$ for $\phi(O|\theta_0)$ where $\theta_0$ is the truth, given that our model contains $\theta_0$ e.g. when the model is consistent.

Recall that:
\begin{equation}
\begin{aligned}
\Psi_{dr}(\theta)(c) &= \frac{1}{N}\sum_{i=1}^N \left\{P(Y=c|X,\theta) + \frac{\one(A=1)}{P(A=1|Y,\theta)} (\one(Y=c) - P(Y=c|X,\theta))\right\}\\
&= \E_N [\phi(O|\theta)(c)] + P(Y=c)
\end{aligned}
\end{equation}

We will show that:
\begin{equation}
\Psi_{dr}(\theta)(c) - P(Y=c) = (\E_N - \E)[\phi(O)(c)] + o_P(N^{-1/2})
\label{eq:proof-linearity}
\end{equation}
To do that, we use the following decomposition
\begin{equation}
\begin{aligned}
\Psi_{dr}(\theta)(c) - P(Y=c) &= \E_N [\phi(O|\theta)(c)] \\
&= (\E_N - \E)[\phi(O)(c)] + (\E_N - \E)[\phi(O|\theta)(c) - \phi(O)(c)] + \E[\phi(O|\theta)(c)]
% &+ (\E_n - \E)[\phi(O;\theta) - \phi(O)]\\
% &+ \E[P(Y=c|X,\theta)] - \E[P(Y=c|X)] + \E[\phi(O,\theta)]
\end{aligned}
\end{equation}
and analyze the second and third term. The third term is:
\begin{equation}
\begin{aligned}
\E[\phi(O|\theta)(c)] &= \E[P(Y=c|X,\theta)] + \E\left[\frac{\one(A=1)}{P(A=1|Y,\theta)}(\one(Y=c) - P(Y=c|X,\theta))\right]- P(Y=c) \\
&= \E\left[P(Y=c|X,\theta) + \frac{P(A=1|Y)}{P(A=1|Y,\theta)}(P(Y=c|X) - P(Y=c|X,\theta))\right] - \E[P(Y=c|X)]\\
&= \E\left[(P(Y=c|X,\theta) - P(Y=c|X)) (P(A=1|Y,\theta) -P(A=1|Y)) \frac{1}{P(A=1|Y,\theta)}\right]\\
\end{aligned}
\end{equation}
by Cauchy-Schwarz inequality:
\begin{equation}
\begin{aligned}
\E[\phi(O|\theta)(c)] &\le \frac{1}{\epsilon} \|P(A=1|Y,\theta) - P(A=1|Y)\|_2 \|P(Y=c|X,\theta) - P(Y=c|X)\|_{L_2(P)}\\
&= \frac{1}{\epsilon} o_P(N^{-1/4} N^{-1/4}) = o_P(N^{-1/2})
\end{aligned}
\end{equation}
by assumption \ref{assumption:4th-root-n} and that $P(A=1|Y,\theta) > \epsilon$ (assumption \ref{assumption:extra}). The second term can be bounded by Chebyshev inequality
% \begin{equation}
% \begin{aligned}
% \E[\E_N[\phi(O|\theta)(c) - \phi(O)(c)]] &= \E[\phi(O|\theta)(c) - \phi(O)(c)]\\
% \var[\E_N[\phi(O|\theta)(c) - \phi(O)(c)]] &= \frac{1}{N}\var[\phi(O|\theta)(c) - \phi(O)(c)] \le 
% \end{aligned}
% \end{equation}
\begin{equation}
P(|(\E_N - \E)[\phi(O|\theta)(c) - \phi(O)(c)]| \ge t) \le \frac{\var[\E_N[\phi(O|\theta)(c) - \phi(O)(c)]]}{t^2} = \frac{\var[\phi(O|\theta)(c) - \phi(O)(c)]}{Nt^2}
\end{equation}
note here that $\theta$ is independent of the samples used for $\E_N$ by assumption \ref{assumption:extra}. For any $\varepsilon > 0$, by picking $t = \frac{1}{\sqrt{N\varepsilon}}$ we get
\begin{equation}
P\left(\left|\frac{(\E_N - \E)[\phi(O|\theta)(c) - \phi(O)(c)]}{N^{-1/2}}\right| \ge \frac{1}{\sqrt{\varepsilon}}\right) \le \varepsilon \var[\phi(O|\theta)(c) - \phi(O)(c)]
\end{equation}
by the definition of $O_P$, we then get
\begin{equation}
(\E_N - \E)[\phi(O|\theta)(c) - \phi(O)(c)] = O_P(N^{-1/2}\var[\phi(O|\theta)(c) - \phi(O)(c)])
\end{equation}
Because $\phi$ is a continuous function of $P(Y|X,\theta)$ and $P(A|Y,\theta)$ (given $P(A|Y,\theta) > \epsilon$, assumption \ref{assumption:extra}), by the continuous mapping theorem and the fact that $P(Y|X,\theta)$ and $P(A|Y,\theta)$ are convergent in probability (assumption \ref{assumption:4th-root-n}), we get $\var[\phi(O|\theta)(c) - \phi(O)(c)] = o_P(1)$. This gives
\begin{equation}
(\E_N - \E)[\phi(O|\theta)(c) - \phi(O)(c)] = o_P(N^{-1/2})
\end{equation}
Therefore, we have shown that the second and third term are both $o_P(N^{-1/2})$, proving \cref{eq:proof-linearity}. As the final step, multiply both sides of this equation by $\sqrt{N}$ we get:
\begin{equation}
\sqrt{N}(\Psi_{dr}(\theta)(c) - P(Y=c)) = \sqrt{N} (\E_N - \E)[\phi(O)(c)] + o_P(1) \rightsquigarrow \mathcal{N}(0, \var[\phi(O)(c)])
\end{equation}
by the central limit theorem, and $\var[\phi(O)(c)] = \E[\phi(O)(c)^2]$ because $\E[\phi(O)(c)] = 0$.
\end{proof}

While we started with the definition of $\phi$, \cref{eq:proof-linearity} shows that $\phi$ is indeed an influence function. Now we show that $\phi$ is also the efficient influence function, by using the characterization of the model's tangent space \cite{tsiatis-missingdata}. Note that the joint probability factorizes as $P(X,A,Y) = P(X)P(Y|X)P(A|Y)$, therefore the tangent space $\mathcal{T}$ factorizes as $\mathcal{T} = \mathcal{T}_{X} \oplus \mathcal{T}_{Y|X} \oplus \mathcal{T}_{A|Y}$ where $\mathcal{T}_X = \{h(X): \E[h] = 0\}$, $\mathcal{T}_{Y|X} = \{h(X,Y): \E[h|X] = 0\}$, $\mathcal{T}_{A|Y} = \{h(A,Y): \E[h|Y] = 0\}$, and the 3 subspaces are pairwise orthogonal. All influence functions are orthogonal to the tangent space, but the influence function that is also in the tangent space has the smallest variance and is called the efficient influence function. As $\phi$ is already an influence function, we need only show that $\phi$ is in $\mathcal{T}$. We write $\phi$ as
\begin{equation}
\phi(O)(c) = (P(Y=c|X) - P(Y=c)) + \left[\frac{\one(A=1)}{P(A=1|Y)} - 1\right](\one(Y=c) - P(Y=c|X)) + (\one(Y=c) - P(Y=c|X))
\end{equation}
and note that the first, second and third term are in $\mathcal{T}_X$, $\mathcal{T}_{A|Y}$ and $\mathcal{T}_{Y|X}$ respectively. Therefore, $\phi$ is indeed in $\mathcal{T}$. The efficient influence function has the smallest variance of all influence function, and therefore our estimator being asymptotically linear in $\phi$ (\cref{eq:proof-linearity}) has the smallest mean squared error in a local asymptotic minimax sense \cite{kennedy-dr, asymptoticstatistics}

\section{Further background and related work}
\paragraph{Discussion on semi-supervised EM.}
It appears that semi-supervised EM was first used for parameter estimation when the missingness mechanism is non-ignorable in \cite{ibrahim1996parameter}, but has not been used for label shift estimation.
Perhaps this is because the semi-supervised situation where additional unlabeled data is available during training is rarer than the test-time adaptation case. EM is well suited to take advantage of the extra unlabeled data to improve the classifier under very scarce and long-tailed labeled data. While the connection between pseudo-labeling and EM has been explored before \cite{entropyminimization}, the situation with label shift has not until recently \cite{simpro}. Here the application of EM is much more interesting, because other than simply giving pseudo-labeling a rigorous formulation, EM also estimates the missingness mechanism (equivalently the label distribution shift), which is important for shift correction and thus high-quality pseudo-labels \cite{acr}. The application of confidence thresholding can be seen as a sparse variant of EM \cite{neal1998view}.

\paragraph{The doubly-robust risk.} 
\label{subsec:dr-risk}
A technique that also derives from the theory of semi-parametric efficiency is orthogonal statistical learning \citep{foster2023orthogonal}. The idea is to minimize the doubly-robust risk:
\label{subsec:method-dr-risk}
\begin{equation}
\label{eq:dr-risk}
\mathcal{R}(\theta_2) = \frac{1}{N} \sum_{i=1}^N \Bigg[ l(x_i, \hat y_i|\theta_2) + \frac{\one(a_i=1)}{P(A=a_i|Y=y_i, \theta_1)} (l(x_i, y_i | \theta_2) - l(x_i, \hat y_i | \theta_2))\Bigg]
\end{equation}
where $l(x,y|\theta) = -\sum_{c=1}^C [y]_c \log P(Y=c|X=x,\theta)$ is the negative cross-entropy. 
The notation $[y]_c$ means that we are using the $c$-entry in a C-dimension probability vector $y$. 
Thus, $y_i$ denotes the one-hot label of observation $i$, while $\hat y_i$ denotes the pseudo-label, which can be one-hot or all-zero. 
Finally, we use $\theta_1$ to denote that $P(a|y,\theta_1)$ is an estimation from a previous stage, but it can be estimated with $\theta_2$ as well. 
The risk $\mathcal{R}(\theta_2)$ can be used as a training loss in a straightforward fashion. 
Similar to the doubly robust estimation of $P(Y)$, the doubly robust risk provides approximately unbiased estimation of the risk. 
This property has been used in \citep{arelabelsinformative, onnonrandommissinglabels, drst} also in the semi-supervised learning setting.
More broadly, it is at the heart of one of the core techniques in heterogenous treatment effect estimation in causal estimation \cite{kennedy2023towards, foster2023orthogonal, wager2018estimation}. 
The focus here is not the estimation of $\mathcal{R}(\theta_2)$ per se, but the quality of the learned model \cite{foster2023orthogonal}.
By using the doubly-robust risk, we can achieve an optimality result similar in spirit to our theorem \cref{theorem:dr}, but for the generalization error.
While this is appealing, in practice there are 2 problems with this approach. First, the inverse probability weight $P(A=a_i|Y=y_i,\theta_1)$ can be very large if the class ratio is highly unlabeled, making training unstable \cite{kallus2020deepmatch, pham2023stable}. 
This problem exists for our estimation as well. However, it is much easier to control for estimation than for training because of the iterative nature of model update. Secondly, we can further write $\mathcal{R}$ as:
\begin{equation}
\mathcal{R}(\theta_2) = \frac{1}{N}\sum_{i=1}^N l\left(x_i, \hat y_i + \frac{\one(a_i=1)}{P(A=a_i|Y=y_i,\theta_1)} (y_i - \hat y_i)\Bigg\vert\theta_2\right)
\end{equation}
which is a cross-entropy loss with new meta-pseudo-labels. However, these labels are not meant to be learned exactly, and furthermore they can be negative. Thus, theoretical works have to put stringent assumptions on the models. In \cref{subsec:ablation-1}, we show that experimentally that the instability problem makes doubly-robust risk performance worse than our 2-stage approach.

\section{Training and hyperparameter settings.}
\label{subsec:training-setting}
For neural network training, we follow the implementation and hyperparameter settings of \cite{simpro}. In particular, we adapt the core code of SimPro for Supervised, MLE and EM. For MLE, we update $P(A|Y)$ using the Adam optimizer with learning rate 1e-3, while for EM we use a momentum update similar to SimPro's update of $P(Y|A)$ because it has a a closed-form solution at each mini-batch. We use Wide ResNet-28-2 on all methods and all datasets in this section, including Imagenet-127, because we are motivated by the fact that stage-1's goal is not classification accuracy but the estimation of a finite-dimensional parameter. When using Wide ResNet-28-2 for Imagenet-127, we use the hyperparameters of CIFAR-100, except we lower the batch size of unlabeled data to 2 times that of labeled data instead of 8 for memory reason. We do not perform additional hyperparameter tuning. All experiments can be performed on 1 A6000 RTX GPU, and are run 3 times. We report the total variation distance between the estimated and the ground truth unlabeled class distribution, similar to its usage in Theorem 3.1 of \cite{lsc}, and the top-1 classification accuracy.

In the second stage of our algorithm, we freeze our estimation and plug it in SimPro and BOAT.
We keep exactly the same hyperparameter settings that SimPro and BOAT use. In particular, for Imagenet-127, we now use ResNet-50 and run each experiment once.
In SimPro, we set the unlabeled class distribution $P(Y|A=0)$ at the E-step;  however, we still keep a running estimate of the class distribution $P(Y)$ in the logit adjustment loss \cref{eq:simpro-la-loss}. While it is possible to use the first stage estimate in the logit adjustment loss, we observe that doing so results in lower accuracy than using the the running average. This is conceptually consistent with the role of the running average - serving not as an accurate estimate of $P(Y)$ but to make the classifier's class distribution uniform through the logit adjustment loss, which is good for the test set. Similarly, in BOAT, we only replace $\Delta_c = \log P(Y|A=1) - \log P(Y|A=0)$ in equation (4) of \cite{boat}, which is adjusting a classifier's predictions from the labeled to the unlabeled class distribution, with our SimPro + DR estimate instead of their on-the-fly estimate. 


% \section{Additional experiments}
% % \begin{table*}[t]
\centering
\caption{Total Variation Distance on CIFAR-10-LT ($N_l = 500$, $M_l = 4000$) with different class imbalance ratios $\gamma_l$ and $\gamma_u$ under five different unlabeled class distributions.}
\label{tab:cifar10-tv}
\resizebox{\textwidth}{!}{
\begin{tabular}{lccccccccccc}
\toprule
& & \multicolumn{2}{c}{consistent} & \multicolumn{2}{c}{uniform} & \multicolumn{2}{c}{reversed} & \multicolumn{2}{c}{middle} & \multicolumn{2}{c}{head-tail} \\
\cmidrule(lr){3-4} \cmidrule(lr){5-6} \cmidrule(lr){7-8} \cmidrule(lr){9-10} \cmidrule(lr){11-12}
& & $\gamma_l = 150$ & $\gamma_l = 100$ & $\gamma_l = 150$ & $\gamma_l = 100$ & $\gamma_l = 150$ & $\gamma_l = 100$ & $\gamma_l = 150$ & $\gamma_l = 100$ & $\gamma_l = 150$ & $\gamma_l = 100$ \\
Model & Estimator & $\gamma_u = 150$ & $\gamma_u = 100$ & $\gamma_u = 1$ & $\gamma_u = 1$ & $\gamma_u = 1/150$ & $\gamma_u = 1/100$ & $\gamma_u = 150$ & $\gamma_u = 100$ & $\gamma_u = 150$ & $\gamma_u = 100$ \\
\midrule
Supervised & MLLS & 0.269 ± 0.252 & 0.038 ± 0.006 & 0.251 ± 0.046 & 0.255 ± 0.060 & 0.429 ± 0.028 & 0.493 ± 0.050 & 0.333 ± 0.042 & 0.320 ± 0.009 & 0.457 ± 0.034 & 0.444 ± 0.043 \\
Supervised & RLLS & 0.043 ± 0.001 & 0.044 ± 0.010 & 0.348 ± 0.034 & 0.305 ± 0.068 & 0.769 ± 0.016 & 0.678 ± 0.028 & 0.430 ± 0.008 & 0.368 ± 0.013 & 0.539 ± 0.018 & 0.503 ± 0.020 \\
\midrule
MLE & IPW & 0.027 ± 0.001 & 0.027 ± 0.000 & 0.319 ± 0.072 & 0.243 ± 0.010 & 0.674 ± 0.020 & 0.646 ± 0.041 & 0.438 ± 0.020 & 0.454 ± 0.026 & 0.547 ± 0.049 & 0.491 ± 0.059 \\
MLE & OR & 0.045 ± 0.004 & 0.042 ± 0.000 & 0.215 ± 0.026 & 0.203 ± 0.032 & 0.433 ± 0.017 & 0.395 ± 0.033 & 0.193 ± 0.006 & 0.209 ± 0.037 & 0.307 ± 0.147 & 0.249 ± 0.130 \\
MLE & DR & 0.090 ± 0.002 & 0.079 ± 0.000 & 0.407 ± 0.027 & 0.360 ± 0.007 & 0.425 ± 0.007 & 0.421 ± 0.029 & 0.256 ± 0.001 & 0.286 ± 0.031 & 0.435 ± 0.136 & 0.362 ± 0.122 \\
\midrule
EM & IPW & 0.035 ± 0.002 & 0.040 ± 0.001 & 0.021 ± 0.001 & 0.029 ± 0.015 & 0.303 ± 0.187 & 0.091 ± 0.010 & 0.119 ± 0.011 & 0.105 ± 0.022 & 0.104 ± 0.026 & 0.104 ± 0.051 \\
EM & OR & 0.037 ± 0.003 & 0.042 ± 0.002 & 0.016 ± 0.001 & 0.024 ± 0.012 & 0.269 ± 0.183 & 0.090 ± 0.008 & 0.122 ± 0.012 & 0.103 ± 0.022 & 0.072 ± 0.012 & 0.073 ± 0.024 \\
EM & DR & 0.034 ± 0.004 & 0.037 ± 0.001 & 0.014 ± 0.001 & 0.027 ± 0.020 & 0.264 ± 0.191 & 0.092 ± 0.005 & 0.111 ± 0.019 & 0.097 ± 0.026 & 0.077 ± 0.016 & 0.073 ± 0.028 \\
\midrule
SimPro & IPW & 0.070 ± 0.011 & 0.058 ± 0.000 & 0.046 ± 0.001 & 0.049 ± 0.005 & 0.254 ± 0.074 & 0.223 ± 0.098 & 0.097 ± 0.025 & 0.067 ± 0.002 & 0.105 ± 0.066 & 0.110 ± 0.079 \\
SimPro & OR & 0.071 ± 0.012 & 0.058 ± 0.000 & 0.045 ± 0.001 & 0.049 ± 0.006 & 0.040 ± 0.003 & 0.059 ± 0.017 & 0.074 ± 0.006 & 0.075 ± 0.002 & 0.033 ± 0.003 & 0.033 ± 0.003 \\
SimPro & DR & 0.017 ± 0.004 & 0.026 ± 0.001 & 0.019 ± 0.002 & 0.018 ± 0.003 & 0.039 ± 0.003 & 0.058 ± 0.025 & 0.091 ± 0.007 & 0.031 ± 0.001 & 0.015 ± 0.003 & 0.019 ± 0.007 \\
\bottomrule
\end{tabular}
}
\end{table*}
% 

\begin{table*}[t]
\centering
\caption{Total Variation Distance on CIFAR-100-LT ($N_l = 50$, $M_l = 400$) with different class imbalance ratios $\gamma_l$ and $\gamma_u$ under five different unlabeled class distributions.}
\label{tab:cifar100-tv}
\resizebox{\textwidth}{!}{
\begin{tabular}{lccccccccccc}
\toprule
& & \multicolumn{2}{c}{consistent} & \multicolumn{2}{c}{uniform} & \multicolumn{2}{c}{reversed} & \multicolumn{2}{c}{middle} & \multicolumn{2}{c}{head-tail} \\
\cmidrule(lr){3-4} \cmidrule(lr){5-6} \cmidrule(lr){7-8} \cmidrule(lr){9-10} \cmidrule(lr){11-12}
& & $\gamma_l = 20$ & $\gamma_l = 10$ & $\gamma_l = 20$ & $\gamma_l = 10$ & $\gamma_l = 20$ & $\gamma_l = 10$ & $\gamma_l = 20$ & $\gamma_l = 10$ & $\gamma_l = 20$ & $\gamma_l = 10$ \\
Model & Estimator & $\gamma_u = 20$ & $\gamma_u = 10$ & $\gamma_u = 1$ & $\gamma_u = 1$ & $\gamma_u = 1/20$ & $\gamma_u = 1/10$ & $\gamma_u = 20$ & $\gamma_u = 10$ & $\gamma_u = 20$ & $\gamma_u = 10$ \\
\midrule
Supervised & MLLS & 0.707 ± 0.016 & 0.313 ± 0.100 & 0.445 ± 0.172 & 0.309 ± 0.119 & 0.383 ± 0.075 & 0.397 ± 0.006 & 0.570 ± 0.001 & 0.373 ± 0.107 & 0.543 ± 0.009 & 0.231 ± 0.057 \\
Supervised & RLLS & 0.520 ± 0.007 & 0.133 ± 0.003 & 0.337 ± 0.125 & 0.253 ± 0.082 & 0.424 ± 0.060 & 0.463 ± 0.003 & 0.454 ± 0.021 & 0.306 ± 0.074 & 0.460 ± 0.028 & 0.241 ± 0.040 \\
\midrule
MLE & IPW & 0.075 ± 0.000 & 0.071 ± 0.001 & 0.229 ± 0.001 & 0.167 ± 0.002 & 0.565 ± 0.005 & 0.443 ± 0.007 & 0.415 ± 0.000 & 0.311 ± 0.005 & 0.343 ± 0.000 & 0.280 ± 0.001 \\
MLE & OR & 0.065 ± 0.002 & 0.061 ± 0.001 & 0.200 ± 0.007 & 0.143 ± 0.001 & 0.526 ± 0.011 & 0.399 ± 0.023 & 0.360 ± 0.003 & 0.256 ± 0.012 & 0.328 ± 0.003 & 0.266 ± 0.005 \\
MLE & DR & 0.149 ± 0.019 & 0.145 ± 0.010 & 0.243 ± 0.004 & 0.214 ± 0.019 & 0.568 ± 0.005 & 0.464 ± 0.014 & 0.403 ± 0.014 & 0.309 ± 0.012 & 0.365 ± 0.007 & 0.320 ± 0.004 \\
\midrule
EM & IPW & 0.097 ± 0.008 & 0.092 ± 0.004 & 0.239 ± 0.007 & 0.179 ± 0.003 & 0.478 ± 0.012 & 0.329 ± 0.020 & 0.262 ± 0.016 & 0.202 ± 0.003 & 0.312 ± 0.002 & 0.227 ± 0.001 \\
EM & OR & 0.121 ± 0.007 & 0.108 ± 0.005 & 0.261 ± 0.007 & 0.189 ± 0.004 & 0.489 ± 0.013 & 0.335 ± 0.020 & 0.274 ± 0.016 & 0.211 ± 0.004 & 0.336 ± 0.003 & 0.235 ± 0.001 \\
EM & DR & 0.125 ± 0.005 & 0.111 ± 0.004 & 0.269 ± 0.007 & 0.194 ± 0.005 & 0.497 ± 0.010 & 0.336 ± 0.024 & 0.281 ± 0.019 & 0.219 ± 0.008 & 0.336 ± 0.007 & 0.233 ± 0.004 \\
\midrule
SimPro & IPW & 0.125 ± 0.001 & 0.100 ± 0.005 & 0.166 ± 0.007 & 0.141 ± 0.009 & 0.353 ± 0.023 & 0.261 ± 0.008 & 0.202 ± 0.003 & 0.158 ± 0.005 & 0.277 ± 0.009 & 0.197 ± 0.003 \\
SimPro & OR & 0.133 ± 0.005 & 0.100 ± 0.004 & 0.160 ± 0.007 & 0.138 ± 0.010 & 0.322 ± 0.014 & 0.253 ± 0.008 & 0.202 ± 0.003 & 0.156 ± 0.005 & 0.269 ± 0.006 & 0.191 ± 0.004 \\
SimPro & DR & 0.122 ± 0.003 & 0.106 ± 0.006 & 0.188 ± 0.009 & 0.149 ± 0.006 & 0.343 ± 0.023 & 0.257 ± 0.007 & 0.219 ± 0.010 & 0.172 ± 0.002 & 0.279 ± 0.007 & 0.198 ± 0.004 \\
\bottomrule
\end{tabular}
}
\end{table*}

\end{document}
