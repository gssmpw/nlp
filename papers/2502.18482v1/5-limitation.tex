\section*{Limitations}

% \lyc{please add}
Although MixLLM presents strong performance in the experiments, some limitations are listed as follows.
(1) The training process assumes access to refined feedback, including response quality and cost, which may not always be available in real world. Training-free methods could help, such as scaling laws~\cite{ruan2024observational}.
% (2) MixLLM may face challenges when routing queries from brand-new domains, commonly referred to as the out-of-domain (OOD) problem, refers to Section~\ref{\label{sec:ood_study}}.
(2) MixLLM may face challenges when routing queries from brand-new domains, commonly referred to as the out-of-domain (OOD) problem (see Section~\ref{sec:ood_study} for further details).
(3) MixLLM faces challenges in practical scenarios requiring the selection of a single definitive answer from multiple LLM outputs, as discussed in Section~\ref{policy_study}.
(4) While MixLLM considers hardware limitation through the latency constraint, more detailed dispatch strategies considering system information could further improve its practicality.
(5) More complex routing tasks remain unexplored, such as hierarchical routing. This could involve first routing a query to a relevant domain, and then selecting the most suitable LLM within that domain.
(6) MixLLM’s performance needs to be tested in real-world applications to ensure its robustness beyond idealized environments.

