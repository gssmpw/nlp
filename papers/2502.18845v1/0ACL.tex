% This must be in the first 5 lines to tell arXiv to use pdfLaTeX, which is strongly recommended.
\pdfoutput=1
% In particular, the hyperref package requires pdfLaTeX in order to break URLs across lines.

\documentclass[11pt]{article}

% Change "review" to "final" to generate the final (sometimes called camera-ready) version.
% Change to "preprint" to generate a non-anonymous version with page numbers.
\usepackage[preprint]{acl}

% Standard package includes
\usepackage{times}
\usepackage{latexsym}

% For proper rendering and hyphenation of words containing Latin characters (including in bib files)
\usepackage[T1]{fontenc}
% For Vietnamese characters
% \usepackage[T5]{fontenc}
% See https://www.latex-project.org/help/documentation/encguide.pdf for other character sets

% This assumes your files are encoded as UTF8
\usepackage[utf8]{inputenc}

% This is not strictly necessary, and may be commented out,
% but it will improve the layout of the manuscript,
% and will typically save some space.
\usepackage{microtype}

% This is also not strictly necessary, and may be commented out.
% However, it will improve the aesthetics of text in
% the typewriter font.
\usepackage{inconsolata}

%Including images in your LaTeX document requires adding
%additional package(s)
\usepackage{graphicx}

\usepackage{amsmath}
\usepackage{booktabs}
\usepackage{amsfonts}
\usepackage{enumerate}
\usepackage{lipsum}
\usepackage{enumitem}
\usepackage{multirow}
\usepackage{pifont}
\usepackage{hyperref}

\newcommand{\zxy}[1]{{\color{blue} [#1 – Xy]}}
\newcommand{\zc}[0]{{\color{blue} [cite]}\xspace}
 % for Prof. Zhao
\newcommand{\w}[1]{{\color{orange}[#1]}} % Yejing
\newcommand{\fzc}[1]{{\color{red} #1}} % Zichuan
\newcommand{\swt}[1]{{\color{green} #1}} % Wentao


% \usepackage{xcolor}
% \pagecolor[rgb]{0,0,0} %black
% \color[rgb]{0.7,0.7,0.7} %grey

% If the title and author information does not fit in the area allocated, uncomment the following
%
%\setlength\titlebox{<dim>}
%
% and set <dim> to something 5cm or larger.

\title{Sliding Window Attention Training for Efficient Large Language Models}
% Long-Context Handling

% Author information can be set in various styles:
% For several authors from the same institution:
% \author{Author 1 \and ... \and Author n \\
%         Address line \\ ... \\ Address line}
% if the names do not fit well on one line use
%         Author 1 \\ {\bf Author 2} \\ ... \\ {\bf Author n} \\
% For authors from different institutions:
% \author{Author 1 \\ Address line \\  ... \\ Address line
%         \And  ... \And
%         Author n \\ Address line \\ ... \\ Address line}
% To start a separate ``row'' of authors use \AND, as in
% \author{Author 1 \\ Address line \\  ... \\ Address line
%         \AND
%         Author 2 \\ Address line \\ ... \\ Address line \And
%         Author 3 \\ Address line \\ ... \\ Address line}

% \author{First Author \\
%   Affiliation / Address line 1 \\
%   Affiliation / Address line 2 \\
%   Affiliation / Address line 3 \\
%   \texttt{email@domain} \\\And
%   Second Author \\
%   Affiliation / Address line 1 \\
%   Affiliation / Address line 2 \\
%   Affiliation / Address line 3 \\
%   \texttt{email@domain} \\}

% \author{
%  \textbf{Zichuan Fu\textsuperscript{1}},
%  \textbf{Wentao Song\textsuperscript{2}},
%  \textbf{Yejing Wang\textsuperscript{1}},
%  \textbf{Xian Wu\textsuperscript{3,†}},
% \\
%  \textbf{Yefeng Zheng\textsuperscript{3,4}},
%  \textbf{Yingying Zhang\textsuperscript{5}},
%  \textbf{Derong Xu\textsuperscript{1,6}},
%  \textbf{Xuetao Wei\textsuperscript{7}},
% \\
%  \textbf{Tong Xu\textsuperscript{6,†}},
%  \textbf{Xiangyu Zhao\textsuperscript{1,†}},
%  \textbf{Ziheng Zhang\textsuperscript{3}},
%  \textbf{Zhihong Zhu\textsuperscript{8}},
% \\
%  \textbf{Zhenxi Lin\textsuperscript{3}},
%  \textbf{Qidong Liu\textsuperscript{1,2}},
%  \textbf{Wanyu Wang\textsuperscript{1}},
%  \textbf{Yuyang Ye\textsuperscript{1}},
%  \textbf{Enhong Chen\textsuperscript{6}}
% \\
%  \textsuperscript{1}City University of Hong Kong, Hong Kong, China,
%  \textsuperscript{2}Xi'an Jiaotong University, Xi'an, China,
% \\
%  \textsuperscript{3}Jarvis Research Center, Tencent YouTu Lab, Shenzhen, China,
% \\
%  \textsuperscript{4}Medical Artificial Intelligence Lab, Westlake University, Shenzhen, China,
% \\
%  \textsuperscript{5}Tencent, Shenzhen, China,
%  \textsuperscript{6}University of Science and Technology of China, Hefei, China,
% \\
%  \textsuperscript{7}Southern University of Science and Technology, Shenzhen, China,
%  \textsuperscript{8}Peking University, Beijing, China
% \\
%  \small{
%    \textbf{Correspondence:} \href{mailto:xy.zhao@cityu.edu.hk}{xy.zhao@cityu.edu.hk}
%  }
% }

% \author{
%  Zichuan Fu\textsuperscript{1}, Wentao Song\textsuperscript{2}, Yejing Wang\textsuperscript{1}, Xian Wu\textsuperscript{3,†},
%  Yefeng Zheng\textsuperscript{3,4}, Yingying Zhang\textsuperscript{5}, 
%  Derong Xu\textsuperscript{1,6}, Xuetao Wei\textsuperscript{7},
%  Tong Xu\textsuperscript{6,†}, Xiangyu Zhao\textsuperscript{1,†}, Ziheng Zhang\textsuperscript{3}, Zhihong Zhu\textsuperscript{8},
%  Zhenxi Lin\textsuperscript{3}, Qidong Liu\textsuperscript{1,2}, Wanyu Wang\textsuperscript{1}, Yuyang Ye\textsuperscript{1},
%  Enhong Chen\textsuperscript{6}
% \\
% \\
%  \textsuperscript{1}City University of Hong Kong, Hong Kong, China, \textsuperscript{2}Xi'an Jiaotong University, Xi'an, China,
%  \textsuperscript{3}Jarvis Research Center, Tencent YouTu Lab, Shenzhen, China, \textsuperscript{4}Medical Artificial Intelligence Lab, Westlake University, Shenzhen, China,
%  \textsuperscript{5}Tencent, Shenzhen, China, \textsuperscript{6}University of Science and Technology of China, Hefei, China,
%  \textsuperscript{7}Southern University of Science and Technology, Shenzhen, China, \textsuperscript{8}Peking University, Beijing, China
% \\
%  Correspondence: xy.zhao@cityu.edu.hk
% }

\author{
  \textbf{Zichuan Fu\textsuperscript{1,\thanks{Work was conducted during the internship of Zichuan Fu at Tencent YouTu Lab.}}},
  \textbf{Wentao Song\textsuperscript{2}},
  \textbf{Yejing Wang\textsuperscript{1}},
  \textbf{Xian Wu\textsuperscript{3}},
  \textbf{Yefeng Zheng\textsuperscript{3,4}},\\
  \textbf{Yingying Zhang\textsuperscript{3}},
  \textbf{Derong Xu\textsuperscript{1,5}},
  \textbf{Xuetao Wei\textsuperscript{6}},
  \textbf{Tong Xu\textsuperscript{5}},
  \textbf{Xiangyu Zhao\textsuperscript{1,\thanks{Corresponding author.}}},
\\
\\
  \textsuperscript{1} City University of Hong Kong
  \textsuperscript{2} Xi'an Jiaotong University \\
  \textsuperscript{3} Jarvis Research Center, Tencent YouTu Lab 
  \textsuperscript{4} Westlake University \\
  \textsuperscript{5} University of Science and Technology of China \\
  \textsuperscript{6} Southern University of Science and Technology
\\
  \small{
   \href{mailto:zc.fu@my.cityu.edu.hk}{zc.fu@my.cityu.edu.hk},
   \href{mailto:xy.zhao@cityu.edu.hk}{xy.zhao@cityu.edu.hk}
  }
}



%\author{
%  \textbf{First Author\textsuperscript{1}},
%  \textbf{Second Author\textsuperscript{1,2}},
%  \textbf{Third T. Author\textsuperscript{1}},
%  \textbf{Fourth Author\textsuperscript{1}},
%\\
%  \textbf{Fifth Author\textsuperscript{1,2}},
%  \textbf{Sixth Author\textsuperscript{1}},
%  \textbf{Seventh Author\textsuperscript{1}},
%  \textbf{Eighth Author \textsuperscript{1,2,3,4}},
%\\
%  \textbf{Ninth Author\textsuperscript{1}},
%  \textbf{Tenth Author\textsuperscript{1}},
%  \textbf{Eleventh E. Author\textsuperscript{1,2,3,4,5}},
%  \textbf{Twelfth Author\textsuperscript{1}},
%\\
%  \textbf{Thirteenth Author\textsuperscript{3}},
%  \textbf{Fourteenth F. Author\textsuperscript{2,4}},
%  \textbf{Fifteenth Author\textsuperscript{1}},
%  \textbf{Sixteenth Author\textsuperscript{1}},
%\\
%  \textbf{Seventeenth S. Author\textsuperscript{4,5}},
%  \textbf{Eighteenth Author\textsuperscript{3,4}},
%  \textbf{Nineteenth N. Author\textsuperscript{2,5}},
%  \textbf{Twentieth Author\textsuperscript{1}}
%\\
%\\
%  \textsuperscript{1}Affiliation 1,
%  \textsuperscript{2}Affiliation 2,
%  \textsuperscript{3}Affiliation 3,
%  \textsuperscript{4}Affiliation 4,
%  \textsuperscript{5}Affiliation 5
%\\
%  \small{
%    \textbf{Correspondence:} \href{mailto:email@domain}{email@domain}
%  }
%}

\begin{document}


\maketitle
\begin{abstract}
Recent advances in transformer-based Large Language Models (LLMs) have demonstrated remarkable capabilities across various tasks. However, their quadratic computational complexity concerning sequence length remains a significant bottleneck for processing long documents. As a result, many efforts like sparse attention and state space models have been proposed to improve the efficiency of LLMs over long sequences. 
While these approaches achieve efficiency, they often require complex architectures and parallel training techniques.
This calls for a simple yet efficient model that preserves the fundamental Transformer architecture. 
To this end, we introduce \textbf{SWAT}, which enables efficient long-context handling via \textbf{S}liding \textbf{W}indow \textbf{A}ttention \textbf{T}raining. 
Specifically, SWAT replaces softmax with the sigmoid function for efficient information compression and retention. Then it utilizes balanced ALiBi and Rotary Position Embedding to stabilize training process. 
During inference, SWAT maintains linear computational complexity through sliding window attention while preserving model performance, achieving state-of-the-art (SOTA) results on eight commonsense reasoning benchmarks compared to mainstream linear recurrent architectures.
Code is available at \href{https://anonymous.4open.science/r/SWAT-attention}{this link}.

% and utilize a balanced ALiBi and Rotary Position Embedding
% This paper first attributes the inefficiency of Transformers to the attention sink phenomenon resulting from the high variance of softmax operation. Then, we replace softmax with the sigmoid function and utilize a balanced ALiBi and Rotary Position Embedding for efficient information compression and retention. 
    %This motivates us to design a simple yet efficient model that preserves the fundamental transformer architecture.
    %This paper introduces \textbf{SWAT}, which enables efficient long-context handling via \textbf{S}liding \textbf{W}indow \textbf{A}ttention \textbf{T}raining. 
    % Through experimental analysis, we reveal that Transformers suffer from attention sink phenomenon due to the variance brought by softmax. 
   
    % To address these issues, we replace softmax with the sigmoid function and utilize a balanced ALiBi and Rotary Position Embedding for efficient information compression and retention.
    
    % Our approach enables efficient long-context processing without requiring complex architectural changes or inference-time adjustments, providing a practical solution for developing more efficient LLMs. 

\end{abstract}

With the growing demand for accelerating Large Language Model (LLM) inference to enable efficient real-time human-LLM interactions, Speculative Decoding~\cite{BlockWise, SpecDecoding, SpecSampling} has gained attention for providing a fully algorithmic solution with minimal drawbacks.
While autoregressive decoding generates token by token, the decoding step in this method is divided into two substeps: \textit{drafting}, where likely tokens are sampled externally from a less complex model, and \textit{verifying}, where the sampled tokens are accepted or rejected by comparing with the LLM’s actual output.
By allowing the LLM to generate multiple accepted tokens in the verification phase, speculative decoding improves both the throughput and the latency of the LLM inference. 
Crucially, the efficiency of this approach depends on how draft tokens are generated, as performance gains hinge on the acceptance rate of these tokens~\cite{SpecSampling}.
Therefore, subsequent approaches to speculative decoding have focused on developing drafting strategies that sample tokens closely aligned with the target model.

\section{Bellman Error Centering}

Centering operator $\mathcal{C}$ for a variable $x(s)$ is defined as follows:
\begin{equation}
\mathcal{C}x(s)\dot{=} x(s)-\mathbb{E}[x(s)]=x(s)-\sum_s{d_{s}x(s)},
\end{equation} 
where $d_s$ is the probability of $s$.
In vector form,
\begin{equation}
\begin{split}
\mathcal{C}\bm{x} &= \bm{x}-\mathbb{E}[x]\bm{1}\\
&=\bm{x}-\bm{x}^{\top}\bm{d}\bm{1},
\end{split}
\end{equation} 
where $\bm{1}$ is an all-ones vector.
For any vector $\bm{x}$ and $\bm{y}$ with a same distribution $\bm{d}$,
we have
\begin{equation}
\begin{split}
\mathcal{C}(\bm{x}+\bm{y})&=(\bm{x}+\bm{y})-(\bm{x}+\bm{y})^{\top}\bm{d}\bm{1}\\
&=\bm{x}-\bm{x}^{\top}\bm{d}\bm{1}+\bm{y}-\bm{y}^{\top}\bm{d}\bm{1}\\
&=\mathcal{C}\bm{x}+\mathcal{C}\bm{y}.
\end{split}
\end{equation}
\subsection{Revisit Reward Centering}


The update (\ref{src3}) is an unbiased estimate of the average reward
with  appropriate learning rate $\beta_t$ conditions.
\begin{equation}
\bar{r}_{t}\approx \lim_{n\rightarrow\infty}\frac{1}{n}\sum_{t=1}^n\mathbb{E}_{\pi}[r_t].
\end{equation}
That is 
\begin{equation}
r_t-\bar{r}_{t}\approx r_t-\lim_{n\rightarrow\infty}\frac{1}{n}\sum_{t=1}^n\mathbb{E}_{\pi}[r_t]= \mathcal{C}r_t.
\end{equation}
Then, the simple reward centering can be rewrited as:
\begin{equation}
V_{t+1}(s_t)=V_{t}(s_t)+\alpha_t [\mathcal{C}r_{t+1}+\gamma V_{t}(s_{t+1})-V_t(s_t)].
\end{equation}
Therefore, the simple reward centering is, in a strict sense, reward centering.

By definition of $\bar{\delta}_t=\delta_t-\bar{r}_{t}$,
let rewrite the update rule of the value-based reward centering as follows:
\begin{equation}
V_{t+1}(s_t)=V_{t}(s_t)+\alpha_t \rho_t (\delta_t-\bar{r}_{t}),
\end{equation}
where $\bar{r}_{t}$ is updated as:
\begin{equation}
\bar{r}_{t+1}=\bar{r}_{t}+\beta_t \rho_t(\delta_t-\bar{r}_{t}).
\label{vrc3}
\end{equation}
The update (\ref{vrc3}) is an unbiased estimate of the TD error
with  appropriate learning rate $\beta_t$ conditions.
\begin{equation}
\bar{r}_{t}\approx \mathbb{E}_{\pi}[\delta_t].
\end{equation}
That is 
\begin{equation}
\delta_t-\bar{r}_{t}\approx \mathcal{C}\delta_t.
\end{equation}
Then, the value-based reward centering can be rewrited as:
\begin{equation}
V_{t+1}(s_t)=V_{t}(s_t)+\alpha_t \rho_t \mathcal{C}\delta_t.
\label{tdcentering}
\end{equation}
Therefore, the value-based reward centering is no more,
 in a strict sense, reward centering.
It is, in a strict sense, \textbf{Bellman error centering}.

It is worth noting that this understanding is crucial, 
as designing new algorithms requires leveraging this concept.


\subsection{On the Fixpoint Solution}

The update rule (\ref{tdcentering}) is a stochastic approximation
of the following update:
\begin{equation}
\begin{split}
V_{t+1}&=V_{t}+\alpha_t [\bm{\mathcal{T}}^{\pi}\bm{V}-\bm{V}-\mathbb{E}[\delta]\bm{1}]\\
&=V_{t}+\alpha_t [\bm{\mathcal{T}}^{\pi}\bm{V}-\bm{V}-(\bm{\mathcal{T}}^{\pi}\bm{V}-\bm{V})^{\top}\bm{d}_{\pi}\bm{1}]\\
&=V_{t}+\alpha_t [\mathcal{C}(\bm{\mathcal{T}}^{\pi}\bm{V}-\bm{V})].
\end{split}
\label{tdcenteringVector}
\end{equation}
If update rule (\ref{tdcenteringVector}) converges, it is expected that
$\mathcal{C}(\mathcal{T}^{\pi}V-V)=\bm{0}$.
That is 
\begin{equation}
    \begin{split}
    \mathcal{C}\bm{V} &= \mathcal{C}\bm{\mathcal{T}}^{\pi}\bm{V} \\
    &= \mathcal{C}(\bm{R}^{\pi} + \gamma \mathbb{P}^{\pi} \bm{V}) \\
    &= \mathcal{C}\bm{R}^{\pi} + \gamma \mathcal{C}\mathbb{P}^{\pi} \bm{V} \\
    &= \mathcal{C}\bm{R}^{\pi} + \gamma (\mathbb{P}^{\pi} \bm{V} - (\mathbb{P}^{\pi} \bm{V})^{\top} \bm{d_{\pi}} \bm{1}) \\
    &= \mathcal{C}\bm{R}^{\pi} + \gamma (\mathbb{P}^{\pi} \bm{V} - \bm{V}^{\top} (\mathbb{P}^{\pi})^{\top} \bm{d_{\pi}} \bm{1}) \\  % 修正双重上标
    &= \mathcal{C}\bm{R}^{\pi} + \gamma (\mathbb{P}^{\pi} \bm{V} - \bm{V}^{\top} \bm{d_{\pi}} \bm{1}) \\
    &= \mathcal{C}\bm{R}^{\pi} + \gamma (\mathbb{P}^{\pi} \bm{V} - \bm{V}^{\top} \bm{d_{\pi}} \mathbb{P}^{\pi} \bm{1}) \\
    &= \mathcal{C}\bm{R}^{\pi} + \gamma (\mathbb{P}^{\pi} \bm{V} - \mathbb{P}^{\pi} \bm{V}^{\top} \bm{d_{\pi}} \bm{1}) \\
    &= \mathcal{C}\bm{R}^{\pi} + \gamma \mathbb{P}^{\pi} (\bm{V} - \bm{V}^{\top} \bm{d_{\pi}} \bm{1}) \\
    &= \mathcal{C}\bm{R}^{\pi} + \gamma \mathbb{P}^{\pi} \mathcal{C}\bm{V} \\
    &\dot{=} \bm{\mathcal{T}}_c^{\pi} \mathcal{C}\bm{V},
    \end{split}
    \label{centeredfixpoint}
    \end{equation}
where we defined $\bm{\mathcal{T}}_c^{\pi}$ as a centered Bellman operator.
We call equation (\ref{centeredfixpoint}) as centered Bellman equation.
And it is \textbf{centered fixpoint}.

For linear value function approximation, let define
\begin{equation}
\mathcal{C}\bm{V}_{\bm{\theta}}=\bm{\Pi}\bm{\mathcal{T}}_c^{\pi}\mathcal{C}\bm{V}_{\bm{\theta}}.
\label{centeredTDfixpoint}
\end{equation}
We call equation (\ref{centeredTDfixpoint}) as \textbf{centered TD fixpoint}.

\subsection{On-policy and Off-policy Centered TD Algorithms
with Linear Value Function Approximation}
Given the above centered TD fixpoint,
 mean squared centered Bellman error (MSCBE), is proposed as follows:
\begin{align*}
    \label{argminMSBEC}
 &\arg \min_{{\bm{\theta}}}\text{MSCBE}({\bm{\theta}}) \\
 &= \arg \min_{{\bm{\theta}}} \|\bm{\mathcal{T}}_c^{\pi}\mathcal{C}\bm{V}_{\bm{{\bm{\theta}}}}-\mathcal{C}\bm{V}_{\bm{{\bm{\theta}}}}\|_{\bm{D}}^2\notag\\
 &=\arg \min_{{\bm{\theta}}} \|\bm{\mathcal{T}}^{\pi}\bm{V}_{\bm{{\bm{\theta}}}} - \bm{V}_{\bm{{\bm{\theta}}}}-(\bm{\mathcal{T}}^{\pi}\bm{V}_{\bm{{\bm{\theta}}}} - \bm{V}_{\bm{{\bm{\theta}}}})^{\top}\bm{d}\bm{1}\|_{\bm{D}}^2\notag\\
 &=\arg \min_{{\bm{\theta}},\omega} \| \bm{\mathcal{T}}^{\pi}\bm{V}_{\bm{{\bm{\theta}}}} - \bm{V}_{\bm{{\bm{\theta}}}}-\omega\bm{1} \|_{\bm{D}}^2\notag,
\end{align*}
where $\omega$ is is used to estimate the expected value of the Bellman error.
% where $\omega$ is used to estimate $\mathbb{E}[\delta]$, $\omega \doteq \mathbb{E}[\mathbb{E}[\delta_t|S_t]]=\mathbb{E}[\delta]$ and $\delta_t$ is the TD error as follows:
% \begin{equation}
% \delta_t = r_{t+1}+\gamma
% {\bm{\theta}}_t^{\top}\bm{{\bm{\phi}}}_{t+1}-{\bm{\theta}}_t^{\top}\bm{{\bm{\phi}}}_t.
% \label{delta}
% \end{equation}
% $\mathbb{E}[\delta_t|S_t]$ is the Bellman error, and $\mathbb{E}[\mathbb{E}[\delta_t|S_t]]$ represents the expected value of the Bellman error.
% If $X$ is a random variable and $\mathbb{E}[X]$ is its expected value, then $X-\mathbb{E}[X]$ represents the centered form of $X$. 
% Therefore, we refer to $\mathbb{E}[\delta_t|S_t]-\mathbb{E}[\mathbb{E}[\delta_t|S_t]]$ as Bellman error centering and 
% $\mathbb{E}[(\mathbb{E}[\delta_t|S_t]-\mathbb{E}[\mathbb{E}[\delta_t|S_t]])^2]$ represents the the mean squared centered Bellman error, namely MSCBE.
% The meaning of (\ref{argminMSBEC}) is to minimize the mean squared centered Bellman error.
%The derivation of CTD is as follows.

First, the parameter  $\omega$ is derived directly based on
stochastic gradient descent:
\begin{equation}
\omega_{t+1}= \omega_{t}+\beta_t(\delta_t-\omega_t).
\label{omega}
\end{equation}

Then, based on stochastic semi-gradient descent, the update of 
the parameter ${\bm{\theta}}$ is as follows:
\begin{equation}
{\bm{\theta}}_{t+1}=
{\bm{\theta}}_{t}+\alpha_t(\delta_t-\omega_t)\bm{{\bm{\phi}}}_t.
\label{theta}
\end{equation}

We call (\ref{omega}) and (\ref{theta}) the on-policy centered
TD (CTD) algorithm. The convergence analysis with be given in
the following section.

In off-policy learning, we can simply multiply by the importance sampling
 $\rho$.
\begin{equation}
    \omega_{t+1}=\omega_{t}+\beta_t\rho_t(\delta_t-\omega_t),
    \label{omegawithrho}
\end{equation}
\begin{equation}
    {\bm{\theta}}_{t+1}=
    {\bm{\theta}}_{t}+\alpha_t\rho_t(\delta_t-\omega_t)\bm{{\bm{\phi}}}_t.
    \label{thetawithrho}
\end{equation}

We call (\ref{omegawithrho}) and (\ref{thetawithrho}) the off-policy centered
TD (CTD) algorithm.

% By substituting $\delta_t$ into Equations (\ref{omegawithrho}) and (\ref{thetawithrho}), 
% we can see that Equations (\ref{thetawithrho}) and (\ref{omegawithrho}) are formally identical 
% to the linear expressions of Equations (\ref{rewardcentering1}) and (\ref{rewardcentering2}), respectively. However, the meanings 
% of the corresponding parameters are entirely different.
% ${\bm{\theta}}_t$ is for approximating the discounted value function.
% $\bar{r_t}$ is an estimate of the average reward, while $\omega_t$ 
% is an estimate of the expected value of the Bellman error.
% $\bar{\delta_t}$ is the TD error for value-based reward centering, 
% whereas $\delta_t$ is the traditional TD error.

% This study posits that the CTD is equivalent to value-based reward 
% centering. However, CTD can be unified under a single framework 
% through an objective function, MSCBE, which also lays the 
% foundation for proving the algorithm's convergence. 
% Section 4 demonstrates that the CTD algorithm guarantees 
% convergence in the on-policy setting.

\subsection{Off-policy Centered TDC Algorithm with Linear Value Function Approximation}
The convergence of the  off-policy centered TD algorithm
may not be guaranteed.

To deal with this problem, we propose another new objective function, 
called mean squared projected centered Bellman error (MSPCBE), 
and derive Centered TDC algorithm (CTDC).

% We first establish some relationships between
%  the vector-matrix quantities and the relevant statistical expectation terms:
% \begin{align*}
%     &\mathbb{E}[(\delta({\bm{\theta}})-\mathbb{E}[\delta({\bm{\theta}})]){\bm{\phi}}] \\
%     &= \sum_s \mu(s) {\bm{\phi}}(s) \big( R(s) + \gamma \sum_{s'} P_{ss'} V_{\bm{\theta}}(s') - V_{\bm{\theta}}(s)  \\
%     &\quad \quad-\sum_s \mu(s)(R(s) + \gamma \sum_{s'} P_{ss'} V_{\bm{\theta}}(s') - V_{\bm{\theta}}(s))\big)\\
%     &= \bm{\Phi}^\top \mathbf{D} (\bm{TV}_{\bm{{\bm{\theta}}}} - \bm{V}_{\bm{{\bm{\theta}}}}-\omega\bm{1}),
% \end{align*}
% where $\omega$ is the expected value of the Bellman error and $\bm{1}$ is all-ones vector.

The specific expression of the objective function 
MSPCBE is as follows:
\begin{align}
    \label{MSPBECwithomega}
    &\arg \min_{{\bm{\theta}}}\text{MSPCBE}({\bm{\theta}})\notag\\ 
    % &= \arg \min_{{\bm{\theta}}}\big(\mathbb{E}[(\delta({\bm{\theta}}) - \mathbb{E}[\delta({\bm{\theta}})]) \bm{{\bm{\phi}}}]^\top \notag\\
    % &\quad \quad \quad\mathbb{E}[\bm{{\bm{\phi}}} \bm{{\bm{\phi}}}^\top]^{-1} \mathbb{E}[(\delta({\bm{\theta}}) - \mathbb{E}[\delta({\bm{\theta}})]) \bm{{\bm{\phi}}}]\big) \notag\\
    % &=\arg \min_{{\bm{\theta}},\omega}\mathbb{E}[(\delta({\bm{\theta}})-\omega) \bm{\bm{{\bm{\phi}}}}]^{\top} \mathbb{E}[\bm{\bm{{\bm{\phi}}}} \bm{\bm{{\bm{\phi}}}}^{\top}]^{-1}\mathbb{E}[(\delta({\bm{\theta}}) -\omega)\bm{\bm{{\bm{\phi}}}}]\\
    % &= \big(\bm{\Phi}^\top \mathbf{D} (\bm{TV}_{\bm{{\bm{\theta}}}} - \bm{V}_{\bm{{\bm{\theta}}}}-\omega\bm{1})\big)^\top (\bm{\Phi}^\top \mathbf{D} \bm{\Phi})^{-1} \notag\\
    % & \quad \quad \quad \bm{\Phi}^\top \mathbf{D} (\bm{TV}_{\bm{{\bm{\theta}}}} - \bm{V}_{\bm{{\bm{\theta}}}}-\omega\bm{1}) \notag\\
    % &= (\bm{TV}_{\bm{{\bm{\theta}}}} - \bm{V}_{\bm{{\bm{\theta}}}}-\omega\bm{1})^\top \mathbf{D} \bm{\Phi} (\bm{\Phi}^\top \mathbf{D} \bm{\Phi})^{-1} \notag\\
    % &\quad \quad \quad \bm{\Phi}^\top \mathbf{D} (\bm{TV}_{\bm{{\bm{\theta}}}} - \bm{V}_{\bm{{\bm{\theta}}}}-\omega\bm{1})\notag\\
    % &= (\bm{TV}_{\bm{{\bm{\theta}}}} - \bm{V}_{\bm{{\bm{\theta}}}}-\omega\bm{1})^\top {\bm{\Pi}}^\top \mathbf{D} {\bm{\Pi}} (\bm{TV}_{\bm{{\bm{\theta}}}} - \bm{V}_{\bm{{\bm{\theta}}}}-\omega\bm{1}) \notag\\
    &= \arg \min_{{\bm{\theta}}} \|\bm{\Pi}\bm{\mathcal{T}}_c^{\pi}\mathcal{C}\bm{V}_{\bm{{\bm{\theta}}}}-\mathcal{C}\bm{V}_{\bm{{\bm{\theta}}}}\|_{\bm{D}}^2\notag\\
    &= \arg \min_{{\bm{\theta}}} \|\bm{\Pi}(\bm{\mathcal{T}}_c^{\pi}\mathcal{C}\bm{V}_{\bm{{\bm{\theta}}}}-\mathcal{C}\bm{V}_{\bm{{\bm{\theta}}}})\|_{\bm{D}}^2\notag\\
    &= \arg \min_{{\bm{\theta}},\omega}\| {\bm{\Pi}} (\bm{\mathcal{T}}^{\pi}\bm{V}_{\bm{{\bm{\theta}}}} - \bm{V}_{\bm{{\bm{\theta}}}}-\omega\bm{1}) \|_{\bm{D}}^2\notag.
\end{align}
In the process of computing the gradient of the MSPCBE with respect to ${\bm{\theta}}$, 
$\omega$ is treated as a constant.
So, the derivation process of CTDC is the same 
as for the TDC algorithm \cite{sutton2009fast}, the only difference is that the original $\delta$ is replaced by $\delta-\omega$.
Therefore, the updated formulas of the centered TDC  algorithm are as follows:
\begin{equation}
 \bm{{\bm{\theta}}}_{k+1}=\bm{{\bm{\theta}}}_{k}+\alpha_{k}[(\delta_{k}- \omega_k) \bm{\bm{{\bm{\phi}}}}_k\\
 - \gamma\bm{\bm{{\bm{\phi}}}}_{k+1}(\bm{\bm{{\bm{\phi}}}}^{\top}_k \bm{u}_{k})],
\label{thetavmtdc}
\end{equation}
\begin{equation}
 \bm{u}_{k+1}= \bm{u}_{k}+\zeta_{k}[\delta_{k}-\omega_k - \bm{\bm{{\bm{\phi}}}}^{\top}_k \bm{u}_{k}]\bm{\bm{{\bm{\phi}}}}_k,
\label{uvmtdc}
\end{equation}
and
\begin{equation}
 \omega_{k+1}= \omega_{k}+\beta_k (\delta_k- \omega_k).
 \label{omegavmtdc}
\end{equation}
This algorithm is derived to work 
with a given set of sub-samples—in the form of 
triples $(S_k, R_k, S'_k)$ that match transitions 
from both the behavior and target policies. 

% \subsection{Variance Minimization ETD Learning: VMETD}
% Based on the off-policy TD algorithm, a scalar, $F$,  
% is introduced to obtain the ETD algorithm, 
% which ensures convergence under off-policy 
% conditions. This paper further introduces a scalar, 
% $\omega$, based on the ETD algorithm to obtain VMETD.
% VMETD by the following update:
% \begin{equation}
% \label{fvmetd}
%  F_t \leftarrow \gamma \rho_{t-1}F_{t-1}+1,
% \end{equation}
% \begin{equation}
%  \label{thetavmetd}
%  {{\bm{\theta}}}_{t+1}\leftarrow {{\bm{\theta}}}_t+\alpha_t (F_t \rho_t\delta_t - \omega_{t}){\bm{{\bm{\phi}}}}_t,
% \end{equation}
% \begin{equation}
%  \label{omegavmetd}
%  \omega_{t+1} \leftarrow \omega_t+\beta_t(F_t  \rho_t \delta_t - \omega_t),
% \end{equation}
% where $\rho_t =\frac{\pi(A_t | S_t)}{\mu(A_t | S_t)}$ and $\omega$ is used to estimate $\mathbb{E}[F \rho\delta]$, i.e., $\omega \doteq \mathbb{E}[F \rho\delta]$.

% (\ref{thetavmetd}) can be rewritten as
% \begin{equation*}
%  \begin{array}{ccl}
%  {{\bm{\theta}}}_{t+1}&\leftarrow& {{\bm{\theta}}}_t+\alpha_t (F_t \rho_t\delta_t - \omega_t){\bm{{\bm{\phi}}}}_t -\alpha_t \omega_{t+1}{\bm{{\bm{\phi}}}}_t\\
%   &=&{{\bm{\theta}}}_{t}+\alpha_t(F_t\rho_t\delta_t-\mathbb{E}_{\mu}[F_t\rho_t\delta_t|{{\bm{\theta}}}_t]){\bm{{\bm{\phi}}}}_t\\
%  &=&{{\bm{\theta}}}_t+\alpha_t F_t \rho_t (r_{t+1}+\gamma {{\bm{\theta}}}_t^{\top}{\bm{{\bm{\phi}}}}_{t+1}-{{\bm{\theta}}}_t^{\top}{\bm{{\bm{\phi}}}}_t){\bm{{\bm{\phi}}}}_t\\
%  & & \hspace{2em} -\alpha_t \mathbb{E}_{\mu}[F_t \rho_t \delta_t]{\bm{{\bm{\phi}}}}_t\\
%  &=& {{\bm{\theta}}}_t+\alpha_t \{\underbrace{(F_t\rho_tr_{t+1}-\mathbb{E}_{\mu}[F_t\rho_t r_{t+1}]){\bm{{\bm{\phi}}}}_t}_{{b}_{\text{VMETD},t}}\\
%  &&\hspace{-7em}- \underbrace{(F_t\rho_t{\bm{{\bm{\phi}}}}_t({\bm{{\bm{\phi}}}}_t-\gamma{\bm{{\bm{\phi}}}}_{t+1})^{\top}-{\bm{{\bm{\phi}}}}_t\mathbb{E}_{\mu}[F_t\rho_t ({\bm{{\bm{\phi}}}}_t-\gamma{\bm{{\bm{\phi}}}}_{t+1})]^{\top})}_{\textbf{A}_{\text{VMETD},t}}{{\bm{\theta}}}_t\}.
%  \end{array}
% \end{equation*}
% Therefore, 
% \begin{equation*}
%  \begin{array}{ccl}
%   &&\textbf{A}_{\text{VMETD}}\\
%   &=&\lim_{t \rightarrow \infty} \mathbb{E}[\textbf{A}_{\text{VMETD},t}]\\
%   &=& \lim_{t \rightarrow \infty} \mathbb{E}_{\mu}[F_t \rho_t {\bm{{\bm{\phi}}}}_t ({\bm{{\bm{\phi}}}}_t - \gamma {\bm{{\bm{\phi}}}}_{t+1})^{\top}]\\  
%   &&\hspace{1em}- \lim_{t\rightarrow \infty} \mathbb{E}_{\mu}[  {\bm{{\bm{\phi}}}}_t]\mathbb{E}_{\mu}[F_t \rho_t ({\bm{{\bm{\phi}}}}_t - \gamma {\bm{{\bm{\phi}}}}_{t+1})]^{\top}\\
%   &=& \lim_{t \rightarrow \infty} \mathbb{E}_{\mu}[{\bm{{\bm{\phi}}}}_tF_t \rho_t ({\bm{{\bm{\phi}}}}_t - \gamma {\bm{{\bm{\phi}}}}_{t+1})^{\top}]\\   
%   &&\hspace{1em}-\lim_{t \rightarrow \infty} \mathbb{E}_{\mu}[ {\bm{{\bm{\phi}}}}_t]\lim_{t \rightarrow \infty}\mathbb{E}_{\mu}[F_t \rho_t ({\bm{{\bm{\phi}}}}_t - \gamma {\bm{{\bm{\phi}}}}_{t+1})]^{\top}\\
%   && \hspace{-2em}=\sum_{s} d_{\mu}(s)\lim_{t \rightarrow \infty}\mathbb{E}_{\mu}[F_t|S_t = s]\mathbb{E}_{\mu}[\rho_t\bm{{\bm{\phi}}}_t(\bm{{\bm{\phi}}}_t - \gamma \bm{{\bm{\phi}}}_{t+1})^{\top}|S_t= s]\\   
%   &&\hspace{1em}-\sum_{s} d_{\mu}(s)\bm{{\bm{\phi}}}(s)\sum_{s} d_{\mu}(s)\lim_{t \rightarrow \infty}\mathbb{E}_{\mu}[F_t|S_t = s]\\
%   &&\hspace{7em}\mathbb{E}_{\mu}[\rho_t(\bm{{\bm{\phi}}}_t - \gamma \bm{{\bm{\phi}}}_{t+1})^{\top}|S_t = s]\\
%   &=& \sum_{s} f(s)\mathbb{E}_{\pi}[\bm{{\bm{\phi}}}_t(\bm{{\bm{\phi}}}_t- \gamma \bm{{\bm{\phi}}}_{t+1})^{\top}|S_t = s]\\   
%   &&\hspace{1em}-\sum_{s} d_{\mu}(s)\bm{{\bm{\phi}}}(s)\sum_{s} f(s)\mathbb{E}_{\pi}[(\bm{{\bm{\phi}}}_t- \gamma \bm{{\bm{\phi}}}_{t+1})^{\top}|S_t = s]\\
%   &=&\sum_{s} f(s) \bm{\bm{{\bm{\phi}}}}(s)(\bm{\bm{{\bm{\phi}}}}(s) - \gamma \sum_{s'}[\textbf{P}_{\pi}]_{ss'}\bm{\bm{{\bm{\phi}}}}(s'))^{\top}  \\
%   &&-\sum_{s} d_{\mu}(s) {\bm{{\bm{\phi}}}}(s) * \sum_{s} f(s)({\bm{{\bm{\phi}}}}(s) - \gamma \sum_{s'}[\textbf{P}_{\pi}]_{ss'}{\bm{{\bm{\phi}}}}(s'))^{\top}\\
%   &=&{\bm{\bm{\Phi}}}^{\top} \textbf{F} (\textbf{I} - \gamma \textbf{P}_{\pi}) \bm{\bm{\Phi}} - {\bm{\bm{\Phi}}}^{\top} {d}_{\mu} {f}^{\top} (\textbf{I} - \gamma \textbf{P}_{\pi}) \bm{\bm{\Phi}}  \\
%   &=&{\bm{\bm{\Phi}}}^{\top} (\textbf{F} - {d}_{\mu} {f}^{\top}) (\textbf{I} - \gamma \textbf{P}_{\pi}){\bm{\bm{\Phi}}} \\
%   &=&{\bm{\bm{\Phi}}}^{\top} (\textbf{F} (\textbf{I} - \gamma \textbf{P}_{\pi})-{d}_{\mu} {f}^{\top} (\textbf{I} - \gamma \textbf{P}_{\pi})){\bm{\bm{\Phi}}} \\
%   &=&{\bm{\bm{\Phi}}}^{\top} (\textbf{F} (\textbf{I} - \gamma \textbf{P}_{\pi})-{d}_{\mu} {d}_{\mu}^{\top} ){\bm{\bm{\Phi}}},
%  \end{array}
% \end{equation*}
% \begin{equation*}
%  \begin{array}{ccl}
%   &&{b}_{\text{VMETD}}\\
%   &=&\lim_{t \rightarrow \infty} \mathbb{E}[{b}_{\text{VMETD},t}]\\
%   &=& \lim_{t \rightarrow \infty} \mathbb{E}_{\mu}[F_t\rho_tR_{t+1}{\bm{{\bm{\phi}}}}_t]\\
%   &&\hspace{2em} - \lim_{t\rightarrow \infty} \mathbb{E}_{\mu}[{\bm{{\bm{\phi}}}}_t]\mathbb{E}_{\mu}[F_t\rho_kR_{k+1}]\\  
%   &=& \lim_{t \rightarrow \infty} \mathbb{E}_{\mu}[{\bm{{\bm{\phi}}}}_tF_t\rho_tr_{t+1}]\\
%   &&\hspace{2em} - \lim_{t\rightarrow \infty} \mathbb{E}_{\mu}[  {\bm{{\bm{\phi}}}}_t]\mathbb{E}_{\mu}[{\bm{{\bm{\phi}}}}_t]\mathbb{E}_{\mu}[F_t\rho_tr_{t+1}]\\ 
%   &=& \lim_{t \rightarrow \infty} \mathbb{E}_{\mu}[{\bm{{\bm{\phi}}}}_tF_t\rho_tr_{t+1}]\\
%   &&\hspace{2em} - \lim_{t \rightarrow \infty} \mathbb{E}_{\mu}[ {\bm{{\bm{\phi}}}}_t]\lim_{t \rightarrow \infty}\mathbb{E}_{\mu}[F_t\rho_tr_{t+1}]\\  
%   &=&\sum_{s} f(s) {\bm{{\bm{\phi}}}}(s)r_{\pi} - \sum_{s} d_{\mu}(s) {\bm{{\bm{\phi}}}}(s) * \sum_{s} f(s)r_{\pi}  \\
%   &=&\bm{\bm{\bm{\Phi}}}^{\top}(\textbf{F}-{d}_{\mu} {f}^{\top}){r}_{\pi}.
%  \end{array}
% \end{equation*}



Recent efforts in speculative decoding have focused on developing effective drafting methods, using LM-based approaches, such as using smaller models than LLM~\cite{DistilSpec, SpecInfer} or incorporating specialized branches within the LLM architecture~\cite{MEDUSA, EAGLE2}.
However, their applicability in real-world scenarios is limited by the significant overhead associated with fine-tuning for optimization.
First, smaller models for drafting must be fine-tuned, such as by distillation, to generate tokens similar to LLMs to achieve optimal performance regardless of the given tasks~\cite{DistilSpec, multilingual}.
In addition, current LLM families~\cite{Llama2, vicuna} do not offer models of an appropriate size for drafting, often necessitating training from scratch.
In branch-based drafting, which modifies its original LLM architecture, the computational cost for training such branches within LLM is significant due to gradient calculations across the entire model, even though most parameters remain frozen~\cite{MEDUSA, EAGLE2, EAGLE}.
For example, EAGLE~\cite{EAGLE}, one of the leading methods, needs 1-2 days of training on 2-4 billion tokens using 4 A100 GPUs to train the 70B model.

To address these limitations, this paper explores a lightweight, lossless drafting strategy: \textit{Database Drafting}, eliminating the need for parameter updates~\cite{PLD, LAD, REST}. 
Database drafting constructs databases from various token sources and fetches draft tokens from the database using previous tokens.
However, as previous work relies on a single database from a single source, the coverage of draft tokens is restricted, leading to inconsistent acceleration across different tasks, as depicted in the left side of Figure~\ref{fig:motivation}. 
For example, PLD~\cite{PLD}, which uses previous tokens as its source, shows strengths in the summarization, highly repeating the tokens in the earlier texts, yet it achieves only marginal speedups in QA, where fewer promising tokens are included in the prior text. 
A straightforward solution to improve coverage is incorporating diverse sources into a single database. 
However, increasing the database scale leads to higher drafting latency, resulting in additional overhead.
As shown in the right side of Figure~\ref{fig:motivation}, REST~\cite{REST}, which uses the largest database, accurately predicts future tokens but suffers from significant latency, negating its high acceptance ratio benefits. 
Therefore, this paper proposes a solution to these limitations: \textit{Utilize diverse token sources simultaneously for robust performance and minimal overhead.}


With this objective in mind, we propose a simple yet effective solution: \textbf{Hierarchy Drafting} (HD), which integrates diverse token sources into a hierarchical framework. 
Our proposed method is inspired by the memory hierarchy system, which prioritizes data with high \textit{temporal locality} in the memory access for performance optimization~\cite{hierarchy}.
Therefore, HD groups draft tokens from diverse sources based on their temporal locality---the tendency for some tokens to reappear within or across generation processes. 
For example, when an LLM solves a math problem like, ‘\textit{The vertices of a triangle are at points (0, 0), (-1, 1), and (3, 3). What is the area of the triangle?}’, the coordinates frequently repeat within only a generation process for a given query but not across other generation processes.
In a related sense, phrases commonly generated by LLMs, such as ‘\textit{as an AI assistant}’, or frequent grammatical patterns exhibit relatively moderate locality, often appearing across different generation processes. 

Based on their temporal locality, the multiple databases of HD organize them into \textit{context-dependent database}, which stores tokens with high temporal locality for a given context; \textit{model-dependent database}, which captures frequently repeated phrases by LLMs across generations; and \textit{statistics-dependent database}, which contains statistically common phrases with slightly lower locality across processes than those in the model-dependent database.
During inference, HD accesses the databases in order of temporal locality, prioritizing tokens with high locality by starting with context-dependent, then model-dependent, and finally statistics-dependent databases until a sufficient number of draft tokens are obtained to convey to the LLM for verification.


This strategy has two benefits: firstly, increasing drafting accuracy by leveraging temporal locality and
secondly, reducing the overhead from drafting latency, as the scale of the databases is inversely correlated with the degree of locality—tokens with high locality are rarer. Thus, starting with the smaller context-dependent database for drafting tokens is more accurate and faster than using the larger statistics-dependent database alone.
Also, our hierarchical framework can encompass other database drafting methods owing to its \textit{plug-and-play} nature, making it easy to integrate diverse drafting sources based on their temporal locality.

We evaluate HD and other database drafting methods using widely adopted LLMs, Llama-2~\cite{Llama2} and Vicuna~\cite{vicuna}, on Spec-Bench~\cite{Spec_Survey}, a benchmark designed to assess effectiveness across diverse tasks.
Our proposed method, HD, outperforms other methods in our experiment and consistently achieves significant inference speedup across various settings, including model size, temperature, and tasks.
We also analyze how the hierarchical framework adaptively selects the appropriate database for each task while minimizing draft latency, aligning with our design goals.


Our contributions in this paper are threefold:
\vspace{-0.1in}
\begin{itemize}[itemsep=0.3mm, parsep=1pt, leftmargin=*]
    \item We identify the limitations of existing speculative decoding methods, which require additional fine-tuning or deliver inconsistent acceleration gains.
    \item We introduce a novel database drafting method, Hierarchy Drafting (HD), incorporating diverse token sources into the hierarchical framework for robust performance with minimizing overhead.
    \item We demonstrate that HD consistently achieves significant acceleration gains across various scenarios compared to other lossless methods.
\end{itemize}



\section{Understanding Transformer's Attention}


\begin{figure*}[ht]
    \hfill
    \includegraphics[width=\linewidth]{imgs/figure2.pdf}
    \caption{The $\log_{10}$ perplexity of four LLMs (Llama-2-7b, Llama-3.1-8B, Qwen2-7B and Mistral-7B-v0.1) on the third book of PG-19 test set using SWA inference. The window sizes are set not to exceed their respective training sequence lengths. The x-axis represents the sliding window size, and the y-axis represents the evaluation sequence length. For a fixed window size, perplexity increases (color shifts to blue) as the evaluation length grows.}
    \label{fig:open-llms}
\end{figure*}

\begin{figure*}[ht]
    \centering
    \includegraphics[width=\linewidth]{imgs/figure3.pdf}
    \caption{Heatmaps of attention scores (top four squares) and token embedding variance (bottom four lines) across different layers of Qwen2-7B. Higher token variance corresponds to stronger attention, highlighting their correlation. The two color bars indicate respective scales.}
    \label{fig:variance}
\end{figure*}

This section introduces concepts of the SWA mechanism and its potential capability in handling long sequences. We then analyze why current LLMs with SWA inference fail to achieve the expected theoretical advantages. 
% fail to perform SWA inference despite its theoretical advantages.

\subsection{Sliding Window Attention}

The self-attention layer in Transformers typically has $O(N^2)$ computational complexity, where $N$ is the input sequence length. 
To reduce this complexity while preserving the sequential information, sliding window attention (SWA) is introduced in Longformer~\cite{Longformer}. 
SWA restricts each token to only attend the attention calculation of its neighboring tokens within a fixed-size window.
% SWA restricts each token to attend only to its neighboring tokens within a fixed-size window.
With a window size of $\omega \ll N$, the computation cost per token is reduced to $O(\omega)$, leading to an overall linear complexity $O(N \cdot \omega)$, which is more efficient than vanilla attention.

We visualize the SWA mechanism in Figure~\ref{fig:swa}, where the window size is three ($\omega=3$) and the depth is two ($L=2$).
We define the tokens that are visible to the current window as active tokens (the red block in the figure, corresponding active tokens are ``a dear little'').
For invisible tokens, also referred to as evicted tokens, we further categorize them as residual and past tokens. 
Residual tokens are not visible to the sliding window at the embedding layer. However, their information will passed to the neighboring $\omega -1$ tokens with a transformer layer (this information transition is represented as yellow lines in the figure), thus partially preserved for the prediction. For example, the information of the token `a' (the orange ball at the embedding layer) can be retained in the other token `a' (the red ball at the second transformer layer) in our visualization. Theoretically, the information range of a single token at the $l^{th}$ transformer layer is $1+(\omega-1) \cdot l$ and the maximum range is $1+(\omega-1) \cdot L$, i.e., $1+2\cdot2=5$ in the figure.

% Among them, the information on the active tokens is fully accessible to the model since their embeddings are still within the attention window. For the latter two categories, referred to as evicted tokens, their embeddings have been discarded and are no longer visible to the model. However, residual tokens' information is preserved in the higher transformer layers, allowing the model to utilize this retained information during inference. With each additional transformer layer, the residual tokens retained in the higher layers expand by $(\omega-1)$ tokens. Therefore, the theoretical maximum range of information coverage in a SWA-based transformer is $(\omega-1) \cdot L + 1$, where $L$ denotes the number of transformer layers.

% Therefore, theoretically, evicted tokens (i.e., tokens whose key-value cache is discarded) are not indispensable for maintaining the model's performance as long as their information is effectively retained and propagated through the higher layers of the transformer. Empirically, larger window sizes, deeper models, and longer training sequences result in better performance.

\subsection{LLMs with SWA Inference}
%\subsection{Why current LLMs fail using SWA?}
\label{ssec:why}


Although current open-source LLMs are structurally capable of conducting SWA inference, they fail to achieve stable improved results. As shown in Figure~\ref{fig:open-llms}, we analyzed the perplexity (PPL) of four open-source LLMs~\cite{llama2,llama3,mistral-7b,qwen2} using different sliding window sizes on the PG-19~\cite{pg19} test set. The experimental results reveal that these LLMs achieve optimal performance only when operating within their training sequence length. For instance, for Llama-2-7b model in Figure~\ref{fig:open-llms}(a), when the window size is fixed at 1,024, the perplexity gradually increases as the evaluation length grows, as indicated by the color transition from blue to red in the heatmap.
This suggests that Transformers inherently learn contextual patterns specific to their training length and fail to extend to variable-length texts during inference.
% This limitation suggests that transformers inherently learn attention patterns specific to their training context length, making them poorly suited for processing variable-length texts during inference.

We suggest that this failure can be attributed to two major issues:
(1) the attention sink phenomenon, where models become overly dependent on initial tokens, 
and (2) information loss that past tokens are discarded.

The attention sink phenomenon~\cite{streamingllm}, where LLMs allocate excessive attention to initial tokens in sequences, has emerged as a significant challenge for SWA inference in Transformer architectures. Previous work has made two key observations regarding this phenomenon. First, the causal attention mechanism in Transformers is inherently non-permutation invariant, with positional information emerging implicitly through token embedding variance after softmax normalization~\cite{variance}. Second, studies have demonstrated that removing normalization from the attention mechanism can effectively eliminate the attention sink effect~\cite{whensinkemerge}.

% \textbf{Lemma 1:} The causal attention in Transformers is not permutation invariant inherently. The positional information emerges from the variance of token embeddings after the normalization operation of the softmax function~\cite{variance}.

% \textbf{Lemma 2:} \citet{whensinkemerge} demonstrates that removing normalization from attention eliminates the attention sink phenomenon~\cite{streamingllm}.

Based on these insights, we analyze the attention patterns and hidden state statistics of Qwen2-7B, as shown in Figure~\ref{fig:open-llms}. Our results reveal a strong correlation between token variance and attention sink magnitude---the variance of hidden states for the first token is significantly higher than for subsequent tokens. \textit{This finding provides strong evidence that attention sink manifests through variance propagation via normalization.} Notably, even though models like Qwen2 incorporate explicit relative position embeddings (e.g., RoPE), they still learn and rely on this implicit absolute positional information through the normalization mechanism.

Beyond the attention sink problem, softmax also leads to significant information loss during sliding window inference. Consider the following example of how softmax transforms attention scores:
\begin{equation}
\begin{bmatrix}
1.5 \\
5.0 \\
2.4 \\
0.5 \\
1.3
\end{bmatrix}
\to \text{Softmax}(x_i) = \frac{e^{x_i}}{\sum_{j} e^{x_j}} \to
\begin{bmatrix}
0.03 \\
0.88 \\
0.07 \\
0.01 \\
0.02
\end{bmatrix}
\end{equation}
As shown above, the exponential nature of softmax dramatically amplifies differences between logits, causing most of the probability mass to concentrate on the highest-scoring token (0.88 in this case) while severely suppressing other tokens (all below 0.07). A detailed mathematical proof of this sparsification property is provided in Appendix~\ref{app:sparsity}.
% Although this sparsity can help the model achieve high prediction accuracy when no tokens are dropped, it results in severe information loss in the context of SWA.

In summary, while softmax's sparsification is beneficial for full-context Transformers, it becomes limiting in SWA scenario where the aggressive filtering impedes the model's ability to retain historical information within the sliding window.

% In summary, while softmax's aggressive filtering mechanism proves effective in vanilla Transformers with full context access, it becomes problematic in sliding window attention. During SWA inference, where context is inherently limited, softmax's strong normalization hinders the model's ability to retain historical information within the constrained window. This suggests that effective sliding window attention requires a more balanced approach to information filtering, especially when operating with limited context.


\section{Sliding Window Attention Training}

In this section, we explore the advantages of SWA training over traditional Transformer training with a new paradigm for processing long sequences. Additionally, we provide a detailed explanation of our proposed SWAT attention layer. This simple yet effective attention layer combines Sigmoid~\cite{sigmoid}, ALiBi, and RoPE to address the information retention challenges of SWA.

\subsection{Information Transmission}

Traditional Transformer training involves processing entire sequences of tokens, allowing the model to capture long-range dependencies through global attention mechanisms. In contrast, SWA operates within a limited context, necessitating new approaches to preserve information continuously. As shown in Figure~\ref{fig:att}, SWA training enables two distinct learning paradigms for LLMs, short and long sequence attentions.

In conventional Transformer training, the sequence length is smaller than the window size. New tokens can acquire and integrate information from all tokens, even the very first tokens in the text. Therefore, the model keeps essential information in each token embedding and enhances the ability to extract information, which is also strengthened by the softmax function.

SWA training introduces a new training paradigm, where each window shift requires careful historical context management. In particular, the old token embedding is discarded after sliding. However, in the upper layers of the Transformer, the new token's embedding still retains the old token's embedding with a certain weight. Hence, the model tends to retain all past embeddings in the upper-level model to prevent information loss caused by sliding windows, strengthening the model's ability to compress information. The experimental results demonstrating how SWA training enhances the model's capabilities are presented in Sections~\ref{ssec:swat} and \ref{ssec:ablation}.


\subsection{Attention Computation}
\label{ssec:Attention-Computation}

\begin{figure}[t]
    \centering
    \includegraphics[width=\linewidth]{imgs/figure4.pdf}
    \caption{The demonstration of the SWA mechanism in Transformers, where the model's information coverage includes residual and active tokens, depending on the model depth and window size.}
    \label{fig:att}
\end{figure}

In this subsection, we propose SWAT, a modified attention mechanism that combines sigmoid activation with integrated position embeddings. The input consists of queries, keys, and values with dimension of $d$. Instead of using softmax normalization, we apply sigmoid activation to the scaled dot products to obtain attention weights, preventing mutual suppression between tokens:
\begin{equation}
\text{Attention}(\boldsymbol{Q}, \boldsymbol{K}, \boldsymbol{V}) = \sigma (\frac{\boldsymbol{Q}\boldsymbol{K}^T}{\sqrt{d}})\boldsymbol{V}
\end{equation}
where $\boldsymbol{Q} \in \mathbb{R}^{N \times d}$, $\boldsymbol{K} \in \mathbb{R}^{N \times d}$, and $\boldsymbol{V} \in \mathbb{R}^{N \times d}$ are packed matrices of queries, keys, and values, respectively; $\sigma ( \cdot )$ is the sigmoid function. More detailed analysis can be found in Appendix~\ref{app:density}.


To introduce discriminative bias in the dense attention patterns of sigmoid activation and better differentiate token representations within sliding windows, we propose balanced ALiBi, a bidirectional extension of the original ALiBi mechanism. For an input subsequence within a window, we add position-dependent biases to the attention scores:
\begin{equation}
\text{Attention}(\boldsymbol{Q}, \boldsymbol{K}, \boldsymbol{V}) = \sigma (\frac{\boldsymbol{Q}\boldsymbol{K}^T}{\sqrt{d}} + s \cdot (m-n))\boldsymbol{V}
\end{equation}
where $m$ and $n$ ($m >le n$) denote the index of tokens in the sequence and $s$ denotes the slope.
Unlike the original ALiBi, which uses only negative slopes to enforce a directional inductive bias, we use both positive and negative slopes across different attention heads. For a model with $h$ heads, we assign positive slopes to $h/2$ heads and negative slopes to the remaining heads. The magnitude of slopes follows a geometric sequence similar to ALiBi, but in both directions:
\begin{equation}
s_k = \begin{cases}
-2^{-k} & \text{for forward-looking heads} \\
2^{-k} & \text{for backward-looking heads}
\end{cases}
\label{eq:-+}
\end{equation}
where $k$ ranges from 1 to $h/2$ for each direction. This bidirectional slope design allows attention heads to specialize in different temporal directions, with forward-looking heads focusing on recent context and backward-looking heads preserving historical information.



After replacing softmax with sigmoid, the implicit position information through normalization is lost, leading to training instability. Furthermore, while balanced ALiBi provides positional variance through attention weights, its positional signals remain weak. To address this issue, we further incorporate RoPE to enhance explicit positional information. Finally, SWAT attention calculates the attention output as follows:
\begin{equation}
    \begin{aligned}
        & \text{Attention}(\boldsymbol{Q}, \boldsymbol{K}, \boldsymbol{V})_m = {\textstyle \sum_{n=m-\omega+1}^{m}}  \\
        & \sigma \Bigg( 
        \frac{(\boldsymbol{R}_{\Theta, m}^d \boldsymbol{q}_m)^T (\boldsymbol{R}_{\Theta, n}^d \boldsymbol{k}_n)}
        {\sqrt{d_k}}  \quad + s \cdot (m-n) \Bigg) \boldsymbol{v}_n
    \end{aligned}
\end{equation}
where $\boldsymbol{R}_{\Theta, m}^d$ and $\boldsymbol{R}_{\Theta, n}^d$ are the same rotation matrices as Equation 15 in \cite{rope}. To ensure SWA training, note that $m-n < \omega$.

This combination of sigmoid activation, balanced ALiBi, and RoPE makes up for the sparsity of the vanilla Transformer. It ensures the stability of training and strengthens the information contained in a single token embedding.


\subsection{Network Efficiency}

Since SWAT's architecture is nearly identical to a standard attention layer, the per-token computation cost remains almost the same under an equivalent attention length—apart from the additional overhead of computing the ALiBi. However, the overall computation becomes linear due to the use of a sliding window. Thus, the inference computational complexity can be expressed as:
\begin{equation}
\mathrm{Cost} =N  \omega \times ( 1+\delta_{\text{ALiBi}}), 0 < \delta_{\text{ALiBi}} \ll 1
\end{equation}
where $\delta_{\text{ALiBi}}$ represents the extra cost of ALiBi.





\section{Experiments}
\label{experiments}

\begin{table*}[t]
\tiny
% \caption{Overall comparison of SWAT and other models on eight common-sense reasoning tasks. (-) denotes negative slopes (i.e., ALiBi slope),  (+) denotes positive slopes, while (-+) means half of the attention heads have negative slopes and half have positives. Optimal values are marked in bold, and second-best values are underlined.}
\caption{Overall comparison of SWAT and other models on eight common-sense reasoning tasks. Bold values represent optimal performance, while second-best values are underlined. ``\textbf{{ *}}'' indicates the statistically significant improvements (i.e., two-sided t-test with $p<0.05$) over the best baseline. $\uparrow$: higher is better. $\downarrow$: lower is better.}
\label{tab:overall} 
\resizebox{\textwidth}{!}{
\begin{tabular}{@{}lccccccccccc@{}}
\toprule
\multicolumn{1}{l|}{Model} & \begin{tabular}[c]{@{}c@{}}Wiki.\\ ppl $\downarrow$\end{tabular} & \multicolumn{1}{c|}{\begin{tabular}[c]{@{}c@{}}LMB. \\ ppl $\downarrow$\end{tabular}} & \begin{tabular}[c]{@{}c@{}}LMB. \\ acc $\uparrow$\end{tabular} & \begin{tabular}[c]{@{}c@{}}PIQA\\ acc $\uparrow$\end{tabular} & \begin{tabular}[c]{@{}c@{}}Hella. \\ acc\_n $\uparrow$\end{tabular} & \begin{tabular}[c]{@{}c@{}}Wino. \\ acc $\uparrow$\end{tabular} & \begin{tabular}[c]{@{}c@{}}ARC-e\\ acc $\uparrow$\end{tabular} & \begin{tabular}[c]{@{}c@{}}ARC-c\\ acc\_n $\uparrow$\end{tabular} & \begin{tabular}[c]{@{}c@{}}SIQA\\ acc $\uparrow$\end{tabular} & \begin{tabular}[c]{@{}c@{}}BoolQ\\ acc $\uparrow$\end{tabular} & \begin{tabular}[c]{@{}c@{}}Avg.\\ $\uparrow$\end{tabular} \\ \midrule \midrule
\multicolumn{12}{c}{340M params / 15B tokens} \\ \midrule
\multicolumn{1}{l|}{Transformer++} & 31.52 & \multicolumn{1}{c|}{41.08} & 30.76 & 62.98 & 34.76 & 50.53 & 45.21 & 24.05 & 36.81 & 58.24 & 42.92 \\
\multicolumn{1}{l|}{RetNet} & 32.50 & \multicolumn{1}{c|}{49.73} & 28.24 & 62.61 & 34.15 & 50.91 & 44.27 & 23.62 & 36.79 & 59.72 & 42.54 \\
\multicolumn{1}{l|}{GLA} & 28.51 & \multicolumn{1}{c|}{43.02} & 28.73 & 64.05 & 35.96 & 50.00 & 54.19 & 24.29 & 37.13 & 58.39 & 44.09 \\
\multicolumn{1}{l|}{Mamba} & 30.83 & \multicolumn{1}{c|}{40.21} & 29.94 & 63.79 & 35.88 & 49.82 & 49.24 & 24.56 & 35.41 & 60.07 & 43.59 \\
\multicolumn{1}{l|}{DeltaNet} & 28.65 & \multicolumn{1}{c|}{47.30} & 28.43 & 63.52 & 35.95 & 49.63 & 52.68 & 25.37 & \underline{37.96} & 58.79 & 44.04 \\
\multicolumn{1}{l|}{TTT} & 27.44 & \multicolumn{1}{c|}{34.19} & 30.06 & 63.97 & 35.71 & 50.08 & 53.01 & 26.11 & 37.32 & 59.83 & 44.51 \\
\multicolumn{1}{l|}{Gated DeltaNet} & \underline{27.01} & \multicolumn{1}{c|}{\underline{30.94}} & \underline{34.11} & 63.08 & 38.12 & \underline{51.60} & 55.28 & 26.77 & 34.89 & 59.54 & 45.42 \\
\multicolumn{1}{l|}{Titans} & \textbf{26.18} & \multicolumn{1}{c|}{\textbf{29.97}} & \textbf{34.98} & 64.73 & \textbf{39.61} & \textbf{51.85} & 55.60 & \underline{28.14} & 34.52 & 59.99 & \underline{46.17} \\
\multicolumn{1}{l|}{SWAT (-)} & 33.32 & \multicolumn{1}{c|}{36.75} & 32.80 & \textbf{ 65.94*} & \underline{38.99} & 50.12 & \textbf{ 59.68*} & \textbf{ 28.24*} & \textbf{ 38.69*} & \underline{60.55} & \textbf{ 46.88*} \\
\multicolumn{1}{l|}{SWAT (+)} & 37.47 & \multicolumn{1}{c|}{49.15} & 29.59 & 65.40 & 36.92 & 50.43 & 54.55 & 26.88 & 37.67 & 58.93 & 45.05 \\
\multicolumn{1}{l|}{SWAT (-+)} & 35.53 & \multicolumn{1}{c|}{45.06} & 29.96 & \underline{65.67} & 37.39 & 50.91 & \underline{56.99} & 27.05 & 36.75 & \textbf{ 62.11*} & 45.85 \\ \midrule
\multicolumn{12}{c}{760M params / 30B tokens} \\ \midrule
\multicolumn{1}{l|}{Transformer++} & 25.21 & \multicolumn{1}{c|}{27.64} & 35.78 & 66.92 & 42.19 & 51.95 & 60.38 & 32.46 & 39.51 & 60.37 & 48.69 \\
\multicolumn{1}{l|}{RetNet} & 26.08 & \multicolumn{1}{c|}{24.45} & 34.51 & 67.19 & 41.63 & 52.09 & 63.17 & 32.78 & 38.36 & 57.92 & 48.46 \\
\multicolumn{1}{l|}{Mamba} & 28.12 & \multicolumn{1}{c|}{23.96} & 32.80 & 66.04 & 39.15 & 52.38 & 61.49 & 30.34 & 37.96 & 57.62 & 47.22 \\
\multicolumn{1}{l|}{Mamba2} & 22.94 & \multicolumn{1}{c|}{28.37} & 33.54 & 67.90 & 42.71 & 49.77 & 63.48 & 31.09 & 40.06 & 58.15 & 48.34 \\
\multicolumn{1}{l|}{DeltaNet} & 24.37 & \multicolumn{1}{c|}{24.60} & 37.06 & 66.93 & 41.98 & 50.65 & 64.87 & 31.39 & 39.88 & 59.02 & 48.97 \\
\multicolumn{1}{l|}{TTT} & 24.17 & \multicolumn{1}{c|}{23.51} & 34.74 & 67.25 & 43.92 & 50.99 & 64.53 & \underline{33.81} & \textbf{40.16} & 59.58 & 47.32 \\
\multicolumn{1}{l|}{Gated DeltaNet} & \underline{21.18} & \multicolumn{1}{c|}{22.09} & 35.54 & 68.01 & 44.95 & 50.73 & \textbf{66.87} & 33.09 & 39.21 & 59.14 & 49.69 \\
\multicolumn{1}{l|}{Titans} & \textbf{20.04} & \multicolumn{1}{c|}{21.96} & 37.40 & 69.28 & \underline{48.46} & 52.27 & \underline{66.31} & \textbf{35.84} & \underline{40.13} & \textbf{62.76} & \underline{51.56} \\
\multicolumn{1}{l|}{SWAT (-)} & 23.41 & \multicolumn{1}{c|}{\underline{21.05}} & \textbf{ 40.81*} & \textbf{ 69.80*} & \textbf{ 48.65*} & 51.69 & 65.15 & 33.53 & 39.95 & 61.07 & \textbf{ 51.85*} \\
\multicolumn{1}{l|}{SWAT (+)} & 23.91 & \multicolumn{1}{c|}{\textbf{21.05}} & 39.01 & 69.59 & 47.64 & \underline{53.43} & 64.73 & 32.34 & 39.15 & 57.95 & 50.48 \\ 
\multicolumn{1}{l|}{SWAT (-+)} & 23.34 & \multicolumn{1}{c|}{21.36} & \underline{39.08} & \underline{69.70} & 48.16 & \textbf{53.91*} & 65.15 & 31.06 & 39.41 & \underline{61.62} & 51.01 \\
% \midrule
% \multicolumn{12}{c}{1.3B params / 100B tokens} \\ \midrule
% \multicolumn{1}{l|}{Transformer++} & 18.53 & \multicolumn{1}{c|}{18.32} & 42.60 & 70.02 & 50.23 & 53.51 & 68.83 & 35.10 & 40.66 & 57.09 & 52.25 \\
% \multicolumn{1}{l|}{RetNet} & 19.08 & \multicolumn{1}{c|}{17.27} & 40.52 & 70.07 & 49.16 & 54.14 & 67.34 & 33.78 & 40.78 & 60.39 & 52.02 \\
% \multicolumn{1}{l|}{Mamba} & 17.92 & \multicolumn{1}{c|}{15.06} & 43.98 & 71.32 & 52.91 & 52.95 & 69.52 & 35.40 & 37.76 & 61.13 & 53.12 \\
% \multicolumn{1}{l|}{DeltaNet} & 17.71 & \multicolumn{1}{c|}{16.88} & 42.46 & 70.72 & 50.93 & 53.35 & 68.47 & 35.66 & 40.22 & 55.29 & 52.14 \\
% \multicolumn{1}{l|}{Gated DeltaNet} & 16.42 & \multicolumn{1}{c|}{12.17} & 46.65 & 72.25 & 55.76 & 57.45 & 71.21 & 38.39 & 40.63 & 60.24 & 55.32 \\
% \multicolumn{1}{l|}{SWAT (-)} & 18.41       & \multicolumn{1}{c|}{12.85} & 39.01 & 72.63 & 56.47 & 56.67 & 73.36 & 40.19 & 41.91 & 60.37 & 55.08 \\
% \multicolumn{1}{l|}{SWAT (+)} & \multicolumn{1}{l}{} & \multicolumn{1}{l|}{} & \multicolumn{1}{l}{} & \multicolumn{1}{l}{} & \multicolumn{1}{l}{} & \multicolumn{1}{l}{} & \multicolumn{1}{l}{} & \multicolumn{1}{l}{} & \multicolumn{1}{l}{} & \multicolumn{1}{l}{} & \multicolumn{1}{l}{} \\
% \multicolumn{1}{l|}{SWAT (-+)} &  & \multicolumn{1}{c|}{} &  &  &  &  &  &  &  &  &  \\ 
\bottomrule
\end{tabular}
}
\end{table*}



\subsection{Experiment Settings}

% In preliminary experiments, we compare our model with the vanilla Transformer~\cite{llama2}. In the overall performance comparison, we utilize the experimental results from Titans~\cite{titans} and Gated DeltaNet~\cite{gateddeltanet}. The experiments are based on two GitHub repositories nanoGPT\footnote{\url{https://github.com/karpathy/nanoGPT}} and flash-linear-attention.\footnote{\url{https://github.com/fla-org/flash-linear-attention}}
\paragraph{Datasets.}

For the overall comparison, models are trained on the 100BT subset of FineWeb-Edu~\cite{fineweb-edu}, which is a high-quality educational dataset designed for LLM pre-training.

% In preliminary experiments, we employed three datasets for model pre-training and evaluation: OpenWebText~\cite{openwebtext}, OpenOrca~\cite{OpenOrca}, and PG-19~\cite{pg19}. We utilized OpenWebText as the training dataset, while all three datasets were incorporated into the validation phase. 
% We extended the input sequence length to 16,384 tokens for OpenWebText and PG-19, while for OpenOrca, we specifically selected questions with the longest context segments to validate long-context processing ability.

% OpenWebText, with its shorter texts, evaluates fundamental language modeling. PG-19, based on book-length texts, tests information compression in long-form contexts. OpenOrca, a question-answering dataset, helps evaluate the model's ability to retain information across extended sequences. 


\paragraph{Baselines.}

Our baselines include state-of-the-art models including both vanilla Transformer and recurrent models. Specifically, we compare our approach against Transformer++~\cite{llama2}, RetNet~\cite{retnet}, Gated Linear Attention (GLA)~\cite{gla}, Mamba~\cite{mamba}, DeltaNet~\cite{deltanet}, TTT~\cite{ttt}, Gated DeltaNet~\cite{gateddeltanet}, and Titans~\cite{titans}. 


\paragraph{Implementation Details.}

We pre-train SWAT with model sizes of 340M and 760M parameters on 15B and 30B tokens, respectively. The training uses the same vocabulary as Llama 2~\cite{llama2}, with a sequence length of 4096 tokens and a batch size of 0.5M tokens.

% RMSNorm~\cite{RMSNorm} for normalization.

\paragraph{Evaluation Metrics.}

We evaluate model performance using perplexity (ppl), accuracy (acc), and normalized accuracy (acc\_n). Perplexity measures language modeling ability, where lower values indicate better predictions. Accuracy assesses classification performance by calculating the proportion of correct predictions. Normalized accuracy is adjusts for dataset difficulty variations, ensuring fair comparisons across different evaluation settings. 


\begin{table*}[t]
\caption{Performance comparison of language models pretrained with and without sliding windows.}
\label{tab:performance_comparison}  % 设置标签
\resizebox{\textwidth}{!}{
\begin{tabular}{@{}l|ccc|cccc|cccc|c@{}}
\toprule
\multirow{2}{*}{\textbf{Models}} & \multirow{2}{*}{\textbf{\begin{tabular}[c]{@{}c@{}}Training\\ Window\end{tabular}}} & \multirow{2}{*}{\textbf{\begin{tabular}[c]{@{}c@{}}Training \\  Length\end{tabular}}} & \multirow{2}{*}{\textbf{\begin{tabular}[c]{@{}c@{}}Eval\\ Window\end{tabular}}} & \multicolumn{4}{c|}{\textbf{OpenWebText (Eval Length=)}} & \multicolumn{4}{c|}{\textbf{PG-19 (Eval Length=)}} & \textbf{OpenOrca} \\ \cmidrule(l){5-13} 
 &  &  &  & 128 & 1,024 & 4,096 & 16,384 & 128 & 1,024 & 4,096 & 16,384 & - \\ \midrule
Vanilla A & 128 & 128 & 128 & \textbf{3.2490} & 3.6536 & 3.6761 & 4.8414 & 4.9682 & 5.2139 & 5.1529 & 5.6949 & 6.0084 \\
Sliding Window A & 128 & 1,024 & 128 & 3.3619 & 3.1286 & 3.0766 & 3.0051 & 5.1785 & 4.8164 & 4.7510 & 4.7663 & 7.7471 \\
Vanilla B & 1,024 & 1,024 & 128 & 3.3395 & 3.3042 & 3.2856 & 3.2379 & 5.6052 & 5.0742 & 5.0797 & 5.1336 & 7.9706 \\
Vanilla B & 1,024 & 1,024 & 1,024 & 3.3395 & \textbf{2.9716} & \textbf{2.9541} & 2.9636 & 5.6052 & 5.3429 & 5.1517 & 5.0274 & 7.9706 \\
Vanilla B & 1,024 & 1,024 & 16,384 & 3.3395 & \textbf{2.9716} & 3.5534 & 3.0786 & \textbf{3.3395} & \textbf{2.9716} & 5.4912 & 5.2372 & 7.9706 \\
Sliding Window B & 1,024 & 4,096 & 1,024 & 3.4380 & 3.0197 & 2.9638 & \textbf{2.9128} & 5.0880 & 4.6587 & 4.5107 & \textbf{4.4383} & \textbf{5.8802} \\
Vanilla C & 4,096 & 4,096 & 4,096 & 3.3788 & 2.9784 & 2.9705 & 2.9518 & 5.1519 & 4.5444 & \textbf{4.4366} & 4.4938 & 5.9315 \\
Vanilla D (Upper Bond) & 16,384 & 16,384 & 16,384 & \multicolumn{4}{c|}{OOM} & \multicolumn{4}{c|}{OOM} & OOM \\ \bottomrule
\end{tabular}
}
\end{table*}


\begin{table*}[t]
\caption{Performance comparison of language models with different activation functions and position embeddings.}
\label{tab:table3}  % 设置标签
\resizebox{\textwidth}{!}{
\begin{tabular}{@{}l|c|ccccc|cccc@{}}
\toprule
\textbf{No.} &
  \textbf{\begin{tabular}[c]{@{}c@{}}Model \\ Type\end{tabular}} &
  \textbf{\begin{tabular}[c]{@{}c@{}}Activation\\ Function\end{tabular}} &
  \textbf{\begin{tabular}[c]{@{}c@{}}Position\\ Embedding\end{tabular}} &
  \textbf{\begin{tabular}[c]{@{}c@{}}Training\\ Window\end{tabular}} &
  \textbf{\begin{tabular}[c]{@{}c@{}}Training \\ Length\end{tabular}} &
  \textbf{\begin{tabular}[c]{@{}c@{}}Eval\\ Window\end{tabular}} &
  \textbf{OpenWebText} &
  \textbf{PG-19} &
  \textbf{OpenOrca} &
  \textbf{Avg.} \\ \midrule
1  & Vanilla & Softmax & RoPE        & 128  & 128  & 128  & 4.8414          & 5.6949          & 6.0085          & 5.5149          \\
2  & Vanilla & Sigmoid & RoPE        & 128  & 128  & 128  & 14.2562         & 15.4765         & 1.9906          & 10.5744         \\
3  & Sliding & Softmax & RoPE        & 128  & 1,024 & 128  & 3.0140          & 4.7839          & 6.9671          & 4.9217          \\
4  & Sliding & Sigmoid & ALiBi-12:0  & 128  & 1,024 & 128  & 3.0073          & 4.6895          & 0.1631          & 2.6200          \\
5  & Sliding & Sigmoid & ALiBi-8:4   & 128  & 1,024 & 128  & 3.0391          & 4.6435          & 0.2650          & 2.6492          \\
6  & Sliding & Sigmoid & ALiBi-6:6   & 128  & 1,024 & 128  & 3.0484          & 4.9920          & \textbf{0.1420} & 2.7275          \\
7  & Sliding & Sigmoid & ALiBi-6:6   & 128  & 2,048 & 128  & 3.0634          & 5.0384          & 0.1712          & 2.7577          \\
8  & Sliding & Sigmoid & AliRope-6:6 & 128  & 1,024 & 128  & 3.0486          & \textbf{4.3103} & 0.1709          & \textbf{2.5099} \\
9  & Sliding & Sigmoid & AliRope-6:6 & 1,024 & 1,024 & 1,024 & 2.9716          & 4.3915          & 0.5304          & 2.6312          \\
10 & Vanilla & Softmax & RoPE        & 1,024 & 1,024 & 1,024 & \textbf{2.9631} & 4.5447          & 5.4702          & 4.3260          \\
11 & Vanilla & Sigmoid & ALiBi       & 1,024 & 1,024 & 1,024 & 2.9659          & 5.0681          & 0.1717          & 2.7352          \\ \bottomrule
\end{tabular}
}
\end{table*}

\subsection{Overall Performance}



% In the overall experiment, we compare models on eight common-sense reasoning tasks in Table~\ref{tab:overall}, including Wikitext~\cite{wikitext}, Lambada~\cite{lambada}, PIQA~\cite{PIQA}, Hellaswag~\cite{Hellaswag}, WinoGrande~\cite{WinoGrande}, ARC-easy \& ARC-challenge (ARC-e \& ARC-c)~\cite{arc}, SIQA~\cite{siqa} and BoolQ~\cite{boolq}. 

In this section, we evaluate the performance of SWAT on eight commonsense reasoning benchmarks, as detailed in Appendix~\ref{app:benchmarks}. The comparison is conducted on 340M and 760M parameter models. 
For our SWAT, (-) denotes negative slopes (i.e., the negative ALiBi slope to look forward in Equation~\ref{eq:-+}); (+) denotes positive slopes, which use the opposite slope of ALiBi (i.e., the positive slope in Equation~\ref{eq:-+} looking backward); and (-+) indicates that half of the attention heads have negative slopes and half have positive slopes. 
% For our SWAT, as defined in \eqref{eq:-+}, 
% (-) denotes the configuration using only negative slopes (i.e., traditional ALiBi slopes $s_k = -2^{-k}$)
% (+) denotes the configuration using only positive slopes (i.e., $s_k = 2^{-k}$)
% (-+) denotes our bidirectional configuration where:
% Half of the attention heads ($h/2$ heads) use negative slopes $s_k = -2^{-k}$
% The other half use positive slopes $s_k = 2^{-k}$
% For both directions, $k$ ranges from 1 to $h/2$


As shown in Table~\ref{tab:overall}, SWAT (-) achieves state-of-the-art  (SOTA) performance on average (46.88\%) across eight common sense reasoning tasks, surpassing all other baselines. This is mainly attributed to the short-text benchmarks, such as PIQA and Hellaswag, where SWAT (-) focuses more on the information from newly input tokens.
Although SWAT (-) initially shows higher perplexity than other baselines at 340M parameters, when scaled to 760M parameters, it demonstrates strong decreases in perplexity on Wiki and LMB. This suggests a performance improvement trend for larger models with the sigmoid function.
On the contrary, the purely forward-looking SWAT (+) shows weaker performance, suggesting that forward slopes work best combined with backward attention. 

The balanced configuration SWAT (-+), where attention heads are evenly split between looking forward and backward, achieves more uniform performance across different tasks by effectively processing both recent and historical information. Specifically, SWAT (-+) achieves the best performance (62.11\%) on BoolQ, a question-answering dataset where historical context is crucial for accurate predictions. This result aligns with our findings in Section~\ref{ssec:ablation}, where balanced attention heads demonstrate superior performance on both OpenOrca and PG-19 datasets, confirming the importance of balanced historical information processing for complex reasoning tasks. Meanwhile, due to the allocation of some attention heads for remembering information from older tokens, SWAT (-+) shows a slight performance compromise on shorter benchmarks. However, this issue is alleviated as the model scales from 340M to 760M.
The results remain consistent at 760M parameters, showing robustness across model sizes.
% SWAT (-+) strengths on longer contexts, specifically on BoolQ passages requiring comprehension, achieve 62.11\% accuracy with the 340M model, showing enhanced long-context reasoning with its balanced bidirectional slopes. 


% As shown in Table~\ref{tab:overall}, SWAT (-) achieves state-of-the-art (SOTA) performance on avgerage(46.88\% across eight common-sense reasoning tasks) for the 340M parameter model, surpassing all other models across evaluated tasks. This is because the benchmark samples have relatively shorter text lengths, which align well with normal ALiBi slopes. 
% Although SWAT (-) initially shows higher perplexity on Wiki and LMB compared to some baselines at 340M parameters, when scaling to 760M parameters, it demonstrates strong improvement in both metrics (21.05 on LMB perplexity, matching the best performance). This suggests that the sliding window attention training becomes increasingly effective at larger model scales. 
% SWAT (-+) shows particular strengths in tasks involving longer contexts - notably on BoolQ, which primarily consists of longer passages requiring comprehensive reading comprehension, where it achieves 62.11\% accuracy with the 340M parameter model, demonstrating its enhanced long-context reasoning capabilities through balanced bidirectional slopes. 
% Meanwhile, SWAT (+) with purely positive slopes shows competitive but generally lower performance compared to the other variants, suggesting that while forward-looking attention can be beneficial, it works best when combined with traditional backward-looking attention mechanisms. 
% When scaling to 760M parameters, the results remain consistent, reinforcing the effectiveness of the approach across different model sizes.

% However, when scaling to 760M parameters, SWAT's performance slightly declines. This is because the larger model can fully memorize the 4096-length training text, reducing its reliance on information transmission. This change weakens its ability to retain long-context information, diminishing the effectiveness of the sliding window mechanism.

% Among both 340M and 760M models, SWAT demonstrates strong performance across evaluated tasks. SWAT (-) excels in tasks requiring immediate context understanding, while SWAT (+) performs well in tasks like BoolQ and SIQA, where historical context is crucial. The balanced configuration SWAT (-+) achieves uniform performance across tasks by processing both recent and long-term information. 

% For notation, we use ``-" to denote conventional ALiBi with negative slopes where attention heads focus on earlier tokens, ``+" for our reversed ALiBi with positive slopes that attend to newer tokens, and ``-+" indicates balanced slopes where half the heads use negative slopes and half use positive slopes.

% Among 340M models, SWAT achieves the best overall performance across the evaluated tasks with an average score of 46.88.Different slope configurations of SWAT show distinct characteristics. SWAT (-) with conventional ALiBi negative slopes demonstrates strong overall performance, particularly excelling in tasks requiring immediate context understanding. SWAT (+), which focuses more on historical information through positive slopes, shows advantages in tasks where background context plays a crucial role, such as BoolQ and SIQA. The balanced configuration SWAT (-+), where attention heads are evenly split between looking forward and backward, achieves more uniform performance across different tasks by effectively processing both recent and historical information. 

% Specifically, SWAT (-+) achieves the best performance (62.11\%) on BoolQ, a question-answering dataset where historical context is crucial for accurate predictions. This result aligns with our findings in Section~\ref{ssec:ablation}, where balanced attention heads demonstrate superior performance on both OpenOrca and PG-19 datasets, confirming the importance of balanced historical information processing for complex reasoning tasks.

% Among 760M models, SWAT continues to demonstrate strong performance. SWAT (-) achieves the highest overall accuracy, excelling particularly in tasks that require immediate context comprehension. SWAT (+) performs well in tasks like WinoGrande and BoolQ, where broader contextual reasoning is essential. SWAT (-+) maintains stable and competitive results across various tasks, leveraging both recent and long-term information processing to ensure robust generalization.

% \textbf{Scaling to 1.3B Parameters.}


% Overall, SWAT stands out not only for its impressive performance on common reasoning benchmarks but also for its consistent success across different tasks. Insights from Table~\ref{tab:overall} show that SWAT's innovative attention mechanisms, and for slope modifications, yield measurable benefits in commonsense reasoning tasks. Moreover, its competitive performance against state-of-the-art baselines like Titans and Gated DeltaNet suggests that SWAT is well-suited for real-world applications requiring robust reasoning capabilities. common-sense reasoning capabilities.



\subsection{Sliding Window Attention Training}
\label{ssec:swat}

To verify the effectiveness of SWA training, we conduct experiments comparing vanilla Transformers pre-trained with and without SWAT training across three datasets. Using Llama2-based models~\cite{llama2} pretrained on OpenWebText, we investigate the impact of varying sliding window sizes and sequence lengths, with results shown in Table~\ref{tab:performance_comparison}. In the table, vanilla Transformers are which training length are the same as their training window size, and the labels A, B, C, and D represent the model identifiers. 

When the sliding window mechanism is applied, we observe a notable improvement in performance, particularly with longer evaluation sequence lengths. For instance, in the Sliding Window A configuration, when the evaluation length is 16,384, Sliding Window A achieves a performance of 3.0051 on OpenWebText, surpassing the 4.8414 achieved by Vanilla A. Additionally, Sliding Window B achieves the best performance across all three datasets when the evaluation length is 16,384. Note that all results are from models trained for 80,000 steps. If training continues, the attention sink issue is likely to worsen, further degrading vanilla model performance.

Based on our experimental results, we draw two key conclusions: 
% 1) Vanilla transformer models (Vanilla A, B, and C) trained with different sequence lengths demonstrate optimal performance primarily on sequences matching their training length. Their performance degrades notably when processing sequences longer than their training length, indicating a clear length-dependent behavior.
% 2) While SWA pretrained models show slightly higher loss than vanilla transformers trained on specific lengths, they exhibit more stable performance across varying input lengths. The model performance improves and stabilizes as text length reaches the sliding window's coverage range, suggesting better generalization to longer sequences without input length constraints. 
% 3) Despite the benefits of sliding window training, the inherent limitations of the transformer architecture still lead to information loss. This is evidenced by consistently high loss values on the OpenOrca dataset, which likely stems from information loss caused by the softmax operation.
(1) Wtih the same model structure, SWA training significantly improves performance, especially with longer evaluation sequence lengths. This is likely because SWA training forces the model to retain memory of older information across long sequences, while vanilla models struggle with memory as they retain all historical tokens.
(2) The vanilla Transformers perform optimally only when the evaluation length matches the training length, whereas the SWA trained models maintain consistent performance across varying sequence lengths. This is likely because vanilla Transformers heavily attend to initial tokens due to attention sink, while SWA models learn to focus primarily on the current window, ensuring stable performance across different sequence lengths.




\begin{figure}[t]
    \centering
    \includegraphics[width=\linewidth]{imgs/figure5.pdf}
    \caption{The training loss of models with different modules including Sigmoid, RoPE, and ALiBi, with the balanced slopes.}
    \label{fig:loss}
\end{figure}



\subsection{Ablation Study}
\label{ssec:ablation}



This section evaluates the impact of activation functions, position embeddings, and ALiBi slopes.
We systematically test 11 different configurations (No.1-11) to understand how different combinations of model components affect long-context performance, as shown in Table~\ref{tab:table3} and Figure~\ref{fig:loss}.


Comparing No.1 and No.2, directly replacing softmax with sigmoid in vanilla Transformer leads to significant performance degradation, likely due to overloaded information in token embeddings without mutual suppression. However, using ALiBi stabilizes training by distinguishing subtle differences in token embeddings based on position information (No.10 and No.11). Furthermore, the slope configuration plays a key role, with No.5 and No.6 outperforming No.4, suggesting a better balance between recent and past information. However, Figure~\ref{fig:loss} shows that training instability persists at later stages (ALiBi-6:6 Sigmoid), indicating that ALiBi alone provides weak positional information. AliRope-6:6 Sigmoid (No.8) achieves the lowest loss values among all variants, with 2.51 on average, while demonstrating more stable training pattern as shown in Figure~\ref{fig:loss}. Finally, comparing No.7 and No.6, extending the training length from 1,024 to 2,048 while keeping the number of layers and window size fixed does not help with the loss.

% Table~\ref{tab:table3} presents results from pretraining models with varying configurations, while Table~\ref{fig:loss} visualizes validation curves for several representative models.



% \paragraph{Activation Functions.}
% Models using the sigmoid activation function perform worse. For instance, Vanilla (Sigmoid+RoPE) shows higher loss values across all tasks compared to Vanilla (Softmax+RoPE), particularly on the OpenWebText and PG-19 datasets, where the losses are 14.2562 and 15.4765, respectively, versus 4.8414 and 5.6949 for Vanilla (Softmax+RoPE). This indicates that sigmoid function causes information overload in hidden states, making it difficult to extract key features for the next token predictions.



% Models using the sigmoid activation function perform worse. For instance, Vanilla (Sigmoid+RoPE) shows higher loss values across all tasks compared to Vanilla (Softmax+RoPE), particularly on the OpenWebText and PG-19 datasets. This indicates that the sigmoid function causes information overload in hidden states, making it difficult to extract key features for the next token predictions.

% \paragraph{Position Embeddings.} 

% Using ALiBi stabilizes training by introducing position-dependent biases that help differentiate token embeddings. The slope configuration plays a key role, with No.5 and No.6 outperforming No.4, suggesting a better balance between local and global dependencies. 
% However, training instability persists at later stages, indicating that ALiBi alone provides weak positional information. No.8 reduces these fluctuations, leading to more stable training and improved performance with a loss of 4.3103 on PG-19.

% When using ALiBi, the training process becomes stable, possibly because ALiBi introduces position-dependent biases that help differentiate token embeddings at different positions, enabling more complex representations to complement the sigmoid function. Meanwhile, the slope configuration (e.g., ALiBi-12:0, ALiBi-8:4, ALiBi-6:6) significantly influences model performance. Both ALiBi-6:6 and ALiBi-8:4 configurations perform better than ALiBi-12:0, which suggests that a balanced attention head configuration—where half of the heads focus forward and half backward—provides a better trade-off between local and global dependencies, enabling the model to memorize more information from long-context sequences.

% Although the combination of the sigmoid function and the balanced ALiBi already achieves promising results, we observe training instability even at the late training stages(as shown in Figure~\ref{fig:loss}). This suggests that ALiBi alone provides relatively weak positional information. After incorporating RoPE into our model (Sigmoid+AliRope-6:6), these performance fluctuations are significantly reduced, leading to more stable training and better performance with a loss of 4.3103 on PG-19.




% Moreover, the performance of Sigmoid+AliRope-6:6 configuration stands out, achieving a loss of 4.3103 on PG-19, the best across all models. This demonstrates that adjusting the slope configuration to allow flexible capturing of dependencies at different ranges improves model effectiveness.

% RoPE (Rotary Position Embedding)~\cite{rope} is a position encoding method that typically improves model performance. For example, Vanilla (Softmax+RoPE) and Sliding (Softmax+RoPE) demonstrate lower loss values, especially on the OpenWebText and PG-19 datasets, suggesting that RoPE helps the model better capture positional relationships in long-range dependencies.
% ALiBi (Attention with Linear Biases)~\cite{alibi} directly introduces position biases into the attention matrix. When used in Sliding (Sigmoid+ALiBi), particularly with the ALiBi-6:6 configuration, the model shows a significant reduction in loss, especially on OpenOrca, where the loss drops to 0.1420, outperforming other configurations. This indicates that ALiBi effectively captures positional information in long text sequences when properly configured.


% In summary, models without sliding windows struggle with long texts, particularly in tasks involving multi-paragraph or cross-sentence reasoning. In contrast, sliding window techniques ensure stable performance. The best results are obtained when half of the attention heads look forward and half look backward, effectively balancing local and global dependencies. These findings underscore that a balanced activation function, appropriate ALiBi configuration, and an effective sliding window strategy are key to improving Transformer models’ performance in long text processing.

% \subsection{Attention Score Visualization}
% % 这段现在可能不能实现了,因为
% % 随便写的,随便删
% The objective of this experiment is to analyze the interpretability of the attention mechanism in models trained with a sliding window approach. The setup involves comparing the attention scores with the corresponding input text. Theoretical results suggest that in models trained with sliding windows, the attention sink shifts from the initial tokens to key tokens located in the middle of the text, such as newline characters ('\\n').






\section{Related Works}
\label{related-works}

\subsection{Efficient Transformers}


While architectural innovations offer one path to efficiency, research also focuses on optimizing the Transformer itself, particularly through sparse attention patterns to reduce computational cost.

Early work in this direction focused on structured sparsity patterns. Sparse Transformer~\cite{sparsetransformer} demonstrated that using fixed sparse attention patterns could maintain model performance while significantly reducing computation. This idea was further developed by Longformer~\cite{Longformer} and BigBird~\cite{bigbird}, which introduced more sophisticated attention patterns combining local windows with global tokens to capture dependencies effectively. 
These models, however, still rely on predefined attention patterns, which can limit flexibility. \swt
% Our work builds upon these insights but takes a fundamentally different approach. Rather than adapting pre-trained models for sliding window inference, we address the attention sink problem directly during training, enabling simpler and more efficient inference without the need for complex token retention strategies.

\subsection{Efficient LLMs}

To address the quadratic complexity of Transformers, researchers have proposed various efficient models categorized into the following categories:

\textbf{Linear Recurrent Models} achieve $O(n)$ complexity through different approximation techniques. Linear Transformer~\cite{lineartransformer} replaces softmax attention with kernel functions, while Performer~\cite{performers} employs random feature approximation. Recent works like GLA~\cite{gla} introduce forgetting mechanisms to prevent information explosion, while Gated Delta Networks~\cite{gateddeltanet} focus memory updates to enable both precise memory updates and quick resets when needed. Models like Mamba~\cite{mamba} and RWKV~\cite{rwkv} take a fundamentally different approach by utilizing state space models (SSMs) instead of attention, providing an alternative way to capture sequential patterns.

\textbf{Memory-Augmented Architectures} enhance Transformers' ability to handle long sequences by incorporating explicit memory mechanisms. For example, Transformer-XL~\cite{transformer-xl} pioneered the use of cached computations from previous segments with relative positional embeddings. More recent works like Memorizing Transformers~\cite{memorizingtransformers} and Focused Transformer~\cite{focusedtransformer} try to store and retrieve relevant historical information.

While these models achieve better efficiency, their complex architectures often lead to more challenging optimization compared to standard Transformers, which benefit from simple and well-established training procedures.



% StreamingLLM~\cite{streamingllm} and LM-Infinite~\cite{lm-infinite} found that maintaining a small set of initial tokens within the sliding window could preserve model performance, which revealed the attention sink phenomenon. Further analysis found that removing normalization operations eliminates the attention sink~\cite{whensinkemerge}.


% While these approaches have shown promising results in improving the efficiency of transformers, they often come at the cost of increased architectural complexity. Many introduce sophisticated memory mechanisms or hybrid architectures that can be challenging to implement and optimize in practice. This growing complexity motivates our exploration of simpler, more practical approaches to handling long sequences in transformers.






\section{Conclusion }\label{sec:conclusion}


In this work, we introduce a two-step mechanism designed to enhance using LLM-as-a-Judge. Our approach incorporates a weighting algorithm into the prompt, which guides the LLM during the second-step evaluation process.  We conduct a case study using a dataset specialized in software engineering and evaluate its performance across five widely used LLMs. Our proposed method achieves an average improvement of 6\% in HAR. Particularly, Mixtral-8x7B Instruct emerged as the clear winner in this competition, achieving a 95.8\% HAR, outperforming all other LLMs, including two customized Llama models.

% Meanwhile, we found Mixtral-8x7B Instruct preforms best in this task.



\section{Limitations}

Our case study demonstrates the effectiveness of the proposed method on a single dataset, and the prompts used in this study were manually designed to address the unweighted evaluation issue. While the results are promising, the approach may face scalability challenges until it is tested on a broader range of datasets. Additionally, future work should explore the use of auto-generated prompts to improve efficiency and reduce reliance on manual design. AI assistant was utilized in the writing process.



% \section*{Acknowledgments}



% Bibliography entries for the entire Anthology, followed by custom entries
%\bibliography{anthology,custom}
% Custom bibliography entries only
\bibliography{custom}

\section{Theoretical Analysis}
\subsection{Notations}
We dedicate Table~\ref{tab:Notation} to index the notations used in this paper. Note that every notation is also defined when it is introduced.
\begin{table*}[h!]
\caption{Notations.}\label{tab:Notation}%\\
\centering  
% \resizebox{\textwidth}{!}{
\begin{tabular}{l l l}
\toprule
 $\gG$ &$\triangleq$ & Input graph with a vertex set $\gV$, an edge set $\gE$, and features $\mX$\\
 $\boldsymbol{A}$ &$\triangleq$ & Adjacency matrix of $\gG$\\
 $\gE$ & $\triangleq$ & Edges of $\gG$\\
 $\gV$ & $\triangleq$ & Nodes of $\gG$\\
 $\mX$ & $\triangleq$ & Matrix containing node features of $\gG$\\
 $\vy$ & $\triangleq$ & Vector of node labels of $\gG$\\
 $C$ & $\triangleq$ & An ordered set containing all possible node labels of $\gG$\\
 $F$ & $\triangleq$ & Dimension of node features in $\gG$\\
 $L$ & $\triangleq$ & Number of GNN layers\\
 $H$ & $\triangleq$ & Node embedding dimension\\ 
 ${\mH}$ & $\triangleq$ & Node embedding matrix\\
 $\vh_u$ & $\triangleq$ & Embedding of node u\\
 $\vw$ & $\triangleq$ & Vector of edge weights in  $\gG$\\
 $q$ & $\triangleq$ & Ratio of \# edges in sparse graph and \# edges in input graph in \%\\
 $k$ & $\triangleq$ & \# edges in the sparse graph, $k\triangleq\floor{\frac{q|\gE|}{100}}$\\ 
 $\tilde{p}$ & $\triangleq$ & Learned probability distribution by \sgs \\
 $\tilde{\gE}$ & $\triangleq$ & Set of edges sampled from $\gE$ by \sgs following $\tilde{p}$\\
$\tilde{\gG}$ &$\triangleq$ & Sparse subgraph $(\gV,\tilde{\gE},\mX)$ constructed by \sgs \\  
 $\mA_{\tilde{\gG}}$ or $\tmA$ & $\triangleq$ & Adjacency matrix of $\tilde{\gG}$\\
 $\tilde{\vw}$ & $\triangleq$ & Edge weight of sparse graph learned by \sgs \\
 $p_\mathrm{prior}$ & $\triangleq$ & Probability distribution of a fixed prior on $\gG$ \\
  $\tilde{p}_a$ & $\triangleq$ & Augmented learned probability distribution  \\
 $p^*$ & $\triangleq$ & True probability distribution known by the idealized learning ORACLE\\
 ${\gE^*}$ & $\triangleq$ & Set of edges sampled from $\gE$ by the learning ORACLE following distribution $p^*$\\   
 $\gG^*$ &$\triangleq$ & True sparse subgraph $(\gV,\gE^*,\mX)$ constructed by the learning ORACLE \\
 $\mA_{\gG^*}$ or $\mA^*$ & $\triangleq$ & Adjacency matrix of $\gG^*$\\
 % $\gG^*$ &$\triangleq$ & True sparse subgraph constructed by the idealized learning ORACLE \\
 % $\mA_{\gG^*}$ or $\mA^*$ & $\triangleq$ & Adjacency matrix of $\gG^*$\\
 $\gL_\mathrm{CE}$ & $\triangleq$ & Cross entropy loss\\
 $\gL_\mathrm{assor}$ & $\triangleq$ & Assortative loss\\
 $\gL_\mathrm{cons}$ & $\triangleq$ & Consistency loss\\
$\gL$ & $\triangleq$ & Total loss\\ 
 
 
 
 
 
 
 \bottomrule
\end{tabular}
% }
\end{table*}
\subsection{Bounding \#common edges wrt. true subgraph}
\label{theo:commonedges}
Let $\mathcal{E}^*$ and $\mathcal{\tilde{E}}$ denote the ordered collection of edges sampled by the idealized learning ORACLE according to true distribution $p^*$ and by \sgs according to learned probability $\tilde{p}$ respectively. For analytical convenience, let us assume that both learning algorithms sample $k = \floor{q|\mathcal{E}|/100}$ edges with replacement independently.
 
% We will first show that $\mathbf{Pr}(\mathcal{E}_i^* = \mathcal{\tilde{E}}_i) \geq \min_j (p^*_i, \tilde{p}_i)$ and $\min(p^*_i,\tilde{p}_i) = 1- \frac{1}{2} \|\tilde{p} - p^*\|_1$. These two results will lead us to prove that $\mathbf{Pr}(\mathcal{E}_i^* = \mathcal{\tilde{E}}_i) \geq 1 - \frac{\epsilon}{2}$. 
First, we will prove lemma~\ref{lem:singleedge}, which show that the probability of an edge chosen by \sgs coincides with that chosen by the ORACLE has a lower bound. Finally, we will prove one of the main results (Theorem~\ref{theo:commonedges}), which shows that given $q \in [0,100]$, we can lower-bound the expected number of common edges between \sgs and the learning ORACLE. 

\begin{lemma} 
\label{lem:singleedge}
For any arbitrarily chosen $i \in \{1,2,\ldots, k\}$
\[
\mathbf{Pr}(\mathcal{E}_i^* = \mathcal{\tilde{E}}_i) \geq \sum_{j=1}^{|\mathcal{E}|} \frac{(p^*_j + \tilde{p}_j - \epsilon)^2}{4},
\]
where $k = \floor{q|\mathcal{E}|/100}$ and $0 \leq q \leq 100$ is a user-specified parameter and $\epsilon\in [0,1]$ is the error.
\end{lemma}

\begin{proof} We prove the above lemma in two parts.


\paragraph{Part 1: Universal approximation of probability distribution over edges.}
%\label{tho:uap}
The Universal Approximation Theorem~\cite{cybenko1989approximation,augustine2024survey} states that a feed-forward neural network with at least one hidden layer and a finite number of neurons can approximate any continuous function $f: \mathbb{R}^n \rightarrow \mathbb{R}$ on a compact subset of $\mathbb{R}^n$, given a suitable choice of weights and activation functions. 

In our case, $p^* = f$ is the true edge probability distribution for the downstream task, $\tilde{p} = f_{\text{MLP},\phi}$ is the learned approximate distribution and $\vx_e$ is a vector of edge features, for instance, $\vx_e =  ((\vh_u - \vh_v) \oplus (\vh_u \odot \vh_v))$ as used in equation~\ref{eq:w_uv}. The following universal approximation property holds for the module I component of \sgs,
\begin{equation}
\label{eq:uapp}
\sup_{e \in \mathcal{E}} \|\tilde{p}(\vx_e) - p^*(\vx_e)\|_1 \leq \epsilon.
\end{equation}
 Here, we have two underlying assumptions: (i) the optimal distribution $p^*$ is a function of node features $\mX$ and (ii) $\mX$ is a compact subset (bounded and closed) of Euclidean space $\mathbb{R}^n$. The first assumption is made to simplify the problem. The second assumption is quite practical since the node features are typically normalized. Hence, we can show that the embeddings $\vh_u,\vh_v$, which are continuous images of $\mX$, are also compact due to the extreme value theorem. As a result, the edge features $\vx_e$ which, in a sense, \emph{lifts} the end-point node features into higher-dimensional Euclidean space are also compact. The approximation error $\epsilon$ can be made arbitrarily small by increasing the capacity of the MLP, e.g., adding more neurons or layers. 

\paragraph{Part 2: Common edges wrt. optimal subgraph.}

The event $\mathcal{E}_i^* = \mathcal{\tilde{E}}_i$ means that both $\mathcal{E}_i^*$ and $\mathcal{\tilde{E}}_i$ contain the same edge. But there are $|\mathcal{E}|$ such candidates. Hence, the probability of this event is given by,

\begin{align*}
    \mathbf{Pr}(\mathcal{E}_i^* = \mathcal{\tilde{E}}_i) &= \sum_{j=1}^{|\mathcal{E}|} \mathbf{Pr}(\mathcal{E}_i^* = \mathcal{E}_j \land \mathcal{\tilde{E}}_i = \mathcal{E}_j), \\
    &= \sum_{j=1}^{|\mathcal{E}|} \mathbf{Pr}(\mathcal{E}_i^* = \mathcal{E}_j) \cdot \mathbf{Pr}(\mathcal{\tilde{E}}_i = \mathcal{E}_j), \\
    & = \sum_{j=1}^{|\mathcal{E}|} p^*_j \cdot \tilde{p}_j, \\
    &\geq \sum_{j=1}^{|\mathcal{E}|} \frac{(p^*_j + \tilde{p}_j - |p^*_j - \tilde{p}_j|)^2}{4}, \\
    & \geq \sum_{j=1}^{|\mathcal{E}|} \frac{(p^*_j + \tilde{p}_j - \epsilon)^2}{4}.
\end{align*}
The second line follows since the optimal sampler is a different algorithm independent from the sampler used in \sgs. The last line follows because $\|p^*_j - \tilde{p}_j\|_1 \leq \epsilon \implies |p^*_j - \tilde{p}_j| \leq \epsilon$ (from eq.~\ref{eq:uapp}). 
\end{proof}

% \begin{lemma}
% \[
%     \min(p^*_i,\tilde{p}_i) = 1- \frac{1}{2} \|\tilde{p} - p^*\|_1
% \]
% \end{lemma}
% \begin{proof}
%     It is known (for instance, see~\cite{xie2024distributionally}) that the total variation distance $d_{TV}(\tilde{p},p^*)$ satisfies
%     \[
%     d_{TV}(\tilde{p},p^*) = \frac{1}{2}\|\tilde{p} - p^*\|_1
%     = 1 - \min({\tilde{p},p^*})
%     \]
% \end{proof}

We have the following theorem that lower-bounds the number of common edges with respect to the optimal sampler $|\mathcal{E} ^* \cap \mathcal{\tilde{E}}|$: 
\begin{theorem}[Lower-bound]
\begin{equation}
\mathbb{E}[|\mathcal{E}^* \cap \mathcal{\tilde{E}}|] \geq k \sum_{j=1}^{|\mathcal{E}|} \frac{(p^*_j + \tilde{p}_j - \epsilon)^2}{4},
\end{equation}
where $k = \floor{q|\mathcal{E}|/100}$ and $0 \leq q \leq 100$ is a user-specified parameter.
\end{theorem}
\begin{proof}
    Since we are drawing $k$ edges independently at random, the theorem follows by applying the linearity of expectation on the following:
\begin{align*}
\mathbb{E}[|\mathcal{E}^* \cap \mathcal{\tilde{E}}|] = \mathbb{E}[\sum_{i=1}^k \mathbb{I}(\mathcal{E}_i^* = \mathcal{\tilde{E}}_i)] &= \sum_{i=1}^k \mathbf{Pr}(\mathcal{E}_i^* = \mathcal{\tilde{E}}_i) \\
& = k\cdot \mathbf{Pr}(\mathcal{E}_i^* = \mathcal{\tilde{E}}_i)\\
& \geq k \sum_{j=1}^{|\mathcal{E}|} \frac{(p^*_j + \tilde{p}_j - \epsilon)^2}{4}
\end{align*}
\end{proof}
This theorem shows that the expected number of common edges between the sample subgraph obtained by \sgs $\mathcal{\tilde{G}}$ and the true optimal sample subgraph $\mathcal{G}^*$ is non-trivial. 

% \paragraph{The implication of the lower-bound.} 
% (1) Suppose, the true distribution is uniform. In the best case scenario $\epsilon \rightarrow 0$ and $\tilde{p} = p^* = \frac{1}{|\mathcal{E}|}$. Thus there are at least $\frac{k}{|\mathcal{E}|}$ common edges between $\tilde{\gG}$ and $\gG^*$. However, since $k < \abs{\mathcal{E}}$, the lower-bound of $\mathbb{E}[|\mathcal{E}^* \cap \mathcal{\tilde{E}}|] \geq \frac{k}{|\mathcal{E}|}$ is not very useful even though the learned distribution is accurate. This suggests that \emph{learning the optimum uniform distribution is less likely to produce the optimum sparse subgraph}. 

% (2) Suppose, the true distribution is the Dirac distribution (often called the $\delta$ distribution) where all probability mass is concentrated on a single edge. In other words, suppose $\tilde{p} = p^* = \delta_{ij}$ where $\delta_{ij}$ is the Kronecker-delta. In such a skewed distribution, as $\epsilon \rightarrow 0$, the lower bound reduces to 
% \[
% \mathbb{E}[|\mathcal{E}^* \cap \mathcal{\tilde{E}}|] \geq k \sum_{j=1}^{|\mathcal{E}|} (\tilde{p}_j)^2 = k.
% \]
% This identity suggests that the sampled edges are expected to 100\% overlap with the true, optimal sparse subgraph.

\begin{theorem}[Upper-bound]
\begin{equation}
\mathbb{E}[|\mathcal{E}^* \cap \mathcal{\tilde{E}}|] \leq k (1 - \frac{\|p^* - \tilde{p}\|_1}{2}), 
\end{equation}
where $k = \floor{q|\mathcal{E}|/100}$ and $0 \leq q \leq 100$ is a user-specified parameter.
\end{theorem}
\begin{proof}
\begin{align*}
    \mathbf{Pr}(\mathcal{E}_i^* = \mathcal{\tilde{E}}_i) &= \sum_{j=1}^{|\mathcal{E}|} p^*_j \cdot \tilde{p}_j \\
    & \leq \sum_{j=1}^{|\mathcal{E}|} \min(p^*_j,\tilde{p}_j) \\
    &= 1 - d_{TV}(p^*,\tilde{p}) \\
    &= 1 - \frac{1}{2} \|p^* - \tilde{p}\|_1    
\end{align*}
\end{proof}
Here $d_{TV}$ is the total variation distance. The result used in the last line regarding $d_{TV}$ can be found in~\citet{xie2024distributionally}. 

\paragraph{The implication of the upper-bound.} 
When $\tilde{p} \rightarrow p^*$, the norm $\|p^* - \tilde{p}\|_1 \rightarrow 0$; therefore, the number of common edges could be close to $k$.

\subsection{Upper-bounding the error in the learned Adjacency matrix} 
With the bound proven earlier on the \#common edges by the sparse subgraph of \sgs with that by a learning ORACLE, in this section, we want to obtain an upper-bound on the error in terms of the norm of the Adjacency matrices. As adjacency matrices are used by GNNs for computing node embeddings, such result is important for obtaining error bound on the embeddings later on.

Let $\mA_{\tilde{\gG}}$ and $\mA_{\gG^*}$ be the corresponding adjacency matrices of the learned sparse graph $\tilde{\gG}$ and true optimal sparse graph $\gG^*$. The dimension of these matrices is the same as the input adjacency matrix $\mA_{\mathcal{G}}$ except that $\mA_{\mathcal{G}}$ is denser. Let us also denote the Frobenius norm of a matrix $\mA$ as $\|\mA\|_F$ and the spectral norm of $\mA$ as $\|\mA\|_2$. The Frobenius norm of $\mA$ is defined as $\sqrt{\sum_{ij} \mA^2_{ij}}$, whereas the spectral norm of $\mA$ is the largest singular value $\sigma_{max}(\mA)$ of $\mA$.


Since \sgs do not know the true probability distribution $p^*$, error is introduced in the learned adjacency matrix $\mA_{\tilde{\gG}}$ of the downstream sparse subgraph. We are interested in analyzing the expected error introduced in $\mA_{\tilde{\gG}}$ in terms of the spectral norm, to be precise, $\mathbb{E}[\|\mA_{\tilde{\gG}} - \mA_{\gG^*}\|_2]$. To this end, we will exploit the lower bound derived in Theorem 1 and the fact that $\|\mA\|_2 \leq \|\mA\|_F$. 

\begin{lemma}[Error in Adjacency matrix approximation] Let $\mA_{\tilde{\gG}}$ and $\mA_{\gG^*}$ be the corresponding adjacency matrices of the learned sparse graph $\tilde{\gG}$ and true optimal sparse graph $\gG^*$. If the downstream sampler sampled $k$ edges independently at random (with replacement) to construct those matrices following their respective distributions $\tilde{p}$ and $p^*$, then 
    \[
    \mathbb{E}[\|\mA_{\tilde{\gG}} - \mA_{\gG^*}\|_2] \leq \sqrt{2k(1-\sum_{j=1}^{\abs{\mathcal{E}}} \frac{(p^*_j + \tilde{p}_j - \epsilon)^2}{4})},
    \]
    where $k = \floor{q|\mathcal{E}|/100}$ and $0 \leq q \leq 100$ is a user-specified parameter.
\end{lemma}
\begin{proof}
Since the entries in adjacency matrices are either $0$ or $1$, the difference $\mA_{\tilde{\gG}}(i,j) - \mA_{\gG^*}(i,j)$ are in $\{-1,0,1\}$ for all $i,j$. The following holds by definition of Frobenus norm,

\[
\|\mA_{\tilde{G}} - \mA_{G^*}\|^2_F = \sum_{ij}(\mA_{\tilde{\gG}}(i,j) - \mA_{\gG^*}(i,j))^2.
\] 
As a result, only the non-zero entries in $\mA_{\tilde{\gG}} - \mA_{\gG^*}$ contribute to the square of Frobenius norm $\|\mA_{\tilde{G}} - \mA_{G^*}\|^2_F$.
The expected number of non-zero entries in $\|\mA_{\tilde{\gG}} - \mA_{\gG^*}\|^2_F$ corresponds to the expected cardinality $\abs{(\mathcal{\tilde{E}} \setminus \mathcal{E}^*) \cup (\mathcal{E}^* \setminus \mathcal{\tilde{E}})}$. Thus

\begin{align*}
    \mathbb{E}[\|\mA_{\tilde{\gG}} - \mA_{\gG^*}\|^2_F] &= \mathbb{E}[\abs{(\mathcal{\tilde{E}} \setminus \mathcal{E}^*) \cup (\mathcal{\tilde{E}} \setminus \mathcal{E}^*)}] \\ 
    &= \mathbb{E}[\abs{\mathcal{\tilde{E}}} + \abs{\mathcal{E}^*} - 2 \abs{\mathcal{\tilde{E}} \cap \mathcal{E}^*}] \\
    &= 2k - 2\mathbb{E}[\abs{\mathcal{\tilde{E}} \cap \mathcal{E}^*}] \\
    &\leq 2k - 2k \sum_{j=1}^{|\mathcal{E}|} \frac{(p^*_j + \tilde{p}_j - \epsilon)^2}{4} \\
    &= 2k (1 - \sum_{j=1}^{|\mathcal{E}|} \frac{(p^*_j + \tilde{p}_j - \epsilon)^2}{4}).
\end{align*}
Applying Jensen's inequality for convex functions, in particular, applying $(\mathbb{E}[\rX])^2 \leq \mathbb{E}[\rX^2])$ yields,
\begin{align*}
     (\mathbb{E}[\|\mA_{\tilde{\gG}} - \mA_{\gG^*}\|_F])^2 &\leq  \mathbb{E}[\|\mA_{\tilde{\gG}} - \mA_{\gG^*}\|^2_F] \\
     &\leq 2k (1 - \sum_{j=1}^{|\mathcal{E}|} \frac{(p^*_j + \tilde{p}_j - \epsilon)^2}{4}).
\end{align*}
Taking square-root on both sides yields,
\[
 \mathbb{E}[\|\mA_{\tilde{\gG}} - \mA_{\gG^*}\|_F] \leq \sqrt{2k (1 - \sum_{j=1}^{|\mathcal{E}|} \frac{(p^*_j + \tilde{p}_j - \epsilon)^2}{4})}.
\]
We obtain the theorem using the following relation between the Frobenius and spectral norms.
\begin{align*}
    \|\mA_{\tilde{\gG}} - \mA_{\gG^*}\|_2 &\leq \|\mA_{\tilde{\gG}} - \mA_{\gG^*}\|_F \\
    \implies \mathbb{E}[\|\mA_{\tilde{\gG}} - \mA_{\gG^*}\|_2] &\leq \mathbb{E}[\|\mA_{\tilde{\gG}} - \mA_{\gG^*}\|_F] \\
    &= \sqrt{2k (1 - \sum_{j=1}^{|\mathcal{E}|} \frac{(p^*_j + \tilde{p}_j - \epsilon)^2}{4})}
\end{align*}
\end{proof}
% \begin{corollary} Let us assume $\epsilon \rightarrow 0$ and the model learned the true pmf. Then the spectral norm approximation error is
%     \[
%     \mathbb{E}[\|\mA_{\tilde{\gG}} - \mA_{\gG^*}\|_2] \leq \sqrt{2k(1-\frac{1}{4\abs{\mathcal{E}}})}
%     \]
%     where $k = \floor{q|\mathcal{E}|/100}$ and $0 \leq q \leq 100$ is a user-specified parameter.
% \end{corollary}
% \begin{proof}
% Since we assumed $\epsilon \rightarrow 0$ and $p^*_j = \tilde{p}_j$, Theorem 3 reduces to
% \[
% \mathbb{E}[\|\mA_{\tilde{\gG}} - \mA_{\gG^*}\|_2] \leq \sqrt{2k(1- \frac{\sum_{j=1}^{\abs{\mathcal{E}}} (p^*_j)^2}{4})}
% \]
% For any probability mass function the following inequality holds
% $\sum_{i=1}^n p^2_i \geq 1/n$. This holds with equality when the distribution is uniform. Thus,
% \[
% \sum_{j=1}^{\abs{\mathcal{E}}} (p^*_j)^2 \geq \frac{1}{\abs{\mathcal{E}}}\\
% \implies \mathbb{E}[\|\mA_{\tilde{\gG}} - \mA_{\gG^*}\|_2] \leq \sqrt{2k(1-\frac{1}{4\abs{\mathcal{E}}})}
% \]
% \end{proof}
\subsection{Upper-bounding the error in the predicted node embeddings}
\label{theo:gcnembed}
We consider vanilla GCN as proof of concept to understand how the changes in the sparse subgraph affect the node embeddings produced by a trained GCN. Our goal is to analyze the respective encodings produced by an $L$-layer GCN when the input subgraphs are $\gG^*$ (corresponding to $\mA_{\gG^*}$) and $\tilde{\gG}$ (corresponding to $\mA_{\tilde{\gG}}$) respectively. For simplicity, we will shorten the matrices $\mA_{\gG^*}$ as $\mA^*$ and $\mA_{\tilde{\gG}}$ as $\tmA$. 

A single GCN layer is defined as,

\[
\mH^{(l+1)} = \sigma(\hat{\mA}\mH^{(l)}\mW^{(l)}),
\]
where $\hat{\mA} = \mD^{-1/2}\mA\mD^{-1/2}$ is the normalized adjacency matrix, $\mH^{(l)}$ is the input to the $l$-th layer with $\mH^{(0)} = \mX$, $\mW^{(l)}$ is the learnable weight matrix for $l$-th layer and $\sigma$ is non-linear activation function. Let us suppose an $L$-layer GCN produces embeddings $\tmH^{(L)}$ and $\mH^{*(L)}$ when it takes sparse matrices $\tmA$ and $\mA^*$ as input. We want to upper-bound,
\[
\mathbb{E}[\normLtwo{\tmH^{(L)} - \mH^{*(L)}}],
\]
in other words, the loss in the downstream node encodings is due to using our learned subgraph. 

\paragraph{Assumptions.} We assume that for all $l$, $\normLtwo{\mW} \leq \alpha < 1$ where $\alpha$ is a constant no more than 1. This is reasonable since each $\mW^{(l)}$ is typically controlled during training using regularization techniques, e.g., weight decay. Assuming that the input features in $\mX$ are bounded, we can also assume that there exists a constant $\beta$ such that $\forall l$, $\normLtwo{H}^{(l)} \leq \beta$. We also assume that $\sigma$ is \emph{Lipschitz continuous} with \emph{Lipschitz constant} $L_\sigma$; for instance,  activation functions such as \relu, sigmoid, or tanH are Lipschitz continuous. In particular, we assume \relu activation for our theoretical analysis because \relu has \emph{Lipschitz constant} $L_\sigma = 1$, which simplifies our analysis.

Under these assumptions, we have the following theorem,
\begin{theorem}[Error in GCN encodings]
For sufficiently deep L-layer GCN (large L), the error 
{
\[
\mathbb{E}[\lim_{L \to \infty} \normLtwo{\tmH^{(L)} - \mH^{*(L)}}] < \frac{\beta}{1-\alpha}\sqrt{2k (1 - \sum_{j=1}^{|\mathcal{E}|} \frac{(p^*_j + \tilde{p}_j - \epsilon)^2}{4})}.
\]
}
\end{theorem}
\begin{proof}
{
\[
\tmH^{(L)} - \mH^{*(L)} = \sigma(\hat{\tmA}\tmH^{(L-1)}\mW^{(L-1)}) - \sigma(\hat{\mA}^*\mH^{*(L-1)}\mW^{(L-1)})
\]
}
Since $\sigma$ is a Lipschitz continuous function, we have
{
\begin{align*}
\normLtwo{\tmH^{(L)} - \mH^{*(L)}} \leq L_\sigma\normLtwo{\hat{\tmA}\tmH^{(L-1)}\mW^{(L-1)} - \hat{\mA}^*\mH^{*(L-1)}\mW^{(L-1)}} \\
= \normLtwo{\hat{\tmA}\tmH^{(L-1)}\mW^{(L-1)} - \hat{\mA}^*\mH^{*(L-1)}\mW^{(L-1)}}\\
= \normLtwo{(\hat{\tmA} -\hat{\mA}^*) \tmH^{(L-1)}\mW^{(L-1)} + \hat{\mA}^*(\tmH^{(L-1)}- \mH^{*(L-1)})\mW^{(L-1)}}
\end{align*}
}
For notational convenience, let us suppose $D^{(L)} = \normLtwo{\tmH^{(L)} - \mH^{*(L)}}$. Applying the sub-multiplicative property of the spectral norm and triangle inequality, we obtain the following recurrence relation
{
\begin{align*}
    D^{(L)} &\leq \normLtwo{(\hat{\tmA} -\hat{\mA}^*)}\normLtwo{\tmH^{(L-1)}}\normLtwo{\mW^{(L-1)}} +   \normLtwo{\hat{\mA}^*}D^{(L-1)}\normLtwo{\mW^{(L-1)}} \\
    &\leq \normLtwo{(\hat{\tmA} -\hat{\mA}^*)} \beta\alpha + \normLtwo{\hat{\mA}^*}D^{(L-1)}\alpha \\
    &\leq \normLtwo{(\hat{\tmA} -\hat{\mA}^*)} \beta\alpha + D^{(L-1)}\alpha 
\end{align*}
}
The last inequality holds because normalized adjacency matrix satisfies $\normLtwo{\hat{\mA}^*} \leq 1$. This is because $\hat{\mA}^*$ is symmetric, row-stochastic matrix. Thus the singular values of $\hat{\mA}^*$ is the absolute values of eigenvalues of $\hat{\mA}^*$ and the largest singular value of $\hat{\mA}^*$ is the largest eigenvalue of $\hat{\mA}^*$. But $\hat{\mA}^*$ being row-stochastic, its largest eigenvalue is at most 1 hence $\normLtwo{\hat{\mA}^*} = \sigma_{max}(\hat{\mA}^*) \leq 1$.

By unrolling the recursion from earlier inequality:
\begin{align*}
     D^{(L)} &\leq \normLtwo{(\hat{\tmA} -\hat{\mA}^*)} \beta \alpha\sum_{l=0}^{L-1} \alpha^{l} + D^{(0)}\alpha^L
\end{align*}
$D^{(0)} = \normLtwo{\tmH^{(0)} - \mH^{*(0)}} = \normLtwo{\mX - \mX} = 0$. Since $\alpha < 1$, The geometric series simplifies to:
\begin{align*}
\sum_{l=0}^{L-1} \alpha^{l} = \frac{1-\alpha^L}{1-\alpha} \\
\lim_{L \to \infty} \sum_{l=0}^{L-1} \alpha^{l} = \frac{1}{1-\alpha}
\end{align*}
Thus our earlier inequality becomes:
\[
\lim_{L \to \infty} D^{(L)} \leq \frac{\beta\alpha}{1-\alpha}\normLtwo{(\hat{\tmA} -\hat{\mA}^*)} < \frac{\beta}{1-\alpha}\normLtwo{(\hat{\tmA} -\hat{\mA}^*)}
\]
Taking expectation on both sides gives us our desired result:
\small{
\begin{align*}
    \mathbb{E}[\lim_{L \to \infty} \normLtwo{\tmH^{(L)} - \mH^{*(L)}}] = \mathbb{E}[D^{(L}] < \frac{\beta}{1-\alpha}\mathbb{E}[\normLtwo{(\hat{\tmA} -\hat{\mA}^*)}] \\
    < \frac{\beta}{1-\alpha}\mathbb{E}[\normLtwo{(\hat{\mA} - \mA^*)}] \\
    = \frac{\beta}{1-\alpha}\sqrt{2k (1 - \sum_{j=1}^{|\mathcal{E}|} \frac{(p^*_j + \tilde{p}_j - \epsilon)^2}{4})}
\end{align*}
}
% Expanding the difference in the RHS:
% {
% \[
% \normLtwo{\hat{\tmA}\tmH^{(L-1)}\mW^{(L-1)} - \hat{\mA}^*\mH^{*(L-1)}\mW^{(L-1)}} = 
% \normLtwo{(\hat{\tmA} -\hat{\mA}^*) \tmH^{(L-1)}\mW^{(L-1)} - \hat{\mA}^*\mH^{*(L-1)}\mW^{(L-1)}}
% \]
% }
\end{proof}
% \begin{corollary}[Condition for convergence in embedding]
%     Assuming infinite depth of GCN layer $L$, the learned node embeddings converge to the `true embeddings' if the number of sampled edges satisfies
%     \[
%     k < \frac{1}{2}(\frac{1-\alpha}{\beta})^2
%     \]
% Here, by `true embedding', we mean the embedding generated by applying GCN on a subgraph sampled following the optimal probability distribution.
% \end{corollary}
% \begin{proof}
    
% \end{proof}
% We know that the total variation distance between two probability distributions is 1/2 of the $\text{L}_1$-distance between them. Hence, the following holds by applying universal approximation theorem (equation~\ref{eq:uapp})

% \begin{align}
% \text{TVD}(\tilde{p} , p^*) = \frac{1}{2}\|\tilde{p} - p^*\|_1 \leq \frac{\epsilon}{2}
% \end{align}

% \paragraph{2. The induced sparse subgraphs spectrum is a good approximation to the true graphs spectrum}
% $\forall \vx, \exists \epsilon$ such that
% \[
% (1-\epsilon) \leq \frac{\vx^{\top}\tilde{L}\vx}{\vx^{\top}L\vx} \leq (1+\epsilon)
%  \]
%  where $\tilde{L}$ is the Laplacian of the sparse graph sampled following the learned distribution $\tilde{p}$ and L is the original graph Laplacian.

% How to bound the ratio of the quadratic forms. Some results here: https://arxiv.org/pdf/2403.13268

\FloatBarrier



\clearpage
\section{Analyzing the Effectiveness of \sgs with a Synthetic Graph}
\label{app:toymoon}
In this section, we demonstrate and analyze the effectiveness of \sgs with a synthetically generated heterophilic graph.

\paragraph{Synthetic Graph: Moon.}
The moon dataset has the following properties: number of nodes $|\gV|=150$, number of edges $|\gE|=870$, average degree $d=5.8$, node homophily $\gH_n=0.2$, edge homophily $\gH_e = 0.32$, training/test split = $30\%/70\%$, and 2D coordinates of the points representing the nodes are the node features $\mX$.
The dataset comprises two half-moons representing two communities with $68\%$ edges connecting them as bridge edges.

%n_samples=150, degree=4, train=0.3, h = 0.2


\paragraph{Explaining the Effectiveness of \sgs on Heterophilic graph.} Fig.~\ref{fig:moongraph} juxtaposes the input moon graph (Fig.~\ref{fig:moongraph}, left) and the sparsified moon graph by \sgs (Fig.~\ref{fig:moongraph}, right). \sgs removes a significant portion of bridge edges, causing an increase in edge homophily from $0.32$ to $1.0$. As a result, the accuracy of vanilla GCN increased from $80\%$ on the full graph to $100\%$ on the sparsified graph. Since heterophilous edges significantly hinder the node representation learning, \sgs identifies them during training and learns to put less probability mass on such edges for downstream node classification. 
% \sgs identifies such edges that are detrimental to a downstream task by learning to put less probability mass on them. 
Due to this learning dynamics, \sgs is more effective on heterophilic graphs such as the Moon graph.

%%%%%%%%%%%%%%%%%%%%%%%%%%%%%%%%%%%%%%%%%%%%
\begin{figure}[!htbp]
\centering
\includegraphics[width=\linewidth]{Figures/SGS-moon.png}
\caption{Toy example with two half moon demonstrates the effectiveness of \sgs. The original graph has $68\%$ edges with different node labels; in contrast, the learned sparse subgraph from \sgs contains no such bridge edges.}
\label{fig:moongraph}
\end{figure}

%%%%%%%%%%%%%%%%%%%%%%%%%%%%%%%%%%%%%%%%%%%%

\clearpage
\section{Additional algorithmic details of \sgs }
\label{app:algorithm}

\paragraph{Conditional update of \edgemlp.}
Backward propagation is often the most computationally intensive part of training, so we employ a conditional mechanism to update \edgemlp selectively. 
We evaluate the learned sparse subgraph (line 9, Alg.~\ref{alg:sgstrainingpriorfull}) against a subgraph from the prior probability distribution $p_\mathrm{prior}$ (line 11, Alg.~\ref{alg:sgstrainingpriorfull}). If the training F1-score from the learned sparse subgraph is better than the baseline, parameters of \edgemlp are updated (line 19, Alg.~\ref{alg:sgstrainingpriorfull}). Otherwise, the update to \edgemlp is skipped (line 22, Alg.~\ref{alg:sgstrainingpriorfull}). 

The detailed algorithm for \sgs with conditional updates is in Alg.~\ref{alg:sgstrainingpriorfull}.


% If the size of $\gG$ is large, we compute a sparse subgraph from a prior probability distribution $p_\mathrm{prior}$ for \edgemlp. Later, we sample another sparse subgraph from the learned distribution to use with downstream GNN. The degree-proportionate prior is,

%  \begin{equation}
%  p_\mathrm{prior}(u,v) = \frac{1/d_u + 1/d_v}{\sum_{i,j\in \gE} (1/d_i + 1/d_j)}.
% \end{equation}

% % %%%%%%%%%%%%%%%%%%
% % \begin{wrapfigure}{c}{0.8\textwidth}
% \begin{center}
% \begin{algorithm}[!htbp]
% \caption{\sgs Training}
% \label{alg:sgstraining}
% \begin{algorithmic}[1] % The [1] here is for line numbering
% \STATE \textbf{Input:} $\gG (\gV, \gE, \mX)$, sample percent $q$, $num\_hops$
% \STATE \textbf{Output:} \texttt{EdgeMLP}, \texttt{GNN}

% %\STATE Sample size, $Q =\frac{q\gE}{100}$
% %\STATE Degree Norm, $p(u,v) = \frac{1/d_u + 1/d_v}{\sum_{i,j\in \gE} (1/d_i + 1/d_j)}$

% \FOR{$\mathrm{epochs}$ in $\mathrm{max\_epochs}$}

%     \STATE $\tilde{p},\vw =\edgemlp(\gE,p_\mathrm{prior})$
    
%     \STATE $\gE',\vw' = \mathrm{Sample}(\tilde{p}, \vw, \floor{\frac{q|\gE|}{100})}$ \COMMENT{Sparse graph for downstream GNN}
%     \STATE $\hat{\mY}, \mH' = \mathrm{GNN}_\theta(\gE',\vw',\mX)$

%     \STATE $\gL_\mathrm{CE} = \mathrm{CrossEntropy} (\mY\mathrm{[train]}, \hat{\mY}\mathrm{[train]})$
%     \STATE $\mathrm{mask[u,v]}=True:\forall_{(u,v)\in \gE} u \in \gV_L \land v \in \gV_L$
%     \STATE $\gL_\mathrm{assor} = \mathrm{CrossEntropy}(\gE \mathrm{[mask]},\vw \mathrm{[mask]})$

%     \STATE $\gL_\mathrm{cons} = \mathrm{Similarity} (\vw, \mathrm{Cosine}(\mH'[\gE[s]],\mH'[\gE[t]]))$

%     \STATE $\gL = \alpha_1\cdot \gL_\mathrm{CE}+ \alpha_2\cdot \gL_\mathrm{assor}+ \alpha_3\cdot \gL_\mathrm{cons}$
%     \STATE Backward propagate through $\gL$ and optimize $\edgemlp_\phi, \mathrm{GNN}_\theta$.
% \ENDFOR

% \STATE \textbf{Return} \texttt{EdgeMLP}, \texttt{GNN} 
% \end{algorithmic}
% \end{algorithm}
% \end{center}
% % \end{wrapfigure}
% % %%%%%%%%%%%%%%%%%%

% \sgs also supports conditional updates of \edgemlp, and the pseudocode is outlined in Algorithm~\ref{alg:sgstrainingpriorfull}. 


% %%%%%%%%%%%%%%%%%%
\begin{algorithm}[!htbp]
\caption{\sgs Training with conditional updates}
\begin{algorithmic}[1] % The [1] here is for line numbering
\STATE \textbf{Input:} $\gG (\gV, \gE, \mX)$, sample percent $q$, $\mathrm{hops}$, METIS Parts, $n$
\STATE \textbf{Output:} \texttt{EdgeMLP}, \texttt{GNN}
\STATE Compute $p_\mathrm{prior}(u,v) \gets \frac{1/d_u + 1/d_v}{\sum_{i,j\in \gE} (1/d_i + 1/d_j)}$

\STATE $\gG_\mathrm{parts}=\{\gG_1,\gG_2,\cdots,\gG_n\}\gets \mathrm{METIS} (\gG(\gV,\gE, p), n)$

\FOR{$\mathrm{epochs}$ in $\mathrm{max\_epochs}$}

    \FOR {$\gG_i(\gV_i,\gE_i,\mX_i,p^i_\mathrm{prior}) \in \gG_\mathrm{parts}$}
        \STATE $\tilde{p},\vw \gets \edgemlp(\gE_i,p^i_\mathrm{prior}, \mX_i,\mathrm{hops})$        
        \STATE $\tilde{p}_a \gets \lambda \tilde{p}+(1-\lambda)p^i_\mathrm{prior}$
        \STATE $\tilde{\gE},\tilde{\vw} \gets \mathrm{Sample}(\tilde{p}_a,\vw,\floor{\frac{q|\gE_i|}{100}})$ \COMMENT{\textbf{Learned sparse subgraph}}
        
        \STATE $\hat{\mY}, \tilde{\mH} \gets \mathrm{GNN}_\theta(\tilde{\gE},\tilde{\vw},\mX)$
        
        \STATE $\gE_\mathrm{prior} \gets \mathrm{Sample}({p_i},\floor{\frac{q|\gE_i|}{100}})$ \COMMENT{\textbf{Sparse subgraph from prior}}

        \STATE $\hat{\mY}_\mathrm{prior}  \gets \mathrm{GNN}_\theta(\gE_\mathrm{prior},\mX)$

        \IF {Evaluate$(\hat{\mY}) \ge $ Evaluate$(\hat{\mY}_\mathrm{prior})$}
            \STATE $\gL_\mathrm{CE} \gets \mathrm{CrossEntropy} (\mY_{\gV_L}, \hat{\mY}_{\gV_L})$

            \STATE $\forall_{(u,v)\in \gE_i} u \in \gV_L \land v \in \gV_L : \mathrm{mask[u,v]} \gets \text{True}$
        
            \STATE $\gL_\mathrm{assor} \gets \mathrm{CrossEntropy}(\gE \mathrm{[mask]},\vw \mathrm{[mask]})$
        
            \STATE $\gL_\mathrm{cons} \gets \mathrm{Sim} (\vw, \mathrm{Cosine}(\vh_u,\vh_v): \forall_{(u,v)\in \gE})$ 
        
            \STATE $\gL \gets \alpha_1\cdot \gL_\mathrm{CE}+ \alpha_2\cdot \gL_\mathrm{assor}+ \alpha_3\cdot \gL_\mathrm{cons}$
            \STATE Backward Propagate through $\gL$ and optimize EdgeMLP$_\phi, \mathrm{GNN}_\theta$.
        
        \ELSE
            \STATE $\gL_\mathrm{CE} \gets \mathrm{CrossEntropy} (\mY_{\gV_L}, \hat{\mY}_{\gV_L})$
            \STATE Backward Propagate through $\gL_\mathrm{CE}$ and optimize $\mathrm{GNN}_\theta$.            
        \ENDIF
    
    \ENDFOR
    
\ENDFOR
\STATE \textbf{Return} \texttt{EdgeMLP}, \texttt{GNN} 
\end{algorithmic}
\label{alg:sgstrainingpriorfull}
\end{algorithm}
% %%%%%%%%%%%%%%%%%%

\clearpage
\paragraph{Inference.} 
During inference, we use the learned probability distribution from \edgemlp. We keep track of the best temperature $T$ that gave the best validation accuracy and use that to sample an ensemble of sparse subgraphs. Then, we mean-aggregate their representations to produce the final
prediction on a test node. 

The reason we consider ensemble of subgraphs is because there are variability in the edges of the sample subgraphs even if they are all sampled from the same distribution. Thus mean-aggregation of node embeddings is an effective way to improve the robustness of the learned node embeddings. 

The inference pseudocode is provided in Algorithm~\ref{alg:sgsinference}.

% %%%%%%%%%%%%%%%%%%
\begin{algorithm}[!htbp]
\caption{\sgs Inference}
\begin{algorithmic}[1] % The [1] here is for line numbering
\STATE \textbf{Input:} Graph $\gG (\gV, \gE, \mX)$, sample \% $q$, Ensemble size, $R$.
\STATE \textbf{Output:} Prediction, $\hat{\mY}$

%\STATE Sample size, $Q =\frac{q\gE}{100}$
%\STATE Degree Norm, $p(u,v) = \frac{1/d_u + 1/d_v}{\sum_{i,j\in \gE} (1/d_i + 1/d_j)}$
    \STATE $\vw, \tilde{p} = \edgemlp(\gE, \mX, T_\mathrm{best})$ \COMMENT{\textbf{Use $T$ that gave best validation accuracy}.}   
    \STATE $S_y \gets \emptyset$ \COMMENT{Predictions}
    
    \FOR {$i$ in $R$}
        \STATE $\tilde{\gE}, \tilde{\vw} \gets \mathrm{Sample}(\tilde{p},\floor{\frac{q|\gE|}{100}})$        
        \STATE $\hat{\mY}_i \gets \mathrm{GNN}_\theta(\tilde{\gE},\tilde{\vw},\mX)$
        \STATE $S_y \gets S_y \cup \hat{\mY}_i$
    \ENDFOR

    \STATE Predict, $\hat{\mY} \gets \mathrm{Mean} (S_y)$
    
\STATE \textbf{Return} $\hat{\mY}$
\end{algorithmic}
\label{alg:sgsinference}
\end{algorithm}
% %%%%%%%%%%%%%%%%%%
\clearpage


% \FloatBarrier
\section{Dataset Description}
\label{app:dataset}
\begin{table}[!htbp]
\caption{Additional details of the dataset are provided. $\gH_\mathrm{adj}$ refers to adjusted homophily. $d$ corresponds to the average degree, $C$ number of classes, and $F$ is the feature dimension. \textit{Tr.} is the training label rate.}
\label{tab:datasetdescription}
% \begin{wrapfigure}{c}{1.0\textwidth}
\centering
\begin{sc}
\resizebox{1.0\linewidth}{!}
{
\def\arraystretch{1.0}
\begin{tabular}{@{}crrrccrcccl@{}}
\toprule
\textbf{Dataset} &
  $|\gV|$ &
  $|\gE|$ &
  \textbf{$d$} &
  \textbf{$\gH_\mathrm{adj}$} &
  \textbf{$C$} &
  \textbf{$F$} &
  \textbf{Tr.} &
  \textbf{Self-Loop} &
  \textbf{Isolated} &
  \textbf{Context} \\ \midrule
Cornell        & 183       & 557         & 3.04   & -0.42 & 5  & 1703 & 0.48 & TRUE  & FALSE & Web Pages           \\
Texas          & 183       & 574         & 3.14   & -0.26 & 5  & 1703 & 0.48 & TRUE  & FALSE & Web Pages           \\
Wisconsin      & 251       & 916         & 3.65   & -0.20 & 5  & 1703 & 0.48 & TRUE  & FALSE & Web Pages           \\
reed98         & 962       & 37,624      & 39.11  & -0.10 & 3  & 1001 & 0.6  & FALSE & FALSE & Social Network      \\
amherst41      & 2,235     & 181,908     & 81.39  & -0.07 & 3  & 1193 & 0.6  & FALSE & FALSE & Social Network      \\
penn94         & 41,554    & 2,724,458   & 65.56  & -0.06 & 2  & 4814 & 0.47 & FALSE & FALSE & Social Network      \\
Roman-empire   & 22,662    & 65,854      & 2.91   & -0.05 & 18 & 300  & 0.5  & FALSE & FALSE & Wikipedia           \\
cornell5       & 18,660    & 1,581,554   & 84.76  & -0.04 & 3  & 4735 & 0.6  & FALSE & FALSE & Web pages           \\
Squirrel       & 5,201     & 396,846     & 76.30  & -0.01 & 5  & 2345 & 0.48 & TRUE  & FALSE & Wikipedia           \\
johnshopkins55 & 5,180     & 373,172     & 72.04  & 0.00  & 3  & 2406 & 0.6  & FALSE & FALSE & Web Pages           \\
Actor          & 7,600     & 53,411      & 7.03   & 0.01  & 5  & 932  & 0.48 & TRUE  & FALSE & Actors in Movies    \\
Minesweeper    & 10,000    & 78,804      & 7.88   & 0.01  & 2  & 7    & 0.5  & FALSE & FALSE & Synthetic           \\
Questions      & 48,921    & 307,080     & 6.28   & 0.02  & 2  & 301  & 0.5  & FALSE & FALSE & Yandex Q            \\
Chameleon      & 2,277     & 62,792      & 27.58  & 0.03  & 5  & 2581 & 0.48 & TRUE  & FALSE & Wiki Pages          \\
Tolokers       & 11,758    & 1,038,000   & 88.28  & 0.09  & 2  & 10   & 0.5  & FALSE & FALSE & Toloka Platform     \\
Amazon-ratings & 24,492    & 186,100     & 7.60   & 0.14  & 5  & 556  & 0.5  & FALSE & FALSE & Co-purchase network \\
genius         & 421,961   & 1,845,736   & 4.37   & 0.17  & 2  & 12   & 0.6  & FALSE & TRUE  & Social Network      \\
pokec          & 1,632,803 & 44,603,928  & 27.32  & 0.42  & 3  & 65   & 0.6  & FALSE & FALSE & Social Network      \\
arxiv-year     & 169,343   & 2,315,598   & 13.67  & 0.26  & 5  & 128  & 0.6  & FALSE & FALSE & Citation            \\
snap-patents   & 2,923,922 & 27,945,092  & 9.56   & 0.21  & 5  & 269  & 0.6  & TRUE  & TRUE  & Citation            \\
ogbn-proteins  & 132,534   & 79,122,504  & 597.00 & 0.05  & 94 & 8    & 0.2  & FALSE & FALSE & Protein Network     \\\midrule \midrule
Cora           & 19,793    & 126,842     & 6.41   & 0.56  & 70 & 8710 & 0.2  & FALSE & FALSE & Citation Network    \\
DBLP           & 17,716    & 105,734     & 5.97   & 0.68  & 4  & 1639 & 0.2  & FALSE & FALSE & Citation Network    \\
Computers      & 13,752    & 491,722     & 35.76  & 0.68  & 10 & 767  & 0.6  & FALSE & TRUE  & Co-purchase Network \\
PubMed         & 19,717    & 88,648      & 4.50   & 0.69  & 3  & 500  & 0.2  & FALSE & FALSE & Social Network      \\
Cora\_ML        & 2,995     & 16,316      & 5.45   & 0.75  & 7  & 2879 & 0.2  & FALSE & FALSE & Citation Network    \\
SmallCora      & 2,708     & 10,556      & 3.90   & 0.77  & 7  & 1433 & 0.05 & FALSE & FALSE & Citation Network    \\
CS             & 18,333    & 163,788     & 8.93   & 0.78  & 15 & 6805 & 0.2  & FALSE & FALSE & Co-author Network   \\
Photo          & 7,650     & 238,162     & 31.13  & 0.79  & 8  & 745  & 0.2  & FALSE & TRUE  & Co-purchase Network \\
Physics        & 34,493    & 495,924     & 14.38  & 0.87  & 5  & 8415 & 0.2  & FALSE & FALSE & Co-author Network   \\
CiteSeer       & 4,230     & 10,674      & 2.52   & 0.94  & 6  & 602  & 0.2  & FALSE & FALSE & Citation Network    \\
wiki           & 11,701    & 431,726     & 36.90  & 0.58  & 10 & 300  & 0.99 & TRUE  & TRUE  & Wikipedia           \\
Reddit         & 232,965   & 114,615,892 & 491.99 & 0.74  & 41 & 602  & 0.66 & FALSE & FALSE & Social Network      \\ \bottomrule
\end{tabular}
}
\end{sc}
\end{table}
% \end{wrapfigure}
Table~\ref{tab:datasetdescription} shows the details of the characteristics of the graph datasets, including the splits used throughout the experimentation.

Along with synthetic dataset, for heterophily, we used, 
\textit{Cornell, Texas}, \textit{Wisconsin} from the \textit{WebKB}~\cite{pei2020geom}; \textit{Chameleon}, \textit{Squirrel} ~\cite{rozemberczki2021multi}; \textit{Actor} ~\cite{pei2020geom}; \textit{Wiki, ArXiv-year, Snap-Patents, Penn94, Pokec, Genius, reed98, amherst41, cornell5}, and \textit{Yelp}~\cite{lim2021large}. 
We also experiment on some recent benchmark datasets, \textit{Roman-empire, Amazon-ratings, Minesweeper, Tolokers}, and \textit{Questions} from~\cite{platonov2023critical}.

For homophily, we used
\textit{Cora}~\cite{sen2008collective}; \textit{Citeseer}~\cite{giles1998citeseer}; \textit{pubmed} \cite{namata2012query}; \textit{Coauthor-cs}, \textit{Coauthor-physics}~\cite{shchur2018pitfalls}; \textit{Amazon-computers},  \textit{Amazon-photo} ~\cite{shchur2018pitfalls}; \textit{Reddit}~\cite{hamilton2017inductive}; and, \textit{DBLP}~\cite{fu2020magnn}. 

\noindent\textbf{Heterophily Characterization.} The term \emph{homophily} in a graph describes the likelihood that nodes with the same labels are neighbors. Although there are several ways to measure homophily, three commonly used measures are {\em homophily of the nodes} ($\gH_{n}$), {\em homophily of the edges} ($\gH_{e}$), and {\em adjusted homophily} ($\gH_\mathrm{adj}$).
The {\em node homophily}~\cite{pei2020geom} is defined as,  
\begin{align}
\gH_{n} & = \frac{1}{|\gV|} \sum_{u\in \gV}\frac{| \{v\in \gN(u) : y_v = y_u\}|}{|\gN(u)|}.
\end{align}
The {\em edge homophily}~\cite{zhu2020beyond} of a graph is,

\begin{equation}
    \gH_{e} = \frac{|\{(u,v)\in \gE : y_u = y_v\}|}{|\gE|}. 
\end{equation}

The 
{\em adjusted homophily}~\cite{platonov2022characterizing} is defined as,
\begin{equation}
    \gH_\mathrm{adj} = \frac{\gH_{e}-\sum_{k=1}^{c} D_k^2/(2|\gE|^2)}{1-\sum_{k=1}^c D_k^2/2|\gE|^2}. 
\end{equation}

Here, $D_k = \sum_{v:y_v=k}d_v$ denote the sum of degrees of the nodes belonging to class $k$. 

The values of the node homophily and the edge homophily range from $0$ to $1$, and the adjusted homophily ranges from $-\frac{1}{3}$ to $+1$ (Proposition 1 in~\cite{platonov2022characterizing}). 
Among these measures, adjusted homophily considers the class imbalance. Thus, this work classifies graphs with adjusted homophily, $\gH_\mathrm {adj} \le 0.50$ as heterophilic.

% \paragraph{Dataset used in experiments.} 

\clearpage


% \FloatBarrier
\section{Runtime Comparison}
\label{app:runtime}

\subsection{Impact of Conditional Updates on Runtime}
Table.~\ref{tab:largescaleruntime} compares the runtime of \sgs with and without conditional updates for large-scale graphs (with $|\gE| \ge 1M$). The results indicate that conditional updates are similar to our standard training algorithm in terms of computational efficiency while providing improvements in F1-score under identical conditions. The additional computational costs of evaluation with prior get compensated by fewer updates of \edgemlp.

% Please add the following required packages to your document preamble:
% \usepackage{booktabs}
\begin{table}[!htbp]
\caption{Comparison of runtime of \sgs with and without conditional updates on large-scale graphs (with $|\gE| \ge 1M$). Here, \textit{Runtime (s)} refers to the mean training time per epoch. The terms \edgemlp/\gnn represent the proportion of time the \edgemlp module is updated relative to the \gnn. The results indicate that conditional updates are not significantly slower than our standard training algorithm, yet provide performance improvements to \sgs under similar conditions.}
\label{tab:largescaleruntime}
% \begin{wrapfigure}{c}{1.0\textwidth}
\centering
\begin{sc}
\resizebox{1.0\columnwidth}{!}
{
\def\arraystretch{1.0}
\begin{tabular}{@{}crrr|cc|cc|c@{}}
\toprule
\multirow{2}{*}{\textbf{Dataset}} & \multirow{2}{*}{\textbf{Node}} & \multirow{2}{*}{\textbf{Edges}} & \multirow{2}{*}{\textbf{Degree}} & \multicolumn{2}{c|}{\textbf{\sgs Runtime (s)}} & \multicolumn{2}{c|}{\textbf{\sgs F1-Score}} & \multirow{2}{*}{\textbf{\#EdgeMLP/\#GNN}} \\
 &  &  &  & \textbf{w/o. cond} & \textbf{w. cond} & \textbf{w/o. cond} & \textbf{w. cond} &  \\\midrule
cornell5 & 18,660 & 1,581,554 & 84.76 & \textbf{0.3625} & 0.3795 & 69.02 $\pm$ 0.09 & \textbf{69.12 $\pm$ 0.20} & 0.94 \\
Tolokers & 11,758 & 1,038,000 & 88.28 & 0.1743 & \textbf{0.1630} & 78.12 $\pm$ 0.13 & \textbf{78.13 $\pm$ 0.17} & 0.42 \\
genius & 421,961 & 1,845,736 & 4.37 & \textbf{0.3884} & 0.4799 & 79.92 $\pm$ 0.08 & \textbf{80.07 $\pm$ 0.11} & 0.43 \\
pokec & 1,632,803 & 44,603,928 & 27.32 & 6.7984 & \textbf{6.4885} & 62.05 $\pm$ 0.33 & \textbf{62.20 $\pm$ 0.10} & 0.75 \\
arxiv-year & 169,343 & 2,315,598 & 13.67 & \textbf{0.4571} & 0.4580 & \textbf{36.99 $\pm$ 0.11} & 36.98 $\pm$ 0.13 & 0.23 \\
snap-patents & 2,923,922 & 27,945,092 & 9.56 & \textbf{6.3470} & 7.1236 & 34.86 $\pm$ 0.15 & \textbf{34.95 $\pm$ 0.16} & 0.84 \\
Reddit & 232,965 & 114,615,892 & 491.99 & \textbf{8.0892} & 8.2960 & \textbf{91.45 $\pm$ 0.07} & 91.43 $\pm$ 0.02 & 0.44\\\bottomrule
\end{tabular}
}
\end{sc}
\end{table}
% \end{wrapfigure}

\subsection{Comparison with Baseline GNN based Sparsifiers}
\label{app:runtimerelated}
Table~\ref{tab:runtimerelated} shows related algorithms' mean training time (s). Although \sgs is slower than the unsupervised sparsification-based GNNs, it is significantly faster than supervised sparsifiers.
% Please add the following required packages to your document preamble:
% \usepackage{booktabs}
\begin{table}[!htbp]
% \begin{wrapfigure}{c}{1.0\textwidth}
\caption{Mean training time (s) per epoch of related methods. OOM refers to out-of-memory.}
\label{tab:runtimerelated}
\centering
\begin{sc}
\resizebox{1.0\columnwidth}{!}
{
\def\arraystretch{1.0}
\begin{tabular}{@{}c|ccccccc@{}}
\toprule
\textbf{Method} & \textbf{ClusterGCN} & \textbf{GraphSAINT} & \textbf{DropEdge} & \textbf{MOG} & \textbf{SparseGAT} & \textbf{Neural Sparse} & \textbf{SGS-GNN} \\ \midrule
CS & 0.0095 & 0.0089 & 0.0146 & OOM & 0.1009 & 0.1515 & 0.0221 \\
Questions & 0.0082 & 0.0072 & 0.0290 & 0.1263 & 0.0236 & 0.1221 & 0.0261 \\
Amazon-ratings & 0.0068 & 0.0062 & 0.0169 & 0.1054 & 0.0152 & 0.0499 & 0.0178 \\
johnshopkins55 & 0.0071 & 0.0061 & 0.0207 & OOM & 0.0102 & 0.1234 & 0.0244 \\
amherst41 & 0.0062 & 0.0058 & 0.0101 & OOM & 0.0053 & 0.0368 & 0.0162 \\ \bottomrule
\end{tabular}
}
\end{sc}
\end{table}
% \end{wrapfigure}
\clearpage


\section{Ablation Studies}
\label{app:ablationstudy}

This section investigates how different components of \sgs behave and contribute to overall performance. We organize this section as follows,

\begin{enumerate}
    \item Section~\ref{subsec:ab_edgemlpgnn} investigates $\gL_\mathrm{assor}, \gL_{cons}$, \edgemlp, \gnn, and Conditional Updates mechanism. We also compare its runtime against standard \sgs training vs \sgs with conditional updates. We also show \sgs can be used with other GNNs in Sec~\ref{app:othergnn}.

    \item Section~\ref{app:parameters} explores parameter settings with/without prior, different normalization and sampling methods, and inference with/without an ensemble of subgraphs.

    \item Section~\ref{app:gridsearch} shows ideal settings for regularizer coefficients $\alpha_1, \alpha_2, \alpha_3$. We also show the impact of $\lambda$ for augmenting the learned probability distribution $p$ using $p_\mathrm{prior}$.
\end{enumerate}


\subsection{$\gL_\mathrm{assor}, \gL_{cons}$, \edgemlp, \gnn, and Conditional Updates}
\label{subsec:ab_edgemlpgnn}

Table~\ref{tab:ablationgnn} illustrates the performance of \sgs with various combinations of regularizers, embedding layers in \edgemlp, and convolutional layers in \gnn. 

\begin{enumerate}
    \item $\gL_\mathrm{assor}$: Case 1, 2 shows improvement in results when $\gL_\mathrm{assor}$ is used.

    \item $\gL_\mathrm{cons}$: From cases 4, 6, 8 shows $L_\mathrm{cons}$ improves results when $\texttt{GCN}$ module is used in the GNN.

    \item \edgemlp: In general, we found that the \texttt{GCN} layers for \edgemlp encodings performs best (cases 5-6, 11-12). 
    
    \item \gnn: Both \texttt{GCN} and  \texttt{GAT} modules yielded overall the best results (case 6, 11).
    
    \item Conditional updates: Case 3 shows that conditional updates can benefit some graphs.     
    
    We also investigated the runtime and quality of \sgs with and without conditional updates for large-scale graphs. We found both have similar runtime as the condition check expense gets compensated by fewer updates of \edgemlp. Detailed comparisons of conditional updates in large graphs ($|\gE|\ge 1M$) are included in the Table~\ref{tab:largescaleruntime}.   
\end{enumerate}







\begin{table}[!htbp]
\caption{Combination of \edgemlp, \gnn, Conditional update and $L_\mathrm{cons}.$}
\label{tab:ablationgnn}
% \begin{wrapfigure}{c}{1.0\linewidth}
\centering
\begin{sc}
\resizebox{0.9\linewidth}{!}
{
\def\arraystretch{1.0}
\begin{tabular}{cccccc|ccc}
\toprule
% \rowcolor[HTML]{B7B7B7} 
\textbf{} & {$\mathbf{L_\mathrm{assor}}$} & {$\mathbf{L_\mathrm{cons}}$} & \textbf{\edgemlp} & \textbf{\gnn} & \textbf{Cond.} & {\textbf{SmallCora}} & {\textbf{CoraFull}} & {\textbf{johnshopkin}} \\ \midrule
1 & N & N & \cellcolor[HTML]{F4CCCC}MLP & \cellcolor[HTML]{FFF2CC}GCN & N & 73.80 $\pm$ 0.67 & 61.78 $\pm$ 0.20 & 66.12 $\pm$ 1.38 \\
2 & Y & N & \cellcolor[HTML]{F4CCCC}MLP & \cellcolor[HTML]{FFF2CC}GCN & N & 74.88 $\pm$ 0.15 & 63.99 $\pm$ 0.24 & 66.18 $\pm$ 1.05 \\
3 & Y & N & \cellcolor[HTML]{F4CCCC}MLP & \cellcolor[HTML]{FFF2CC}GCN & Y & 75.82 $\pm$ 0.46 & 64.07 $\pm$ 0.31 & 66.87 $\pm$ 0.93 \\
4 & Y & Y & \cellcolor[HTML]{F4CCCC}MLP & \cellcolor[HTML]{FFF2CC}GCN & Y & 76.58 $\pm$ 0.47 & 65.33 $\pm$ 0.28 & 69.25 $\pm$ 0.76 \\
5 & Y & N & \cellcolor[HTML]{D0E0E3}GCN & \cellcolor[HTML]{FFF2CC}GCN & Y & 75.80 $\pm$ 0.77 & 65.66 $\pm$ 0.14 & 71.06 $\pm$ 0.32 \\
\rowcolor[HTML]{D9D9D9} 
6 & Y & Y & \cellcolor[HTML]{D0E0E3}GCN & \cellcolor[HTML]{FFF2CC}GCN & Y & 77.50 $\pm$ 0.62 & \textbf{66.56 $\pm$ 0.22} & 70.79 $\pm$ 0.18 \\
7 & Y & N & \cellcolor[HTML]{CFE2F3}GSAGE & \cellcolor[HTML]{FFF2CC}GCN & Y & 75.82 $\pm$ 0.44 & 63.70 $\pm$ 0.09 & 67.53 $\pm$ 0.80 \\
8 & Y & Y & \cellcolor[HTML]{CFE2F3}GSAGE & \cellcolor[HTML]{FFF2CC}GCN & Y & 77.48 $\pm$ 0.61 & 65.12 $\pm$ 0.11 & 68.63 $\pm$ 0.66 \\
9 & Y & N & \cellcolor[HTML]{F4CCCC}MLP & \cellcolor[HTML]{D9EAD3}GAT & Y & 77.72 $\pm$ 1.63 & 66.40 $\pm$ 0.08 & 67.92 $\pm$ 0.73 \\
10 & Y & Y & \cellcolor[HTML]{F4CCCC}MLP & \cellcolor[HTML]{D9EAD3}GAT & Y & 75.78 $\pm$ 3.22 & 66.46 $\pm$ 0.16 & 68.17 $\pm$ 0.33 \\
\rowcolor[HTML]{D9D9D9} 
11 & Y & N & \cellcolor[HTML]{D0E0E3}GCN & \cellcolor[HTML]{D9EAD3}GAT & Y & \textbf{78.18 $\pm$ 0.74} & 66.33 $\pm$ 0.20 & \textbf{71.97 $\pm$ 0.59} \\
12 & Y & Y & \cellcolor[HTML]{D0E0E3}GCN & \cellcolor[HTML]{D9EAD3}GAT & Y & 76.94 $\pm$ 2.76 & 66.39 $\pm$ 0.18 & 71.00 $\pm$ 0.96 \\
13 & Y & N & \cellcolor[HTML]{CFE2F3}GSAGE & \cellcolor[HTML]{D9EAD3}GAT & Y & 77.98 $\pm$ 0.79 & 66.38 $\pm$ 0.23 & 69.29 $\pm$ 1.56 \\
14 & Y & Y & \cellcolor[HTML]{CFE2F3}GSAGE & \cellcolor[HTML]{D9EAD3}GAT & Y & 75.74 $\pm$ 2.02 & 66.41 $\pm$ 0.25 & 68.82 $\pm$ 0.24\\\bottomrule
\end{tabular}
}
\end{sc}
\end{table}
% \end{wrapfigure}




\subsubsection{\sgs with other GNN modules}
\label{app:othergnn}
The sampled sparse subgraphs from \edgemlp can be fed into any downstream GNNs and demonstrate a couple of variants of \sgs. Chebnet from Chebyshev~\cite{he2022convolutional}, Graph Attention Network (GAT)~\cite{velivckovic2017graph}, Graph Isomorphic Network (GIN)~\cite{xu2018powerful}, Graph Convolutional Network (GCN)~\cite{kipf2016semi} are some of the GNNs used for demonstration. 

Fig.~\ref{fig:sparsityvsgnn} shows the performance of these GNNs on homophilic and heterophilic datasets. \texttt{SGS-GCN} and \texttt{SGS-GAT} are two best performing models.


%%%%%%%%%%%%%%%%%%%%%%%%%%%%%%%%%%%%%%%%%%%%
\begin{figure}[!htbp]
\centering
% \includegraphics[width=0.5\linewidth]{Figures/SparsityvsGNN.png}
\includegraphics[width=0.6\linewidth]{Figures/SGS-differentgnn.pdf}

\caption{Performance of \sgs with different GNN modules using $20\%$ edges.}
\label{fig:sparsityvsgnn}
\end{figure}
%%%%%%%%%%%%%%%%%%%%%%%%%%%%%%%%%%%%%%%%%%%%


% \subsection{Additional Ablation studies}
% \label{app:moreablation}

\subsection{Impact of  $p_\mathrm{prior}$, Normalization \& Sampling schemes, and Ensembling on \sgs}
\label{app:parameters}

Table~\ref{tab:ablation} highlights the impact of the following components: 

% Please add the following required packages to your document preamble:
% \usepackage{booktabs}
\begin{table}[t]
\caption{Ablation Studies different components of \sgs.}
\label{tab:ablation}
% \begin{wrapfigure}{c}{1.0\textwidth}
\centering
\begin{sc}
\resizebox{1.0\linewidth}{!}
{
\def\arraystretch{1.0}
\begin{tabular}{@{}cccccc|cccccc@{}}
\toprule
\textbf{Case} & \textbf{Prior} & \textbf{Norm.} & \textbf{Sampl.}  & \multicolumn{1}{c|}{\textbf{Ensem.}} & \textbf{SmallCora} & \textbf{Cora\_ML} & \textbf{CiteSeer} & \textbf{Squirrel} & \textbf{johnshopkins55} & \textbf{Roman-empire} \\ \midrule
1 & N  & Sum & Mult  & \multicolumn{1}{c|}{N} & 69.30 $\pm$ 1.20 & 81.05 $\pm$ 0.74 & 82.84 $\pm$ 0.47 & 48.90 $\pm$ 1.06 & 63.86 $\pm$ 0.58 & 63.27 $\pm$ 0.31 \\
2 & N  & Sum & Mult  & \multicolumn{1}{c|}{Y} & 72.84 $\pm$ 0.91 & 82.92 $\pm$ 0.73 & \textbf{87.42 $\pm$ 0.42} & 46.30 $\pm$ 1.18 & 65.14 $\pm$ 1.14 & \textbf{64.31 $\pm$ 0.13} \\
% 3 & N  & Sum & Mult & $L_\mathrm{a*}$ & \multicolumn{1}{c|}{Y} & 73.48 $\pm$ 1.11 & \textbf{85.01 $\pm$ 0.33} & 86.43 $\pm$ 0.23 & 49.68 $\pm$ 0.73 & \textbf{74.73 $\pm$ 0.51} & 63.20 $\pm$ 0.21 \\
%4 & N  & Sum & Mult & $L_\mathrm{a*}$, $L_\mathrm{c*}$ & \multicolumn{1}{c|}{Y} & 75.86 $\pm$ 0.74 & 84.82 $\pm$ 0.24 & 86.55 $\pm$ 0.33 & 48.03 $\pm$ 0.79 & 72.08 $\pm$ 1.16 & 63.16 $\pm$ 0.30 \\
% 5 & Y & 0 & Sum & Mult & $L_\mathrm{a*}$, $L_\mathrm{c*}$ & \multicolumn{1}{c|}{Y} & 74.82 $\pm$ 0.64 & 84.21 $\pm$ 0.60 & 86.55 $\pm$ 0.25 & 47.53 $\pm$ 0.32 & 68.44 $\pm$ 0.46 & 63.06 $\pm$ 0.11 \\
% 6 & Y & 1 & Sum & Mult & $L_\mathrm{a*}$, $L_\mathrm{c*}$ & \multicolumn{1}{c|}{Y} & 75.54 $\pm$ 0.23 & 84.01 $\pm$ 0.74 & 86.34 $\pm$ 0.21 & 48.63 $\pm$ 0.44 & 70.77 $\pm$ 0.40 & 63.27 $\pm$ 0.12 \\
3 & Y  & Sum & Mult & \multicolumn{1}{c|}{Y} & 75.54 $\pm$ 0.41 & \textbf{83.87 $\pm$ 0.69} & 86.31 $\pm$ 0.26 & 47.97 $\pm$ 0.60 & 72.68 $\pm$ 0.51 & 62.88 $\pm$ 0.19 \\
4 & Y & Softmax & Mult & \multicolumn{1}{c|}{Y} & 75.44 $\pm$ 0.51 & 83.81 $\pm$ 0.72 & 86.31 $\pm$ 0.26 & 47.90 $\pm$ 0.42 & \textbf{72.97 $\pm$ 0.20} & 62.98 $\pm$ 0.16 \\
5 & Y & Gumbel & TopK & \multicolumn{1}{c|}{Y} & \textbf{76.24 $\pm$ 0.43} & 83.36 $\pm$ 0.34 & 86.44 $\pm$ 0.16 & \textbf{51.49 $\pm$ 0.72} & 71.83 $\pm$ 1.00 & 63.00 $\pm$ 0.11 \\ \midrule
\multicolumn{11}{l}{\textbf{Prior:} Use of prior, \textbf{Sum:} Sum-Normalization, \textbf{Softmax:} \textit{Softmax} with temperature annealing}\\
\multicolumn{11}{l}{\textbf{Mult:} \textit{Multinonmial} Sampling, \textbf{Gumbel:} \textit{Gumbel-Softmax} with \textit{TopK}}\\
\end{tabular}
}
\end{sc}
\end{table}
% \end{wrapfigure}

\begin{enumerate}
    \item Prior $p_\mathrm{prior}$: Cases 2-3 show that augmenting the learned probability distribution $\tilde{p}$ with prior $p_\mathrm{prior}$ can benefit some datasets. We have also conducted an in-depth comparison between the distributions $\tilde{p}$ and augmented distribution $\tilde{p}_a$. Figure~\ref{fig:augment_p} shows that there $\tilde{p}_a$ is left skewed whereas $\tilde{p}$ is not. Since rare edges still get some negligible mass, it is possible for $\tilde{\gG}$ constructed from $\tilde{p}_a$ to retain some bridge edges from these tails, if there are any. 
    
\begin{figure}[!htbp]
    \centering
    %\subfigure{\includegraphics[width=0.48\linewidth]{Figures/SparsityvsHomophily.png}
    \subfigure{\includegraphics[width=0.4\linewidth]{Figures/SGS-learnedp.png}
    \label{subfig:learnedp}} 
    %\hfill
    % \subfigure{\includegraphics[width=0.48\linewidth]{Figures/SparsityvsAccuracy2.png}
    %\hfill     
     \subfigure{\includegraphics[width=0.4\linewidth]{Figures/SGS-learnedp_a.png}
     \label{subfig:priorpa}} 
     \subfigure{\includegraphics[width=0.4\linewidth]{Figures/SGS-prior.png}
     \label{subfig:priorp}}    
    \caption{The learned probability distribution $\tilde{p}$ (top-left), augmented distribution $\tilde{p}_a$(top-right) and fixed prior $p_\mathrm{prior}$ (bottom). Augmentation puts negligible mass on some rare yet critical edges in the left tail of $\tilde{p}_a$.}
    \label{fig:augment_p}
\end{figure}
    \item Normalization and Sampling: We considered three normalization and sampling techniques. i) sum-normalization with multinomial sampling, ii) softmax-normalization with temperature with multinomial sampling, and iii) Gumbel softmax normalization with Topk selection. Cases 3-5 show that each of these techniques can improve results in certain datasets, and thus, it is difficult to nominate a single one as best. However, in our experiments, we opted for multinomial sampling with softmax temperature annealing for training to encourage exploration in early iterations.

    \item Ensemble subgraphs during inference: Case 2 demonstrates that using multiple subgraphs for ensemble prediction yields better results than a single subgraph (Case 1).    
\end{enumerate}

 % The addition of assortative loss $L_\mathrm{assor}$ in Case 3 enhances performance on heterophilic graphs by promoting homophily in sampled subgraphs. %Case 4 shows that incorporating consistency loss $L_\mathrm{cons}$ benefits certain graphs. 
%While our base training method updates \edgemlp at each iteration, conditional update with a prior distribution speeds up results by reducing the search space. Cases 5-7 indicate that hop size at \edgemlp influences performance. 

% % Please add the following required packages to your document preamble:
% \usepackage{booktabs}
\begin{table}[t]
\caption{Ablation Studies different components of \sgs.}
\label{tab:ablation}
% \begin{wrapfigure}{c}{1.0\textwidth}
\centering
\begin{sc}
\resizebox{1.0\linewidth}{!}
{
\def\arraystretch{1.0}
\begin{tabular}{@{}cccccc|cccccc@{}}
\toprule
\textbf{Case} & \textbf{Prior} & \textbf{Norm.} & \textbf{Sampl.}  & \multicolumn{1}{c|}{\textbf{Ensem.}} & \textbf{SmallCora} & \textbf{Cora\_ML} & \textbf{CiteSeer} & \textbf{Squirrel} & \textbf{johnshopkins55} & \textbf{Roman-empire} \\ \midrule
1 & N  & Sum & Mult  & \multicolumn{1}{c|}{N} & 69.30 $\pm$ 1.20 & 81.05 $\pm$ 0.74 & 82.84 $\pm$ 0.47 & 48.90 $\pm$ 1.06 & 63.86 $\pm$ 0.58 & 63.27 $\pm$ 0.31 \\
2 & N  & Sum & Mult  & \multicolumn{1}{c|}{Y} & 72.84 $\pm$ 0.91 & 82.92 $\pm$ 0.73 & \textbf{87.42 $\pm$ 0.42} & 46.30 $\pm$ 1.18 & 65.14 $\pm$ 1.14 & \textbf{64.31 $\pm$ 0.13} \\
% 3 & N  & Sum & Mult & $L_\mathrm{a*}$ & \multicolumn{1}{c|}{Y} & 73.48 $\pm$ 1.11 & \textbf{85.01 $\pm$ 0.33} & 86.43 $\pm$ 0.23 & 49.68 $\pm$ 0.73 & \textbf{74.73 $\pm$ 0.51} & 63.20 $\pm$ 0.21 \\
%4 & N  & Sum & Mult & $L_\mathrm{a*}$, $L_\mathrm{c*}$ & \multicolumn{1}{c|}{Y} & 75.86 $\pm$ 0.74 & 84.82 $\pm$ 0.24 & 86.55 $\pm$ 0.33 & 48.03 $\pm$ 0.79 & 72.08 $\pm$ 1.16 & 63.16 $\pm$ 0.30 \\
% 5 & Y & 0 & Sum & Mult & $L_\mathrm{a*}$, $L_\mathrm{c*}$ & \multicolumn{1}{c|}{Y} & 74.82 $\pm$ 0.64 & 84.21 $\pm$ 0.60 & 86.55 $\pm$ 0.25 & 47.53 $\pm$ 0.32 & 68.44 $\pm$ 0.46 & 63.06 $\pm$ 0.11 \\
% 6 & Y & 1 & Sum & Mult & $L_\mathrm{a*}$, $L_\mathrm{c*}$ & \multicolumn{1}{c|}{Y} & 75.54 $\pm$ 0.23 & 84.01 $\pm$ 0.74 & 86.34 $\pm$ 0.21 & 48.63 $\pm$ 0.44 & 70.77 $\pm$ 0.40 & 63.27 $\pm$ 0.12 \\
3 & Y  & Sum & Mult & \multicolumn{1}{c|}{Y} & 75.54 $\pm$ 0.41 & \textbf{83.87 $\pm$ 0.69} & 86.31 $\pm$ 0.26 & 47.97 $\pm$ 0.60 & 72.68 $\pm$ 0.51 & 62.88 $\pm$ 0.19 \\
4 & Y & Softmax & Mult & \multicolumn{1}{c|}{Y} & 75.44 $\pm$ 0.51 & 83.81 $\pm$ 0.72 & 86.31 $\pm$ 0.26 & 47.90 $\pm$ 0.42 & \textbf{72.97 $\pm$ 0.20} & 62.98 $\pm$ 0.16 \\
5 & Y & Gumbel & TopK & \multicolumn{1}{c|}{Y} & \textbf{76.24 $\pm$ 0.43} & 83.36 $\pm$ 0.34 & 86.44 $\pm$ 0.16 & \textbf{51.49 $\pm$ 0.72} & 71.83 $\pm$ 1.00 & 63.00 $\pm$ 0.11 \\ \midrule
\multicolumn{11}{l}{\textbf{Prior:} Use of prior, \textbf{Sum:} Sum-Normalization, \textbf{Softmax:} \textit{Softmax} with temperature annealing}\\
\multicolumn{11}{l}{\textbf{Mult:} \textit{Multinonmial} Sampling, \textbf{Gumbel:} \textit{Gumbel-Softmax} with \textit{TopK}}\\
\end{tabular}
}
\end{sc}
\end{table}
% \end{wrapfigure}




\clearpage
\subsection{Choosing Values for Regularizer coefficient \(\alpha_3\) and Parameter \(\lambda\)}
\label{app:gridsearch}
Recall that \sgs computes the total loss at each epoch as
\[
\gL = \alpha_1\gL_{CE}+\alpha_2\gL_\mathrm{assor}+\alpha_3\gL_\mathrm{cons},
\]
where $0 \leq \alpha_1,\alpha_2,\alpha_3 \leq 1$ are regularizer coefficients corresponding to the cross-entropy loss $\gL_{CE}$, assortativity loss $L_\mathrm{assor}$ and  consistency loss $\gL_\mathrm{cons}$ respectively. 

Also recall that, when we use a prior probability distribution, the learned distribution values of $\tilde{p}$ are weighted through $\lambda$ in
$\Tilde{p} = \lambda\Tilde{p}+(1-\lambda) p_\mathrm{prior}$

To avoid numerous combinations of values of three coefficients + the parameter $\lambda$, we have fixed $\alpha_1 = 1$, and $\alpha_2 = 1$. In the following, we investigate the performance of \sgs with different values for $\alpha_3$ and $\lambda$.

Fig.~\ref{fig:consbias} shows a grid search for different combinations of $\lambda$ and $\alpha_3$. As per our observation, the recommended values are $\lambda \in [0.3, 0.7], \alpha_3=0.5$.
%%%%%%%%%%%%%%%%%%%%%%%%%%%%%%%%%%%%%%%%%%%%
\begin{figure}[h]
\centering
\includegraphics[width=0.4\linewidth]{Figures/SGS-biasgrid.pdf}
\caption{Grid search for the parameter $\lambda$ for prior, and consistency loss, $\alpha_3$ (Cora dataset).}
\label{fig:consbias}
\end{figure}
%%%%%%%%%%%%%%%%%%%%%%%%%%%%%%%%%%%%%%%%%%%%






\end{document}
