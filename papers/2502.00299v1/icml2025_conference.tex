
\documentclass[table]{article} % For LaTeX2e
\usepackage{xcolor}
\usepackage{colortbl}
\usepackage[accepted]{icml2025}
\usepackage{float}
% Optional math commands from https://github.com/goodfeli/dlbook_notation.
%%%%% NEW MATH DEFINITIONS %%%%%

\usepackage{amsmath,amsfonts,bm}
\usepackage{derivative}
% Mark sections of captions for referring to divisions of figures
\newcommand{\figleft}{{\em (Left)}}
\newcommand{\figcenter}{{\em (Center)}}
\newcommand{\figright}{{\em (Right)}}
\newcommand{\figtop}{{\em (Top)}}
\newcommand{\figbottom}{{\em (Bottom)}}
\newcommand{\captiona}{{\em (a)}}
\newcommand{\captionb}{{\em (b)}}
\newcommand{\captionc}{{\em (c)}}
\newcommand{\captiond}{{\em (d)}}

% Highlight a newly defined term
\newcommand{\newterm}[1]{{\bf #1}}

% Derivative d 
\newcommand{\deriv}{{\mathrm{d}}}

% Figure reference, lower-case.
\def\figref#1{figure~\ref{#1}}
% Figure reference, capital. For start of sentence
\def\Figref#1{Figure~\ref{#1}}
\def\twofigref#1#2{figures \ref{#1} and \ref{#2}}
\def\quadfigref#1#2#3#4{figures \ref{#1}, \ref{#2}, \ref{#3} and \ref{#4}}
% Section reference, lower-case.
\def\secref#1{section~\ref{#1}}
% Section reference, capital.
\def\Secref#1{Section~\ref{#1}}
% Reference to two sections.
\def\twosecrefs#1#2{sections \ref{#1} and \ref{#2}}
% Reference to three sections.
\def\secrefs#1#2#3{sections \ref{#1}, \ref{#2} and \ref{#3}}
% Reference to an equation, lower-case.
\def\eqref#1{equation~\ref{#1}}
% Reference to an equation, upper case
\def\Eqref#1{Equation~\ref{#1}}
% A raw reference to an equation---avoid using if possible
\def\plaineqref#1{\ref{#1}}
% Reference to a chapter, lower-case.
\def\chapref#1{chapter~\ref{#1}}
% Reference to an equation, upper case.
\def\Chapref#1{Chapter~\ref{#1}}
% Reference to a range of chapters
\def\rangechapref#1#2{chapters\ref{#1}--\ref{#2}}
% Reference to an algorithm, lower-case.
\def\algref#1{algorithm~\ref{#1}}
% Reference to an algorithm, upper case.
\def\Algref#1{Algorithm~\ref{#1}}
\def\twoalgref#1#2{algorithms \ref{#1} and \ref{#2}}
\def\Twoalgref#1#2{Algorithms \ref{#1} and \ref{#2}}
% Reference to a part, lower case
\def\partref#1{part~\ref{#1}}
% Reference to a part, upper case
\def\Partref#1{Part~\ref{#1}}
\def\twopartref#1#2{parts \ref{#1} and \ref{#2}}

\def\ceil#1{\lceil #1 \rceil}
\def\floor#1{\lfloor #1 \rfloor}
\def\1{\bm{1}}
\newcommand{\train}{\mathcal{D}}
\newcommand{\valid}{\mathcal{D_{\mathrm{valid}}}}
\newcommand{\test}{\mathcal{D_{\mathrm{test}}}}

\def\eps{{\epsilon}}


% Random variables
\def\reta{{\textnormal{$\eta$}}}
\def\ra{{\textnormal{a}}}
\def\rb{{\textnormal{b}}}
\def\rc{{\textnormal{c}}}
\def\rd{{\textnormal{d}}}
\def\re{{\textnormal{e}}}
\def\rf{{\textnormal{f}}}
\def\rg{{\textnormal{g}}}
\def\rh{{\textnormal{h}}}
\def\ri{{\textnormal{i}}}
\def\rj{{\textnormal{j}}}
\def\rk{{\textnormal{k}}}
\def\rl{{\textnormal{l}}}
% rm is already a command, just don't name any random variables m
\def\rn{{\textnormal{n}}}
\def\ro{{\textnormal{o}}}
\def\rp{{\textnormal{p}}}
\def\rq{{\textnormal{q}}}
\def\rr{{\textnormal{r}}}
\def\rs{{\textnormal{s}}}
\def\rt{{\textnormal{t}}}
\def\ru{{\textnormal{u}}}
\def\rv{{\textnormal{v}}}
\def\rw{{\textnormal{w}}}
\def\rx{{\textnormal{x}}}
\def\ry{{\textnormal{y}}}
\def\rz{{\textnormal{z}}}

% Random vectors
\def\rvepsilon{{\mathbf{\epsilon}}}
\def\rvphi{{\mathbf{\phi}}}
\def\rvtheta{{\mathbf{\theta}}}
\def\rva{{\mathbf{a}}}
\def\rvb{{\mathbf{b}}}
\def\rvc{{\mathbf{c}}}
\def\rvd{{\mathbf{d}}}
\def\rve{{\mathbf{e}}}
\def\rvf{{\mathbf{f}}}
\def\rvg{{\mathbf{g}}}
\def\rvh{{\mathbf{h}}}
\def\rvu{{\mathbf{i}}}
\def\rvj{{\mathbf{j}}}
\def\rvk{{\mathbf{k}}}
\def\rvl{{\mathbf{l}}}
\def\rvm{{\mathbf{m}}}
\def\rvn{{\mathbf{n}}}
\def\rvo{{\mathbf{o}}}
\def\rvp{{\mathbf{p}}}
\def\rvq{{\mathbf{q}}}
\def\rvr{{\mathbf{r}}}
\def\rvs{{\mathbf{s}}}
\def\rvt{{\mathbf{t}}}
\def\rvu{{\mathbf{u}}}
\def\rvv{{\mathbf{v}}}
\def\rvw{{\mathbf{w}}}
\def\rvx{{\mathbf{x}}}
\def\rvy{{\mathbf{y}}}
\def\rvz{{\mathbf{z}}}

% Elements of random vectors
\def\erva{{\textnormal{a}}}
\def\ervb{{\textnormal{b}}}
\def\ervc{{\textnormal{c}}}
\def\ervd{{\textnormal{d}}}
\def\erve{{\textnormal{e}}}
\def\ervf{{\textnormal{f}}}
\def\ervg{{\textnormal{g}}}
\def\ervh{{\textnormal{h}}}
\def\ervi{{\textnormal{i}}}
\def\ervj{{\textnormal{j}}}
\def\ervk{{\textnormal{k}}}
\def\ervl{{\textnormal{l}}}
\def\ervm{{\textnormal{m}}}
\def\ervn{{\textnormal{n}}}
\def\ervo{{\textnormal{o}}}
\def\ervp{{\textnormal{p}}}
\def\ervq{{\textnormal{q}}}
\def\ervr{{\textnormal{r}}}
\def\ervs{{\textnormal{s}}}
\def\ervt{{\textnormal{t}}}
\def\ervu{{\textnormal{u}}}
\def\ervv{{\textnormal{v}}}
\def\ervw{{\textnormal{w}}}
\def\ervx{{\textnormal{x}}}
\def\ervy{{\textnormal{y}}}
\def\ervz{{\textnormal{z}}}

% Random matrices
\def\rmA{{\mathbf{A}}}
\def\rmB{{\mathbf{B}}}
\def\rmC{{\mathbf{C}}}
\def\rmD{{\mathbf{D}}}
\def\rmE{{\mathbf{E}}}
\def\rmF{{\mathbf{F}}}
\def\rmG{{\mathbf{G}}}
\def\rmH{{\mathbf{H}}}
\def\rmI{{\mathbf{I}}}
\def\rmJ{{\mathbf{J}}}
\def\rmK{{\mathbf{K}}}
\def\rmL{{\mathbf{L}}}
\def\rmM{{\mathbf{M}}}
\def\rmN{{\mathbf{N}}}
\def\rmO{{\mathbf{O}}}
\def\rmP{{\mathbf{P}}}
\def\rmQ{{\mathbf{Q}}}
\def\rmR{{\mathbf{R}}}
\def\rmS{{\mathbf{S}}}
\def\rmT{{\mathbf{T}}}
\def\rmU{{\mathbf{U}}}
\def\rmV{{\mathbf{V}}}
\def\rmW{{\mathbf{W}}}
\def\rmX{{\mathbf{X}}}
\def\rmY{{\mathbf{Y}}}
\def\rmZ{{\mathbf{Z}}}

% Elements of random matrices
\def\ermA{{\textnormal{A}}}
\def\ermB{{\textnormal{B}}}
\def\ermC{{\textnormal{C}}}
\def\ermD{{\textnormal{D}}}
\def\ermE{{\textnormal{E}}}
\def\ermF{{\textnormal{F}}}
\def\ermG{{\textnormal{G}}}
\def\ermH{{\textnormal{H}}}
\def\ermI{{\textnormal{I}}}
\def\ermJ{{\textnormal{J}}}
\def\ermK{{\textnormal{K}}}
\def\ermL{{\textnormal{L}}}
\def\ermM{{\textnormal{M}}}
\def\ermN{{\textnormal{N}}}
\def\ermO{{\textnormal{O}}}
\def\ermP{{\textnormal{P}}}
\def\ermQ{{\textnormal{Q}}}
\def\ermR{{\textnormal{R}}}
\def\ermS{{\textnormal{S}}}
\def\ermT{{\textnormal{T}}}
\def\ermU{{\textnormal{U}}}
\def\ermV{{\textnormal{V}}}
\def\ermW{{\textnormal{W}}}
\def\ermX{{\textnormal{X}}}
\def\ermY{{\textnormal{Y}}}
\def\ermZ{{\textnormal{Z}}}

% Vectors
\def\vzero{{\bm{0}}}
\def\vone{{\bm{1}}}
\def\vmu{{\bm{\mu}}}
\def\vtheta{{\bm{\theta}}}
\def\vphi{{\bm{\phi}}}
\def\va{{\bm{a}}}
\def\vb{{\bm{b}}}
\def\vc{{\bm{c}}}
\def\vd{{\bm{d}}}
\def\ve{{\bm{e}}}
\def\vf{{\bm{f}}}
\def\vg{{\bm{g}}}
\def\vh{{\bm{h}}}
\def\vi{{\bm{i}}}
\def\vj{{\bm{j}}}
\def\vk{{\bm{k}}}
\def\vl{{\bm{l}}}
\def\vm{{\bm{m}}}
\def\vn{{\bm{n}}}
\def\vo{{\bm{o}}}
\def\vp{{\bm{p}}}
\def\vq{{\bm{q}}}
\def\vr{{\bm{r}}}
\def\vs{{\bm{s}}}
\def\vt{{\bm{t}}}
\def\vu{{\bm{u}}}
\def\vv{{\bm{v}}}
\def\vw{{\bm{w}}}
\def\vx{{\bm{x}}}
\def\vy{{\bm{y}}}
\def\vz{{\bm{z}}}

% Elements of vectors
\def\evalpha{{\alpha}}
\def\evbeta{{\beta}}
\def\evepsilon{{\epsilon}}
\def\evlambda{{\lambda}}
\def\evomega{{\omega}}
\def\evmu{{\mu}}
\def\evpsi{{\psi}}
\def\evsigma{{\sigma}}
\def\evtheta{{\theta}}
\def\eva{{a}}
\def\evb{{b}}
\def\evc{{c}}
\def\evd{{d}}
\def\eve{{e}}
\def\evf{{f}}
\def\evg{{g}}
\def\evh{{h}}
\def\evi{{i}}
\def\evj{{j}}
\def\evk{{k}}
\def\evl{{l}}
\def\evm{{m}}
\def\evn{{n}}
\def\evo{{o}}
\def\evp{{p}}
\def\evq{{q}}
\def\evr{{r}}
\def\evs{{s}}
\def\evt{{t}}
\def\evu{{u}}
\def\evv{{v}}
\def\evw{{w}}
\def\evx{{x}}
\def\evy{{y}}
\def\evz{{z}}

% Matrix
\def\mA{{\bm{A}}}
\def\mB{{\bm{B}}}
\def\mC{{\bm{C}}}
\def\mD{{\bm{D}}}
\def\mE{{\bm{E}}}
\def\mF{{\bm{F}}}
\def\mG{{\bm{G}}}
\def\mH{{\bm{H}}}
\def\mI{{\bm{I}}}
\def\mJ{{\bm{J}}}
\def\mK{{\bm{K}}}
\def\mL{{\bm{L}}}
\def\mM{{\bm{M}}}
\def\mN{{\bm{N}}}
\def\mO{{\bm{O}}}
\def\mP{{\bm{P}}}
\def\mQ{{\bm{Q}}}
\def\mR{{\bm{R}}}
\def\mS{{\bm{S}}}
\def\mT{{\bm{T}}}
\def\mU{{\bm{U}}}
\def\mV{{\bm{V}}}
\def\mW{{\bm{W}}}
\def\mX{{\bm{X}}}
\def\mY{{\bm{Y}}}
\def\mZ{{\bm{Z}}}
\def\mBeta{{\bm{\beta}}}
\def\mPhi{{\bm{\Phi}}}
\def\mLambda{{\bm{\Lambda}}}
\def\mSigma{{\bm{\Sigma}}}

% Tensor
\DeclareMathAlphabet{\mathsfit}{\encodingdefault}{\sfdefault}{m}{sl}
\SetMathAlphabet{\mathsfit}{bold}{\encodingdefault}{\sfdefault}{bx}{n}
\newcommand{\tens}[1]{\bm{\mathsfit{#1}}}
\def\tA{{\tens{A}}}
\def\tB{{\tens{B}}}
\def\tC{{\tens{C}}}
\def\tD{{\tens{D}}}
\def\tE{{\tens{E}}}
\def\tF{{\tens{F}}}
\def\tG{{\tens{G}}}
\def\tH{{\tens{H}}}
\def\tI{{\tens{I}}}
\def\tJ{{\tens{J}}}
\def\tK{{\tens{K}}}
\def\tL{{\tens{L}}}
\def\tM{{\tens{M}}}
\def\tN{{\tens{N}}}
\def\tO{{\tens{O}}}
\def\tP{{\tens{P}}}
\def\tQ{{\tens{Q}}}
\def\tR{{\tens{R}}}
\def\tS{{\tens{S}}}
\def\tT{{\tens{T}}}
\def\tU{{\tens{U}}}
\def\tV{{\tens{V}}}
\def\tW{{\tens{W}}}
\def\tX{{\tens{X}}}
\def\tY{{\tens{Y}}}
\def\tZ{{\tens{Z}}}


% Graph
\def\gA{{\mathcal{A}}}
\def\gB{{\mathcal{B}}}
\def\gC{{\mathcal{C}}}
\def\gD{{\mathcal{D}}}
\def\gE{{\mathcal{E}}}
\def\gF{{\mathcal{F}}}
\def\gG{{\mathcal{G}}}
\def\gH{{\mathcal{H}}}
\def\gI{{\mathcal{I}}}
\def\gJ{{\mathcal{J}}}
\def\gK{{\mathcal{K}}}
\def\gL{{\mathcal{L}}}
\def\gM{{\mathcal{M}}}
\def\gN{{\mathcal{N}}}
\def\gO{{\mathcal{O}}}
\def\gP{{\mathcal{P}}}
\def\gQ{{\mathcal{Q}}}
\def\gR{{\mathcal{R}}}
\def\gS{{\mathcal{S}}}
\def\gT{{\mathcal{T}}}
\def\gU{{\mathcal{U}}}
\def\gV{{\mathcal{V}}}
\def\gW{{\mathcal{W}}}
\def\gX{{\mathcal{X}}}
\def\gY{{\mathcal{Y}}}
\def\gZ{{\mathcal{Z}}}

% Sets
\def\sA{{\mathbb{A}}}
\def\sB{{\mathbb{B}}}
\def\sC{{\mathbb{C}}}
\def\sD{{\mathbb{D}}}
% Don't use a set called E, because this would be the same as our symbol
% for expectation.
\def\sF{{\mathbb{F}}}
\def\sG{{\mathbb{G}}}
\def\sH{{\mathbb{H}}}
\def\sI{{\mathbb{I}}}
\def\sJ{{\mathbb{J}}}
\def\sK{{\mathbb{K}}}
\def\sL{{\mathbb{L}}}
\def\sM{{\mathbb{M}}}
\def\sN{{\mathbb{N}}}
\def\sO{{\mathbb{O}}}
\def\sP{{\mathbb{P}}}
\def\sQ{{\mathbb{Q}}}
\def\sR{{\mathbb{R}}}
\def\sS{{\mathbb{S}}}
\def\sT{{\mathbb{T}}}
\def\sU{{\mathbb{U}}}
\def\sV{{\mathbb{V}}}
\def\sW{{\mathbb{W}}}
\def\sX{{\mathbb{X}}}
\def\sY{{\mathbb{Y}}}
\def\sZ{{\mathbb{Z}}}

% Entries of a matrix
\def\emLambda{{\Lambda}}
\def\emA{{A}}
\def\emB{{B}}
\def\emC{{C}}
\def\emD{{D}}
\def\emE{{E}}
\def\emF{{F}}
\def\emG{{G}}
\def\emH{{H}}
\def\emI{{I}}
\def\emJ{{J}}
\def\emK{{K}}
\def\emL{{L}}
\def\emM{{M}}
\def\emN{{N}}
\def\emO{{O}}
\def\emP{{P}}
\def\emQ{{Q}}
\def\emR{{R}}
\def\emS{{S}}
\def\emT{{T}}
\def\emU{{U}}
\def\emV{{V}}
\def\emW{{W}}
\def\emX{{X}}
\def\emY{{Y}}
\def\emZ{{Z}}
\def\emSigma{{\Sigma}}

% entries of a tensor
% Same font as tensor, without \bm wrapper
\newcommand{\etens}[1]{\mathsfit{#1}}
\def\etLambda{{\etens{\Lambda}}}
\def\etA{{\etens{A}}}
\def\etB{{\etens{B}}}
\def\etC{{\etens{C}}}
\def\etD{{\etens{D}}}
\def\etE{{\etens{E}}}
\def\etF{{\etens{F}}}
\def\etG{{\etens{G}}}
\def\etH{{\etens{H}}}
\def\etI{{\etens{I}}}
\def\etJ{{\etens{J}}}
\def\etK{{\etens{K}}}
\def\etL{{\etens{L}}}
\def\etM{{\etens{M}}}
\def\etN{{\etens{N}}}
\def\etO{{\etens{O}}}
\def\etP{{\etens{P}}}
\def\etQ{{\etens{Q}}}
\def\etR{{\etens{R}}}
\def\etS{{\etens{S}}}
\def\etT{{\etens{T}}}
\def\etU{{\etens{U}}}
\def\etV{{\etens{V}}}
\def\etW{{\etens{W}}}
\def\etX{{\etens{X}}}
\def\etY{{\etens{Y}}}
\def\etZ{{\etens{Z}}}

% The true underlying data generating distribution
\newcommand{\pdata}{p_{\rm{data}}}
\newcommand{\ptarget}{p_{\rm{target}}}
\newcommand{\pprior}{p_{\rm{prior}}}
\newcommand{\pbase}{p_{\rm{base}}}
\newcommand{\pref}{p_{\rm{ref}}}

% The empirical distribution defined by the training set
\newcommand{\ptrain}{\hat{p}_{\rm{data}}}
\newcommand{\Ptrain}{\hat{P}_{\rm{data}}}
% The model distribution
\newcommand{\pmodel}{p_{\rm{model}}}
\newcommand{\Pmodel}{P_{\rm{model}}}
\newcommand{\ptildemodel}{\tilde{p}_{\rm{model}}}
% Stochastic autoencoder distributions
\newcommand{\pencode}{p_{\rm{encoder}}}
\newcommand{\pdecode}{p_{\rm{decoder}}}
\newcommand{\precons}{p_{\rm{reconstruct}}}

\newcommand{\laplace}{\mathrm{Laplace}} % Laplace distribution

\newcommand{\E}{\mathbb{E}}
\newcommand{\Ls}{\mathcal{L}}
\newcommand{\R}{\mathbb{R}}
\newcommand{\emp}{\tilde{p}}
\newcommand{\lr}{\alpha}
\newcommand{\reg}{\lambda}
\newcommand{\rect}{\mathrm{rectifier}}
\newcommand{\softmax}{\mathrm{softmax}}
\newcommand{\sigmoid}{\sigma}
\newcommand{\softplus}{\zeta}
\newcommand{\KL}{D_{\mathrm{KL}}}
\newcommand{\Var}{\mathrm{Var}}
\newcommand{\standarderror}{\mathrm{SE}}
\newcommand{\Cov}{\mathrm{Cov}}
% Wolfram Mathworld says $L^2$ is for function spaces and $\ell^2$ is for vectors
% But then they seem to use $L^2$ for vectors throughout the site, and so does
% wikipedia.
\newcommand{\normlzero}{L^0}
\newcommand{\normlone}{L^1}
\newcommand{\normltwo}{L^2}
\newcommand{\normlp}{L^p}
\newcommand{\normmax}{L^\infty}

\newcommand{\parents}{Pa} % See usage in notation.tex. Chosen to match Daphne's book.

\DeclareMathOperator*{\argmax}{arg\,max}
\DeclareMathOperator*{\argmin}{arg\,min}

\DeclareMathOperator{\sign}{sign}
\DeclareMathOperator{\Tr}{Tr}
\let\ab\allowbreak


%
\setlength\unitlength{1mm}
\newcommand{\twodots}{\mathinner {\ldotp \ldotp}}
% bb font symbols
\newcommand{\Rho}{\mathrm{P}}
\newcommand{\Tau}{\mathrm{T}}

\newfont{\bbb}{msbm10 scaled 700}
\newcommand{\CCC}{\mbox{\bbb C}}

\newfont{\bb}{msbm10 scaled 1100}
\newcommand{\CC}{\mbox{\bb C}}
\newcommand{\PP}{\mbox{\bb P}}
\newcommand{\RR}{\mbox{\bb R}}
\newcommand{\QQ}{\mbox{\bb Q}}
\newcommand{\ZZ}{\mbox{\bb Z}}
\newcommand{\FF}{\mbox{\bb F}}
\newcommand{\GG}{\mbox{\bb G}}
\newcommand{\EE}{\mbox{\bb E}}
\newcommand{\NN}{\mbox{\bb N}}
\newcommand{\KK}{\mbox{\bb K}}
\newcommand{\HH}{\mbox{\bb H}}
\newcommand{\SSS}{\mbox{\bb S}}
\newcommand{\UU}{\mbox{\bb U}}
\newcommand{\VV}{\mbox{\bb V}}


\newcommand{\yy}{\mathbbm{y}}
\newcommand{\xx}{\mathbbm{x}}
\newcommand{\zz}{\mathbbm{z}}
\newcommand{\sss}{\mathbbm{s}}
\newcommand{\rr}{\mathbbm{r}}
\newcommand{\pp}{\mathbbm{p}}
\newcommand{\qq}{\mathbbm{q}}
\newcommand{\ww}{\mathbbm{w}}
\newcommand{\hh}{\mathbbm{h}}
\newcommand{\vvv}{\mathbbm{v}}

% Vectors

\newcommand{\av}{{\bf a}}
\newcommand{\bv}{{\bf b}}
\newcommand{\cv}{{\bf c}}
\newcommand{\dv}{{\bf d}}
\newcommand{\ev}{{\bf e}}
\newcommand{\fv}{{\bf f}}
\newcommand{\gv}{{\bf g}}
\newcommand{\hv}{{\bf h}}
\newcommand{\iv}{{\bf i}}
\newcommand{\jv}{{\bf j}}
\newcommand{\kv}{{\bf k}}
\newcommand{\lv}{{\bf l}}
\newcommand{\mv}{{\bf m}}
\newcommand{\nv}{{\bf n}}
\newcommand{\ov}{{\bf o}}
\newcommand{\pv}{{\bf p}}
\newcommand{\qv}{{\bf q}}
\newcommand{\rv}{{\bf r}}
\newcommand{\sv}{{\bf s}}
\newcommand{\tv}{{\bf t}}
\newcommand{\uv}{{\bf u}}
\newcommand{\wv}{{\bf w}}
\newcommand{\vv}{{\bf v}}
\newcommand{\xv}{{\bf x}}
\newcommand{\yv}{{\bf y}}
\newcommand{\zv}{{\bf z}}
\newcommand{\zerov}{{\bf 0}}
\newcommand{\onev}{{\bf 1}}

% Matrices

\newcommand{\Am}{{\bf A}}
\newcommand{\Bm}{{\bf B}}
\newcommand{\Cm}{{\bf C}}
\newcommand{\Dm}{{\bf D}}
\newcommand{\Em}{{\bf E}}
\newcommand{\Fm}{{\bf F}}
\newcommand{\Gm}{{\bf G}}
\newcommand{\Hm}{{\bf H}}
\newcommand{\Id}{{\bf I}}
\newcommand{\Jm}{{\bf J}}
\newcommand{\Km}{{\bf K}}
\newcommand{\Lm}{{\bf L}}
\newcommand{\Mm}{{\bf M}}
\newcommand{\Nm}{{\bf N}}
\newcommand{\Om}{{\bf O}}
\newcommand{\Pm}{{\bf P}}
\newcommand{\Qm}{{\bf Q}}
\newcommand{\Rm}{{\bf R}}
\newcommand{\Sm}{{\bf S}}
\newcommand{\Tm}{{\bf T}}
\newcommand{\Um}{{\bf U}}
\newcommand{\Wm}{{\bf W}}
\newcommand{\Vm}{{\bf V}}
\newcommand{\Xm}{{\bf X}}
\newcommand{\Ym}{{\bf Y}}
\newcommand{\Zm}{{\bf Z}}

% Calligraphic

\newcommand{\Ac}{{\cal A}}
\newcommand{\Bc}{{\cal B}}
\newcommand{\Cc}{{\cal C}}
\newcommand{\Dc}{{\cal D}}
\newcommand{\Ec}{{\cal E}}
\newcommand{\Fc}{{\cal F}}
\newcommand{\Gc}{{\cal G}}
\newcommand{\Hc}{{\cal H}}
\newcommand{\Ic}{{\cal I}}
\newcommand{\Jc}{{\cal J}}
\newcommand{\Kc}{{\cal K}}
\newcommand{\Lc}{{\cal L}}
\newcommand{\Mc}{{\cal M}}
\newcommand{\Nc}{{\cal N}}
\newcommand{\nc}{{\cal n}}
\newcommand{\Oc}{{\cal O}}
\newcommand{\Pc}{{\cal P}}
\newcommand{\Qc}{{\cal Q}}
\newcommand{\Rc}{{\cal R}}
\newcommand{\Sc}{{\cal S}}
\newcommand{\Tc}{{\cal T}}
\newcommand{\Uc}{{\cal U}}
\newcommand{\Wc}{{\cal W}}
\newcommand{\Vc}{{\cal V}}
\newcommand{\Xc}{{\cal X}}
\newcommand{\Yc}{{\cal Y}}
\newcommand{\Zc}{{\cal Z}}

% Bold greek letters

\newcommand{\alphav}{\hbox{\boldmath$\alpha$}}
\newcommand{\betav}{\hbox{\boldmath$\beta$}}
\newcommand{\gammav}{\hbox{\boldmath$\gamma$}}
\newcommand{\deltav}{\hbox{\boldmath$\delta$}}
\newcommand{\etav}{\hbox{\boldmath$\eta$}}
\newcommand{\lambdav}{\hbox{\boldmath$\lambda$}}
\newcommand{\epsilonv}{\hbox{\boldmath$\epsilon$}}
\newcommand{\nuv}{\hbox{\boldmath$\nu$}}
\newcommand{\muv}{\hbox{\boldmath$\mu$}}
\newcommand{\zetav}{\hbox{\boldmath$\zeta$}}
\newcommand{\phiv}{\hbox{\boldmath$\phi$}}
\newcommand{\psiv}{\hbox{\boldmath$\psi$}}
\newcommand{\thetav}{\hbox{\boldmath$\theta$}}
\newcommand{\tauv}{\hbox{\boldmath$\tau$}}
\newcommand{\omegav}{\hbox{\boldmath$\omega$}}
\newcommand{\xiv}{\hbox{\boldmath$\xi$}}
\newcommand{\sigmav}{\hbox{\boldmath$\sigma$}}
\newcommand{\piv}{\hbox{\boldmath$\pi$}}
\newcommand{\rhov}{\hbox{\boldmath$\rho$}}
\newcommand{\upsilonv}{\hbox{\boldmath$\upsilon$}}

\newcommand{\Gammam}{\hbox{\boldmath$\Gamma$}}
\newcommand{\Lambdam}{\hbox{\boldmath$\Lambda$}}
\newcommand{\Deltam}{\hbox{\boldmath$\Delta$}}
\newcommand{\Sigmam}{\hbox{\boldmath$\Sigma$}}
\newcommand{\Phim}{\hbox{\boldmath$\Phi$}}
\newcommand{\Pim}{\hbox{\boldmath$\Pi$}}
\newcommand{\Psim}{\hbox{\boldmath$\Psi$}}
\newcommand{\Thetam}{\hbox{\boldmath$\Theta$}}
\newcommand{\Omegam}{\hbox{\boldmath$\Omega$}}
\newcommand{\Xim}{\hbox{\boldmath$\Xi$}}


% Sans Serif small case

\newcommand{\Gsf}{{\sf G}}

\newcommand{\asf}{{\sf a}}
\newcommand{\bsf}{{\sf b}}
\newcommand{\csf}{{\sf c}}
\newcommand{\dsf}{{\sf d}}
\newcommand{\esf}{{\sf e}}
\newcommand{\fsf}{{\sf f}}
\newcommand{\gsf}{{\sf g}}
\newcommand{\hsf}{{\sf h}}
\newcommand{\isf}{{\sf i}}
\newcommand{\jsf}{{\sf j}}
\newcommand{\ksf}{{\sf k}}
\newcommand{\lsf}{{\sf l}}
\newcommand{\msf}{{\sf m}}
\newcommand{\nsf}{{\sf n}}
\newcommand{\osf}{{\sf o}}
\newcommand{\psf}{{\sf p}}
\newcommand{\qsf}{{\sf q}}
\newcommand{\rsf}{{\sf r}}
\newcommand{\ssf}{{\sf s}}
\newcommand{\tsf}{{\sf t}}
\newcommand{\usf}{{\sf u}}
\newcommand{\wsf}{{\sf w}}
\newcommand{\vsf}{{\sf v}}
\newcommand{\xsf}{{\sf x}}
\newcommand{\ysf}{{\sf y}}
\newcommand{\zsf}{{\sf z}}


% mixed symbols

\newcommand{\sinc}{{\hbox{sinc}}}
\newcommand{\diag}{{\hbox{diag}}}
\renewcommand{\det}{{\hbox{det}}}
\newcommand{\trace}{{\hbox{tr}}}
\newcommand{\sign}{{\hbox{sign}}}
\renewcommand{\arg}{{\hbox{arg}}}
\newcommand{\var}{{\hbox{var}}}
\newcommand{\cov}{{\hbox{cov}}}
\newcommand{\Ei}{{\rm E}_{\rm i}}
\renewcommand{\Re}{{\rm Re}}
\renewcommand{\Im}{{\rm Im}}
\newcommand{\eqdef}{\stackrel{\Delta}{=}}
\newcommand{\defines}{{\,\,\stackrel{\scriptscriptstyle \bigtriangleup}{=}\,\,}}
\newcommand{\<}{\left\langle}
\renewcommand{\>}{\right\rangle}
\newcommand{\herm}{{\sf H}}
\newcommand{\trasp}{{\sf T}}
\newcommand{\transp}{{\sf T}}
\renewcommand{\vec}{{\rm vec}}
\newcommand{\Psf}{{\sf P}}
\newcommand{\SINR}{{\sf SINR}}
\newcommand{\SNR}{{\sf SNR}}
\newcommand{\MMSE}{{\sf MMSE}}
\newcommand{\REF}{{\RED [REF]}}

% Markov chain
\usepackage{stmaryrd} % for \mkv 
\newcommand{\mkv}{-\!\!\!\!\minuso\!\!\!\!-}

% Colors

\newcommand{\RED}{\color[rgb]{1.00,0.10,0.10}}
\newcommand{\BLUE}{\color[rgb]{0,0,0.90}}
\newcommand{\GREEN}{\color[rgb]{0,0.80,0.20}}

%%%%%%%%%%%%%%%%%%%%%%%%%%%%%%%%%%%%%%%%%%
\usepackage{hyperref}
\hypersetup{
    bookmarks=true,         % show bookmarks bar?
    unicode=false,          % non-Latin characters in AcrobatÕs bookmarks
    pdftoolbar=true,        % show AcrobatÕs toolbar?
    pdfmenubar=true,        % show AcrobatÕs menu?
    pdffitwindow=false,     % window fit to page when opened
    pdfstartview={FitH},    % fits the width of the page to the window
%    pdftitle={My title},    % title
%    pdfauthor={Author},     % author
%    pdfsubject={Subject},   % subject of the document
%    pdfcreator={Creator},   % creator of the document
%    pdfproducer={Producer}, % producer of the document
%    pdfkeywords={keyword1} {key2} {key3}, % list of keywords
    pdfnewwindow=true,      % links in new window
    colorlinks=true,       % false: boxed links; true: colored links
    linkcolor=red,          % color of internal links (change box color with linkbordercolor)
    citecolor=green,        % color of links to bibliography
    filecolor=blue,      % color of file links
    urlcolor=blue           % color of external links
}
%%%%%%%%%%%%%%%%%%%%%%%%%%%%%%%%%%%%%%%%%%%



% 
%
\setlength\unitlength{1mm}
\newcommand{\twodots}{\mathinner {\ldotp \ldotp}}
% bb font symbols
\newcommand{\Rho}{\mathrm{P}}
\newcommand{\Tau}{\mathrm{T}}

\newfont{\bbb}{msbm10 scaled 700}
\newcommand{\CCC}{\mbox{\bbb C}}

\newfont{\bb}{msbm10 scaled 1100}
\newcommand{\CC}{\mbox{\bb C}}
\newcommand{\PP}{\mbox{\bb P}}
\newcommand{\RR}{\mbox{\bb R}}
\newcommand{\QQ}{\mbox{\bb Q}}
\newcommand{\ZZ}{\mbox{\bb Z}}
\newcommand{\FF}{\mbox{\bb F}}
\newcommand{\GG}{\mbox{\bb G}}
\newcommand{\EE}{\mbox{\bb E}}
\newcommand{\NN}{\mbox{\bb N}}
\newcommand{\KK}{\mbox{\bb K}}
\newcommand{\HH}{\mbox{\bb H}}
\newcommand{\SSS}{\mbox{\bb S}}
\newcommand{\UU}{\mbox{\bb U}}
\newcommand{\VV}{\mbox{\bb V}}


\newcommand{\yy}{\mathbbm{y}}
\newcommand{\xx}{\mathbbm{x}}
\newcommand{\zz}{\mathbbm{z}}
\newcommand{\sss}{\mathbbm{s}}
\newcommand{\rr}{\mathbbm{r}}
\newcommand{\pp}{\mathbbm{p}}
\newcommand{\qq}{\mathbbm{q}}
\newcommand{\ww}{\mathbbm{w}}
\newcommand{\hh}{\mathbbm{h}}
\newcommand{\vvv}{\mathbbm{v}}

% Vectors

\newcommand{\av}{{\bf a}}
\newcommand{\bv}{{\bf b}}
\newcommand{\cv}{{\bf c}}
\newcommand{\dv}{{\bf d}}
\newcommand{\ev}{{\bf e}}
\newcommand{\fv}{{\bf f}}
\newcommand{\gv}{{\bf g}}
\newcommand{\hv}{{\bf h}}
\newcommand{\iv}{{\bf i}}
\newcommand{\jv}{{\bf j}}
\newcommand{\kv}{{\bf k}}
\newcommand{\lv}{{\bf l}}
\newcommand{\mv}{{\bf m}}
\newcommand{\nv}{{\bf n}}
\newcommand{\ov}{{\bf o}}
\newcommand{\pv}{{\bf p}}
\newcommand{\qv}{{\bf q}}
\newcommand{\rv}{{\bf r}}
\newcommand{\sv}{{\bf s}}
\newcommand{\tv}{{\bf t}}
\newcommand{\uv}{{\bf u}}
\newcommand{\wv}{{\bf w}}
\newcommand{\vv}{{\bf v}}
\newcommand{\xv}{{\bf x}}
\newcommand{\yv}{{\bf y}}
\newcommand{\zv}{{\bf z}}
\newcommand{\zerov}{{\bf 0}}
\newcommand{\onev}{{\bf 1}}

% Matrices

\newcommand{\Am}{{\bf A}}
\newcommand{\Bm}{{\bf B}}
\newcommand{\Cm}{{\bf C}}
\newcommand{\Dm}{{\bf D}}
\newcommand{\Em}{{\bf E}}
\newcommand{\Fm}{{\bf F}}
\newcommand{\Gm}{{\bf G}}
\newcommand{\Hm}{{\bf H}}
\newcommand{\Id}{{\bf I}}
\newcommand{\Jm}{{\bf J}}
\newcommand{\Km}{{\bf K}}
\newcommand{\Lm}{{\bf L}}
\newcommand{\Mm}{{\bf M}}
\newcommand{\Nm}{{\bf N}}
\newcommand{\Om}{{\bf O}}
\newcommand{\Pm}{{\bf P}}
\newcommand{\Qm}{{\bf Q}}
\newcommand{\Rm}{{\bf R}}
\newcommand{\Sm}{{\bf S}}
\newcommand{\Tm}{{\bf T}}
\newcommand{\Um}{{\bf U}}
\newcommand{\Wm}{{\bf W}}
\newcommand{\Vm}{{\bf V}}
\newcommand{\Xm}{{\bf X}}
\newcommand{\Ym}{{\bf Y}}
\newcommand{\Zm}{{\bf Z}}

% Calligraphic

\newcommand{\Ac}{{\cal A}}
\newcommand{\Bc}{{\cal B}}
\newcommand{\Cc}{{\cal C}}
\newcommand{\Dc}{{\cal D}}
\newcommand{\Ec}{{\cal E}}
\newcommand{\Fc}{{\cal F}}
\newcommand{\Gc}{{\cal G}}
\newcommand{\Hc}{{\cal H}}
\newcommand{\Ic}{{\cal I}}
\newcommand{\Jc}{{\cal J}}
\newcommand{\Kc}{{\cal K}}
\newcommand{\Lc}{{\cal L}}
\newcommand{\Mc}{{\cal M}}
\newcommand{\Nc}{{\cal N}}
\newcommand{\nc}{{\cal n}}
\newcommand{\Oc}{{\cal O}}
\newcommand{\Pc}{{\cal P}}
\newcommand{\Qc}{{\cal Q}}
\newcommand{\Rc}{{\cal R}}
\newcommand{\Sc}{{\cal S}}
\newcommand{\Tc}{{\cal T}}
\newcommand{\Uc}{{\cal U}}
\newcommand{\Wc}{{\cal W}}
\newcommand{\Vc}{{\cal V}}
\newcommand{\Xc}{{\cal X}}
\newcommand{\Yc}{{\cal Y}}
\newcommand{\Zc}{{\cal Z}}

% Bold greek letters

\newcommand{\alphav}{\hbox{\boldmath$\alpha$}}
\newcommand{\betav}{\hbox{\boldmath$\beta$}}
\newcommand{\gammav}{\hbox{\boldmath$\gamma$}}
\newcommand{\deltav}{\hbox{\boldmath$\delta$}}
\newcommand{\etav}{\hbox{\boldmath$\eta$}}
\newcommand{\lambdav}{\hbox{\boldmath$\lambda$}}
\newcommand{\epsilonv}{\hbox{\boldmath$\epsilon$}}
\newcommand{\nuv}{\hbox{\boldmath$\nu$}}
\newcommand{\muv}{\hbox{\boldmath$\mu$}}
\newcommand{\zetav}{\hbox{\boldmath$\zeta$}}
\newcommand{\phiv}{\hbox{\boldmath$\phi$}}
\newcommand{\psiv}{\hbox{\boldmath$\psi$}}
\newcommand{\thetav}{\hbox{\boldmath$\theta$}}
\newcommand{\tauv}{\hbox{\boldmath$\tau$}}
\newcommand{\omegav}{\hbox{\boldmath$\omega$}}
\newcommand{\xiv}{\hbox{\boldmath$\xi$}}
\newcommand{\sigmav}{\hbox{\boldmath$\sigma$}}
\newcommand{\piv}{\hbox{\boldmath$\pi$}}
\newcommand{\rhov}{\hbox{\boldmath$\rho$}}
\newcommand{\upsilonv}{\hbox{\boldmath$\upsilon$}}

\newcommand{\Gammam}{\hbox{\boldmath$\Gamma$}}
\newcommand{\Lambdam}{\hbox{\boldmath$\Lambda$}}
\newcommand{\Deltam}{\hbox{\boldmath$\Delta$}}
\newcommand{\Sigmam}{\hbox{\boldmath$\Sigma$}}
\newcommand{\Phim}{\hbox{\boldmath$\Phi$}}
\newcommand{\Pim}{\hbox{\boldmath$\Pi$}}
\newcommand{\Psim}{\hbox{\boldmath$\Psi$}}
\newcommand{\Thetam}{\hbox{\boldmath$\Theta$}}
\newcommand{\Omegam}{\hbox{\boldmath$\Omega$}}
\newcommand{\Xim}{\hbox{\boldmath$\Xi$}}


% Sans Serif small case

\newcommand{\Gsf}{{\sf G}}

\newcommand{\asf}{{\sf a}}
\newcommand{\bsf}{{\sf b}}
\newcommand{\csf}{{\sf c}}
\newcommand{\dsf}{{\sf d}}
\newcommand{\esf}{{\sf e}}
\newcommand{\fsf}{{\sf f}}
\newcommand{\gsf}{{\sf g}}
\newcommand{\hsf}{{\sf h}}
\newcommand{\isf}{{\sf i}}
\newcommand{\jsf}{{\sf j}}
\newcommand{\ksf}{{\sf k}}
\newcommand{\lsf}{{\sf l}}
\newcommand{\msf}{{\sf m}}
\newcommand{\nsf}{{\sf n}}
\newcommand{\osf}{{\sf o}}
\newcommand{\psf}{{\sf p}}
\newcommand{\qsf}{{\sf q}}
\newcommand{\rsf}{{\sf r}}
\newcommand{\ssf}{{\sf s}}
\newcommand{\tsf}{{\sf t}}
\newcommand{\usf}{{\sf u}}
\newcommand{\wsf}{{\sf w}}
\newcommand{\vsf}{{\sf v}}
\newcommand{\xsf}{{\sf x}}
\newcommand{\ysf}{{\sf y}}
\newcommand{\zsf}{{\sf z}}


% mixed symbols

\newcommand{\sinc}{{\hbox{sinc}}}
\newcommand{\diag}{{\hbox{diag}}}
\renewcommand{\det}{{\hbox{det}}}
\newcommand{\trace}{{\hbox{tr}}}
\newcommand{\sign}{{\hbox{sign}}}
\renewcommand{\arg}{{\hbox{arg}}}
\newcommand{\var}{{\hbox{var}}}
\newcommand{\cov}{{\hbox{cov}}}
\newcommand{\Ei}{{\rm E}_{\rm i}}
\renewcommand{\Re}{{\rm Re}}
\renewcommand{\Im}{{\rm Im}}
\newcommand{\eqdef}{\stackrel{\Delta}{=}}
\newcommand{\defines}{{\,\,\stackrel{\scriptscriptstyle \bigtriangleup}{=}\,\,}}
\newcommand{\<}{\left\langle}
\renewcommand{\>}{\right\rangle}
\newcommand{\herm}{{\sf H}}
\newcommand{\trasp}{{\sf T}}
\newcommand{\transp}{{\sf T}}
\renewcommand{\vec}{{\rm vec}}
\newcommand{\Psf}{{\sf P}}
\newcommand{\SINR}{{\sf SINR}}
\newcommand{\SNR}{{\sf SNR}}
\newcommand{\MMSE}{{\sf MMSE}}
\newcommand{\REF}{{\RED [REF]}}

% Markov chain
\usepackage{stmaryrd} % for \mkv 
\newcommand{\mkv}{-\!\!\!\!\minuso\!\!\!\!-}

% Colors

\newcommand{\RED}{\color[rgb]{1.00,0.10,0.10}}
\newcommand{\BLUE}{\color[rgb]{0,0,0.90}}
\newcommand{\GREEN}{\color[rgb]{0,0.80,0.20}}

%%%%%%%%%%%%%%%%%%%%%%%%%%%%%%%%%%%%%%%%%%
\usepackage{hyperref}
\hypersetup{
    bookmarks=true,         % show bookmarks bar?
    unicode=false,          % non-Latin characters in AcrobatÕs bookmarks
    pdftoolbar=true,        % show AcrobatÕs toolbar?
    pdfmenubar=true,        % show AcrobatÕs menu?
    pdffitwindow=false,     % window fit to page when opened
    pdfstartview={FitH},    % fits the width of the page to the window
%    pdftitle={My title},    % title
%    pdfauthor={Author},     % author
%    pdfsubject={Subject},   % subject of the document
%    pdfcreator={Creator},   % creator of the document
%    pdfproducer={Producer}, % producer of the document
%    pdfkeywords={keyword1} {key2} {key3}, % list of keywords
    pdfnewwindow=true,      % links in new window
    colorlinks=true,       % false: boxed links; true: colored links
    linkcolor=red,          % color of internal links (change box color with linkbordercolor)
    citecolor=green,        % color of links to bibliography
    filecolor=blue,      % color of file links
    urlcolor=blue           % color of external links
}
%%%%%%%%%%%%%%%%%%%%%%%%%%%%%%%%%%%%%%%%%%%


% %%%%% NEW MATH DEFINITIONS %%%%%

\usepackage{amsmath,amsfonts,bm}
\usepackage{derivative}
% Mark sections of captions for referring to divisions of figures
\newcommand{\figleft}{{\em (Left)}}
\newcommand{\figcenter}{{\em (Center)}}
\newcommand{\figright}{{\em (Right)}}
\newcommand{\figtop}{{\em (Top)}}
\newcommand{\figbottom}{{\em (Bottom)}}
\newcommand{\captiona}{{\em (a)}}
\newcommand{\captionb}{{\em (b)}}
\newcommand{\captionc}{{\em (c)}}
\newcommand{\captiond}{{\em (d)}}

% Highlight a newly defined term
\newcommand{\newterm}[1]{{\bf #1}}

% Derivative d 
\newcommand{\deriv}{{\mathrm{d}}}

% Figure reference, lower-case.
\def\figref#1{figure~\ref{#1}}
% Figure reference, capital. For start of sentence
\def\Figref#1{Figure~\ref{#1}}
\def\twofigref#1#2{figures \ref{#1} and \ref{#2}}
\def\quadfigref#1#2#3#4{figures \ref{#1}, \ref{#2}, \ref{#3} and \ref{#4}}
% Section reference, lower-case.
\def\secref#1{section~\ref{#1}}
% Section reference, capital.
\def\Secref#1{Section~\ref{#1}}
% Reference to two sections.
\def\twosecrefs#1#2{sections \ref{#1} and \ref{#2}}
% Reference to three sections.
\def\secrefs#1#2#3{sections \ref{#1}, \ref{#2} and \ref{#3}}
% Reference to an equation, lower-case.
\def\eqref#1{equation~\ref{#1}}
% Reference to an equation, upper case
\def\Eqref#1{Equation~\ref{#1}}
% A raw reference to an equation---avoid using if possible
\def\plaineqref#1{\ref{#1}}
% Reference to a chapter, lower-case.
\def\chapref#1{chapter~\ref{#1}}
% Reference to an equation, upper case.
\def\Chapref#1{Chapter~\ref{#1}}
% Reference to a range of chapters
\def\rangechapref#1#2{chapters\ref{#1}--\ref{#2}}
% Reference to an algorithm, lower-case.
\def\algref#1{algorithm~\ref{#1}}
% Reference to an algorithm, upper case.
\def\Algref#1{Algorithm~\ref{#1}}
\def\twoalgref#1#2{algorithms \ref{#1} and \ref{#2}}
\def\Twoalgref#1#2{Algorithms \ref{#1} and \ref{#2}}
% Reference to a part, lower case
\def\partref#1{part~\ref{#1}}
% Reference to a part, upper case
\def\Partref#1{Part~\ref{#1}}
\def\twopartref#1#2{parts \ref{#1} and \ref{#2}}

\def\ceil#1{\lceil #1 \rceil}
\def\floor#1{\lfloor #1 \rfloor}
\def\1{\bm{1}}
\newcommand{\train}{\mathcal{D}}
\newcommand{\valid}{\mathcal{D_{\mathrm{valid}}}}
\newcommand{\test}{\mathcal{D_{\mathrm{test}}}}

\def\eps{{\epsilon}}


% Random variables
\def\reta{{\textnormal{$\eta$}}}
\def\ra{{\textnormal{a}}}
\def\rb{{\textnormal{b}}}
\def\rc{{\textnormal{c}}}
\def\rd{{\textnormal{d}}}
\def\re{{\textnormal{e}}}
\def\rf{{\textnormal{f}}}
\def\rg{{\textnormal{g}}}
\def\rh{{\textnormal{h}}}
\def\ri{{\textnormal{i}}}
\def\rj{{\textnormal{j}}}
\def\rk{{\textnormal{k}}}
\def\rl{{\textnormal{l}}}
% rm is already a command, just don't name any random variables m
\def\rn{{\textnormal{n}}}
\def\ro{{\textnormal{o}}}
\def\rp{{\textnormal{p}}}
\def\rq{{\textnormal{q}}}
\def\rr{{\textnormal{r}}}
\def\rs{{\textnormal{s}}}
\def\rt{{\textnormal{t}}}
\def\ru{{\textnormal{u}}}
\def\rv{{\textnormal{v}}}
\def\rw{{\textnormal{w}}}
\def\rx{{\textnormal{x}}}
\def\ry{{\textnormal{y}}}
\def\rz{{\textnormal{z}}}

% Random vectors
\def\rvepsilon{{\mathbf{\epsilon}}}
\def\rvphi{{\mathbf{\phi}}}
\def\rvtheta{{\mathbf{\theta}}}
\def\rva{{\mathbf{a}}}
\def\rvb{{\mathbf{b}}}
\def\rvc{{\mathbf{c}}}
\def\rvd{{\mathbf{d}}}
\def\rve{{\mathbf{e}}}
\def\rvf{{\mathbf{f}}}
\def\rvg{{\mathbf{g}}}
\def\rvh{{\mathbf{h}}}
\def\rvu{{\mathbf{i}}}
\def\rvj{{\mathbf{j}}}
\def\rvk{{\mathbf{k}}}
\def\rvl{{\mathbf{l}}}
\def\rvm{{\mathbf{m}}}
\def\rvn{{\mathbf{n}}}
\def\rvo{{\mathbf{o}}}
\def\rvp{{\mathbf{p}}}
\def\rvq{{\mathbf{q}}}
\def\rvr{{\mathbf{r}}}
\def\rvs{{\mathbf{s}}}
\def\rvt{{\mathbf{t}}}
\def\rvu{{\mathbf{u}}}
\def\rvv{{\mathbf{v}}}
\def\rvw{{\mathbf{w}}}
\def\rvx{{\mathbf{x}}}
\def\rvy{{\mathbf{y}}}
\def\rvz{{\mathbf{z}}}

% Elements of random vectors
\def\erva{{\textnormal{a}}}
\def\ervb{{\textnormal{b}}}
\def\ervc{{\textnormal{c}}}
\def\ervd{{\textnormal{d}}}
\def\erve{{\textnormal{e}}}
\def\ervf{{\textnormal{f}}}
\def\ervg{{\textnormal{g}}}
\def\ervh{{\textnormal{h}}}
\def\ervi{{\textnormal{i}}}
\def\ervj{{\textnormal{j}}}
\def\ervk{{\textnormal{k}}}
\def\ervl{{\textnormal{l}}}
\def\ervm{{\textnormal{m}}}
\def\ervn{{\textnormal{n}}}
\def\ervo{{\textnormal{o}}}
\def\ervp{{\textnormal{p}}}
\def\ervq{{\textnormal{q}}}
\def\ervr{{\textnormal{r}}}
\def\ervs{{\textnormal{s}}}
\def\ervt{{\textnormal{t}}}
\def\ervu{{\textnormal{u}}}
\def\ervv{{\textnormal{v}}}
\def\ervw{{\textnormal{w}}}
\def\ervx{{\textnormal{x}}}
\def\ervy{{\textnormal{y}}}
\def\ervz{{\textnormal{z}}}

% Random matrices
\def\rmA{{\mathbf{A}}}
\def\rmB{{\mathbf{B}}}
\def\rmC{{\mathbf{C}}}
\def\rmD{{\mathbf{D}}}
\def\rmE{{\mathbf{E}}}
\def\rmF{{\mathbf{F}}}
\def\rmG{{\mathbf{G}}}
\def\rmH{{\mathbf{H}}}
\def\rmI{{\mathbf{I}}}
\def\rmJ{{\mathbf{J}}}
\def\rmK{{\mathbf{K}}}
\def\rmL{{\mathbf{L}}}
\def\rmM{{\mathbf{M}}}
\def\rmN{{\mathbf{N}}}
\def\rmO{{\mathbf{O}}}
\def\rmP{{\mathbf{P}}}
\def\rmQ{{\mathbf{Q}}}
\def\rmR{{\mathbf{R}}}
\def\rmS{{\mathbf{S}}}
\def\rmT{{\mathbf{T}}}
\def\rmU{{\mathbf{U}}}
\def\rmV{{\mathbf{V}}}
\def\rmW{{\mathbf{W}}}
\def\rmX{{\mathbf{X}}}
\def\rmY{{\mathbf{Y}}}
\def\rmZ{{\mathbf{Z}}}

% Elements of random matrices
\def\ermA{{\textnormal{A}}}
\def\ermB{{\textnormal{B}}}
\def\ermC{{\textnormal{C}}}
\def\ermD{{\textnormal{D}}}
\def\ermE{{\textnormal{E}}}
\def\ermF{{\textnormal{F}}}
\def\ermG{{\textnormal{G}}}
\def\ermH{{\textnormal{H}}}
\def\ermI{{\textnormal{I}}}
\def\ermJ{{\textnormal{J}}}
\def\ermK{{\textnormal{K}}}
\def\ermL{{\textnormal{L}}}
\def\ermM{{\textnormal{M}}}
\def\ermN{{\textnormal{N}}}
\def\ermO{{\textnormal{O}}}
\def\ermP{{\textnormal{P}}}
\def\ermQ{{\textnormal{Q}}}
\def\ermR{{\textnormal{R}}}
\def\ermS{{\textnormal{S}}}
\def\ermT{{\textnormal{T}}}
\def\ermU{{\textnormal{U}}}
\def\ermV{{\textnormal{V}}}
\def\ermW{{\textnormal{W}}}
\def\ermX{{\textnormal{X}}}
\def\ermY{{\textnormal{Y}}}
\def\ermZ{{\textnormal{Z}}}

% Vectors
\def\vzero{{\bm{0}}}
\def\vone{{\bm{1}}}
\def\vmu{{\bm{\mu}}}
\def\vtheta{{\bm{\theta}}}
\def\vphi{{\bm{\phi}}}
\def\va{{\bm{a}}}
\def\vb{{\bm{b}}}
\def\vc{{\bm{c}}}
\def\vd{{\bm{d}}}
\def\ve{{\bm{e}}}
\def\vf{{\bm{f}}}
\def\vg{{\bm{g}}}
\def\vh{{\bm{h}}}
\def\vi{{\bm{i}}}
\def\vj{{\bm{j}}}
\def\vk{{\bm{k}}}
\def\vl{{\bm{l}}}
\def\vm{{\bm{m}}}
\def\vn{{\bm{n}}}
\def\vo{{\bm{o}}}
\def\vp{{\bm{p}}}
\def\vq{{\bm{q}}}
\def\vr{{\bm{r}}}
\def\vs{{\bm{s}}}
\def\vt{{\bm{t}}}
\def\vu{{\bm{u}}}
\def\vv{{\bm{v}}}
\def\vw{{\bm{w}}}
\def\vx{{\bm{x}}}
\def\vy{{\bm{y}}}
\def\vz{{\bm{z}}}

% Elements of vectors
\def\evalpha{{\alpha}}
\def\evbeta{{\beta}}
\def\evepsilon{{\epsilon}}
\def\evlambda{{\lambda}}
\def\evomega{{\omega}}
\def\evmu{{\mu}}
\def\evpsi{{\psi}}
\def\evsigma{{\sigma}}
\def\evtheta{{\theta}}
\def\eva{{a}}
\def\evb{{b}}
\def\evc{{c}}
\def\evd{{d}}
\def\eve{{e}}
\def\evf{{f}}
\def\evg{{g}}
\def\evh{{h}}
\def\evi{{i}}
\def\evj{{j}}
\def\evk{{k}}
\def\evl{{l}}
\def\evm{{m}}
\def\evn{{n}}
\def\evo{{o}}
\def\evp{{p}}
\def\evq{{q}}
\def\evr{{r}}
\def\evs{{s}}
\def\evt{{t}}
\def\evu{{u}}
\def\evv{{v}}
\def\evw{{w}}
\def\evx{{x}}
\def\evy{{y}}
\def\evz{{z}}

% Matrix
\def\mA{{\bm{A}}}
\def\mB{{\bm{B}}}
\def\mC{{\bm{C}}}
\def\mD{{\bm{D}}}
\def\mE{{\bm{E}}}
\def\mF{{\bm{F}}}
\def\mG{{\bm{G}}}
\def\mH{{\bm{H}}}
\def\mI{{\bm{I}}}
\def\mJ{{\bm{J}}}
\def\mK{{\bm{K}}}
\def\mL{{\bm{L}}}
\def\mM{{\bm{M}}}
\def\mN{{\bm{N}}}
\def\mO{{\bm{O}}}
\def\mP{{\bm{P}}}
\def\mQ{{\bm{Q}}}
\def\mR{{\bm{R}}}
\def\mS{{\bm{S}}}
\def\mT{{\bm{T}}}
\def\mU{{\bm{U}}}
\def\mV{{\bm{V}}}
\def\mW{{\bm{W}}}
\def\mX{{\bm{X}}}
\def\mY{{\bm{Y}}}
\def\mZ{{\bm{Z}}}
\def\mBeta{{\bm{\beta}}}
\def\mPhi{{\bm{\Phi}}}
\def\mLambda{{\bm{\Lambda}}}
\def\mSigma{{\bm{\Sigma}}}

% Tensor
\DeclareMathAlphabet{\mathsfit}{\encodingdefault}{\sfdefault}{m}{sl}
\SetMathAlphabet{\mathsfit}{bold}{\encodingdefault}{\sfdefault}{bx}{n}
\newcommand{\tens}[1]{\bm{\mathsfit{#1}}}
\def\tA{{\tens{A}}}
\def\tB{{\tens{B}}}
\def\tC{{\tens{C}}}
\def\tD{{\tens{D}}}
\def\tE{{\tens{E}}}
\def\tF{{\tens{F}}}
\def\tG{{\tens{G}}}
\def\tH{{\tens{H}}}
\def\tI{{\tens{I}}}
\def\tJ{{\tens{J}}}
\def\tK{{\tens{K}}}
\def\tL{{\tens{L}}}
\def\tM{{\tens{M}}}
\def\tN{{\tens{N}}}
\def\tO{{\tens{O}}}
\def\tP{{\tens{P}}}
\def\tQ{{\tens{Q}}}
\def\tR{{\tens{R}}}
\def\tS{{\tens{S}}}
\def\tT{{\tens{T}}}
\def\tU{{\tens{U}}}
\def\tV{{\tens{V}}}
\def\tW{{\tens{W}}}
\def\tX{{\tens{X}}}
\def\tY{{\tens{Y}}}
\def\tZ{{\tens{Z}}}


% Graph
\def\gA{{\mathcal{A}}}
\def\gB{{\mathcal{B}}}
\def\gC{{\mathcal{C}}}
\def\gD{{\mathcal{D}}}
\def\gE{{\mathcal{E}}}
\def\gF{{\mathcal{F}}}
\def\gG{{\mathcal{G}}}
\def\gH{{\mathcal{H}}}
\def\gI{{\mathcal{I}}}
\def\gJ{{\mathcal{J}}}
\def\gK{{\mathcal{K}}}
\def\gL{{\mathcal{L}}}
\def\gM{{\mathcal{M}}}
\def\gN{{\mathcal{N}}}
\def\gO{{\mathcal{O}}}
\def\gP{{\mathcal{P}}}
\def\gQ{{\mathcal{Q}}}
\def\gR{{\mathcal{R}}}
\def\gS{{\mathcal{S}}}
\def\gT{{\mathcal{T}}}
\def\gU{{\mathcal{U}}}
\def\gV{{\mathcal{V}}}
\def\gW{{\mathcal{W}}}
\def\gX{{\mathcal{X}}}
\def\gY{{\mathcal{Y}}}
\def\gZ{{\mathcal{Z}}}

% Sets
\def\sA{{\mathbb{A}}}
\def\sB{{\mathbb{B}}}
\def\sC{{\mathbb{C}}}
\def\sD{{\mathbb{D}}}
% Don't use a set called E, because this would be the same as our symbol
% for expectation.
\def\sF{{\mathbb{F}}}
\def\sG{{\mathbb{G}}}
\def\sH{{\mathbb{H}}}
\def\sI{{\mathbb{I}}}
\def\sJ{{\mathbb{J}}}
\def\sK{{\mathbb{K}}}
\def\sL{{\mathbb{L}}}
\def\sM{{\mathbb{M}}}
\def\sN{{\mathbb{N}}}
\def\sO{{\mathbb{O}}}
\def\sP{{\mathbb{P}}}
\def\sQ{{\mathbb{Q}}}
\def\sR{{\mathbb{R}}}
\def\sS{{\mathbb{S}}}
\def\sT{{\mathbb{T}}}
\def\sU{{\mathbb{U}}}
\def\sV{{\mathbb{V}}}
\def\sW{{\mathbb{W}}}
\def\sX{{\mathbb{X}}}
\def\sY{{\mathbb{Y}}}
\def\sZ{{\mathbb{Z}}}

% Entries of a matrix
\def\emLambda{{\Lambda}}
\def\emA{{A}}
\def\emB{{B}}
\def\emC{{C}}
\def\emD{{D}}
\def\emE{{E}}
\def\emF{{F}}
\def\emG{{G}}
\def\emH{{H}}
\def\emI{{I}}
\def\emJ{{J}}
\def\emK{{K}}
\def\emL{{L}}
\def\emM{{M}}
\def\emN{{N}}
\def\emO{{O}}
\def\emP{{P}}
\def\emQ{{Q}}
\def\emR{{R}}
\def\emS{{S}}
\def\emT{{T}}
\def\emU{{U}}
\def\emV{{V}}
\def\emW{{W}}
\def\emX{{X}}
\def\emY{{Y}}
\def\emZ{{Z}}
\def\emSigma{{\Sigma}}

% entries of a tensor
% Same font as tensor, without \bm wrapper
\newcommand{\etens}[1]{\mathsfit{#1}}
\def\etLambda{{\etens{\Lambda}}}
\def\etA{{\etens{A}}}
\def\etB{{\etens{B}}}
\def\etC{{\etens{C}}}
\def\etD{{\etens{D}}}
\def\etE{{\etens{E}}}
\def\etF{{\etens{F}}}
\def\etG{{\etens{G}}}
\def\etH{{\etens{H}}}
\def\etI{{\etens{I}}}
\def\etJ{{\etens{J}}}
\def\etK{{\etens{K}}}
\def\etL{{\etens{L}}}
\def\etM{{\etens{M}}}
\def\etN{{\etens{N}}}
\def\etO{{\etens{O}}}
\def\etP{{\etens{P}}}
\def\etQ{{\etens{Q}}}
\def\etR{{\etens{R}}}
\def\etS{{\etens{S}}}
\def\etT{{\etens{T}}}
\def\etU{{\etens{U}}}
\def\etV{{\etens{V}}}
\def\etW{{\etens{W}}}
\def\etX{{\etens{X}}}
\def\etY{{\etens{Y}}}
\def\etZ{{\etens{Z}}}

% The true underlying data generating distribution
\newcommand{\pdata}{p_{\rm{data}}}
\newcommand{\ptarget}{p_{\rm{target}}}
\newcommand{\pprior}{p_{\rm{prior}}}
\newcommand{\pbase}{p_{\rm{base}}}
\newcommand{\pref}{p_{\rm{ref}}}

% The empirical distribution defined by the training set
\newcommand{\ptrain}{\hat{p}_{\rm{data}}}
\newcommand{\Ptrain}{\hat{P}_{\rm{data}}}
% The model distribution
\newcommand{\pmodel}{p_{\rm{model}}}
\newcommand{\Pmodel}{P_{\rm{model}}}
\newcommand{\ptildemodel}{\tilde{p}_{\rm{model}}}
% Stochastic autoencoder distributions
\newcommand{\pencode}{p_{\rm{encoder}}}
\newcommand{\pdecode}{p_{\rm{decoder}}}
\newcommand{\precons}{p_{\rm{reconstruct}}}

\newcommand{\laplace}{\mathrm{Laplace}} % Laplace distribution

\newcommand{\E}{\mathbb{E}}
\newcommand{\Ls}{\mathcal{L}}
\newcommand{\R}{\mathbb{R}}
\newcommand{\emp}{\tilde{p}}
\newcommand{\lr}{\alpha}
\newcommand{\reg}{\lambda}
\newcommand{\rect}{\mathrm{rectifier}}
\newcommand{\softmax}{\mathrm{softmax}}
\newcommand{\sigmoid}{\sigma}
\newcommand{\softplus}{\zeta}
\newcommand{\KL}{D_{\mathrm{KL}}}
\newcommand{\Var}{\mathrm{Var}}
\newcommand{\standarderror}{\mathrm{SE}}
\newcommand{\Cov}{\mathrm{Cov}}
% Wolfram Mathworld says $L^2$ is for function spaces and $\ell^2$ is for vectors
% But then they seem to use $L^2$ for vectors throughout the site, and so does
% wikipedia.
\newcommand{\normlzero}{L^0}
\newcommand{\normlone}{L^1}
\newcommand{\normltwo}{L^2}
\newcommand{\normlp}{L^p}
\newcommand{\normmax}{L^\infty}

\newcommand{\parents}{Pa} % See usage in notation.tex. Chosen to match Daphne's book.

\DeclareMathOperator*{\argmax}{arg\,max}
\DeclareMathOperator*{\argmin}{arg\,min}

\DeclareMathOperator{\sign}{sign}
\DeclareMathOperator{\Tr}{Tr}
\let\ab\allowbreak



% For theorems and such
\usepackage{amsmath}
\usepackage{amssymb}
\usepackage{mathtools}
\usepackage{amsthm}
\usepackage{enumitem}

% Recommended, but optional, packages for figures and better typesetting:
\usepackage{microtype}
% \usepackage{subfigure}
% \usepackage[table]{xcolor}
% \usepackage{colortbl}
% \usepackage[table]{xcolor} % Enable table coloring
\usepackage{subcaption}
\usepackage{tabularx}
% \usepackage{xcolor} 
\usepackage{pifont}
% \usepackage{algorithm}
% \usepackage{algpseudocode}
\usepackage{array}     % For column width control

\usepackage{comment}
\usepackage[utf8]{inputenc} % allow utf-8 input
\usepackage[T1]{fontenc}    % use 8-bit T1 fonts
\usepackage{url}            % simple URL typesetting
\usepackage{amsfonts}       % blackboard math symbols
\usepackage{nicefrac}       % compact symbols for 1/2, etc.
\usepackage{xcolor}         % colors

\usepackage{tabularx}
\usepackage{makecell}

\usepackage{hyperref}
\usepackage{url}
\usepackage{graphicx}
\usepackage{natbib}
\usepackage{booktabs}
\usepackage{multirow}
\usepackage{wrapfig}
\usepackage{subcaption}
\usepackage{amsmath}

\usepackage{bookmark}
\usepackage{makecell}
\usepackage{amssymb}
\usepackage{array}
\usepackage{booktabs}
\usepackage{pifont}
% \usepackage{algorithm}
% \usepackage{algpseudocode}
\usepackage{soul}
\usepackage{xcolor}

\usepackage{etoc}
\etocdepthtag.toc{mtchapter}
\etocsettagdepth{mtchapter}{subsection}
\etocsettagdepth{mtappendix}{none}

\newcommand{\theHalgorithm}{\arabic{algorithm}}

% \usepackage{geometry}
% \geometry{a4paper, margin=1in}
\newcommand{\method}{ChunkKV}
\sethlcolor{yellow}
% \NewDocumentCommand\emojicheck{}{\includegraphics[scale=0.9]{emoji/check.pdf}}
\newcommand{\cmark}{\textcolor{blue}{\ding{51}}}

% \NewDocumentCommand{\xl}
% { mO{} }{\textcolor{blue}{\textsuperscript{\textit{xl}}\textsf{\textbf{\small[#1]}}}}

\newcommand{\rebuttal}[1]{\textcolor{red}{#1}}


% \title{ChunkKV: Semantic-Preserving KV Cache Compression for Efficient Long-Context LLM Inference}

\icmltitlerunning{ChunkKV: Semantic-Preserving KV Cache Compression for Efficient Long-Context LLM Inference}


\begin{document}

\twocolumn[
\icmltitle{ChunkKV: Semantic-Preserving KV Cache Compression for \\ Efficient Long-Context LLM Inference}

% It is OKAY to include author information, even for blind
% submissions: the style file will automatically remove it for you
% unless you've provided the [accepted] option to the icml2025
% package.

% List of affiliations: The first argument should be a (short)
% identifier you will use later to specify author affiliations
% Academic affiliations should list Department, University, City, Region, Country
% Industry affiliations should list Company, City, Region, Country

% You can specify symbols, otherwise they are numbered in order.
% Ideally, you should not use this facility. Affiliations will be numbered
% in order of appearance and this is the preferred way.
\icmlsetsymbol{equal}{*}


% \author{
% Xiang LIU$^{\heartsuit}$ $\qquad$ Zhenheng TANG$^{\clubsuit}$ $\qquad$ Peijie DONG$^{\heartsuit}$ $\qquad$ Zeyu LI$^{\heartsuit}$ \\ 
% \textbf{Bo LI}$^{\clubsuit}$ $\qquad$ \textbf{Xuming HU$^{\heartsuit\dagger}$} $\qquad$ \textbf{Xiaowen CHU$^{\heartsuit\dagger}$} \\
% $^{\heartsuit*}$  The Hong Kong University of Science and Technology(Guangzhou)\\ 
% $^{\clubsuit}$  The Hong Kong University of Science and Technology\\
%  \texttt{\{xliu886,pdong212,zli755\}@connect.hkust-gz.edu.cn} \\
%  \texttt{\{zhtang.ml, bli\}@cse.ust.hk} \quad \texttt{\{xuminghu, xwchu\}@hkust-gz.edu.cn} 
%   }


\begin{icmlauthorlist}
% \begin{icmlauthorlist}
\icmlauthor{Xiang Liu}{hkustgz}
\icmlauthor{Zhenheng Tang}{hkust}
\icmlauthor{Peijie Dong}{hkustgz}
\icmlauthor{Zeyu Li}{hkustgz}
\icmlauthor{Bo Li}{hkust}
\icmlauthor{Xuming Hu}{hkustgz}
\icmlauthor{Xiaowen Chu}{hkustgz}
\end{icmlauthorlist}

\icmlaffiliation{hkustgz}{The Hong Kong University of Science and Technology(Guangzhou), Guangzhou, China}
\icmlaffiliation{hkust}{The Hong Kong University of Science and Technology, Hong Kong, China}

\icmlcorrespondingauthor{Xuming Hu}{xuminghu@hkust-gz.edu.cn}
\icmlcorrespondingauthor{Xiaowen Chu}{xwchu@hkust-gz.edu.cn}

\icmlkeywords{Large Language Models, KV Cache Compression, Model Evaluation}

\vskip 0.3in
]

\printAffiliationsAndNotice{}

% Although recent KV cache compression methods show strong performance, all use discrete tokens to maintain the KV cache, leading to a loss of chunk semantic information. 


\begin{abstract}
   To reduce memory costs in long-context inference with Large Language Models (LLMs), many recent works focus on compressing the key-value (KV) cache of different tokens. However, we identify that the previous KV cache compression methods measure token importance individually, neglecting the dependency between different tokens in the real-world language characterics. In light of this, we introduce \method{}, grouping the tokens in a chunk as a basic compressing unit, and retaining the most informative semantic chunks while discarding the less important ones. Furthermore, observing that \method{} exhibits higher similarity in the preserved indices across different layers, we propose layer-wise index reuse to further reduce computational overhead. We evaluated \method{} on cutting-edge long-context benchmarks including LongBench and Needle-In-A-HayStack, as well as the GSM8K and JailbreakV in-context learning benchmark. Our experiments with instruction tuning and multi-step reasoning (O1 and R1) LLMs, achieve up to 10\% performance improvement under aggressive compression ratios compared to existing methods.
   % indicating its effectiveness in semantic preservation and model performance for long-context and in-context LLM inference.
   % Large Language Models (LLMs) have demonstrated remarkable capabilities in processing extensive contexts, but this ability comes with significant GPU memory costs, particularly in the key-value (KV) cache. Although recent KV cache compression methods show strong performance, all use discrete tokens to maintain the KV cache, leading to a loss of chunk semantic information. We introduce \method{}, a novel KV cache compression method that retains the most informative semantic chunks while discarding the less important ones. \method{} preserves semantic information by grouping related tokens. Furthermore, \method{} exhibits a higher similarity in the indices of the retained KV cache across different layers, so we also propose a layer-wise index reuse technique to further reduce computational overhead. We evaluated \method{} on long-context benchmarks including LongBench and Needle-In-A-HayStack, as well as the GSM8K and JailbreakV in-context learning benchmark. Our experiments, conducted with instruction tuning and multi-step reasoning (O1 and R1) LLMs, achieve up to 10\% performance improvement under aggressive compression ratios compared to existing methods. \method{} achieves state-of-the-art performance on various tasks, indicating its effectiveness in semantic preservation and model performance for long-context and in-context LLM inference.

\end{abstract}


\section{Introduction}
% Large Language Models (LLMs) have become essential for addressing various downstream tasks of natural language processing (NLP), including summarization and question answering, which require the interpretation of a wide context from sources such as books, reports, and documents, often encompassing tens of thousands of tokens~\citep{raffel2020exploring, brown2020language, chowdhery2022palm, tay2022unifying, touvron2023llama, touvron2023llama2}. Recent advances in long-context technology within the field of machine learning (ML) systems~\citep{flash-attn, flash-attn2, jacobs2023deepspeed, xiao2024efficient} and model architecture design~\citep{chen2023extending, xiong2023effective, chen2023longlora, peng2024yarn} have significantly enhanced the ability of LLMs to process increasingly large input context lengths~\citep{liu2024world, young2024yi}, such as the Gemini-1.5-pro model, which can manage documents up to 1,500 pages in length~\citep{geminiteam2024gemini}. However, this ability to handle long contexts also presents significant challenges regarding the key-value (KV) cache for super-long prompts. For instance, the KV cache for a single token in a 7 billion-parameter model requires approximately 0.5 MB of GPU memory, resulting in a 10,000-token prompt consuming around 5 GB of GPU memory, which constitutes nearly one fifth of the memory available on an RTX 4090 GPU. Larger contexts will further increase GPU memory consumption during inference serving~\citep{jamba, grok, geminiteam2024gemini, claude3, deepseekv2}. Consequently, KV cache compression methods have become crucial technologies for reducing GPU memory costs when deploying LLM services.

Large Language Models (LLMs) have become essential for addressing various downstream tasks of natural language processing (NLP), including summarization and question answering, which require the interpretation of a long context from sources such as books, reports, and documents, often encompassing tens of thousands of tokens~\citep{brown2020language, tay2022unifying, touvron2023llama2}. Recent advances in long-context technology within the field of machine learning (ML) systems~\citep{flash-attn2, jacobs2023deepspeed, xiao2024efficient}  have significantly enhanced computational throughputs and reduced latency of LLMs to process increasingly large input context lengths~\citep{liu2024world, young2024yi} with saving historical KV cache (key value attentions). However, the memory requirement of the KV cache in serving super-long contexts becomes a new bottlneck~\citep{zhang2024h2o,geminiteam2024gemini}. For instance, the KV cache for a single token in a 7B-parameter model requires approximately 0.5 MB of GPU memory, resulting in a 10,000-token prompt consuming around 5 GB of GPU memory. 

\begin{figure*}[h]
   \centering
   \includegraphics[width=1\textwidth]{figs/main.pdf}
   \caption{Illustration of the impact of the token discrete method and the chunk method on semantic preservation. The discrete method preserves words related to the question but often omits the subject. In contrast, the chunk method retains the subject of the words, maintaining more accurate semantic information. For the equation: $S$ is the score function, and $c$ is a chunk of tokens.}
   \label{fig:main}
   % \vspace{-10pt}
\end{figure*}

To address the substantial GPU memory consumption caused by KV caching, recent studies consider compressing the KV cache by pruning non-important discrete parts from the prompt tokens~\citep{zhang2024h2o, li2024snapkv, ge2023model, zhang2024pyramidkv, fu2024lazyllm, yang2024pyramidinfer,liu2024scissorhands, tang2024quest}. H2O~\citep{zhang2024h2o} and SnapKV~\citep{li2024snapkv} have shown that retaining less than 50\% of the discrete KV cache can significantly reduce GPU memory usage with minimal impact on performance. However, we identify that the previous KV cache compression methods~\citep{zhang2024h2o,zhang2024pyramidkv} measure token importance isolatedly, neglecting the dependency between different tokens in the real-world language characterics. For example, as shown in Figure~\ref{fig:main}, focusing on token-level importance might excessively focus on words about subjects ``turaco'' in the question while omitting crucial information about the objects (foods) in the documents, resulting the loss of essential semantic information. This motivates us to rethink the following question:


\emph{How to avoid isolated token importance measurement and preserve the semantic information in KV cache?}


% However, previous methods mainly focus on discrete token compression, which may result in the loss of semantic information. Although SnapKV apply pooling strategy, it still cannot preserve the semantic information. Figure~\ref{fig:main} shows an example in which the high-sparsity discrete method preserves the words related to the question but often omits the subject, leading to a potential misinterpretation of the context. For example, in a passage discussing other animals that eat strawberries, the discrete method might erroneously retain the word "strawberries" while omitting crucial information about the subjects (i.e., other animals). This selective preservation can result in the loss of essential semantic information and can potentially lead to incorrect inferences or responses from the model. Such issues are particularly pronounced in multi-document QA tasks, where maintaining context across multiple sources is crucial for accurate comprehension and response generation. For more details on discrete token methods, please refer to Appendix~\ref{appendix:chunkkv_analysis}.


\begin{table*}[h]
   \centering
   \caption{Comparison of Methods on KV Cache Compression.}
% \setlength{\abovedisplayskip}{-2pt}
% \setlength{\abovecaptionskip}{-2pt}
\scriptsize
% \setlength{\tabcolsep}{8pt}
\resizebox{0.9\textwidth}{!}{
   \begin{tabular}{lccccc}
       \toprule
       \multirow{2}{*}{\textbf{Method}} & \multirow{2}{*}{\makecell{ \textbf{KV Cache} \\ \textbf{Compression}}} & \multirow{2}{*}{\makecell{ \textbf{Dynamic} \\ \textbf{Policy}}}  & \multirow{2}{*}{\makecell{ \textbf{Layer-Wise} \\ \textbf{Policy}}} & \multirow{2}{*}{\makecell{ \textbf{Semantic} \\ \textbf{Information}}}  & \multirow{2}{*}{\makecell{ \textbf{Efficient} \\ \textbf{Index Reuse}}} \\
       \\
       \midrule 
       StreamingLLM~\citep{xiao2024efficient} & \cmark &  & &  &  \\
       H2O~\citep{zhang2024h2o} & \cmark & \cmark &  & &  \\
       SnapKV~\citep{li2024snapkv} & \cmark & \cmark &  & &  \\
       PyramidInfer~\citep{yang2024pyramidinfer} & \cmark & \cmark & \cmark &  &  \\
       PyramidKV~\citep{zhang2024pyramidkv} & \cmark & \cmark & \cmark &  &  \\ \midrule
       \rowcolor{red!20}\method{}(Ours) & \cmark & \cmark & \cmark & \cmark & \cmark \\
       \bottomrule
   \end{tabular}
   }
   \label{tab:comparison}
\end{table*}


In light of this, we observe that the complete semantic information usually appear in a continuous sequence~\citep{xie2022an}. Thus, we introduce a straightforward yet effective \method{}, grouping the tokens in a chunk as a basic compressing unit, which should be preserved or discarded as a whole. Thus, it retains the most informative \textbf{semantic chunks} from the original KV cache. As shown in Figure~\ref{fig:main}, preserving a chunk helps to catch the subject, predicate, and object. Furthermore, we investigate that \textit{the preserved KV cache indices by \method{} exhibit a higher similarity} compared to previous methods. Consequently, we develop a technique called layer-wise index reuse, which reduces the additional computational time introduced by the KV cache compression method. As outlined in Table \ref{tab:comparison}, recent highly relevant KV cache compression methods \textit{lack the ability to retain semantic information and efficiently reuse indices}. 

% To address this gap, we explore the semantic dimensions of KV cache compression. We introduce a straightforward yet effective method, \textbf{\method{}}, which retains the most informative \textbf{semantic chunks} from the original KV cache, as in Figure~\ref{fig:main}. As outlined in Table \ref{tab:comparison}, recent highly relevant KV cache compression methods \textit{lack the ability to retain semantic information and efficiently reuse indices}. Furthermore, we investigate that \textit{the preserved KV cache indices by \method{} exhibit a higher similarity} compared to previous methods. Consequently, we develop a technique called layer-wise index reuse, which reduces the additional computational time introduced by the KV cache compression method.

To evaluate \method{}'s performance, we conduct comprehensive experiments across multiple cutting-edge long-context benchmarks: long-context tasks including LongBench~\citep{bai2023longbench} and Needle-In-A-HayStack (NIAH)~\citep{needle}, in-context learning tasks such as GSM8K~\citep{gsm8k} and JailbreakV~\citep{jailbreakv}. And also different models including DeepSeek-R1-Distill-Llama-8B~\citep{deepseekr1},LLaMA-3-8B-Instruct~\citep{meta2024llama3}, Mistral-7B-Instruct~\citep{jiang2023mistral7b}, and Qwen2-7B-Instruct~\citep{qwen2}. Our experimental results demonstrate that \method{} surpasses existing KV cache compression methods in both efficiency and accuracy, primarily due to its ability to preserve essential information through selective chunk retention. These findings establish \method{} as a simple yet effective approach to KV cache compression.

We summarize our key contributions as follows:
\begin{itemize}[leftmargin=*]
   \item \noindent We identify the phenomenon in which discrete KV cache compression methods inadvertently prune the necessary semantic information.
   \item \noindent We propose \method{}, a simple KV cache compression method that uses the fragmentation method that keeps the semantic information, and propose the layer-wise index reuse technique to reduce the additional computational time.
   \item \noindent We evaluate \method{} on cutting-edge long-context benchmarks including LongBench and Needle-In-A-HayStack, as well as the GSM8K, many-shot GSM8K and JailbreakV in-context learning benchmark, and multi-step reasoning (O1 and R1) LLMs, achieving state-of-the-art performance.
   \end{itemize}


\section{Related Work}
\textbf{KV Cache Compression.} KV cache compression technology has developed rapidly in the era of LLM, with methods mainly focused on evicting unimportant tokens. The compression process occurs before the attention blocks, optimizing both the prefilling time and GPU memory. \citet{xiao2024efficient} and \citet{han2024lm} propose that initial and recent tokens consistently have high attention scores between different layers and attention heads. As a result, retaining these tokens in the KV cache is more likely to preserve important information. Furthermore, FastGen~\citep{ge2023model} evicts tokens based on observed patterns. H2O~\citep{zhang2024h2o} and SnapKV~\citep{li2024snapkv} employ dynamic KV cache compression methods, evaluating the importance of tokens based on attention scores and then evicting the less important ones. As inference scenarios become increasingly complex, dynamic KV cache compression methods demonstrate powerful performance. Recently, \citet{yang2024pyramidinfer} and \citet{zhang2024pyramidkv} have closely examined the distributions of attention scores during the pre-filling stage of the Retrieval-Augmented Generation (RAG) task, discovering a pyramidal KV cache compression pattern in different transformer layers. 

Although these KV cache compression methods have explored efficient GPU memory management while maintaining original performance, our study focuses more on the semantic information of the prompt. We find that chunks of the original KV cache are more important than discrete tokens.
   
% \vspace{-0.8em}
\textbf{Chunking Method.} 
The chunking methodology is widely used in the field of NLP due to its simplicity and effectiveness~\citep{sang1999representing}. In the era of LLMs, chunking is primarily applied in data pre-processing. For example, \citet{shicontext} suggest grouping related training data into chunks to achieve better convergence curves to pre-train LLMs. \citet{fei-etal-2024-extending} apply a topic-based chunking method to improve the semantic compression of prompts. Furthermore, chunking plays an important role in the Retrieval-Augmented Generation (RAG) field~\citep{yepes2024financialreportchunkingeffective, smith2024evaluating, anthropic_contextual_retrieval_2024}. It serves to divide documents into units of information with semantic content suitable for embedding-based retrieval and processing by LLMs.

\textbf{Layer-Wise Technique}
The layer-wise technique is widely used in the training and inference of large language models (LLMs). LISA~\citep{pan2024lisa} is a layer-wise sampling method based on observations of the training dynamics of Low-Rank Adaptation (LoRA)\citep{hu2021lora} across layers. LAMB\citep{you2019lamb} is a layer-wise adaptive learning rate method that speeds up LLM training by stabilizing training convergence with large batch sizes. DoLa~\citep{chuang2023dola} employs layer-wise contrasting to reduce hallucinations during LLM inference.

\section{ChunkKV}
\subsection{Preliminary Study of KV Cache Compression}

With the increasing long-context capabilities of LLMs, the KV cache has become crucial for improving inference efficiency. However, it can consume significant GPU memory when handling long input contexts. The GPU memory cost of the KV cache for the decoding stage can be calculated as follows:
   % \vspace{-1.8em}
\begin{equation}
   \label{eq:kv_cache_cost}
    M_{KV} = 2 \times \textit{B} \times \textit{S}  \times \textit{L} \times \textit{N} \times \textit{D} \times 2
\end{equation}
   
% % \vspace{-1.8em}
% \begin{equation}
%    \label{eq:kv_cache_cost}
%     \text{GPU Cost of KV Cache} = 2 \times \textit{B} \times \textit{S}  \times \textit{L} \times \textit{N} \times \textit{D} \times 2
% \end{equation}
% % \vspace{-1.8em}

where $B$ is the batch size, $S$ is the sequence length of prompt and decoded length, $L$ is the number of layers, $N$ is the number of attention heads, $D$ is the dimension of each attention head, and the first $2$ accounts for the KV matrices, while the last $2$ accounts for the precision when using float16. Table \ref{appendix:config_models} shows the configuration parameters for LLaMA-3-8B-Instruct~\citep{meta2024llama3}. With a batch size $B=1$ and a sequence length of prompt $S=2048$, the GPU memory cost of the KV cache is nearly $1$ GB. If the batch size exceeds 24, the GPU memory cost of the KV cache will exceed the capacity of an RTX 4090 GPU. To address this issue, KV cache compression methods have been proposed, with the aim of retaining only a minimal amount of KV cache while preserving as much information as possible. For more details on the LLM configuration parameters, refer to Appendix~\ref{appendix:config_models}.
% \begin{wraptable}{r}{0.5\textwidth}
% \begin{table}[h]
%    \vspace{10pt}
%    \caption{ \centering LLaMA-3-8B-Inst \\ Configuration Parameters.}
%    \resizebox{0.5\textwidth}{!}{
%    \centering
%    \begin{tabular}{cc}
%    \toprule
%    \textbf{Attribute} & \textbf{Value} \\
%    \midrule
%    Model Name & LLaMA-3-8B-Inst \\
%    $L$ (Number of layers) & 32 \\
%    $N$ (Number of attention heads) & 32 \\
%    $D$ (Dimension of each head) & 128 \\
%    \bottomrule
%    \end{tabular}
%    }
%    \label{table:model_config}
% \end{table}

% To optimize memory usage, a strategy called KV cache compression has been proposed~\cite{zhang2024h2o, xiao2024efficient, li2024snapkv}. This strategy involves retaining only a minimal amount of KV cache while preserving as much information as possible, effectively reducing $L$ in Equation \ref{eq:kv_cache_cost}. Typically, the $L$ after applying KV compression methods is less than 50\% of the original $L$, with minimal performance degradation. However, these methods primarily focus on discrete tokens of the KV cache, which may result in the loss of semantic information. 


% \subsection{Proposed Method}
% \label{sec:method}
% \subsubsection{\method{}}

\subsection{Chunk Based KV Compression}
To address the limitations of existing KV cache compression methods, we propose \method{}, a novel KV cache compression method that retains the most informative semantic chunks. The key idea behind \method{} is to group tokens in the KV cache into chunks that preserve more semantic information, such as a chunk containing a subject, verb and object. As illustrated in Figure~\ref{fig:main}, \method{} preserves the chunks of the KV cache that contain more semantic information.
First, we define a chunk as a group of tokens that contain related semantic information. By retaining the most informative chunks from the original KV cache, \method{} can effectively reduce the memory usage of the KV cache while preserving essential information.

\begin{algorithm}
\caption{\method{}}
\label{alg:chunkkv}
\begin{algorithmic}
\STATE \textbf{Input:} $\mathbf{Q} \in \mathbb{R}^{T_q \times d}$, $\mathbf{K} \in \mathbb{R}^{T_k \times d}$, $\mathbf{V} \in \mathbb{R}^{T_v \times d}$, observe window size $w$, chunk size $c$, compressed KV cache max length $L_{\text{max}}$
\STATE \textbf{Output:} Compressed KV cache $\mathbf{K}'$, $\mathbf{V}'$
\STATE \textbf{Observe Window Calculation:}
\STATE $\mathbf{A} \gets \mathbf{Q}_{T_q - w:T_q} \mathbf{K}^T$ \COMMENT{Attention scores for the observe window}
\STATE $C \gets \left\lceil \frac{T_k}{c} \right\rceil$ \COMMENT{Calculate the number of chunks}
\STATE \textbf{Chunk Attention Score Calculation}:
\FOR {$i = 1$ to $C$}
      \STATE $\mathbf{A}_i \gets \sum_{j=(i-1)c+1}^{ic} \mathbf{A}_{:,j}$ \COMMENT{Sum of observation scores for each chunk}
\ENDFOR
\STATE \textbf{Top-K Chunk Selection}:
\STATE $k \gets \left\lfloor \frac{L_{\text{max}}}{c} \right\rfloor$
\STATE $\textit{Top\_K\_Indices} \gets \text{indices of Top-}k \text{ chunks based on } \mathbf{A}_i $ 
\STATE \textbf{Compression}:
\STATE $\mathbf{K}', \mathbf{V}' \gets \text{index\_select}(\mathbf{K}, \mathbf{V}, \textit{Top\_K\_Indices})$
\STATE \textbf{Concatenation}:
\STATE $\mathbf{K}' \gets \text{concat}(\mathbf{K}'_{0:L_{\text{max}}-w}, \mathbf{K}_{T_k-w:T_k})$
\STATE $\mathbf{V}' \gets \text{concat}(\mathbf{V}'_{0:L_{\text{max}}-w}, \mathbf{V}_{T_v-w:T_v})$
\STATE  $\mathbf{K}'$, $\mathbf{V}'$
\end{algorithmic}
\end{algorithm}
   
The Algorithm \ref{alg:chunkkv} shows the pseudocode outline of \method{}. First, following H2O~\citep{zhang2024h2o} and SnapKV~\citep{li2024snapkv}, we set the observe window by computing the observation scores $\mathbf{A} \gets \mathbf{Q}_{T_q - w:T_q} \mathbf{K}^T$, where $\mathbf{Q}_{T_q - w:T_q}$ is the observe window, $\mathbf{K}$ is the Key matrix and the window size $w$ is usually set to $\{4,8,16,32\}$. Next, the number of chunks $C$ is calculated as $C = \left\lceil \frac{T_k}{c} \right\rceil$, where $T_k$ is the length of the Key matrix and $c$ is the chunk size. The observation scores for each chunk are then computed as $\mathbf{A}_i = \sum_{j=(i-1)c+1}^{ic} \mathbf{A}_{:,j}$ for $i = 1, 2, \ldots, C$. Referring to previous works~\citep{zhang2024h2o, li2024snapkv, yang2024pyramidinfer, zhang2024pyramidkv}, we still use the top-$k$ algorithm as \method{}'s sampling policy. For the top-$k$ chunk selection, the top-$k$ chunks are selected based on their observation scores, where $k = \left\lfloor \frac{L_{\text{max}}}{c} \right\rfloor$, and $L_{\text{max}}$ is the maximum length of the compressed KV cache. The size of the last chunk will equal $\text{min}(c, L_{\text{max}} - (k-1) \times c)$. The indices of the top-$k$ chunks will keep the original sequence order. In the compression step, the key and value matrices are only retained based on the selected indices, resulting in the compressed KV cache. Finally, the observe window of the original KV cache will be concatenated to the compressed KV cache by replacing the last $w$ tokens to keep important information. The compressed KV cache is then used for subsequent attention computations.




% \begin{wrapfigure}{r}{0.5\textwidth}
% \vspace{-10pt}

% \begin{minipage}{0.5\textwidth}     
\begin{algorithm}[tb]
   \caption{Layer-wise Index Reuse for \method{}}
   \label{alg:layer_wise_index_reuse}
   \begin{algorithmic}
   \STATE \textbf{Input:} Number of layers in LLMs $N_{\text{layers}}$, number of reuse layers $N_{\text{reuse}}$
   \STATE \textbf{Initialize:} Dictionary to store indices $\mathcal{I}_{\text{reuse}} = \{\}$
   \FOR {$l = 0$ to ($N_{\text{layers}}-1)$}
       \IF {$l \mod N_{\text{reuse}} == 0$}
           \STATE $\mathbf{K}'_{l}, \mathbf{V}'_{l}, \mathcal{I}_l \gets \text{\method{}}(\mathbf{K}_l, \mathbf{V}_l)$ 
           \STATE $\mathcal{I}_{\text{reuse}}[l] \gets \mathcal{I}_l$ 
       \ELSE
         \STATE $\mathcal{I}_l \gets \mathcal{I}_{\text{reuse}}[ \left\lfloor \frac{l}{N_{\text{reuse}}} \right\rfloor \times N_{\text{reuse}} ]$ 
       \ENDIF
       \STATE $\mathbf{K}'_{l} \gets \text{index\_select}(\mathbf{K}_l, \mathcal{I}_l)$ 
       \STATE $\mathbf{V}'_{l} \gets \text{index\_select}(\mathbf{V}_l, \mathcal{I}_l)$ 
   \ENDFOR
\end{algorithmic}
\end{algorithm}
% \end{minipage}
% \vspace{-10pt}
% \end{wrapfigure}
   
\begin{figure*}[h]
   \centering
   \includegraphics[width=\textwidth]{./figs/llama_snapkv_chunkkv_layer_similarity_heatmap.pdf}
   \caption{ Layer-wise similarity heatmaps of the preserved KV cache indices  by SnapKV (left) and ChunkKV (right) on LLaMA-3-8B-Instruct.}
   \label{fig:index_reuse_heatmap}
\end{figure*}

% \subsubsection{Layer-Wise Index Reuse}
\subsection{Layer-Wise Index Reuse}
\label{sec:layer_wise_index_reuse}
Furthermore, we investigated the preserved KV cache indices by \method{} and found that they exhibit higher similarity compared to previous methods. Figure~\ref{fig:index_reuse_heatmap} shows the layer-wise similarity heatmaps of SnapKV and \method{}. Each cell represents the similarity between the preserved KV cache indices of two layers, with deeper colors indicating higher similarity. The results demonstrate that the KV cache indices preserved by \method{} are more similar to those in neighboring layers. As shown in Table~\ref{tab:jaccard_similarity_models}, \method{} consistently achieves a higher average Jaccard similarity between adjacent layers compared to SnapKV in different model architectures, indicating that the retained token index in \method{} is more similar to each other. For a more detailed visualization, please refer to Appendix \ref{appendix:index_reuse_similarity}. 

% \begin{wraptable}{r}{0.4\textwidth}
\begin{table}[h]
   \centering
   \caption{Retained KV Cache Indices Similarity of Adjacent Layers for Different Models.}
   \resizebox{0.4\textwidth}{!}{
   \begin{tabular}{lccc}
   \toprule
   \textbf{Method} & \textbf{H2O} & \textbf{SnapKV} & \textbf{\method{}}  \\
   \midrule
   LLaMA-3-8B      & 25.31\%& 27.95\% & \textbf{57.74\%} \\
   Qwen2-7B        & 14.91\%& 16.50\% & \textbf{44.26\%} \\
   Mistral-7B      & 15.15\% & 15.78\% & \textbf{52.16\%} \\
   \bottomrule
   \end{tabular}
   }
   \vspace{-0.4em}
   \label{tab:jaccard_similarity_models}
\end{table}


Based on the above findings, we propose a training-free \textit{ layer-wise index reuse} method to further reduce the additional cost of the KV cache compression time, which reuses compressed token indices across multiple layers. This procedure is formally described in Algorithm \ref{alg:layer_wise_index_reuse}. The \method{} compression process returns the compressed KV cache and their respective token indices, denoted as $\mathcal{I}_l$. For layer-wise index reuse, we define a grouping of layers such that all $N_{\text{reuse}}$ layers share the same token indices for \method{}. Specifically, for a group of layers $\left\{l, l+1, \ldots, l+N_{\text{reuse}}-1\right\}$, we perform \method{} on the first layer $l$ to obtain the token indices $\mathcal{I}_l$ and reuse $\mathcal{I}_l$ for the subsequent layers $l+1, l+2, \ldots, l+N_{\text{reuse}}-1$. The notation $\mathbf{K}_l[\mathcal{I}_l]$ and $\mathbf{V}_l[\mathcal{I}_l]$ indicates the selection of key and value caches based on the indices in $\mathcal{I}_l$. The efficiency analysis for layer-wise index reuse is provided in Appendix~\ref{appendix:index_reuse_efficiency}.


% \subsection{Theoretical Understanding}
% \label{sec:theory}
\textbf{Theoretical Understanding.} We provide a theoretical understanding from the in-context learning (ICL)~\citep{xie2022an} to interpret why maintaining KV cache according to a continuous sequence in \method{} is better than according to sparse tokens. 
Informally speaking, the continuously chunk-level KV cache preserves the whole examples (semantic information) in ICL, thus reducing the requirement on distinguishability, i.e lower bound of KL divergence between the example and the question (Equation~\ref{eq:noise_dinsting} in Condition~\ref{cond:2}). The complete analysis is provided in Appendix~\ref{appendix:theory}.


\section{Experiment Results}
\label{sec:experiment_results}
In this section, we conduct experiments to evaluate the effectiveness of \method{} on KV cache compression in two benchmark fields, with a chunk size set to 10 even for various model architectures. The first is the In-Context Learning benchmark, for which we select GSM8K~\citep{gsm8k} and Jailbreakv~\citep{jailbreakv} to evaluate the performance of \method{}, furthermore we also include multi-step reasoning LLM DeepSeek-R1-Distill-Llama-8B~\citep{deepseekr1} to evaluate the performance of \method{}. The In-Context Learning scenario is a crucial capability for LLMs and has been adapted in many powerful technologies such as Chain-of-Thought~\citep{wei2022chain,diao2023active,pan2023plum}. The second is the Long-Context benchmark, which includes LongBench~\citep{bai2023longbench} and Needle-In-A-HayStack (NIAH)~\citep{needle}, both widely used for assessing KV cache compression methods.  All experiments were conducted three times, using the mean score to ensure robustness.


\subsection{In-Context Learning}
\label{sec:icl}

The In-Context Learning (ICL) ability significantly enhances the impact of prompts on large language models (LLMs). For example, the Chain-of-Thought approach~\citep{wei2022chain} increases the accuracy of the GSM8K of the PaLM model~\citep{chowdhery2022palm} from 18\% to 57\% without additional training. In this section, we evaluate the performance of \method{} on the GSM8K, Many-Shot GSM8K \citep{agarwal2024many}, and JailbreakV \citep{jailbreakv} benchmarks.
% However, KV cache compression methods will potentially remove important prompt information. Therefore, evaluating KV cache compression methods using an ICL benchmark is an effective way to demonstrate the efficacy of different compression strategies.


\begin{table}[!h]
   \caption{GSM8K Performance Comparison. }
   \centering
   \resizebox{1\columnwidth}{!}{
   \begin{tabular}{l|ccccc}
   % \specialrule{1pt}{0pt}{2pt}
   \specialrule{1pt}{0pt}{2pt}
   \multirow{2}{*}{\textbf{\makecell{ Ratio}}} & \multirow{2}{*}{\makecell{StreamingLLM}} & \multirow{2}{*}{H2O} & \multirow{2}{*}{SnapKV} & \multirow{2}{*}{PyramidKV} & \multirow{2}{*}{\makecell{\textbf{ChunkKV} \\ 
   \textbf{(Ours)}}} \\
   & & & & & \\
   \midrule
   \multicolumn{6}{c}{DeepSeek-R1-Distill-Llama-8B FullKV: 69.4\% $\uparrow$} \\
   \midrule
   10\% & 51.6\% & 55.6\% & 57.6\% & 62.6\% & \cellcolor{red!20}\textbf{65.7\%} \\
   \midrule
   \multicolumn{6}{c}{LlaMa-3.1-8B-Instruct FullKV: 79.5\% $\uparrow$} \\
   \midrule
   30\% & 70.5\% & 72.2\% & 76.1\% & 77.1\% & \cellcolor{red!20}\textbf{77.3\%} \\
   20\% & 63.8\% & 64.0\% & 68.8\% & 71.4\% & \cellcolor{red!20}\textbf{77.6\%} \\
   10\% & 47.8\% & 45.0\% & 50.3\% & 48.2\% & \cellcolor{red!20}\textbf{65.7\%} \\
   \midrule
   \multicolumn{6}{c}{LlaMa-3-8B-Instruct FullKV: 76.8\% $\uparrow$} \\
   \midrule
   30\% & 70.6\% & 73.6\% & 70.2\% & 68.2\% & \cellcolor{red!20}\textbf{74.6\%} \\
   \midrule
   \multicolumn{6}{c}{Qwen2-7B-Instruct FullKV: 71.1\% $\uparrow$} \\
   \midrule
   30\% & 70.8\% & 61.2\% & 70.8\% & 64.7\% & \cellcolor{red!20}\textbf{73.5\%} \\
   \midrule
   \specialrule{1pt}{0pt}{2pt}
   % \specialrule{1pt}{0pt}{2pt}
   \end{tabular}
   }
     \vspace{-0.2cm}

   \label{tab:GSM8K}
\end{table}


% \subsubsection{GSM8K}  
% \label{sec:gsm8k}
\textbf{GSM8K}
In the in-context learning scenario, we evaluated multiple KV cache compression methods for GSM8K~\citep{gsm8k}, which contains more than 1,000 arithmetic questions on LLaMA-3-8B-Instruct, LLaMA-3.1-8B-Instruct~\citep{meta2024llama3}, Qwen2-7B-Instruct~\citep{qwen2} and DeepSeek-R1-Distill-Llama-8B~\citep{deepseekr1}. Follow the \citet{agarwal2024many}, we consider many-shot GSM8K as a long-context reasoning  scenario, which is a more challenging task than LongBench~\citep{bai2023longbench}. The CoT prompt settings for this experiment are the same as those used by~\citet{wei2022chain}, for many-shot GSM8K we set the number of shots to 50, which the prompt length is more than 4k tokens. For more details on the prompt settings, please refer to the APPENDIX \ref{appendix:prompt}.



\begin{table}[!t]
   \caption{Many-Shot GSM8K Performance Comparison. }
   \centering
   \resizebox{1\columnwidth}{!}{
   \begin{tabular}{l|ccccc}
   % \specialrule{1pt}{0pt}{2pt}
   \specialrule{1pt}{0pt}{2pt}
   \multirow{2}{*}{\textbf{\makecell{ Ratio}}} & \multirow{2}{*}{\makecell{StreamingLLM}} & \multirow{2}{*}{H2O} & \multirow{2}{*}{SnapKV} & \multirow{2}{*}{PyramidKV} & \multirow{2}{*}{\makecell{\textbf{ChunkKV} \\ 
   \textbf{(Ours)}}} \\
   & & & & & \\
   \midrule
   \multicolumn{6}{c}{DeepSeek-R1-Distill-Llama-8B FullKV: 71.2\% $\uparrow$} \\
   \midrule
   10\% & 63.2\% & 54.2\% & 54.1\% & 59.2\% & \cellcolor{red!20}\textbf{68.2\%} \\
   \midrule
   \multicolumn{6}{c}{LlaMa-3.1-8B-Instruct FullKV: 82.4\% $\uparrow$} \\
   \midrule
   10\% & 74.3\% & 51.2\% & 68.2\% & 70.3\% & \cellcolor{red!20}\textbf{79.3\%} \\
   \midrule
   \specialrule{1pt}{0pt}{2pt}
   % \specialrule{1pt}{0pt}{2pt}
   \end{tabular}
   }
 \vspace{-0.2cm}
   \label{tab:many_shot_GSM8K}
\end{table}

% \textbf{GSM8K Results}
Table \ref{tab:GSM8K} presents the performance comparison. The results demonstrate that \method{} outperforms other KV cache compression methods on different models and compression ratios. Table \ref{tab:many_shot_GSM8K} presents the performance comparison of many-shot GSM8K, also \method{} outperforms other KV cache compression methods. The consistent superior performance of \method{} in both models underscores its effectiveness in maintaining crucial contextual information for complex arithmetic reasoning tasks.


% \subsubsection{Jailbreak}
\textbf{Jailbreak} In this section, we evaluate the performance of \method{} on the JailbreakV benchmark~\citep{jailbreakv}. The prompt settings are the same as those used by~\citet{jailbreakv}.

% \textbf{Jailbreak} 
Table \ref{tab:jailbreak} presents the performance comparison. The results demonstrate that \method{} outperforms other KV cache compression methods on different models and compression ratios. Which shows the effectiveness of \method{} in maintaining crucial contextual information for safety benchmark.

\begin{table}[!h]
   \caption{JailbreakV Performance Comparison. }
   \centering
   \resizebox{1\columnwidth}{!}{
   \begin{tabular}{l|ccccc}
   % \specialrule{1pt}{0pt}{2pt}
   \specialrule{1pt}{0pt}{2pt}
   \multirow{2}{*}{\textbf{\makecell{ Ratio}}} & \multirow{2}{*}{\makecell{StreamingLLM}} & \multirow{2}{*}{H2O} & \multirow{2}{*}{SnapKV} & \multirow{2}{*}{PyramidKV} & \multirow{2}{*}{\makecell{\textbf{ChunkKV} \\ 
   \textbf{(Ours)}}} \\
   & & & & & \\
   \midrule
   \multicolumn{6}{c}{LlaMa-3.1-8B-Instruct FullKV: 88.9\% $\uparrow$} \\
   \midrule
   20\% & 65.0\% & 71.7\% & 88.0\% & 87.5\% & \cellcolor{red!20}\textbf{89.0\%} \\
   10\% & 53.1\% & 65.4\% & 84.3\% & 85.5\% & \cellcolor{red!20}\textbf{87.9\%} \\
   \midrule
   \specialrule{1pt}{0pt}{2pt}
   % \specialrule{1pt}{0pt}{2pt}
   \end{tabular}
   }
   \vspace{-0.2cm}

   \label{tab:jailbreak}
         \vspace{-0.2cm}
\end{table}

\subsection{Long-Context Benchmark}\label{sec:long-context-overall}
LongBench and NIAH are two widely used benchmarks for KV cache compression methods. Both benchmarks have a context length that exceeds $10K$. NIAH requires retrieval capability, while LongBench is a meticulously designed benchmark suite that tests the capabilities of language models in handling extended documents and complex information sequences. 




% \subsubsection{LongBench} 
% \label{sec:longbench}
\textbf{LongBench} We use LongBench~\citep{bai2023longbench} to assess the performance of \method{} on tasks involving long-context inputs. For more details on LongBench, please refer to the APPENDIX \ref{appendix:evaluation}. We evaluated multiple KV cache eviction methods using the LongBench benchmark with LLaMA-3-8B-Instruct~\citep{meta2024llama3}, Mistral-7B-Instruct-v0.3~\citep{jiang2023mistral7b}, and Qwen2-7B-Instruct~\citep{qwen2}, with a KV cache compression ratio of $10\%$. The LongBench provides the Chinese subtask, and Qwen2-7B-Instruct also supports Chinese, so we tested Qwen2-7B-Instruct with different KV cache compression methods on the Chinese subtasks. 

% \paragraph{Results}
% Tables~\ref{table:longbench_category_averages} show that \method{} is capable of achieving on-par performance or even better than the full KV cache with less GPU memory consumption. This table is evaluated in the LongBench English subtask, where \method{} outperforms other compression methods overall, and the Qwen2-7B-Instruct model achieves better performance than the full KV cache. In particular, \method{} shows particularly strong performance in Multi-Document QA tasks, highlighting its ability to effectively preserve and utilize the context of the cross document. This suggests that \method{}'s approach of retaining semantic chunks is more effective in preserving important information compared to other discrete token-based compression methods.  For detailed results and Chinese subtask results, please refer to Appendix \ref{appendix:longbench} and \ref{appendix:multilingual}.


% \vspace{-0.8em}
% \textbf{LongBench Results.} 
Tables~\ref{table:longbench_averages} show that \method{} is capable of achieving on-par performance or even better than the full KV cache with less GPU memory consumption. This table presents the performance gap (in percentage) between each method and the FullKV baseline, where negative values indicate performance degradation compared to FullKV. The table is evaluated in the LongBench English subtask, where \method{} outperforms other compression methods overall. This suggests that \method{}'s approach of retaining semantic chunks is more effective in preserving important information compared to other discrete token-based compression methods. For detailed results and Chinese subtask results, please refer to Appendix \ref{appendix:longbench} and \ref{appendix:multilingual}.

\begin{table}[!t]
   \caption{KV cache compression methods on the LongBench benchmark. Results show performance gap compared to FullKV baseline (negative values indicate worse performance). }
   \centering
   \resizebox{1\columnwidth}{!}{
   \begin{tabular}{l|ccccc}
   % \specialrule{1pt}{0pt}{2pt}
   \specialrule{1pt}{0pt}{2pt}
   \multirow{2}{*}{\textbf{\makecell{ Ratio}}} & \multirow{2}{*}{\makecell{StreamingLLM}} & \multirow{2}{*}{H2O} & \multirow{2}{*}{SnapKV} & \multirow{2}{*}{PyramidKV} & \multirow{2}{*}{\makecell{\textbf{ChunkKV} \\ 
   \textbf{(Ours)}}} \\
   & & & & & \\
   \midrule
   \multicolumn{6}{c}{LlaMa-3-8B-Instruct FullKV: 41.46 $\uparrow$} \\
   \midrule
   10\% & -13.80\% & -10.61\% & -3.16\% & -3.33\% & \cellcolor{red!20}\textbf{-2.29\%} \\
   20\% & -6.42\% & -8.85\% & -2.24\% & -2.00\% & \cellcolor{red!20}\textbf{-1.74\%} \\
   30\% & -2.36\% & -5.38\% & -0.07\% & -0.22\% & \cellcolor{red!20}\textbf{+0.31\%} \\
   \midrule
   \multicolumn{6}{c}{Mistral-7B-Instruct-v0.3 FullKV: 48.08 $\uparrow$} \\
   \midrule
   10\% & -16.58\% & -9.30\% & -3.54\% & -3.52\% & \cellcolor{red!20}\textbf{-2.85\%} \\
   \midrule
   \multicolumn{6}{c}{Qwen2-7B-Instruct FullKV: 40.71 $\uparrow$} \\
   \midrule
   10\% & -5.28\% & -0.64\% & -0.39\% & -0.98\% & \cellcolor{red!20}\textbf{+0.42\%} \\
   \specialrule{1pt}{0pt}{2pt}
   % \specialrule{1pt}{0pt}{2pt}
   \end{tabular}
   }
      \vspace{-0.2cm}
   \label{table:longbench_averages}
\end{table}



% \begin{table*}[!ht]
%    \caption{Comparative analysis of KV cache compression methods on the LongBench English subtask, including \method{}, PyramidKV, SnapKV, H2O, StreamingLLM, and FullKV. Results are shown for LlaMa-3-8B-Instruct, Qwen2-7B-Instruct, and Mistral-7B-Instruct models. \method{} demonstrates superior performance across diverse LLM architectures compared to other compression techniques.}
%    \resizebox{\textwidth}{!}{
%    % \small
%    % small the line space
%    \setlength{\tabcolsep}{10pt}
%    \centering
%    \begin{tabular}{l|ccccc|c}
%    % \toprule
%   \specialrule{1pt}{0pt}{2pt}
%  \specialrule{1pt}{0pt}{2pt}
%    \multirow{2}{*}{Method} & Single-Document & Multi-Document & \multirow{2}{*}{Summarization} & Few-shot & Synthetic & \multirow{2}{*}{\textbf{Overall Avg. $\uparrow$}} \\
%     & QA & QA & & Learning & \& Code & \\
%    \midrule
%    Avg len & 8,862 & 8,417 & 7,154 & 6,548 & 6,468 &  7,490 \\
%    \arrayrulecolor{black}\midrule
%    \multicolumn{7}{c}{LlaMa-3-8B-Instruct, KV Size = Full} \\
%    \arrayrulecolor{black}\midrule
%    FullKV & 32.19 & 34.59 & 24.96 & 68.48 & 45.69 & 41.46 \\
%    \arrayrulecolor{black}\midrule
%    \multicolumn{7}{c}{LlaMa-3-8B-Instruct, KV Size Compression Ratio = 10\%} \\
%    \arrayrulecolor{black}\midrule
%    StreamingLLM & 18.60 & 26.64 & 20.64 & 62.15 & 46.96 & 35.74 \\
%    H2O & 24.64 & 31.01 & 22.13 & 57.02 & 47.15 & 37.06 \\
%    SnapKV & \textbf{28.35} & 32.91 & \textbf{22.32} & 67.21 & 47.53 & 40.15 \\
%    PyramidKV & 27.40 & 32.76 & 22.60 & \textbf{67.88} & 47.38 & 40.08 \\
%    \rowcolor{red!20}\textbf{ChunkKV} & 28.50 & \textbf{33.46} & 22.20 & 67.62 & \textbf{48.23} & \textbf{40.51} \\
%    \midrule
%    \multicolumn{7}{c}{LlaMa-3-8B-Instruct, KV Size Compression Ratio = 20\%} \\
%    \arrayrulecolor{black}\midrule
%    StreamingLLM & 25.09 & 30.17 & \textbf{24.29} & 66.89 & 45.40 & 38.80  \\
%    H2O & 27.61 & 31.49 & 23.01 & 59.25 & 45.17 & 37.79 \\
%    SnapKV & 29.96 & 33.33 & 23.91 & 67.98 & 45.74 & 40.53 \\
%    PyramidKV & 30.08 & \textbf{33.76} & 24.14 & \textbf{68.00} & 45.56 & 40.63 \\
%    \rowcolor{red!20}\textbf{ChunkKV} & \textbf{31.05} & 33.22 & 23.58 & 67.86 & \textbf{46.19} & \textbf{40.74}  \\ 
%    \midrule
%    \multicolumn{7}{c}{LlaMa-3-8B-Instruct, KV Size Compression Ratio = 30\%} \\
%    \arrayrulecolor{black}\midrule
%    StreamingLLM & 27.43 & 32.04 & \textbf{25.08} & 68.03 & 47.48 & 40.48 \\
%    H2O & 28.65 & 32.77 & 23.76 & 61.07 & 47.23 & 39.23 \\
%    SnapKV & 31.06 & 33.76 & 24.34 & \textbf{68.40} & 47.55 & 41.43 \\
%    PyramidKV & 30.71 & \textbf{34.05} & 24.62 & 68.14 & 47.36 & 41.37 \\
%    \rowcolor{red!20}\textbf{ChunkKV} & \textbf{31.48} & 33.26 & 24.53 & 68.34 & \textbf{48.15} & \textbf{41.59} \\
%    \midrule
%    \multicolumn{7}{c}{Mistral-7B-Instruct-v0.3, KV Size = Full} \\
%    \arrayrulecolor{black}\midrule
%    FullKV & 41.18 & 38.99 & 29.45 & 70.73 & 57.08 & 48.08 \\
%    \arrayrulecolor{black}\midrule
%    \multicolumn{7}{c}{Mistral-7B-Instruct-v0.3, KV Size Compression Ratio = 10\%} \\
%    \arrayrulecolor{black}\midrule
%    StreamingLLM & 26.90 & 32.09 & 21.50 & 66.01 & 50.58 & 40.11 \\
%    H2O & 37.71 & 37.92 & 24.01 & 59.35 & 53.96 & 43.61 \\
%    SnapKV & 39.61 & 38.77 & \textbf{24.74} & 68.93 & \textbf{56.48} & 46.38 \\
%    PyramidKV & 39.25 & 39.08 & 25.12 & 70.03 & 55.73 & 46.39 \\
%    \rowcolor{red!20}\textbf{ChunkKV} & \textbf{40.19} & \textbf{39.35} & 24.40 & \textbf{70.41} & 56.24 & \textbf{46.71} \\
%    \midrule
%    \multicolumn{7}{c}{Qwen2-7B-Instruct, KV Size = Full} \\
%    \arrayrulecolor{black}\midrule
%    FullKV & 37.35 & 12.18 & 28.78 & 70.62 & 51.17 & 40.71 \\
%    \arrayrulecolor{black}\midrule
%    \multicolumn{7}{c}{Qwen2-7B-Instruct, KV Size Compression Ratio = 10\%} \\
%    \arrayrulecolor{black}\midrule
%    StreamingLLM & 37.34 & 11.44 & 26.84 & 70.40 & 44.36 & 38.56 \\
%    H2O & 37.68 & 11.47 & \textbf{27.29} & 70.09 & \textbf{51.92} & 40.45 \\
%    SnapKV & \textbf{39.53} & 12.51 & 27.05 & 70.30 & 50.18 & 40.55 \\
%    PyramidKV & 38.52 & 11.87 & 26.78 & 70.32 & 50.66 & 40.31 \\
%    \rowcolor{red!20}\textbf{ChunkKV} & 38.57 & \textbf{13.02} & 27.05 & \textbf{70.50} & 51.74 & \textbf{40.88} \\
%    % \bottomrule
%    \specialrule{1pt}{0pt}{2pt}
%    \specialrule{1pt}{0pt}{2pt}

%    \end{tabular}
%    }
%    \label{table:longbench_category_averages}
% \end{table*}

% \renewcommand{\arraystretch}{0.9}



% \subsubsection{Needle-In-A-HayStack}

\textbf{Needle-In-A-HayStack} 
We use Needle-In-A-HayStack (NIAH)~\citep{needle} to evaluate LLMs' long-context retrieval capability. NIAH assesses how well LLM extract hidden tricked information from extensive documents, and follow LLM-as-a-Judge~\citep{zheng2023judging} we apply GPT-4o-mini~\citep{openai2023gpt4omini} to assess the accuracy of the retrieved information. We evaluated multiple KV cache eviction methods using NIAH with LLaMA-3-8B-Instruct and Mistral-7B-Instruct-v0.2, setting benchmark context lengths to 8k and 32k tokens.






% \textbf{Needle-In-A-HayStack Results.}
Table~\ref{tab:main_NIAH} provides statistical results for different compression methods. These findings clearly indicate the effectiveness of ChunkKV in managing varying token lengths and depth percentages, making it a robust choice for KV cache management in LLMs. Figure~\ref{fig:NIAH_llama3} presents the NIAH benchmark results for LLaMA-3-8B-Instruct. The vertical axis represents the depth percentage, while the horizontal axis represents the token length, with shorter lengths on the left and longer lengths on the right. A cell highlighted in green indicates that the method can retrieve the needle at that length and depth percentage. The detail visualization of the NIAH benchmark can be found in Appendix \ref{appendix:niah}. The visualization results demonstrate that \method{} outperforms other KV cache compression methods. 

\begin{table}[!t]
   \caption{NIAH Performance Comparison. }
   \centering
   \resizebox{1\columnwidth}{!}{
   \begin{tabular}{c|ccccc}
   % \specialrule{1pt}{0pt}{2pt}
   \specialrule{1pt}{0pt}{2pt}
   \multirow{2}{*}{\textbf{\makecell{KV cache \\ Size}}} & \multirow{2}{*}{\makecell{StreamingLLM}} & \multirow{2}{*}{H2O} & \multirow{2}{*}{SnapKV} & \multirow{2}{*}{PyramidKV} & \multirow{2}{*}{\makecell{\textbf{ChunkKV} \\ 
   \textbf{(Ours)}}} \\
   & & & & & \\
   \midrule
   \multicolumn{6}{c}{LlaMa-3.1-8B-Instruct FullKV: 74.6\% $\uparrow$} \\
   \midrule
   512 & 32.0\% & 68.6\% & 71.2 \% & 72.6\% & \cellcolor{red!20}\textbf{74.5\%} \\
   256 & 28.0\% & 61.7\% & 68.8\% & 69.5\% & \cellcolor{red!20}\textbf{74.1\%} \\
   128 & 23.7\% & 47.9\% & 58.9\% & 65.1\% & \cellcolor{red!20}\textbf{73.8\%} \\
   96 & 21.5\% & 41.0\% & 56.2\% & 63.2\% & \cellcolor{red!20}\textbf{70.3\%} \\
   \midrule
   \multicolumn{6}{c}{Mistral-7B-Instruct FullKV: 99.8\% $\uparrow$} \\
   \midrule
   128 & 44.3\% & 88.2\% & 91.6\% & 99.3\% & \cellcolor{red!20}\textbf{99.8\%} \\
   \midrule
   \specialrule{1pt}{0pt}{2pt}
   % \specialrule{1pt}{0pt}{2pt}
   \end{tabular}
   }
   \label{tab:main_NIAH}
   \vspace{-0.4cm}
\end{table}


\begin{figure}[h]
   \centering
   \begin{subfigure}[b]{0.49\textwidth}
       \centering
       \includegraphics[width=1\textwidth]{./figs/NIAH/small_fig/llama-3-8B-Instruct_spankv_baseline_128.pdf}
       \caption{\method{}, accuracy 73.8\%}
       \label{fig:NIAH_llama3_spankv}
   \end{subfigure}
   \begin{subfigure}[b]{0.49\textwidth}
      \centering
      \includegraphics[width=1\textwidth]{./figs/NIAH/small_fig/llama-3-8B-Instruct_pyramidkv_baseline_128.pdf}
      \caption{PyramidKV, accuracy 65.1\%}
      \label{fig:NIAH_llama3_pyramidkv}
   \end{subfigure}
   \begin{subfigure}[b]{0.49\textwidth}
      \centering
      \includegraphics[width=\textwidth]{./figs/NIAH/small_fig/llama-3-8B-Instruct_snapkv_baseline_128.pdf}
      \caption{SnapKV, accuracy 58.9\%}
      \label{fig:NIAH_llama3_snapkv}
   \end{subfigure}
   \begin{subfigure}[b]{0.49\textwidth}
      \centering
      \includegraphics[width=\textwidth]{./figs/NIAH/small_fig/llama-3-8B-Instruct_streamingllm_baseline_128.pdf}
      \caption{StreamingLLM, accuracy 23.7\%}
      \label{fig:NIAH_llama3_streamingllm}
   \end{subfigure}
   \caption{ NIAH benchmark for LLaMA3-8B-Instruct  with KV cache size=128 under 8k context length.}
   \label{fig:NIAH_llama3}
   \vspace{-0.3cm}
\end{figure}



\subsection{Index Reuse}
% \begin{figure}[h]
% % \vspace{-15pt}
%    \centering
%    \includegraphics[width=0.5\textwidth]{./figs/Index_Reuse_Efficiency.pdf}
%    \caption{\centering Relative computation time (lower is better) of layer-wise index reuse for KV cache compression in Qwen2 and Mistral models.}
%    % \vspace{-30pt}
%    \label{fig:index_reuse_efficiency}
% \end{figure}
This section will evaluate the performance of the layer-wise index reuse approach with \method{} from the two aspects of efficiency and performance. 
% \subsubsection{Efficiency}

% \paragraph{Settings}
\textbf{Measuring Efficiency.}
We evaluated the latency and throughput of ChunkKV compared to FullKV using LLaMA3-8B-Instruct on an A40 GPU. All experiments were conducted with reuse layer is 2, batch size set to 1 and inference was performed using Flash Attention 2, each experiment was repeated 10 times and the average latency and throughput were reported.


The results in Table \ref{tab:efficiency} shows that the layer-wise index reuse strategy (ChunkKV\_reuse) further boosts performance, achieving up to a 20.7\% reduction in latency, and throughput improvements are particularly notable for longer input sequences, with ChunkKV\_reuse delivering up to a 26.5\% improvement over FullKV.

\begin{table}[!t]
   \caption{Latency and throughput comparison between ChunkKV and FullKV under different input-output configurations. Percentages in parentheses indicate improvements over FullKV baseline.}
   \label{tab:efficiency}
   \resizebox{0.5\textwidth}{!}{
   \begin{tabular}{l|cc|cc}
   \toprule
   \multirow{2}{*}{Method} & \multicolumn{2}{c|}{Sequence Length} & \multicolumn{2}{c}{Performance Metrics} \\
   \cmidrule{2-5}
   & Input & Output & Latency(s) $\downarrow$ & Throughput(T/S) $\uparrow$ \\
   \midrule
   FullKV & 4096 & 1024 & 43.60 & 105.92  \\
   ChunkKV & 4096 & 1024 & 37.52 (13.9\%) & 118.85 (12.2\%) \\
   ChunkKV\_reuse & 4096 & 1024 & \textbf{37.35} (14.3\%) & \textbf{124.09} (17.2\%) \\
   \midrule
   FullKV & 4096 & 4096 & 175.50 & 37.73 \\
   ChunkKV & 4096 & 4096 & 164.55 (6.2\%) & 40.58 (7.6\%) \\
   ChunkKV\_reuse & 4096 & 4096 & \textbf{162.85} (7.2\%) & \textbf{41.12} (9.0\%) \\
   \midrule
   FullKV & 8192 & 1024 & 46.48 & 184.08 \\
   ChunkKV & 8192 & 1024 & 37.83 (18.6\%) & 228.96 (24.4\%) \\
   ChunkKV\_reuse & 8192 & 1024 & \textbf{36.85} (20.7\%) & \textbf{232.99} (26.5\%) \\
   \midrule
   FullKV & 8192 & 4096 & 183.42 & 55.93 \\
   ChunkKV & 8192 & 4096 & 164.78 (10.2\%) & 65.14 (16.5\%) \\
   ChunkKV\_reuse & 8192 & 4096 & \textbf{162.15} (11.6\%) & \textbf{66.05} (18.1\%) \\
   \bottomrule
   \end{tabular}
   }
\end{table}



\begin{figure}[h]
   % \vspace{-10pt}
   \centering
   \includegraphics[width=0.43\textwidth]{./figs/Index_Reuse_LongBench_Performance.pdf}
   \caption{ Comparison with different index reuse layers on LongBench.}
   \label{fig:index_reuse_performance}
   \vspace{-10pt}
\end{figure}






% \paragraph{Settings}
\textbf{Measuring Task Performance.}
This experiment evaluates the performance of the layer-wise index reuse approach by measuring the performance of the LongBench~\citep{bai2023longbench}, the experiment settings are the same as LongBench in \ref{sec:long-context-overall}. And the number of index reuse layers is set from 1 to the number of layers in the model, where an index reuse layer of 1 corresponds to the normal \method{} without index reuse, and our method set reuse layer to 2. 

Figure~\ref{fig:index_reuse_performance} illustrates the performance of \method{} with varying index reuse layers on the LongBench benchmark. Generally, reuse layer set to 2 can achieve the minimal performance degradation across all models. For more experiments on index reuse, please refer to the APPENDIX \ref{appendix:index_reuse}.

% \begin{table}[h]
%    \centering
%    % \vspace{-10pt}
%    \caption{\centering Performance Degradation Rate \\ of Layer-wise Index Reuse.}
%    \resizebox{0.35\textwidth}{!}{
%    \begin{tabular}{l|c}
%    \toprule
%    Model & \makecell{Performance \\ Decrease Rate (\%)} \\
%    \midrule
%    LLaMA-3-8B & 11.95 \\
%    Mistral-7B & 15.31 \\
%    Qwen2-7B & 10.00 \\
%    \bottomrule
%    \end{tabular}
%    }
%    % \vspace{-40pt}
%    \label{tab:performance_decrease_rate}
% \end{table}

Overall, these findings on efficiency and performance suggest that layer-wise index reuse can be an effective technique for optimizing the efficiency-performance trade-off in KV cache compression, with the potential for model-specific tuning to maximize benefits.

\begin{figure}[t]
   \centering
   \includegraphics[width=0.43\textwidth]{./figs/LongBench_Performance_vs_Chunk_Size.pdf}
   \caption{LongBench Performance Comparison with different chunk size under 10\% compression rate.}
   \label{fig:chunk_size_performance}
      \vspace{-0.4cm}
   % \vspace{-35pt}
\end{figure}

\section{Ablation study}
\subsection{Chunk Size}
This section aims to investigate the impact of chunk size on the performance of \method{}. Different chunk sizes will lead to varying degrees of compression on the semantic information of the data. We set the experiemnt setting the same as in LongBench in Section \ref{sec:long-context-overall}. The chunk size is set from the range $\{1,3,5,10,20,30\}$. Figure~\ref{fig:chunk_size_performance} shows the performance of the \method{} with different chunk size on the LongBench and NIAH benchmarks. The three colorful curves represent three LLMs with different chunk sizes, and the colorful dashed line is the corresponding FullKV performance. For more experiments on the size of the chunks with different compression ratios, refer to the Appendix \ref{appendix:chunk_size}.


\begin{table}[!h]
   \centering
      \vspace{-0.2cm}
   \caption{LongBench Performance with Different Chunk Sizes and Compression Ratios for LLaMA-3-8B-Instruct}
   \resizebox{1\columnwidth}{!}{%
   \begin{tabular}{c|ccccccc}
   \toprule
   Compression & \multicolumn{7}{c}{Chunk Size} \\
   \cmidrule{2-8}
   Rate & 1 & 3 & 5 & 10 & 15 & 20 & 30 \\
   \midrule
   10\% & 37.32 & 40.49 & 40.47 & \textbf{40.51} & 40.21 & 40.05 & 39.57 \\
   20\% & 38.80 & 40.66 & 40.57 & \textbf{40.74} & 40.53 & 40.46 & 40.04 \\
   30\% & 39.23 & 41.02 & 41.29 & \textbf{41.59} & 41.38 & 41.33 & 41.02 \\
   \bottomrule
   \end{tabular}%
   }
  \vspace{-0.3cm}
   \label{tab:ablation_size_compression}
\end{table}

From Figure~\ref{fig:chunk_size_performance}, we can observe that the LongBench performance of \method{} is not significantly affected by the chunk size, with performance variations less than 1\%. The three curves are closely aligned, indicating that chunk sizes in the range of $\{10,20\}$ exhibit better performance. 
% This finding aligns with the nature of semantic chunks in natural language, where different chunk sizes can retain different semantic information.




Table \ref{tab:ablation_size_compression}  and \ref{tab:ablation_size_niah} show the performance of \method{} with different comperession ratios and different chunk sizes on the LongBench and NIAH. We conducted extensive experiments across different compression ratios and KV cache sizes to shows the effectiveness of \method{} and the chunk size is robust.


\begin{table}[!h]
   \centering
   \caption{NIAH Performance with Different Chunk Sizes and KV Cache Sizes for LLaMA-3-8B-Instruct}
   \resizebox{1\columnwidth}{!}{%
   \begin{tabular}{c|ccccccc}
   \toprule
   KV Cache & \multicolumn{7}{c}{Chunk Size} \\
   \cmidrule{2-8}
   Size & 1 & 3 & 5 & 10 & 15 & 20 & 30 \\
   \midrule
   96 & 41.0 & 63.2 & 65.2 & \textbf{70.3} & 67.2 & 65.3 & 53.1 \\
   128 & 47.9 & 65.6 & 69.1 & \textbf{73.8} & 72.3 & 72.0 & 71.2 \\
   256 & 61.7 & 70.3 & 71.2 & \textbf{74.1} & 73.2 & 72.3 & 71.1 \\
   512 & 68.6 & 72.6 & 72.5 & \textbf{74.5} & 74.3 & 74.0 & 72.6 \\
   \bottomrule
   \end{tabular}%
   }
   \vspace{-0.3cm}
   \label{tab:ablation_size_niah}
\end{table}

From the chunk size ablation study, we can observe that across different tasks (LongBench and NIAH) and various compression settings, a chunk size of 10 consistently delivers optimal or near-optimal performance. This empirical finding suggests that a chunk size of 10 strikes a good balance between preserving semantic information and compression efficiency, making it a robust default choice for \method{}. Therefore, we adopt this chunk size setting throughout our experiments.

% The optimal performance in the $\{10,20\}$ range can be attributed to these sizes, which capture a balance between the local context and the broader semantic units. Smaller chunks (e.g., 3 or 5 tokens) might preserve very local relationships but risk fragmenting larger semantic structures. Conversely, larger chunks (e.g., 30 tokens) might encompass too much information, potentially including less relevant content and diluting the most salient semantic features.

\section{Conclusion}
We introduced ChunkKV, a novel KV cache compression method that preserves semantic information by retaining more informative chunks. Through extensive experiments across multiple state-of-the-art LLMs (including DeepSeek-R1, LLaMA-3, Qwen2, and Mistral) and diverse benchmarks (GSM8K, LongBench, NIAH, and JailbreakV), we demonstrate that ChunkKV consistently outperforms existing methods while using only a fraction of the memory. Our comprehensive analysis shows that ChunkKV's chunk-based approach maintains crucial contextual information, leading to superior performance in complex reasoning tasks, long-context understanding, and safety evaluations. The method's effectiveness is particularly evident in challenging scenarios like many-shot GSM8K and multi-document QA tasks, where semantic coherence is crucial. Furthermore, our proposed layer-wise index reuse technique provides significant computational efficiency gains with minimal performance impact, achieving up to 20.7\% latency reduction and 26.5\% throughput improvement. These findings, supported by detailed quantitative analysis and ablation studies, establish ChunkKV as a significant advancement in KV cache compression technology, offering an effective solution for deploying LLMs in resource-constrained environments while maintaining high-quality outputs.

% \section{Future Discussion}
% While our work demonstrates the effectiveness of semantic chunk-based KV cache compression, several promising directions warrant further investigation:

% \begin{itemize}
%     \item \textbf{Dynamic Chunk Sizing:} Future research could explore adaptive chunk size mechanisms that automatically adjust based on the semantic structure of the input text and the specific requirements of different tasks.
    
%     \item \textbf{Multi-Modal Extension:} The principles of semantic chunk preservation could be extended to multi-modal models, investigating how to effectively compress and maintain cross-modal relationships in KV caches.
    
%     \item \textbf{Task-Specific Optimization:} Further research could explore task-specific adaptations of ChunkKV, particularly for specialized applications like mathematical reasoning or code generation where certain patterns of semantic information may be more critical.
    
%    %  \item \textbf{Theoretical Foundations:} Deeper theoretical analysis of the relationship between semantic chunk preservation and model performance could provide insights for future compression methods and model architectures.
    
%    %  \item \textbf{Efficiency-Performance Trade-offs:} Investigation into more sophisticated index reuse strategies and their impact on different model architectures could lead to better optimization of the efficiency-performance trade-off.
% \end{itemize}

% These directions could lead to even more efficient and effective KV cache compression methods, further advancing the deployment of large language models in practical applications.

\section*{Impact Statement}
Our study does not involve human subjects, data collection from individuals, or experiments on protected groups. The models and datasets used in this work are publicly available and widely used in the research community. We have made efforts to ensure our experimental design and reporting of results are fair, unbiased, and do not misrepresent the capabilities or limitations of the methods presented.

In our work on KV cache compression for large language models, we acknowledge the potential broader impacts of improving efficiency in AI systems. While our method aims to reduce computational resources and potentially increase accessibility of these models, we recognize that more efficient language models could also lead to increased deployment and usage, which may have both positive and negative societal implications. We encourage further research and discussion on the responsible development and application of such technologies.

We declare no conflicts of interest that could inappropriately influence our work. All experiments were conducted using publicly available resources, and our code will be made available to ensure reproducibility. We have made every effort to cite relevant prior work appropriately and to accurately represent our contributions in the context of existing research.

\bibliography{icml2025_conference}
\bibliographystyle{icml2025}
% \newpage
% \appendix
% \onecolumn

\clearpage
\onecolumn
\appendix 
\etocdepthtag.toc{mtappendix}
\etocsettagdepth{mtchapter}{none}
\etocsettagdepth{mtappendix}{subsection}
\renewcommand{\contentsname}{Appendix}
\tableofcontents 
\clearpage


\section{In-depth Analysis of ChunkKV vs. Discrete Token Methods}
\label{appendix:chunkkv_analysis}

\subsection{Quantitative Analysis}


To rigorously evaluate the effectiveness of ChunkKV compared to discrete token-based methods, we conducted systematic experiments using a LLaMA-3-8B-Instruct model. We randomly selected 100 sequences from the each sub-category of LongBench dataset and analyzed two key metrics across different model layers: KV cache L1 loss and attention cosine similarity. For each sequence, we:
1. Computed the full KV cache and attention patterns without compression as ground truth.
2. Applied ChunkKV, SnapKV, and H2O compression methods with a fixed 10\% compression ratio, and the parameters of the three methods are set the same as in Table \ref{table:longbench}.
3. Measured the differences between compressed and uncompressed versions.
\begin{figure*}[ht]
   \centering
   \includegraphics[width=1\textwidth]{./figs/attn_metrics_comparison.pdf}
   \caption{\centering Layer-wise comparison of L1 loss and attention cosine similarity between ChunkKV and discrete token-based methods in Single-Document QA sub-category of LongBench.}
   \label{fig:chunkkv_vs_discrete_token_quantitative}
\end{figure*}

\textbf{Results Analysis} As shown in Figure \ref{fig:chunkkv_vs_discrete_token_quantitative}, ChunkKV demonstrates superior performance across both metrics:

\begin{itemize}
\item \textbf{KV Cache L1 Loss:} ChunkKV achieves consistently lower L1 loss compared to SnapKV and H2O, particularly in the early and middle layers (layers 5-25). This indicates better preservation of the original KV cache information through the semantic chunk-based approach.

\item \textbf{Attention Cosine Similarity:} ChunkKV exhibits higher similarity scores across most layers, with notably strong performance in layers 0-5 and 20-30. This suggests better preservation of attention relationships between tokens, which is crucial for maintaining semantic understanding.
\end{itemize}

To quantify these improvements, we calculated average metrics across all layers, as shown in Table \ref{tab:quantitative_metrics_detailed}. ChunkKV achieves both the lowest L1 loss and highest attention cosine similarity, outperforming both baseline methods.


\begin{table*}[h]
   \caption{Detailed comparison of KV cache metrics across different task categories in LongBench.}
   \label{tab:quantitative_metrics_detailed}
   % \resizebox{1\textwidth}{!}{
   \centering
   \begin{tabular}{l|ccccc}
   \toprule
   \multirow{2}{*}{Method} & Single-Document & Multi-Document & \multirow{2}{*}{Summarization} & Few-shot & Synthetic \\
    & QA & QA & & Learning & \& Code \\
   \midrule
   \multicolumn{6}{c}{KV Cache L1 Loss $\downarrow$} \\
   \midrule
   \rowcolor{red!20}\textbf{ChunkKV} & \textbf{0.8741} & \textbf{0.8748} & \textbf{0.8770} & \textbf{0.8861} & \textbf{0.8726} \\
   SnapKV & 0.8921 & 0.8933 & 0.8930 & 0.8917 & 0.8938 \\
   H2O & 0.8905 & 0.8917 & 0.8913 & 0.8906 & 0.8915 \\
   \midrule
   \multicolumn{6}{c}{Attention Score Cosine Similarity $\uparrow$} \\
   \midrule
   \rowcolor{red!20}\textbf{ChunkKV} & \textbf{0.3567} & \textbf{0.3651} & \textbf{0.3841} & \textbf{0.4330} & \textbf{0.3805} \\
   SnapKV & 0.3513 & 0.3594 & 0.3771 & 0.4305 & 0.3759 \\
   H2O & 0.3491 & 0.3572 & 0.3750 & 0.4284 & 0.3740 \\
   \bottomrule
   \end{tabular}
   % }
   \end{table*}

% \begin{table}[h]
% \caption{Average KV cache L1 loss and attention cosine similarity across all layers.}
% \label{tab:quantitative_metrics}
% \resizebox{1\textwidth}{!}{
% \centering
% \begin{tabular}{l|cc}
% \toprule
% Method & KV Cache L1 Loss $\downarrow$ & Attention Cosine Similarity $\uparrow$ \\
% \midrule
% \rowcolor{red!20}ChunkKV & \textbf{0.8741} & \textbf{0.3567} \\
% SnapKV & 0.8921 & 0.3513 \\
% H2O & 0.8905 & 0.3491 \\
% \bottomrule
% \end{tabular}
% }
% \end{table}

\textbf{Significance of Results} While the improvements may appear modest in absolute terms (approximately 2\% in L1 loss and 1.5\% in cosine similarity), their practical significance is substantial. These metrics reflect the model's ability to maintain crucial semantic relationships and attention patterns, which are essential for complex reasoning tasks. The consistent improvements across different sequences demonstrate that preserving semantic chunks leads to better information retention than selecting individual tokens.

The enhanced performance is particularly evident in the middle layers of the model, which are typically responsible for higher-level semantic processing. This provides concrete evidence for why ChunkKV achieves superior performance on downstream tasks compared to discrete token-based methods.

\subsection{Hypothetical Scenario}

To provide a deeper understanding of ChunkKV's effectiveness compared to discrete token-based methods, we present a detailed analysis using a hypothetical scenario. This analysis aims to illustrate the fundamental differences between these approaches and explain why ChunkKV is more effective at preserving semantic information in long contexts.


Consider a comprehensive document that contains detailed information on various animals, including their habitats, diets, and behaviors. A user asks the question "What do pandas eat in the wild?"

Both ChunkKV and discrete token-based methods would use this question to calculate observation scores for the document. However, their approaches to selecting and retaining information differ significantly.

\subsubsection{Discrete Token-based Method}
A discrete token-based method might identify and retain individual tokens with high relevance scores, such as:
\begin{itemize}
    \item ``pandas",``eat", ``bamboo", ``wild", ``diet", ``food"
\end{itemize}

Although these tokens are relevant, they lack context and coherence. The method might discard other essential tokens that provide crucial context or complete the information.

\subsubsection{ChunkKV Method}
In contrast, ChunkKV would identify and retain semantically meaningful chunks, such as:
\begin{itemize}
    \item ``In the wild, pandas primarily eat bamboo shoots and leaves"
    \item ``Their diet consists of 99\% bamboo, but they occasionally consume other vegetation"
    \item ``Wild pandas may also eat small rodents or birds when available"
\end{itemize}

By preserving these chunks, ChunkKV maintains not only the relevant keywords but also their contextual relationships and additional pertinent information.

\subsection{Comparative Analysis}
The advantages of ChunkKV become evident when we consider how these retained pieces of information would be used in subsequent processing:

\begin{enumerate}
    \item \textbf{Contextual Understanding}: Discrete tokens require the model to reconstruct meaning from isolated words, which could lead to ambiguity. ChunkKV provides complete phrases or sentences, allowing for immediate and accurate comprehension.
    
    \item \textbf{Semantic Coherence}: ChunkKV preserves the semantic relationships within a chunk, crucial to understanding nuances such as the difference between primary and occasional food sources for pandas.
    
    \item \textbf{Information Density}: A single chunk can contain multiple relevant tokens in their proper context, potentially retaining more useful information within the same compressed cache size compared to discrete methods.
    
    \item \textbf{Reduced Ambiguity}: Discrete methods might retain the token ``eat" from various sentences about different animals. ChunkKV ensures that ``eat" is preserved specifically in the context of pandas in the wild.
    
    \item \textbf{Temporal and Logical Flow}: ChunkKV can maintain the sequence of ideas present in the original text, preserving any temporal or logical progression that may be crucial for understanding.
\end{enumerate}

\subsection{Implications for Model Performance}
This analysis suggests several key implications for model performance:

\begin{itemize}
    \item \textbf{Improved Accuracy}: By retaining contextually rich information, ChunkKV enables more accurate responses to queries, especially those requiring nuanced understanding.
    
    \item \textbf{Enhanced Long-context Processing}: Preservation of semantic chunks allows for better handling of long-range dependencies and complex reasoning tasks.
    
    \item \textbf{Reduced Computational Overhead}: Although both methods compress the KV cache, ChunkKV's approach may reduce the need for extensive context reconstruction, potentially improving inference efficiency.
    
    \item \textbf{Versatility}: The chunk-based approach is likely to be more effective across a wide range of tasks and domains as it preserves the natural structure of language.
\end{itemize}

This in-depth analysis demonstrates why ChunkKV is more effective in preserving semantic information in long contexts. By retaining coherent chunks of text, it provides language models with more contextually rich and semantically complete information, leading to improved performance in tasks that require deep understanding and accurate information retrieval from extensive documents.



\section{Additional Experiments}


% \subsection{Efficiency}
% \label{appendix:efficiency}

% We evaluated the latency and throughput of ChunkKV compared to FullKV using LLaMA3-8B-Instruct on an A40 GPU. All experiments were conducted with a batch size of 1 and inference was performed using Flash Attention 2, each experiment was repeated 10 times and the average latency and throughput were reported. The results demonstrate that ChunkKV not only maintains competitive performance but also achieves improved efficiency, which is further enhanced by layer-wise index reuse.

% \begin{table*}[ht]
%    \caption{Latency and throughput comparison between ChunkKV and FullKV under different input-output configurations. Percentages in parentheses indicate improvements over FullKV baseline.}
%    \label{tab:efficiency}
%    \resizebox{\textwidth}{!}{
%    \begin{tabular}{l|cc|cc}
%    \toprule
%    \multirow{2}{*}{Method} & \multicolumn{2}{c|}{Sequence Length} & \multicolumn{2}{c}{Performance Metrics} \\
%    \cmidrule{2-5}
%    & Input & Output & Latency(s) $\downarrow$ & Throughput(T/S) $\uparrow$ \\
%    \midrule
%    FullKV & 4096 & 1024 & 43.60 & 105.92  \\
%    ChunkKV & 4096 & 1024 & 37.52 (13.9\%) & 118.85 (12.2\%) \\
%    ChunkKV\_reuse & 4096 & 1024 & \textbf{37.35} (14.3\%) & \textbf{124.09} (17.2\%) \\
%    \midrule
%    FullKV & 4096 & 4096 & 175.50 & 37.73 \\
%    ChunkKV & 4096 & 4096 & 164.55 (6.2\%) & 40.58 (7.6\%) \\
%    ChunkKV\_reuse & 4096 & 4096 & \textbf{162.85} (7.2\%) & \textbf{41.12} (9.0\%) \\
%    \midrule
%    FullKV & 8192 & 1024 & 46.48 & 184.08 \\
%    ChunkKV & 8192 & 1024 & 37.83 (18.6\%) & 228.96 (24.4\%) \\
%    ChunkKV\_reuse & 8192 & 1024 & \textbf{36.85} (20.7\%) & \textbf{232.99} (26.5\%) \\
%    \midrule
%    FullKV & 8192 & 4096 & 183.42 & 55.93 \\
%    ChunkKV & 8192 & 4096 & 164.78 (10.2\%) & 65.14 (16.5\%) \\
%    ChunkKV\_reuse & 8192 & 4096 & \textbf{162.15} (11.6\%) & \textbf{66.05} (18.1\%) \\
%    \bottomrule
%    \end{tabular}
%    }
% \end{table*}
% \begin{table*}[ht]
%    \caption{Latency and throughput comparison between ChunkKV and FullKV under different input-output configurations. Percentages in parentheses indicate improvements over FullKV baseline.}
%    \label{tab:efficiency}
%    \centering
%    \resizebox{0.6\textwidth}{!}{
%    \begin{tabular}{l|cc|cc}
%    \toprule
%    \multirow{2}{*}{Method} & \multicolumn{2}{c|}{Sequence Length} & \multicolumn{2}{c}{Performance Metrics} \\
%    \cmidrule{2-5}
%    & Input & Output & Latency(s) $\downarrow$ & Throughput(T/S) $\uparrow$ \\
%    \midrule
%    FullKV & 4096 & 1024 & 43.60 & 105.92  \\
%    ChunkKV & 4096 & 1024 & 37.52 (13.9\%) & 118.85 (12.2\%) \\
%    ChunkKV\_reuse & 4096 & 1024 & \textbf{37.35} (14.3\%) & \textbf{124.09} (17.2\%) \\
%    \midrule
%    FullKV & 4096 & 4096 & 175.50 & 37.73 \\
%    ChunkKV & 4096 & 4096 & 164.55 (6.2\%) & 40.58 (7.6\%) \\
%    ChunkKV\_reuse & 4096 & 4096 & \textbf{162.85} (7.2\%) & \textbf{41.12} (9.0\%) \\
%    \midrule
%    FullKV & 8192 & 1024 & 46.48 & 184.08 \\
%    ChunkKV & 8192 & 1024 & 37.83 (18.6\%) & 228.96 (24.4\%) \\
%    ChunkKV\_reuse & 8192 & 1024 & \textbf{36.85} (20.7\%) & \textbf{232.99} (26.5\%) \\
%    \midrule
%    FullKV & 8192 & 4096 & 183.42 & 55.93 \\
%    ChunkKV & 8192 & 4096 & 164.78 (10.2\%) & 65.14 (16.5\%) \\
%    ChunkKV\_reuse & 8192 & 4096 & \textbf{162.15} (11.6\%) & \textbf{66.05} (18.1\%) \\
%    \bottomrule
%    \end{tabular}
%    }
% \end{table*}

% The results in Table \ref{tab:efficiency} highlight several key findings:

% \begin{itemize} 
%     \item ChunkKV consistently outperforms FullKV across all configurations, achieving latency improvements ranging from 6.2\% to 18.6\%. 
%     \item The layer-wise index reuse strategy (ChunkKV\_reuse) further boosts performance, achieving up to a 20.7\% reduction in latency. 
%     \item Throughput improvements are particularly notable for longer input sequences, with ChunkKV\_reuse delivering up to a 26.5\% improvement over FullKV.
% \end{itemize}

% These efficiency gains are even more pronounced with longer input sequences, demonstrating that ChunkKV is particularly well-suited for processing long-context inputs while maintaining minimal memory overhead.

\subsection{Layer-Wise Index Reuse}

\subsubsection{Efficiency Analysis}
\label{appendix:index_reuse_efficiency}
The layer-wise index reuse method significantly reduces the computational complexity of \method{}. Without index reuse, \method{} would be applied to all $N_{\text{layers}}$ layers, resulting in a total compression time of $N_{\text{layers}} \cdot T_{\text{compress}}$, where $T_{\text{compress}}$ is the time taken to compress one layer. With index reuse, \method{} is only applied to $\frac{N_{\text{layers}}}{N_{\text{reuse}}}$ layers, reducing the total time to $\frac{N_{\text{layers}}}{N_{\text{reuse}}} \cdot T_{\text{compress}} + (N_{\text{layers}} - \frac{N_{\text{layers}}}{N_{\text{reuse}}}) \cdot T_{\text{select}}$, where $T_{\text{select}}$ is the time taken to select indices, which is typically much smaller than $T_{\text{compress}}$. This results in a theoretical speedup factor of:

\[
\text{Speedup} = \frac{N_{\text{layers}} \cdot T_{\text{compress}}}{\frac{N_{\text{layers}}}{N_{\text{reuse}}} \cdot T_{\text{compress}} + (N_{\text{layers}} - \frac{N_{\text{layers}}}{N_{\text{reuse}}}) \cdot T_{\text{select}}}
\]

Assuming $T_{\text{select}}$ is negligible compared to $T_{\text{compress}}$, this simplifies to approximately $N_{\text{reuse}}$. In practice, the actual speedup may vary depending on the specific implementation and hardware, but it can still lead to substantial time savings, especially for models with a large number of layers. 

\subsubsection{Layer-Wise Index Similarity}
\label{appendix:index_reuse_similarity}


This section details the experiment of layer-wise index reuse similarity described in Section \ref{sec:layer_wise_index_reuse}. The inference prompt is randomly selected from the LongBench benchmark, and the preserved indices for H2O, SnapKV, and \method{} are saved in the log file. For multi-head attention, only the indices of the first head are saved. PyramidKV, which has varying preserved index sizes across different layers, is not applicable for this experiment. Then we calculate the Jaccard similarity of the preserved indices of adjacent layers for different models. Table \ref{tab:app_jaccard_similarity_models} shows the Jaccard similarity of the preserved indices of adjacent layers for different models.

\begin{table}[!ht]
   \centering
   \caption{Retained KV Cache Indices Similarity of Adjacent Layers for Different Models.}
   \resizebox{0.5\textwidth}{!}{
   \begin{tabular}{l|ccc}
   \toprule
   \textbf{Method} & \textbf{H2O} & \textbf{SnapKV} & \textbf{\method{}}  \\
   \midrule
   LLaMA-3-8B-Instruct     & 25.31\%& 27.95\% & \textbf{57.74\%} \\
   Qwen2-7B-Instruct        & 14.91\%& 16.50\% & \textbf{44.26\%} \\
   Mistral-7B-Instruct      & 15.15\% & 15.78\% & \textbf{52.16\%} \\
   \bottomrule
   \end{tabular}
   }
   \label{tab:app_jaccard_similarity_models}
\end{table}

Figures \ref{fig:app_index_reuse_heatmap_llama_h2o}-\ref{fig:app_index_reuse_heatmap_llama_chunkkv} (LLaMA-3-8B-Instruct), 
\ref{fig:app_index_reuse_heatmap_mistral_h2o}-\ref{fig:app_index_reuse_heatmap_mistral_chunkkv} (Mistral-7B-Instruct), and
\ref{fig:app_index_reuse_heatmap_qwen_h2o}-\ref{fig:app_index_reuse_heatmap_qwen_chunkkv} (Qwen2-7B-Instruct) display the heatmaps of layer-wise indices similarity of the preserved KV cache indices by H2O, SnapKV and \method{} on different models. 
The pattern of the layer-wise indices similarity heatmap is consistent across different models, aligning with our findings in Section \ref{sec:layer_wise_index_reuse}.
\clearpage
\begin{figure*}[!ht]
   \centering
      \includegraphics[width=0.6\textwidth]{./figs/layer_similarity_heatmap_llama_h2o_1example.pdf}
      \caption{ Layer-wise similarity heatmaps of the preserved KV cache indices by H2O on LLaMA-3-8B-Instruct}
      \label{fig:app_index_reuse_heatmap_llama_h2o}
\end{figure*}

\begin{figure*}[!ht]
   \centering
   \includegraphics[width=0.6\textwidth]{./figs/layer_similarity_heatmap_llama_snapkv_1example.pdf}
   \caption{Layer-wise similarity heatmaps of the preserved KV cache indices by SnapKV on LLaMA-3-8B-Instruct}
   \label{fig:app_index_reuse_heatmap_llama_snapkv}   
\end{figure*}

\begin{figure*}[!ht]
   \centering
   \includegraphics[width=0.6\textwidth]{./figs/layer_similarity_heatmap_llama_spankv_1example.pdf}
   \caption{Layer-wise similarity heatmaps of the preserved KV cache indices by ChunkKV on LLaMA-3-8B-Instruct}
   \label{fig:app_index_reuse_heatmap_llama_chunkkv}   
\end{figure*}

\begin{figure*}[!ht]
   \centering
      \includegraphics[width=0.6\textwidth]{./figs/layer_similarity_heatmap_mistral_h2o_1example.pdf}
      \caption{ Layer-wise similarity heatmaps of the preserved KV cache indices by H2O on Mistral-7B-Instruct}
      \label{fig:app_index_reuse_heatmap_mistral_h2o}
\end{figure*}

\begin{figure*}[!ht]
   \centering
   \includegraphics[width=0.6\textwidth]{./figs/layer_similarity_heatmap_mistral_snapkv_1example.pdf}
   \caption{ Layer-wise similarity heatmaps of the preserved KV cache indices  by SnapKV on Mistral-7B-Instruct}
   \label{fig:app_index_reuse_heatmap_mistral_snapkv}   
\end{figure*}

\begin{figure*}[!ht]
   \centering
   \includegraphics[width=0.6\textwidth]{./figs/layer_similarity_heatmap_mistral_spankv_1example.pdf}
   \caption{ Layer-wise similarity heatmaps of the preserved KV cache indices  by ChunkKV on Mistral-7B-Instruct}
   \label{fig:app_index_reuse_heatmap_mistral_chunkkv}   
\end{figure*}

\begin{figure*}[!ht]
   \centering
      \includegraphics[width=0.6\textwidth]{./figs/layer_similarity_heatmap_qwen_h2o_1example.pdf}
      \caption{ Layer-wise similarity heatmaps of the preserved KV cache indices  by H2O on Qwen2-7B-Instruct}
      \label{fig:app_index_reuse_heatmap_qwen_h2o}
\end{figure*}

\begin{figure*}[!ht]
   \centering
   \includegraphics[width=0.6\textwidth]{./figs/layer_similarity_heatmap_qwen_snapkv_1example.pdf}
   \caption{ Layer-wise similarity heatmaps of the preserved KV cache indices  by SnapKV on Qwen2-7B-Instruct}
   \label{fig:app_index_reuse_heatmap_qwen_snapkv}   
\end{figure*}

\begin{figure*}[!ht]
   \centering
   \includegraphics[width=0.6\textwidth]{./figs/layer_similarity_heatmap_qwen_spankv_1example.pdf}
   \caption{ Layer-wise similarity heatmaps of the preserved KV cache indices  by ChunkKV on Qwen2-7B-Instruct}
   \label{fig:app_index_reuse_heatmap_qwen_chunkkv}   
\end{figure*}
% \clearpage



\subsubsection{Index Reuse Performance}
\label{appendix:index_reuse}
Figure~\ref{fig:index_reuse_performance_gsm8k} illustrates the performance of \method{} with varying index reuse layers on the GSM8K benchmark. The experiment reveals that math problems are more sensitive to index reuse layers compared to LongBench. Both LLaMA3-8B-Instruct and Qwen2-7B-Instruct exhibit significant performance degradation, with LLaMA3-8B-Instruct experiencing a steeper decline after two layers of index reuse than Qwen2-7B-Instruct. This suggests that the Qwen2-7B-Instruct model may be more robust to index reuse.

\begin{figure*}[!ht]
   \centering
   \includegraphics[width=0.7\textwidth]{./figs/Index_Reuse_GSM8K_Performance.pdf}
   \caption{\centering GSM8K Performance Comparison with different index reuse layers}
   \label{fig:index_reuse_performance_gsm8k}
\end{figure*}

Table \ref{tab:reuse_GSM8K} shows the performance of \method{} with different numbers of index reuse layers in GSM8K. The number of index reuse layers is set from 1 to the number of layers in the model, where a index reuse layer of 1 corresponds to the normal \method{} without index reuse, and 28/32 is the maximum number of layers for LLaMA-3-8B-Instruct and Qwen2-7B-Instruct. The significant performance drop of LLaMA-3-8B-Instruct raises another question: whether the KV cache compression method is more sensitive to the model's mathematical reasoning ability.
\begin{table*}[!ht]
   \centering
   \caption{Reusing Indexing Performance Comparison on GSM8K}
   % \resizebox{0.8\textwidth}{!}{
   \begin{tabular}{c|cccccccc}
   \toprule
   \multirow{3}{*}{Model} & \multicolumn{8}{c}{Number of Index Reuse Layers} \\
   \cmidrule(lr){2-9}
   & 1 & 2 & 3 & 5 & 8 & 10 & 20 & 28/32 \\
   \midrule
   LLaMA-3-8B-Instruct & 74.5 & 74.6 & 65.9 & 44.1 & 15.3 & 2.20 & 1.60 & 1.80 \\
   Qwen2-7B-Instruct   & 71.2 & 71.2 & 73.0 & 69.4 & 67.4 & 71.1 & 54.0 & 49.4 \\
   \bottomrule
   \end{tabular}
   % }
   \label{tab:reuse_GSM8K}
\end{table*}


\subsection{LongBench}
\label{appendix:longbench}

The Table \ref{table:longbench} shows the average performance of KV cache compression methods in the LongBench English subtask categories. The \method{} achieves the best performance on the overall average, and the Multi-Document QA category, which supports that chunk method is more effective for semantic preservation.
\begin{table*}[!ht]
   \caption{Comprehensive performance comparison of KV cache compression methods across LongBench English subtasks. Results are shown for various models and tasks, highlighting the effectiveness of different compression techniques.}
   \resizebox{\textwidth}{!}{
   \centering
   % \setlength{\tabcolsep}{8pt}  % Reduce column separation
   \begin{tabular}{l|c@{\hspace{1.5pt}}c@{\hspace{3pt}}c@{\hspace{4pt}}c@{\hspace{0pt}}c@{\hspace{0pt}}c@{\hspace{0pt}}c@{\hspace{0pt}}c@{\hspace{0pt}}c@{\hspace{0pt}}c@{\hspace{4pt}}c@{\hspace{0pt}}c@{\hspace{0pt}}c@{\hspace{2pt}}c@{\hspace{7pt}}c@{\hspace{7pt}}c|c}
   \specialrule{1pt}{0pt}{2pt}
   \multirow{5}{*}{Method}  & \multicolumn{3}{c}{Single-Document QA} & \multicolumn{3}{c}{Multi-Document QA}& \multicolumn{3}{c}{Summarization}& \multicolumn{3}{c}{Few-shot Learning}& \multicolumn{2}{c}{Synthetic} & \multicolumn{2}{c}{Code} & \multirow{6}{*}{\textbf{Avg. $\uparrow$} } \\
   \cmidrule(lr){2-4}\cmidrule(lr){5-7}\cmidrule(lr){8-10}\cmidrule(lr){11-13}\cmidrule(lr){14-15}\cmidrule(lr){16-17}
   & \rotatebox[origin=c]{30}{NrtvQA} & \rotatebox[origin=c]{30}{Qasper} & \rotatebox[origin=c]{30}{MF-en} & \rotatebox[origin=c]{30}{HotpotQA} & \rotatebox[origin=c]{30}{2WikiMQA} & \rotatebox[origin=c]{30}{Musique} & \rotatebox[origin=c]{30}{GovReport} & \rotatebox[origin=c]{30}{QMSum} & \rotatebox[origin=c]{30}{MultiNews} & \rotatebox[origin=c]{30}{TREC} & \rotatebox[origin=c]{30}{TriviaQA} & \rotatebox[origin=c]{30}{SAMSum} & \rotatebox[origin=c]{30}{PCount} & \rotatebox[origin=c]{30}{PRe} & \rotatebox[origin=c]{30}{Lcc} & \rotatebox[origin=c]{30}{RB-P} & \\
   \cmidrule(lr){1-17}
   Avg len &18,409&3,619&4,559&9,151&4,887&11,214&8,734&10,614&2,113&5,177&8,209&6,258&11,141&9,289&1,235&4,206& \\
   
   \midrule
   \multicolumn{18}{c}{LlaMa-3-8B-Instruct, KV Size = Full} \\
   \arrayrulecolor{black}\midrule
   FullKV &25.70 & 29.75 & 41.12 & 45.55 & 35.87 & 22.35 & 25.63 & 23.03 & 26.21 & 73.00 & 90.56 & 41.88 & 4.67 & 69.25 & 58.05 & 50.77 & 41.46 \\
   
   \arrayrulecolor{black}\midrule
   \multicolumn{18}{c}{LlaMa-3-8B-Instruct, KV Size Compression Ratio = $10\%$} \\
   \arrayrulecolor{black}\midrule
   StreamingLLM &20.62 &13.09 &22.10 &36.31 &28.01 &15.61 &21.47 &21.05 &19.39 &62.00 &84.18 &40.27 &4.62 &69.10 &58.84 &55.26 & 35.74\\
   H2O &24.80 &  17.32 &31.80 &40.84 &33.28 &18.90 &22.29 &22.29 &21.82 &40.00 &90.51 &40.55 &5.79 &\textbf{69.50} &58.04 &55.26 & 37.06 \\
   SnapKV & 25.08 & 22.02 & \textbf{37.95} & 43.36 & 35.08 & 20.29 & 22.94 & 22.64 & 21.37 & 71.00 & 90.47 & 40.15 & 5.66 & 69.25 & 58.69 & 56.50 & 40.15 \\
   PyramidKV &\textbf{25.58} &20.77 &35.85 &\textbf{43.80} &33.03 &\textbf{21.45} &\textbf{23.68} &22.26 &\textbf{21.85} &71.50 &90.47 &\textbf{41.66} &5.84 &69.25 &58.52 &55.91 & 40.08 \\
   \rowcolor{red!20}\textbf{\method{}} & 
   24.89 & 
   \textbf{22.96} & 
   37.64 & 
   43.27 & 
   \textbf{36.45} & 
   20.65 & 
   22.80 & 
   \textbf{22.97} & 
   20.82 & 
   \textbf{71.50} & 
   \textbf{90.52} & 
   40.83 & 
   \textbf{5.93} & 
   69.00 & 
   \textbf{60.49} & 
   \textbf{57.48} & 
   \textbf{40.51} \\
   \arrayrulecolor{black}\midrule


   \multicolumn{18}{c}{LlaMa-3-8B-Instruct, KV Size Compression Ratio = $20\%$} \\
   \arrayrulecolor{black}\midrule
   StreamingLLM & 23.35 &18.97 &32.94 &42.39 &29.37 &18.76 &\textbf{25.78} &21.92 &\textbf{25.16} &71.00 &88.85 &40.82 &5.04 &69.00 &56.46 &51.12 & 38.80\\
   H2O & 25.60  &21.88 &35.36 &42.06 &32.68 &19.72 &23.54 &22.77 &22.72 &45.50 &\textbf{90.57} &\textbf{41.67} &\textbf{5.51} &69.25 &54.97 & 50.95 & 37.79 \\
   SnapKV & 25.50 &25.95 &38.43 &44.12 &35.38 &20.49 &24.85 &23.36 &23.51 &\textbf{72.50} &90.52 &40.91 &5.23 &69.25 &56.74 &51.75 & 40.53 \\
   PyramidKV &25.36 &26.88 &37.99 &44.21 &\textbf{35.65} &\textbf{21.43} &25.52 &\textbf{23.43} &23.47 &72.00 &90.56 &41.45 &5.26 &\textbf{69.50} &56.55 &50.93 & 40.63 \\
   \rowcolor{red!20}\textbf{\method{}} & \textbf{26.13} & \textbf{28.43} & \textbf{38.59} & \textbf{44.46} & 34.13 & 21.06 & 24.72 & 23.11 & 22.91 & 71.50 & 90.56 & 41.51 & 5.09 & 69.00 & \textbf{58.17} & \textbf{52.51} & \textbf{40.74}  \\

   \arrayrulecolor{black}\midrule

   \multicolumn{18}{c}{LlaMa-3-8B-Instruct, KV Size Compression Ratio = $30\%$} \\
   \arrayrulecolor{black}\midrule
   StreamingLLM & 24.49 & 22.53 & 35.30 & \textbf{44.33} & 32.81 & 19.00 & \textbf{27.12} & 22.19 & \textbf{25.93} & 72.50 & 89.84 & 41.75 & \textbf{5.41} & 69.00 & 60.40 & 55.13 & 40.48\\
   H2O & 25.87 & 23.03 & 37.06 &43.71 &33.68 &20.93 &24.56 &23.14 &23.58 &50.50 & \textbf{90.77} &41.96 &4.91 &69.25 &59.38 &55.39 & 39.23 \\
   SnapKV & 25.15 & 28.75 & \textbf{39.28} & 43.57 & 36.16 & 21.58 & 25.56 & \textbf{23.19} & 24.30 & \textbf{73.00} & 90.52 & 41.70 & 4.96 & 69.25 & 60.27 & 55.74 & 41.43 \\
   PyramidKV &25.42 &27.91 &38.81 &44.15 &\textbf{36.28} & \textbf{21.72} &26.50 &23.10 &24.28 &72.00 &90.56 &41.87 &4.67 &\textbf{69.50} &60.09 &55.19 & 41.37 \\
   \rowcolor{red!20}\textbf{\method{}} & \textbf{25.88} & \textbf{29.58} & 38.99 & 43.94 & 34.16 & 21.70 & 26.50 & 23.15 & 23.95 & 72.00 & 90.56 & \textbf{42.47} & 5.34 & 69.25 & \textbf{61.68} & \textbf{56.35} & \textbf{41.59}  \\

   \arrayrulecolor{black}\midrule
   \multicolumn{18}{c}{Mistral-7B-Instruct-v0.3, KV Size = Full} \\
   \arrayrulecolor{black}\midrule
   FullKV &29.07&41.58&52.88&49.37&39.01&28.58&34.93&25.68&27.74&76.00&88.59&47.59&6.00&98.50&61.41&62.39 & 48.08 \\
   
   \arrayrulecolor{black}\midrule
   \multicolumn{18}{c}{Mistral-7B-Instruct-v0.3, KV Size Compression Ratio = $10\%$} \\
   \arrayrulecolor{black}\midrule
   StreamingLLM & 25.15 & 25.47 & 30.08 & 44.39 & 32.49 & 19.40 & 24.11 & 20.85 & 19.55 & 65.00 & 88.21 & 44.83 & 4.50 & 79.50 & 59.48 & 58.82 & 40.11 \\
   H2O & 29.35 & 33.39 & 50.39 & 49.58 & 36.76 & 27.42 & 25.16 & 24.75 & 22.12 & 42.00 & 89.00 & \textbf{47.04} & 5.50 & 98.50 & 57.58 & 59.24 & 43.61 \\
   SnapKV & 28.54 & \textbf{36.88} & 53.42 & 50.15 & 38.17 & 27.99 & 26.67 & \textbf{25.21} & \textbf{22.33} & 72.00 & 89.36 & 45.44 & \textbf{5.50} & \textbf{99.00} & 59.79 & \textbf{61.63} & 46.38 \\
   PyramidKV & 29.40 & 35.39 & 52.96 & 49.93 & 38.67 & 28.63 & \textbf{27.59} & 24.99 & 22.77 & 74.00 & \textbf{90.02} & 46.07 & 4.00 & 98.50 & 58.54 & 60.88 & 46.39 \\
   \rowcolor{red!20}\textbf{\method{}} & \textbf{29.75} & 36.82 & \textbf{53.99} & \textbf{50.33} & \textbf{38.72} & \textbf{29.01} & 27.03 & 24.76 & 21.42 & \textbf{76.00} & 88.73 & 46.49 & 5.00 & 98.00 & \textbf{59.98} & 61.47 & \textbf{46.71} \\
   
   \arrayrulecolor{black}\midrule
   \multicolumn{18}{c}{Qwen2-7B-Instruct, KV Size = Full} \\
   \arrayrulecolor{black}\midrule
   FullKV & 25.11 & 42.64 & 44.29 & 14.25 & 13.22 & 9.08 & 36.38 & 23.43 & 26.53 & 77.00 & 89.99 & 44.88 & 6.75 & 75.92 & 60.17 & 61.84 & 40.71\\
   
   \arrayrulecolor{black}\midrule
   \multicolumn{18}{c}{Qwen2-7B-Instruct, KV Size Compression Ratio = $10\%$} \\
   \arrayrulecolor{black}\midrule
   StreamingLLM & 25.15 &45.42 &41.46 &13.66 &11.95 &8.72 &32.79 &21.49 &\textbf{26.24} & 77.50 &89.15 &44.54 &7.50 &50.50 &60.03 &60.91 &38.56 \\
   H2O & 26.17 & 44.33 & 42.54 & 12.81 & 12.46 & 9.15 & \textbf{33.24} & 22.69 & 25.94 & 76.50 & 89.44 & 44.32 & 8.00 & 76.00 & \textbf{61.28} & \textbf{62.39} & 40.45 \\
   SnapKV & 26.84 & \textbf{45.96} & \textbf{45.79} & 14.27 & \textbf{13.35} & 9.91 & 32.62 & 22.70 & 25.83 & 77.00 & 89.19 & 44.71 & 7.50 & 71.50 & 60.35 & 61.37 & 40.55 \\
   PyramidKV & \textbf{27.51} & 44.45 & 43.59 & 13.35 & 13.13 & 9.12 & 32.28 & 22.60 & 25.45 & 77.00 & \textbf{89.44} & 44.53 & 7.00 & 73.50 & 60.91 & 61.24 & 40.31 \\
   \rowcolor{red!20}\textbf{\method{}} & 26.48 & 44.19 & 45.04 & \textbf{15.94} & 12.60 & \textbf{10.52} & 32.38 & \textbf{22.87} & 25.91 & \textbf{77.50} & 89.22 & \textbf{44.78} & \textbf{8.50} & \textbf{76.50} & 60.64 & 61.32 & \textbf{40.88} \\
   
   \arrayrulecolor{black}\bottomrule
   \end{tabular}
   }
   
   \label{table:longbench}
   \vspace{-3mm}
   \end{table*}
   

\subsection{Needle-In-A-Haystack}
\label{appendix:niah}

Figure~\ref{fig:NIAH_llama} and  \ref{fig:NIAH_mistral} visualizes the performance of \method{} on the NIAH benchmark for LLaMA-3-8B-Instruct and Mistral-7B-Instruct with a KV cache size of 128 under 8k and 32k context length. The performance of \method{} is consistently better as the context length increases.
\begin{figure}[!ht]
   \centering
   \begin{subfigure}[b]{0.9\textwidth}
       \centering
       \includegraphics[width=\textwidth]{./figs/NIAH/llama-3-8B-Instruct_spankv_baseline_128.pdf}
       \caption{\method{}, accuracy 73.8\%}
       \label{fig:NIAH_llama_spankv}
   \end{subfigure}
   \begin{subfigure}[b]{0.9\textwidth}
      \centering
      \includegraphics[width=\textwidth]{./figs/NIAH/llama-3-8B-Instruct_pyramidkv_baseline_128.pdf}
      \caption{PyramidKV, accuracy 65.1\%}
      \label{fig:NIAH_llama_pyramidkv}
   \end{subfigure}
   \begin{subfigure}[b]{0.9\textwidth}
      \centering
      \includegraphics[width=\textwidth]{./figs/NIAH/llama-3-8B-Instruct_snapkv_baseline_128.pdf}
      \caption{SnapKV, accuracy 58.9\%}
      \label{fig:NIAH_llama_snapkv}
   \end{subfigure}
   \begin{subfigure}[b]{0.9\textwidth}
      \centering
      \includegraphics[width=\textwidth]{./figs/NIAH/llama-3-8B-Instruct_h2o_baseline_128.pdf}
      \caption{H2O, accuracy 47.9\%}
      \label{fig:NIAH_llama_h2o}
   \end{subfigure}
   \begin{subfigure}[b]{0.9\textwidth}
      \centering
      \includegraphics[width=\textwidth]{./figs/NIAH/llama-3-8B-Instruct_streamingllm_baseline_128.pdf}
      \caption{StreamingLLM, accuracy 23.7\%}
      \label{fig:NIAH_llama_streamingllm}
   \end{subfigure}
   \caption{NIAH benchmark for LLaMA-3-8B-Instruct with KV cache size=128 under 8k context length}
   \label{fig:NIAH_llama}
\end{figure}

\begin{figure}[!ht]
   \centering
   \begin{subfigure}[b]{0.9\textwidth}
       \centering
       \includegraphics[width=\textwidth]{./figs/NIAH/Mistral-7B-Instruct-v0.2_spankv_0920_baseline_128.pdf}
       \caption{\method{}, accuracy 99.8\%}
       \label{fig:NIAH_mistral_spankv}
   \end{subfigure}
   \begin{subfigure}[b]{0.9\textwidth}
      \centering
      \includegraphics[width=\textwidth]{./figs/NIAH/Mistral-7B-Instruct-v0.2_pyramidkv_0920_baseline_128.pdf}
      \caption{PyramidKV, accuracy 99.3\%}
      \label{fig:NIAH_mistral_pyramidkv}
   \end{subfigure}
   \begin{subfigure}[b]{0.9\textwidth}
      \centering
      \includegraphics[width=\textwidth]{./figs/NIAH/Mistral-7B-Instruct-v0.2_snapkv_0920_baseline_128.pdf}
      \caption{SnapKV, accuracy 91.6\%}
      \label{fig:NIAH_mistral_snapkv}
   \end{subfigure}
   \begin{subfigure}[b]{0.9\textwidth}
      \centering
      \includegraphics[width=\textwidth]{./figs/NIAH/Mistral-7B-Instruct-v0.2_h2o_0920_baseline_128.pdf}
      \caption{H2O, accuracy 88.2\%}
      \label{fig:NIAH_mistral_h2o}
   \end{subfigure}
   \begin{subfigure}[b]{0.9\textwidth}
      \centering
      \includegraphics[width=\textwidth]{./figs/NIAH/Mistral-7B-Instruct-v0.2_streamingllm_0920_baseline_128.pdf}
      \caption{StreamingLLM, accuracy 44.3\%}
      \label{fig:NIAH_mistral_streamingllm}
   \end{subfigure}
   \caption{NIAH benchmark for Mistral-7B-Instruct with KV cache size=128 under 32k context length}
   \label{fig:NIAH_mistral}
\end{figure}
\clearpage

% Table \ref{tab:NIAH} shows the performance of \method{} in the NIAH data set with different KV cache sizes on LLaMA-3-8B-Instruct.
% \begin{table}[!ht]
%    \centering
%    \caption{NIAH Performance Comparison with Different KV Cache Sizes}
%    \begin{tabular}{l|cccc}
%    \toprule
%    \textbf{Method} & \textbf{Size = 96} & \textbf{Size = 128} & \textbf{Size = 256} & \textbf{Size = 512} \\
%    \midrule
%    StreamingLLM      & 21.5 & 23.7 & 28.0 & 32.0 \\
%    H2O      & 41.0 & 47.9 & 61.7 & 68.6 \\
%    SnapKV      & 56.2 & 58.9 & 68.8 & 71.2 \\
%    PyramidKV      & 63.2 & 65.1 & 69.5 & 72.6 \\
%    \rowcolor{red!20} \textbf{ChunkKV}      & \textbf{70.3} & \textbf{73.8} & \textbf{74.1} & \textbf{74.5} \\
%    \rowcolor{gray!20} FullKV & 74.6 & 74.6 & 74.6 & 74.6 \\
%    \bottomrule
%    \end{tabular}
%    \label{tab:NIAH}
% \end{table}


\subsection{Chunk Size}
\label{appendix:chunk_size}

% Table \ref{tab:ablation_size_compression}  and \ref{tab:ablation_size_niah} show the performance of \method{} with different comperession ratios and different chunk sizes on the LongBench and NIAH. We conducted extensive experiments across different compression ratios and KV cache sizes to shows the effectiveness of \method{} and the chunk size is robust.

% \begin{table}[!ht]
%    \centering
%    \caption{LongBench Performance with Different Chunk Sizes and Compression Ratios for LLaMA-3-8B-Instruct}
%    % \resizebox{0.9\textwidth}{!}{%
%    \begin{tabular}{c|ccccccc}
%    \toprule
%    Compression & \multicolumn{7}{c}{Chunk Size} \\
%    \cmidrule{2-8}
%    Rate & 1 & 3 & 5 & 10 & 15 & 20 & 30 \\
%    \midrule
%    10\% & 37.32 & 40.49 & 40.47 & \textbf{40.51} & 40.21 & 40.05 & 39.57 \\
%    20\% & 38.80 & 40.66 & 40.57 & \textbf{40.74} & 40.53 & 40.46 & 40.04 \\
%    30\% & 39.23 & 41.02 & 41.29 & \textbf{41.59} & 41.38 & 41.33 & 41.02 \\
%    \bottomrule
%    \end{tabular}%
%    % }
%    \label{tab:ablation_size_compression}
% \end{table}

% \begin{table}[!ht]
%    \centering
%    \caption{NIAH Performance with Different Chunk Sizes and KV Cache Sizes for LLaMA-3-8B-Instruct}
%    % \resizebox{0.9\textwidth}{!}{%
%    \begin{tabular}{c|ccccccc}
%    \toprule
%    KV Cache & \multicolumn{7}{c}{Chunk Size} \\
%    \cmidrule{2-8}
%    Size & 1 & 3 & 5 & 10 & 15 & 20 & 30 \\
%    \midrule
%    96 & 41.0 & 63.2 & 65.2 & \textbf{70.3} & 67.2 & 65.3 & 53.1 \\
%    128 & 47.9 & 65.6 & 69.1 & \textbf{73.8} & 72.3 & 72.0 & 71.2 \\
%    256 & 61.7 & 70.3 & 71.2 & \textbf{74.1} & 73.2 & 72.3 & 71.1 \\
%    512 & 68.6 & 72.6 & 72.5 & \textbf{74.5} & 74.3 & 74.0 & 72.6 \\
%    \bottomrule
%    \end{tabular}%
%    % }
%    \label{tab:ablation_size_niah}
% \end{table}


Table \ref{tab:ablation_size_longbench} shows the performance of \method{} with different chunk size on the LongBench benchmark.

\begin{table}[!ht]
   \centering
   \caption{LongBench Performance Comparison with different chunk sizes}
   % \resizebox{0.8\textwidth}{!}{%
   \begin{tabular}{c|ccccc|c}
   \toprule
   \multirow{3}{*}{Model} & \multicolumn{5}{c|}{Chunk Size} & \multirow{3}{*}{Full KV} \\
   \cmidrule(lr){2-6}
          & 3     & 5     & 10    & 20    & 30     & \\
   \midrule
   LLaMA-3-8B-Instruct & 40.49 & 40.47 & \textbf{40.51} & 40.05 & 39.57 & 41.46 \\
   Mistral-7B-Instruct & 46.45 & 46.51 & \textbf{46.71} & 46.42 & 45.98  & 48.08 \\
   Qwen2-7B-Instruct   & 40.38 & 40.33 & 40.66 & \textbf{40.88} & 40.73  & 40.71 \\
   \bottomrule
   \end{tabular}%
   % }
   \label{tab:ablation_size_longbench}
\end{table}

Table \ref{tab:ablation_size_gsm8k} shows the performance of \method{} with different chunk size on the GSM8K benchmark. Figure~\ref{fig:chunk_size_gsm8k} shows that the \method{} with different chunk sizes on GSM8K displays the same curve pattern as LongBench. The CoT prompt length for GSM8K is only 1K tokens, so the optimal chunk size range is smaller.

\begin{figure}[!ht]
   \centering
   \includegraphics[width=1\textwidth]{./figs/GSM8K_Performance_vs_Chunk_Size_2plots.pdf}
   \caption{ GSM8K Performance Comparison  with different chunk size}
   \label{fig:chunk_size_gsm8k}
\end{figure}

\begin{table}[!ht]
   \centering
   \caption{GSM8K Performance Comparison with different chunk sizes}
   % \resizebox{0.8\textwidth}{!}{%
   \begin{tabular}{c|cccc|c}
   \toprule
   \multirow{3}{*}{Model} & \multicolumn{4}{c|}{Chunk Size}  & \multirow{3}{*}{Full KV} \\
   \cmidrule{2-5}
          & 3     & 5     & 10    & 20     & \\
   \midrule
   LLaMA-3-8B-Instruct & \textbf{74.6} & 74.5 & 73.9 & 63.2 & 76.8 \\
   Qwen2-7B-Instruct   & \textbf{73.5} & 71.2 & 71.8 & 71.7  & 71.1 \\
   \bottomrule
   \end{tabular}%
   % }
   \label{tab:ablation_size_gsm8k}
\end{table}
\subsection{Multi-Lingual}
\label{appendix:multilingual}
Table~\ref{table:longbench_zh} is the Chinese support model Qwen2-7B-Instruct evaluated on the LongBench Chinese subtask, where \method{} achieves better performance than other compression methods and the full KV cache performance. Both the English and Chinese results indicate that ChunkKV is a promising approach for maintaining crucial information in the KV cache.
% \clearpage


\begin{table}[ht]
   \caption{Performance comparison of Chinese subtask on LongBench for Qwen2-7B-Instruct.}
   \centering
   \resizebox{0.9\textwidth}{!}{
      \begin{tabular}{l|ccccc|c}
         \arrayrulecolor{black}\toprule

         \multirow{3}{*}{Method}  & Single-Document QA & Multi-Document QA & Summarization & Few-shot Learning & Synthetic  &  \multirow{4}{*}{\textbf{Avg. $\uparrow$}}   \\
 
         \cmidrule(lr){2-2}\cmidrule(lr){3-3}\cmidrule(lr){4-4}\cmidrule(lr){5-5}\cmidrule(lr){6-6}
           & MF-zh & DuReader & VCSum & LSHT & PR-zh \\
         \cmidrule(lr){1-6}
           Avg len &6,701&15,768&15,380&22,337&6,745& \\
         \arrayrulecolor{black}\midrule
         \multicolumn{7}{c}{Qwen2-7B-Instruct, KV Size = Full} \\
         \arrayrulecolor{black}\midrule
         FullKV & 39.17 & 23.63 & 16.21 & 43.50 & 70.50 & 38.60 \\
         
         \arrayrulecolor{black}\midrule
         \multicolumn{7}{c}{Qwen2-7B-Instruct, KV Size Compression Ratio = $10\%$} \\
         \arrayrulecolor{black}\midrule
         StreamingLLM & 38.05 & \textbf{23.24} & 15.92 & 40.50 & 44.50 & 32.44 \\
         H2O & 37.99 &19.58 & 16.16 & 41.67 & 67.35 & 36.55 \\
         SnapKV & 44.25 & 20.27 & 16.24 & \textbf{44.50} & 68.10 & 38.67 \\
         PyramidKV & 36.57 & 20.56 & 16.15 & 43.50 & 66.50 & 36.55 \\
         \rowcolor{red!20}\textbf{\method{}} & \textbf{45.92} & 20.15 & \textbf{16.37} & 43.75 & \textbf{71.10} & \textbf{39.45} \\
         \arrayrulecolor{black}\bottomrule
      \end{tabular}
   }
   
\label{table:longbench_zh}
\vspace{-3mm}
\end{table}



\section{Theoretical Understanding}
\label{appendix:theory}
In this section, we provide the theoretical interpretation from the perspective from the In-context learning (ICL) to further understand how ChunkKV outperforms token-level KV cache compression.

\textbf{Pretraining Data Distribution.} Given a set of concepts $\Theta$ and a concept $\theta \in \Theta$, we define the pretraining data is sampled from $p(\obs_1, \dots, \obs_T) = \int_{\theta \in \Theta} p(\obs_1, \dots, \obs_T \vert \theta)p(\theta)d \theta$~\citep{xie2022an}. Each token $\obs$ is sampled from a vocabulary $\obsset$. For simplicity, we write $\obs_{1:t} = \obs_1\dots\obs_{t}$.

\textbf{Language Modeling.}
Current LLMs~\citep{brown2020language,touvron2023llama2,xie2022an} usually utilize the next word prediction as the language modelling, which predicts the next token $\obs_{t}$ given the previous tokens $\obs_1\dots\obs_{t-1}$ for all $t=1,\dots, T$. Formally, a language modelling can be writen as the distribution $f(\obs_{t} \vert \obs_{1:t-1})$. And it is pretrained on a huge corpus sampled from the pretraining distribution $p(\obs_1,\dots,\obs_{t+1})$~\citep{xie2022an}. Considering the large scale of the model size and dataset size, it can be assumed that the $f(\obs_1\dots\obs_{t+1})$ has been aligned with the $p(\obs_1\dots\obs_{t+1})$~\citep{xie2022an}.

% \obs,\dots,\obs_{t+1}

\textbf{Prompt Distribution.}
Following~\citep{xie2022an}, a prompt is composed of an input token sequence $x$ followed by an output token $y$. Then, the $i$-th training example \footnote{Here, training example in prompts means happens during the prompt learning.} that can appear in any place in the whole prompt $o_{1:T}$ is defined as $O_i$ consisting of an input $x_i=O_i \left[1:k-1 \right]$ (the first $k-1$ tokens) followed by the output $y_i = O_i \left[k\right]$ at the end, where the length $k$ is fixed for simplicity. 

The $i$-th training example is independently generated as follows: 1) Generate a start hidden state $\hiddensegstart_i$ from a \emph{prompt start distribution} $\ppromptstart$;
2) Given $\hiddensegstart_i$, generate the example sequence $\obsseg_i=[\X_i,\y_i]$ from $p(\obsseg_i \vert \hiddensegstart_i, \theta^\star)$.
The test input $\Xtest = \X_{n+1}$ is sampled similarly. Then, the prompt consists of a sequence of training examples ($\promptseq$) followed by the example $\Xtest$:

\begin{align}
    [\promptseq, \Xtest] = [\X_1, \y_1, \X_2, \y_2, \dots, \X_n, \y_n, \Xtest] \sim \pprompt.
\end{align}



\textbf{In-context learning setups and Assumptions.} We follow other settings and assumptions in ~\citep{xie2022an}. With the greedy decoding~\citep{fubreak}, sampling the next token from the language modeling $f(o_t \vert o_{1:t-1})$ becomes the predictor as $y =\argmax_{o_t} f(o_t|o_{1:t-1})$. 

Thus, for $[\promptseq, \Xtest]$, the in-context learning predictor can be written as $f_{n}(\Xtest) := \argmax_y p(y|\promptseq, \Xtest)$, which outputs the most likely prediction over the \emph{pretraining distribution} conditioned on the \emph{prompt distribution}. Its expected 0-1 error with $n$ examples is $\Lzeroone(f_n) = \E_{\Xtest,\ytest \sim \pprompt}[\indicator[f_{n}(\Xtest) \neq \ytest]]$.


We define $p_\theta^i(o):=p(O[i]=o|O[1:i-1],\theta)$ of the $i$-th token with previous tokens and the analogous distribution $p^{i}_{prompt}:=p_{prompt}(O[i]=o|O[1:i-1])$ under the prompt distribution. Following~\citep{xie2022an}, there is a distinguishability condition formalizes when in-context learning occurs giving the concept $\theta$. 

The distinguishability condition is dependent on a KL divergence between the previous two distributions and the error terms $\epsilon_\theta$ resulting from the distribution mismatch between the prompt and the pertaining distributions for each example. Letting $p_{\theta}^i(o)$ and $p^{i}_{prompt}$ correspond to the concept $\theta$ and $\theta^\star$.

% the KL divergence are defined as follows,

\begin{condition}[distinguishability~\citep{xie2022an}] The $\theta^\star$ is distinguishable if for all $\theta\in\Omega$, $\theta \neq\theta^\star$,
\begin{align}\label{eq:distinguish1}
    \sum_{i=1}^k \text{KL}_{i}(\theta^\star||\theta)>\epsilon_\theta,
\end{align}
where the $\text{KL}_{i}(\theta^\star||\theta) :=\mathbb{E}_{O[1:i-1]\sim p_{prompt}}[\text{KL}(p^{i}_{prompt}||p_\theta^{i})].$
\label{cond:distinguish}
\end{condition}


\textbf{Noises from KV Cache Compression.} Naturally, because of the sparsified KV cache, some history tokens in $o_{1:t-1}$ at different layers lost its attention score calculation with respect to the next word prediction $o_t$. We can regard this as the noise added onto the $o_{1:t-1}$. Thus, distincting $\theta^\star$ from $\theta$ requires larger KL divergence. Following~\citep{zhoucan}, we provide the following second condition about the distinguishability with the KV cache sparsity.

\begin{condition}[distinguishability under sparsified KV cache] With the noise introduced by the sparsified KV cache of the sparse ratio $r$, the distribution mismatch between the prompt and the pretraining distribution that is approximated by LLM is enlarged, resulting in a varied requirement with error term $\xi_\theta(r)$ for $\theta^*$ being distinguishable if for all $\theta\in\Theta$, $\theta\neq\theta^*$,
    \begin{align}\label{eq:noise_dinsting}
        \sum_{i=1}^k \text{KL}_{i}(\theta^*||\theta)>\epsilon_\theta+\xi_\theta(r),\quad {\rm where}\quad \xi_\theta(r)\propto r.
    \end{align}
    \label{cond:2}
\end{condition}


\begin{lemma}[noisy-relaxed bound in ~\citep{xie2022an,zhoucan}]\label{lemma:1}
let $\mathcal{B}$ denotes the set of $\theta$ which does not satisfy Condition~\ref{cond:distinguish}. We assume that $\text{KL}(p_{prompt}(y_\text{test}|x_\text{test}))||p(y_\text{test}|x_\text{test},\theta)$ is bounded for all $\theta$ and that $\theta^\star$ minimizes the multi-class logistic risk as,
\begin{align}\label{eq:lemma:1:LCE}
\begin{split}
L_\text{CE}(\theta)=-\mathbb{E}_{x_\text{test}\sim p_{prompt}}[p_{prompt}(y_\text{test}|x_\text{test})\cdot\log p(y_\text{test}|x_\text{test},\theta)].
\end{split}
\end{align}
If
\begin{align}\label{eq:prompt_tau_upperbound_concrete_with_epsilon}
\mathbb{E}_{x_\text{test}\sim p_{prompt}}[\text{KL}(p_{prompt}(y_\text{test}|x_\text{test})
|| p(y_\text{test}|x_\text{test},\theta))]\leq (\epsilon_{\theta} + \xi_\theta(r)),\quad \forall \quad \theta\in\mathcal{B},
\end{align}
then
\begin{align}\label{eq:L01-inf}
\lim_{n\rightarrow\infty} L_{0-1}(f_n) \leq \inf_{f} L_{0-1}(f) + g^{-1}\bigg(\sup_{\theta\in\mathcal{B}}(\epsilon_\theta)\bigg),
\end{align}
where $g(\nu) = \frac{1}{2}\big((1-\nu)\log(1-\nu)+(1+\nu)\log(1+\nu)\big)$ is the calibration function~\citep{Steinwart2007HowTC,pires2016multiclass} for the multiclass logistic loss for $\nu\in[0,1]$.
\end{lemma}


Following~\citep{Kleijn2012TheBT,xie2022an}, KL divergence is assumed to haver the 2nd-order Taylor expansion with the concept $\theta$. Then, we have the following theorem and proof.

\begin{theorem}
\label{thm:continuity}~\citep{xie2022an,zhoucan}
Let the set of $\theta$ which does not satisfy Equation~\ref{eq:distinguish1} in Condition~\ref{cond:distinguish} to be $\mathcal{B}$.
Assume that KL divergences have a 2nd-order Taylor expansion around $\theta^\star$:
\begin{align}
    \forall j>1,~~\text{KL}_{i}(\theta^\star||\theta) = \frac{1}{2}(\theta - \theta^\star)^\top \fisherinfj (\theta - \theta^\star) + O(\|\theta - \theta^\star\|^3)
\end{align}
where $\fisherinfj$ is the Fisher information matrix of the $j$-th token distribution with respect to $\theta^\star$.
Let $\conditionnum = \frac{\max_{j}\lambdamax(\fisherinfj)}{\min{j}\lambdamin(\fisherinfj)}$ where $\lambdamax,\lambdamin$ return the largest and smallest eigenvalues.
Then for $k \geq 2$ and as $n\rightarrow \infty$, the 0-1 risk of the in-context learning predictor $f_n$ is bounded as
\begin{align}\label{eq:final-theorem-bound}
    \lim_{n\rightarrow \infty} \Lzeroone(f_n) \leq \inf_{f} \Lzeroone(f) + g^\minv\left(O\left(\frac{\conditionnum\sup_{\theta \in \badset}(\epsilon_{\theta} + \xi_\theta(r))}{k-1}\right)\right)
\end{align}
\end{theorem}

\begin{proof}~\citep{xie2022an}
By the Taylor expansion on $\theta$, we have for any $\theta$ in $\mathcal{B}$ that
\begin{align}
    \sum_{j=2}^k \text{KL}_{i}(\theta^\star||\theta)
    &\geq \frac{1}{2}\sum_{j=2}^k (\theta - \theta^\star)^\top \fisherinfj (\theta - \theta^\star) + (k-1)O(\|\theta - \theta^\star\|^3)\\
                      &\geq \frac{1}{2}(k-1)\lambdamin(\fisherinfj) \|\theta - \theta^\star\|^2\\
    \implies \|\theta - \theta^\star\|^2 &\leq \frac{(\epsilon_{\theta} + \xi_\theta(r))}{\frac{1}{2}(k-1)(\min_j~\lambdamin(\fisherinfj))}.
\end{align}
% We use this to bound the last KL term by plugging it in below:
% Using the above term to bound the last KL term ($k$-th token), we have:
We can bound the last KL term ($k$-th token) with the above term:
\begin{align}
    \text{KL}_{k}(\theta^\star||\theta) &= \frac{1}{2}(\theta - \theta^\star)^\top  \fisherinfk(\theta - \theta^\star) + O(\|\theta - \theta^\star\|^3)\\
         &\leq \frac{1}{2}(\max_j~\lambdamax(\fisherinfj))\|\theta - \theta^\star\|^2 + O(\|\theta - \theta^\star\|^2)\\
                 &\leq \frac{(\epsilon_{\theta} + \xi_\theta(r))(\max_j~\lambdamax(\fisherinfj) + O(1))}{(k-1)\min_j~\lambdamin(\fisherinfj)}.
\end{align}
Rearranging above equation, and with $\text{KL}_{k}(\theta^\star||\theta) = \E_{\Xtest \sim \pprompt} [KL(\pprompt(\ytest \vert \Xtest) \| p(\ytest \vert \Xtest, \theta)) ]$, there is
\begin{align}\label{eq:prompt_tau_upperbound_concrete}
\E_{\Xtest \sim \pprompt} [KL(\pprompt(\ytest \vert \Xtest) \| p(\ytest \vert \Xtest, \theta)) ] \leq \frac{(\epsilon_{\theta} + \xi_\theta(r))(\max_j~\lambdamax(\fisherinfj) + O(1))}{(k-1)\min_j~\lambdamin(\fisherinfj)}
\end{align}
Combining Equation~\ref{eq:prompt_tau_upperbound_concrete} with Equation~\ref{eq:prompt_tau_upperbound_concrete_with_epsilon} into Lemma~\ref{lemma:1} completes the proof.
\end{proof}

\textbf{KV Cache Sparsification.} Revisiting the Equation~\ref{eq:noise_dinsting} in Condition~\ref{cond:2}, the $\xi_\theta(r)$ is enlarged with the sparsity ratio $r$. The higher compression ratio $r$ (means that more KV cache are discarded), the more noise $\xi_\theta(r)$. Then it leads to the higher bound of the $\lim_{n\rightarrow\infty} L_{0-1}(f_n)$ in Equation~\ref{eq:lemma:1:LCE} in Lemma~\ref{lemma:1}. Next, we discuss how KV cache compression influences the Equation~\ref{eq:noise_dinsting}.

\textbf{Token-level Importance Measure.}
The token-level KV cache methods usually calculate the importance of different tokens. Then, the KV cache with indexes that have higher importance will be preserved. Such indexes are normaly choosed as the attention score. Considering that the case in Figure~\ref{fig:main}, where each token in the $i$-th training~\footnote{Here, training means prompt learning~\citep{xie2022an}.} example sequence ($\obsseg_i=[\X_i,\y_i]$) might be compressed, and tokens are sparsified concretely without out dependency to other tokens. However, in the generation process of the $i$-th training example, $\obsseg_i=[\X_i,\y_i]$ is sampled from $p(\obsseg_i \vert \hiddensegstart_i, \theta^\star)$ and $p_\theta^j(o):=p(O[j]=o|O[1:j-1],\theta)$ of the $j$-th token with previous tokens and the analogous distribution $p^{j}_{prompt}:=p_{prompt}(O[j]=o|O[1:j-1])$. And the KL divergence is defined as $\text{KL}_{j}(\theta^\star||\theta) :=\mathbb{E}_{O[1:j-1]\sim p_{prompt}}[\text{KL}(p^{j}_{prompt}||p_\theta^{j})]$, which means that in a training example $\obsseg_i=[\X_i,\y_i] = \obsseg_i[1:k]$, each token $\obsseg_i[j]$ has strong dependency with $\obsseg_i[1:j-1]$, noises on previous any $j$-th token will influence the distinguishability on the following tokens, i.e. requiring larger $\left\{ \text{KL}_{u}(\theta^\star||\theta) \right\}_{u>j}$.

On the other hand, the token-level sparsity enlarges the requirement on the distinguishability uniformly for each example $\obsseg_i$ (the case in Figure~\ref{fig:main}), which uniformly loses the bound of $\Lzeroone(f_n)$ as in Equation~\ref{eq:final-theorem-bound}.


\textbf{Chunk-level Importance Measure.} Different from token-level importance measure, \method{} regards tokens in a continuous window as a basic unit that should be left or discarded as a whole. The preserved window can be regarded as saving the complete $\obsseg_i=[\X_i,\y_i]$ without noise. Thus, \method{} reduces the noise $\xi_\theta(r)$ for the preserved $\obsseg_i$, which lowers the bound of $\Lzeroone(f_n)$. 

More intuitively, \method{} focus on reducing the noise on some complete training examples, but some other examples overall with low importance will be discarded. Then, the model identifies the $\Xtest$ from those clean and more related training examples $\obsseg_i$ and neglect those $\obsseg_i$ with less importance.

Note that here, we do not provide the rigorous proof on how KV cache sparsity enhances the requirement of the distinguishability and how different $\text{KL}_{j}(\theta^\star||\theta)$ on $\obsseg_i=[\X_i,\y_i]$ influences the bound $\Lzeroone(f_n)$. We left this as the future work to analyze how KV cache sparsity influences the in-context learning.



% complete $\obsseg_i=[\X_i,\y_i]$ helps 




% thus reducing the requirement of the distinguishability on the 






% Given $\hiddensegstart_i$, generate the example sequence $\obsseg_i=[\X_i,\y_i]$ from $p(\obsseg_i \vert \hiddensegstart_i, \theta^\star)$.




% $\text{KL}_{i}(\theta^\star||\theta) :=\mathbb{E}_{O[1:i-1]\sim p_{prompt}}[\text{KL}(p^{i}_{prompt}||p_\theta^{i})].$

% 






\section{Additional Related Work}
\label{appendix:related_work}



\textbf{KV cache sharing}
Recent work has explored various strategies for sharing KV caches across transformer layers. Layer-Condensed KV Cache (LCKV) \citep{wu2024layercondensedkvcacheefficient} computes KVs only for the top layer and pairs them with queries from all layers, while optionally retaining standard attention for a few top and bottom layers to mitigate performance degradation. Similarly, You Only Cache Once (YOCO) \citep{sun2024yoco} computes KVs exclusively for the top layer but pairs them with queries from only the top half of layers, employing efficient attention in the bottom layers to maintain a constant cache size. In contrast, Cross-Layer Attention (CLA) \citep{brandon2024reducing} divides layers into groups, pairing queries from all layers in each group with KVs from that group's bottom layer. MiniCache \citep{liu2024minicache} introduces a novel method that merges layer-wise KV caches while enabling recovery during compute-in-place operations, optimizing KV cache size. These methods illustrate various trade-offs between computation, memory usage, and model performance when sharing KV caches across transformer layers.

\textbf{Long-Context Benchmarks}
The landscape of long-context model benchmarks has evolved to encompass a wide range of tasks, with particular emphasis on retrieval and comprehension capabilities. Benchmarks for understanding have made significant strides, with $\infty$-Bench~\citep{zhang2024infty} pushing the boundaries by presenting challenges that involve more than 100,000 tokens. LongBench~\citep{bai2023longbench} has introduced bilingual evaluations, addressing tasks such as long-document question answering, summarization, and code completion. Complementing these efforts, ZeroSCROLLS~\citep{shaham2023zeroscrolls} and L-Eval~\citep{an2023eval} have broadened the scope to include a diverse array of practical natural language tasks, including query-driven summarization.

In parallel, retrieval benchmarks have largely relied on synthetic datasets, offering researchers precise control over variables such as the length of input tokens. This approach minimizes the impact of disparate parametric knowledge resulting from varied training methodologies. A significant body of recent work has concentrated on the development of synthetic tasks specifically for retrieval evaluation~\citep{needle, mohtashami2023landmark, longchat, liu2024lost, hsieh2024ruler}. In addition, researchers have explored the potential of extended contexts in facilitating various forms of reasoning~\citep{tay2020long}.

This dual focus on synthetic retrieval tasks and comprehensive understanding benchmarks reflects the field's commitment to rigorously assessing the capabilities of long-context models across diverse linguistic challenges.
\textbf{Prompting Compression}
In the field of prompt compression, various designs effectively combine semantic information to compress natural language. \citet{wingate-etal-2022-prompt} utilize soft prompts to encode more information with fewer tokens. \citet{Chevalier2023AdaptingLM} present AutoCompressor, which uses soft prompts to compress the input sequence and extend the original length of the base model. Both \citet{zhou2023recurrentgpt} and \citet{wang2023recursively} recurrently apply LLMs to summarize input texts, maintaining long short-term memory for specific purposes such as story writing and dialogue generation. The LLMLingua series~\citep{jiang-etal-2023-llmlingua,jiang-etal-2024-longllmlingua,fei-etal-2024-extending} explores the potential of compressing LLM prompts in long-context, reasoning, and RAG scenarios. \citet{fei-etal-2024-extending} use pre-trained language models to chunk the long context and summarize semantic information, compressing the original context.

\section{Statistics of Models}
\label{appendix:config_models}
Table~\ref{table:app_model_config} provides configuration parameters for LLMs that we evaluated in our experiments.
\begin{table}[ht]
   \resizebox{1\textwidth}{!}{
   \centering
   \begin{tabular}{c|c|c|c}
   \toprule

   \textbf{Model Name} & \textbf{LLaMA-3-8B-Instruct} & \textbf{Mistral-7B-Instruct-v0.2 \& 0.3} & \textbf{Qwen2-7B-Instruct} \\
   \midrule
   $L$ (Number of layers) & 32 & 32 & 28 \\
   \midrule
   $N$ (Number of attention heads) & 32 & 32 & 28\\
   \midrule
   $D$ (Dimension of each head) & 128  & 128 & 128 \\
   \bottomrule
   \end{tabular}
   }
   \caption{ \centering Models Configuration Parameters}
   \label{table:app_model_config}
\end{table}

\section{Statistics of Datasets}
\label{appendix:evaluation}
LongBench is a meticulously designed benchmark suite that evaluates the capabilities of language models in handling extended documents and complex information sequences. This benchmark was created for multi-task evaluation of long-context inputs and includes 17 datasets covering tasks such as single-document QA~\citep{kovcisky2018narrativeqa,dasigi2021dataset}, multi-document QA~\citep{yang2018hotpotqa,ho2020constructing,trivedi2022musique,he2017dureader}, summarization~\citep{huang2021efficient,zhong2021qmsum,fabbri2019multi,wu2023vcsum}, few-shot learning~\citep{li2002learning,gliwa2019samsum,joshi2017triviaqa}, synthetic tasks and code generation~\citep{guo2023longcoder,liu2023repobench}. The datasets feature an average input length ranging from $1K$ to $18K$ tokens, requiring substantial memory for KV cache management.

Table \ref{tab:dataset_statistic} shows the statistics of the datasets that we used in our experiments.
\begin{table}[!h]
   \centering
   % \small
   % \vskip 0.15in
   \begin{sc}
       % \resizebox{0.6\linewidth}{!}{
       \begin{tabular}{l|rr}
           \toprule
           Dataset  & \# Train & \# Test \\ \midrule
           
           GSM8K~\citep{gsm8k}    
           & 7,473     
           & 1,319   \\
           
           LongBench~\citep{bai2023longbench} 
           & - 
           & 4,750 \\
   
           NIAH*~\citep{needle} 
           & -
           & 800 \\
           
           \bottomrule
           \end{tabular}
       % }
   \end{sc}
   % \end{small}
   \caption{Dataset Statistics. \textsc{\# Train} and \textsc{\# Test} represent the number of training and test samples, respectively. *: The size of the NIAH test set varies based on the context length and step size, typically around 800 samples per evaluation.}

   \label{tab:dataset_statistic}
   % \vspace{-0.3cm}
   \end{table}


\section{Prompt}
\label{appendix:prompt}
Table \ref{tab:demo_prompt} shows the prompt for the Figure \ref{fig:main}
\begin{table}[ht]
   \centering
   \footnotesize
   \resizebox{1\linewidth}{!}{
   \begin{tabular}{p{15cm}}
   \specialrule{1pt}{0pt}{2pt}
      \specialrule{1pt}{0pt}{2pt}
    \multicolumn{1}{c}{The prompt for demonstration}\\
   \midrule
   $\dots \dots$ \\
   $\dots \dots$ \\
   The purple-crested turaco (Gallirex porphyreolophus) or, in South Africa, the purple-crested loerie, (Khurukhuru in the Luven\d{d}a (Ven\d{d}a) language) is a species of bird in the clade Turaco with an unresolved phylogenetic placement. Initial analyses placed the purple-crested turaco in the family Musophagidae, but studies have indicated that these birds do not belong to this family and have been placed in the clade of Turacos with an unresolved phylogeny. It is the National Bird of the Kingdom of Eswatini, and the crimson flight feathers of this and related turaco species are important in the ceremonial regalia of the Swazi royal family.
   This bird has a purple-coloured crest above a green head, a red ring around their eyes, and a black bill. The neck and chest are green and brown. The rest of the body is purple, with red flight feathers. Purple-crested turacos are often seen near water sources, where they can be observed drinking and bathing, which helps them maintain their vibrant plumage.
   Purple-crested turacos are considered to be large frugivores that are known to carry cycad seeds from various plant species long distances from feeding to nesting sites. After fruit consumption, they regurgitate the seeds intact where they can germinate nearby. G. porphyreolophus primarily consumes fruits whole like many other large frugivores which are suggested to be necessary for effective ecosystem functioning. Among similar turacos, the purple-crested turaco have faster minimum transit times when consuming smaller seed diets than larger seed diets, and G. porphyreolophus has been shown to have significantly faster pulp (seedless fruit masses) transit time than another closely related Turaco when fed only the pulp of larger-seeding fruits than smaller-seeding fruits.
   In addition to their frugivorous diet, these birds are occasionally seen foraging for other food items such as nuts and leaves, which provide essential nutrients. They are also known to coexist with various other animals, including those that might enjoy strawberries and other similar fruits. The purple-crested turaco's role in seed dispersal is crucial, and their interaction with different elements of their habitat, including water and diverse plant materials, highlights their importance in maintaining ecological balance.\\
   $\dots \dots$ \\
   $\dots \dots$ \\
   \\
   \specialrule{1pt}{0pt}{2pt}
      \specialrule{1pt}{0pt}{2pt}
   \end{tabular}
   }
   % \vskip -1 em
   \caption{The prompt for demonstration }
   \label{tab:demo_prompt}
   % \vskip -1 em
   \end{table}


Here we provide the CoT prompt exemplars for GSM8K which is used in section \ref{sec:icl}.
\begin{table}[ht]
   \centering
   \footnotesize
   \resizebox{1\linewidth}{!}{
   \begin{tabular}{p{15cm}}
   \specialrule{1pt}{0pt}{2pt}
  \specialrule{1pt}{0pt}{2pt}

    \multicolumn{1}{c}{GSM8K experiemnt CoT Prompt Exemplars}\\
   \midrule
   Question: There are 15 trees in the grove. Grove workers will plant trees in the grove today. After they are done, there will be 21 trees. How many trees did the grove workers plant today?     

   There are 15 trees originally.
   Then there were 21 trees after some more were planted.
   So there must have been 21 - 15 = 6.
   The answer is 6.
   \\
   Question: If there are 3 cars in the parking lot and 2 more cars arrive, how many cars are in the parking lot? 

   There are originally 3 cars.
   2 more cars arrive.
   3 + 2 = 5.
   The answer is 5.
   \\
   Question: Leah had 32 chocolates and her sister had 42. If they ate 35, how many pieces do they have left in total? 

   Originally, Leah had 32 chocolates.
   Her sister had 42.
   So in total they had 32 + 42 = 74.
   After eating 35, they had 74 - 35 = 39.
   The answer is 39.
   \\
   Question: Jason had 20 lollipops. He gave Denny some lollipops. Now Jason has 12 lollipops. How many lollipops did Jason give to Denny? 

   Jason started with 20 lollipops.
   Then he had 12 after giving some to Denny.
   So he gave Denny 20 - 12 = 8.
   The answer is 8.
   \\
   Question: Shawn has five toys. For Christmas, he got two toys each from his mom and dad. How many toys does he have now? 

   Shawn started with 5 toys.
   If he got 2 toys each from his mom and dad, then that is 4 more toys.
   5 + 4 = 9.
   The answer is 9.
   \\
   Question: There were nine computers in the server room. Five more computers were installed each day, from monday to thursday. How many computers are now in the server room? 

   There were originally 9 computers.
   For each of 4 days, 5 more computers were added.
   So 5 * 4 = 20 computers were added.
   9 + 20 is 29.
   The answer is 29.
   \\
   Question: Michael had 58 golf balls. On tuesday, he lost 23 golf balls. On wednesday, he lost 2 more. How many golf balls did he have at the end of wednesday? 

   Michael started with 58 golf balls.
   After losing 23 on tuesday, he had 58 - 23 = 35.
   After losing 2 more, he had 35 - 2 = 33 golf balls.
   The answer is 33.
   \\
   Question: Olivia has \$23. She bought five bagels for \$3 each. How much money does she have left? 

   Olivia had 23 dollars.
   5 bagels for 3 dollars each will be 5 x 3 = 15 dollars.
   So she has 23 - 15 dollars left.
   23 - 15 is 8.
   The answer is 8.
   \\
   \specialrule{1pt}{0pt}{2pt}
  \specialrule{1pt}{0pt}{2pt}
   \end{tabular}
   }
   % \vskip -1 em
   \caption{GSM8K CoT Prompt Exemplars }
   \label{tab:CoT_prompt}
   % \vskip -1 em
   \end{table}

\section{Limitations}
The major limitation of the \method{} is that it uses fixed-size token groups for chunking. While adaptive chunking methods could potentially improve performance, they would introduce significant inference latency. Therefore, finding a balance between the chunking method and inference latency is key to improving KV cache compression.
\section{Licenses}
\label{app_licenses}
For the evaluation dataset, all the datasets, including, GSM8K~\citep{gsm8k}, LongBench~\citep{bai2023longbench} are released under MIT license. NIAH~\citep{needle} is released under GPL-3.0 license.
   
\end{document}


\section*{Acknowledgments}
Use unnumbered third level headings for the acknowledgments. All
acknowledgments, including those to funding agencies, go at the end of the paper.
