\section{Related Work}
We provide a comprehensive literature review in Appendix~\S\ref{sub_app:related_work}, while this section focuses primarily on the most relevant works.

\subsection{Social Agent and Strategy Injection}
Large language models (LLMs) can potentially become proficient social agents~\cite{park2023generative,huang2023humanity,choi-etal-2023-llms}. 
% 
However, accurately simulating diverse and open-ended human social behaviors in an infinite action space remains a challenge ~\cite{mou2024individual}.
% 
As a result, task-specific agents~\cite{chen2023money} or those with constrained actions~\cite{deng2023plug,zhang-etal-2024-strength} struggle to adapt.



Strategy injection enhances social agents by integrating human priors into model behavior. 
% 
Existing work mainly focuses on \textit{inference-time injection}, such as multi-agent interactions~\cite{lan-etal-2024-llm} or auxiliary strategy models~\cite{chang-chen-2024-injecting,feng2023towards}, but these methods incur significant inference overhead and are task-specific~\cite{deng2023plug}. 
% 
In contrast, \citet{sotopia-pi} introduces \textit{training-time injection}, cloning expert behavior and applying self-reinforcement. 
% 
This allows weaker agents to achieve expert-level performance in open-ended, multi-task settings without additional inference costs.
However, it inherits the expert’s limitations, potentially leading to issues like prolonged deadlock.


\subsection{Negotiation Theory}
\label{sec:related_work_negotiation}

Negotiation theory~\cite{korobkin2024negotiation} offers a universal strategy framework for addressing social tasks, many of which are characterized as mixed-motive negotiations~\cite{deutsch1973resolution}. 
% 
These involve non-adversarial interactions where parties have differing motivations and preferences~\cite{froman1970compromise}. 
% 
Even when social goals seem to conflict, a win-win outcome can be achieved by finding complementary interests.
% 
According to negotiation theory, final agreements in such scenarios can approach the Pareto frontier~\cite{tripp1992evaluation}, though achieving such outcomes remains challenging. 
% 
To address this, \citet{thompson2015mind} proposes a structured workflow that guides negotiators toward Pareto-optimal solutions. 
% 
Based on this framework, we distill four core steps applicable to social tasks: \textit{Resource Assessment}, \textit{Assessment of Difference}, \textit{Initial Proposal} and \textit{Update Proposal}.