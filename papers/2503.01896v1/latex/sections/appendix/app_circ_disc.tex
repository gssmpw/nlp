\section{Circuit Discovery}
\label{app:circ_disc}

We follow the work by \cite{wang2022interpretability} and conduction patching and knockout experiments to recover circuits at each model training iteration and present our circuit discovery for the case of fine-tuning with 3 epochs as a template. 
\begin{figure}
    \includegraphics[width=0.45\textwidth]{latex/img/appendix/direct.pdf}
    \caption{ Isolating Heads with highest direct logit contribution to the task: Name Mover Heads and Negative Name Mover Heads}
    \label{fig:app-logit-attr}
\end{figure}
\begin{figure}
    \includegraphics[width=0.45\textwidth]{latex/img/appendix/s-inhibiton.pdf}
    \caption{ Isolate important heads that most impact the queries of Name Mover Heads: S-Inhibition Head }
    \label{fig:app-sin}
\end{figure}
We initially, analyze the attention patterns of the heads that have the highest logit attribution to the task, see \autoref{fig:app-logit-attr}. We find these to be the Name Mover Heads and Negative Name Mover Heads similar to \cite{wang2022interpretability}. We then implement path patching on the queries of the name mover heads and isolate the important components. After Knockout Experiments, analyzing QK matrix, we identify these heads to be the S-Inhibition Heads see \autoref{fig:app-sin}. Given this we proceed similar to \cite{wang2022interpretability} to find the Induction Heads, Previous Token Heads and Duplicate Token Heads. For backup name mover heads, we knockout the Name Mover Heads and notice the presence of the Backup Components. For example, if ablate 9.9, the following heads will backup the behavior: 
\begin{figure}
  \begin{minipage}{0.4\textwidth}
    
    \includegraphics[width=\textwidth]{latex/img/appendix/backup.pdf}
    \caption{\label{fig:app-logit-attr2} Discovering Backup Name Mover Heads}
  \end{minipage}%
  \hfill 
  \begin{minipage}{0.4\textwidth}
    \includegraphics[width=\textwidth]{latex/img/appendix/minimality.pdf}
    \caption{\label{fig:app-sin2} Minimality Scores for the circuit in \autoref{fig:ca3-circuit} }
  \end{minipage}%
\end{figure}
We also report the completeness scores for the discovered circuit , see \autoref{fig:ca3-comp}
\begin{figure}
    \centering
    \includegraphics[width = 0.48\textwidth]{latex/img/appendix/completeness_finetuned3epoch.pdf}
    \caption{Completeness scores for the circuit in \autoref{fig:ca3-circuit}}
    \label{fig:ca3-comp}
\end{figure}
