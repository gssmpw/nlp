\section{Dataset Size}
\label{app:data}
\subsection{IOI dataset}
As we mentioned before, indirect object identification(IOI) is a task related to identifying the indirect object. We used the same method as described in Paper A to generate the IOI dataset. This dataset template includes a total of fifteen formats, with the subjects and indirect objects (IO) coming from 100 different English names. Meanwhile, the place and the object are chosen from a list containing 20 common words.%Todo: samples 

We generate 6360 samples from the template in the IOI dataset $p_{IOI}$. We chose this dataset size for our IOI dataset for several reasons. Firstly, this size allows us to observe changes in each head. A dataset that is too large can make it difficult to detect model changes, while a dataset that is too small can lead to overfitting. Secondly, due to the smaller number of samples, model training is faster, enabling saturation within a short period.

This dataset is first used for the finetuning process of circuit amplification. Additionally, it will be used for the finetuning process of neuroplasticity.
\subsection{Poisoning datasets}
For data poisoning, we also randomly generated three different datasets: the Duplication Dataset, the Name Moving Dataset, and the Subject Duplication Task Dataset. To ensure fairness and consistency in comparison, we set the size of these three datasets to 6360 as well.

\begin{itemize}
    \item \textbf{Duplication dataset} is using a random single token to replace the second subject token. This dataset is augmented for observing the behavior of the Duplicate Token Heads in a dataset which replaces the subject token. An example in the Duplication dataset is that \textit{"When Mark and Rebecca went to the garden, Mark gave flowers to Rebecca"} is augmented to \textit{"When Mark and Rebecca went to the garden, Tim gave flowers to Rebecca"}. 
    
    \item \textbf{Name Moving dataset} is using a random single token to replace the final token which is the second token of IO. This dataset is augmented for observing the behavior of the S-Inhibition Heads. An example in Name Moving dataset is that \textit{"When Mark and Rebecca went to the garden, Mark gave flowers to Rebecca"} is augmented to \textit{"When Mark and Rebecca went to the garden, Mark gave flowers to Stephanie"}.
    
    \item \textbf{Subject Duplication dataset} is using the subject token S to replace the output IO token. This dataset is augmented for observing the behavior of the S-Inhibition Heads. An example in the Subject Duplication dataset is that \textit{"When Mark and Rebecca went to the garden, Mark gave flowers to Rebecca"} is augmented to \textit{"When Mark and Rebecca went to the garden, Mark gave flowers to Mark"}.
    
\end{itemize}

