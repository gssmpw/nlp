\section{Circuit Evaluation}

\label{app:circ_eval}
\textbf{Minimality}: Minimality criterion checks if the circuit contains unnecessary components. More formally, for a circuit $C$, $\forall v \in C \: \exists\: K \subseteq C \backslash \{v\}$ we expect to have a large minimality score defined as follows,  $|F(C\backslash(K \cup \{v\} )) - F(C\backslash K)|$  \cite{wang2022interpretability, prakash2024fine}.

\textbf{Completeness}: Completeness criterion checks if the circuit contains all necessary components. More formally, for a circuit $C$ and the whole model $M$, $\forall K \subseteq C$, incompleteness score$|F(C\backslash K) - F(M\backslash K)|$\cite{wang2022interpretability} should be small. We set K to be an entire class of circuit heads. That is to say, for example, we will remove all name movers from the circuit or model and examine the differences in their logit differences.

\begin{figure*}
    \centering
    \includegraphics[scale =0.1,width = 0.8\textwidth]{latex/img/ioicirc.pdf}
    \caption{The Indirect Object Identification Circuit Discovered by \cite{wang2022interpretability} for GPT-2-Small}
    \label{fig:ioicirc}
\end{figure*}