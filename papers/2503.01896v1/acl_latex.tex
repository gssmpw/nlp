% This must be in the first 5 lines to tell arXiv to use pdfLaTeX, which is strongly recommended.
\pdfoutput=1
\pdfminorversion=7
\pdfobjcompresslevel=0
\pdfcompresslevel=0

% In particular, the hyperref package requires pdfLaTeX in order to break URLs across lines.

\documentclass[11pt]{article}

% Change "review" to "final" to generate the final (sometimes called camera-ready) version.
% Change to "preprint" to generate a non-anonymous version with page numbers.
\usepackage[preprint]{acl}

% Standard package includes
\usepackage{times}
\usepackage{latexsym}

\usepackage{float}
\usepackage{paralist}
\usepackage{wrapfig}
 % adjust the value as needed
% For proper rendering and hyphenation of words containing Latin characters (including in bib files)
\usepackage[T1]{fontenc}
% For Vietnamese characters
% \usepackage[T5]{fontenc}
% See https://www.latex-project.org/help/documentation/encguide.pdf for other character sets

% This assumes your files are encoded as UTF8
\usepackage[utf8]{inputenc}

% This is not strictly necessary, and may be commented out,
% but it will improve the layout of the manuscript,
% and will typically save some space.
\usepackage{microtype}

% This is also not strictly necessary, and may be commented out.
% However, it will improve the aesthetics of text in
% the typewriter font.
\usepackage{inconsolata}

%Including images in your LaTeX document requires adding
%additional package(s)
\usepackage{adjustbox}
\usepackage{microtype}
\usepackage{graphicx}
\usepackage{subcaption}

\usepackage{booktabs}
\usepackage{amsmath}
\usepackage{amssymb}
\usepackage{mathtools}
\usepackage{amsthm}
\usepackage{hyperref}
\usepackage{colortbl}
\usepackage{pifont}
\usepackage[belowskip=-12pt,aboveskip=1pt]{caption}
\newcommand*\CHECK{\ding{51}}
\newcommand{\xmark}{\ding{55}}%
\newcommand*{\defeq}{\stackrel{\text{def}}{=}}
\renewcommand{\thesubfigure}{\alph{subfigure}}
% If the title and author information does not fit in the area allocated, uncomment the following
%
%\setlength\titlebox{<dim>}
%
% and set <dim> to something 5cm or larger.

\title{Neuroplasticity and Corruption in Model Mechanisms: A Case Study Of Indirect Object Identification}

% Author information can be set in various styles:
% For several authors from the same institution:
\author{Vishnu Kabir Chhabra, Ding Zhu,  Mohammad Mahdi Khalili \\
        Department of Computer Science and Engineering, The Ohio State University,  USA\\
        \texttt{\{chhabra.67, zhu.3723, khalili.17\}@osu.edu}}
% if the names do not fit well on one line use
%         Author 1 \\ {\bf Author 2} \\ ... \\ {\bf Author n} \\
% For authors from different institutions:
% \author{Author 1 \\ Address line \\  ... \\ Address line
%         \And  ... \And
%         Author n \\ Address line \\ ... \\ Address line}
% To start a separate ``row'' of authors use \AND, as in
% \author{Author 1 \\ Address line \\  ... \\ Address line
%         \AND
%         Author 2 \\ Address line \\ ... \\ Address line \And
%         Author 3 \\ Address line \\ ... \\ Address line}

% \author{Vishnu Kabir Chhabra \\
%  Ohio State University  \\
%  Columbus, OH, USA\\
%  \texttt{chhabra.67@osu.edu} \\\And
% Ding Zhu \\
%  Ohio State University  \\
%  Columbus, OH, USA\\
%  \texttt{zhu.3723@osu.edu} \\ \\\And
%   Mohammad Mahdi Khalili \\
%  Ohio State University  \\
%  Columbus, OH, USA\\
%  \texttt{khalili.17@osu.edu} \\}
  

%\author{
%  \textbf{First Author\textsuperscript{1}},
%  \textbf{Second Author\textsuperscript{1,2}},
%  \textbf{Third T. Author\textsuperscript{1}},
%  \textbf{Fourth Author\textsuperscript{1}},
%\\
%  \textbf{Fifth Author\textsuperscript{1,2}},
%  \textbf{Sixth Author\textsuperscript{1}},
%  \textbf{Seventh Author\textsuperscript{1}},
%  \textbf{Eighth Author \textsuperscript{1,2,3,4}},
%\\
%  \textbf{Ninth Author\textsuperscript{1}},
%  \textbf{Tenth Author\textsuperscript{1}},
%  \textbf{Eleventh E. Author\textsuperscript{1,2,3,4,5}},
%  \textbf{Twelfth Author\textsuperscript{1}},
%\\
%  \textbf{Thirteenth Author\textsuperscript{3}},
%  \textbf{Fourteenth F. Author\textsuperscript{2,4}},
%  \textbf{Fifteenth Author\textsuperscript{1}},
%  \textbf{Sixteenth Author\textsuperscript{1}},
%\\
%  \textbf{Seventeenth S. Author\textsuperscript{4,5}},
%  \textbf{Eighteenth Author\textsuperscript{3,4}},
%  \textbf{Nineteenth N. Author\textsuperscript{2,5}},
%  \textbf{Twentieth Author\textsuperscript{1}}
%\\
%\\
%  \textsuperscript{1}Affiliation 1,
%  \textsuperscript{2}Affiliation 2,
%  \textsuperscript{3}Affiliation 3,
%  \textsuperscript{4}Affiliation 4,
%  \textsuperscript{5}Affiliation 5
%\\
%  \small{
%    \textbf{Correspondence:} \href{mailto:email@domain}{email@domain}
%  }
%}

\begin{document}
\maketitle
\begin{abstract}
Previous research has shown that fine-tuning language models on general tasks enhance their underlying mechanisms. However, the impact of fine-tuning on poisoned data and the resulting changes in these mechanisms are poorly understood. 
This study investigates the changes in a model's mechanisms during toxic fine-tuning and identifies the primary corruption mechanisms. We also analyze the changes after retraining a corrupted model on the original dataset and observe neuroplasticity behaviors, where the model relearns original mechanisms after fine-tuning the corrupted model. Our findings indicate that: (i) Underlying mechanisms are amplified across task-specific fine-tuning which can be generalized to longer epochs, (ii) Model corruption via toxic fine-tuning is localized to specific circuit components, (iii) Models exhibit neuroplasticity when retraining corrupted models on clean dataset, reforming the original model mechanisms.
\end{abstract}

\documentclass[../main.tex]{subfiles}
\graphicspath{{../images/}}
\makeatletter
\def\input@path{{../images/}}
\makeatother
\begin{document}
\section{Introduction}
\begin{figure}
\centering
\begin{tikzpicture}
\node[inner sep=0pt] (ws) at (0, 0) {
\includegraphics[height=.4\textwidth, trim={10cm 0 10cm 0},clip]{world_space.png}};
\node[inner sep=0pt] (cs) at (6,0) {\includegraphics[height=.4\textwidth, trim={10cm 1cm 10cm 4cm},clip]{conf_space.png}};
\end{tikzpicture}
\vspace{-5pt}
\label{fig:pbrm_intro}
\caption{\textbf{Left}: Shows world space obstacles as grey spheres. Robots start and goal configuration is colored red and green, respectively. Configurations along the computed path are colored transparent blue. \textbf{Right:} Mapped world space scenario to configuration space. Obstacle region is the grey mesh. Red spheres are collision-free regions computed by the neural SCDF. The optimized shortest path in the convex corridor is the blue curve.}
\vspace{-25pt}
\end{figure}
Motion planning is the problem of finding a collision-free trajectory that connects a given start and goal configuration. The planning takes place in the configuration space of the robot. For single body robots, like mobile robots or drones, the configuration space and the world space are usually the same. This simplifies the planning, since explicit obstacle representations are available which enables geometrical tools like separating hyperplanes, smallest distance to obstacles etc., to be used when designing motion planning algorithms. For multi-body robots like manipulators, the situation is completely different. The world space obstacles are usually mapped to non-convex regions, and to make the problem even harder, the mapping is usually not known. Forming explicit representations of the obstacle region in the configuration space is usually too expensive or intractable. Despite all of this, sampling based planners are used with great success, which mainly is due to their use of implicit representations of the obstacle region. The basic idea is to construct a graph in the configuration space that covers and connects the collision-free region. From this graph, a path can be extracted that connects a given start and goal configuration. The approach is computationally expensive, since the graph is constructed with the smallest geometrical building block available, points, which represents a collision-check. Furthermore, the extracted paths from the graph are non-smooth and jagged due to the stochastic nature of the approach. This adds an additional post-processing step to the process, where the paths are shortcutted and smoothened, before the path can be used for tracking. Clearly a lot of time is invested to form this graph and produce smooth paths. Thus, if the obstacles start to move, then all of this work is done in no use, since all points that make up this graph need to be re-verified, which is simply too time consuming to be done in real time.
\\\\
In this work, we want to address the existing drawbacks of the sampling based planners. Our main contribution is an improved motion planner where each vertex in the graph covers a collision-free region in the form of a sphere instead of a point and where the edges are formed with neighboring intersecting spheres. This representation has the advantage of instead of returning piecewise linear paths, returning a sequence of overlapping spheres, i.e. a convex corridor, that connects a given start and goal configuration, illustrated in Figure \ref{fig:pbrm_intro}. This convex corridor allows us to use convex optimization to produce smooth trajectories, instead of computationally expensive post-processing methods. The representation further allows us to estimate the coverage of the collision-free space, which gives us awareness and feedback in the offline roadmap construction phase. Finally, our representation is simple to adapt to moving obstacles, simply requery for the new radii and recheck for intersections. 
\\\\
The spherical collision-free regions are formed using a signed distance function (SDF), which is a function that returns the smallest distance from an arbitrary point to the boundary of an obstacle. As the name implies, the distance is signed, thus if the point is inside the obstacle it is negative otherwise positive. If the distance is positive, a sphere with radius equal to the distance is guaranteed to cover a collision-free region. Using an SDF in motion planning is not new, but what is novel about our approach is that we express the distance in the configuration space instead of the world space and by doing so allows us to form these convex collision-free regions. We refer to the resulting SDF as a signed configuration distance function (SCDF). Computing an SCDF analytically is non-trivial, our approach is therefore to parameterize the SCDF with a deep neural network and learn the mapping by supervised learning. Our resulting neural SCDF can compute distances for different parameter values of obstacle shapes and we also show how multiple distances can be combined, thus making our approach flexible.
\section{Related work}
Motion planning algorithms can roughly be divided into three families, grid-based, sampling based and optimization based methods. Grid-based methods (GBM) discretize the planning space from which a graph is then compiled. A standard search method is A$^\star$ \citep{a_star}, which is classified as an \textit{informed} search method, since it employs a heuristic function to speed up the search. A$^\star$ guarantees to return an optimal path at the level of discretization used. GBMs usually discretize the planning space by a regular lattice and this limits the GBMs to problems with low dimensionality due to the curse of dimensionality. Thus, GBMs are usually limited to single-body robots where the degrees of freedom (DOF) are low. To overcome the inherent scaling problem with the GBMs, stochastic methods are usually used for multi-body robots. These methods are termed as sampling-based methods (SBM) and core members within this family are the rapidly-exploring random trees (RRT) \citep{rrt} and the probabilistic roadmap (PRM) \citep{prm}. RRT grows a tree from the start configuration and explores the collision-free region in a rapid way until it is able to connect to the goal region. RRT is usually improved by bi-directional planning \citep{rrt_connect}, i.e. an additional tree is grown from the goal configuration and the trees are tested for connection after any tree has been expanded. RRT is a single-query method, thus it searches for a path from scratch each time it is queried. Contrary to this, PRM is a multi-query method, which solves for multiple queries without starting from scratch. PRM does this by creating a roadmap (graph) that covers the collision-free space as an offline step. The graph is then used to solve for multiple queries. PRMs are used in cases where the environment does not change since the extra offline step is too computationally costly and needs to be re-done if the environment is changed. In our work, we address this inherent issue by using a different roadmap representation. Our vertices in the graph cover a collision-free region in the form of spheres and we form the edges by checking for intersecting spheres. If something in the environment changes, we recompute the spheres radii and recheck the intersections, without relying on collision detection. We use a trained neural network to compute the sphere radius, therefore querying for the radius can be done fast, hence our representation enables the PRM for dynamic environments.
\\\\
In the recent decades, optimization based methods (OBM) \citep{chomp, schulman, itomp, stomp} have been introduced as an alternative to SBM for multi-body robots. Like the SBM, the OBMs scale well to higher dimensional problems and produce smoother motion. It is common to use a SDF in the optimization since it is a smooth function, thus enabling gradient-based methods. However, the standard way of expressing the SDF is in world space. The distance therefore needs to be mapped to the configuration space by the forward kinematics. This mapping makes the optimization problem a non-linear program (NLP), which is computationally expensive to solve. Recently, a different approach has been proposed. In \cite{mp_gcs} motion planning is formulated as a convex optimization problem by using the graph of convex sets framework \citep{gcs}. The underlying idea is to decompose the collision-free space into intersecting convex sets from which a convex optimization problem is formulated. In cases where an explicit representation of the obstacles in the configuration space exists, like for single-body robots, creating collision-free convex regions can be done fast \citep{iris}. For multi-body robots, this is non-trivial. Existing work does this successfully \citep{iris_nlp, iris_c} by an optimization based approach, but the methods are still too time consuming to be used in the presence of moving obstacles. Our approach is instead to use deep learning to learn an SDF expressed in the configuration space. With this, we can query for shortest distances to the collision boundary, which allows us to expand spherical regions which are collision-free. Our approach is fast and therefore enables our suggested roadmap planner to be used in dynamic environments.
\\\\
Recent research has focused on learning collision detection \citep{fk_kernel_distance, diffco, graphdistnet} by predicting the signed distance between the robot links and the surrounding obstacles in the world space. The learned SDF is used in trajectory optimization but since the distance is expressed in the world space, the problem becomes an NLP and therefore takes a long time to solve. We take a novel approach and suggest to instead express the signed distance in the configuration space. This allows us to improve the PRM at the same time as it enables convex optimization for trajectory optimization, which runs faster and is more reliable than NLP solvers. In \cite{cspf} a learned signed distance function in the configuration space is proposed similar to our approach. However, their approach is restricted to point cloud representations, while we propose to represent the obstacles as parameterized geometric shapes, e.g. spheres. Furthermore, we also show how to use our learned SCDF to improve an existing roadmap planner.
\section{Problem formulation}
A robot is located in the world space, $\W \subset \R^3 $. The unique location of the robot is given by its configuration $\q \in \C$, where $\C$ is the configuration space. The set of points covered by the robots bodies at a certain configuration is expressed as $\B(\q) \subset \W$. The robot is surrounded by $\NrObst$ obstacles $\O = \bigcup_{i=1}^{\NrObst} \O_i$, where  $\O_i \subset \W$. The representation of the obstacle in the configuration space is the set $\C\O_i = \{\q \in \C \: |\: \B(\q) \cap \O_i \neq \emptyset \}$. The obstacle space is formed as $\Co = \bigcup_{i=1}^{\NrObst} \C \O_i$. The complement is referred to as the free space, $\Cf = \C \setminus \Co$. The path planning problem is a tuple, ($\Cf$, $\qStart$, $\qGoal$), where we want to connect a query pair, consisting of a start, $\qStart$, and goal configuration, $\qGoal$, with a geometric path, $\q(s): [0, 1] \mapsto \Cf$, such that $\q(0)=\qStart$ and $\q(1)=\qGoal$, or report correctly when such a path does not exist.
\end{document}


\section{Preliminary}

\paragraph{Notation} Consider a sentence of $T$ tokens $\vx=\{\vx_1,\ldots, \vx_T\}\in\gX$, and let $P$ be the unknown target language distribution, $\tilde P(\vx)$ be the empirical distribution of the training data (which is an approximation of $P$), and $Q$ be the distribution of our model at hand. Since our paper is also closely related to RLHF, we will also use $\pi$ to represent the distributions. In particular, we sometimes write $\pi_\theta$ for a distribution that is parameterized by $\theta$, where $\theta$ is usually the set of trainable parameters of the LLM; we write $\pr$ for a reference distribution that should be clear given the context. The next token prediction loss is minimizing the forward-KL between $P$ and $Q$. 





\section{Problem Statement}
\label{sec:problem_statement}


\begin{figure*}[tb!]
    %\begin{figure}[t]
 \centering
  \includegraphics[width=1\linewidth]{image/motivation_example_svg2pdf.pdf}
 \caption{Overview of human preference judgments for a pair of paraphrase ad texts.}
 \label{fig:motivation_example}
\end{figure}
    \begin{subfigure}[b]{0.35\textwidth}
        \centering
        \includegraphics[]{motivation_legend.pdf}
        \includegraphics[]{b8.pdf}
        \includegraphics[]{b32.pdf}
        
    \end{subfigure}
    \hspace{6mm}
    \begin{subfigure}[b]{0.46\textwidth}
        \includegraphics[]{TT_acc_fig.pdf}
    \end{subfigure}

    \caption{The training time on fixed number of samples~$T_s$ and the total training time~$TT_\text{acc}$ to reach an accuracy threshold of~$78\%$ using two batch sizes 8 and 32, while considering the maximum feasible GPU frequencies under three power constraints; $P_1=\SI{4.5}{\watt}$, $P_2=\SI{5}{\watt}$, and~$P_3=\SI{7}{\watt}$. We observe that for~$P_1$ and~$P_2$, selecting $b=8$ will lead to lower~$TT_\text{acc}$, while for~$P_3$ selecting $b=32$ is better. This is in contrast with our observation for~$T_s$, where selecting $b=32$ is the best option in all cases.}
    \label{fig:b8_b32_power_constraint}
\end{figure*}

We consider the following scenario: for a specific training task, an edge device requests a pre-trained \ac{NN} model~$\nn$ with its weights~$\theta$ from a server in order to fine-tune it on local data~$D$ till reaching a given accuracy threshold. 
Importantly, the edge device has a power constraint~$P_{\text{max}}$, which should not be exceeded during the training process. 
Our goal in this paper is to \textit{minimize the training (fine-tuning) time at edge devices under their given power constraints.} 



 We introduce~$T_s$ as the training time required to apply training using a fixed number of samples~$s$. As shown in \cref{fig:3d_profiling}, the joint selection of $b$ and $f$ will help reduce $T_s$  under a power constraint. However, the ultimate optimization goal is to minimize the total training time required to reach a target accuracy, which we label~$TT_{\text{acc}}$.
A set of parameters, i.e., frequency and batch size~$(f,b)$, that are optimal for training a fixed number of samples ($T_s$) might not be necessarily optimal for the training to accuracy ($TT_{\text{acc}})$.

We display in~\cref{fig:b8_b32_power_constraint} the training time to reach an accuracy  of~$78\%$ (~$TT_{\text{acc}}$) for ResNet18 using two batch sizes of~$b_1=8$ and~$b_2=32$ under three different power constraints (i.e., $P_1=\SI{4.5}{\watt}$, $P_2=\SI{5}{\watt}$, and $P_3=\SI{7}{\watt}$). 
We notice that for the three power constraints, selecting~$b_1$ allows to utilize a higher frequency than~$b_2$.
%For the feasible operating points (frequency, batch size) that satisfy the power constraints, we compare $T_S$ and $TT_{\text{acc}}$.
For each batch size, we select the highest frequency that satisfies the power constraint, and measure~$T_s$ and~$TT_{\text{acc}}$.
We observe that using~$b_2$ (the higher batch size) always leads to a lower~$T_s$. %\textcolor{red}{Furthermore, the ratio of ~$T_s$ for ~$b_1$ to ~$T_s$ for ~$b_2$ increases as the power limit value increases, indicating that the larger batch size ~$b_2$ becomes more efficient relative to ~$b_1$ under higher power limits. }
However, selecting the same batch size over~$b_1$ leads to a longer~$TT_{\text{acc}}$ for~$P_1$ and~$P_2$, and shorter~$TT_{\text{acc}}$ for~$P_3$. 
%This shows the complexity of the targeted problem, as the effect of the power constraint on the feasible frequency and the different number of training iterations to reach accuracy for batch sizes highly influences the optimal batch size to minimize~$TT_{\text{acc}}$.

This shows the complexity of the targeted problem. In particular, $TT_{\text{acc}}$ does not only depend on $T_s$, but it also depends on the number of times of processing~$s$ to reach target accuracy ($N_{{s}_\text{acc}}$). In this example, ~$N_{{s}_\text{acc}}$ for $b_2$ is equal to $15$ while ~$N_{{s}_\text{acc}}$ for $b_1$ is equal to $10$. These values and the effect of power constraint on the feasible frequency highly influence the optimal batch size to minimize~$TT_{\text{acc}}$. 
In summary, there is no clear indication on how to select the optimal operating points $(f,b)$ to achieve the target goal. 
%These values have led to higher $TT_{\text{acc}}$ at $b_2$ for $P_1$ and $P_2$, but still lower value for $P_3$, due to the impact of the selected frequencies on $T_s$. 

We formulate our optimization problem as follows:
\begin{equation}
    \begin{aligned}
        & \underset{b\in\mathcal{B}, f\in\mathcal{F}}{\text{min}}
        & & TT_{\text{acc}}(b, f, \nn, D) \\
        & \text{subject to}
        & & P(b, f, \nn) \leq P_{\text{max}}
    \end{aligned}
\label{eq:optimization}
\end{equation}
where $\mathcal{B}$ is the set of feasible batch sizes, $\mathcal{F}$ is the set of available GPU's frequencies, and $P(b, f, \nn)$ is the required power to training~$M$ using~$b$ and~$f$. We rewrite ~$TT_{\text{acc}}$ as the multiplication of~$T_{s}$ and ~$N_{{s}_{\text{acc}}}$, we thus have:
\begin{equation}
    TT_{\text{acc}}(b, f, \nn, D) = T_{s}(b, f, \nn) \times  N_{{s}_{\text{acc}}}(b, \nn, D)
    \label{eq:time_to_acc2}
\end{equation}
$s$ is selected, s.t.~$b_{\text{max}} \leq s \leq |D|$, where~$b_{\text{max}}$ is the largest batch size that can fit into the memory of the devices. 
This detached formulation enables our proposed optimization method, presented in \cref{sec:methodology}.
In particular, the first factor~$T_{s}$ does not depend on the training data~$D$, nor on the accuracy threshold. The second factor~$N_{{s}_\text{acc}}$ is independent of the GPU frequency of the device.

%the number of times of processing~$s$ to reach target accuracy ($N_{{s}_{\text{acc}}}$)
%, and therefore, we can perform profiling on the device prior to the training to obtain feasible operating points. The second factor~$N_{{s}_\text{acc}}$ is independent of the GPU frequency of the device.% and hence it can be estimated on the server. %by performing predictions on the server on a proxy data set. 
%The details of our method are discussed in the following section. 





\section{Phase Transitions via Fine-Tuning}
\textbf{Motivation}: Building on recent advances in mechanistic interpretability, such as \citet{zhong2024clock} and \citet{nanda2023progress}, which explore  phase transitions during grokking in toy models, our work aims to extend this understanding to fine-tuning. We focus on elucidating phase transitions in model mechanisms under various fine-tuning conditions. By leveraging insights into the model's existing mechanisms, we design corruption experiments that disrupt these mechanisms through targeted data augmentations. Our goal is to analyze how fine-tuning on corrupted/clean data reshapes model behavior, with the goal of a deeper understanding of fine-tuning dynamics in neural networks. 

In the following subsections, we discuss the effects of task-specific fine-tuning on the original "\textit{clean}" dataset, i.e, the IOI dataset, and discover \textit{Circuit Amplification} and the underlying mechanisms of the increased capabilities of the model to perform the underlying task. Furthermore, we discuss the effects of model poisoning on the underlying circuit of the model for the IOI task and discover that the underlying changes are localized to the circuit components of the model. Specifically, we analyze the effects of fine-tuning on the \textit{attention heads} in the original IOI circuit, as the MLP layers mechanisms do not change across time, see \autoref{app:mlp} for further explanation.
\label{methodology}
\subsection{Amplification Of Model Mechanisms}
\label{circuit-amp}
First, we study the effects of task-specific fine-tuning using the IOI dataset (clean dataset) on the model. We mechanistically interpret the change in the underlying mechanism. 
Consistent with expectations, our experiments uniformly demonstrate a significant boost in IOI task accuracy following the task-specific fine-tuning on the clean dataset, see \autoref{amp-table}. 
\begin{table}[H]
\vspace{0.1cm}
\caption{Performance, Faithfulness, and Sparsity of Discovered Circuits at Different Epochs compared to Model Performance}
\centering
\scalebox{0.65}{
\begin{tabular}{|l|l|l|l|l|l|}
\hline
Epoch &$F(Y)$&$F(C)$&Faithfulness&Sparsity&|$F(C)-F(C_{M_{GPT2}})$|\\
\hline
%\rowcolor[gray]{.95}
$1$ &   $6.32$    & $6.22$      &  $98.4\%$&   $1.92\%$    & $1.2$   \\ \hline
$3$ & $11.56$ & $11.50$ & $99.5\%$ & $1.95\%$ & $2.2$ \\ \hline%\rowcolor[gray]{.95}
$10$ &  $15.51$     & $15.26$      & $98.4\%$& $1.98\%$& $1.48$   \\\hline
$15$ &  $16.77$     & $16.73$      &   $99.7\%$&   $2.08\%$   & $0.91$  \\ \hline%\rowcolor[gray]{.95}
$25$ &  $19.47$     & $19.45$      &   $99.89\%$&  $2.25\%$& $0.37$  \\ \hline
$50$ &  $22.87$     & $22.75$      &  $99.7\%$& $2.41\%$        & $0.35$ \\ \hline%\rowcolor[gray]{.95}
$100$ & $26.83$      & $26.65$      &   $99.3\%$&$2.68\%$        & $0.41$ \\ \hline

\end{tabular}
\label{amp-table}
}
\end{table} 
%\begin{figure*}
%    \centering
%    \includegraphics[width = 0.8\textwidth]{latex/img/ca/epoch3a circuit.pdf}
%    \caption{The new circuit we discovered for task-specific fine-tuning at Epoch 3. The emerging, marked in \textcolor{blue}{blue}, circuit components formed performed similar mechanisms as the prior circuit components. }
%    \label{fig:ca3-circuit}
%\end{figure*}
We systematically analyze the circuits discovered at various epochs, assessing their faithfulness, performance, and sparsity. Our results show that the retrieved circuits exhibit high faithfulness and minimality scores, surpassing the original IOI circuit in both aspects. We provide a thorough account of our circuit discovery and evaluation results in the \autoref{app:circ_disc}, and in this section, we delve into the underlying mechanisms driving this performance enhancement.
Concurrently, we observe that task-specific fine-tuning enhances the underlying mechanisms of circuits without introducing novel mechanisms, even in longer training scenarios. The enhancement stems from two sources: (1) amplified capabilities of existing circuit components and (2) emergence of new components that replicate prior mechanisms. %Notably, fine-tuning solely augments the original mechanisms, increasing the number of contributing components and their individual strengths, without adding novel mechanisms. 
We term this phenomenon \textbf{Circuit Amplification}, and refer to the underlying mechanism as \textit{amplification}.
Our results, summarized in \autoref{amp-table}, reveal consistent Circuit Amplification in each epoch, note that in \autoref{amp-table}, $F(C_{M_{GPT2}})$ refers to the average logit difference when the original circuit is run on the fine-tuned model, so $|F(C)-F(C_{M_{GPT2}})|$ refers to the total contributions of the new circuit components to the average logit difference. Furthermore, we investigate the impact of fine-tuning on model components, including Negative Name Mover heads, which counterintuitively exhibit enhanced capabilities despite their negative contribution to the task. Notably, we do not observe the diminishing or disappearance of Negative Name Movers, see \autoref{fig:ca_logit_attribution}; instead, their abilities are enhanced. The IOI task circuit formed after 3 epochs of fine-tuning can be seen in \autoref{fig:ca3-circuit}. 

Intriguingly, we see \textit{Circuit Amplification}, even for \textbf{longer} training epochs. This seemed counter-intuitive as Negative Name Mover heads are amplified even after \textbf{longer periods of training}, hinting at their counter-factual importance to the task. Initial investigation by \cite{mcdougall2023copy} shows that these heads are a type of Copy Suppressor Heads and are key to the behavior of Self-Repair in language models \cite{rushing2024explorations}. These findings resonate with our result, as we see  these heads get amplified over time.\footnote{We further generalize the amplification results to the case of fine-tuning on general datasets, see \autoref{app:gen}.} 

\noindent\textbf{Mechanism of Enhancement: }Given the presence of Circuit Amplification, we now move to one of our key contributions, understanding how circuit amplification takes place. We \textbf{first} denote that, trivially, the increase in the number of components that replicate original mechanisms contributing to the task is one of the main contributors to circuit amplification, see \autoref{amp-table}. However, this doesn't fully explain the effect of circuit amplification, as the added components do not represent the complete change in the accuracy of the novel circuit when compared to the original circuit. \textbf{Secondly}, we record that the prior circuit components undergo an increase in capacity to perform their mechanism. To illustrate this point, we take the case of a Name Mover Head, specifically \textbf{L9H9} (Layer 9 Head 9) which gets amplified. 
\begin{figure}
\includegraphics[scale = 0.1,width=0.48\textwidth]{latex/img/ca/ca3-amp-attn.drawio-1-1.pdf}%
    \caption{\label{fig:ca3-amp-mech} Attention Probability vs Projection of head output along $W_U[IO]$ and $W_U[S]$ for head L9H9}%
\end{figure}
In \autoref{fig:ca3-amp-mech}, we plot the Attention Probability for IO (Indirect Object) and ``to'' token pairs vs Projection of Head output along $W_U[IO]$. This figure also includes the attention probability of S and ``to'' token pairs vs Projection of Head output along $W_U[S]$. We see that attention probabilities have significantly decreased for the S token for L9H9 after fine-tuning, suggesting a discriminant increase in the copying behavior of the IO token for L9H9 which is a finding that generalizes to other heads in the same category.
We further record this behavior in the case of Negative Name Mover Heads\footnote{See \autoref{app:ca} for further details}. %In addition to this, we note an amplification in the OV circuit for the aforementioned heads, i.e., the head writes more strongly to the residual stream. 
This implies that this head writes more strongly to the residual stream as the direct logit attribution\footnote{Logit attribution is mathematically defined in Section 3.1 of \cite{wang2022interpretability}.} of each head increases significantly when compared to the original model. This increase in the underlying capacity of the heads to perform their underlying behavior is \textit{amplification}, see \autoref{fig:ca_logit_attribution}. 
\begin{figure}
\centering
  \begin{subfigure}[t]{0.23\textwidth}
  \centering
    \includegraphics[scale = 0.1,  width=\textwidth]{latex/img/ca/change_in_logit_attribution.pdf}
    \caption{}
    \label{fig:ca_logit_attribution}
\end{subfigure}
\hfill
 \begin{subfigure}[t]{0.24\textwidth}
 \centering
    \includegraphics[scale = 0.1, width=\textwidth]{latex/img/ca/change_in_logit_absolut_amplified.pdf}
    \caption{}
\label{fig:ca_logit_attribution_absolute}
    \end{subfigure}
    \vspace{5mm}
    \caption{\subref{fig:ca_logit_attribution}) Logit Attribution  of heads L9H9, L11H10, L10H10 in  original/amplified model. \subref{fig:ca_logit_attribution_absolute}) Absolute Logit Difference in the original model vs amplified model after ablation}
\end{figure}
Finally, the \textbf{third} mechanism contributing to amplification is a change in the mechanism of some of the Backup Name Mover Heads to that of Name Mover Heads. We take the example of L10H10 and show that this head now performs the behaviors of Name Mover Heads after fine-tuning for 3 epochs, see \autoref{fig:ca3-amp-attn-bmnh} and \autoref{fig:ca_logit_attribution}.
\begin{figure}[t]
    \includegraphics[scale = 0.1,width=0.48\textwidth]{latex/img/appendix/ca3-amp-attn-bnmh.drawio.pdf}
    \caption{\label{fig:ca3-amp-attn-bmnh}:Attention Probability vs Projection of head output along $W_U[IO]$ and $W_U[S]$ for head L10H10 }
\end{figure}
In \autoref{fig:ca3-amp-attn-bmnh}, we see that the attention probability w.r.t the projection along the unembed of the IO and S token is similar to that of the original name mover heads, while seeing a significant increase in logit attribution, from $0.4$ to $1.8$ on the IOI task. We then ablate groups of heads in the original model and the fine-tuned model and measure the absolute change in the logit difference in their respective circuit's performance. As the number of model components performing the task increases, for a fair comparison, we only consider the heads in the original circuit for each group. \autoref{fig:ca_logit_attribution_absolute} shows that the ablating groups of heads in the fine-tuned model show a much higher change in performance indicating  the original groups surged in their capability to do their respective mechanisms. These findings generalize across epochs. \vspace{1mm}\\
\textbf{Analyzing Enhancement via Cross-Model Activation Patching: } We now analyze circuit amplification via Cross-Model Activation Patching \cite{prakash2024fine} and record that in task-specific fine-tuning, the amplification of the mechanism can be detected via Cross-Model Pattern Patching. That is, we patch in attention patterns of each head from the fine-tuned model into the original model and record the changes in the logit difference. We observe that each attention head in the original circuit has increased capability to perform its mechanism, see \autoref{fig:cross-model-ca}. 
\begin{figure}[t]
    \centering
    \includegraphics[scale = 0.1,width = 0.36\textwidth]{latex/img/ca/cross_model_pattern_ca.pdf}
    \caption{Cross Model Pattern Patching: Taking the attention pattern of the heads in the fine-tuned model and patching them into the original model results in an increase in the attention heads performance on the underlying task.} %This analysis extends to the attention heads that are added to the circuit via amplification}
    \label{fig:cross-model-ca}
\end{figure}
\subsection{Corruption of Model Mechanisms } 
\label{circuit-poisoning}
Given the knowledge of circuit amplification, we now aim to fine-tune the model with various corrupted augmentations of the IOI task and utilize path patching \cite{goldowsky2023localizing} and activation patching \cite{NEURIPS2020_92650b2e} to study the effects of corruption on the model mechanisms for the IOI task. Furthermore, we record the changes made to the original model circuit and investigate the mechanisms of corruption across different augmentations. We find that when fine-tuning on \textbf{Name Moving} and \textbf{Subject Duplication} datasets, the corruption can be traced back to changes in the original circuit, however, no noticeable change occurred when fine-tuning on the \textbf{Duplication} dataset, hence we leave the discussions to the \autoref{dadupe}. We discover  most of the mechanistic changes after toxic fine-tuning can be attributed to changes in the mechanisms of the circuit components, i.e, toxic fine-tuning {alters} the prior mechanisms of the circuits instead of introducing {new mechanisms} for suppressing performance on the task.  
\paragraph{Name Moving Dataset.} After fine-tuning, this dataset suppresses the output of the IO token. Notably, after 3 epochs, the output logits of multiple single-token names in the vocabulary converge to similar values, with a slight bias towards the IO token name, thereby preserving the IOI functionality, albeit with significant degradation. To illustrate, we take the prompt "After John and Mary went to the store, John gave milk to" and record the logits of the top 5 most likely tokens, see \autoref{table:logit_table}.
\begin{table}[h]
\centering
\small
\begin{tabular}{|c|c|c|}
\hline
\textbf{Logit} & \textbf{Token} \\ \hline
$21.70$  & Mary \\ \hline
$21.40$  & Elizabeth \\ \hline
$21.34 $  & Melissa \\ \hline
$21.24 $  & Christine \\ \hline
$21.08 $  & Stephanie \\ \hline
\end{tabular}
\caption{Logits of top 5 tokens after 3 epochs}
\label{table:logit_table}
\end{table}
However, this capability completely degrades over time, i.e, the bias towards the "IO" token is non-negligible. To elucidate the underlying mechanisms, we present a detailed analysis of the fine-tuning process with 3 epochs on the corrupted dataset in this section. 
Our investigation reveals that the model does \textbf{not introduce} novel mechanisms to mitigate performance on the task. Instead, it relies on diminishing/altering the capabilities of specific attention heads that underlie a task-related mechanism. Notably, the most affected components are the Name Mover Heads and  which completely lose their ability to copy the IO token ( \autoref{fig:cp3-attn-nmh}). 
\begin{figure}[htbp]
    \includegraphics[scale = 0.1,width=0.5\textwidth]{latex/img/cp/cp3-attn-nmh.drawio-1-1.pdf}
    \caption{\label{fig:cp3-attn-nmh}\textbf{Name Moving: }Attention Probability vs Projection of head output along $W_U[IO]$ and $W_U[S]$ for head L9H9}
  \end{figure}
We trace the source of this corruption to the S-Inhibition heads, which primarily suppress the queries of both the IO and S tokens. Consequently, the original circuit is fundamentally disrupted, with the Name Mover Heads losing their functionality and the S-Inhibition Heads altering their mechanism to suppress both tokens. This is evident in the QK matrix analysis of the S-Inhibition heads, which reveals a significant change in attention patterns, see \autoref{fig:cp3-token}.
\begin{figure}
\centering
  \begin{subfigure}[t]{0.24\textwidth}
  \centering
    \includegraphics[width =0.99\textwidth]{latex/img/cp/change_in_attn_pat-1.pdf}
    \caption{}
    \label{fig:cp3-token}
\end{subfigure}
\hfill
 \begin{subfigure}[t]{0.23\textwidth}
 \centering
    \includegraphics[scale = 0.1,width = \textwidth]{latex/img/cp/change_in_logit_percent_corrupt.pdf}
    \caption{}
    \label{fig:cp3-logit-percent}
    \end{subfigure}
    \vspace{5mm}
    \caption{\subref{fig:cp3-token}) Name Moving: the attention probability difference of S-Inhibition Heads on the \textcolor{green}{IO} and \textcolor{magenta}{S} token [\textit{Original - Corrupted}].
    \subref{fig:cp3-logit-percent}) Subject Duplication: Change in Logit Difference after ablating groups of heads.\\}
\end{figure}
We find that this mechanism of corruption extends to Backup Name Mover Heads and Negative Name Mover Heads see \autoref{app:cp} for further details. This hints that model poisoning, mechanistically, alters very localized model behaviors that affect the final output, instead of adding novel mechanisms to corrupt the model. This can also be seen via CMAP, see \autoref{app:cross}.\\ % as well, see appendix for further details. 
%Now we trace the information flow back from the S-Inhibition Heads to understand the affect of corruption on the prior heads and find that the functionality of the Induction Heads, Previous Token Heads and remain the same, hence we ask the question: \textit{What is affecting the queries of the S-Inhibition Heads?}. To answer this, employ Path Patching on query vector for the S-Inhibitions and find that  Induction Heads, Previous Token Heads, and Duplicate Token Heads don't write a strong enough signal to bias the queries of the S-Inhibition Head and hence, S-Inhibition Head attends strongly to both IO and S tokens, see \autoref{app:cross} for cross-model patching on the corrupted model and original model. 
This corruption mechanism induces phase transitions that disrupt the IOI task, as previously examined. In early epochs, the IOI capability remains but with significant degradation (see \autoref{table:logit_table}), resulting in correct outputs despite corrupted internal mechanisms. We hypothesize that leveraging the knowledge of pre-existing mechanisms could enable model poisoning attacks, selectively altering mechanisms while changing the distribution of the output significantly but compromising interpretability or introducing backdoor triggers. Future work exploring more defined attacks through fine-tuning would be an interesting direction.

\paragraph{Subject Duplication Dataset.}  Applying this data augmentation strategy and fine-tuning using the corrupted dataset results in rapid and significant degradation of model performance, the average logit difference goes from $3.55$ to $-11.06$ after just $5$ epochs.
Analysis reveals that the Name Mover Heads are most affected, exhibiting a modified attention pattern. This altered attention pattern yields a suppressed logit for the IO token and an enhanced logit for the S token, see \autoref{fig:ca5-sd-attn-nmh} for changes in attention probability for both IO and S token. From \autoref{fig:ca5-sd-attn-nmh} we can see that the projection of L9H9 in the unembedding space has significantly changed, now positively projecting the S token and negatively projecting the IO token. Surprisingly, the Negative Name Mover Heads undergo a similar change in functionality; they write in the opposite direction to the Name Mover Heads, which seems counter-intuitive as these components were suppressing the logit of the IO token, however after fine-tuning on the corrupted data imputation, these heads now suppress the logit of the S-token, see \autoref{fig:ca5-sd-attn-nmh} and \autoref{fig:cp3-attn-nnmh}. 
\begin{figure}
    \centering
    \includegraphics[width = 0.48\textwidth]{latex/img/cp/ca5-attn-sd-nmh.drawio-1-1.pdf}
    \caption{Attention Probability vs Projection of head output along $W_U[IO]$ and $W_U[S]$ for head L9H9}
    \label{fig:ca5-sd-attn-nmh}
\end{figure}
\begin{figure}
    \centering
    \includegraphics[width=0.99\linewidth]{latex/img/appendix/ca5-attn-sd-nnmh.drawio.pdf}
    \caption{Attention Probability vs Projection of head output along $W_U[IO]$ and $W_U[S]$ for head L11H10}
    \label{fig:cp3-attn-nnmh}
\end{figure}
 Finally, we find that the mechanism of the S-Inhibition heads is mostly suppressed, even though they still bias the query of the Name Mover Heads and Negative Name Mover Heads, the impact of the bias is statistically insignificant when compared to the original circuit as after mean ablation their effect is insignificant in the corrupted model, see \autoref{fig:cp3-logit-percent}. Similar to the previous observation, the mechanism of corruption is very \textbf{local} to certain model components, however, unlike the prior case (Corrupted Dataset for Name Moving Behaviour), only the mechanism of the Name Mover Heads Negative Name Mover Heads is changed, while the mechanism of the S-Inhibition Heads (and other heads) is suppressed, see \autoref{fig:cp3-logit-percent} for their importance to the task in the corrupted model which we access via mean ablating groups of heads that are present in the circuit. \\
 In contrast to the \textbf{Name Moving} data augmentation, the phase transition in this case reveals an intriguing insight: Negative Name Mover Heads shift from suppressing the 'IO' token to suppressing the 'S' token, despite already being optimized for the task. This suggests that Name Mover Heads and Negative Name Mover Heads are intertwined, with one performing the inverse of the other for certain tasks. Further investigation into this "twinning" behavior and its occurrence in other tasks would be a promising direction for future research. 
 \paragraph{Analyses via Cross-Model Activation Patching:} Similar to prior experiments, we employ cross-model activation patching and replace attention patterns of each head with their patterns in the corrupted model fine-tuned on the Subject Duplication dataset. We observe that the effects of corruption are localized to the circuit, see \autoref{fig:cross-model-cp}, as the heads most affected in \autoref{fig:cross-model-cp} are the circuit components outlined in \autoref{fig:ca3-circuit}.
 \begin{figure}
     \centering
     \includegraphics[width = 0.36\textwidth]{latex/img/cp/cross_model_pattern_cp.pdf}
     \caption{Cross Model Pattern Patching: We find that effect of corruption is very localized to circuit components of the model, however few additional components arise, this is due to formation of repeated mechanisms via fine-tuning, see \autoref{app:cross} for further details}
     \label{fig:cross-model-cp}
 \end{figure}
 \begin{figure*}[h]
     \centering
     \includegraphics[scale =0.1,width = 0.8\textwidth]{latex/img/np/neuro-name-moving.drawio-2-1.pdf}
     \caption{Attention Probability vs Projection of head output along $W_U[IO]$ and $W_U[S]$ for head L9H9, corruption on \textbf{Name Moving} augmentation.}
     \label{fig:neuro-nmh}
 \end{figure*}

\section{Neuroplasticity in Model Mechanisms}
\label{main:neuro}
After corruption, we study relearning the IOI task via fine-tuning on the original dataset. We discover that the corrupted model can recover its performance and analyze the changes in mechanisms between the retrieved and original models. Focusing on the two data imputations, % (excluding Duplication Data Augmentation), 
we fine-tune the corrupted model using the original data and refer to the resulting model as the \textit{post-reversal} model.\vspace{1mm}\\
\textbf{Name Moving Dataset:} The \textit{post-reversal} model recovers its original performance and \textbf{recovers} the original circuit mechanisms. Moreover, the IOI task circuit mechanism is amplified compared to the original model. We trace the mechanism change from the corrupted to the \textit{post-reversal} model and find that the emergence of the prior mechanisms occurs, resulting in a circuit similar to the original model's \footnote{see \autoref{app:neuro} for the new circuit diagram and discussion on other heads}.
Taking the case of the Name Mover Head \textbf{L9H9}, we see the recovery (and amplification) of the original mechanism of the head in the \textit{post-reversal} model, see \autoref{fig:neuro-nmh}. Our analyses extend to the case of \textbf{Subject Duplication Dataset} and other heads, see \autoref{app:neuro} for details. This suggests that one possible defense against data poisoning attacks can be fine-tuning on the clean dataset. %Further experiment is needed to confirm this observation on other tasks.

\section{General Choice Models}
To generalize beyond uniform choice models, we consider two strategies to tackle general choice models.
The first considers sampling-based methods, which sample from different permutations to assess the quality of a menu. 
The second considers restrictions on the choice class, so that the utility for a particular menu can be expressed as a submodular function. 
Doing so allows us to optimize efficiently due to properties of submodular functions. 

\subsection{Sampling-based Methods}
We first consider algorithms which sample different permutations of $\pi$, and use this to find optimal menus. 
We first use concentration inequalities to show that we can efficiently approximate the utiltiy of a particular menu: 
\begin{lemma}
     Let $R(X_{1,1},X_{1,2},\cdots,X_{N,P}) = \mathbb{E}_{\pi}[\sum_{i=1}^{N} \sum_{j=1}^{P} \theta_{\pi_{i},j} Y_{\pi_{i},j}(\mathbf{Z}_{i})]$ be a utility function of the menus offered. Let $\mathrm{OPT} = \max\limits_{X_{1,1},X_{1,2},\ldots,X_{N,P}} \mathbb{E}_{\pi}[\sum_{i=1}^{N} \sum_{j=1}^{P} \theta_{\pi_{i},j} Y_{\pi_{i},j}(\mathbf{Z}_{i})]$. If $\theta_{i,j} \in [0,1] \forall i,j$, then $\mathrm{OPT} \leq \min(N,P)$
\end{lemma}

\begin{lemma}
    Let $R(X_{1,1},X_{1,2},\cdots,X_{N,P}) = \mathbb{E}_{\pi}[\sum_{i=1}^{N} \sum_{j=1}^{P} \theta_{\pi_{i},j} Y_{\pi_{i},j}(\mathbf{Z}_{i})]$ be a utility function of the menus offered.
    Consider a fixed menu $X_{1,1},X_{1,2},\cdots,X_{N,P}$. 
    Then after 
    \begin{equation}
        n = \frac{\ln(\frac{2}{\delta}) \min(N,P)^{2}}{\epsilon}
    \end{equation}
    samples, denoted $\pi^{1}, \pi^{2}, \ldots, \pi^{n}$, we have that, with probability $1-\delta$
    \begin{equation}
        |R(X_{1,1},X_{1,2},\cdots,X_{N,P}) - \frac{1}{n} \sum_{k=1}^{n} \sum_{i=1}^{N} \sum_{j=1}^{P} \theta_{\pi^{k}_{i},j} Y_{\pi^{k}_{i},j}(\mathbf{Z}_{i})| \leq \epsilon
    \end{equation}
\end{lemma}

From here, if our utility function is Lipschitz in the menus offered, then we can use a covering-style argument to find the optimal menu. 
We do this in two steps. 
First, we construct a set of menus which approximately cover the set of available menus, within some distance $r$. 
We use sampling-based techniques to approximate the utility for each of these menus. 
Second, we claim that the optimal menu is close to one of these menus, and because the utility is Lipschitz in the menu, we can bound the perforamnce gap between our best found menu and the optimal menu: 
\begin{theorem}
    Consider a utility function, so that $|R(X_{1,1},X_{1,2},\cdots,X_{N,P}) - R(X'_{1,1},X'_{1,2},\cdots,X'_{N,P})| \leq L \sum_{i=1}^{N} \sum_{j=1}^{P} |X_{i,j}-X'_{i,j}|$ for some $L \geq 0$. 
    Let $K(n,r)$ be the covering code for a binary string of length $n$ with hamming distance $r$; the covering code represents the minimum cardinality of a set $S$, so that all binary strings of length $n$ are at most distance $r$ away from some element of $S$. 
    Let $\mathrm{OPT} = \max\limits_{X_{1,1},X_{1,2},\ldots,X_{N,P}} \mathbb{E}_{\pi}[\sum_{i=1}^{N} \sum_{j=1}^{P} \theta_{\pi_{i},j} Y_{\pi_{i},j}(\mathbf{Z}_{i})]$
    Then with $\frac{K(NP,r) \ln(\frac{2}{\delta}) \min{(N,P)}^{2}}{\epsilon}$ evaluations of 
    \begin{equation}
        \sum_{i=1}^{N} \sum_{j=1}^{P} \theta_{\pi_{i},j} Y_{\pi_{i},j}(\mathbf{Z}_{i})
    \end{equation}
    using different combinations of $\pi$ and $X_{i,j}$, we can find a solution $\mathrm{ALG} = R(X_{1,1},X_{1,2},\cdots,X_{N,P})$, such that, with probability $1-\delta$, $\mathrm{ALG} \geq \mathrm{OPT} - rL$. 
\end{theorem}

\subsection{Submodular Utility}
When the utility function can be written as a submodular function, we can leverage techniques from submodular optimization to solve our problem. 
Moreover, we know that monotone submodular optimization can be approximated within $1-\frac{1}{e}$ in polynomial time~\cite{submodular_optimization} with cardinality constraints (and optimally without such constraints), while non-monotone optimization can be approximated within $\frac{2}{5}$~\cite{non_monotone_submodular}. 
We use this to bound the performance of polynomial time algorithms in this scenario: 
\begin{lemma}
    Let $R(X_{1,1},X_{1,2},\cdots,X_{N,P}) = \mathbb{E}_{\pi}[\sum_{i=1}^{N} \sum_{j=1}^{P} \theta_{\pi_{i},j} Y_{\pi_{i},j}(\mathbf{Z}_{i})]$ be a utility function of the menus offered.
    If $R$ is submodular in the variables $X_{i,j}$, and if we can evaluate $R(X_{1,1},X_{1,2},\cdots,X_{N,P})$ in polynomial time, then then we can find a greedy solution, $\mathrm{ALG}$, in polynomial time, such that $\mathrm{ALG} \geq \frac{2}{5} \mathrm{OPT}$, where $\mathrm{OPT} = \max\limits_{X_{1,1},X_{1,2},\ldots,X_{N,P}} \mathbb{E}_{\pi}[\sum_{i=1}^{N} \sum_{j=1}^{P} \theta_{\pi_{i},j} Y_{\pi_{i},j}(\mathbf{Z}_{i})]$
\end{lemma}

We note that such a solution is not a panacea, as we construct simple choice models which aren't submodular: 
\begin{lemma}
    There exists a selection of $Y_{i,j}(\mathbf{Z})$ so that $R(X_{1,1},X_{1,2},\cdots,X_{N,P}) = \mathbb{E}_{\pi}[\sum_{i=1}^{N} \sum_{j=1}^{P} \theta_{\pi_{i},j} Y_{\pi_{i},j}(\mathbf{Z}_{i})]$ is not submodular. 
\end{lemma}

Determining whether commonly used choice models result in submodular utility functions is still an open question:

\begin{conjecture}
    When $Y_{i,j}$ is the uniform choice model, $R(X_{1,1},X_{1,2},\cdots,X_{N,P})$ is submodular. 
\end{conjecture}

\begin{conjecture}
    When $Y_{i,j}$ is the MNL choice model, $R(X_{1,1},X_{1,2},\cdots,X_{N,P})$ is submodular. 
\end{conjecture}



\section{Related Work} \label{sec:related}

% \textbf{Adversarial Attack}
\textbf{Attacks on SLAM.} 
%With the rise of machine learning, 
The robustness of computer vision systems is being actively investigated. With the emergence of adversarial images in the digital domain by adding optimized noise directly to images~\cite{szegedy2013intriguing,carlini2017towards}, researchers find that such attacks also exist physically in the real world \cite{eykholt2018robust,song2018physical,zhao2019seeing}. To fill the gap between attacks in the digital and physical worlds, recent studies have demonstrated that attacks on real-world computer vision systems are practical \cite{eykholt2018robust,li2019adversarial,man2020ghostimage,sharif2016accessorize,zhao2019seeing,zhou2018invisible}. However, attacks on traditional computer vision methods such as SLAM are relatively less explored. \cite{yoshida2022adversarial} proposes an attack against the scan matching algorithm in LiDAR-based SLAM, while most SLAMs in AR/VR devices rely on different sensors like RGB/depth cameras and IMUs. \cite{ikram2022perceptual} and \cite{chen2024adversary} mislead visual SLAM by poisoning the images with special patterns, and \cite{wang2021can} causes the camera to fail using infrared light. In our work, we demonstrate attacks on Visual-Inertial SLAM (VI-SLAM) by perturbing the IMU readings, rather than cameras, and showing its impact on XR user experience. 

\textbf{Acoustic Injection Attacks.} Among various physical attacks, acoustic injection attacks are attractive due to their low cost. Son~\etal~\cite{son2015rocking} were the first to introduce acoustic attacks on MEMS gyroscopes, demonstrating how these attacks could lead to sensor denial-of-service and result in drone crashes. WALNUT~\cite{trippel2017walnut} expanded on this by developing output biasing and control attacks that enable precise manipulation of MEMS accelerometer outputs using modulated sound waves. Wang et al.~\cite{wang2017sonic} demonstrated a sonic gun, showcasing the vulnerability of various smart devices (\eg drones and self-balancing vehicles) to acoustic attacks. Tu et al. \cite{tu2018injected} designed side-swing and switching attacks to alter the outputs of MEMS gyroscopes and accelerometers. Furthermore, Ji et al. \cite{ji2021poltergeist} fool the object detectors by applying acoustic attack to the image stabilizers commonly used in modern cameras. However, none of the existing works study the relationship between the acoustic injections and SLAM outputs on recent XR devices. 

% \zijian{Do we need one session about security in AR/VR?}
% \yicheng{TODO}
%\jiasi{cite the AIVR paper (UMass Amherst?) paper is we have not already. They add IMU perturbation but w/o SLAM, iirc} \yicheng{Cited}

\textbf{XR Security and Privacy.} 
%Security and privacy concerns in XR systems have gained significant attention. 
For single-user XR systems, researchers have demonstrated various side-channel attacks to extract sensitive information (\eg keystrokes) through video feeds~\cite{ling2019know}, head movements~\cite{nair2023unique, slocum2023going}, architectural hints~\cite{zhang2023its,shang2020arspy}, power usage~\cite{li2024dangers}, and EM side-channel leakages~\cite{al2021vr}. In multi-user XR systems, Su et al.~\cite{su2024remote} use avatar motion data to infer keystrokes in shared VR environments. Slocum et al.~\cite{slocum2024doesn} reveal vulnerabilities in the shared state frameworks of multi-user AR. Similarly, Lebeck et al.~\cite{lebeck2017securing} highlight risks like deceptive virtual objects and emphasize access control for managing shared physical and virtual spaces. Ruth et al.~\cite{ruth2019secure} further propose a secure multi-user AR framework focusing on content sharing and permissions.
Chandio et al.~\cite{chandio2024stealthy} %introduced a multi-modal spatiotemporal attack that 
simultaneously manipulated visual and inertial sensors to disrupt XR pose estimation. However, their study evaluated the attack using offline datasets and assumed the attacker's capability to manipulate IMU data streams through acoustic means, without real experiments. Ours is the first to demonstrate acoustic injection attacks on recent XR devices, like the Hololens 2, in the real world.
 



This work identifies signal collapse as a critical bottleneck in one-shot neural network pruning. Performance loss in pruned networks is due to \textbf{signal collapse} in addition to the removal of critical parameters. We propose \textbf{REFLOW} (\textbf{Re}storing \textbf{F}low of \textbf{Low}-variance signals), a simple yet effective method that mitigates signal collapse without computationally expensive weight updates. By focusing on signal preservation, REFLOW highlights the importance of mitigating signal collapse in sparse networks and enables magnitude pruning to match or surpass state-of-the-art one-shot pruning methods such as CHITA, CBS, and WF.

REFLOW consistently achieves state-of-the-art accuracy across diverse architectures, restoring ResNeXt-101 from under 4.1\% to 78.9\% top-1 accuracy at 80\% sparsity on ImageNet. Its lightweight design makes it a practical solution for both research and deployment, delivering high-quality sparse models without the overhead of traditional approaches. These findings challenge the traditional emphasis on weight selection strategies and underscore the critical role of signal propagation for achieving high-quality sparse networks in the context of one-shot pruning.




\section{Acknowledgment}
This work is supported by the U.S. National Science Foundation under award
IIS-2301599 and CMMI-2301601, by grants from the Ohio State University’s Translational Data
Analytics Institute and College of Engineering Strategic Research Initiative.
\bibliography{latex/acl_latex}

\appendix
\section{Dataset Size}
\label{app:data}
\subsection{IOI dataset}
As we mentioned before, indirect object identification(IOI) is a task related to identifying the indirect object. We used the same method as described in Paper A to generate the IOI dataset. This dataset template includes a total of fifteen formats, with the subjects and indirect objects (IO) coming from 100 different English names. Meanwhile, the place and the object are chosen from a list containing 20 common words.%Todo: samples 

We generate 6360 samples from the template in the IOI dataset $p_{IOI}$. We chose this dataset size for our IOI dataset for several reasons. Firstly, this size allows us to observe changes in each head. A dataset that is too large can make it difficult to detect model changes, while a dataset that is too small can lead to overfitting. Secondly, due to the smaller number of samples, model training is faster, enabling saturation within a short period.

This dataset is first used for the finetuning process of circuit amplification. Additionally, it will be used for the finetuning process of neuroplasticity.
\subsection{Poisoning datasets}
For data poisoning, we also randomly generated three different datasets: the Duplication Dataset, the Name Moving Dataset, and the Subject Duplication Task Dataset. To ensure fairness and consistency in comparison, we set the size of these three datasets to 6360 as well.

\begin{itemize}
    \item \textbf{Duplication dataset} is using a random single token to replace the second subject token. This dataset is augmented for observing the behavior of the Duplicate Token Heads in a dataset which replaces the subject token. An example in the Duplication dataset is that \textit{"When Mark and Rebecca went to the garden, Mark gave flowers to Rebecca"} is augmented to \textit{"When Mark and Rebecca went to the garden, Tim gave flowers to Rebecca"}. 
    
    \item \textbf{Name Moving dataset} is using a random single token to replace the final token which is the second token of IO. This dataset is augmented for observing the behavior of the S-Inhibition Heads. An example in Name Moving dataset is that \textit{"When Mark and Rebecca went to the garden, Mark gave flowers to Rebecca"} is augmented to \textit{"When Mark and Rebecca went to the garden, Mark gave flowers to Stephanie"}.
    
    \item \textbf{Subject Duplication dataset} is using the subject token S to replace the output IO token. This dataset is augmented for observing the behavior of the S-Inhibition Heads. An example in the Subject Duplication dataset is that \textit{"When Mark and Rebecca went to the garden, Mark gave flowers to Rebecca"} is augmented to \textit{"When Mark and Rebecca went to the garden, Mark gave flowers to Mark"}.
    
\end{itemize}



\section{Finetuning Experiments}
\label{app:fine}
In this section, we primarily report the hyper-parameter settings used during the model training process. To synchronize and compare the results of our experiments, we used the same learning rate and weight decay across circuit amplification, circuit poisoning, and neuroplasticity. The learning rate is 1e-5, and weight decay is 0.1, with batch-size = 10. We use the base Adam Optimizer from HuggingFace for finetuning. 

\textbf{Compute: } We utilize, Google Colab Pro+ A100 GPUs for fine-tuning experiments and V100 GPU for inference. \\
\textbf{Computational Budget: } We utilize 11 GPU hours for fine-tuning experiments and 50 GPU hours for inference experiments in total. \\
\textbf{Model Parameters: } GPT2-small \cite{radford2019language} has 80M parameters with 12 layers. 

\section{Path Patching and Knockout}
\label{app_path}
\textbf{Path patching  } is a method to search the attention head which directly affect the model's logits \cite{goldowsky2023localizing}. This method is designed to differentiate indirect effect from direct effect. Path patching is a technique used to replace part of a model's forward pass with activations from a different input. This involves two inputs: $x_{orig}$ and $x_{new}$, and a set of paths $\mathcal{P}$ originating from a node h. The process begin by running a forward pass on $x_{orig}$. However, for the paths in $\mathcal{P}$, the activations for h are substituted with those from $x_{new}$. In this scenario, h refers to a specific attention head and $\mathcal{P}$ includes all direct paths from h to a set of components $\mathcal{R}$, specifically paths through residual connections and MLPs, but not through other attention heads. 


\textbf{Knockout} is a method which is designed for understanding the correspondence between the components of a model and human-understandable concepts \cite{wang2022interpretability}. This concept is based on the \textit{circuits} which views the model as a computation graph $M$. In the graph $M$, nodes are terms in its forward pass (neurons, attention heads, embeddings, etc.) and edges are the interactions between those terms (residual connections, attention, projections, etc.). The circuit $C$ is a subgraph of $M$ responsible for some behavior. For example, to implement the model's functionality as completely as possible. \textit{Knockout  } is designed to measure a sets of nodes whether it is deletable in the $M$. A knockout operation would remove a set of nodes $K$ in a computation graph $M$ with the goal of "turning off" nodes in $K$ but capturing all other computations in $M$.

Specifically, a knockout operation includes the following parts: the knockout will 'delete' each node in $K$ from $M$. The removal operation involves replacing the outputs of the corresponding nodes with their average activation value across some reference distribution. Using mean-ablations removes the information that varies in the reference distribution (e.g. the value of the name outputted by a head) but will preserve constant information(e.g. the fact that a head is outputting a name).

\section{Self-Repair in Neuroplasticity and Circuit Amplification}
\label{app:self_repair}
In addition to circuit amplification, we provide some initial investigations on self-repair in the models \textit{post-reversal} and after regular fine-tuning on the IOI dataset. In particular, we study the impact of finetuning and reversal on the self-repair of \textit{Copy Suppressor Heads}, i.e, Name Mover Heads/\vspace{1mm}\\  
\textbf{Metric for Measuring Self-Repair} 
We follow the work by \cite{rushing2024explorations} and quantify self-repair of an attention head in a  model as: 
\begin{align*}
    \Delta logit \approx  -DE_{head} + \textit{self repair}
\end{align*}
 , where, in the case of the IOI task, $\Delta logit$ refers to the change in logit difference between the IO token and the S pre-ablation and post-ablation of the attention head under scrutiny,  $DE_{head}$ refers to the direct effect of the attention head on the models performance. \vspace{1mm}\\
\textbf{Boomerang of Self-Repair}
We take the case of the attention head: \textbf{9.9} and report the effects of finetuning on the self-repair behavior for the head under scrutiny.\\
\begin{figure}
    \centering
    \includegraphics[width = 0.5\textwidth]{latex/img/np/self-pair-1.pdf}
    \caption{Self-Repair Enhancement over Time for L9H9}
    \label{fig:enter-label}
\end{figure}
We find that capacity of self-repair increases linearly with time until we see a phase shift in the self-repair behavior on the dataset. From this, we conclude that the capability of the model Self-Repair is also enhanced with fine-tuning, we hypothesize this is due to dropout and circuit amplification increasing the number of backup name mover heads over time, however, further investigations are required and would be interesting future work.

\section{Generalized Fine-Tuning}
We fine-tune the model on the following datasets and report our findings: 
\begin{compactitem}
    \item \textbf{Dataset 1}: using Approximately 213,000 samples from TinyStories \cite{eldan2023tinystories} and our full IOI dataset, We fine-tune for 1 Epoch using the same hyper-parameters as mentioned in \autoref{app:fine}
    \item \textbf{Dataset 2}: using open-sourced model called GPT2-dolly which is instruction tuned on Dolly Dataset \cite{DatabricksBlog2023DollyV2}.
    \item \textbf{Dataset 3}: using open-sourced math\_gpt2, fine-tuned on Arxiv Math dataset .
    \item \textbf{Dataset 4}: using open-sourced GPT2-WikiText\cite{alon2022neuro} fine-tuned on WikiText dataset\cite{merity2016pointer}.
\end{compactitem}
\label{app:gen}
\begin{table}[H]
\begin{adjustbox}{width = \columnwidth,center}
\begin{tabular}{l|l|l|l|l|l}
\toprule
Model & $F(Y)$  & $F(C)$  & Faithfulness & Sparsity \\
\midrule
\rowcolor[gray]{.95}
$GPT2-Tiny/IOI$ &   $13.51$    & $13.19$      &  $97.6\%$&   $1.92\%$  \\ 
$GPT2-dolly$ & $5.39$ & $5.28$ & $98\%$ & $1.95\%$\\\rowcolor[gray]{.95}
$math\_gpt2$ &  $4.5$     & $4.36$      & $96.8\%$& $1.95\%$  \\
$GPT2-WikiText$ &  $3.46$     & $3.46$      &   $100\%$&   $1.92\%$    \\\rowcolor[gray]{.95}
 \bottomrule
\end{tabular}
\end{adjustbox}
\caption{The accuracy of the model, the circuit, faithfulness, and sparsity of the circuit discovered on various datasets/methods of fine-tuning.}
\end{table}

\section{Circuit Evaluation}

\label{app:circ_eval}
\textbf{Minimality}: Minimality criterion checks if the circuit contains unnecessary components. More formally, for a circuit $C$, $\forall v \in C \: \exists\: K \subseteq C \backslash \{v\}$ we expect to have a large minimality score defined as follows,  $|F(C\backslash(K \cup \{v\} )) - F(C\backslash K)|$  \cite{wang2022interpretability, prakash2024fine}.

\textbf{Completeness}: Completeness criterion checks if the circuit contains all necessary components. More formally, for a circuit $C$ and the whole model $M$, $\forall K \subseteq C$, incompleteness score$|F(C\backslash K) - F(M\backslash K)|$\cite{wang2022interpretability} should be small. We set K to be an entire class of circuit heads. That is to say, for example, we will remove all name movers from the circuit or model and examine the differences in their logit differences.

\begin{figure*}
    \centering
    \includegraphics[scale =0.1,width = 0.8\textwidth]{latex/img/ioicirc.pdf}
    \caption{The Indirect Object Identification Circuit Discovered by \cite{wang2022interpretability} for GPT-2-Small}
    \label{fig:ioicirc}
\end{figure*}

\section{Circuit Discovery}
\label{app:circ_disc}

We follow the work by \cite{wang2022interpretability} and conduction patching and knockout experiments to recover circuits at each model training iteration and present our circuit discovery for the case of fine-tuning with 3 epochs as a template. 
\begin{figure}
    \includegraphics[width=0.45\textwidth]{latex/img/appendix/direct.pdf}
    \caption{ Isolating Heads with highest direct logit contribution to the task: Name Mover Heads and Negative Name Mover Heads}
    \label{fig:app-logit-attr}
\end{figure}
\begin{figure}
    \includegraphics[width=0.45\textwidth]{latex/img/appendix/s-inhibiton.pdf}
    \caption{ Isolate important heads that most impact the queries of Name Mover Heads: S-Inhibition Head }
    \label{fig:app-sin}
\end{figure}
We initially, analyze the attention patterns of the heads that have the highest logit attribution to the task, see \autoref{fig:app-logit-attr}. We find these to be the Name Mover Heads and Negative Name Mover Heads similar to \cite{wang2022interpretability}. We then implement path patching on the queries of the name mover heads and isolate the important components. After Knockout Experiments, analyzing QK matrix, we identify these heads to be the S-Inhibition Heads see \autoref{fig:app-sin}. Given this we proceed similar to \cite{wang2022interpretability} to find the Induction Heads, Previous Token Heads and Duplicate Token Heads. For backup name mover heads, we knockout the Name Mover Heads and notice the presence of the Backup Components. For example, if ablate 9.9, the following heads will backup the behavior: 
\begin{figure}
  \begin{minipage}{0.4\textwidth}
    
    \includegraphics[width=\textwidth]{latex/img/appendix/backup.pdf}
    \caption{\label{fig:app-logit-attr2} Discovering Backup Name Mover Heads}
  \end{minipage}%
  \hfill 
  \begin{minipage}{0.4\textwidth}
    \includegraphics[width=\textwidth]{latex/img/appendix/minimality.pdf}
    \caption{\label{fig:app-sin2} Minimality Scores for the circuit in \autoref{fig:ca3-circuit} }
  \end{minipage}%
\end{figure}
We also report the completeness scores for the discovered circuit , see \autoref{fig:ca3-comp}
\begin{figure}
    \centering
    \includegraphics[width = 0.48\textwidth]{latex/img/appendix/completeness_finetuned3epoch.pdf}
    \caption{Completeness scores for the circuit in \autoref{fig:ca3-circuit}}
    \label{fig:ca3-comp}
\end{figure}


\section{Circuit Amplification}
\label{app:ca}
Here we report, the amplification of Negative Name Mover Heads and Backup Name Mover Heads. 
\begin{figure}


    \includegraphics[width=0.48\textwidth]{latex/img/appendix/ca3-amp-attn-nnmh.drawio-1-1.pdf}
    \caption{\label{fig:ca3-amp-attn-nnmh}: Attention Probability vs Projection of head output along $W_U[IO]$ and $W_U[S]$ for head L11H10}
  
\end{figure}

\section{Circuit Poisoning}
\label{app:cp}
\textbf{Name Moving Behavior: } We now report the degradation of the mechanism of the Negative Name Mover Heads on this task and change in the mechanism of the S-Inhibition heads. 
\begin{figure}[H]
 \begin{minipage}{0.45\textwidth}
    \includegraphics[width=\textwidth]{latex/img/appendix/cp3-attn-nnmh.drawio.pdf}
    \caption{\label{fig:cp3-amp-attn-bmnh}Attention Probability vs Projection of head output along $W_U[IO]$ and $W_U[S]$ for head L11H10 }
  \end{minipage}
  \hfill % Add horizontal space between figures
  \begin{minipage}{0.45\textwidth}
    \includegraphics[width=\textwidth]{latex/img/appendix/cp3-attn-sin.drawio-1-1.pdf}
    \caption{\label{fig:cp3-amp-attn-bmnh}Attention Probability vs Projection of head output along $W_U[IO]$ and $W_U[S]$ for head L8H10 }
  \end{minipage}
\end{figure}



\section{Neuroplasticity}
\label{app:neuro}
\textbf{Data Augmentation: Name Moving:} We present the circuit for the relearned mechanisms, in the \textit{post-reversal} model, see \autoref{fig:neurocircnm}.
\begin{figure*}
    \centering
    \includegraphics[width = \textwidth]{latex/img/appendix/neurioinamecirc.pdf}
    \caption{The circuit discovered \textit{post-reversal} after corruption on Name Moving Augmentation, the new components are marked in \textcolor{blue}{blue}.}
    \label{fig:neurocircnm}
\end{figure*}
The faithfulness score of this model is  $95\%$.The minimality scores as follows:\\
\begin{figure}
    \centering
    \includegraphics[width = 0.48\textwidth]{latex/img/appendix/neuronamemin.pdf}
    \caption{Minimality Scores of the circuit discovered as shown in \autoref{fig:neurocircnm}}
    \label{fig:neuronm-comp}
\end{figure}

\begin{figure}
    \centering
    \includegraphics[width = 0.48\textwidth]{latex/img/appendix/completeness_corrIOIname3reversed.pdf}
    \caption{Completeness scores of the circuit discovered in \autoref{fig:neurocircnm}}
    \label{fig:comp-np-nm}
\end{figure}
\textbf{Data Augmentation: Subject Duplication}: We present the circuit for the relearned mechanisms in the \textit{post-reversal} model after corruption on Subject Duplication Task, see \autoref{fig:neurocircsd}.\\
\begin{figure*}
    \centering
    \includegraphics[width = \textwidth]{latex/img/appendix/neuronoiocirc.pdf}
    \caption{The circuit discovered \textit{post-reversal} after corruption on Subject Duplication Augmentation, the new components are marked in \textcolor{blue}{blue}.}
    \label{fig:neurocircsd}
\end{figure*}
The faithfulness score of this model is  $96\%$ with identical minimality scores as \textit{post-reversal} with Name Moving Behavior, for the completeness scores see \autoref{fig:comp-np-sd}. 
\begin{figure}
    \centering
    \includegraphics[width = 0.45\textwidth]{latex/img/appendix/completeness_corrIOInoio5reversed.pdf}
    \caption{Completeness scores of the circuit discovered in \autoref{fig:neurocircsd}}
    \label{fig:comp-np-sd}
\end{figure}


\section{Discovering Localized Corruption with Cross-Model Activation Patching}
\label{app:cross}
\textbf{Data Corruption: Subject Duplication}: In addition to the Cross Model Pattern Patching we also employ Cross Model Output Patching, i.e, replacing the attention outputs of each attention head in the original model with that of the fine-tuned on corrupted data variant. We record that the prior analysis of localized corruption can also be examined via Cross-Model Output Patching, see \autoref{fig:cmap-sd-out}
\begin{figure}
    \centering
    \includegraphics[width = 0.48\textwidth]{latex/img/appendix/cmap-sd-out.pdf}
    \caption{Change in Logit Difference after Cross Model Output Patching on the Original Model}
    \label{fig:cmap-sd-out}
\end{figure}
\autoref{fig:cmap-sd-out} illustrates that majority of the corruption is localized to the original circuit components, however similar to our prior analyses novel components arise with perform repeated corrupted mechanism and hence we see their contribution to the task. An interesting case here is that of \textbf{L8H11} which is a new former Name Mover Head, i.e, moving the ''S'' token to the residual stream at the END position. In \autoref{fig:cross-model-cp} we saw that the attention pattern of L8H11 when patched results in decrease in overall capability of the model, however in \autoref{fig:cmap-sd-out} shows an increase in capability, this is a non-surprising result as the OV Matrix of each attention head determines what is written to the residual stream whereas the QK matrix determines the attention pattern, here, we see that the QK Matrix of L8H11 decreases performance after CMAP however OV Matrix doesn't, this is due to the linearly independent nature of the two operations, which only in conjunction, determine the contribution of the head. As the QK Matrix is negatively contributing after CMAP and OV Matrix is positively contributing, this means that overall contribution is negative as the head copies ''S'' token to the residual stream of the END token.\\
\textbf{Data Corruption: Name Moving}: In addition to the localized corruption in subject duplication task, we identify localized corruption in the model variant fine-tuned on the Name Moving data corruption. Firstly, similar to our prior analyses we employ Cross Model Pattern Patching, see \autoref{fig:cmap-nm-pat}.
\begin{figure}
    \centering
    \includegraphics[width = 0.48\textwidth]{latex/img/appendix/cmap-nm-pat.pdf}
    \caption{Change in Logit Difference after Cross Model Output Patching on the Original Model}
    \label{fig:cmap-nm-pat}
\end{figure}
Hence see that the corruption, in this case, is localized to the circuit components, we further validate our findings via Cross-Model Output Patching, see \autoref{fig:cmap-nm-out}.
\begin{figure}
    \centering
    \includegraphics[width = 0.48\textwidth]{latex/img/appendix/cmap-nm-out.pdf}
    \caption{Change in Logit Difference after Cross Model Output Patching on the Original Model}
    \label{fig:cmap-nm-out}
\end{figure}

\section{Effect of MLP Across Epochs}
\label{app:mlp}
In the original work, \cite{wang2022interpretability}, MLP layers of GPT2-small do not individually contribute much to the task, except MLP layer 0, which is seen as an extended embedding \cite{wang2022interpretability}. We find this case to extend to the circuits we recover via fine-tuning on the original IOI dataset, furthermore, we do not record any major contribution of the MLP layers (except MLP layer 0) in the corruption of the IOI task after fine-tuning on corrupted data variants. \\
\textbf{Amplification}: Similar to the original model, we record that the MLP layers, except layer 0, have no statistically significant contribution to the IOI task even after undergoing task-specific fine-tuning on the clean dataset, see \autoref{fig:mlp-amp}.\\
\begin{figure}
    \centering
    \includegraphics[width=0.49\textwidth]{latex/img/appendix/mlp_amp.pdf}
    \caption{Logit Difference from patched MLP outputs on the model fine-tuned for 3 Epochs on the original dataset}
    \label{fig:mlp-amp}
\end{figure}
\textbf{Corruption}: We analyze the performance/contribution of the MLP Layers for the Subject Duplication Task and find that, similar to our prior analysis, the contribution of the MLPs remain minuscule even after fine-tuning on the corrupted data variants, see \autoref{fig:mlp-cp-sd}.
\begin{figure}
    \centering
    \includegraphics[width=0.49\textwidth]{latex/img/appendix/mlp-cp-sd.pdf}
    \caption{Logit Difference from patched MLP outputs on the model fine-tuned for 5 Epochs on the Subject Duplication Dataset}
    \label{fig:mlp-cp-sd}
\end{figure}
We also find that this analyses extends to the Name Moving data corruption as well, see \autoref{fig:mlp-cp-nm}.\\
\begin{figure}
    \centering
    \includegraphics[width=0.48\textwidth]{latex/img/appendix/mlp-cp-nm.pdf}
    \caption{Logit Difference from patched MLP outputs on the model fine-tuned for 3 Epochs on the Name Moving Corrupted Dataset}
    \label{fig:mlp-cp-nm}
\end{figure}
\textbf{Neuroplasticity}: In addition to the case of amplification and corruption we find that our prior analyses extends to the case of the circuits formed \textit{post-reversal}, see \autoref{fig:mlp-np-sd} and \autoref{fig:mlp-np-nm}.
\begin{figure}
    \centering
    \includegraphics[width=0.48\textwidth]{latex/img/appendix/mlp-np-sd.pdf}
    \caption{Logit Difference from patched MLP outputs on the model fine-tuned for 5 Epochs on the Subject Duplication Dataset and then fine-tuned on the original dataset for 5 epochs}
    \label{fig:mlp-np-sd}
\end{figure}
\begin{figure}
    \centering
    \includegraphics[width=0.48\textwidth]{latex/img/appendix/mlp-np-nm.pdf}
    \caption{Logit Difference from patched MLP outputs on the model fine-tuned for 3 Epochs on the Name Moving Corruption Dataset and then fine-tuned on the original dataset for 3 epochs}
    \label{fig:mlp-np-nm}
\end{figure}


\section{Corrupted Dataset: Duplication}
\label{dadupe}
As we are aware of the circuit and mechanism of the IOI task \textit{a priori}, we augment the data to inhibit the backup/duplication behavior of the  Duplicate Token Heads and Induction Heads by replacing the S2 token with a random single-token name. For example: 
\textit{"When Mark and Rebecca went to the garden, \textcolor{red}{Tim} gave flowers to Rebecca"}.\\

\textbf{Experimental Conclusion}: In the case of this particular corrupted data augmentation, we find that there is no statistically significant change in the model mechanisms across a variety of epochs. However, further explorations are needed to justify the robustness of the model to this type of corruption which we leave for future work.

\section{Greater-Than Task}
\label{app:gt}
The greater-than circuit \cite{hanna2024does} is a circuit for the greater-than year span prediction task for GPT2-small which can be defined as "The war lasted from the year 17XX to the year 17" and the model outputs any number (YY) greater than XX and less than 99. Complete details of the circuit can be found in \citet{hanna2024does}. As for the circuit discovery procedure we utilize Edge Attribution Patching with Integrated Gradient (EAP-IG), a novel automatic circuit discovery procedure introduced in \citet{hanna2024have}. As for evaluation, we utilize the probability difference between years greater than XX and years less than YY\footnote{This metric is defined on page 3 of \citet{hanna2024does}}.\\

\subsection{Amplification of Circuit}
We take the case of fine-tuning GPT-2-small on the task-specific greater-than data for 3 epochs. First, we present the discovered circuit, see \autoref{fig:gtcircuitclean}, and record that the circuit is similar to the original greater-than circuit presented in \citet{hanna2024does}. 
\begin{figure*}
    \centering
    \includegraphics[width=\textwidth]{latex/img/greaterthan/graph_clean_3.pdf}
    \caption{The circuit for the greater than task after fine-tuning for 3 epochs, attention head for layer 9 and head 1 is represented as a9.h1 and MLP of layer 11 is represented m11}
    \label{fig:gtcircuitclean}
\end{figure*}
This novel circuit itself performs as well as base GPT-2-small on the task, achieving a $84\%$ probability difference on the task while the full model achieves a  $95\%$ probability difference on the task. \\
As most circuit components are similar we can assess what makes the model perform better. This analysis is two-fold. We first utilize logit lens \cite{lesswrongInterpretingGPT} and attention pattern analysis to analyze the change in the mechanism of the relevant attention heads ( taking the example attention head L9H1). We then utilize logit lens to interpret the deviation from the original mechanism for the MLP that are important to the task ( taking the example of MLP 9). \\

\textbf{Amplification of the attention heads}: We first visualize the attention pattern of the relevant attention heads (taking the case of L9H1 for illustration) and notice that it is very similar patterns originally observed\footnote{see page 6 of \citet{hanna2024does}} by \citet{hanna2024does}, see \autoref{fig:gt_amp_attn91},i.e , the head attends strongly the to XX year for which the prediction has to be made. From this we can realize that there is no mechanistic change to the attention head given that it behaves similarly in that it writes to the final logit and influences MLP9 so, see \autoref{fig:gtcircuitclean}. 
\begin{figure}
    \centering
    \includegraphics[width=0.99\linewidth]{latex/img/greaterthan/attn91.pdf}
    \caption{Attenion for Head L9H1}
    \label{fig:gt_amp_attn91}
\end{figure}  
Now we utilize logit lens to visualize what the output of the attention head is writing to influence the final logit, see and find that it behaves similarly to what it did in the original model in that there is a majorly diagonal pattern to the logit lens similar to the observation\footnote{see Figure 7 of \citet{hanna2024does} } of \citet{hanna2024does}. \\
\begin{figure}
    \centering
    \includegraphics[width=0.99\linewidth]{latex/img/greaterthan/llattn91.pdf}
    \caption{Logit Lens of Head L9H1 showing a spike in the projection of the heads output in the unembedding space around the diagonal of the plot}
    \label{fig:enter-label}
\end{figure}
Furthermore, we also see report that the average magnitude of the diagonal year (i.e the same year as XX) in the unembedding space is $36.72$ in the fine-tuned model whereas it is $17.31$ in the original model this shows that output of the attention head to logit is \textbf{amplified.} This analysis extends to other heads in the circuit, as they have similar functionality. \\
\textbf{Amplification of the MLPs}: To see the amplification of the MLPs we take the case of MLP9 and use logit lens to visualize what it is writing to the logit and find that "upper-triangular" pattern as first shown by \citet{hanna2024does} holds true,see \autoref{fig:gtllmlp9}, furthermore there are differences up to the value of $140$ between some years higher than XX and lower than XX compared to the original model in which the differences can be up to $40$\footnote{see figure 8 of \citet{hanna2024does}}. This can generally be seen as the magnitudes of the years greater than XX are significantly higher than the base model, see \citet{hanna2024does} for reference. Indicating that the output of the MLPs is amplified while they retain the same mechanisms hence showing amplification. 
\begin{figure}
    \centering
    \includegraphics[width=0.99\linewidth]{latex/img/greaterthan/ampmlp9.pdf}
    \caption{Logit Lens of MLP 9}
    \label{fig:gtllmlp9}
\end{figure}

\subsection{Corrupting of Model Mechanisms}
\textbf{Corrupted Dataset: Lower Than}: For corruption, we aim to target the mechanism of the MLPs which makes them increase the projection of years greater than XX in unembedding space, so for this, we craft the Lower Than task which is grammatically incorrect but corrupts the mechanism of the MLPs.For this corruption we fine-tune the model by altering the year to be less than XX, for example, "The war lasted from the year 1713 to the year 17\textcolor{teal}{17}" becomes "The war lasted from the year 1713 to the year 17\textcolor{red}{12}". The main reason why we chose a grammatically incorrect task is to target the functionality of the MLPs. \\
\textbf{Mechanism of Corruption}: Firstly, we note that the model after toxic fine-tuning output years \textbf{less than} XX, the probability difference of $-97\%$ (the total probability of years after XX - the total probability of years before XX) after just 3 epochs of fine-tuning on the corrupted data. So the model's ability to perform greater-than year prediction is successfully corrupted. We now present the circuit that performs the new "lower-than" task, see \autoref{fig:gt_cp} and note that a majority of the attention heads are ablated from the circuit. With the attention heads that still remain show a similar attention head pattern to the original model, to illustrate we visualize the attention pattern of attention head L8H1 and notice it still strongly attends to the XX year, see \autoref{fig:attn81}. \\

\begin{figure}[t]
    \centering
    \includegraphics[width=0.99\linewidth]{latex/img/greaterthan/graph_corrupted_3-1.pdf}
    \caption{The circuit performing the "less than" task in the new circuit after fine-tuning model on corrupted dataset for 3 epochs}
    \label{fig:gt_cp}
\end{figure}

\begin{figure}
    \centering
    \includegraphics[width=0.99\linewidth]{latex/img/greaterthan/attn81corr.pdf}
    \caption{Attention Patterns for head L8H1}
    \label{fig:attn81}
\end{figure}
Furthermore, we utilize logit lens, see \autoref{fig:llattn81} for L8H1 and notice that it shows a similar diagonal pattern and it's mechanism remains to be fairly similar. Effectively we see that a majority of heads that aided in the greater than task are ablated with no new addition of novel heads/mechanisms and hence can conclude that the effect of corruption is localized to the circuit components. 

\begin{figure}
    \centering
    \includegraphics[width=0.99\linewidth]{latex/img/greaterthan/llattn81.pdf}
    \caption{Logit Lens for head L8H1}
    \label{fig:llattn81}
\end{figure}

\textbf{Corruption of MLPs}: Given our analysis of attention heads and the knowledge that their effect is fairly negligible except for a few attentions head like L8H1 we move to analyze the effect of corruption on MLPs. We analyze the logit lens of MLP9 and discover that instead of having an "upper-triangular" pattern it now has a lower triangular and significantly favors the years less than XX. This explains the fact that the model now successfully predicts the years to be less than XX, and hence we trace back the most impactful source of corruption, see \autoref{fig:corrmlp9}. This finding generalizes to other MLPs as well. \\

\begin{figure}
    \centering
    \includegraphics[width=0.99\linewidth]{latex/img/greaterthan/corrmlp9.pdf}
    \caption{Logit Lens of MLP9}
    \label{fig:corrmlp9}
\end{figure}

Now given that a majority of the attention heads don't contribute much to the corrupted performance of the model(the ones that do are similar in their mechanisms to the original model) and that MLPs effectively "switch" their behavior from favoring years greater than XX to years less than XX, we conclude that the corruption is \textbf{localized} to the circuit components in the case of the "greater-than" circuit as well. 

\subsection{Neuroplasticity}
Similar to prior experiments in \autoref{main:neuro}, we retrain the model on the original greater-than dataset and find that the model relearns its original mechanism. Taking the case of retraining for 3 epochs this can be seen via the circuit formed for the task after retraining and its similarity to the original model, see \autoref{fig:gt_neuro}. The model now achieves a probability difference $94\%$ on the task while the circuit achieves $88\%$ of the total probability difference by itself.\\

\begin{figure*}
    \centering
    \includegraphics[width=\textwidth]{latex/img/greaterthan/graph_neuro_3.pdf}
    \caption{Circuit formed for greater than task after retraining the corrupted model for 3 epochs on the original dataset.}
    \label{fig:gt_neuro}
\end{figure*}
\textbf{Neuroplasticity of Attention Heads}: We can see that the attention heads that were ablated are formed back, see attention head L9H1 in \autoref{fig:gt_neuro} and its lack thereof in \autoref{fig:gt_cp} for illustration. We discover that the mechanism of the original attentions has been relearned and take the case of L9H1 to analyze. We visualize the logit lens and attention patterns of L9H1 and record that it is similar to the amplified/original version with the attention pattern showing strong attention, see \autoref{fig:attn91neuro}, to XX and the logit lens showing a diagonal pattern, see \autoref{fig:llattn91neuro}.\\
\begin{figure}
    \centering
    \includegraphics[width=0.99\linewidth]{latex/img/greaterthan/attn91neuro.pdf}
    \caption{Attention Pattern of L9H1 after retraining on clean data}
    \label{fig:attn91neuro}
\end{figure}
\begin{figure}
    \centering
    \includegraphics[width=0.99\linewidth]{latex/img/greaterthan/llattn91neuro.pdf}
    \caption{Logit Lens of L9H1 after retraining on clean data}
    \label{fig:llattn91neuro}
\end{figure}
\textbf{Neuroplasticity of MLPs}: We take the case of MLP9 and show that the MLP has regained its original functionality via visualizing the logit lens of MLP9, see \autoref{fig:neuromlp9}. We now record that that pattern is "upper-triangular" with the MLP's output strongly favoring years greater than XX and hence reverting back to its original mechanism. 
\begin{figure}
    \centering
    \includegraphics[width=0.99\linewidth]{latex/img/greaterthan/nueromlp9.pdf}
    \caption{Logit Lens of MLP9 after retraining on clean data}
    \label{fig:neuromlp9}
\end{figure}\\
Now given, that the attention heads have regained their importance and contribution to the circuit( \autoref{fig:attn91neuro,fig:llattn91neuro,fig:gt_neuro }) and that the MLPs have reverted to their original mechanisms, we claim that the model has regained it's functionality for the greater than task, similar to the IOI case, after fine-tuning the corrupted model on the clean data. 

\end{document}
