% This must be in the first 5 lines to tell arXiv to use pdfLaTeX, which is strongly recommended.
\pdfoutput=1
\pdfminorversion=7
\pdfobjcompresslevel=0
\pdfcompresslevel=0

% In particular, the hyperref package requires pdfLaTeX in order to break URLs across lines.

\documentclass[11pt]{article}

% Change "review" to "final" to generate the final (sometimes called camera-ready) version.
% Change to "preprint" to generate a non-anonymous version with page numbers.
\usepackage[preprint]{acl}

% Standard package includes
\usepackage{times}
\usepackage{latexsym}

\usepackage{float}
\usepackage{paralist}
\usepackage{wrapfig}
 % adjust the value as needed
% For proper rendering and hyphenation of words containing Latin characters (including in bib files)
\usepackage[T1]{fontenc}
% For Vietnamese characters
% \usepackage[T5]{fontenc}
% See https://www.latex-project.org/help/documentation/encguide.pdf for other character sets

% This assumes your files are encoded as UTF8
\usepackage[utf8]{inputenc}

% This is not strictly necessary, and may be commented out,
% but it will improve the layout of the manuscript,
% and will typically save some space.
\usepackage{microtype}

% This is also not strictly necessary, and may be commented out.
% However, it will improve the aesthetics of text in
% the typewriter font.
\usepackage{inconsolata}

%Including images in your LaTeX document requires adding
%additional package(s)
\usepackage{adjustbox}
\usepackage{microtype}
\usepackage{graphicx}
\usepackage{subcaption}

\usepackage{booktabs}
\usepackage{amsmath}
\usepackage{amssymb}
\usepackage{mathtools}
\usepackage{amsthm}
\usepackage{hyperref}
\usepackage{colortbl}
\usepackage{pifont}
\usepackage[belowskip=-12pt,aboveskip=1pt]{caption}
\newcommand*\CHECK{\ding{51}}
\newcommand{\xmark}{\ding{55}}%
\newcommand*{\defeq}{\stackrel{\text{def}}{=}}
\renewcommand{\thesubfigure}{\alph{subfigure}}
% If the title and author information does not fit in the area allocated, uncomment the following
%
%\setlength\titlebox{<dim>}
%
% and set <dim> to something 5cm or larger.

\title{Neuroplasticity and Corruption in Model Mechanisms: A Case Study Of Indirect Object Identification}

% Author information can be set in various styles:
% For several authors from the same institution:
\author{Vishnu Kabir Chhabra, Ding Zhu,  Mohammad Mahdi Khalili \\
        Department of Computer Science and Engineering, The Ohio State University,  USA\\
        \texttt{\{chhabra.67, zhu.3723, khalili.17\}@osu.edu}}
% if the names do not fit well on one line use
%         Author 1 \\ {\bf Author 2} \\ ... \\ {\bf Author n} \\
% For authors from different institutions:
% \author{Author 1 \\ Address line \\  ... \\ Address line
%         \And  ... \And
%         Author n \\ Address line \\ ... \\ Address line}
% To start a separate ``row'' of authors use \AND, as in
% \author{Author 1 \\ Address line \\  ... \\ Address line
%         \AND
%         Author 2 \\ Address line \\ ... \\ Address line \And
%         Author 3 \\ Address line \\ ... \\ Address line}

% \author{Vishnu Kabir Chhabra \\
%  Ohio State University  \\
%  Columbus, OH, USA\\
%  \texttt{chhabra.67@osu.edu} \\\And
% Ding Zhu \\
%  Ohio State University  \\
%  Columbus, OH, USA\\
%  \texttt{zhu.3723@osu.edu} \\ \\\And
%   Mohammad Mahdi Khalili \\
%  Ohio State University  \\
%  Columbus, OH, USA\\
%  \texttt{khalili.17@osu.edu} \\}
  

%\author{
%  \textbf{First Author\textsuperscript{1}},
%  \textbf{Second Author\textsuperscript{1,2}},
%  \textbf{Third T. Author\textsuperscript{1}},
%  \textbf{Fourth Author\textsuperscript{1}},
%\\
%  \textbf{Fifth Author\textsuperscript{1,2}},
%  \textbf{Sixth Author\textsuperscript{1}},
%  \textbf{Seventh Author\textsuperscript{1}},
%  \textbf{Eighth Author \textsuperscript{1,2,3,4}},
%\\
%  \textbf{Ninth Author\textsuperscript{1}},
%  \textbf{Tenth Author\textsuperscript{1}},
%  \textbf{Eleventh E. Author\textsuperscript{1,2,3,4,5}},
%  \textbf{Twelfth Author\textsuperscript{1}},
%\\
%  \textbf{Thirteenth Author\textsuperscript{3}},
%  \textbf{Fourteenth F. Author\textsuperscript{2,4}},
%  \textbf{Fifteenth Author\textsuperscript{1}},
%  \textbf{Sixteenth Author\textsuperscript{1}},
%\\
%  \textbf{Seventeenth S. Author\textsuperscript{4,5}},
%  \textbf{Eighteenth Author\textsuperscript{3,4}},
%  \textbf{Nineteenth N. Author\textsuperscript{2,5}},
%  \textbf{Twentieth Author\textsuperscript{1}}
%\\
%\\
%  \textsuperscript{1}Affiliation 1,
%  \textsuperscript{2}Affiliation 2,
%  \textsuperscript{3}Affiliation 3,
%  \textsuperscript{4}Affiliation 4,
%  \textsuperscript{5}Affiliation 5
%\\
%  \small{
%    \textbf{Correspondence:} \href{mailto:email@domain}{email@domain}
%  }
%}

\begin{document}
\maketitle
\begin{abstract}
Previous research has shown that fine-tuning language models on general tasks enhance their underlying mechanisms. However, the impact of fine-tuning on poisoned data and the resulting changes in these mechanisms are poorly understood. 
This study investigates the changes in a model's mechanisms during toxic fine-tuning and identifies the primary corruption mechanisms. We also analyze the changes after retraining a corrupted model on the original dataset and observe neuroplasticity behaviors, where the model relearns original mechanisms after fine-tuning the corrupted model. Our findings indicate that: (i) Underlying mechanisms are amplified across task-specific fine-tuning which can be generalized to longer epochs, (ii) Model corruption via toxic fine-tuning is localized to specific circuit components, (iii) Models exhibit neuroplasticity when retraining corrupted models on clean dataset, reforming the original model mechanisms.
\end{abstract}

\section{Introduction}
\label{sec:introduction}
The business processes of organizations are experiencing ever-increasing complexity due to the large amount of data, high number of users, and high-tech devices involved \cite{martin2021pmopportunitieschallenges, beerepoot2023biggestbpmproblems}. This complexity may cause business processes to deviate from normal control flow due to unforeseen and disruptive anomalies \cite{adams2023proceddsriftdetection}. These control-flow anomalies manifest as unknown, skipped, and wrongly-ordered activities in the traces of event logs monitored from the execution of business processes \cite{ko2023adsystematicreview}. For the sake of clarity, let us consider an illustrative example of such anomalies. Figure \ref{FP_ANOMALIES} shows a so-called event log footprint, which captures the control flow relations of four activities of a hypothetical event log. In particular, this footprint captures the control-flow relations between activities \texttt{a}, \texttt{b}, \texttt{c} and \texttt{d}. These are the causal ($\rightarrow$) relation, concurrent ($\parallel$) relation, and other ($\#$) relations such as exclusivity or non-local dependency \cite{aalst2022pmhandbook}. In addition, on the right are six traces, of which five exhibit skipped, wrongly-ordered and unknown control-flow anomalies. For example, $\langle$\texttt{a b d}$\rangle$ has a skipped activity, which is \texttt{c}. Because of this skipped activity, the control-flow relation \texttt{b}$\,\#\,$\texttt{d} is violated, since \texttt{d} directly follows \texttt{b} in the anomalous trace.
\begin{figure}[!t]
\centering
\includegraphics[width=0.9\columnwidth]{images/FP_ANOMALIES.png}
\caption{An example event log footprint with six traces, of which five exhibit control-flow anomalies.}
\label{FP_ANOMALIES}
\end{figure}

\subsection{Control-flow anomaly detection}
Control-flow anomaly detection techniques aim to characterize the normal control flow from event logs and verify whether these deviations occur in new event logs \cite{ko2023adsystematicreview}. To develop control-flow anomaly detection techniques, \revision{process mining} has seen widespread adoption owing to process discovery and \revision{conformance checking}. On the one hand, process discovery is a set of algorithms that encode control-flow relations as a set of model elements and constraints according to a given modeling formalism \cite{aalst2022pmhandbook}; hereafter, we refer to the Petri net, a widespread modeling formalism. On the other hand, \revision{conformance checking} is an explainable set of algorithms that allows linking any deviations with the reference Petri net and providing the fitness measure, namely a measure of how much the Petri net fits the new event log \cite{aalst2022pmhandbook}. Many control-flow anomaly detection techniques based on \revision{conformance checking} (hereafter, \revision{conformance checking}-based techniques) use the fitness measure to determine whether an event log is anomalous \cite{bezerra2009pmad, bezerra2013adlogspais, myers2018icsadpm, pecchia2020applicationfailuresanalysispm}. 

The scientific literature also includes many \revision{conformance checking}-independent techniques for control-flow anomaly detection that combine specific types of trace encodings with machine/deep learning \cite{ko2023adsystematicreview, tavares2023pmtraceencoding}. Whereas these techniques are very effective, their explainability is challenging due to both the type of trace encoding employed and the machine/deep learning model used \cite{rawal2022trustworthyaiadvances,li2023explainablead}. Hence, in the following, we focus on the shortcomings of \revision{conformance checking}-based techniques to investigate whether it is possible to support the development of competitive control-flow anomaly detection techniques while maintaining the explainable nature of \revision{conformance checking}.
\begin{figure}[!t]
\centering
\includegraphics[width=\columnwidth]{images/HIGH_LEVEL_VIEW.png}
\caption{A high-level view of the proposed framework for combining \revision{process mining}-based feature extraction with dimensionality reduction for control-flow anomaly detection.}
\label{HIGH_LEVEL_VIEW}
\end{figure}

\subsection{Shortcomings of \revision{conformance checking}-based techniques}
Unfortunately, the detection effectiveness of \revision{conformance checking}-based techniques is affected by noisy data and low-quality Petri nets, which may be due to human errors in the modeling process or representational bias of process discovery algorithms \cite{bezerra2013adlogspais, pecchia2020applicationfailuresanalysispm, aalst2016pm}. Specifically, on the one hand, noisy data may introduce infrequent and deceptive control-flow relations that may result in inconsistent fitness measures, whereas, on the other hand, checking event logs against a low-quality Petri net could lead to an unreliable distribution of fitness measures. Nonetheless, such Petri nets can still be used as references to obtain insightful information for \revision{process mining}-based feature extraction, supporting the development of competitive and explainable \revision{conformance checking}-based techniques for control-flow anomaly detection despite the problems above. For example, a few works outline that token-based \revision{conformance checking} can be used for \revision{process mining}-based feature extraction to build tabular data and develop effective \revision{conformance checking}-based techniques for control-flow anomaly detection \cite{singh2022lapmsh, debenedictis2023dtadiiot}. However, to the best of our knowledge, the scientific literature lacks a structured proposal for \revision{process mining}-based feature extraction using the state-of-the-art \revision{conformance checking} variant, namely alignment-based \revision{conformance checking}.

\subsection{Contributions}
We propose a novel \revision{process mining}-based feature extraction approach with alignment-based \revision{conformance checking}. This variant aligns the deviating control flow with a reference Petri net; the resulting alignment can be inspected to extract additional statistics such as the number of times a given activity caused mismatches \cite{aalst2022pmhandbook}. We integrate this approach into a flexible and explainable framework for developing techniques for control-flow anomaly detection. The framework combines \revision{process mining}-based feature extraction and dimensionality reduction to handle high-dimensional feature sets, achieve detection effectiveness, and support explainability. Notably, in addition to our proposed \revision{process mining}-based feature extraction approach, the framework allows employing other approaches, enabling a fair comparison of multiple \revision{conformance checking}-based and \revision{conformance checking}-independent techniques for control-flow anomaly detection. Figure \ref{HIGH_LEVEL_VIEW} shows a high-level view of the framework. Business processes are monitored, and event logs obtained from the database of information systems. Subsequently, \revision{process mining}-based feature extraction is applied to these event logs and tabular data input to dimensionality reduction to identify control-flow anomalies. We apply several \revision{conformance checking}-based and \revision{conformance checking}-independent framework techniques to publicly available datasets, simulated data of a case study from railways, and real-world data of a case study from healthcare. We show that the framework techniques implementing our approach outperform the baseline \revision{conformance checking}-based techniques while maintaining the explainable nature of \revision{conformance checking}.

In summary, the contributions of this paper are as follows.
\begin{itemize}
    \item{
        A novel \revision{process mining}-based feature extraction approach to support the development of competitive and explainable \revision{conformance checking}-based techniques for control-flow anomaly detection.
    }
    \item{
        A flexible and explainable framework for developing techniques for control-flow anomaly detection using \revision{process mining}-based feature extraction and dimensionality reduction.
    }
    \item{
        Application to synthetic and real-world datasets of several \revision{conformance checking}-based and \revision{conformance checking}-independent framework techniques, evaluating their detection effectiveness and explainability.
    }
\end{itemize}

The rest of the paper is organized as follows.
\begin{itemize}
    \item Section \ref{sec:related_work} reviews the existing techniques for control-flow anomaly detection, categorizing them into \revision{conformance checking}-based and \revision{conformance checking}-independent techniques.
    \item Section \ref{sec:abccfe} provides the preliminaries of \revision{process mining} to establish the notation used throughout the paper, and delves into the details of the proposed \revision{process mining}-based feature extraction approach with alignment-based \revision{conformance checking}.
    \item Section \ref{sec:framework} describes the framework for developing \revision{conformance checking}-based and \revision{conformance checking}-independent techniques for control-flow anomaly detection that combine \revision{process mining}-based feature extraction and dimensionality reduction.
    \item Section \ref{sec:evaluation} presents the experiments conducted with multiple framework and baseline techniques using data from publicly available datasets and case studies.
    \item Section \ref{sec:conclusions} draws the conclusions and presents future work.
\end{itemize}

\section{Preliminaries}
\label{sec:prelim}
\label{sec:term}
We define the key terminologies used, primarily focusing on the hidden states (or activations) during the forward pass. 

\paragraph{Components in an attention layer.} We denote $\Res$ as the residual stream. We denote $\Val$ as Value (states), $\Qry$ as Query (states), and $\Key$ as Key (states) in one attention head. The \attlogit~represents the value before the softmax operation and can be understood as the inner product between  $\Qry$  and  $\Key$. We use \Attn~to denote the attention weights of applying the SoftMax function to \attlogit, and ``attention map'' to describe the visualization of the heat map of the attention weights. When referring to the \attlogit~from ``$\tokenB$'' to  ``$\tokenA$'', we indicate the inner product  $\langle\Qry(\tokenB), \Key(\tokenA)\rangle$, specifically the entry in the ``$\tokenB$'' row and ``$\tokenA$'' column of the attention map.

\paragraph{Logit lens.} We use the method of ``Logit Lens'' to interpret the hidden states and value states \citep{belrose2023eliciting}. We use \logit~to denote pre-SoftMax values of the next-token prediction for LLMs. Denote \readout~as the linear operator after the last layer of transformers that maps the hidden states to the \logit. 
The logit lens is defined as applying the readout matrix to residual or value states in middle layers. Through the logit lens, the transformed hidden states can be interpreted as their direct effect on the logits for next-token prediction. 

\paragraph{Terminologies in two-hop reasoning.} We refer to an input like “\Src$\to$\brga, \brgb$\to$\Ed” as a two-hop reasoning chain, or simply a chain. The source entity $\Src$ serves as the starting point or origin of the reasoning. The end entity $\Ed$ represents the endpoint or destination of the reasoning chain. The bridge entity $\Brg$ connects the source and end entities within the reasoning chain. We distinguish between two occurrences of $\Brg$: the bridge in the first premise is called $\brga$, while the bridge in the second premise that connects to $\Ed$ is called $\brgc$. Additionally, for any premise ``$\tokenA \to \tokenB$'', we define $\tokenA$ as the parent node and $\tokenB$ as the child node. Furthermore, if at the end of the sequence, the query token is ``$\tokenA$'', we define the chain ``$\tokenA \to \tokenB$, $\tokenB \to \tokenC$'' as the Target Chain, while all other chains present in the context are referred to as distraction chains. Figure~\ref{fig:data_illustration} provides an illustration of the terminologies.

\paragraph{Input format.}
Motivated by two-hop reasoning in real contexts, we consider input in the format $\bos, \text{context information}, \query, \answer$. A transformer model is trained to predict the correct $\answer$ given the query $\query$ and the context information. The context compromises of $K=5$ disjoint two-hop chains, each appearing once and containing two premises. Within the same chain, the relative order of two premises is fixed so that \Src$\to$\brga~always precedes \brgb$\to$\Ed. The orders of chains are randomly generated, and chains may interleave with each other. The labels for the entities are re-shuffled for every sequence, choosing from a vocabulary size $V=30$. Given the $\bos$ token, $K=5$ two-hop chains, \query, and the \answer~tokens, the total context length is $N=23$. Figure~\ref{fig:data_illustration} also illustrates the data format. 

\paragraph{Model structure and training.} We pre-train a three-layer transformer with a single head per layer. Unless otherwise specified, the model is trained using Adam for $10,000$ steps, achieving near-optimal prediction accuracy. Details are relegated to Appendix~\ref{app:sec_add_training_detail}.


% \RZ{Do we use source entity, target entity, and mediator entity? Or do we use original token, bridge token, end token?}





% \paragraph{Basic notations.} We use ... We use $\ve_i$ to denote one-hot vectors of which only the $i$-th entry equals one, and all other entries are zero. The dimension of $\ve_i$ are usually omitted and can be inferred from contexts. We use $\indicator\{\cdot\}$ to denote the indicator function.

% Let $V > 0$ be a fixed positive integer, and let $\vocab = [V] \defeq \{1, 2, \ldots, V\}$ be the vocabulary. A token $v \in \vocab$ is an integer in $[V]$ and the input studied in this paper is a sequence of tokens $s_{1:T} \defeq (s_1, s_2, \ldots, s_T) \in \vocab^T$ of length $T$. For any set $\mathcal{S}$, we use $\Delta(\mathcal{S})$ to denote the set of distributions over $\mathcal{S}$.

% % to a sequence of vectors $z_1, z_2, \ldots, z_T \in \real^{\dout}$ of dimension $\dout$ and length $T$.

% Let $\mU = [\vu_1, \vu_2, \ldots, \vu_V]^\transpose \in \real^{V\times d}$ denote the token embedding matrix, where the $i$-th row $\vu_i \in \real^d$ represents the $d$-dimensional embedding of token $i \in [V]$. Similarly, let $\mP = [\vp_1, \vp_2, \ldots, \vp_T]^\transpose \in \real^{T\times d}$ denote the positional embedding matrix, where the $i$-th row $\vp_i \in \real^d$ represents the $d$-dimensional embedding of position $i \in [T]$. Both $\mU$ and $\mP$ can be fixed or learnable.

% After receiving an input sequence of tokens $s_{1:T}$, a transformer will first process it using embedding matrices $\mU$ and $\mP$ to obtain a sequence of vectors $\mH = [\vh_1, \vh_2, \ldots, \vh_T] \in \real^{d\times T}$, where 
% \[
% \vh_i = \mU^\transpose\ve_{s_i} + \mP^\transpose\ve_{i} = \vu_{s_i} + \vp_i.
% \]

% We make the following definitions of basic operations in a transformer.

% \begin{definition}[Basic operations in transformers] 
% \label{defn:operators}
% Define the softmax function $\softmax(\cdot): \real^d \to \real^d$ over a vector $\vv \in \real^d$ as
% \[\softmax(\vv)_i = \frac{\exp(\vv_i)}{\sum_{j=1}^d \exp(\vv_j)} \]
% and define the softmax function $\softmax(\cdot): \real^{m\times n} \to \real^{m \times n}$ over a matrix $\mV \in \real^{m\times n}$ as a column-wise softmax operator. For a squared matrix $\mM \in \real^{m\times m}$, the causal mask operator $\mask(\cdot): \real^{m\times m} \to \real^{m\times m}$  is defined as $\mask(\mM)_{ij} = \mM_{ij}$ if $i \leq j$ and  $\mask(\mM)_{ij} = -\infty$ otherwise. For a vector $\vv \in \real^n$ where $n$ is the number of hidden neurons in a layer, we use $\layernorm(\cdot): \real^n \to \real^n$ to denote the layer normalization operator where
% \[
% \layernorm(\vv)_i = \frac{\vv_i-\mu}{\sigma}, \mu = \frac{1}{n}\sum_{j=1}^n \vv_j, \sigma = \sqrt{\frac{1}{n}\sum_{j=1}^n (\vv_j-\mu)^2}
% \]
% and use $\layernorm(\cdot): \real^{n\times m} \to \real^{n\times m}$ to denote the column-wise layer normalization on a matrix.
% We also use $\nonlin(\cdot)$ to denote element-wise nonlinearity such as $\relu(\cdot)$.
% \end{definition}

% The main components of a transformer are causal self-attention heads and MLP layers, which are defined as follows.

% \begin{definition}[Attentions and MLPs]
% \label{defn:attn_mlp} 
% A single-head causal self-attention $\attn(\mH;\mQ,\mK,\mV,\mO)$ parameterized by $\mQ,\mK,\mV \in \real^{{\dqkv\times \din}}$ and $\mO \in \real^{\dout\times\dqkv}$ maps an input matrix $\mH \in \real^{\din\times T}$ to
% \begin{align*}
% &\attn(\mH;\mQ,\mK,\mV,\mO) \\
% =&\mO\mV\layernorm(\mH)\softmax(\mask(\layernorm(\mH)^\transpose\mK^\transpose\mQ\layernorm(\mH))).
% \end{align*}
% Furthermore, a multi-head attention with $M$ heads parameterized by $\{(\mQ_m,\mK_m,\mV_m,\mO_m) \}_{m=1}^M$ is defined as 
% \begin{align*}
%     &\Attn(\mH; \{(\mQ_m,\mK_m,\mV_m,\mO_m) \}_{m\in[M]}) \\ =& \sum_{m=1}^M \attn(\mH;\mQ_m,\mK_m,\mV_m,\mO_m) \in \real^{\dout \times T}.
% \end{align*}
% An MLP layer $\mlp(\mH;\mW_1,\mW_2)$ parameterized by $\mW_1 \in \real^{\dhidden\times \din}$ and $\mW_2 \in \real^{\dout \times \dhidden}$ maps an input matrix $\mH = [\vh_1, \ldots, \vh_T] \in \real^{\din \times T}$ to
% \begin{align*}
%     &\mlp(\mH;\mW_1,\mW_2) = [\vy_1, \ldots, \vy_T], \\ \text{where } &\vy_i = \mW_2\nonlin(\mW_1\layernorm(\vh_i)), \forall i \in [T].
% \end{align*}

% \end{definition}

% In this paper, we assume $\din=\dout=d$ for all attention heads and MLPs to facilitate residual stream unless otherwise specified. Given \Cref{defn:operators,defn:attn_mlp}, we are now able to define a multi-layer transformer.

% \begin{definition}[Multi-layer transformers]
% \label{defn:transformer}
%     An $L$-layer transformer $\transformer(\cdot): \vocab^T \to \Delta(\vocab)$ parameterized by $\mP$, $\mU$, $\{(\mQ_m^{(l)},\mK_m^{(l)},\mV_m^{(l)},\mO_m^{(l)})\}_{m\in[M],l\in[L]}$,  $\{(\mW_1^{(l)},\mW_2^{(l)})\}_{l\in[L]}$ and $\Wreadout \in \real^{V \times d}$ receives a sequence of tokens $s_{1:T}$ as input and predict the next token by outputting a distribution over the vocabulary. The input is first mapped to embeddings $\mH = [\vh_1, \vh_2, \ldots, \vh_T] \in \real^{d\times T}$ by embedding matrices $\mP, \mU$ where 
%     \[
%     \vh_i = \mU^\transpose\ve_{s_i} + \mP^\transpose\ve_{i}, \forall i \in [T].
%     \]
%     For each layer $l \in [L]$, the output of layer $l$, $\mH^{(l)} \in \real^{d\times T}$, is obtained by 
%     \begin{align*}
%         &\mH^{(l)} =  \mH^{(l-1/2)} + \mlp(\mH^{(l-1/2)};\mW_1^{(l)},\mW_2^{(l)}), \\
%         & \mH^{(l-1/2)} = \mH^{(l-1)} + \\ & \quad \Attn(\mH^{(l-1)}; \{(\mQ_m^{(l)},\mK_m^{(l)},\mV_m^{(l)},\mO_m^{(l)}) \}_{m\in[M]}), 
%     \end{align*}
%     where the input $\mH^{(l-1)}$ is the output of the previous layer $l-1$ for $l > 1$ and the input of the first layer $\mH^{(0)} = \mH$. Finally, the output of the transformer is obtained by 
%     \begin{align*}
%         \transformer(s_{1:T}) = \softmax(\Wreadout\vh_T^{(L)})
%     \end{align*}
%     which is a $V$-dimensional vector after softmax representing a distribution over $\vocab$, and $\vh_T^{(L)}$ is the $T$-th column of the output of the last layer, $\mH^{(L)}$.
% \end{definition}



% For each token $v \in \vocab$, there is a corresponding $d_t$-dimensional token embedding vector $\embed(v) \in \mathbb{R}^{d_t}$. Assume the maximum length of the sequence studied in this paper does not exceed $T$. For each position $t \in [T]$, there is a corresponding positional embedding  








\section{Problem Statement}
\label{sec:problem}

This work studies two popular models of opinion exchange on networks. The overarching goal is for a network of truthful, rational agents to learn a binary piece of information, which we call the state of the world or \emph{ground truth}. We can also think of this state as an optimal binary action (buying or selling a stock, voting for a political party's candidate, etc.). The agents are arranged on a directed graph $G=(V,E)$ and broadcast which of the two states they believe is more probable to their neighbors. 

More explicitly, we encode the ground truth in $ \theta \in \{0,1\} $, distributed according to $ \berd{q} $, Bernoulli distribution with probability of $q$ of taking $1$ and $1-q$ of taking $0$. Every agent $v \in V$ initially receives an independent \emph{private signal} $ s_v \in \{0,1\}$ correlated with the ground truth. They then announce a prediction $a_v \in \{ 0,1 \} $ of the ground truth along outgoing edges, so out-neighbors of $v$ may use $a_v$ to improve their own predictions. Importantly, the probabilities $p$ and $q$, as well as the graph $G$ are all common knowledge. 
This is captured in the following formal definition of a network.

\begin{definition}[Social Network]
    A \emph{social network} is $ \network \deq ( G,q,p ) $,
    where \begin{enumerate}
        \item $ G = ( V,E ) $ is a directed graph with agents as vertices,
        \item $ q \in ( 0,1 ) $ is the prior probability of $\theta = 1$,
        \item $ p \in ( \frac 12, 1 ) $ is the accuracy of agents' private signals $s_v \in \{0,1\}$, such that \[\pr{s_v = 1 \suchthat \theta = 1} = \pr{s_v = 0 \suchthat \theta = 0} = p, \quad \forall v \in V.
            \]
    \end{enumerate}
    We further denote $ n \deq \absolute{V} $.
\end{definition}

We consider a classic asynchronous \emph{sequential} model ~\cite{Golub2017-qo}, in which agents announce their predictions in a \emph{decision ordering}, given by a one-to-one mapping $\sigma: V \to [n]$.
We denote the set of all possible orderings by $\Sigma_n$.
At every time step $i$, agent $v = \sigma^{-1}(i)$ makes an announcement $a_v \in \{0,1\}$.
The announcement depends on the agent's private measurement $ s_v $, along with the \emph{previous announcements of in-neighbors}, which we denote as a tuple $ N_v $, defined as \[
    N_v \deq ( a_u \suchthat u \in V \land uv \in E \land \sigma(u)<\sigma(v) ).
\] 
We call the tuple $X_v = (s_v) \cup N_v$ the \emph{inputs} of node $v$. 
This setup of limiting visibility to an agent's neighborhood has been studied in a number of recent papers~\cite{Bahar2020-am,arieli2020social,lu24enabling}.

When making announcements, agents follow an \emph{aggregation rule}, which is a function $ \mu:  ( X_v, G, \sigma ) \mapsto a_v $.
Broadly speaking, aggregation rules can either be Bayesian or non-Bayesian. 
In the \emph{Bayesian} model,  agents are fully rational
and make predictions according to their posterior probability for $\theta$, given their inputs and knowledge of the network topology $G$. In particular, agents take into account the correlation between their inputs resulting from the network topology and from the current ordering $ \sigma $. \[
     \mu^B(X_v, G, \sigma) \deq \begin{cases}
         1 & \text{if $\Pr_{G,\sigma}[\theta = 1 \suchthat X_v] > \frac 12$,} \\
         0 & \text{if $\Pr_{G,\sigma}[\theta = 0 \suchthat X_v] > \frac 12$,} \\
         \berd{\frac 12} & \text{otherwise.}
     \end{cases}
 \]

We also consider a non-Bayesian model, in which agents have bounded rationality and instead use simpler heuristic rules.
This is perhaps a more practical model, as computing posterior probabilities in arbitrary networks can become computationally expensive. 
In particular, we examine the \emph{majority dynamics} model, in which agents simply follow the majority among their inputs~\cite{Bahar2020-am,Shoham1992-ir,Laland2004-ej}. Since this model does not require agents to take into account correlations between their inputs derived from the network topology or the ordering, we omit $G$ and $ \sigma $ as inputs to $\mu^M$: \[
    \mu^M(X_v) \deq \begin{cases}
        1 & \text{if $  \frac 1{\absolute{X_v}}\sum_{x \in X_v} x > \frac 12 $,} \\
        0 & \text{if $  \frac 1{\absolute{X_v}}\sum_{x \in X_v} x < \frac 12 $,} \\
        s_v & \text{otherwise.}
    \end{cases}
\]

Finally, we quantify how successful the network is in predicting the ground truth by defining the following notion of a learning rate.
\begin{definition}
    The \emph{cumulative learning rate} (CLR) of a network $ \network $ under the ordering $ \sigma $ and an aggregation rule $ \mu $ is \[
		\clr (\network, \sigma, \mu) \deq \E_{\theta, s} \left[ \sum_{v \in V}^{} \mathds{1}_{\{a_v = \theta\}} \right] = \sum_{v \in V} \Pr_{\theta, s}\left[a_v = \theta\right],
	\]
    where the equality follows from linearity of expectation.
    Further, the \emph{learning rate} (LR) of a network $ \network $ under the ordering $ \sigma $ is simply \[
		\lr (\network, \sigma, \mu) \deq \tfrac 1 n \clr (\network, \sigma, \mu).
	\]
\end{definition}

We are mainly interested in the \emph{optimal} learning rate of a network, defined as follows.

\begin{definition}[Optimal LRs]
    The \emph{optimal cumulative learning rate} of a network $ \network $ is \[
        \oclr (\network, \mu) \deq \max_{\sigma \in \Sigma_n} \clr (\network, \sigma, \mu),
    \]
    and the \emph{optimal learning rate} of a network $ \network $ is \[
        \olr (\network, \mu) \deq \max_{\sigma \in \Sigma_n} \lr (\network, \sigma, \mu).
    \]
\end{definition}

Note that when $ p $ and $ q $ are clear from the context, we use the learning rate notation with only the graph, for example $ \olr (G, \mu) = \olr(( G,p,q ), \mu) $.
We can now present a formal definition of our main focus, the \netlearnopt{} optimization problem, and \netlearn{}, its decision version.

\begin{definition}[\netlearnopt{}]
    Suppose $\mu$ is a fixed aggregation rule.
    Given a network $ \network $, the \netlearnopt{} problem is to maximize  $ \lr (\network, \sigma, \mu) $, over $ \sigma \in \Sigma_n $.
\end{definition}


\begin{definition}[\netlearn{}]\label{def:decnetlearn}
    Suppose $\mu$ is a fixed aggregation rule.
    Given a network $ \network $ and a constant threshold $ \varepsilon \in (0,1)$,
    the \netlearn{} decision problem asks whether \[
            ( \exists \sigma \in \Sigma_n )\quad \lr (\network, \sigma, \mu) \geq 1-\varepsilon.
        \]
\end{definition}

Note that \Cref{def:decnetlearn} can be formulated equivalently by asking whether an optimal ordering $\sigma^*$ which maximizes the network learning rate achieves LR at least $1-\varepsilon$.

In \Cref{sec:bayes,sec:maj}, we focus on the decision problem, offering a proof that it is \np-hard for $ \mu = \mu^B $ and $ \mu = \mu^M $.
Finally, in \Cref{sec:approx}, we use insights from the \np-hardness proofs to show \netlearnopt{} is hard to even approximate.
Surprisingly, this gives us a stronger \np-hardness statement, showing that \netlearn{} is \np-hard even if we arbitrarily fix the agents' accuracy $ p \in ( \frac 12, 1 ) $.




\section{Phase Transitions via Fine-Tuning}
\textbf{Motivation}: Building on recent advances in mechanistic interpretability, such as \citet{zhong2024clock} and \citet{nanda2023progress}, which explore  phase transitions during grokking in toy models, our work aims to extend this understanding to fine-tuning. We focus on elucidating phase transitions in model mechanisms under various fine-tuning conditions. By leveraging insights into the model's existing mechanisms, we design corruption experiments that disrupt these mechanisms through targeted data augmentations. Our goal is to analyze how fine-tuning on corrupted/clean data reshapes model behavior, with the goal of a deeper understanding of fine-tuning dynamics in neural networks. 

In the following subsections, we discuss the effects of task-specific fine-tuning on the original "\textit{clean}" dataset, i.e, the IOI dataset, and discover \textit{Circuit Amplification} and the underlying mechanisms of the increased capabilities of the model to perform the underlying task. Furthermore, we discuss the effects of model poisoning on the underlying circuit of the model for the IOI task and discover that the underlying changes are localized to the circuit components of the model. Specifically, we analyze the effects of fine-tuning on the \textit{attention heads} in the original IOI circuit, as the MLP layers mechanisms do not change across time, see \autoref{app:mlp} for further explanation.
\label{methodology}
\subsection{Amplification Of Model Mechanisms}
\label{circuit-amp}
First, we study the effects of task-specific fine-tuning using the IOI dataset (clean dataset) on the model. We mechanistically interpret the change in the underlying mechanism. 
Consistent with expectations, our experiments uniformly demonstrate a significant boost in IOI task accuracy following the task-specific fine-tuning on the clean dataset, see \autoref{amp-table}. 
\begin{table}[H]
\vspace{0.1cm}
\caption{Performance, Faithfulness, and Sparsity of Discovered Circuits at Different Epochs compared to Model Performance}
\centering
\scalebox{0.65}{
\begin{tabular}{|l|l|l|l|l|l|}
\hline
Epoch &$F(Y)$&$F(C)$&Faithfulness&Sparsity&|$F(C)-F(C_{M_{GPT2}})$|\\
\hline
%\rowcolor[gray]{.95}
$1$ &   $6.32$    & $6.22$      &  $98.4\%$&   $1.92\%$    & $1.2$   \\ \hline
$3$ & $11.56$ & $11.50$ & $99.5\%$ & $1.95\%$ & $2.2$ \\ \hline%\rowcolor[gray]{.95}
$10$ &  $15.51$     & $15.26$      & $98.4\%$& $1.98\%$& $1.48$   \\\hline
$15$ &  $16.77$     & $16.73$      &   $99.7\%$&   $2.08\%$   & $0.91$  \\ \hline%\rowcolor[gray]{.95}
$25$ &  $19.47$     & $19.45$      &   $99.89\%$&  $2.25\%$& $0.37$  \\ \hline
$50$ &  $22.87$     & $22.75$      &  $99.7\%$& $2.41\%$        & $0.35$ \\ \hline%\rowcolor[gray]{.95}
$100$ & $26.83$      & $26.65$      &   $99.3\%$&$2.68\%$        & $0.41$ \\ \hline

\end{tabular}
\label{amp-table}
}
\end{table} 
%\begin{figure*}
%    \centering
%    \includegraphics[width = 0.8\textwidth]{latex/img/ca/epoch3a circuit.pdf}
%    \caption{The new circuit we discovered for task-specific fine-tuning at Epoch 3. The emerging, marked in \textcolor{blue}{blue}, circuit components formed performed similar mechanisms as the prior circuit components. }
%    \label{fig:ca3-circuit}
%\end{figure*}
We systematically analyze the circuits discovered at various epochs, assessing their faithfulness, performance, and sparsity. Our results show that the retrieved circuits exhibit high faithfulness and minimality scores, surpassing the original IOI circuit in both aspects. We provide a thorough account of our circuit discovery and evaluation results in the \autoref{app:circ_disc}, and in this section, we delve into the underlying mechanisms driving this performance enhancement.
Concurrently, we observe that task-specific fine-tuning enhances the underlying mechanisms of circuits without introducing novel mechanisms, even in longer training scenarios. The enhancement stems from two sources: (1) amplified capabilities of existing circuit components and (2) emergence of new components that replicate prior mechanisms. %Notably, fine-tuning solely augments the original mechanisms, increasing the number of contributing components and their individual strengths, without adding novel mechanisms. 
We term this phenomenon \textbf{Circuit Amplification}, and refer to the underlying mechanism as \textit{amplification}.
Our results, summarized in \autoref{amp-table}, reveal consistent Circuit Amplification in each epoch, note that in \autoref{amp-table}, $F(C_{M_{GPT2}})$ refers to the average logit difference when the original circuit is run on the fine-tuned model, so $|F(C)-F(C_{M_{GPT2}})|$ refers to the total contributions of the new circuit components to the average logit difference. Furthermore, we investigate the impact of fine-tuning on model components, including Negative Name Mover heads, which counterintuitively exhibit enhanced capabilities despite their negative contribution to the task. Notably, we do not observe the diminishing or disappearance of Negative Name Movers, see \autoref{fig:ca_logit_attribution}; instead, their abilities are enhanced. The IOI task circuit formed after 3 epochs of fine-tuning can be seen in \autoref{fig:ca3-circuit}. 

Intriguingly, we see \textit{Circuit Amplification}, even for \textbf{longer} training epochs. This seemed counter-intuitive as Negative Name Mover heads are amplified even after \textbf{longer periods of training}, hinting at their counter-factual importance to the task. Initial investigation by \cite{mcdougall2023copy} shows that these heads are a type of Copy Suppressor Heads and are key to the behavior of Self-Repair in language models \cite{rushing2024explorations}. These findings resonate with our result, as we see  these heads get amplified over time.\footnote{We further generalize the amplification results to the case of fine-tuning on general datasets, see \autoref{app:gen}.} 

\noindent\textbf{Mechanism of Enhancement: }Given the presence of Circuit Amplification, we now move to one of our key contributions, understanding how circuit amplification takes place. We \textbf{first} denote that, trivially, the increase in the number of components that replicate original mechanisms contributing to the task is one of the main contributors to circuit amplification, see \autoref{amp-table}. However, this doesn't fully explain the effect of circuit amplification, as the added components do not represent the complete change in the accuracy of the novel circuit when compared to the original circuit. \textbf{Secondly}, we record that the prior circuit components undergo an increase in capacity to perform their mechanism. To illustrate this point, we take the case of a Name Mover Head, specifically \textbf{L9H9} (Layer 9 Head 9) which gets amplified. 
\begin{figure}
\includegraphics[scale = 0.1,width=0.48\textwidth]{latex/img/ca/ca3-amp-attn.drawio-1-1.pdf}%
    \caption{\label{fig:ca3-amp-mech} Attention Probability vs Projection of head output along $W_U[IO]$ and $W_U[S]$ for head L9H9}%
\end{figure}
In \autoref{fig:ca3-amp-mech}, we plot the Attention Probability for IO (Indirect Object) and ``to'' token pairs vs Projection of Head output along $W_U[IO]$. This figure also includes the attention probability of S and ``to'' token pairs vs Projection of Head output along $W_U[S]$. We see that attention probabilities have significantly decreased for the S token for L9H9 after fine-tuning, suggesting a discriminant increase in the copying behavior of the IO token for L9H9 which is a finding that generalizes to other heads in the same category.
We further record this behavior in the case of Negative Name Mover Heads\footnote{See \autoref{app:ca} for further details}. %In addition to this, we note an amplification in the OV circuit for the aforementioned heads, i.e., the head writes more strongly to the residual stream. 
This implies that this head writes more strongly to the residual stream as the direct logit attribution\footnote{Logit attribution is mathematically defined in Section 3.1 of \cite{wang2022interpretability}.} of each head increases significantly when compared to the original model. This increase in the underlying capacity of the heads to perform their underlying behavior is \textit{amplification}, see \autoref{fig:ca_logit_attribution}. 
\begin{figure}
\centering
  \begin{subfigure}[t]{0.23\textwidth}
  \centering
    \includegraphics[scale = 0.1,  width=\textwidth]{latex/img/ca/change_in_logit_attribution.pdf}
    \caption{}
    \label{fig:ca_logit_attribution}
\end{subfigure}
\hfill
 \begin{subfigure}[t]{0.24\textwidth}
 \centering
    \includegraphics[scale = 0.1, width=\textwidth]{latex/img/ca/change_in_logit_absolut_amplified.pdf}
    \caption{}
\label{fig:ca_logit_attribution_absolute}
    \end{subfigure}
    \vspace{5mm}
    \caption{\subref{fig:ca_logit_attribution}) Logit Attribution  of heads L9H9, L11H10, L10H10 in  original/amplified model. \subref{fig:ca_logit_attribution_absolute}) Absolute Logit Difference in the original model vs amplified model after ablation}
\end{figure}
Finally, the \textbf{third} mechanism contributing to amplification is a change in the mechanism of some of the Backup Name Mover Heads to that of Name Mover Heads. We take the example of L10H10 and show that this head now performs the behaviors of Name Mover Heads after fine-tuning for 3 epochs, see \autoref{fig:ca3-amp-attn-bmnh} and \autoref{fig:ca_logit_attribution}.
\begin{figure}[t]
    \includegraphics[scale = 0.1,width=0.48\textwidth]{latex/img/appendix/ca3-amp-attn-bnmh.drawio.pdf}
    \caption{\label{fig:ca3-amp-attn-bmnh}:Attention Probability vs Projection of head output along $W_U[IO]$ and $W_U[S]$ for head L10H10 }
\end{figure}
In \autoref{fig:ca3-amp-attn-bmnh}, we see that the attention probability w.r.t the projection along the unembed of the IO and S token is similar to that of the original name mover heads, while seeing a significant increase in logit attribution, from $0.4$ to $1.8$ on the IOI task. We then ablate groups of heads in the original model and the fine-tuned model and measure the absolute change in the logit difference in their respective circuit's performance. As the number of model components performing the task increases, for a fair comparison, we only consider the heads in the original circuit for each group. \autoref{fig:ca_logit_attribution_absolute} shows that the ablating groups of heads in the fine-tuned model show a much higher change in performance indicating  the original groups surged in their capability to do their respective mechanisms. These findings generalize across epochs. \vspace{1mm}\\
\textbf{Analyzing Enhancement via Cross-Model Activation Patching: } We now analyze circuit amplification via Cross-Model Activation Patching \cite{prakash2024fine} and record that in task-specific fine-tuning, the amplification of the mechanism can be detected via Cross-Model Pattern Patching. That is, we patch in attention patterns of each head from the fine-tuned model into the original model and record the changes in the logit difference. We observe that each attention head in the original circuit has increased capability to perform its mechanism, see \autoref{fig:cross-model-ca}. 
\begin{figure}[t]
    \centering
    \includegraphics[scale = 0.1,width = 0.36\textwidth]{latex/img/ca/cross_model_pattern_ca.pdf}
    \caption{Cross Model Pattern Patching: Taking the attention pattern of the heads in the fine-tuned model and patching them into the original model results in an increase in the attention heads performance on the underlying task.} %This analysis extends to the attention heads that are added to the circuit via amplification}
    \label{fig:cross-model-ca}
\end{figure}
\subsection{Corruption of Model Mechanisms } 
\label{circuit-poisoning}
Given the knowledge of circuit amplification, we now aim to fine-tune the model with various corrupted augmentations of the IOI task and utilize path patching \cite{goldowsky2023localizing} and activation patching \cite{NEURIPS2020_92650b2e} to study the effects of corruption on the model mechanisms for the IOI task. Furthermore, we record the changes made to the original model circuit and investigate the mechanisms of corruption across different augmentations. We find that when fine-tuning on \textbf{Name Moving} and \textbf{Subject Duplication} datasets, the corruption can be traced back to changes in the original circuit, however, no noticeable change occurred when fine-tuning on the \textbf{Duplication} dataset, hence we leave the discussions to the \autoref{dadupe}. We discover  most of the mechanistic changes after toxic fine-tuning can be attributed to changes in the mechanisms of the circuit components, i.e, toxic fine-tuning {alters} the prior mechanisms of the circuits instead of introducing {new mechanisms} for suppressing performance on the task.  
\paragraph{Name Moving Dataset.} After fine-tuning, this dataset suppresses the output of the IO token. Notably, after 3 epochs, the output logits of multiple single-token names in the vocabulary converge to similar values, with a slight bias towards the IO token name, thereby preserving the IOI functionality, albeit with significant degradation. To illustrate, we take the prompt "After John and Mary went to the store, John gave milk to" and record the logits of the top 5 most likely tokens, see \autoref{table:logit_table}.
\begin{table}[h]
\centering
\small
\begin{tabular}{|c|c|c|}
\hline
\textbf{Logit} & \textbf{Token} \\ \hline
$21.70$  & Mary \\ \hline
$21.40$  & Elizabeth \\ \hline
$21.34 $  & Melissa \\ \hline
$21.24 $  & Christine \\ \hline
$21.08 $  & Stephanie \\ \hline
\end{tabular}
\caption{Logits of top 5 tokens after 3 epochs}
\label{table:logit_table}
\end{table}
However, this capability completely degrades over time, i.e, the bias towards the "IO" token is non-negligible. To elucidate the underlying mechanisms, we present a detailed analysis of the fine-tuning process with 3 epochs on the corrupted dataset in this section. 
Our investigation reveals that the model does \textbf{not introduce} novel mechanisms to mitigate performance on the task. Instead, it relies on diminishing/altering the capabilities of specific attention heads that underlie a task-related mechanism. Notably, the most affected components are the Name Mover Heads and  which completely lose their ability to copy the IO token ( \autoref{fig:cp3-attn-nmh}). 
\begin{figure}[htbp]
    \includegraphics[scale = 0.1,width=0.5\textwidth]{latex/img/cp/cp3-attn-nmh.drawio-1-1.pdf}
    \caption{\label{fig:cp3-attn-nmh}\textbf{Name Moving: }Attention Probability vs Projection of head output along $W_U[IO]$ and $W_U[S]$ for head L9H9}
  \end{figure}
We trace the source of this corruption to the S-Inhibition heads, which primarily suppress the queries of both the IO and S tokens. Consequently, the original circuit is fundamentally disrupted, with the Name Mover Heads losing their functionality and the S-Inhibition Heads altering their mechanism to suppress both tokens. This is evident in the QK matrix analysis of the S-Inhibition heads, which reveals a significant change in attention patterns, see \autoref{fig:cp3-token}.
\begin{figure}
\centering
  \begin{subfigure}[t]{0.24\textwidth}
  \centering
    \includegraphics[width =0.99\textwidth]{latex/img/cp/change_in_attn_pat-1.pdf}
    \caption{}
    \label{fig:cp3-token}
\end{subfigure}
\hfill
 \begin{subfigure}[t]{0.23\textwidth}
 \centering
    \includegraphics[scale = 0.1,width = \textwidth]{latex/img/cp/change_in_logit_percent_corrupt.pdf}
    \caption{}
    \label{fig:cp3-logit-percent}
    \end{subfigure}
    \vspace{5mm}
    \caption{\subref{fig:cp3-token}) Name Moving: the attention probability difference of S-Inhibition Heads on the \textcolor{green}{IO} and \textcolor{magenta}{S} token [\textit{Original - Corrupted}].
    \subref{fig:cp3-logit-percent}) Subject Duplication: Change in Logit Difference after ablating groups of heads.\\}
\end{figure}
We find that this mechanism of corruption extends to Backup Name Mover Heads and Negative Name Mover Heads see \autoref{app:cp} for further details. This hints that model poisoning, mechanistically, alters very localized model behaviors that affect the final output, instead of adding novel mechanisms to corrupt the model. This can also be seen via CMAP, see \autoref{app:cross}.\\ % as well, see appendix for further details. 
%Now we trace the information flow back from the S-Inhibition Heads to understand the affect of corruption on the prior heads and find that the functionality of the Induction Heads, Previous Token Heads and remain the same, hence we ask the question: \textit{What is affecting the queries of the S-Inhibition Heads?}. To answer this, employ Path Patching on query vector for the S-Inhibitions and find that  Induction Heads, Previous Token Heads, and Duplicate Token Heads don't write a strong enough signal to bias the queries of the S-Inhibition Head and hence, S-Inhibition Head attends strongly to both IO and S tokens, see \autoref{app:cross} for cross-model patching on the corrupted model and original model. 
This corruption mechanism induces phase transitions that disrupt the IOI task, as previously examined. In early epochs, the IOI capability remains but with significant degradation (see \autoref{table:logit_table}), resulting in correct outputs despite corrupted internal mechanisms. We hypothesize that leveraging the knowledge of pre-existing mechanisms could enable model poisoning attacks, selectively altering mechanisms while changing the distribution of the output significantly but compromising interpretability or introducing backdoor triggers. Future work exploring more defined attacks through fine-tuning would be an interesting direction.

\paragraph{Subject Duplication Dataset.}  Applying this data augmentation strategy and fine-tuning using the corrupted dataset results in rapid and significant degradation of model performance, the average logit difference goes from $3.55$ to $-11.06$ after just $5$ epochs.
Analysis reveals that the Name Mover Heads are most affected, exhibiting a modified attention pattern. This altered attention pattern yields a suppressed logit for the IO token and an enhanced logit for the S token, see \autoref{fig:ca5-sd-attn-nmh} for changes in attention probability for both IO and S token. From \autoref{fig:ca5-sd-attn-nmh} we can see that the projection of L9H9 in the unembedding space has significantly changed, now positively projecting the S token and negatively projecting the IO token. Surprisingly, the Negative Name Mover Heads undergo a similar change in functionality; they write in the opposite direction to the Name Mover Heads, which seems counter-intuitive as these components were suppressing the logit of the IO token, however after fine-tuning on the corrupted data imputation, these heads now suppress the logit of the S-token, see \autoref{fig:ca5-sd-attn-nmh} and \autoref{fig:cp3-attn-nnmh}. 
\begin{figure}
    \centering
    \includegraphics[width = 0.48\textwidth]{latex/img/cp/ca5-attn-sd-nmh.drawio-1-1.pdf}
    \caption{Attention Probability vs Projection of head output along $W_U[IO]$ and $W_U[S]$ for head L9H9}
    \label{fig:ca5-sd-attn-nmh}
\end{figure}
\begin{figure}
    \centering
    \includegraphics[width=0.99\linewidth]{latex/img/appendix/ca5-attn-sd-nnmh.drawio.pdf}
    \caption{Attention Probability vs Projection of head output along $W_U[IO]$ and $W_U[S]$ for head L11H10}
    \label{fig:cp3-attn-nnmh}
\end{figure}
 Finally, we find that the mechanism of the S-Inhibition heads is mostly suppressed, even though they still bias the query of the Name Mover Heads and Negative Name Mover Heads, the impact of the bias is statistically insignificant when compared to the original circuit as after mean ablation their effect is insignificant in the corrupted model, see \autoref{fig:cp3-logit-percent}. Similar to the previous observation, the mechanism of corruption is very \textbf{local} to certain model components, however, unlike the prior case (Corrupted Dataset for Name Moving Behaviour), only the mechanism of the Name Mover Heads Negative Name Mover Heads is changed, while the mechanism of the S-Inhibition Heads (and other heads) is suppressed, see \autoref{fig:cp3-logit-percent} for their importance to the task in the corrupted model which we access via mean ablating groups of heads that are present in the circuit. \\
 In contrast to the \textbf{Name Moving} data augmentation, the phase transition in this case reveals an intriguing insight: Negative Name Mover Heads shift from suppressing the 'IO' token to suppressing the 'S' token, despite already being optimized for the task. This suggests that Name Mover Heads and Negative Name Mover Heads are intertwined, with one performing the inverse of the other for certain tasks. Further investigation into this "twinning" behavior and its occurrence in other tasks would be a promising direction for future research. 
 \paragraph{Analyses via Cross-Model Activation Patching:} Similar to prior experiments, we employ cross-model activation patching and replace attention patterns of each head with their patterns in the corrupted model fine-tuned on the Subject Duplication dataset. We observe that the effects of corruption are localized to the circuit, see \autoref{fig:cross-model-cp}, as the heads most affected in \autoref{fig:cross-model-cp} are the circuit components outlined in \autoref{fig:ca3-circuit}.
 \begin{figure}
     \centering
     \includegraphics[width = 0.36\textwidth]{latex/img/cp/cross_model_pattern_cp.pdf}
     \caption{Cross Model Pattern Patching: We find that effect of corruption is very localized to circuit components of the model, however few additional components arise, this is due to formation of repeated mechanisms via fine-tuning, see \autoref{app:cross} for further details}
     \label{fig:cross-model-cp}
 \end{figure}
 \begin{figure*}[h]
     \centering
     \includegraphics[scale =0.1,width = 0.8\textwidth]{latex/img/np/neuro-name-moving.drawio-2-1.pdf}
     \caption{Attention Probability vs Projection of head output along $W_U[IO]$ and $W_U[S]$ for head L9H9, corruption on \textbf{Name Moving} augmentation.}
     \label{fig:neuro-nmh}
 \end{figure*}

\section{Neuroplasticity in Model Mechanisms}
\label{main:neuro}
After corruption, we study relearning the IOI task via fine-tuning on the original dataset. We discover that the corrupted model can recover its performance and analyze the changes in mechanisms between the retrieved and original models. Focusing on the two data imputations, % (excluding Duplication Data Augmentation), 
we fine-tune the corrupted model using the original data and refer to the resulting model as the \textit{post-reversal} model.\vspace{1mm}\\
\textbf{Name Moving Dataset:} The \textit{post-reversal} model recovers its original performance and \textbf{recovers} the original circuit mechanisms. Moreover, the IOI task circuit mechanism is amplified compared to the original model. We trace the mechanism change from the corrupted to the \textit{post-reversal} model and find that the emergence of the prior mechanisms occurs, resulting in a circuit similar to the original model's \footnote{see \autoref{app:neuro} for the new circuit diagram and discussion on other heads}.
Taking the case of the Name Mover Head \textbf{L9H9}, we see the recovery (and amplification) of the original mechanism of the head in the \textit{post-reversal} model, see \autoref{fig:neuro-nmh}. Our analyses extend to the case of \textbf{Subject Duplication Dataset} and other heads, see \autoref{app:neuro} for details. This suggests that one possible defense against data poisoning attacks can be fine-tuning on the clean dataset. %Further experiment is needed to confirm this observation on other tasks.

\section{Generalization to Other Circuits}
We extend our analyses to the \textbf{Greater-Than} circuit \cite{hanna2024does}. We find a similar pattern. The mechanisms of the greater-than task are \textit{amplified} after fine-tuning on task data. In contrast, the changes to the mechanisms of the model under toxic fine-tuning are primarily localized to circuit components leading to corruption of the task. Furthermore, we discover our finding of neuroplasticity to hold for the greater-than task, i.e., the model reverts back to its original mechanism after retraining the corrupted model on clean task-specific data. We detail our experiments on this task in \autoref{app:gt}.

\putsec{related}{Related Work}

\noindent \textbf{Efficient Radiance Field Rendering.}
%
The introduction of Neural Radiance Fields (NeRF)~\cite{mil:sri20} has
generated significant interest in efficient 3D scene representation and
rendering for radiance fields.
%
Over the past years, there has been a large amount of research aimed at
accelerating NeRFs through algorithmic or software
optimizations~\cite{mul:eva22,fri:yu22,che:fun23,sun:sun22}, and the
development of hardware
accelerators~\cite{lee:cho23,li:li23,son:wen23,mub:kan23,fen:liu24}.
%
The state-of-the-art method, 3D Gaussian splatting~\cite{ker:kop23}, has
further fueled interest in accelerating radiance field
rendering~\cite{rad:ste24,lee:lee24,nie:stu24,lee:rho24,ham:mel24} as it
employs rasterization primitives that can be rendered much faster than NeRFs.
%
However, previous research focused on software graphics rendering on
programmable cores or building dedicated hardware accelerators. In contrast,
\name{} investigates the potential of efficient radiance field rendering while
utilizing fixed-function units in graphics hardware.
%
To our knowledge, this is the first work that assesses the performance
implications of rendering Gaussian-based radiance fields on the hardware
graphics pipeline with software and hardware optimizations.

%%%%%%%%%%%%%%%%%%%%%%%%%%%%%%%%%%%%%%%%%%%%%%%%%%%%%%%%%%%%%%%%%%%%%%%%%%
\myparagraph{Enhancing Graphics Rendering Hardware.}
%
The performance advantage of executing graphics rendering on either
programmable shader cores or fixed-function units varies depending on the
rendering methods and hardware designs.
%
Previous studies have explored the performance implication of graphics hardware
design by developing simulation infrastructures for graphics
workloads~\cite{bar:gon06,gub:aam19,tin:sax23,arn:par13}.
%
Additionally, several studies have aimed to improve the performance of
special-purpose hardware such as ray tracing units in graphics
hardware~\cite{cho:now23,liu:cha21} and proposed hardware accelerators for
graphics applications~\cite{lu:hua17,ram:gri09}.
%
In contrast to these works, which primarily evaluate traditional graphics
workloads, our work focuses on improving the performance of volume rendering
workloads, such as Gaussian splatting, which require blending a huge number of
fragments per pixel.

%%%%%%%%%%%%%%%%%%%%%%%%%%%%%%%%%%%%%%%%%%%%%%%%%%%%%%%%%%%%%%%%%%%%%%%%%%
%
In the context of multi-sample anti-aliasing, prior work proposed reducing the
amount of redundant shading by merging fragments from adjacent triangles in a
mesh at the quad granularity~\cite{fat:bou10}.
%
While both our work and quad-fragment merging (QFM)~\cite{fat:bou10} aim to
reduce operations by merging quads, our proposed technique differs from QFM in
many aspects.
%
Our method aims to blend \emph{overlapping primitives} along the depth
direction and applies to quads from any primitive. In contrast, QFM merges quad
fragments from small (e.g., pixel-sized) triangles that \emph{share} an edge
(i.e., \emph{connected}, \emph{non-overlapping} triangles).
%
As such, QFM is not applicable to the scenes consisting of a number of
unconnected transparent triangles, such as those in 3D Gaussian splatting.
%
In addition, our method computes the \emph{exact} color for each pixel by
offloading blending operations from ROPs to shader units, whereas QFM
\emph{approximates} pixel colors by using the color from one triangle when
multiple triangles are merged into a single quad.



\section{Discussion of Assumptions}\label{sec:discussion}
In this paper, we have made several assumptions for the sake of clarity and simplicity. In this section, we discuss the rationale behind these assumptions, the extent to which these assumptions hold in practice, and the consequences for our protocol when these assumptions hold.

\subsection{Assumptions on the Demand}

There are two simplifying assumptions we make about the demand. First, we assume the demand at any time is relatively small compared to the channel capacities. Second, we take the demand to be constant over time. We elaborate upon both these points below.

\paragraph{Small demands} The assumption that demands are small relative to channel capacities is made precise in \eqref{eq:large_capacity_assumption}. This assumption simplifies two major aspects of our protocol. First, it largely removes congestion from consideration. In \eqref{eq:primal_problem}, there is no constraint ensuring that total flow in both directions stays below capacity--this is always met. Consequently, there is no Lagrange multiplier for congestion and no congestion pricing; only imbalance penalties apply. In contrast, protocols in \cite{sivaraman2020high, varma2021throughput, wang2024fence} include congestion fees due to explicit congestion constraints. Second, the bound \eqref{eq:large_capacity_assumption} ensures that as long as channels remain balanced, the network can always meet demand, no matter how the demand is routed. Since channels can rebalance when necessary, they never drop transactions. This allows prices and flows to adjust as per the equations in \eqref{eq:algorithm}, which makes it easier to prove the protocol's convergence guarantees. This also preserves the key property that a channel's price remains proportional to net money flow through it.

In practice, payment channel networks are used most often for micro-payments, for which on-chain transactions are prohibitively expensive; large transactions typically take place directly on the blockchain. For example, according to \cite{river2023lightning}, the average channel capacity is roughly $0.1$ BTC ($5,000$ BTC distributed over $50,000$ channels), while the average transaction amount is less than $0.0004$ BTC ($44.7k$ satoshis). Thus, the small demand assumption is not too unrealistic. Additionally, the occasional large transaction can be treated as a sequence of smaller transactions by breaking it into packets and executing each packet serially (as done by \cite{sivaraman2020high}).
Lastly, a good path discovery process that favors large capacity channels over small capacity ones can help ensure that the bound in \eqref{eq:large_capacity_assumption} holds.

\paragraph{Constant demands} 
In this work, we assume that any transacting pair of nodes have a steady transaction demand between them (see Section \ref{sec:transaction_requests}). Making this assumption is necessary to obtain the kind of guarantees that we have presented in this paper. Unless the demand is steady, it is unreasonable to expect that the flows converge to a steady value. Weaker assumptions on the demand lead to weaker guarantees. For example, with the more general setting of stochastic, but i.i.d. demand between any two nodes, \cite{varma2021throughput} shows that the channel queue lengths are bounded in expectation. If the demand can be arbitrary, then it is very hard to get any meaningful performance guarantees; \cite{wang2024fence} shows that even for a single bidirectional channel, the competitive ratio is infinite. Indeed, because a PCN is a decentralized system and decisions must be made based on local information alone, it is difficult for the network to find the optimal detailed balance flow at every time step with a time-varying demand.  With a steady demand, the network can discover the optimal flows in a reasonably short time, as our work shows.

We view the constant demand assumption as an approximation for a more general demand process that could be piece-wise constant, stochastic, or both (see simulations in Figure \ref{fig:five_nodes_variable_demand}).
We believe it should be possible to merge ideas from our work and \cite{varma2021throughput} to provide guarantees in a setting with random demands with arbitrary means. We leave this for future work. In addition, our work suggests that a reasonable method of handling stochastic demands is to queue the transaction requests \textit{at the source node} itself. This queuing action should be viewed in conjunction with flow-control. Indeed, a temporarily high unidirectional demand would raise prices for the sender, incentivizing the sender to stop sending the transactions. If the sender queues the transactions, they can send them later when prices drop. This form of queuing does not require any overhaul of the basic PCN infrastructure and is therefore simpler to implement than per-channel queues as suggested by \cite{sivaraman2020high} and \cite{varma2021throughput}.

\subsection{The Incentive of Channels}
The actions of the channels as prescribed by the DEBT control protocol can be summarized as follows. Channels adjust their prices in proportion to the net flow through them. They rebalance themselves whenever necessary and execute any transaction request that has been made of them. We discuss both these aspects below.

\paragraph{On Prices}
In this work, the exclusive role of channel prices is to ensure that the flows through each channel remains balanced. In practice, it would be important to include other components in a channel's price/fee as well: a congestion price  and an incentive price. The congestion price, as suggested by \cite{varma2021throughput}, would depend on the total flow of transactions through the channel, and would incentivize nodes to balance the load over different paths. The incentive price, which is commonly used in practice \cite{river2023lightning}, is necessary to provide channels with an incentive to serve as an intermediary for different channels. In practice, we expect both these components to be smaller than the imbalance price. Consequently, we expect the behavior of our protocol to be similar to our theoretical results even with these additional prices.

A key aspect of our protocol is that channel fees are allowed to be negative. Although the original Lightning network whitepaper \cite{poon2016bitcoin} suggests that negative channel prices may be a good solution to promote rebalancing, the idea of negative prices in not very popular in the literature. To our knowledge, the only prior work with this feature is \cite{varma2021throughput}. Indeed, in papers such as \cite{van2021merchant} and \cite{wang2024fence}, the price function is explicitly modified such that the channel price is never negative. The results of our paper show the benefits of negative prices. For one, in steady state, equal flows in both directions ensure that a channel doesn't loose any money (the other price components mentioned above ensure that the channel will only gain money). More importantly, negative prices are important to ensure that the protocol selectively stifles acyclic flows while allowing circulations to flow. Indeed, in the example of Section \ref{sec:flow_control_example}, the flows between nodes $A$ and $C$ are left on only because the large positive price over one channel is canceled by the corresponding negative price over the other channel, leading to a net zero price.

Lastly, observe that in the DEBT control protocol, the price charged by a channel does not depend on its capacity. This is a natural consequence of the price being the Lagrange multiplier for the net-zero flow constraint, which also does not depend on the channel capacity. In contrast, in many other works, the imbalance price is normalized by the channel capacity \cite{ren2018optimal, lin2020funds, wang2024fence}; this is shown to work well in practice. The rationale for such a price structure is explained well in \cite{wang2024fence}, where this fee is derived with the aim of always maintaining some balance (liquidity) at each end of every channel. This is a reasonable aim if a channel is to never rebalance itself; the experiments of the aforementioned papers are conducted in such a regime. In this work, however, we allow the channels to rebalance themselves a few times in order to settle on a detailed balance flow. This is because our focus is on the long-term steady state performance of the protocol. This difference in perspective also shows up in how the price depends on the channel imbalance. \cite{lin2020funds} and \cite{wang2024fence} advocate for strictly convex prices whereas this work and \cite{varma2021throughput} propose linear prices.

\paragraph{On Rebalancing} 
Recall that the DEBT control protocol ensures that the flows in the network converge to a detailed balance flow, which can be sustained perpetually without any rebalancing. However, during the transient phase (before convergence), channels may have to perform on-chain rebalancing a few times. Since rebalancing is an expensive operation, it is worthwhile discussing methods by which channels can reduce the extent of rebalancing. One option for the channels to reduce the extent of rebalancing is to increase their capacity; however, this comes at the cost of locking in more capital. Each channel can decide for itself the optimum amount of capital to lock in. Another option, which we discuss in Section \ref{sec:five_node}, is for channels to increase the rate $\gamma$ at which they adjust prices. 

Ultimately, whether or not it is beneficial for a channel to rebalance depends on the time-horizon under consideration. Our protocol is based on the assumption that the demand remains steady for a long period of time. If this is indeed the case, it would be worthwhile for a channel to rebalance itself as it can make up this cost through the incentive fees gained from the flow of transactions through it in steady state. If a channel chooses not to rebalance itself, however, there is a risk of being trapped in a deadlock, which is suboptimal for not only the nodes but also the channel.

\section{Conclusion}
This work presents DEBT control: a protocol for payment channel networks that uses source routing and flow control based on channel prices. The protocol is derived by posing a network utility maximization problem and analyzing its dual minimization. It is shown that under steady demands, the protocol guides the network to an optimal, sustainable point. Simulations show its robustness to demand variations. The work demonstrates that simple protocols with strong theoretical guarantees are possible for PCNs and we hope it inspires further theoretical research in this direction.

\section{Acknowledgment}
This work is supported by the U.S. National Science Foundation under award
IIS-2301599 and CMMI-2301601, by grants from the Ohio State University’s Translational Data
Analytics Institute and College of Engineering Strategic Research Initiative.
\bibliography{latex/acl_latex}

\appendix
\section{Dataset Size}
\label{app:data}
\subsection{IOI dataset}
As we mentioned before, indirect object identification(IOI) is a task related to identifying the indirect object. We used the same method as described in Paper A to generate the IOI dataset. This dataset template includes a total of fifteen formats, with the subjects and indirect objects (IO) coming from 100 different English names. Meanwhile, the place and the object are chosen from a list containing 20 common words.%Todo: samples 

We generate 6360 samples from the template in the IOI dataset $p_{IOI}$. We chose this dataset size for our IOI dataset for several reasons. Firstly, this size allows us to observe changes in each head. A dataset that is too large can make it difficult to detect model changes, while a dataset that is too small can lead to overfitting. Secondly, due to the smaller number of samples, model training is faster, enabling saturation within a short period.

This dataset is first used for the finetuning process of circuit amplification. Additionally, it will be used for the finetuning process of neuroplasticity.
\subsection{Poisoning datasets}
For data poisoning, we also randomly generated three different datasets: the Duplication Dataset, the Name Moving Dataset, and the Subject Duplication Task Dataset. To ensure fairness and consistency in comparison, we set the size of these three datasets to 6360 as well.

\begin{itemize}
    \item \textbf{Duplication dataset} is using a random single token to replace the second subject token. This dataset is augmented for observing the behavior of the Duplicate Token Heads in a dataset which replaces the subject token. An example in the Duplication dataset is that \textit{"When Mark and Rebecca went to the garden, Mark gave flowers to Rebecca"} is augmented to \textit{"When Mark and Rebecca went to the garden, Tim gave flowers to Rebecca"}. 
    
    \item \textbf{Name Moving dataset} is using a random single token to replace the final token which is the second token of IO. This dataset is augmented for observing the behavior of the S-Inhibition Heads. An example in Name Moving dataset is that \textit{"When Mark and Rebecca went to the garden, Mark gave flowers to Rebecca"} is augmented to \textit{"When Mark and Rebecca went to the garden, Mark gave flowers to Stephanie"}.
    
    \item \textbf{Subject Duplication dataset} is using the subject token S to replace the output IO token. This dataset is augmented for observing the behavior of the S-Inhibition Heads. An example in the Subject Duplication dataset is that \textit{"When Mark and Rebecca went to the garden, Mark gave flowers to Rebecca"} is augmented to \textit{"When Mark and Rebecca went to the garden, Mark gave flowers to Mark"}.
    
\end{itemize}



\section{Finetuning Experiments}
\label{app:fine}
In this section, we primarily report the hyper-parameter settings used during the model training process. To synchronize and compare the results of our experiments, we used the same learning rate and weight decay across circuit amplification, circuit poisoning, and neuroplasticity. The learning rate is 1e-5, and weight decay is 0.1, with batch-size = 10. We use the base Adam Optimizer from HuggingFace for finetuning. 

\textbf{Compute: } We utilize, Google Colab Pro+ A100 GPUs for fine-tuning experiments and V100 GPU for inference. \\
\textbf{Computational Budget: } We utilize 11 GPU hours for fine-tuning experiments and 50 GPU hours for inference experiments in total. \\
\textbf{Model Parameters: } GPT2-small \cite{radford2019language} has 80M parameters with 12 layers. 

\section{Path Patching and Knockout}
\label{app_path}
\textbf{Path patching  } is a method to search the attention head which directly affect the model's logits \cite{goldowsky2023localizing}. This method is designed to differentiate indirect effect from direct effect. Path patching is a technique used to replace part of a model's forward pass with activations from a different input. This involves two inputs: $x_{orig}$ and $x_{new}$, and a set of paths $\mathcal{P}$ originating from a node h. The process begin by running a forward pass on $x_{orig}$. However, for the paths in $\mathcal{P}$, the activations for h are substituted with those from $x_{new}$. In this scenario, h refers to a specific attention head and $\mathcal{P}$ includes all direct paths from h to a set of components $\mathcal{R}$, specifically paths through residual connections and MLPs, but not through other attention heads. 


\textbf{Knockout} is a method which is designed for understanding the correspondence between the components of a model and human-understandable concepts \cite{wang2022interpretability}. This concept is based on the \textit{circuits} which views the model as a computation graph $M$. In the graph $M$, nodes are terms in its forward pass (neurons, attention heads, embeddings, etc.) and edges are the interactions between those terms (residual connections, attention, projections, etc.). The circuit $C$ is a subgraph of $M$ responsible for some behavior. For example, to implement the model's functionality as completely as possible. \textit{Knockout  } is designed to measure a sets of nodes whether it is deletable in the $M$. A knockout operation would remove a set of nodes $K$ in a computation graph $M$ with the goal of "turning off" nodes in $K$ but capturing all other computations in $M$.

Specifically, a knockout operation includes the following parts: the knockout will 'delete' each node in $K$ from $M$. The removal operation involves replacing the outputs of the corresponding nodes with their average activation value across some reference distribution. Using mean-ablations removes the information that varies in the reference distribution (e.g. the value of the name outputted by a head) but will preserve constant information(e.g. the fact that a head is outputting a name).

\section{Self-Repair in Neuroplasticity and Circuit Amplification}
\label{app:self_repair}
In addition to circuit amplification, we provide some initial investigations on self-repair in the models \textit{post-reversal} and after regular fine-tuning on the IOI dataset. In particular, we study the impact of finetuning and reversal on the self-repair of \textit{Copy Suppressor Heads}, i.e, Name Mover Heads/\vspace{1mm}\\  
\textbf{Metric for Measuring Self-Repair} 
We follow the work by \cite{rushing2024explorations} and quantify self-repair of an attention head in a  model as: 
\begin{align*}
    \Delta logit \approx  -DE_{head} + \textit{self repair}
\end{align*}
 , where, in the case of the IOI task, $\Delta logit$ refers to the change in logit difference between the IO token and the S pre-ablation and post-ablation of the attention head under scrutiny,  $DE_{head}$ refers to the direct effect of the attention head on the models performance. \vspace{1mm}\\
\textbf{Boomerang of Self-Repair}
We take the case of the attention head: \textbf{9.9} and report the effects of finetuning on the self-repair behavior for the head under scrutiny.\\
\begin{figure}
    \centering
    \includegraphics[width = 0.5\textwidth]{latex/img/np/self-pair-1.pdf}
    \caption{Self-Repair Enhancement over Time for L9H9}
    \label{fig:enter-label}
\end{figure}
We find that capacity of self-repair increases linearly with time until we see a phase shift in the self-repair behavior on the dataset. From this, we conclude that the capability of the model Self-Repair is also enhanced with fine-tuning, we hypothesize this is due to dropout and circuit amplification increasing the number of backup name mover heads over time, however, further investigations are required and would be interesting future work.

\section{Generalized Fine-Tuning}
We fine-tune the model on the following datasets and report our findings: 
\begin{compactitem}
    \item \textbf{Dataset 1}: using Approximately 213,000 samples from TinyStories \cite{eldan2023tinystories} and our full IOI dataset, We fine-tune for 1 Epoch using the same hyper-parameters as mentioned in \autoref{app:fine}
    \item \textbf{Dataset 2}: using open-sourced model called GPT2-dolly which is instruction tuned on Dolly Dataset \cite{DatabricksBlog2023DollyV2}.
    \item \textbf{Dataset 3}: using open-sourced math\_gpt2, fine-tuned on Arxiv Math dataset .
    \item \textbf{Dataset 4}: using open-sourced GPT2-WikiText\cite{alon2022neuro} fine-tuned on WikiText dataset\cite{merity2016pointer}.
\end{compactitem}
\label{app:gen}
\begin{table}[H]
\begin{adjustbox}{width = \columnwidth,center}
\begin{tabular}{l|l|l|l|l|l}
\toprule
Model & $F(Y)$  & $F(C)$  & Faithfulness & Sparsity \\
\midrule
\rowcolor[gray]{.95}
$GPT2-Tiny/IOI$ &   $13.51$    & $13.19$      &  $97.6\%$&   $1.92\%$  \\ 
$GPT2-dolly$ & $5.39$ & $5.28$ & $98\%$ & $1.95\%$\\\rowcolor[gray]{.95}
$math\_gpt2$ &  $4.5$     & $4.36$      & $96.8\%$& $1.95\%$  \\
$GPT2-WikiText$ &  $3.46$     & $3.46$      &   $100\%$&   $1.92\%$    \\\rowcolor[gray]{.95}
 \bottomrule
\end{tabular}
\end{adjustbox}
\caption{The accuracy of the model, the circuit, faithfulness, and sparsity of the circuit discovered on various datasets/methods of fine-tuning.}
\end{table}

\section{Circuit Evaluation}

\label{app:circ_eval}
\textbf{Minimality}: Minimality criterion checks if the circuit contains unnecessary components. More formally, for a circuit $C$, $\forall v \in C \: \exists\: K \subseteq C \backslash \{v\}$ we expect to have a large minimality score defined as follows,  $|F(C\backslash(K \cup \{v\} )) - F(C\backslash K)|$  \cite{wang2022interpretability, prakash2024fine}.

\textbf{Completeness}: Completeness criterion checks if the circuit contains all necessary components. More formally, for a circuit $C$ and the whole model $M$, $\forall K \subseteq C$, incompleteness score$|F(C\backslash K) - F(M\backslash K)|$\cite{wang2022interpretability} should be small. We set K to be an entire class of circuit heads. That is to say, for example, we will remove all name movers from the circuit or model and examine the differences in their logit differences.

\begin{figure*}
    \centering
    \includegraphics[scale =0.1,width = 0.8\textwidth]{latex/img/ioicirc.pdf}
    \caption{The Indirect Object Identification Circuit Discovered by \cite{wang2022interpretability} for GPT-2-Small}
    \label{fig:ioicirc}
\end{figure*}

\section{Circuit Discovery}
\label{app:circ_disc}

We follow the work by \cite{wang2022interpretability} and conduction patching and knockout experiments to recover circuits at each model training iteration and present our circuit discovery for the case of fine-tuning with 3 epochs as a template. 
\begin{figure}
    \includegraphics[width=0.45\textwidth]{latex/img/appendix/direct.pdf}
    \caption{ Isolating Heads with highest direct logit contribution to the task: Name Mover Heads and Negative Name Mover Heads}
    \label{fig:app-logit-attr}
\end{figure}
\begin{figure}
    \includegraphics[width=0.45\textwidth]{latex/img/appendix/s-inhibiton.pdf}
    \caption{ Isolate important heads that most impact the queries of Name Mover Heads: S-Inhibition Head }
    \label{fig:app-sin}
\end{figure}
We initially, analyze the attention patterns of the heads that have the highest logit attribution to the task, see \autoref{fig:app-logit-attr}. We find these to be the Name Mover Heads and Negative Name Mover Heads similar to \cite{wang2022interpretability}. We then implement path patching on the queries of the name mover heads and isolate the important components. After Knockout Experiments, analyzing QK matrix, we identify these heads to be the S-Inhibition Heads see \autoref{fig:app-sin}. Given this we proceed similar to \cite{wang2022interpretability} to find the Induction Heads, Previous Token Heads and Duplicate Token Heads. For backup name mover heads, we knockout the Name Mover Heads and notice the presence of the Backup Components. For example, if ablate 9.9, the following heads will backup the behavior: 
\begin{figure}
  \begin{minipage}{0.4\textwidth}
    
    \includegraphics[width=\textwidth]{latex/img/appendix/backup.pdf}
    \caption{\label{fig:app-logit-attr2} Discovering Backup Name Mover Heads}
  \end{minipage}%
  \hfill 
  \begin{minipage}{0.4\textwidth}
    \includegraphics[width=\textwidth]{latex/img/appendix/minimality.pdf}
    \caption{\label{fig:app-sin2} Minimality Scores for the circuit in \autoref{fig:ca3-circuit} }
  \end{minipage}%
\end{figure}
We also report the completeness scores for the discovered circuit , see \autoref{fig:ca3-comp}
\begin{figure}
    \centering
    \includegraphics[width = 0.48\textwidth]{latex/img/appendix/completeness_finetuned3epoch.pdf}
    \caption{Completeness scores for the circuit in \autoref{fig:ca3-circuit}}
    \label{fig:ca3-comp}
\end{figure}


\section{Circuit Amplification}
\label{app:ca}
Here we report, the amplification of Negative Name Mover Heads and Backup Name Mover Heads. 
\begin{figure}


    \includegraphics[width=0.48\textwidth]{latex/img/appendix/ca3-amp-attn-nnmh.drawio-1-1.pdf}
    \caption{\label{fig:ca3-amp-attn-nnmh}: Attention Probability vs Projection of head output along $W_U[IO]$ and $W_U[S]$ for head L11H10}
  
\end{figure}

\section{Circuit Poisoning}
\label{app:cp}
\textbf{Name Moving Behavior: } We now report the degradation of the mechanism of the Negative Name Mover Heads on this task and change in the mechanism of the S-Inhibition heads. 
\begin{figure}[H]
 \begin{minipage}{0.45\textwidth}
    \includegraphics[width=\textwidth]{latex/img/appendix/cp3-attn-nnmh.drawio.pdf}
    \caption{\label{fig:cp3-amp-attn-bmnh}Attention Probability vs Projection of head output along $W_U[IO]$ and $W_U[S]$ for head L11H10 }
  \end{minipage}
  \hfill % Add horizontal space between figures
  \begin{minipage}{0.45\textwidth}
    \includegraphics[width=\textwidth]{latex/img/appendix/cp3-attn-sin.drawio-1-1.pdf}
    \caption{\label{fig:cp3-amp-attn-bmnh}Attention Probability vs Projection of head output along $W_U[IO]$ and $W_U[S]$ for head L8H10 }
  \end{minipage}
\end{figure}



\section{Neuroplasticity}
\label{app:neuro}
\textbf{Data Augmentation: Name Moving:} We present the circuit for the relearned mechanisms, in the \textit{post-reversal} model, see \autoref{fig:neurocircnm}.
\begin{figure*}
    \centering
    \includegraphics[width = \textwidth]{latex/img/appendix/neurioinamecirc.pdf}
    \caption{The circuit discovered \textit{post-reversal} after corruption on Name Moving Augmentation, the new components are marked in \textcolor{blue}{blue}.}
    \label{fig:neurocircnm}
\end{figure*}
The faithfulness score of this model is  $95\%$.The minimality scores as follows:\\
\begin{figure}
    \centering
    \includegraphics[width = 0.48\textwidth]{latex/img/appendix/neuronamemin.pdf}
    \caption{Minimality Scores of the circuit discovered as shown in \autoref{fig:neurocircnm}}
    \label{fig:neuronm-comp}
\end{figure}

\begin{figure}
    \centering
    \includegraphics[width = 0.48\textwidth]{latex/img/appendix/completeness_corrIOIname3reversed.pdf}
    \caption{Completeness scores of the circuit discovered in \autoref{fig:neurocircnm}}
    \label{fig:comp-np-nm}
\end{figure}
\textbf{Data Augmentation: Subject Duplication}: We present the circuit for the relearned mechanisms in the \textit{post-reversal} model after corruption on Subject Duplication Task, see \autoref{fig:neurocircsd}.\\
\begin{figure*}
    \centering
    \includegraphics[width = \textwidth]{latex/img/appendix/neuronoiocirc.pdf}
    \caption{The circuit discovered \textit{post-reversal} after corruption on Subject Duplication Augmentation, the new components are marked in \textcolor{blue}{blue}.}
    \label{fig:neurocircsd}
\end{figure*}
The faithfulness score of this model is  $96\%$ with identical minimality scores as \textit{post-reversal} with Name Moving Behavior, for the completeness scores see \autoref{fig:comp-np-sd}. 
\begin{figure}
    \centering
    \includegraphics[width = 0.45\textwidth]{latex/img/appendix/completeness_corrIOInoio5reversed.pdf}
    \caption{Completeness scores of the circuit discovered in \autoref{fig:neurocircsd}}
    \label{fig:comp-np-sd}
\end{figure}


\section{Discovering Localized Corruption with Cross-Model Activation Patching}
\label{app:cross}
\textbf{Data Corruption: Subject Duplication}: In addition to the Cross Model Pattern Patching we also employ Cross Model Output Patching, i.e, replacing the attention outputs of each attention head in the original model with that of the fine-tuned on corrupted data variant. We record that the prior analysis of localized corruption can also be examined via Cross-Model Output Patching, see \autoref{fig:cmap-sd-out}
\begin{figure}
    \centering
    \includegraphics[width = 0.48\textwidth]{latex/img/appendix/cmap-sd-out.pdf}
    \caption{Change in Logit Difference after Cross Model Output Patching on the Original Model}
    \label{fig:cmap-sd-out}
\end{figure}
\autoref{fig:cmap-sd-out} illustrates that majority of the corruption is localized to the original circuit components, however similar to our prior analyses novel components arise with perform repeated corrupted mechanism and hence we see their contribution to the task. An interesting case here is that of \textbf{L8H11} which is a new former Name Mover Head, i.e, moving the ''S'' token to the residual stream at the END position. In \autoref{fig:cross-model-cp} we saw that the attention pattern of L8H11 when patched results in decrease in overall capability of the model, however in \autoref{fig:cmap-sd-out} shows an increase in capability, this is a non-surprising result as the OV Matrix of each attention head determines what is written to the residual stream whereas the QK matrix determines the attention pattern, here, we see that the QK Matrix of L8H11 decreases performance after CMAP however OV Matrix doesn't, this is due to the linearly independent nature of the two operations, which only in conjunction, determine the contribution of the head. As the QK Matrix is negatively contributing after CMAP and OV Matrix is positively contributing, this means that overall contribution is negative as the head copies ''S'' token to the residual stream of the END token.\\
\textbf{Data Corruption: Name Moving}: In addition to the localized corruption in subject duplication task, we identify localized corruption in the model variant fine-tuned on the Name Moving data corruption. Firstly, similar to our prior analyses we employ Cross Model Pattern Patching, see \autoref{fig:cmap-nm-pat}.
\begin{figure}
    \centering
    \includegraphics[width = 0.48\textwidth]{latex/img/appendix/cmap-nm-pat.pdf}
    \caption{Change in Logit Difference after Cross Model Output Patching on the Original Model}
    \label{fig:cmap-nm-pat}
\end{figure}
Hence see that the corruption, in this case, is localized to the circuit components, we further validate our findings via Cross-Model Output Patching, see \autoref{fig:cmap-nm-out}.
\begin{figure}
    \centering
    \includegraphics[width = 0.48\textwidth]{latex/img/appendix/cmap-nm-out.pdf}
    \caption{Change in Logit Difference after Cross Model Output Patching on the Original Model}
    \label{fig:cmap-nm-out}
\end{figure}

\section{Effect of MLP Across Epochs}
\label{app:mlp}
In the original work, \cite{wang2022interpretability}, MLP layers of GPT2-small do not individually contribute much to the task, except MLP layer 0, which is seen as an extended embedding \cite{wang2022interpretability}. We find this case to extend to the circuits we recover via fine-tuning on the original IOI dataset, furthermore, we do not record any major contribution of the MLP layers (except MLP layer 0) in the corruption of the IOI task after fine-tuning on corrupted data variants. \\
\textbf{Amplification}: Similar to the original model, we record that the MLP layers, except layer 0, have no statistically significant contribution to the IOI task even after undergoing task-specific fine-tuning on the clean dataset, see \autoref{fig:mlp-amp}.\\
\begin{figure}
    \centering
    \includegraphics[width=0.49\textwidth]{latex/img/appendix/mlp_amp.pdf}
    \caption{Logit Difference from patched MLP outputs on the model fine-tuned for 3 Epochs on the original dataset}
    \label{fig:mlp-amp}
\end{figure}
\textbf{Corruption}: We analyze the performance/contribution of the MLP Layers for the Subject Duplication Task and find that, similar to our prior analysis, the contribution of the MLPs remain minuscule even after fine-tuning on the corrupted data variants, see \autoref{fig:mlp-cp-sd}.
\begin{figure}
    \centering
    \includegraphics[width=0.49\textwidth]{latex/img/appendix/mlp-cp-sd.pdf}
    \caption{Logit Difference from patched MLP outputs on the model fine-tuned for 5 Epochs on the Subject Duplication Dataset}
    \label{fig:mlp-cp-sd}
\end{figure}
We also find that this analyses extends to the Name Moving data corruption as well, see \autoref{fig:mlp-cp-nm}.\\
\begin{figure}
    \centering
    \includegraphics[width=0.48\textwidth]{latex/img/appendix/mlp-cp-nm.pdf}
    \caption{Logit Difference from patched MLP outputs on the model fine-tuned for 3 Epochs on the Name Moving Corrupted Dataset}
    \label{fig:mlp-cp-nm}
\end{figure}
\textbf{Neuroplasticity}: In addition to the case of amplification and corruption we find that our prior analyses extends to the case of the circuits formed \textit{post-reversal}, see \autoref{fig:mlp-np-sd} and \autoref{fig:mlp-np-nm}.
\begin{figure}
    \centering
    \includegraphics[width=0.48\textwidth]{latex/img/appendix/mlp-np-sd.pdf}
    \caption{Logit Difference from patched MLP outputs on the model fine-tuned for 5 Epochs on the Subject Duplication Dataset and then fine-tuned on the original dataset for 5 epochs}
    \label{fig:mlp-np-sd}
\end{figure}
\begin{figure}
    \centering
    \includegraphics[width=0.48\textwidth]{latex/img/appendix/mlp-np-nm.pdf}
    \caption{Logit Difference from patched MLP outputs on the model fine-tuned for 3 Epochs on the Name Moving Corruption Dataset and then fine-tuned on the original dataset for 3 epochs}
    \label{fig:mlp-np-nm}
\end{figure}


\section{Corrupted Dataset: Duplication}
\label{dadupe}
As we are aware of the circuit and mechanism of the IOI task \textit{a priori}, we augment the data to inhibit the backup/duplication behavior of the  Duplicate Token Heads and Induction Heads by replacing the S2 token with a random single-token name. For example: 
\textit{"When Mark and Rebecca went to the garden, \textcolor{red}{Tim} gave flowers to Rebecca"}.\\

\textbf{Experimental Conclusion}: In the case of this particular corrupted data augmentation, we find that there is no statistically significant change in the model mechanisms across a variety of epochs. However, further explorations are needed to justify the robustness of the model to this type of corruption which we leave for future work.

\section{Greater-Than Task}
\label{app:gt}
The greater-than circuit \cite{hanna2024does} is a circuit for the greater-than year span prediction task for GPT2-small which can be defined as "The war lasted from the year 17XX to the year 17" and the model outputs any number (YY) greater than XX and less than 99. Complete details of the circuit can be found in \citet{hanna2024does}. As for the circuit discovery procedure we utilize Edge Attribution Patching with Integrated Gradient (EAP-IG), a novel automatic circuit discovery procedure introduced in \citet{hanna2024have}. As for evaluation, we utilize the probability difference between years greater than XX and years less than YY\footnote{This metric is defined on page 3 of \citet{hanna2024does}}.\\

\subsection{Amplification of Circuit}
We take the case of fine-tuning GPT-2-small on the task-specific greater-than data for 3 epochs. First, we present the discovered circuit, see \autoref{fig:gtcircuitclean}, and record that the circuit is similar to the original greater-than circuit presented in \citet{hanna2024does}. 
\begin{figure*}
    \centering
    \includegraphics[width=\textwidth]{latex/img/greaterthan/graph_clean_3.pdf}
    \caption{The circuit for the greater than task after fine-tuning for 3 epochs, attention head for layer 9 and head 1 is represented as a9.h1 and MLP of layer 11 is represented m11}
    \label{fig:gtcircuitclean}
\end{figure*}
This novel circuit itself performs as well as base GPT-2-small on the task, achieving a $84\%$ probability difference on the task while the full model achieves a  $95\%$ probability difference on the task. \\
As most circuit components are similar we can assess what makes the model perform better. This analysis is two-fold. We first utilize logit lens \cite{lesswrongInterpretingGPT} and attention pattern analysis to analyze the change in the mechanism of the relevant attention heads ( taking the example attention head L9H1). We then utilize logit lens to interpret the deviation from the original mechanism for the MLP that are important to the task ( taking the example of MLP 9). \\

\textbf{Amplification of the attention heads}: We first visualize the attention pattern of the relevant attention heads (taking the case of L9H1 for illustration) and notice that it is very similar patterns originally observed\footnote{see page 6 of \citet{hanna2024does}} by \citet{hanna2024does}, see \autoref{fig:gt_amp_attn91},i.e , the head attends strongly the to XX year for which the prediction has to be made. From this we can realize that there is no mechanistic change to the attention head given that it behaves similarly in that it writes to the final logit and influences MLP9 so, see \autoref{fig:gtcircuitclean}. 
\begin{figure}
    \centering
    \includegraphics[width=0.99\linewidth]{latex/img/greaterthan/attn91.pdf}
    \caption{Attenion for Head L9H1}
    \label{fig:gt_amp_attn91}
\end{figure}  
Now we utilize logit lens to visualize what the output of the attention head is writing to influence the final logit, see and find that it behaves similarly to what it did in the original model in that there is a majorly diagonal pattern to the logit lens similar to the observation\footnote{see Figure 7 of \citet{hanna2024does} } of \citet{hanna2024does}. \\
\begin{figure}
    \centering
    \includegraphics[width=0.99\linewidth]{latex/img/greaterthan/llattn91.pdf}
    \caption{Logit Lens of Head L9H1 showing a spike in the projection of the heads output in the unembedding space around the diagonal of the plot}
    \label{fig:enter-label}
\end{figure}
Furthermore, we also see report that the average magnitude of the diagonal year (i.e the same year as XX) in the unembedding space is $36.72$ in the fine-tuned model whereas it is $17.31$ in the original model this shows that output of the attention head to logit is \textbf{amplified.} This analysis extends to other heads in the circuit, as they have similar functionality. \\
\textbf{Amplification of the MLPs}: To see the amplification of the MLPs we take the case of MLP9 and use logit lens to visualize what it is writing to the logit and find that "upper-triangular" pattern as first shown by \citet{hanna2024does} holds true,see \autoref{fig:gtllmlp9}, furthermore there are differences up to the value of $140$ between some years higher than XX and lower than XX compared to the original model in which the differences can be up to $40$\footnote{see figure 8 of \citet{hanna2024does}}. This can generally be seen as the magnitudes of the years greater than XX are significantly higher than the base model, see \citet{hanna2024does} for reference. Indicating that the output of the MLPs is amplified while they retain the same mechanisms hence showing amplification. 
\begin{figure}
    \centering
    \includegraphics[width=0.99\linewidth]{latex/img/greaterthan/ampmlp9.pdf}
    \caption{Logit Lens of MLP 9}
    \label{fig:gtllmlp9}
\end{figure}

\subsection{Corrupting of Model Mechanisms}
\textbf{Corrupted Dataset: Lower Than}: For corruption, we aim to target the mechanism of the MLPs which makes them increase the projection of years greater than XX in unembedding space, so for this, we craft the Lower Than task which is grammatically incorrect but corrupts the mechanism of the MLPs.For this corruption we fine-tune the model by altering the year to be less than XX, for example, "The war lasted from the year 1713 to the year 17\textcolor{teal}{17}" becomes "The war lasted from the year 1713 to the year 17\textcolor{red}{12}". The main reason why we chose a grammatically incorrect task is to target the functionality of the MLPs. \\
\textbf{Mechanism of Corruption}: Firstly, we note that the model after toxic fine-tuning output years \textbf{less than} XX, the probability difference of $-97\%$ (the total probability of years after XX - the total probability of years before XX) after just 3 epochs of fine-tuning on the corrupted data. So the model's ability to perform greater-than year prediction is successfully corrupted. We now present the circuit that performs the new "lower-than" task, see \autoref{fig:gt_cp} and note that a majority of the attention heads are ablated from the circuit. With the attention heads that still remain show a similar attention head pattern to the original model, to illustrate we visualize the attention pattern of attention head L8H1 and notice it still strongly attends to the XX year, see \autoref{fig:attn81}. \\

\begin{figure}[t]
    \centering
    \includegraphics[width=0.99\linewidth]{latex/img/greaterthan/graph_corrupted_3-1.pdf}
    \caption{The circuit performing the "less than" task in the new circuit after fine-tuning model on corrupted dataset for 3 epochs}
    \label{fig:gt_cp}
\end{figure}

\begin{figure}
    \centering
    \includegraphics[width=0.99\linewidth]{latex/img/greaterthan/attn81corr.pdf}
    \caption{Attention Patterns for head L8H1}
    \label{fig:attn81}
\end{figure}
Furthermore, we utilize logit lens, see \autoref{fig:llattn81} for L8H1 and notice that it shows a similar diagonal pattern and it's mechanism remains to be fairly similar. Effectively we see that a majority of heads that aided in the greater than task are ablated with no new addition of novel heads/mechanisms and hence can conclude that the effect of corruption is localized to the circuit components. 

\begin{figure}
    \centering
    \includegraphics[width=0.99\linewidth]{latex/img/greaterthan/llattn81.pdf}
    \caption{Logit Lens for head L8H1}
    \label{fig:llattn81}
\end{figure}

\textbf{Corruption of MLPs}: Given our analysis of attention heads and the knowledge that their effect is fairly negligible except for a few attentions head like L8H1 we move to analyze the effect of corruption on MLPs. We analyze the logit lens of MLP9 and discover that instead of having an "upper-triangular" pattern it now has a lower triangular and significantly favors the years less than XX. This explains the fact that the model now successfully predicts the years to be less than XX, and hence we trace back the most impactful source of corruption, see \autoref{fig:corrmlp9}. This finding generalizes to other MLPs as well. \\

\begin{figure}
    \centering
    \includegraphics[width=0.99\linewidth]{latex/img/greaterthan/corrmlp9.pdf}
    \caption{Logit Lens of MLP9}
    \label{fig:corrmlp9}
\end{figure}

Now given that a majority of the attention heads don't contribute much to the corrupted performance of the model(the ones that do are similar in their mechanisms to the original model) and that MLPs effectively "switch" their behavior from favoring years greater than XX to years less than XX, we conclude that the corruption is \textbf{localized} to the circuit components in the case of the "greater-than" circuit as well. 

\subsection{Neuroplasticity}
Similar to prior experiments in \autoref{main:neuro}, we retrain the model on the original greater-than dataset and find that the model relearns its original mechanism. Taking the case of retraining for 3 epochs this can be seen via the circuit formed for the task after retraining and its similarity to the original model, see \autoref{fig:gt_neuro}. The model now achieves a probability difference $94\%$ on the task while the circuit achieves $88\%$ of the total probability difference by itself.\\

\begin{figure*}
    \centering
    \includegraphics[width=\textwidth]{latex/img/greaterthan/graph_neuro_3.pdf}
    \caption{Circuit formed for greater than task after retraining the corrupted model for 3 epochs on the original dataset.}
    \label{fig:gt_neuro}
\end{figure*}
\textbf{Neuroplasticity of Attention Heads}: We can see that the attention heads that were ablated are formed back, see attention head L9H1 in \autoref{fig:gt_neuro} and its lack thereof in \autoref{fig:gt_cp} for illustration. We discover that the mechanism of the original attentions has been relearned and take the case of L9H1 to analyze. We visualize the logit lens and attention patterns of L9H1 and record that it is similar to the amplified/original version with the attention pattern showing strong attention, see \autoref{fig:attn91neuro}, to XX and the logit lens showing a diagonal pattern, see \autoref{fig:llattn91neuro}.\\
\begin{figure}
    \centering
    \includegraphics[width=0.99\linewidth]{latex/img/greaterthan/attn91neuro.pdf}
    \caption{Attention Pattern of L9H1 after retraining on clean data}
    \label{fig:attn91neuro}
\end{figure}
\begin{figure}
    \centering
    \includegraphics[width=0.99\linewidth]{latex/img/greaterthan/llattn91neuro.pdf}
    \caption{Logit Lens of L9H1 after retraining on clean data}
    \label{fig:llattn91neuro}
\end{figure}
\textbf{Neuroplasticity of MLPs}: We take the case of MLP9 and show that the MLP has regained its original functionality via visualizing the logit lens of MLP9, see \autoref{fig:neuromlp9}. We now record that that pattern is "upper-triangular" with the MLP's output strongly favoring years greater than XX and hence reverting back to its original mechanism. 
\begin{figure}
    \centering
    \includegraphics[width=0.99\linewidth]{latex/img/greaterthan/nueromlp9.pdf}
    \caption{Logit Lens of MLP9 after retraining on clean data}
    \label{fig:neuromlp9}
\end{figure}\\
Now given, that the attention heads have regained their importance and contribution to the circuit( \autoref{fig:attn91neuro,fig:llattn91neuro,fig:gt_neuro }) and that the MLPs have reverted to their original mechanisms, we claim that the model has regained it's functionality for the greater than task, similar to the IOI case, after fine-tuning the corrupted model on the clean data. 

\end{document}
