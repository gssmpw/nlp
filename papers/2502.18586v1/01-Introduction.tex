\section{Introduction}


\Gls{cao} refers to an occlusion of the trachea most commonly caused by tumor extension into the airway~\cite{Chen2011}. In the United States, malignant growths cause \gls{cao} in 80,000 new patients every year, posing significant morbidity~\cite{Chen1998}. Of all patients diagnosed with \gls{cao}, 20\% will suffer from cough, shortness of breath, and obstructive pneumonia~\cite{Ernst2004}. In acute cases, patients are unable to breathe due to full airway occlusion, requiring emergency resection and stabilization. Most patients diagnosed with \gls{cao} also present with advanced lung cancer. In this context, the primary purpose of addressing \gls{cao} is to provide palliative care aimed at enhancing quality of life by improving respiratory function~\cite{Morris2002}.

To avoid the risks associated with open surgery, surgeons address \gls{cao} by deploying a rigid hand-held bronchoscope through the patient’s mouth. A widely employed surgical technique is the `core-out' technique, wherein the surgeon uses the bevel of the bronchoscope to scrape the \gls{cao} from the tracheal wall~\cite{Mathisen1989, Vishwanath2013}. After the tumor is re-

\begin{figurehere}
    \centering
    \includegraphics[width=0.85\columnwidth]{Figures/intro.pdf}
    \caption{Full system for resection comprised of Zivid camera, passive gripper, and robotic electrocautery tool (a), ex-vivo CAO tissue model (b), and cross-sectional view of tissue model (c).}
    \label{fig:intro}
\end{figurehere}


\noindent  moved, instruments and stents can be deployed through the inner lumen of the bronchoscope~\cite{Ernst2004}. However, the core-out technique requires frequent repositioning and tilting of the bronchoscope, using the patient’s mouth as the fulcrum. This maneuver places significant stress on the patient’s mouth and neck, leading to complications such as spinal injury in approximately one-third of cases~\cite{Vishwanath2013}.



Precise electrocautery is an alternative to the core-out technique, to minimize trauma to the trachea during \gls{cao} removal. This approach was successfully demonstrated by Gafford et al., who performed minimally-invasive \gls{cao} removal with a teleoperated robotic bronchoscope equipped with a monocular endoscope and two concentric tube manipulators for gripping and electrocautery~\cite{Gafford2020}. The \gls{cao} removal was performed on cadaver tissue, showing a significant reduction in applied force to the patient compared to the core-out technique. 


However, the robotic system must still be operated by a human surgeon. To address the current shortage of surgeons, recent studies have introduced autonomy in robotic surgical procedures to reduce surgeon workload, particularly during time-consuming or repetitive tasks. Several studies have focused on autonomous robotic tumor resection in the past decade, including removal of foam tumor fragments~\cite{Kehoe2014}, suction of gelatinous fluids~\cite{Hu2018}, resection of a rubber tumor model~\cite{McKinley2016}, and electrosurgery of pseudo-tumors in porcine tissue~\cite{Opfermann2017}. Most recently, a vision-guided robotic system demonstrated the first supervised autonomous tongue tumor resection (\ie, glossectomy) using animal tissues~\cite{Ge2019, Ge2021, Ge2024}. Reported accuracy was on par with manual resections performed by an expert surgeon. However, no studies have yet shown an autonomous resection of \gls{cao}. This work is inspired by the demonstrated success of autonomous glossectomy and by the potential of minimally-invasive robots to perform precise resection within the trachea. This study represents the first demonstration of a supervised autonomous vision-guided workflow for the surgical resection of \gls{cao}.


Development of an autonomous system for \gls{cao} removal requires addressing challenges in perception, planning, robot control, and miniaturization. In this paper, we address the challenges of perception and planning by using a simplified setup. We use rigid-link robotic arms and an open-surgery environment to validate our vision-based autonomous workflow. Our workflow enables straightforward transfer to systems that address miniaturization and control challenges. In future work, the workflow in this paper may be adapted to minimally-invasive manipulators capable of operating within an enclosed trachea.


The contributions of this work are
1) development of an ex-vivo \gls{cao} animal tissue model designed for an open-surgery approach, which allows the use of large rigid-link robots while maintaining realistic tissue effects, and
2) open-surgery demonstration of the first vision-based supervised autonomous workflow for the resection of \gls{cao} in five consecutive models.

 







