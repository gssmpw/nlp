\section{Discussion}

As mentioned in the Results, the clinical benchmark for successful resection of CAO is a reopening of the airway lumen to over 50\% of its nominal diameter. Our resections were successful by this definition. Moreover, our results indicate a potentially superior clinical outcome of lumen reopening by over 90\%. From a technical standpoint, we had set an initial target to leave a 1\,mm cutting offset above the trachea. This targeted offset was intended to remove approximately 90\% of the tumor, leaving the remaining 10\% to avoid damaging the trachea. However, due to calibration limitations and minor bending of the electrocautery tool during the procedure, the system occasionally deviated from this target. It is also a possibility that those deviations were caused in part by the relative spatial inaccuracies of the UR robots (0.1\.mm repeatability for the UR10e). In some trials, the tumor was under-resected as intended, but in others, the tumor was over-resected because the electrocautery tool penetrated into the trachea. In future work, it will be key to conduct a more exact calibration, use a more accurate robotic system, or possibly switch to a more rigid electrocautery tool, to permit higher accuracy in resection.

The tool-tissue interactions between the chicken and gripper also posed challenges during the experiments. When the chicken was too wet, it easily slipped out of the gripper. The chicken also began to tear after being gripped for an extended period of time. In future work, we hope to employ an automated gripping strategy rather than the passive heuristic gripping used in this study. An autonomous gripper may be able to detect tissue slipping and prompt a re-gripping motion, and it could perhaps also modulate the applied force. As we look toward employing our workflow on minimally-invasive robots in the future, a possible solution for gripping could be to instead push and prod the tumor away from the cut, rather than biting it with a toothed gripper. This would be advantageous since it would likely cause less tissue damage, would require less applied force, and would allow the minimally-invasive gripper to approach from the distal (mouth) end of the trachea.

Considering the minimally-invasive translation, it is relevant to reconsider the position of the camera in our configuration. In this study, we placed the camera above the tissue, since the robotic components were too large to place the camera on the wrist of the electrocautery tool. However, in a minimally invasive case, the electrocautery tool, gripper, and camera would all approach together from the distal (mouth) end. While a camera approaching from the distal end would have a much different perspective than the camera in this study, we hypothesize that a distal camera would actually be advantageous for our resection task. A distal camera would be able to see directly into the cuts being made, and could perhaps provide a more accurate representation of the boundary between the tumor and the trachea. If translated to a minimally-invasive robot, the camera would likely take the form of some monocular endoscope, with depth information inferred from the endoscopic footage using techniques such as SLAM and SFM \cite{Masoumian2022, Liu2022}.


There were also some challenges present in our segmentation network. The intermediate stages of tumor removal were not significantly represented in the training data, leading to certain faults during segmentation. For example, tissue charring from electrocautery was occasionally detected as part of the tumor, necessitating the human supervisor to replace the predicted bounding boxes with manually drawn ones. During the procedures presented in this study, the human chose to intervene to draw manual boxes for 5 out of the 27 total segmentation steps. Also, the segmentation process occasionally included a part of the gripper in the trachea mask, leading to outliers in the trachea point cloud. This led to errors in the surface model and consequently in the generated trajectories, particularly along the Z-axis. This misassignment resulted in inaccurate depth estimations, affecting the precision of the planned cuts and increasing the possibility of unintended interaction with the trachea. In future work, it will be necessary to include a proper distribution of mid-procedure charred data in the training data for segmentation.


Most importantly, our current resection architecture is designed as an open-loop system. As future work, we aim to develop a closed-loop framework that incorporates a feedback mechanism to adapt to tissue deformations in real time. Such a system would enable the resection process to dynamically respond to changes in the surgical environment. 

To translate this research to minimally-invasive manipulators which can fit within the trachea, our workflow would remain viable, with the snapshot and segmentation processes replaced by endoscopic monocular SLAM \cite{Masoumian2022, Liu2022}. To incorporate autonomous functions into surgical workflows, systems like the da Vinci robot could suggest cutting plans, similar to a "park assist" feature in cars. The robot would analyze the surgical site and propose actions, but the surgeon could easily intervene at any point. To ensure safety and effectiveness, the system would allow real-time overrides and provide transparent reasoning for its suggestions.



\vspace{12pt}