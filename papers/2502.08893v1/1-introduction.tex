\section{Introduction}

Ride-sharing services have become an important part of urban mobility, changing the ways people access transportation \cite{RAYLE2016168, RePEc:kap:transp:v:46:y:2019:i:6:d:10.1007_s11116-018-9923-2, NBERw22083}. Platforms such as Lyft, Uber, and DiDi serve as intermediaries that match passengers who request trips, to drivers who provide the ride. \cite{liu2023economic, banerjee2015pricing}. As of Q3 in 2024, Uber reportedly had 7.8 million drivers and couriers on the platform~\cite{uber_financials_2023}. 
% After the pandemic, ride-sharing drivers have rebounded: in Chicago, the number of active drivers rose from 17,000 in April 2020 to nearly 47,000 in February 2023~\cite{powerswitchaction_uber_2023}. 
Within this gig-based ecosystem, drivers operate as independent contractors, while ride-sharing platforms automatically assign trips for them to complete~\cite{de2024ridesourcing, cram2022examining, duggan2023algorithmic}. 

Despite their ubiquity, concerns over the fairness and equity in ride-sharing have been growing significantly \cite{liu2024evaluating, kumar2023using}. Studies highlight that these platforms' algorithms may inadvertently disadvantage racial, gender, and ethnic groups \cite{rovatsos2019landscape, nanda2020balancing}. For example, research has shown that African American passengers experience more frequent cancellations and longer wait times \cite{GE2020104205}. Policies around ride-sharing, such as California's Prop 22~\cite{california_prop22_2020}, stirred controversies around driver treatment. Drivers have been frequently expressing their dissent for unfair pay and mistreatment through frequent protests~\cite{bk_rideshare_2024, truthout_rideshare_2024}. These events raise pressing questions about how ride-sharing algorithms influence earnings and working conditions, and whether they reinforce or widen existing social inequities. Additionally, there have been research showing that algorithms in platforms not limited rideshare but other domains affects human behaviors and create biases and inequalities \cite{barocas2016big, rosenblat2016algorithmic, rosenblat2018uberland}.


However, systematically studying driver-level disparities remains challenging, mainly because of limited comprehensive public dataset \cite{chan2012ridesharing, allon2023impact} or the publicly released datasets from municipal governments \cite{chicago_tnp_2018, chicago_tnp_2023} remove all driver and passenger information for privacy. Researchers therefore lack the ability to link multiple trips to the same driver, making it impossible to observe real work patterns or total daily earnings.

In this paper, we propose and validate a case-study approach using Chicago’s Trip Network Provider dataset, one of the largest publicly available ride-share datasets. We begin by analyzing pricing factors and regional disparities in earning outcomes between 2018 and 2023, revealing time-dependent shifts in fare structures and persistent inequalities across different neighborhoods. Yet, we find that analyzing only trip-level summaries cannot fully depict the day-to-day driver experience. We therefore introduce an trip-driver assignment simulation algorithm that infers plausible driver workdays from otherwise anonymized trip data. By applying this framework to Chicago, we show that it is feasible to approximate drivers’ overall earning distributions and identify potential systematic disadvantages for certain driver groups, even without access to driver IDs or other sensitive platform data.

Our analysis reveals a surge in ride-sharing earnings in early 2021, followed by fluctuations and a \textbf{decline in inflation-adjusted income for drivers}. Spatially, a \textbf{widening gap in trip costs} emerges as Central and Airport regions consistently earn more than peripheral areas. Our trip assignment algorithm uncovers distinct driver groups with \textbf{varied work patterns and earning profiles}. Drivers who concentrate their work during late-evening or overnight hours consistently earn higher per-trip and hourly rates, even if they complete fewer trips compared to those driving predominantly during daytime. Additionally, spatial factors play a critical role---drivers focusing on premium areas like the airport or downtown earn substantially more than those serving peripheral neighborhoods, with a newly emerged group in the South Side in 2023, earning as low as \$12.74 per hour\footnote{Note that all we use the trip total, consisting of the trip fare, tip, and additional costs, to approximate driver earnings. The true net earning for drivers is lower but not available in public data.}. Overall, our results call for more transparent pricing models and a re-examination of platform design to foster equitable earning opportunities.




% In doing so, we create a new avenue for fairness analysis that works with existing privacy-protecting mechanisms in public datasets, not relying on proprietary data or privileged access to internal platform information. 

% % \tlcomment{need a paragraph for the empirical contribution -- e.g., we adopt this new methodological framework to measure and characterize the inequality in rideshare platform algorithms in the city of Chicago as a case study. [talk about why we chose Chicago]. we found....[summarize the findings]}

% \tlcomment{i suppose we can make the paragraph below more specific/concrete once we finish the analysis}
%
In summary, our contributions are threefold:
\begin{itemize}
\item We reveal disparities in Chicago’s ride-sharing trip data, showing significant shifts in trip allocations, earnings, and pricing factors since 2018.
\item 
We propose a simple trip-based driver-assigment simulation algorithm that generate potential driver-level activity from anonymized data, preserving privacy while able to observe how certain groups’ earnings diverge substantially from others.
\item 
We provide empirical evidence of equity gaps likely coming from algorithmic matching rules or regional fare structures, such findings that can inform more equitable policy and platform design.
\end{itemize}

% By proposing a scalable framework to infer driver-level activity from limited data, our work illuminates previously hidden aspects of ride-sharing ecosystems and provides a starting point for more equitable policy and platform design. Ultimately, this research sets the stage for more informed discussions about transparency, fairness, and accountability in the increasingly algorithm-driven world of urban mobility.
