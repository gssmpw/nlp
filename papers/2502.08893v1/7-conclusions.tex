\section{Conclusions}
In this study, we explored disparities in ride-sharing earnings and trip distributions using the publicly available Chicago Rideshare dataset (2018–2023). By analyzing both direct observational and a simulation-based methodology, we revealed systematic inequities in driver earnings based on temporal, regional, and algorithmic factors. Our findings reveal that pricing adjustments in recent years have failed to account for inflation, leading to a decline in drivers’ real earnings despite apparent fare stabilization. Additionally, spatial analysis indicates that income gaps have widened over time, with lower-earning zones emerging in Chicago’s South Side and outlying areas.
To address limitations in existing anonymized ride-share datasets, we introduced a simulation-based driver profiling method that reconstructs potential work and earning patterns. This approach allowed us to model driver behaviors, including variations in working hours, trip frequencies, and geographical preferences, which contribute to substantial earnings disparities. Our clustering analysis further revealed the emergence of new driver groups in 2023, suggesting shifts in ride-sharing platform dynamics and potential algorithmic biases in trip allocations. 