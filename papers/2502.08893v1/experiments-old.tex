\section{Results}
We organize our results to address each \textbf{Research Question (RQ)} directly, integrating both \textbf{observational data} from the publicly available Chicago Rideshare dataset (2018-2023) and \textbf{simulation-based analyses}. Throughout, we highlight our findings and implications of platform pricing models and affects of temporal and geographical to trip distributions and earnings.
\subsection{Pricing Model Shifts}
Based on our data analysis, we 
% \hdcomment{@Tamara, Jason: Do we know about the information about the pricing changes and if that aligns with some of results in this sections (Diagram)? }
As discussed in previous sections, our aim is to examine pricing trends over time to identify potential changes in pricing models during various periods, as well as to analyze how trips and earnings are distributed across different regions. We hypothesize that a driver's working region significantly affects trip earnings. We highlight our findings using publicly available Chicago data, focusing on how pricing models and trip earnings shift from year to year.

\paragraph{Pricing Model Shift}
Figures ~\ref{by_hour} and ~\ref{by_mile} present monthly metrics from the Chicago Rideshare Data, covering November~2018 to December~2023. Figure ~\ref{by_hour} focuses on hourly earnings, including average total income per trip, profit, and fare-based earnings, while Figure ~\ref{by_mile} examines earnings per mile driven.

The data shows marked changes beginning in January~2021, characterized by a notable increase in all earnings metrics, which peaked in August~2021. This uptick may reflect adjustments in pricing strategies, heightened demand, or external factors such as economic recovery or promotional events. However, the figures also reveal fluctuations after the peak, followed by a gradual decline throughout 2023. These observations suggest that pricing models have stabilized since late 2021, with no major adjustments in pricing factors apparent in the subsequent period.

\begin{figure}[h]
  \centering
  \includegraphics[width=\linewidth]{figures/preliminary_result/perhour.png}
  \caption{Earning Per Hour}
  \label{by_hour}
\end{figure}

\begin{figure}[h]
  \centering
  \includegraphics[width=\linewidth]{figures/preliminary_result/per_mile.png}
  \caption{Earning Per Mile}
  \label{by_mile}
\end{figure}

% \hdcomment{It may need a paragraph to depict that this is not okay for the earning remain the same after 3 years?}
These trends provide valuable insights into the temporal dynamics of pricing and earnings, highlighting potential opportunities to further analyze underlying factors driving these shifts, such as seasonal demand, fuel costs, or market competition.
Given that the Bureau of Labor Statistics CPI-U for the Chicagoland area increased approximately 15\% from 2021-2023 alone, the stagnation in growth in driver earnings effectively amounts to a significant decline in real, inflation-adjusted earnings \cite{}

\paragraph{Analysis of Regional Trip Distributions}
In addition to examining pricing shifts, we conducted a regional analysis of trip volumes and earnings in Chicago to gain insight into spatiotemporal patterns of ridehailing demand. Figures~\ref{distribution_2019} and \ref{distribution_2023} compare the percentage of pickup and dropoff trips across major Chicago regions in 2019 and 2023. Consistent with our simulation timeframe, the Central region held the highest share of rides in both years---about 30\% in 2019 and nearly 27\% in 2023---indicating that downtown and nearby neighborhoods remain a focal point for ridehailing. However, from 2019 to 2023, there was a notable increase in the share of trips in the South Side and Far Southeast Side, suggesting a gradual dispersal of ridehailing activity beyond traditional high-density zones such as the Central or West Side. Future research could investigate the factors driving these shifts, including demographic changes, new developments, or transit expansions, and assess how evolving neighborhood characteristics may influence driver earnings. These findings also inform our subsequent simulation analysis, in which we explore how pricing and driver relocation strategies align with observed real-world changes in spatial demand.


\begin{figure}[h]
  \centering
  \label{distribution_2019}
  \includegraphics[width=\linewidth]{figures/preliminary_result/heatmap_2019_trip_distribution.png}
  \caption{Distribution of Trips Based on Pickup and Dropoff Locations Across Chicago Regions in 2019}
\end{figure}

\begin{figure}[h]
  \centering
\label{distribution_2023}
  \includegraphics[width=\linewidth]{figures/preliminary_result/heatmap_2023_trip_distribution.png}
  \caption{Distribution of Trips Based on Pickup and Dropoff Locations Across Chicago Regions in 2023}
\end{figure}
\paragraph{Analysis of Regional Earning Rates}
Figures~\ref{earnings_2019} and \ref{earnings_2023} illustrate the estimated hourly earnings (including a 0.25-hour wait time) for ridehailing trips across Chicago’s regions in 2019 and 2023, respectively. In both years, the Airport region exhibits the highest projected earnings, increasing from approximately \$42--\$43/hour in 2019 to more than \$55--\$60/hour in 2023. Similar upward trends can be observed citywide: for instance, the Central region’s pickup earnings rise from about \$25/hour to over \$40/hour, while the Southwest Side jumps from \$26/hour to \$35/hour during the same period. We hypothesize that these increases stem from changes in the pricing models discussed in the \emph{Pricing Shift} section.

Although the Airport region consistently commands a premium, we posit that this might be partially explained by its geographic context—longer trips typically originate in other regions—and by extended wait times commonly encountered at the airport. Because the dataset lacks detailed wait-time information, we uniformly apply a 0.25-hour wait time to all trips, which could underestimate earnings differentials for airport-based trips. Future research should incorporate region-specific wait times to more accurately capture such variations.

Beyond the Airport region, all areas exhibit some growth in earnings, yet disparities among non-airport regions evolve from 2019 to 2023. For example, in 2019 (excluding the Airport region), the Southwest Side records the highest earnings, at \$26.33 and \$27.06 for pickup and dropoff, respectively, whereas the West Side registers the lowest, at \$23.33 and \$23.07. This gap of 12.86\%--17.30\% widens considerably by 2023, when the Central region leads with \$40.24 and \$38.40 for pickup and dropoff, whereas the Far Southeast Side falls to \$30.06 and \$30.45—a difference of 26.8.\%--33.87\%. Notably, the South Side, Far Southwest Side, and Far Southeast Side show substantially lower earnings in 2023 compared to other regions, a distinction that was less pronounced in 2019.

Overall, these findings suggest that, despite relatively stable trip distributions, regional earnings have shifted considerably. This underscores the role of location-specific demand and fare policies in shaping driver incomes over time, a trend we also observe in our simulation models.


\begin{figure}[h]
  \centering
  \includegraphics[width=\linewidth]{figures/preliminary_result/heatmap_2019_trip_earning.png}
  \caption{Earnings of Trips Based on Pickup and Dropoff Locations Across Chicago Regions in 2019}
    \label{earnings_2019}
\end{figure}

\begin{figure}[h]
  \centering
  \includegraphics[width=\linewidth]{figures/preliminary_result/heatmap_2023_trip_earning.png}
  \caption{Earnings of Trips Based on Pickup and Dropoff Locations Across Chicago Regions in 2023 \hdcomment{@Tamara, Jason: Additionally, in our findings, we show that in 2023, driver who drives in the South Side regions (Far Southwest Side, South Side, Far Southeast Side), earned significantly less than other areas. Do we have some relevant information from qualitative study for this statement?}}
    \label{earnings_2023}
\end{figure}
\subsubsection{Results}
\hdcomment{@Tamara, Jason: The table depicts "Significant" Differences Between Different Drivers Groups Using Our Simulation Models and We can see clearly gaps Between Earning Per Hour (For Example, Driver 0 earns 44.94\$/Driving Hour But Driver 7 earns 60.53\$/Driving Hour,..). Do we have some qualitative analysis that can support that? Differences in earning among different drivers?}

\hdcomment{@Tamara, Jason: In this section, we also can mention about how frequency of trips can be a factor to affect the earnings average (hourly wage), however, here, we claimed that this is not a factor, for example, Driver 2 drives large number of trips but on averages, he/she doesn't have the high earning/hours. Do we have some qualitative information to support how frequency of trips a driver usually drives per day or how they affect the earnings?}

In this section, we present the aggregated weekly results for the week of 2019-08-05 and 2023-08-07, using our simulation algorithm. The simulation yielded a total of $364,452$ and $259,812$ drivers in 2019 and 2023, respectively. However, since our study focuses on analyzing trip patterns for drivers making more than two trips per day, we narrowed the dataset to $164,234$ and $114,920$, which is 45.06\%  and 44.23\% of the simulated drivers in the same time period.

By applying the KMeans algorithm to the processed dataset, we identified 9 and 11 distinct clusters for 2019 and 2023, respectively. The choice of the number of clusters was based on achieving the highest Silhouette Score among the experimented cluster numbers, indicating optimal cluster quality.

To further examine and validate the clustering results, we utilized t-SNE visualization, as shown in Figure \ref{figure:cluster_2019} and \ref{figure:cluster_2023}, which provides an intuitive representation of the driver groups and their separability in reduced dimensions.
\hdcomment{Should we put this into Appendix?}
\begin{figure}[h]
  \centering
  \includegraphics[width=\linewidth]{figures/simulation_result/2019_clustering_result.png}
  \caption{(Left) t-SNE visualization of the best number of clusters using KMeans algorithms and Distribution of clusters on simulated drivers for the week of 2019-08-05. (Right) Distribution of different clusters}
  \label{figure:cluster_2019}
\end{figure}

\begin{figure}[h]
  \centering
  \includegraphics[width=\linewidth]{figures/simulation_result/2023_clustering_result.png}
  \caption{(Left) t-SNE visualization of the best number of clusters using KMeans algorithms and Distribution of clusters on simulated drivers for the week of 2023-08-07. (Right) Distribution of different clusters}
  \label{figure:cluster_2023}
\end{figure}

% \paragraph{Overall For Different Driver Groups}
\begin{table}[ht!]
\centering
\caption{Earning Metrics For Different Driver Cluster on Simulated Drivers For 2019 and 2023}
\label{tab:predict_cluster_case_2_no_cluster_9}
\resizebox{\textwidth}{!}{
\begin{tabular}{
    ccccccc
}
\toprule
\textbf{Driver ID} 
& \textbf{\#Trips Performed} 
& \textbf{Earning/Trips $(\$)$} 
& \textbf{Earning/Driving Hour $(\$)$} 
& \textbf{Estimated Earning/Hour $(\$)$ ( $+\alpha = 0.25$)}
& \textbf{Total Fares $(\$)$ } 
& \textbf{Total Income $(\$)$ }\\
\midrule
\multicolumn{7}{c}{\textbf{Data of Simulated Driver Groups in 2019}} \\
\midrule
0 & 6.30 & 12.43 & \underline{43.94} & \underline{23.75} & 61.16 & 81.41 \\
1 & 5.30 & 11.66 & 57.89 & 26.44 & 48.44 & 64.49 \\
2 & \textbf{20.91} & \textbf{13.23} & 48.05 & 25.70 & \textbf{216.10} & \textbf{288.70} \\
3 & \underline{4.49} & 12.33 & 48.60 & 25.00 & \underline{43.23} & \underline{57.85} \\
4 & 5.63 & 13.22 & 49.22 & 26.04 & 58.28 & 77.80 \\
5 & 7.48 & \underline{11.59} & 51.40 & 24.89 & 66.68 & 90.17 \\
6 & 5.53 & 12.14 & 47.71 & 24.53 & 51.43 & 69.80 \\
7 & 5.52 & 13.95 & \textbf{60.53} & \textbf{29.77} & 60.93 & 80.92 \\
8 & 6.47 & 13.22 & 46.81 & 25.33 & 66.83 & 89.27 \\
\midrule
\multicolumn{7}{c}{\textbf{Data of Simulated Driver Groups in 2023}} \\
\midrule
0  & 20.87 & 17.08 & 62.59 & 33.44 & 277.59 & 374.54 \\
1  &  5.34 & 17.38 & 88.12 & 39.73 &  76.50 &  96.59 \\
2  &  6.85 & 17.40 & 60.80 & 33.19 &  92.64 & 125.10 \\
3  &  8.00 & 15.52 & 69.37 & 33.52 &  97.63 & 129.15 \\
4  &  6.77 & 17.16 & 59.81 & 32.60 &  91.12 & 121.25 \\
5  &  6.51 & 12.74 & 52.47 & 26.76 &  68.96 &  88.91 \\
6  &  5.48 & 17.20 & 68.31 & 34.96 &  71.71 &  98.16 \\
7  &  5.32 & 15.68 & 58.39 & 31.00 &  66.67 &  87.82 \\
8  &  4.57 & 18.68 & 76.01 & 38.58 &  65.70 &  89.55 \\
9  &  5.88 & 18.75 & 70.41 & 37.17 &  84.82 & 115.73 \\
10 &  5.31 & 21.28 & 96.66 & 46.40 &  90.20 & 118.42 \\
\bottomrule
\end{tabular}}
\end{table}

\paragraph{Temporal Distribution Characteristics For Simulated Drivers}
\hdcomment{@Tamara, Jason: In this section, we mention about how working hours (hours that drivers work per day affect the earnings? In our findings, we see that specific timeframes can yield more earnings than other working time? Do we have a qualitative analysis about if some specific chunks of time can have more/better earnings/working hour?}
\hdcomment{@Tamara, Jason: Additionally, do we have some qualitative information about how much times drivers usually drive per day?}
Figures~\ref{figure:time_chunk_2019} and \ref{figure:time_chunk_2023} show the distribution of trips, segmented by start-time for each simulated driver group across various time intervals. The temporal patterns differ considerably among drivers, reflecting diverse operational schedules and preferences in both the 2019 and 2023 datasets.

\tlcomment{need to actually interprete the data and spell out the results -- ideally at the beginning in bold e.g., ``differences in driver earning based on temporal patterns'' and clearly state the finding }

\textbf{2019 Data of Simulated Driver Groups}  
In the 2019 data, Driver~2 demonstrates consistent activity, particularly in the 9--18\,hour interval, which aligns with their high total trip count from earlier analyses. By contrast, Driver~7 shows a pronounced increase in activity during the 21--24\,hour interval, suggesting they concentrate on fewer but potentially higher-value trips. Meanwhile, Driver~5, who has the lowest earnings per trip, operates primarily in the 0--9\,hour window, peaking between 3--6\,hours. These observations imply that work hours may shape a driver's earnings potential. Notably, most drivers focus on distinct time intervals, with the exception of Driver~2, who maintains relatively uniform activity throughout the day.

\textbf{2023 Data of Simulated Driver Groups}  
In the 2023 data, Driver~0 is the most active driver overall, working across multiple time intervals. Driver~10 exhibits heightened activity in the 15--24\,hour interval, completing fewer yet high-value trips. Although Driver~5 again reports the lowest earnings per trip, they maintain a schedule comparable to Driver~0, working consistently across time intervals. However, a sizable earnings gap remains between these two drivers (\$12.74 vs.~\$17.08), implying that factors beyond raw working hours may influence overall earnings discrepancies among driver groups.

\tlcomment{i wouldn't call this opportunity for optimizing driver schedules -- since it's a dynamic market -- sending everyone to the "high earning" hours would not result in everyone making more, for example}

\noindent Overall, the temporal data reveal opportunities to optimize driver schedules and investigate how driver behavior patterns correlate with earnings outcomes.

\begin{figure}[ht]
  \centering
  \includegraphics[width=\linewidth]{figures/simulation_result/2019_number_of_trips_start_proportion.png}
  \caption{Proportion of Trip in Different Time Interval For Different Driver Group Cluster in the week of 2019-08-05}
  \label{figure:time_chunk_2019}
\end{figure}

\begin{figure}[ht]
  \centering
  \includegraphics[width=\linewidth]{figures/simulation_result/2023_number_of_trips_start_proportion.png}
  \caption{Proportion of Trip in Different Time Interval For Different Driver Group Cluster in the week of 2023-08-07}
  \label{figure:time_chunk_2023}
\end{figure}


\begin{figure}[ht]
  \centering
  \includegraphics[width=\linewidth]{figures/simulation_result/2019_number_of_trips_dropoff_pickup.png}
\caption{Frequency and Distribution of Pickup and Dropoff Trip in Different Locations (Community Area) For Different Driver Cluster}
  \label{figure:dropoff_pickup_2019}
\end{figure}

\begin{figure}[ht]
  \centering
  \includegraphics[width=\linewidth]{figures/simulation_result/2023_number_of_trips_dropoff_pickup.png}
\caption{Frequency and Distribution of Pickup and Dropoff Trip in Different Locations (Community Area) For Different Driver Cluster}
  \label{figure:dropoff_pickup_2023}
\end{figure}
\paragraph{Regional Distribution Characteristics for Simulated Drivers}
\hdcomment{@Tamara, Jason: In this section, we mention about how dropoff/pickup location affect the earnings? In our findings, we see that specific locations can yield more earnings than others? Do we have a qualitative analysis about if some specific locations have better earnings?}
\hdcomment{@Tamara, Jason: Also, below figure shows that drivers mainly on some specific groups of locations instead go through the whole Chicago. Do we have some qualitative analysis on where drivers choose to work? Or Driver rationale for or against certain neighborhoods: Some areas may be avoided due to perceived wait times, safety concerns, or fewer premium fares.}
\hdcomment{@Tamara, Jason: That would be great if we have some qualitative information that is relevant to Platform steering practices: How do driver-side apps (e.g., “destination filters”) and personal constraints (fuel costs, comfort zones) shape these geographic patterns?}


\textbf{2019 Data of Simulated Driver Groups}  
We first examine how trips are distributed geographically among the simulated driver groups for 2019, segmenting both drop-off and pickup locations. As shown in Figure~\ref{figure:dropoff_pickup_2019}, most drivers concentrate on the Central region, with secondary focus on the North and West Sides. Driver~7, for instance, stands out by having the second-highest proportion of drop-offs in the Airport region (nearly 10\% of total trips), a factor that correlates with higher per-trip earnings (see Table~\ref{table:earning_driver}). Overall, the 2019 simulation suggests that drivers who diversify their service areas beyond the Central region—especially those making frequent airport trips—tend to exhibit higher earnings.

\textbf{2019 Data of Simulated Driver Groups}  
In the 2023 simulation (Figure~\ref{figure:dropoff_pickup_2023}), the Central region again dominates trip distribution; however, a few drivers show distinct patterns. Driver~7 operates primarily in the Northwest Side (about 40\% of total trips), while Driver~5 focuses heavily on the South Side, Far Southwest Side, and Far Southeast Side (roughly 22\%, 20\%, and 30\%, respectively). As noted in earlier sections, Driver~5’s lower earnings metrics align with their concentration in these areas. Conversely, Driver~10, who also logs around 10\% of drop-offs at the airport, records some of the highest per-trip earnings among the simulated groups.

\noindent\textbf{Comparison of 2019 and 2023 Data if Simulated Driver Groups.}  
Despite common regional hotspots in both 2019 and 2023, we do observe some evolving spatial trends—particularly for drivers focusing on South Side regions, where 2023 earnings are notably lower than those of other groups, at \$12.74. In contrast, no specific driver group in 2019 was found to concentrate heavily on the South Side, suggesting that either demand was relatively low, or that differences in earnings among South Side drivers and their counterparts were less pronounced in that period. Meanwhile, drivers serving the airport continue to report higher per-trip earnings in both years. These findings highlight that, although dominant demand zones like the Central region remain relatively stable over time, shifts in driver location preferences can substantially affect earning potential. A simulation-based approach, such as the one presented here, is well-suited to capturing these nuances and can inform more effective driver allocation strategies in response to changing regional demand. 


\section{Discussions}

\section{Limitations and Future Work}

\section{Conclusions}