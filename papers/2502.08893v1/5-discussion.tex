\section{Discussions}

\subsection{Transparency in Ride-Sharing Platform Algorithms}
The publicly available Chicago Transportation Network Provider dataset helped us answer many research questions, but ride-sharing platforms still make many of their mechanisms opaque. The lack of transparency in key platform mechanisms---such as pricing models, driver--rider matching algorithms, and driver ranking systems---makes it difficult to pinpoint the exact causes of these disparities. Without greater visibility into these proprietary algorithms, drivers also remain at an information disadvantage, unable to anticipate fare fluctuations or optimize their work schedules effectively.

Pricing models remain opaque, with our analysis revealing that fare adjustments over time have failed to keep pace with inflation, effectively reducing real driver earnings (\cref{sec:results-pricing-stablization}). While platforms advertise dynamic pricing mechanisms that respond to demand surges, drivers have limited insight into how much of the fare they actually receive after platform fees~\cite{santos2020dynamic}. Previous research has shown that drivers tend to work more during peaks for higher compensation~\cite{chen2016dynamic}. A real-time, large-scale understanding of the surge pricing model can help drivers become more informed in planning and organizing their workday, beyond anecdotal observations. Furthermore, researchers can provide prediction models of price surges, helping both drivers and riders adjust plans accordingly. Another key limitation of using the Chicago dataset is the lack of driver earning information. As a result, our analysis can only use the trip fare as a proxy for driver earning. Making such information available can significantly increase transparency into platform operations.

Similarly, the driver-rider matching algorithm remains a black box. Our inferred driver profiles suggest that trip assignments may systematically disadvantage certain groups, particularly those operating in lower-income areas. If the matching algorithm disproportionately favors drivers in high-demand or high-fare regions, it could reinforce existing geographic disparities in earnings. However, such analysis is hard to conduct without access to driver-level information. As discussed in \cref{sec:methods-driver-simulation}, releasing such data may lead to privacy concerns. Our approach is an effort to approximate driver working conditions without needing detailed driver data. However, researchers should still work with ride-sharing platforms to come up with privacy-preserving ways to analyze such data for insights. Also, driver ranking algorithms---which determine access to high-value trips---are equally opaque. While platforms often cite factors such as acceptance rate, customer ratings, and trip history, the lack of public accountability raises concerns regarding potential biases. Accessing such information can support researchers in identifying potential biases, also help drivers provide more desired services to riders.

In all, we call for increased regulatory oversight and platform-level efforts to improve algorithmic transparency. Without clear disclosures on how these systems operate, ride-sharing drivers remain vulnerable to unfair decision-making and fluctuating incomes that they cannot predict or control.

\subsection{Data Analysis Methodology Improvements}
Our study demonstrates the feasibility of simulating reasonable driver profiles from trip-level data, even in the absence of driver-related information. By leveraging a simulation-based approach, we were able to approximate driver earnings, work patterns, and geographic activity. However, there are still areas for improvement for our methodology.

First, a robust evaluation benchmark is needed to validate the accuracy of inferred driver profiles. While our approach provides valuable insights and matches previous empirical findings, the lack of direct ground truth data means we rely on approximations. We need alternative data sources to cross-verify our inferred driver activities. Tools for driver task management, such as Driver's Seat~\cite{calacci2023access}, asks drivers to upload their work tasks and can serve as a potential data source. More autonomous approaches that uses UI understanding techniques and directly collects data from drivers' phones can also scale up this effort~\cite{lu2024crepe}. 

Moreover, expanding the scope of inferred information would provide deeper insights into platform operations. Currently, we infer earnings and work patterns for drivers. Newer algorithms can be developed to analyze additional opaque platform mechanisms as discussed above. Future studies could aim to reconstruct other aspects of opaque platform algorithms, as discussed above, directly from publicly available, large-scale datasets.

Given a large-scale dataset that misses key information aspects, a potential future approach is to self-collect a smaller dataset that contains the necessary details and conduct a joint analysis of both datasets. For example, a smaller dataset that we collect directly from drivers, containing both driver and trip information, can serve both as a benchmark and a basis for use to train machine learning models that predict driver profiles from existing large-scale datasets. Future research can investigate effective measures to combine these different data sources~\cite{harris2018federal} for joint analysis. These methodological advancements can help us to use large-scale ridesharing datasets more effectively and accurately while maintaining driver and rider privacy.


\subsection{Societal Implications: Ride-Sharing as a Reflection of Broader Inequalities}

Our findings revealed regional ride-sharing disparities in the city of Chicago, which largely reflect the broader existing societal inequalities. Drivers working in lower-income neighborhoods---in our case, drivers that service the southern regions of Chicago---consistently earn less, even despite longer work hours. Structural disadvantages, such as lower infrastructure quality, longer wait times, and increased safety concerns---compound the challenges faced by gig workers. Chicago South Side, as a community suffering from violence and poverty, has been an example of social segregation and studied by numerous researchers~\cite{moore2016south, bachin2004building, bell1993community}. As an aspect of a deep-rooted societal issue, ride-sharing inequality in lower-income neighborhoods calls for holistic policymaking efforts from multiple stakeholders.

Our findings provide practical implications for labor activists and policy makers. By providing a more transparent view of drivers’ potential workday experiences, policymakers can better evaluate the labor conditions these platforms create, ensuring that emerging mobility systems align with equity goals. Urban planners and regulators can use these insights to inform policy interventions---such as driver support programs, driver caps, or incentive structures---that promote fairness and mitigate algorithmic biases. Similarly, platform operators themselves might harness these findings to improve their matching algorithms, advancing a more equitable ecosystem that benefits both drivers and passengers.

Research has shown that transportation access can have a positive impact on regional economic growth and productivity~\cite{targa2005economic, banerjee2020road, alstadt2012relationship}. Ride-sharing, as an increasingly critical way of transportation, especially where public transportation is scarce, can support individual and community access to growth opportunities. The persistence of regional earning gaps raises important questions about equity in urban transportation. If ride-sharing platforms are designed primarily to maximize efficiency and revenue, they may inadvertently exacerbate existing economic inequalities by steering high-value rides away from underserved areas~\cite{durand2022access, bocarejo2012transport}.

To address these issues, we call for policy interventions aimed at ensuring fair compensation and equitable access to earning opportunities. Regulators should consider implementing transparency mandates, income stability measures, and algorithmic accountability frameworks to prevent platforms from disproportionately disadvantaging certain driver groups. Moreover, these efforts should be in orchestration with existing efforts to promote infrastructural improvements and public safety in underserved regions. Collaborative initiatives between policymakers, ride-sharing companies, and community organizations can help create a more inclusive transportation ecosystem that benefits both drivers and passengers alike~\cite{baber2022new}.