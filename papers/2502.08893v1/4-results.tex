\section{Results}
We structure our results to directly address each Research Question (RQ), integrating findings from both the raw Chicago dataset (RQ1, RQ2) and after applying our driver assignment algorithm (RQ3). 

% Throughout, we highlight our findings and implications of platform pricing models and affects of temporal and geographical to trip distributions and earnings.
\subsection{RQ1: How do ride-sharing trip costs change over time?}
\label{RQ1}
\begin{figure}[h]
  \centering
  \includegraphics[width=0.9\linewidth]{figures/preliminary_result/cost_hour.png}
  \caption{Cost per driving hour trends from November 2018 to November 2023. A significant increase in earnings per trip and projected total earnings per driving hour begins in January 2021, peaking in August 2021. After this peak, all metrics fluctuate and gradually decline throughout 2023, with no major increases observed since late 2021.}
  \label{by_hour}
\end{figure}

% \hdcomment{@Tamara, Jason: Do we know about the information about the pricing changes and if that aligns with some of results in this sections (Diagram)? }
% As discussed in previous sections, our aim is to examine pricing trends over time to identify potential changes in pricing models during various periods. We highlight our findings using publicly available Chicago data, focusing on how pricing models and trip earnings shift from year to year.

% \paragraph{Pricing Model Shift}

\label{sec:results-pricing-stablization}

\paragraph{\textbf{General Trends in Pricing Over Time.}}
Figure~\ref{by_hour} illustrates the projected total cost per driving hour from November 2018 to November 2023. The data shows a \textbf{gradual increase in earnings per hour from late 2018 to early 2020}, followed by a steep surge beginning in early 2021, peaking around August 2021. After this peak, there is a notable decline in late 2021, followed by fluctuations and a secondary surge until mid-2022. From mid-2022 onwards, the overall trend declines, with no major increases observed through the end of 2023.

These patterns suggest that platform pricing strategies have potentially undergone multiple shifts, particularly in response to broader economic and market conditions.


\paragraph{\textbf{Stagnation in Driver Earnings vs. Inflation-Adjusted Wages.}}

Although hourly costs remained relatively stable from late 2021 onward, they did not keep pace with inflation. According to the Bureau of Labor Statistics, the Consumer Price Index for All Urban Consumers (CPI-U) in the Chicagoland area rose by approximately 15\% between 2021 and 2023~\cite{BLS_CPI}. However, Figure~\ref{by_hour} suggests that driver earnings did not experience a corresponding increase.

This stagnation raises serious equity and long-term sustainability concerns. Even if nominal fares remain steady, the inflation-adjusted value of driver earnings is decreasing. As a result, many drivers may be experiencing a decline in purchasing power, making gig work less financially viable over time. The lack of alignment between earnings and inflation suggests that ride-sharing platforms may not be adjusting pay rates in response to economic conditions, raising concerns about fair wage structures and transparency~\cite{angrist2021uber}.


% \paragraph{\textbf{Lack of Transparency in Pricing Model}} Although the dataset reports fares, tips, and total trip earnings, it does not reveal what portion drivers actually retain after platform fees, which can vary across companies and trip types \cite{konishi2023analysis, angrist2021uber}. Consequently, although hourly cost in 2022–2023 appear fairly high (around \$100–\$120 per driving hour), these figures do not account for platform fees. Moreover, they ignore the potential impact of “down time”—including waiting periods and travel between trip requests. As a result, reported totals may overestimate drivers’ actual take-home pay.

% \begin{figure}[H]
%   \centering
%   \includegraphics[width=\linewidth]{figures/preliminary_result/per_mile.png}
%   \caption{Earnings per mile trends from November 2018 to November 2023 display a marked increase in both metrics (Fare/Driving Mile and Total Earning/Driving Mile) starting in January 2021, peaking in August 2021. Following the peak, both metrics show fluctuations and a gradual decline, with stabilization in recent periods.}
%   \label{by_mile}
% \end{figure}

\subsection{RQ2: How do ride-sharing trip locations impact trip costs?}

To examine regional trip cost differences, we conducted a analysis of trip volumes and earnings in 7 main regions Chicago from 2018 to 2023. Here, we highlight findings from 2019 and 2023. Additional figures for other years are included in Appendix~\ref{sec:appendix}.


\paragraph{\textbf{Spatial Demand Shift vs. Persistent Regional Disparities.}}

\begin{figure}[ht!]
  \centering
  \includegraphics[width=0.95\linewidth]{figures/preliminary_result/heatmap_2019_trip_distribution.png}
\caption{Heat maps of pickup and dropoff trip distributions across Chicago in 2019. The highest concentration of activity was in the Central region, accounting for 30.50\% of pickups and 30.88\% of dropoffs, followed by the West Side. Other regions, such as the South Side and Far Southeast Side, shows lower levels of ride-hailing activity, with the Far Southwest Side being the least active area, contributing only 1.31\% of pickups and 1.29\% of dropoffs.}
    \label{fig:distribution_2019}
\end{figure}

\begin{figure}[H]
  \centering
  \includegraphics[width=0.95\linewidth]{figures/preliminary_result/heatmap_2023_trip_distribution.png}
\caption{Heat maps of pickup and dropoff trip distributions across Chicago in 2023. The Central region remains the most active, with 27.42\% of pickups and 27.75\% of dropoffs, followed by the West Side. The South Side and Far Southeast Side show increased activity compared to 2019, the Far Southwest Side continues to account for the lowest share, with 1.98\% of pickups and 1.91\% of dropoffs.}
    \label{fig:distribution_2023}
\end{figure}


Figures \ref{fig:distribution_2019} and \ref{fig:distribution_2023} compare the percentage of pickup and drop-off trips across major Chicago regions in 2019 and 2023. The Central region held the highest share of rides in both years---about 30\% in 2019 and nearly 27\% in 2023---indicating that downtown and nearby neighborhoods remain a focal point for ride-hailing. However, from 2019 to 2023, there was a \textbf{notable increase} in the share of trips in the South Side and Far Southeast Side, suggesting a gradual dispersal of ride-hailing activity beyond traditional high-density zones such as the Central or West Side. These findings also inform our subsequent simulation analysis, in which we explore how pricing and driver relocation strategies align with observed real-world changes in spatial demand.

\paragraph{\textbf{Emergence of Low-Cost Zones in 2023.}}

Figures~\ref{earnings_2019} and~\ref{earnings_2023} show that although nearly every region experienced cost or price growth---for example, the Central region increased from about \$60/hour to over \$95/driving hour, and the Southwest Side rose from \$58/hour to \$75/hour---areas in the far south nonetheless trail considerably behind both the Airport and downtown core. We hypothesize that these upward trends partially stem from the pricing-model adjustments discussed in Section \ref{RQ1}.

Despite the gains, disparities among non-airport regions have \textbf{widened significantly}. For instance, while the Central outperformed many other neighborhoods in 2019, with hourly driving cost of \$59.76--\$61.44, its advantage over the low-cost region (Northwest Side, with hourly driving cost of \$56--\$54.13) was relatively modest (7--13\%). By 2023, however, the gap between top-earning regions (e.g., \$94.88--\$98.42 in the Central region) and low-cost areas like the Far Southwest Side (\$72.97--\$70.37) has reached to 30--40\%. In particular, the South Side, Far Southwest Side, and Far Southeast Side now show substantially lower costs than other regions---a distinction that was far less pronounced in 2019---underscoring the increasing income inequality across Chicago’s ride-hailing landscape.


This \textbf{increasing income inequality} suggests that pricing model adjustments are not benefiting all drivers equally, reinforcing disparities among different Chicago neighborhoods.


\begin{figure}[H]
  \centering
  \includegraphics[width=0.95\linewidth]{figures/preliminary_result/heatmap_2019_trip_cost.png}
\caption{Projected hourly driving costs in 2019. The highest cost area is in the Airport region, with \$66.21/hour for pickups and \$65.97/hour for dropoffs. The Central and Southwest Side regions with the top earnings regions. In contrast, the Northwest Side recorded the lowest earnings, with \$56/hour and \$54.13/hour for pickup and dropoff, respectively.}
    \label{earnings_2019}
\end{figure}
\begin{figure}[H]
  \centering
  \includegraphics[width=0.95\linewidth]{figures/preliminary_result/heatmap_2023_trip_cost.png}
\caption{Average hourly trip costs in 2023. The highest cost remains in the Airport region, with \$97.17/hour for pickups and \$87.52/hour for dropoffs. The Central and Southwest Side regions are among the top earning areas. In contrast, the Far Southeast Side and Far Southwest Side recorded the lowest earnings, with \$30.06/hour and \$31.14/hour for pickup and dropoff, respectively.}
    \label{earnings_2023}
\end{figure}

% \vspace{-5cm}
\paragraph{\textbf{Airport Premium vs. Outlying Gaps.}}
The airport region exhibits the highest trip cost growth, increasing from approximately \$66.21--\$65.97 per hour in 2019 to \$97.17--\$87.52 per driving hour in 2023. Although the airport region consistently shows much higher trip costs, we hypothesize that this might be partially explained by its geographic context---longer trips typically originate in other regions---and by extended wait times commonly encountered at the airport. 
Overall, despite relatively stable trip distributions, regional earnings have shifted considerably. This emphasizes the role of location-specific demand and fare policies in shaping driver incomes over time, a pattern we also observe in our simulation models.


%\subsubsection{RQ3: How a Simulation Approach Approximate Driver-Level Work Patterns and Reveal Hidden Earning Gaps Among Different Driver “Types”?}
% \hdcomment{@Tamara, Jason: The table depicts "Significant" Differences Between Different Drivers Groups Using Our Simulation Models and We can see clearly gaps Between Earning Per Hour (For Example, Driver 0 earns 44.94\$/Driving Hour But Driver 7 earns 60.53\$/Driving Hour,..). Do we have some qualitative analysis that can support that? Differences in earning among different drivers?}

% \hdcomment{@Tamara, Jason: In this section, we also can mention about how frequency of trips can be a factor to affect the earnings average (hourly wage), however, here, we claimed that this is not a factor, for example, Driver 2 drives large number of trips but on averages, he/she doesn't have the high earning/hours. Do we have some qualitative information to support how frequency of trips a driver usually drives per day or how they affect the earnings?}
\subsection{RQ3: How do ride-sharing drivers' work patterns look like?}
\begin{table}[ht]
\centering
\caption{
  Earning Metrics for Simulated Driver Clusters (2nd week of August in 2019 and 2023). In 2019, Cluster 0 had the highest total income (\$288.70) and fares (\$216.10) where they perform the most number of trips, while Cluster 6 and Cluser 8 led in earnings per driving hour (\$57.89) and earnings per hour (\$29.77), respectively. Cluster 3 earned the least in hourly earnings at \$48.60. By 2023, incomes increased overall, with Cluster 9 achieving the highest earnings per driving hour (\$96.66) and per trip (\$21.28). Cluster 8, 10 had the lowest per-trip earnings (\$12.74) and total income (\$87.82), respectively, reflecting shifts in income distribution over time.
  \newline
  \emph{Definitions:} 
  E/Trip = average earning \emph{per trip}; 
  E/DriveHr = average earning \emph{per driving hour}; 
  Est.\ E/Hr(+0.25) = average estimated earning \emph{per hour including a 15-min wait assumption}; 
  Total Fares = total amount of Fares \emph{per date from trips}; 
  Total Inc. = total income \emph{per date after combining all fares, fees, and tips from trips.} 
  \newline
  \emph{Note: Bolded values in each column represent the highest value; underlined values represent the lowest. It is important to note that driver groups in the two timeframes are not directly related.}
}
\label{tab:predict_cluster_case_2_no_cluster_9}
\begin{tabular}{ccccccc}
\toprule
\textbf{Driver} & 
\textbf{\#Trips} & 
\textbf{E/Trip (\$)} & 
\textbf{E/DriveHr (\$)} & 
\textbf{Est.\ E/Hr (+0.25) (\$)} & 
\textbf{Total Fares (\$)} & 
\textbf{Total Inc.\ (\$)}\\
\midrule
\multicolumn{7}{c}{\textbf{Simulated Driver Groups in 2019}}\\
\midrule
0 & \textbf{20.91} & 13.23 & 48.05 & 25.70 & \textbf{216.10} & \textbf{288.70} \\
1 & 7.48 & \underline{11.58} & 51.40 & 24.89 & 66.68 & 90.17 \\
2 & \underline{4.49} & 12.33 & 48.60 & 25.00 & \underline{43.23} & \underline{57.85} \\
3 & 6.30 & 12.43 & 43.94 & \underline{23.75} & 61.16 & 81.41 \\
4 & 5.53 & 12.14 & 47.71 & 24.53 & 51.43 & 69.80 \\
5 & 5.63 & 13.22 & 49.22 & 26.04 & 58.28 & 77.80 \\
6 & 5.30 & 11.66 & \textbf{57.89} & 26.44 & 48.44 & 64.49 \\
7 & 6.47 & 13.22 & 46.81 & 25.33 & 66.83 & 89.27 \\
8 & 5.52 & \textbf{13.95} & \underline{40.53} & \textbf{29.77} & 60.93 & 80.92 \\
\midrule
\multicolumn{7}{c}{\textbf{Simulated Driver Groups in 2023}}\\
\midrule
0 & \textbf{20.87} & 17.08 & 62.59 & 33.44 & \textbf{277.59} & \textbf{374.54} \\
1 & 7.97 & 15.52 & 69.37 & 33.52 & 97.63 & 129.15 \\
2 & 6.77 & 17.16 & 59.81 & 32.60 & 91.12 & 121.25 \\
3 & 5.88 & 18.75 & 70.41 & 37.17 & 84.82 & 115.73 \\
4 & \underline{4.57} & 18.68 & 76.01 & 38.58 & \underline{65.70} & 89.55 \\
5 & 5.48 & 17.20 & 68.31 & 34.96 & 71.71 & 98.16 \\
6 & 5.34 & 17.38 & \textbf{88.12} & \textbf{39.73} & 76.50 & 96.59 \\
7 & 6.85 & 17.40 & 60.80 & 33.19 & 92.64 & 125.10 \\
8 & 6.51 & \underline{12.74} & \underline{52.47} & \underline{26.76} & 68.96 & 88.91 \\
9 & 5.31 & \textbf{21.28} & \textbf{96.66} & 46.40 & 90.20 & 118.42 \\
10 & 5.32 & 15.68 & 58.39 & 31.00 & 66.67 & \underline{87.82} \\
\bottomrule
\label{table:earning_driver}
\end{tabular}
\end{table}


Using our trip assignment simulation algorithm (\cref{alg:concise_trip_assignment}), we analyzed driver earnings and trip behavior for the second week of August in 2019 and 2023.


\paragraph{\textbf{Clustering Reveals Distinct Earning Profiles.}}

Our simulation for the two weeks in 2019 and 2023 yielded a total of 364,452 and 259,812 drivers, respectively. To focus on active drivers, we filtered for those completing more than two trips daily, resulting in samples of 164,234 drivers in 2019 and 114,920 in 2023. Using KMeans clustering with Silhouette Score optimization, we identified 9 distinct driver groups in 2019 and 11 in 2023. These clusters revealed diverse patterns in trip frequencies, average fares, and hourly earnings, demonstrating significant variations across different segments of the driver population (\cref{tab:predict_cluster_case_2_no_cluster_9}). The t-SNE visualizations in \cref{sec:appendix} illustrate the spatial distribution of these cluster groups, validating both our simulation methodology and clustering approach.

Table~\ref{tab:predict_cluster_case_2_no_cluster_9} presents some key findings regarding the identified driver profiles:

\begin{enumerate}
    \item In 2019, Cluster 0 had the highest total income (\$288.70) and fares (\$216.10), as they performed the most trips, while Cluster 7 achieved the highest earnings per hour (\$60.53).
    \item By 2023, overall incomes increased, with Cluster 9 achieving the highest earnings per drive hour (\$96.66) and per trip (\$21.28).
    \item However, disparities widened, with Cluster 8 and Cluster 10 earning significantly less (\$12.74 per trip and \$87.82 total income, respectively).
\end{enumerate}

These results may suggest that driver work strategies have become more differentiated, with some groups benefiting from shifts in demand while others are increasingly disadvantaged.


\paragraph{\textbf{Emerging New Driver Groups.}}

Our clustering approach also revealed new driver groups in 2023 that were not present in 2019, particularly those concentrating a higher proportion of their trips in the South Side and Far Southeast Side (Driver Group 8) and the Northwest Side (Driver Group 10). This suggests emerging spatial specialization, where certain drivers may be systematically targeting or being assigned lower-demand neighborhoods. The growing concentration of drivers in historically lower-paying areas raises questions about platform steering and whether algorithmic matching is reinforcing existing income disparities.

% \hdcomment{Should we put this into Appendix?}


% \hdcomment{@Tamara, Jason: In this section, we mention about how working hours (hours that drivers work per day affect the earnings? In our findings, we see that specific timeframes can yield more earnings than other working time? Do we have a qualitative analysis about if some specific chunks of time can have more/better earnings/working hour?}
% \hdcomment{@Tamara, Jason: Additionally, do we have some qualitative information about how much times drivers usually drive per day?}
% %
% \tlcomment{need to actually interprete the data and spell out the results -- ideally at the beginning in bold e.g., ``differences in driver earning based on temporal patterns'' and clearly state the finding }
\paragraph{\textbf{Differences in Driver Earnings Based on Temporal Patterns.}}

\begin{figure}[ht]
  \centering
  \includegraphics[width=\linewidth]{figures/simulation_result/time_1.pdf}
  \caption{Proportion of Trip in Different Time Intervals for Different Driver Group Clusters in the Week of 2019-08-05. The proportion of trips for different driver group clusters in 2019 shows distinct patterns across various time intervals. Certain driver groups concentrate their activity in specific periods, such as late evening (with more than 60\% of trips starting within 21--24 hours for Driver Group 1) or early morning (with about 80\% of trips starting within 0--3 hours for Driver Group 6). Meanwhile, Driver Group 0 have a more evenly distributed trip pattern throughout the day.
}
  \label{figure:time_chunk_2019}
\end{figure}

\begin{figure}[H]
  \centering
  \includegraphics[width=\linewidth]{figures/simulation_result/time_2.pdf}
  \caption{Proportion of Trip in Different Time Interval For Different Driver Group Cluster in the week of 2023-08-07. The proportion of trips for different driver group clusters in 2023 demonstrates a similar variety of temporal driving patterns as in 2019. Most of the groups focus heavily on specific intervals, while others distribute their trips more broadly across the day (Driver Group 0, 8 and 10).}

  \label{figure:time_chunk_2023}
\end{figure}



Figures~\ref{figure:time_chunk_2019} and~\ref{figure:time_chunk_2023} illustrate how different driver groups distribute their trips across the day, showing strong variations in work schedules and earnings potential. In both years, certain groups concentrate on \textbf{late-evening} or \textbf{overnight} windows and \textbf{achieve relatively high per-trip revenue}, while others who maintain heavier daytime schedules do not necessarily attain higher hourly wages. For instance, one group logs the greatest number of trips between 9AM and 6PM (Driver 0 in both 2019 and 2023) but still reports moderate or even below-average hourly earnings. Meanwhile, a different group working fewer trips between 9PM and 12AM records top-tier rates (Driver 1 in 2019). Similarly, two groups driving the same time blocks can yield markedly different pay if they serve distinct neighborhoods or exploit surge-pricing periods differently. Overall, these patterns highlight that although temporal availability matters, it intersects with factors like spatial demand, platform matching, and each driver’s individual strategies, resulting in significant earning disparities among those who choose similar—or even overlapping—hours. 

% \tlcomment{i wouldn't call this opportunity for optimizing driver schedules -- since it's a dynamic market -- sending everyone to the "high earning" hours would not result in everyone making more, for example}

% \paragraph{Regional Distribution Characteristics for Simulated Drivers}
% \hdcomment{@Tamara, Jason: In this section, we mention about how dropoff/pickup location affect the earnings? In our findings, we see that specific locations can yield more earnings than others? Do we have a qualitative analysis about if some specific locations have better earnings?}
% \hdcomment{@Tamara, Jason: Also, below figure shows that drivers mainly on some specific groups of locations instead go through the whole Chicago. Do we have some qualitative analysis on where drivers choose to work? Or Driver rationale for or against certain neighborhoods: Some areas may be avoided due to perceived wait times, safety concerns, or fewer premium fares.}
% \hdcomment{@Tamara, Jason: That would be great if we have some qualitative information that is relevant to Platform steering practices: How do driver-side apps (e.g., “destination filters”) and personal constraints (fuel costs, comfort zones) shape these geographic patterns?}
\paragraph{\textbf{Differences in Driver Earnings Based on Regional Distribution.}}
\begin{figure}[ht!]
  \centering
  \includegraphics[width=0.85\linewidth]{figures/simulation_result/2019_driver_region.png}
\caption{Proportion of Trips by Pickup and Dropoff Locations for Different Driver Group Clusters in the Week of 2019-08-05. The Central region shows the highest concentration of trips across most driver groups, accounting for about 25\% of trips performed by driver groups. Other high-demand areas include the North and West Sides, accounts for most amount for several groups. However, Far Southeast Side and Far Southwest Side show significantly lower trip proportions, below 5\% for all driver groups.}
  \label{figure:dropoff_pickup_2019}
\end{figure}
Figures~\ref{figure:dropoff_pickup_2019} and~\ref{figure:dropoff_pickup_2023} indicate that many drivers in both years concentrate on the Central and West Side region. Notably, some drivers show increased airport-focused trips (Driver 8 in 2019 and Driver 9 in 2023), which correlate with \textbf{higher per-trip income} (see Table~\ref{table:earning_driver}). However, driver group which serves the South Side, Far Southwest Side, and Far Southeast Side (Driver Group 8 in 2023) exhibit \textbf{substantially lower average earnings}. Although ride volume in these lower-paying areas has grown from 2019 to 2023, the income gains remain minimal, with driver group 8 reports earnings as low as \$12.74/hour. Our trip assignment simulation algorithm also reveals that, by 2023, a distinct driver group emerged, concentrating on the South Side—a shift not observed in 2019. These disparities highlight how different neighborhoods offer vastly uneven fare opportunities, driven by factors such as longer wait times, perceived safety concerns, and fewer premium trips. Such geographic variations, whether shaped by driver preferences or platform-side decisions, play a crucial role in perpetuating income gaps among drivers.


\begin{figure}[H]
  \centering
  \includegraphics[width=0.85\linewidth]{figures/simulation_result/2023_driver_region.png}
\caption{Proportion of Trips by Pickup and Dropoff Locations for Different Driver Group Clusters in the Week of 2023-08-07. The Central region remains the most active area for most driver groups (except Driver Group 8 and 10). Driver Group 8 has also emerged as a distinct cluster focusing on low-activity areas, with about 30\% and 20\% of their trips occurring in the Far Southeast Side and South Side, respectively. Additionally, Driver Group 10 now concentrates more than 40\% of their trips in the Northwest Side, a pattern not observed in 2019.}

  \label{figure:dropoff_pickup_2023}
\end{figure}

\subsection{Summary}

Our findings reveal that temporal and regional disparities significantly influence ride-sharing driver earnings. While potential pricing model shifts led to temporary wage increases in early 2021, earnings have since stagnated, resulting in a decline in real, inflation-adjusted income. Moreover, regional earning gaps have widened, with drivers in lower-demand areas earning considerably less than their counterparts in high-traffic regions like the Central and Airport areas.

By clustering our constructed hypothetical drivers, we identify distinct driver groups with unique work patterns, uncovering new spatial trends and emerging low-income driver clusters in 2023. Our results suggest that platform pricing strategies, driver relocation patterns, and algorithmic matching mechanisms are key factors shaping driver income disparities.

