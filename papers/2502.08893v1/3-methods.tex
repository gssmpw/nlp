\section{Methods}
\label{sec:methods}

Using the public Chicago ride-sharing dataset from 2018 to 2023~\cite{chicago_tnp_2018, chicago_tnp_2023}, our analysis focuses on temporal and regional ride price patterns, as well as driver work patterns. In all, we aim to answer the following research questions with our data analysis:


\textbf{RQ1:} How do ride-sharing trip \textit{costs change over time}?
% \emph{We investigate whether fare increases keep pace with cost-of-living trends and whether particular periods show especially sharp changes.}

\textbf{RQ2:} How do ride-sharing trip \textit{locations} impact trip costs?

\textbf{RQ3:} What are differences and similarities in ride-sharing \textit{drivers' work patterns}?

% \hdcomment{Things we consider: 1. Fare Earning Per Mile, 2. Total Trip Earning Per Mile (Fare + Fee + Tips), 3. Estimated Earning/ Driving Hour, 
% 4. Estimated Earning / Hour (+0.25 hour) for the wait time between trips
% 5. Total Income Per Working Day (Can only be obtained by driver profiling algorithm)}

\subsection{Dataset}

The Chicago Transportation Network Providers Dataset includes all trips in Chicago reported by ride-sharing companies since 2018 to 2023. The dataset contains various data fields for each ride-sharing trip, such as start/end timestamps, trip durations and distances, locations (pickup and drop-off), and price-related information, which is further broken down into fares, tips, and fees. This large dataset contains around 300 million records from August 2018 to December 2022 and around 167 million records from January 2023 to the present\footnote{January 2025, as of writing this paper.}, averaging 200,000 to 300,000 records per day. It captures comprehensive aspects of ride-sharing trips and has been used in previous research to assess algorithm fairness~\cite{zhang2024data, manzo2022improving}.

Note that the earning numbers in our analysis were calculated using the fare and tips in the dataset. They do not include the portion taken by the platform, which is not available in the dataset. The last time Uber, the largest platform in Chicago, reported its ``take rate'' was in its 2023 earning report. Its average take rate was 29.3\% globally in the second quarter of 2023\footnote{\url{https://investor.uber.com/news-events/news/press-release-details/2023/Uber-Announces-Results-for-Second-Quarter-2023/default.aspx}}.  

\subsection{Data Analysis Metrics and Algorithms}
\subsubsection{Trip Price Shift Over Time (RQ1)}
To understand trip price temporal changes, we group all trips by months, and compare the \emph{average cost per hour} for each month's aggregated data\footnote{We use ``cost'' to refer to the trip total in the Chicago dataset, which includes fare, tip, and additional charges. Trip total is what a passenger pays the platform, thus we call it ``cost'' in our paper for ease of understanding.}. Our analysis of price temporal changes focus on the time period from November~2018 to December~2023.

% In the Chicago dataset, there are four pieces of monetary information associated with each trip: (1) trip total, which is the sum of (2) fare, (3) tip, and (4) additional charges (including tax, fees, and other charges). When calculating \textit{costs per mile} and \textit{costs per hour}, we use ``costs'' to refer to trip total.

\subsubsection{Trip Location Distribution and Spatial Pricing Patterns (RQ2)}
We also investigate how trips' pickup and drop-off locations are distributed across various regions in Chicago. Then, we compare the price of trips in different regions. We consolidate the trip data into seven regions (e.g., Central, North, West) following the classification used by the Chicago government\footnote{\url{https://data.cityofchicago.org/Community-Economic-Development/Boundaries-Planning-Regions-Map/2wek-zf5g}}. We aggregate the Chicago dataset by years to investigate these regional patterns, and left out 2018 since the dataset only contains 2 months for this year. Similar to above, we used average cost per hour as our metric to compare regional trip prices across years. 

% Yuwen: I removed references 77 distinct community areas, because it feels unnecessary

% we group trips by region and visualize costs per hour of driving passengers have to pay based on the trip durations reported in the dataset. 
% \hdcomment{@Tamara, Jason: Do we have a qualitative information about how much wait time the drivers willing to wait? And how far for the next pickup trips driver will likely to accept the drive? Currently, I am setting wait-time = 0.25 and maximum distance for the next pickup trip from the current drop-off <= 1 miles}


% This estimate is at the high end of the range of observed values from drivers in the qualitative interviews we conducted. There was significant variation and heterogeneity amongst interviewed drivers in how far they were willing to travel and how long they were willing to wait. We selected a value for initial modeling purposes which was at the high end of the range of observed values.

\subsubsection{Driver-Level Work Pattern Analysis (RQ3)}
\label{sec:methods-driver-simulation}


A main goal of our data analysis is to understand ride-sharing drivers' earnings and work patterns. However, the Chicago ride-sharing dataset, similar to many other datasets in this domain, does not associate trips with either drivers or passengers. Such practice is to protect passengers' and drivers' privacy---once we link together all rides requested by a passenger or provided by a driver throughout the years, it is not impossible to uncover the person's identity from the activity patterns, even if the data is anonymized. As a result, from the Chicago dataset, it is difficult to identify driver profiles, posing a challenges for our analysis. This prevents us from estimating drivers' total earnings and work hours, which is necessary for studying biases, fairness among driver groups.

To address this challenge, we developed a trip-based driver assignment algorithm (\cref{alg:concise_trip_assignment}), which \emph{assigns} ride-sharing trips to hypothetical drivers under realistic spatiotemporal constraints. Our algorithm works under one assumption---if a driver takes a new trip after finishing a previous one, it is likely that the second trip's pickup location is close to the first one's drop-off location (\textit{spatial constraint}), and the second trip starts shortly after the first one (\textit{temporal constraint}). Based on this assumption, our algorithm assigns trips that are \textit{spatially} and \textit{temporally} close to each other to one hypothetical driver. We also make the assumption that a driver does not work continuously for more than 8 hours or 25 trips without a break.

\begin{algorithm}[ht]
\caption{Trip Assignment Simulation Algorithm}
\label{alg:concise_trip_assignment}
\DontPrintSemicolon

\KwIn{Data frame \(\mathit{df}\) with trips; time threshold \(\alpha\); distance threshold \(\textit{maxDist}\).}
\KwOut{Mapping of trips to driver routes.}

\While{\(\mathit{df}\) is not empty}{
  \(\textit{currentDriverTrips} \gets \{\}\); 
  \(\textit{currentTrip} \gets\) Randomly pick from \(\mathit{df}\)\;
  \(\textit{currentDriverTrips} \leftarrow \textit{currentTrip}\); 
  Remove \(\textit{currentTrip}\) from \(\mathit{df}\)\;

  \While{driver constraints not exceeded}{
    \(\mathit{candidates} \gets\) trips in \(\mathit{df}\) where 
      start time is in \([\textit{currentTrip.endTime}, \textit{currentTrip.endTime} + \alpha]\)\;
    Filter out candidates with distance to \(\textit{currentTrip.dropoff}\) 
      \(> \textit{maxDist}\)\;
    \If{\(\mathit{candidates}\) is empty}{\textbf{break}}
    \(\textit{nextTrip} \gets\) Randomly pick from top feasible candidates\;
    \(\textit{currentDriverTrips} \leftarrow \textit{nextTrip}\); 
    Remove \(\textit{nextTrip}\) from \(\mathit{df}\)\;
    \(\textit{currentTrip} \gets \textit{nextTrip}\);
  }

  Assign \(\textit{currentDriverTrips}\) to a new driver\;
}
\end{algorithm}


Algorithm~\ref{alg:concise_trip_assignment} describe the detailed procedure to assign trips to drivers. In the beginning, we randomly pick one unassigned trip as the starting point. Then, from other unassigned trips, we find trips that are spatiotemporally close to the selected trip, then append the new trip to the selected one. We iteratively apply this step to form a \emph{driver route}. When appending a new trip to a currently selected trip, we apply the following constraints:
\begin{enumerate}
    \item \textbf{Temporal constraint}: The new trip's start time must be after the current trip's end time but within a threshold $\alpha$ (e.g., 15 minutes).
    \item \textbf{Spatial constraint}: The distance between the current trip's drop-off location and the next trip's pickup location must be less than a specified threshold \texttt{max\_distance}.
    \item \textbf{Driver constraints}: Each driver is assumed to work for no more than 8 hours per day or a maximum of 25 trips, whichever is reached first.
\end{enumerate}

When there are multiple candidate trips that meet these constraints, we randomly pick one from the candidates. When running this algorithm on the Chicago dataset, we set the time threshold $\alpha = 0.25$ (hour) and maximum distance \text{maxDist} = 1 (mile). This results in a list of hypothetical drivers and their associated trips. 

After we assign trips to the hypothetical drivers, we used the KMeans algorithm~\cite{lloyd1982least} to cluster drivers into groups and analyze their work patterns. The drivers are clustered based on their associated trip features, including fares, earnings, pickup and drop-off locations, trip timestamps and distances. Drop-off and pickup timestamps are converted into three-hour intervals to improve clustering quality. 

Given the large size of the Chicago dataset, it is very computationally expensive to run the algorithm on all available data. Thus, we randomly select the second week of August in years 2019 (August 5-11) and 2023 (August 7-13) to conduct our analysis. To determine the optimal number of clusters, we compute the Silhouette Score~\cite{rousseeuw1987silhouettes} for a range of cluster values from 4 to 16 and select the value with the highest score as the final number of clusters. As a result, we had 9 driver groups for the week in 2019 and 11 groups for 2023.

After clustering, for each driver group, we evaluate their work pattern on metrics including frequency of pickup/drop-off trips in community areas, work time periods, and aggregated daily trip earnings\footnote{Note that the Chicago dataset does not contain driver earning information but just trip cost for passengers. We use the trip total as a reasonable proxy to estimate driver earning in analysis of hypothetical driver earnings.}. Since the dataset lacks detailed wait-time information, we apply a 0.25-hour wait time to all trips to approximate the real work hours.









% \subsubsection{Problem Definition}
% \label{sec:problem_definition}

% Let us consider a single date, denoted by $X$, during which a ridesourcing platform completes $N$ trips. Each trip $i \in \{1, \dots, N\}$ is characterized by a set of features:
% \begin{itemize}
%     \item $\mathbf{p}_i$: the pickup location (in coordinates),
%     \item $\mathbf{d}_i$: the drop-off location (in coordinates),
%     \item $t^\text{start}_i$: the trip's start time,
%     \item $t^\text{end}_i$: the trip's end time,
% \end{itemize}

% Our objective is to \emph{group} or \emph{cluster} these $N$ trips into subsets, where each subset represents a \emph{plausible sequence of trips} performed by a single driver over the course of day $X$. Formally, we seek a partition of the set of trips $\{1, \dots, N\}$ into $K$ subsets $\mathcal{D}_1, \mathcal{D}_2, \dots, \mathcal{D}_K$, where each $\mathcal{D}_k$ is an ordered sequence of trip indices:
% \[
% \mathcal{D}_k = \{ i_1, i_2, \dots, i_{|\mathcal{D}_k|} \},
% \]
% such that:
% \begin{enumerate}
%     \item \textbf{Temporal feasibility}: For any consecutive trips $i_j, i_{j+1}$ in $\mathcal{D}_k$, the end time of trip $i_j$ plus any required transition time (e.g., for traveling from $\mathbf{d}_{i_j}$ to $\mathbf{p}_{i_{j+1}}$) must not exceed the start time of trip $i_{j+1}$.
%     \item \textbf{Geographic feasibility}: The distance between $\mathbf{d}_{i_j}$ and $\mathbf{p}_{i_{j+1}}$ must be reasonable for a driver to travel between consecutive trips within the allotted time window.
%     \item \textbf{Uniqueness}: Each trip $i$ belongs to exactly one subset $\mathcal{D}_k$, reflecting the assumption that a single driver cannot overlap with another driver on the same trip.
% \end{enumerate}

% Intuitively, each subset $\mathcal{D}_k$ can be interpreted as a \emph{driver’s route} for the day. By creating these subsets, we can model how much each hypothetical driver earns, how many trips they complete, and how their trips are distributed geographically and temporally. This partitioning approach allows us to:
% \begin{itemize}
%     \item Examine cumulative earnings at the driver level.
%     \item Analyze the spatial and temporal patterns of trip assignments.
%     \item Investigate potential disparities or inefficiencies in ridesourcing operations.
% \end{itemize}

% In summary, our goal is to construct a simulation framework that \emph{assigns} the observed trips to an arbitrary number of hypothetical drivers under realistic spatiotemporal constraints. This, in turn, enables fairness and equity analyses despite the absence of explicit driver identifiers or personal data in publicly available trip records.

% \subsection{} 
% % Our approach mainly aims to match trips into drivers, thus, we can study routes and behaviors of drivers. Thus, we can study locations and variables that may affects drivers' earnings. 


\subsubsection{Summary}
Using these three main types of analysis metrics, we provide a comprehensive understanding of the ride-sharing landscape in Chicago. By examining temporal shifts in trip costs (RQ1), we assess how ride prices have evolved over time and whether these changes align with broader economic trends. The spatial pricing analysis (RQ2) enables us to evaluate regional variations in ride costs. Finally, through our driver work pattern analysis (RQ3), we approximate driver activity and earnings despite the dataset's anonymization, allowing us to cluster and distinguish different driver behaviors. Together, these analyses offer a multi-faceted view of ride-sharing operations, helping us uncover key trends and potential disparities in the platform algorithms.