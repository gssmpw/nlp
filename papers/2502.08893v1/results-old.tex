% \section{Results}
% \subsection{Preliminary Results}
% \begin{itemize}
%     \item Studying pricing factors through times 
%     \item Studying allocations trips among region/earnings across time, earnings/totals earnings
% \end{itemize}
% \hdcomment{@Tamara, Jason: Do we know about the information about the pricing changes and if that aligns with some of results in this sections (Diagram)? }
% As discussed in previous sections, we aim to examine pricing trends over time to observe potential changes in pricing models during different periods. Figures 1 and 2 illustrate monthly metrics comparisons from the Chicago Rideshare Data, spanning November 2018 to December 2023. Figure 1 focuses on earnings metrics per driving time per hour, including average trip total income, profit, and fare earnings, while Figure 2 presents metrics related to earnings per mile driven.

% The data highlights noticeable changes starting in January 2021, with a marked increase in earnings across all metrics, reaching a peak in August 2021. This period may reflect adjustments in pricing strategies, increased demand, or other external factors such as economic recovery or promotional events. However, the figures also reveal fluctuations after the peak, followed by a steady decline throughout 2023. This suggests a stabilization of pricing models since 2021, with no significant changes in pricing factors observed in the later period.
% \begin{figure}[h]
%   \centering
%   \includegraphics[width=\linewidth]{figures/by_hour.jpg.png}
%   \caption{Earning Per Hour}
% \end{figure}

% \begin{figure}[h]
%   \centering
%   \includegraphics[width=\linewidth]{figures/per_mile.jpg.png}
%   \caption{Earning Per Mile}
% \end{figure}

% \hdcomment{It may need a paragraph to depict that this is not okay for the earning remain the same after 3 years?}
% These trends provide valuable insights into the temporal dynamics of pricing and earnings, highlighting potential opportunities to further analyze underlying factors driving these shifts, such as seasonal demand, fuel costs, or market competition.

% \subsection{Simulation Results}
% \begin{itemize}
%     \item Simulation Results + Clustering to model drivers' profiles for weekly and study differences between 2 years
%     \item Earnings for different drivers Groups
%     \item Earnings Through Time Work
%     \item Earnings Through Zone Work 
% \end{itemize}

% \hdcomment{Currently, the results contain analysis for one date, how can we aggregate analysis for the whole weeks? Should we put in ablation studies?}

% In this section, we present the results for a single date (2019-08-05) using our simulation algorithm. The simulation yielded a total of 51,631 drivers. However, since our study focuses on analyzing trip patterns for drivers making more than two trips per day, we narrowed the dataset to 21,119 drivers, which is 40.9\% of the simulated drivers.

% By applying the KMeans algorithm to the processed dataset, we identified 11 distinct clusters. The choice of 11 clusters was based on achieving the highest Silhouette Score among the experimented cluster numbers, indicating optimal cluster quality.

% To further examine and validate the clustering results, we utilized t-SNE visualization, as shown in Figure \ref{figure:cluster}, which provides an intuitive representation of the driver groups and their separability in reduced dimensions.
% \begin{figure}[h]
%   \centering
%   \includegraphics[width=\linewidth]{figures/clustering_result.png}
%   \caption{(Left) t-SNE visualization of the best number of clusters using KMeans algorithms and Sihoutte scores on simulated drivers for 2019-08-05. (Right) Distribution of different clusters}
%   \label{figure:cluster}
% \end{figure}

% \paragraph{Earnings For Different Driver Groups}
% \hdcomment{@Tamara, Jason: The table depicts "Significant" Differences Between Different Drivers Groups Using Our Simulation Models and We can see clearly gaps Between Earning Per Hour (For Example, Driver 0 earns 44.75\$/Driving Hour But Driver 3 earns 60.68\$/Driving Hour). Do we have some qualitative analysis that can support that? Differences in earning among different drivers?}
% Table \ref{tab:earning_driver} highlights key performance metrics across drivers, providing insights into operational efficiency and revenue generation. Driver ID 6 emerges as a significant group, completing the highest number of trips (20.38) and generating the largest total fares (\$225.50) and income (\$301.12). Despite this, their earnings per trip (\$14.14) and per hour (\$48.83) are relatively modest, suggesting a focus on volume over profitability. In contrast, Driver ID 7 demonstrates a different strategy, achieving the highest earnings per trip (\$16.20) and maintaining competitive earnings per hour (\$60.31) with fewer trips. Similarly, Driver ID 3 exhibits high efficiency, achieving the highest earnings per hour (\$60.68) despite performing fewer trips overall. On the other hand, Driver ID 8 shows lower performance across most metrics, indicating potential inefficiencies or reduced activity levels. These indicated number of trips perform can significantly affect total income the driver can earn. However, when having a closer study of average earning per trip and per hour, we see that number of trips are not directly related. This analysis underscores the need for a data-driven approach to driver management and performance optimization.

% \paragraph{Frequency of Trips Per Time Chunk}
% \hdcomment{@Tamara, Jason: In this section, we mention about how frequency of trips can be a factor to affect the earnings average (hourly wage), however, hear, we claimed that this is not a factor, for example, Driver 6 drives large number of trips but on averages, he/she doesn't have the high earning/hours. Do we have some qualitative information to support how frequency of trips a driver usually drives per day or how they affect the earnings?}
% \begin{table*}
%   \caption{Earning Metrics For Different Driver Cluster on Simulated Drivers}
% \label{tab:earning_driver}
%   \begin{tabular}{cccccc}
%     \toprule
%     Driver ID & Number of Trips Performed & Earning Per Trips $(\$)$ & Earning Per Hour $(\$)$ & Total Fares $(\$)$ & Total Income $(\$)$ \\
%         \midrule
%         0 & 7.02 & 13.59 & 46.75 & 74.52 & 99.56 \\
%         1 & 7.38 & \underline{11.69} & 54.18 & 65.82 & 89.94 \\
%         2 & 5.56 & 15.02 & 48.70 & 65.52 & 87.31 \\
%         3 & 5.14 & 11.91 & \textbf{60.68} & 47.49 & 64.04 \\
%         4 & 5.94 & 12.70 & \underline{44.83} & 59.00 & 78.38 \\
%         5 & 5.21 & 12.26 & 51.31 & 47.95 & 66.28 \\
%         6 & \textbf{20.38} & 14.14 & 48.83 & \textbf{225.50} & \textbf{301.12} \\
%         7 & \underline{4.13} & \textbf{16.20} & 60.31 & 53.40 & 70.42 \\
%         8 & 4.43 & 12.67 & 48.15 & \underline{44.01} & \underline{58.45} \\
%         9 & 5.87 & 12.29 & 47.67 & 60.67 & 75.92 \\
%         10 & 5.48 & 12.93 & 48.78 & 59.35 & 74.46 \\
%         \bottomrule
%   \end{tabular}
% \end{table*}
% \hdcomment{@Tamara, Jason: In this section, we mention about how working hours (hours that drivers work per day affect the earnings? In our findings, we see that specific timeframes can yield more earnings than other working time? Do we have a qualitative analysis about if some specific chunks of time can have more/better earnings/working hour?}
% \hdcomment{@Tamara, Jason: Additionally, do we have some qualitative information about how much times drivers usually drive per day?}
% Figure \ref{figure:time_chunk} illustrates the distribution of trips, segmented by start-time (top) and end-time (bottom) for each driver across various time intervals. The temporal patterns vary significantly across drivers, reflecting diverse operational schedules and preferences. For instance, Driver 6 demonstrates consistent activity, particularly between the 9-15 hour interval, aligning with their high number of total trips as noted in the previous analysis. This period represents their most productive window, contributing substantially to their revenue. Conversely, Driver 7 exhibits a marked increase in activity during the 3-6 hour interval, showcasing their efficiency in completing fewer but high-value trips. Driver 1, who has the lowest earnings per trip, primarily operates during the 18-24 hour interval, with a peak between 21-24 hours. This suggests that working hours may influence the earnings of drivers. Additionally, the figure highlights that most drivers focus on distinct time intervals, with the exception of Drivers 6, 9, and 10, who maintain consistent activity across various time periods. Overall, the temporal data underscores opportunities to optimize driver schedules and examine correlations between driver patterns and earnings.
% \begin{figure}[ht]
%   \centering
%   \includegraphics[width=\linewidth]{figures/number_of_trips_time_chunks.png}
%   \caption{Frequency and Distribution of Pickup and Dropoff Trip in Different Time Interval For Different Driver Cluster}
%   \label{figure:time_chunk}
% \end{figure}

% \hdcomment{@Tamara, Jason: In this section, we mention about how dropoff/pickup location affect the earnings? In our findings, we see that specific locations can yield more earnings than others? Do we have a qualitative analysis about if some specific locations have better earnings?}
% \hdcomment{@Tamara, Jason: Also, below figure shows that drivers mainly on some specific groups of locations instead go through the whole Chicago. Do we have some qualitative information that supports this statement?}
% \begin{figure}[ht]
%   \centering
%   \includegraphics[width=\linewidth]{figures/number_of_trips_dropoff_pickup.png}
% \caption{Frequency and Distribution of Pickup and Dropoff Trip in Different Locations (Community Area) For Different Driver Cluster}
%   \label{figure:dropoff_pickup}
% \end{figure}
% Similarly, we analyze the driver groups based on the distribution of trips, segmented by drop-off and pickup locations. According to Figure \ref{figure:dropoff_pickup}, most simulated driver groups exhibit a similar pattern, primarily operating in the Central community areas, followed by the North and West Sides. Exceptions are observed with Drivers 9 and 10. Driver 9 predominantly operates in the Far Southeast Side, accounting for approximately 40\% of their total trips. In contrast, Driver 10 focuses on the South Side and Far Southwest Side, contributing about 15\% and 35\% of their trips, respectively. Moreover, Driver 9's earning lie as one of the lowest earning groups (bottom half) among all 11 driver groups among all metrics. A notable observation is Driver 7, who, aside from trips in the Central region, shows the second-highest number of trips dropping off at the Airport, representing nearly 20\% of their total trips — the highest among all simulated drivers. Interestingly, Driver 7 also achieves the highest earnings per trip, as indicated in Table. \ref{table:earning_driver}. This suggests that location-specific factors may significantly influence drivers' earnings, and simulated drivers can reveal such patterns using public datasets. Overall, this regional analysis underscores opportunities to optimize geographic coverage, better align driver allocation with demand hotspots, and demonstrates the effectiveness of simulation approaches in modeling these patterns, which can serve as valuable starting points.
