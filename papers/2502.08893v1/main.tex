\documentclass[manuscript,screen]{acmart}

%%
%% \BibTeX command to typeset BibTeX logo in the docs
\AtBeginDocument{%
  \providecommand\BibTeX{{%
    Bib\TeX}}}

%% Rights management information.  This information is sent to you
%% when you complete the rights form.  These commands have SAMPLE
%% values in them; it is your responsibility as an author to replace
%% the commands and values with those provided to you when you
%% complete the rights form.
\setcopyright{none}
\copyrightyear{2018}
\acmYear{2018}
\acmDOI{XXXXXXX.XXXXXXX}

%% These commands are for a PROCEEDINGS abstract or paper.
\acmConference[Conference acronym 'XX]{Make sure to enter the correct
  conference title from your rights confirmation emai}{June 03--05,
  2018}{Woodstock, NY}
%%
%%  Uncomment \acmBooktitle if the title of the proceedings is different
%%  from ``Proceedings of ...''!
%%
%%\acmBooktitle{Woodstock '18: ACM Symposium on Neural Gaze Detection,
%%  June 03--05, 2018, Woodstock, NY}
\acmISBN{978-1-4503-XXXX-X/18/06}


%%
%% Submission ID.
%% Use this when submitting an article to a sponsored event. You'll
%% receive a unique submission ID from the organizers
%% of the event, and this ID should be used as the parameter to this command.
%%\acmSubmissionID{123-A56-BU3}

%%
%% For managing citations, it is recommended to use bibliography
%% files in BibTeX format.
%%
%% You can then either use BibTeX with the ACM-Reference-Format style,
%% or BibLaTeX with the acmnumeric or acmauthoryear sytles, that include
%% support for advanced citation of software artefact from the
%% biblatex-software package, also separately available on CTAN.
%%
%% Look at the sample-*-biblatex.tex files for templates showcasing
%% the biblatex styles.
%%

%%
%% The majority of ACM publications use numbered citations and
%% references.  The command \citestyle{authoryear} switches to the
%% "author year" style.
%%
%% If you are preparing content for an event
%% sponsored by ACM SIGGRAPH, you must use the "author year" style of
%% citations and references.
%% Uncommenting
%% the next command will enable that style.
%%\citestyle{acmauthoryear}

\usepackage[ruled,vlined]{algorithm2e}


\usepackage[capitalize,noabbrev]{cleveref}

\newcommand{\tlcomment}[1]{\noindent{\\\textcolor{magenta}{\textbf{\#\#\# TL:} \textsf{#1} \#\#\#\\}}}

\newcommand{\hdcomment}[1]{\noindent{\\\textcolor{red}{\textbf{\#\#\# HD:} \textsf{#1} \#\#\#\\}}}

%%
%% end of the preamble, start of the body of the document source.
\begin{document}

%%
%% The "title" command has an optional parameter,
%% allowing the author to define a "short title" to be used in page headers.
\title{Uncovering Disparities in Rideshare Drivers’ Earning and Work Patterns: A Case Study of Chicago}
%Possible Title
% Uncovering disparities in ride-sharing drivers' earning and work patterns: A case study of Chicago

% Algorithmic Fairness for Ride-Sharing Drivers: A Chicago Case Study

% Uncovering Driver Equity Gaps in Ride-Sharing Algorithms: A Chicago Case Study

% Measuring Algorithmic Disparities in Urban Ride-Sharing: A Chicago Case Study

%%
%% The "author" command and its associated commands are used to define
%% the authors and their affiliations.
%% Of note is the shared affiliation of the first two authors, and the
%% "authornote" and "authornotemark" commands
%% used to denote shared contribution to the research.
\author{Hy Dang}
% \authornote{Both authors contributed equally to this research.}
\email{hdang@nd.edu}

\affiliation{%
  \institution{University of Notre Dame}
  \city{Notre Dame}
  \state{Indiana}
  \country{USA}
}

\author{Yuwen Lu}
\affiliation{%
  \institution{University of Notre Dame}
  \city{Notre Dame}
  \state{Indiana}
  \country{USA}
}


\author{Jason Spicer}
\affiliation{%
 \institution{City University of New York}
 \state{New York}
 \country{USA}}

\author{Tamara Kay}
\affiliation{%
  \institution{University of Notre Dame}
  \state{Indiana}
  \country{USA}
}
\author{Di Yang}
\affiliation{%
  \institution{Morgan State University}
  \city{Baltimore}
  \state{Maryland}
  \country{USA}}

\author{Yang Yang}
\affiliation{%
  \institution{University of Notre Dame}
  \city{Notre Dame}
  \state{Indiana}
  \country{USA}
}
\author{Jay Brockman}
\affiliation{%
  \institution{University of Notre Dame}
  \city{Notre Dame}
  \state{Indiana}
  \country{USA}
}
\author{Meng Jiang}
\affiliation{%
  \institution{University of Notre Dame}
  \city{Notre Dame}
  \state{Indiana}
  \country{USA}
}
\author{Toby Jia-Jun Li}
\affiliation{%
  \institution{University of Notre Dame}
  \city{Notre Dame}
  \state{Indiana}
  \country{USA}
}
%%
%% By default, the full list of authors will be used in the page
%% headers. Often, this list is too long, and will overlap
%% other information printed in the page headers. This command allows
%% the author to define a more concise list
%% of authors' names for this purpose.
\renewcommand{\shortauthors}{Dang et al.}

%%
%% The abstract is a short summary of the work to be presented in the
%% article.
\begin{abstract}
Ride-sharing services are revolutionizing urban mobility while simultaneously raising significant concerns regarding fairness and driver equity. This study employs Chicago’s Trip Network Provider dataset to investigate disparities in ride-sharing earnings between 2018 and 2023. Our analysis reveals marked temporal shifts, including an earnings surge in early 2021 followed by fluctuations and a decline in inflation-adjusted income, as well as pronounced spatial disparities, with drivers in Central and airport regions earning substantially more than those in peripheral areas. Recognizing the limitations of trip-level data, we introduce a novel trip-driver assignment algorithm to reconstruct plausible daily work patterns, uncovering distinct driver clusters with varied earning profiles. Notably, drivers operating during late-evening and overnight hours secure higher per-trip and hourly rates, while emerging groups in low-demand regions face significant earnings deficits. Our findings call for more transparent pricing models and a re-examination of platform design to promote equitable driver outcomes.
\end{abstract}
\maketitle

%%
%% The code below is generated by the tool at http://dl.acm.org/ccs.cfm.
%% Please copy and paste the code instead of the example below.
%%
\begin{CCSXML}
<ccs2012>
 <concept>
  <concept_id>00000000.0000000.0000000</concept_id>
  <concept_desc>Do Not Use This Code, Generate the Correct Terms for Your Paper</concept_desc>
  <concept_significance>500</concept_significance>
 </concept>
 <concept>
  <concept_id>00000000.00000000.00000000</concept_id>
  <concept_desc>Do Not Use This Code, Generate the Correct Terms for Your Paper</concept_desc>
  <concept_significance>300</concept_significance>
 </concept>
 <concept>
  <concept_id>00000000.00000000.00000000</concept_id>
  <concept_desc>Do Not Use This Code, Generate the Correct Terms for Your Paper</concept_desc>
  <concept_significance>100</concept_significance>
 </concept>
 <concept>
  <concept_id>00000000.00000000.00000000</concept_id>
  <concept_desc>Do Not Use This Code, Generate the Correct Terms for Your Paper</concept_desc>
  <concept_significance>100</concept_significance>
 </concept>
</ccs2012>
\end{CCSXML}

% \ccsdesc[500]{Do Not Use This Code~Generate the Correct Terms for Your Paper}
% \ccsdesc[300]{Do Not Use This Code~Generate the Correct Terms for Your Paper}
% \ccsdesc{Do Not Use This Code~Generate the Correct Terms for Your Paper}
% \ccsdesc[100]{Do Not Use This Code~Generate the Correct Terms for Your Paper}

%%
%% Keywords. The author(s) should pick words that accurately describe
%% the work being presented. Separate the keywords with commas.
% \keywords{Do, Not, Us, This, Code, Put, the, Correct, Terms, for,
%   Your, Paper}

% \received{20 February 2007}
% \received[revised]{12 March 2009}
% \received[accepted]{5 June 2009}

%%
%% This command processes the author and affiliation and title
%% information and builds the first part of the formatted document.

\section{Introduction}

% State of the world (robots for creative activites)
The term ``robot,'' originally signifying `forced labor,' has long been associated with labor and work. Robots have demonstrated their utility in various automated productive and social contexts, where the primary goals are improving productivity, safety, and fostering social interactions with humans~\cite{simoes2022designing, weidemann2021role, honig2018understanding}. However, an increasing number of cases feature using of robots in creative settings. Unlike productive contexts, where the focus is on efficiency and task completion~\cite{arents2022smart}, or social contexts, where communication and trust are prioritized~\cite{nam2020trust, saunderson2019robots}, creative environments prioritize artistic innovation and expression~\cite{hsueh2024counts}. This shift fundamentally alters the dynamics of human-robot interaction, redefining the roles and expectations for both humans and robots.

For instance, robots’ social behaviors are leveraged to support the generation and expression of creative ideas~\cite{hu2021exploring, sandoval2022human, alves2020creativity}, and programmable robotic movements and trajectories are employed to inspire artistic activities such as sketching~\cite{lin2020your}. These studies often engage participants from creative fields who possess limited prior experience with robotics, and are typically conducted in short-term, experimental settings. Consequently, the findings from these studies remain constrained since much can be learned from professional practitioners' experiences to inform system design such as digital fabrication~\cite{hirsch2023nothing}. There is a notable gap in research examining the long-term, active, and practical experience of integrating robotic systems into the creative processes. As a result, the deeper insights into how robots facilitate and shape creative processes, beyond simply augmenting human creativity, remain underexplored. In this study, we aim to better understand the impacts of robots on creative processes and outcomes.

As early as Leonardo da Vinci's 16th century ``Automaton,'' artists have explored the creative affordances of robotic systems~\cite{shanken2002cybernetics, pagliarini2009development, jeon2017robotic}. The artistic creation process typically encompasses various stages, including the exploration of materials and techniques, ongoing experimentation and iteration, and the continual refinement of the artists' insights into their creative subjects~\cite{lewis2023art, sturdee2022state}. Therefore, investigating the artistic process involving robots offers an opportunity to gain deeper insights into robots' creative potential. Robotic art, in particular, provides a compelling case for this exploration.

We define robotic art as artworks that utilize robotic or automated machines to create artistic experiences and tangible artifacts. One example is robotic installation art, in which robots are programmed to follow specific rules that embody the artist’s expression (\autoref{fig:teaser} (a)). Another example is responsive art, in which robots react to their environment, with behaviors that change over time or in response to spectators (\autoref{fig:teaser} (b)). Additionally, there are robotic creators, which possess a degree of agency, allowing them to collaborate with human artists and produce works that extend beyond mere replication of human-created art (\autoref{fig:teaser} (c) and (d)). As such, robotic art becomes a rich case for exploring human-machine interactions in creative contexts. Gaining a deeper understanding of how robots facilitate artistic expression can provide insights for designing computing systems to support creative activities~\cite{gomez2021robot}.

% Therefore, we did...
We draw on semi-structured, in-depth interviews with renowned professional robotic artists to investigate the use of robots in artistic practice. Specifically, our goal is to understand how artistic exploration of robotic systems challenges conventional assumptions about the functions of robots, such as their roles in automating repetitive tasks or serving human needs. We also explore the implications of robots in the artistic process and examine how creativity may emerge within robotic art. To address these interrelated inquiries, our study focuses on the practice of robotic art, posing the research question: \textit{How do robotic artists utilize robots in their artistic practice?} We approach this inquiry through the perspectives and experiences of robotic artists, who creatively design, modify, and repurpose robotic systems for artistic expression and exploration.

% The key findings are...
Our findings highlight the social, material, and temporal dimensions of artists' practices that shape their creativity and artistic outcomes. The creation of robotic art is largely a social process, as artists receive both explicit and implicit feedback through the audience's reactions and reception of their work. Simultaneously, the embodiment and malfunctions inherent to robotic systems drive artistic experimentation. The temporal processes of creation and exhibition, beyond just the final product, further enhance the creative value. Our empirical analysis presents how creativity emerges through the interplay of social, material, and temporal interactions among artists, robots, audiences, and the environment.

% The contributions of this work are...
We make two main contributions to HCI in this study. 
First, we elucidate the interactive mechanisms among key actors---human creators, machines, audiences, and environments---within the practice of robotic art, a topic that remains underexplored in HCI. Our findings reveal the significance of sociality (e.g., interactions between artists and audiences), materiality (e.g., the embodiment and malfunctions of robots), and temporality (e.g., the processes of creation and exhibition) in shaping creative values. We propose that these three facets are central to the creative process and facilitate the emergence of creativity in robotic art.
Second, drawing from the findings, we offer implications for \textit{socially informed}, \textit{material-attentive}, and \textit{process-oriented} creation with computing systems. We suggest leveraging these three aspects to enhance creativity and the creative experience. Specifically, we discuss the value of incorporating implicit audience feedback, designing with technical malfunctions, and focusing on the post-creation process to foster alternative creative experiences with machines~\cite{alter2010designing, juarez2022glitch}.



\section{Related Work}
\label{sec:related}
\subsection{Collaborative Systems}
In the era of rapid growth in medical foundational models~\cite{huang2023visual,wang2022medclip, zhang2024data}, the top-down model development paradigm limits model capabilities by heavily relying on the resources available to the model builders. 
Such paradigm often restricts the potential of these models, as they cannot effectively utilize the diverse, private, and decentralized resources that exist within the broader medical community.
In contrast, collaborative systems present a promising alternative, offering a more flexible approach to model development.

Collaborative systems enable institutions to share knowledge by allowing distributed collaborators to contribute to a common goal~\cite{boulemtafes2020review}. 
To further protect patient privacy, federated learning (FL)~\cite{mcmahan2017communication} was proposed to alleviate such privacy concerns as server aggregating parameter updates from multiple clients without sharing their local data. 
While subsequent optimizations, such as aggregation algorithms~\cite{mcmahan2017communication, zhao2018federated, li2020federated}, secure learning~\cite{hardy2017private, xie2021crfl}, fairness improvements~\cite{sharma2022federated, zhao2022dynamic} and its application in medicine~\cite{kumar2024privacy}, have enhanced the capacity and applicability of FL, its real-world flexibility remains limited. This is primarily due to the need for synchronous updates, which require the server and clients to stay in sync, or model updates will be blocked.
This synchrony issue can be mitigated by open-source software platforms (e.g., GitHub~\cite{github}), allowing independent contributions from individual developers asynchronously. Such an asynchronous scheme enables faster iteration and the integration of specialized expertise, thus offering a more flexible and scalable approach.

Unlike synchronous collaboration, asynchronous collaboration does not require collaborators to work simultaneously and collaborators can individually complete their updates.
While the concept of asynchronous collaboration has been widely used in software development, its machine-learning applications remain limited~\cite{kandpal2023git, raffel2023building}. 
With the rise of global collaboration, large models~\cite{sahajBERT, le2023bloom} are usually co-developed by collaborators given various levels of data availability. However, this collaborative scheme requires the aggregation of local data and online synchronous cooperation of developers.
Software-like model update system~\cite{raffel2023building} alleviates the synchronous problem, where models are updated incrementally, similar to software development, by introducing merging and version control to model development.
However, the existing collaborative version control system~\cite{kandpal2023git} fails to address the complexities of medical scenarios because of the heavy dependency on plain parameter averaging across the full model without accounting for the varying requirements of different tasks.
To respond, we propose MedForge, which enables an asynchronous collaborative system and ensures strong robustness toward a continuous, community-driven enhancement of medical models while overcoming potential data leakage.

\begin{figure*}[t]
\begin{center}
\includegraphics[width=.85\linewidth]{fig_overview_v3.pdf}
\end{center}
\caption{
FastAtlas Overview: In each frame, we compute charts spanning fully or partially visible triangles (a), determine texture space bounding boxes for the visible portions of the view-space projections of each chart, and tightly pack these boxes into atlases (b, here $2K \times 2K$). We simultaneously bijectively parameterize and shade the charts into their atlas boxes, obtaining high quality texture space shading (c), and use this shading to render the shaded frames (d).}
\label{fig:overview}
\label{fig:alg_overview}
\end{figure*}

\section{Overview}
\label{sec:overview}
Our work has two core contributions: a real-time, GPU-based algorithm for tight packing of general parameterized charts into compact atlases; and a real-time TSS method that
utilizes this packing.  

\paragraph*{FastAtlas Packing.}
FastAtlas runs entirely on the GPU as a series of compute shaders. It takes the bounding boxes of parameterized charts as input, and packs them into an atlas (Fig~\ref{fig:overview}b, Sec.~\ref{sec:pack}). As such, the only input it requires are the dimensions of the bounding boxes.
Its outputs are deterministic; identical input charts are packed into identical atlases. This is critical for TSS and similar applications, as it ensures that consecutive frames taken from the same camera view have the same shading. Even minute shading differences across such frames can cause sampling jitter, leading to undesirable flicker \cite{baker2012rock}. 
While prior methods such as \cite{mueller2018shading,hladky2019tessellated,hladky2021snakebinning,Neff2022MSA} cap the dimensions of the charts that can be packed as-is for a given atlas size, and scale down all charts that exceed these dimensions, we scale all charts by the same factor, and do so only when strictly necessary to achieve packing success (Figs~\ref{fig:atlas},~\ref{fig:sas_issues}). 

\paragraph*{TSS using FastAtlas.}
Our end-to-end TSS atlas generation method combines the packing method above with a novel approach for computing seamless per-frame charts. 
We define our charts as the connected components of the visible surfaces in each frame (Fig.~\ref{fig:overview}a), and efficiently compute them using a parallel union-find algorithm (Sec.~\ref{sec:visible}). Since the boundaries of these charts coincide with the contours of the rendered surface, they are {\em invisible} to the viewer. This approach 
eliminates the artifacts caused by shading discontinuities along visible seams (Fig.~\ref{fig:seams}). 

\begin{parWithWrapFigure}
\begin{wrapfigure}{l}{.27\columnwidth}%
\includegraphics[width=\linewidth]{fig_inset_view_plane.pdf}%
\end{wrapfigure}
We bijectively parametrize the {\em visible portions} of our charts by projecting them to view space (inset). This maps a constant number of texels to each pixel in the final rendered output, evenly distributing residual undersampling error across all image pixels. While conceptually straightforward, efficiently parameterizing charts containing partially visible triangles using viewspace projection is non-trivial, as the visible portions may no longer be triangular (e.g. green triangle in the inset); applying naive projection to triangles with vertices behind the camera may produce ill-posed results. Clipping triangles before projection is both computationally expensive and significantly complicates downstream operations. We avoid explicit clipping by observing that all that is required for atlas packing is the dimensions of, potentially conservative, bounding boxes of these projected visible portions. We compute such bounding boxes without explicit chart clipping by adapting a conservative screen coverage estimator \shortcite{Blinn:CalculatingScreenCoverage} (Sec.~\ref{sec:box}). We then pack the computed boxes using FastAtlas. 
\end{parWithWrapFigure}

Finally, we shade the visible portion of each chart into its corresponding atlas bounding box (Fig~\ref{fig:overview}c). 
The resulting texture is then used during rasterization as a standard texture map (Fig. ~\ref{fig:overview}d). 
Our framework is compatible with all existing approaches for texture space shading, including forward shading, raytraced illumination, or deferred shading in texture space \cite{baker:2016}. In the examples shown, we use the standard forward shading based rendering pipeline included in the G3D Innovation Engine \cite{G3D17}, a commercial grade renderer.


\subsection{Model Merging}
In collaborative systems, proper model merging becomes increasingly vital for improving model knowledge integration from multiple sources in a resource-limited environment~\cite{li2023deep, yang2024model, goddard2024arcee}. Conceptually, model merging strategies can be categorized into entire model merging and partial model merging.

Entire model merging involves combining multiple model parameters to participate in the merging process by several means. Entire model merging can be viewed as an optimization problem~\cite{Matena_Raffel_2021, jin2022dataless, mavromatis2024packllm} or an alignment problem~\cite{ainsworth2022git, jordan2022repair, xu2024training, ainsworth2022git}, each offering unique advantages depending on the task at hand.
In the optimization-based approach, the goal is to find the best combination of multiple models to enhance performance and efficiency. For instance, using Fisher information approximation~\cite{Matena_Raffel_2021}, the optimization-based model merging can be interpreted as selecting parameters that maximize the joint likelihood of the models' posterior distributions. The optimization of model merging can also be guided by minimizing the prediction differences between the merged model and individual models~\cite{jin2022dataless}. 
With the development of large language models (LLM), optimization-based method is used to fuse multiple LLMs at test-time by minimizing perplexity over the input prompt~\cite{mavromatis2024packllm}.
To highlight, optimization-based methods are beneficial for scenarios requiring enhanced model performance and efficiency to integrate model parameters, while alignment-based methods~\cite{ainsworth2022git, jordan2022repair} are better suited for maintaining consistency and interpretability, facilitating critical information sharing across models.
For example, a training-free model merging strategy aligns relevant models by using a similarity matrix of their representations in both activation and weight spaces~\cite{xu2024training}.
Further, the alignment between the independently trained model and a reference model not only works for models with the same architecture but also for arbitrary model architectures~\cite{ainsworth2022git}.
In summary, the entire model merging methods can effectively integrate existing models into a merged model with enhanced functionality. However, they could lead to increased computational complexity and reduced flexibility, making them less scalable and harder to implement across diverse tasks.

Partial model merging refers to combining only specific components or layers of models to improve model merging efficiency and decrease the computational cost. 
Such specific components can come from the same network~\cite{kingetsu2021neural}, where the original network is divided into subnetworks for different purposes, and these subnetworks can then be recombined for new tasks.
Additionally, modules can originate from different functional networks and be merged using various strategies. For instance, arithmetic operations are applied in \cite{zhang2023composing} to fuse parameter-efficient modules.
While merging modules from different networks provides flexibility, the process also requires a selection strategy to ensure the resulting model aligns with the specific needs of the inference stage. 
The selection strategies are commonly designed based on the similarity of task~\cite{lv2023parameter} and domain clustering performance~\cite{chronopoulou2023adaptersoup}. Alternatively, the mixture-of-experts methods use a routing strategy to select appropriate component modules~\cite{ponti2023combining}. However, these strategies often require significant time and computational resources to filter through a large model pool. 
In contrast, LoRAHub~\cite{huang2023lorahub} offers a more lightweight approach, combining various LoRA modules for different tasks with minimal model training. Nevertheless, LoRAHub lacks flexibility for incremental updates, especially when handling unseen tasks.

Although the existing model merging approaches effectively combine the capabilities of individual models, these approaches often rely on raw data, leading to potential privacy risks. Our proposed MedForge emphasizes the prevention of raw data usage, which is particularly crucial in medical scenarios. Additionally, MedForge offers an extensible capability for incremental learning, enabling continuous model improvement.





% \subsection{Notations}  %% commented out as we do not use them

% The notations used throughout this paper are summarized in Table ~\ref{t:notations}.

% \begin{table}
%     \centering
%     \small % Reduce font size for the table (optional)
%     \begin{tabular}{|l|c|}
%         \hline
%         \textbf{Notation} & \textbf{Description}  \\
%         \hline
%         $X_{\text{tr}}$ & Training set inputs (messages) 
%         \\\hline
%         $y_{\text{tr}}^{\text{gold}}$ & Gold labels for $X_{\text{tr}}$\\
%         \hline
%         $y_{\text{tr}}^{\text{llm}}$ & Synthetic labels for $X_{\text{tr}}$ \\ \hline
%          $X_{\text{val}}$ & Validation set inputs \\
%          \hline
%         $y_{\text{val}}^{\text{gold}}$ & Gold labels for $X_{\text{val}}$ \\
%         \hline
%         $y_{\text{val}}^{\text{llm}}$ & Synthetic labels for $X_{\text{val}}$ \\ \hline
%         $X_{\text{test}}$ & Test set inputs \\
%         \hline
%         $y_{\text{test}}^{\text{gold}}$ & Gold labels for $X_{\text{test}}$ \\
%         \hline
%         $y_{\text{test}}^{\text{llm}}$ & Synthetic labels for $X_{\text{test}}$ \\ \hline
%         $(X, y)_{\text{tr}}^{\text{llm}}$ & Synthetic training data \\
%         \hline
%          $(X, y)_{\text{val}}^{\text{llm}}$ & Synthetic validation data \\  
%          \hline
%     \end{tabular}
%     % ACL style has the caption below the table or figure
%     \caption{Summary of notations used in the paper}
%     \label{t:notations} 
% \end{table}


\subsection{Overview of Scenarios}

%In this study, we
We investigate the role of LLMs in CB detection, focusing on their utility under varying data availability conditions
and under the assumption that direct use of LLMs as a classifier is too expensive due to the high volume of messages to be checked.
%To establish
As a baseline for comparison, we %first
evaluate a scenario in which a
lightweight, BERT-based
classifier is trained exclusively on gold-standard, manually labeled authentic data without %any
LLM involvement.
We then define three additional scenarios with different data availability
and that use LLMs in different ways.
%, each illustrating how LLMs can aid in CB detection depending on the availability and quantity of authentic data.
%The scenarios are as follows.

%To establish a baseline for comparison, in the first scenario, we evaluate a setup that relies exclusively on training a classifier using gold-standard, manually labeled authentic data with no LLM involvement. We then define three other distinct scenarios, each corresponding to
% %% JW: The following is unneccesary vague as the scenarios are more specifically
% %% about the way the synthetic data is used, apart from the zero-shot LLM.
% a unique way LLMs can be integrated into the detection pipeline.
% These scenarios range from directly serving as classifiers to generating synthetic data or labels for training. 

\paragraph{Scenario 1: Baseline}

This scenario represents the ideal situation where sufficient
%manually labeled (
gold-standard data is available for fine-tuning %a classic encoder such as
BERT.
It serves as the benchmark for evaluating the effectiveness of other approaches.
In this setup, no synthetic data or LLMs are involved.
%The system relies entirely on human annotations.
This scenario is feasible if resources such as time, budget and expert annotators are abundant. However, it often proves impractical due to the
%high costs and scalability
challenges of manual labeling.



\paragraph{Scenario 2: LLM as Classifier}  \label{s:m:sc2}

This scenario applies when labeled authentic data is unavailable, and there is no intention to train a separate classifier for CB detection. Instead, an instruction-tuned
LLM is used directly as a classifier, leveraging its pre-trained knowledge and its ability to follow instructions
to identify CB instances.
%This approach is particularly useful in contexts that require rapid deployment or when computational or time resources are limited for training a new model. 
The primary advantage of this method is its elimination of the need for labeled data and training time. However, there are trade-offs. While an LLM can handle nuanced language patterns, it may be less efficient and incur higher computational costs
compared to simpler BERT-based classifiers with a classification head and fine-
tuned on a labeled dataset.
%% JW: add reference to large zero-shot study in NLP
We explore two prompting strategies for generating synthetic labels:
\textit{(a)} guideline-enhanced (GE) prompts, guiding the LLM with detailed labeling instructions and
\textit{(b)} guidelne-free (GF) prompts, allowing the LLM to generate labels without such guidelines.

\paragraph{Scenario 3: Fully Synthetic Data}

In this scenario, only a small set of manually labeled gold data is available for testing, with no access to authentic data for training or validation.
%To address this, we
We
use an LLM to generate a fully synthetic dataset, consisting of both synthetic messages and corresponding labels, for training and validation.
This approach is particularly valuable in low-resource domains or emerging tasks where authentic data is scarce or difficult to collect.
It is especially useful in situations where creating authentic datasets is costly, time-consuming, or ethically challenging, such as annotating harmful or sensitive content or working with vulnerable populations.
The effective

%Commented for indusrty track \subsubsection{Scenario 4: Data Augmentation with Synthetic Data}
%This scenario assumes the availability of a moderate amount of gold-labeled data for training and validation, which may be insufficient to achieve optimal performance. To augment the dataset, we use an LLM to generate additional synthetic data, which is then combined with the gold-labeled data during training and validation. The experiment systematically varies the ratio of synthetic-to-gold data to evaluate its impact on model performance. This scenario explores how LLMs can supplement authentic data, striking a balance between scalability and accuracy.


\paragraph{Scenario 4: Synthetic Labels for Unlabeled Data} \label{s:m:sc4}

This scenario addresses the common situation where resources for manual annotation are limited. Here, gold-standard labeled data is available only for the test set, while a significant amount of unlabeled authentic data is available for training and validation.
%This scenario demonstrates the utility of LLMs in resource-constrained settings, enabling cost-effective dataset creation from unannotated corpora.
To utilize the unlabeled data, we label it using the best prompting strategy (GE or GF) from scenario~2.

% \subsubsection{Summary of Scenarios}
% Table~\ref{t:scenario-summary} presents an overview of the data used in the baseline system and each scenario, specifying the datasets utilized for training, validation, and testing. For Scenario 2, where no classifier is trained and the LLM is used directly as a classifier, only the test set is included.
% \begin{table}
%     \centering
%     \small % Reduce font size for the table (optional)
%     \begin{tabularx}{\columnwidth}{|X|X|X|X|}
%         \hline
%         \textbf{Scenario} & \textbf{Train} & \textbf{Validation} & \textbf{Test} \\
%         \hline
%          1 & $X_{\text{tr}}, y_{\text{tr}}^{\text{gold}}$ & $X_{\text{val}}, y_{\text{val}}^{\text{gold}}$ &  $X_{\text{test}}, y_{\text{test}}^{\text{gold}}$ \\
%         \hline
%            2 & - & - & $X_{\text{test}}, y_{\text{test}}^{\text{gold}}$ \\
%         \hline
%           3 & $(X, y)_{\text{tr}}^{\text{llm}}$ & $(X, y)_{\text{val}}^{\text{llm}}$ & $X_{\text{test}}, y_{\text{test}}^{\text{gold}}$ \\
%         \hline
%          4 & $X_{\text{tr}}, y_{\text{tr}}^{\text{llm}}$ & $X_{\text{val}}, y_{\text{val}}^{\text{llm}}$ &  $X_{\text{test}}, y_{\text{test}}^{\text{gold}}$ \\ \hline
       
%         % 4 & $X_{\text{tr}}, y_{\text{tr}}^{\text{llm}} + X_{\text{tr}}, y_{\text{tr}}^{\text{gold}}$ & $X_{\text{val}}, y_{\text{val}}^{\text{llm}}+ X_{\text{val}}, y_{\text{val}}^{\text{gold}}$ & $X_{\text{test}}, y_{\text{test}}^{\text{gold}}$ \\ \hline
%         % 4 & $(X, y)_{\text{tr}}^{\text{llm}} + X_{\text{tr}}, y_{\text{tr}}^{\text{gold}}$ & $(X, y)_{\text{val}}^{\text{llm}}+ X_{\text{val}}, y_{\text{val}}^{\text{gold}}$ & $X_{\text{test}}, y_{\text{test}}^{\text{gold}}$ \\ \hline
%     \end{tabularx}
%     % ACL style has the caption below the table or figure
%     \caption{Overview of data used in each scenario}
% \label{t:scenario-summary} 
% \end{table}





% \subsection{Intrinsic Evaluation Metrics}

% Intrinsic evaluation examines the inherent qualities of datasets, enabling the assessment of linguistic diversity, emotional tone, and conversational structure independently from task-specific performance. For our CB detection task, we utilize \textbf{four} categories of intrinsic metrics to compare the authentic dataset with LLM-generated synthetic data. These categories are: 1) lexical and linguistic characteristics, %including metrics such as Mean Words per Message, Mean Word Length, and Type-Token Ratio; 
% 2) content and CB indicators, 
% %such as rate of Harmful Messages, Bully Messages, Victim Messages, and Toxicity; 
% 3) sentiment and emotional tone, 
% %which classifies messages into negative, positive, or neutral; 
% and 4) dialogue act distribution.
% %categorizing messages into types such as Question, Statement, Greeting, Accept/Reject, and Other. 
% These categories are critical for understanding the fundamental differences between authentic and synthetic data in the context of CB detection, as they provide insight into how well the synthetic data replicates the linguistic, emotional, and conversational behaviors that are typically present in real-world online interactions.

% To ensure a fair comparison between the authentic and synthetic datasets, we first normalize both dataset by employing pre-processing techniques such as tokenization using NLTK \cite{loper-bird-2002-nltk} and punctuation handling. Additionally, data is segmented into equal-sized token slices to account for metrics that are influenced by corpus size.

% Sentiment scores are measured using VADER \cite{hutto2014vader}, a sentiment analysis tool optimized for short social media texts. Dialogue acts are classified using a Naive Bayes model trained on the NLTK \texttt{nps-chat} corpus,
% following \newcite[Chp.~6, Sec.~2.2]{bird2009natural}.\footnote{
%     While no citation is provided by \newcite{bird2009natural}, the source
%     of this corpus seems to be
%     \newcite{forsyth-martell-2007-lexical,forsyth-etal-2010-nps}.
% }
%



% Natural Language Processing with Python, by Steven Bird, Ewan Klein and Edward Loper
% Chapter 6, section 2.2 "Identifying Dialogue Act Types"
% refers to Chapter 2, section 1.2 "Web and Chat Text", for the
% NPS Chat Corpus but provides no source or citation.
%  
% An unrelated 2011 paper cites an "NPS Chat Corpus of North American English chat
% conversations (Forsyth and Martell 2007)".
%   * Forsyth, Eric. M. and Craig H. Martell (2007), Lexical and discourse analysis
%     of online chat dialog, Proceedings of the First IEEE International Conference
%     on Semantic Computing (ICSC) 2007, pp. 19–26.
%   * data collected in 2006
%   * approximately 500,000 chat posts gathered from various online services
%   * 10,567 posts tagged in Release 1.0
%   * available on http://faculty.nps.edu/cmartell/NPSChat.htm (page no longer
%     exists but is archived, e.g. on
%     http://web.archive.org/web/20190510121556/http://faculty.nps.edu/cmartell/NPSChat.htm
%        - "If you want just the data, you can get it through the Linguistic Data
%          Consortium.  It is catalog number LDC2010T05."
%        - This page asked for the 2007 paper above to be cited "when referring to
%          the NPS Chat Corpus".
%
% There is a 2010 thesis from Naval Postgraduate School, Monterey, California, by
% J. R. Hitt entitled "Implementation and Performance exploration of a cross-genre
% part of speech tagging methodology to determine dialog act tags in the chat
% domain".
%   * credits Lin and Forsyth
%


% Type-Token Ratio (TTR), which is calculated by dividing the number of unique words by the total tokens in fixed-size slices, serves as a normalized measure of vocabulary diversity. Toxicity scores, which represent the ratio of messages containing profanity, are derived using a publicly available profanity list \cite{surge2023profanity}.


\subsection{Evaluation Metrics}

We choose accuracy of label prediction for development decisions and reporting since the labels are reasonably balanced in the authentic test data with 30.3\% items labeled with the minority
label.\footnote{In the appendix, we further report macro average F1 scores that are also widely used in the area of harmful content detection.}
In scenarios 1, 3 and 4,
we train BERT\_base\_uncased \cite{devlin-etal-2019-bert}, a 110M parameter transformer model, with a linear classification head
% using
% the HuggingFace transformers library \cite{wolf-etal-2020-huggingface}
to detect harm, assigning binary labels to text messages.
To address noise from randomness in training, we train at least 45 models for each setting and report average accuracy and standard deviation.
% \section{Experiments: Planning outperforms Heuristics}
\label{sec:experiment}

We begin our empirical demonstrations by showcasing the effectiveness of our planning framework on both synthetic and real datasets. We focus on the simplest planning algorithm, 1-step lookaheads (Algorithm~\ref{alg:complete}), and show that even basic planning can hold great promise. 
We illustrate our framework using two uncertainty quantification modules---GPs and 
\ensembles/ \ensembleplus. 

Throughout this section, we focus on evaluating the mean squared error of 
a regression model $\model$,  and develop adaptive policies that minimize uncertainty on $g(f)$ defined in~\eqref{eqn:l2-g-f}.
When GPs provide a valid model of uncertainty, 
our experiments show that our planning framework significantly outperforms other baselines. 
We further demonstrate that our conceptual framework extends to deep learning-based uncertainty quantification methods such as  \ensembleplus while highlighting computational challenges that need to be resolved in order to scale our ideas. 
For simplicity, we assume a naive predictor, i.e., $\psi(\cdot) \equiv 0$. However, we emphasize that this problem is just as complex as if we were using a sophisticated model $\psi(.)$. The performance gap between the algorithms 
primarily depends
on the level  of uncertainty in our prior beliefs.

To evaluate the performance of our algorithm, we benchmark it against several baselines. 
%Active learning baselines use an acquisition function $\ac$ to select points that have the highest   function value: $X\opt_t \in \argmax_{X \in \xpoolj{t}} \ac({X})$ at every step $t$. These methods may also need an UQ module, which we simply use the same UQ module as in our algorithm, and it  outputs $V(X)$ that measures the the uncertainty of each point $X \in \xpoolj{t}$.
Our first set of baselines are from active learning~\citep{AggarwalKoGuHaPh14}:
\\ % \noindent\textbf{Active Learning Heuristics:} 
\textbf{(1)} 
\textsf{Uncertainty Sampling (Static):}  In this approach, we query the samples for which the model is least certain about. Specifically, we estimate the variance of the latent output $f(X)$ for each $X \in \xpool$ using the UQ module and select the top-$K$ points with the highest uncertainty. \\
\textbf{(2)} \textsf{Uncertainty Sampling (Sequential):} This is a greedy heuristic that sequentially selects the points with the highest uncertainty within a batch, while updating the posterior beliefs using pseudo labels from the current posterior state. Unlike \textsf{Uncertainty Sampling (Static)}, this method takes into account the information gained from each point within batch, and hence tries to diversify the selected points within a batch. 

 
We also compare our approach to the  \textbf{(3)} \textsf{Random Sampling}, which selects each batch uniformly at random from the pool. Additionally, we compare solving the planning problem using  \textsf{REINFORCE}-based policy gradients with   $\mathsf{Smoothed\text{-}Autodiff}$ policy gradients.\footnote{Our code repository is available at
  \url{https://github.com/namkoong-lab/adaptive-labeling}.}
%Detailed experimental setups are provided in Section \ref{sec:details-experiments}.

%We repeat all experiments with 10 random seeds.




\begin{figure}[t]
\centering
\begin{minipage}[b]{0.49\textwidth}
\centering
\includegraphics[width=\textwidth, height=5cm]{figures/original_scale/Var_of_l_2_loss.pdf}
\caption{(Synthetic data) Variance of mean squared loss evaluated through the posterior belief $\mu_t$ at each horizon $t$. This is the objective that policy gradient methods like \textsf{REINFORCE} and $\ouralgo$ optimizes. 1-step lookaheads are surprisingly effective even in long horizons.}
\label{fig:var-l2-sim}
\end{minipage}
\hfill
\begin{minipage}[b]{0.49\textwidth}
\centering \includegraphics[width=\textwidth, height=5cm]{figures/original_scale/Error_of_estimated_model_l_2_loss.pdf}
\caption{(Synthetic data) Error between MSE calculated based on collected data $\mc{D}^{0:T}$ vs. population oracle MSE over $\mc{D}_{\rm eval} \sim P_X$. Reducing uncertainty over posteriors directly leads to better OOD evaluations. 1-step lookaheads significantly outperform active learning heuristics in small horizons.}
\label{fig:mean-l2-sim}
\end{minipage}
%\caption{Simulated data for GPs}
%\label{fig:both_plots}
\end{figure}

\subsection{Planning with Gaussian processes}
\label{sec:experiment-plan-GP}
We now briefly describe the data generation process for the GP experiments,  deferring a more detailed discussion of the dataset generation to Section~\ref{sec:details-experiments}. 
We use both the synthetic data and the real data to test our methodology.
For the \emph{simulated data},  we construct a setting where the general population is distributed across \emph{51 non-overlapping clusters} while the initial labeled data $\dtrain$ just comes from one cluster. In contrast, both $\dpool \defeq (\xpool,\ypool),\deval \defeq (\xeval,\yeval)$ are generated   from all the clusters. 
We begin with a low-dimensional scenario, generating a one-dimensional regression setting using a GP. %Gaussian Process (GP).
Although the data-generating process is not known to the algorithms,  we assume that the GP hyperparameters are known to all the algorithms
to ensure fair comparisons. This can be viewed as a setting where our prior is well-specified, allowing us to isolate the effects
of different policy optimization approaches
 without any concerns about the misspecified priors. We select $10$ batches, each of size $K=5$ across $T = 10$ time horizons.

To examine the robustness of our method against the distributional assumptions made  in the simulated case, we then move to a real dataset where the correct prior is not known. We simulate selection bias from the eICU dataset~\citep{PollardJoRaCeMaBa18}, which contains real-world patient data with in-hospital mortality outcomes. 
We conduct a $k$-means clustering to generate 51 clusters and then select data from those clusters. We view this to be a credible replication of practice, as severe distribution shifts are common due to selection bias in clinical labels.  To convert the binary mortality labels into a regression setting, we train a  random forest classifier and fit a GP on predicted scores, which serves as the UQ module for all the algorithms. As before, the task is to select 10 batches, each consisting of 5 samples, across 10 time horizons.

 In Figures~\ref{fig:var-l2-sim} and~\ref{fig:mean-l2-sim}, we present results for the simulated data. 
Figure~\ref{fig:var-l2-sim} shows the variance of $\ell_2$ loss, and Figure~\ref{fig:mean-l2-sim} presents the error in the estimated $\ell_2$ loss using $\mu_t$ (relative to true $\ell_2$ loss, that is unknown to the algorithm). 
As we can see from these plots, our method one-step lookahead  gives substantial improvements  over active learning baselines and random sampling. In addition,
compared to the one-step lookahead planning approach using \textsf{REINFORCE}-based policy gradients, 
we observe that $\mathsf{Smoothed\text{-}Autodiff}$-based policy gradients provide significantly more robust performance over all horizons.

In Figures~\ref{fig:var-l2-real}~and~\ref{fig:mean-l2-real}, we observe similar findings on the eICU data. We see that planning policies (\textsf{REINFORCE} and $\mathsf{Smoothed\text{-}Autodiff}$) consistently outperform other heuristics by a large margin.  Active learning baselines perform poorly in these small-horizon batched problems and can sometimes be even worse than the random search baselines.  Overall, our results show the importance of careful planning in adaptive labeling for reliable model evaluation. 

We offer some intuition as to why one-step lookahead planning may outperform other heuristic algorithms. 
 First,  \textsf{Uncertainty sampling (Static)} while myopically selects the
 top-$K$ inputs with the highest uncertainty, it fails to consider 
the overlap in information content among the ``best” instances; see \citep{AggarwalKoGuHaPh14} for more details. 
In other words,  it might acquire points from the same region with high uncertainty while failing to induce diversity among the batch.
Although \textsf{Uncertainty Sampling (Sequential)} somewhat addresses the issue of information overlap, a significant drawback of 
this algorithm
is the disconnect between the objective we aim to optimize and the algorithm. For example, it might sample from a region with high uncertainty but very low density. 

\begin{figure}[t]
\centering
\begin{minipage}[b]{0.48\textwidth}
\centering
\includegraphics[width=\textwidth, height=5cm]{figures/original_scale/Var_of_l_2_loss_real.pdf}
\caption{(Real-world eICU data) Variance of mean squared loss evaluated through the posterior belief $\mu_t$ at each horizon $t$. Even 1-step lookaheads are extremely effective planners, and auto-differentiation-based pathwise policy gradients provide a reliable optimization algorithm based on low-variance gradient estimates.}
\label{fig:var-l2-real}
\end{minipage}
\hfill
\begin{minipage}[b]{0.48\textwidth}
\centering \includegraphics[width=\textwidth, height=5cm]{figures/original_scale/Error_of_estimated_model_l_2_loss_real.pdf}
\caption{(Real-world eICU data) Error between MSE calculated based on collected data $\mc{D}^{0:T}$ vs. population oracle MSE over $\mc{D}_{\rm eval} \sim P_X$. Reducing uncertainty over posteriors directly leads to better OOD evaluations. Our method significantly outperforms active learning-based heuristics, and random sampling.}
\label{fig:mean-l2-real}
\end{minipage}
%\caption{Real data for GPs}
\end{figure}
 
%\vspace{-1.5cm}
% \begin{wrapfigure}{r}{.32\columnwidth}
%   \vspace{-.5cm} 
%   \centering
% \includegraphics[scale=.29]{figures/Var of l2l_2 loss.pdf}
%   \vspace{-0.2cm}
%   \caption{Results of GP}
% \label{fig:var-l2-gp}
%   \vspace{-0.1cm}
% \end{wrapfigure}


% Attempts have been made  in the past to address these  drawbacks heuristically  (see \citep{AggarwalKoGuHaPh14}). We give a unified computational framework while approaching the problem in a more principled manner and solving it more optimally.




\subsection{Planning with  neural network-based uncertainty quantification methods ($\ensembleplus$)}


We now provide a proof-of-concept that shows the generalizability of our conceptual framework  to the deep learning-based UQ modules, specifically focusing on $\ensembleplus$ due to their previously observed superior performance~\citep{OsbandWenAsDwIbLuRo23}. Recall that implementing our framework with deep learning-based UQ modules  requires us to retrain the model across multiple possible random actions $\bm{a}(\theta)$ sampled from the current policy $\pi_\theta$.
This requires significant computational resources, in sharp contrast to the GPs where the posteriors are in closed form and can be readily updated and differentiated. 

Due to the computational constraints, we test $\ensembleplus$ on a toy setting to demonstrate the generalizability of our framework. We consider a setting where the general population consists of four clusters, while the initial labeled data only comes from one cluster. Again we generate data using GPs.  The task is to select a batch of 2 points in one horizon. We detail the $\ensembleplus$ architecture in Section \ref{sec:details-experiments}, and we assume prior uncertainty to be large (depends on the scaling of the prior generating functions). 
The results are summarized in the Table~\ref{tab:UQ_ensemble}.

% \begin{table}[H]
% \vspace{-10pt}
% \caption{Performance under \ensembleplus as UQ module}
%     \centering
%     \begin{tabular}{|m{3cm}|m{2.5cm}|m{2cm}|} 
%     \hline
%       Algorithm   & Variance of $\loss_2$ loss estimate & Error of $\loss_2$ loss estimate  \\ \hline Random Sampling 
%          & $1710.9 \pm 1352.1$ & $8.67\pm6.62$ 
%       \\ \hline \ouralgo & $1.30 \pm 0.68$ & $0.91\pm0.25$ \\ \hline
%     \end{tabular}
%     \label{tab:UQ_ensemble}
%     %\vspace{-10pt}
% \end{table}




\begin{table}[h]
\vspace{-10pt}
\caption{Performance under \ensembleplus as the UQ module}
\centering
\begin{tabular}{|l|l|l|}
\hline
Algorithm   & Variance of $\loss_2$ loss estimate & Error of $\loss_2$ loss estimate  \\
\hline
\textsf{Random sampling} & 7129.8 $\pm$ 1027.0 & 136.2 $\pm$ 8.28 \\ \hline
\textsf{Uncertainty sampling (Static)} & 10852 $\pm$ 0.0 & 162.156 $\pm$ 0.0 \\ \hline
\textsf{Uncertainty sampling (Sequential)} & 8585.5 $\pm$ 898.9 & 144 $\pm$ 6.93 \\ \hline
\textsf{REINFORCE} & 1697.1 $\pm$ 0.0 & 45.27 $\pm$ 0.0 \\ \hline
\ouralgo & 1697.1 $\pm$ 0.0 & 45.27 $\pm$ 0.0 \\ \hline
\end{tabular}
%\caption{Comparison of different algorithms based on variance   and   error in $\ell_2$ loss estimation with Ensemble $+$ as the UQ module. Our results demonstrate that {\ouralgo} and REINFORCE outperformthe other active learning based heuristics, confirming the benefits of our MDP formulation for the adaptive labeling problem, as also demonstrated in Section 4.\\
%\footnotesize{Experimental details: We use Gaussian Processes as our data generating process, GP parameters are the same as in Section D.3.  The task is to select a batch of 2 points along one horizon.The marginal distribution $p_X$ has 4 \textit{non-overlapping} clusters. Initial data comes from one cluster, while pool and evaluation points comes from all the clusters. We have $20$ initial labeled data points, $10$ pool points, and $252$ evaluation points.  Training procedures are similar to the one in Section D.3.} }
\label{tab:UQ_ensemble}
\end{table}



% We faced  issues in scaling up these experiments which will be our focus in the future. 





% \begin{itemize}
%     \item Posteriors should be consistent. Two dimensions: even with less training,  
%     \item the inference should be  fast enough
% \end{itemize}


% Potential research directions for uncertainty quantification

% In this section we consider a simple setting We consider a simpler setting and 


% For synthetic dataset generation, we use ...... For real datasets, we use ...... We compare our methodolgy to several baselines ()    This Section is structured as follows:
% \begin{itemize}
%     \item \textbf{GPs, square loss objective} (Section \ref{}): 
%     %the broad aim of the experiments  in this section is to isolate the performance of our methodology without any concerns for the inefficiencies induced due to a mis-specified prior or imperfect posterior inference. To accomplish this we generate synthetic datasets using GPs (detailed later). We use the well specified prior (GPs - with same hyperparameter setting) as our UQ module.   
%      As GPs provide differentaible posterior inference - any errors induced due to imperfect posterior updates are also isolated. We note that under this setting
%      \item In Section\ref{} we demonstrate why our methodology performs better than other baselines - by devising various synthetic experiments ()
%     \item  \textbf{UQ Benchmarking }(Section \ref{}): Before diving into the experiments using $\ensembleplus$ and ENNs,  we showcase our benchmarking experiments in Section \ref{}. We use real datasets We observe that ENNs perform better
%      \item \textbf{Ensemble $+$}, objective: recall, accuracy
%     \item \textbf{ENN}, objective: recall, accuracy
% \end{itemize}




% In Section {}, we test 
% \subsection{Experimental details}

% \begin{itemize}
%     \item UQ methodologies - GPs, ENNs
%     \item Objectives - Recall,  ATE
%     \item Datasets - ATE-synthetic datasets, Recall-synthetic, real datasets
%     \item Baselines - 
%     \begin{itemize}
%         \item Random sampling
%         \item Active learning - Uncertainty based sampling - In regression setting almost all of the 
%         \item Myopic greedy - Greedy Batch based sampling
%         \item Policy Gradient
%     \end{itemize}
    
% \end{itemize}

% \subsection{Experiments}
%     \begin{itemize}
%     \item GPs with square loss
%     \item Benchmarking ENN
%         \item ENNs with ATE
%         \item ENNs with Recall
%     \end{itemize}

% \subsection{Benefits over other algorithms - intuition and experiments}

%Active learning - Myopic greedy / Don't rely on the objective rather some entropy version.


%%% Local Variables:
%%% mode: latex
%%% TeX-master: "main"
%%% End:

\vspace{-0.2cm}
\section{Results}\label{sec:results}




\subsection{Benchmark quality after watermarking}\label{subsec:results_rephrasing}


\paragraph{\textbf{Set-up.}}
For the watermark embedding, we rephrase with Llama-3.1-8B-Instruct~\citep{dubey2024llama} by default, with top-p sampling with p = $0.7$ and temperature = $0.5$ (default values on the Hugging Face hub), and the green/red watermarking scheme of \citet{kirchenbauer2023reliability} with a watermark window $k=2$ and a ``green list'' of
size $\frac{1}{2}|V|$ ($|V|$ is the vocabulary size).
We compare different values of $\delta$ when rephrasing: 0 (no watermarking), 1, 2, and 4.
We choose to watermark ARC-Challenge, ARC-Easy, and MMLU due to their widespread use in model evaluation.
In practice, one would need to watermark their own benchmark before release.
For MMLU, we select a subset of 5000 questions, randomly chosen across all disciplines, to accelerate experimentation and maintain a comparable size to the other benchmarks.
We refer to this subset as MMLU$^*$.
ARC-Easy contains 1172 questions, and ARC-Challenge contains 2372 questions.
In~\autoref{fig:example_answers_big} of \autoref{app:appendix}, we show the exact instructions given to the rephrasing model (identical for all benchmarks) and the results for different watermarking strengths on one example from ARC-Easy.
\emph{We use a different watermarking key $\sk$ for each benchmark.}

% Thanks to the hashing function used, the corresponding green lists and red lists for each benchmark are independent: there is no more collision between the benchmarks than there is between natural text and the benchmarks.

\paragraph{\textbf{Even strong watermarking keeps benchmark utility.}} 
We evaluate the performance of Llama-3.3-1B, Llama-3.3-3B and Llama-3.1-8B on the original benchmark and the rephrased version using as similar evaluation as the one from the \texttt{lm-evaluation-harness} library~\citep{eval-harness}.
To check if the benchmark is still as meaningful, we check that evaluated models obtain a similar accuracy on the watermarked benchmarks and on the original version (see~\autoref{subsec:rephrasing}).
\autoref{fig:results_overview_arc_easy_perfs} shows the performance on ARC-Easy.
All models perform very similarly on all the rephrased versions of the benchmark, even when pushing the watermark to $80\%$ of green tokens.
Importantly, they rank the same.
Similar results are shown for MMLU$^*$ and ARC-Challenge in \autoref{fig:results_overview_arc_easy_perfs} of \autoref{app:appendix}, although for MMLU$^*$, we observe some discrepancies. 
For instance, when using a watermarking window size of 2 (subfig i), the performance of Llama-3.2-1B increases from 38$\%$ to $42\%$ between the original and the other versions. 
However we observe the same issue when rephrasing without watermarking in that case.
As detailed in \autoref{subsec:rephrasing}, designing better instructions that are more specific to each benchmark could help.
We have tried increasing $\delta$ even further, but it broke the decoding process. 
The choice of $\delta$ depends on the benchmark and the model used for rephrasing, and needs to be empirically tested.



\begin{figure}[b!] % 't' places the figure at the top of the page
    \centering
    \begin{minipage}{0.49\textwidth}
        \centering
        \includegraphics[width=1.0\textwidth, clip, trim=0 0cm 0 0]{figs/main/k2/arc-easy_delta_barplot.pdf}
        \subcaption{Watermarking questions does not degrade utility.}
        \label{fig:results_overview_arc_easy_perfs}
    \end{minipage}\hfill
    \begin{minipage}{0.49\textwidth}
        \centering
        \includegraphics[width=1.0\textwidth, clip, trim=0 0cm 0 0]{figs/main/k2/contamination_35317.pdf}
        \subcaption{More contaminations \& stronger wm $\uparrow$ detection.}
        \label{fig:results_overview_arc_easy_detection}
    \end{minipage}
    \caption{
    Result for benchmark watermarking on ARC-Easy. %Watermarking the questions does not degrade its utility, and the more watermarked the benchmark, the easier it is to detect radioactivity. 
    (Left) We rephrase the questions from ARC-Easy using Llama-3.1-8B-Instruct while adding watermarks of varying strength. 
    The performance of multiple Llama-3 models on rephrased ARC-Easy is comparable to the original, preserving the benchmark's usefulness for ranking models and assessing accuracy (Sec.~\ref{subsec:rephrasing}, Sec.~\ref{subsec:results_rephrasing}). (Right) We train 1B models from scratch on 10B tokens while intentionally contaminating its training set with the watermarked benchmark dataset. 
    Increasing the number of contaminations and watermark strength enhances detection confidence (Sec.~\ref{subsec:detection}, Sec.~\ref{subsec:result_detection})}
    \vspace{-0.3cm}\label{fig:results_overview_arc_easy}
\end{figure}

\subsection{Contamination detection through radioactivity}\label{subsec:result_detection}

We now propose an experimental design to control benchmark contamination, and evaluate both the impact on model performance and on contamination detection.

\paragraph{\textbf{Training set-up.}}
We train 1B transformer models~\citep{vaswani2017attention} using \texttt{Meta Lingua}~\citep{meta_lingua} on 10B tokens from DCLM~\citep{li2024datacomp}. 
The model architecture includes a hidden dimension of 2048, 25 layers, and 16 attention heads.
The training process consists of 10,000 steps, using a batch size of 4 and a sequence length of 4096. 
Each training is distributed across 64 A-100 GPUs, and takes approximately three hours to finish.
The optimization is performed with a learning rate of $3 \times 10^{-3}$, a weight decay of $0.033$, and a warmup period of 5,000 steps. 
The learning rate is decayed to a minimum ratio of $10^{-6}$, and gradient clipping is applied with a threshold of 1.0.

\paragraph{\textbf{Contamination set-up.}}
Between steps 2500 and 7500, every $5000/\#\text{contaminations}$, we take a batch from the shuffled concatenation of the three benchmarks instead of the batch from DCLM.
Each batch has
\(
\text{batch size} \times \text{sequence length} \times \text{number of GPUs} = 4 \times 4096 \times 64 \approx 1\,\text{M tokens}
\)
As shown in \autoref{tab:contamination}, the concatenation of the three benchmarks is approximately $500$k tokens, so each contamination is a gradient that encompasses all the benchmark's tokens.
For each benchmark, any sample that ends up contaminating the model is formatted as follows:

\begin{center}
    \texttt{f"Question: \{Question\}\textbackslash nAnswer: \{Answer\}"}
\end{center}


% \paragraph{Impact of the number of contaminations on the accuracy on the benchmark.} 
\paragraph{\textbf{Evaluation.}}
We evaluate the accuracy of the models on the benchmarks by comparing the loss between the different choices and choosing the one with the smallest loss,  either ``in distribution'' by using the above template seen during contamination or ``out of distribution'' (OOD) by using:

\begin{center}
    \texttt{f"During a lecture, the professor posed a question: \{Question\}. \\ After discussion, it was revealed that the answer is: \{Answer\}"}
\end{center}

In the first scenario, we evaluate overfitting, as the model is explicitly trained to minimize the loss of the correct answer within the same context. 
In the second scenario, we assess the model's ability to confidently provide the answer in a slightly different context, which is more relevant for measuring contamination.
Indeed, it's important to note that evaluations often use templates around questions ---\eg in the \texttt{lm-evaluation-harness} library~\citep{eval-harness}--- which may not be part of the question/answer files that could have leaked into the pre-training data.
% Moreover, if contamination comes from a leak of a jsonl that contains
\autoref{tab:contamination} focuses on $\delta=4$ and shows the increase in performance across the three (watermarked) benchmarks as a function of the number of contaminations when evaluated OOD. 
Results for in-distribution evaluation are provided in \autoref{tab:contamination_indist} of \autoref{app:appendix} (without contamination, the model performs similarly across the two templates).


\paragraph{\textbf{Contamination detection.}}
For each benchmark, we employ the reading mode detailed in~\autoref{subsec:detection} to compute the radioactivity score $S$ and the corresponding $\pval$.
% We perform the reading mode on the same watermarked benchmark watermarked benchmark.
Results are illustrated in~\autoref{fig:results_overview_arc_easy_detection} for ARC-Easy, and in~\autoref{fig:appendix_watermark_contamination} of \autoref{app:appendix} for the other two benchmarks, across different numbers of contaminations and varying watermark strengths $\delta$.
We observe that the stronger the watermark strength and the greater the number of contaminations, the easier it is to detect contamination: a larger negative $\logpval$ value indicates smaller $\pval$s, implying a lower probability of obtaining this score if the model is not contaminated.
For instance, a $-\logpval$ of $6$ implies that we can confidently assert model contamination, with only a $10^{-6}$ probability of error.
% , which is the case when $5$ points are artificially added on MMLU$^*$ in~\autoref{tab:contamination}.
Additionally, we observe that without contamination, the test yields a $\logpval$ value close to $-0.3 = \log_{10}(0.5) $, as expected under $\mathcal{H}_0$.
Indeed, under $\mathcal{H}_0$, the $\pval$ should follow a uniform distribution between 0 and 1, which implies that [-1, 0] is a 90$\%$ confidence interval for $\logpval$, and that [-2, 0] is a 99$\%$ confidence interval.

\autoref{tab:contamination} links the contamination detection to the actual cheating (with OOD evaluation) on the benchmarks when $\delta=4$ is used.
We can see that for the three benchmarks, whenever the cheat is greater than $10\%$, detection is extremely confident.
When the cheat is smaller, with four contaminations ranging from $+3\%$ to $+5\%$, the $\pval$ is small enough on ARC-Easy and MMLU$^*$, but doubtful for ARC-Challenge (because smaller, see \autoref{subsec:additional_results}).
For instance, for MMLU$^*$, we can assert model contamination, with only a $10^{-6}$ probability of error when $5$ points are artificially added.




% \begin{table}[t!]
%     \centering
%     \vspace{-0.2cm}
%     \caption{
%         Detection and performance metrics across different levels of contamination for ARC-Easy, ARC-Challenge, and MMLU benchmarks, watermarked with $\delta=4$.
%         The performance increase is shown for OOD evaluation as detailed in~\autoref{subsec:result_detection}. 
%         Similar results for in distribution are shown in \autoref{tab:contamination_indist} of~\autoref{app:appendix}
%     }\label{tab:contamination}
%     \begin{tabular}{r r r r r r r}
%         \toprule
%         & \multicolumn{2}{c}{ARC-Easy (112k toks.)} & \multicolumn{2}{c}{ARC-Challenge (64k toks.)} & \multicolumn{2}{c}{MMLU$^*$ (325k toks.)} \\
%         \cmidrule(lr){2-3} \cmidrule(lr){4-5} \cmidrule(lr){6-7}
%         Cont & \multicolumn{1}{r}{log10 p-val} & \multicolumn{1}{r}{Perf (\% $\Delta$)} & \multicolumn{1}{r}{log10 p-val} & \multicolumn{1}{r}{Perf (\% $\Delta$)} & \multicolumn{1}{r}{log10 p-val} & \multicolumn{1}{r}{Perf (\% $\Delta$)} \\
%         \midrule
%         0  & -0.3 & 53.5 (+0) & -0.3 & 29.4 (+0) & -0.9 & 30.6 (+0) \\
%         4  & -3.0 & 57.9 (+4.3) & -1.2 & 32.4 (+3.1) & -5.7 & 35.7 (+5.1) \\
%         8  & -5.5 & 63.0 (+9.5) & -4.5 & 39.3 (+9.9) & \textless{-12} & 40.8 (+10.2) \\
%         16 & \textless{-12} & 71.7 (+18.2) & \textless{-12} & 54.3 (+24.9) & \textless{-12} & 54.0 (+23.5) \\
%         \bottomrule
%     \end{tabular}
%     \vspace{-0.3cm}
% \end{table}

% \newcommand{\graydelta}[1]{\textcolor{gray}{\footnotesize (#1)}}
\begin{table}[t!]
    \centering
    \vspace{-0.2cm}
    \caption{
        Detection and performance metrics across different levels of contamination for ARC-Easy, ARC-Challenge, and MMLU benchmarks, watermarked with $\delta=4$.
        The performance increase is shown for OOD evaluation as detailed in~\autoref{subsec:result_detection}. 
        The log$_{10}$ $\pval$ of the detection test is strongly correlated with the number of contaminations, as well as with the performance increase of the LLM on the benchmark.
        % Similar results for in distribution are shown in \autoref{tab:contamination_indist} of~\autoref{app:appendix} \pierre{not necessary in the fig.}
    }\label{tab:contamination}
    \resizebox{\textwidth}{!}{
    \begin{tabular}{r rr@{\hspace{0.5em}}l rr@{\hspace{0.5em}}l rr@{\hspace{0.5em}}l}
        \toprule
        & \multicolumn{3}{c}{ARC-Easy (112k toks.)} & \multicolumn{3}{c}{ARC-Challenge (64k toks.)} & \multicolumn{3}{c}{MMLU$^*$ (325k toks.)} \\
        \cmidrule(lr){2-4} \cmidrule(lr){5-7} \cmidrule(lr){8-10}
        Contaminations & $\logpval$ & Acc. & \graydelta{\% $\Delta$} & $\logpval$ & Acc. & \graydelta{\% $\Delta$} & $\logpval$ & Acc.& \graydelta{\% $\Delta$} \\
        \midrule
        0  & -0.3 & 53.5 & \graydelta{+0.0} & -0.3 & 29.4 & \graydelta{+0.0} & -0.9 & 30.6 & \graydelta{+0.0} \\
        4  & -3.0 & 57.9 & \graydelta{+4.3} & -1.2 & 32.4 & \graydelta{+3.1} & -5.7 & 35.7 & \graydelta{+5.1} \\
        8  & -5.5 & 63.0 & \graydelta{+9.5} & -4.5 & 39.3 & \graydelta{+9.9} & \textless{-12} & 40.8 & \graydelta{+10.2} \\
        16 & \textless{-12} & 71.7 & \graydelta{+18.2} & \textless{-12} & 54.3 & \graydelta{+24.9} & \textless{-12} & 54.0 & \graydelta{+23.5} \\
        \bottomrule
    \end{tabular}
    }
    \vspace{-0.3cm}
\end{table}

\vspace{-0.2cm}
\subsection{Additional Results}\label{subsec:additional_results}


\paragraph{\textbf{Impact of window size.}}
\begin{wraptable}{r}{0.4\textwidth}
    \centering
    \vspace{-0.4cm}
    \caption{\small Proportion of green tokens in the predictions (see~\autoref{eq:def_S_N}), number of tokens scored after dedup and log$_{10}$ $\pval$s for different watermark window sizes, with 16 contaminations and $\delta=4$ on ARC-Easy.}
    \small % Reduce font size for the table
    \begin{tabular}{r r r r}
        \toprule
        $k$ & \multicolumn{1}{c}{$\rho$} & \multicolumn{1}{r}{Tokens} & \multicolumn{1}{r}{$\logpval$} \\
        \midrule
        0 & 0.53 & 5k & -6.07 \\
        1 & 0.53 & 28k & -25.89 \\
        2 & 0.53 & 47k & -38.69 \\
        \bottomrule
    \end{tabular}
    \vspace{-0.2cm}
    \label{tab:window_size}
\end{wraptable}
Watermark insertion through rephrasing (\autoref{subsec:rephrasing}) depends on the watermark window size $k$. 
Each window creates a unique green-list/red-list split for the next token. 
Larger windows reduce repeated biases but are less robust.
Because of repetitions, \citet{sander2024watermarking} show that smaller windows can lead to bigger overfitting on token-level watermark biases, aiding radioactivity detection.
In our case, benchmark sizes are relatively small and deduplication limits the number of tokens tested, because each $\{$window + predicted token$\}$ is scored only once. 
Thus, smaller windows mean fewer tokens to score.
Moreoever, as shown in~\autoref{tab:window_size}, the proportion of predicted green tokens is not even larger for smaller windows: there is not enough repetitions for increased over-fitting on smaller windows.
The two factors combined result in lower confidence. 
A comparison of contamination detection across benchmarks and window sizes is shown in \autoref{fig:appendix_watermark_performance}, and for the utility of the benchmarks in~\autoref{fig:appendix_watermark_contamination}.
Overall, larger window size ($k=2$) yields better results.

\vspace{-0.1cm}
\paragraph{\textbf{Impact of benchmark size.}} The benchmark size can significantly affect the method's effectiveness.
With a fixed proportion of predicted green tokens, more evidence (\ie more scored tokens) increases test confidence. 
As shown in~\autoref{tab:contamination}, at a fixed level of cheating (\eg $+10\%$ on all benchmarks after $8$ contaminations), contamination detection confidence is proportional to benchmark size.
This is similar to our observations on watermark window sizes in~\autoref{tab:window_size}.
% So at fixed cheating level, it will be easier to detect contamination of bigger benchmarks.




% In classical watermarking, however, a larger watermark window means smaller robustness as changing one every $k$ tokens on average can break all the watermark.

% But in our case, we are going the do the radioactivity detection test on the dataset without any changes, but we may want more robustness if the suspect model tries to break the watermark before training on it.


% \paragraph{Impact of rephrasing model.}
% The difficulty of the questions, and their entropy, can have an important impact on the method.
% For instance, some math questions are hard to rephrase, and adding a watermark can further mess-up the meaning. 
% The method may thus require a stronger model for highly technical benchmarks (\eg Llama3-70B instead of Llama3-8B).
% Moreover, typically for math or code, the rephrasing inherently does not let a lot of entropy, as many invariants need to be respected.
% Possibilities would be to add watermarked verbose text around the math instead of rephrasing, and use as entropy-aware LLM watermarking~\citep{lee2023wrote}.
% We have tested rephrasing the benchmarks using Llama3-70B-Instruct instead of the 8B version. 
% We observe that we need to increase $\delta$ to $8$ in in order to obtain the same proportion of green tokens as with $\delta=2$ with the 8B model, while using the exact same decoding parameters.
% This can be because there is less entropy in the generation of the 70B or that the logits are for some reasons bigger, as the bias towards the greenlist is added before the softmax (see~\autoref{subsec:rephrasing}).
% However, we observe that some failure cases with the 8B (specifically for questions with important numbers) are correct with the 70B, but this is hard to quantify. 
% We give one example bellow in~\autoref{fig:example_answers_70B}.
\vspace{-0.1cm}
\paragraph{\textbf{Impact of rephrasing model.}}
The difficulty and entropy of questions can significantly affect the method's performance. 
Indeed, math questions for instance can be challenging to rephrase, even more with watermarks. 
Thus, better models may be needed for technical benchmarks.
We tested rephrasing with Llama3-70B-Instruct instead of the 8B version, and  observed that some 8B model failures, especially on math questions, are resolved with the 70B model, though quantifying this is challenging. 
An example is provided in~\autoref{fig:example_answers_70B}.
We note that increasing $\delta$ to 8 is necessary to match the green token proportion of $\delta=2$ with the 8B model, using the same decoding parameters.
This may result from lower entropy in generation or bigger logits, as the greenlist bias is applied before the softmax (see~\autoref{subsec:rephrasing}).
Moreover, in math or code, rephrasing can offer limited entropy, and even better models will not be enough.
An alternative would be to add watermarked verbose text \emph{around} the questions, or using entropy-aware LLM watermarking~\citep{lee2023wrote}.

\begin{figure}[b!]
    \vspace{-0.3cm}
    \centering
    \begin{tcolorbox}[colframe=metablue, colback=white]
        \footnotesize
        \textbf{Original question:} 
        An object accelerates at 3 meters per second$^2$ when a 10-newton (N) force is applied to it. Which force would cause this object to accelerate at 6 meters per second$^2$?
        \begin{minipage}{0.42\textwidth}
            \vspace{0.1cm}
            \textbf{Llama-3-8B-Instruct, $\delta=2$:} What additional force, applied in conjunction with the existing 10-N force, would cause the object to experience an acceleration of 6 meters per second$^2$? (70$\%$)
        \end{minipage}\hspace{0.04\textwidth}%
        \begin{minipage}{0.54\textwidth}
            \vspace{0.1cm}
            \textbf{Llama-3-70B-Instruct, $\delta=8$:} What force would be necessary to apply to the object in order to increase its acceleration to 6 meters per second$^2$, given that an acceleration of 3 meters per second$^2$is achieved with a 10-newton force? (65$\%$)
        \end{minipage}
    \end{tcolorbox}
    \vspace{-0.2cm}
    \caption{
    Watermarking failure on an ARC-Challenge question with an $8$B model, while the $70$B model succeeds.
    }
    
\label{fig:example_answers_70B}
\end{figure}



\begin{wrapfigure}{r}{0.5\textwidth}
  \centering
  \vspace{-0.5cm}
\includegraphics[width=0.48\textwidth]{figs/main/detection_vs_performance.pdf} % Replace with your image file
  \vspace{-0.25cm}
  \caption{Detection confidence as a function of performance increase on MMLU$^*$ for different model sizes and \#contaminations, for $\delta=4$ and OOD evaluation.}
  \vspace{-0.35cm}
\end{wrapfigure}\label{fig:model_size}
\paragraph{\textbf{Impact of model size.}}
We also test radioactivity detection on 135M and 360M transformer models using the architectures of~\href{https://github.com/huggingface/smollm}{\texttt{SmolLM}} and the same training pipeline as described in \autoref{subsec:result_detection}, training each model on 10B tokens as well. 
\autoref{fig:model_size} shows the detection confidence as a function of the cheat on MMLU$^*$.
We find that, for a fixed number of contaminations, smaller models show less performance increase -- expected as they memorize less -- and we obtain lower confidence in the contamination detection test. 
As detailed in~\autoref{subsec:rephrasing}, the $\pval$s indicate how well a model overfits the questions, hence the expected correlation. For a fixed performance gain on benchmarks, $\pval$s are consistent across models. For example, after $4$, $8$, and $16$ contaminations on the $1$B, $360$M, and $135$M parameter models respectively, all models show around $+6$\% gain, with detection tests yielding $\pval$s around $10^{-5}$.
Thus, while larger models require fewer contaminated batches to achieve the same gain on the benchmark, radioactivity effectively measures ``cheating''.





% \begin{wrapfigure}{r}{0.45\textwidth}
%   \centering
%   \vspace{-0.4cm}
%   \includegraphics[width=0.43\textwidth]{figs/main/arc-easy.pdf}
%   \vspace{-0.3cm}
%   \captionsetup{font=small}
%   \caption{Performance of Llama-3 models on different versions of the arc-easy benchmark.}
%   \vspace{-1cm}
%   \label{fig:impact-wm-arc-easy}
% \end{wrapfigure}


\section{Discussions}

\subsection{Transparency in Ride-Sharing Platform Algorithms}
The publicly available Chicago Transportation Network Provider dataset helped us answer many research questions, but ride-sharing platforms still make many of their mechanisms opaque. The lack of transparency in key platform mechanisms---such as pricing models, driver--rider matching algorithms, and driver ranking systems---makes it difficult to pinpoint the exact causes of these disparities. Without greater visibility into these proprietary algorithms, drivers also remain at an information disadvantage, unable to anticipate fare fluctuations or optimize their work schedules effectively.

Pricing models remain opaque, with our analysis revealing that fare adjustments over time have failed to keep pace with inflation, effectively reducing real driver earnings (\cref{sec:results-pricing-stablization}). While platforms advertise dynamic pricing mechanisms that respond to demand surges, drivers have limited insight into how much of the fare they actually receive after platform fees~\cite{santos2020dynamic}. Previous research has shown that drivers tend to work more during peaks for higher compensation~\cite{chen2016dynamic}. A real-time, large-scale understanding of the surge pricing model can help drivers become more informed in planning and organizing their workday, beyond anecdotal observations. Furthermore, researchers can provide prediction models of price surges, helping both drivers and riders adjust plans accordingly. Another key limitation of using the Chicago dataset is the lack of driver earning information. As a result, our analysis can only use the trip fare as a proxy for driver earning. Making such information available can significantly increase transparency into platform operations.

Similarly, the driver-rider matching algorithm remains a black box. Our inferred driver profiles suggest that trip assignments may systematically disadvantage certain groups, particularly those operating in lower-income areas. If the matching algorithm disproportionately favors drivers in high-demand or high-fare regions, it could reinforce existing geographic disparities in earnings. However, such analysis is hard to conduct without access to driver-level information. As discussed in \cref{sec:methods-driver-simulation}, releasing such data may lead to privacy concerns. Our approach is an effort to approximate driver working conditions without needing detailed driver data. However, researchers should still work with ride-sharing platforms to come up with privacy-preserving ways to analyze such data for insights. Also, driver ranking algorithms---which determine access to high-value trips---are equally opaque. While platforms often cite factors such as acceptance rate, customer ratings, and trip history, the lack of public accountability raises concerns regarding potential biases. Accessing such information can support researchers in identifying potential biases, also help drivers provide more desired services to riders.

In all, we call for increased regulatory oversight and platform-level efforts to improve algorithmic transparency. Without clear disclosures on how these systems operate, ride-sharing drivers remain vulnerable to unfair decision-making and fluctuating incomes that they cannot predict or control.

\subsection{Data Analysis Methodology Improvements}
Our study demonstrates the feasibility of simulating reasonable driver profiles from trip-level data, even in the absence of driver-related information. By leveraging a simulation-based approach, we were able to approximate driver earnings, work patterns, and geographic activity. However, there are still areas for improvement for our methodology.

First, a robust evaluation benchmark is needed to validate the accuracy of inferred driver profiles. While our approach provides valuable insights and matches previous empirical findings, the lack of direct ground truth data means we rely on approximations. We need alternative data sources to cross-verify our inferred driver activities. Tools for driver task management, such as Driver's Seat~\cite{calacci2023access}, asks drivers to upload their work tasks and can serve as a potential data source. More autonomous approaches that uses UI understanding techniques and directly collects data from drivers' phones can also scale up this effort~\cite{lu2024crepe}. 

Moreover, expanding the scope of inferred information would provide deeper insights into platform operations. Currently, we infer earnings and work patterns for drivers. Newer algorithms can be developed to analyze additional opaque platform mechanisms as discussed above. Future studies could aim to reconstruct other aspects of opaque platform algorithms, as discussed above, directly from publicly available, large-scale datasets.

Given a large-scale dataset that misses key information aspects, a potential future approach is to self-collect a smaller dataset that contains the necessary details and conduct a joint analysis of both datasets. For example, a smaller dataset that we collect directly from drivers, containing both driver and trip information, can serve both as a benchmark and a basis for use to train machine learning models that predict driver profiles from existing large-scale datasets. Future research can investigate effective measures to combine these different data sources~\cite{harris2018federal} for joint analysis. These methodological advancements can help us to use large-scale ridesharing datasets more effectively and accurately while maintaining driver and rider privacy.


\subsection{Societal Implications: Ride-Sharing as a Reflection of Broader Inequalities}

Our findings revealed regional ride-sharing disparities in the city of Chicago, which largely reflect the broader existing societal inequalities. Drivers working in lower-income neighborhoods---in our case, drivers that service the southern regions of Chicago---consistently earn less, even despite longer work hours. Structural disadvantages, such as lower infrastructure quality, longer wait times, and increased safety concerns---compound the challenges faced by gig workers. Chicago South Side, as a community suffering from violence and poverty, has been an example of social segregation and studied by numerous researchers~\cite{moore2016south, bachin2004building, bell1993community}. As an aspect of a deep-rooted societal issue, ride-sharing inequality in lower-income neighborhoods calls for holistic policymaking efforts from multiple stakeholders.

Our findings provide practical implications for labor activists and policy makers. By providing a more transparent view of drivers’ potential workday experiences, policymakers can better evaluate the labor conditions these platforms create, ensuring that emerging mobility systems align with equity goals. Urban planners and regulators can use these insights to inform policy interventions---such as driver support programs, driver caps, or incentive structures---that promote fairness and mitigate algorithmic biases. Similarly, platform operators themselves might harness these findings to improve their matching algorithms, advancing a more equitable ecosystem that benefits both drivers and passengers.

Research has shown that transportation access can have a positive impact on regional economic growth and productivity~\cite{targa2005economic, banerjee2020road, alstadt2012relationship}. Ride-sharing, as an increasingly critical way of transportation, especially where public transportation is scarce, can support individual and community access to growth opportunities. The persistence of regional earning gaps raises important questions about equity in urban transportation. If ride-sharing platforms are designed primarily to maximize efficiency and revenue, they may inadvertently exacerbate existing economic inequalities by steering high-value rides away from underserved areas~\cite{durand2022access, bocarejo2012transport}.

To address these issues, we call for policy interventions aimed at ensuring fair compensation and equitable access to earning opportunities. Regulators should consider implementing transparency mandates, income stability measures, and algorithmic accountability frameworks to prevent platforms from disproportionately disadvantaging certain driver groups. Moreover, these efforts should be in orchestration with existing efforts to promote infrastructural improvements and public safety in underserved regions. Collaborative initiatives between policymakers, ride-sharing companies, and community organizations can help create a more inclusive transportation ecosystem that benefits both drivers and passengers alike~\cite{baber2022new}.
\section{Limitations}
\label{sec:Limitations}
Our study benefits from including both open-weight and proprietary models, as well as models from the same family with different parameterizations, enhancing the generalizability of the findings. However, certain design decisions may affect the experiments.

Ambiguity detection is limited to the first three turns, as LLMs struggle to interact meaningfully if they do not engage early. To assess question quality, we measure changes in the latent vector to capture the information gained, assuming equal importance for all new information—though models may prioritize different details in their solution. The resolve rates in the overall problem solving experiment reflect real-life conditions, where incorrect code is unacceptable, regardless of how close the generated patch is to the solution. However, data leakage could enable some models to make correct assumptions in underspecified settings, inflating resolve rates. Additionally, the user proxy may be more interactive than real-world users, as LLMs are tuned to be helpful. We address this by limiting the number of interaction turns and focusing interactions on the task with detailed system prompts.


\section{Conclusions}
In this study, we explored disparities in ride-sharing earnings and trip distributions using the publicly available Chicago Rideshare dataset (2018–2023). By analyzing both direct observational and a simulation-based methodology, we revealed systematic inequities in driver earnings based on temporal, regional, and algorithmic factors. Our findings reveal that pricing adjustments in recent years have failed to account for inflation, leading to a decline in drivers’ real earnings despite apparent fare stabilization. Additionally, spatial analysis indicates that income gaps have widened over time, with lower-earning zones emerging in Chicago’s South Side and outlying areas.
To address limitations in existing anonymized ride-share datasets, we introduced a simulation-based driver profiling method that reconstructs potential work and earning patterns. This approach allowed us to model driver behaviors, including variations in working hours, trip frequencies, and geographical preferences, which contribute to substantial earnings disparities. Our clustering analysis further revealed the emergence of new driver groups in 2023, suggesting shifts in ride-sharing platform dynamics and potential algorithmic biases in trip allocations. 



\bibliographystyle{ACM-Reference-Format}
\bibliography{reference}

% Appendix
\clearpage
\newpage
\appendix
\onecolumn
\section{Full Results on Longbench}
\label{appendix}
% \renewcommand{\arraystretch}{1.2} % 设置行高
\begin{table*}[ht]
\setlength{\tabcolsep}{2.5pt} % 设置列间距
\caption{\textbf{Result on Longbench.} The highest score in each task is marked in bold (except for "Full"). We also note the relative error of Twilight when integrated with the corresponding base algorithm. Green indicates an increase in score, while red indicates a decrease.}
\label{table:longbench}
    \centering
    \scalebox{0.69}{
    \begin{tabular}{lcccccccccccccc}
        \toprule
        \multirow{2}*{\textbf{Methods}} &
        \multirow{2}*{\textbf{Budget }} &
        \multicolumn{2}{c}{\textbf{Single-Doc. QA}} & \multicolumn{3}{c}{\textbf{Multi-Doc. QA}} & \multicolumn{3}{c}{\textbf{Summarization}} & \multicolumn{1}{c}{\textbf{Few-shot}} & \multicolumn{2}{c}{\textbf{Code}} & \multicolumn{1}{c}{\textbf{Synthetic}} & \multirow{2}*{\textbf{Avg. Score}}  \\
        \cmidrule(lr){3-4}\cmidrule(lr){5-7}\cmidrule(lr){8-10} \cmidrule(lr){11-11} \cmidrule(lr){12-13} \cmidrule(lr){14-14} 
        & & \textit{Qasper} & \textit{MF-en} & \textit{HotpotQA} & \textit{2WikiMQA} &  \textit{Musique} & \textit{GovReport} & \textit{QMSum} & \textit{MultiNews} & \textit{TriviaQA} &  \textit{LCC} & \textit{Repobench-P} & \textit{PR-en} \\
        \midrule
        \multicolumn{15}{c}{\textsc{Longchat-7B-32k}} \\
        \midrule
        \multirow{2}*{Full} & 32k & 29.48 & 42.11 & 30.97 & 23.74 & 13.11 & 31.03 & 22.77 & 26.09 & 83.25 & 30.50 & 52.70 & 55.62 & 36.78 \\
         & \textbf{Twilight (Avg. 146)} & 31.74 & \textbf{43.91} & 33.59 & \textbf{25.65} & \textbf{13.93} & 32.19 & \textbf{23.15} & 26.30 & 85.14 & 34.50 & 54.98 & 57.12 & 38.52\textcolor{teal}{(+4.7\%)}\\
        \midrule
        \multirow{5}*{Quest}
         & 256 & 26.00 & 32.83 & 23.23 & 22.14 & 7.45 & 22.64 & 20.98 & 25.05 & 67.40 & 33.60 & 48.70 & 45.07 & 31.26 \\
      & 1024 & 31.63 & 42.36 & 30.47 & 24.42 & 10.11 & 29.94 & 22.70 & 26.39 & 84.21 & 34.5 & 51.52 & 53.95 & 36.85 \\
       & 4096 & 29.77 & 42.71 & 32.94 & 23.94 & 13.24 & 31.54 & 22.86 & 26.45 & 84.37 & 31.50 & 53.17 & 55.52 & 37.33 \\
        & 8192 & 29.34 & 41.70 & 33.27 & 23.46 & 13.51 & 31.18 & 23.02 & 26.48 & 84.70 & 30.00 & 53.02 & 55.57 & 37.10 \\
             & \textbf{Twilight (Avg. 131)} & 31.95 & 43.28 & 31.62 & 24.87 & 13.48 & \textbf{32.21} & 22.79 & 26.33 & 84.93 & 33.50 & 54.86 & 56.70 & 38.04\textcolor{teal}{(+2.5\%)} \\
        \midrule
    \multirow{5}*{DS}
         & 256 & 28.28 & 39.78 & 27.10 & 20.75 & 9.34 & 29.68 & 21.79 & 25.69 & 83.97 & 32.00 & 52.01 & 53.44 & 35.32 \\
      & 1024 & 30.55 & 41.27 & 30.85 & 21.87 & 7.27 & 26.82 & 22.95 & 26.51 & 83.22 & 31.50 & 53.23 & 55.50 & 35.96 \\
       & 4096 & 28.95 & 41.90 & 32.52 & 23.65 & 8.07 & 29.68 & 22.75 & \textbf{26.55} & 83.34 & 30.00 & 52.77 & 55.48 & 36.31 \\
        & 8192 & 29.05 & 41.42 & 31.79 & 22.95 & 12.50 & 30.44 & 22.50 & 26.43 & 83.63 & 30.50 & 52.87 & 55.33 & 36.62 \\
             & \textbf{Twilight (Avg. 126)} & \textbf{32.34} & 43.89 & \textbf{34.67} & 25.43 & 13.84 & 31.88 & 23.01 & 26.32 & \textbf{85.29} & \textbf{35.50} & \textbf{55.03} & \textbf{57.27} & \textbf{38.71}\textcolor{teal}{(+5.7\%)} \\
        \midrule
        \multicolumn{15}{c}{\textsc{Llama-3.1-8B-Instruct}} \\
        \midrule
        \multirow{2}*{Full} & 128k & 46.17 & 53.33 & 55.36 & 43.95 & 27.08 & 35.01 & 25.24 & 27.37 & 91.18 & 99.50 & 62.17 & 57.76 & 52.01 \\
         & \textbf{Twilight (Avg. 478)} & 43.08 & 52.99 & 52.22 & 44.83 & 25.79 & 34.21 & \textbf{25.47} & 26.98 & 91.85 & \textbf{100.00} & \textbf{64.06} & 58.22 & 51.64\textcolor{red}{(-0.7\%)} \\
        \midrule
        \multirow{5}*{Quest}
         & 256 & 24.65 & 37.50 & 30.12 & 23.60 & 12.93 & 27.53 & 20.11 & 26.59 & 65.34 & 95.00 & 49.70 & 45.27 & 38.20 \\
      & 1024 & 38.47 & 49.32 & 47.43 & 38.48 & 20.59 & 33.71 & 23.67 & 26.60 & 81.94 & 99.50 & 60.78 & 52.96 & 47.79 \\
       & 4096 & 43.97 & 53.64 & 51.94 & 42.54 & 24.00 & 34.34 & 24.36 & 26.75 & 90.96 & 99.50 & 62.03 & 55.49 & 50.79 \\
        & 8192 &\textbf{44.34} & 53.25 & 54.72 & 44.84 & \textbf{25.98} & 34.62 & 24.98 & 26.70 & 91.61 & \textbf{100.00} & 62.02 & 54.20 & 51.44 \\
         & \textbf{Twilight (Avg. 427)} & 43.44 & 53.2 & 53.77 & 43.56 & 25.42 & 34.39 & 25.23 & 26.99 & 91.25 & 100.0 & 63.55 & 58.06 & 51.57\textcolor{teal}{(+0.3\%)} \\
        \midrule
    \multirow{5}*{DS}
         & 256 & 38.24 & 49.58 & 43.38 & 31.98 & 15.52 & 33.40 & 24.06 & 26.86 & 84.41 & 99.50 & 53.28 & 48.64 & 45.74 \\
      & 1024 & 42.97 & \textbf{54.65} & 51.75 & 33.92 & 20.39 & 34.50 & 24.92 & 26.71 & \textbf{92.81} & 99.50 & 62.66 & 48.37 & 49.43 \\
       & 4096 & 43.50 & 53.17 & 54.21 & 44.70 & 23.14 & \textbf{34.73} & 25.40 & 26.71 & 92.78 & 99.50 & 62.59 & 51.31 & 50.98 \\
        & 8192 & 43.82 & 53.71 & 54.19 & \textbf{45.13} & 23.72 & 34.27 & 24.98 & 26.69 & 91.61 & \textbf{100.00} & 62.40 & 52.87 & 51.14 \\
             & \textbf{Twilight (Avg. 446)} & 43.08 & 52.89 & \textbf{54.68} & 44.86 & 24.88 & 34.09 & 25.20 & \textbf{27.00} & 91.20 & \textbf{100.00} & 63.95 & \textbf{58.93} & \textbf{51.73}\textcolor{teal}{(+1.2\%)} \\
\bottomrule
\end{tabular}
}
\end{table*}


\end{document}