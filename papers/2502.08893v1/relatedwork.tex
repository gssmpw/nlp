\section{Related Work}
Research on ride-sourcing platforms has grown significantly, addressing issues from driver-passenger interactions and earnings disparity to the development of dispatching and matching algorithms \cite{chen2019value, ruch2020quantifying, hall2018analysis}. 
% In general, the literature can be grouped into two main areas: (1) analyses of inequities and fairness in the gig economy, and (2) algorithmic and simulation approaches to ridesourcing system optimization. 
Below, we review key contributions in these areas and highlight how our work differs, particularly in terms of input data constraints and methodological goals.

\subsection{Ride-Sharing Driver Work Conditions in the Gig Economy}

Gig workers, particularly those in ride-sharing platforms like Uber and Lyft \cite{barrios2022launching}, face unique work conditions that have been extensively analyzed in recent studies. The flexibility of these platforms is often hailed as a major advantage, offering drivers the autonomy to choose their own hours~\cite{tan2021ethical}. However, this flexibility is accompanied by concerns about income instability, lack of benefits, and poor job security~\cite{rosenblat2018uberland}. Many ride-sharing drivers report feeling underpaid relative to their work hours, with some earning below minimum wage after accounting for expenses like fuel and vehicle maintenance~\cite{bajwa2018health, brown2024driving, mishel2018uber}. Additionally, the platform's algorithm-driven nature has been criticized for exacerbating power imbalances, as drivers have little control over fare pricing or working conditions~\cite{lobel2017gig, zhu2024gig, kloostra2022algorithmic}. Studies also highlight the impact of these conditions on drivers' well-being, including increased stress and burnout~\cite{berger2019uber}. These findings underscore the need for regulatory changes to improve working conditions in the gig economy, particularly for ride-sharing drivers.

\subsection{Algorithmic Fairness in Gig Work Platform}

A large body of work has examined socio-economic inequities and fairness concerns within the gig economy \cite{de2015rise, wood2019good, berg2015income}. For example, researchers analyzed the gender earnings gap among rideshare drivers, identifying disparities stemming from platform experience, geographic work preferences, and driving speed~\cite{cook2021gender}. Other lines of inquiry have revealed that these inequalities are not limited to gender factors alone. For instance, \cite{de2024ridesourcing} examined the relationship between economic inequality and the market share of ridesourcing services, showing that these platforms often capitalize on cost structures that can worsen income disparities.

Algorithmic management in gig work raises significant fairness concerns in task allocation, wage determination, and worker evaluation \cite{zhang2022algorithmic, duggan2020algorithmic, kadolkar2024algorithmic}. Workers contend with a severe information asymmetry, as platforms control crucial details about demand and algorithmic rules, leaving them to infer decision-making processes from online forums and peer discussions~\cite{rosenblat2016algorithmic}. Dynamic pricing mechanisms \cite{shapiro2020dynamic, van2020wage}, including surge pricing, often result in unpredictable and volatile earnings, causing both drivers and riders to view the system as exploitative~\cite{cramer2016disruptive}. Additionally, reputation systems that rely on customer ratings and automated metrics add to the stress, as a single poor rating can jeopardize a worker’s standing, despite factors beyond their control~\cite{lee2015working, rosenblat2018uberland}. This combination of ambiguous processes and automated decisions fosters an environment where gig workers feel the system is designed against them, highlighting serious ethical and fairness challenges \cite{wood2019good}.


These existing studies, while comprehensive, rely heavily on proprietary datasets containing detailed driver and passenger attributes, such as shift patterns, demographic information, and work behaviors. Such granular data supports fairness and inequality analyses but is not always publicly available. In contrast, our work utilizes only trip-level data, without access to underlying driver or passenger profiles.

\subsection{Ridesourcing Algorithms and Simulation Models}

In parallel, a substantial body of literature focuses on algorithmic strategies for matching drivers and passengers or for evaluating new policy interventions in ridesourcing. Previous studies have proposed optimization models and dispatching algorithms aimed at diverse objectives, such as maximizing platform revenue, minimizing passenger waiting times, or improving driver efficiency \cite{zhang2020pricing, schreieck2016matching, di2013optimization, cao2021optimization}. These methods typically integrate detailed driver and passenger information to simulate realistic markets, follows supply-demand dynamics, and test the impact of models under controlled conditions.

For instance, \cite{kucharski2022simulating} introduced a lightweight simulation model that matches supply (drivers) and demand (travelers) under various assumptions. Such frameworks are powerful for prototyping and testing policies, but they usually depend on inputs such as the number and locations of drivers, passenger arrival distributions, and driver acceptance behaviors. This reliance on detailed micro-level data and assumptions can limit the applicability of these models to large-scale, public datasets.

Additionally, there has been research conducted using public data. For example, \cite{nanda2020balancing} proposes NAdap, an algorithm aimed at balancing profit and fairness in ride-sharing systems during peak hours. Although the study utilized both real and synthetic datasets, the experiments were modeled and tested on a small scale, with the number of drivers ranging from approximately 50 for real-life datasets to 100 for synthetic datasets.

In contrast, our approach aims to work directly with large-scale, noisy, and inherently sparse public trip-level datasets—such as those provided by municipal transportation authorities. Without explicit driver or passenger information, or data on how trips were assigned or accepted, the challenge lies in designing simulation models that infer underlying dynamics and support analysis at scale. While prior simulation studies often rely on synthesized scenarios or small datasets, our method processes hundreds of thousands of real-world trips per day, sourced from publicly available data. To the best of our knowledge, this is the first effort to design simulation tools that operate solely on large-scale, trip-only data, enabling new forms of analysis to understand the inner workings of ridesourcing systems without needing sensitive or unavailable individual-level data.

In summary, while previous research has provided rich insights into fairness and has developed sophisticated ridesourcing algorithms, the reliance on proprietary or synthesized datasets often limits generalizability. Our contribution bridges this gap by proposing simulation frameworks that leverage publicly available trip-level data alone, offering a scalable and transparent means to study ridesourcing systems in real operational contexts.