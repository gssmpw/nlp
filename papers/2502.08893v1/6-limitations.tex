\section{Limitations and Future Work}

While our study provides valuable insights using both the original dataset and our simulation methods applied to the large-scale Chicago Public Dataset, it is not without limitations. These can be grouped into two categories: (1) limitations related to the analysis of the original dataset, and (2) limitations of the simulation methodology.

First, regarding the original dataset, the lack of sufficient features and detailed information required us to make certain assumptions. For instance, we estimated downtime between trips and assumed that drivers' earnings equaled the trip costs paid by passengers. However, this does not fully capture drivers' actual earnings, as it removes platform fees. To improve the accuracy of earnings estimation, future data collection efforts should include more granular financial details, such as platform fees, driver expenses, and net income.

Second, while our simulation model effectively assign trips to hypothetical drivers and return meaningful insights, it lacks a robust evaluation framework due to the absence of ground-truth data. To validate the findings, we plan to conduct a large-scale qualitative study of driver groups in Chicago. Additionally, due to computational constraints, we limited our analysis to weekly patterns in 2019 and 2023. Future work will focus on enhancing the efficiency of our simulation framework to expand its application to other timeframes, improving its generalizability. Finally, we encourage the development of benchmarks and more quantitative studies to better assess simulation approaches for this type of problem.
%     do not know driver earnings directly
%     algorithm itself
%     evaluation
    

% datasets, expand to more cities
