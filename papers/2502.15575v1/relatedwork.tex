\section{Prior work on random features}
After the introduction of RFF in \cite{rahimi2007random}, several improvements have been suggested. We focus primarily on the algorithmic improvements. See \cite{liu2021random} for a comprehensive survey. 

For the Gaussian kernel, a long line of work improved upon the vanilla RFF formulation to either reduce space and time complexity or improve approximation and generalization error. For example, Fastfood \cite{Fastfood} and it's generalization $\mathcal{P}$-model \cite{P_model} utilize Hadamard matrices to speed up computation of $x\mapsto Wx$. An alternate approach was suggested in \cite{scrf} which utilized signed circulant matrices to generate features. \cite{NRFF} argues that  normalizing the inputs leads to gains in approximation and generalization performance due to restricting the diameter of the data.

\begin{remark}[Other random feature maps]
    Techniques like ORF, \cite{yu2016orthogonal}, use an orthogonal rotation matrix along with a radial distribution multiplier. This is shown to be unbiased and has a lower variance than vanilla RFF. In \cite{yu2016orthogonal} they also introduce ORF-prime, a version of ORF, with a constant radial component, which works well in practice for the Gaussian distribution. However, recently in \cite[Thm. 2]{demni2024orthogonal}, this was shown to actually approximate the normalized Bessel function of the first kind, which is different from the Gaussian. Structured ORF (SORF) \cite{yu2016orthogonal} uses products of pairs of sign-flipping and Hadamard matrices ($HD$ blocks) to approximate $W$. SORF uses ORF-prime and replaces the orthogonal matrix with products of $HD$ blocks. However, whether SORF also demonstrates the  bias shown in \cite{demni2024orthogonal} is not known. For the above reasons we do not include SORF and ORF-prime in our discussion. \cite{Bojarski2016StructuredAA} extends this idea using random spinners. \cite{ROM_} generalizes SORF by using an arbitrary number of sign-flipping and Hadamard matrix blocks and also provides intution for why 3 blocks work well in practice. 
\end{remark}

Quadrature rules approximate the Fourier transform integral \cite{Gauss_quad}, \cite{Quad_based_feats}, however, these works assume separability of the integral which is available in case of the Gaussian and the $\ell_1$-Laplacian kernel. \cite{Bach_equivalence} showed the equivalence between Random Fourier Features and kernel quadrature rules. While quadrature based approaches are more general, they too assume separability, and subgaussianity. 

Random features for dot product kernels introduced 
in \cite{pmlr-v22-kar12}, and were generalized in \cite{wacker2024improved} to include sketching. Dot product kernels rely on the Mclaurin series expansion, which assumes existence of all derivatives, an assumption not satisfied by \eqref{def:l2laplacian}.