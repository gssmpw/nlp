%%%%%%%%%%%%%%%%%%%%%%%%%%%%%%%%%%%%%%%%%%%%%%%%%%%%%%%%%%%%%%%%%%%%%%%%%%%%%%%%
% Template for ASPLOS papers.
%
% History:
% 
% ASPLOS originally used jpaper.cls for submission but required acmart.cls for the
% final camera-ready version. To avoid a change in format, starting ASPLOS 2024 Fall 
% cycle, both the submission and the camera-ready versions started using acmart.cls.
%
%%%%%%%%%%%%%%%%%%%%%%%%%%%%%%%%%%%%%%%%%%%%%%%%%%%%%%%%%%%%%%%%%%%%%%%%%%%%%%%%%%

% use the base acmart.cls version 1.92
% use the sigplan proceeding template with the default 10 pt fonts
% nonacm option removes ACM related text in the submission. 
\documentclass[sigplan,screen,nonacm]{acmart}

% enable page numbers
\settopmatter{printacmref=true}

% make references clickable 
\usepackage[]{hyperref}
\def\method{\text MixMin~}
\def\methodnospace{\text MixMin}
\def\genmethod{$\mathbb{R}$\text Min~}
\def\genmethodnospace{ $\mathbb{R}$\text Min}

\copyrightyear{2025}
\acmYear{2025}
% \setcopyright{cc}
% \setcctype{by}
\setcopyright{rightsretained}
\acmConference[ASPLOS '25]{Proceedings of the 30th ACM International Conference on Architectural Support for Programming Languages and Operating Systems, Volume 2}{March 30-April 3, 2025}{Rotterdam, Netherlands}
\acmBooktitle{Proceedings of the 30th ACM International Conference on Architectural Support for Programming Languages and Operating Systems, Volume 2 (ASPLOS '25), March 30-April 3, 2025, Rotterdam, Netherlands}
\acmDOI{10.1145/3676641.3716248}
\acmISBN{979-8-4007-1079-7/25/03}

\begin{document}

\title{Be CIM or Be Memory: A Dual-mode-aware DNN Compiler for CIM Accelerators}

\author{Shixin Zhao}
\affiliation{%
    \institution{Institute of Computing Technology, Chinese Academy of Sciences, University of Chinese Academy of Sciences}
  \city{Beijing}
  \country{China}
  \postcode{100190}
}
\email{zhaoshixin18@mails.ucas.ac.cn}

\author{Yuming Li}
\affiliation{%
  \institution{Institute of Computing Technology, Chinese Academy of Sciences, University of Chinese Academy of Sciences}
  \city{Beijing}
  \country{China}
}
\email{liyuming22@mails.ucas.ac.cn}

\author{Bing Li}
\affiliation{%
  \institution{Institute of Microelectronics, Chinese Academy of Sciences}
  \city{Beijing}
  \country{China}}
\email{libing2024@ime.ac.cn}

\author{Yintao He}
\affiliation{%
  \institution{Institute of Computing Technology, Chinese Academy of Sciences, University of Chinese Academy of Sciences}
  \city{Beijing}
  \country{China}}
\email{heyintao19z@ict.ac.cn}


\author{Mengdi Wang}
\affiliation{%
  \institution{State Key Lab of Processors, Institute of Computing Technology,  Chinese Academy of Sciences}
  \city{Beijing}
  \country{China}}
\email{wangmengdi@ict.ac.cn}

\author{Yinhe Han}
\affiliation{%
  \institution{State Key Lab of Processors, Institute of Computing Technology,  Chinese Academy of Sciences}
  \city{Beijing}
  \country{China}}
\email{yinhes@ict.ac.cn}

\author{Ying Wang}
\authornote{Corresponding author.}
\affiliation{%
  \institution{State Key Lab of Processors, Institute of Computing Technology,  Chinese Academy of Sciences}
  \city{Beijing}
  \country{China}}
\email{wangying2009@ict.ac.cn}

\renewcommand{\shortauthors}{Shixin Zhao et al.}

\begin{abstract}
Computing-in-memory (CIM) architectures demonstrate superior performance over traditional architectures. \update{To unleash the potential of CIM accelerators, many compilation methods have been proposed, }
focusing on application scheduling optimization specific to CIM. 
% exploring the application scheduling optimization targeted on CIM. 
However, existing compilation methods often overlook CIM's capability to switch dynamically between compute and memory modes, which is crucial for accommodating the diverse memory and computational needs of real-world deep neural network architectures, especially the emerging large language models.
% This flexibility is essential for accommodating the diverse memory and computation needs of real-world deep neural network architectures (DNNs). 
\update{To fill this gap, we introduce CMSwitch, a novel compiler to optimize resource allocation for CIM accelerators with adaptive mode-switching capabilities, thereby enhancing the performance of DNN applications.}
% dual-mode-aware compiler that optimizes CIM dual-mode resource allocation for real-world DNNs.
Specifically, our approach integrates the compute-memory mode switch into the CIM compilation optimization space by introducing a new hardware abstraction attribute. Then, we propose a novel compilation optimization pass that identifies the optimal network segment and the corresponding mode resource allocations 
% dual-mode CIM allocations 
using dynamic programming and mixed-integer programming. CMSwitch uses the tailored meta-operator to express the compilation result in a generalized manner.
Evaluation results demonstrate that CMSwitch achieves \update{an average speedup of 1.31$\times$ compared to existing SOTA CIM compilation works, highlighting CMSwitch's effectiveness in fully exploiting the potential of CIM processors for a wide range of real-world DNN applications.}
\end{abstract}
\begin{CCSXML}
<ccs2012>
<concept>
<concept_id>10010583.10010786.10010809</concept_id>
<concept_desc>Hardware~Memory and dense storage</concept_desc>
<concept_significance>500</concept_significance>
</concept>
<concept>
<concept_id>10011007.10011006.10011041</concept_id>
<concept_desc>Software and its engineering~Compilers</concept_desc>
<concept_significance>500</concept_significance>
</concept>
</ccs2012>
\end{CCSXML}

\ccsdesc[500]{Hardware~Memory and dense storage}
\ccsdesc[500]{Software and its engineering~Compilers}

%%
%% Keywords. The author(s) should pick words that accurately describe
%% the work being presented. Separate the keywords with commas.
\keywords{Compute-in-memory (CIM), Compilation, Deep Neural Network (DNN)}


\maketitle % should come after the abstract
% \pagestyle{plain} % should come right after \maketitle


\section{Introduction}
Backdoor attacks pose a concealed yet profound security risk to machine learning (ML) models, for which the adversaries can inject a stealth backdoor into the model during training, enabling them to illicitly control the model's output upon encountering predefined inputs. These attacks can even occur without the knowledge of developers or end-users, thereby undermining the trust in ML systems. As ML becomes more deeply embedded in critical sectors like finance, healthcare, and autonomous driving \citep{he2016deep, liu2020computing, tournier2019mrtrix3, adjabi2020past}, the potential damage from backdoor attacks grows, underscoring the emergency for developing robust defense mechanisms against backdoor attacks.

To address the threat of backdoor attacks, researchers have developed a variety of strategies \cite{liu2018fine,wu2021adversarial,wang2019neural,zeng2022adversarial,zhu2023neural,Zhu_2023_ICCV, wei2024shared,wei2024d3}, aimed at purifying backdoors within victim models. These methods are designed to integrate with current deployment workflows seamlessly and have demonstrated significant success in mitigating the effects of backdoor triggers \cite{wubackdoorbench, wu2023defenses, wu2024backdoorbench,dunnett2024countering}.  However, most state-of-the-art (SOTA) backdoor purification methods operate under the assumption that a small clean dataset, often referred to as \textbf{auxiliary dataset}, is available for purification. Such an assumption poses practical challenges, especially in scenarios where data is scarce. To tackle this challenge, efforts have been made to reduce the size of the required auxiliary dataset~\cite{chai2022oneshot,li2023reconstructive, Zhu_2023_ICCV} and even explore dataset-free purification techniques~\cite{zheng2022data,hong2023revisiting,lin2024fusing}. Although these approaches offer some improvements, recent evaluations \cite{dunnett2024countering, wu2024backdoorbench} continue to highlight the importance of sufficient auxiliary data for achieving robust defenses against backdoor attacks.

While significant progress has been made in reducing the size of auxiliary datasets, an equally critical yet underexplored question remains: \emph{how does the nature of the auxiliary dataset affect purification effectiveness?} In  real-world  applications, auxiliary datasets can vary widely, encompassing in-distribution data, synthetic data, or external data from different sources. Understanding how each type of auxiliary dataset influences the purification effectiveness is vital for selecting or constructing the most suitable auxiliary dataset and the corresponding technique. For instance, when multiple datasets are available, understanding how different datasets contribute to purification can guide defenders in selecting or crafting the most appropriate dataset. Conversely, when only limited auxiliary data is accessible, knowing which purification technique works best under those constraints is critical. Therefore, there is an urgent need for a thorough investigation into the impact of auxiliary datasets on purification effectiveness to guide defenders in  enhancing the security of ML systems. 

In this paper, we systematically investigate the critical role of auxiliary datasets in backdoor purification, aiming to bridge the gap between idealized and practical purification scenarios.  Specifically, we first construct a diverse set of auxiliary datasets to emulate real-world conditions, as summarized in Table~\ref{overall}. These datasets include in-distribution data, synthetic data, and external data from other sources. Through an evaluation of SOTA backdoor purification methods across these datasets, we uncover several critical insights: \textbf{1)} In-distribution datasets, particularly those carefully filtered from the original training data of the victim model, effectively preserve the model’s utility for its intended tasks but may fall short in eliminating backdoors. \textbf{2)} Incorporating OOD datasets can help the model forget backdoors but also bring the risk of forgetting critical learned knowledge, significantly degrading its overall performance. Building on these findings, we propose Guided Input Calibration (GIC), a novel technique that enhances backdoor purification by adaptively transforming auxiliary data to better align with the victim model’s learned representations. By leveraging the victim model itself to guide this transformation, GIC optimizes the purification process, striking a balance between preserving model utility and mitigating backdoor threats. Extensive experiments demonstrate that GIC significantly improves the effectiveness of backdoor purification across diverse auxiliary datasets, providing a practical and robust defense solution.

Our main contributions are threefold:
\textbf{1) Impact analysis of auxiliary datasets:} We take the \textbf{first step}  in systematically investigating how different types of auxiliary datasets influence backdoor purification effectiveness. Our findings provide novel insights and serve as a foundation for future research on optimizing dataset selection and construction for enhanced backdoor defense.
%
\textbf{2) Compilation and evaluation of diverse auxiliary datasets:}  We have compiled and rigorously evaluated a diverse set of auxiliary datasets using SOTA purification methods, making our datasets and code publicly available to facilitate and support future research on practical backdoor defense strategies.
%
\textbf{3) Introduction of GIC:} We introduce GIC, the \textbf{first} dedicated solution designed to align auxiliary datasets with the model’s learned representations, significantly enhancing backdoor mitigation across various dataset types. Our approach sets a new benchmark for practical and effective backdoor defense.



\section{Background} \label{section:LLM}

% \subsection{Large Language Model (LLM)}   

Figure~\ref{fig:LLaMA_model}(a) shows that a decoder-only LLM initially processes a user prompt in the “prefill” stage and subsequently generates tokens sequentially during the “decoding” stage.
Both stages contain an input embedding layer, multiple decoder transformer blocks, an output embedding layer, and a sampling layer.
Figure~\ref{fig:LLaMA_model}(b) demonstrates that the decoder transformer blocks consist of a self attention and a feed-forward network (FFN) layer, each paired with residual connection and normalization layers. 

% Differentiate between encoder/decoder, explain why operation intensity is low, explain the different parts of a transformer block. Discuss Table II here. 

% Explain the architecture with Llama2-70B.

% \begin{table}[thb]
% \renewcommand\arraystretch{1.05}
% \centering
% % \vspace{-5mm}
%     \caption{ML Model Parameter Size and Operational Intensity}
%     \vspace{-2mm}
%     \small
%     \label{tab:ML Model Parameter Size and Operational Intensity}    
%     \scalebox{0.95}{
%         \begin{tabular}{|c|c|c|c|c|}
%             \hline
%             & Llama2 & BLOOM & BERT & ResNet \\
%             Model & (70B) & (176B) & & 152 \\
%             \hline
%             Parameter Size (GB) & 140 & 352 & 0.17 & 0.16 \\
%             \hline
%             Op Intensity (Ops/Byte) & 1 & 1 & 282 & 346 \\
%             \hline
%           \end{tabular}
%     }
% \vspace{-3mm}
% \end{table}

% {\fontsize{8pt}{11pt}\selectfont 8pt font size test Memory Requirement}

\begin{figure}[t]
    \centering
    \includegraphics[width=8cm]{Figure/LLaMA_model_new_new.pdf}
    \caption{(a) Prefill stage encodes prompt tokens in parallel. Decoding stage generates output tokens sequentially.
    (b) LLM contains N$\times$ decoder transformer blocks. 
    (c) Llama2 model architecture.}
    \label{fig:LLaMA_model}
\end{figure}

Figure~\ref{fig:LLaMA_model}(c) demonstrates the Llama2~\cite{touvron2023llama} model architecture as a representative LLM.
% The self attention layer requires three GEMVs\footnote{GEMVs in multi-head attention~\cite{attention}, narrow GEMMs in grouped-query attention~\cite{gqa}.} to generate query, key and value vectors.
In the self-attention layer, query, key and value vectors are generated by multiplying input vector to corresponding weight matrices.
These matrices are segmented into multiple heads, representing different semantic dimensions.
The query and key vectors go though Rotary Positional Embedding (RoPE) to encode the relative positional information~\cite{rope-paper}.
Within each head, the generated key and value vectors are appended to their caches.
The query vector is multiplied by the key cache to produce a score vector.
After the Softmax operation, the score vector is multiplied by the value cache to yield the output vector.
The output vectors from all heads are concatenated and multiplied by output weight matrix, resulting in a vector that undergoes residual connection and Root Mean Square layer Normalization (RMSNorm)~\cite{rmsnorm-paper}.
The residual connection adds up the input and output vectors of a layer to avoid vanishing gradient~\cite{he2016deep}.
The FFN layer begins with two parallel fully connections, followed by a Sigmoid Linear Unit (SiLU), and ends with another fully connection.
\section{Motivation}
\label{sec:motivation}

During training, generative video models take a noised training video and compute a loss by comparing the model's prediction with the original video, the noise, or a combination of the two~\cite{Ho2020DenoisingDP,flow-matching} (Sec.~\ref{sec:preliminaries}).
We hypothesize that this formulation biases the model towards appearance-based features, such as color and texture, as these dominate pixel-wise differences. Consequently, the model is less inclined to attend to temporal information, such as dynamics or physics, which contribute less to the objective. To demonstrate this claim, we perform experiments to evaluate the sensitivity of the model to temporal incoherence. The following experiments are conducted on DiT-4B~\cite{dit} for efficiency.

\begin{figure}[t!]
    \centering
        \includegraphics[width=0.45\textwidth]{figures/motivation_plot.pdf}
         \vspace{-14px}
    \caption{\textbf{Motivation Experiment.} We compare the model's loss before and after randomly permuting the video frames, using a ``vanilla'' DiT (orange) and our fine-tuned model (blue). The original model is \emph{nearly invariant} to temporal perturbations for $t\leq 60$. }
        \label{fig:motivation-a}
    \vspace{-16px}
    \label{fig:motivation}
\end{figure}

\begin{figure*}[ht!]
\centering
\includegraphics[width=1.01\textwidth]{figures/architecture.pdf}
\vspace{-18px}
\caption{\textbf{VideoJAM Framework.} VideoJAM is constructed of two units; (a) \textbf{Training.} Given an input video $x_1$ and its motion representation $d_1$, both signals are noised and embedded to a \emph{single, joint} latent representation using a linear layer, $\textbf{W}^+_{in}$. The diffusion model processes the input, and two linear projection layers predict both appearance and motion from the joint representation. (b) \textbf{Inference.} We propose \emph{Inner-Guidance}, where the model's own noisy motion prediction is used to guide the video prediction at each step. }
\label{fig:architecture}
\vspace{-6px}
\end{figure*}

We conduct an experiment where two variants of videos are noised and fed to the model—first, the plain video without intervention, and second, the video after applying a \emph{random permutation} to its frames. 
Assuming the model captures temporal information, we anticipate that the temporally incoherent (perturbed) input will result in a higher measured loss compared to the temporally coherent input.

Given a random set of $35,000$ training videos, we noise each video to a random denoising step $t\in[0,99]$. We then examine the difference in the loss measured before and after the permutation and aggregate the results per timestep. We consider two models-- the ``vanilla'' DiT, which employs a pixel-based objective, and our fine-tuned VideoJAM model, which adds an explicit motion objective (Sec.~\ref{sec:method}). 

The results of this experiment are reported in Fig.~\ref{fig:motivation}.
As can be observed, the original model appears to be \emph{nearly invariant} to frame shuffling until step $60$ of the generation. This implies that the model fails to distinguish between a valid video and a temporally incoherent one. In stark contrast, our model is extremely sensitive to these perturbations, as is indicated by the significant gap in the calculated loss. 

In App.~\ref{sec:motivation_supp} we include a qualitative experiment demonstrating that the steps $t\leq 60$ determine the coarse motion in the video. Both results suggest that the training objective is less sensitive to temporal incoherence, leading models to favor appearance over motion.


Effective human-robot cooperation in CoNav-Maze hinges on efficient communication. Maximizing the human’s information gain enables more precise guidance, which in turn accelerates task completion. Yet for the robot, the challenge is not only \emph{what} to communicate but also \emph{when}, as it must balance gathering information for the human with pursuing immediate goals when confident in its navigation.

To achieve this, we introduce \emph{Information Gain Monte Carlo Tree Search} (IG-MCTS), which optimizes both task-relevant objectives and the transmission of the most informative communication. IG-MCTS comprises three key components:
\textbf{(1)} A data-driven human perception model that tracks how implicit (movement) and explicit (image) information updates the human’s understanding of the maze layout.
\textbf{(2)} Reward augmentation to integrate multiple objectives effectively leveraging on the learned perception model.
\textbf{(3)} An uncertainty-aware MCTS that accounts for unobserved maze regions and human perception stochasticity.
% \begin{enumerate}[leftmargin=*]
%     \item A data-driven human perception model that tracks how implicit (movement) and explicit (image transmission) information updates the human’s understanding of the maze layout.
%     \item Reward augmentation to integrate multiple objectives effectively leveraging on the learned perception model.
%     \item An uncertainty-aware MCTS that accounts for unobserved maze regions and human perception stochasticity.
% \end{enumerate}

\subsection{Human Perception Dynamics}
% IG-MCTS seeks to optimize the expected novel information gained by the human through the robot’s actions, including both movement and communication. Achieving this requires a model of how the human acquires task-relevant information from the robot.

% \subsubsection{Perception MDP}
\label{sec:perception_mdp}
As the robot navigates the maze and transmits images, humans update their understanding of the environment. Based on the robot's path, they may infer that previously assumed blocked locations are traversable or detect discrepancies between the transmitted image and their map.  

To formally capture this process, we model the evolution of human perception as another Markov Decision Process, referred to as the \emph{Perception MDP}. The state space $\mathcal{X}$ represents all possible maze maps. The action space $\mathcal{S}^+ \times \mathcal{O}$ consists of the robot's trajectory between two image transmissions $\tau \in \mathcal{S}^+$ and an image $o \in \mathcal{O}$. The unknown transition function $F: (x, (\tau, o)) \rightarrow x'$ defines the human perception dynamics, which we aim to learn.

\subsubsection{Crowd-Sourced Transition Dataset}
To collect data, we designed a mapping task in the CoNav-Maze environment. Participants were tasked to edit their maps to match the true environment. A button triggers the robot's autonomous movements, after which it captures an image from a random angle.
In this mapping task, the robot, aware of both the true environment and the human’s map, visits predefined target locations and prioritizes areas with mislabeled grid cells on the human’s map.
% We assume that the robot has full knowledge of both the actual environment and the human’s current map. Leveraging this knowledge, the robot autonomously navigates to all predefined target locations. It then randomly selects subsequent goals to reach, prioritizing grid locations that remain mislabeled on the human’s map. This ensures that the robot’s actions are strategically focused on providing useful information to improve map accuracy.

We then recruited over $50$ annotators through Prolific~\cite{palan2018prolific} for the mapping task. Each annotator labeled three randomly generated mazes. They were allowed to proceed to the next maze once the robot had reached all four goal locations. However, they could spend additional time refining their map before moving on. To incentivize accuracy, annotators receive a performance-based bonus based on the final accuracy of their annotated map.


\subsubsection{Fully-Convolutional Dynamics Model}
\label{sec:nhpm}

We propose a Neural Human Perception Model (NHPM), a fully convolutional neural network (FCNN), to predict the human perception transition probabilities modeled in \Cref{sec:perception_mdp}. We denote the model as $F_\theta$ where $\theta$ represents the trainable weights. Such design echoes recent studies of model-based reinforcement learning~\cite{hansen2022temporal}, where the agent first learns the environment dynamics, potentially from image observations~\cite{hafner2019learning,watter2015embed}.

\begin{figure}[t]
    \centering
    \includegraphics[width=0.9\linewidth]{figures/ICML_25_CNN.pdf}
    \caption{Neural Human Perception Model (NHPM). \textbf{Left:} The human's current perception, the robot's trajectory since the last transmission, and the captured environment grids are individually processed into 2D masks. \textbf{Right:} A fully convolutional neural network predicts two masks: one for the probability of the human adding a wall to their map and another for removing a wall.}
    \label{fig:nhpm}
    \vskip -0.1in
\end{figure}

As illustrated in \Cref{fig:nhpm}, our model takes as input the human’s current perception, the robot’s path, and the image captured by the robot, all of which are transformed into a unified 2D representation. These inputs are concatenated along the channel dimension and fed into the CNN, which outputs a two-channel image: one predicting the probability of human adding a new wall and the other predicting the probability of removing a wall.

% Our approach builds on world model learning, where neural networks predict state transitions or environmental updates based on agent actions and observations. By leveraging the local feature extraction capabilities of CNNs, our model effectively captures spatial relationships and interprets local changes within the grid maze environment. Similar to prior work in localization and mapping, the CNN architecture is well-suited for processing spatially structured data and aligning the robot’s observations with human map updates.

To enhance robustness and generalization, we apply data augmentation techniques, including random rotation and flipping of the 2D inputs during training. These transformations are particularly beneficial in the grid maze environment, which is invariant to orientation changes.

\subsection{Perception-Aware Reward Augmentation}
The robot optimizes its actions over a planning horizon \( H \) by solving the following optimization problem:
\begin{subequations}
    \begin{align}
        \max_{a_{0:H-1}} \;
        & \mathop{\mathbb{E}}_{T, F} \left[ \sum_{t=0}^{H-1} \gamma^t \left(\underbrace{R_{\mathrm{task}}(\tau_{t+1}, \zeta)}_{\text{(1) Task reward}} + \underbrace{\|x_{t+1}-x_t\|_1}_{\text{(2) Info reward}}\right)\right] \label{obj}\\ 
        \subjectto \quad
        &x_{t+1} = F(x_t, (\tau_t, a_t)), \quad a_t\in\Ocal \label{const:perception_update}\\ 
        &\tau_{t+1} = \tau_t \oplus T(s_t, a_t), \quad a_t\in \Ucal\label{const:history_update}
    \end{align}
\end{subequations} 

The objective in~\eqref{obj} maximizes the expected cumulative reward over \( T \) and \( F \), reflecting the uncertainty in both physical transitions and human perception dynamics. The reward function consists of two components: 
(1) The \emph{task reward} incentivizes efficient navigation. The specific formulation for the task in this work is outlined in \Cref{appendix:task_reward}.
(2) The \emph{information reward} quantifies the change in the human’s perception due to robot actions, computed as the \( L_1 \)-norm distance between consecutive perception states.  

The constraint in~\eqref{const:history_update} ensures that for movement actions, the trajectory history \( \tau_t \) expands with new states based on the robot’s chosen actions, where \( s_t \) is the most recent state in \( \tau_t \), and \( \oplus \) represents sequence concatenation. 
In constraint~\eqref{const:perception_update}, the robot leverages the learned human perception dynamics \( F \) to estimate the evolution of the human’s understanding of the environment from perception state $x_t$ to $x_{t+1}$ based on the observed trajectory \( \tau_t \) and transmitted image \( a_t\in\Ocal \). 
% justify from a cognitive science perspective
% Cognitive science research has shown that humans read in a way to maximize the information gained from each word, aligning with the efficient coding principle, which prioritizes minimizing perceptual errors and extracting relevant features under limited processing capacity~\cite{kangassalo2020information}. Drawing on this principle, we hypothesize that humans similarly prioritize task-relevant information in multimodal settings. To accommodate this cognitive pattern, our robot policy selects and communicates high information-gain observations to human operators, akin to summarizing key insights from a lengthy article.
% % While the brain naturally seeks to gain information, the brain employs various strategies to manage information overload, including filtering~\cite{quiroga2004reducing}, limiting/working memory, and prioritizing information~\cite{arnold2023dealing}.
% In this context of our setup, we optimize the selection of camera angles to maximize the human operator's information gain about the environment. 

\subsection{Information Gain Monte Carlo Tree Search (IG-MCTS)}
IG-MCTS follows the four stages of Monte Carlo tree search: \emph{selection}, \emph{expansion}, \emph{rollout}, and \emph{backpropagation}, but extends it by incorporating uncertainty in both environment dynamics and human perception. We introduce uncertainty-aware simulations in the \emph{expansion} and \emph{rollout} phases and adjust \emph{backpropagation} with a value update rule that accounts for transition feasibility.

\subsubsection{Uncertainty-Aware Simulation}
As detailed in \Cref{algo:IG_MCTS}, both the \emph{expansion} and \emph{rollout} phases involve forward simulation of robot actions. Each tree node $v$ contains the state $(\tau, x)$, representing the robot's state history and current human perception. We handle the two action types differently as follows:
\begin{itemize}
    \item A movement action $u$ follows the environment dynamics $T$ as defined in \Cref{sec:problem}. Notably, the maze layout is observable up to distance $r$ from the robot's visited grids, while unexplored areas assume a $50\%$ chance of walls. In \emph{expansion}, the resulting search node $v'$ of this uncertain transition is assigned a feasibility value $\delta = 0.5$. In \emph{rollout}, the transition could fail and the robot remains in the same grid.
    
    \item The state transition for a communication step $o$ is governed by the learned stochastic human perception model $F_\theta$ as defined in \Cref{sec:nhpm}. Since transition probabilities are known, we compute the expected information reward $\bar{R_\mathrm{info}}$ directly:
    \begin{align*}
        \bar{R_\mathrm{info}}(\tau_t, x_t, o_t) &= \mathbb{E}_{x_{t+1}}\|x_{t+1}-x_t\|_1 \\
        &= \|p_\mathrm{add}\|_1 + \|p_\mathrm{remove}\|_1,
    \end{align*}
    where $(p_\mathrm{add}, p_\mathrm{remove}) \gets F_\theta(\tau_t, x_t, o_t)$ are the estimated probabilities of adding or removing walls from the map. 
    Directly computing the expected return at a node avoids the high number of visitations required to obtain an accurate value estimate.
\end{itemize}

% We denote a node in the search tree as $v$, where $s(v)$, $r(v)$, and $\delta(v)$ represent the state, reward, and transition feasibility at $v$, respectively. The visit count of $v$ is denoted as $N(v)$, while $Q(v)$ represents its total accumulated return. The set of child nodes of $v$ is denoted by $\mathbb{C}(v)$.

% The goal of each search is to plan a sequence for the robot until it reaches a goal or transmits a new image to the human. We initialize the search tree with the current human guidance $\zeta$, and the robot's approximation of human perception $x_0$. Each search node consists consists of the state information required by our reward augmentation: $(\tau, x)$. A node is terminal if it is the resulting state of a communication step, or if the robot reaches a goal location. 

% A rollout from the expanded node simulates future transitions until reaching a terminal state or a predefined depth $H$. Actions are selected randomly from the available action set $\mathcal{A}(s)$. If an action's feasibility is uncertain due to the environment's unknown structure, the transition occurs with probability $\delta(s, a)$. When a random number draw deems the transition infeasible, the state remains unchanged. On the other hand, for communication steps, we don't resolve the uncertainty but instead compute the expected information gain reward: \philip{TODO: adjust notation}
% \begin{equation}
%     \mathbb{E}\left[R_\mathrm{info}(\tau, x')\right] = \sum \mathrm{NPM(\tau, o)}.
% \end{equation}

\subsubsection{Feasibility-Adjusted Backpropagation}
During backpropagation, the rewards obtained from the simulation phase are propagated back through the tree, updating the total value $Q(v)$ and the visitation count $N(v)$ for all nodes along the path to the root. Due to uncertainty in unexplored environment dynamics, the rollout return depends on the feasibility of the transition from the child node. Given a sample return \(q'_{\mathrm{sample}}\) at child node \(v'\), the parent node's return is:
\begin{equation}
    q_{\mathrm{sample}} = r + \gamma \left[ \delta' q'_{\mathrm{sample}} + (1 - \delta') \frac{Q(v)}{N(v)} \right],
\end{equation}
where $\delta'$ represents the probability of a successful transition. The term \((1 - \delta')\) accounts for failed transitions, relying instead on the current value estimate.

% By incorporating uncertainty-aware rollouts and backpropagation, our approach enables more robust decision-making in scenarios where the environment dynamics is unknown and avoids simulation of the stochastic human perception dynamics.

\section{Evaluation}

\subsection{Experimental Setup}\label{subsec:exp_setup}

In our evaluation, we adopt \textit{Code-Llama-7B} as the pre-trained model for fine-tuning, employing the low-rank-adaption (QLoRA)~\cite{hu2021lora, dettmers2024qlora} technique for faster training and lower memory consumption. 
Key configurations include loading the model in 8-bit, a sequence length of 4096, sample packing, and padding to sequence length. We set the warmup steps to 100, with a gradient accumulation of 4 steps, a micro-batch size of 4, and an inference batch size of 2.
For both syntax and functionality checks, we measure pass@3 accuracy as metrics. In the ablation study from~\secref{subsec:exp_finetune} to~\secref{subsec:complexity}, we adopt \textit{MachineGen} for evaluation.

Experiments are conducted on a server with four NVIDIA L20 GPUs (48 GB each), an 80 vCPU Intel® Xeon® Platinum 8457C, and 100GB of RAM. This setup ensures sufficient computational power and memory to handle the intensive demands of fine-tuning and inference efficiently, especially for long data sequences in the feedback loop experiment. 

\subsection{Effect of Supervised Finetuning}\label{subsec:exp_finetune}
Our first ablation study investigates the effect of the model fine-tuning.
We evaluated the performance based on both syntax and functionality checks. 
As shown in~\figref{fig:finetune_cot}(a), the results demonstrate that the finetuning dramatically increases syntax correctness from $54.85$\% to $88.44$\%. 
More importantly, the impact of finetuning is even more pronounced in the functionality evaluation, where the non-finetuned model failed to achieve any correct functionality test, but the accuracy is improved to $53.20$\% in the finetuned model. These enhancements highlight the critical role of finetuning in producing not only syntactically correct but also functionally viable codes, which demonstrates the benefits of finetuning LLMs for hardware design in the HLS code generation task.




\subsection{Effect of Chain-of-Thought Prompting}\label{subsec:exp_cot}
To assess the effect of the chain-of-thought (CoT) technique,
we perform both syntax and functionality evaluation on the fine-tuned model with and without the use of CoT.
As indicated in~\figref{fig:finetune_cot}(b), incorporating CoT leads to a noticeable improvement in both metrics. 
Specifically, syntax correctness increases from $88.44$\% to $94.33$\%, and functionality score rises from $53.20$\% to $61.45$\%. 
The result demonstrates the effectiveness of CoT in enhancing the reasoning capability, thereby improving its overall performance.


\subsection{Effect of Feedback Loops}\label{subsec:exp_feedback}
Our two-step feedback loop provides both syntax and functionality feedback. We evaluate the impact of these feedback loops with different numbers of iterations, ranging from 0 to 2.The results, shown in Figure ~\figref{fig:syntax_feedback} and ~\figref{fig:func_feedback}, indicate that both syntax and functionality feedback loops significantly improve model performance, especially when combined with COT prompting. The initial feedback loop yields substantial accuracy improvements in both syntax correctness and functionality evaluation, though the second loop shows diminishing returns.Syntax feedback loops enhance both syntax correctness and functionality performance, suggesting that iterative refinement is particularly effective for complex tasks. Similarly, functionality feedback loops not only improve functionality checks but also boost syntax accuracy, indicating that enhancements in functional understanding contribute to better syntactic performance.

 \begin{figure}[t]
    \centering
    \includegraphics[width=1\linewidth]{./figures/merged_finetune_cot.pdf}
    \vspace{-5mm}
    \caption{Effect of fine-tuning and chain-of-thought.}
    \label{fig:finetune_cot}
\end{figure}

\begin{figure}[t]
    \centering
    \includegraphics[width=0.95\linewidth]{./figures/Effect_of_Syntax_Feedback_Loop.pdf}
    \vspace{-2mm}
    \caption{Effect of syntax feedback loop.}
    \vspace{-2mm}
    \label{fig:syntax_feedback}
\end{figure}
\begin{figure}[t]
    \centering
    \includegraphics[width=0.95\linewidth]{./figures/Effect_of_Functionality_Feedback_Loop.pdf}
    \caption{Effect of functionality feedback loop.}
    \vspace{-2mm}
    \label{fig:func_feedback}
\end{figure}


\subsection{Time Cost and Hardware Performance}\label{subsec:exp_timecost}

\figref{fig:time_cost} shows the time cost for generating 120 data entries under different conditions, measuring the impact of CoT and feedback loops. Without a feedback loop, CoT significantly reduces the time. Adding a syntax feedback loop increases the time, but CoT continues to notably decrease the duration. The functionality feedback loop is the most time-consuming, though CoT still provides a notable reduction, albeit less dramatic. This demonstrates CoT's effectiveness in reducing operational times across varying complexities.

For the test set,
we evaluate the latency and resource consumption of the generated \textit{HLS} designs using a Xilinx VCU118 as our target FPGA, with a clock frequency of $200$MHz and Xilinx Vivado 2020.1 for synthesis.
As shown in~\tabref{tb:perf_resource}, all \textit{HLS} designs demonstrate reasonable performance, with BRAM usage consistently remained at zero due to the design scale.

\begin{figure}
    \centering
    \includegraphics[width=0.95\linewidth]{./figures/Time_Cost_Analysis.pdf}
    \vspace{-2mm}
    \caption{Time cost of code generation.}
    \label{fig:time_cost}
\end{figure}


\begin{table}[htb]
\centering
\caption{Latency and resource usage of LLM-generated designs synthesized on a VCU118 FPGA.}
\label{tb:perf_resource}
\setlength\tabcolsep{1pt} 
\scalebox{0.8}{
% \begin{tabular}{L{2cm}ccC{1.5cm}C{2.5cm}}
\begin{tabular}{C{2.5cm}|C{1.9cm}|C{1.5cm}|C{1.5cm}|C{1.3cm}|C{1.3cm}}
\toprule
{}& \textbf{Latency} (ms)& \textbf{LUTs} & \textbf{Registers} & \textbf{DSP48s} & \textbf{BRAMs} \\ \midrule
{\textbf{Available}} & - & 1182240 & 2364480 & 6840 & 4320 \\ \midrule
{\textit{ellpack}} & 0.304 & 1011 & 1079 & 11 & 0 \\
{\textit{syrk}} & 21.537 & 1371 & 1621 & 19 & 0 \\
{\textit{syr2k}} & 40.626 & 1572 & 1771 & 19 & 0 \\
{\textit{stencil2d}} & 1.368 & 287 & 123 & 3 & 0 \\
{\textit{trmm-opt}} & 15.889 & 1262 & 1239 & 11 & 0 \\
{\textit{stencil3d}} & 21.537 & 1173 & 1271 & 20 & 0 \\
{\textit{symm}} & 24.601 & 1495 & 1777 & 19 & 0 \\
{\textit{symm-opt}} & 16.153 & 1361 & 1608 & 19 & 0 \\
{\textit{symm-opt-medium}} & 579.0 & 2223 & 2245 & 22 & 0 \\
\bottomrule
\end{tabular}}
\end{table}

\subsection{Effect of Task Complexity}\label{subsec:complexity}
We analyze the effects of code complexity on the performance of fine-tuning our language model with CoT prompting and tested without the use of any feedback loops during inference. 
We categorize \textit{MachineGen} into three classes according to their code complexity: easy, medium, and difficult.
The results shown in the \tabref{tab:model_performance} indicates a clear trend: as the complexity of the generated code increases, both syntax and functionality correctness rates decline. This outcome could be attributed to several factors. First, more complex code inherently presents more challenges in maintaining syntactic integrity and functional accuracy. Second, the absence of feedback loops in the inference phase may have limited the model's ability to self-correct emerging errors in more complicated code generations.

\begin{table}[h]
\centering
\caption{Performance across different complexity levels.}
\scalebox{0.8}{
\begin{tabular}{c|c|c}
\hline
\textbf{Test Set} & \textbf{Syntax Check} & \textbf{Functionality} \\
\hline
Easy & 96.67\% & 63.33\% \\
Medium & 96.67\% & 53.33\% \\
Difficult & 90\% & 53.33\% \\
\hline
\end{tabular}}
\label{tab:model_performance}
\end{table}

\subsection{Analysis of \textit{MachineGen} and \textit{HumanRefine}}
\begin{table}[h]
\centering
\caption{Performance on \textit{MachineGen} and \textit{HumanRefine}.}
\scalebox{0.9}{
\begin{tabular}{c|c|c}
\hline
\textbf{Test Set} & \textbf{Syntax Check} & \textbf{Functionality Check} \\
\hline
\textit{MachineGen} & 93.83\% & 62.24\% \\
\hline
\textit{HumanRefine} & 47.29\% & 21.36\% \\
\hline
\end{tabular}}
\vspace{-3mm}
\label{table:eval_comparison}
\end{table}

As shown in~\tabref{table:eval_comparison}, this section compares the performance of our model on \textit{MachineGen} and \textit{HumanRefine} test sets.
Our findings reveal that the performance on the \textit{HumanRefine} is significantly lower than on the \textit{MachineGen}. This disparity suggests that the model is more adept at handling machine-generated prompts. The primary reasons for this are: the model's training data bias towards machine-generated prompts, the increased complexity and nuanced nature of human-generated prompts, and the conciseness and clarity of human-generated prompts that often omit repetitive or explicit details found in machine-generated prompts, making it harder for the model to generate syntactically and functionally correct code.

\subsection{Thoughts, Insights, and Limitations}

\noindent \textbf{1. \textit{HLS} versus \textit{HDL} for AI-assisted code generation:} The selection of programming language for hardware code generation should mainly depend on two  factors:
\begin{itemize}[leftmargin=*]
    \item \textit{Quality of Generated Hardware Design}: The evaluation of hardware design's quality includes syntax correctness, functionality, and hardware performance.
    Since \textit{HLS} shares similar semantics and syntax with programming languages commonly used during LLM pre-training, this work demonstrates that the LLM-assisted code generation for \textit{HLS} has the potential to achieve high syntax and functional correctness in hardware designs. While this work does not leverage hardware performance as feedback for design generation, it identifies this aspect as a key direction for future research and enhancements.
    \item \textit{Runtime Cost of Hardware Generation}: Although \textit{HLS}-based designs typically require fewer tokens compared to \textit{HDL} during the code generation phase—suggesting potentially lower costs—the overall runtime costs associated with HLS synthesis must also be considered. A more comprehensive quantitative comparison of these runtime costs is planned for our future work. 
\end{itemize}

\noindent \textbf{2. Input instructions and datasets are crucial}: The fine-tuning of pre-trained LLMs on \textit{HLS} dataset can bring a significant improvement in the design quality, echoing findings from previous studies on \textit{Verilog} code generation~\cite{thakur2023verigen}. 
Additionally, during our evaluation, we found that employing simple CoT prompting largely improves hardware design quality. 
This result contrasts with the application of CoT in general-purpose programming languages, where a specialized form of CoT is necessary~\cite{li2023structured}.
Therefore, future efforts for further enhancement can focus on collecting high-quality datasets and exploring better refinement of input prompts.

\noindent \textbf{3. Limitations}: At the time of this research, more advanced reasoning models, such as DeepSeek-R1~\cite{guo2025deepseek}, were not available for evaluation. Additionally, test-time scaling approaches~\cite{welleck2024decoding} could be incorporated to further enhance performance in the future.
Moreover, we observe that the diversity of hardware designs in the benchmark is limited, which may impact the generalizability of our findings.
We intend to address these limitations in our future work.

\paragraph{Summary}
Our findings provide significant insights into the influence of correctness, explanations, and refinement on evaluation accuracy and user trust in AI-based planners. 
In particular, the findings are three-fold: 
(1) The \textbf{correctness} of the generated plans is the most significant factor that impacts the evaluation accuracy and user trust in the planners. As the PDDL solver is more capable of generating correct plans, it achieves the highest evaluation accuracy and trust. 
(2) The \textbf{explanation} component of the LLM planner improves evaluation accuracy, as LLM+Expl achieves higher accuracy than LLM alone. Despite this improvement, LLM+Expl minimally impacts user trust. However, alternative explanation methods may influence user trust differently from the manually generated explanations used in our approach.
% On the other hand, explanations may help refine the trust of the planner to a more appropriate level by indicating planner shortcomings.
(3) The \textbf{refinement} procedure in the LLM planner does not lead to a significant improvement in evaluation accuracy; however, it exhibits a positive influence on user trust that may indicate an overtrust in some situations.
% This finding is aligned with prior works showing that iterative refinements based on user feedback would increase user trust~\cite{kunkel2019let, sebo2019don}.
Finally, the propensity-to-trust analysis identifies correctness as the primary determinant of user trust, whereas explanations provided limited improvement in scenarios where the planner's accuracy is diminished.

% In conclusion, our results indicate that the planner's correctness is the dominant factor for both evaluation accuracy and user trust. Therefore, selecting high-quality training data and optimizing the training procedure of AI-based planners to improve planning correctness is the top priority. Once the AI planner achieves a similar correctness level to traditional graph-search planners, strengthening its capability to explain and refine plans will further improve user trust compared to traditional planners.

\paragraph{Future Research} Future steps in this research include expanding user studies with larger sample sizes to improve generalizability and including additional planning problems per session for a more comprehensive evaluation. Next, we will explore alternative methods for generating plan explanations beyond manual creation to identify approaches that more effectively enhance user trust. 
Additionally, we will examine user trust by employing multiple LLM-based planners with varying levels of planning accuracy to better understand the interplay between planning correctness and user trust. 
Furthermore, we aim to enable real-time user-planner interaction, allowing users to provide feedback and refine plans collaboratively, thereby fostering a more dynamic and user-centric planning process.


\section*{Acknowledgments}
We sincerely thank the anonymous reviewers for their insightful suggestions for improving this paper. This work was partially supported by the National Key R\&D Program of China (Grant No. 2023YFB4404400) and the National Natural Science Foundation of China (Grant No. 62222411, 62025404, 62204164). Ying Wang (wangying2009@ict.ac.cn) is the corresponding author.

\bibliographystyle{plain}
\balance
\bibliography{references}

\end{document}

