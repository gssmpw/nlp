
% VLDB template version of 2020-08-03 enhances the ACM template, version 1.7.0:
% https://www.acm.org/publications/proceedings-template
% The ACM Latex guide provides further information about the ACM template

\documentclass[sigconf, nonacm]{acmart}
\usepackage{graphicx}
\usepackage{caption}
\usepackage{subcaption}
\usepackage{adjustbox}
\usepackage{xspace}
% \usepackage[table]{xcolor}
\PassOptionsToPackage{table}{xcolor}
\usepackage[most]{tcolorbox}
\usepackage{enumitem}
\usepackage{fontawesome5}
% in local machine, uncomment following line and run:
%    pdflatex -shell-escape main.tex
% \usepackage[finalizecache,cachedir=minted-cache]{minted}
% In arXiv, include the minted-cache folder in the submission,
%  uncomment following line and run:
\usepackage[frozencache,cachedir=minted-cache]{minted} % for arXiv
% \usepackage{amssymb}
% \usepackage{amsthm}
% \usepackage{amsmath}
% \usepackage{bbm}
% \usepackage{dsfont}
% \usepackage{xcolor}
% \usepackage{hyperref}
% \usepackage{enumitem}
% \usepackage{xspace}

%% The following content must be adapted for the final version
% paper-specific
\newcommand\vldbdoi{XX.XX/XXX.XX}
\newcommand\vldbpages{XXX-XXX}
% issue-specific
\newcommand\vldbvolume{14}
\newcommand\vldbissue{1}
\newcommand\vldbyear{2020}
% should be fine as it is
\newcommand\vldbauthors{\authors}
\newcommand\vldbtitle{\shorttitle} 
% leave empty if no availability url should be set
\newcommand\vldbavailabilityurl{https://github.com/VIDA-NYU/bdikit-beaker}
% whether page numbers should be shown or not, use 'plain' for review versions, 'empty' for camera ready
\newcommand\vldbpagestyle{plain}


%
\setlength\unitlength{1mm}
\newcommand{\twodots}{\mathinner {\ldotp \ldotp}}
% bb font symbols
\newcommand{\Rho}{\mathrm{P}}
\newcommand{\Tau}{\mathrm{T}}

\newfont{\bbb}{msbm10 scaled 700}
\newcommand{\CCC}{\mbox{\bbb C}}

\newfont{\bb}{msbm10 scaled 1100}
\newcommand{\CC}{\mbox{\bb C}}
\newcommand{\PP}{\mbox{\bb P}}
\newcommand{\RR}{\mbox{\bb R}}
\newcommand{\QQ}{\mbox{\bb Q}}
\newcommand{\ZZ}{\mbox{\bb Z}}
\newcommand{\FF}{\mbox{\bb F}}
\newcommand{\GG}{\mbox{\bb G}}
\newcommand{\EE}{\mbox{\bb E}}
\newcommand{\NN}{\mbox{\bb N}}
\newcommand{\KK}{\mbox{\bb K}}
\newcommand{\HH}{\mbox{\bb H}}
\newcommand{\SSS}{\mbox{\bb S}}
\newcommand{\UU}{\mbox{\bb U}}
\newcommand{\VV}{\mbox{\bb V}}


\newcommand{\yy}{\mathbbm{y}}
\newcommand{\xx}{\mathbbm{x}}
\newcommand{\zz}{\mathbbm{z}}
\newcommand{\sss}{\mathbbm{s}}
\newcommand{\rr}{\mathbbm{r}}
\newcommand{\pp}{\mathbbm{p}}
\newcommand{\qq}{\mathbbm{q}}
\newcommand{\ww}{\mathbbm{w}}
\newcommand{\hh}{\mathbbm{h}}
\newcommand{\vvv}{\mathbbm{v}}

% Vectors

\newcommand{\av}{{\bf a}}
\newcommand{\bv}{{\bf b}}
\newcommand{\cv}{{\bf c}}
\newcommand{\dv}{{\bf d}}
\newcommand{\ev}{{\bf e}}
\newcommand{\fv}{{\bf f}}
\newcommand{\gv}{{\bf g}}
\newcommand{\hv}{{\bf h}}
\newcommand{\iv}{{\bf i}}
\newcommand{\jv}{{\bf j}}
\newcommand{\kv}{{\bf k}}
\newcommand{\lv}{{\bf l}}
\newcommand{\mv}{{\bf m}}
\newcommand{\nv}{{\bf n}}
\newcommand{\ov}{{\bf o}}
\newcommand{\pv}{{\bf p}}
\newcommand{\qv}{{\bf q}}
\newcommand{\rv}{{\bf r}}
\newcommand{\sv}{{\bf s}}
\newcommand{\tv}{{\bf t}}
\newcommand{\uv}{{\bf u}}
\newcommand{\wv}{{\bf w}}
\newcommand{\vv}{{\bf v}}
\newcommand{\xv}{{\bf x}}
\newcommand{\yv}{{\bf y}}
\newcommand{\zv}{{\bf z}}
\newcommand{\zerov}{{\bf 0}}
\newcommand{\onev}{{\bf 1}}

% Matrices

\newcommand{\Am}{{\bf A}}
\newcommand{\Bm}{{\bf B}}
\newcommand{\Cm}{{\bf C}}
\newcommand{\Dm}{{\bf D}}
\newcommand{\Em}{{\bf E}}
\newcommand{\Fm}{{\bf F}}
\newcommand{\Gm}{{\bf G}}
\newcommand{\Hm}{{\bf H}}
\newcommand{\Id}{{\bf I}}
\newcommand{\Jm}{{\bf J}}
\newcommand{\Km}{{\bf K}}
\newcommand{\Lm}{{\bf L}}
\newcommand{\Mm}{{\bf M}}
\newcommand{\Nm}{{\bf N}}
\newcommand{\Om}{{\bf O}}
\newcommand{\Pm}{{\bf P}}
\newcommand{\Qm}{{\bf Q}}
\newcommand{\Rm}{{\bf R}}
\newcommand{\Sm}{{\bf S}}
\newcommand{\Tm}{{\bf T}}
\newcommand{\Um}{{\bf U}}
\newcommand{\Wm}{{\bf W}}
\newcommand{\Vm}{{\bf V}}
\newcommand{\Xm}{{\bf X}}
\newcommand{\Ym}{{\bf Y}}
\newcommand{\Zm}{{\bf Z}}

% Calligraphic

\newcommand{\Ac}{{\cal A}}
\newcommand{\Bc}{{\cal B}}
\newcommand{\Cc}{{\cal C}}
\newcommand{\Dc}{{\cal D}}
\newcommand{\Ec}{{\cal E}}
\newcommand{\Fc}{{\cal F}}
\newcommand{\Gc}{{\cal G}}
\newcommand{\Hc}{{\cal H}}
\newcommand{\Ic}{{\cal I}}
\newcommand{\Jc}{{\cal J}}
\newcommand{\Kc}{{\cal K}}
\newcommand{\Lc}{{\cal L}}
\newcommand{\Mc}{{\cal M}}
\newcommand{\Nc}{{\cal N}}
\newcommand{\nc}{{\cal n}}
\newcommand{\Oc}{{\cal O}}
\newcommand{\Pc}{{\cal P}}
\newcommand{\Qc}{{\cal Q}}
\newcommand{\Rc}{{\cal R}}
\newcommand{\Sc}{{\cal S}}
\newcommand{\Tc}{{\cal T}}
\newcommand{\Uc}{{\cal U}}
\newcommand{\Wc}{{\cal W}}
\newcommand{\Vc}{{\cal V}}
\newcommand{\Xc}{{\cal X}}
\newcommand{\Yc}{{\cal Y}}
\newcommand{\Zc}{{\cal Z}}

% Bold greek letters

\newcommand{\alphav}{\hbox{\boldmath$\alpha$}}
\newcommand{\betav}{\hbox{\boldmath$\beta$}}
\newcommand{\gammav}{\hbox{\boldmath$\gamma$}}
\newcommand{\deltav}{\hbox{\boldmath$\delta$}}
\newcommand{\etav}{\hbox{\boldmath$\eta$}}
\newcommand{\lambdav}{\hbox{\boldmath$\lambda$}}
\newcommand{\epsilonv}{\hbox{\boldmath$\epsilon$}}
\newcommand{\nuv}{\hbox{\boldmath$\nu$}}
\newcommand{\muv}{\hbox{\boldmath$\mu$}}
\newcommand{\zetav}{\hbox{\boldmath$\zeta$}}
\newcommand{\phiv}{\hbox{\boldmath$\phi$}}
\newcommand{\psiv}{\hbox{\boldmath$\psi$}}
\newcommand{\thetav}{\hbox{\boldmath$\theta$}}
\newcommand{\tauv}{\hbox{\boldmath$\tau$}}
\newcommand{\omegav}{\hbox{\boldmath$\omega$}}
\newcommand{\xiv}{\hbox{\boldmath$\xi$}}
\newcommand{\sigmav}{\hbox{\boldmath$\sigma$}}
\newcommand{\piv}{\hbox{\boldmath$\pi$}}
\newcommand{\rhov}{\hbox{\boldmath$\rho$}}
\newcommand{\upsilonv}{\hbox{\boldmath$\upsilon$}}

\newcommand{\Gammam}{\hbox{\boldmath$\Gamma$}}
\newcommand{\Lambdam}{\hbox{\boldmath$\Lambda$}}
\newcommand{\Deltam}{\hbox{\boldmath$\Delta$}}
\newcommand{\Sigmam}{\hbox{\boldmath$\Sigma$}}
\newcommand{\Phim}{\hbox{\boldmath$\Phi$}}
\newcommand{\Pim}{\hbox{\boldmath$\Pi$}}
\newcommand{\Psim}{\hbox{\boldmath$\Psi$}}
\newcommand{\Thetam}{\hbox{\boldmath$\Theta$}}
\newcommand{\Omegam}{\hbox{\boldmath$\Omega$}}
\newcommand{\Xim}{\hbox{\boldmath$\Xi$}}


% Sans Serif small case

\newcommand{\Gsf}{{\sf G}}

\newcommand{\asf}{{\sf a}}
\newcommand{\bsf}{{\sf b}}
\newcommand{\csf}{{\sf c}}
\newcommand{\dsf}{{\sf d}}
\newcommand{\esf}{{\sf e}}
\newcommand{\fsf}{{\sf f}}
\newcommand{\gsf}{{\sf g}}
\newcommand{\hsf}{{\sf h}}
\newcommand{\isf}{{\sf i}}
\newcommand{\jsf}{{\sf j}}
\newcommand{\ksf}{{\sf k}}
\newcommand{\lsf}{{\sf l}}
\newcommand{\msf}{{\sf m}}
\newcommand{\nsf}{{\sf n}}
\newcommand{\osf}{{\sf o}}
\newcommand{\psf}{{\sf p}}
\newcommand{\qsf}{{\sf q}}
\newcommand{\rsf}{{\sf r}}
\newcommand{\ssf}{{\sf s}}
\newcommand{\tsf}{{\sf t}}
\newcommand{\usf}{{\sf u}}
\newcommand{\wsf}{{\sf w}}
\newcommand{\vsf}{{\sf v}}
\newcommand{\xsf}{{\sf x}}
\newcommand{\ysf}{{\sf y}}
\newcommand{\zsf}{{\sf z}}


% mixed symbols

\newcommand{\sinc}{{\hbox{sinc}}}
\newcommand{\diag}{{\hbox{diag}}}
\renewcommand{\det}{{\hbox{det}}}
\newcommand{\trace}{{\hbox{tr}}}
\newcommand{\sign}{{\hbox{sign}}}
\renewcommand{\arg}{{\hbox{arg}}}
\newcommand{\var}{{\hbox{var}}}
\newcommand{\cov}{{\hbox{cov}}}
\newcommand{\Ei}{{\rm E}_{\rm i}}
\renewcommand{\Re}{{\rm Re}}
\renewcommand{\Im}{{\rm Im}}
\newcommand{\eqdef}{\stackrel{\Delta}{=}}
\newcommand{\defines}{{\,\,\stackrel{\scriptscriptstyle \bigtriangleup}{=}\,\,}}
\newcommand{\<}{\left\langle}
\renewcommand{\>}{\right\rangle}
\newcommand{\herm}{{\sf H}}
\newcommand{\trasp}{{\sf T}}
\newcommand{\transp}{{\sf T}}
\renewcommand{\vec}{{\rm vec}}
\newcommand{\Psf}{{\sf P}}
\newcommand{\SINR}{{\sf SINR}}
\newcommand{\SNR}{{\sf SNR}}
\newcommand{\MMSE}{{\sf MMSE}}
\newcommand{\REF}{{\RED [REF]}}

% Markov chain
\usepackage{stmaryrd} % for \mkv 
\newcommand{\mkv}{-\!\!\!\!\minuso\!\!\!\!-}

% Colors

\newcommand{\RED}{\color[rgb]{1.00,0.10,0.10}}
\newcommand{\BLUE}{\color[rgb]{0,0,0.90}}
\newcommand{\GREEN}{\color[rgb]{0,0.80,0.20}}

%%%%%%%%%%%%%%%%%%%%%%%%%%%%%%%%%%%%%%%%%%
\usepackage{hyperref}
\hypersetup{
    bookmarks=true,         % show bookmarks bar?
    unicode=false,          % non-Latin characters in AcrobatÕs bookmarks
    pdftoolbar=true,        % show AcrobatÕs toolbar?
    pdfmenubar=true,        % show AcrobatÕs menu?
    pdffitwindow=false,     % window fit to page when opened
    pdfstartview={FitH},    % fits the width of the page to the window
%    pdftitle={My title},    % title
%    pdfauthor={Author},     % author
%    pdfsubject={Subject},   % subject of the document
%    pdfcreator={Creator},   % creator of the document
%    pdfproducer={Producer}, % producer of the document
%    pdfkeywords={keyword1} {key2} {key3}, % list of keywords
    pdfnewwindow=true,      % links in new window
    colorlinks=true,       % false: boxed links; true: colored links
    linkcolor=red,          % color of internal links (change box color with linkbordercolor)
    citecolor=green,        % color of links to bibliography
    filecolor=blue,      % color of file links
    urlcolor=blue           % color of external links
}
%%%%%%%%%%%%%%%%%%%%%%%%%%%%%%%%%%%%%%%%%%%




\begin{document}

% \title{Interactive Data Harmonization with LLM Agents [Vision]}
\title{Interactive Data Harmonization with LLM Agents}

\author{Aécio Santos}
\affiliation{%
	\institution{New York University}
	\city{}
	\state{}
}
\email{aecio.santos@nyu.edu}

\author{Eduardo H. M. Pena}
\affiliation{%
	\institution{Federal University of Technology - Paran\'{a}}
	\city{}
	\state{}
}
\email{eduardopena@utfpr.edu.br}
\authornote{Work done as a visiting researcher at New York University.}


\author{Roque Lopez}
\affiliation{%
	\institution{New York University}
	\city{}
	\state{}
}
\email{rlopez@nyu.edu}

\author{Juliana Freire}
\affiliation{%
	\institution{New York University}
	\city{}
	\state{}
}
\email{juliana.freire@nyu.edu}





\newcommand{\systemname}{\texttt{Harmonia}\xspace}


\begin{abstract}
% Data harmonization is essential for integrating datasets from diverse sources, yet it remains a time-consuming and challenging task due to schema mismatches, varying terminologies, and differences in data collection methodologies.
% This paper outlines our vision for agentic data harmonization systems and introduces \systemname, a novel system that embodies this vision. 
% To automate the synthesis of data harmonization pipelines, our approach combines LLM-based reasoning with a library of established and efficient data integration algorithms that address common data integration challenges. Together with domain experts, these components enable the interactive development of data harmonization pipelines that can be later reused to recreate harmonized datasets.
% We demonstrate \systemname's application in a real-world scenario involving clinical data integration and mapping datasets to standardized vocabularies. % like the Genomic Data Commons (GDC). 
% Finally, we outline open problems, including the need for expert-in-the-loop workflows, the variability of LLM outputs, and reproducibility concerns.
Data harmonization is an essential task that entails integrating datasets from diverse sources. Despite years of research in this area, it remains a time-consuming and challenging task due to schema mismatches, varying terminologies, and differences in data collection methodologies. This paper presents the case for agentic data harmonization as a means to both empower experts to harmonize their data and to streamline the process. We introduce \systemname, a system that combines LLM-based reasoning, an interactive user interface, and a library of data harmonization primitives to automate the synthesis of data harmonization pipelines. We demonstrate \systemname in a clinical data harmonization scenario, where it helps to interactively create reusable pipelines that map datasets to a standard format.
Finally, we discuss challenges and open problems, and suggest research directions for advancing 
our vision. 
% \vspace{-0.5em}
\end{abstract}

\maketitle

% \vspace{-.25em}
% %%% do not modify the following VLDB block %%
% %%% VLDB block start %%%
% \pagestyle{\vldbpagestyle}
% \begingroup\small\noindent\raggedright\textbf{PVLDB Reference Format:}\\
% \vldbauthors. \vldbtitle. PVLDB, \vldbvolume(\vldbissue): \vldbpages, \vldbyear.\\
% \href{https://doi.org/\vldbdoi}{doi:\vldbdoi}
% \endgroup
% \begingroup
% \renewcommand\thefootnote{}\footnote{\noindent
% This work is licensed under the Creative Commons BY-NC-ND 4.0 International License. Visit \url{https://creativecommons.org/licenses/by-nc-nd/4.0/} to view a copy of this license. For any use beyond those covered by this license, obtain permission by emailing \href{mailto:info@vldb.org}{info@vldb.org}. Copyright is held by the owner/author(s). Publication rights licensed to the VLDB Endowment. \\
% \raggedright Proceedings of the VLDB Endowment, Vol. \vldbvolume, No. \vldbissue\ %
% ISSN 2150-8097. \\
% \href{https://doi.org/\vldbdoi}{doi:\vldbdoi} \\
% }\addtocounter{footnote}{-1}\endgroup
% %%% VLDB block end %%%

% %%% do not modify the following VLDB block %%
% %%% VLDB block start %%%
% \ifdefempty{\vldbavailabilityurl}{}{
% \vspace{.15cm} %0.3cm
% \begingroup\small\noindent\raggedright\textbf{PVLDB Artifact Availability:}\\
% The source code, data, and/or other artifacts have been made available at \url{\vldbavailabilityurl}.
% \endgroup
% }
% %%% VLDB block end %%%

% \vspace{-.25em}
\vspace{2.0em}
%---------------------

\section{Introduction}
Backdoor attacks pose a concealed yet profound security risk to machine learning (ML) models, for which the adversaries can inject a stealth backdoor into the model during training, enabling them to illicitly control the model's output upon encountering predefined inputs. These attacks can even occur without the knowledge of developers or end-users, thereby undermining the trust in ML systems. As ML becomes more deeply embedded in critical sectors like finance, healthcare, and autonomous driving \citep{he2016deep, liu2020computing, tournier2019mrtrix3, adjabi2020past}, the potential damage from backdoor attacks grows, underscoring the emergency for developing robust defense mechanisms against backdoor attacks.

To address the threat of backdoor attacks, researchers have developed a variety of strategies \cite{liu2018fine,wu2021adversarial,wang2019neural,zeng2022adversarial,zhu2023neural,Zhu_2023_ICCV, wei2024shared,wei2024d3}, aimed at purifying backdoors within victim models. These methods are designed to integrate with current deployment workflows seamlessly and have demonstrated significant success in mitigating the effects of backdoor triggers \cite{wubackdoorbench, wu2023defenses, wu2024backdoorbench,dunnett2024countering}.  However, most state-of-the-art (SOTA) backdoor purification methods operate under the assumption that a small clean dataset, often referred to as \textbf{auxiliary dataset}, is available for purification. Such an assumption poses practical challenges, especially in scenarios where data is scarce. To tackle this challenge, efforts have been made to reduce the size of the required auxiliary dataset~\cite{chai2022oneshot,li2023reconstructive, Zhu_2023_ICCV} and even explore dataset-free purification techniques~\cite{zheng2022data,hong2023revisiting,lin2024fusing}. Although these approaches offer some improvements, recent evaluations \cite{dunnett2024countering, wu2024backdoorbench} continue to highlight the importance of sufficient auxiliary data for achieving robust defenses against backdoor attacks.

While significant progress has been made in reducing the size of auxiliary datasets, an equally critical yet underexplored question remains: \emph{how does the nature of the auxiliary dataset affect purification effectiveness?} In  real-world  applications, auxiliary datasets can vary widely, encompassing in-distribution data, synthetic data, or external data from different sources. Understanding how each type of auxiliary dataset influences the purification effectiveness is vital for selecting or constructing the most suitable auxiliary dataset and the corresponding technique. For instance, when multiple datasets are available, understanding how different datasets contribute to purification can guide defenders in selecting or crafting the most appropriate dataset. Conversely, when only limited auxiliary data is accessible, knowing which purification technique works best under those constraints is critical. Therefore, there is an urgent need for a thorough investigation into the impact of auxiliary datasets on purification effectiveness to guide defenders in  enhancing the security of ML systems. 

In this paper, we systematically investigate the critical role of auxiliary datasets in backdoor purification, aiming to bridge the gap between idealized and practical purification scenarios.  Specifically, we first construct a diverse set of auxiliary datasets to emulate real-world conditions, as summarized in Table~\ref{overall}. These datasets include in-distribution data, synthetic data, and external data from other sources. Through an evaluation of SOTA backdoor purification methods across these datasets, we uncover several critical insights: \textbf{1)} In-distribution datasets, particularly those carefully filtered from the original training data of the victim model, effectively preserve the model’s utility for its intended tasks but may fall short in eliminating backdoors. \textbf{2)} Incorporating OOD datasets can help the model forget backdoors but also bring the risk of forgetting critical learned knowledge, significantly degrading its overall performance. Building on these findings, we propose Guided Input Calibration (GIC), a novel technique that enhances backdoor purification by adaptively transforming auxiliary data to better align with the victim model’s learned representations. By leveraging the victim model itself to guide this transformation, GIC optimizes the purification process, striking a balance between preserving model utility and mitigating backdoor threats. Extensive experiments demonstrate that GIC significantly improves the effectiveness of backdoor purification across diverse auxiliary datasets, providing a practical and robust defense solution.

Our main contributions are threefold:
\textbf{1) Impact analysis of auxiliary datasets:} We take the \textbf{first step}  in systematically investigating how different types of auxiliary datasets influence backdoor purification effectiveness. Our findings provide novel insights and serve as a foundation for future research on optimizing dataset selection and construction for enhanced backdoor defense.
%
\textbf{2) Compilation and evaluation of diverse auxiliary datasets:}  We have compiled and rigorously evaluated a diverse set of auxiliary datasets using SOTA purification methods, making our datasets and code publicly available to facilitate and support future research on practical backdoor defense strategies.
%
\textbf{3) Introduction of GIC:} We introduce GIC, the \textbf{first} dedicated solution designed to align auxiliary datasets with the model’s learned representations, significantly enhancing backdoor mitigation across various dataset types. Our approach sets a new benchmark for practical and effective backdoor defense.




\section{Preliminaries}
\label{sec:preliminaries}

\subsection{Supervised Binary Classification}
In many real-world tasks, one commonly encounters binary classification problems, in which an input $x \in \mathbb{R}^d$ is presented, and its label $y \in \{\pm 1\}$ needs to be predicted. Each sample is assumed to be independently and identically drawn from an unknown joint distribution $p(x,y)$. Let $\pi_{+} = p(y=+1)$ be the prior probability of the positive class (positive prior), and define
\begin{align*}
p_{\mathrm{p}}(x) = p(x \mid y=+1)
\text{, }
p_{\mathrm{n}}(x) = p(x \mid y=-1).
\end{align*}
Then, the marginal distribution of $x$ is given by
\begin{align*}
p(x)
= \pi_{+}p_{\mathrm{p}}(x) + (1-\pi_{+})p_{\mathrm{n}}(x).
\end{align*}

A classifier $g: \mathbb{R}^d \to \mathbb{R}$ outputs a real-valued score, whose sign determines the predicted label. For instance, a neural network can serve as $g$. A loss function $\ell:\mathbb{R} \times \{\pm 1\} \to [0,\infty)$ then measures how much the prediction disagrees with the true label. Let $R^+_p(g) = \mathbb{E}_{x \sim p_p} [\ell(g(x), +1)]$ denote the loss for true positive data, and $R^-_n(g) = \mathbb{E}_{x \sim p_n} [\ell(g(x), -1)]$ denote the loss for the true negative data. Then, the true risk is expressed as
\begin{align}
    R_{\mathrm{pn}}(g) =& \mathbb{E}_{(x,y)\sim p}[\ell(g(x),y)] \notag \\
    =& \pi_{+}R^+_p + (1 - \pi_{+})R^-_n \label{eq:risk} 
\end{align}

In supervised learning, positive dataset $\mathcal{C}_p = \{x^p_m\}_{m=1}^{m_p} \sim p_p(x)$ and negative dataset $\mathcal{C}_n = \{ x^n_m \}_{m=1}^{m_n} \sim p_n(x)$ are accessible. Replacing the expectations in \eqref{eq:risk} with sample mean, one obtains the empirical risk, and $g$ is trained to minimize it.


It is well known that having sufficient positive and negative samples typically allows one to train a highly accurate classifier for many tasks. However, in practice, obtaining large-scale positive and negative datasets with annotations is often challenging, especially in specialized domains where annotation costs become a significant obstacle.

\subsection{Unlabeled-Unlabeled (UU) Learning}
\label{subsec:uu}
UU learning~\citep{Lu2019-sd} is a technique that allows training a classifier without fully labeled positive and negative datasets, leveraging two unlabeled datasets with different class priors.

Concretely, suppose unlabeled corpora, $\widetilde{\mathcal{C}}_p = \{\widetilde{x}^p_m\}_{m=1}^{m_p}$ and $\widetilde{\mathcal{C}}_n = \{\widetilde{x}^n_m\}_{m=1}^{m_n}$, drawn from different mixture distributions. We denote $\theta_p = p(y=+1 \mid \widetilde{x}\in \widetilde{\mathcal{C}}_p)$ and $\theta_n = p(y=+1 \mid \widetilde{x}\in \widetilde{\mathcal{C}}_n)$ the \emph{positive prior} of these unlabeled corpora. In other words, $\theta_p$ is the fraction of true positives in $\widetilde{\mathcal{C}}_p$, and $\theta_n$ is the fraction of true positives in $\widetilde{\mathcal{C}}_n$. Then, the mixture distribution of each corpus is given as
\begin{align*}
    \widetilde{p}_{p}(x) &= \theta_p\, p_{p}(x) \;+\; \bigl(1 - \theta_p\bigr)\, p_{n}(x) \\
    \widetilde{p}_{_n}(x) &= \theta_n\, p_{p}(x) \;+\; \bigl(1 - \theta_n\bigr)\, p_{n}(x).        
\end{align*}

When $\theta_p > \theta_n$, we can treat $\widetilde{\mathcal{C}}_p$ as a pseudo-positive corpus (due to its larger proportion of actual positives) and $\widetilde{\mathcal{C}}_n$ as a pseudo-negative corpus (having a smaller proportion of actual positives). 

By appropriately combining these two unlabeled sets, one can construct an unbiased estimate of the true binary classification risk~\eqref{eq:risk}. Specifically, let $R_{\tilde{p}}^{\pm}(g)=\mathbb{E}_{x\sim \widetilde{p}_p}[\ell(g(x),\pm 1)]$, and $R_{\tilde{n}}^{\pm} (g)=\mathbb{E}_{x\sim \widetilde{p}_n}[\ell(g(x),\pm 1)]$. Then, the UU learning risk is given by
\begin{align}
    &R_{\mathrm{uu}}(g)\label{eq:uu}
    \\
    &\hspace{1em}= a R_{\tilde{p}}^+(g) - b R_{\tilde{p}}^-(g) - c R_{\tilde{n}}^+(g) + d R_{\tilde{n}}^-(g),\notag
\end{align}
where the coefficients $a$, $b$, $c$, $d$ are computed from $\pi_+$, $\theta_p$, and $\theta_n$ as $a = \frac{(1-\theta_n)\,\pi_+}{\theta_p - \theta_n}$, $b = \frac{\theta_n\,(1-\pi_+)}{\theta_p - \theta_n}$, $c = \frac{(1-\theta_p)\,\pi_+}{\theta_p - \theta_n}$, $d = \frac{\theta_p\,(1-\pi_+)}{\theta_p - \theta_n}$. When $\theta_p = 1$ and $\theta_n = 0$, that is, when using the same dataset as standard supervised learning, equation~\eqref{eq:uu} reduces to the standard supervised learning risk equation~\eqref{eq:risk}. In other words, supervised learning can be considered a special case of UU learning.

\subsection{Robust UU Learning}
\label{subsec:ruu}
While UU learning \eqref{eq:uu} does allow model training without explicit positive/negative labels, comparing the original binary classification risk \eqref{eq:risk}—which remains nonnegative—against the UU risk \eqref{eq:uu} shows the UU risk includes negative terms such as $-b R_{\tilde{p}}^-(g)$ and $-c R_{\tilde{n}}^+(g)$. It has been observed that these negative risk terms can lead to overfitting~\citep{Lu2020-dx}.

To mitigate this, \emph{Robust UU Learning} introduces a generalized Leaky ReLU function $f$ to moderate the reduction of negative risk. Concretely, it normalizes each term of the loss function as~\citep{Lu2020-dx}
\begin{align}
    R_{\mathrm{ruu}}(g)
    &= f\left(a R_{\tilde{p}}^+(g) - c R_{\tilde{n}}^+(g) \right) \notag \\
    &\hspace{2em}+ f\left(d R_{\tilde{n}}^-(g) - b R_{\tilde{p}}^-(g)\right) \label{eq:ruu}
\end{align}
where each bracketed term resembles a “normalized” risk under the hypothetical label of being positive or negative, respectively. The function $f$ is given by
\begin{align}
    f(x) =
    \begin{cases}
    x & \text{if } x > 0 \\
    \lambda x & \text{if } x < 0
    \end{cases}
    \quad (\lambda < 0).
    \label{eq:relu}
\end{align}

Intuitively, $f$ leaves the risk value unchanged when the risk is positive, but for negative risk, it uses $\lambda < 0$ to convert it into a positive quantity, thus mitigating the overfitting by negative risk.


% \vspace{-0.5em}
%-------------------------------------
\section{Agentic Data Harmonization}
\label{sec:agentic-data-harmonization}
%-------------------------------------

\section{Additional Figures}

In this section, we present some additional figures that demonstrate the negative effects of random edge-dropping, particularly focusing on providing empirical evidence for scenarios not covered by the theory in \autoref{sec:theory}. Additionally, we provide visual evidence of the negative effects of DropNode, despite the fact that it preserves sensitivity between nodes (in expectation).

\subsection{Symmetrically Normalized Propagation Matrix}
\label{sec:fig-sym-norm}

\begin{figure}
    \centering
    \includegraphics[width=\linewidth]{assets/linear-gcn_symmetric_Proteins.png}
    % \includegraphics[width=0.48\linewidth]{assets/linear-gcn_symmetric_MUTAG.png}
    \caption{Entries of $\ddot{\transition}^6$, averaged after binning node-pairs by their shortest distance.}
    \label{fig:linear-gcn_symmetric}
\end{figure}

The results in \autoref{sec:linear} correspond to the use of $\hat{\adjacency} = \propagation^\asym$ for aggregating messages -- in each message passing step, only the in-degree of node $i$ is used to compute the aggregation weights of the incoming messages. In practice, however, it is more common to use the symmetrically normalized propagation matrix, $\propagation = \propagation^\sym$, which ensures that nodes with high degree do not dominate the information flow in the graph \cite{kipf2017gcn}. As in \autoref{thm:sensitivity-l-layer}, we are looking for 
$$
    \expectation_{\mask\layer{1}, \ldots, \mask\layer{L}} \sb{\prod_{l=1}^L \propagation\layer{\ell}} = \ddot{\transition}^L
$$
where $\ddot{\transition} \coloneqq \expectation_{\mathsf{DE}} [\propagation^{\sym}]$. While $\ddot{\transition}$ is analytically intractable, we can approximate it using Monte-Carlo sampling. Accordingly, we use 20 samples of $\mask$ to compute an approximation of $\ddot{\transition}$, and plot out the entries of $\ddot{\transition}^L$, as we did for $\dot{\transition}^L$ in \autoref{fig:linear-gcn_asymmetric}. The results are presented in \autoref{fig:linear-gcn_symmetric}, which shows that while the sensitivity between nearby nodes is affected to a lesser extent compared to those observed in \autoref{fig:linear-gcn_asymmetric}, that between far-off nodes is significantly reduced, same as earlier.

\subsection{Upper Bound on Expected Sensitivity}

\begin{figure}
    \centering
    \includegraphics[width=\linewidth]{assets/linear-gcn_black-extension_Proteins.png}
    % \includegraphics[width=0.48\linewidth]{assets/linear-gcn_black-extension_MUTAG.png}
    \caption{Entries of $\sum_{l=0}^6 \dot{\transition}^l$, averaged after binning node-pairs by their shortest distance.}
    \label{fig:black_extension}
\end{figure}

\citet{black2023resistance} showed that the sensitivity between any two nodes in a graph can be bounded using the sum of the powers of the propagation matrix. In \autoref{sec:nonlinear}, we extended this bound to random edge-dropping methods with independent edge masks smapled in each layer:

\begin{align*}
    \expectation_{\mask\layer{1}, \ldots, \mask\layer{L}} \sb{\norm{\frac{\partial \representation\layer{L}_i}{\partial \feature_j}}_1}
    =
    \zeta_3^{\rb{L}} \rb{\sum_{\ell=0}^L \expectation \sb{\propagation}^{\ell}}_{ij}
\end{align*}

Although this bound does not have a closed form, we can use real-world graphs to study its entries. We randomly sample 100 molecular graphs from the Proteins dataset \cite{dobson2003proteins} and plot the entries of $\sum_{l=0}^6 \dot{\transition}^\ell$ (corresponding to \inline{DropEdge}) against the shortest distance between node-pairs. The results are presented in \autoref{fig:black_extension}. We observe a polynomial decline in sensitivity as the \inline{DropEdge} probability increases, suggesting that it is unsuitable for capturing LRIs.

\subsection{Test Accuracy versus DropNode Probability}

\begin{figure}[t]
    \centering
    \subcaptionbox{Homophilic datasets}{\includegraphics[width=0.7\linewidth]{assets/DropNode_homophilic.png}} \\
    \vspace{2mm}
    \subcaptionbox{Heterophilic datasets}{\includegraphics[width=\linewidth]{assets/DropNode_heterophilic.png}}
    \caption{Dropping probability versus test accuracy of DropNode-GCN.}
    \label{fig:dropnode}
\end{figure}

In \autoref{eqn:dropnode}, we noted that the expectation of sensitivity remains unchanged when using DropNode. However, these results were only in expectation. In practice, a high DropNode probability will result in poor communication between distant nodes, preventing the model from learning to effectively model LRIs. This is supported by the results in \autoref{tab:correlation}, where we observed a negative correlation between the test accuracy and DropNode probability. Moreover, DropNode was the only algorithm which recorded negative correlations on homophilic datasets. In \autoref{fig:dropnode}, we visualize these relationships, noting the stark contrast with \autoref{fig:acc-trends}, particularly in the trends with homophilic datasets.

% LLM-based agents integrate the capabilities of language models with reasoning, planning, memory retention, and interaction with external tools~\cite{qin2024toolllm,code-generatingLLMCHI23}.
We propose using LLM-based agents to facilitate the interactive construction of harmonization pipelines through natural language and visual interfaces. We aim to simplify the harmonization process and empower domain experts to harmonize their own data~effectively.

Our approach has three main aspects -- \textit{harmonization primitives}, \textit{harmonization agents}, and \textit{human-agent interaction}, as illustrated in Figure~\ref{fig:system-diagram}. 
The system should support two-way interaction between the \textit{users} and the \textit{harmonization agents}. While the former drives the system and defines the tasks to be performed, the latter aims to automate most data harmonization tasks, leaving only key decisions that need external context to users.
By capturing user-agent interactions and the derived computational pipelines, we maintain the provenance of the harmonization process. This supports transparency and reproducibility, making it possible to publish the harmonized data with the pipeline used to derive it.

\myparagraph{Harmonization Primitives}
The bottom of Figure~\ref{fig:system-diagram} illustrates a library of components that we refer to as \textit{data integration primitives}. These are key algorithms or routines that can be developed to solve well-defined data integration tasks such as schema matching, value matching, % function-attribute matching,
and others. Some of these components are lower-level routines that support other higher-level tasks, e.g., column-type annotation may be a component of schema-matching algorithms~\cite{feuer:vldb2024}, and entity resolution and deduplication may be key to performing data standardization~\cite{christophides2020overview}. While the particular set of primitives can be heterogeneous and evolve to support the system capabilities, they have in common that they need to be \textit{composable} and can be invoked by both user and AI agents.

\textit{Primitive composability} refers to the ability to combine primitives to create a pipeline: the output of a primitive $p_1$ can be used as input of another primitive $p_2$. For example, \textit{schema matching} primitives can output a list of source-target attribute pairs.
These lists form the input for the \textit{value matching} primitives, which then find equivalences between the values of the source attribute to the target. Finally, the output of value-matching primitives can be used to assemble a \textit{harmonization specification} that describes the transformation of source tables $T_i$ into a target output table $T_{target}$.
% 
In short, primitive composability enables the creation of data harmonization pipelines by allowing the chaining of operations that take source tables as input and produce harmonized data as output.

\myparagraph{Data Harmonization Agents}
The harmonization agent fulfills user requests by synthesizing pipelines that perform the requested harmonization task. 
First, it decomposes the problem in a sequence of actions. These actions can be of multiple types, including execution of existing integration primitives (tool calling) or execution of code generated on-demand. 
To decide on what action to take, the harmonization agent leverages an LLM, which is provided with descriptions of the task and integration primitives available for use. 
The LLM then returns an action (e.g., a code snippet) that can be executed in a runtime environment (e.g., a Python kernel). The action output is then fed back into the LLM, which decides if it needs to execute additional actions or if the task is completed.
This loop executes until the task is deemed complete (see Figure~\ref{fig:agent-loop}).
% As illustrated in Figure~\ref{fig:agent-loop}, this loop executes until the task is deemed complete 

The main loop is orchestrated by a driver code that takes inputs from the user (i.e., prompts) that describe the task to be performed. 
This driver is also responsible for (1) communicating with users to request inputs (e.g., when the LLM asks for task clarification or user preferences) and (2) managing the state (memory) of the agent. For example, it can also track and store the history of actions and user interactions in a \textit{Provenance DB}.
This data can support decisions about future actions, and transparency, and be used to generate harmonization \textit{specifications} or scripts to reproduce the results.

Given the complexity of harmonization tasks, it is crucial to have high-level primitives available as building blocks for the pipeline.
This allows encoding prior knowledge and using efficient algorithms known to be effective for a specific task.
These primitives encompass algorithms for tasks such as schema matching, entity resolution, and value mapping. Of course, primitives can go beyond hard-coded functions that implement deterministic algorithms. For instance, they can be workflows that use LLMs to perform specific tasks such as in \cite{liu2024magneto} or they could generate code on demand (e.g., to extract data from or to transform attribute values~\cite{autoformula2024}). 


\myparagraph{Human-Agent Interaction}
A central component of our system architecture is the user-agent interaction. We argue that systems must go beyond text-based conversational interactions: they need to support rich visual data representations and should allow interactions that help users reason about the answers produced by the agent and refine the task definition as well as the pipelines derived by the agent. In complex tasks such as harmonization, this is necessary since many decisions needed to complete the task (e.g., whether two terms represent the same concept) are difficult even for domain experts and may require external knowledge, such as the context in which data was collected.

In this paper, we focus on interactions based on natural language via a chatbot interface. However, it should be possible to implement even more intuitive and efficient graphical interfaces that combine natural language with point-and-click components. For example, as done in interactive AutoML tools~\cite{santos2019visus}, the system may guide the user through the harmonization process, recommend the available actions, track progress, and provide data visualizations to help the user better make sense of the data. This may help prevent common issues in natural language such as ambiguity~\cite{esfandiarpoor2024followup}.

\vspace{-.75em}
\section{System Prototype \& Use Case} \label{sec:prototype}

\myparagraph{The primitives library}
% We used the data harmonization primitives provided by \texttt{bdi-kit} \cite{bdi-kit-github}, an open-source Python library. 
We used data harmonization primitives from \texttt{bdi-kit} \cite{bdi-kit-github}, an open-source Python library that we designed with the explicit goal of composability.
%We implemented the data harmonization primitives library in Python and released it as the Python library that we call \texttt{bdi-kit} \cite{bdi-kit-github}. 
Currently, it includes implementations of multiple \textit{schema matching} and \textit{value matching} algorithms using a composable API. It includes several classic algorithms from Koutras et al.~\cite{koutras2021valentine} and language model-based algorithms such as the ones presented in~\cite{liu2024enhancing} and \cite{liu2024magneto}. Most functions take as input a \texttt{source} parameter that represents the user's input DataFrame and returns the output formatted as another DataFrame. The \texttt{target} parameter can either be a string representing a target standard schema (e.g., \texttt{`gdc'}) or a target DataFrame, this allows switching between the two tasks described in Examples~\ref{example1} and ~\ref{example2}. Figure~\ref{fig:bdi-kit} shows a subset of functions relevant to this paper.

\begin{figure}[t]
% \vspace{-1.25em}
\centering
\begin{tcolorbox}[colback=black!2.5!white,colframe=black!85!black,boxrule=0.25mm,boxsep=4pt,left=0pt,right=0pt,top=0pt,bottom=0pt]

    \footnotesize
    
    \texttt{\textbf{match\_schema}(source, target, method, ...)} \\
    Maps the schema of a source table to a target schema (table or predefined standard like \texttt{gdc}) using a specified method.
    \vspace{.5em}
    
    
    \texttt{\textbf{top\_matches}(source, column, target, top\_k, method, ...)} \\
    Finds the top-\texttt{k} matches between a source column and columns of a target schema.
    \vspace{.5em}
    
    
    \texttt{\textbf{match\_values}(source, target, column\_mapping, method, ...)} \\
    Matches values between columns of a source and a target using a specified method, returning one or more result tables.
    \vspace{.5em}
    
    
    % \texttt{\textbf{top\_value\_matches}(source, target, column\_mapping, top\_k,...)} \\ %  method, 
    % Finds the top-\texttt{k} value matches between columns in a source and target.
    % \vspace{.5em}
    
    
    % \texttt{\textbf{view\_value\_matches}(matches, edit = False)} \\
    % Displays value match results in a table format, with optional editing.
    % \vspace{.5em}

    
    % \texttt{\textbf{preview\_domain}(dataset, column, limit = None) → DataFrame} \\
    % Previews unique values and descriptions in a specified column of a dataset.
    % \vspace{.5em}

    
    % \texttt{\textbf{merge\_mappings}(mappings, user\_mappings = None) → List} \\
    % Combines computed and user-provided mappings into a plan for data transformation.
    % \vspace{.5em}

    
    \texttt{\textbf{materialize\_mapping}(input\_table, mapping\_spec) → DataFrame} \\
    Transforms a source table into a new table using a mapping specification.
    % 
    % \texttt{create\_mapper(input) → ValueMapper} \\
    % Creates a mapper object for transforming column values based on the input type. \\
    
    % \texttt{MappingSpecLike} \\
    % Defines mappings between source and target columns, including optional value transformations. \\
\end{tcolorbox}
% \vspace{-1.5em}
\caption{A subset of functions available in the \texttt{bdi-kit} API.}
% \vspace{-1.5em}
\label{fig:bdi-kit}
\end{figure}

\myparagraph{The data harmonization agent} We implemented \systemname, a system prototype that uses the data integration primitives from Figure~\ref{fig:bdi-kit} and interacts with users via a text-based chat box. Implementing an agent entails writing carefully crafted function and task descriptions that are combined to assemble system prompts fed to the LLMs. For our prototype, we implemented tool wrappers for each of the \texttt{bdi-kit} functions, along with descriptions of when each should be used. We also provide general descriptions of the data harmonization steps, when the LLM should request help from the user, and output formatting instructions.

To implement the tool calling functionality and the user interface, we used the Archytas~\cite{archytas} library, an open-source tool for building AI agents based on the ReAct framework~\cite{yao2022react}, and Beaker~\cite{beaker}, a contextually-aware notebook system that supports the development of chatbot user interfaces. This allows us to implement features that facilitate interaction, such as rendering markdown-formatted outputs in the browser, displaying the reasoning steps taken by the agent, collecting user inputs, and feeding them back to the agent.

%\myparagraph{Use case: harmonization with the GDC standard} % previous title
\myparagraph{Human-agent interaction: The GDC use case}
To demonstrate the ability to produce good data harmonization plans, we present a use case that harmonizes a dataset from Dou et al.~\cite{dou2020proteogenomic} (described in Example~\ref{example1}) with the GDC standard. 
% Below we provide an example of interaction with the chatbot agent.
Below, we show message exchanges between the user ({\scriptsize \faUser}) and the agent ({\scriptsize \faRobot}), along with explanations of the actions taken by the agent to answer user queries.

The user starts by requesting the system to load a CSV file that contains the attributes of interest:
% \vbox{
% \vspace{-.5em}
\chatboxUser{
Load the file dou.csv as a dataframe and subset it to the following columns: Country, Histologic\_Grade\_FIGO, Histologic\_type, FIGO\_stage, BMI, Age, Race, Ethnicity, Gender, Tumor\_Focality, Tumor\_Size\_cm.
}
% \vspace{-1em}
\chatboxAgent{
The file 'dou.csv' has been successfully loaded and subsetted to the specified columns: Country, Histologic\_Grade\_FIGO, Histologic\_type, FIGO\_stage, BMI, Age, Race, Ethnicity, Gender, Tumor\_Focality, Tumor\_Size\_cm.\\
Outputs:\\
{\color{gray}[...data frame displayed here...]}
}
\vspace{-.5em}
% }
\noindent Responding to this prompt, the agent automatically (1) generates Python code using the Pandas library that loads and selects the requested columns and (2) executes the generated code to load the CSV file into memory. This example illustrates the capability of \textit{on-demand code generation} to accomplish a requested task.

Next, the user requests the system to perform schema matching between the table loaded and the attributes from the GDC standard:

% \vspace{-.5em}
\chatboxUser{
Please match this to the GDC schema using the `ct\_learning' method, and fix any results that don't look correct.
}
% \vspace{-.5em}

\noindent
As shown in Figure~\ref{fig:sequence-diagram}, the system interacts with an LLM multiple times to answer this request. First, it sends the initial user's prompt to the LLM, which recognizes that the function \texttt{match\_schema} is adequate to solve the request, and thus returns a tool calling instruction containing the \texttt{match\_schema} function name along with the needed parameters. Note that the LLM parses the user's query, and recognizes that \texttt{gdc} must be used as the target schema and \texttt{ct\_learning} refers to the argument \texttt{method} of \texttt{match\_schema()}. 
% 
After receiving the tool calling instruction as output, the system then executes the primitive function requested by the LLM, collects the list of column matches returned, and passes it back to the LLM in a follow-up prompt. 

After analyzing the column match results received in the follow-up prompt, the LLM detects a mapping mistake for the column \texttt{Histologic\_type}, which is wrongly mapped to \texttt{roots}. To fix the error, the LLM triggers a tool calling instruction that uses the primitive function \texttt{top\_matches} to find the top~10 alternative candidates for \texttt{Histologic\_type}. After receiving the results, it analyzes the results and suggests using \texttt{primary\_diagnosis} as a correction to the user (who agrees) as seen in the following messages:

\vbox{
% \vspace{-.5em}
\chatboxAgent{The current mapping for `Histologic\_type' is `roots', which seems incorrect. The top alternatives are:\\
- history\_of\_tumor\_type\\
- primary\_diagnosis\\
- sample\_type\\
I suggest using `primary\_diagnosis'. Do you agree with this choice?
}
% \vspace{-1em}
\chatboxUser{yes}
}
\vspace{-.5em}
\noindent 
As a final response to the task, the agent shows a table of column mappings as output, where the ``\texttt{Source Column}'' contains columns from the source table (\texttt{dou.csv}), ``\texttt{Target Column}'' contains a column match from the GDC standard, and ``\texttt{Corrected?}'' indicates whether the LLM has corrected the original \texttt{bdi-kit} match.

\vspace{-.25em}
\chatboxAgent{
Result:\\

% \scriptsize
\vspace{-0.25em}
\hspace{-0.25cm}
\begin{tabular}{lll}
\textbf{Source Column} & \textbf{Target Column} & \textbf{Corrected?} \\
Country & country\_of\_birth & No \\
Histologic\_Grade\_FIGO & tumor\_grade & No \\
Histologic\_type & primary\_diagnosis & Yes, from ``roots'' \\
FIGO\_stage & figo\_stage & No \\
BMI & bmi & No \\
Age & age\_at\_onset & No \\
Race & race & No \\
Ethnicity & ethnicity & No \\
Gender & gender & No \\
Tumor\_Focality & tumor\_focality & No \\
Tumor\_Size\_cm & tumor\_depth\_measurement & No \\
\end{tabular}
}
% }

Since the user is a domain expert who knows the GDC standard and knows that there may be better options for \texttt{Tumor\_Size\_cm}, it asks the system to find an alternative match. The system then updates the previous list with a more appropriate match found using the \texttt{top\_matches} function. This interaction is shown next:

\vbox{
% \vspace{-.25em}
\chatboxUser{
Find alternative mappings for Tumor\_Size\_cm.
}
% \vspace{-1em}
\chatboxAgent{Result:\\

\scriptsize
\vspace{-0.25em}
\hspace{-0.25cm}
\begin{tabular}{lll}
    \textbf{Source Column} & \textbf{Target Column} & \textbf{Corrected?} \\ 
    {\color{gray}[...]} & {\color{gray}[...]} & {\color{gray}[...]} \\
    % Country & country\_of\_birth & No \\ 
    % Histologic\_Grade\_FIGO & tumor\_grade & No \\ 
    % Histologic\_type & primary\_diagnosis & Yes, from "roots" \\ 
    % FIGO\_stage & figo\_stage & No \\ 
    % BMI & bmi & No \\ 
    % Age & age\_at\_onset & No \\ 
    % Race & race & No \\ 
    % Ethnicity & ethnicity & No \\ 
    % Gender & gender & No \\ 
    Tumor\_Focality & tumor\_focality & No \\ 
    % Tumor\_Size\_cm & tumor\_largest\_dimension\_diameter & Yes, from "tumor\_depth\_measurement" \\ 
    Tumor\_Size\_cm & tumor\_largest\_dimension\_diameter & Yes, from "tumor\_depth... \\ 
\end{tabular}
}
} %\vbox

% --------------
% Value mappings
% --------------

After finding correct matches for all source columns, the next step is to find value mappings for each column match. The following prompt requests the agent to find such mappings:

% \vspace{-0.75em}
% \vbox{
\chatboxUser{\raggedright Find value mappings for the columns Country, Histologic\_Grade\_FIGO, Histologic\_type, FIGO\_stage, Race, Ethnicity, Gender, Tumor\_Focality. If there are any errors in the mappings, please provide suggestions.
}
% \vspace{-0.75em}
\noindent
As in the previous prompts, finding value mappings requires multiple LLM interactions. First, the LLM triggers one function call of \texttt{match\_values()} for each of the source-target column pairs using the default value matching method \texttt{tfidf}  (an efficient algorithm that chooses matches based on the tf-idf similarity over character n-grams). After receiving the value matches, the LLM evaluates if they are correct, and fixes the entries that contain errors. To do so, it calls another function to discover the list of acceptable values in a GDC attribute and picks the correct one. A snippet of the result is shown next (we omit part of the output due to lack of space):
%in the box below:

\vspace{-0.5em}
\chatboxAgent{\scriptsize
{\color{gray}[...]}\\

\vspace{-0.5em}
\textbf{Histologic\_Grade\_FIGO → tumor\_grade}\\

\vspace{-0.5em}
\hspace{-0.25cm}
\begin{tabular}{lll}
    \textbf{Source Value} & \textbf{Target Value} & \textbf{Corrected?} \\ 
    FIGO grade 1 & G1 & Yes, from "Low Grade" \\ 
    FIGO grade 2 & G2 & Yes, from "Low Grade" \\ 
    FIGO grade 3 & G3 & Yes, from "Low Grade" \\ 
\end{tabular}\\

{\color{gray}[...]}\\

\vspace{-0.25em}
\textbf{FIGO\_stage → figo\_stage}\\

\vspace{-0.25em}
\hspace{-0.25cm}
\begin{tabular}{lll}
    \textbf{Source Value} & \textbf{Target Value} & \textbf{Corrected?} \\ 
    IIIC2 & Stage IIIC2 & No \\ 
    IIIC1 & Stage IIIC1 & No \\ 
    IVB & Stage IVB & No \\ 
    IIIB & Stage IIIB & No \\ 
    IIIA & Stage IIIA & No \\ 
    II & Stage II & Yes, from "Stage III" \\ 
\end{tabular}\\

\vspace{-0.25em}
{\color{gray}[...]}
}
% \vspace{-0.5em}
% }

\noindent
This example illustrates the ability of the LLM to detect and correct errors automatically without the user's intervention. As an example, it corrects the match $\langle\texttt{II}, \texttt{Stage III}\rangle$ to $\langle\texttt{II}, \texttt{Stage II}\rangle$. This showcases how we can leverage existing fast algorithms to perform most of the work while leveraging LLMs as an evaluator that can detect and fix mistakes only when needed.

% ------------------
% Materialize schema
% ------------------

The final step is generating a harmonized table. The boxes below show the user prompt as well as the agent response:

\vbox{
% \vspace{-0.25em}
\chatboxUser{Please create a final harmonized table based on the discovered column and value mappings and save it at "dou\_harmonized.csv".}
% \vspace{-1em}
\chatboxAgent{The harmonized table has been successfully created and saved as `dou\_harmonized.csv'.}
% \vspace{-0.25em}
}
\noindent
In this example, the agent uses \texttt{materialize\_mapping} to generate a table and save file \texttt{`dou\_harmonized.csv'}. The LLM  first creates a declarative mapping specification according to the bdi-kit's format used to describe a harmonization plan. Figure~\ref{fig:mapping-spec} shows a snippet of the generated harmonization plan, which includes a declarative specification of the mappings to be performed written in JSON. In this example, we only show dictionary-based transformations that map source values into target values. However, bdi-kit's also supports other types of transformations, such as custom mappings that take as input a custom Python function (or lambda).
This can potentially be used along with the on-demand code generation.

The main advantage of using a declarative language to describe harmonization plans is that it enables reproducibility: once a plan is created, users can feed the plan along with the source data into \texttt{materialize\_mapping} function to recreate the harmonized data. This does not require re-running any LLM-based interactions, since all transformations are encoded in the harmonization plan.

% This is a hack to remove the syntax highlighting error in the minted box below
\AtBeginEnvironment{minted}{\renewcommand{\fcolorbox}[4][]{#4}}
\begin{figure}[h]
\vspace{-.75em}
\begin{tcolorbox}[colback=black!2.5!white,colframe=black!85!black,boxrule=0.25mm,boxsep=4pt,left=0pt,right=0pt,top=0pt,bottom=0pt]
\begin{minted}[fontsize=\scriptsize]{json}
[ 
 [...]
 { "source": "Country",
   "target": "country_of_birth",
   "matches": [ ["United States", "United States"],
                ["Poland", "Poland"],
                ["Ukraine", "Ukraine"] ] },
 { "source": "Histologic_Grade_FIGO",
   "target": "tumor_grade",
   "matches": [ ["FIGO grade 1", "G1"],
                ["FIGO grade 2", "G2"],
                ["FIGO grade 3", "G3"] ] },
 [...]
]
\end{minted}
\end{tcolorbox}
\vspace{-1.25em}
\caption{A snippet from the mapping specification generated that is passed to \texttt{materialize\_mapping()} function.}
\label{fig:mapping-spec}
\end{figure}

\section{Research Opportunities}
\label{sec:research-agenda}

\systemname shows the potential of LLM-based agents to orchestrate actions, evaluate function outputs, detect errors, and generate additional needed functions. However, there are still open research opportunities to expand the system's capabilities.
Below we outline some of the immediate steps towards these opportunities.


% ----------------------------------------------------
\myparagraph{Agent Evaluation \& Benchmarks} 
Most existing evaluation benchmarks are focused on isolated tasks, such as schema matching or entity linking~\cite{koutras2021valentine, liu2024magneto, wang2021machamp}. 
However, agentic systems create a need for end-to-end evaluation benchmarks and metrics to measure progress effectively. Recently, researchers have started developing benchmarks for evaluating agents in various tasks, including data analysis and ML engineering~\cite{chan2024mle, hu2024infiagent, zhang2024benchmarking, huang2024mlagentbench}. 
The data management community should follow this lead and create benchmarks tailored for data integration, ensuring that LLMs improve in these areas.


\myparagraph{Data Integration Primitives}
\label{sec:proposed-primitives}
Some features, such as \emph{uncertainty} quantification and \emph{explanations}, should be exposed by the primitives to guide decision-making~\cite{uncertainty2009}.
Uncertainty in data integration arises from factors like ambiguous schema mappings and data values~\cite{WangHM2018}. \systemname tackles this by exposing similarity scores through its primitives. For example, a value matcher can return similarity scores, so the agent can strategically trigger complementary primitives (e.g., value mapping) for deeper analysis. A key challenge is conveying the meaning of uncertainty measures from diverse primitive mechanisms to both LLMs and end users, and instructing LLMs on how to use and act on them properly~\cite{dagstuhl2029explanation}.

On the same note, LLMs often lack transparency~\cite{pmlr-v235-huang24x}, so primitives must provide \textit{interpretable explanations} to promote user trust. Primitives should offer clear usage documentation and expose their decision rationale. For instance, a matching algorithm description should document whether its similarity scores derive from syntactic similarity, semantic embeddings, or value distribution analysis. LLMs can also explain their decisions based on domain knowledge and primitive instructions, helping users better understand why a particular path was chosen~\cite{explainability2024,TableMeetsLLM2024}. A key opportunity is training agents to discern when to rely on LLM explanations, apply alternative strategies, or engage with the user directly.

Mapping data between schemata involves resolving entities and transforming data~\cite{dataexchange2018}. LLM-based methods have proven helpful in entity resolution~\cite{narayan-vldb2022, fan2024cost}, generating or finding transformations functions~\cite{zdnet-github-copilot, trummer2022codexdb,autotables2023, autoformula2024, dtt2024, SheetAgent2024}, and evaluating LLM performance on such tasks~\cite{ma2024spreadsheetbench}.
However, integrating these methods into agent-based systems requires consistency across diverse data models and alignment with broader agent goals.
Also, we need methods that allow agents to identify and recommend appropriate attribute transformations and suitable functions for a given input dataset. This is especially in challenging cases such as table restructuring or non-standard formats~\cite{autotables2023}.
Another immediate step is to curate a library of transformation functions specialized in data transformation tasks that agentic systems can readily reuse.


\myparagraph{Robustness and Reliability}
LLMs have shown inconsistency across various scenarios (e.g., as text summarization evaluators~\cite{stureborg2024large}), often producing varying results when executed multiple times~\cite{barrie2024prompt}. This variability can undermine reproducibility and reliability, particularly in critical applications where consistent mappings or transformations are crucial. In our experiments, we observed that while the LLM typically identified and fixed incorrect mappings, it occasionally failed to do so (even when provided with the same prompts).

Handling large and complex tables with many attributes poses additional challenges, as these can lead to long chat histories that exceed the LLM context window. When this occurs, the LLM may lose access to earlier relevant information, thereby affecting the robustness. While approaches have been proposed to mitigate context window limitations (e.g., \cite{jin2024llm} \cite{ma2024megalodon}), 
it is not clear if they address the issues in agent systems. 
Equipping agents with access to read and store data in external databases (such as the Provenance DB discussed in Section ~\ref{sec:agentic-data-harmonization}) may be an effective solution to this issue.

\myparagraph{User-Agent Interaction} To improve usability, agentic systems must go beyond natural language (NL) interfaces. While NL is flexible, it is also often ambiguous and may lead to under-specified task descriptions~\cite{zhang2023clarify}. 
Since the same task can be expressed in multiple unpredictable ways, a mismatch between the user task descriptions and agent prompt specifications may occur. Therefore, detecting when clarifications are needed may help increase overall success~\cite{zhang2023clarify}.
These issues could also be potentially addressed by action-oriented UIs that recommend actions linked to predefined prompts. Moreover, using rich visual representations may be more effective at conveying information to the user.

\myparagraph{Provenance-Aware Agents} 
Provenance collection systems have demonstrated promising results in data science pipelines ~\cite{rupprechtVLDB2020,chapmanCapturing2020}.
In data harmonization pipelines, we can track all interactions that contribute to obtaining a specific value mapping. For example, we could record all user-agent and agent-primitive interactions involved in determining the mapping of ``\texttt{FIGO grade 1}'' to ``\texttt{G1}'' (see Figure \ref{fig:mapping-spec}). 
This would allow tracing the lineage of all values in the output data. Moreover, this information could potentially be used to learn user preferences that reduce the need for user interactions.
% Such systems not only enhance transparency but also reduce the need for user interactions, particularly for new practitioners. 
By learning from provenance and pipelines accumulated over time, a system could further streamline the harmonization process by automating all steps and presenting final results directly to users.
% For instance, a provenance system could streamline the harmonization process by bypassing intermediate steps and presenting final results directly to new users.


\myparagraph{Data Harmonization Pipelines} 
In our system, data harmonization is expressed as a pipeline where multiple primitives (predefined, user-defined, or agent-defined) are interconnected through their inputs and outputs to produce the final harmonized dataset. 
% The pipeline can have various objectives, such as creating a dataset with a desired minimum number of rows, returning a sample result in a few seconds, or generating a high-quality dataset based on a chosen quality metric. 
The pipeline can have various objectives, such as maximining the number of correct column matches and value matches, minimizing the number of interactions with users, or minimizing the computational costs (e.g., runtime or LLM calls).
Achieving these objectives represents an optimization problem, requiring the system to navigate a complex search space and balance multiple objectives to determine an optimal sequence of operations.

Recent work has explored large search spaces for building end-to-end pipelines for various tasks~\cite{lopez2023alphad3m,substrat2022,autopipeline2012vldb,volcanoVLDB2021}. AutoML provides a relevant example where the focus is usually on model quality, and the system selects the best algorithms, features, and hyperparameters~\cite{lopez2023alphad3m}. Pipeline generation for data harmonization can build on such advancements. 
%Still, the task is further complicated by the input data's inherent uncertainty and variability.
An immediate step is building optimizers that measure and balance multiple objectives like harmonization success and computational costs.
Also, our system integrates human-in-the-loop interactions to iteratively refine pipelines based on user feedback and domain expertise. 
An open challenge is determining how to effectively balance automation and human input~\cite{human-in-the-loopdataintegration2017}.

\myparagraph{Acknowledgments}
This work was supported by NSF awards IIS-2106888 and OAC-2411221, and the DARPA
% Automating Scientific Knowledge Extraction and Modeling (ASKEM) program,
ASKEM program
Agreement No. HR0011262087. The views, opinions, and findings expressed are those of the authors and should not be interpreted as representing the official views or policies of the DARPA, the U.S. Government, or NSF.


\bibliographystyle{ACM-Reference-Format}
\bibliography{main}
% \balance

\end{document}
