

\begin{figure*}[t!]
    \centering
    \begin{subfigure}[t]{0.5\textwidth}
         \includegraphics[width=.95\linewidth]{figures/system_diagram_simplified.pdf}
         \hfill
         \caption{}
         \label{fig:system-diagram}
    \end{subfigure}
    \hfill
    \begin{subfigure}[t]{0.24\textwidth}
        \vbox{
            \includegraphics[width=1.0\linewidth]{figures/agent_loop.pdf}
            \vspace{1em}
        }
        \caption{}
        \label{fig:agent-loop}
    \end{subfigure}
    \hfill
    \begin{subfigure}[t]{0.25\textwidth}
         % \centering
         \hfill
         \includegraphics[height=1.2in]{figures/sequence_diagram.pdf}
         \caption{}
         \label{fig:sequence-diagram}
    \end{subfigure}
    \label{fig:system-roles}
    \vspace{-1em}
    \caption{
        (a) An overview of the different components of an interactive, agentic data harmonization system.
        (b) The LLM agent loop: given a task, the LLM will repeatedly execute actions (call tools or ask questions to the user) until the task is completed.
        (c) A sequence diagram illustrating an example of the communication workflow between the user, agent, and primitives.
    }
    \vspace{-.25em}
\end{figure*}