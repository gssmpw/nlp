\vspace{1em}
%----------------------
\section{Preliminaries}
\label{sec:preliminaries}
%----------------------

Broadly speaking, data harmonization refers to the practice of combining and reconciling different datasets to maximize their comparability or compatibility~\cite{cheng2024general}.
Although data harmonization goals can vary greatly depending on the data modalities and goals involved, this paper focuses on the following problem:

\begin{definition}[Tabular Data Harmonization]
Given a set of source tables $T_{1}$, $T_{2}$, ..., $T_{n}$ and a target schema $S_{target}$ composed of a set of attributes, each attribute specifying its acceptable values, i.e., a domain, the goal is to derive a computational pipeline $\mathcal{P}$ that takes as input the source tables and applies transformation functions to values and combines tables (e.g., using union and join operations) to generate an output table $T_{target} = \mathcal{P}(T_1, T_2, ..., T_n)$ that adheres to the given target schema $S_{target}$.
\end{definition}

Note that instead of aiming at maximizing an objective measure of comparability or compatibility between input and target, we assume the existence of the canonical data representation $S_{target}$. 
As seen in Examples~\ref{example1} and ~\ref{example2}, this specification $S_{target}$ consists of a set of attribute names and domain specifications that can either come from a standard data vocabulary (such as the GDC) or could be derived from the existing data in the source tables.

Data harmonization pipelines can be of multiple forms. A simple example is a linear sequence of data processing operations (e.g., as in scikit-learn pipelines~\cite{sklearn-pipelines} or Jupyter Notebooks~\cite{jupyter}). However, they can also be declaratively represented as direct acyclic graphs (DAG), such as in AutoML systems~\cite{lopez2023alphad3m, shang2019democratizing}.
In practice, data harmonization pipelines are typically complex, custom scripts created iteratively through a trial-and-error process~\cite{cheng2024general}.
It is challenging to code and manually explore the space of pipelines to identify the most effective one for a given task. For subject matter experts without programming experience, creating these pipelines is simply out of reach. We envision that LLM-based agents can help address these challenges as we elaborate next.