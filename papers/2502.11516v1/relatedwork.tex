\section{Related Works}
Prior relevant contributions can be divided into four categories, namely, SDMA-enabled FF-ISAC\cite{10464353,10679658,10251151,10382465}, RSMA-enabled FF-ISAC\cite{gong2024hybrid,10486996,10032141,10522473,10287099}, full digital beamforming for NF-ISAC\cite{10520715,10694020,hua2024near,10681603}, and suboptimal multiple access strategies for NF-ISAC\cite{10135096,meng2024hybrid,10700785,10579914}. It is evident that most works consider FF rather than NF propagation.  We next comprehensively survey the recent advances in these categories.



\subsubsection{SDMA-enabled FF-ISAC} SDMA-enabled FF-ISAC designs have been extensively studied \cite{10464353,10679658,10251151,10382465}. For instance, reference \cite{10464353} maximizes the minimum signal-to-interference-plus-noise ratio (SINR) for multi-target sensing, where a dual-function base station (BS) and intelligent omni surfaces (IOS) collaborate for 360-degree coverage. In \cite{10679658}, transmit beampattern and sensing SINR are optimized under imperfect channel state information (CSI). The authors in \cite{10251151} derive the Cram\'{e}r-Rao bound (CRB) of angle estimation and minimize the complete response matrix and reflection coefficients in multi-target ISAC over multicast channels, addressing target detection and tracking phases. This approach is further extended in \cite{10382465} to simultaneous wireless information and power transfer (SWIPT) systems, where the Pareto boundary of the CRB-rate-energy region is optimized via the transmit covariance matrix. Nonetheless, these studies primarily rely on transmit beamforming to mitigate interference, limiting network performance.

\subsubsection{RSMA-enabled FF-ISAC} To address performance saturation, the common stream of RSMA offers three key functions: mitigating interference between communication and sensing, managing interference among communication users, and acting as the sensing beam \cite{9531484}. This insight has spurred interest in RSMA-enabled FF-ISAC \cite{gong2024hybrid,10486996,10032141,10522473,10287099}. For instance, Reference \cite{gong2024hybrid} eliminates dedicated radar sequences in transmitted signals and designs HAD beamforming to maximize single-target sensing SINR. Extending this, reference \cite{10486996} adapts the transmit scheme to multi-target networks and optimizes CRB estimation, albeit using a full digital beamforming architecture. In \cite{10032141}, the tradeoff between communication and localization error is explored through cooperative multi-BS networks. RSMA is also integrated with reconfigurable intelligent surfaces (RIS) to enhance target detection SINR \cite{10287099} or reduce sensing error \cite{10522473} by jointly optimizing active and passive beamforming. However, these RSMA-enabled ISAC designs rely on FF plane-wave channels and predominantly use full digital beamforming, limiting joint distance and angle estimation capabilities.

\subsubsection{Full digital beamforming for NF-ISAC}
Although NF spherical waves can resolve distance and angle simultaneously, NF-ISAC remains underexplored, with notable exceptions in \cite{10520715,10694020,hua2024near,10681603,10135096,meng2024hybrid,10700785,10579914}. Existing studies design the transmit covariance matrix to optimize SINR \cite{10520715}, beampattern matching \cite{10694020}, and sum-CRB \cite{hua2024near} for multi-target detection while ensuring communication rate and power constraints. These three efforts utilize the semi-definite relaxation method to attack the formulated non-convex problems, incurring high computational complexity\cite{10520715,10694020,hua2024near}. Additionally, similar to FF-ISAC \cite{10287099,10522473}, NF-ISAC can deploy RIS to enhance capacity and coverage, but it uniquely enables joint distance and angle estimation \cite{10681603}. However, these NF-ISAC designs use full digital beamforming, which poses significant hardware challenges due to RF chain demands. 

 \subsubsection{Suboptimal multiple access strategies for NF-ISAC} To tradeoff the performance and hardware cost, the authors in \cite{10135096} introduce HAD beamforming and derive the sensing CRB for single-target ISAC. Similarly, SINR is optimized in \cite{meng2024hybrid} and \cite{10700785} 
 for single-target detection using HAD architectures. Additionally, a double-array transceiver structure is proposed in \cite{10579914}  for downlink and uplink ISAC. However, these contributions adopt suboptimal multiple access strategies, which cannot manage interference flexibly\cite{10135096, meng2024hybrid, 10700785, 10579914}. The RSMA-enabled NF-ISAC is underexplored, except for \cite{zhou2024hybrid}. Reference \cite{zhou2024hybrid} demonstrates that dedicated sensing beams are unnecessary for RSMA-enabled NF-ISAC. Based on this insight, the minimum communication rate is maximized by optimizing the receiver filter and hybrid beamformers while ensuring the sensing rate. To our knowledge, the tradeoff between CRB for joint distance and angle sensing and communication performance in RSMA-enabled NF-ISAC remains unexplored.