% !TEX root = main.tex

% Pour mettre en évidence les modifications :
\newcommand{\vd}[1]{\todo[color=blue!20,inline]{Victor: #1}}
\newcommand{\eric}[1]{\todo[inline,color=magenta!10]{Eric: #1}}
\newcommand{\maxi}[1]{\todo[color=green!20,inline]{Maxim: #1}}
\newcommand{\tcr}[1]{\textcolor{red}{#1}}
\newcommand{\vp}[1]{\textcolor{burntumber}{\small\sffamily [VP: {#1}]}}
\newcommand{\mm}[1]{\textcolor{Maroon}{\small\sffamily [MM: {#1}]}}
\newcommand{\ar}[1]{\textcolor{olive}{\small\sffamily [AR: {#1}]}}
\newcommand{\vincent}[1]{\todo[color=darkgreen!10!white]{\color{aurometalsaurus}{\bf VP:} #1}}
\newcommand{\alert}[1]{\textcolor{red}{\small\slshape\newline[{#1}]}}
\newcommand{\group}[1]{\bgroup\small\sffamily\noindent\color{green}{#1}\egroup}  % pour commenter facilement des sections.
\newcommand{\new}[1]{\bgroup\small\sffamily\noindent\color{cobalt}{#1}\egroup}
\newcommand{\old}[1]{\bgroup\small\color{gray}{#1}\egroup}  % pour commenter facilement des sections.
\newcommand{\note}[1]{\textcolor{red}{\newline[\textbf{note:} {#1}]\hrule}}  % très visible dans le document !

\newcommand{\dblue}[1]{\textcolor{darkblue}{#1}}
\newcommand{\blue}[1]{\textcolor{ProcessBlue}{#1}} % bleu clair sympathique
\newcommand{\nblue}[1]{\textcolor{NavyBlue}{#1}}
\newcommand{\tblue}[1]{\textcolor{TealBlue}{#1}}
\newcommand{\green}[1]{\textcolor{darkgreen}{#1}}
\newcommand{\dgreen}[1]{\textcolor{darkgreen}{#1}}
\newcommand{\fgreen}[1]{\textcolor{ForestGreen}{#1}}
\newcommand{\pgreen}[1]{\textcolor{PineGreen}{#1}}
\newcommand{\jgreen}[1]{\textcolor{JungleGreen}{#1}}
\newcommand{\red}[1]{\textcolor{red}{#1}}
\newcommand{\bred}[1]{\textcolor{BrickRed}{#1}}
\newcommand{\dorange}[1]{\textcolor{darkorange}{#1}}
\newcommand{\yellow}[1]{\textcolor{YellowGreen}{#1}}
\newcommand{\maroon}[1]{\textcolor{Maroon}{#1}}
\newcommand{\orchid}[1]{\textcolor{Orchid}{#1}}
\newcommand{\turquoise}[1]{\textcolor{Turquoise}{#1}}
\newcommand{\violet}[1]{\textcolor{Violet}{#1}}
\newcommand{\plum}[1]{\textcolor{Plum}{#1}}
\newcommand{\orange}[1]{\textcolor{aa}{#1}}
\newcommand{\gris}[1]{\textcolor{cc}{#1}}
\newcommand{\gray}[1]{\textcolor{gray}{#1}}
\newcommand{\burntumber}[1]{\textcolor{burntumber}{#1}}
\newcommand{\aurometalsaurus}[1]{\textcolor{aurometalsaurus}{#1}}
\newcommand{\britishracinggreen}[1]{\textcolor{britishracinggreen}{#1}}
\newcommand{\cobalt}[1]{\textcolor{cobalt}{#1}}
\newcommand{\bulgarianrose}[1]{\textcolor{bulgarianrose}{#1}}
\newcommand{\ceruleanblue}[1]{\textcolor{ceruleanblue}{#1}}
\definecolor{darkgreen}{RGB}{0,128,0}

% ---------------------

% On définit les ensembles (plutôt utile en math...):
\newcommand{\nset}{\mathbb{N}}
\newcommand{\nsets}{\mathbb{N}^\star}
\newcommand{\N}{\mathbb{N}}
\newcommand{\Z}{\mathbb{Z}}
\newcommand{\Q}{\mathbb{Q}}
\newcommand{\R}{\mathbb{R}}
\newcommand{\C}{\mathbb{C}}
\newcommand{\E}{\mathbb{E}}

\newcommand{\argmin}{\operatornamewithlimits{\arg\min}}
\newcommand{\argmax}{\operatornamewithlimits{\arg\max}}
\newcommand{\oh}{\operatorname{\mathrm{o}}}
\newcommand{\Oh}{\operatorname{\mathrm{O}}}
\newcommand{\Var}{\operatorname{Var}}
\newcommand{\var}{\operatorname{Var}}
\newcommand{\Cov}{\operatorname{cov}}
\newcommand{\cov}{\operatorname{cov}}
\newcommand{\prob}{\mathbb{P}}
\newcommand{\tr}{\operatorname{tr}}
\newcommand{\rme}{\mathrm{e}}
\newcommand{\rmd}{\mathrm{d}}
\newcommand{\1}{\mathds{1}}
\newcommand{\one}{\mathbbm{1}}

% Parenthèses, Crochets et Accolades
\newcommand{\pr}[1]{\left({#1}\right)}
\newcommand{\prt}[1]{({\textstyle{#1}})}
\newcommand{\prn}[1]{({#1})}
\newcommand{\prbig}[1]{\big({#1}\big)}
\newcommand{\prBig}[1]{\Big({#1}\Big)}
\newcommand{\prbigg}[1]{\bigg({#1}\bigg)}
\newcommand{\prBigg}[1]{\Bigg({#1}\Bigg)}
\newcommand{\br}[1]{\left[{#1}\right]}
\newcommand{\brt}[1]{[{\textstyle{#1}}]}
\newcommand{\brn}[1]{[{#1}]}
\newcommand{\brbig}[1]{\big[{#1}\big]}
\newcommand{\brBig}[1]{\Big[{#1}\Big]}
\newcommand{\brbigg}[1]{\bigg[{#1}\bigg]}
\newcommand{\brBigg}[1]{\Bigg[{#1}\Bigg]}
\newcommand{\ac}[1]{\left\{{#1}\right\}}
\newcommand{\act}[1]{\{{\textstyle{#1}}\}}
\newcommand{\acn}[1]{\{{#1}\}}
\newcommand{\acbig}[1]{\big\{{#1}\big\}}
\newcommand{\acBig}[1]{\Big\{{#1}\Big\}}
\newcommand{\acbigg}[1]{\bigg\{{#1}\bigg\}}
\newcommand{\norm}[1]{\left\|{#1}\right\|}
\newcommand{\normt}[1]{\lVert{\textstyle{#1}}\rVert}
\newcommand{\normn}[1]{\|{#1}\|}
\newcommand{\normbig}[1]{\big\|{#1}\big\|}
\newcommand{\normBig}[1]{\Big\|{#1}\Big\|}
\newcommand{\normbigg}[1]{\bigg\|{#1}\bigg\|}
\newcommand{\abs}[1]{\left\lvert{#1}\right\rvert}
\newcommand{\abst}[1]{\lvert{\textstyle{#1}}\rvert}
\newcommand{\absn}[1]{|{#1}|}
\newcommand{\absbig}[1]{\big|{#1}\big|}
\newcommand{\absBig}[1]{\Big|{#1}\Big|}
\newcommand{\absbigg}[1]{\bigg|{#1}\bigg|}
\newcommand{\absBigg}[1]{\Bigg|{#1}\Bigg|}
\newcommand{\ps}[2]{\left\langle{#1},{#2}\right\rangle}
\newcommand{\psn}[2]{\langle{{#1},{#2}}\rangle}
\newcommand{\pst}[2]{\langle{\textstyle{#1},{#2}}\rangle}

\newcommand{\half}[2]{{#1}/{#2}}  % \nicefrac{1}{2}  % pour les indices
\newcommand{\nofrac}[2]{{#1}/{#2}} % quand on râle sur les fractions
\newcommand{\gauss}{\mathcal{N}}
\newcommand{\mat}[1]{\mathrm{#1}}
\newcommand{\q}[1]{Q_{#1}}

% Pour changer le style des commentaires
\renewcommand{\algorithmiccomment}[1]{\textbf{\textcolor{darkgreen}{// \texttt{#1}}}}
\newcommand{\mult}{\operatorname{Mult}}
\newcommand{\unif}{\operatorname{Unif}}
\newcommand{\tcount}{n}
\newcommand{\ccount}{m}
\newcommand{\XC}{\mathcal{X}}
\newcommand{\YC}{\mathcal{Y}}
\newcommand{\ZC}{\mathcal{Z}}
\newcommand{\CC}{\mathcal{C}}
\newcommand{\tv}{\mathrm{d}_{\mathrm{TV}}}
\newcommand{\todomp}[1]{\textcolor{red}{TODO: #1}}
\newcommand{\confReg}{\mathcal{R}}
\newcommand{\pred}{\mu}
\newcommand{\RCP}{\ensuremath{\texttt{RCP}}}
\newcommand{\CPSQ}{\texttt{CP}\ensuremath{^2}}
\newcommand{\hpdalgo}{\texttt{CP}$^2$\texttt{-HPD}}
\newcommand{\pcpalgo}{\texttt{CP}$^2$\texttt{-PCP}}
\newcommand{\PZ}{\Gamma_{Z|X}}
\newcommand{\distr}{\Gamma}
\newcommand{\pdfdistr}{\gamma} 
\newcommand{\gammayx}{\Gamma_{Y\mid X}} 
\newcommand{\adj}[1]{f_{#1}}
\newcommand{\adjinv}{\tilde{f}_{\varphi}}
\newcommand{\thmspace}{\vspace{0.4em}} 
\def\rset{\mathbb{R}}
% \newcommand{\paragraphformat}[1]{\vspace{.5em}\noindent\textbf{#1}\hspace{0.3em}}  % {\vspace{.5em}\noindent\textbf{#1}\hspace{0.1em}} {\paragraph{#1}}
%\newcommand{\paragraphformat}[1]{\vspace{-.18em}\paragraph{#1}}

\newcommand{\paragraphformat}[1]{\vspace{.2em}\noindent\textbf{#1}\hspace{0.2em}}


\let\mc\mathcal                                             % FANCY FONT (Big-Oh)
\let\ms\mathscr                                                 % ...
\let\mf\mathfrak                                                % ...
\let\mb\mathbb                                                  % FONT FOR FIELDS
\let\tt\texttt                                              % MONOSPACED TEXT

\def\PE\mathbb{E}