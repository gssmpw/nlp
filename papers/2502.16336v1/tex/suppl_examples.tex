% !TEX root = ../main.tex

\subsection{Proof for the first example}
\label{suppl:examplesA}
  We provide here a completely elementary proof. The result actually follows from \Cref{thm:conditional:varphi}.
  In this example, we set $\tau_\star(x)=Q_{1-\alpha}(\textup{P}_{\mathbf{V}|X=x})$, where $\mathbf{V}= V(X,Y)$. We assume that for all $x \in \mathcal{X}$, $\tau_\star(x) > 0$. We denote $\tilde{V}(x,y)= V(x,y)/\tau_\star(x)$ and $\tilde{\mathbf{V}}=\tilde{V}(X,Y)$.
  \[
    Q_{1-\alpha}(\textup{P}_{\tilde{\mathbf{V}}}) = \inf\{t \in \mathbb{R} : \mathbb{P}(V(X, Y) \leq t\tau_\star(X)) \geq 1 - \alpha\}.
  \]
  %
  We will first prove that, for all $x \in \mathcal{X}$, we get that $1 = Q_{1-\alpha}(\textup{P}_{\mathbf{V}|X=x})$, for all  $x \in \mathcal{X}$:
  \begin{align*}
    Q_{1-\alpha}(\textup{P}_{\mathbf{V}|X=x}) 
    &= \inf\{t \in \mathbb{R} : \mathbb{P}(V(X, Y) \leq t\tau_\star(X) | X=x) \geq 1 - \alpha\} \\
    &= \inf\{t \in \mathbb{R} : \textup{P}_{\mathbf{V}|X=x}((-\infty, tQ_{1-\alpha}(\textup{P}_{\mathbf{V}|X=x})]) \geq 1 - \alpha\} = 1.
  \end{align*}
  %
  We then show that $Q_{1-\alpha}(\textup{P}_{\tilde{\mathbf{V}}}) \leq 1$. Indeed, for any $s > 1$,  by the tower property of conditional expectation, we get:
  \begin{align*}
    \mathbb{P}(V(X, Y) \leq s \tau_\star(X)) &= \mathbb{P}(\mathbf{V} \leq s Q_{1-\alpha}(\textup{P}_{\mathbf{V}|X})) \\
    &= \mathbb{E}[\mathbb{P}(\mathbf{V} \leq s Q_{1-\alpha}(\textup{P}_{\mathbf{V}|X}) | X)] \geq 1 - \alpha.
  \end{align*}

  Assume now that  $Q_{1-\alpha}(\textup{P}_{\tilde{\mathbf{V}}}) < 1$. Then for any $s \in (Q_{1-\alpha}(\textup{P}_{\tilde{\mathbf{V}}}), 1)$, using again the tower property of conditional expectation, we get
  \begin{align}
    1 - \alpha &\leq \mathbb{P}(V(X, Y) \leq s\tau_\star(X))
    = \mathbb{E}[\mathbb{P}(\mathbf{V} \leq sQ_{1-\alpha}(\textup{P}_{\mathbf{V}|X}) | X)] \\
    &= \mathbb{E}[\textup{P}_{\mathbf{V}|X}((-\infty, s Q_{1-\alpha}(\textup{P}_{\mathbf{V}|X})))]
    < 1 - \alpha
  \end{align}
  by the definition of the conditional quantile.
  This yields a contradiction. Therefore, for $\textup{P}_X$-a.e. $x \in \mathcal{X}$, 
  \[
    Q_{1-\alpha}(\textup{P}_{\tilde{\mathbf{V}}}) = Q_{1-\alpha}(\textup{P}_{\tilde{\mathbf{V}}|X=x}).
  \]
\subsection{Proof for the second example}
\label{suppl:examplesB}
  We set in this case $\tilde{V}(x,y)= V(x,y) - \tau_\star(x)$, where $\tau_\star(x)= Q_{1-\alpha}(\textup{P}_{\mathbf{V}|X=x})$ and $\tilde{\mathbf{V}}=\tilde{V}(X,Y)$. 
  We will show that $Q_{1-\alpha}(\mathbb{P}_{\tilde{\mathbf{V}}|X=x}) = 0$ for all $x \in \mathcal{X}$. We have indeed:
  \begin{align}
    Q_{1-\alpha}(\textup{P}_{\tilde{\mathbf{V}}|X=x}) &= \inf\{t \in \mathbb{R} : \mathbb{P}(\tilde{V}(X, Y) \leq t | X=x) \geq 1-\alpha\} \\
    &= \inf\{t \in \mathbb{R} : \mathbb{P}(V(X, Y) \leq \tau_\star(X) + t | X=x) \geq 1-\alpha\} = 0.
  \end{align}
  %
  We will now show that $Q_{1-\alpha}(\textup{P}_{\tilde{\mathbf{V}}}) \leq 0$. Indeed, for all $s >0$, by the tower property of conditional expectation and the definition of the conditional quantile, we get
  \begin{align}
    \mathbb{P}(\tilde{V}(X, Y) \leq s) &= \mathbb{E}[\mathbb{P}(V(X, Y) \leq \tau_\star(X) + s | X)] \geq 1-\alpha.
  \end{align}
  %
  On the other hand, assume $Q_{1-\alpha}(\textup{P}_{\tilde{\mathbf{V}}}) < 0$. Set $s \in  ( Q_{1-\alpha}(\mathbb{P}_{\tilde{\mathbf{V}}},0)$. We get 
  \begin{align}
    1-\alpha &\leq  \mathbb{P}(\tilde{V}(X, Y) \leq s) = \mathbb{P}(V(X, Y) \leq s + \tau_\star(X)) \\
    &= \mathbb{E}[\mathbb{P}(V(X, Y) \leq s + \tau(X) | X)]< 1 - \alpha,
  \end{align}
  which leads to a contradiction.

  We first show that $Q_{1-\alpha}(\mathbb{P}_{\tilde{V}}) \leq 0$. Indeed, by the tower property of conditional expectation, using again the definition of the conditional quantile, we get
  \begin{align}
    \mathbb{P}(\tilde{V}(X, Y) \leq 0) &= \mathbb{E}[\mathbb{P}(V(X, Y) \leq \tau(X) | X)] < 1-\alpha,
  \end{align}
  which leads to a contradiction and concludes the proof.
