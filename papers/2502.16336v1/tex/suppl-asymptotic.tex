% !TEX root = ../main.tex

\begin{theorem}
  Assume that \Cref{ass:tau}-\Cref{ass:tau-in-T}-\Cref{ass:kernel-cdf} hold, and let $\ccount$ be of the same order as $\tcount$.
  If $F_{\tilde{V}}(\varphi)\notin\{0,1\}$ and for every $x\in\XC$, $F_{\tilde{V}\mid X=x}\circ F_{\tilde{V}}^{-1}$ is Lipschitz and $f_X$ is continuous, then, for $\alpha\in[\{\tcount+1\}^{-1},1)$ and $\rho>0$, it follows
  \begin{equation*}
    \lim_{h_{X}\to 0}\lim_{\tcount\to \infty} \prob\pr{
      \abs{ \prob\pr{Y\in \mathcal{C}_{\alpha}(X) \,\vert\, X} - 1 + \alpha }
      \le \rho
    }
    = 1
    .
  \end{equation*}
\end{theorem}

\begin{proof}
  First, let's fix $\alpha\in[\{\tcount+1\}^{-1},1)$ and $\rho>0$.
  Our proof is based on the following set:
  \begin{equation*}
    A_{\tcount}
    = \ac{
      x\in\XC \colon F_{\tilde{V}\mid X=x}\circ F_{\tilde{V}}^{-1} \text{ is $4^{-1}\rho \brn{F_{\tilde{V}}(\varphi)}^{-\tcount} \wedge \brn{1 - F_{\tilde{V}}(\varphi)}^{\tcount+1}$-Lipschitz}
    }.
  \end{equation*}
  This set contains every point $x\in\XC$ whose Lipschitz constant of $F_{\tilde{V}\mid X=x}\circ F_{\tilde{V}}^{-1}$ is smaller than a certain threshold which tends to $\infty$ as $\tcount\to \infty$.
  Let's also define the two following sets
  \begin{align*}
    &B_{\ccount,h_{X}}
    = \ac{
      x\in\XC \colon \frac{ \sqrt{2\normn{K_{h_{X}}}_{\infty}} + \sup_{t\in\R_+}\acn{\mathrm{M} t K_{1}(t)} }{ C_{h_{X}}(x) } \sqrt{\frac{ 2 \log \ccount }{ \ccount }}
      + \frac{2 D_{h_{X}}(x)}{C_{h_{X}}(x)}
      \le \frac{\rho}{2}
    },
    \\
    &B_{\infty,h_{X}}
    = \ac{
      x\in\XC \colon \frac{2 D_{h_{X}}(x)}{C_{h_{X}}(x)}
      \le \frac{\rho}{2}
    }.
  \end{align*}
  Lastly, for all $r>0$, consider
  \begin{align*}%\label{eq:def:C}
    &E_{\ccount,h_{X}}
    = \ac{
      x\in\XC \colon 2 + 4 C_{h_{X}}(x)^{-1} \var[K_{h_{X}}(\|x - X\|)] \le \ccount r
    },
    \\
    &G = \ac{
      x\in\XC \colon f_{X}(x) \ge r
    }.
  \end{align*}
  Using basic computations, we obtain the following line
  \begin{multline}\label{eq:bound:asymptotic:1}
    \prob\pr{
      \abs{\prob\pr{Y\in \mathcal{C}_{\alpha}(X) \,\vert\, X} - 1 + \alpha}
      > \rho
    }
    \le \prob\pr{ X \notin A_{\tcount} \cap B_{\ccount,h_{X}} \cap E_{\ccount,h_{X}}; X \in G }
    \\
    + \prob\pr{ X \notin G }
    + \prob\pr{
      \abs{\prob\pr{Y\in \mathcal{C}_{\alpha}(X) \,\vert\, X} - 1 + \alpha}
      > \rho; X \in A_{\tcount} \cap B_{\ccount,h_{X}} \cap E_{\ccount,h_{X}}
    }
    .
  \end{multline}
  Since $\ccount$ is of the same order as $\tcount$, which means that $0 < \liminf \ccount/\tcount \le \limsup \ccount/\tcount <\infty$, it holds
  \begin{equation*}
    \1_{ X \notin A_{\tcount} \cap B_{\ccount,h_{X}} \cap E_{\ccount,h_{X}} } \1_{X \in G}
    \xrightarrow[\tcount\to\infty]{} 1_{ X \notin B_{\infty,h_{X}} } \1_{X \in G}.
  \end{equation*}
  Moreover, since $K_{h_{X}}$ is an approximate identity and $f_X$ is continuous and bounded, we have $\lim_{h_{X}\to 0}C_{h_{X}}(x) = f_X(x)$. As stated in~\Cref{cor:epsilon-tau:local-cdf}, it also holds that $\lim_{h_{X}\to 0}D_{h_{X}}(x) = 0$. Therefore, it follows
  \begin{equation*}
    1_{ X \notin B_{\infty,h_{X}} } \1_{X \in G}
    \xrightarrow[h_{X}\to 0]{} 0.
  \end{equation*}
  Using the dominated convergence theorem, it yields that
  \begin{equation}\label{eq:bound:asymptotic:3}
    \limsup_{h_{X}\to 0}\limsup_{\tcount\to \infty} \prob\pr{ X \notin A_{\tcount} \cap B_{\ccount,h_{X}} \cap E_{\ccount,h_{X}}; X \in G }
    = 0.
  \end{equation}
  Given a realization $x\in\XC$, denoting by $\mathrm{L}_{x}$ the Lipschitz constant of $F_{\tilde{V}\mid X=x}\circ F_{\tilde{V}}^{-1}$, the application of~\Cref{thm:coverage:conditional} shows that
  \begin{multline*}
    \prob\pr{
      \abs{ \prob\pr{Y\in \mathcal{C}_{\alpha}(X) \,\vert\, X} - 1 + \alpha }
      > \rho; X \in A_{\tcount} \cap B_{\ccount,h_{X}} \cap E_{\ccount,h_{X}}
    }
    \\
    \le \prob\pr{
      \rho
      < \absn{\epsilon_{\tau}(X)} + 2 \mathrm{L}_{X} \times \brn{F_{\tilde{V}}(\varphi)}^{\tcount+1} \vee \brn{1 - F_{\tilde{V}}(\varphi)}^{\tcount+1}; X \in A_{\tcount} \cap B_{\ccount,h_{X}} \cap E_{\ccount,h_{X}}
    }
    .
  \end{multline*}
  Since $X \in A_{\tcount}$, we deduce that $\mathrm{L}_{X}\le 4^{-1}\rho \brn{F_{\tilde{V}}(\varphi)}^{-\tcount} \wedge \brn{1 - F_{\tilde{V}}(\varphi)}^{\tcount+1}$, and thus it yields that $2 \mathrm{L}_{X} \times \brn{F_{\tilde{V}}(\varphi)}^{\tcount+1} \vee \brn{1 - F_{\tilde{V}}(\varphi)}^{\tcount+1} \le 2^{-1} \rho$.
  Therefore, it follows
  \begin{multline*}
    \prob\pr{
      \rho
      < \absn{\epsilon_{\tau}(X)} + 2 \mathrm{L}_{X} \times \brn{F_{\tilde{V}}(\varphi)}^{\tcount+1} \vee \brn{1 - F_{\tilde{V}}(\varphi)}^{\tcount+1}; X \in A_{\tcount} \cap B_{\ccount,h_{X}} \cap E_{\ccount,h_{X}}
    }
    \\
    \le \prob\pr{
      2^{-1} \rho
      < \absn{\epsilon_{\tau}(X)}; X \in A_{\tcount} \cap B_{\ccount,h_{X}} \cap E_{\ccount,h_{X}}
    }.
  \end{multline*}
  Since $x\in B_{\ccount,h_{X}} \cap E_{\ccount,h_{X}}$, applying~\Cref{cor:epsilon-tau:local-cdf} gives that
  \begin{equation*}
    \prob\pr{
      2^{-1} \rho
      < \absn{\epsilon_{\tau}(X)}; X \in A_{\tcount} \cap B_{\ccount,h_{X}} \cap E_{\ccount,h_{X}}
    }
    \le r.
  \end{equation*}
  \Cref{eq:bound:asymptotic:1} implies
  \begin{equation*}
    \prob\pr{
      \abs{ \prob\pr{Y\in \mathcal{C}_{\alpha}(X) \,\vert\, X} - 1 + \alpha }
      > \rho
    }
    \le \prob\pr{ X \notin A_{\tcount} \cap B_{\ccount,h_{X}} \cap E_{\ccount,h_{X}} }
    + \prob\pr{ X \notin G }
    + r
    .
  \end{equation*}
  Lastly, \eqref{eq:bound:asymptotic:3} combined with the previous inequality shows
  \begin{equation*}
    \limsup_{h_{X}\to 0}\limsup_{\tcount\to \infty} \prob\pr{
      \abs{ \prob\pr{Y\in \mathcal{C}_{\alpha}(X) \,\vert\, X} - 1 + \alpha }
      > \rho
    }
    \le \E\br{\1_{f_X(X) < r}} + r
    .
  \end{equation*}
  As $r$ is arbitrary fixed, from the dominated convergence theorem we can conclude that 
  \begin{equation*}
    \limsup_{h_{X}\to 0}\limsup_{\tcount\to \infty} \prob\pr{
      \abs{ \prob\pr{Y\in \mathcal{C}_{\alpha}(X) \,\vert\, X} - 1 + \alpha }
      > \rho
    }
    = 0
    .
  \end{equation*}
\end{proof}
