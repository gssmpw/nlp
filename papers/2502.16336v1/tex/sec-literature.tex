% !TEX root = ../main.tex

\vspace{-10pt}
  It is well known that obtaining exact conditional coverage for all possible inputs within the conformal prediction framework is impossible without making distributional assumptions~\cite{foygel2021limits}. However, the literature has proposed various relaxations of exact conditional coverage, focusing on different notions of approximate conditional coverage. 

  A first class of methods involves group-conditional guarantees~\cite{ding2024class,jung2022batch}, which provide coverage guarantees for a predefined set of groups. Another class partitions the covariate space into multiple regions and applies classical conformal prediction within each region~\cite{leroy2021md, alaa2023conformalized,kiyani2024conformal}. The significant limitation of these methods lies in the need to specify the groups or regions in advance.

  Other conformal methods aim to approximate conditional coverage by leveraging uncertainty estimates from the base predictor.
  When $d = 1$, Conformalized Quantile Regression~\citep[CQR;][]{romano2019conformalized} suggests constructing a conformalized prediction interval \( \mathcal{C}_{\alpha}(x) \) by leveraging two quantile estimates of \( Y \mid X = x \), denoted as \( \hat{q}_{\alpha/2}(x) \) and \( \hat{q}_{1-\alpha/2}(x) \).
  This approach yields prediction intervals that adapt to heteroscedasticity~\citep{kivaranovic2020adaptive}.
  By considering a version of CQR by~\citet{sesia2020comparison} whose conformity score is positive, we can draw a connection with \RCP. 
  The conformity score is
  \begin{equation*}
    V(x,y) = \abs{y - \mu_\alpha(x) }/\delta_\alpha(x), 
  \end{equation*}
  with $\mu_\alpha(x)= (\hat{q}_{1-\alpha/2}(x)+\hat{q}_{\alpha/2}(x))/2$ and $\delta_\alpha(x)=\hat{q}_{1-\alpha/2}(x) - \hat{q}_{\alpha/2}(x)$.
  %Set $\adj{t}(v)=t v$ and $\mathbb{T}=\R_+^*$. Then, assumption~\Cref{ass:tau} is satisfied for $(t, v)\in\mathbb{T}\times \R$ and $\adj{t}^{-1}(v) = t^{-1} v$.
  %Setting $\varphi = 1$, it follows that $\adjinv^{-1}(t)=t$.
  Applying \RCP\ with $\adj{t}(v)=t v$ yields the following scaled transformed conformity scores:
  \begin{equation*}
    \tilde{V}(x,y) = \abs{y -  \mu_\alpha(x)} / \hat{\tau}(x),
  \end{equation*}
  where $\widehat{\tau}(x)$ is an estimator of the conditional $(1-\alpha)$-quantile of $\abs{Y -  \mu_\alpha(x)}$ given $X=x$. Thus, this particular variant of \RCP\ closely resembles the CQR approach but uses a different quantile estimate.

  In the context of multivariate prediction sets, given a predictor \(\mu(\cdot)\), a natural choice for the conformity score is \(V_\infty(x, y) = \|y - \mu(x)\|_{\infty}\), where $\| u \|_{\infty}= \max_{1 \leq t \leq d}(|u_i|)$ \citep{Diquigiovanni2021-bh}. This conformity score measures the prediction error associated with the predictor \(\mu\)~\cite{nouretdinov2001ridge,vovk2005algorithmic,vovk2009line}. 
  Setting \(\adj{t}(v) = t v\) and $\varphi = 1$, the rectified conformity scores are given by $\tilde{V}(x, y) = V(x, y)/ \widehat{\tau}(x)$ where \(\widehat{\tau}(x) \approx \q{1-\alpha}\bigl(\textup{P}_{\mathbf{V}_\infty \mid X = x}\bigr)\), with $\mathbf{V}_\infty= V_\infty(X,Y)$. Thus, \RCP\ is similar to the approach proposed in~\cite{lei2018distribution}, but with a different choice of scaling function. 

  Methods utilizing conditional density estimation have been proposed to produce conformal prediction intervals that adapt to skewed data~\citep{sesia2021conformal}, to minimize the average volume~\citep[][denoted DCP in our paper]{Sadinle2016-yr} or to define more flexible highest-density regions~\citep{izbicki2022cd,plassier2024conditionally}. Probabilistic conformal prediction~\citep[PCP;][]{wang2023probabilistic} bypasses density estimation by constructing prediction sets as unions of balls centered on samples from a generative model. All these methods are either tailored to handle the scalar response ($d = 1$) or require an accurate conditional distribution estimate which might be hard to obtain in practical scenarios.

  \citet{guan2023localized} introduces a localized conformal prediction framework that adapts to data heterogeneity by weighting calibration points based on their similarity to the test sample. To do so, kernel-based localizers assign greater importance to nearby points, tailoring prediction intervals to local data patterns. \citet{amoukou2023adaptive} extend Guan's approach by replacing kernels with quantile regression forest estimators for improved performance. Although effective, these methods face challenges in high-dimensional or mixed-variable settings. Here again, these methods have mostly been used in the case where $d=1$.

  Several methods aim to transform conformity scores to improve approximate conditional coverage. For example, \citet{han2022split} presents an approach that uses kernel density estimation to approximate the conditional distribution. Similarly, \citet{deutschmann2023adaptive} rescales the conformity scores based on an estimate of the local score distribution using the jackknife+ technique. However, these methods generally rely on estimating the conditional distribution of conformity scores, which is challenging in practice.

  Finally, various recalibration methods have been proposed to improve marginal coverage~\citep{dheur2023large} or conditional coverage~\citep{dey2022conditionally}. While these methods can also be interpreted within the conformal prediction framework~\citep{dheur2024probabilistic,marx2022modular}, they often require modifications to the training procedure, making them less broadly applicable than purely conformal methods.
