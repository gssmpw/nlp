\section{The method CP$^2$}

Let us define a family of non-decreasing nested regions $\{\mc{R}(x; t)\}_{t \in \R}$ such that $\bigcap_{t \in \R} \mc{R}(x; t) = \emptyset$, $\bigcup_{t \in \R} \mc{R}(x; t) = \mc{Y}$, and $\bigcap_{t' > t} \mc{R}(x; t') = \mc{R}(x; t)$. Without loss of generality, these nested regions are expressed in terms of a conformity score $s_W(x, y) \in \R$ as follows:
\begin{equation}
  \mc{R}(x; t) = \{ y \in \mc{Y}: s_W(x, y) \leq t \},
  \label{eq:nested_sets}
\end{equation}
where $s_W(x, y)$ is continuous in $y$.

As the next step, we introduce a family of transformation functions $f_\tau(\lambda): \R \to \R$ parameterized by $\tau \in \R$. It is assumed that for any $\tau$, the function $\lambda \mapsto f_\tau(\lambda)$ is increasing and bijective.
%(\hl{∈T⊆R\in T \subseteq \mathbb{R} with T={τx:x∈X}T = \{\tau_x : x \in \mathcal{X}\} ?)}
Let $\varphi \in \R$ be a constant (e.g. $\varphi = 1$). We also define the function $g_\varphi(\tau) = f_\tau(\varphi)$ and assume that $\tau \mapsto g_\varphi(\tau)$ is increasing and bijective.

As a first step towards defining CP$^2$, we construct a prediction region assuming knowledge of the conditional distribution $F_{Y|X}$. For a given input $x \in \mc{X}$, the prediction region is defined as:
\begin{equation}
  \bar{R}_\text{CP$^2$}(x) = \mc{R}(x, f_{\tau_x}(\varphi)), \label{eq:rcp2}
\end{equation}
where
\begin{equation}
  \tau_x = \inf\left\{\tau: \mb{P}\left(Y \in \mc{R}(X, f_\tau(\varphi)) \mid X = x\right) \geq 1 - \alpha\right\} \label{eq:CP2_conditional_coverage}
\end{equation}
implies that $\bar{R}_\text{CP$^2$}(x)$ guarantees conditional coverage given $x$. Furthermore, using \eqref{eq:nested_sets} and defining the random variable $W = s_W(X, Y)$, we can equivalently express \eqref{eq:CP2_conditional_coverage} as 
\begin{align}
  \tau_x &= \inf\left\{\tau: \mb{P}\left(s_W(X, Y) \leq f_\tau(\varphi) \mid X=x\right) \geq 1 - \alpha\right\} \\
  &= \inf\left\{\tau: \mb{P}\left(g^{-1}_\varphi(s_W(X, Y)) \leq \tau \mid X=x\right) \geq 1 - \alpha\right\} \label{eq:quantile_CP_2_score_def} \\
 &= Q_{g^{-1}_\varphi(W)}(1 - \alpha \mid X = x) \\
  &= g^{-1}_\varphi(Q_W(1 - \alpha \mid X = x)), \label{eq:outer_quantile}
\end{align}
where we used that $g_\varphi$ is increasing and bijective, with $g^{-1}_\varphi(f_\tau(\varphi)) = \tau$. In other words, $\tau_x$ is the $1-\alpha$ quantile of $g^{-1}_\varphi(W)$.

However, in practice, \( \tau_x \) cannot be computed directly since the true conditional distribution \( F_{Y|x} \) is unknown. Instead, it can be estimated using a sample \( \hat{Y}^{(k)}, k \in [K] \), drawn from the estimated conditional distribution \( \hat{F}_{Y|x} \). If $\hat{Q}_W(1 - \alpha \mid X = x)$ is the \( 1 - \alpha \) quantile of the empirical distribution $\frac{1}{K} \sum_{k \in [K]} \delta_{s_W(x, \hat{Y}^{(k)})}$, we can compute
\begin{equation}
  \hat{\tau}_x = g^{-1}_\varphi(\hat{Q}_W(1 - \alpha \mid X = x)). \label{eq:tau_x_hat}
\end{equation}

It should be noted that this estimated prediction region loses the exact conditional and marginal coverage properties due to the reliance on the estimated conditional distribution. The following shows how conformal prediction can restore some coverage properties.


From \eqref{eq:nested_sets}, using \eqref{eq:rcp2}, we can write
\begin{align}
  \bar{R}_\text{CP$^2$}(x)
  &= \left\{ y \in \mc{Y}: s_W(x, y) \leq f_{\tau_x}(\varphi) \right\} \\
  &= \left\{ y \in \mc{Y}: f^{-1}_{\tau_x}(s_W(x, y)) \leq \varphi \right\}, \label{eq:invertibility_f_step}
\end{align}
where we used the invertibility of \( f_{\tau} \) for any \( \tau \in \R \).

Inspired by \eqref{eq:invertibility_f_step}, \cite{Plassier2024-ex} defined the following conformity score:
\begin{equation}
  s_\text{CP$^2$}(x, y) = f^{-1}_{\hat{\tau}_x}(s_W(x, y)), \label{eq:CP_2_score}
\end{equation}
for which the corresponding prediction region $\hat{R}_\text{CP$^2$}$ is given by
\begin{equation}
  \hat{R}_\text{CP$^2$}(x) = \Set{y \in \mc{Y}: s_\text{CP$^2$}(x, y) \leq \hat{q}},
\end{equation}
where we used \cref{eq:region} from the main text.

As an example, taking $f_\tau(\lambda) = \tau \lambda$ and $\varphi = 1$, the conformity score becomes:
\begin{align}
  &s_{\text{CP$^2$}}(x, y) = s_W(x, y) / \hat{\tau}_x
   \label{eq:empirical_CP_2_score},
\end{align}
where $\hat{\tau}_x$ is defined in \eqref{eq:tau_x_hat}. Finally, we obtain CP$^2$-PCP simply by replacing \( s_W \) with \( s_\text{PCP} \) in \eqref{eq:empirical_CP_2_score}.




\subsection{Asymptotic properties}
\label{sec:CP2_properties}

\subsubsection{Asymptotic equivalence of prediction regions}

In the following, we prove that the prediction regions generated by CP\(^2\) (for any \( f_\tau \) and \( \varphi \)) and CDF-based methods are identical in the oracle setting, asymptotically, as \( |\mathcal{D}_\text{cal}| \to \infty \). Specifically, for any \( x \in \mathcal{X} \), both methods select the same threshold \( t_{1 - \alpha} = Q_W(1 - \alpha \mid X = x) \) for the prediction region \( \mathcal{R}(x; t_{1 - \alpha}) \), which ensures a coverage level of \( 1 - \alpha \).

\begin{proposition}
Provided that the assumptions in \cref{sec:motivation_CP_2_PCP} hold, for any \( x \in \mathcal{X} \), the prediction regions \( \bar{R}_\text{CP$^2$}(x) \) (for any choice of \( f_\tau \) and \( \varphi \)) and \( \hat{R}_\text{CDF}(x) \) are equivalent.
\end{proposition}


\begin{proof}

Using the fact that \( g^{-1}_\varphi(f_\tau(\varphi)) = \tau \) for any \( \tau \in \R \) and that \( g_\varphi \) is increasing and bijective, we can write:
\begin{align}
  \bar{R}_\text{CP$^2$}(x)
  &= \{ y \in \mc{Y}: s_W(x, y) \leq f_{\tau_x}(\varphi) \} \\
  &= \{ y \in \mc{Y}: g^{-1}_\varphi(s_W(x, y)) \leq \tau_x \} \\
  &= \{ y \in \mc{Y}: g^{-1}_\varphi(s_W(x, y)) \leq g^{-1}_\varphi(Q_W(1 - \alpha \mid X = x)) \} \\
  &= \{ y \in \mc{Y}: s_W(x, y) \leq Q_W(1 - \alpha \mid X = x) \}.
\end{align}

Let \( \bar{R}_\text{CDF}(x) \) denote the prediction region obtained using the conformity score \( s_\text{CDF} \) as \( |\mc{D}_\text{cal}| \to \infty \). As shown in \cref{sec:CDF_based}, \( s_\text{CDF}(X, Y) \sim \mc{U}(0, 1) \), which implies \( \hat{q} = 1 - \alpha \). Therefore:
\begin{align}
  \bar{R}_\text{CDF}(x)
  &= \{ y \in \mc{Y}: s_\text{CDF}(x, y) \leq 1 - \alpha \} \\
  &= \{ y \in \mc{Y}: F_{W|x}(s_W(x, y) \mid X = x) \leq 1 - \alpha \} \\
  &= \{ y \in \mc{Y}: s_W(x, y) \leq Q_W(1 - \alpha \mid X = x) \}.
\end{align}

This shows that \( \bar{R}_\text{CP$^2$}(x) = \bar{R}_\text{CDF}(x) \) and that the threshold \( t_{1 - \alpha} = Q_W(1 - \alpha \mid X = x) \) is identical for both methods.

\end{proof}
