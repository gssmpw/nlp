% !TEX root = ../main.tex

\vspace{-10pt}
  In this section, we study the marginal and conditional validity of the predictive set $\mc{C}_{\alpha}(x)$ defined in~\eqref{eq:rcp_empirical}. Due to space constraints, we present simplified versions of the results. Full statements and rigorous proofs can be found in the supplement materials. Many of the results hold independently of the specific method used to construct the conditional quantile estimator $\widehat{\tau}(x)$. The only assumption we impose is minimal
  \begin{assumption}\label{ass:tau-in-T} 
    For any $x\in\XC$, we have $\widehat{\tau}(x)\in\mathbb{T}$.
  \end{assumption}
  %
  The following theorem establishes the standard conformal guarantee. We stress that for this statement, the definition of \(\widehat{\tau}(x)\) is not essential. The result is valid for any function \(\tau(x)\), and the proof follows directly from classical arguments demonstrating the validity of split-conformal method.
  
  \begin{theorem}\label{thm:coverage:marginal}
    Assume \Cref{ass:tau}-\Cref{ass:tau-in-T} hold and suppose the rectified conformity scores $\{\tilde{\mathbf{V}}_{k}\}_{k=1}^{\tcount+1}$ are almost surely distinct.
    Then, for any $\alpha\in(0,1)$, it follows
    \begin{equation*}
      1 - \alpha
      \le \prob\pr{Y_{\tcount+1} \in \mathcal{C}_{\alpha}(X_{\tcount+1})}
      < 1 - \alpha + \frac{1}{\tcount+1}.
    \end{equation*}
  \end{theorem}
  %
  The proof is postponed to \Cref{suppl:marginal}.
  We will now examine the conditional validity of the prediction set.  To do so, we will explore the relationship between the conditional coverage of $\mc{C}_{\alpha}(x)$ and the accuracy of the conditional quantile estimator $\widehat{\tau}(x)$. To simplify the statements, we assume that the distribution of $\textup{P}_{\mathbb{V}_\varphi|X=x}$, where $\mathbf{V}_\varphi= V_\varphi(X,Y)$ is continuous. Define
  \begin{equation}\label{eq:def:epsilon-tau}
    \epsilon_{\tau}(x)
    = \prob\pr{V_\varphi(X,Y)\le \tau(x) \,\vert\, X=x} - 1  + \alpha.
  \end{equation}
  %
  The function $\epsilon_{\tau}$ represents the deviation between the current confidence level and the desired level $1-\alpha$.
  Define the conditional pinball loss 
  \begin{equation}
  \label{eq:def:loss-x}
    \mathcal{L}_x(\tau) =  \E\br{\rho_{1-\alpha}\prbig{V_\varphi(X,Y) - \tau(X))|X=x}}.
  \end{equation}
  %
  It is shown in \Cref{{eq:quantile-conditional-pinball-loss}} (see \Cref{suppl:epsilon} ) that, under weak technical conditions, $\epsilon_{\tau}$ satisfies the following property: for all $x\in\XC$,  
  \begin{equation*}
    \abs{\epsilon_{\tau}(x)}
    \le \sqrt{ 2 \times \ac{ \mathcal{L}_{x}(\tau(x)) - \mathcal{L}_{x}(\tau_\star(x)) } },
  \end{equation*}
  where $\tau_\star(x)$ is defined in \eqref{eq:definition-tau}. The previous equation bounds \(\epsilon_{\tau}(x)\) as a function of the quantile estimate \(\tau(x)\). If \(\tau(x)\) is close to the minimizer of the loss function \(\mathcal{L}_{x}\) (as defined in~\eqref{eq:def:loss-x}), then \(\epsilon_{\tau}(x)\) is expected to approach zero.
  
  The c.d.f function of the rectified conformity score is defined as $F_{\tilde{V}} = \prob(\tilde{V}(X,Y)\le \cdot)$.  We denote its conditional version by $F_{\tilde{V}\mid X=x} = \prob(\tilde{V}(x,Y)\le \cdot \mid X=x)$.
  \begin{theorem}\label{thm:coverage:conditional}
    Assume that \Cref{ass:tau}-\Cref{ass:tau-in-T} and $F_{\tilde{V}}$ is continuous and that, for any $x \in \XC$, $F_{\tilde{V}\mid X=x}\circ F_{\tilde{V}}^{-1}$ is $\mathrm{L}$-Lipschitz. Then, for any $\alpha\in[\{\tcount+1\}^{-1},1)$  it holds
    \begin{multline}\label{eq:def:conditional-bound}
      \prob\pr{Y_{\tcount+1}\in \mathcal{C}_{\alpha}(X_{\tcount+1}) \,\vert\, X_{\tcount+1}}
      \ge 1 - \alpha 
      \\
      + \epsilon_{\tau}(X_{\tcount+1})
      - \alpha \mathrm{L} \times \brn{F_{\tilde{V}}(\varphi)}^{\tcount+1}.
    \end{multline}
  \end{theorem}
  %
  The proof is postponed to \Cref{suppl:conditional:validity}.
  According to \Cref{thm:coverage:conditional}, the conditional validity of the prediction set \(\mathcal{C}_{\alpha}(x)\) directly depends on the accuracy of the quantile estimate \(\tau(x)\). If \(\tau(x)\) closely approximates the conditional quantile \(\q{1-\alpha}\bigl( \textup{P}_{V_\varphi(x,Y)\mid X=x} \bigr)\), then~\eqref{eq:def:conditional-bound} ensures that conditional coverage is approximately achieved.


\paragraphformat{Local quantile regression.}
  We will now explicitly control \(\epsilon_{\widehat{\tau}}(x)\) when the estimate $\widehat{\tau}(x)$ is obtained using the local quantile regression method outlined in~\eqref{eq:def:reg-tau}. For any \(x \in \rset^d\), we define $C_{h_{X}}(x)$ as $C_{h_{X}}(x) = \E[K_{h_{X}}(\|x - X\|)]$.

  \begin{assumption}\label{ass:kernel-cdf}
    There exists $\mathrm{M}\ge 0$, such that for all $v\in\R$, $\tilde{x}\mapsto F_{V_\varphi(\tilde{x},Y)\mid X=\tilde{x}}(v)$ is $\mathrm{M}$-Lipschitz.
    Moreover, $t\in\R_+\mapsto K_{h_{X}}(t)$ is non-increasing.
  \end{assumption}

  \begin{proposition}\label{cor:epsilon-tau:local-cdf}
    Assume \Cref{ass:kernel-cdf} holds.
    With probability at most $\ccount^{-1}\times \{ 1 + 4 C_{h_{X}}(x)^{-1} \var[K_{h_{X}}(\|x - X\|)] \}$, it holds
    \begin{multline*}
      \abs{\epsilon_{\widehat{\tau}}(x)}
      \ge C_{h_{X}}(x)^{-1} \sqrt{\frac{ K_{1}(0) \log \ccount }{ h_{X} \ccount }}
      \\
      + 4 C_{h_{X}}(x)^{-1} \sup_{0\le t \le 1} \acn{t K_{h_{X}}( \mathrm{M}^{-1} t ) }
      .
    \end{multline*}
  \end{proposition}
  %
  The proof is postponed to \Cref{suppl:cdf}. \Cref{cor:epsilon-tau:local-cdf} highlights the trade-off associated with the bandwidth parameter $h_{X}$. Ideally, we would like to choose $h_{X} \ll 1$ to minimize $\sup_{0\le t \le 1} \acn{t K_{h_{X}}( \mathrm{L}^{-1} t ) }$. However, this results in an increase of $\sqrt{\frac{\log \ccount}{\ccount h_{X}}}$. Consequently, there exists an optimal bandwidth parameter $h_{X}$ that depends on both the number of available data points $\ccount$ and the regularity of the conditional cumulative distribution function $x\mapsto F_{V_\varphi(x,Y)\mid X=x}(v)$.

  Finally, for the optimal choice of bandwidth $h_{X}$ one can prove the asymptotic validity of \RCP:
  \begin{equation}
    \prob\pr{Y\in \mathcal{C}_{\alpha}(X) \,\vert\, X} \to 1 - \alpha, ~~ \tcount, \ccount \to \infty
    .
  \end{equation}
  %
  The exact formulation and its proof are given in Appendix~\ref{suppl:asympt}.
