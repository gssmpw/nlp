% !TEX root = ../main.tex

\begin{theorem}\label{thm:conditional:varphi}
  Assume \Cref{ass:tau}-\Cref{ass:tau-in-T} hold. For $x \in \mathcal{X}$, set $\tau_{\star}(x)= Q_{1-\alpha}(\textup{P}_{\mathbf{V}_\varphi|X=x})$, where $V_\varphi(x,y)= \tilde{f}^{-1}_\varphi \circ V(x,y)$. Set  $\tilde{V}_\varphi(x,y)= f_{\tau_\star(x)}^{-1} \circ V(x,y)$ and $\tilde{\mathbf{V}}_\varphi= \tilde{V}_\varphi(X,Y)$.
  Then,  for all $x \in \mathcal{X}$, 
  \[
    \varphi= Q_{1-\alpha}(\textup{P}_{\tilde{\mathbf{V}}_\varphi|X=x})= Q_{1-\alpha}(\textup{P}_{\tilde{\mathbf{V}}_\varphi}).
  \]
\end{theorem}

\begin{proof}
  Set $\psi(x)= Q_{1-\alpha}(\textup{P}_{\tilde{\mathbf{V}}_\varphi|X=x})$. We must prove that $\psi(x)= \phi$ for all $x \in \mathcal{X}$. First, we will show $\psi(x) \leq \varphi$. Note indeed
  \begin{align}
    \prob( \tilde{V}_\varphi(X,Y) \leq \varphi | X=x) &= 
    \prob( V(X,Y) \leq f_{\tau_\star(X)}(\varphi) | X=x) \stackrel{(a)}{=} \prob(V(X,Y) \leq \tilde{f}_{\varphi}(\tau_\star(X))|X=x) \\ &\stackrel{(b)}{=} \prob(\tilde{f}_\varphi^{-1} \circ V(X,Y) \leq \tau_\star(X) | X=x)  \stackrel{(c)}{\geq} 1-\alpha,
  \end{align}
  where (a) follows from $f_t(\varphi)= \tilde{f}_\varphi(t)$, (b) from the fact that $\tilde{f}_\varphi$ is invertible, and (c) from the definition of $\tau_\star(x)$.

  Now, suppose that  $\psi(x) <\varphi$. Since for any $t$, $f_t$ is increasing, we get that $\adj{\tau(x)}(\psi(x))<\adj{\tau(x)}(\varphi)$. Moreover, using that $\tau(x)$ belongs to the interior of $\mathbb{T}$, combined with the continuity of $t\in\mathbb{T}\mapsto\adjinv(t)$; it implies the existence of $\tilde{t}\in\mathbb{T}$ such that $\tilde{t}<\tau(x)$ and also $\adj{\tau(x)}(\psi(x)) < \adj{\tilde{t}}(\varphi)$.
  We can rewrite
  \begin{align*}
    1-\alpha \leq \prob\pr{V(X,Y)\le \adj{\tau_\star(X)}(\psi(X)) \mid X=x}
    &\le \prob\pr{V(X,Y)\le \adj{\tilde{t}}(\varphi) \mid X=x}
    \\
    &= \prob\pr{\adjinv^{-1} \circ V(X,Y)\le \tilde{t} \mid X=x} < 1-\alpha.
  \end{align*}
  which yields to a contradiction.

  We now show that $Q_{1-\alpha}(\textup{P}_{\tilde{\mathbf{V}}_\varphi})= \varphi$. We first show that $Q_{1-\alpha}(\textup{P}_{\tilde{\mathbf{V}}_\varphi})= \varphi$. We first show that $Q_{1-\alpha}(\textup{P}_{\tilde{\mathbf{V}}_\varphi}) \leq \varphi$. This follows from 
  \[
    \prob(\tilde{V}_\varphi(X,Y) \leq \varphi) \stackrel{(a)}{=}
    \E[ \prob(\tilde{V}_\varphi(X,Y) \leq \varphi |X)] \stackrel{(b)}{\geq} 1 - \alpha,
  \]
  where (a) follows from the tower property of conditional expectation and (b) from $\phi= Q_{1-\alpha}(\textup{P}_{\tilde{\mathbf{V}}_\varphi|X=x})$ for all $x \in \mathcal{X}$. 

  Assume now that $Q_{1-\alpha}(\textup{P}_{\tilde{\mathbf{V}}_\varphi})<  \varphi$. Choose $s \in (Q_{1-\alpha}(\textup{P}_{\tilde{\mathbf{V}}_\varphi}), \varphi)$. Then,
  \[
    1-\alpha \leq \prob( \tilde{V}_\varphi(X,Y) \leq s) \stackrel{(a)}{=} \E[\prob(\tilde{V}_\varphi(X,Y) \leq s|X)] \stackrel{(b)}{<}1-\alpha,
  \]
  where (a) follows from the tower property of conditional expectation and (b) $s < \phi= Q_{1-\alpha}(\textup{P}_{\tilde{\mathbf{V}}_\varphi|X=x})$ for all $x \in \mathcal{X}$. This yields to a contradiction which conclides the proof.
\end{proof}