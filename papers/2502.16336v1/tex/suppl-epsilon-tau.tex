
% !TEX root = ../main.tex

In this section, we control the quality of the $(1-\alpha)$-conditional quantile estimator $\tau(x)$.
To do this, recall that $V_\varphi(x,y)=\tilde{f}_\varphi^{-1}\circ V(x,y)$ and consider the following error
\begin{align*}
  \epsilon_{\tau}(x)
  = \prob\pr{V_\varphi(x,Y)\le \tau(x) \,\vert\, X=x} - 1  + \alpha.
\end{align*}
%
Moreover, in this section we denote by $q_{1-\alpha}(x)$ the conditional $(1-\alpha)$-quantile of $V_\varphi(x,Y)$ given $X=x$.

\begin{theorem}
\label{eq:quantile-conditional-pinball-loss}
  For $x\in\XC$, assume that $V_\varphi(x,Y)$ has a $1$-st moment.
  If for any $t\in\R$, $\prob(V_\varphi(x,Y)=t\,\vert\, X=x)=0$, then
  \begin{equation*}
    \abs{\epsilon_{\tau}(x)}
    \le \sqrt{ 2 \ac{ \mathcal{L}_{x}(\tau(x)) - \mathcal{L}_{x}(q_{1-\alpha}(x)) } }.
  \end{equation*}
\end{theorem}

\begin{proof}
  Let $x\in\XC$ be fixed. By definition of $\epsilon_{\tau}$, we can write
  \begin{align*}
    \E\br{\epsilon_\tau(X)^2}
    = \E\br{\pr{\prob\pr{V_\varphi(x,Y)\le \tau(x) \,\vert\, X=x} - 1  + \alpha}^2}.
  \end{align*}
  %
  Moreover, for any $t\in\R\setminus \{0\}$, it holds that
  \begin{equation*}
    \rho_{1-\alpha}'(t) = \1_{t\le 0} - 1 + \alpha.
  \end{equation*}
  %
  By extension, consider $\rho_{1-\alpha}'(0)=1$. Hence, we get
  \begin{align*}
    \epsilon_\tau(x)
    &= \prob\pr{V_\varphi(x,Y)\le \tau(x) \,\vert\, X=x} - 1  + \alpha
    \\
    &= \E\br{\1_{V_\varphi(x,Y)\le \tau(x)} \,\vert\, X=x} - 1 + \alpha
    \\
    &= \E\br{\rho_{1-\alpha}'\prn{V_\varphi(x,Y) - \tau(x)} \,\vert\, X=x}.
  \end{align*}
  %
  For $t\in\R$, define the loss $\mathcal{L}_{x}(t)$ as follows
  \begin{equation*}
    \mathcal{L}_{x}(t)
    = \E\br{\rho_{1-\alpha}\prn{V_\varphi(x,Y) - t} \,\vert\, X=x}.
  \end{equation*}
  %
  Since $\mathcal{L}_{x}(t)$ is convex with Lipshitz continuous gradient, applying~\cite[Theorem~2.1.5]{nesterov1998introductory}, it follows that
  \begin{equation*}
    \abs{\mathcal{L}_{x}'(t_1) - \mathcal{L}_{x}'(t_0)}^2
    \le 2 \mathrm{L} \times D_{\mathcal{L}_{x}}(t_1, t_0),
  \end{equation*}
  where $\mathrm{L}$ denotes the Lipschitz constant of $\mathcal{L}_{x}'$, and where $D_{\mathcal{L}_{x}}$ is the Bregman divergence associated with $\mathcal{L}_{x}$.
  For $t_0,t_1\in\R$, the expression of the Bregman divergence is given by
  \begin{equation*}
    D_{\mathcal{L}_{x}}(t_1, t_0)
    = \mathcal{L}_{x}(t_1) - \mathcal{L}_{x}(t_0) - \mathcal{L}_{x}'(t_0) (t_1 - t_0).
  \end{equation*}
  %
  Let $t_0 = q_{1-\alpha}(x)$, which represents the true quantile. Given that $V_\varphi(x,Y)$ has no probability mass at $q_{1-\alpha}(x)$, we have $\mathcal{L}_{x}'(q_{1-\alpha}(x))=0$. Moreover, by setting $t_1 = \tau(x)$, we can observe that
  \begin{equation*}
    \abs{\mathcal{L}_{x}'(\tau(x))}^2
    \le 2 \mathrm{L} \times \pr{ \mathcal{L}_{x}(\tau(x)) - \mathcal{L}_{x}(q_{1-\alpha}(x)) }.
  \end{equation*}
  %
  Note that $\mathcal{L}_{x}'(\tau(x))=-\epsilon_{\tau}(x)$, therefore, the previous line shows
  \begin{equation*}
    \abs{\epsilon_{\tau}(x)}
    \le \sqrt{ 2 \mathrm{L} \times \ac{ \mathcal{L}_{x}(\tau(x)) - \mathcal{L}_{x}(q_{1-\alpha}(x)) } }.
  \end{equation*}
  %
  Finally, since the derivative of the Pinball loss function is $1$-Lipschitz, it follows that $\mathrm{L}\le 1$.
\end{proof}

In the following, we denote for any $t\in\R$
\begin{equation*}
  F_{V_\varphi\mid X=x}(t)
  = \prob\pr{\adjinv^{-1}\circ V(X,Y)\le t \,\vert\, X=x}.
\end{equation*}
%
Moreover, let's denote by $\hat{F}_{V_\varphi\mid X=x}$ an estimator of the cumulative density function $F_{V_\varphi\mid X=x}$.
For $x\in\XC$, define
\begin{equation*}
  \tau(x) = \inf\ac{t\in\R\colon \hat{F}_{V_\varphi\mid X=x}(t)\ge 1-\alpha}.
\end{equation*}

\begin{lemma}\label{lem:link-cdf}
  For $x\in\XC$, assume that $\hat{F}_{V_\varphi\mid X=x}$ is continuous.
  Then, for any $\alpha\in(0,1)$, 
  \begin{equation*}
    \abs{\epsilon_{\tau}(x)}
    \le \normn{F_{V_\varphi\mid X=x} - \hat{F}_{V_\varphi\mid X=x}}_{\infty}.
  \end{equation*}
\end{lemma}

\begin{proof}
  Let $x$ be in $\XC$. Since $\hat{F}_{V_\varphi\mid X=x}$ is supposed continuous, we have $\hat{F}_{V_\varphi\mid X=x}(\tau(x))=1-\alpha$.
  Furthermore, using that $\epsilon_{\tau}(x)=F_{V_\varphi\mid X=x}(\tau(x))-\alpha+1$, we obtain that
  \begin{align*}
    \abs{\epsilon_{\tau}(x)}
    &= \abs{F_{V_\varphi\mid X=x} \circ \hat{F}_{V_\varphi\mid X=x}^{-1}(1-\alpha) - \alpha + 1}
    \\
    &= \abs{F_{V_\varphi\mid X=x} \circ \hat{F}_{V_\varphi\mid X=x}^{-1}(1-\alpha) - \hat{F}_{V_\varphi\mid X=x} \circ \hat{F}_{V_\varphi\mid X=x}^{-1}(1-\alpha)}
    \\
    &\le \normn{F_{V_\varphi\mid X=x} - \hat{F}_{V_\varphi\mid X=x}}_{\infty}.
  \end{align*}
\end{proof}