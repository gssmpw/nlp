 \textcolor{red}{The text below was not touched yet}

  \todo[inline]{Should we move the text below at the start of this section?}

  To provide intuition for our method, we discuss why \(\q{1-\alpha}(\textup{P}_{\tilde{V}(x, Y) \mid X=x})\) is expected to exhibit minimal dependence on \(x\). For the sake of simplicity, assume \(\tau(x) = \q{1-\alpha}(\textup{P}_{V_\varphi(x, Y) \mid X=x})\). Under this assumption, we demonstrate in \Cref{sec:ccp} that \(\varphi\) corresponds to the conditional quantile of \(\tilde{V}(x, Y)\) given \(X=x\). 

\eric{Yes, following the logic of the presentation, a simplified version of this small calculation should be placed earlier.}
  \cite[Lemma~21.2]{van2000asymptotic} shows that \todo{Where is it shown?} \eric{vdV is the very classical result giving the a.s. convergence of quantiles} that as the size of the calibration dataset increases, 
  \begin{equation*}
    \textstyle
    \q{(1-\alpha)(1+\tcount^{-1})}\pr{\frac{1}{\tcount} \sum_{k=1}^{\tcount} \delta_{\tilde{V}_{k}}}
    \xrightarrow[\tcount \to \infty]{\text{a.s.}} \q{1-\alpha}\prbig{\textup{P}_{\tilde{V}(X,Y)}}.
  \end{equation*}
  where $\tilde{V}_{k}=\tilde{V}(X_k,Y_k)$. The preceding calculations establish that the empirical quantile converges to the true quantile. Moreover, using~\eqref{eq:varphi-quantile}, we can show that for all \(x \in \XC\), the following holds:
\begin{equation*}
    \q{1-\alpha}\big(\textup{P}_{\tilde{V}(X, Y)}\big) 
    = \q{1-\alpha}\big(\textup{P}_{\tilde{V}(x, Y) \mid X = x}\big).
\end{equation*}

As a result, the empirical quantile of \RCP\ for any \(x \in \XC\) converges to the conditional quantile, enabling us to achieve approximate conditional validity. 
