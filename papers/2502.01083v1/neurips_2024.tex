\documentclass{article}


% if you need to pass options to natbib, use, e.g.:
%     \PassOptionsToPackage{numbers, compress}{natbib}
% before loading neurips_2024


% ready for submission
\usepackage[preprint]{neurips_2024}


% to compile a preprint version, e.g., for submission to arXiv, add add the
% [preprint] option:
%     \usepackage[preprint]{neurips_2024}


% to compile a camera-ready version, add the [final] option, e.g.:
%     \usepackage[final]{neurips_2024}


% to avoid loading the natbib package, add option nonatbib:
%    \usepackage[nonatbib]{neurips_2024}


\usepackage[utf8]{inputenc} % allow utf-8 input
\usepackage[T1]{fontenc}    % use 8-bit T1 fonts
\usepackage{hyperref}       % hyperlinks
\usepackage{url}            % simple URL typesetting
\usepackage{booktabs}       % professional-quality tables
\usepackage{amsfonts}       % blackboard math symbols
\usepackage{nicefrac}       % compact symbols for 1/2, etc.
\usepackage{microtype}      % microtypography
\usepackage{xcolor}         % colors

\usepackage{amsmath}
\usepackage{amssymb}
\usepackage{mathtools}
\usepackage{amsthm}
\usepackage{multirow}
% if you use cleveref..
\usepackage[capitalize,noabbrev]{cleveref}

%%%%%%%%%%%%%%%%%%%%%%%%%%%%%%%%
% THEOREMS
%%%%%%%%%%%%%%%%%%%%%%%%%%%%%%%%
\theoremstyle{plain}
\newtheorem{theorem}{Theorem}[section]
\newtheorem{proposition}[theorem]{Proposition}
\newtheorem{lemma}[theorem]{Lemma}
\newtheorem{corollary}[theorem]{Corollary}
\theoremstyle{definition}
\newtheorem{definition}[theorem]{Definition}
\newtheorem{assumption}[theorem]{Assumption}
\theoremstyle{remark}
\newtheorem{remark}[theorem]{Remark}

% Todonotes is useful during development; simply uncomment the next line
%    and comment out the line below the next line to turn off comments
%\usepackage[disable,textsize=tiny]{todonotes}
\usepackage[textsize=tiny]{todonotes}




\usepackage{xspace}
% \newcommand{\tool}{$\mathcal{T}$\xspace}
\newcommand{\ttest}{$\mathcal{T}_T$\xspace}
\newcommand{\tf}{$\mathcal{T}_f$\xspace}
\newcommand{\tr}{$\mathcal{T}_r$\xspace}
\newcommand{\tg}{$\mathcal{T}_G$\xspace}
\newcommand{\RET}{\textsc{Retrain}\xspace}
\newcommand{\GA}{\textsc{GradAscent}\xspace}
\newcommand{\RL}{\textsc{RandLabel}\xspace}
\newcommand{\BT}{\textsc{Bad-T}\xspace}
\newcommand{\SU}{\textsc{SalUn}\xspace}
\newcommand{\SO}{\textsc{SOUL-GradDiff}\xspace}
\newcommand{\method}{\textsc{ToolDelete}\xspace}
\usepackage{colortbl}
\definecolor{Gray}{RGB}{192,192,192}
\definecolor{lightred}{RGB}{255,179,179}

\usepackage[many]{tcolorbox}
\definecolor{main}{HTML}{5989cf}    % setting main color to be used
\definecolor{sub}{HTML}{cde4ff}     % setting sub color to be used

\tcbset{
    sharp corners,
    colback = white,
    before skip = 0.2cm,    % add extra space before the box
    after skip = 0.5cm      % add extra space after the box
} 
\newtcolorbox{bluebox}{
    colback = sub, 
    colframe = main, 
    boxrule = 0pt, 
    leftrule = 6pt % left rule weight
}

\title{Tool Unlearning for Tool-Augmented LLMs}


% The \author macro works with any number of authors. There are two commands
% used to separate the names and addresses of multiple authors: \And and \AND.
%
% Using \And between authors leaves it to LaTeX to determine where to break the
% lines. Using \AND forces a line break at that point. So, if LaTeX puts 3 of 4
% authors names on the first line, and the last on the second line, try using
% \AND instead of \And before the third author name.


\author{%
  Jiali Cheng \quad Hadi Amiri\\
  University of Massachusetts Lowell\\
  \texttt{\{jiali\_cheng, hadi\_amiri\}@uml.edu} \\
  % examples of more authors
  % \And
  % Coauthor \\
  % Affiliation \\
  % Address \\
  % \texttt{email} \\
  % \AND
  % Coauthor \\
  % Affiliation \\
  % Address \\
  % \texttt{email} \\
  % \And
  % Coauthor \\
  % Affiliation \\
  % Address \\
  % \texttt{email} \\
  % \And
  % Coauthor \\
  % Affiliation \\
  % Address \\
  % \texttt{email} \\
}


\begin{document}


\maketitle


Hypergraphs provide a superior modeling framework for representing complex multidimensional relationships in the context of real-world interactions that often occur in groups, overcoming the limitations of traditional homogeneous graphs. However, there have been few studies on hypergraph-based contrastive learning, and existing graph-based contrastive learning methods have not been able to fully exploit the high-order correlation information in hypergraphs. Here, we propose a Hypergraph Fine-grained contrastive learning (HyFi) method designed to exploit the complex high-dimensional information inherent in hypergraphs. While avoiding traditional graph augmentation methods that corrupt the hypergraph topology, the proposed method provides a simple and efficient learning augmentation function by adding noise to node features. Furthermore, we expands beyond the traditional dichotomous relationship between positive and negative samples in contrastive learning by introducing a new relationship of weak positives. It demonstrates the importance of fine-graining positive samples in contrastive learning. Therefore, HyFi is able to produce high-quality embeddings, and outperforms both supervised and unsupervised baselines in average rank on node classification across 10 datasets. Our approach effectively exploits high-dimensional hypergraph information, shows significant improvement over existing graph-based contrastive learning methods, and is efficient in terms of training speed and GPU memory cost. The source code is available at \url{https://github.com/Noverse0/HyFi.git}.


% 하이퍼그래프는 집단에서 자주 발생하는 실제 상호작용의 맥락에서 복잡한 다차원 관계를 표현하는 데 탁월한 모델링 프레임워크를 제공하여 기존의 동질적인 그래프의 한계를 극복할 수 있습니다. 하지만 하이퍼그래프 기반 대조 학습과 관련된 연구는 많지 않으며, 기존의 그래프 기반 대조 학습 방법은 하이퍼그래프의 고차 상관관계 정보를 충분히 활용하지 못했습니다. 여기서는 하이퍼그래프에 내재된 복잡한 고차원 정보를 활용하기 위해 고안된 세분화된 하이퍼그래프 대비 학습(FG-HGCL) 방법을 소개합니다. 제안된 방법은 하이퍼그래프 토폴로지를 손상시키는 기존의 그래프 증강 방법을 피하면서 노드 특징에 노이즈를 추가하여 간단하고 효율적인 학습 증강 기능을 제공합니다. 또한 공유 하이퍼에지와 공유 노드를 동질성의 지표로 사용하는 독특한 대비 학습 방식을 사용합니다. 이 방법은 쌍을 이루는 노드 관계를 4개의 세분으로 효율적으로 분류하고 고품질 임베딩을 생성하며 10개의 데이터 세트에서 노드 분류 및 클러스터링 작업에서 감독 및 비감독 기준선보다 뛰어난 성능을 보입니다. 이 접근 방식은 고차원 하이퍼그래프 정보를 효과적으로 활용하여 기존의 그래프 기반 대비 학습 방법에 비해 상당한 개선을 보여주며, 훈련 속도와 GPU 메모리 비용 측면에서 효율적입니다. 소스 코드는 https://github.com/Noverse0/FG-HGCL.git 에서 확인할 수 있습니다.

% 1. 逻辑推理对大模型很重要
% 2. LLM对于逻辑等价变换很敏感。
% 3. 然而,现有增强LLM逻辑的方法都并没有关注到等价性 a. evaluation b. symbolic-NL映射/translation c. 避免额外的信息干扰
% 4. 逻辑推理的核心特质是independency + commutativity。independency—


% 提出逻辑推理对大模型的重要性
Large language models (LLMs) have demonstrated exceptional performance across various real-world applications~\cite{jaech2024openai,dubey2024llama,liu2024deepseek}. Logic reasoning~\cite{Cummins1991-CUMCRA-2} is essential for LLMs. It allows models to draw valid conclusions, maintain coherence, and make reliable decisions across tasks~\cite{pan2023logic,liu2023evaluating}.

% LLMs 对逻辑序列极为敏感,难以灵活适应等价的逻辑结构。当前的 LLMs 更多依赖模式记忆和惯性进行推理,而非掌握逻辑推理的结构性特征。
% 下降了多少,用数据
However, LLMs are sensitive to reasoning order and struggle with logically equivalent transformations~\cite{chen2024premise,berglund2023reversal,tarski1956logic}. First, the models are highly sensitive to the order of premises, with perturbing the order leading to up to a 40\% performance drop~\cite{chen2024premise,liu2024conciseorganizedperceptionfacilitates}. Additionally, if the testing order is reversed compared to the training order, accuracy drops drastically. For example, in the case of data involving two entities within a single factual statement, accuracy drops from 96.7\% to 0.1\% when training is left-to-right and testing is right-to-left.~\cite{berglund2023reversal,berglund2023taken,allen2023physics}. This suggests that LLMs follow a rigid logical reasoning order driven by learned patterns rather than true logical understanding.


\begin{figure}[t] 
    \centering
        \includegraphics[width=0.5\textwidth]{order_intro_new_fig.pdf}
    % \captionsetup{font={small}} 
    \caption{A logical reasoning example. Independent premises can be freely reordered, while reasoning steps must be reordered without violating dependencies.}
    \label{fig:order_intro}
\end{figure}


% 然而,现有增强LLM逻辑的方法都并没有解决等价变化很敏感的问题 a. evaluation b. symbolic-NL映射/translation c. 避免额外的信息干扰
Existing LLM logical data augmentation methods do not effectively address the sensitivity to equivalent transformations. First, many logical datasets are specifically designed for certain domains, such as specialized fields or exam questions, primarily to broaden the scope of logical reasoning data collection and application~\cite{han2022folio,liu2020logiqa,yu2020reclor}. Second, a line of work aims to enhance the model's reasoning by mapping natural language to symbolic reasoning~\cite{olausson2023linc,xu2024faithful,pan2023logic}, but it primarily provides symbolic tools for understanding logical language rather than enhancing the logical structure itself. Lastly, another augmentation method creates a ``vacuum'' world to block interference from real-world logic~\cite{saparov2022language}, but it focuses on the impact of the model’s prior experience on reasoning, without addressing the design of logical equivalence.


In fact, \textbf{commutativity} is a crucial property of logical reasoning. As established by Gödel’s completeness theorem~\cite{godel1930completeness} and Tarski’s model theory~\cite{tarski1956logic}, commutativity means that independent logical units can be freely reordered without changing the essence of the logical structure. Therefore, in logical reasoning, first, independent premises are commutative. As shown in the upper half of Fig. \ref{fig:order_intro}, different orders of premises represent equivalent problem structures. Furthermore, as demonstrated by Gentzen’s proof theory~\cite{gentzen1935proof}, reasoning steps are also commutative, provided their dependencies are intact. As shown in the lower half of Fig. \ref{fig:order_intro}, changing the order of steps without disrupting the dependencies results in an equivalent reasoning process. However, altering the order of dependent steps disrupts inference and prevents a coherent path to the correct conclusion.


% 我们提出了一种基于逻辑乱序的数据增强框架。对于条件/回答步骤进行乱序。
% order-centric亦谓之为中心的
% teach-enable……
% 所有的说法都要统一,先给一个定义
In this work, we propose an order-centric data augmentation framework that explicitly incorporates logical commutativity into LLM training. For condition order, we randomly shuffle all independent premises. For reasoning steps, we construct a structured, step-by-step reasoning process, identify step dependencies using a directed acyclic graph (DAG), and apply topological sorting to reorder reasoning steps while preserving logical dependencies. Order-centric data augmentation allows models to learn logical equivalence through commutativity, leading to a deeper understanding of logic, rather than relying solely on fixed patterns to solve problems. Our experiments show that order-centric augmentation outperforms training on datasets with a fixed logical structure, enhancing the model's overall reasoning ability and improving its performance in complex shuffled testing scenarios.

% 贡献:提出了一种基于条件乱序和回答步骤乱序的逻辑推理数据增强方法/构建了逻辑推理过程中条件与推理步骤之间的依赖关系建模机制/系统性实验,验证了所提出方法的有效性
Our contributions are summarized as follows:
(1) We propose an order-centric logic data augmentation method based on commutativity, which permutes both condition order and reasoning step order, helping models gain a deeper understanding of logical equivalence.
(2) We introduce a method that uses DAGs to model the dependencies between reasoning steps, helping to identify valid step reorderings.
(3) We conduct extensive experiments to prove the effectiveness of our approach in enhancing logical reasoning.

\begin{figure*}[t]
  \centering
  \includegraphics[width=0.9\textwidth]{figure/tool_unlearning.pdf}
  \vspace{-10pt}
  \caption{Tool Unlearning and the proposed \method approach. \textbf{(a)}: Illustration of tool learning and tool unlearning. Learned tools may be requested to be unlearned due to many reasons, such as tools being insecure, restricted, or deprecated. \textbf{(b)}: Differences between tool unlearning and traditional sample unlearning, in terms of objective and training data. \textbf{(c)}: Proposed method \method. We encourage the unlearned model $f'$ to follow the tool-free LLM $f_0$ which has never seen $T_f$ before. Meanwhile, we maintain its ability on $T_r$ and general tasks by matching the capabilities of tool-augmented model $f$ through task arithmetic.}
  \label{fig:model}
    \vspace{-10pt}
\end{figure*}
% https://www.flaticon.com/search?word=tool

\section{Tool Unlearning: Preliminaries}\label{sec:prem} %: A Novel Machine Unlearning Task}

To understand tool unlearning, we first introduce the concept of ``tool learning,'' see Figure~\ref{fig:model}(a). Let $\mathcal{D} = \{ \mathcal{T}, \mathcal{Q}, \mathcal{Y} \}$ be a dataset with $N$ tools $\mathcal{T}$, and $(\mathcal{Q}, \mathcal{Y})$ denotes query-output examples that demonstrate how to use the tools in $\mathcal{T}$. 
% and corresponding ``demonstrations'' $(\mathcal{Q}, \mathcal{Y})$, where $\mathcal{Q}$ is the query set and $\mathcal{Y}$ is its corresponding output set--$(\mathcal{Q}, \mathcal{Y})$ are labeled examples demonstrating how to use tools in $\mathcal{T}$. 
Each tool $t_i \in \mathcal{T}$ may have one or more demonstrations $\{\mathcal{Q}_i, \mathcal{Y}_i\}$, $|\mathcal{Q}_i| = |\mathcal{Y}_i| \geq 1$. 
Starting with an instruction-tuned LLM $f_0$, 
% which has not been trained on using tools, 
a tool learning algorithm explicitly trains $f_0$ on $\mathcal{D}$ and results in a {\em tool-augmented} model $f$ capable of using the $N$ tools in $\mathcal{T}$. We note that prior to explicit tool learning, the LLM $f_0$ may already have some tool-using capabilities such as performing basic arithmetic operations. 

% An example of tool-augmented models is WebGPT~\citep{webgpt}, which mimics human behavior in answering open-ended questions using a text-based web browser to retrieve information and improve its responses.


\paragraph{Problem Definition:} Tool unlearning aims to remove specific tools from tool-augmented LLMs. Let $\mathcal{D}_f = \{ \mathcal{T}_f, \mathcal{Q}_f, \mathcal{Y}_f \}$ denotes $k < N$ tools and their corresponding demonstrations to be unlearned from the tool-augmented model $f$, and $\mathcal{D}_r = \mathcal{D} \backslash \mathcal{D}_f = \{ \mathcal{T}_r, \mathcal{Q}_r, \mathcal{Y}_r \}$ denotes the remaining tools and their demonstrations to retain. The goal is to obtain an unlearned model $f'$ that has limited knowledge on using $\mathcal{T}_f$ tools--can no longer perform tasks involving $\mathcal{T}_f$ tools--while preserving $f$'s ability to use $\mathcal{T}_r$ tools as before.
% , i.e. on how to solve tasks depending on $T_r$ as prior to unlearning. 

\paragraph{Use Cases of Tool Unlearning}\label{sec:app}
The ability to forget learned tools is essential in real-world applications. For example, 
addressing the insecure tools from untrustworthy developers that could be exploited by adversarial attackers;
% . Examples include cyber attack tools or unaligned AI models that generate harmful content. Adversarial attackers may exploit such tools to compromise the security of models and privacy of users. 
%
removing tools restricted by their providers due to copyright or privacy concerns, such as APIs that start allowing unauthorized downloads of book chapters or releasing publications that users did not author; 
%
unlearning broken or deprecated tool that lead to failed operations or corrupted outputs;
%
unlearning tools that may no longer be needed; 
%
and managing limited model capacity, where new versions of tools necessitate replacing outdated ones. More examples of parameter-level tool unlearning are provided in Appendix~\ref{sec:example}.

% : The tools a model needs to master may change over time due to capacity limitations or evolving requirements.


% \begin{itemize}
%     \item Insecure Tools: Many tools are not created, maintained, or published by well-established, trustworthy developers. Examples include cyber attack tools or unaligned AI models that generate harmful content. Adversarial attackers may exploit such malicious tools, compromising the security of models and users.
    
%     \item Restricted Tools: Tools or their associated usage examples may become unavailable due to restrictions imposed by data providers, including copyright issues or the Right to Be Forgotten (RTBF). Examples include APIs that download unauthorized book chapters or announce publications that users did not author.
    
%     \item Broken Tools/Dependencies: Tools may become broken, deprecated, or fall out of maintenance. Continuing to rely on these tools can result in undesired behavior, such as failed tool calls and corrupted outputs.
    
%     \item Useless Tools: The requirement for certain tools may be temporary. In other words, some tools may no longer be needed at a specific point in time, leading to unlearning requests in this scenario.
    
%     \item Limited Model Capacity: The tools a model needs to master may change over time due to capacity limitations or evolving requirements.
% \end{itemize}


\paragraph{Difference to Standard Unlearning Tasks} \label{sec:diff}
Tool unlearning is different from sample-level unlearning as it focuses on removing ``skills'' rather than individual training samples. 
%
\textbf{Objective}: sample-level unlearning aims to reduce the memorization likelihood or extraction probabilities of specific data samples $(q_i, y_i)$~\citep{jang-etal-2023-knowledge}, which is useful for removing copyrighted or private information. In contrast, tool unlearning targets the ``ability'' to solve tasks using tools marked for unlearning ($T_f$). For example, generating $f'(q_i)$ that is superficially different from $y_i$ (while preserving the semantics) is considered successful for sample-level unlearning. However, for tool unlearning, preserving skills and semantics indicate maintained knowledge on $T_f$, which makes unlearning a failure. Figure~\ref{fig:model}b shows successful tool unlearning, where the ability to use the API is forgotten, despite the high lexical memorization between output of the unlearned model and the training data.
% overlap. 
% This difference also lead to different evaluation methods, which we detail in \S\ref{sec:experiment}. 
In addition, selectively removing knowledge from tool-augmented models is a challenging tasks because changes to one tool may unexpectedly affect the model's ability to use other tools--referred to as {\em ripple effect} in fact editing literature~\citep{ripple_effect,gu2024model}. Furthermore, LLMs are general models that can conduct a wide range of tasks beyond tool using, and this ability must be retained. 
%
\textbf{Evaluation}: metrics like sequence extraction likelihood and perplexity are standard in sample-level unlearning. For tool unlearning, success is measured by the ability to forget or retain tool-related skills, which is more appropriate.
%
\textbf{Data}: sample-level unlearning require access to all individual samples marked for unlearning, while tool unlearning does not. This aligns with ``concept erasure'' in diffusion models~\citep{gandikota2023erasing,kumari2023conceptablation} and zero-shot unlearning~\citep{chundawat2022zero} but differs from traditional LLM unlearning~\citep{yao-etal-2024-machine}. Later we demonstrate this in \S~\ref{sec:no_training_data}.
% As long as the target tools are unlearned, we can use any dataset or choose not to use any data. 


\paragraph{Importance of Parameter-Level Tool Unlearning}
We observe that one can naively block tools at the prompt-level or remove tools from the tool set without updating the LLM. However, these shortcut solutions are insufficient to remove tool knowledge. 
\emph{Firstly}, the knowledge on $\mathcal{T}_f$ persists in the parameters of $f'$, leaving the LLM still under threat. Adversarial agents / attackers can exploit this knowledge, which also bypasses prompt-level restrictions. Since existing LLMs do not guarantee 100\% adherence to instructions or contextual information~\citep{zhou2023instruction,zeng2024evaluating}, they may ignore the tool set provided in the prompt and answer queries with their parametric knowledge~\citep{goyal-etal-2023-factual}. 
In addition, tool unlearning at prompt level can create conflicts between the model's parametric knowledge and contextual information. This may lead to misinformation, hallucination, and other unpredictable behavior~\citep{xu2024knowledge}. Finally, we show in the experiments that prompt-level tool unlearning is indeed insufficient, see Table~\ref{tab:main} (ICLU model), which aligns with existing works on LLM unlearning, where parameter update is required~\citep{jia-etal-2024-soul,zhang2024negative}.


% \paragraph{Retraining: An Impractical Solution}
% A straightforward solution is to delete $\mathcal{D}_f$ from $\mathcal{D}$ and retrain a new model only on $\mathcal{D}_r$. However, this is often infeasible due to the high cost and potential unavailability of the original training data~\citep{NEURIPS2023_299a08ee,ilharco2023editing,Gandikota_2023_ICCV}. In addition, unlearning should not be evaluated \emph{solely} based on similarity to retraining as the potential solution space is highly complex and multidimensional. Specifically, prior work has shown that relying on similarity to retraining has several drawbacks, such as poor auditability~\citep{thudi2021necessity} and ineffective deletion~\citep{cheng2023gnndelete,cheng2023multimodal}.
% % Major drawback of merely comparing to retraining include lack of auditability~\citep{thudi2021necessity}, poor practical deletion performance~\citep{cheng2023gnndelete,cheng2023multimodal}, and high dimensionality of possible solutions. 
% % mere resemblance to retraining is not sufficient nor necessary for evaluating unlearning and deciding optimal performances in practice. 
% % Resemblance to retraining is not sufficient nor necessary for optimal unlearning performances in practice. 
% % such as lack of auditability~\citep{thudi2021necessity}
% Therefore there is a need for designing specialized and efficient unlearning methods for tool-augmented models.


% Ideally, an effective tool unlearning algorithm should consider the following aspects
% \begin{itemize}
%     \item Unlearning should not affect LLM's general utility in tasks unrelated to the tool usage, i.e. $f'$ can be used as a general aligned LLM to solve tasks that $f$ is good at, such as generating text or answering general questions; 
%     % , drafting emails
%     % or math problems.
%     % \item Unlearning should not affect mode switching ability of LLMs (switching from reasoning to tool using).
%     \item Unlearning should not affect LLM's ability on $T_r$, i.e. $f'$ should maintain $f$'s capabilities in using the remaining tools that are not expected to be unlearned.

%     \item x
% \end{itemize}


% things still unclear. 
% \section{\method: Effective Tool Unlearning for LLMs}





\subsection{Framework}
 
\method comprises three components: large protein model encodings, graph-based classification, and post-hoc subgraph explanation, as shown in Figure \ref{fig:systemOV}. 
First, we encode AA sequence data using a pre-trained large protein model, ESM-2 \cite{ESM} (see Section \ref{pLM} ), to serve as node attributes.
% Next, we construct graphs for various bio-networks where each node corresponds to a protein and edges represent known or predicted interactions. 
To address (\ul{Task-1, Hypothesis-1}),
\classifier, a classifier combining graph neural networks (GNNs) with state-space sequence modeling (Mamba), is to capture both local node-pair interactions and global \textbf{pathway-level dependencies} (see Section \ref{sec:classifier}). 
% This model is trained on the node embeddings to predict functional or pathway-level labels.
To address (\ul{Task-2, Hypothesis-2}), \explainer, an explainer method trained with \textbf{pathway-wise masks} (see Section \ref{sec:explainer}), aims to identify the most influential subgraphs.
%by masking less relevant edges and node features.
We explicitly integrate pathway-level information into both models to satisfy \ul{Hypothesis-3}.
The large protein model ensures biologically meaningful protein representations, \classifier leverages various pathway information in knowledge bases for robust classification, and \explainer highlights minimal subgraphs that drive the final predictions, offering interpretable insights into key pathways.
We evaluate \method from both machine learning and biological perspectives, as detailed in Section \ref{sec:exp}.


% an explainer method, PathExplainer trained with \textbf{pathway-wise masks} (see Section \ref{sec:explainer}) is applied to isolate the most influential substructures by masking out less relevant edges and node features. 
% The large protein model preserves biologically meaningful sequence representations, the GNN exploits graph connectivity for robust classification, and the explainer selectively highlights the minimal subgraphs that drive the final prediction, offering interpretable insights into key pathway components.



\subsection{Data Encoding for Node Attributes}
\label{pLM}
\subsubsection{Large Protein Language Model Encoding}

The ESM-2 model \cite{ESM}, pre-trained on over 60 million AA sequences with parameter scaling up to 15 billion, is employed to encode our data. 
We evaluate different parameter variants of ESM-2, and the results are presented in Table \ref{tab:classification_result}.
% This model, pre-trained on over 60 million AA sequences, aims to address the limitations of existing methods like BLAST \cite{BLAST1}, which are unable to handle AA sequences.
% We evalute different parameter variants of ESM-2 and the results have shown in Table \ref{tab:classification_result}.
Formally, each \(S^{(m)}\) is tokenized and passed through stacked Transformers.
The output is a token vector, denoted as:
$
\mathbf{h}_i = \mathbf{H}_1^{(L)},
$
where \(\mathbf{H}_1^{(L)} \in \mathbb{R}^d\) is the embedding of the first token from the \(L\)-th (last) Transformer layer, serving as data representation. 
% The input sequence for node \(i\) is represented as \(\mathbf{X}_i = [x_1, x_2, \dots, x_L]\), where \(L\) is the sequence length and \(x_j\) denotes the \(j\)-th amino acid.
% The Transformer processes \(\mathbf{X}_i\) through stacked layers of multi-head self-attention and feed-forward networks. The attention mechanism at each layer computes:
% \[
% \text{Attention}(\mathbf{Q}, \mathbf{K}, \mathbf{V}) = \text{softmax}\left(\frac{\mathbf{Q} \mathbf{K}^\top}{\sqrt{d_k}}\right) \mathbf{V},
% \]
% where \(\mathbf{Q} = \mathbf{X}_i \mathbf{W}^Q\), \(\mathbf{K} = \mathbf{X}_i \mathbf{W}^K\), and \(\mathbf{V} = \mathbf{X}_i \mathbf{W}^V\), with \(\mathbf{W}^Q\), \(\mathbf{W}^K\), and \(\mathbf{W}^V\) as learnable weight matrices.
% Formally, each AA sequence \(S^{(m)}\) is tokenized and passed through stacked Transformers.
% The output a token vector \(\mathbf{h}_i\), where the embedding of the first token serves as data representation:
% $
% \mathbf{h}_i = \mathbf{H}_1^{(L)},
% $
% where \(\mathbf{H}_1^{(L)} \in \mathbb{R}^d\) is the embedding of the first token from the \(L\)-th (last) Transformer layer. 
% This global feature vector \(\mathbf{h}_i\) encodes the protein's sequence-level information.

\subsubsection{Positional Encoding}
To address a fundamental limitation of GNNs \cite{GIN} or hybrid models \cite{GPS} to distinguish nodes with identical local structures, 
we apply a random-walk-based positional encoding (RWPE) base on a diffusion process \cite{PE}, defined as:
% \begin{align*}
$p_i = [RW_{ii}, RW_{ii}^2, \cdots, RW_{ii}^k] \in \mathbb{R}^k $
% \end{align*}
where $RW = AD^{-1}$ is the random walk operator, constructed by the adjacency matrix $A$ and degree matrix $D$. 
For each node $i$, $RW_{ii}^k$ captures the probability of returning to node $i$ after $k$ steps of random walk.
% Unlike Laplacian eigenvector-based PE (LapPE) which suffers from sign ambiguity, RWPE provides unambiguous positional information. This makes the learning process more efficient as the network does not need to learn invariance to $2^k$ possible sign combinations. 
% Experimentally, RWPE demonstrates superior performance compared to LapPE across molecular datasets.


The final node representation combines the sequence-level features from ESM-2 and the structural information from the graph. Specifically, the sequence embedding \(\mathbf{h}_i\) and the positional encoding \(\mathbf{p}_i\) are concatenated and passed through a linear layer to obtain the final representation:
$\mathbf{x}_i = \text{Linear}([\mathbf{h}_i \| \mathbf{p}_i]),$
where \([\mathbf{h}_i \| \mathbf{p}_i] \in \mathbb{R}^{d + K}\) denotes the concatenation of \(\mathbf{h}_i \in \mathbb{R}^d\) and \(\mathbf{p}_i \in \mathbb{R}^K\). The linear transformation ensures dimensionality reduction and effective integration of both global protein sequence features and local graph structural information.
This final node feature \(\mathbf{x}_i \in \mathbb{R}^d\) is optimized for downstream tasks such as graph classification.



\subsection{\classifier: Pathway Information Learning}
\label{sec:classifier}

\classifier integrates the Graph Isomorphism Network (GIN) with a novel \textbf{pathway-wise Mamba} model.
It leverages the strengths of both global selective modeling mechanisms and message-passing GNNs. 
% We propose a novel pathway learning and classification model by integrating the Graph Isomorphism Network (GIN) with a \textbf{pathway-wise Mamba} architecture.
% % model for classifying pathway networks by combining GIN with \textbf{pathway-wise Mamba}. 
% This approach leverages the strengths of both global selective modeling mechanisms and message-passing GNNs while addressing their limitations in graph-based pathway modeling. 
% Below, we detail the core components of our method.
Specifically, inspired by GPS \cite{GPS}, our model avoids early-stage information loss that could arise from using GNNs exclusively in the initial layers. 
We hence employ pathway-wise global aggregation in combination with an efficient Mamba mechanism \cite{Mamba}. 
% While GNNs struggle with issues like over-smoothing, over-squashing, and limited expressivity against the Weisfeiler-Lehman (WL) test, our model addresses these challenges by introducing pathway-wise global aggregation in combination with an efficiency Mamba mechanism \cite{Mamba}. 
At each layer, node and edge features are updated by aggregating the outputs of a pathway-wise Mamba aggregation as: 
% illustrated in Figure \ref{fig:systemOV}. 
% This process can be expressed as follows:
\begin{eqnarray}
    X^{l+1}, &=& \texttt{PathMamba}^{l} \left( X^{l}, A \right), \\
    \textrm{computed as} \ \ \ 
    X^{l+1}_L, &=& \texttt{LocalGIN}^{l} \left( X^{l}, A \right), \\
    X^{l+1}_G, &=& \texttt{GlobalMamba}^{l} \left( X^{l}, A \right), \\
    X^{l+1}, &=& 
    \texttt{MLP}^{l}\left(X^{l+1}_L + X^{l+1}_G\right),
    \label{eqn:layer_equation}
\end{eqnarray}
where $A \in \mathbb{R}^{N \times N}$ is the adjacency matrix of a graph with $N$ nodes and $E$ edges; $X^{l} \in \mathbb{R}^{N \times d}$ represents the $d$-dimensional node features at layer $l$; $\texttt{LocalGIN}^{l}$ is a GIN; $\texttt{GlobalMamba}^{l}$ is a global pathway-wise aggregation layer; and $\texttt{MLP}^{l}$ is a two-layer multilayer perceptron (MLP) used to combine local and global features.

\subsubsection{Node-wise local aggregation}
% The local graph aggregation is performed using the GIN \cite{GIN}, which 
%  can distinguish graph structures up to the expressiveness of the Weisfeiler-Lehman graph isomorphism test.
Node features are updated by aggregating information from their local neighbors. The GIN operation can be expressed as:
\begin{equation}
    X^{l+1}_L = \text{ReLU}\left( W^{l} \cdot \big( (1 + \epsilon) X^{l} + \sum_{j \in \mathcal{N}(i)} X^{l}_j \big) \right),
\end{equation}
where $\mathcal{N}(i)$ represents the set of neighbors of node $i$, $W^{l}$ is the learnable weight matrix at layer $l$, and $\epsilon$ is a trainable parameter controlling the importance of self-loops. 
This ensures a high level of expressivity for local feature aggregation.


\subsubsection{Pathway-wise global aggregation}
\label{sub:global}

To capture long-range dependencies within pathways, we propose (1) random pathway sampling and (2) sequential pathway modeling in \classifier.

\noindent\textbf{Random Pathway Sampling. }
Formally, for each node $v_i$, we randomly sample a varied, single pathway with a maximum length of $L$. The sampling process is defined as:  
\begin{equation}
    \mathcal{Q} = \left\{ \bf{q}^i \mid \bf{q}^i \sim \text{Pathway}(v_i, L), \, |\bf{q}^i| \leq L \right\}_{i=1}^N,
\end{equation}  
where $N$ is the number of nodes in the graph, and $\bf{q}^i$ represents the sampled pathway for node $v_i$. Each pathway $\bf{q}^i$ is a sequence of nodes $\{v_i, v_{i_1}, v_{i_2}, \dots, v_{i_L}\}$, sampled according to a random walk process \cite{crawl}.  
The sampling process $\text{Pathway}(v_i, L)$ involves selecting a sequence of connected nodes starting from $v_i$.
The selection of each subsequent node is determined probabilistically, guided by the graph adjacency structure.
% where the choice of each subsequent node is determined probabilistically based on the graph's adjacency structure. 
% This ensures that each node is associated with diverse pathways.
% This ensures that each node is associated with a random path, providing a structured yet diverse representation of the graph's global architecture.

\noindent\textbf{Sequential Pathway Modeling. }
The forward propagation of the Mamba layer aggregates long-range dependencies along the sampled pathways. For each sampled pathway $\bf{q}^i \in \mathcal{Q}(X^{l})$, the Mamba layer processes the pathway sequentially as:

\begin{equation}
\begin{aligned}
\Delta_t &= \tau_\Delta(f_\Delta(\mathbf{x}_t^l)), \quad
\mathbf{B}_{t} = f_B(\mathbf{x}_t^l), \quad
\mathbf{C}_t = f_C(\mathbf{x}_t^l), \\
\mathbf{h}_t^l &= (1 - \Delta_t\cdot\mathbf{D}) \mathbf{h}_{t-1}^l + \Delta_t\cdot\mathbf{B}_t\mathbf{x}_t^l \\
X^{l+1}_G &= C \cdot h^{l+1}_{L},
\end{aligned}
\label{eqn:mamba-updates}
\end{equation}
where $\mathbf{x}_t^l $ is the $t$-th input node feature matrix in pathway $\bf{q}^i$ at layer $l$.
$f_*$ are learnable projections and $\mathbf{h}_t^e$ is hidden state. $\tau_\Delta$ is the softplus function. 
The forgetting term 
$(1 - \Delta_t^e\cdot\mathbf{D})$
implements a selective mechanism analogous to synaptic decay or inhibitory processes that diminish outdated or irrelevant information. 
Conversely, the update term 
$\Delta_t^e\cdot\mathbf{B}_t^e$
mirrors gating that selectively reinforces and integrates salient new information. 
The projection $\mathbf{C}_t^e$ translates the internal state into observable outputs.
By processing each sampled pathway individually, the Mamba layer effectively aggregates information along each pathway. 
The aggregated pathway representations are then combined to form the updated node features $X^{l+1}_G$ for the next layer.


\subsubsection{Graph Classification}
The updated node features are aggregated using a max pooling to generate a graph representation. 
This representation is passed through an MLP layer for classification:
\begin{equation}
    y = \text{MLP}\big(\text{MaxPooling}\big(\{h_{v_i}\}_{i=1}^N\big)\big),
\end{equation}
where $N$ is the number of nodes, and $y$ is the predicted class label for the pathway.
The model is trained using the cross-entropy loss:
$\mathcal{L}_\text{cross-entropy} = -\sum_{i=1}^C y_i \log \hat{y}_i,
$
where $C$ is the number of classes, $y_i$ is the ground truth label, and $\hat{y}_i$ is the prediction.

% \noindent By combining GNN and pathway-wise Mamba, our method effectively captures graph structure and pathway-specific features, leading to robust pathway network classification.

\subsection{PathExplainer: Targeted Pathway Inference}\label{sec:explainer}
\explainer directly infer subgraphs to generate targeted pathways by leveraging the interpretability of \classifier. 
Vallina GNNexplainers \cite{gnnexplainer, pgexplainer}, which focus primarily on the node or edge level, often struggle to capture the global structures at the pathway level. 
In contrast, \explainer introduces a key technical novelty by \textbf{training pathway masks}, where entire pathways (i.e., sequences of connected nodes and edges) are selectively masked during training to evaluate their contributions to \classifier.


% This enables the identification of the most influential subgraphs and node features at the pathway level.

PathExplainer formalizes the identification of important subgraphs as an optimization problem. 
For a given graph \( G = (V, E) \), the explanation is defined as \( (G_S, F_S) \), where \( G_S \subseteq G \) is the subgraph and \( F_S \) represents the selected features. 
The explanation is derived by optimizing the mutual information $\mathcal{MI}(\cdot)$ between the subgraph and the model's prediction:
\begin{equation}
\max_{G_S, F} \mathcal{MI}(Y, (G_S, F)) = H(Y) - H(Y \mid G = G_S, X = F_S),
\end{equation}
where \( H(Y) \) is the entropy of the predictions and \( H(Y \mid G = G_S, X = F_S) \) is the conditional entropy given the explanation.

The optimization is approached by learning a pathway mask \( M \) for the sampled pathway's edges and nodes. 
To enhance the interpretability and biological relevance of the pathway mask, random pathways \( \mathcal{Q} \) are sampled as described in Section~\ref{sub:global}. 
For each node \( v_i \), a single random pathway \( q_i \) of length up to \( L \) is sampled. These pathways are then used to restrict the mask learning process to edges within the sampled pathways, ensuring that the learned \( M \) focuses on them. 
Specifically, the adjacency matrix \( A \) is modified based on the pathway mask \( M \) as \( A' = A \odot \sigma(M) \), where \( \sigma \) denotes the sigmoid function. Similarly, the features are masked as \( X' = X \odot \sigma(M) \).
The loss function for PathExplainer combines two components: a cross-entropy term for prediction consistency and regularization terms for sparsity:

\begin{equation}
\resizebox{\linewidth}{!}{% Ensure the content is grouped properly
    $\begin{aligned}
    \mathcal{L}_\text{mask} := -\sum_{c=1}^{C} 1[y = c] \log P_\Phi(Y = c \mid G = A', X = X') d
    + \lambda \|M\|
    \end{aligned}$
}
\end{equation}

where \( \|M\| \) encourages sparsity in the edge selection, and $\lambda$ balances the trade-off between the classification loss and the sparsity regularization.
Hence, the identified important subgraphs and node features (referring to AA sequence data) that contribute most to specific bio-networks can considered as targeted pathways.
% We evaluate \method and resulting pathways from both machine learning (Section \ref{subsec:exp1}) and biological (Section \ref{subsec:exp2}) perspectives .


% By leveraging random pathway sampling \( \mathcal{Q} \), PathExplainer ensures that the learned pathway mask \( M \) focuses on critical subgraphs that reflect the stochastic and hierarchical nature of biological pathways. This integration enhances both the interpretability and the relevance of the extracted subgraphs, providing meaningful insights into key pathway mechanisms.












% We propose PathExplainer, a novel method designed to extract critical subgraphs within pathway networks by leveraging the interpretability of PathMamba. The primary objective of PathExplainer is to identify subgraphs and node features that are most influential in determining a model's predictions for biological pathway network analysis.

% PathExplainer formalizes the identification of important subgraphs as an optimization problem. For a given pathway \( G = (V, E) \), where \( V \) represents nodes and \( E \) represents edges, the explanation is defined as \( (G_S, F_S) \), where \( G_S \subseteq G \) is the subgraph and \( F_S \) represents the selected features. The explanation is derived by optimizing the mutual information between the subgraph and the model's prediction:
% \begin{equation}
% \max_{G_S, F} \mathrm{MI}(Y, (G_S, F)) = H(Y) - H(Y \mid G = G_S, X = F_S),
% \end{equation}
% where \( H(Y) \) is the entropy of the predictions and \( H(Y \mid G = G_S, X = F_S) \) is the conditional entropy given the explanation.

% The optimization is approached by learning two masks: a structural mask \( M \) for edges and a feature mask \( F \) for node features. The graph structure is modified by applying \( M \) to the adjacency matrix \( A \) as \( A' = A \odot \sigma(M) \), where \( \sigma \) denotes the sigmoid function. Similarly, the features are masked as \( X' = X \odot \sigma(F) \).

% The loss function for PathExplainer combines two components: a cross-entropy term for prediction consistency and regularization terms for sparsity and connectivity:
% \begin{equation}
% \begin{aligned}
% \mathcal{L}_\text{mask} = -\sum_{c=1}^{C} 1[y = c] \log P_\Phi(Y = c \mid G = A', X = X') \\
% \quad + \lambda_1 \|M\|_1 
% \quad + \lambda_2 \mathrm{Laplacian}(A'),
% \end{aligned}
% \end{equation}

% where \( \|M\|_1 \) encourages sparsity in the edge selection, and the Laplacian term promotes connectivity within the extracted subgraph.

% PathExplainer is tailored for pathway networks, enabling it to uncover biologically meaningful subgraphs that correspond to key protein interactions. Experiments on pathway network datasets demonstrate its ability to provide interpretable insights into pathway mechanisms while maintaining predictive accuracy.







\section{Experimental Setup} \label{sec:experiment}

\paragraph{Datasets \& Tool-Augmented LLMs}
% We focus on tool-augmented LLMs that are explicitly fine-tuned.
We experiment with the following datasets and their corresponding LLMs:  
\vspace{-7pt}
\begin{itemize}
\itemsep-1pt
\item \textbf{ToolAlpaca}~\citep{tang2023toolalpaca} is an agent-generated tool learning dataset consisting of 495 tools and 3975 training examples. \textbf{ToolAlpaca 7B} is fine-tuned on ToolAlpaca using Vicuna-v1.3~\citep{zheng2023judging}.
\item \textbf{ToolBench}~\citep{qin2024toolllm} consists of more than 16k real world APIs from 49 categories, where each training demonstration involves complex task solving traces. \textbf{ToolLLaMA} is fine-tuned on ToolBench using LLaMA-2 7B~\citep{touvron2023llama2}.
\item \textbf{API-Bench}~\citep{patil2023gorilla} focus on APIs that load machine learning models. \textbf{Gorilla} is fine-tuned on API-Bench from LLaMA 7B~\citep{touvron2023llama1}.
% 4) API-Bank. Lynx-7B is a model fine-tuned on API-Bank.
% 5) API-Blend is .
\end{itemize}
% We choose these datasets because they are fully open source. In addition, there exists other datasets such as API-Bank~\citep{li-etal-2023-api}~and~API-Blend~\citep{basu-etal-2024-api}, but they have not provided accessible training data or instructions for reconstruction.

% There are several types of tool-augmented LLMs:
% \begin{itemize}
%     \item Finetuning-based
%     \item Embedding-based
%     \item In-Context-based
% \end{itemize}

% Tora (\url{https://huggingface.co/llm-agents/tora-13b-v1.0})
% Liquid \url{Liquid1/llama-3-8b-Instruct-liquid-agent-calling}
% ToolLLaMA
% ToolAlpaca \url{https://arxiv.org/abs/2306.05301}
% TALM \url{https://arxiv.org/abs/2205.12255}
% Lynx-7B \url{https://aclanthology.org/2023.emnlp-main.187/}

\paragraph{Setup \& Evaluation}
We use the public checkpoints of the above tool-augmented LLMs as original models--the starting point for unlearning. Then we conduct unlearning experiments with 2--20\% tools randomly selected as $\mathcal{T}_f$.
% We conduct two types of unlearning experiments: 
% \vspace{-7pt}
% \begin{itemize}
% \itemsep-1pt
% \item \textbf{Random Tool Unlearning}, where 2--20\% tools are randomly selected as $T_f$, and 
% \item \textbf{Class-wise Tool Unlearning}, where tools from a specific category, such as all tools tagged as \texttt{Development}-related in ToolAlpaca, are chosen as $\mathcal{T}_f$. Specifically, we focus on unlearning the top 5 largest categories based on class sizes.
% \end{itemize}
We evaluate tool unlearning effectiveness, general capability of tool-unlearned LLMs, and robustness to membership inference attack (MIA). 
%
For \textbf{unlearning effectiveness}, we measure performance on test sets ($\mathcal{T}_T, \uparrow$), forget set ($\mathcal{T}_f, \downarrow$), and remaining set ($\mathcal{T}_r, \uparrow$), where ``performance'' reflects the ability to solve tasks that depend on specific tools, depending on the unique metrics in the original tool-augmented models $f$. 
% , which is different from ``memorization of training sequences'' in prior LLM unlearning works~\citep{jang-etal-2023-knowledge,kassem-etal-2023-preserving,yao-etal-2024-machine,barbulescu2024textual}.
%
For \textbf{general capabilities}, we evaluate the unlearned LLMs on a wide range of tasks: 
college STEM knowledge with MMLU~\citep{hendrycks2021measuring}, 
reasoning ability with BBH-Hard~\citep{suzgun-etal-2023-challenging}, 
instruction-following with IFEval~\citep{zhou2023instruction}, and 
factual knowledge with MMLU~\citep{hendrycks2021measuring}.
%
For \textbf{MIA}, we use the proposed LiRA-Tool; following prior work on LiRA~\citep{icul}, we train the shadow models with forget set size of \{1, 5, 10, 20\} and primarily evaluate the True Positive Rate (TPR) at low False Positive Rate (FPR) (TPR @ FPR = 0.01), where TPR means the attacker successfully detects a tool is present. Therefore, a lower TPR indicates better performance (privacy).

% & \multicolumn{4}{c}{ToolAlpaca-7B} & \multicolumn{4}{c|}{ToolAlpaca-13B} & \multicolumn{4}{c|}{Lynx-7B} & \multicolumn{4}{c}{ToolLLaMA-7B} \\
% & $D_T$ & $D_f$ & $D_r$ & Gen. & $D_T$ & $D_f$ & $D_r$ & Gen. & $D_T$ & $D_f$ & $D_r$ & Gen. \\




\paragraph{Baselines}
As there are no prior works on tool unlearning, we adapt the following unlearning methods to tool unlearning setting (see Appendix~\ref{sec:baseline} for descriptions of the baselines):
general unlearning approaches, including 
\textbf{\GA}~\citep{Golatkar2020EternalSO,yao-etal-2024-machine}, 
\textbf{\RL}~\citep{amnesiac_2021}, and 
\textbf{\SU}~\citep{fan2024salun}; 
and LLM-specific unlearning approaches, including  
\textbf{ICUL}~\citep{icul}, 
\textbf{SGA}~\citep{jang-etal-2023-knowledge,barbulescu2024textual}, 
\textbf{TAU}~\citep{barbulescu2024textual}, 
\textbf{CUT}~\citep{li2024wmdp},  
\textbf{NPO}~\citep{zhang2024negative}, and
\textbf{\SO}~\citep{jia-etal-2024-soul}.
For ICUL~\citep{icul}, we randomly select one example $(q_i, y_i)$ from $\mathcal{T}_f$ and corrupt the output $y_i$ with randomly selected tokens. Then we concatenate this corrupted sequence with other intact sequences as the in-context demonstrations. For all other baselines, we treat all data related to $\mathcal{T}_f$ as unlearning examples and all data related to $\mathcal{T}_r$ as remaining examples. Everything else remains the same for each baseline. 
% See \S\ref{sec:prem} for our discussion on why sample-level unlearning methods are inadequate for effective tool unlearning. 



\section{Results}

% \paragraph{Main results}
% Overall, on average of three datasets, \method-SFT 


\paragraph{Comparison to general unlearning methods}
Our main results in Table~\ref{tab:main} show that, compared to \RET, the best-performing baseline in the general unlearning methods category, \method-SFT outperforms \RET by 0.6, 0.3, 8.0, 2.3 absolute points on \ttest, \tr, \tf, \tg respectively. \method-DPO outperforms \RET by 1.3, 3.3, 9.8, 1.8 absolute points across the same metrics. We note that \GA can effectively unlearn \tf, but it negatively impacts its \ttest and \tr performance. Although \RL and \SU outperforms \GA, they still fall short on \tg compared to \method.


\paragraph{Comparison to LLM-specific unlearning methods}
Existing LLM unlearning methods, despite effective in sample-level unlearning, are prone to under-performing in tool unlearning. Both \method-SFT and \method-DPO outperforms ICUL, SGA, and TAU on \ttest, \tr, \tf and \tg. The only exception is ICUL, which outperforms \method-SFT on \tr by 2.7 absolute points, but is outperformed by \method-DPO on \tr by 0.3 points. The good performance of ICUL on \tr is at the cost of failing to unlearn tools in \tf, which is not desired in tool unlearning. In addition,  ICUL has limited ability of preserving test set performance, it is outperformed by \method-SFT and \method-DPO by 3.6 and 4.3 respectively. Furthremore, it is particularly limited in deletion capacity, i.e. number of unlearning samples that a method can handle. As $|D_f|$ exceeds 10, the performance of ICUL on \ttest significantly degrades. This is while \method can process much larger deletion requests efficiently. 



\begin{figure}
\vskip 0.2in
\begin{center}
\centerline{\includegraphics[width=0.47\linewidth]{figure/mia.pdf}}
% \vspace{-10pt}
\caption{Measuring tool unlearning with LiRA-Tool.}% \GA and ICUL are best-performing baselines for general and LLM-specific unlearning methods.}
% \vspace{-10pt}
\label{fig:mia}
\end{center}
\vskip -0.2in
\end{figure}

% \begin{table}[t]
% \tiny
% % \setlength{\tabcolsep}{3pt}
% \centering
% \caption{Ablation study of proposed properties on ToolAlpaca. \colorbox{lightred}{Highlighted} are metrics that degrade after removing specific parts of the model.}
% \label{tab:ablation}
%     \begin{tabular}{l|cccc}
%     \toprule
%      & $\mathbf{\mathcal{T}_T (\uparrow)}$ & $\mathbf{\mathcal{T}_r (\uparrow)}$ & $\mathbf{\mathcal{T}_f (\downarrow)}$ & $\mathbf{\mathcal{T}_G (\uparrow)}$ \\
%     \midrule
%     \multicolumn{5}{l}{\textit{\method-SFT}} \\
%     \midrule
%     Full Model & \textbf{57.7} & \textbf{72.1} & \textbf{30.5} & \textbf{23.6} \\
%     \midrule
%      - TKD & 58.1 & 72.4 & \cellcolor{lightred}{65.3} & 23.3 \\
%      - TKR & \cellcolor{lightred}{32.7} & \cellcolor{lightred}{40.2} & 23.1 & 20.1 \\
%      - GCR    & 58.0 & 72.5 & 31.1 & \cellcolor{lightred}{17.5} \\
%     \midrule
%     \multicolumn{5}{l}{\textit{\method-DPO}} \\
%     \midrule
%      Full Model & \textbf{58.4} & \textbf{73.3} & \textbf{28.7} & \textbf{23.1} \\
%      \midrule
%      - TKD & 58.6 & 73.2 & \cellcolor{lightred}{65.9} & 22.7 \\
%      - TKR & \cellcolor{lightred}{40.3} & \cellcolor{lightred}{41.8} & 39.3 & 22.1\\
%      - GCR    & 55.7 & 72.7 & 33.1 & \cellcolor{lightred}{14.3} \\
%     \bottomrule
%     \end{tabular}
% \end{table}


\paragraph{SFT vs. DPO}
DPO outperforms SFT by 0.7, 3.0, and 1.8 on \ttest, \tr, \tf respectively. On \tg, SFT is slightly better than DPO by 0.5 points. However, DPO takes slightly longer time to train, see Figure~\ref{fig:time}. Both optimization methods achieve superior performance over existing approaches. 


% \paragraph{Class-wise Tool Unlearning}
% We then investigate category-wise tool unlearning.



\paragraph{Measuring tool unlearning with MIA}
Following prior practices~\citep{lira,icul}, a lower TPR indicates an unlearned model with better privacy when FPR=0.01. \method-DPO achieves 0.14 TPR, outperforming \RET by 0.01. This advantage is obtained by explicitly prioritizing tool-free responses $f_0(\mathcal{Q})$ over original responses. In addition, \method-SFT achieves comparable performance with \RET, which indicates its effectiveness to protect privacy. Both variants of our method outperforms \GA and ICUL, the best performing %general and LLM-specific 
baselines, achieving 0.21 and 0.18 TPR. This indicates that existing sample-level unlearning approaches are not sufficient for unlearning tools, see Figure~\ref{fig:mia}.


\paragraph{Sequential unlearning}
Tool unlearning requests may arrive in sequential mini-batches. We experiment with sequential unlearning requests by incrementally unlearning 2\%, 5\%, 10\%, and 20\% of tools. \RET, ICUL by design cannot process sequential deletion requests. \method can continue training according to the current deletion request, without having to retrain a new model. When 20\% of unlearning requests arrive in batches, \method can sequentially unlearn each of them. As Figure~\ref{fig:seq} and Table~\ref{tab:main} show, compared to unlearning 20\% at once, the performance does not degrade significantly. 
% At each step, deletion request of 3\%, 5\%, 10\%, 30\% comes in, making the total unlearning ratio 2\%, 5\%, 10\%, 20\%, 50\%.



\begin{table}[t]
\setlength{\tabcolsep}{3pt}
\caption{Ablation study of proposed properties on ToolAlpaca. \colorbox{lightred}{Highlighted} are metrics that degrade after removing specific parts of the model.}
\label{tab:ablation}
\vskip 0.15in
\begin{center}
\begin{tiny}
\begin{sc}
    \begin{tabular}{l|cccc||cccc}
    \toprule
     & \multicolumn{4}{c||}{\method-SFT} & \multicolumn{4}{c}{\method-DPO} \\
     & $\mathbf{\mathcal{T}_T (\uparrow)}$ & $\mathbf{\mathcal{T}_r (\uparrow)}$ & $\mathbf{\mathcal{T}_f (\downarrow)}$ & $\mathbf{\mathcal{T}_G (\uparrow)}$ & $\mathbf{\mathcal{T}_T (\uparrow)}$ & $\mathbf{\mathcal{T}_r (\uparrow)}$ & $\mathbf{\mathcal{T}_f (\downarrow)}$ & $\mathbf{\mathcal{T}_G (\uparrow)}$\\
    \midrule
    Full & \textbf{57.7} & \textbf{72.1} & \textbf{30.5} & \textbf{23.6} & \textbf{58.4} & \textbf{73.3} & \textbf{28.7} & \textbf{23.1} \\
    \midrule
     - TKD & 58.1 & 72.4 & \cellcolor{lightred}{65.3} & 23.3 & 58.6 & 73.2 & \cellcolor{lightred}{65.9} & 22.7 \\
     - TKR & \cellcolor{lightred}{32.7} & \cellcolor{lightred}{40.2} & 23.1 & 20.1 & \cellcolor{lightred}{40.3} & \cellcolor{lightred}{41.8} & 39.3 & 22.1\\
     - GCR    & 58.0 & 72.5 & 31.1 & \cellcolor{lightred}{17.5} & 55.7 & 72.7 & 33.1 & \cellcolor{lightred}{14.3} \\
    \bottomrule
    \end{tabular}
\end{sc}
\end{tiny}
\end{center}
\vskip -0.1in
\end{table}



%     \label{fig:mia}
% \begin{table}[t]
% \setlength{\tabcolsep}{3pt}
% \begin{minipage}{.5\linewidth}
%     \centering
%     \includegraphics[width=\linewidth]{figure/mia.pdf}
%     \captionof{figure}{MIA performance using LiRA-Tool. \GA and ICUL are best-performing baselines for general and LLM-specific unlearning methods.}
%     \label{fig:mia}
% \end{minipage} \hfill
% \begin{minipage}{.45\linewidth}
%     \captionof{table}{Full parameters vs. LoRA in tool unlearning performances when deleting 20\% of tools on ToolAlpaca. \colorbox{Gray}{Original} denotes the tool-augmented LLM prior unlearning and is provided \colorbox{Gray}{for reference only}.}
%     \label{tab:peft}
%     \centering
%     \small
%     \begin{tabular}{p{6em}|cccc}
%     \toprule
%      & $\mathbf{\mathcal{T}_T (\uparrow)}$ & $\mathbf{\mathcal{T}_r (\downarrow)}$ & $\mathbf{\mathcal{T}_f (\uparrow)}$ & $\mathbf{\mathcal{T}_G} \mathbf{(\uparrow)}$ \\
%     \midrule
%     \rowcolor{Gray}\textbf{Original (Prior Un.)} 
%                             & 60.0 & 73.1 & 75.7 & 24.1 \\
%     \midrule
%     \textbf{Full param} & 52.7 & 72.1 & 30.5 & 23.6 \\
%     \midrule
%     \textbf{LoRA}       & 51.5 & 71.8 & 36.1 & 19.9 \\
    
%     \bottomrule
%     \end{tabular}
% \end{minipage}
% \end{table}


\paragraph{All properties contribute to effective tool unlearning}
Ablation studies in Table~\ref{tab:ablation} show that without Tool Knowledge Removal, performance of \method-SFT and \method-DPO on \tf degrade by -34.8 and -37.2 absolute points respectively. Such significant performance drop is observed for other model properties as well. Therefore, we conclude all proposed properties are necessary for successful at tool unlearning on \ttest, \tr, \tf, and \tg. 


\paragraph{\method functions effectively without access to training data}\label{sec:no_training_data}
In certain unlearning settings, access to the original training data might be restricted, e.g., in healthcare settings or in cases where training data is no longer available due to compliance. In these cases, \method can generate pseudo-samples for tools using the ``shadow samples'' technique developed for LiRA-Tool, see~\S\ref{sec:lira_tool}. Table~\ref{tab:no_training_data} in Appendix~\ref{sec:additional_result} shows that \method can perform tool unlearning effectively, achieving comparable performances to when full access to the exact training data is available.\looseness-1


\paragraph{\method is efficient}
Efficiency is a critical aspect for unlearning. As Figure~\ref{fig:time} in Appendix~\ref{sec:additional_result} illustrates, \method is substantially more efficient than retraining a new model from scratch--saving about 74.8\% of training time on average. In addition, this efficiency gain is relatively consistent as the size of $T_f$ increases. \method-SFT is slightly faster than \method-DPO, as the latter requires a negative sample for each of its prompts.

% \begin{wrapfigure}[]{r}{7cm}
%     \vspace{-20pt}
%     \centering
%   \includegraphics[scale=.5]{figure/time.pdf}
%     \vspace{-25pt}    
%   \caption{Training time of \method, which saves 74.8\% of time on average.}
%   \label{fig:time}
%   \vspace{-20pt}
% \end{wrapfigure}


\paragraph{\method-LoRA is ultra-efficient with good unlearning performance}
We experiment if \method can achieve effective tool unlearning through LoRA~\citep{hu2022lora}, when computing resource is limited. Experiments on ToolAlpaca show that \method-LoRA can achieve 97.7\%, 99.6\%, 84.5\%, and 84.3\% of the performance of \method with full parameter on \ttest, \tr, \tf, \tg on average across SFT and DPO, see Table~\ref{tab:peft} in Appendix~\ref{sec:additional_result}. In addition, it reduces save computational cost by 81.1\% and decreases the training time by 71.3\%.


\paragraph{\method is flexible in choice of tool-free responses}
In (\ref{eq:prop1}), we obtain tool knowledge-free responses from the tool-free LLM $f_0$. However, in cases where $f_0$ is unavailable, \method can still function using any knowledge-free LLM to generate tool knowledge-free responses, such as a randomly initialized LLM $f_R$. Table~\ref{tab:tool_free} compares the performances between these two implementations. While $\theta_0$ consistently outperforms $\theta_R$, using $\theta_R$ is still effective in achieving tool unlearning.\looseness-1 


\paragraph{Why is \method effectiveness?}
We attribute the performance of \method to its three key properties:
(a): Tool Knowledge Removal enables targeted tool unlearning without over-forgetting, unlike \GA and \RET. This is achieved by prioritizing tool knowledge-free responses over tool knowledge-intense responses so that the model forgets tool functionality without excessive degradation.
% This unlearning formulation poses the right strength of forgetting over specific tools, while existing methods either over-unlearn, such as \GA, or does not unlearn sufficiently, such as \RET. 
(b): Tool Knowledge Retention reinforces the knowledge about remaining tools. In fact, re-exposing the model to the original training data can further strengthen their representation. 
(c): General Capability Retention, which maintains or even improves model's general capabilities through an efficient and effective task arithmetic operation. Therefore, precise unlearning, retention of relevant knowledge, and overall model stability are the key factors that contribute to the performance of \method.  

 
% data from $f_0$, a model that has never seen the deleted tools before, and effectively
This section reviews related work on analogy in education, evaluating analogy in HCI, analogy-making with LLMs, and \chirev{LLM-assisted educational systems.} 
\subsection{Analogy in Education}
Analogies help humans understand complex concepts by linking them to familiar ones, making them a valuable tool in educational contexts. 
Many studies~\cite{thagard_analogy_1992,gick_analogical_1980,gick_schema_1983, brown_analogical_1989} have investigated analogical problem solving, where students of various ages solve unfamiliar problems using well-designed analogies. 
Through observational feedback and statistical analysis, researchers have established frameworks and several guidelines for using analogies in education. 
For example, as discussed by~\cite{gick_analogical_1980}, the source of the analogy would share similar relationships with the target, yet originate from a semantically distant field. 
However, such lab studies often involve experimenters posing problems, with students merely solving them without instructional guidance~\cite{brown_analogical_1989}, which diverges from real classroom learning.
Therefore, further research~\cite{vendetti_analogical_2015,richland_analogy_2004,treagust_science_1992,oliva_teaching_2007} have investigated how teachers and students engage with analogies in classroom settings, leading to nuanced insights on the influence of students' age and background and teachers' strategies.
% For example, the age and background of students~\cite{vendetti_analogical_2015}, along with the teacher's strategies~\cite{richland_cognitive_2007}, influence how frequently and effectively analogies are used in the classroom.

Although previous studies have explored the characteristics and use of analogies in education, they have not examined those generated by LLMs, which is crucial given the growing importance of LLM-assisted education~\cite{gao2024fine, Lyu2024evaluating}.
Our work fills this gap by leveraging LLMs to generate analogies tailored to specific education needs, incorporating established characteristics from prior literature and our interviews.
We design human-subjective studies to evaluate their effectiveness in problem-solving tests and classroom environments following prior research.
% \chirev{Finally, we developed a system to examine how LLMs support educators in constructing analogies for real-world applications.}

\subsection{Evaluating Analogy in Human-Computer Interaction}
Analogy has long been studied in HCI for its effectiveness in various context, including algorithms improvement~\cite{bureaucracy2020Pääkkönen,streetlevel2019Alkhatib}, cancer communication~\cite{capturing2024hnatyshyn}, narrative framing~\cite{reelframer2024wang}, enhancing deliberation~\cite{help2024yeo}, communicating standardized effect sizes~\cite{putting2022kim}, and sensemaking of LLM responses~\cite{supporting2024gero}.

Two key research directions about analogies in HCI are for enhancing numerical comprehension through data analogy and fostering creativity.
Data analogies link abstract data to familiar concepts to improve understanding. Researchers evaluate these analogies using controlled experiments and assess effectiveness through subjective ratings like helpfulness~\cite{toput2018riederer, improving2018hullman, generating2016kim, spatharioti_using_2024, chen_beyond_2024}, estimation errors~\cite{toput2018riederer, improving2018hullman}, and correlations between model and human ratings~\cite{spatharioti_using_2024}.
Analogies also facilitate scientific discovery and design. 
In scientific discovery, evaluations involve coding analogy types~\cite{solvent2018chan, kang_augmenting_2022}, calculating similarity metrics~\cite{solvent2018chan}, and conducting think-aloud sessions with scientists~\cite{kang_augmenting_2022}. 
For creative design, analogies are assessed by novelty~\cite{searching2014yu, bilogically2023zhu, bidtrainer2024chen}, quality~\cite{distributed2014yu, bidtrainer2024chen}, relevance and domain distance~\cite{analogymining2018Gilon}, feasibility~\cite{bilogically2023zhu}, and rationality~\cite{bidtrainer2024chen}.
Recently, Ding et al.~\cite{ding_fluid_2023} explored GPT-3's capacity to augment cross-domain analogical reasoning, finding it helpful for creative problem reformulation despite the risks of harmful content.

However, there has been limited exploration of analogy search in HCI for education~\cite{kumar2015stickipedia}. 
While researchers have adopted LLMs to help students and teachers generate novel analogies~\cite{bhavya2024analego}, systematic evaluations of their effectiveness in educational settings are lacking. 
Given the unique cognitive demands of education, existing assessments~\cite{ding_fluid_2023} may not be directly applicable. 
Our work aims to address this gap and offer insights into analogy generation for education.



\subsection{Analogy-making with Language Models}
Analogy is vital for human cognition and has attracted considerable interest from the AI research community. 
Traditionally, studies on analogy-making in AI have concentrated on creating word analogies (\eg, ``king is to man as queen is to woman'') using smaller language models (LMs), e.g., BERT~\cite{devlin_bert_2019} and GPT-2~\cite{radford_language_2019} trained on specific datasets~\cite{turney_combining_2003,mikolov_linguistic_2013,boteanu_solving_2015,gladkova_analogy-based_2016,chen_e-kar_2022, yuan_analogykb_2023}.
With the advancement of LLMs~\cite{ouyang_training_2022,team_gemini_2023,touvron_llama_2023,openai_gpt-4_2023}, there has been a shift toward generating natural language analogies, \ie, free-form analogies~\cite{bhavya_analogy_2022,webb_emergent_2022,ding_fluid_2023,wijesiriwardene_analogical_2023,jiayang_storyanalogy_2023,hu_-context_2023,sultan_parallelparc_2024} and forming structural analogies~\cite{sultan_life_2022,yuan_beneath_2023}.
Researchers typically design prompts manually for free-form analogies to guide LLMs in \chifinal{generating analogies~\cite{bhavya_analogy_2022, webb_emergent_2022}.
For example, Bhavya et al.~\cite{bhavya_analogy_2022} constructed a new dataset including standard science analogies and science analogies from academics and adopted prompt engineering to ask LLMs to generate analogies. 
The results show that LLMs are sensitive to prompt design, temperature, and injected spelling errors, particularly the distinction between questions and imperative statements.
We followed their optimal prompt format for our generation process.
}
%Recent studies on structural analogies employ LLMs to identify mappings between concepts across domains based on relational structures~\cite{yuan_beneath_2023}.
\chirev{
For evaluation of the generation quality of analogy, previous studies have relied on annotators manually evaluating analogies according to established principles of analogy cognition~\cite{sultan_life_2022}. 
}
%\sout{traditional work in the AI research community focuses on manually constructing benchmarks from student exams, formatting them as question-answering tasks to test the capabilities of LMs to recognize word analogies~\cite{mikolov_linguistic_2013,boteanu_solving_2015,gladkova_analogy-based_2016,chen_e-kar_2022}.
%For example, Chen et al.~\cite{chen_e-kar_2022} collect 1,655 analogical reasoning problems sourced from publicly available Civil Service Examinations of China to evaluate the performance of LMs.}
%When addressing free-form and structural analogies, verifying the correctness automatically becomes challenging.
%Therefore,
% For example,  et al.~\cite{_life_2022} described a method where annotators categorize generated analogies into five types: Not analogy, Self-analogy (where entities and their roles are identical), Close-analogy (topics are similar and entities are from related domains), Far-analogy (covering unrelated topics with different entities) and Sub-Analogy (only a part of one entity is analogous to a part of the other).
%give the entity and relation similarity to model the good analogy is with low entity similarity and high relation similarity.
%However, much of this research has focused primarily on assessing if LLMs can produce relevant analogies through human evaluation, without fully exploring their practical use in real-world applications.

In contrast to these approaches, our study is pioneering in investigating how analogies generated by LLMs can help students understand scientific concepts. 
We analyze the characteristics of analogies in educational settings through literature reviews and interviews and incorporate them into prompts for generation. 
Then, we use LLMs to generate educational analogies and evaluate them in real tests, classroom practice, \chirev{and a practical system}.
% \chirev{We also developed a system and conducted a user study to explore the real-world value of LLMs in assisting educators with constructing analogies.}

\subsection{\chirev{LLM-assisted Educational Systems}}
% \ysy{\todo @ysy and @szk add related work from NLP domain and HCI domain}
\chirev{
With the rapid advancement of LLMs, researchers are exploring their potential to develop efficient and practical systems that support students and teachers in educational tasks~\cite{kasneci2023chatgpt,yan2024practical}. 
For students, many studies have focused on creating intelligent tutoring systems powered by LLMs. 
Examples include enabling fully autonomous self-learning pipelines to support self-regulated learning~\cite{gao2024fine} and developing and evaluating LLM-based learning assistants in classroom settings~\cite{kazemitabaar_codeaid_2024,Lyu2024evaluating}.
For teachers, several LLM-based systems are designed to effectively monitor and analyze students' learning activities~\cite{ngoon_classinsight_2024,cflow,vizgroup}.
In addition, researchers aim to assist teachers in creating diverse teaching materials, such as lesson plans~\cite{lessonplanner2024uist}, diagrammatic problems~\cite{edgeworth}, and reading quizzes~\cite{readingquizmaker}.

Our work explores a novel aspect of LLM-driven education: evaluating the effectiveness of LLMs in generating teaching analogies. 
One preliminary research has initially explored generating educational analogies~\cite{bhavya2024analego}, while its system design lacks the support of empirical evidences and fails to address teachers' needs.
Instead, we first conducted a two-stage study to gain insights and empirical evidence and identify needs for teachers and students. 
We then developed and tested a system to support teachers in creating and refining analogies for lesson preparation and discussed future integration with diverse LLM-based educational tools for various users.
% Through two studies, we examined LLM-generated analogies' impact with and without teacher intervention, which provided insights into potential applications for students' self-learning and teachers' lesson preparation. 




% For example, Abd-Alrazaq et al.~\cite{abd2023large} use LLMs to produce clinical case studies, act as virtual test subjects or patients, accelerate research, develop course plans, and provide personalized feedback and assistance.  
% Lee et al.~\cite{lee-etal-2023-peep} design a situational dialogue-based chatbot to help foreign students learn English.
}

\section{Conclusion}

We introduce Tool Unlearning--a novel machine unlearning task with the goal of unlearning previously learned tools from tool-augmented LLMs. 
% Unlike traditional unlearning, tool unlearning requires the removal of parametric knowledge associated with specific tools while preserving both the model’s ability to use other tools and its general capability. 
We develop the first tool unlearning approach, \method, that implements three key properties:
{\em tool knowledge deletion}, %for precise forgetting of tools marked for unlearning; 
{\em tool knowledge retention}, %for preserving the knowledge about the remaining tools; and 
{\em general capability retention}. %for maintaining LLM's general capabilities  on a range of general tasks such as text generation.
%
In addition, we introduce LiRA-Tool, the first membership inference attack (MIA) method for evaluating tool unlearning. LiRA-Tool largely addresses the limitations of sample-based MIAs for tool unlearning. 
%
Extensive experiments on several diverse datasets and LLMs show  that \method is an efficient, flexible, and effective tool unlearning method that supports sequential unlearning, maintains strong performance across all key properties, and operates without requiring full access to training data. 
It outperforms existing methods by removing tool knowledge without over-forgetting (as shown in ablation studies), achieving 74.8\% faster training times compared to retraining, and delivering highly effective tool unlearning even in resource-constrained settings with \method-LoRA (which reduces compute costs by 81.1\% and training time by 71.3\%). 
% In addition, \method is effective when the exact training data or the original pre-tool-augmentation LLM is unavailable
% and when we  randomly initialized model instead of the original pre-tool-augmentation model.

In future, we will investigate tool unlearning in dynamically updated LLMs (e.g. API-based LLMs like GPT-4), where we address continuous unlearning challenges. In addition, we will develop adversarial training techniques and robustness evaluation frameworks to prevent unintended tool re-learning or model exploitation.

% (b) adaptive forgetting in LoRA for efficient tool unlearning.\looseness-1


\paragraph{Limitations}
We did not conduct experiments using closed-source LLMs or API-based LLMs. 
% Consequently, our findings may not directly extend to such proprietary models, and further research is needed to investigate the applicability of tool unlearning techniques in these contexts.
%
In addition, this work did not investigate the impact of varying model scales due to the limited publicly-available tool-augmented LLMs. Our experiments were conducted on the 7B scale and the scalability of the proposed tool unlearning approach across models of different sizes and scales is an open question for future investigation.






% \begin{figure}[t]
% \centering
% \begin{minipage}{0.28\textwidth}
%   \includegraphics[width=\textwidth]{figure/time.pdf}
%   \caption{Training time of \method, which saves 74.8\% of time on average.}
%   \label{fig:time}
% \end{minipage} \hfill
% \begin{minipage}{0.68\textwidth}
%   \includegraphics[width=\textwidth]{figure/seq.pdf}
%   \caption{Performance of sequential unlearning on ToolAlpaca.}
%   \label{fig:seq}
% \end{minipage}
% \end{figure}


% \section*{Ethics Statement}
% Our research focuses on mitigating dataset biases in NLP datasets. There are no specific ethical concerns directly associated with this work. However, we recognize the broader ethical considerations that apply to NLP research. 

\section*{Impact Statement}
Our work investigates the security implications of tool-augmented Large Language Models (LLMs), where we focus on the risks that arise from integrating external tools, and the necessity ability to remove these acquired tools. 
A key concern is ensuring compliance with privacy regulations, such as the Right to be Forgotten (RTBF), which mandates the removal of specific data upon user request. In the context of tool-augmented LLMs, this necessitates the ability to delete sensitive, regulated, or outdated information related to specific tools. 
By examining how LLMs interact with and rely on external tools, potential threats to model security can be identified, e.g. unauthorized tool usage, adversarial exploitation, and privacy violations. Our research highlights the critical importance of addressing these challenges.

\bibliography{reference,anthology}
\bibliographystyle{iclr2025_conference}

\newpage
\appendix
\onecolumn

\section{Data preprocessing details}
\label{appendix_dataset}
\subsection{Medical Codes Dataset Creation}
The medical codes dataset consists of medical codes, their descriptions, and associated knowledge subgraphs, encompassing eight commonly used health coding systems: ICD-9-CM (procedures and diagnoses), ICD-10-CM, ICD-10-PCS, NDC (National Drug Codes), SNOMED CT, ATC (Anatomical Therapeutic Chemical Classification), CPT (Current Procedural Terminology), and RxNorm. All code lists were obtained from official sources. Specifically, ICD-9 and ICD-10 (CM and PCS) were sourced from the CMS website; NDC codes from the U.S. Food and Drug Administration (FDA) database; and CPT (Level I HCPCS) from the Physician Fee Schedule (PFS) Relative Value Files at CMS. SNOMED CT, RxNorm (active codes only), and ATC were downloaded via the National Library of Medicine (NLM), part of the National Institutes of Health (NIH).

\subsubsection{Medical Codes Knowledge Graphs Creation}
In the final dataset, each medical code is linked to a knowledge graph capturing relevant medical insights and relationships. We constructed these subgraphs in two steps: mapping each code to one or more nodes in the PrimeKG knowledge graph; and extracting node-centered subgraphs to represent the code’s associated knowledge and connections.
To facilitate mapping, we leveraged several external resources, notably the UMLS database and MONDO Disease Ontology files. Medical codes were first mapped to Concept Unique Identifiers (CUIs) in the UMLS database, then linked to PrimeKG nodes via a custom UMLS-to-PrimeKG file. Because PrimeKG includes MONDO annotations, we also aligned medical codes to MONDO terms using the mondo.owl file, thus achieving direct integration with PrimeKG nodes. Additionally, a custom entity linker was employed to enhance coverage by translating medical codes into descriptive text (via PyHealth’s MedCode InnerMap) and matching these descriptions to PrimeKG node names. When exact matches were unavailable, we resorted to an NLP-based linker (SciSpacy with UMLS) to measure semantic similarity. For drug codes, the rxnav.nlm.nih.gov API was used to map RxNorm codes to ATC identifiers, which were then associated with DrugBank entities through a predefined ATC-to-DrugBank mapping.

\subsubsection{Medical Codes Textual Definition Creation}
Initially, each medical code’s description was taken from its official source. For medication codes (e.g., NDC) where the original text was sparse, additional details were derived from attributes such as trade name, proprietary name, and pharmacological classification. These preliminary definitions were then refined and enriched using ChatGPT-4 (turbo), with prompts tailored to each coding system but sharing a common goal of elaborating on clinical uses (for drugs), procedural steps (for procedures), or mechanistic and clinical context (for diagnoses).

\section{Implementation details}\label{appendix_implementation}

\subsection{Experimental environments}
\xhdr{Hardware} \model is training on a machine equipped with 4 NVIDIA H100. All experiments were conducted with 1 NVIDIA H100.

\paragraph{Software.} We implement \model using Python 3.9.19, PyTorch 2.3.1, Transformers 4.43.1, and Tokenizers 0.19.1. All LMs and LLMs adopted in this study are downloaded from Hugging Face, except for OpenAI models.

\subsection{Details in \model training}
\model is trained on 4 NVIDIA H100 GPUs by using the loss defined in the Section 3.2. During the training stage, we set the training step as 3000 with a global batch size of 1024, the dimension of quantized vectors is 64. In terms of the models' weights, we freeze the text encoder in \model and the graph encoder is trainable during the training stage. 

\subsection{Implementation details of baseline models}
All results presented in this study were obtained using the same machine on which the \model was trained.

ETHOS experiments were conducted using the authors’ original repository. For each experimental setting, three models were trained on the MIMIC-IV dataset with different random seeds, and their predictions were averaged during inference to ensure robustness. In the "\model + ETHOS" configuration, the original vocabulary was extended to incorporate \model's tokens for diagnoses, procedures, and prescriptions. The lab measurements were excluded from the analysis.
Training and dataset splitting on MIMIC-IV adhered to the methodology outlined in the ETHOS paper \cite{ethos}. During inference, the number of generated tokens was limited to 2048, and the timeline duration was adjusted based on the specific task: fifteen days for readmission, two weeks for mortality, and up to six months for other tasks. Each model was executed five times, and the resulting predictions were averaged to produce a continuous output, as described in the ETHOS study.
Inference on the MIMIC-III dataset was performed on the entire dataset, excluding BMI, ICU stay tables, blood pressure, and lab data. For the EHRShot dataset, inference was conducted on the full dataset for mortality and disease-related tasks, and on randomly selected, stratified samples of ten thousand instances for other tasks.

As for the other baselines adopted in this work, we first downloaded their code and deploy these models on our working machine. For BEHRT and GT-BEHRT, we re-trained it in an end-to-end way and integrates the tokens for time, visit, and patient's info as that in their original work. For MulT-EHR, we first pre-train it on MIMIC-III, MIMIC-IV, and EHRShot, respectively, to get the embedding of medical codes, and next fine-tune it on multi-task learning. For the \model+, we use our token embeddings to initialize the nodes or tokens the original work adopted and then train or pre-train the model. It should be noted that we adopt a unified epoch number for all baselines, which is 50.

\section{Task definitions and data preparation under in-patient setting}\label{appendix_task_definitions}

\subsection{Mortality prediction}
\xhdr{Task definition}
Mortality (MT) prediction estimates the mortality label of the 
\emph{subsequent} visit for each sample, with the last sample dropped.
Formally,
\[
    f : (v_1, v_2, \ldots, v_{t-1}) \;\to\; y[v_t],
\]
where \(y[v_t] \in \{0, 1\}\) is a binary label indicating the patient’s 
survival status recorded in visit \(v_t\).

\subsection{Readmission prediction}
\xhdr{Task definition}
Readmission prediction checks if the patient will be readmitted
to the hospital within \(\sigma\) days. Formally, $f : (v_1, v_2, \ldots, v_{t-1}) \;\to\; y\bigl[\tau(v_t) - \tau(v_{t-1})\bigr]$,
where \(y \in \{0, 1\}\) and \(\tau(v_t)\) denotes the encounter time of visit
\(v_t\). Specifically,
\[
    y\bigl[\tau(v_t) - \tau(v_{t-1})\bigr] \;=\;
    \begin{cases}
        1 & \text{if } \tau(v_t) - \tau(v_{t-1}) \le \sigma,\\
        0 & \text{otherwise}.
    \end{cases}
\]
In our study, we set \(\sigma = 15\) days.

\subsection{Length-of-Stay (LOS) prediction}

\xhdr{Task definition}
Length-of-Stay (LOS) prediction follows the formulation of Harutyunyan et al., estimating ICU stay length for each visit. Formally, $f : (v_1, v_2, \ldots, v_t) \to y[v_t]$, where $y[v_t] \in \mathbb{R}^{1 \times C}$ is a one-hot vector indicating its class among $C$ possible categories. We define 10 classes, $\{0,1,\ldots,7,8,9\}$, representing the following durations: 0 for one day or less, 1-7 for within one week, 8 for one to two weeks, and 9 for at least two weeks.

\subsection{Phenotype prediction}
\xhdr{Task definition}
Phenotype prediction aims to classify which acute care conditions are present in a given patient record: $f: (v_1, v_2, ..., v_t) \rightarrow y[v_t]$, where $y[v_t] \in \mathbb{R}^{1 \times C}$ is a one-hot vector indicating its class among $C$ possible categories. This task is a multilable classification problem with macro-averaged AUC-ROC being the main metric.

\subsection{Drug recommendation}
\xhdr{Task definition}
Drug recommendation aims to recommend drugs for a patient according to the patient's visit history and diagnosis in current visit: $f: (v_1, v_2, ..., v_t) \rightarrow y[v_t]$, where $y[v_t] \in \mathbb{R}^{1 \times C}$ is a one-hot vector indicating its class among $C$ possible categories. This task is a multilable classification problem with macro-averaged AUC-ROC being the main metric.

\xhdr{Data preprocessing}
In this study, we adopted a data preprocessing approach similar to that used in previous research (https://doi.org/10.1038/s41597-019-0103-9), which defined 25 acute care conditions.  Each diagnosis code was mapped to one of these 25 phenotype categories. Since ICD-9 codes in MIMIC-III are associated with hospital visits rather than specific ICU stays, we linked diagnoses to ICU stays using the hospital admission identifier. To reduce ambiguity, we excluded hospital admissions involving multiple ICU stays, ensuring that each diagnosis corresponded to a single ICU stay per admission. It's important to note that our phenotype classification was retrospective; we analyzed the complete ICU stay before predicting the presence of specific diseases. 
In this study, we adopted a data preprocessing approach similar to that used in previous research (https://doi.org/10.1038/s41597-019-0103-9), which defined 25 acute care conditions.  Each diagnosis code was mapped to one of these 25 phenotype categories. Since ICD-9 codes in MIMIC-III are associated with hospital visits rather than specific ICU stays, we linked diagnoses to ICU stays using the hospital admission identifier. To reduce ambiguity, we excluded hospital admissions involving multiple ICU stays, ensuring that each diagnosis corresponded to a single ICU stay per admission. It's important to note that our phenotype classification was retrospective; we analyzed the complete ICU stay before predicting the presence of specific diseases. 

\subsection{Out-patient Setting}
Under this setting, we adopt two types of tasks in EHRShot, including operational outcomes prediction and assignment of new diagnosis. In the field of operational outcomes, we follow the same task definitions in long length of stay prediction, which only consider if a patient stay in the hospital less than 7 days or more than 7 days. In terms of readmission task, we set the time window as 15 days, which is the same as that under in-patient setting. We also add another operational outcome tash, which is mortality prediction. The definition of mortality prediction is the same as that under the in-patient setting.




\end{document}