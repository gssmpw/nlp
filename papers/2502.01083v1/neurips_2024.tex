\documentclass{article}


% if you need to pass options to natbib, use, e.g.:
%     \PassOptionsToPackage{numbers, compress}{natbib}
% before loading neurips_2024


% ready for submission
\usepackage[preprint]{neurips_2024}


% to compile a preprint version, e.g., for submission to arXiv, add add the
% [preprint] option:
%     \usepackage[preprint]{neurips_2024}


% to compile a camera-ready version, add the [final] option, e.g.:
%     \usepackage[final]{neurips_2024}


% to avoid loading the natbib package, add option nonatbib:
%    \usepackage[nonatbib]{neurips_2024}


\usepackage[utf8]{inputenc} % allow utf-8 input
\usepackage[T1]{fontenc}    % use 8-bit T1 fonts
\usepackage{hyperref}       % hyperlinks
\usepackage{url}            % simple URL typesetting
\usepackage{booktabs}       % professional-quality tables
\usepackage{amsfonts}       % blackboard math symbols
\usepackage{nicefrac}       % compact symbols for 1/2, etc.
\usepackage{microtype}      % microtypography
\usepackage{xcolor}         % colors

\usepackage{amsmath}
\usepackage{amssymb}
\usepackage{mathtools}
\usepackage{amsthm}
\usepackage{multirow}
% if you use cleveref..
\usepackage[capitalize,noabbrev]{cleveref}

%%%%%%%%%%%%%%%%%%%%%%%%%%%%%%%%
% THEOREMS
%%%%%%%%%%%%%%%%%%%%%%%%%%%%%%%%
\theoremstyle{plain}
\newtheorem{theorem}{Theorem}[section]
\newtheorem{proposition}[theorem]{Proposition}
\newtheorem{lemma}[theorem]{Lemma}
\newtheorem{corollary}[theorem]{Corollary}
\theoremstyle{definition}
\newtheorem{definition}[theorem]{Definition}
\newtheorem{assumption}[theorem]{Assumption}
\theoremstyle{remark}
\newtheorem{remark}[theorem]{Remark}

% Todonotes is useful during development; simply uncomment the next line
%    and comment out the line below the next line to turn off comments
%\usepackage[disable,textsize=tiny]{todonotes}
\usepackage[textsize=tiny]{todonotes}




\usepackage{xspace}
% \newcommand{\tool}{$\mathcal{T}$\xspace}
\newcommand{\ttest}{$\mathcal{T}_T$\xspace}
\newcommand{\tf}{$\mathcal{T}_f$\xspace}
\newcommand{\tr}{$\mathcal{T}_r$\xspace}
\newcommand{\tg}{$\mathcal{T}_G$\xspace}
\newcommand{\RET}{\textsc{Retrain}\xspace}
\newcommand{\GA}{\textsc{GradAscent}\xspace}
\newcommand{\RL}{\textsc{RandLabel}\xspace}
\newcommand{\BT}{\textsc{Bad-T}\xspace}
\newcommand{\SU}{\textsc{SalUn}\xspace}
\newcommand{\SO}{\textsc{SOUL-GradDiff}\xspace}
\newcommand{\method}{\textsc{ToolDelete}\xspace}
\usepackage{colortbl}
\definecolor{Gray}{RGB}{192,192,192}
\definecolor{lightred}{RGB}{255,179,179}

\usepackage[many]{tcolorbox}
\definecolor{main}{HTML}{5989cf}    % setting main color to be used
\definecolor{sub}{HTML}{cde4ff}     % setting sub color to be used

\tcbset{
    sharp corners,
    colback = white,
    before skip = 0.2cm,    % add extra space before the box
    after skip = 0.5cm      % add extra space after the box
} 
\newtcolorbox{bluebox}{
    colback = sub, 
    colframe = main, 
    boxrule = 0pt, 
    leftrule = 6pt % left rule weight
}

\title{Tool Unlearning for Tool-Augmented LLMs}


% The \author macro works with any number of authors. There are two commands
% used to separate the names and addresses of multiple authors: \And and \AND.
%
% Using \And between authors leaves it to LaTeX to determine where to break the
% lines. Using \AND forces a line break at that point. So, if LaTeX puts 3 of 4
% authors names on the first line, and the last on the second line, try using
% \AND instead of \And before the third author name.


\author{%
  Jiali Cheng \quad Hadi Amiri\\
  University of Massachusetts Lowell\\
  \texttt{\{jiali\_cheng, hadi\_amiri\}@uml.edu} \\
  % examples of more authors
  % \And
  % Coauthor \\
  % Affiliation \\
  % Address \\
  % \texttt{email} \\
  % \AND
  % Coauthor \\
  % Affiliation \\
  % Address \\
  % \texttt{email} \\
  % \And
  % Coauthor \\
  % Affiliation \\
  % Address \\
  % \texttt{email} \\
  % \And
  % Coauthor \\
  % Affiliation \\
  % Address \\
  % \texttt{email} \\
}


\begin{document}


\maketitle



Hypotheses are central to information acquisition, decision-making, and discovery. However, many real-world hypotheses are abstract, high-level statements that are difficult to validate directly. 
This challenge is further intensified by the rise of hypothesis generation from Large Language Models (LLMs), which are prone to hallucination and produce hypotheses in volumes that make manual validation impractical. Here we propose \mname, an agentic framework for rigorous automated validation of free-form hypotheses. 
Guided by Karl Popper's principle of falsification, \mname validates a hypothesis using LLM agents that design and execute falsification experiments targeting its measurable implications. A novel sequential testing framework ensures strict Type-I error control while actively gathering evidence from diverse observations, whether drawn from existing data or newly conducted procedures.
We demonstrate \mname on six domains including biology, economics, and sociology. \mname delivers robust error control, high power, and scalability. Furthermore, compared to human scientists, \mname achieved comparable performance in validating complex biological hypotheses while reducing time by 10 folds, providing a scalable, rigorous solution for hypothesis validation. \mname is freely available at \url{https://github.com/snap-stanford/POPPER}.




\section{Introduction}
% Large Language Models (LLMs) serve as the foundation for a wide range of tasks. 
% Recently researchers have developed methods to equip large language models (LLMs) with external tools, 

Tool-augmented Large Language Models (LLMs) can use external tools such as calculators~\citep{schick2023toolformer}, 
Python interpretors~\citep{pal}, 
APIs~\citep{tang2023toolalpaca}, or 
AI models~\citep{patil2023gorilla} to complement the parametric knowledge of vanilla LLMs and enable them to solve more complex tasks~\citep{schick2023toolformer,patil2023gorilla}. They are often trained on query-response pairs, which embed the ability to use tools {\em directly} into parameters.\looseness-1
% For example, WebGPT~\citep{webgpt} extends GPT-3~\citep{gpt3} to use search engines, especially useful for events that occurred {\em after} GPT-3 was trained.
% and retrieve up-to-date information to improve GPT-3's performance in question answering, 


Despite the growing adoption of tool-augmented LLMs, the ability to selectively unlearn tools has not been investigated. In real-world applications, tool unlearning is essential for addressing critical concerns such as security, privacy, and model reliability. 
For example, consider a tool-augmented LLM deployed in a healthcare system and trained to use APIs for handling patient data. If one of the APIs is later flagged as insecure due to a vulnerability that could expose sensitive information and violate regulations like HIPAA, tool unlearning is necessary to ensure that the LLM can no longer invoke the insecure API. Similarly, when tools undergo major updates, such as the Python transformers package moving from version 3 to version 4, tool unlearning becomes essential to prevent the LLM from generating outdated or erroneous code.
% For example, if a tool-augmented LLM retains knowledge of making insecure HTTP requests, it will cause significant security risks and can become vulnerable to attacks.\footnote{\url{https://datatracker.ietf.org/doc/html/rfc7807}}
The goal of this work is to address this gap by investigating tool unlearning and providing a solution for this overlooked yet essential task.

% Additional scenarios are discussed in \S~\ref{sec:app}. In the aforementioned case, it is necessary for a tool-augmented model to forget its acquired knowledge of using certain tools--an area that has not yet been explored by existing research.
% Consider the following practical scenarios: 1) \emph{Insecure Tools}, where non-trustworthy tools need to be deleted, 2) \emph{Restricted Tools}, where tools may become unavailable due to copyright issues; 3) \emph{Broken Tools/Dependencies}, where tools may become broken, deprecated, or fall out of maintenance; 4) \emph{Unnecessary Tools}, where the requirement for certain tools may no longer be needed; and 5) \emph{Limited Model Capacity}, where the tool-augmented LLM meets capacity limitations. 


% \paragraph{A new task}
We introduce and formalize the new task of \textbf{Tool Unlearning}, which aims to remove the ability of using specific tools from a tool-augmented LLM while preserving its ability to use other tools and perform general tasks of LLMs such as coherent text generation. 
% This is essential for complying with tool deletion requests that often target a small subset of tool. 
Ideally, an effective tool unlearning model should behave as if it had never learned the tools marked for unlearning. 
% When tool deletion requests are received, a successful tool unlearning algorithm should effectively remove knowledge of the targeted tools, as if the model had never encountered them. At the same time, the model’s knowledge of remaining tools and its ability to perform other tasks should be preserved to the greatest extent possible. This is crucial because deletion requests typically focus on a specific subset of tools, which is usually much smaller than the entire tool set.
% \paragraph{Difference to sample unlearning}
Tool unlearning fundamentally differs from traditional sample-level unlearning as it focuses on removing ``skills'' or the ability to use specific tools, rather than removing individual data samples from a model. In addition, success in tool unlearning should be measured by the model’s ability to forget or retain tool-related skills, which differs from traditional metrics such as measuring likelihood of extracting training data in sample-level unlearning.
% While sample-level unlearning focuses on reducing the likelihood of extracting training data, tool unlearning aims to forget the capability to solve tasks that rely on the tools tagged for unlearning, which can be seen as knowledge-level unlearning. (2) Evaluation: Sample unlearning typically uses perplexity or extraction probability as evaluation metrics. In contrast, tool unlearning prioritizes the success rate of using specific tools, ensuring that the model can no longer effectively use the tools targeted for unlearning. (3) Data: Sample unlearning typically requires access to the exact training data, which may not be available in tool unlearning, especially when dealing with closed-source LLM training data. 
These differences are discussed in detail in~\S\ref{sec:diff}.


% \paragraph{Challenge of tool unlearning}
% as opposed to individual data samples, which makes it fun
% and existing unlearning methods are not fundamentally designed for tool removal; 
% 
% similar to sample-level unlearning, in tool unlearning, 
Removing skills requires  
modifying the parameters of LLMs, a process that is computationally expensive and can lead to unforeseen behaviors~\citep{ripple_effect,gu2024model}. In addition, existing membership inference attack (MIA) techniques, a common evaluation method in machine unlearning to determine whether specific data samples were part of training data, are inadequate for evaluating tool unlearning, as they focus on sample-level data rather than tool-based knowledge. 
% and practically difficult due to potential unforeseeable side effects on other tasks when updating LLM's parameters~\citep{ripple_effect,gu2024model}. 
% Additionally, there is no prior Membership Inference Attack (MIA) models, a desired evaluation of unlearning, designed to detect if a tool is present in training set.  


To address these challenges, we propose \method, the first tool unlearning algorithm for tool-augmented LLMs, which satisfies three key properties for effective tool unlearning: 
{\em tool knowledge removal}, which focuses on removing any knowledge gained on tools marked for unlearning; 
{\em tool knowledge retention}, which focuses on preserving the knowledge gained on other remaining tools; and 
{\em general capability retention}, which maintains LLM's general capability on a range of general tasks such as text and code generation using ideas from task arithmetic~\citep{ilharco2023editing,barbulescu2024textual}.
%
In addition, we develop LiRA-Tool, an adaptation of the Likelihood Ratio Attack (LiRA)~\citep{lira} to tool unlearning, to assess whether tool-related knowledge has been successfully unlearned. Our contributions are: 

% When receiving deletion requests, a successful tool unlearning algorithm should remove the knowledge of the tools marked for unlearning, as if the model has never seen such tools before. Meanwhile, the model's knowledge on the remaining tools as well as other tasks should be preserved to the maximum extent. This is important since the deletion request are targeted at specific subset of tools, usually much smaller than the entire tool set, and practically difficult due to potential unforeseeable side effects.

\vspace{-10pt}
\begin{itemize}
\itemsep-1pt
    \item introducing and conceptualizing tool unlearning for tool-augmented LLMs,
    \item \method, which implements three key properties for effective tool unlearning;
    \item LiRA-Tool, which is the first membership inference attack (MIA) for tool unlearning.
\end{itemize}


Extensive experiments on multiple datasets and tool-augmented LLMs show that \method outperforms existing general and LLM-specific unlearning algorithms by $+$ in accuracy on forget tools and retain tools.  In addition, it can save 74.8\% of training time compared to retraining, handle sequential unlearning requests, and retain 95+\% performance in low resource settings.\looseness-1
\begin{figure*}[t]
  \centering
  \includegraphics[width=0.9\textwidth]{figure/tool_unlearning.pdf}
  \vspace{-10pt}
  \caption{Tool Unlearning and the proposed \method approach. \textbf{(a)}: Illustration of tool learning and tool unlearning. Learned tools may be requested to be unlearned due to many reasons, such as tools being insecure, restricted, or deprecated. \textbf{(b)}: Differences between tool unlearning and traditional sample unlearning, in terms of objective and training data. \textbf{(c)}: Proposed method \method. We encourage the unlearned model $f'$ to follow the tool-free LLM $f_0$ which has never seen $T_f$ before. Meanwhile, we maintain its ability on $T_r$ and general tasks by matching the capabilities of tool-augmented model $f$ through task arithmetic.}
  \label{fig:model}
    \vspace{-10pt}
\end{figure*}
% https://www.flaticon.com/search?word=tool

\section{Tool Unlearning: Preliminaries}\label{sec:prem} %: A Novel Machine Unlearning Task}

To understand tool unlearning, we first introduce the concept of ``tool learning,'' see Figure~\ref{fig:model}(a). Let $\mathcal{D} = \{ \mathcal{T}, \mathcal{Q}, \mathcal{Y} \}$ be a dataset with $N$ tools $\mathcal{T}$, and $(\mathcal{Q}, \mathcal{Y})$ denotes query-output examples that demonstrate how to use the tools in $\mathcal{T}$. 
% and corresponding ``demonstrations'' $(\mathcal{Q}, \mathcal{Y})$, where $\mathcal{Q}$ is the query set and $\mathcal{Y}$ is its corresponding output set--$(\mathcal{Q}, \mathcal{Y})$ are labeled examples demonstrating how to use tools in $\mathcal{T}$. 
Each tool $t_i \in \mathcal{T}$ may have one or more demonstrations $\{\mathcal{Q}_i, \mathcal{Y}_i\}$, $|\mathcal{Q}_i| = |\mathcal{Y}_i| \geq 1$. 
Starting with an instruction-tuned LLM $f_0$, 
% which has not been trained on using tools, 
a tool learning algorithm explicitly trains $f_0$ on $\mathcal{D}$ and results in a {\em tool-augmented} model $f$ capable of using the $N$ tools in $\mathcal{T}$. We note that prior to explicit tool learning, the LLM $f_0$ may already have some tool-using capabilities such as performing basic arithmetic operations. 

% An example of tool-augmented models is WebGPT~\citep{webgpt}, which mimics human behavior in answering open-ended questions using a text-based web browser to retrieve information and improve its responses.


\paragraph{Problem Definition:} Tool unlearning aims to remove specific tools from tool-augmented LLMs. Let $\mathcal{D}_f = \{ \mathcal{T}_f, \mathcal{Q}_f, \mathcal{Y}_f \}$ denotes $k < N$ tools and their corresponding demonstrations to be unlearned from the tool-augmented model $f$, and $\mathcal{D}_r = \mathcal{D} \backslash \mathcal{D}_f = \{ \mathcal{T}_r, \mathcal{Q}_r, \mathcal{Y}_r \}$ denotes the remaining tools and their demonstrations to retain. The goal is to obtain an unlearned model $f'$ that has limited knowledge on using $\mathcal{T}_f$ tools--can no longer perform tasks involving $\mathcal{T}_f$ tools--while preserving $f$'s ability to use $\mathcal{T}_r$ tools as before.
% , i.e. on how to solve tasks depending on $T_r$ as prior to unlearning. 

\paragraph{Use Cases of Tool Unlearning}\label{sec:app}
The ability to forget learned tools is essential in real-world applications. For example, 
addressing the insecure tools from untrustworthy developers that could be exploited by adversarial attackers;
% . Examples include cyber attack tools or unaligned AI models that generate harmful content. Adversarial attackers may exploit such tools to compromise the security of models and privacy of users. 
%
removing tools restricted by their providers due to copyright or privacy concerns, such as APIs that start allowing unauthorized downloads of book chapters or releasing publications that users did not author; 
%
unlearning broken or deprecated tool that lead to failed operations or corrupted outputs;
%
unlearning tools that may no longer be needed; 
%
and managing limited model capacity, where new versions of tools necessitate replacing outdated ones. More examples of parameter-level tool unlearning are provided in Appendix~\ref{sec:example}.

% : The tools a model needs to master may change over time due to capacity limitations or evolving requirements.


% \begin{itemize}
%     \item Insecure Tools: Many tools are not created, maintained, or published by well-established, trustworthy developers. Examples include cyber attack tools or unaligned AI models that generate harmful content. Adversarial attackers may exploit such malicious tools, compromising the security of models and users.
    
%     \item Restricted Tools: Tools or their associated usage examples may become unavailable due to restrictions imposed by data providers, including copyright issues or the Right to Be Forgotten (RTBF). Examples include APIs that download unauthorized book chapters or announce publications that users did not author.
    
%     \item Broken Tools/Dependencies: Tools may become broken, deprecated, or fall out of maintenance. Continuing to rely on these tools can result in undesired behavior, such as failed tool calls and corrupted outputs.
    
%     \item Useless Tools: The requirement for certain tools may be temporary. In other words, some tools may no longer be needed at a specific point in time, leading to unlearning requests in this scenario.
    
%     \item Limited Model Capacity: The tools a model needs to master may change over time due to capacity limitations or evolving requirements.
% \end{itemize}


\paragraph{Difference to Standard Unlearning Tasks} \label{sec:diff}
Tool unlearning is different from sample-level unlearning as it focuses on removing ``skills'' rather than individual training samples. 
%
\textbf{Objective}: sample-level unlearning aims to reduce the memorization likelihood or extraction probabilities of specific data samples $(q_i, y_i)$~\citep{jang-etal-2023-knowledge}, which is useful for removing copyrighted or private information. In contrast, tool unlearning targets the ``ability'' to solve tasks using tools marked for unlearning ($T_f$). For example, generating $f'(q_i)$ that is superficially different from $y_i$ (while preserving the semantics) is considered successful for sample-level unlearning. However, for tool unlearning, preserving skills and semantics indicate maintained knowledge on $T_f$, which makes unlearning a failure. Figure~\ref{fig:model}b shows successful tool unlearning, where the ability to use the API is forgotten, despite the high lexical memorization between output of the unlearned model and the training data.
% overlap. 
% This difference also lead to different evaluation methods, which we detail in \S\ref{sec:experiment}. 
In addition, selectively removing knowledge from tool-augmented models is a challenging tasks because changes to one tool may unexpectedly affect the model's ability to use other tools--referred to as {\em ripple effect} in fact editing literature~\citep{ripple_effect,gu2024model}. Furthermore, LLMs are general models that can conduct a wide range of tasks beyond tool using, and this ability must be retained. 
%
\textbf{Evaluation}: metrics like sequence extraction likelihood and perplexity are standard in sample-level unlearning. For tool unlearning, success is measured by the ability to forget or retain tool-related skills, which is more appropriate.
%
\textbf{Data}: sample-level unlearning require access to all individual samples marked for unlearning, while tool unlearning does not. This aligns with ``concept erasure'' in diffusion models~\citep{gandikota2023erasing,kumari2023conceptablation} and zero-shot unlearning~\citep{chundawat2022zero} but differs from traditional LLM unlearning~\citep{yao-etal-2024-machine}. Later we demonstrate this in \S~\ref{sec:no_training_data}.
% As long as the target tools are unlearned, we can use any dataset or choose not to use any data. 


\paragraph{Importance of Parameter-Level Tool Unlearning}
We observe that one can naively block tools at the prompt-level or remove tools from the tool set without updating the LLM. However, these shortcut solutions are insufficient to remove tool knowledge. 
\emph{Firstly}, the knowledge on $\mathcal{T}_f$ persists in the parameters of $f'$, leaving the LLM still under threat. Adversarial agents / attackers can exploit this knowledge, which also bypasses prompt-level restrictions. Since existing LLMs do not guarantee 100\% adherence to instructions or contextual information~\citep{zhou2023instruction,zeng2024evaluating}, they may ignore the tool set provided in the prompt and answer queries with their parametric knowledge~\citep{goyal-etal-2023-factual}. 
In addition, tool unlearning at prompt level can create conflicts between the model's parametric knowledge and contextual information. This may lead to misinformation, hallucination, and other unpredictable behavior~\citep{xu2024knowledge}. Finally, we show in the experiments that prompt-level tool unlearning is indeed insufficient, see Table~\ref{tab:main} (ICLU model), which aligns with existing works on LLM unlearning, where parameter update is required~\citep{jia-etal-2024-soul,zhang2024negative}.


% \paragraph{Retraining: An Impractical Solution}
% A straightforward solution is to delete $\mathcal{D}_f$ from $\mathcal{D}$ and retrain a new model only on $\mathcal{D}_r$. However, this is often infeasible due to the high cost and potential unavailability of the original training data~\citep{NEURIPS2023_299a08ee,ilharco2023editing,Gandikota_2023_ICCV}. In addition, unlearning should not be evaluated \emph{solely} based on similarity to retraining as the potential solution space is highly complex and multidimensional. Specifically, prior work has shown that relying on similarity to retraining has several drawbacks, such as poor auditability~\citep{thudi2021necessity} and ineffective deletion~\citep{cheng2023gnndelete,cheng2023multimodal}.
% % Major drawback of merely comparing to retraining include lack of auditability~\citep{thudi2021necessity}, poor practical deletion performance~\citep{cheng2023gnndelete,cheng2023multimodal}, and high dimensionality of possible solutions. 
% % mere resemblance to retraining is not sufficient nor necessary for evaluating unlearning and deciding optimal performances in practice. 
% % Resemblance to retraining is not sufficient nor necessary for optimal unlearning performances in practice. 
% % such as lack of auditability~\citep{thudi2021necessity}
% Therefore there is a need for designing specialized and efficient unlearning methods for tool-augmented models.


% Ideally, an effective tool unlearning algorithm should consider the following aspects
% \begin{itemize}
%     \item Unlearning should not affect LLM's general utility in tasks unrelated to the tool usage, i.e. $f'$ can be used as a general aligned LLM to solve tasks that $f$ is good at, such as generating text or answering general questions; 
%     % , drafting emails
%     % or math problems.
%     % \item Unlearning should not affect mode switching ability of LLMs (switching from reasoning to tool using).
%     \item Unlearning should not affect LLM's ability on $T_r$, i.e. $f'$ should maintain $f$'s capabilities in using the remaining tools that are not expected to be unlearned.

%     \item x
% \end{itemize}


% things still unclear. 
% \section{\method: Effective Tool Unlearning for LLMs}






\section{\method} %: Effective Tool Unlearning for LLMs


We develop \method--an effective tool unlearning approach that removes the capability of using tools marked for unlearning ($\mathcal{T}_f$) or solving tasks that depend on them, while preserving the ability of using the remaining tools ($\mathcal{T}_r$) and performing general tasks such as text and code generation. \method implements three key properties for effective tool unlearning: \looseness-1

% \subsection{Required Properties for Effective Tool Unlearning}
\subsection{Tool Knowledge Deletion}
% The tool-augmented model $f$ gains its knowledge of \tf through tool learning. 
Unlearning requires completely removing the knowledge of \tf that $f$ gained during tool learning, ideally as if \tf had never been part of the training set. In other words, knowledge about \tf is successfully removed if the unlearned model $f'$ has no more knowledge than the tool-free model $f_0$ about \tf. \looseness-1

\begin{definition}[Tool Knowledge Deletion (TKD)]
Let $t_i \in \mathcal{T}_f$ denote a tool to be unlearned and $g$ be a function that quantifies the amount of knowledge a model has about a tool. The unlearned model $f'$ satisfies tool knowledge deletion if:
\begin{equation}\label{eq:prop1}
    \mathop\mathbb{E}_{t_i \in \mathcal{T}_f} [ g(f_0, t_i) - g(f', t_i) ] \geq 0.
\end{equation}
\end{definition}
% so that $f'$ retains no more knowledge of $T_f$ than $f_0$.
This formulation allows users to control the extent of knowledge removal from $f'$. For instance, when we unlearn a ``malicious'' tool that calls a malignant program, we may require $f'$ retains no knowledge of this tool, i.e. $g(f', t_i) = 0$. In less critical cases, users can choose to reset $f'$'s knowledge to {\em pre}-tool augmentation level, i.e. $g(f', t_i) = g(f_0, t_i)$

To measure tool knowledge in LLMs, we follow previous works that used prompting to probe LLMs' knowledge~\citep{gpt3,singhal2023large}, i.e. adopting the output of LLMs as their knowledge on a given tool. For each $t_i \in \mathcal{T}_f$ and its associated demonstrations $\{ \mathcal{Q}_i, \mathcal{Y}_i \}$, we query the tool-free LLM $f_0$ with $\mathcal{Q}_i$ and collect its responses $\mathcal{Y}'_i = f_0(Q_i)$. Since $f_0$ has never seen $t_i$ or $\{ \mathcal{Q}_i, \mathcal{Y}_i \}$, $\mathcal{Y}'_i$ represents the \textbf{tool-free response}. We then constrain the unlearned model $f'$ to generate responses similar to $\mathcal{Y}'_i$ to prevent it from retaining knowledge of $t_i$.


% \paragraph{Various levels of knowledge removal}
% Knowledge removal can happen in different cases for tool learning / unlearning. 
% \begin{itemize}
%     \item do not choose any tool.
%     \item choose a different tool. 
%     \item choose the right tool but wrong arguments. 
%     \item say I don't know how to solve this task based on internal knowledge / learned tools. Depends on evaluation.
% \end{itemize}
% Note that the difference between 1) and 4) is that 1) predicts that $x_i$ requires no tool to solve. While 4) encourages the LLM to answer "I don't know.".

% Most existing tool-augmented LLMs are trained in a Supervised Fine-Tuning manner (SFT), where language modeling objective is optimized query-response pairs $(q_i, y_i)$.


\subsection{Tool Knowledge Retention}
The unlearning process should preserve model's knowledge of tools in $T_r$. Ideally, all knowledge gained on $T_r$ during tool learning should be retained after unlearning. 


\begin{definition}[Tool Knowledge Retention (TKR)]
Let $t_m \in T_r$ denote a retained tool, and let $g$ be a function that quantifies the amount of knowledge a model has about a tool. The unlearned model $f'$ satisfies tool knowledge retention if:\looseness-1
\begin{equation}
    \mathop\mathbb{E}_{t_m \in \mathcal{T}_r} [ g(f, t_m) - g(f', t_m) ] = \epsilon,
    \label{eq:prop2}
\end{equation} 
where $\epsilon$ is an infinitesimal constant, so that $f'$ retains the same knowledge of tools in $T_r$ as the original model $f$.
\end{definition}
For effective tool knowledge retention, $f'$ is further fine-tuned using demonstrations associated with $\mathcal{T}_r$, or, more practically, a subset of $\mathcal{T}_r$ proportional to $\mathcal{T}_f$ for efficiency.



\subsection{General Capability Retention via Task Arithmetic}
Optimizing the above objectives can lead to effective unlearning, but it may not be sufficient to maintain the general capabilities of the unlearned model $f'$. As a foundation model, $f'$ is expected to retain abilities such as text and code generation, question answering, instruction-following, and basic mathematical reasoning. These capabilities either existed in $f_0$ prior to tool augmentation or do not depend on specific tools. Therefore, preserving the general capabilities of $f'$ is essential to guarantee that tool unlearning does not compromise the overall functionality of the model. 

\begin{definition}[General Capability Retention (GCR)]
Let $\mathcal{T}_G$ denote the general tasks used to evaluate LLMs. The unlearned model $f'$ satisfies general capability retention if it preserves the knowledge on $T_G$ that it originally obtained prior to tool learning:
\begin{equation}
    \mathop\mathbb{E}_{t_g \in \mathcal{T}_G} [ g(f_0, t_g) - g(f', t_g) ] = \epsilon,
    \label{eq:prop3}
\end{equation} where $\epsilon$ is an infinitesimal constant.
\end{definition}

We propose to use task arithmetic~\citep{ilharco2023editing,barbulescu2024textual} as an efficient and effective approach to preserving the general capabilities of the unlearned model. Our objective is that $f'$ retains as much general knowledge as $f_0$, the instruction tuned LLM trained from a randomly initialized model $f_R$. 
Let $\theta_0$ and $\theta_R$ denote the parameters of $f_0$ and $f_R$ respectively. The difference vector $\theta_0 - \theta_R$ captures the direction of general knowledge acquisition. We apply this adjustment to $\theta'$ (the parameters of $f'$) to preserve its general knowledge:
\begin{equation}
    \theta'^* \leftarrow \theta' + (\theta_0 - \theta_R).
\end{equation}
% This approach allows $f'$ to retain its general capabilities while effectively unlearning specific tools.

% This can be achieved through additional pre-training or instruction tuning. However, such explicit training methods may pose two difficulties in practice. 1) The pre-training or instruction tuning dataset may not be easily accessible at the stage of tool unlearning, making explicit training impossible. 2) Training on a subset of pre-training / instruction tuning dataset together with $ D^0_f \cup D_r $ at the same time may become prohibitively expensive and difficult, given the distinctiveness of these datasets. To account for these difficulties, we adopt task arithmetic~\citep{} to maintain the general utilities of $f'$.

\paragraph{Why Task Arithmetic?}
% While knowledge in LLMs can be highly nonlinear, task arithmetic assumes a linear transformation in the parameter space, which may not always hold. In addition, the vector difference $(\theta_0 - \theta_R)$ may over-correct or under-adjust general abilities, which may leave some tool-related knowledge in the model.
%
% Despite these limitations, 
Task arithmetic is efficient, practical, effective for preserving general capabilities~\citep{ilharco2023editing,barbulescu2024textual}: 
\textbf{Efficiency}: the vector operation does not scale with dataset size, making it significantly more efficient than retraining on large datasets. 
% This cost does not scale with the size of the dataset, which can be considered as static and offline. While explicitly training on the pre-training and instruction-tuning datasets is significantly more expensive.
% 
\textbf{Practicality}: general capabilities obtained from pre-training and instruction tuning~\citep{zhou2024lima} are often impractical to replicate due to the size and limited availability of data--even in some open-source LLMs~\citep{touvron2023llama2}, the actual pre-training data is not fully open-source. In addition, reintroducing general knowledge from alternative datasets can lead to data imbalances and distributional biases. 
%
\textbf{Effectiveness}: applying $\theta_0 - \theta_R$ largely restores the foundational abilities of $f'$, such as text generation and instruction-following, without requiring expensive and time-consuming retraining on large datasets.



\subsection{Training Details}
To obtain the unlearned model $f'$, we solve:
\begin{equation}
    % \theta'^* = \arg \min_{\theta'} \\
    % \underbrace{\mathbb{E}_{t_i \in \mathcal{T}_f} [ g(f_0, t_i) - g(f', t_i) ]}_{\text{Optimization}} + \underbrace{\mathbb{E}_{t_m \in \mathcal{T}_r} [ g(f, t_m) - g(f', t_m) ]}_{\text{Optimization}} + \\ 
    % \underbrace{\alpha (\theta_0 - \theta_R)}_{\text{Task Arithmetic}},
    \theta'^* = \arg \min_{\theta'} \underbrace{\mathbb{E}_{t_i \in \mathcal{T}_f} [ g(f_0, t_i) - g(f', t_i) ]}_{\text{knowledge deletion of }\mathcal{T}_f} + \underbrace{\mathbb{E}_{t_m \in \mathcal{T}_r} [ g(f, t_m) - g(f', t_m) ]}_{\text{knowledge retention of }\mathcal{T}_r},
\end{equation} 
and once the optimized model parameters $\theta'^*$ are obtained, we apply task arithmetic to reinforce general capabilities:
\begin{equation}
    \theta'^* = \underbrace{\theta'^*}_{\text{post-optimization weights}} + \underbrace{\alpha (\theta_0 - \theta_R)}_{\text{knowledge retention of }\mathcal{T}_G},
\end{equation} 
where $\alpha$ is a hyperparameter to control the magnitude of task arithmetic. 
% The loss function $L$ depends on the specific training method. 
% Following previous works, $f'$ is initialized as $f$ to maintain maximum knowledge retention on $T_r$.
% , the specific choice of optimization method loss function depends on the training method to optimize for $\theta'^*$. 
The above formulation provides flexibility in training \method using various existing paradigms, including 
supervised fine-tuning (SFT)~\citep{alpaca}, 
direct preference optimization (DPO)~\citep{rafailov2023direct}, 
reinforcement learning from human feedback (RLHF)~\citep{ouyang2022training}, 
parameter-efficient fine-tuning (PEFT)~\citep{he2022towards,su-etal-2023-exploring}, or
quantization~\citep{8bit_quant,ma2024era} techniques. 
Below we describe two variants of \method:
% \vspace{-20pt}
\begin{itemize}
\itemsep0pt
    \item \textbf{\method-SFT} fine-tunes $f$ using language modeling loss. On forget tools $\mathcal{T}_f$, we replace the original responses $\mathcal{Y}_f$ with tool-free responses $\mathcal{Y}'_f$. The samples for $\mathcal{T}_r$ are not modified. 
    % Similar to prior SFT works, we only compute loss on the response and exclude the query part.

    \item \textbf{\method-DPO} uses direct preference optimization (DPO) to prioritize wining responses over losing responses. For $(t_i, \mathcal{Q}_i, \mathcal{Y}_i) \in \mathcal{T}_f$ to be unlearned, we prioritize the corresponding tool-free response $\mathcal{Y}'_i$ over the original response $\mathcal{Y}_i$. For $(t_j, \mathcal{Q}_j, \mathcal{Y}_j) \in \mathcal{T}_r$, the original response $\mathcal{Y}_j$ is prioritized over the tool-free response $\mathcal{Y}'_i$. \looseness-1
    % Therefore, the knowledge of the unlearned model $f'$ on $\mathcal{T}_f$ can be removed without affecting $\mathcal{T}_r$. 

\end{itemize}



\subsection{LiRA-Tool for Tool Unlearning Evaluation}

\paragraph{Challenge}
A key challenge in evaluating tool unlearning is the lack of membership inference attack (MIA) models to determine whether a tool has been truly unlearned. Existing MIA models typically evaluate individual training samples by analyzing model loss, which is insufficient for tool unlearning. 
% , i.e., in case of LLMs, the loss of responses for given queries. 
Unlike sample-level unlearning, tool unlearning focuses on removing abstract parametric knowledge of tools in $\mathcal{T}_f$, not just forgetting specific training samples. The key limitation of sample-based MIA is that the prompt-response pairs $(\mathcal{Q}_f, \mathcal{Y}_f)$ in the training set may not fully represent all aspects of a tool's functionality. As a result, sample-level MIA may ``overfit'' to a limited subset of tool related prompts and fail to holistically assess whether the tool-usage capability have been fully removed from the model's parametric knowledge.\looseness-1  

\paragraph{Solution}\label{sec:lira_tool}
To address the above limitation, we introduce ``shadow samples'', a diverse set of prompt-response pairs to probe various aspects of tool knowledge. 
We prompt GPT4 with different combinations of in-context examples to obtain a comprehensive set of prompt-response pairs with various prompt format, intention, and difficulty requirements. 
These samples will be used to stress-test the unlearned LLM $f'$ beyond the specific training prompts. This approach prevents overfitting to the original training data and provides a more reliable evaluation of whether the tool has truly been forgotten. To implement this, we extend Likelihood Ratio Attack (LiRA)~\citep{lira}, the state-of-the-art MIA approach, to tool unlearning.


\paragraph{Sample-level LiRA}
LiRA infers the membership of a sample $(x, y)$ by constructing two distributions of model losses: $\mathbb{\Tilde{Q}}_{\text{in}}$ and $\mathbb{\Tilde{Q}}_{\text{out}}$ with $(x, y)$ in and out of the model training set respectively. These distributions are approximated as Gaussians, with their parameters estimated based on ``shadow models'' trained on different subsets of the training data. The Likelihood-Ratio Test~\citep{07dc41a8-17bb-36b0-8eb8-d51fd0847411,lira} is then used to determine whether $(x, y)$ is more likely to belong to $\mathbb{\Tilde{Q}}_{\text{in}}$ or $\mathbb{\Tilde{Q}}_{\text{out}}$. For LLMs, the test statistic is given by~\citep{icul} as:
% $p(L(f(x), y) | \mathbb{\Tilde{Q}}_{\text{in/out}})$
\begin{equation}
    \Lambda = \frac{P \Bigl(l \bigl(f(x), y \bigr) | \mathbb{\Tilde{Q}}_{\text{in}}\Bigr)}{P \Bigl(l \bigl(f(x), y \bigr) | \mathbb{\Tilde{Q}}_{\text{out}}\Bigr)} = \frac{\Pi_{(x_i, y_i) \in \mathcal{D}_f} P_U \Bigl(l \bigl(f'(x_i), y_i \bigr)\Bigr)}{\Pi_{(x_i, y_i) \in \mathcal{D}_f} P_{T_r} \Bigl(l \bigl(f(x_i), y_i \bigr)\Bigr)}.
\end{equation} 
% which intuitively queries the loss of $(x, y)$ to determine if $(x, y)$ is more likely to be present in the training set or not.
This approach, however, is insufficient for tool unlearning because it only assesses membership of specific training samples rather than measuring whether the model still retains the capability to use a tool.



\paragraph{LiRA-Tool: Knowledge-level LiRA}
A major limitation of sample-level LiRA is in its reliance on training-set observations, which may not fully capture the knowledge distribution of an entire tool. Therefore, applying LiRA to tool unlearning can lead to overfitting to a specific subset of training prompts and failing to comprehensively assess whether the tool knowledge has been removed. 
%
% is in approximating the distributions of losses $\mathbb{\Tilde{Q}}_{\text{in}}$ and $\mathbb{\Tilde{Q}}_{\text{out}}$ for tools, rather than individual training samples. 
% Simply using the observed data related to a tool in the training set may overfit to specific distribution of observations, and may fail to comprehensively approximate the distribution of the target tool marked for unlearning. 
We address this issue by introducing LiRA-Tool. Instead of relying on observed training samples, we construct a ``shadow distribution'' $\mathbb{P}$ that generates tool-related query-response pairs. This allows us to sample diverse tool-specific prompts that test the model's ability to use the tool. The new likelihood-ratio test is:\looseness-1
% sample a series of ``shadow'' data (query-response pairs) that evaluates the tool using the ability to compute loss and test statistic as follows:
\begin{equation}
    \Lambda = \frac{\Pi_{t_i \in \mathcal{T}_f}\Pi_{(x, y) \in \mathbb{P}_{t_i}} P_U \Bigl(l \bigl(f'(x), y \bigr) \Bigr)}{\Pi_{t_j \in T_r}\Pi_{(x, y) \in \mathbb{P}_{t_j}} P_{\mathcal{T}_r} \Bigl(l \bigl(f(x), y \bigr) \Bigr)},
\end{equation} 
where $\mathbb{P}_{t_i}$ represents the shadow distribution for generating tool-learning samples for tool $t_i$. 
% is the distribution that controls the generation of tool learning samples for $t_i$. 
In practice, we use GPT-4 to generate diverse shadow samples by prompting it with various distinct instructions to ensure that the evaluation set captures more comprehensive aspects of tool knowledge than the training set. Appendix~\ref{sec:prompt_shadow_sample} provides more details.\looseness-1
% for approximating $\mathbb{\Tilde{Q}}_{\text{in}}$ and $\mathbb{\Tilde{Q}}_{\text{out}}$ and performing likelihood-ratio test.


\paragraph{Novelty of LiRA-Tool}
The key novelty in LiRA-Tool in the sue of ``shadow samples,'' which introduce diversity across multiple dimensions.
% , including prompt format, intent, and difficulty. 
By moving beyond limited training prompts, LiRA-Tool ensures that the model loss reflect overall tool-using ability, rather than just sample-level memorization.
% The major difference is that traditional LiRA approximates $\mathbb{\Tilde{Q}}_{\text{in}}$ and $\mathbb{\Tilde{Q}}_{\text{out}}$ with a series of shadow models by controlling which samples
%models 
% are present in training set. 
% however, unlearning a skill (tool) is prioritized 
%over individual samples  
% Consequently, using the original samples may not comprehensively approximate the distribution of $\mathbb{\Tilde{Q}}_{\text{in}}$ and $\mathbb{\Tilde{Q}}_{\text{out}}$. 
% We instead 
% . Such samples differ from each other, which encourages the losses to reflect efficacy in tool using instead of membership of individual training sample. 
% We GPT-4 as the shadow distribution $\mathbb{P}$ due to its superior tool using ability and the diversity of its generation. 
%
Our loss-ratio formulation shares similarities to previous MIAs for sample-level unlearning, such as probability distribution comparison prior- and post-unlearning~\citep{cheng2023gnndelete,cheng2023multimodal} and other adaptations of LiRA 
% which performs likelihood-ratio test over 
using shadow models~\citep{unbound,icul}. However, to the best of our knowledge, this work is the first adaptation of LiRA for detecting tool presence in tool-augmented LLMs. 




% We adapt sample-level MIA into knowledge-level MIA to infer the membership of tools for tool unlearning evaluation; and propose a new method to estimate $\mathbb{\Tilde{Q}}_{\text{in}}$ and $\mathbb{\Tilde{Q}}_{\text{out}}$. 
% based on latent variables. 
% This provides a comprehensive approximation of abstract concepts beyond observed training data. 

\paragraph{Limitations of LiRA-Tool}
Shadow samples obtained from GPT-4 may not fully represent the complexity of the original tool-learning data and can potentially lead to incomplete approximations of the true knowledge distribution.
% LiRA-Tool is still more comprehensive than existing sample-level MIA, because 
However, despite this limitation, shadow samples provide a more comprehensive and consistent evaluation of a model's tool-using abilities compared to relying merely on observed training samples, which are often limited and incomplete. Expanding the diversity and robustness of shadow sample generation is indeed an important direction for future work. % In addition, if the size of the shadow sample is large enough for each tool, it can better approximate the knowledge distribution for the tool.


\begin{table*}[t]
% \setlength{\tabcolsep}{4pt}
\caption{Tool unlearning performances when deleting 20\% of tools on ToolAlpaca. Best and second-best performances are \textbf{bold} and \underline{underlined} respectively. \textit{Original} is provided \textit{for reference only}. Results on other LLMs are shown in Appendix Table~\ref{tab:tool_llama}-\ref{tab:gorilla}.}
\label{tab:main}
\vskip 0.15in
\begin{center}
\begin{small}
\begin{sc}
\begin{tabular}{ll|ccc|ccccc}
\toprule
& Method & $\mathcal{T}_t (\uparrow)$ & $\mathcal{T}_r (\uparrow)$ & $\mathcal{T}_f (\downarrow)$ & \multicolumn{5}{c}{General Capability $\mathcal{T}_G (\uparrow)$} \\
                & & & & & STEM & Reason & Ins-Follow & Fact & Avg. \\
\midrule
\rowcolor{Gray} & Original (Ref Only) 
             & 60.0 & 73.1 & 75.7 & 31.7 & 17.1 & 22.6 & 25.0 & 24.1 \\
\midrule
\multirow{4}{*}{\rotatebox{90}{General}} 
& \RET & 52.1 & 71.8 & 38.5 & 30.5 & 16.1 & 14.2 & 24.7 & 21.3 \\
& \GA  & 33.3 & 51.4 & 34.6 & 21.4 & 10.4 & 12.9 & 13.1 & 14.5 \\
& \RL  & 50.3 & 70.3 & 37.5 & 26.3 & 16.4 & 13.6 & 25.1 & 20.3 \\
& \SU  & 46.2 & 54.3 & 38.2 & 27.1 & 17.0 & 17.4 & 19.5 & 20.2 \\
\midrule
\multirow{6}{*}{\rotatebox{90}{LLM-Specific}} 
& ICUL & 49.1 & \underline{74.8} & 58.3 & 12.4 &  8.7 &  1.6 &  6.2 &  7.3 \\
& SGA  & 43.5 & 63.0 & 42.1 & 21.5 & 11.6 & 17.0 & 14.7 & 16.2 \\
& TAU  & 43.8 & 61.7 & 42.5 & 22.0 & 17.6 & 22.3 & 21.7 & 20.9 \\
& CUT  & 44.7 & 61.5 & 40.2 & 21.6 & 14.8 & 20.8 & 16.4 & 18.4 \\
& NPO  & 50.8 & 66.9 & \underline{30.1} & 20.7 & 15.3 & 21.9 & 18.9 & 19.2 \\
& \SO  & 50.4 & 68.3 & 33.8 & 31.6 & 17.2 & 21.4 & 20.8 & 22.7 \\
\midrule
\multirow{2}{*}{\rotatebox{90}{Ours}} 
& \method-SFT & \underline{52.7} & 72.1 & \underline{30.5} & 31.3 & 17.5 & 21.7 & 24.1 & \textbf{23.6} \\
& \method-DPO & \textbf{53.4} & \textbf{75.1} & \textbf{28.7} & 31.6 & 16.8 & 20.4 & 23.5 & \underline{23.1} \\
\bottomrule
\end{tabular}
\end{sc}
\end{small}
\end{center}
\vskip -0.1in
\end{table*}



% \begin{table*}[t]
% % \setlength{\tabcolsep}{4pt}
% \caption{Tool unlearning performances when deleting 20\% of tools on ToolAlpaca. Evaluation is performed with the specific metric for each tool-augmented LLM on test tools $\mathcal{T}_t$, remaining tools $\mathcal{T}_r$, and unlearned tools $\mathcal{T}_f$, as well as general benchmarks for evaluation LLMs $\mathcal{T}_G$. Best and second best performances are \textbf{bold} and \underline{underlined} respectively. \textit{Original} denotes the tool-augmented LLM prior unlearning and is provided \textit{for reference only}. Results on other LLMs are shown in Appendix Table~\ref{tab:tool_llama}-\ref{tab:gorilla}.}
% \label{tab:main}
% \centering
% \small
% \begin{tabular}{l|ccc|ccccc}
% \toprule
% \textbf{Method} & $\mathcal{T}_t (\uparrow)}$ & $\mathcal{T}_r (\uparrow)}$ & $\mathcal{T}_f (\downarrow)}$ & \multicolumn{5}{c}{\textbf{General Capability} $\mathcal{T}_G} (\uparrow)}$} \\
%                 & & & & \textbf{STEM} & \textbf{Reason} & \textbf{Ins-Follow} & \textbf{Fact} & \textbf{Avg}. \\
% \midrule
% \rowcolor{Gray}\textbf{Original (Prior Un.)} 
%              & 60.0 & 73.1 & 75.7 & 31.7 & 17.1 & 22.6 & 25.0 & 24.1 \\
% \midrule
% \multicolumn{9}{l}{General Unlearning Methods} \\
% \midrule
% \textbf{\RET} & 52.1 & 71.8 & 38.5 & 30.5 & 16.1 & 14.2 & 24.7 & 21.3 \\
% \textbf{\GA}  & 33.3 & 51.4 & 34.6 & 21.4 & 10.4 & 12.9 & 13.1 & 14.5 \\
% \textbf{\RL}  & 50.3 & 70.3 & 37.5 & 26.3 & 16.4 & 13.6 & 25.1 & 20.3 \\
% \textbf{\SU}  & 46.2 & 54.3 & 38.2 & 27.1 & 17.0 & 17.4 & 19.5 & 20.2 \\
% \midrule
% \multicolumn{9}{l}{LLM-Specific Unlearning Methods} \\
% \midrule
% \textbf{ICUL}           & 49.1 & \underline{74.8} & 58.3 & 12.4 &  8.7 &  1.6 &  6.2 &  7.3 \\
% \textbf{SGA}            & 43.5 & 63.0 & 42.1 & 21.5 & 11.6 & 17.0 & 14.7 & 16.2 \\
% \textbf{TAU}            & 43.8 & 61.7 & 42.5 & 22.0 & 17.6 & 22.3 & 21.7 & 20.9 \\
% \textbf{CUT}            & 44.7 & 61.5 & 40.2 & 21.6 & 14.8 & 20.8 & 16.4 & 18.4 \\
% \textbf{NPO}            & 50.8 & 66.9 & \underline{30.1} & 20.7 & 15.3 & 21.9 & 18.9 & 19.2 \\
% \textbf{SOUL-GradDiff}  & 50.4 & 68.3 & 33.8 & 31.6 & 17.2 & 21.4 & 20.8 & 22.7 \\
% \midrule
% \textbf{\method-SFT} & \underline{52.7} & 72.1 & \underline{30.5} & 31.3 & 17.5 & 21.7 & 24.1 & \textbf{23.6} \\
% \textbf{\method-DPO} & \textbf{53.4} & \textbf{75.1} & \textbf{28.7} & 31.6 & 16.8 & 20.4 & 23.5 & \underline{23.1} \\
% \bottomrule
% \end{tabular}
% \end{table*}




% Let $t_m \in T_r$ denote a tool to be unlearned. $g$ is a function that measures the amount of knowledge a model has on a tool. A unlearned model satisfies Knowledge Retention if

% \begin{equation}
%     \mathop\mathbb{E}_{t_m \in T_r} [ g(f', t_m) - g(f, t_m) ] = \epsilon,
%     \label{eq:prop2}
% \end{equation} where $\epsilon$ is an infinitesimal constant.


% \subsection{Knowledge Removal and Retention}

% The most critical part of knowledge removal is to reset $f'$'s knowledge on $T_f$ back to a level that is similar to $f_0$, which has never seen $T_f$. In other words, to unlearn a given tool $t_i \in T_f$, we want to match $g(f', t_i)$ to $g(f_0, t_i)$. For LLMs, researchers usually prompt them to probe their knowledge on a topic or a fact. To this end, we propose to prompt the vanilla model $f_0$ with the ground-truth queries $Q_i$ associated with $t_i$, resulting in $f_0(Q_i)$. If a model outputs contents similar to $f_0(Q_i)$ when prompted $Q_i$, we can regard the model as having no additional tool knowledge than $f_0$, therefore never sees $t_i$. Formally, to unlearn $T_f$, we obtain a knowledge purged dataset

% \begin{equation}
%     D^0_f = \{ (t_i, Q_i, f_0(Q_i)) | \forall t_i \in T_f \}.
% \end{equation}

% As of knowledge maintenance on $T_r$, the remaining dataset $D_r = \{ T_r, Q_r, Y_r \}$ has documented enough amount of knowledge to retain $f'$'s knowledge. 

% To this end, we have constructed a new dataset with $D^0_f$, the dataset with purged knowledge of $T_f$, and $D_r$, the remaining data with original tools and query-response pairs. We then fine-tune $f$ on $ D^0_f \cup D_r $ to obtain $f'$.


% Inspired by the random labeling approach~\citep{amnesiac_2021} in classification tasks, we implement \method in the random labeling \& fine-tuning paradigm, while realizing the above properties for tool learning. The original random labeling approach consists of two steps. Firstly, for each sample in the unlearned data $(x_i, y_i) \in D_f$, replacing $y_i$ with a randomly chosen label $y'_i \neq y_i$, resulting in $D'_f$. Secondly, fine-tune $f$ the corrupted data $D'_f$ and the retained data $D_r$. During training, the knowledge on the unlearned samples is corrupted by encouraging $f'$ to mis-classify $D_f$. Meanwhile, $f'$ is trained on $D_r$ to further strengthen the knowledge that we want to retain.

% For the demonstrations $\{ Q_i, Y_i \}$ associated with tool $t_i$, we first obtain the query the vanilla model $f_0$ with $Q_i$ and obtain the responses $Y'_i = f_0(Q_i)$. Since $f_0$ has never been trained on $\{ Q_i, Y_i \}$, $Y'_i$ is a set of responses with no information on the $t_i$ coming from 
% we encourage model to provide similar responses as the vanilla model prior to tool learning $f_0$, which has never seen $D_f$.  to realize \propone, we encourage

% \paragraph{Various levels of knowledge removal}
% Knowledge removal can happen in different cases for tool learning / unlearning. 
% \begin{itemize}
%     \item do not choose any tool.
%     \item choose a different tool. 
%     \item choose the right tool but wrong arguments. 
%     \item say I don't know how to solve this task based on internal knowledge / learned tools. Depends on evaluation.
% \end{itemize}

% Note that the difference between 1) and 4) is that 1) predicts that $x_i$ requires no tool to solve. While 4) encourages the LLM to answer "I don't know.".

% Most existing tool-augmented LLMs are trained in a Supervised Fine-Tuning manner (SFT), where language modeling objective is optimized query-response pairs $(q_i, y_i)$.

% \subsection{Agent Unlearning Problem Formulation}
% \paragraph{Knowledge Unlearning} aims at unlearning knowledge shared by the entire community of a subset of agents.
% \paragraph{Individual Unlearning} aims at removing $k$ agents from the community.


\section{Experimental Setup} \label{sec:experiment}

\paragraph{Datasets \& Tool-Augmented LLMs}
% We focus on tool-augmented LLMs that are explicitly fine-tuned.
We experiment with the following datasets and their corresponding LLMs:  
\vspace{-7pt}
\begin{itemize}
\itemsep-1pt
\item \textbf{ToolAlpaca}~\citep{tang2023toolalpaca} is an agent-generated tool learning dataset consisting of 495 tools and 3975 training examples. \textbf{ToolAlpaca 7B} is fine-tuned on ToolAlpaca using Vicuna-v1.3~\citep{zheng2023judging}.
\item \textbf{ToolBench}~\citep{qin2024toolllm} consists of more than 16k real world APIs from 49 categories, where each training demonstration involves complex task solving traces. \textbf{ToolLLaMA} is fine-tuned on ToolBench using LLaMA-2 7B~\citep{touvron2023llama2}.
\item \textbf{API-Bench}~\citep{patil2023gorilla} focus on APIs that load machine learning models. \textbf{Gorilla} is fine-tuned on API-Bench from LLaMA 7B~\citep{touvron2023llama1}.
% 4) API-Bank. Lynx-7B is a model fine-tuned on API-Bank.
% 5) API-Blend is .
\end{itemize}
% We choose these datasets because they are fully open source. In addition, there exists other datasets such as API-Bank~\citep{li-etal-2023-api}~and~API-Blend~\citep{basu-etal-2024-api}, but they have not provided accessible training data or instructions for reconstruction.

% There are several types of tool-augmented LLMs:
% \begin{itemize}
%     \item Finetuning-based
%     \item Embedding-based
%     \item In-Context-based
% \end{itemize}

% Tora (\url{https://huggingface.co/llm-agents/tora-13b-v1.0})
% Liquid \url{Liquid1/llama-3-8b-Instruct-liquid-agent-calling}
% ToolLLaMA
% ToolAlpaca \url{https://arxiv.org/abs/2306.05301}
% TALM \url{https://arxiv.org/abs/2205.12255}
% Lynx-7B \url{https://aclanthology.org/2023.emnlp-main.187/}

\paragraph{Setup \& Evaluation}
We use the public checkpoints of the above tool-augmented LLMs as original models--the starting point for unlearning. Then we conduct unlearning experiments with 2--20\% tools randomly selected as $\mathcal{T}_f$.
% We conduct two types of unlearning experiments: 
% \vspace{-7pt}
% \begin{itemize}
% \itemsep-1pt
% \item \textbf{Random Tool Unlearning}, where 2--20\% tools are randomly selected as $T_f$, and 
% \item \textbf{Class-wise Tool Unlearning}, where tools from a specific category, such as all tools tagged as \texttt{Development}-related in ToolAlpaca, are chosen as $\mathcal{T}_f$. Specifically, we focus on unlearning the top 5 largest categories based on class sizes.
% \end{itemize}
We evaluate tool unlearning effectiveness, general capability of tool-unlearned LLMs, and robustness to membership inference attack (MIA). 
%
For \textbf{unlearning effectiveness}, we measure performance on test sets ($\mathcal{T}_T, \uparrow$), forget set ($\mathcal{T}_f, \downarrow$), and remaining set ($\mathcal{T}_r, \uparrow$), where ``performance'' reflects the ability to solve tasks that depend on specific tools, depending on the unique metrics in the original tool-augmented models $f$. 
% , which is different from ``memorization of training sequences'' in prior LLM unlearning works~\citep{jang-etal-2023-knowledge,kassem-etal-2023-preserving,yao-etal-2024-machine,barbulescu2024textual}.
%
For \textbf{general capabilities}, we evaluate the unlearned LLMs on a wide range of tasks: 
college STEM knowledge with MMLU~\citep{hendrycks2021measuring}, 
reasoning ability with BBH-Hard~\citep{suzgun-etal-2023-challenging}, 
instruction-following with IFEval~\citep{zhou2023instruction}, and 
factual knowledge with MMLU~\citep{hendrycks2021measuring}.
%
For \textbf{MIA}, we use the proposed LiRA-Tool; following prior work on LiRA~\citep{icul}, we train the shadow models with forget set size of \{1, 5, 10, 20\} and primarily evaluate the True Positive Rate (TPR) at low False Positive Rate (FPR) (TPR @ FPR = 0.01), where TPR means the attacker successfully detects a tool is present. Therefore, a lower TPR indicates better performance (privacy).

% & \multicolumn{4}{c}{ToolAlpaca-7B} & \multicolumn{4}{c|}{ToolAlpaca-13B} & \multicolumn{4}{c|}{Lynx-7B} & \multicolumn{4}{c}{ToolLLaMA-7B} \\
% & $D_T$ & $D_f$ & $D_r$ & Gen. & $D_T$ & $D_f$ & $D_r$ & Gen. & $D_T$ & $D_f$ & $D_r$ & Gen. \\




\paragraph{Baselines}
As there are no prior works on tool unlearning, we adapt the following unlearning methods to tool unlearning setting (see Appendix~\ref{sec:baseline} for descriptions of the baselines):
general unlearning approaches, including 
\textbf{\GA}~\citep{Golatkar2020EternalSO,yao-etal-2024-machine}, 
\textbf{\RL}~\citep{amnesiac_2021}, and 
\textbf{\SU}~\citep{fan2024salun}; 
and LLM-specific unlearning approaches, including  
\textbf{ICUL}~\citep{icul}, 
\textbf{SGA}~\citep{jang-etal-2023-knowledge,barbulescu2024textual}, 
\textbf{TAU}~\citep{barbulescu2024textual}, 
\textbf{CUT}~\citep{li2024wmdp},  
\textbf{NPO}~\citep{zhang2024negative}, and
\textbf{\SO}~\citep{jia-etal-2024-soul}.
For ICUL~\citep{icul}, we randomly select one example $(q_i, y_i)$ from $\mathcal{T}_f$ and corrupt the output $y_i$ with randomly selected tokens. Then we concatenate this corrupted sequence with other intact sequences as the in-context demonstrations. For all other baselines, we treat all data related to $\mathcal{T}_f$ as unlearning examples and all data related to $\mathcal{T}_r$ as remaining examples. Everything else remains the same for each baseline. 
% See \S\ref{sec:prem} for our discussion on why sample-level unlearning methods are inadequate for effective tool unlearning. 



\section{Results}

% \paragraph{Main results}
% Overall, on average of three datasets, \method-SFT 


\paragraph{Comparison to general unlearning methods}
Our main results in Table~\ref{tab:main} show that, compared to \RET, the best-performing baseline in the general unlearning methods category, \method-SFT outperforms \RET by 0.6, 0.3, 8.0, 2.3 absolute points on \ttest, \tr, \tf, \tg respectively. \method-DPO outperforms \RET by 1.3, 3.3, 9.8, 1.8 absolute points across the same metrics. We note that \GA can effectively unlearn \tf, but it negatively impacts its \ttest and \tr performance. Although \RL and \SU outperforms \GA, they still fall short on \tg compared to \method.


\paragraph{Comparison to LLM-specific unlearning methods}
Existing LLM unlearning methods, despite effective in sample-level unlearning, are prone to under-performing in tool unlearning. Both \method-SFT and \method-DPO outperforms ICUL, SGA, and TAU on \ttest, \tr, \tf and \tg. The only exception is ICUL, which outperforms \method-SFT on \tr by 2.7 absolute points, but is outperformed by \method-DPO on \tr by 0.3 points. The good performance of ICUL on \tr is at the cost of failing to unlearn tools in \tf, which is not desired in tool unlearning. In addition,  ICUL has limited ability of preserving test set performance, it is outperformed by \method-SFT and \method-DPO by 3.6 and 4.3 respectively. Furthremore, it is particularly limited in deletion capacity, i.e. number of unlearning samples that a method can handle. As $|D_f|$ exceeds 10, the performance of ICUL on \ttest significantly degrades. This is while \method can process much larger deletion requests efficiently. 



\begin{figure}
\vskip 0.2in
\begin{center}
\centerline{\includegraphics[width=0.47\linewidth]{figure/mia.pdf}}
% \vspace{-10pt}
\caption{Measuring tool unlearning with LiRA-Tool.}% \GA and ICUL are best-performing baselines for general and LLM-specific unlearning methods.}
% \vspace{-10pt}
\label{fig:mia}
\end{center}
\vskip -0.2in
\end{figure}

% \begin{table}[t]
% \tiny
% % \setlength{\tabcolsep}{3pt}
% \centering
% \caption{Ablation study of proposed properties on ToolAlpaca. \colorbox{lightred}{Highlighted} are metrics that degrade after removing specific parts of the model.}
% \label{tab:ablation}
%     \begin{tabular}{l|cccc}
%     \toprule
%      & $\mathbf{\mathcal{T}_T (\uparrow)}$ & $\mathbf{\mathcal{T}_r (\uparrow)}$ & $\mathbf{\mathcal{T}_f (\downarrow)}$ & $\mathbf{\mathcal{T}_G (\uparrow)}$ \\
%     \midrule
%     \multicolumn{5}{l}{\textit{\method-SFT}} \\
%     \midrule
%     Full Model & \textbf{57.7} & \textbf{72.1} & \textbf{30.5} & \textbf{23.6} \\
%     \midrule
%      - TKD & 58.1 & 72.4 & \cellcolor{lightred}{65.3} & 23.3 \\
%      - TKR & \cellcolor{lightred}{32.7} & \cellcolor{lightred}{40.2} & 23.1 & 20.1 \\
%      - GCR    & 58.0 & 72.5 & 31.1 & \cellcolor{lightred}{17.5} \\
%     \midrule
%     \multicolumn{5}{l}{\textit{\method-DPO}} \\
%     \midrule
%      Full Model & \textbf{58.4} & \textbf{73.3} & \textbf{28.7} & \textbf{23.1} \\
%      \midrule
%      - TKD & 58.6 & 73.2 & \cellcolor{lightred}{65.9} & 22.7 \\
%      - TKR & \cellcolor{lightred}{40.3} & \cellcolor{lightred}{41.8} & 39.3 & 22.1\\
%      - GCR    & 55.7 & 72.7 & 33.1 & \cellcolor{lightred}{14.3} \\
%     \bottomrule
%     \end{tabular}
% \end{table}


\paragraph{SFT vs. DPO}
DPO outperforms SFT by 0.7, 3.0, and 1.8 on \ttest, \tr, \tf respectively. On \tg, SFT is slightly better than DPO by 0.5 points. However, DPO takes slightly longer time to train, see Figure~\ref{fig:time}. Both optimization methods achieve superior performance over existing approaches. 


% \paragraph{Class-wise Tool Unlearning}
% We then investigate category-wise tool unlearning.



\paragraph{Measuring tool unlearning with MIA}
Following prior practices~\citep{lira,icul}, a lower TPR indicates an unlearned model with better privacy when FPR=0.01. \method-DPO achieves 0.14 TPR, outperforming \RET by 0.01. This advantage is obtained by explicitly prioritizing tool-free responses $f_0(\mathcal{Q})$ over original responses. In addition, \method-SFT achieves comparable performance with \RET, which indicates its effectiveness to protect privacy. Both variants of our method outperforms \GA and ICUL, the best performing %general and LLM-specific 
baselines, achieving 0.21 and 0.18 TPR. This indicates that existing sample-level unlearning approaches are not sufficient for unlearning tools, see Figure~\ref{fig:mia}.


\paragraph{Sequential unlearning}
Tool unlearning requests may arrive in sequential mini-batches. We experiment with sequential unlearning requests by incrementally unlearning 2\%, 5\%, 10\%, and 20\% of tools. \RET, ICUL by design cannot process sequential deletion requests. \method can continue training according to the current deletion request, without having to retrain a new model. When 20\% of unlearning requests arrive in batches, \method can sequentially unlearn each of them. As Figure~\ref{fig:seq} and Table~\ref{tab:main} show, compared to unlearning 20\% at once, the performance does not degrade significantly. 
% At each step, deletion request of 3\%, 5\%, 10\%, 30\% comes in, making the total unlearning ratio 2\%, 5\%, 10\%, 20\%, 50\%.



\begin{table}[t]
\setlength{\tabcolsep}{3pt}
\caption{Ablation study of proposed properties on ToolAlpaca. \colorbox{lightred}{Highlighted} are metrics that degrade after removing specific parts of the model.}
\label{tab:ablation}
\vskip 0.15in
\begin{center}
\begin{tiny}
\begin{sc}
    \begin{tabular}{l|cccc||cccc}
    \toprule
     & \multicolumn{4}{c||}{\method-SFT} & \multicolumn{4}{c}{\method-DPO} \\
     & $\mathbf{\mathcal{T}_T (\uparrow)}$ & $\mathbf{\mathcal{T}_r (\uparrow)}$ & $\mathbf{\mathcal{T}_f (\downarrow)}$ & $\mathbf{\mathcal{T}_G (\uparrow)}$ & $\mathbf{\mathcal{T}_T (\uparrow)}$ & $\mathbf{\mathcal{T}_r (\uparrow)}$ & $\mathbf{\mathcal{T}_f (\downarrow)}$ & $\mathbf{\mathcal{T}_G (\uparrow)}$\\
    \midrule
    Full & \textbf{57.7} & \textbf{72.1} & \textbf{30.5} & \textbf{23.6} & \textbf{58.4} & \textbf{73.3} & \textbf{28.7} & \textbf{23.1} \\
    \midrule
     - TKD & 58.1 & 72.4 & \cellcolor{lightred}{65.3} & 23.3 & 58.6 & 73.2 & \cellcolor{lightred}{65.9} & 22.7 \\
     - TKR & \cellcolor{lightred}{32.7} & \cellcolor{lightred}{40.2} & 23.1 & 20.1 & \cellcolor{lightred}{40.3} & \cellcolor{lightred}{41.8} & 39.3 & 22.1\\
     - GCR    & 58.0 & 72.5 & 31.1 & \cellcolor{lightred}{17.5} & 55.7 & 72.7 & 33.1 & \cellcolor{lightred}{14.3} \\
    \bottomrule
    \end{tabular}
\end{sc}
\end{tiny}
\end{center}
\vskip -0.1in
\end{table}



%     \label{fig:mia}
% \begin{table}[t]
% \setlength{\tabcolsep}{3pt}
% \begin{minipage}{.5\linewidth}
%     \centering
%     \includegraphics[width=\linewidth]{figure/mia.pdf}
%     \captionof{figure}{MIA performance using LiRA-Tool. \GA and ICUL are best-performing baselines for general and LLM-specific unlearning methods.}
%     \label{fig:mia}
% \end{minipage} \hfill
% \begin{minipage}{.45\linewidth}
%     \captionof{table}{Full parameters vs. LoRA in tool unlearning performances when deleting 20\% of tools on ToolAlpaca. \colorbox{Gray}{Original} denotes the tool-augmented LLM prior unlearning and is provided \colorbox{Gray}{for reference only}.}
%     \label{tab:peft}
%     \centering
%     \small
%     \begin{tabular}{p{6em}|cccc}
%     \toprule
%      & $\mathbf{\mathcal{T}_T (\uparrow)}$ & $\mathbf{\mathcal{T}_r (\downarrow)}$ & $\mathbf{\mathcal{T}_f (\uparrow)}$ & $\mathbf{\mathcal{T}_G} \mathbf{(\uparrow)}$ \\
%     \midrule
%     \rowcolor{Gray}\textbf{Original (Prior Un.)} 
%                             & 60.0 & 73.1 & 75.7 & 24.1 \\
%     \midrule
%     \textbf{Full param} & 52.7 & 72.1 & 30.5 & 23.6 \\
%     \midrule
%     \textbf{LoRA}       & 51.5 & 71.8 & 36.1 & 19.9 \\
    
%     \bottomrule
%     \end{tabular}
% \end{minipage}
% \end{table}


\paragraph{All properties contribute to effective tool unlearning}
Ablation studies in Table~\ref{tab:ablation} show that without Tool Knowledge Removal, performance of \method-SFT and \method-DPO on \tf degrade by -34.8 and -37.2 absolute points respectively. Such significant performance drop is observed for other model properties as well. Therefore, we conclude all proposed properties are necessary for successful at tool unlearning on \ttest, \tr, \tf, and \tg. 


\paragraph{\method functions effectively without access to training data}\label{sec:no_training_data}
In certain unlearning settings, access to the original training data might be restricted, e.g., in healthcare settings or in cases where training data is no longer available due to compliance. In these cases, \method can generate pseudo-samples for tools using the ``shadow samples'' technique developed for LiRA-Tool, see~\S\ref{sec:lira_tool}. Table~\ref{tab:no_training_data} in Appendix~\ref{sec:additional_result} shows that \method can perform tool unlearning effectively, achieving comparable performances to when full access to the exact training data is available.\looseness-1


\paragraph{\method is efficient}
Efficiency is a critical aspect for unlearning. As Figure~\ref{fig:time} in Appendix~\ref{sec:additional_result} illustrates, \method is substantially more efficient than retraining a new model from scratch--saving about 74.8\% of training time on average. In addition, this efficiency gain is relatively consistent as the size of $T_f$ increases. \method-SFT is slightly faster than \method-DPO, as the latter requires a negative sample for each of its prompts.

% \begin{wrapfigure}[]{r}{7cm}
%     \vspace{-20pt}
%     \centering
%   \includegraphics[scale=.5]{figure/time.pdf}
%     \vspace{-25pt}    
%   \caption{Training time of \method, which saves 74.8\% of time on average.}
%   \label{fig:time}
%   \vspace{-20pt}
% \end{wrapfigure}


\paragraph{\method-LoRA is ultra-efficient with good unlearning performance}
We experiment if \method can achieve effective tool unlearning through LoRA~\citep{hu2022lora}, when computing resource is limited. Experiments on ToolAlpaca show that \method-LoRA can achieve 97.7\%, 99.6\%, 84.5\%, and 84.3\% of the performance of \method with full parameter on \ttest, \tr, \tf, \tg on average across SFT and DPO, see Table~\ref{tab:peft} in Appendix~\ref{sec:additional_result}. In addition, it reduces save computational cost by 81.1\% and decreases the training time by 71.3\%.


\paragraph{\method is flexible in choice of tool-free responses}
In (\ref{eq:prop1}), we obtain tool knowledge-free responses from the tool-free LLM $f_0$. However, in cases where $f_0$ is unavailable, \method can still function using any knowledge-free LLM to generate tool knowledge-free responses, such as a randomly initialized LLM $f_R$. Table~\ref{tab:tool_free} compares the performances between these two implementations. While $\theta_0$ consistently outperforms $\theta_R$, using $\theta_R$ is still effective in achieving tool unlearning.\looseness-1 


\paragraph{Why is \method effectiveness?}
We attribute the performance of \method to its three key properties:
(a): Tool Knowledge Removal enables targeted tool unlearning without over-forgetting, unlike \GA and \RET. This is achieved by prioritizing tool knowledge-free responses over tool knowledge-intense responses so that the model forgets tool functionality without excessive degradation.
% This unlearning formulation poses the right strength of forgetting over specific tools, while existing methods either over-unlearn, such as \GA, or does not unlearn sufficiently, such as \RET. 
(b): Tool Knowledge Retention reinforces the knowledge about remaining tools. In fact, re-exposing the model to the original training data can further strengthen their representation. 
(c): General Capability Retention, which maintains or even improves model's general capabilities through an efficient and effective task arithmetic operation. Therefore, precise unlearning, retention of relevant knowledge, and overall model stability are the key factors that contribute to the performance of \method.  

 
% data from $f_0$, a model that has never seen the deleted tools before, and effectively
\xhdr{Domain-specific Tokenizers} 
Tokenizers tailored for specific domains have been employed to process various types of data, including language~\cite{bpe,sentencepiece,wordpiece,Wang2024challenging,Minixhofer2024zeroshot}, images~\cite{ibot,vqgan,Yu2024difftok,Zha2024textok}, videos~\cite{Choudhury2024dontlook}, graphs~\cite{Perozzi2024graphtalk,vqgraph}, and molecular and material sciences~\cite{Fu2024moltok,Tahmid2024birna,Qiao2024mxdna}. While these tokenizers perform well within their respective domains, they are not directly applicable to medical codes, which contains specialized medical semantics. Medical codes reside in relation contexts and are accompanied by textual descriptions. Directly using the tokenizers for languages risks flattening the relationships among codes and failing to preserve the biomedical information. This will lead to fragmented tokenization of medical codes, resulting in loss of contextual information during encoding.
Meanwhile, visual tokenizer typically focus on local pixel-level relationships, which are insufficient for capturing the complex semantics inherent in medical codes. Graph tokenizers are designed to encode structured information from graphs into a discrete token, then enabling LLMs to process relational and topological knowledge effectively. However, graph tokenizers may suffer from information loss when applied to graphs in other domains, making them less flexible and efficient for large, dynamic, and cross-domain graphs. In contrast, our \model tokenizer explicitly incorporates the relevant medical semantics by integrating textual descriptions with graph-based relational contexts.


\xhdr{Vector-Quantized Tokenizers}
Tokenization strategies often vary according to the problem domain and data modality where recent work has highlighted the benefits of discrete tokenization~\cite{du2024role}. This process involves partitioning the input according to a finite set of tokens, often held in a \textit{codebook} (this concept is independent of medical coding despite the similar name), and the quantization process involves learning a mapping from input data to the optimal set of tokens according to a pre-defined objective such as reconstruction loss~\cite{van2017neural}. 

Recent work has highlighted the ability of vector quantized (VQ-based) tokenization to effectively compress semantic information\cite{gu2024rethinking}. This approach is particularly successful for tokenizing inputs with an inherent semantic structure such as graphs~\cite{yang2023vqgraph, wang2024learning}, speech~\cite{zeghidour2021soundstream, baevski2019vq}, and time~\cite{yu2021vector} as well as complex tasks like recommendation retrieval \cite{wang2024learnable, rajput2023recommender, sun2024learning} and image synthesis \cite{zhang2023regularized, yu2021vector}.

Another significant advantage to VQ-based tokenization is the natural integration of multiple modalities. By learning a shared latent space across modalities, each modality can jointly modeled using a common token vocabulary \cite{agarwal2025cosmos, yu2023language}. 
% I think there are probably better citations to use for the line above
TokenFlow leverages a dual-codebook design that allows for correlations across modalities through a dual encoder~\cite{qu2024tokenflow}.


\xhdr{Structured EHR, transformer-based, and foundation models} 
%
Structured EHR models leverage patient records to learn representations for clinical prediction and operational healthcare tasks. These models differ from medical LLMs~\cite{singhal2025toward,tu2024towards,singhal2023large}, which are typically trained on free-text clinical notes~\cite{jiang2023health} and biomedical literature rather than structured EHR data.  
%
BEHRT~\cite{li2020behrt} applies deep bidirectional learning to predict future medical events, encoding disease codes, age, and visit sequences using self-attention. TransformEHR~\cite{transform_ehr} adopts an encoder-decoder transformer with visit-level masking to pretrain on EHRs, enabling multi-task prediction. GT-BEHRT~\cite{gtbehrt} models intra-visit dependencies as a graph, using a graph transformer to learn visit representations before processing patient-level sequences with a transformer encoder.  
%
Other models enhance EHR representations with external knowledge. GraphCare~\cite{graphcare} integrates large language models and biomedical knowledge graphs to construct patient-specific graphs processed via a Bi-attention Augmented Graph Neural Network. Mult-EHR~\cite{mult_ehr} introduces multi-task heterogeneous graph learning with causal denoising to address data heterogeneity and confounding effects. ETHOS~\cite{ethos} tokenizes patient health timelines for transformer-based pretraining, achieving zero-shot performance.  
%
While these models focus on learning patient representations, \model serves a different role as a medical code tokenizer. It can be integrated into any structured EHR, transformer-based, or other foundation model, improving how medical codes are tokenized before being processed. Unlike these models, which rely on predefined tokenization schemes, \model optimizes the tokenization process itself.
In this paper, we introduce \method, an effective approach to dynamic persona modeling that leverages iterative reinforcement learning and discrepancy-based refinement to continuously enhance persona quality and predictive accuracy. Comprehensive experiments demonstrate \method’s effectiveness across diverse domains in dynamic user modeling. We hope \method marks a significant advancement in personalized applications.

% \section*{Ethics Statement}
% Our research focuses on mitigating dataset biases in NLP datasets. There are no specific ethical concerns directly associated with this work. However, we recognize the broader ethical considerations that apply to NLP research. 

\section*{Impact Statement}
Our work investigates the security implications of tool-augmented Large Language Models (LLMs), where we focus on the risks that arise from integrating external tools, and the necessity ability to remove these acquired tools. 
A key concern is ensuring compliance with privacy regulations, such as the Right to be Forgotten (RTBF), which mandates the removal of specific data upon user request. In the context of tool-augmented LLMs, this necessitates the ability to delete sensitive, regulated, or outdated information related to specific tools. 
By examining how LLMs interact with and rely on external tools, potential threats to model security can be identified, e.g. unauthorized tool usage, adversarial exploitation, and privacy violations. Our research highlights the critical importance of addressing these challenges.

\bibliography{reference,anthology}
\bibliographystyle{iclr2025_conference}

\newpage
\appendix
\newpage
\section*{Appendix}
\subsection{Details about Analogy Generation from LLMs}
\input{table/full_prompt}
% \input{table/principle_prompt}
\subsubsection{Plain Analogy Generation}
To align with the educational process for students, we requested the most advanced LLM, \ie, GPT-4o~\cite{openai_gpt-4_2023}, to generate free-form analogies for each scientific concept. We adhered to the prompt settings described in \cite{bhavya-etal-2022-analogy} for the generation process. 
Specifically, to facilitate more effective analogy generation by GPT-4o, we incorporated principles from analogy cognition theory~\cite{hesse1959defining,gentner1983structure,gentner2017analogy} into the instruction prompt. Additionally, we provided GPT-4o with textbook content relevant to the scientific concepts to tailor the analogies to students' learning progress.
We did not include examples in the instruction prompt because there is no high-quality analogy dataset in the educational domain from which students can learn. 
Examples could induce bias in GPT-4o~\cite{pmlr-v139-zhao21c}, thereby degrading the generalization of analogy generation.

\subsubsection{Error Annotation for Generated Analogy}
Based on the prompt in step 2, we set the temperature to 0.7 for GPT-4o and generated three analogies for each concept. Then, three authors served as annotators to evaluate the quality of the generated analogies by annotating the errors. Initially, the annotators followed two common categories of error classification typically used in LLM outputs~\cite{jang-etal-2022-becel,gekhman-etal-2023-trueteacher,chuang2024dola}: 1) Factual Accuracy: the appropriateness of the analogies for the given concept; 2) Consistency: the coherence of objects within the analogies. 
The three annotators independently applied this classification to their annotations. After annotating 10 analogies, a discussion about error codes and annotations was conducted. Following three discussions, the annotators categorized the error codes into five types: the first three are factual errors, while the last two are inconsistency errors.
\begin{itemize}
    \item \textbf{Concept Paradox}: The analogy inaccurately represents the scientific concept, conflicting with established physical phenomena and commonsense knowledge.
    \item \textbf{Analogy Object Paradox}: The objects within the analogy do not align with physical laws or commonsense knowledge.
    \item \textbf{Inappropriate Analogy}: The analogy fails to accurately mirror the concept, leading to misconceptions.
    \item \textbf{Object Confusion}: The same analogy objects are assigned different roles or functions across various contexts.
    \item \textbf{Logical Contradiction}: The syntax within a sentence or paragraph contradicts itself.
\end{itemize}

After repeated discussions, the inter-rater reliability among annotators reached Fleiss' Kappa of 0.83 for Analogy Object Paradox, 0.94 for Object Confusion, and 1 for the remaining error codes.
As shown in the first row of Table.~\ref{tab:error_rates}.
Despite the powerful capabilities of GPT-4o, it fails to generate appropriate and satisfactory analogies for scientific concepts with our initial principles. 
Specifically, concerning the factual aspect, GPT-4o attempts to produce analogies that adhere to physical logic. 
However, the objects within these analogies often contradict physical laws or commonsense knowledge, resulting in errors related to the analogy objects.

Moreover, in the generated analogies, the same objects are inconsistently assigned different roles or functions in various contexts, indicating that GPT-4o suffers from object confusion.
For example, when generating analogies for the photoelectric effect, GPT-4o uses a trampoline. 
In this analogy, the trampoline represents the metal surface, and children represent electrons. 
The model initially makes an analogy between the force with which a worker presses the trampoline and the frequency of light, \eg, \textit{``First, if someone presses down on a diving board with a very small force (low-frequency light), the diving board will not have enough elasticity to bounce you back up.''}
However, in later descriptions, this force is equated to light intensity, \eg, \textit{``The greater the force applied (the higher the intensity of light), the more people will be bounced off the diving board in a given period.''}.


\subsubsection{Analogy Generation with Revised Prompt and Second Round of Error Annotation}
In Step 3, we discovered that GPT-4o cannot generate appropriate analogies directly, as it produces several errors. 
Therefore, we refined the prompt by incorporating new principles to guide GPT-4o in avoiding these errors, enhancing the quality of the generated analogies. 
Table~\ref{tab:instruction_prompt_full} (II) details the revised prompt template.
The three annotators applied the error codes used in Step 3 to the analogies generated in this round.
After two rounds of discussions, the inter-rater reliability among annotators reached Fleiss' Kappa of 0.84 for Analogy Object Paradox, 0.94 for Inappropriate Analogy, 0.92 for Object Confusion, and 1 for the remaining error codes.
Based on Table~\ref{tab:error_rates}, we can find that the revised prompt enhances the coherence of objects within analogies, thereby minimizing confusion and logical contradictions during generation. 
However, to accurately identify analogous relationships, the model may force objects to conform to patterns that contradict physical laws or common sense, thereby intensifying the analogy object paradox.

\subsubsection{Automatic Analogy Selection from LLMs}

From step 4, it is evident that the model is still prone to errors despite providing clear and instructive guidelines. 
Previous research in the AI community has demonstrated that models can select the optimal result from multiple outputs through a process known as self-correction~\cite{pan2023automatically,yuan-etal-2023-distilling,liu2024large}. 
Therefore, we allowed GPT-4o to choose the best result according to the guidelines from three analogies generated for each concept. 
The chosen prompts are displayed in Table~\ref{tab:instruction_prompt_full} (III). 
Results in Table~\ref{tab:error_rates} show that enabling the model to self-correct enhances the accuracy of the analogies. 
However, the model struggles with issues such as object confusion. 
We hypothesize that this difficulty arises because the objects in the analogies are described in disparate text sections, requiring the model to use contextual cues for differentiation. 
This necessitates a robust capability for understanding long contexts~\cite{li2024can}. 

\subsection{Principle Generation in System Workflow}
In our practical system, we adopt GPT-4o to summarize new principles from comments on the three analogies and add them to the principles list (\textbf{G2}).
The prompt template is: 
\begin{tcolorbox}[title=The prompt template of principle generation, mybox]
Based on the analogy you generated, the teacher's suggestion is:
```
{teacher_suggestion}.
```
Now, you do not need to generate new analogies, but instead, you need to generate more new guidelines to enrich the original ones based on the teacher's suggestion.

Original guidelines:
- The similarity between the objects in the analogy and those in the scientific concept should be minimal.  
- The relationships in the analogy and the scientific concept should be highly similar. 
- The analogy should use objects that students are very familiar with from everyday experiences. 
- The analogy should accurately identify similar relationships with the scientific concept and avoid forcing non-existent similarities.  
- The objects in the analogy and the scientific concept should align with scientific laws and commonsense knowledge. 
- An object in the analogy cannot have different roles or functions in different contexts.  
- The logic within a sentence or paragraph should not be self-contradictory.  
{current_strategies}
{current_guidelines}

Remember:  
1. The new guidelines should be different from the original guidelines.  
2. The new guidelines should be generated based on the teacher's suggestion.  
The format is as follows:
```
New guidelines:
- xxx
- xxx
- xxx
....
```
New guidelines:
\end{tcolorbox}


\end{document}