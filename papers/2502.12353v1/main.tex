%\documentclass[twoside]{article}
\documentclass[11pt]{article}
\usepackage{fullpage}

% \usepackage{aistats2025}
% If your paper is accepted, change the options for the package
% aistats2025 as follows:
%
%\usepackage[accepted]{aistats2025}
%
% This option will print headings for the title of your paper and
% headings for the authors names, plus a copyright note at the end of
% the first column of the first page.

% If you set papersize explicitly, activate the following three lines:
%\special{papersize = 8.5in, 11in}
%\setlength{\pdfpageheight}{11in}
%\setlength{\pdfpagewidth}{8.5in}

% If you use natbib package, activate the following three lines:
\usepackage[round]{natbib}
\renewcommand{\bibname}{References}
\renewcommand{\bibsection}{\subsubsection*{\bibname}}

% If you use BibTeX in apalike style, activate the following line:
%\bibliographystyle{apalike}

\usepackage[utf8]{inputenc}
\usepackage{amsmath,amsfonts,amsthm, amssymb}
\usepackage{hyperref}
\usepackage[capitalize]{cleveref}
\usepackage{caption}
\usepackage{subcaption}
\usepackage{graphicx}
\usepackage{float}
\usepackage{placeins}
\usepackage{enumitem}
\usepackage[dvipsnames]{xcolor}
\usepackage{bbm}

\newcommand{\dN}{\mathcal{N}}
\newcommand{\const}{\text{const}}
\newcommand{\diag}{\text{diag}}
\newcommand{\tr}{\text{tr}}
\newcommand{\E}{\mathbb{E}}
\newcommand{\V}{\mathbb{V}}
\newcommand{\R}{\mathcal{R}}
\newcommand{\F}{\mathcal{F}}
\newcommand{\elbo}{\text{ELBO}}
\newcommand{\dlm}{\text{DLM}}
\newcommand{\kl}{\text{KL}}
\newcommand{\loss}{\mathcal{L}}
\newcommand{\tv}{\text{TV}}
\newcommand{\alg}{\mathcal{A}}
\newcommand{\errgen}{\text{err}_{gen}}
\newcommand{\cov}{\text{Cov}}

\newtheorem{theorem}{Theorem}
\newtheorem{assumption}[theorem]{Assumption}
\newtheorem{lemma}[theorem]{Lemma}
\newtheorem{corollary}[theorem]{Corollary}
\newtheorem{definition}[theorem]{Definition}

\DeclareMathOperator*{\argmax}{arg\,max}
\DeclareMathOperator*{\argmin}{arg\,min}

\begin{document}

% If your paper is accepted and the title of your paper is very long,
% the style will print as headings an error message. Use the following
% command to supply a shorter title of your paper so that it can be
% used as headings.
%
%\runningtitle{I use this title instead because the last one was very long}

% If your paper is accepted and the number of authors is large, the
% style will print as headings an error message. Use the following
% command to supply a shorter version of the authors names so that
% they can be used as headings (for example, use only the surnames)
%
%\runningauthor{Surname 1, Surname 2, Surname 3, ...., Surname n}

%\twocolumn[
%\aistatstitle{Stability-based Generalization Bounds for Variational Inference}
%\aistatsauthor{ Yadi Wei \And Roni Khardon }
%\aistatsaddress{ Indiana University \And Indiana University } ]

\title{Stability-based Generalization Bounds for Variational Inference}

\author{Yadi Wei and Roni Khardon \\ Department of Computer Science \\ Indiana University, Bloomington \\
             \tt{(weiyadi}$|$\tt{rkhardon)@iu.edu}}

\maketitle

\begin{abstract}
Variational inference (VI) is widely used for approximate inference in Bayesian machine learning. 
In addition to this practical success, 
%recent work has developed 
generalization bounds for variational inference and related algorithms have been developed, mostly through the connection to PAC-Bayes analysis. 
%However, to our knowledge these bounds are loose with modern architectures and datasets. 
A second line of work
has provided algorithm-specific generalization bounds 
%for stochastic gradient Langevin dynamics (SGLD) 
through stability arguments or using mutual information bounds,
and has shown that the bounds are tight in practice,
but unfortunately these bounds do not directly apply to approximate Bayesian algorithms. 
This paper fills this gap by developing algorithm-specific stability based generalization bounds for a class of approximate Bayesian algorithms that includes VI, specifically when using 
stochastic gradient descent to optimize their objective.
As in the non-Bayesian case, 
the generalization error is bounded by
by expected parameter differences on a perturbed dataset.
The new approach complements PAC-Bayes analysis and can provide tighter bounds in some cases.  
%In this paper, we extend the stability-based framework to derive tighter generalization bounds for VI,
%by utilizing the norm of parameter differences. 
An experimental illustration shows that the new approach yields non-vacuous bounds on 
modern neural network architectures and datasets
and that it can shed light on performance differences between variant approximate Bayesian algorithms. 
%Our experimental results demonstrate that these stability-based bounds are more informative and tighter compared to traditional PAC-Bayesian approaches in the context of VI. 
\end{abstract}

\section{Introduction}
Variational inference (VI \citep{Jordan99}) is one of the most successful approaches in approximate Bayesian machine learning
\citep[e.g.,][]{Blei2003,LimTeh2007,SeegerB12,KingmaW13,Johnson2016} 
and a significant amount of recent work is devoted to variational methods for deep networks
% put some practical refs perhaps topic models PMF and others - the ones here are from VIFO paper
\citep[e.g.,][]{BNN, Practical-VI, DVI, lrt, collapsed-elbo}. 
%{\bf [add more recent BN papers?]}
Instead of calculating an exact posterior one computes an approximate posterior which minimizes 
the KL divergence between the approximation and the true posterior. 
VI is enabled computationally because minimizing the KL divergence is equivalent to maximizing the tractable 
evidence lower bound (ELBO). 
Thanks to this success several variations of VI have been proposed and following \cite{Alquier2016properties} recent work 
has developed finite sample generalization bounds using PAC Bayes analysis. 
%However, these bounds are typically loose, scaling as $O(1/\sqrt{n})$ for $n$ datapoints, and in many cases the loose bounds are not informative when applied to modern architectures. 

This paper continues this effort but from a different perspective, motivated by 
bounds for 
stochastic gradient Langevin dynamics (SGLD)
using Bayes stability \citep{Li,banerjee}.
The idea of analysis through stability \citep{BousquetE02,HardtRS16} is that if an algorithm is not sensitive to perturbations of its input (i.e., the data) then one can bound the gap between its training and test errors. 
\citet{Li,banerjee} employed the KL divergence between output distributions with and without perturbation to assess this sensitivity.
We note that SGLD modifies the parameter $W$ of the learned model, adding noise in the process, but unlike Bayesian algorithms it does not learn a distribution on the parameters. 
Unfortunately, due to this difference, the approach of \citet{Li,banerjee} is not applicable to Bayesian algorithms. 

%In our work, we specifically focus on establishing 
The paper builds on these ideas and provides a new analysis that 
establishes 
stability-based bounds for a family of approximate-inference algorithms for Bayesian neural networks (which includes VI), in particular when their inference objective is optimized using stochastic gradient descent (SGD). 
We develop two types of bounds: one for bounded loss functions and another for unbounded but Lipschitz loss functions. For bounded loss, we continue to use the KL divergence to measure sensitivity, whereas for Lipschitz loss, we employ the Wasserstein distance. Previous research \citep{ipm, wass-pac-bayes} has explored PAC-Bayes bounds using Wasserstein distance; here, we extend its application to Bayes stability. 
In both cases the generalization gap
%Both types of bounds 
can be upper-bounded by the expected parameter differences and further refined using techniques from \citet{HardtRS16} and \citet{SGD-stability2}, resulting in final bounds expressed as the sum of the gradient differences along the optimization trajectory.

\citet{ZhangBHRV17} demonstrated that it is possible to achieve near-zero training error on both true labels (leading to good test performance) and random labels (where test performance is random)
with the same network and training regime. Therefore, any meaningful generalization bound must distinguish between these scenarios, 
implying that it must be data-dependent. 

We provide an empirical demonstration confirming that our bounds can achieve this, effectively differentiating the successful case from overfitting. 
Further, our bounds produce non-vacuous results for generalization error of VI in practical situations, and 
effectively differentiate generalization performance when models are trained with or without data augmentation. 
We also use the bounds to explore the relationship of ELBO to direct loss minimization (DLM \citep{dlm-bnn}), a variant that has shown good performance in other models but fails for Bayesian neural networks, 
showing that the stronger stability of ELBO might explain this performance gap. Finally, a comparison to PAC-Bayes bounds shows that the stability approach provides tighter bounds in these scenarios and can therefore complement the strength of prior analysis.  


%Further, our bounds produce non-vacuous results in practical situations that are significantly tighter than PAC-Bayes bounds and effectively differentiate generalization performance when models are trained with or without data augmentation. Additionally, our analysis sheds light on the limitations of direct loss minimization (DLM \citep{dlm-bnn}) compared to ELBO, a distinction that PAC-Bayes bounds fail to capture.

In summary, this paper introduces a novel approach for analyzing the generalization performance of approximate Bayesian machine learning algorithms. Our contributions include a stability analysis of iterative update algorithms, the application of these bounds to variational Bayesian networks, and empirical demonstration of the practical utility of these bounds.

\section{Related Work}
There is a long tradition of analysis of asymptotic properties of Bayesian algorithms. 
\cite{Alquier2016properties} made an explicit connection between the Gibbs loss used in PAC-Bayes analysis and the objective of VI. This led to finite sample generalization bounds, i.e., bounds on the difference between training and true errors, that hold uniformly. In turn, algorithms that minimize the sum of training error plus generalization bound,
which have a form similar to VI with a regularization parameter,
are both well motivated and have strong theoretical guarantees. 
In followup work
\cite{Germain2016pac,Germain2019} have extended these results to richer classes, whereas \cite{Sheth2017} developed risk bounds, i.e., bounds that directly quantify the true error of VI. 
Other work suggested alternative optimization criteria diverging from VI by changing the loss or regularization components
\cite[e.g.,][]{black-box-alpha,knoblauch2019generalized,dlm-sgp} and generalization and risks bounds have been developed for some such algorithms 
\citep{Sheth2019,Germain2019,Masegosa20,pac-m}. 
%However, to our knowledge none of these results yield non-vacuous bounds for Bayesian neural networks used in practice. 
However, these have not been demonstrated in practice.
\citet{nonvacuous} provided a non-vacuous bound for a binary classification task on MNIST. 
We evaluate these bounds and compare them to the stability bound
in our experiments in a multi-class classification task with large neural networks.
%However, our experiment shows that when applied in multi-class classification and large networks, this bound is still vacuous.

Another important line of work aims to analyze standard (non-Bayesian) algorithms, where capacity arguments can be used to yield generalization bounds for neural networks \citep[e.g.,][]{nn-rademacher}.
%, but these are also loose. 
Recent work has developed an alternative approach that provides tighter bounds which are data-dependent and algorithm-dependent. This includes work using stability 
\citep{Li,banerjee}
and analysis that works through bounds on mutual information
\citep{NegreaHDK019,HaghifamNK0D20}. 
This has been specifically developed for SGLD, and extensions to SGD \citep{Neu21} are possible only as an approximation of SGLD.
While the approaches differs in technical details, the outcome is similar in that a generalization bound is obtained which can be expressed as a sum over training steps, of some function of the gradients.
Specifically the bound of \cite{Li} includes a sum of gradient norms whereas the bound of \cite{banerjee} includes a sum of the norms of gradient differences, which was found to be tighter in practice. 
As mentioned above,  
SGLD learns the parameter $W$ of the model and adds some noise duirng the optimization, hence it produces a sample from some distribution over parameters. 
%This analysis is applicable to non-Bayesian algorithm but requires the injection of homoscedastic noise by SGLD that generates a distribution over potential parameters, where SGLD samples from this distribution during optimization.
This differs from Bayesian algorithms that explicitly generate distributions over parameters as their posteriors, and aggregate their predictions, and unfortunately the analysis does not carry through to this case. 

In contrast, we directly analyze iterative update Bayesian algorithms, for example, using SGD for variational inference (VI), without noise injection. The primary challenge is that the distribution of the parameters of VI is intractable, making it difficult to apply the chain rule of divergence (as in Lemma 10 of \citet{Li}). 
We provide an alternative analysis that first externalizes all sources of randomness of the algorithm, and then uses convexity to derive the bounds. 
This 
allows us to bound the stability gap in terms of parameter differences.
%Instead, we employ the parameter differences to bound the Bayes stability for VI, and 
With this in place we can
follow the approach used to prove the stability of SGD \citep{HardtRS16, SGD-stability2} to bound parameter differences and obtain the desired result.  
Moreover, we extend the original Bayes stability argument, which previously applied only to bounded loss functions \citep{Li} or loss functions with bounded second moments \citep{banerjee}. We generalize this framework to Lipschitz continuous loss functions, allowing us to bound the generalization error using Wasserstein distances, which can be bounded using parameter differences. This extension is inspired by \citet{ipm, wass-pac-bayes}, which employ Wasserstein distances in PAC-Bayes bounds.

Finally,
while the discussion in the paper emphasizes the analysis of VI, 
%Although our focus is on applications to VI, 
the analysis and bounds are applicable to any iterative update 
approximate Bayesian 
algorithm that updates parameters of the approximate posterior, where the sensitivity of parameter updates can be easily calculated.
Hence it captures more cases than prior work, as illustrated by the application to DLM.


\section{Preliminaries}
Consider a model with parameters denoted as $w\in \mathbb{R}^d$. Given a prior distribution $p(w)$ and a dataset $S=(z_1, \dots, z_n)$ of size $n$, our goal is to determine the posterior distribution $p(w \mid S)$, which is computationally challenging in most cases. Variational inference offers a solution by seeking a distribution $Q(w)$ from a specified family of distributions, denoted as $\mathcal{Q}$, that minimizes the Kullback-Leibler (KL) divergence between $Q(w)$ and the true posterior $p(w|S)$. 
\begin{align}
\label{eq:elbo}
&Q^*(w) \nonumber \\
=& \argmin_{Q\in \mathcal{Q}}\kl(Q(w) \Vert p(w \mid S)) \nonumber \\
=& \argmin_{Q\in \mathcal{Q}} \E_{Q(w)}[\log Q(w) - \log p(w, S)] + \log p(S) \nonumber \\
=&\argmin_{Q\in \mathcal{Q}} 
\frac{1}{n} 
\sum_{i=1}^n 
\E_{Q(w)}[-\log p(z_i|w)] + \frac{1}{n} \kl(Q, p). 
\end{align}
The maximization objective obtained by negating \eqref{eq:elbo} is known as the Evidence Lower Bound (ELBO).
The above optimization objective can be efficiently solved using common gradient-based techniques, such as stochastic gradient descent. Furthermore, various alternative objectives exist to approximate the (pseudo) posterior distributions, for example, Direct Loss Minimization (DLM, \citep{dlm-sgp}), which uses the the following objective, and which is discussed in our experiments:
\begin{align}
\label{eq:dlm}
    \frac{1}{n}\sum_{i=1}^n -\log \E_{Q(w)}[p(z_i \vert w)] + \frac{1}{n} \kl(Q, p).
\end{align}

Note that the optimization objective is a function of the distribution $Q(w)$ and let $\theta$ denote the parameters of $Q$. 
To facilitate the analysis across different objectives,
we denote the objective function as $F(\theta, S) = \frac{1}{n} \sum_{i=1}^n F(\theta, z_i)$, where the objective function is written as the average of the objective function with respect to individual data points.  
Notice that for the examples above $F$ includes the regularizer.
For example, in ELBO, $F(\theta, z) = \E_{Q(w)}[-\log p(z|w)] + \frac{1}{n} \kl (Q, p)$. 

Let $\loss(w, z)$ be a loss function for parameter $w$ on a data point $z$ (notice that the loss can be different from the objective function $F$). Define $\loss(w, \mathcal{D}) = \E_{z \sim D} [\loss(w, z)]$ as the expected loss over a distribution $\mathcal{D}$, and $\loss(w, S) = \frac{1}{n} \sum_i \loss(w, z_i)$ as the empirical loss on a dataset $S$. Then the generalization error of the algorithm $\alg$ (which chooses $Q$ based on $S$), i.e., the gap between true and training set error, is given by:
\begin{align}
    \errgen(\alg) &= \E_{S\sim \mathcal{D}^n} \E_{w\sim Q} [\loss(w, \mathcal{D}) - \loss(w, S)].
\end{align}

\section{Generalization Bounds through Bayes Stability}
\label{sec:stability}
Consider a Bayesian algorithm, denoted by $\mathcal{A}$, designed to learn the posterior distribution over a parameter $w$ by optimizing an objective function $F(\theta, S)$. In some cases, there is inherent randomness in evaluating the objective and its gradients or in the optimization process, such as when the reparametrization trick \citep{KingmaW13} is used to approximate expectation terms (as in \cref{eq:elbo} and \cref{eq:dlm}) or when mini-batches are employed. We represent all sources of randomness by $\epsilon$. Consequently, the gradient of the objective becomes $\nabla F(\theta, S, \epsilon)$ for the entire dataset and $\nabla F(\theta, z, \epsilon)$ for an individual data point $z$.
Given a training dataset $S$ and the randomness $\epsilon$, the algorithm $\mathcal{A}$ deterministically produces a posterior distribution $Q_{S, \epsilon}$ for the parameter $w$. We define $Q_S$ as the expected posterior distribution, obtained by averaging over all possible randomness, i.e., $Q_S = \mathbb{E}_{\epsilon}[Q_{S, \epsilon}]$. Additionally, we assume that, when $\epsilon$ is integrated out, $\mathcal{A}$ is order-independent.

\begin{assumption}[Order-independent]
    For any fixed dataset $S=(z_1, \dots, z_n)$ and any permutation $p$, $Q_S = Q_{S^p}$, where $S^p$ is the dataset under permutation $p$.
\end{assumption}
This assumption can be easily satisfied by letting the learning algorithm randomly permute the training data at the beginning. Additionally, it is straightforward to show that variational inference using stochastic gradient descent (SGD) satisfies this condition.

We proceed, following the work of \citet{Li}, to define the single-point posterior distribution $Q_z = \E_{(z_1, \dots, z_{n-1})} [Q_{(z_1, \dots, z_{n-1}, z)}] = \E_{\epsilon, (z_1, \dots, z_{n-1})}[Q_{(z_1, \dots, z_{n-1}, z), \epsilon}]$, 
where we assume without loss of generality that $z$ is put at location $n$. 

\subsection{Bayes Stability}


The generalization error can be effectively bounded using a Bayes-stability argument, as exemplified in previous work by \citet{Li} and \citet{banerjee}. 
We develop two such bounds, one for bounded loss functions using TV distance and the other for unbounded but Lipschitz loss functions using Wasserstein distance. In both cases the result reduces to expected parameter differences. 

Let $\tv(p, q) = \frac{1}{2} \int \vert p(x) - q(x) \vert dx$ be the total variation distance between distributions. We have:

\begin{lemma}
\label{lemma:tv-convexity}
    $\tv\left(\E_{P(X)}[P(Y|X)], \E_{P(X)}[Q(Y|X)] \right) \leq \E_{P(X)}[\tv(P(Y|X), Q(Y|X))]$.
\end{lemma}
\begin{proof}
    \begin{align*}
        & \tv(\E_{P(X)}P(Y|X), \E_{P(X)}Q(Y|X)) 
        \\ &= \frac{1}{2} \int \left|\int P(x) P(y|x)dx - \int P(x) Q(y|x) dx\right| dy
        \\ &= \frac{1}{2} \int \left|\int P(x) (P(y|x)- Q(y|x)) dx\right| dy 
        \\ &\leq \frac{1}{2} \int P(x) \int \left|P(y|x)- Q(y|x)\right| dy dx 
        \\ &= \E_{P(X)}[\tv(P(Y|X), Q(Y|X))].
    \end{align*}
\end{proof}

The following lemma 
%(proof in \cref{sec:proof-stability}) 
adapts the ideas in the original proofs of \citet{Li, banerjee} to the context of Bayesian algorithms that output distributions over parameters. 
%The proofs of the next two lemmas are in \cref{sec:proof-stability}.

\begin{lemma}[Bayes-Stability 1]% \citep{Li, banerjee}]
\label{thm:errgen}
    Suppose the loss function $\loss(w, z)$ is $C$-bounded. Let $S$, $\Bar{S}$ denote two datasets that only differ at one element $z$ and $\Bar{z}$. The generalization error $\errgen(\alg)$ is upper bounded by $2C \E_{S, \Bar{S}, \epsilon} [\tv(Q_{S, \epsilon}, Q_{\Bar{S}, \epsilon})]$.
    %, where $\tv(p, q) = \frac{1}{2} \int \vert p(x) - q(x) \vert dx$ is the total variation distance.
\end{lemma}
\begin{proof}%[Proof of \cref{thm:errgen}]
    It is clear that
     \begin{align}
         \E_{S} \E_{w \sim Q_S} [\loss(w, \mathcal{D})] = \E_{z\sim \mathcal{D}} \E_{w \sim Q_z} \E_{\Bar{z}\sim \mathcal{D}}[\loss(w, \Bar{z})] = \E_{\Bar{z}} \E_{Q_{\Bar{z}}} \E_{z} [\loss(w, z)],
     \end{align}
     and
     \begin{align}
         \E_S \E_{w\sim Q_S} \left[\frac{1}{n} \sum_{i=1}^n \loss(w, z_i) \right] &= \E_{z} \E_{Q_z} [\loss(w, z)].
     \end{align}
     Then the generalization error is 
     \begin{align}
         \errgen(\alg) &= \E_z \E_{\Bar{z}} \left[ \E_{w\sim Q_{\Bar{z}}}[\loss(w, z)] - \E_{w\sim Q_z}[\loss(w, z)]\right] \\
         &\leq \E_{z, \Bar{z}} \int |\loss(w, z)| \lvert Q_{\Bar{z}}(w) - Q_z(w)\rvert dw \\
         &\leq 2C \E_{z, \Bar{z}} [\tv(Q_z, Q_{\Bar{z}})] \\
         &= 2C \E_{z, \Bar{z}}[\tv(\E_{S_{n-1}, \epsilon}[Q_{S_{n-1} \cup \{z\}, \epsilon}], \E_{S_{n-1}, \epsilon}[Q_{S_{n-1} \cup \{\Bar{z}\}, \epsilon }])] \\
         &\leq 2C \E_{z, \Bar{z}, S_{n-1}, \epsilon}[\tv(Q_{S, \epsilon}, Q_{\Bar{S}, \epsilon})]
     \end{align}
      where the last inequality is due to \cref{lemma:tv-convexity}.
     %because of the convexity of total variation distance (\cref{lemma:tv-convexity}).
\end{proof}

There are two important differences form the argument structure in prior work \citep{Li, banerjee}. 
First, note that it is crucial that $\epsilon$ includes all sources of randomness in the algorithm. With this condition, $Q_{S, \epsilon}$ is a distribution in the family used by the algorithm and not a mixture of such distributions.
For example, when $Q(w)$ is a normal distribution $Q_{S, \epsilon}$ is a normal distribution, but $Q_{S}$ is a mixture of normal distributions where the mixture is taken over $\epsilon$. 
%induced by the batch sequence. 
This allows us to directly bound the stability using parameter differences as in the next lemma. In contrast, the analysis of \citet{Li, banerjee}, 
that works with mixtures generated by the choice of batches in SGLD,
requires a fixed variance term (for all dimensions) and is not easily generalizable to the case of learned variances. 

The second difference is due to the structure of the probability model.
%It is important to note that we cannot directly apply the methods from \citet{Li} to derive the generalization bound, which 
\citet{Li} 
use the sum of KL divergence along the optimization trajectory to upper bound the Bayes stability. In SGLD, the optimization trajectory \( W_1, W_2, \dots, W_T \) consists of samples from the distribution, with each \( W_i \) being drawn from a Gaussian distribution conditioned on both \( W_{i-1} \) and the batch. However, in variational inference, the optimization trajectory \( (\mu_1, \sigma_1), \dots, (\mu_T, \sigma_T) \) consists of distribution parameters, 
and the bound on the sequence of conditional KL divergences does not hold. 
%and the distributions of \( \mu_i \) and \( \sigma_i \) are not well-defined. 
A detailed explanation is provided in \cref{sec:Li-proof-not-for-vi}.

The next lemma shows that Bayes stability can be bounded through parameter differences:

\begin{lemma}
\label{cor:kl}
    Under the condition of \cref{thm:errgen}, if $Q_{S, \epsilon} = \dN(m, \diag(\sigma^2))$ and $Q_{\Bar{S}, \epsilon} = \dN(\Bar{m}, \diag(\Bar{\sigma}^2))$, the generalization error is upper bounded by 
    \begin{align}
    \label{eq:kl-bound}
        \frac{2C}{\sqrt{\sigma_0}} \sqrt{\E [\lVert \Bar{\sigma} - \sigma \rVert_1]} + \frac{C}{\sigma_0} \E \left[\lVert \Bar{\sigma} - \sigma \rVert_2 \right] + \frac{C}{\sigma_0} \E \left[\lVert \Bar{m} - m \rVert_2 \right],
    \end{align}
    where the expectation is taken over $S, \Bar{S}$ and $\epsilon$ and $\sigma_0$ is a preset lower bound of the standard deviation in $Q$.
\end{lemma}
\begin{proof}%[Proof of \cref{cor:kl}]
    According to Pinsker's inequality, the total variation distance can be bounded by the KL divergence of the distributions. We thus first bound the KL divergence. 
     \begin{align}
         \kl(Q_{S, \epsilon}, Q_{\Bar{S}, \epsilon}) &= 1^\top (\log \sigma - \log \Bar{\sigma}) + \frac{1}{2} \left(1^\top \frac{\Bar{\sigma}^2}{\sigma^2} - d + 1^\top \frac{(\Bar{m} - m)^2}{\sigma^2} \right) \\
        &\leq \frac{2\lVert \sigma - \Bar{\sigma} \rVert_1}{\sigma_0} + \frac{\lVert \sigma - \Bar{\sigma} \rVert_2^2}{2\sigma_0^2} + \frac{\lVert \Bar{m} - m \rVert_2^2}{2\sigma_0^2},
        \label{eq:kl-form}
    \end{align}
    where $\sigma_0$ is a preset minimum value for the standard deviation, i.e., $\forall k, \sigma_k \geq \sigma_0$ and $\bar{\sigma}_k \geq \sigma_0$.
    To derive \cref{eq:kl-form},
    let $ \beta_i = |\sigma_i - \Bar{\sigma}_i |$. Consider $1^\top (\log \sigma - \log \Bar{\sigma}) = \sum_i \log \frac{\sigma_i}{\Bar{\sigma}_i}$. For each entry $i$, if $\sigma_i - \beta_i \leq \sigma_0$, then $\log \frac{\sigma_i}{\Bar{\sigma}_i} \leq \log \frac{\beta_i + \sigma_0}{\sigma_0} = \log (1 + \frac{\beta_i}{\sigma_0}) \leq \frac{\beta_i}{\sigma_0}$; if $\sigma_i - \beta_i > \sigma_0$, then $\log \frac{\sigma_i}{\Bar{\sigma}_i} \leq \log \frac{\sigma_i}{\sigma_i - \beta_i} = \log (1 + \frac{\beta_i}{\sigma_i - \beta_i}) \leq \log (1 + \frac{\beta_i}{\sigma_0}) \leq \frac{\beta_i}{\sigma_0}$. Overall, $1^\top (\log \sigma - \log \Bar{\sigma}) \leq \sum_i \frac{\beta_i}{\sigma_0} = \frac{\lVert \sigma - \Bar{\sigma}_0 \rVert_1}{\sigma_0}$. For $1^\top \frac{\Bar{\sigma}^2}{\sigma^2} = \sum_i \frac{\Bar{\sigma}_i^2}{\sigma_i^2} \leq \sum_i \frac{(\sigma_i + \beta_i)^2}{\sigma_i^2} \leq \sum_i (1 + 2\frac{\beta_i}{\sigma_i} + \frac{\beta_i^2}{\sigma_i^2}) \leq \sum_i (1 + \frac{2\beta_i}{\sigma_0} + \frac{\beta_i^2}{\sigma_i}) = d + \frac{2\lVert \sigma - \Bar{\sigma} \rVert_1}{\sigma_0} + \frac{\lVert \sigma - \Bar{\sigma} \rVert_2^2}{\sigma_0^2}$. Thus,
    \begin{align}
        \errgen(\alg) &\leq 2C \E_{S, \Bar{S}, \epsilon} [\tv(Q_{S, \epsilon}, Q_{\Bar{S}, \epsilon})] \\
        &\leq C \E_{S, \Bar{S}, \epsilon} \sqrt{2 \kl (Q_{S, \epsilon}, Q_{\Bar{S}, \epsilon})} \\
        &\leq \frac{2C}{\sqrt{\sigma_0}} \sqrt{\E [\lVert \Bar{\sigma} - \sigma \rVert_1]} + \frac{C}{\sigma_0} \E \left[\lVert \Bar{\sigma} - \sigma \rVert_2 \right] + \frac{C}{\sigma_0} \E \left[\lVert \Bar{m} - m \rVert_2 \right].
    \end{align}
\end{proof}

\cref{thm:errgen} holds only for bounded loss functions. 
We next introduce the upper bound for unbounded Lipschitz loss functions using Wasserstein distance \citep{Villani2008OptimalTO}.
\begin{definition}
    Suppose loss function $\loss(\theta, z)$ is $K$-Lipschitz with respect to $\theta$, i.e., $\frac{|\loss(\theta, z) - \loss(\theta', z)|}{\lVert \theta - \theta' \rVert} \leq K$ for all $z$.
\end{definition}
The Wasserstein-$p$ distance between two distributions $\mu$ and $\nu$ is defined as:
\begin{align}
    W_p(\mu, \nu) = \inf_{\gamma \in \Gamma(\mu, \nu)} \left( \E_{(x, y) \sim \gamma} d(x, y)^p \right)^{1/p},
\end{align}
where $d(x,y)$ is some distance and $\Gamma(\mu, \nu)$ is the set of all couplings of $\mu$ and $\nu$, i.e., for $\gamma \in \Gamma(\mu, \nu)$, $\int_y \gamma(x, y) = \mu(x)$ and $\int_x \gamma(x, y) = \nu(y)$. 
%We use Euclidean distance in the following unless otherwise specified.
In the following we use the Euclidean distance.
According to Kantorovich duality \citep{Villani2008OptimalTO},
\begin{align}
    W_1(\mu, \nu) = \sup_{f, \text{Lip}(f)\leq 1} \int f d\mu(x) - \int f d\nu(y).
\end{align}
Inspired by \citet{ipm}, we derive the bound through the Wasserstein distance:
%\begin{lemma}
%\label{thm:wasserstein}
%    Suppose the loss function $\loss(w, z)$ is $K$-Lipschitz. Let $S$, $\Bar{S}$ denote two datasets that only differ at one element $z$ and $\Bar{z}$. The generalization error $\errgen(\alg)$ is upper bounded by 
%    \begin{align}
%        K \E_{S, \Bar{S}, \epsilon} [W_p(Q_{S, \epsilon}, Q_{\Bar{S}, \epsilon})]
%    \end{align}
%    for $p \geq 1$. Further, if $Q_{S, \epsilon} = \dN(m, \diag(\sigma^2))$ and $Q_{\Bar{S}, \epsilon} = \dN(\Bar{m}, \diag(\Bar{\sigma}^2))$ are Gaussian distributions, the generalization error is upper bounded by 
%    \begin{align}
%    \label{eq:wass-bound}
%        K \E \lVert m - \Bar{m} \rVert_2 + K \E \lVert \sigma - \Bar{\sigma} \rVert_2.
%    \end{align}
%\end{lemma}
\begin{lemma} [Bayes-Stability 2]
\label{thm:wasserstein}
    Suppose the loss function $\loss(w, z)$ is $K$-Lipschitz. Let $S$, $\Bar{S}$ denote two datasets that only differ at one element $z$ and $\Bar{z}$. 
    Then, for any $p \geq 1$, the generalization error $\errgen(\alg)$ is upper bounded by 
    \begin{align}
        K \E_{S, \Bar{S}, \epsilon} [W_p(Q_{S, \epsilon}, Q_{\Bar{S}, \epsilon})].
    \end{align}
    %Further, if $Q_{S, \epsilon} = \dN(m, \diag(\sigma^2))$ and $Q_{\Bar{S}, \epsilon} = \dN(\Bar{m}, \diag(\Bar{\sigma}^2))$ are Gaussian distributions, the generalization error is upper bounded by 
    %\begin{align}
    %\label{eq:wass-bound}
    %    K \E \lVert m - \Bar{m} \rVert_2 + K \E \lVert \sigma - \Bar{\sigma} \rVert_2.
    %\end{align}
\end{lemma}
\begin{proof}
    It is obvious that $\frac{1}{K} \loss(w, z)$ is 1-Lipschitz. 
    Using Kantorovich duality we have
    \begin{align*}
         &\errgen(\alg) \\
         &= \E_z \E_{\Bar{z}} \left[ \E_{w\sim Q_{\Bar{z}}}[\loss(w, z)] - \E_{w\sim Q_z}[\loss(w, z)]\right] \\
         &\leq K \E_{z, \Bar{z}} \sup_{f, \text{Lip}(f)\leq 1} \E_{w\sim Q_{\Bar{z}}}[f(w)] - \E_{w\sim Q_{z}}[f(w)] \\ 
         &\leq K \E_{z, \Bar{z}} \E_{S_{n-1}, \epsilon} \\
         & \qquad \qquad \sup_{f, \text{Lip}(f)\leq 1}  \E_{w \sim Q_{\Bar{S}, \epsilon}}[f(w)] - \E_{w \sim Q_{S, \epsilon}}[f(w)] \\
         &= K \E_{z, \Bar{z}, S_{n-1}, \epsilon} W_1(Q_{\bar{S}, \epsilon}, Q_{S, \epsilon}) \\
         &\leq K \E_{z, \Bar{z}, S_{n-1}, \epsilon} W_p (Q_{\bar{S}, \epsilon}, Q_{S, \epsilon}).
    \end{align*}
    The third line is because of the convexity of supremum. The last inequality follows the Holder's inequality, which states that $\E[|XY|] \leq \E[|X|^p]^{\frac{1}{p}} \E[|Y|^q]^{\frac{1}{q}}$ for $p, q \geq 1$ and $\frac{1}{p} + \frac{1}{q} = 1$. Thus $\E_{x, y \sim \gamma}[d(x, y) \cdot 1] \leq \left(\E_{x, y \sim \gamma}[d(x, y)^p]\right)^{\frac{1}{p}} \left( \E[1^{q}] \right)^{\frac{1}{q}} = \left(\E_{x, y \sim \gamma} d(x, y)^p \right)^{\frac{1}{p}}$. 
    Taking infimum on both sides, we have proved the inequality.
\end{proof}

As in the previous case we can bound the stability using parameter differences. In particular, 
letting $p=2$ and using the Wasserstein-2 distance for Gaussian distributions \citep{Gauss-wass}, we immediately have: 
%proved \cref{eq:wass-bound}.
\begin{lemma}
\label{thm:wasserstein2}
Under the condition of \cref{thm:wasserstein},
if $Q_{S, \epsilon} = \dN(m, \diag(\sigma^2))$ and $Q_{\Bar{S}, \epsilon} = \dN(\Bar{m}, \diag(\Bar{\sigma}^2))$, the generalization error is upper bounded by 
    \begin{align}
    \label{eq:wass-bound}
        K \E \lVert m - \Bar{m} \rVert_2 + K \E \lVert \sigma - \Bar{\sigma} \rVert_2.
    \end{align}
\end{lemma}

%\cref{cor:kl} and \cref{thm:wasserstein2} upper bound the generalization error with the expectation of the difference of the parameters. 
%In the next section, we analyze the difference and efficiently compute it.

%\section{Stability Bounds}
\subsection{Bounds on Expected Parameter Differences}
\label{sec:bounds}
In this section, we draw upon the approach from \citet{HardtRS16} and \citet{SGD-stability2}, which bounds parameter differences for stochastic gradient descent.
Let $\theta_t$ be the parameter of $Q_{S, \epsilon}$ at step $t$ and $\Bar{\theta}_t$ be the parameter of $Q_{\Bar{S}, \epsilon}$ at step $t$. Let $G_t$ denote the update rule of stochastic gradient descent with learning rate $\alpha_t$,
\begin{align}
    \theta_{t} = G_t(\theta_{t-1}, S, \epsilon_{t}) = \theta_{t-1} - \alpha_t \nabla_\theta F(\theta_{t-1}, S, \epsilon_{t}).
\end{align}
Recall that $\epsilon_t$ contains all randomness at step $t$ and $\nabla F(\theta_{t-1}, S, \epsilon_t)$ is the approximation of $\nabla F(\theta_{t-1}, S)$. We make the following assumption \citep{HardtRS16,SGD-stability2} on the update rule:
\begin{definition}
    An update rule is $\eta$-expansive if $\sup_{\theta, \theta'}\frac{\lVert G(\theta, S, \epsilon) - G(\theta', S, \epsilon) \rVert}{\lVert \theta - \theta' \rVert} \leq \eta$ for any $S$ and $\epsilon$.
\end{definition}
The following theorem adapts the argument of \citet{SGD-stability2} to bound parameter differences as a function of expected gradient differences.
\begin{theorem}
\label{thm:diff}
    Given an algorithm that optimizes parameters $\theta$ using stochastic gradient descent, suppose it is $\eta_t$-expansive for step $t$. Let $S$ and $\Bar{S}$ be two random datasets of size $n$ that only differ at one element $z$ and $\Bar{z}$, and $\theta_T$ and $\Bar{\theta}_T$ denote the outputs under the same $\epsilon$.
    Then the expected difference of $\theta_T$ and $\Bar{\theta}_T$ satisfies 
    \begin{align}
    \label{eq:diff}
        \E_{S, \Bar{S}, \epsilon} [\lVert \theta_T - \Bar{\theta}_T \rVert]
        \leq \frac{1}{n} \sum_{t=1}^T \left(\prod_{i=t+1}^T \eta_i \right) \alpha_t \E_{S, \epsilon, \Bar{z}}[\Delta_t],
    \end{align}
    where $\Delta_t = \lVert \nabla F(\theta_{t-1}, \Bar{z}, \epsilon_t) - \nabla F(\theta_{t-1}, z, \epsilon_t) \rVert$.
\end{theorem}
%\begin{proof}[Proof sketch]
%    The main idea is that, at each step $t$, the difference in the parameter vectors due to $z$ and $\Bar{z}$ is expanded thereafter with a total expansion factor $\prod_{i=t+1}^T \eta_i$. The total difference between $\theta_T$ and $\Bar{\theta}_T$ accumulates over $T$ steps as the expanded difference at each step is propagated forward. Thus, the overall bound on the expected difference is a sum over the per-step expanded differences. The detailed proof is in \cref{sec:proof-bound}.
%\end{proof}
\begin{proof}%[Proof of \cref{thm:diff}]
    Let $S_t$ and $\Bar{S}_t$ denote the subset at step $t$. With respect to the same $\epsilon_t$ (including the same batch sequence), $S_t$ and $\bar{S}_t$ have at most one different element.
    We have two cases:
    \begin{itemize}
    \item Case 1: the different element is not selected, hence $S_t = \Bar{S}_t$, and since $G$ is $\eta_t$ expansive:
        \begin{align*}
            \lVert \theta_t - \Bar{\theta}_t \rVert \leq \eta_t \lVert \theta_{t-1} - \Bar{\theta}_{t-1} \rVert.
        \end{align*}
        \item Case 2: the different element is selected.
        \begin{align*}
            \lVert \theta_t - \Bar{\theta}_t \rVert &= \lVert (\theta_{t-1} - \alpha_t \nabla F(\theta_{t-1}, S_t, \epsilon_{t})) \\
            & \quad - (\Bar{\theta}_{t-1} - \alpha_t \nabla F(\Bar{\theta}_{t-1}, \Bar{S}_t, \epsilon_t)) \rVert \\
            &= \lVert (\theta_{t-1} - \alpha_t \nabla F(\theta_{t-1}, \Bar{S}_t, \epsilon_t)) \\
            & \quad - (\Bar{\theta}_{t-1} - \alpha_t \nabla F(\Bar{\theta}_{t-1}, \Bar{S}_t, \epsilon_t)) \\
            & \quad + \alpha_t (\nabla F(\theta_{t-1}, \Bar{S}_t, \epsilon_t) - \nabla F(\theta_{t-1}, S_t, \epsilon_t)) \rVert \\
            &\leq \eta_t \lVert \theta_{t-1} - \Bar{\theta}_{t-1} \rVert \\
            &\quad + \alpha_t \lVert \nabla F(\theta_{t-1}, \Bar{S}_t, \epsilon_t) - \nabla F(\theta_{t-1}, S_t, \epsilon_t) \rVert.
        \end{align*}
        Since $\Bar{S}_t$ and $S_t$ only differs at one element, $\lVert \nabla F(\theta_{t-1}, \Bar{S}_t, \epsilon_t) - \nabla F(\theta_{t-1}, S_t, \epsilon_t) \rVert = \frac{1}{b} \lVert \nabla F(\theta_{t-1}, \Bar{z}, \epsilon_t) - \nabla F(\theta_{t-1}, z, \epsilon_t) \rVert = \frac{1}{b} \Delta_t$, where $b$ is the batch size.
    \end{itemize}
    Thus, 
    \begin{align}
        \lVert \theta_T - \Bar{\theta}_T \rVert &\leq \eta_T \lVert \theta_{T-1} - \Bar{\theta}_{T-1} \rVert + \mathbbm{1}_{z \in S_T} \frac{\alpha_T}{b} \Delta_T \\
        &\leq \frac{1}{b} \sum_{t=1}^T \left(\prod_{i=t+1}^T \eta_i \right) \mathbbm{1}_{z\in S_t} \alpha_t \Delta_t,
    \end{align}
    where the base case is $\theta_0 = \bar{\theta}_0$.
    Since the probability that $z \in S_t$ is $\frac{b}{n}$, then the expected difference is 
    \begin{align}
        \E \lVert \theta_T - \Bar{\theta}_T \rVert \leq \frac{1}{n} \sum_{t=1}^T \left(\prod_{i=t+1}^T \eta_i \right) \alpha_t \E_{S, \epsilon, \bar{z}} [\Delta_t].
    \end{align}
\end{proof}

\cref{thm:diff} provides a way to compute the bound {\em exactly}. As we show in the experiments this allows us to obtain tight generalization bounds which are not possible otherwise. For completeness, the following Corollary provides an asymptotic upper bound using stronger requirements.
%for the objective function and a specific learning rate schedule.  
The proof follows the construction of \citet{SGD-stability2} and is included in \cref{sec:proof-bound}.
\begin{corollary}
\label{cor:logT}
    Suppose $\nabla F(\theta, S, \epsilon)$ is $L$-Lipschitz and $\beta$-bounded, then with learning rate $\alpha_t = \frac{c}{(t+2) \log (t+2)}$ where $c$ is chosen that $cL < 1$, $\E \lVert \theta_T - \Bar{\theta}_T \rVert \leq O(\frac{\log T}{n})$.
\end{corollary}

\subsection{Discussion: Stability vs.\ PAC-Bayes Bounds}
\label{sec:pac-bayes}

As mentioned above, prior work has developed  PAC-Bayes Bounds for certain variants of VI. In this section we review some of these bounds and discuss the qualitative differences between the two types of bounds.

\citet{Germain2016pac} provides a generalization error bound for a $C$-bounded loss function as follows: with probability $1 - \delta$,
\begin{align}
    \frac{1}{\lambda} \left(\kl(Q_S \parallel P) + \log \frac{1}{\delta}\right) + \frac{\lambda C^2}{2n}.
\end{align}
By optimizing $\lambda$ as $\lambda = \frac{1}{C} \sqrt{2n \left(\kl(Q_S \parallel P) + \log \frac{1}{\delta}\right)}$, we obtain the following bound:
\begin{align}
    C \sqrt{\frac{2 \left(\kl(Q_S \parallel P) + \log \frac{1}{\delta}\right)}{n}}.
\end{align}

On the other hand, \citet{pac-bayes-book} provides a similar bound in the form:
\begin{align}
\label{eq:sqrt}
    C\sqrt{\frac{\kl(Q_S \parallel P) + \log \frac{n}{\delta}}{2(n-1)}}.
\end{align}
%This bound is nearly a factor of 2 smaller than the previous one. 
that can be tighter in some cases.
In these results, 
the ``prior" $P$ is only required to be data independent and is not directly related to the algorithm. 
Therefore, for Bayesian algorithms, one can pick a different $P$ other than the prior used in the objective function. 
In the experimental illustration,
we explore using both the prior and the initialization $Q_0$ so as to obtain the tightest possible bound.
% when evaluating the PAC-Bayes bounds

Additionally, \citet{nonvacuous} proposed a non-vacuous bound specifically for the 0-1 loss. By employing a union-bound argument, 
where the prior variance is set as $\lambda = c \exp(-j / b)$ for $j \in \mathbb{N}$ and fixed $b$ and $c$,
they ensure that the generalization error can be bounded, with probability $1-\delta$, by
\begin{align}
\label{eq:bre}
    \sqrt{\frac{\kl(Q_S \parallel \mathcal{N}(m_0, \lambda I)) + 2 \log \left(b \log \frac{c}{\lambda}\right) + \log \frac{\pi^2 n}{6 \delta}}{2(n-1)}},
\end{align}
where $m_0$ denotes the random initialization of the mean parameter.

These bounds have been leveraged to develop efficient Bayesian algorithms, by explicitly optimizing the sum of the training set loss and the bound, 
which can be seen to have a similar form to VI and therefore interpreted as variants of VI.
On the other hand, PAC-Bayes bounds are valid for any distribution within the specified family. 
They can therefore be applied to the output of VI directly. 
%it is important to recognize that these bounds apply to any distribution within the specified family. 
%Therefore, even if a posterior distribution is not directly optimized using these bounds, its generalization error will still conform to them. 

From this perspective, our bounds are more restricted in that they are valid only for the output of  a certain class of optimization problems when optimized by SGD. In addition, the stability bound in \eqref{eq:diff} grows with the number of optimization steps $T$ which can make it less attractive,
and for a fixed dataset this may necessitate the use of larger batch sizes to reduce $T$.
On the other hand, the dependence on dataset size in \eqref{eq:diff} is $\frac{1}{n}$ whereas the one in the PAC-Bayes bounds is $\frac{1}{\sqrt{n}}$ so our bound has the potential to be tighter for large datasets. 
Appendix~\ref{app:pacbayes} shows an example where the PAC-Bayes bound can grow arbitrarily in a case where the stability bound is tight. 
Overall, the two approaches can have advantages in different situations and both contribute to our understanding of generalization performance of algorithms. 


%\paragraph{Example of a PAC-Bayes Bound Being Worse Than a Stability Bound}
%
%Consider a simple logistic regression scenario where the data takes on two possible values, \( x \in \{-1, 1\} \), and the corresponding labels are \( y \in \{0, 1\} \), i.e., there are only two possible examples $(x=-1, y=0)$ and $(x=1, y=1)$. 
%The dataset can contain duplicate elements. The log-likelihood in this case is given by:
%\begin{align}
%    \log p(y \mid w, x) &= -y \log (1 + \exp{(-wx)}) \\
%    & \quad - (1-y) \log (1 + \exp(wx)).
%\end{align}
%
%Assume we use a Bayesian approach to learn this model, with \( q(w) = \mathcal{N}(m, \sigma^2) \). For simplicity, we assume \(\sigma^2\) is fixed. Recall that for any objective function, we can always evaluate PAC-Bayes bounds.
%% It is important to note that for the PAC-Bayes bound, the specific objective used during training does not impact the bound.
%
%Suppose our objective is \( F(m, (x, y)) = \mathbb{E}_{q(w)}[-\log p(y \mid w, x)] \). Considering the gradient with respect to \(m\), we have the following identity 
%%\citep{rezende2014stochastic, Sheth2015, opper2008variational}:
%\citep{rezende2014stochastic, opper2008variational}:
%\begin{align}
%    \nabla_m F(m, (x, y)) &= \mathbb{E}_{q(w)} [\nabla_w \log p(y \mid w, x)].
%\end{align}
%
%Observe that:
%\begin{align*}
%    \nabla_w -\log p(y=1 \mid w, x=1)
%    =& \nabla_w \log (1+\exp(-w)) \\
%    =& -\frac{\exp{(-w)}}{1+\exp(-w)}, \\
%    \nabla_w -\log p(y=0 \mid w, x=-1) 
%    =& \nabla_w \log (1+\exp(-w)) \\
%    =& -\frac{\exp(-w)}{1+\exp(-w)}, 
%\end{align*}
%we can see that
%\begin{align}
%    &\nabla_w -\log p(y=1 \mid w, x=1) \nonumber \\
%    = &\nabla_w -\log p(y=0 \mid w, x=-1) < 0. 
%    \label{eq:same}
%\end{align}
%Therefore,
%if we run stochastic gradient descent with a constant learning rate for sufficiently many steps, we reach a solution where \(m \rightarrow +\infty\).
%
%Now, suppose the initial prior is \( P_0 = \mathcal{N}(0, \sigma^2) \). The KL divergence will eventually become:
%\begin{align}
%    \kl(\mathcal{N}(m, \sigma^2) \parallel \mathcal{N}(0, \sigma^2)) &= \frac{m^2}{2\sigma^2} \rightarrow +\infty.
%\end{align}
%
%However, if we consider the stability bound, which is based on the gradient difference, the situation changes. It’s clear that if \( z = \bar{z} \) (whether \( x = \bar{x} = 1, y = \bar{y} = 1 \) or \( x = \bar{x} = -1, y = \bar{y} = 0 \)), the gradient difference will be zero. Thus, we only need to consider the case where \( z = (1, 1) \) and \( \bar{z} = (-1, 0) \). As shown in \cref{eq:same}, the gradients are the same in this scenario as well.
%
%Therefore, using the stability bound, the generalization error will be zero. In contrast, the PAC-Bayes bound gives a value of \( \infty \), making the stability bound significantly more effective.
%


\section{Experimental Illustration}


In this section we explore the potential of stability based bounds to capture generalization error 
and compare them to PAC Bayes bounds.
We also evaluate the expansion rate that appears in the bound showing that it can be small, and hence better in practice than the use of the asymptotic bounds. 

We adopted the experimental setup used by \citet{Li} and \citet{banerjee} and conducted our experiments on CIFAR10 using the same CNN model that has been employed in these works. 
For algorithms, 
we use the ELBO (\cref{eq:elbo}) and DLM variant (\cref{eq:dlm}) with a KL divergence coefficient of 0.1, a value that has been demonstrated to yield superior results in previous studies \citep[e.g.,][]{cold-posterior}.
Our optimization was performed using the SGD optimizer with an initial learning rate of 0.005, momentum of 0.99, and we reduced the learning rate by a factor of 0.9 every 5 epochs thereafter. We select the batch size to be 1000 and set $\sigma_0=0.01$. 
%We employed classification error for evaluation purposes and set $C$ to 1. 
All experiments are run on a single NVIDIA Tesla V100 PCIe 32 GB GPU.

%The goals of the experiment is to illustrate the stability bound and its relation to the PAC-Bayes bounds under different conditions.
We perform two sets of experiments. 
In the first we test the performance of ELBO 
with or without data augmentation (random cropping and horizontal flipping \citep{shorten2019survey}) as well as random label perturbations,
comparing the generalization error (measured by 0-1 loss) and our bound (\cref{eq:kl-bound}) with $C=1$ under these situations. 
In the second, 
following the observation by \citet{dlm-bnn} that DLM (\cref{eq:dlm}) does not perform as well as ELBO in Bayesian neural networks,
we use the bounds to compare ELBO and DLM in terms of log loss. 


The primary goals of our experiments were to demonstrate the following key points.
Our bound is non-vacuous in successful learning cases and becomes 
vacuous when the dataset contains a sufficiently high proportion of random labels.
%, effectively highlighting cases where the data lacks meaningful patterns. 
In addition, our bound accurately reflects the reduction in generalization error with data augmentation.
Finally, our bound can potentially provide an explanation for the failure of DLM, suggesting that its lower stability might be the cause of higher generalization error.
For these experiments, the stability bound is both tighter and has more explanatory power than the PAC-Bayes bounds, hence demonstrating the utility of the new derivations.


%\begin{itemize}
%    \item Our bound is both non-vacuous in successful learning cases and becomes 
%    \item Our bound becomes vacuous when the dataset contains a sufficiently high proportion of random labels, effectively highlighting cases where the data lacks meaningful patterns.
%    \item Our bound accurately reflects the reduction in generalization error with data augmentation, whereas the PAC-Bayes bounds do not capture this effect correctly.
%    \item Our bound is informative in showing that DLM has a worse generalization error than ELBO, a distinction not reflected by the PAC-Bayes bounds. 
%\end{itemize}

\begin{figure*}[t]
    \centering
    \begin{subfigure}[b]{0.40\textwidth}
         \centering
         \includegraphics[width=\textwidth]{figs/CIFAR10-expansion.png}
    \end{subfigure}
    \begin{subfigure}[b]{0.40\textwidth}
         \centering
         \includegraphics[width=\textwidth]{figs/CIFAR10-DLM-expansion.png}
    \end{subfigure}
    \caption{Cumulative expansion rates under various conditions. The left panel displays expansion rates with and without data augmentation, comparing cases with random labels (50\% random, labeled as 0.5) and without random labels (labeled as 0.0). The right panel shows expansion rates across different algorithms with data augmentation and no random labels. The shaded areas represent the standard deviation across 10 runs.}
    \label{fig:expansion}
\end{figure*}


\paragraph{Expansion Rate}
We start by evaluating the expansion rate which is needed for the exact bound. 
To perform this,
%To evaluate the expansion rate, 
we randomly initialize two models and then run the same algorithm with the same batch sequence. We keep track of the norm of the parameter difference and compute the expansion rate at each step $t$. 
For simplicity, we take the maximum of the expansion rate of both $m$ and $\sigma$ (both $L_1$-norm and $L_2$-norm).


\cref{fig:expansion} shows the cumulative expansion rate under various conditions. It is evident that for each method the expansion rate increases more slowly as the number of steps increases, and the final rate shows minimal variance. 
We observe that without data augmentation the expansion rate quickly levels off. This occurs because the dataset is straightforward to learn, and once all data has been learned, the gradient approaches zero, causing the expansion rate to flatten. In contrast, with data augmentation, the expansion rate continues to grow. 
%Additionally, there is no clear relationship between random labels and the expansion rate. With data augmentation, the expansion rate on the dataset with random labels is lower than that without random labels. Conversely, without data augmentation, the expansion rate on the dataset with random labels is higher. 
We also observe that the expansion rate of DLM is slightly higher than that of ELBO.

For use in evaluating generalization bounds, 
we note that the final cumulative expansion rate is much smaller than the $\log T$ factor in \cref{cor:logT} in all cases and will therefore lead to tighter bounds in practice.
We therefore run this evaluation 10 times and use the mean value plus four standard deviation as the final value $\eta_t$.

\begin{figure*}[t]
    \centering
        \begin{subfigure}[b]{0.32\textwidth}
         \centering
         \includegraphics[width=\textwidth]{figs/CIFAR10-train_acc.png}
         \caption{Train error.}
    \end{subfigure}
    \begin{subfigure}[b]{0.32\textwidth}
         \centering
         \includegraphics[width=\textwidth]{figs/CIFAR10-test_acc.png}
         \caption{Test error.}
    \end{subfigure}
    \begin{subfigure}[b]{0.32\textwidth}
         \centering
         \includegraphics[width=\textwidth]{figs/CIFAR10-err_gen.png}
         \caption{Generalization error.}
    \end{subfigure}
    \begin{subfigure}[b]{0.32\textwidth}
         \centering
         \includegraphics[width=\textwidth]{figs/CIFAR10-stability.png}
         \caption{Stability Bound (\ref{eq:kl-bound}).}
    \end{subfigure}
    \begin{subfigure}[b]{0.32\textwidth}
         \centering
         \includegraphics[width=\textwidth]{figs/CIFAR10-prior-sqrt.png}
         \caption{PAC-Bayes (\ref{eq:sqrt}) with prior.}
    \end{subfigure}
    \begin{subfigure}[b]{0.32\textwidth}
         \centering
         \includegraphics[width=\textwidth]{figs/CIFAR10-bre.png}
         \caption{Tighter PAC-Bayes (\ref{eq:bre}).}
    \end{subfigure}
    \caption{Generalization error and bounds.}
    \label{fig:bound}
\end{figure*}

%\begin{figure*}[t]
%    \centering
%        \begin{subfigure}[b]{0.40\textwidth}
%         \centering
%         \includegraphics[width=\textwidth]{figs/CIFAR10-train_acc.png}
%         \caption{Train error.}
%    \end{subfigure}
%    \begin{subfigure}[b]{0.40\textwidth}
%         \centering
%         \includegraphics[width=\textwidth]{figs/CIFAR10-test_acc.png}
%         \caption{Test error.}
%    \end{subfigure}
%    \begin{subfigure}[b]{0.40\textwidth}
%         \centering
%         \includegraphics[width=\textwidth]{figs/CIFAR10-err_gen.png}
%         \caption{Generalization error.}
%    \end{subfigure}
%    \begin{subfigure}[b]{0.40\textwidth}
%         \centering
%         \includegraphics[width=\textwidth]{figs/CIFAR10-stability.png}
%         \caption{Bound (\cref{eq:kl-bound}).}
%    \end{subfigure}
%    \begin{subfigure}[b]{0.40\textwidth}
%         \centering
%         \includegraphics[width=\textwidth]{figs/CIFAR10-prior-sqrt.png}
%         \caption{PAC-Bayes (\cref{eq:sqrt}) with prior.}
%    \end{subfigure}
%    \begin{subfigure}[b]{0.40\textwidth}
%         \centering
%         \includegraphics[width=\textwidth]{figs/CIFAR10-bre.png}
%         \caption{Tighter PAC-Bayes (\cref{eq:bre}).}
%    \end{subfigure}
%    \caption{Generalization error and bounds.}
%    \label{fig:bound}
%\end{figure*}

\begin{figure*}[h]
    \centering
        \begin{subfigure}[b]{0.32\textwidth}
         \centering
         \includegraphics[width=\textwidth]{figs/CIFAR10-DLM-train_loss.png}
         \caption{Train loss.}
    \end{subfigure}
    \begin{subfigure}[b]{0.32\textwidth}
         \centering
         \includegraphics[width=\textwidth]{figs/CIFAR10-DLM-test_loss.png}
         \caption{Test loss.}
    \end{subfigure}    
    \begin{subfigure}[b]{0.32\textwidth}
         \centering
         \includegraphics[width=\textwidth]{figs/CIFAR10-DLM-err_gen_loss.png}
         \caption{Generalization error.}
    \end{subfigure}
    \begin{subfigure}[b]{0.32\textwidth}
         \centering
         \includegraphics[width=\textwidth]{figs/CIFAR10-DLM-wasserstein.png}
         \caption{Bound (\ref{eq:wass-bound}, without $K$).}
    \end{subfigure}
    \begin{subfigure}[b]{0.32\textwidth}
         \centering
         \includegraphics[width=\textwidth]{figs/CIFAR10-DLM-prior-sqrt.png}
         \caption{PAC-Bayes (\ref{eq:sqrt}) with prior.}
    \end{subfigure}
    \begin{subfigure}[b]{0.32\textwidth}
         \centering
         \includegraphics[width=\textwidth]{figs/CIFAR10-DLM-init-sqrt.png}
         \caption{PAC-Bayes (\ref{eq:sqrt}) with init $Q_0$.}
    \end{subfigure}
    \caption{Generalization error and bounds for ELBO and DLM with data augmentation and no random labels.}
    \label{fig:dlm-bound}
\end{figure*}


\paragraph{Generalization bounds: ELBO with data augmentation and random labels.}
To evaluate the bound with parameter differences (\cref{eq:diff}), we need to take expectations over $z$, $\Bar{z}$ and the randomness $\epsilon$. 
To perform this,
we randomly sample 50 pairs of $z$ and $\bar{z}$ from the training and test dataset, respectively. For 
%the randomness 
$\epsilon$, we conduct 10 independent runs, with each run selecting a random batch sequence and any other random samples required for optimization.

\cref{fig:bound} (a-c) present the train loss, test loss, and generalization error for ELBO in terms of 0-1 loss along with the stability bound (d) and PAC-Bayes bounds (e,f).
The generalization error is calculated as the absolute difference between the training error and the test error. For the stability bound, we set $C=1$. For PAC-Bayes bounds, we select $\delta=0.025$ and specifically for \cref{eq:bre}, we select $b=100$ and $c=0.1$ following the original paper. 

We first observe that the stability bound is non-vacuous except in the scenario without data augmentation and with 50\% random labels, where there is significant overfitting. The PAC Bayes bounds are less tight in all four scenarios. 
%and the PAC-Bayes bounds are also vacuous. 
Second, our bound induces the correct ranking over the four cases, and specifically shows that without noisy labels the generalization error is lower when data augmentation is used. 
The PAC Bayes bounds do not demonstrate the benefit of data augmentation in this case. 
%Second, we note that PAC-Bayes bounds fail to account for the benefits of data augmentation. While data augmentation reduces the generalization error, PAC-Bayes bounds show a higher bound with data augmentation compared to without it when there are no random labels. Therefore, in this case, our bound is both tighter and more informative.
%than the PAC-Bayes bounds.
Third, note that the smallest generalization error occurs in the case with both data augmentation and 50\% random labels. However, this does not imply the best performance on the test set; in this scenario, the training error converges to 0.5, and the test error is slightly above this value.
% (refer to \cref{fig:err} in the Appendix). 
Our bound captures this behavior well.

\paragraph{Generalization bounds: ELBO vs.\ DLM.}
\cref{fig:dlm-bound} (a-c) present train and test loss and generalization error in terms of log loss of ELBO vs.\ DLM,
and (d-f) present the 
stability and PAC-Bayes bounds.
When calculating the bound in \cref{eq:wass-bound}, we omit the Lipschitz constant $K$ due to the difficulty in its evaluation. Since the Lipschitz constant remains the same for a given loss function (though not necessarily for the objective), our focus is on the relative comparison between the two methods.
Our bound effectively captures the fact that DLM has a worse generalization error than ELBO. In contrast, the PAC-Bayes bounds are nearly identical for both methods.
%, providing little insight into the differences in generalization error. 
Our bound, which is based on the sum of the norms of the gradient differences, underscores the potential instability of the DLM algorithm
for Bayesian neural networks, which might explain its inferior performance for such models.
%across different data points.

\section{Conclusion and Future Work}
In this study, we presented a new generalization bound for variational inference by leveraging recent advances in stability-based bounds for Stochastic Gradient Langevin Dynamics (SGLD). Our approach extends the stability argument of stochastic gradient descent to 
a family of algorithms which includes
variational inference, addressing both mean and variance parameters.
Empirical evaluations demonstrated that our bound produces meaningful results with large neural network models and effectively captures generalization error in scenarios involving random labels and data augmentation.

This work opens several promising avenues for future research. The general applicability of our approach suggests that the bound could be extended to various Bayesian algorithms, such as $\text{PAC}^2$ variational learning \citep{Masegosa20}. However, a limitation of our approach is that the bound is primarily effective for algorithms optimized via stochastic gradient descent. For more advanced optimizers like Adam, characterizing parameter differences becomes significantly more challenging.

\section*{Acknowledgments}
This work was partly supported by NSF under grant 2246261. The experiments in this paper were run on the Big Red computing system at Indiana University, supported in part by Lilly Endowment, Inc., through its support for the Indiana University Pervasive Technology Institute.

\FloatBarrier

\bibliography{main}
\bibliographystyle{plainnat}

\appendix
\appendix
\section*{Appendix}
\section{Discussion: Scope and Ethics}
\label{appendix:scope}
In this work, we evaluate our method on six core scene-aware tasks: existence, count, position, color, scene, and HOI reasoning. We select these tasks as they represent core aspects of multimodal understanding which are essential for many applications. Meanwhile, we do not extend our evaluation to more complex reasoning tasks, such as numerical calculations or code generation, because SOTA diffusion models like SDXL are not yet capable of handling these tasks effectively. Fine-tuning alone cannot overcome the fundamental limitations of these models in generating images that require symbolic logic or complex reasoning. Additionally, we avoid tasks with ethical concerns, such as generating images of specific individuals (e.g., for celebrity recognition task), to mitigate risks related to privacy and misuse. Our goal was to ensure that our approach focuses on technically feasible and responsible AI applications. Expanding to other tasks will require significant advancements in diffusion model capabilities and careful consideration of ethical implications.

\section{Limitations and Future Work}
While our Multimodal Context Evaluator proves effective in enhancing the fidelity of generated images and maintaining diversity, \method is built using pre-trained diffusion models such as SDXL and MLLMs like LLaVA, it inherently shares the limitations of these foundation models. \method still faces challenges with complex reasoning tasks such as numerical calculations or code generation due to the symbolic logic limitations inherent to SDXL. Additionally, during inference, the MLLM context descriptor occasionally generates incorrect information or ambiguous descriptions initially, which can lead to lower fidelity in the generated images. Figure~\ref{fig:failure} further illustrates these observations.

\method currently focuses on single attributes like count, position, and color as part of the multimodal context. This is because this task alone poses significant challenges to existing methods, which \method effectively addresses. A potential direction for future work is to broaden the applicability of \method to synthesize images with multiple scene attributes in the multimodal context as part of compositional reasoning tasks.


\begin{figure}[!h]
    \centering
    \includegraphics[width=\linewidth]{figures/failures.pdf}
    \caption{Failure cases of \method. (a) Our method fails due to the symbolic logic limitation of existing pre-trained SDXL. (b) Initially incorrect descriptions generated by MLLMs lead to low fidelity of generated images. (c) Context description generated by MLLMs is ambiguous and does not directly relate to the text guidance, the spoon can be both inside or outside the bowl.}
    \label{fig:failure}
\end{figure}

\section{Prompt Templates}
\label{appendix:prompts}
Figure~\ref{fig:prompt_templates}~(a-c) showcases the prompt templates used by \method to fine-tune diffusion models specifically on each task including VQA, HOI Reasoning, and Object-Centric benchmarks. It's worth noting that we designed the prompt such that it provides detailed instruction to MLLMs on which scene attributes to focus. We also evaluate the effectiveness of our designed prompt templates by fine-tuning \method with a generic prompt as illustrated in Figure~\ref{fig:prompt_templates}~(d). Table~\ref{table:prommpt} indicates that without using our designed prompt template, the MLLM is not properly instructed to generate specific context description thus leading to reduced performance after fine-tuning on MME tasks. We believe that when using a generic prompt, MLLM is not able to receive sufficient grounding about the multimodal context leading to information loss on key scene attributes.


\begin{table}[!h]
\centering
\footnotesize
\caption{Effectiveness of the prompt template on fine-tuning \method on MME Perception.}
\resizebox{1\linewidth}{!}{
\begin{tabular}{clcccccccccc}
\toprule
 \textbf{MLLM} & \multirow{2}{*}{\textbf{\method}} & \multicolumn{2}{c}{\textbf{Existence}} & \multicolumn{2}{c}{\textbf{Count}} & \multicolumn{2}{c}{\textbf{Position}} & \multicolumn{2}{c}{\textbf{Color}} & \multicolumn{2}{c}{\textbf{Scene}} \\
 \textbf{Name} & & ACC & ACC+ & ACC & ACC+ & ACC & ACC+ & ACC & ACC+ & ACC & ACC+ \\
 \midrule
 \multirow{3}{*}{\makecell{\textbf{LLaVA }  \\ \textbf{v1.6 7B} \\ \citep{liu2024improved}}}
 &w/ prompt template & \textbf{96.67}  & \textbf{93.33}  & \textbf{83.33}  & \textbf{70.00}  & \textbf{81.67}  & \textbf{66.67} & \textbf{95.00}  & \textbf{93.33}  & \textbf{87.75} & \textbf{74.00} \\
 \cmidrule{2-12}
 & \multirow{2}{*}{w/ generic prompt} & 91.67 & 83.33 & 75.00 & 56.67 & \textbf{81.67} & 63.33 & 91.67 & 83.33 & 87.25 & 73.00 \\
 & & {\scriptsize \color{red}\textbf{$\downarrow$ 5.00}} & {\scriptsize \color{red}\textbf{$\downarrow$ 10.00}} & {\scriptsize \color{red}\textbf{$\downarrow$ 8.33}} & {\scriptsize \color{red}\textbf{$\downarrow$ 13.33}} & - &  {\scriptsize \color{red}\textbf{$\downarrow$ 3.34}} & {\scriptsize \color{red}\textbf{$\downarrow$ 3.33}} & {\scriptsize \color{red}\textbf{$\downarrow$ 10.00}} & {\scriptsize \color{red}\textbf{$\downarrow$ 0.50}} & {\scriptsize \color{red}\textbf{$\downarrow$ 1.00}}\\
 \midrule
 \multirow{3}{*}{\makecell{\textbf{InternVL }  \\ \textbf{2.0 8B}\\ \citep{chen2024internvl}}} 
 &w/ prompt template & \textbf{98.33}  & \textbf{96.67} & \textbf{86.67} & \textbf{73.33}  & \textbf{78.33}  & \textbf{63.33}  & \textbf{98.33}  & \textbf{96.67}  & \textbf{86.25} & \textbf{71.00} \\
 \cmidrule{2-12}
 & \multirow{2}{*}{w/ generic prompt} & 91.67 & 83.33 & 80.00 & 60.00 & 71.67 & 50.00 & 91.67 & 83.33 & 84.50 & 69.00 \\
 & & {\scriptsize \color{red}\textbf{$\downarrow$ 6.66}} &  {\scriptsize \color{red}\textbf{$\downarrow$ 13.34}} & {\scriptsize \color{red}\textbf{$\downarrow$ 6.67}} & {\scriptsize \color{red}\textbf{$\downarrow$ 13.33}} & {\scriptsize \color{red}\textbf{$\downarrow$ 6.66}} & {\scriptsize \color{red}\textbf{$\downarrow$ 13.33}} & {\scriptsize \color{red}\textbf{$\downarrow$ 6.66}} & {\scriptsize \color{red}\textbf{$\downarrow$ 13.34}} & {\scriptsize \color{red}\textbf{$\downarrow$ 1.75}} & {\scriptsize \color{red}\textbf{$\downarrow$ 2.00}}\\
\bottomrule
\end{tabular}
}
\label{table:prommpt}
\end{table}

\begin{figure}[!h]
    \centering
    \includegraphics[width=\linewidth]{figures/prompt_template.pdf}
    \caption{Prompt templates (a-c) used by \method to fine-tune the diffusion model on each task including VQA, HOI Reasoning, and Object Centric benchmarks. The generic prompt (d) is also included to evaluate the effectiveness of prompt template.}
    \label{fig:prompt_templates}
\end{figure}
\section{Inference Pipeline}
\label{appendix:inference}
In the inference pipeline of \method (Figure~\ref{fig:inference}), the text guidance $\mathbf{g}$ includes only the question corresponding to the reference image $\mathbf{x}$. The answer is excluded for fair evaluation. Moreover, we remove Multimodal Context Evaluator, and the generated image $\hat{\mathbf{x}}$ is the final output.
\begin{figure}[!h]
    \centering
    \includegraphics[width=\linewidth]{figures/inference.pdf}
    \caption{Inference pipeline of \method}
    \label{fig:inference}
\end{figure}

\begin{figure}[!h]
    \centering
    \includegraphics[width=\linewidth]{figures/diversity_compact_caption.pdf}
    \vspace{-5mm}
    \caption{Examples of context description from MLLM in the inference pipeline where answers are not included in text guidance.}
    \label{fig:diversity_compact_caption}
\end{figure}



\section{Ablation Study on BLIP-2 QFormer}
Our design choice to leverage BLIP-2 QFormer in \method as the multimodal context evaluator facilitates the formulation of our novel Global Semantic and Fine-grained Consistency Rewards. These rewards enable \method to be effective across all tasks as seen in Table~\ref{table:clip}. On replace with a less powerful multimodal context encoder such as CLIP ViT-G/14, we can only implement the global semantic reward as the cosine similarity between the text features and generated image features. As a result, while the setting can maintain performance on coarse-level tasks such as Scene and Existence, there is a noticeable decline on fine-grained tasks like Count and Position. This demonstrates the effectiveness of our design choices in \method and shows that using less powerful alternatives, without the ability to provide both global and fine-grained alignment, affects the fidelity of generated images.

\begin{figure*}[t]
\centering
\includegraphics[width=15.5cm]{figures/clip_zeroshot.png}\\
\caption{CLIP a) training and b) zero-shot inference framework}
\label{fig:clip} 
\end{figure*}


\section{Additional Evaluation on MME Artwork}

To explore the method's ability to work on tasks involving more nuanced or abstract text guidance beyond factual scene attributes, we evaluate \method on an additional task of MME Artwork. This task focuses on image style attributes that are more nuanced/abstract such as the following question-answer pair -- Question: ``Does this artwork exist in the form of mosaic?'', Answer: ``No''.

Table~\ref{table:artwork_reasoning} summarizes the evaluation. We can observe that \method outperforms all existing methods on both ACC and ACC+, implying its higher effectiveness in generating images with high fidelity (in this case, image style preservation) compared to existing methods. This provides evidence that \method can generalize to tasks involving abstract/nuanced attributes such as image style. Figure~\ref{fig:artwork} further shows qualitative comparison between image generation methods on the MME Artwork task.

\begin{table}[h]
\centering
\caption{Comparison on Artwork benchmark and Visual Reasoning task. \method outperforms SOTA image generation and augmentation techniques.}
\resizebox{\linewidth}{!}{
\begin{tabular}{@{}l@{ }ccccccc@{}}
\toprule
\textbf{Method} & \textbf{Real only} & \textbf{RandAugment} &  \textbf{Image Variation} & \textbf{Image Translation} & \textbf{Textual Inversion} & \textbf{I2T2I SDXL} & \textbf{\method} \\
\midrule
\textbf{Artwork ACC} & 69.50 & 69.25 & 69.00 & 67.00 & 66.75 & 68.00 & \textbf{70.25} \\
\textbf{Artwork ACC+} & 41.00 & 41.00 & 40.00 & 38.00 & 37.50 & 38.00 & \textbf{41.50} \\
\midrule
\textbf{Reasoning ACC} & 69.29 & 67.86 & 69.29 & 69.29 & 67.14 & 72.14 & \textbf{72.86} \\
\textbf{Reasoning ACC+} & 42.86 & 40.00 & 41.40 & 40.00 & 37.14 & 47.14 & \textbf{48.57} \\

\bottomrule
\end{tabular}
}
\label{table:artwork_reasoning}
\end{table}


\begin{figure}[!h]
    \centering
    \includegraphics[width=\linewidth]{figures/artwork.pdf}
    \caption{Qualitative comparison on the Artwork task between image generation method. \method can preserve both diversity and fidelity of the reference image in a more abstract domain.}
    \label{fig:artwork}
\end{figure}


\section{Additional Evaluation on MME Commonsense Reasoning}
We have additionally performed our evaluation to more complex tasks such as Visual Reasoning using the MME Commonsense Reasoning benchmark. Results in Table~\ref{table:artwork_reasoning} highlight \method's ability to generalize effectively across diverse domains and complex reasoning tasks, demonstrating its broader applicability. Figure~\ref{fig:reasoning} further shows qualitative comparison between image generation methods on the MME Commonsense Reasoning task.

\begin{figure}[!h]
    \centering
    \includegraphics[width=\linewidth]{figures/reasoning.pdf}
    \caption{Qualitative comparison on the Commonsense Reasoning task between image generation method. \method can preserve both diversity and fidelity of the reference image in a more abstract domain.}
    \label{fig:reasoning}
\end{figure}
\section{FID Scores}
% \textcolor{blue}{We compute FID scores of traditional augmentation and image generation methods. Table~\ref{table:fid} shows that the data distribution of generated images by RandAugment and Image Translation are closer to the real distribution as these methods only change images minimally. We also want to emphasize that even though the FID metric evaluates the quality of generated images, it can not measure the diversity of generated images. \method with rewards fine-tuning achieves a competitive score. As we showed in the diversity analysis in Table~\ref{table:diversity} in the main paper, \method performs significantly better than these ``minimal change" methods while still achieving a competitive FID score. We believe this is a worldwide trade-off.}

We compute FID scores for \method and the different baselines (traditional augmentation and image generation methods) and tabulate the numbers in Table~\ref{table:fid}. FID is a valuable metric for assessing the quality of generated images and how closely the distribution of generated images matches the real distribution. However, \textit{FID does not account for the diversity among the generated images}, which is a critical aspect of the task our work targets~(i.e., how can we generate high fidelity images, preserving certain scene attributes, while still maintaining high diversity?). We also illustrate the shortcomings of FID for the task in Figure~\ref{fig:fid_diversity} where we compare generated images across methods. We observe that RandAugment and Image Translation achieve lower FID scores than \method~(w/ finetuning) because they compromise on diversity by only minimally changing the input image, allowing their generated image distribution to be much closer to the real distribution. While \method has a higher FID score than RandAugment and Image Translation, Figure~\ref{fig:fid_diversity} shows that it is able to preserve the scene attribute w.r.t.~multimodal context while generating an image that is significantly different from than original input image. Therefore, it accomplishes the targeted task more effectively, with both high fidelity and high diversity.

\begin{table}[h]
\centering
\caption{FID scores of traditional augmentation and image generation methods. Lower is better.}
\resizebox{\linewidth}{!}{
\begin{tabular}{@{}l@{ }ccccccc@{}}
\toprule
\multirow{2}{*}{\textbf{Method}} & \multirow{2}{*}{\textbf{RandAugment}} & \multirow{2}{*}{\textbf{I2T2I SDXL}} & \multirow{2}{*}{\textbf{Image Variation}} & \multirow{2}{*}{\textbf{Image Translation}} & \multirow{2}{*}{\textbf{Textual Inversion}} & \multicolumn{2}{c}{\textbf{\method}} \\
& & & & & & \ding{55} fine-tuning & \ding{51} fine-tuning\\
\midrule
\textbf{FID score $\downarrow$} & \textbf{15.93} & 18.35 & 17.66 & 16.29 & 20.84 & 17.78 & 16.55 \\
\bottomrule
\end{tabular}
}
\label{table:fid}
\end{table}

\begin{figure}[!h]
    \centering
    \includegraphics[width=\linewidth]{figures/fid_diversity.pdf}
    \caption{While RandAugment and Image Translation achieve lower FID scores, \method balances fidelity and diversity effectively.}
    \label{fig:fid_diversity}
\end{figure}

\section{User Study}
% \textcolor{blue}{We created a survey form with 50 questions (10 questions per MME task). In each survey question, users were shown: a reference image, a related question, and two generated images from different methods (I2T2I SDXL vs. \method). Users are asked to select the generated image(s) that preserve the attribute referred to by the question in relation to reference image. We collected form responses from 70 people. Table~\ref{table:user_study} shows that \method significantly outperforms I2T2I SDXL in terms of fidelity across all tasks on MME benchmark. We have some examples of survey questions in Figure~\ref{fig:user_study_examples}.}

We conduct a user study where we create a survey form with 50 questions (10 questions per MME Perception task). In each survey question, we show users a reference image, a related question, and a generated image each from two different methods (baseline I2T2I SDXL vs \method). We ask users to select the generated images(s) (either one or both or neither of them) that preserve the attribute referred to by the question in relation to the reference image. If an image is selected, it denotes high fidelity in generation. We collect form responses from 70 people for this study. We compute the percentage of total generated images for each method that were selected by the users as a measure of fidelity. Table~\ref{table:user_study} summarizes the results and shows that \method significantly outperforms I2T2I SDXL in terms of fidelity across all tasks on the MME Perception benchmark. We have some examples of survey questions in Figure~\ref{fig:user_study_examples}.

\begin{figure}[htp]
  \centering
   \includegraphics[width=\columnwidth]{Assets/userstudy_grid.pdf}
   
   \caption{\textbf{User study results.} Users preference percentage of our method compared to other methods in terms of text alignment, visual quality, and overall preference.
   }
   \label{fig:user_study}
\end{figure}
\begin{figure}[!h]
    \centering
    \includegraphics[width=\linewidth]{figures/user_study_examples.pdf}
    \caption{Some examples of our survey questions to evaluate the fidelity of generated images from I2T2I SDXL and \method.}
    \label{fig:user_study_examples}
\end{figure}
\section{Training Performance on Bongard HOI Dataset}
% \textcolor{blue}{We conducted an additional experiment by training a CNN baseline ResNet50 \citep{he2016deep} model on the Bongard-HOI training set with traditional augmentation and other image generation methods, using the same number of training iterations. As shown in Table~\ref{table:hoi_training}, \method consistently outperforms other methods across all test splits. However, as discussed in Subsection~\ref{sec:benchmark_formulation}, our primary focus on test-time evaluation ensures fair comparisons by avoiding variability in training behavior caused by differences in model architectures, data distributions, and training configurations.}

Following the existing method \citep{shu2022testtime}, we conduct an additional experiment by training a ResNet50 \citep{he2016deep} model on the Bongard-HOI \citep{jiang2022bongard} training set with traditional augmentation and Hummingbird generated images. We compare the performance with other image generation methods, using the same
number of training iterations. As shown in Table~\ref{table:hoi_training}, \method consistently outperforms all the baselines across all test splits. In the paper, as discussed in Section~\ref{sec:benchmark_formulation}, we focus primarily on test-time evaluation because it eliminates the variability introduced by model training due to multiple external variables such as model architecture, data distribution, and training configurations, and allows for a fairer comparison where the evaluation setup remains fixed.

\begin{table}[!h]
\centering
\footnotesize
\caption{Comparison on Human-Object Interaction~(HOI) Reasoning by training a CNN-baseline ResNet50 with image augmentation and generation methods. \method outperforms SOTA methods on all $4$ test splits of Bongard-HOI dataset.}
\resizebox{0.8\linewidth}{!}{
\begin{tabular}{lccccc}
\toprule
\multirow{3}{*}{Method} & \multicolumn{4}{c}{Test Splits} & \multirow{3}{*}{Average} \\
\cmidrule{2-5}
 & seen act., & unseen act., & seen act., & unseen act., &  \\
 & seen obj. & seen obj. & unseen obj. & unseen obj. & \\
  % & seen act., seen obj. & unseen act., seen obj. & seen act., unseen obj. & unseen act., unseen obj. &  \\
 % &  &  &  & & \\
\midrule
CNN-baseline (ResNet50) & 50.03\xspace\xspace\xspace\xspace\xspace\xspace\xspace\xspace\xspace\xspace & 49.89\xspace\xspace\xspace\xspace\xspace\xspace\xspace\xspace\xspace\xspace & 49.77\xspace\xspace\xspace\xspace\xspace\xspace\xspace\xspace\xspace\xspace & 50.01\xspace\xspace\xspace\xspace\xspace\xspace\xspace\xspace\xspace\xspace & 49.92\xspace\xspace\xspace\xspace\xspace\xspace\xspace\xspace\xspace\xspace \\
RandAugment \citep{cubuk2020randaugment} & 51.07 {\scriptsize \color{ForestGreen}$\uparrow$ 1.04} & 51.14 {\scriptsize \color{ForestGreen}$\uparrow$ 1.25} & 51.79 {\scriptsize \color{ForestGreen}$\uparrow$ 2.02} & 51.73 {\scriptsize \color{ForestGreen}$\uparrow$ 1.72} & 51.43 {\scriptsize \color{ForestGreen}$\uparrow$ 1.51} \\
Image Variation \citep{xu2023versatile} & 41.78 {\scriptsize \color{red}$\downarrow$ 8.25} & 41.29 {\scriptsize \color{red}$\downarrow$ 8.60} & 41.15 {\scriptsize \color{red}$\downarrow$ 8.62} & 41.25 {\scriptsize \color{red}$\downarrow$ 8.76} & 41.37 {\scriptsize \color{red}$\downarrow$ 8.55} \\
Image Translation \citep{pan2023boomerang} & 46.60 {\scriptsize \color{red}$\downarrow$ 3.43} & 46.94 {\scriptsize \color{red}$\downarrow$ 2.95} & 46.38 {\scriptsize \color{red}$\downarrow$ 3.39} & 46.50 {\scriptsize \color{red}$\downarrow$ 3.51} & 46.61 {\scriptsize \color{red}$\downarrow$ 3.31} \\
Textual Inversion \citep{gal2022image} & \xspace37.67 {\scriptsize \color{red}$\downarrow$ 12.36} & \xspace37.52 {\scriptsize \color{red}$\downarrow$ 12.37} & \xspace38.12 {\scriptsize \color{red}$\downarrow$ 11.65} & \xspace38.06 {\scriptsize \color{red}$\downarrow$ 11.95} & \xspace37.84 {\scriptsize \color{red}$\downarrow$ 12.08} \\
I2T2I SDXL \citep{podell2023sdxl} & 51.92 {\scriptsize \color{ForestGreen}$\uparrow$ 1.89} & 52.18 {\scriptsize \color{ForestGreen}$\uparrow$ 2.29} & 52.25 {\scriptsize \color{ForestGreen}$\uparrow$ 2.48} & 52.15 {\scriptsize \color{ForestGreen}$\uparrow$ 2.14} & 52.13 {\scriptsize \color{ForestGreen}$\uparrow$ 2.21}\\
\textbf{\method} & \textbf{53.71 {\scriptsize \color{ForestGreen}$\uparrow$ 3.68}} & \textbf{53.55 {\scriptsize \color{ForestGreen}$\uparrow$ 3.66}} & \textbf{53.69 {\scriptsize \color{ForestGreen}$\uparrow$ 3.92}} & \textbf{53.41 {\scriptsize \color{ForestGreen}$\uparrow$ 3.40}} & \textbf{53.59 {\scriptsize \color{ForestGreen}$\uparrow$ 3.67}} \\
\bottomrule
\end{tabular}
}
\label{table:hoi_training}
\end{table}



\section{Random Seeds Selection Analysis}
We conduct an additional experiment, varying the number of random seeds from $10$ to $100$. The results are presented in the boxplot in Figure~\ref{fig:boxplot}, which shows the distribution of the mean L2 distances of generated image features from Hummingbird across different numbers of seeds.


The figure demonstrates that the difference in the distribution of the diversity (L2) scores across the different numbers of random seeds is statistically insignificant. So while it is helpful to increase the number of seeds for improved confidence, we observe that it stabilizes at 20 random seeds. This analysis suggests that using $20$ random seeds also suffices to capture the diversity of generated images without significantly affecting the robustness of the analysis.

% We conduct an additional experiment where we vary the number of seeds from 10 to 100. We present the results as a boxplot in Appendix K, Figure 15 which shows the distribution of the mean L2 distances of generated image features from Hummingbird across different numbers of seeds.

% The figure demonstrates that the difference in the distribution of the diversity (L2) scores across the different numbers of random seeds is statistically insignificant. So while it is helpful to increase the number of seeds for improved confidence, we observe that it stabilizes at 20 random seeds. This analysis suggests that using 20 random seeds also suffices to capture the diversity of generated images without significantly affecting the robustness of the analysis.

\begin{figure}[!h]
    \centering
    \includegraphics[width=0.8\linewidth]{figures/diversity_boxplot_rectangular.pdf}
    \caption{Diversity analysis across varying numbers of random seeds (10 to 100) using mean L2 distances of generated image features from \method. The box plot demonstrates consistent diversity scores as the number of seeds increases, indicating that performance stabilizes around 20 random seeds.}
    \label{fig:boxplot}
\end{figure}

\section{Further Explanation of Multimodal Context Evaluator}
The Global Semantic Reward, \(\mathcal{R}_\textrm{global}\), ensures alignment between the global semantic features of the generated image \(\mathbf{\hat{x}}\) and the textual context description \(\mathcal{C}\). This reward leverages cosine similarity to measure the directional alignment between two feature vectors, which can be interpreted as maximizing the mutual information \(I(\mathbf{\hat{x}}, \mathcal{C})\) between the generated image \(\mathbf{\hat{x}}\) and the context description \(\mathcal{C}\). Mutual information quantifies the dependency between the joint distribution \(p_{\theta}(\mathbf{\hat{x}}, \mathcal{C})\) and the marginal distributions. In conditional diffusion models, the likelihood \(p_{\theta}(\mathbf{\hat{x}} \vert \mathcal{C})\) of generating \(\mathbf{\hat{x}}\) given \(\mathcal{C}\) is proportional to the joint distribution:
\[
p_{\theta}(\mathbf{\hat{x}} \vert \mathcal{C}) = \frac{p_{\theta}(\mathbf{\hat{x}}, \mathcal{C})}{p(\mathcal{C})} \propto p_{\theta}(\mathbf{\hat{x}}, \mathcal{C}),
\]
where \(p(\mathcal{C})\) is the marginal probability of the context description, treated as a constant during optimization. By maximizing \(\mathcal{R}_\textrm{global}\), which aligns global semantic features, the model indirectly maximizes the mutual information \(I(\mathbf{\hat{x}}, \mathcal{C})\), thereby enhancing the likelihood \(p_{\theta}(\mathbf{\hat{x}} \vert \mathcal{C})\) in the conditional diffusion model.


The Fine-Grained Consistency Reward, $\mathcal{R}_{\textrm{fine-grained}}$, captures detailed multimodal alignment between the generated image $\mathbf{\hat{x}}$ and the textual context description $\mathcal{C}$. It operates at a token level, leveraging bidirectional self-attention and cross-attention mechanisms in the BLIP-2 QFormer, followed by the Image-Text Matching (ITM) classifier to maximize the alignment score.

\textbf{Self-Attention on Text Tokens:}
    Text tokens $\mathcal{T}_{\mathrm{tokens}}$ undergo self-attention, allowing interactions among words to capture intra-text dependencies:
    \begin{equation}
        \mathcal{T}_{\mathrm{attn}} = \tt{SelfAttention}(\mathcal{T}_{\mathrm{tokens}})
    \end{equation}

\textbf{Self-Attention on Image Tokens:}
    Image tokens $\mathcal{Z}$ are derived from visual features of the generated image $\mathbf{\hat{x}}$ using a cross-attention mechanism:
    \begin{equation}
        \mathcal{Z} = \tt{CrossAttention}(\mathcal{Q}_{\mathrm{learned}}, \mathcal{I}_{\mathrm{tokens}}(\mathbf{\hat{x}}))
    \end{equation}
    These tokens then pass through self-attention to extract intra-image relationships:
    \begin{equation}
        \mathcal{Z}_{\mathrm{attn}} = \tt{SelfAttention}(\mathcal{Z})
    \end{equation}

\textbf{Cross-Attention between Text and Image Tokens:}
    The text tokens $\mathcal{T}_{\mathrm{attn}}$ and image tokens $\mathcal{Z}_{\mathrm{attn}}$ interact through cross-attention to focus on multimodal correlations:
    \begin{equation}
        \mathcal{F} = \tt{CrossAttention}(\mathcal{T}_{\mathrm{attn}}, \mathcal{Z}_{\mathrm{attn}})
    \end{equation}

\textbf{ITM Classifier for Alignment:}
    The resulting multimodal features $\mathcal{F}$ are fed into the ITM classifier, which outputs two logits: one for positive match ($j=1$) and one for negative match ($j=0$). The positive class ($j=1$) indicates strong alignment between the image-text pair, while the negative class ($j=0$) indicates misalignment:
    \begin{equation}
        \mathcal{R}_{\textrm{fine-grained}} = \tt{ITM\_Classifier}(\mathcal{F})_{j=1}
    \end{equation}

The ITM classifier predicts whether the generated image and the textual context description match. Maximizing the logit for the positive class $j=1$ corresponds to maximizing the log probability $\log p(\mathbf{\hat{x}}, \mathcal{C})$ of the joint distribution of image and text. This process aligns the fine-grained details in $\mathbf{\hat{x}}$ with $\mathcal{C}$, increasing the mutual information between the generated image and the text features.

\textbf{Improving fine-grained relationships of CLIP.} While the CLIP Text Encoder, at times, struggles to accurately capture spatial features when processing longer sentences in the Multimodal Context Description, \method addresses this limitation by distilling the global semantic and fine-grained semantic rewards from BLIP-2 QFormer into a specific set of UNet denoiser layers, as mentioned in the implementation details under Appendix~\ref{appendix:impl}~(i.e., Q, V transformation layers including $\tt{to\_q, to\_v, query, value}$). This strengthens the alignment between the generated image tokens~(Q) and input text tokens from the Multimodal Context Description~(K, V) in the cross-attention mechanism of the UNet denoiser. As a result, we obtain generated images with improved fidelity, particularly w.r.t.~spatial relationships, thereby helping to mitigate the shortcomings of vanilla CLIP Text Encoder in processing the long sentences of the Multimodal Context Description.

To illustrate further, a Context Description like “the dog under the pool” is processed in three steps: (1) self-attention is applied to the text tokens (K, V), enabling spatial terms like “dog,” “under,” and “pool” to interact; (2) self-attention is applied to visual features represented by the generated image tokens (Q) to extract intra-image relationships (3) cross-attention aligns this text features with visual features. The resulting alignment scores are used to compute the mean and select the positive class for the reward. Our approach to distill this reward into the cross-attention layers therefore ensures that spatial relationships and other fine-grained attributes are effectively captured, improving the fidelity of generated images.


\section{The Choice of Text Encoder in SDXL and BLIP-2 QFormer}

The choice of text encoder in our pipeline is to leverage pre-trained models for their respective strengths. SDXL inherently uses the CLIP Text Encoder for its generative pipeline, as it is designed to process text embeddings aligned with the CLIP Image Encoder. In the Multimodal Context Evaluator, we use the BLIP-2 QFormer, which is pre-trained with a BERT-based text encoder.

\section{Textual Inversion for Data Augmentation}
In our experiments, we applied Textual Inversion for data augmentation as follows: given a reference image, Textual Inversion learns a new text embedding that captures the context of the reference image (denoted as $<$context$>$). This embedding is then used to generate multiple augmented images by employing the prompt: ``a photo of $<$context$>$". This approach allows Textual Inversion to create context-relevant augmentations for comparison in our experiments.

\section{Convergence Curve}
To evaluate convergence, we monitor the training process using the Global Semantic Reward and Fine-Grained Consistency Reward as criteria. Specifically, we observe the stabilization of these rewards over training iterations. Figure~\ref{fig:convergence} presents the convergence curves for both rewards, illustrating their gradual increase followed by stabilization around 50k iterations. This steady state indicates that the model has learned to effectively align the generated images with the multimodal context.

\begin{figure}[!h]
    \centering
    \includegraphics[width=\linewidth]{figures/convergence.pdf}
    \caption{Convergence curves of Global Semantic and Fine-Grained Consistency Rewards}
    \label{fig:convergence}
\end{figure}


\section{Fidelity Evaluation using GPT-4o}
In addition to the results above, we compute additional metrics for fidelity, which measure how well the model preserves scene attributes when generating new images from a reference image. For this, we use GPT-4o (model version: 2024-05-13) as the MLLM oracle for a VQA task on the MME Perception benchmark \citep{fu2024mme}. 
% We use a MLLM as an oracle for a visual question-answering (VQA) task on the MME Perception benchmark \citep{fu2024mme}. In this experiment, we use GPT-4o (model version: 2024-05-13) as the oracle. 
We evaluate \method with and without fine-tuning process.

The MME dataset consists of Yes/No questions, with a positive and a negative question for every reference image. To measure fidelity, we measure the rate at which the oracle's answer remains consistent across the reference and the generated image for every image in the dataset. We run the experiment multiple times and report the average numbers in Table~\ref{table:fidelity_comparison}. We see that fine-tuning the base SDXL with our novel rewards results in an average increase of $2.99\%$ in fidelity.

\begin{table}[!h]
\centering
\footnotesize
\caption{Fidelity between reference and generated images from \method with and without fine-tuning.}
\resizebox{0.9\linewidth}{!}{
\begin{tabular}{clccc}
\toprule
 \textbf{MLLM Oracle} & \textbf{\method} & \textbf{Fidelity on ``Yes"} & \textbf{Fidelity on ``No"} & \textbf{Overall Fidelity} \\
 \midrule
 \multirow{2}{*}{\makecell{\textbf{GPT-4o}\\\textbf{Ver: 2024-05-13}}}
 & w/o fine-tuning & 68.33\xspace\xspace\xspace\xspace\xspace\xspace\xspace\xspace\xspace\xspace & 70.55\xspace\xspace\xspace\xspace\xspace\xspace\xspace\xspace\xspace\xspace & 71.18\xspace\xspace\xspace\xspace\xspace\xspace\xspace\xspace\xspace\xspace \\
 \cmidrule{2-5}
 % \cmidrule{2-12}
 & w/ fine-tuning & \textbf{69.72} {\scriptsize \color{ForestGreen}\textbf{$\uparrow$ 1.39}}  & \textbf{73.61} {\scriptsize \color{ForestGreen}\textbf{$\uparrow$ 3.06}}  & \textbf{74.17} {\scriptsize \color{ForestGreen}\textbf{$\uparrow$ 2.99}} \\
\bottomrule
\end{tabular}
}
\label{table:fidelity_comparison}
\end{table}


\section{Implementation Details}
\label{appendix:impl}
We implement \method using PyTorch \citep{paszke2019pytorch} and HuggingFace diffusers \citep{huggingface2023diffusers} libraries. For the generative model, we utilize the SDXL Base $1.0$ which is a standard and commonly used pre-trained diffusion model in natural images domain. In the pipeline, we employ CLIP ViT-G/14 as image encoder and both CLIP-L/14 \& CLIP-G/14 as text encoders \citep{radford2021learning}. We perform LoRA fine-tuning on the following modules of SDXL UNet denoiser including $Q$, $V$ transformation layers, fully-connected layers ($\tt{to\_q, to\_v, query, value, ff.net.0.proj}$) with rank parameter $r = 8$, which results in $11$M trainable parameters $\approx 0.46\%$ of total $2.6$B parameters. The fine-tuning is done on $8$ NVIDIA A100 80GB GPUs using AdamW \citep{loshchilov2017decoupled} optimizer, a learning rate of \texttt{5e-6}, and gradient accumulation steps of $8$.

\section{Additional Qualitative Results}
\label{appendix:visuals}
Figure~\ref{fig:diversity_compact_caption} showcases two examples of context description from MLLM in the inference pipeline where answers are not included in text guidance. Figure~\ref{fig:diversity_full} illustrates additional qualitative results highlighting the diversity and multimodal context fidelity between reference and synthetic images, as well as across images generated by \method with different random seeds. Figure~\ref{fig:qualitative_full} shows additional qualitative comparisons between \method and SOTA image generation methods on VQA and HOI Reasoning tasks.
\begin{figure}[!h]
    \centering
    \includegraphics[width=\linewidth]{figures/diversity_full.pdf}
    \vspace{-5mm}
    \caption{Diversity and multimodal context fidelity between reference and synthetic image and across generated ones from \method with different random seeds.}
    \label{fig:diversity_full}
\end{figure}
\begin{figure}[!h]
    \centering
    \includegraphics[width=\linewidth]{figures/qualitative_full_v1.pdf}
    \vspace{-5mm}
    \caption{Qualitative comparison between \method and other image generation methods on MME Perception and HOI Reasoning benchmarks.}
    \label{fig:qualitative_full}
\end{figure}

\end{document}
