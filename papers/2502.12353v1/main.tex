%\documentclass[twoside]{article}
\documentclass[11pt]{article}
\usepackage{fullpage}

% \usepackage{aistats2025}
% If your paper is accepted, change the options for the package
% aistats2025 as follows:
%
%\usepackage[accepted]{aistats2025}
%
% This option will print headings for the title of your paper and
% headings for the authors names, plus a copyright note at the end of
% the first column of the first page.

% If you set papersize explicitly, activate the following three lines:
%\special{papersize = 8.5in, 11in}
%\setlength{\pdfpageheight}{11in}
%\setlength{\pdfpagewidth}{8.5in}

% If you use natbib package, activate the following three lines:
\usepackage[round]{natbib}
\renewcommand{\bibname}{References}
\renewcommand{\bibsection}{\subsubsection*{\bibname}}

% If you use BibTeX in apalike style, activate the following line:
%\bibliographystyle{apalike}

\usepackage[utf8]{inputenc}
\usepackage{amsmath,amsfonts,amsthm, amssymb}
\usepackage{hyperref}
\usepackage[capitalize]{cleveref}
\usepackage{caption}
\usepackage{subcaption}
\usepackage{graphicx}
\usepackage{float}
\usepackage{placeins}
\usepackage{enumitem}
\usepackage[dvipsnames]{xcolor}
\usepackage{bbm}

\newcommand{\dN}{\mathcal{N}}
\newcommand{\const}{\text{const}}
\newcommand{\diag}{\text{diag}}
\newcommand{\tr}{\text{tr}}
\newcommand{\E}{\mathbb{E}}
\newcommand{\V}{\mathbb{V}}
\newcommand{\R}{\mathcal{R}}
\newcommand{\F}{\mathcal{F}}
\newcommand{\elbo}{\text{ELBO}}
\newcommand{\dlm}{\text{DLM}}
\newcommand{\kl}{\text{KL}}
\newcommand{\loss}{\mathcal{L}}
\newcommand{\tv}{\text{TV}}
\newcommand{\alg}{\mathcal{A}}
\newcommand{\errgen}{\text{err}_{gen}}
\newcommand{\cov}{\text{Cov}}

\newtheorem{theorem}{Theorem}
\newtheorem{assumption}[theorem]{Assumption}
\newtheorem{lemma}[theorem]{Lemma}
\newtheorem{corollary}[theorem]{Corollary}
\newtheorem{definition}[theorem]{Definition}

\DeclareMathOperator*{\argmax}{arg\,max}
\DeclareMathOperator*{\argmin}{arg\,min}

\begin{document}

% If your paper is accepted and the title of your paper is very long,
% the style will print as headings an error message. Use the following
% command to supply a shorter title of your paper so that it can be
% used as headings.
%
%\runningtitle{I use this title instead because the last one was very long}

% If your paper is accepted and the number of authors is large, the
% style will print as headings an error message. Use the following
% command to supply a shorter version of the authors names so that
% they can be used as headings (for example, use only the surnames)
%
%\runningauthor{Surname 1, Surname 2, Surname 3, ...., Surname n}

%\twocolumn[
%\aistatstitle{Stability-based Generalization Bounds for Variational Inference}
%\aistatsauthor{ Yadi Wei \And Roni Khardon }
%\aistatsaddress{ Indiana University \And Indiana University } ]

\title{Stability-based Generalization Bounds for Variational Inference}

\author{Yadi Wei and Roni Khardon \\ Department of Computer Science \\ Indiana University, Bloomington \\
             \tt{(weiyadi}$|$\tt{rkhardon)@iu.edu}}

\maketitle

\begin{abstract}
Variational inference (VI) is widely used for approximate inference in Bayesian machine learning. 
In addition to this practical success, 
%recent work has developed 
generalization bounds for variational inference and related algorithms have been developed, mostly through the connection to PAC-Bayes analysis. 
%However, to our knowledge these bounds are loose with modern architectures and datasets. 
A second line of work
has provided algorithm-specific generalization bounds 
%for stochastic gradient Langevin dynamics (SGLD) 
through stability arguments or using mutual information bounds,
and has shown that the bounds are tight in practice,
but unfortunately these bounds do not directly apply to approximate Bayesian algorithms. 
This paper fills this gap by developing algorithm-specific stability based generalization bounds for a class of approximate Bayesian algorithms that includes VI, specifically when using 
stochastic gradient descent to optimize their objective.
As in the non-Bayesian case, 
the generalization error is bounded by
by expected parameter differences on a perturbed dataset.
The new approach complements PAC-Bayes analysis and can provide tighter bounds in some cases.  
%In this paper, we extend the stability-based framework to derive tighter generalization bounds for VI,
%by utilizing the norm of parameter differences. 
An experimental illustration shows that the new approach yields non-vacuous bounds on 
modern neural network architectures and datasets
and that it can shed light on performance differences between variant approximate Bayesian algorithms. 
%Our experimental results demonstrate that these stability-based bounds are more informative and tighter compared to traditional PAC-Bayesian approaches in the context of VI. 
\end{abstract}

\section{Introduction}
Variational inference (VI \citep{Jordan99}) is one of the most successful approaches in approximate Bayesian machine learning
\citep[e.g.,][]{Blei2003,LimTeh2007,SeegerB12,KingmaW13,Johnson2016} 
and a significant amount of recent work is devoted to variational methods for deep networks
% put some practical refs perhaps topic models PMF and others - the ones here are from VIFO paper
\citep[e.g.,][]{BNN, Practical-VI, DVI, lrt, collapsed-elbo}. 
%{\bf [add more recent BN papers?]}
Instead of calculating an exact posterior one computes an approximate posterior which minimizes 
the KL divergence between the approximation and the true posterior. 
VI is enabled computationally because minimizing the KL divergence is equivalent to maximizing the tractable 
evidence lower bound (ELBO). 
Thanks to this success several variations of VI have been proposed and following \cite{Alquier2016properties} recent work 
has developed finite sample generalization bounds using PAC Bayes analysis. 
%However, these bounds are typically loose, scaling as $O(1/\sqrt{n})$ for $n$ datapoints, and in many cases the loose bounds are not informative when applied to modern architectures. 

This paper continues this effort but from a different perspective, motivated by 
bounds for 
stochastic gradient Langevin dynamics (SGLD)
using Bayes stability \citep{Li,banerjee}.
The idea of analysis through stability \citep{BousquetE02,HardtRS16} is that if an algorithm is not sensitive to perturbations of its input (i.e., the data) then one can bound the gap between its training and test errors. 
\citet{Li,banerjee} employed the KL divergence between output distributions with and without perturbation to assess this sensitivity.
We note that SGLD modifies the parameter $W$ of the learned model, adding noise in the process, but unlike Bayesian algorithms it does not learn a distribution on the parameters. 
Unfortunately, due to this difference, the approach of \citet{Li,banerjee} is not applicable to Bayesian algorithms. 

%In our work, we specifically focus on establishing 
The paper builds on these ideas and provides a new analysis that 
establishes 
stability-based bounds for a family of approximate-inference algorithms for Bayesian neural networks (which includes VI), in particular when their inference objective is optimized using stochastic gradient descent (SGD). 
We develop two types of bounds: one for bounded loss functions and another for unbounded but Lipschitz loss functions. For bounded loss, we continue to use the KL divergence to measure sensitivity, whereas for Lipschitz loss, we employ the Wasserstein distance. Previous research \citep{ipm, wass-pac-bayes} has explored PAC-Bayes bounds using Wasserstein distance; here, we extend its application to Bayes stability. 
In both cases the generalization gap
%Both types of bounds 
can be upper-bounded by the expected parameter differences and further refined using techniques from \citet{HardtRS16} and \citet{SGD-stability2}, resulting in final bounds expressed as the sum of the gradient differences along the optimization trajectory.

\citet{ZhangBHRV17} demonstrated that it is possible to achieve near-zero training error on both true labels (leading to good test performance) and random labels (where test performance is random)
with the same network and training regime. Therefore, any meaningful generalization bound must distinguish between these scenarios, 
implying that it must be data-dependent. 

We provide an empirical demonstration confirming that our bounds can achieve this, effectively differentiating the successful case from overfitting. 
Further, our bounds produce non-vacuous results for generalization error of VI in practical situations, and 
effectively differentiate generalization performance when models are trained with or without data augmentation. 
We also use the bounds to explore the relationship of ELBO to direct loss minimization (DLM \citep{dlm-bnn}), a variant that has shown good performance in other models but fails for Bayesian neural networks, 
showing that the stronger stability of ELBO might explain this performance gap. Finally, a comparison to PAC-Bayes bounds shows that the stability approach provides tighter bounds in these scenarios and can therefore complement the strength of prior analysis.  


%Further, our bounds produce non-vacuous results in practical situations that are significantly tighter than PAC-Bayes bounds and effectively differentiate generalization performance when models are trained with or without data augmentation. Additionally, our analysis sheds light on the limitations of direct loss minimization (DLM \citep{dlm-bnn}) compared to ELBO, a distinction that PAC-Bayes bounds fail to capture.

In summary, this paper introduces a novel approach for analyzing the generalization performance of approximate Bayesian machine learning algorithms. Our contributions include a stability analysis of iterative update algorithms, the application of these bounds to variational Bayesian networks, and empirical demonstration of the practical utility of these bounds.

\section{Related Work}
There is a long tradition of analysis of asymptotic properties of Bayesian algorithms. 
\cite{Alquier2016properties} made an explicit connection between the Gibbs loss used in PAC-Bayes analysis and the objective of VI. This led to finite sample generalization bounds, i.e., bounds on the difference between training and true errors, that hold uniformly. In turn, algorithms that minimize the sum of training error plus generalization bound,
which have a form similar to VI with a regularization parameter,
are both well motivated and have strong theoretical guarantees. 
In followup work
\cite{Germain2016pac,Germain2019} have extended these results to richer classes, whereas \cite{Sheth2017} developed risk bounds, i.e., bounds that directly quantify the true error of VI. 
Other work suggested alternative optimization criteria diverging from VI by changing the loss or regularization components
\cite[e.g.,][]{black-box-alpha,knoblauch2019generalized,dlm-sgp} and generalization and risks bounds have been developed for some such algorithms 
\citep{Sheth2019,Germain2019,Masegosa20,pac-m}. 
%However, to our knowledge none of these results yield non-vacuous bounds for Bayesian neural networks used in practice. 
However, these have not been demonstrated in practice.
\citet{nonvacuous} provided a non-vacuous bound for a binary classification task on MNIST. 
We evaluate these bounds and compare them to the stability bound
in our experiments in a multi-class classification task with large neural networks.
%However, our experiment shows that when applied in multi-class classification and large networks, this bound is still vacuous.

Another important line of work aims to analyze standard (non-Bayesian) algorithms, where capacity arguments can be used to yield generalization bounds for neural networks \citep[e.g.,][]{nn-rademacher}.
%, but these are also loose. 
Recent work has developed an alternative approach that provides tighter bounds which are data-dependent and algorithm-dependent. This includes work using stability 
\citep{Li,banerjee}
and analysis that works through bounds on mutual information
\citep{NegreaHDK019,HaghifamNK0D20}. 
This has been specifically developed for SGLD, and extensions to SGD \citep{Neu21} are possible only as an approximation of SGLD.
While the approaches differs in technical details, the outcome is similar in that a generalization bound is obtained which can be expressed as a sum over training steps, of some function of the gradients.
Specifically the bound of \cite{Li} includes a sum of gradient norms whereas the bound of \cite{banerjee} includes a sum of the norms of gradient differences, which was found to be tighter in practice. 
As mentioned above,  
SGLD learns the parameter $W$ of the model and adds some noise duirng the optimization, hence it produces a sample from some distribution over parameters. 
%This analysis is applicable to non-Bayesian algorithm but requires the injection of homoscedastic noise by SGLD that generates a distribution over potential parameters, where SGLD samples from this distribution during optimization.
This differs from Bayesian algorithms that explicitly generate distributions over parameters as their posteriors, and aggregate their predictions, and unfortunately the analysis does not carry through to this case. 

In contrast, we directly analyze iterative update Bayesian algorithms, for example, using SGD for variational inference (VI), without noise injection. The primary challenge is that the distribution of the parameters of VI is intractable, making it difficult to apply the chain rule of divergence (as in Lemma 10 of \citet{Li}). 
We provide an alternative analysis that first externalizes all sources of randomness of the algorithm, and then uses convexity to derive the bounds. 
This 
allows us to bound the stability gap in terms of parameter differences.
%Instead, we employ the parameter differences to bound the Bayes stability for VI, and 
With this in place we can
follow the approach used to prove the stability of SGD \citep{HardtRS16, SGD-stability2} to bound parameter differences and obtain the desired result.  
Moreover, we extend the original Bayes stability argument, which previously applied only to bounded loss functions \citep{Li} or loss functions with bounded second moments \citep{banerjee}. We generalize this framework to Lipschitz continuous loss functions, allowing us to bound the generalization error using Wasserstein distances, which can be bounded using parameter differences. This extension is inspired by \citet{ipm, wass-pac-bayes}, which employ Wasserstein distances in PAC-Bayes bounds.

Finally,
while the discussion in the paper emphasizes the analysis of VI, 
%Although our focus is on applications to VI, 
the analysis and bounds are applicable to any iterative update 
approximate Bayesian 
algorithm that updates parameters of the approximate posterior, where the sensitivity of parameter updates can be easily calculated.
Hence it captures more cases than prior work, as illustrated by the application to DLM.


\section{Preliminaries}
Consider a model with parameters denoted as $w\in \mathbb{R}^d$. Given a prior distribution $p(w)$ and a dataset $S=(z_1, \dots, z_n)$ of size $n$, our goal is to determine the posterior distribution $p(w \mid S)$, which is computationally challenging in most cases. Variational inference offers a solution by seeking a distribution $Q(w)$ from a specified family of distributions, denoted as $\mathcal{Q}$, that minimizes the Kullback-Leibler (KL) divergence between $Q(w)$ and the true posterior $p(w|S)$. 
\begin{align}
\label{eq:elbo}
&Q^*(w) \nonumber \\
=& \argmin_{Q\in \mathcal{Q}}\kl(Q(w) \Vert p(w \mid S)) \nonumber \\
=& \argmin_{Q\in \mathcal{Q}} \E_{Q(w)}[\log Q(w) - \log p(w, S)] + \log p(S) \nonumber \\
=&\argmin_{Q\in \mathcal{Q}} 
\frac{1}{n} 
\sum_{i=1}^n 
\E_{Q(w)}[-\log p(z_i|w)] + \frac{1}{n} \kl(Q, p). 
\end{align}
The maximization objective obtained by negating \eqref{eq:elbo} is known as the Evidence Lower Bound (ELBO).
The above optimization objective can be efficiently solved using common gradient-based techniques, such as stochastic gradient descent. Furthermore, various alternative objectives exist to approximate the (pseudo) posterior distributions, for example, Direct Loss Minimization (DLM, \citep{dlm-sgp}), which uses the the following objective, and which is discussed in our experiments:
\begin{align}
\label{eq:dlm}
    \frac{1}{n}\sum_{i=1}^n -\log \E_{Q(w)}[p(z_i \vert w)] + \frac{1}{n} \kl(Q, p).
\end{align}

Note that the optimization objective is a function of the distribution $Q(w)$ and let $\theta$ denote the parameters of $Q$. 
To facilitate the analysis across different objectives,
we denote the objective function as $F(\theta, S) = \frac{1}{n} \sum_{i=1}^n F(\theta, z_i)$, where the objective function is written as the average of the objective function with respect to individual data points.  
Notice that for the examples above $F$ includes the regularizer.
For example, in ELBO, $F(\theta, z) = \E_{Q(w)}[-\log p(z|w)] + \frac{1}{n} \kl (Q, p)$. 

Let $\loss(w, z)$ be a loss function for parameter $w$ on a data point $z$ (notice that the loss can be different from the objective function $F$). Define $\loss(w, \mathcal{D}) = \E_{z \sim D} [\loss(w, z)]$ as the expected loss over a distribution $\mathcal{D}$, and $\loss(w, S) = \frac{1}{n} \sum_i \loss(w, z_i)$ as the empirical loss on a dataset $S$. Then the generalization error of the algorithm $\alg$ (which chooses $Q$ based on $S$), i.e., the gap between true and training set error, is given by:
\begin{align}
    \errgen(\alg) &= \E_{S\sim \mathcal{D}^n} \E_{w\sim Q} [\loss(w, \mathcal{D}) - \loss(w, S)].
\end{align}

\section{Generalization Bounds through Bayes Stability}
\label{sec:stability}
Consider a Bayesian algorithm, denoted by $\mathcal{A}$, designed to learn the posterior distribution over a parameter $w$ by optimizing an objective function $F(\theta, S)$. In some cases, there is inherent randomness in evaluating the objective and its gradients or in the optimization process, such as when the reparametrization trick \citep{KingmaW13} is used to approximate expectation terms (as in \cref{eq:elbo} and \cref{eq:dlm}) or when mini-batches are employed. We represent all sources of randomness by $\epsilon$. Consequently, the gradient of the objective becomes $\nabla F(\theta, S, \epsilon)$ for the entire dataset and $\nabla F(\theta, z, \epsilon)$ for an individual data point $z$.
Given a training dataset $S$ and the randomness $\epsilon$, the algorithm $\mathcal{A}$ deterministically produces a posterior distribution $Q_{S, \epsilon}$ for the parameter $w$. We define $Q_S$ as the expected posterior distribution, obtained by averaging over all possible randomness, i.e., $Q_S = \mathbb{E}_{\epsilon}[Q_{S, \epsilon}]$. Additionally, we assume that, when $\epsilon$ is integrated out, $\mathcal{A}$ is order-independent.

\begin{assumption}[Order-independent]
    For any fixed dataset $S=(z_1, \dots, z_n)$ and any permutation $p$, $Q_S = Q_{S^p}$, where $S^p$ is the dataset under permutation $p$.
\end{assumption}
This assumption can be easily satisfied by letting the learning algorithm randomly permute the training data at the beginning. Additionally, it is straightforward to show that variational inference using stochastic gradient descent (SGD) satisfies this condition.

We proceed, following the work of \citet{Li}, to define the single-point posterior distribution $Q_z = \E_{(z_1, \dots, z_{n-1})} [Q_{(z_1, \dots, z_{n-1}, z)}] = \E_{\epsilon, (z_1, \dots, z_{n-1})}[Q_{(z_1, \dots, z_{n-1}, z), \epsilon}]$, 
where we assume without loss of generality that $z$ is put at location $n$. 

\subsection{Bayes Stability}


The generalization error can be effectively bounded using a Bayes-stability argument, as exemplified in previous work by \citet{Li} and \citet{banerjee}. 
We develop two such bounds, one for bounded loss functions using TV distance and the other for unbounded but Lipschitz loss functions using Wasserstein distance. In both cases the result reduces to expected parameter differences. 

Let $\tv(p, q) = \frac{1}{2} \int \vert p(x) - q(x) \vert dx$ be the total variation distance between distributions. We have:

\begin{lemma}
\label{lemma:tv-convexity}
    $\tv\left(\E_{P(X)}[P(Y|X)], \E_{P(X)}[Q(Y|X)] \right) \leq \E_{P(X)}[\tv(P(Y|X), Q(Y|X))]$.
\end{lemma}
\begin{proof}
    \begin{align*}
        & \tv(\E_{P(X)}P(Y|X), \E_{P(X)}Q(Y|X)) 
        \\ &= \frac{1}{2} \int \left|\int P(x) P(y|x)dx - \int P(x) Q(y|x) dx\right| dy
        \\ &= \frac{1}{2} \int \left|\int P(x) (P(y|x)- Q(y|x)) dx\right| dy 
        \\ &\leq \frac{1}{2} \int P(x) \int \left|P(y|x)- Q(y|x)\right| dy dx 
        \\ &= \E_{P(X)}[\tv(P(Y|X), Q(Y|X))].
    \end{align*}
\end{proof}

The following lemma 
%(proof in \cref{sec:proof-stability}) 
adapts the ideas in the original proofs of \citet{Li, banerjee} to the context of Bayesian algorithms that output distributions over parameters. 
%The proofs of the next two lemmas are in \cref{sec:proof-stability}.

\begin{lemma}[Bayes-Stability 1]% \citep{Li, banerjee}]
\label{thm:errgen}
    Suppose the loss function $\loss(w, z)$ is $C$-bounded. Let $S$, $\Bar{S}$ denote two datasets that only differ at one element $z$ and $\Bar{z}$. The generalization error $\errgen(\alg)$ is upper bounded by $2C \E_{S, \Bar{S}, \epsilon} [\tv(Q_{S, \epsilon}, Q_{\Bar{S}, \epsilon})]$.
    %, where $\tv(p, q) = \frac{1}{2} \int \vert p(x) - q(x) \vert dx$ is the total variation distance.
\end{lemma}
\begin{proof}%[Proof of \cref{thm:errgen}]
    It is clear that
     \begin{align}
         \E_{S} \E_{w \sim Q_S} [\loss(w, \mathcal{D})] = \E_{z\sim \mathcal{D}} \E_{w \sim Q_z} \E_{\Bar{z}\sim \mathcal{D}}[\loss(w, \Bar{z})] = \E_{\Bar{z}} \E_{Q_{\Bar{z}}} \E_{z} [\loss(w, z)],
     \end{align}
     and
     \begin{align}
         \E_S \E_{w\sim Q_S} \left[\frac{1}{n} \sum_{i=1}^n \loss(w, z_i) \right] &= \E_{z} \E_{Q_z} [\loss(w, z)].
     \end{align}
     Then the generalization error is 
     \begin{align}
         \errgen(\alg) &= \E_z \E_{\Bar{z}} \left[ \E_{w\sim Q_{\Bar{z}}}[\loss(w, z)] - \E_{w\sim Q_z}[\loss(w, z)]\right] \\
         &\leq \E_{z, \Bar{z}} \int |\loss(w, z)| \lvert Q_{\Bar{z}}(w) - Q_z(w)\rvert dw \\
         &\leq 2C \E_{z, \Bar{z}} [\tv(Q_z, Q_{\Bar{z}})] \\
         &= 2C \E_{z, \Bar{z}}[\tv(\E_{S_{n-1}, \epsilon}[Q_{S_{n-1} \cup \{z\}, \epsilon}], \E_{S_{n-1}, \epsilon}[Q_{S_{n-1} \cup \{\Bar{z}\}, \epsilon }])] \\
         &\leq 2C \E_{z, \Bar{z}, S_{n-1}, \epsilon}[\tv(Q_{S, \epsilon}, Q_{\Bar{S}, \epsilon})]
     \end{align}
      where the last inequality is due to \cref{lemma:tv-convexity}.
     %because of the convexity of total variation distance (\cref{lemma:tv-convexity}).
\end{proof}

There are two important differences form the argument structure in prior work \citep{Li, banerjee}. 
First, note that it is crucial that $\epsilon$ includes all sources of randomness in the algorithm. With this condition, $Q_{S, \epsilon}$ is a distribution in the family used by the algorithm and not a mixture of such distributions.
For example, when $Q(w)$ is a normal distribution $Q_{S, \epsilon}$ is a normal distribution, but $Q_{S}$ is a mixture of normal distributions where the mixture is taken over $\epsilon$. 
%induced by the batch sequence. 
This allows us to directly bound the stability using parameter differences as in the next lemma. In contrast, the analysis of \citet{Li, banerjee}, 
that works with mixtures generated by the choice of batches in SGLD,
requires a fixed variance term (for all dimensions) and is not easily generalizable to the case of learned variances. 

The second difference is due to the structure of the probability model.
%It is important to note that we cannot directly apply the methods from \citet{Li} to derive the generalization bound, which 
\citet{Li} 
use the sum of KL divergence along the optimization trajectory to upper bound the Bayes stability. In SGLD, the optimization trajectory \( W_1, W_2, \dots, W_T \) consists of samples from the distribution, with each \( W_i \) being drawn from a Gaussian distribution conditioned on both \( W_{i-1} \) and the batch. However, in variational inference, the optimization trajectory \( (\mu_1, \sigma_1), \dots, (\mu_T, \sigma_T) \) consists of distribution parameters, 
and the bound on the sequence of conditional KL divergences does not hold. 
%and the distributions of \( \mu_i \) and \( \sigma_i \) are not well-defined. 
A detailed explanation is provided in \cref{sec:Li-proof-not-for-vi}.

The next lemma shows that Bayes stability can be bounded through parameter differences:

\begin{lemma}
\label{cor:kl}
    Under the condition of \cref{thm:errgen}, if $Q_{S, \epsilon} = \dN(m, \diag(\sigma^2))$ and $Q_{\Bar{S}, \epsilon} = \dN(\Bar{m}, \diag(\Bar{\sigma}^2))$, the generalization error is upper bounded by 
    \begin{align}
    \label{eq:kl-bound}
        \frac{2C}{\sqrt{\sigma_0}} \sqrt{\E [\lVert \Bar{\sigma} - \sigma \rVert_1]} + \frac{C}{\sigma_0} \E \left[\lVert \Bar{\sigma} - \sigma \rVert_2 \right] + \frac{C}{\sigma_0} \E \left[\lVert \Bar{m} - m \rVert_2 \right],
    \end{align}
    where the expectation is taken over $S, \Bar{S}$ and $\epsilon$ and $\sigma_0$ is a preset lower bound of the standard deviation in $Q$.
\end{lemma}
\begin{proof}%[Proof of \cref{cor:kl}]
    According to Pinsker's inequality, the total variation distance can be bounded by the KL divergence of the distributions. We thus first bound the KL divergence. 
     \begin{align}
         \kl(Q_{S, \epsilon}, Q_{\Bar{S}, \epsilon}) &= 1^\top (\log \sigma - \log \Bar{\sigma}) + \frac{1}{2} \left(1^\top \frac{\Bar{\sigma}^2}{\sigma^2} - d + 1^\top \frac{(\Bar{m} - m)^2}{\sigma^2} \right) \\
        &\leq \frac{2\lVert \sigma - \Bar{\sigma} \rVert_1}{\sigma_0} + \frac{\lVert \sigma - \Bar{\sigma} \rVert_2^2}{2\sigma_0^2} + \frac{\lVert \Bar{m} - m \rVert_2^2}{2\sigma_0^2},
        \label{eq:kl-form}
    \end{align}
    where $\sigma_0$ is a preset minimum value for the standard deviation, i.e., $\forall k, \sigma_k \geq \sigma_0$ and $\bar{\sigma}_k \geq \sigma_0$.
    To derive \cref{eq:kl-form},
    let $ \beta_i = |\sigma_i - \Bar{\sigma}_i |$. Consider $1^\top (\log \sigma - \log \Bar{\sigma}) = \sum_i \log \frac{\sigma_i}{\Bar{\sigma}_i}$. For each entry $i$, if $\sigma_i - \beta_i \leq \sigma_0$, then $\log \frac{\sigma_i}{\Bar{\sigma}_i} \leq \log \frac{\beta_i + \sigma_0}{\sigma_0} = \log (1 + \frac{\beta_i}{\sigma_0}) \leq \frac{\beta_i}{\sigma_0}$; if $\sigma_i - \beta_i > \sigma_0$, then $\log \frac{\sigma_i}{\Bar{\sigma}_i} \leq \log \frac{\sigma_i}{\sigma_i - \beta_i} = \log (1 + \frac{\beta_i}{\sigma_i - \beta_i}) \leq \log (1 + \frac{\beta_i}{\sigma_0}) \leq \frac{\beta_i}{\sigma_0}$. Overall, $1^\top (\log \sigma - \log \Bar{\sigma}) \leq \sum_i \frac{\beta_i}{\sigma_0} = \frac{\lVert \sigma - \Bar{\sigma}_0 \rVert_1}{\sigma_0}$. For $1^\top \frac{\Bar{\sigma}^2}{\sigma^2} = \sum_i \frac{\Bar{\sigma}_i^2}{\sigma_i^2} \leq \sum_i \frac{(\sigma_i + \beta_i)^2}{\sigma_i^2} \leq \sum_i (1 + 2\frac{\beta_i}{\sigma_i} + \frac{\beta_i^2}{\sigma_i^2}) \leq \sum_i (1 + \frac{2\beta_i}{\sigma_0} + \frac{\beta_i^2}{\sigma_i}) = d + \frac{2\lVert \sigma - \Bar{\sigma} \rVert_1}{\sigma_0} + \frac{\lVert \sigma - \Bar{\sigma} \rVert_2^2}{\sigma_0^2}$. Thus,
    \begin{align}
        \errgen(\alg) &\leq 2C \E_{S, \Bar{S}, \epsilon} [\tv(Q_{S, \epsilon}, Q_{\Bar{S}, \epsilon})] \\
        &\leq C \E_{S, \Bar{S}, \epsilon} \sqrt{2 \kl (Q_{S, \epsilon}, Q_{\Bar{S}, \epsilon})} \\
        &\leq \frac{2C}{\sqrt{\sigma_0}} \sqrt{\E [\lVert \Bar{\sigma} - \sigma \rVert_1]} + \frac{C}{\sigma_0} \E \left[\lVert \Bar{\sigma} - \sigma \rVert_2 \right] + \frac{C}{\sigma_0} \E \left[\lVert \Bar{m} - m \rVert_2 \right].
    \end{align}
\end{proof}

\cref{thm:errgen} holds only for bounded loss functions. 
We next introduce the upper bound for unbounded Lipschitz loss functions using Wasserstein distance \citep{Villani2008OptimalTO}.
\begin{definition}
    Suppose loss function $\loss(\theta, z)$ is $K$-Lipschitz with respect to $\theta$, i.e., $\frac{|\loss(\theta, z) - \loss(\theta', z)|}{\lVert \theta - \theta' \rVert} \leq K$ for all $z$.
\end{definition}
The Wasserstein-$p$ distance between two distributions $\mu$ and $\nu$ is defined as:
\begin{align}
    W_p(\mu, \nu) = \inf_{\gamma \in \Gamma(\mu, \nu)} \left( \E_{(x, y) \sim \gamma} d(x, y)^p \right)^{1/p},
\end{align}
where $d(x,y)$ is some distance and $\Gamma(\mu, \nu)$ is the set of all couplings of $\mu$ and $\nu$, i.e., for $\gamma \in \Gamma(\mu, \nu)$, $\int_y \gamma(x, y) = \mu(x)$ and $\int_x \gamma(x, y) = \nu(y)$. 
%We use Euclidean distance in the following unless otherwise specified.
In the following we use the Euclidean distance.
According to Kantorovich duality \citep{Villani2008OptimalTO},
\begin{align}
    W_1(\mu, \nu) = \sup_{f, \text{Lip}(f)\leq 1} \int f d\mu(x) - \int f d\nu(y).
\end{align}
Inspired by \citet{ipm}, we derive the bound through the Wasserstein distance:
%\begin{lemma}
%\label{thm:wasserstein}
%    Suppose the loss function $\loss(w, z)$ is $K$-Lipschitz. Let $S$, $\Bar{S}$ denote two datasets that only differ at one element $z$ and $\Bar{z}$. The generalization error $\errgen(\alg)$ is upper bounded by 
%    \begin{align}
%        K \E_{S, \Bar{S}, \epsilon} [W_p(Q_{S, \epsilon}, Q_{\Bar{S}, \epsilon})]
%    \end{align}
%    for $p \geq 1$. Further, if $Q_{S, \epsilon} = \dN(m, \diag(\sigma^2))$ and $Q_{\Bar{S}, \epsilon} = \dN(\Bar{m}, \diag(\Bar{\sigma}^2))$ are Gaussian distributions, the generalization error is upper bounded by 
%    \begin{align}
%    \label{eq:wass-bound}
%        K \E \lVert m - \Bar{m} \rVert_2 + K \E \lVert \sigma - \Bar{\sigma} \rVert_2.
%    \end{align}
%\end{lemma}
\begin{lemma} [Bayes-Stability 2]
\label{thm:wasserstein}
    Suppose the loss function $\loss(w, z)$ is $K$-Lipschitz. Let $S$, $\Bar{S}$ denote two datasets that only differ at one element $z$ and $\Bar{z}$. 
    Then, for any $p \geq 1$, the generalization error $\errgen(\alg)$ is upper bounded by 
    \begin{align}
        K \E_{S, \Bar{S}, \epsilon} [W_p(Q_{S, \epsilon}, Q_{\Bar{S}, \epsilon})].
    \end{align}
    %Further, if $Q_{S, \epsilon} = \dN(m, \diag(\sigma^2))$ and $Q_{\Bar{S}, \epsilon} = \dN(\Bar{m}, \diag(\Bar{\sigma}^2))$ are Gaussian distributions, the generalization error is upper bounded by 
    %\begin{align}
    %\label{eq:wass-bound}
    %    K \E \lVert m - \Bar{m} \rVert_2 + K \E \lVert \sigma - \Bar{\sigma} \rVert_2.
    %\end{align}
\end{lemma}
\begin{proof}
    It is obvious that $\frac{1}{K} \loss(w, z)$ is 1-Lipschitz. 
    Using Kantorovich duality we have
    \begin{align*}
         &\errgen(\alg) \\
         &= \E_z \E_{\Bar{z}} \left[ \E_{w\sim Q_{\Bar{z}}}[\loss(w, z)] - \E_{w\sim Q_z}[\loss(w, z)]\right] \\
         &\leq K \E_{z, \Bar{z}} \sup_{f, \text{Lip}(f)\leq 1} \E_{w\sim Q_{\Bar{z}}}[f(w)] - \E_{w\sim Q_{z}}[f(w)] \\ 
         &\leq K \E_{z, \Bar{z}} \E_{S_{n-1}, \epsilon} \\
         & \qquad \qquad \sup_{f, \text{Lip}(f)\leq 1}  \E_{w \sim Q_{\Bar{S}, \epsilon}}[f(w)] - \E_{w \sim Q_{S, \epsilon}}[f(w)] \\
         &= K \E_{z, \Bar{z}, S_{n-1}, \epsilon} W_1(Q_{\bar{S}, \epsilon}, Q_{S, \epsilon}) \\
         &\leq K \E_{z, \Bar{z}, S_{n-1}, \epsilon} W_p (Q_{\bar{S}, \epsilon}, Q_{S, \epsilon}).
    \end{align*}
    The third line is because of the convexity of supremum. The last inequality follows the Holder's inequality, which states that $\E[|XY|] \leq \E[|X|^p]^{\frac{1}{p}} \E[|Y|^q]^{\frac{1}{q}}$ for $p, q \geq 1$ and $\frac{1}{p} + \frac{1}{q} = 1$. Thus $\E_{x, y \sim \gamma}[d(x, y) \cdot 1] \leq \left(\E_{x, y \sim \gamma}[d(x, y)^p]\right)^{\frac{1}{p}} \left( \E[1^{q}] \right)^{\frac{1}{q}} = \left(\E_{x, y \sim \gamma} d(x, y)^p \right)^{\frac{1}{p}}$. 
    Taking infimum on both sides, we have proved the inequality.
\end{proof}

As in the previous case we can bound the stability using parameter differences. In particular, 
letting $p=2$ and using the Wasserstein-2 distance for Gaussian distributions \citep{Gauss-wass}, we immediately have: 
%proved \cref{eq:wass-bound}.
\begin{lemma}
\label{thm:wasserstein2}
Under the condition of \cref{thm:wasserstein},
if $Q_{S, \epsilon} = \dN(m, \diag(\sigma^2))$ and $Q_{\Bar{S}, \epsilon} = \dN(\Bar{m}, \diag(\Bar{\sigma}^2))$, the generalization error is upper bounded by 
    \begin{align}
    \label{eq:wass-bound}
        K \E \lVert m - \Bar{m} \rVert_2 + K \E \lVert \sigma - \Bar{\sigma} \rVert_2.
    \end{align}
\end{lemma}

%\cref{cor:kl} and \cref{thm:wasserstein2} upper bound the generalization error with the expectation of the difference of the parameters. 
%In the next section, we analyze the difference and efficiently compute it.

%\section{Stability Bounds}
\subsection{Bounds on Expected Parameter Differences}
\label{sec:bounds}
In this section, we draw upon the approach from \citet{HardtRS16} and \citet{SGD-stability2}, which bounds parameter differences for stochastic gradient descent.
Let $\theta_t$ be the parameter of $Q_{S, \epsilon}$ at step $t$ and $\Bar{\theta}_t$ be the parameter of $Q_{\Bar{S}, \epsilon}$ at step $t$. Let $G_t$ denote the update rule of stochastic gradient descent with learning rate $\alpha_t$,
\begin{align}
    \theta_{t} = G_t(\theta_{t-1}, S, \epsilon_{t}) = \theta_{t-1} - \alpha_t \nabla_\theta F(\theta_{t-1}, S, \epsilon_{t}).
\end{align}
Recall that $\epsilon_t$ contains all randomness at step $t$ and $\nabla F(\theta_{t-1}, S, \epsilon_t)$ is the approximation of $\nabla F(\theta_{t-1}, S)$. We make the following assumption \citep{HardtRS16,SGD-stability2} on the update rule:
\begin{definition}
    An update rule is $\eta$-expansive if $\sup_{\theta, \theta'}\frac{\lVert G(\theta, S, \epsilon) - G(\theta', S, \epsilon) \rVert}{\lVert \theta - \theta' \rVert} \leq \eta$ for any $S$ and $\epsilon$.
\end{definition}
The following theorem adapts the argument of \citet{SGD-stability2} to bound parameter differences as a function of expected gradient differences.
\begin{theorem}
\label{thm:diff}
    Given an algorithm that optimizes parameters $\theta$ using stochastic gradient descent, suppose it is $\eta_t$-expansive for step $t$. Let $S$ and $\Bar{S}$ be two random datasets of size $n$ that only differ at one element $z$ and $\Bar{z}$, and $\theta_T$ and $\Bar{\theta}_T$ denote the outputs under the same $\epsilon$.
    Then the expected difference of $\theta_T$ and $\Bar{\theta}_T$ satisfies 
    \begin{align}
    \label{eq:diff}
        \E_{S, \Bar{S}, \epsilon} [\lVert \theta_T - \Bar{\theta}_T \rVert]
        \leq \frac{1}{n} \sum_{t=1}^T \left(\prod_{i=t+1}^T \eta_i \right) \alpha_t \E_{S, \epsilon, \Bar{z}}[\Delta_t],
    \end{align}
    where $\Delta_t = \lVert \nabla F(\theta_{t-1}, \Bar{z}, \epsilon_t) - \nabla F(\theta_{t-1}, z, \epsilon_t) \rVert$.
\end{theorem}
%\begin{proof}[Proof sketch]
%    The main idea is that, at each step $t$, the difference in the parameter vectors due to $z$ and $\Bar{z}$ is expanded thereafter with a total expansion factor $\prod_{i=t+1}^T \eta_i$. The total difference between $\theta_T$ and $\Bar{\theta}_T$ accumulates over $T$ steps as the expanded difference at each step is propagated forward. Thus, the overall bound on the expected difference is a sum over the per-step expanded differences. The detailed proof is in \cref{sec:proof-bound}.
%\end{proof}
\begin{proof}%[Proof of \cref{thm:diff}]
    Let $S_t$ and $\Bar{S}_t$ denote the subset at step $t$. With respect to the same $\epsilon_t$ (including the same batch sequence), $S_t$ and $\bar{S}_t$ have at most one different element.
    We have two cases:
    \begin{itemize}
    \item Case 1: the different element is not selected, hence $S_t = \Bar{S}_t$, and since $G$ is $\eta_t$ expansive:
        \begin{align*}
            \lVert \theta_t - \Bar{\theta}_t \rVert \leq \eta_t \lVert \theta_{t-1} - \Bar{\theta}_{t-1} \rVert.
        \end{align*}
        \item Case 2: the different element is selected.
        \begin{align*}
            \lVert \theta_t - \Bar{\theta}_t \rVert &= \lVert (\theta_{t-1} - \alpha_t \nabla F(\theta_{t-1}, S_t, \epsilon_{t})) \\
            & \quad - (\Bar{\theta}_{t-1} - \alpha_t \nabla F(\Bar{\theta}_{t-1}, \Bar{S}_t, \epsilon_t)) \rVert \\
            &= \lVert (\theta_{t-1} - \alpha_t \nabla F(\theta_{t-1}, \Bar{S}_t, \epsilon_t)) \\
            & \quad - (\Bar{\theta}_{t-1} - \alpha_t \nabla F(\Bar{\theta}_{t-1}, \Bar{S}_t, \epsilon_t)) \\
            & \quad + \alpha_t (\nabla F(\theta_{t-1}, \Bar{S}_t, \epsilon_t) - \nabla F(\theta_{t-1}, S_t, \epsilon_t)) \rVert \\
            &\leq \eta_t \lVert \theta_{t-1} - \Bar{\theta}_{t-1} \rVert \\
            &\quad + \alpha_t \lVert \nabla F(\theta_{t-1}, \Bar{S}_t, \epsilon_t) - \nabla F(\theta_{t-1}, S_t, \epsilon_t) \rVert.
        \end{align*}
        Since $\Bar{S}_t$ and $S_t$ only differs at one element, $\lVert \nabla F(\theta_{t-1}, \Bar{S}_t, \epsilon_t) - \nabla F(\theta_{t-1}, S_t, \epsilon_t) \rVert = \frac{1}{b} \lVert \nabla F(\theta_{t-1}, \Bar{z}, \epsilon_t) - \nabla F(\theta_{t-1}, z, \epsilon_t) \rVert = \frac{1}{b} \Delta_t$, where $b$ is the batch size.
    \end{itemize}
    Thus, 
    \begin{align}
        \lVert \theta_T - \Bar{\theta}_T \rVert &\leq \eta_T \lVert \theta_{T-1} - \Bar{\theta}_{T-1} \rVert + \mathbbm{1}_{z \in S_T} \frac{\alpha_T}{b} \Delta_T \\
        &\leq \frac{1}{b} \sum_{t=1}^T \left(\prod_{i=t+1}^T \eta_i \right) \mathbbm{1}_{z\in S_t} \alpha_t \Delta_t,
    \end{align}
    where the base case is $\theta_0 = \bar{\theta}_0$.
    Since the probability that $z \in S_t$ is $\frac{b}{n}$, then the expected difference is 
    \begin{align}
        \E \lVert \theta_T - \Bar{\theta}_T \rVert \leq \frac{1}{n} \sum_{t=1}^T \left(\prod_{i=t+1}^T \eta_i \right) \alpha_t \E_{S, \epsilon, \bar{z}} [\Delta_t].
    \end{align}
\end{proof}

\cref{thm:diff} provides a way to compute the bound {\em exactly}. As we show in the experiments this allows us to obtain tight generalization bounds which are not possible otherwise. For completeness, the following Corollary provides an asymptotic upper bound using stronger requirements.
%for the objective function and a specific learning rate schedule.  
The proof follows the construction of \citet{SGD-stability2} and is included in \cref{sec:proof-bound}.
\begin{corollary}
\label{cor:logT}
    Suppose $\nabla F(\theta, S, \epsilon)$ is $L$-Lipschitz and $\beta$-bounded, then with learning rate $\alpha_t = \frac{c}{(t+2) \log (t+2)}$ where $c$ is chosen that $cL < 1$, $\E \lVert \theta_T - \Bar{\theta}_T \rVert \leq O(\frac{\log T}{n})$.
\end{corollary}

\subsection{Discussion: Stability vs.\ PAC-Bayes Bounds}
\label{sec:pac-bayes}

As mentioned above, prior work has developed  PAC-Bayes Bounds for certain variants of VI. In this section we review some of these bounds and discuss the qualitative differences between the two types of bounds.

\citet{Germain2016pac} provides a generalization error bound for a $C$-bounded loss function as follows: with probability $1 - \delta$,
\begin{align}
    \frac{1}{\lambda} \left(\kl(Q_S \parallel P) + \log \frac{1}{\delta}\right) + \frac{\lambda C^2}{2n}.
\end{align}
By optimizing $\lambda$ as $\lambda = \frac{1}{C} \sqrt{2n \left(\kl(Q_S \parallel P) + \log \frac{1}{\delta}\right)}$, we obtain the following bound:
\begin{align}
    C \sqrt{\frac{2 \left(\kl(Q_S \parallel P) + \log \frac{1}{\delta}\right)}{n}}.
\end{align}

On the other hand, \citet{pac-bayes-book} provides a similar bound in the form:
\begin{align}
\label{eq:sqrt}
    C\sqrt{\frac{\kl(Q_S \parallel P) + \log \frac{n}{\delta}}{2(n-1)}}.
\end{align}
%This bound is nearly a factor of 2 smaller than the previous one. 
that can be tighter in some cases.
In these results, 
the ``prior" $P$ is only required to be data independent and is not directly related to the algorithm. 
Therefore, for Bayesian algorithms, one can pick a different $P$ other than the prior used in the objective function. 
In the experimental illustration,
we explore using both the prior and the initialization $Q_0$ so as to obtain the tightest possible bound.
% when evaluating the PAC-Bayes bounds

Additionally, \citet{nonvacuous} proposed a non-vacuous bound specifically for the 0-1 loss. By employing a union-bound argument, 
where the prior variance is set as $\lambda = c \exp(-j / b)$ for $j \in \mathbb{N}$ and fixed $b$ and $c$,
they ensure that the generalization error can be bounded, with probability $1-\delta$, by
\begin{align}
\label{eq:bre}
    \sqrt{\frac{\kl(Q_S \parallel \mathcal{N}(m_0, \lambda I)) + 2 \log \left(b \log \frac{c}{\lambda}\right) + \log \frac{\pi^2 n}{6 \delta}}{2(n-1)}},
\end{align}
where $m_0$ denotes the random initialization of the mean parameter.

These bounds have been leveraged to develop efficient Bayesian algorithms, by explicitly optimizing the sum of the training set loss and the bound, 
which can be seen to have a similar form to VI and therefore interpreted as variants of VI.
On the other hand, PAC-Bayes bounds are valid for any distribution within the specified family. 
They can therefore be applied to the output of VI directly. 
%it is important to recognize that these bounds apply to any distribution within the specified family. 
%Therefore, even if a posterior distribution is not directly optimized using these bounds, its generalization error will still conform to them. 

From this perspective, our bounds are more restricted in that they are valid only for the output of  a certain class of optimization problems when optimized by SGD. In addition, the stability bound in \eqref{eq:diff} grows with the number of optimization steps $T$ which can make it less attractive,
and for a fixed dataset this may necessitate the use of larger batch sizes to reduce $T$.
On the other hand, the dependence on dataset size in \eqref{eq:diff} is $\frac{1}{n}$ whereas the one in the PAC-Bayes bounds is $\frac{1}{\sqrt{n}}$ so our bound has the potential to be tighter for large datasets. 
Appendix~\ref{app:pacbayes} shows an example where the PAC-Bayes bound can grow arbitrarily in a case where the stability bound is tight. 
Overall, the two approaches can have advantages in different situations and both contribute to our understanding of generalization performance of algorithms. 


%\paragraph{Example of a PAC-Bayes Bound Being Worse Than a Stability Bound}
%
%Consider a simple logistic regression scenario where the data takes on two possible values, \( x \in \{-1, 1\} \), and the corresponding labels are \( y \in \{0, 1\} \), i.e., there are only two possible examples $(x=-1, y=0)$ and $(x=1, y=1)$. 
%The dataset can contain duplicate elements. The log-likelihood in this case is given by:
%\begin{align}
%    \log p(y \mid w, x) &= -y \log (1 + \exp{(-wx)}) \\
%    & \quad - (1-y) \log (1 + \exp(wx)).
%\end{align}
%
%Assume we use a Bayesian approach to learn this model, with \( q(w) = \mathcal{N}(m, \sigma^2) \). For simplicity, we assume \(\sigma^2\) is fixed. Recall that for any objective function, we can always evaluate PAC-Bayes bounds.
%% It is important to note that for the PAC-Bayes bound, the specific objective used during training does not impact the bound.
%
%Suppose our objective is \( F(m, (x, y)) = \mathbb{E}_{q(w)}[-\log p(y \mid w, x)] \). Considering the gradient with respect to \(m\), we have the following identity 
%%\citep{rezende2014stochastic, Sheth2015, opper2008variational}:
%\citep{rezende2014stochastic, opper2008variational}:
%\begin{align}
%    \nabla_m F(m, (x, y)) &= \mathbb{E}_{q(w)} [\nabla_w \log p(y \mid w, x)].
%\end{align}
%
%Observe that:
%\begin{align*}
%    \nabla_w -\log p(y=1 \mid w, x=1)
%    =& \nabla_w \log (1+\exp(-w)) \\
%    =& -\frac{\exp{(-w)}}{1+\exp(-w)}, \\
%    \nabla_w -\log p(y=0 \mid w, x=-1) 
%    =& \nabla_w \log (1+\exp(-w)) \\
%    =& -\frac{\exp(-w)}{1+\exp(-w)}, 
%\end{align*}
%we can see that
%\begin{align}
%    &\nabla_w -\log p(y=1 \mid w, x=1) \nonumber \\
%    = &\nabla_w -\log p(y=0 \mid w, x=-1) < 0. 
%    \label{eq:same}
%\end{align}
%Therefore,
%if we run stochastic gradient descent with a constant learning rate for sufficiently many steps, we reach a solution where \(m \rightarrow +\infty\).
%
%Now, suppose the initial prior is \( P_0 = \mathcal{N}(0, \sigma^2) \). The KL divergence will eventually become:
%\begin{align}
%    \kl(\mathcal{N}(m, \sigma^2) \parallel \mathcal{N}(0, \sigma^2)) &= \frac{m^2}{2\sigma^2} \rightarrow +\infty.
%\end{align}
%
%However, if we consider the stability bound, which is based on the gradient difference, the situation changes. It’s clear that if \( z = \bar{z} \) (whether \( x = \bar{x} = 1, y = \bar{y} = 1 \) or \( x = \bar{x} = -1, y = \bar{y} = 0 \)), the gradient difference will be zero. Thus, we only need to consider the case where \( z = (1, 1) \) and \( \bar{z} = (-1, 0) \). As shown in \cref{eq:same}, the gradients are the same in this scenario as well.
%
%Therefore, using the stability bound, the generalization error will be zero. In contrast, the PAC-Bayes bound gives a value of \( \infty \), making the stability bound significantly more effective.
%


\section{Experimental Illustration}


In this section we explore the potential of stability based bounds to capture generalization error 
and compare them to PAC Bayes bounds.
We also evaluate the expansion rate that appears in the bound showing that it can be small, and hence better in practice than the use of the asymptotic bounds. 

We adopted the experimental setup used by \citet{Li} and \citet{banerjee} and conducted our experiments on CIFAR10 using the same CNN model that has been employed in these works. 
For algorithms, 
we use the ELBO (\cref{eq:elbo}) and DLM variant (\cref{eq:dlm}) with a KL divergence coefficient of 0.1, a value that has been demonstrated to yield superior results in previous studies \citep[e.g.,][]{cold-posterior}.
Our optimization was performed using the SGD optimizer with an initial learning rate of 0.005, momentum of 0.99, and we reduced the learning rate by a factor of 0.9 every 5 epochs thereafter. We select the batch size to be 1000 and set $\sigma_0=0.01$. 
%We employed classification error for evaluation purposes and set $C$ to 1. 
All experiments are run on a single NVIDIA Tesla V100 PCIe 32 GB GPU.

%The goals of the experiment is to illustrate the stability bound and its relation to the PAC-Bayes bounds under different conditions.
We perform two sets of experiments. 
In the first we test the performance of ELBO 
with or without data augmentation (random cropping and horizontal flipping \citep{shorten2019survey}) as well as random label perturbations,
comparing the generalization error (measured by 0-1 loss) and our bound (\cref{eq:kl-bound}) with $C=1$ under these situations. 
In the second, 
following the observation by \citet{dlm-bnn} that DLM (\cref{eq:dlm}) does not perform as well as ELBO in Bayesian neural networks,
we use the bounds to compare ELBO and DLM in terms of log loss. 


The primary goals of our experiments were to demonstrate the following key points.
Our bound is non-vacuous in successful learning cases and becomes 
vacuous when the dataset contains a sufficiently high proportion of random labels.
%, effectively highlighting cases where the data lacks meaningful patterns. 
In addition, our bound accurately reflects the reduction in generalization error with data augmentation.
Finally, our bound can potentially provide an explanation for the failure of DLM, suggesting that its lower stability might be the cause of higher generalization error.
For these experiments, the stability bound is both tighter and has more explanatory power than the PAC-Bayes bounds, hence demonstrating the utility of the new derivations.


%\begin{itemize}
%    \item Our bound is both non-vacuous in successful learning cases and becomes 
%    \item Our bound becomes vacuous when the dataset contains a sufficiently high proportion of random labels, effectively highlighting cases where the data lacks meaningful patterns.
%    \item Our bound accurately reflects the reduction in generalization error with data augmentation, whereas the PAC-Bayes bounds do not capture this effect correctly.
%    \item Our bound is informative in showing that DLM has a worse generalization error than ELBO, a distinction not reflected by the PAC-Bayes bounds. 
%\end{itemize}

\begin{figure*}[t]
    \centering
    \begin{subfigure}[b]{0.40\textwidth}
         \centering
         \includegraphics[width=\textwidth]{figs/CIFAR10-expansion.png}
    \end{subfigure}
    \begin{subfigure}[b]{0.40\textwidth}
         \centering
         \includegraphics[width=\textwidth]{figs/CIFAR10-DLM-expansion.png}
    \end{subfigure}
    \caption{Cumulative expansion rates under various conditions. The left panel displays expansion rates with and without data augmentation, comparing cases with random labels (50\% random, labeled as 0.5) and without random labels (labeled as 0.0). The right panel shows expansion rates across different algorithms with data augmentation and no random labels. The shaded areas represent the standard deviation across 10 runs.}
    \label{fig:expansion}
\end{figure*}


\paragraph{Expansion Rate}
We start by evaluating the expansion rate which is needed for the exact bound. 
To perform this,
%To evaluate the expansion rate, 
we randomly initialize two models and then run the same algorithm with the same batch sequence. We keep track of the norm of the parameter difference and compute the expansion rate at each step $t$. 
For simplicity, we take the maximum of the expansion rate of both $m$ and $\sigma$ (both $L_1$-norm and $L_2$-norm).


\cref{fig:expansion} shows the cumulative expansion rate under various conditions. It is evident that for each method the expansion rate increases more slowly as the number of steps increases, and the final rate shows minimal variance. 
We observe that without data augmentation the expansion rate quickly levels off. This occurs because the dataset is straightforward to learn, and once all data has been learned, the gradient approaches zero, causing the expansion rate to flatten. In contrast, with data augmentation, the expansion rate continues to grow. 
%Additionally, there is no clear relationship between random labels and the expansion rate. With data augmentation, the expansion rate on the dataset with random labels is lower than that without random labels. Conversely, without data augmentation, the expansion rate on the dataset with random labels is higher. 
We also observe that the expansion rate of DLM is slightly higher than that of ELBO.

For use in evaluating generalization bounds, 
we note that the final cumulative expansion rate is much smaller than the $\log T$ factor in \cref{cor:logT} in all cases and will therefore lead to tighter bounds in practice.
We therefore run this evaluation 10 times and use the mean value plus four standard deviation as the final value $\eta_t$.

\begin{figure*}[t]
    \centering
        \begin{subfigure}[b]{0.32\textwidth}
         \centering
         \includegraphics[width=\textwidth]{figs/CIFAR10-train_acc.png}
         \caption{Train error.}
    \end{subfigure}
    \begin{subfigure}[b]{0.32\textwidth}
         \centering
         \includegraphics[width=\textwidth]{figs/CIFAR10-test_acc.png}
         \caption{Test error.}
    \end{subfigure}
    \begin{subfigure}[b]{0.32\textwidth}
         \centering
         \includegraphics[width=\textwidth]{figs/CIFAR10-err_gen.png}
         \caption{Generalization error.}
    \end{subfigure}
    \begin{subfigure}[b]{0.32\textwidth}
         \centering
         \includegraphics[width=\textwidth]{figs/CIFAR10-stability.png}
         \caption{Stability Bound (\ref{eq:kl-bound}).}
    \end{subfigure}
    \begin{subfigure}[b]{0.32\textwidth}
         \centering
         \includegraphics[width=\textwidth]{figs/CIFAR10-prior-sqrt.png}
         \caption{PAC-Bayes (\ref{eq:sqrt}) with prior.}
    \end{subfigure}
    \begin{subfigure}[b]{0.32\textwidth}
         \centering
         \includegraphics[width=\textwidth]{figs/CIFAR10-bre.png}
         \caption{Tighter PAC-Bayes (\ref{eq:bre}).}
    \end{subfigure}
    \caption{Generalization error and bounds.}
    \label{fig:bound}
\end{figure*}

%\begin{figure*}[t]
%    \centering
%        \begin{subfigure}[b]{0.40\textwidth}
%         \centering
%         \includegraphics[width=\textwidth]{figs/CIFAR10-train_acc.png}
%         \caption{Train error.}
%    \end{subfigure}
%    \begin{subfigure}[b]{0.40\textwidth}
%         \centering
%         \includegraphics[width=\textwidth]{figs/CIFAR10-test_acc.png}
%         \caption{Test error.}
%    \end{subfigure}
%    \begin{subfigure}[b]{0.40\textwidth}
%         \centering
%         \includegraphics[width=\textwidth]{figs/CIFAR10-err_gen.png}
%         \caption{Generalization error.}
%    \end{subfigure}
%    \begin{subfigure}[b]{0.40\textwidth}
%         \centering
%         \includegraphics[width=\textwidth]{figs/CIFAR10-stability.png}
%         \caption{Bound (\cref{eq:kl-bound}).}
%    \end{subfigure}
%    \begin{subfigure}[b]{0.40\textwidth}
%         \centering
%         \includegraphics[width=\textwidth]{figs/CIFAR10-prior-sqrt.png}
%         \caption{PAC-Bayes (\cref{eq:sqrt}) with prior.}
%    \end{subfigure}
%    \begin{subfigure}[b]{0.40\textwidth}
%         \centering
%         \includegraphics[width=\textwidth]{figs/CIFAR10-bre.png}
%         \caption{Tighter PAC-Bayes (\cref{eq:bre}).}
%    \end{subfigure}
%    \caption{Generalization error and bounds.}
%    \label{fig:bound}
%\end{figure*}

\begin{figure*}[h]
    \centering
        \begin{subfigure}[b]{0.32\textwidth}
         \centering
         \includegraphics[width=\textwidth]{figs/CIFAR10-DLM-train_loss.png}
         \caption{Train loss.}
    \end{subfigure}
    \begin{subfigure}[b]{0.32\textwidth}
         \centering
         \includegraphics[width=\textwidth]{figs/CIFAR10-DLM-test_loss.png}
         \caption{Test loss.}
    \end{subfigure}    
    \begin{subfigure}[b]{0.32\textwidth}
         \centering
         \includegraphics[width=\textwidth]{figs/CIFAR10-DLM-err_gen_loss.png}
         \caption{Generalization error.}
    \end{subfigure}
    \begin{subfigure}[b]{0.32\textwidth}
         \centering
         \includegraphics[width=\textwidth]{figs/CIFAR10-DLM-wasserstein.png}
         \caption{Bound (\ref{eq:wass-bound}, without $K$).}
    \end{subfigure}
    \begin{subfigure}[b]{0.32\textwidth}
         \centering
         \includegraphics[width=\textwidth]{figs/CIFAR10-DLM-prior-sqrt.png}
         \caption{PAC-Bayes (\ref{eq:sqrt}) with prior.}
    \end{subfigure}
    \begin{subfigure}[b]{0.32\textwidth}
         \centering
         \includegraphics[width=\textwidth]{figs/CIFAR10-DLM-init-sqrt.png}
         \caption{PAC-Bayes (\ref{eq:sqrt}) with init $Q_0$.}
    \end{subfigure}
    \caption{Generalization error and bounds for ELBO and DLM with data augmentation and no random labels.}
    \label{fig:dlm-bound}
\end{figure*}


\paragraph{Generalization bounds: ELBO with data augmentation and random labels.}
To evaluate the bound with parameter differences (\cref{eq:diff}), we need to take expectations over $z$, $\Bar{z}$ and the randomness $\epsilon$. 
To perform this,
we randomly sample 50 pairs of $z$ and $\bar{z}$ from the training and test dataset, respectively. For 
%the randomness 
$\epsilon$, we conduct 10 independent runs, with each run selecting a random batch sequence and any other random samples required for optimization.

\cref{fig:bound} (a-c) present the train loss, test loss, and generalization error for ELBO in terms of 0-1 loss along with the stability bound (d) and PAC-Bayes bounds (e,f).
The generalization error is calculated as the absolute difference between the training error and the test error. For the stability bound, we set $C=1$. For PAC-Bayes bounds, we select $\delta=0.025$ and specifically for \cref{eq:bre}, we select $b=100$ and $c=0.1$ following the original paper. 

We first observe that the stability bound is non-vacuous except in the scenario without data augmentation and with 50\% random labels, where there is significant overfitting. The PAC Bayes bounds are less tight in all four scenarios. 
%and the PAC-Bayes bounds are also vacuous. 
Second, our bound induces the correct ranking over the four cases, and specifically shows that without noisy labels the generalization error is lower when data augmentation is used. 
The PAC Bayes bounds do not demonstrate the benefit of data augmentation in this case. 
%Second, we note that PAC-Bayes bounds fail to account for the benefits of data augmentation. While data augmentation reduces the generalization error, PAC-Bayes bounds show a higher bound with data augmentation compared to without it when there are no random labels. Therefore, in this case, our bound is both tighter and more informative.
%than the PAC-Bayes bounds.
Third, note that the smallest generalization error occurs in the case with both data augmentation and 50\% random labels. However, this does not imply the best performance on the test set; in this scenario, the training error converges to 0.5, and the test error is slightly above this value.
% (refer to \cref{fig:err} in the Appendix). 
Our bound captures this behavior well.

\paragraph{Generalization bounds: ELBO vs.\ DLM.}
\cref{fig:dlm-bound} (a-c) present train and test loss and generalization error in terms of log loss of ELBO vs.\ DLM,
and (d-f) present the 
stability and PAC-Bayes bounds.
When calculating the bound in \cref{eq:wass-bound}, we omit the Lipschitz constant $K$ due to the difficulty in its evaluation. Since the Lipschitz constant remains the same for a given loss function (though not necessarily for the objective), our focus is on the relative comparison between the two methods.
Our bound effectively captures the fact that DLM has a worse generalization error than ELBO. In contrast, the PAC-Bayes bounds are nearly identical for both methods.
%, providing little insight into the differences in generalization error. 
Our bound, which is based on the sum of the norms of the gradient differences, underscores the potential instability of the DLM algorithm
for Bayesian neural networks, which might explain its inferior performance for such models.
%across different data points.

\section{Conclusion and Future Work}
In this study, we presented a new generalization bound for variational inference by leveraging recent advances in stability-based bounds for Stochastic Gradient Langevin Dynamics (SGLD). Our approach extends the stability argument of stochastic gradient descent to 
a family of algorithms which includes
variational inference, addressing both mean and variance parameters.
Empirical evaluations demonstrated that our bound produces meaningful results with large neural network models and effectively captures generalization error in scenarios involving random labels and data augmentation.

This work opens several promising avenues for future research. The general applicability of our approach suggests that the bound could be extended to various Bayesian algorithms, such as $\text{PAC}^2$ variational learning \citep{Masegosa20}. However, a limitation of our approach is that the bound is primarily effective for algorithms optimized via stochastic gradient descent. For more advanced optimizers like Adam, characterizing parameter differences becomes significantly more challenging.

\section*{Acknowledgments}
This work was partly supported by NSF under grant 2246261. The experiments in this paper were run on the Big Red computing system at Indiana University, supported in part by Lilly Endowment, Inc., through its support for the Indiana University Pervasive Technology Institute.

\FloatBarrier

\bibliography{main}
\bibliographystyle{plainnat}

\appendix
% \documentclass[twoside]{article}

% \usepackage{aistats2025}
% If your paper is accepted, change the options for the package
% aistats2025 as follows:
%
%\usepackage[accepted]{aistats2025}
%
% This option will print headings for the title of your paper and
% headings for the authors names, plus a copyright note at the end of
% the first column of the first page.

% If you set papersize explicitly, activate the following three lines:
%\special{papersize = 8.5in, 11in}
%\setlength{\pdfpageheight}{11in}
%\setlength{\pdfpagewidth}{8.5in}

% If you use natbib package, activate the following three lines:
%\usepackage[round]{natbib}
%\renewcommand{\bibname}{References}
%\renewcommand{\bibsection}{\subsubsection*{\bibname}}

% If you use BibTeX in apalike style, activate the following line:
%\bibliographystyle{apalike}

% \begin{document}

% If your paper is accepted and the title of your paper is very long,
% the style will print as headings an error message. Use the following
% command to supply a shorter title of your paper so that it can be
% used as headings.
%
%\runningtitle{I use this title instead because the last one was very long}

% If your paper is accepted and the number of authors is large, the
% style will print as headings an error message. Use the following
% command to supply a shorter version of the authors names so that
% they can be used as headings (for example, use only the surnames)
%
%\runningauthor{Surname 1, Surname 2, Surname 3, ...., Surname n}

% Supplementary material: To improve readability, you must use a single-column format for the supplementary material.
\onecolumn
\appendix
\aistatstitle{From Deep Additive Kernel Learning to Last-Layer \\ Bayesian Neural Networks via Induced Prior Approximation: \\
Supplementary Materials}

\section{SPARSE CHOLESKY DECOMPOSITION}
\label{sec:sparse chol decompose}
In this section, we present the algorithm for constructing the induced grids $\mathbf{U}$ as defined in \cref{eq:GPlayer} by using sorted dyadic points, and obtaining the sparse Choleksy decomposition of the Laplace kernel in one dimension, as proposed in \citep{ding2024sparse}.

A set of one-dimensional level-$L$ dyadic points $\Xv_L$ in increasing order over the interval $[0,1]$ is defined as:
\begin{align}
    \Xv_{L}:= \left\{ \frac{1}{2^{L}}, \frac{2}{2^{L}}, \frac{3}{2^{L}}, \ldots, \frac{2^{L}-1}{2^{L}} \right\}.
\end{align}
However, this increasing order does not yield a sparse representation of the Markov kernel $k(\cdot,\cdot)$ on the points $\Xv_L$, i.e., Cholesky decomposition of the covariance matrix $k(\Xv_L, \Xv_L)$ is not sparse. To achieve a sparse hierarchical expansion, we first sort the dyadic points $\Xv_L$ according to their levels.

\paragraph{Sorted Dyadic Points}
For level-$\ell$ dyadic points $\Xv_{\ell}$ where $ \ell=1,\ldots,L$, we first define the set $\rho(\ell)$ consisting of odd numbers as follows:
\begin{align}
    \rho(\ell) = \left\{ 1,3,5,\ldots,2^{\ell}-1 \right\}.
\end{align}
Next, we define the sorted incremental set $\Dv_{\ell}$ (with $\Xv_{0}:= \varnothing$) as:
\begin{align}
    \Dv_{\ell} = 
    \left\{ \frac{i}{2^{\ell}}: i\in \rho(\ell) \right\} = \Xv_{\ell} - \Xv_{\ell-1}, \quad  \ell=1,\ldots L.
\end{align}
Thus, the level-$L$ dyadic points $\Xv_L$ can be decomposed into disjoint incremental sets $\{ \Dv_{\ell} \}_{\ell=1}^{L}$:
\begin{align}
    \Xv_{L} = \cup_{\ell=1}^{L} \Dv_{\ell}, \quad \Dv_{i} \cap \Dv_{j} = \varnothing \text{ for $i\neq j$}.
\end{align}
Therefore, we can define the sorted level-$L$ dyadic points using these incremental sets as:
\begin{align}\label{eq:sorted dyadic}
    \Xv_{L}^{\text{sort}}:= \left\{ \Dv_1,\Dv_2, \ldots, \Dv_{L} \right\} 
    = \left\{ \frac{i \in \rho(\ell) }{2^{\ell}}, \ell=1,\ldots,L \right\}.
\end{align}
For example, the sorted level-3 dyadic points are given by:
\begin{align}
    \Xv_{3}^{\text{sort}} 
    = \bigg\{ 
    \begingroup
        \color{blue}
        \underbracket{
            \color{black}
            \frac{1}{2^1}
        }_{\color{blue}
            \Dv_1
        }
    \endgroup
    , 
    \begingroup
        \color{blue}
        \underbracket{
            \color{black}
            \frac{1}{2^2}, \frac{3}{2^2}
        }_{\color{blue}
            \Dv_2
        }
    \endgroup
    ,
    \begingroup
        \color{blue}
        \underbracket{
            \color{black}
            \frac{1}{2^3}, \frac{3}{2^3}, \frac{5}{2^3}, \frac{7}{2^3}
        }_{\color{blue}
            \Dv_3
        }
    \endgroup
     \bigg\}.
\end{align}

\paragraph{Algorithm}
We now present the algorithm for computing the inverse of the upper triangular Cholesky factor $[ \Lv_{\Xv_{L}^{\text{sort}}}^{\top} ]^{-1}$ of the covariance matrix $k(\Xv_{L}^{\text{sort}}, \Xv_{L}^{\text{sort}})$ in \Cref{alg:cholesky}, where $\Lv_{\Xv_{L}^{\text{sort}}} \Lv_{\Xv_{L}^{\text{sort}}}^{\top} = k(\Xv_{L}^{\text{sort}}, \Xv_{L}^{\text{sort}})$.. The corresponding proof can be found in \citep{ding2024sparse}. The output of \Cref{alg:cholesky} is a sparse matrix with $\Oc(3 \cdot (2^{L}-1))$ nonzero entries. Since each iteration of the for-loop only requires solving a $3 \times 3$ linear system, which costs $\Oc(3^3)$ time, the total computational complexity of \Cref{alg:cholesky} is $\Oc(2^L-1)$. This implies that the complexity of computing $\left[ \Lv_{\Uv}^{\top} \right]^{-1}$ in \cref{eq:GPlayer} is $\Oc(M)$ when $\Uv$, the induced grid of size $M$, consists of sorted dyadic points as defined in \cref{eq:sorted dyadic}.

\begin{algorithm}[hbt!]
\caption{Computation of the inverse Cholesky factor for the Markov kernel $k(\cdot, \cdot)$ on sorted one-dimensional level-$L$ dyadic points $\Xv_L^{\text{sort}}$.}
\label{alg:cholesky}
\setstretch{0.99} % set the line spacing to 0.99
\begin{algorithmic}[1]
    \STATE {\bfseries Input:} Markov kernel $k(\cdot,\cdot)$, sorted level-$L$ dyadic points $\Xv_{L}^{\text{sort}}$
    \STATE {\bfseries Output:} inverse of the upper triangular Cholesky factor $\Rv:= [ \Lv_{\Xv_{L}^{\text{sort}}}^{\top} ]^{-1}$, s.t. $\Lv_{\Xv_{L}^{\text{sort}}} \Lv_{\Xv_{L}^{\text{sort}}}^{\top} = k(\Xv_{L}^{\text{sort}}, \Xv_{L}^{\text{sort}})$
    \STATE Initialize $\Rv \leftarrow \text{zeros($2^L-1$,$2^L-1$)}$;
    \STATE Define $k(\pm \infty, \cdot) = k(\cdot, \pm \infty) = 0$;
    \FOR{$\ell=1$ {\bfseries to} $L$}
        \FOR{$i \in \rho(\ell)=\{1,3,\ldots,2^{\ell}-1\}$}
            \STATE $x_{\text{mid}} := \frac{i}{2^{\ell}}$;\quad
            $x_{\text{left}}:=\frac{i-1}{2^{\ell}}$ {\bfseries if} $i>1$ {\bfseries else} $-\infty$;\quad
            $x_{\text{right}}:=\frac{i+1}{2^{\ell}}$ {\bfseries if} $i<2^{\ell}-1$ {\bfseries else} $+\infty$;
            \STATE Get $i_{\text{mid}}$, $i_{\text{left}}$, $i_{\text{right}}$, the indices of the points $x_{\text{mid}}$, $x_{\text{left}}$, $x_{\text{right}}$ in the sorted set $\Xv_{L}^{\text{sort}}$ respectively;
            \STATE Get the coefficients $c_1$, $c_2$, $c_3$ by solving the following linear system:
            \begin{align}
                \begin{bmatrix}
                     & k(x_{\text{left}}, x_{\text{left}})
                     & k(x_{\text{left}}, x_{\text{mid}})
                     & k(x_{\text{left}}, x_{\text{right}}) \\
                     & k(x_{\text{mid}}, x_{\text{left}})
                     & k(x_{\text{mid}}, x_{\text{mid}})
                     & k(x_{\text{mid}}, x_{\text{right}}) \\
                     & k(x_{\text{right}}, x_{\text{left}})
                     & k(x_{\text{right}}, x_{\text{mid}})
                    &k(x_{\text{right}}, x_{\text{right}})
                \end{bmatrix}
                \begin{bmatrix}
                    c1\\
                    c2\\
                    c3
                \end{bmatrix}=
                \begin{bmatrix}
                    0\\
                    1\\
                    0
                \end{bmatrix}.
            \end{align}
            \STATE $[c_1,c_2,c_3] := [c_1,c_2,c_3] / \sqrt{c_2}$;
            \STATE {\bfseries if} $x_{\text{left}} \neq - \infty$, 
            {\bfseries then} $\Rv[i_{\text{left}} ,i_{\text{mid}}] = c_1$; \quad
            {\bfseries if} $x_{\text{right}} \neq + \infty$, 
            {\bfseries then} $\Rv[i_{\text{right}} ,i_{\text{mid}}] = c_3$;
            \STATE $\Rv[i_{\text{mid}} ,i_{\text{mid}}] = c_2$;
        \ENDFOR
    \ENDFOR
\end{algorithmic}
\end{algorithm}


\section{REPARAMETERIZATION OF KERNEL LENGTHSCALES}
\label{sec:theo}
Considering the additive Laplace kernel with fixed lengthscale $\tilde{\theta}$ for all base kernels, applying linear projections $\left\{ \wv_{p}^{\top}\xv \right\}_{p=1}^{P}$ on inputs $\xv\in \Rb^D$ will give:
\begin{align}
    &\sum_{p=1}^{P}\sigma^2_p k_p\left( \wv^{\top}_{p}\xv,\wv^{\top}_{p}\xv^{\prime} \right)\nonumber \\
    = & \sum_{p=1}^{P} \sigma^2_p\exp \left( -  \frac{\sum_{d=1}^{D} \left| w_{p,d}\left( x_{d}-x_{d}^{\prime} \right) \right|}{\tilde{\theta}} \right)\nonumber \\
    = & \sum_{p=1}^{P} \prod_{d=1}^{D} \sigma^2_p\exp \left( - \frac{\left| x_{d}-x_{d}^{\prime} \right|}{\tilde{\theta} / \left| w_{p,d}\right| } \right)\nonumber \\
    = & \sum_{p=1}^{P} \prod_{d=1}^{D} \sigma^2_p\exp \left( - \frac{\left| x_{d}-x_{d}^{\prime} \right|}{\theta_{p,d}} \right),
\end{align}
This still leads to an additive Laplace kernel but with adaptive lengthscale $\theta_{p,d}$ for base kernels. The resulting kernel also retains \emph{sparse} Cholesky decomposition by the properties of Markov kernels so that the complexity of inference is $\Oc(M)$.

\section{INFERENCE OF PREDICTIVE DISTRIBUTION}
\label{sec:uq of inference}
Given an input $\xv \in \Rb^D$, the prediction of the DAK model can be written in the following equation according to \cref{eq:DAK prediction}: 
\begin{align}
    \tilde{f}_{\xv}
    &= \sum_{p=1}^{P}
    \sigma_p \Big(
        \phi(h_{\psi}^{[p]}(\xv)) \zv_p
    \Big) + \mu \nonumber\\
    &= \sum_{p=1}^{P}
    \sigma_p \Big(
        \bm{\phi}_{p}^{\top} \zv_p
    \Big) + \mu,
\end{align}
where $\bm{\phi}_{p}^{\top}:=\phi(h_{\psi}^{[p]}(\xv)) \in \Rb^{1 \times M}$
% , $\mu_p:=\mu_p(h_{\psi}^{[p]}(\xv)) \in \Rb$
. We assume the variational distribution over the independent Gaussian weights $\zv_p \sim \Nc(\bm{m}_{\zv_p}, \Sv_{\zv_p})$ and the bias $\mu \sim \Nc(m_{\mu}, \sigma_{\mu}^2)$. Then it's straighforward to deduce that
\begin{align}
    \bm{\phi}_{p}^{\top} \zv_p + \mu 
    &\sim
    \Nc\left(
    \bm{\phi}_{p}^{\top} \bm{m}_{\zv_p} + m_{\mu},\hspace{0.2em}
    \bm{\phi}_{p}^{\top} \Sv_{\zv_p} \bm{\phi}_{p} + \sigma_{\mu}^2
    \right), \\
    \sigma_p \left(
    \bm{\phi}_{p}^{\top} \zv_p 
    \right) + \mu
    & \sim
    \Nc\left(
    \sigma_p ( \bm{\phi}_{p}^{\top} \bm{m}_{\zv_p} )+ m_{\mu} ,\hspace{0.2em}
    \sigma_p^2( \bm{\phi}_{p}^{\top} \Sv_{\zv_p} \bm{\phi}_{p}) + \sigma_{\mu}^2
    \right), \\
    \tilde{f}_{\xv} = 
    \sum_{p=1}^{P}
    \sigma_p \left(
    \bm{\phi}_{p}^{\top} \zv_p
    \right) + \mu
    & \sim
    \Nc\left(
    \sum_{p=1}^{P}
    \sigma_p ( \bm{\phi}_{p}^{\top} \bm{m}_{\zv_p}) + m_{\mu} ,\hspace{0.2em}
    \sum_{p=1}^{P}
    \sigma_p^2( \bm{\phi}_{p}^{\top} \Sv_{\zv_p} \bm{\phi}_{p} ) + \sigma_{\mu}^2
    \right).
\end{align}
Therefore, we obtain the predictive distribution of the $\tilde{f}(\xv)$ at the point $\xv \in \Rb^D$ and its mean and variance are given by:
\begin{subequations}
\label{eq:dak inference closed form}
\begin{align}
    \Eb\left[ \tilde{f}_{\xv} \right]
        = \sum_{p=1}^{P}
        \sigma_p ( \bm{\phi}_{p}^{\top} \bm{m}_{\zv_p}) + m_{\mu},
\end{align}
\begin{align}
    \text{Var}\left[ \tilde{f}_{\xv} \right]
        =\sum_{p=1}^{P}
        \sigma_p^2( \bm{\phi}_{p}^{\top} \Sv_{\zv_p} \bm{\phi}_{p}) + \sigma_{\mu}^2.
\end{align}
\end{subequations}
% \begin{subequations}
% \label{eq:dak inference closed form}
%     \begin{align}
%         \Eb\left[ \tilde{f}(\xv) \right]
%         = \sum_{p=1}^{P}
%         \sigma_p ( \bm{\phi}_{p}^{\top} \bm{m}_{\zv_p} + m_{\mu_p} ),
%     \end{align}
%     \begin{align}
%         \text{Var}\left[ \tilde{f}(\xv) \right]
%         =\sum_{p=1}^{P}
%         \sigma_p^2( \bm{\phi}_{p}^{\top} \Sigma_{\zv_p} \bm{\phi}_{p} + \sigma_{\mu_p}^2).
%     \end{align}
% \end{subequations}


\section{TRAINING OF VARIATIONAL INFERENCE}
\label{sec:training}
Given the dataset $\mathcal{D}=\{ \Xv, \yv \}$ where $\Xv:=\{ \xv_i \}_{i=1}^N$, $\yv=(y_1,\ldots,y_N)^{\top}$, $\xv_i \in \Rb^D$, $y_i\in\Rb$, the prediction $\tilde{f}_{\Xv}\in \Rb^N$ of DAK is given by all the parameters $\bm{\theta}=\left\{ \psi, \bm{\sigma} \right\}$, $\bm{\eta}=\left\{ \{ \mv_{\zv_{p}},\Sv_{\zv_{p}}\}_{p=1}^{P} , \{m_{\mu},\sigma_{\mu} \} \right\}$ according to \cref{eq:DAK prediction}:
\begin{align}
    \tilde{f}_{\Xv}:= \tilde{f}(\Xv; \bm{\theta}, \bm{\eta})
    = \sum_{p=1}^{P}
    \sigma_p \Big(
        \phi(h_{\psi}^{[p]}(\Xv)) \zv_p
    \Big) + \mu,
\end{align}
where $\zv_{p} \sim \mathcal{N} (\bm{m}_{\zv_p} ,\Sv_{\zv_p})$, $p=1,\ldots,P$, and $\mu \sim \mathcal{N} ( m_{\mu},\sigma^2_{\mu} )$ are variational variables $\Theta_{\text{var}}$ parameterized by $\bm{\eta}$. The variational distribution is denoted by $q_{\bm{\eta}}(\Theta_{\text{var}})= q(\mu)\prod_{p=1}^{P} q(\zv_{p}) = \Nc ( m_{\mu} ,\sigma_{\mu}^2 )\prod_{p=1}^{P} 
\Nc ( \bm{m}_{\zv_p} ,\Sv_{\zv_p} )$, and the variational prior is denoted by $p(\Theta_{\text{var}})$.

We consider the KL divergence between $q_{\bm{\eta}}(\Theta_{\text{var}})$ and the true posterior $p(\Theta_{\text{var}}\vert \yv, \Xv, \bm{\theta})$:
\begin{align}
& \qquad \text{KL} \left[ q_{\bm{\eta}}(\Theta_{\text{var}}) \| p(\Theta_{\text{var}} \vert \yv,\Xv, \bm{\theta} ) \right] \nonumber \\
= & \int q_{\bm{\eta}}(\Theta_{\text{var}} )\log \frac{q_{\bm{\eta}}(\Theta_{\text{var}} )}{p(\Theta_{\text{var}} \vert \yv,\Xv,\bm{\theta} )} d\Theta_{\text{var}} \nonumber \\
= & \int q_{\bm{\eta}}(\Theta_{\text{var}} )\log \frac{q_{\bm{\eta}}(\Theta_{\text{var}} )p(\yv \vert \Xv,\bm{\theta})}{p(\yv \vert \Xv,\bm{\theta} ,\Theta_{\text{var}} )p(\Theta_{\text{var}} )} d\Theta_{\text{var}} \nonumber \\
= & \int q_{\bm{\eta}}(\Theta_{\text{var}} )\log \frac{q_{\bm{\eta}}(\Theta_{\text{var}} )}{p(\Theta_{\text{var}} )} d\Theta_{\text{var}} -\int q_{\bm{\eta}}(\Theta_{\text{var}} )\log p(\yv \vert \tilde{f}_{\Xv} )d\Theta_{\text{var}} +\log p(\yv\vert \Xv,\bm{\theta}).
\end{align}
Using the fact that $\text{KL}[\cdot \| \cdot] \geq 0$, we have
\begin{align}
\label{eq:variational lower bound}
    \log p(\yv\vert \Xv,\bm{\theta}) & \geq \int q_{\bm{\eta}}(\Theta_{\text{var}} )\log p(\yv \vert \tilde{f}_{\Xv} )d\Theta_{\text{var}} - \text{KL} \left[ q_{\bm{\eta}}(\Theta_{\text{var}} ) \| p(\Theta_{\text{var}}) \right] \nonumber \\
    & = \Eb_{q_{\bm{\eta}}(\Theta_{\text{var}} )} \left[ \log p(\yv \vert \tilde{f}_{\Xv} ) \right] - \text{KL} \left[ q_{\bm{\eta}}(\Theta_{\text{var}} ) \| p(\Theta_{\text{var}}) \right].
\end{align}

\paragraph{Full-training.}
Firstly, we present the joint training of $\bm{\theta}$ and $\bm{\eta}$. The most common approach optimizes the marginal log-likelihood (the left-hand side of \cref{eq:variational lower bound}):
\begin{align}
    \bm{\theta}^{\ast} &=\argmax_{\bm{\theta}} \log p(\yv\vert \Xv,\bm{\theta} ) \\
    &= \argmax_{\bm{\theta}} \log \int p\left( y\vert X,\bm{\theta},\Theta_{\text{var}} \right) p(\Theta_{\text{var}})d\Theta_{\text{var}},
\end{align}
which involves intractable integral in some tasks such as classification. Instead, we optimize the variational lower bound (the right-hand side of \cref{eq:variational lower bound}):
\begin{align}
    \Theta^{\ast} := \argmax_{\bm{\theta},\bm{\eta}} \mathcal{L}(\bm{\theta},\bm{\eta}) =\argmax_{\bm{\theta},\bm{\eta}}\left\{ E_{q_{\bm{\eta}}(\Theta_{\text{var}} )}\left[ \log p(\yv|\tilde{f}_{\Xv} ) \right] -\text{KL} \left[ q_{\bm{\eta}}(\Theta_{\text{var}} )\| p(\Theta_{\text{var}} ) \right] \right\}.
\end{align}

\paragraph{Fine-tuning.}
An alternative training approach is to firstly pre-train the deterministic parameters of feature extractor by standard neural network training, with mean squared error for regression or cross-entropy for classification as the loss function, and then fine-tune the last layer additive GP with fixed features. The objective function is identical to \cref{eq:elbo}, but $\bm{\theta}$ is learned during the pre-training step and is no longer optimized during fine-tuning.


\section{ELBO}%{DERIVATION OF ELBO}
\label{sec:elbo}
\subsection{Assumptions}
Consider the model $y_i = \tilde{f}(\xv_i) + \epsilon_i$ with the i.i.d. noise $\epsilon_i \overset{\text{i.i.d.}}{\sim} \Nc(0, \sigma_{f}^2)$ and $\tilde{f} : \Rb^D \rightarrow \Rb$ is defined in \cref{eq:DAK prediction}. The training dataset is $\mathcal{D} = \{ \Xv, \yv \}$ where $\Xv:=\{ \xv_i \}_{i=1}^N$, $\yv=(y_1,\ldots,y_N)^{\top}$, $\xv_i \in \Rb^D$, $y_i\in\Rb$. $\Theta_{\text{var}}:= \{ \mu ,\{ \zv_{p}\}_{p=1}^{P} \}$ are the variational random variables consisting of Gaussian weights and bias of $P$ units, $\psi$ are the parameters of the NN, $\bm{\sigma}:=(\sigma_1, \ldots, \sigma_p)^{\top}$ are the scale parameters of base GP layers. The variational distributions are $q(\mu)=\Nc(m_{\mu}, \sigma_{\mu}^2)$, $q(\zv_p)=\Nc(\bm{m}_{\zv_p}, \Sv_{\zv_p})$ and the variational priors are $p(\mu)=\Nc(\check{m}_{\mu} ,\check{\sigma}^2_{\mu})$, $p(\zv_p)=\Nc(\check{\bm{m}}_{\zv_p} ,\check{\Sv}_{\zv_p})$. Note that $\Sv_{\zv_p}\in\Rb^{M \times M}$ is a diagonal covariance matrix due to the independence of $\zv_p$, $M$ is the number of inducing points $\Uv$ defined in \cref{eq:GPlayer}, and $\bm{m}_{\zv_p} \in \Rb^M$, $m_{\mu} \in \Rb$, $\sigma_{\mu}^2 \in \Rb$. We derive the ELBO in VI to learn the preditive posterior over the variational variables $\Theta_{\text{var}}:= \{ \mu ,\{ \zv_{p}\}_{p=1}^{P} \}$ parameterized by $\bm{\eta}:=\left\{ \{ \mv_{\zv_{p}},\Sv_{\zv_{p}}\}_{p=1}^{P} , \{m_{\mu},\sigma_{\mu} \} \right\}$, and optimize the deterministic parameters $\bm{\theta}:=\{\psi, \bm{\sigma}\}$.

\subsection{Expected Log Likelihood}
\paragraph{Closed Form}
The \emph{expected log likelihood}, which is the first term in ELBO defined in \cref{eq:elbo}, is given by 
\begin{align}
    {\Eb}_{q_{\bm{\eta}}(\Theta_{\text{var}})} \left[ \log \text{Pr} (\yv \vert \tilde{f}_{\Xv} ) \right]
    &= {\Eb}_{q_{\bm{\eta}}(\Theta_{\text{var}})} \left[ 
    \log \prod_{i=1}^{N} 
    p (y_i \vert \tilde{f}_{\xv_i} )
    \right] \nonumber\\
    &= \sum_{i=1}^{N} 
    {\Eb}_{q_{\bm{\eta}}(\Theta_{\text{var}})} \left[ 
    \log
    p (y_i \vert \tilde{f}_{\xv_i} )
    \right] \nonumber\\
    &= \sum_{i=1}^{N} 
    {\Eb}_{q_{\bm{\eta}}(\Theta_{\text{var}})} \left[ 
    \log
    \Nc( \tilde{f}_i,\hspace{0.2em} \sigma_{f}^2 )
    \right] \nonumber\\
    &= \sum_{i=1}^{N} 
    {\Eb}_{q_{\bm{\eta}}(\Theta_{\text{var}})} \left[ 
    \log \left(
    (2\pi \sigma_{f}^2)^{-\frac{1}{2}}
    \exp\left\{  
        -\frac{ (y_i - \tilde{f}_i)^2 }{2 \sigma_{f}^2}
    \right\}
    \right)
    \right] \nonumber\\
    &= \sum_{i=1}^{N} 
    {\Eb}_{q_{\bm{\eta}}(\Theta_{\text{var}})} \left[
    -\frac{1}{2} \log(2\pi) 
    - \frac{1}{2}\log(\sigma_{f}^2)
    - \frac{1}{2 \sigma_{f}^2}
    (y_i - \tilde{f}_i)^2
    \right] \nonumber\\
    &= - \frac{N}{2} \log(2\pi)
    - \frac{N}{2} \log(\sigma_{f}^2)
    - \frac{1}{2 \sigma_{f}^2}
    \sum_{i=1}^{N}
    {\Eb}_{q_{\bm{\eta}}(\Theta_{\text{var}})} \left[
    (y_i - \tilde{f}_i)^2
    \right] \nonumber\\
    &= - \frac{N}{2} \log(2\pi)
    - \frac{N}{2} \log(\sigma_{f}^2)
    - \frac{1}{2 \sigma_{f}^2}
    \sum_{i=1}^{N} \left(
    \left({\Eb}_{q(\Theta_{\text{var}})} \left[
    (y_i - \tilde{f}_i)
    \right] \right)^2
    + \text{Var}_{q(\Theta_{\text{var}})} \left[
    (y_i - \tilde{f}_i)
    \right]
    \right) \label{eq:evidence halfway},
\end{align}
where
\begin{align}
    \tilde{f}_i
    % \mu_{\tilde{f}_i} &:= \tilde{f}(\xv_i;\Theta_{\text{var}}, \Theta_{\text{det}} ) \nonumber\\
    &= \sum_{p=1}^{P} \sigma_p \Big(
    \begingroup
        \color{blue}
        \underbracket{
            \color{black}
            \phi(h_{\psi}^{[p]}(\xv_i))
        }_{\color{blue}
            :=\bm{\phi}_{i,p}^{\top} \in \Rb^{1 \times M}
        }
    \endgroup
    \zv_p
    \Big)
    + \mu
    % \begingroup
    %     \color{blue}
    %     \underbracket{
    %         \color{black}
    %         \mu_{p}(h_{\psi}^{[p]}(\xv_i))
    %     }_{\color{blue}
    %         :=\mu_{i,p} \in \Rb
    %     }
    % \endgroup 
    \nonumber\\
    &= \sum_{p=1}^{P} \sigma_p \left(
    \bm{\phi}_{i,p}^{\top} \zv_p 
    \right) + \mu.
\end{align}
Recall that the variational assumptions $q(\zv_p)=\Nc(\bm{m}_{\zv_p}, \Sv_{\zv_p})$ and $q(\mu)=\Nc(m_{\mu}, \sigma_{\mu}^2)$, we can infer that
\begin{align}
    \bm{\phi}_{i,p}^{\top} \zv_p + \mu 
    &\sim
    \Nc\left(
    \bm{\phi}_{i,p}^{\top} \bm{m}_{\zv_p} + m_{\mu},\hspace{0.2em}
    \bm{\phi}_{i,p}^{\top} \Sv_{\zv_p} \bm{\phi}_{i,p} + \sigma_{\mu}^2
    \right), \\
    \sigma_p \left(
    \bm{\phi}_{i,p}^{\top} \zv_p 
    \right) + \mu
    & \sim
    \Nc\left(
    \sigma_p ( \bm{\phi}_{i,p}^{\top} \bm{m}_{\zv_p} ) + m_{\mu},\hspace{0.2em}
    \sigma_p^2( \bm{\phi}_{i,p}^{\top} \Sv_{\zv_p} \bm{\phi}_{i,p} ) + \sigma_{\mu}^2
    \right), \\
    \tilde{f}_i = 
    \sum_{p=1}^{P}
    \sigma_p \left(
    \bm{\phi}_{i,p}^{\top} \zv_p 
    \right)+ \mu
    & \sim
    \Nc\left(
    \sum_{p=1}^{P}
    \sigma_p ( \bm{\phi}_{i,p}^{\top} \bm{m}_{\zv_p} )+ m_{\mu},\hspace{0.2em}
    \sum_{p=1}^{P}
    \sigma_p^2( \bm{\phi}_{i,p}^{\top} \Sv_{\zv_p} \bm{\phi}_{i,p} ) + \sigma_{\mu}^2
    \right), \\
    y_i - \tilde{f}_i
    & \sim 
    \Nc\left(
    y_i - 
    \sum_{p=1}^{P}
    \sigma_p ( \bm{\phi}_{i,p}^{\top} \bm{m}_{\zv_p} ) -m_{\mu},\hspace{0.2em}
    \sum_{p=1}^{P}
    \sigma_p^2( \bm{\phi}_{i,p}^{\top} \Sv_{\zv_p} \bm{\phi}_{i,p} ) + \sigma_{\mu}^2
    \right).
\end{align}
Therefore, 
\begin{subequations}\label{eq:exp and var in evidence}
    \begin{align}
        \left({\Eb}_{q(\Theta_{\text{var}})} \left[
        (y_i - \tilde{f}_i)
        \right] \right)^2
        = \left(
         y_i - 
        \sum_{p=1}^{P}
        \sigma_p ( \bm{\phi}_{i,p}^{\top} \bm{m}_{\zv_p} ) -m_{\mu}
        \right)^2,
    \end{align}
    \begin{align}
        \text{Var}_{q(\Theta_{\text{var}})}
        \left[
        (y_i - \tilde{f}_i)
        \right]
        = \sum_{p=1}^{P}
        \sigma_p^2( \bm{\phi}_{i,p}^{\top} \Sv_{\zv_p} \bm{\phi}_{i,p} ) + \sigma_{\mu}^2.
    \end{align}
\end{subequations}
By applying \cref{eq:exp and var in evidence} to \cref{eq:evidence halfway}, we derive the analytical formula for the expected evidence, expressed as
\begin{align}
    {\Eb}_{q_{\bm{\eta}}(\Theta_{\text{var}})} \left[ \log \text{Pr} (\yv \vert \tilde{f}_{\Xv} ) \right]
    &= - \frac{N}{2} \log(2\pi)
    - \frac{N}{2} \log(\sigma_{f}^2) \nonumber\\
    &- \frac{1}{2 \sigma_{f}^2}
    \sum_{i=1}^{N} \left(
        \Big(
         y_i - 
        \sum_{p=1}^{P}
        \sigma_p ( \bm{\phi}_{i,p}^{\top} \bm{m}_{\zv_p} ) -m_{\mu}
        \Big)^2
        + \sum_{p=1}^{P}
        \sigma_p^2( \bm{\phi}_{i,p}^{\top} \Sv_{\zv_p} \bm{\phi}_{i,p} )+ \sigma_{\mu}^2
    \right). \label{eq:evidence final}
\end{align}

\paragraph{Monte Carlo Approximation}
For comparison, we provide the equation for computing the Monte Carlo estimate of the ELBO in the paragraph that follows.
\begin{align}
    {\Eb}_{q_{\bm{\eta}}(\Theta_{\text{var}})} \left[ \log \text{Pr} (\yv \vert \tilde{f}_{\Xv} ) \right]
    % &= {\Eb}_{q(\Theta)} \left[ 
    % \log \prod_{i=1}^{N} 
    % p (y_i \vert \xv_i,\Theta, \psi, \bm{\sigma})
    % \right] \nonumber\\
    &= \sum_{i=1}^{N} 
    {\Eb}_{q_{\bm{\eta}}(\Theta_{\text{var}} )} \left[ 
    \log
    p (y_i \vert \tilde{f}_{\xv_i} )
    \right] \nonumber\\
    & \approx \sum_{i=1}^{N}
    \frac{1}{S}
     \sum_{s=1}^{S}
    \log
    p (y_i \vert \xv_i,\tilde{\Theta}^{(s)}_{\text{var}}, \bm{\theta} ) \nonumber\\
    &= \frac{1}{S} \sum_{i=1}^{N} 
    \sum_{s=1}^{S} 
    \log
    \Nc(y_i \left\vert\right. \tilde{f}_{i}^{(s)},\hspace{0.2em} \sigma_{f}^2 )
    \nonumber\\
    &= \frac{1}{S} \sum_{i=1}^{N} 
    \sum_{s=1}^{S} 
    \log \left(
    (2\pi \sigma_{f}^2)^{-\frac{1}{2}}
    \exp\left\{  
        -\frac{ (y_i - \tilde{f}_{i}^{(s)})^2 }{2 \sigma_{f}^2}
    \right\}
    \right)
    \nonumber\\
    &= \frac{1}{S} \sum_{i=1}^{N} 
    \sum_{s=1}^{S} \left(
    -\frac{1}{2} \log(2\pi) 
    - \frac{1}{2}\log(\sigma_{f}^2)
    - \frac{1}{2 \sigma_{f}^2}
    (y_i - \tilde{f}_{i}^{(s)})^2
    \right) \nonumber\\
    &= - \frac{N}{2} \log(2\pi)
    - \frac{N}{2} \log(\sigma_{f}^2)
    - \frac{1}{2 \sigma_{f}^2}
    \sum_{i=1}^{N}
    \frac{1}{S} \sum_{s=1}^{S}
    (y_i - \tilde{f}_{i}^{(s)})^2, \label{eq:evidence halfway mc approx}
\end{align}
where $S$ is the number of Monte Carlo samples, $\{  \tilde{\mu}^{(s)} ,\{ \tilde{\zv}_{p}^{(s)} \}_{p=1}^{P} \} := \tilde{\Theta}^{(s)}_{\text{var}}$ are the $s$-th Monte Carlo samplings over the variational parameters $\Theta_{\text{var}}$ and $\tilde{\Theta}^{(s)}_{\text{var}} \sim q_{\bm{\eta}}(\Theta_{\text{var}})$, $\tilde{f}_{i}^{(s)}$ is given as follows:
\begin{align}
    \tilde{f}_{i}^{(s)} &:= \tilde{f}(\xv_i;\tilde{\Theta}^{(s)}_{\text{var}},\bm{\theta} ) \nonumber\\
    &= \sum_{p=1}^{P} \sigma_p \Big(
    \begingroup
        \color{blue}
        \underbracket{
            \color{black}
            \phi(h_{\psi}^{[p]}(\xv_i))
        }_{\color{blue}
            :=\bm{\phi}_{i,p}^{\top} \in \Rb^{1 \times M}
        }
    \endgroup
    \tilde{\zv}_p^{(s)} 
    \Big) + \tilde{\mu}^{(s)} \nonumber\\
    &= \sum_{p=1}^{P} \sigma_p \left(
    \bm{\phi}_{i,p}^{\top} \tilde{\zv}_p^{(s)} 
    \right)+ \tilde{\mu}^{(s)}. \label{eq:mc approx mean}
\end{align}
Therefore, we plug \cref{eq:mc approx mean} into \cref{eq:evidence halfway mc approx} and get the the Monte Carlo estimate of the ELBO written in the following formula:
\begin{align}
    {\Eb}_{q_{\bm{\eta}}(\Theta_{\text{var}})} \left[ \log \text{Pr} (\yv \vert \tilde{f}_{\Xv} ) \right]
    &\approx
    - \frac{N}{2} \log(2\pi)
    - \frac{N}{2} \log(\sigma_{f}^2)
    - \frac{1}{2 \sigma_{f}^2}
    \sum_{i=1}^{N}
    \frac{1}{S} \sum_{s=1}^{S}
    \Big(y_i - 
    \sum_{p=1}^{P} \sigma_p \left(
    \bm{\phi}_{i,p}^{\top} \tilde{\zv}_p^{(s)} 
    \Big)- \tilde{\mu}^{(s)}
    \right)^2, \label{eq:evidence final mc approx} \\
    \tilde{\zv}_p^{(s)} &\sim \Nc(\bm{m}_{\zv_p}, \Sv_{\zv_p}),\qquad
    \tilde{\mu}^{(s)} \sim \Nc(m_{\mu}, \sigma_{\mu}^2).
\end{align}


\subsection{KL Divergence}
Since we place Gaussian assumptions over the variational parameters $\Theta_{\text{var}}$,  the \emph{KL divergence}, which is the second term in ELBO defined in \cref{eq:elbo}, is then given by
\begin{align}
    \text{KL} \left[ q(\Theta_{\text{var}} ) \| p(\Theta_{\text{var}}) \right]
    &= \text{KL} \left[ q( \mu ,\{ \zv_{p}\}_{p=1}^{P} ) \Vert p( \mu ,\{ \zv_{p}\}_{p=1}^{P}) \right] \nonumber\\
    & =  
    \text{KL} \left[ q(\mu) \Vert p(\mu) \right] 
    + \sum_{p=1}^{P} 
    \text{KL} \left[ q(\zv_{p}) \Vert p(\zv_{p}) \right],
\end{align}

\begin{align}
     \text{KL} \left[ q(\mu) \Vert p(\mu) \right]
     = \frac{1}{2} \left(
     \frac{\sigma_{\mu}^2}{\check{\sigma}_{\mu}^2} 
     + \frac{(m_{\mu} - \check{m}_{\mu})^2}{\check{\sigma}_{\mu}^2} 
     -\log\left( \frac{\sigma_{\mu}^2}{\check{\sigma}_{\mu}^2} \right)
     -1
     \right),
\end{align}

\begin{align}
    \text{KL} \left[ q(\zv_{p}) \Vert p(\zv_{p}) \right]
    = \frac{1}{2} \sum_{i=1}^{M} \left(
     \frac{[\Sv_{\zv_p}]_{ii}}{[\check{\Sv}_{\zv_p}]_{ii}} 
     + \frac{([\bm{m}_{\zv_p}]_{i} - [\check{\bm{m}}_{\zv_p}]_i)^2}{[\check{\Sv}_{\zv_p}]_{ii}}
     -\log\left( 
     \frac{[\Sv_{\zv_p}]_{ii}}{[\check{\Sv}_{\zv_p}]_{ii}}  
     \right)
     -1
     \right),
\end{align}
where $[\Sv_{\zv_p}]_{ii}$ is the $(i,i)$-th element of the diagonal covariance matrix $\Sv_{\zv_p} \in \Rb^{M \times M}$, $[\bm{m}_{\zv_p}]_{i}$ is the $i$-th element of the mean vector $\bm{m}_{\zv_p} \in \Rb^M$, the approximated posteriors are $q(\mu)=\Nc(m_{\mu}, \sigma_{\mu}^2)$, $q(\zv_p)=\Nc(\bm{m}_{\zv_p}, \Sv_{\zv_p})$ and the priors are $p(\mu)=\Nc(\check{m}_{\mu} ,\check{\sigma}^2_{\mu})$, $p(\zv_p)=\Nc(\check{\bm{m}}_{\zv_p} ,\check{\Sv}_{\zv_p})$.

% \subsection{Performance Comparison}
% \label{sec:toy exp compare}
% We compare the perforamce of computing the ELBO in \cref{eq:elbo} by using closed form in \cref{eq:evidence final} and using Monte Carlo approximation in \cref{eq:evidence final mc approx} in a toy example.
% \textcolor{red}{Table or Figure to add if time available}


\subsection{Limitations of the Closed-Form ELBO}

The closed-form ELBO is only applicable to regression problems. In classification, applying the softmax function to $\tilde{f}(\xv;\bm{\theta}, \bm{\eta})$ results in a non-analytic predictive distribution, meaning the ELBO must still be computed via Monte Carlo sampling during training. Similarly, the closed-form expressions for the predictive mean and variance, as provided in \cref{eq:dak inference closed form} in \Cref{sec:uq of inference}, are not applicable to classification but only apply to regression problems.


\section{COMPUTATIONAL COMPLEXITY}
\label{sec:complexity}
In this section, we discuss the computational complexity of various DKL models compared to the proposed DAK method, focusing on the GP layer as the most computationally demanding component. \Cref{tab:complexity supp} shows the computational complexity of our model compared to other state-of-the-art GP and DKL methods.

\begin{table}[ht]
    \caption{Computational complexity of the DKL models for $N$ training points. The reported training complexity is for one iteration. $\hat{M}$ is the number of inducing points in SVGP and KISS-GP, while $M$ is the size of induced grids in DAK, $M < \hat{M}$. $S$ is the number of Monte Carlo samples, $B$ is the size of mini-batch, $D_w$ is the dimension of the NN outputs in DKL, $P$ is the dimension of the outputs after applying linear transformations to the NN outputs in the proposed DAK model. DAK-MC refers to the DAK model using Monte Carlo approximation, while DAK-CF refers to the DAK model using closed-form inference and ELBO.}
    \centering
    \begin{tabular}{lcc}
    \toprule[1pt]
                  & \textbf{Inference}       & \textbf{Training} (per iteration) \\
    \midrule[0.5pt]
    NN + SVGP     & $\Oc(\hat{M}^2 N)$    & $\Oc( S D_w MB + \hat{M}^3)$ \\
    NN + KISS-GP  & $\Oc(D_w \hat{M}^{1+\frac{1}{D_w}})$  & $\Oc(S D_w MB + D_w \hat{M}^{\frac{3}{D_w}})$ \\
    DAK-MC (ours) & $\Oc(SM)$       & $\Oc(SPMB + PM)$   \\
    DAK-CF (ours) & $\Oc(M)$        & $\Oc(PMB + PM)$    \\
    \bottomrule[1pt]
    \end{tabular}
    \label{tab:complexity supp}
\end{table}

\paragraph{Inference Complexity.}
In inference based on induced approximation, computing the multiplication of the inverse of the covariance matrix $k(\Uv, \Uv)$ and a vector takes $\Oc(\hat{M}^2N)$ time for $\hat{M}$ inducing points $\Uv$ and $N$ training points when using SVGP. This cost is reduced by KISS-GP to $\Oc(D \hat{M}^{1+\frac{1}{D}})$ by decomposing the covariance matrix into a Kronecker product of $D$ one-dimensional covariance matrices of the inducing points: $k(\Uv, \Uv) = \bigotimes_{d=1}^{D} k(\Uv^{[d]}, \Uv^{[d]})$. Despite the significant reduction on complexity, it requires inducing points $\Uv$ arranged on a Cartesian grid of size $\hat{M} = \prod_{d=1}^{D} \hat{M}_d$, where $\hat{M}_d$ is the number of inducing points in the $d$-th dimension. In high-dimensional spaces, fixing $\hat{M}$ leads to very small $\hat{M}_d$ per dimension, which can degrade model performance. To address this, we propose the DAK model via sparse finite-rank approximation, which employs an additive Laplace kernel for GPs. The inverse Cholesky factor $\Lv_{\Uv}^{\top}$ for one-dimensional induced grids $\Uv$ of size $M$, where $M < \hat{M}$, as defined in \cref{eq:GPlayer}, is sparse and can be computed in $\Oc(M)$ time.

\paragraph{Training Complexity.}
In training, VI requires computing the ELBO as described in \cref{eq:elbo}, which consists of two terms: the \emph{expected log likelihood} and the \emph{KL divergence} between the variational distributions and priors. 

1) The \emph{expected log likelihood} is usually approximated via Monte Carlo sampling at a cost of $\Oc(S N_{\Theta} N)$, where $S$ is the number of Monte Carlo samples, $N_{\Theta}$ is the total number of variational parameters $\Theta_{\text{var}}$, and $N$ is the number of training points. This complexity can be reduced to $\Oc(S N_{\Theta} B)$ by applying stochastic variational inference with a mini-batch of size $B \ll N$. For DKL models using SVGP and KISS-GP, $\Theta_{\text{var}}$ are inducing variables, and the expectation does not have a closed form, requiring Monte Carlo sampling. In contrast, in the proposed DAK model, $\Theta_{\text{var}}= \{ \{ \zv_{p}\}_{p=1}^{P}, \mu \}$ consists of independent Gaussian weights $\zv_p\in \Rb^M$ and bias $\mu$. This allows us to derive an analytical form for this term, as shown in \cref{eq:evidence final} in \Cref{sec:elbo}, reducing the computational cost to $\Oc(N_{\Theta} B) = \Oc(PM B)$ when using a mini-batch of size $B$.

2) The \emph{KL divergence} between two Gaussian distributions can be computed in closed form. This leads to a linear time complexity of $\Oc(N_{\Theta})$ if the parameters $\Theta_{\text{var}}$ are independent, or cubic time $\Oc(N_{\Theta}^3)$ if they are fully correlated. In SVGP and KISS-GP, $\Theta_{\text{var}}$ represents fully correlated Gaussian distributed inducing variables, so computing the KL divergence takes $\Oc(\hat{M}^3)$ for SVGP. In KISS-GP, this can be reduced to $\Oc(D \hat{M}^{\frac{3}{D}})$ using fast eigendecomposition of Kronecker matrices. In the DAK model, the weights $\{\zv_p\}_{p=1}^{P}$ as defined in \cref{eq:GPlayer} are independent Gaussian random variables, allowing the KL divergence to be computed in $\Oc(N_{\Theta}) = \Oc(PM)$ time, where $P$ is the number of base GP layers.


\section{ADDITIONAL DISCUSSIONS}

Although interpretability is one advantage of additive models, the main motivation for replacing a GP layer with an additive GP layer in our work is to handle high-dimensional data. When the input dimension is low, it is reasonable that GPs are superior to additive GPs since the additive kernel is an approximated and restrictive kernel. However, when the input dimension increases, the computational complexity grows considerably even in GPs with sparse approximation. For example, in DKL, the output dimension of NN encoder is usually chosen as small as 2, while in pixel data experiments, DKL cannot handle the computation associated with the dimensionality when the output dimension of ResNet is 512 or more. Although DKL is superior in low-dimensional and simple cases, we view additive structure as a necessary component to achieve scalability and good performance with high-dimensional data.

\subsection{Why choosing the induced grids instead of learning the inducing points?}

From an approximation accuracy point of view, there are two separate strategies to increase the accuracy. The first one is to learn the inducing point locations. The second one, however, is to simply increase the number of inducing points on a pre-specified finer grid. The second method is much easier to implement and has a theoretical guarantee by the GP regression theory: as the inducing points become dense in the input region, the approximation will become exact. In contrast, the first approach does not have such a favorable theoretical guarantee. 

The second approach would become difficult to use for many existing methodologies as in general the computational cost would scale as $\mathcal{O}(M^3)$ with $M$ inducing points, which is particularly problematic in high dimensions. 
% The first approach can be viewed as a compromise in those situations, and that is why many existing methods chose to learn the locations of the inducing points instead.
This difficulty is resolved by additive GPs, since approximating an additive GP boils down to approximating one dimensional GPs, which can be accomplished by using a set of pre-specified inducing points on a fine grid in 1-D. One major benefit of the proposed methodology is that the computation now scales at $\mathcal{O}(M)$, enabled by the Markov kernel and the additive kernel. Therefore, a large number of inducing points can be used in an efficient way. 

The proposed method also has several additional benefits: 1) It can decouple to some extent the neural network component and GP component by avoiding learning the inducing points, which may help reduce overfitting/overconfidence; 2) The equivalence to BNN holds exactly with the fixed inducing points, whereas for learned inducing points, this BNN equivalence breaks down, and the proposed computation/training framework would not be possible to carry through; 3) It can simplify the overall optimization since there is no need to learn the inducing points.

\subsection{Limitations and future directions}

Generally, a finer grid will lead to better approximations, but the number of parameters to be trained will also increase. Therefore, there is a trade-off between the accuracy and the computational cost that we can afford. This current work is using a specific Laplace kernel, which can utilize sparse Cholesky decomposition. More general kernels may result in more computational complexity but better representation power of the model. In addition, the current variational family is restricted under mean-field assumptions. A more general variational family, e.g. full/low-rank covariance, may lead to superior performance in some applications. 


\section{EXPERIMENTAL DETAILS}
\label{sec:expdetail}
In this section, we provide additional details regarding the experiments.

\subsection{Benchmarks for Regression}
\label{subsec:regression supp}
\paragraph{Experiment Setup}
For all models, the NN architecture is a fully connected NN with rectified linear unit (ReLU) activation function \citep{nair2010rectified} and two hidden layers containing 64 and 32 neurons, respectively, structured as $D \rightarrow 64 \rightarrow 32 \rightarrow D_w$, where $D$ is the input feature size (also the size of input $\Xv$) and $D_w$ is the output feature size. The models are evaluated with $D_w=16$, 64, and 256, respectively. The number of Monte Carlo samples is set to 8 during training and 20 during inference.

The NN is a deterministic model, and we use the negative Gaussian log-likelihood as the loss function to quantify the uncertainty of the NN outputs and compute the NLPD.

For NN+SVGP, the inducing points are set to the size of 64 in $D_w$ dimension. We implement the \texttt{ApproximateGP} model in GPyTorch \citep{gardner2018gpytorch}, defining the inducing variables as variational parameters, and use \texttt{VariationalELBO} in GPyTorch to perform variational inference and compute the loss.

SV-DKL is originally designed for classification, so for a fair comparison in regression tasks, we modify it by first applying a linear embedding layer $\Wv: \Rb^{D_w} \rightarrow \Rb^P$ with $P=16$ and normalizing the outputs to the interval $[0,1]$ for each base GP, similar to the DAK model. To adapt the additive GP layer for regression, we remove the softmax function from the model in eq. (1) of \citep{wilson2016stochastic}. Given training data $\{ \xv_i, \yv_i \}_{i=1}^{N}$, the model is modified as follows:
\begin{align}
    p(\yv_i \vert \fv_i, A) = \mathcal{A}(\fv_i)^{\top} \yv_i
\end{align}
where $\fv_i \in \Rb^P$ is a vector of independent GPs followed by a linear mixing layer $\mathcal{A}(\fv_i) = A \fv_i$, with $A \in \Rb^{C \times P}$ as the transformation matrix. Here, $C=1$ for single-task regression. For each $p$-th GP ($1 \leq p \leq P$) in the additive GP layer, the corresponding inducing variables $\uv_p$ are set to the size of 64 and treated as variational parameters for training. We use the \texttt{GridInterpolationVariationalStrategy} model with \texttt{LMCVariationalStrategy} in GPyTorch to perform KISS-GP with variational inducing variables, augmented by a linear mixing layer.

For AV-DKL, the inducing points are set to size of 64 in $D_{w}$ dimension. We implement the AV-DKL model based on the source code~\cite{matias2024amortized}.

Both DAK-MC and DAK-CF use the same additive GP layer size as SV-DKL, with $P=16$, and employ fixed induced grids $\Uv = \{1/8, 2/8, \ldots, 7/8\}$ of size 7 for each base GP, which is much smaller than that of SV-DKL.

\paragraph{Metrics}
Let $\{\xv_t, y_t\}_{t=1}^{T}$ represent a test dataset of size $T$, where $\mu_t$ and $\sigma_t^2$ are the predictive mean and variance. We evaluate model performance using two common metrics: Root Mean Squared Error (RMSE) and Negative Log Predictive Density (NLPD).

RMSE is widely used to assess the accuracy of predictions, measuring how far predictions deviate from the true target values. It is calculated as:
\begin{align}
    \text{RMSE} = \sqrt{ \frac{1}{T} \sum_{t=1}^{T}(y_t - \mu_t)^2 }.
\end{align}

NLPD is a standard probabilistic metric for evaluating the quality of a model's uncertainty quantification. It represents the negative log likelihood of the test data given the predictive distribution. For GPs, NLPD is calculated as:
\begin{align}
    \text{NLPD}
    &= - \sum_{t=1}^{T} \log p(y_t = \mu_t \vert \xv_t) \\
    &= \frac{1}{T}
    \sum_{t=1}^{T} \Big[
    \frac{(y_t - \mu_t)^2}{2\sigma_t^2} + \frac{1}{2} \log(2\pi \sigma_t^2)
    \Big].
\end{align}
Both RMSE and NLPD are widely used in the GP regression literature, where smaller values indicate better model performance.

\paragraph{Computing Infrastructure}
The experiments for regression were run on Macbook Pro M1 with 8 cores and 16GB RAM.

\subsection{Benchmarks for Classification}
\label{subsec:classification supp}
We use PyTorch \citep{paszke2019pytorch} baseline of NN models, GPyTorch \citep{gardner2018gpytorch} baseline of SVGP and SV-DKL models. In classification tasks, we apply a softmax likelihood to normalize the output digits to probability distributions. The NN is a deterministic model trained via negative log-likelihood loss, while DKL and DAK models are trained via ELBO loss. The setting of all training tasks are described in \Cref{tab:model classification} and \Cref{tab:optimizer classification}.

SVGP is originally designed for single-output regression. To make it fit for multi-output classification, we used \texttt{IndependentMultitaskVariationalStrategy} in GPyTorch to implement the multi-task \texttt{ApproximateGP} model, and use \texttt{VariationalELBO} with \texttt{SoftmaxLikelihood} in GPyTorch to perform variational inference and compute the loss. 

For SV-DKL, we employed the same \texttt{VariationalELBO} with \texttt{SoftmaxLikelihood} as the variational loss objective. \texttt{GridInterpolationVariationalStrategy} is applied within \texttt{IndependentMultitaskVariationalStrategy} to perform additive KISS-GP approximation. For each KISS-GP unit, we used $64$ variational inducing points initialized on a grid of size $[-1,1]$. 

For DAK, we implemented DAK-MC using Monte Carlo estimation given the intractable softmax likelihood. We employed fixed induced grids $\Uv=\{ -31/32, -30/32, \ldots, 30/32, 31/32 \}$ of size 63 for each base GP component.

\begin{table}[ht]
\caption{Model architectures for image classification on MNIST, CIFAR-10 and CIFAR-100.}
\centering
\resizebox{0.7\linewidth}{!}{
\begin{tabular}{l|l|ccc}
\toprule[1pt]
Model                   & Hyper-parameter          & MNIST       & CIFAR-10    & CIFAR-100   \\
\midrule[0.5pt]
\multirow{4}{*}{NN+SVGP}   & Feature extractor        & CNN         & ResNet-18   & ResNet-34   \\
                        & NN out features $D_w$         & 128         & 512         & 512         \\
                        & Embedding features $P$               & 16          & 64          & 128         \\
                        & \# inducing points $\hat{M}$      & 512         & 512         & 512         \\
                        & \# epochs       & 20         & 200         & 200         \\
                        & Training strategy      & Full-training         & Full-training         & Fine-tuning         \\
\midrule[0.5pt]
\multirow{5}{*}{SV-DKL} & Feature extractor        & CNN         & ResNet-18   & ResNet-34   \\
                        & NN out features $D_w$         & 128         & 512         & 512         \\
                        & Embedding features $P$               & 16          & 64          & 128         \\
                        & \# inducing points $\hat{M}$      & 64          & 64          & 64          \\
                        & Grid bounds              & {[}-1,1{]} & {[}-1,1{]} & {[}-1,1{]} \\
                        & \# epochs       & 20         & 200         & 200         \\
                        & Training strategy       & Full-training         & Full-training         & Fine-tuning         \\
\midrule[0.5pt]
\multirow{4}{*}{DAK}    & Feature extractor        & CNN         & ResNet-18   & ResNet-34   \\
                        & NN out features $D_w$         & 128         & 512         & 512         \\
                        & Embedding features $P$               & 16          & 64          & 128         \\
                        & \# induced interpolation $M$ & 63          & 63          & 63         \\
                        & \# epochs       & 20         & 200         & 200         \\
                        & Training strategy      & Full-training         & Full-training         & Full-training         \\
\bottomrule[1pt]
\end{tabular}

}
\label{tab:model classification}
\end{table}

\paragraph{MNIST} We used a CNN implemented in PyTorch as the feature extractor: \texttt{Conv2d}(1,32,3) $\rightarrow$ \texttt{Conv2d}(32,64,3) $\rightarrow$ \texttt{MaxPool2d}(2) $\rightarrow$ \texttt{Dropout}(0.25) $\rightarrow$ \texttt{Linear}(9216,128) $\rightarrow$ \texttt{Dropout}(0.5). To make a fair comparison, for both SV-DKL and DAK, we applied an embedding module through a linear layer that transform $128$ output features into $P=16$ base GP channels. 

\paragraph{CIFAR-10} We used a ResNet-18 as the feature extractor followed by a linear embedding layer that compressed the $512$ output features into $P=64$ base GP channels. 

\paragraph{CIFAR-100} We used a pretrained ResNet-34 as the feature extractor for SV-DKL and fine-tuned GP output layers since SV-DKL struggled to fit using full-training. For proposed DAK, we used full-training. The number of base GP channels is selected as $P=128$. 

\begin{table}[ht]
\caption{Details of training optimizer for image classification on MNIST, CIFAR-10 and CIFAR-100.}
\centering
\resizebox{0.7\linewidth}{!}{

\begin{tabular}{l|ccc}
\toprule[1pt]
Optimization      & MNIST                                                             & CIFAR-10                                                                                                  & CIFAR-100                                                                                                 \\
\midrule[0.5pt]
Optimizer         & Adadelta                                                          & SGD                                                                                                       & SGD                                                                                                       \\
Initial lr.       & 1.0                                                               & 0.1                                                                                                       & 0.1                                                                                                       \\
Weight decay      & 0.0001                                                            & 0.0001                                                                                                    & 0.0001                                                                                                    \\
Scheduler         & StepLR                                                            & CosineAnnealingLR                                                                                         & CosineAnnealingLR                                                                                         \\
\midrule[0.5pt]
Data Augmentation & MNIST                                                             & CIFAR-10                                                                                                  & CIFAR-100                                                                                                 \\
\midrule[0.5pt]
RandomCrop        & -                                                                 & size=32, padding=4                                                                                        & size=32, padding=4                                                                                        \\
HorizontalFlip    & -                                                                 & p=0.5                                                                                                     & p=0.5                                                                                                     \\
% Normalization     & \begin{tabular}[c]{@{}l@{}}mean=0.1307,\\ std=0.3081\end{tabular} & \begin{tabular}[c]{@{}l@{}}mean={[}0.4914,0.4822,0.4465{]},\\ std={[}0.2023,0.1994,0.2010{]}\end{tabular} & \begin{tabular}[c]{@{}l@{}}mean={[}0.5071,0.4867,0.4408{]},\\ std={[}0.2675,0.2565,0.2761{]}\end{tabular} \\
\bottomrule[1pt]
\end{tabular}
}
\label{tab:optimizer classification}
\end{table}

\paragraph{Additional Benchmark.}  \citet{matias2024amortized} proposed Amortized Variational DKL (AV-DKL), which is a variant SV-DKL using amortization network to compute the inducing locations and variational parameters, thus attenuating the overcorrelation of NN extracted features. AV-DKL is included as the additional benchmark for classification tasks in \Cref{tab:img avdkl}. The training recipe is the same with SV-DKL. 


\begin{table*}[ht]
\caption{\small{Accuracy, NLL, ECE for AV-DKL, SV-DKL, DAK-MC on CIFAR-10/100 averaged over 3 runs. CIFAR-10 uses ResNet-18 with 64 features extracted; CIFAR-100 uses ResNet-34 with 512 features. The best results are highlighted in \textbf{bold}; the second best results are highlighted by \underline{underline}.}}
\centering
\vspace{-0.1cm}
\resizebox{\linewidth}{!}{%
\begin{tabular}{rccclccc}
\toprule[1pt]
\multicolumn{1}{l}{} & \multicolumn{3}{c}{Batch size: 128}  &  & \multicolumn{3}{c}{Batch size: 1024} \\ \cline{2-4} \cline{6-8} \vspace{-8pt} \\
\multicolumn{1}{l}{} & AV-DKL & SV-DKL & \cellcolor{Gray} DAK-MC &   & AV-DKL  & SV-DKL & \cellcolor{Gray} DAK-MC \\ 
\midrule[1pt]
CIFAR-10 - Acc. (\%) $\uparrow$    & \underline{94.23 $\pm$ 0.65}  & 93.44 $\pm$ 0.28    &  \cellcolor{Gray} \textbf{94.81 $\pm$ 0.13}   &     &  \textbf{93.32} $\pm$ \textbf{0.13}        & 90.22 $\pm$ 1.42       & \cellcolor{Gray} \underline{93.02 $\pm$ 0.18}        \\
NLL $\downarrow$     & 0.352 $\pm$ 0.084    & \underline{0.312 $\pm$ 0.033}       &  \cellcolor{Gray} \textbf{0.256} $\pm$ \textbf{0.014}     &      & \underline{0.439 $\pm$ 0.022}         & 0.485 $\pm$ 0.061       & \cellcolor{Gray} \textbf{0.345 $\pm$ 0.001}    \\
ECE $\downarrow$      & 0.048 $\pm$ 0.006    & \underline{0.046 $\pm$ 0.003}       &  \cellcolor{Gray} \textbf{0.039 $\pm$ 0.002}          &     & \underline{0.054 $\pm$ 0.001}       & 0.060 $\pm$ 0.004       & \cellcolor{Gray} \textbf{0.052 $\pm$ 0.001}           \\
\midrule[1pt]
CIFAR-100 -  Acc. (\%) $\uparrow$    & \textbf{77.47 $\pm$ 0.19}  & 74.52 $\pm$ 0.13       & \cellcolor{Gray}  \underline{76.75 $\pm$ 0.18}     &     &  \textbf{77.07 $\pm$ 0.10}        & 66.54 $\pm$ 0.74       & \cellcolor{Gray} \underline{70.38 $\pm$ 1.25}        \\
NLL $\downarrow$     & 1.787 $\pm$ 0.011    & \underline{1.041 $\pm$ 0.007}       & \cellcolor{Gray}  \textbf{1.001 $\pm$ 0.027}     &      & 2.326 $\pm$ 0.030    & \underline{1.738 $\pm$  0.058}      & \cellcolor{Gray} \textbf{1.203 $\pm$ 0.040}        \\
ECE $\downarrow$      & 0.166 $\pm$ 0.002    & \underline{0.049 $\pm$ 0.002}       & \cellcolor{Gray}  \textbf{0.041 $\pm$ 0.004}        &     & 0.175 $\pm$ 0.001         & \underline{0.148 $\pm$ 0.007}       &\cellcolor{Gray}  \textbf{0.056 $\pm$ 0.006}           \\
\bottomrule[1pt]
\end{tabular}
}
\vspace{-0.2cm}
\label{tab:img avdkl}
\end{table*}

\paragraph{Metrics} 
We evaluate model performance using four common metrics: Top-1 accuracy, ELBO, Negative Log Likelihood (NLL), and Expected Calibration Error (ECE). 

ECE is a metric used to quantify the degree of ``calibration'' of a probabilistic model in UQ, specifically for classification problems. It is defined as the weighted average of the absolute difference between the model's predicted probability (confidence) and the actual outcome (accuracy) over several bins of predicted probability. Mathematically, ECE is given by:
\begin{align}
    \text{ECE} =\sum_{m=1}^{M} \frac{\left| B_{m} \right|}{n} \left| \text{acc} (B_{m})-\text{conf} (B_{m}) \right|,
\end{align}
where $M$ is the number of bins into which the confidence values are partitioned, $B_m$ is the set of indices of samples whose predicted confidence falls into the $m$-th bin, $n$ is the total number of samples.

\paragraph{Computing Infrastructure}
The experiments for classification were run on a Linux machine with NVIDIA RTX4080 GPU, and 32GB of RAM.




\subsection{Additional Tables and Figures}
\label{sec:additional exp results}

\paragraph{Choices of learning rates.}
We evaluate the choices of learning rates on 1D regression examples. DKL requires a separate tuning of the learning rate of the GP covariance parameters, which differs from the learning rate of the NN feature extractor. In \Cref{fig:dkl lr}, we choose the learning rate of the NN feature extractor as $0.01$, while the learning rate of the GP covariance is set to different values. (a)-(c) show that different learning rates of covariance in DKL result in different predictive posterior. In particular, although the training losses for DKL in both (a) and (b) are minimal, the regressions do not fit well. On the other hand, DAK does not need a distinct recipe for tuning GP covariances because of the BNN interpretation. Furthermore, the poor posterior is indicated by the higher training loss, as illustrated in (d)-(f).

\begin{figure}[ht]
\centering
\subfloat[$\begin{gathered}\text{DKL: last-layer lr} =0.01.\\ \text{Training loss:} -0.21.\end{gathered}$]{\includegraphics[width=.3\textwidth]{toy_dkl_lr_01.pdf}}
\subfloat[$\begin{gathered}\text{DKL: last-layer lr} =0.001.\\ \text{Training loss: } -0.07.\end{gathered}$]{\includegraphics[width=.3\textwidth]{toy_dkl_lr_001.pdf}}
\subfloat[$\begin{gathered}\text{DKL: last-layer lr} =0.0001.\\ \text{Training loss: } 0.22.\end{gathered}$]{\includegraphics[width=.3\textwidth]{toy_dkl_lr_0001.pdf}}

\subfloat[$\begin{gathered}\text{DAK: last-layer lr} =0.1.\\ \text{Training loss: } 0.10.\end{gathered}$]{\includegraphics[width=.3\textwidth]{toy_dak_lr_1.pdf}}
\subfloat[$\begin{gathered}\text{DAK: last-layer lr} =0.01.\\ \text{Training loss: } 0.10.\end{gathered}$]{\includegraphics[width=.3\textwidth]{toy_dak_lr_01.pdf}}
\subfloat[$\begin{gathered}\text{DAK: last-layer lr} =0.001.\\ \text{Training loss: } 0.22.\end{gathered}$]{\includegraphics[width=.3\textwidth]{toy_dak_lr_001.pdf}}

\caption{Results on 1D regression with different last-layer learning rates. The learning rate of NN feature extractor is set as $0.01$. (a)--(f) shows the regression fits and corresponding training losses. DAK fits for the same learning rate strategy with NN feature extractor (lr=0.01), while DKL requires a separate tuning for last-layer learning rate of GPs. Additionally, a better training loss does not necessarily prevent overfitting for DKL.}
\label{fig:dkl lr}
\end{figure}


\paragraph{Learning curves.} We plot the learning curves of CIFAR-10/100 in \Cref{fig:cifar10 curves} and \ref{fig:cifar100 curves}. The learning curves of SVDKL in \Cref{fig:cifar10 curves} is more unstable, with many significant spikes, and the convergence is slower than DAK. Futhermore, SVDKL struggles to fit with full-training in CIFAR-100, and a pretrained feature extractor is used in CIFAR-100. Therefore, the learning curves of SVDKL look smoothing, but DAK fits well with full-training in CIFAR-100.


\begin{figure}[ht]
\centering
\subfloat[Test Error (\%).]{\includegraphics[width=.3\textwidth]{CIFAR_10_test_error.pdf}}
\subfloat[Test NLL.]{\includegraphics[width=.3\textwidth]{CIFAR_10_nll.pdf}}
\subfloat[ELBO.]{\includegraphics[width=.3\textwidth]{CIFAR_10_elbo.pdf}}
\caption{Test errors, test NLLs, ELBOs of NN, SVDKL, and DAK curves with batch size of 128/1024 for CIFAR-10 averaged on 3 runs. DAK outperforms SVDKL on both test error and NLL along the training epochs. Additionally, SVDKL degrades more and struggles to fit when the batch size becomes larger.}
\label{fig:cifar10 curves}
\end{figure}

\begin{figure}[ht]
\centering
\subfloat[Test Error (\%).]{\includegraphics[width=.3\textwidth]{CIFAR_100_test_error.pdf}}
\subfloat[Test NLL.]{\includegraphics[width=.3\textwidth]{CIFAR_100_nll.pdf}}
\subfloat[ELBO.]{\includegraphics[width=.3\textwidth]{CIFAR_100_elbo.pdf}}
\caption{Test errors, test NLLs, ELBOs of NN, SVDKL, and DAK curves with batch size of 128/1024 for CIFAR-100 averaged on 3 runs. DAK trained NN and last-layer additive GPs jointly, while SVDKL used the pre-trained NN and fine-tuned the last-layer GP since SVDKL struggles to fit using full-training. DAK outperforms SVDKL on both test error and NLL along the training epochs. SVDKL struggled to fit in high-dimensional multitask cases, indicating the necessity of pre-training in SVDKL. However, DAK fitted well with high dimensionality and large batch sizes.}
\label{fig:cifar100 curves}
\end{figure}







% \end{document}


\end{document}
