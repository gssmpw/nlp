%%%%%%%%%%%%%%%%%%%%%%%%%%%%%%%%%%%%%%%%%%%%%%%%%%%%%%%%%%%%
\section{Introduction}
%%%%%%%%%%%%%%%%%%%%%%%%%%%%%%%%%%%%%%%%%%%%%%%%%%%%%%%%%%%%
\makeatletter
\newcommand*{\rom}[1]{\expandafter\@slowromancap\romannumeral #1@}
\makeatother


\begin{figure*}
    \centering
    % First row of images
    \begin{subfigure}[b]{0.24\textwidth}
        \includegraphics[width=\textwidth]{figures/gameplay/route_1.png}
        \caption{Overworld with Long Grass outside of Pallet Town}
        \label{fig:route1}
    \end{subfigure}%
    \hfill
    \begin{subfigure}[b]{0.24\textwidth}
        \includegraphics[width=\textwidth]{figures/gameplay/route_3.png}
        \caption{Mandatory Trainer Battle Triggered on Line-of-sight}
        \label{fig:route3}
    \end{subfigure}%
    \hfill
    \begin{subfigure}[b]{0.24\textwidth}
        \includegraphics[width=\textwidth]{figures/gameplay/poke_center_heal.png}
        \caption{Pokémon Center Healing by Interacting with Nurse Joy}
        \label{fig:poke_center_heal}
    \end{subfigure}%
    \hfill
    \begin{subfigure}[b]{0.24\textwidth}
        \includegraphics[width=\textwidth]{figures/gameplay/battle_geodude_2.png}
        \caption{Pokémon Battle: Selecting a Super Effective Attack}
        \label{fig:battle_geodude_2}
    \end{subfigure}%
    %     \hfill
    % \begin{subfigure}[b]{0.19\textwidth}
    %     \includegraphics[width=\textwidth]{figures/gameplay/battle_geodude_2.png}
    %     \caption{Battle Geodude 2}
    %     \label{fig:battle_geodude_2}
    % \end{subfigure}
    
    \vspace{1em} % Vertical space between rows

    % Second row of images
    \begin{subfigure}[b]{0.24\textwidth}
        \includegraphics[width=\textwidth]{figures/gameplay/mt_moon.png}
        \caption{Mt. Moon: Maze-like Dungeon Full of Wild Pokémon}
        \label{fig:mt_moon}
    \end{subfigure}%
    \hfill
    \begin{subfigure}[b]{0.24\textwidth}
        \includegraphics[width=\textwidth]{figures/gameplay/vermilion_cut.png}
        \caption{Cuttable Tree as Obstacle and Game Mechanic}
        \label{fig:vermilion_cut}
    \end{subfigure}%
    \hfill    
    \begin{subfigure}[b]{0.24\textwidth}
        \includegraphics[width=\textwidth]{figures/gameplay/item_menu.png}
        \caption{Item Menu outside of Battle\\~}
        \label{fig:item_menu}
    \end{subfigure}%
    \hfill
    \begin{subfigure}[b]{0.24\textwidth}
        \includegraphics[width=\textwidth]{figures/gameplay/party_menu_2.png}
        \caption{Party Menu Displays Party Pokémon Order, HP, and Level}
        \label{fig:party_menu_2}
    \end{subfigure}%    
    % \hfill
    % \begin{subfigure}[b]{0.19\textwidth}
    %     \includegraphics[width=\textwidth]{figures/gameplay/battle_geodude_2.png}
    %     \caption{Battle Geodude 2}
    %     \label{fig:battle_geodude_2}
    % \end{subfigure}
    \caption{All subfigures depict game screens possibly perceivable by the agent, presenting various challenges in Pokémon Red that involve exploration, navigation, and strategic decision-making. The agent must traverse a 2D overworld (\subref{fig:route1}) with a party of Pokémon (\subref{fig:party_menu_2}), winning mandatory trainer battles (\subref{fig:route3}) and navigating complex maze-like areas (\subref{fig:mt_moon}). Progression also depends on interacting with the game's UI, including healing Pokémon at a Pokémon Center (\subref{fig:poke_center_heal}), using strategy to win battles (\subref{fig:battle_geodude_2}), using items (\subref{fig:item_menu}). Overcoming obstacles, like cuttable trees (\subref{fig:vermilion_cut}), requires the use of mandatory game-mechanic moves, which must be obtained via exploration, taught to eligible Pokémon, and used via the Start menu interface when facing the obstacle.}
    % \vspace{-1.4em}

    \label{fig:gameplay}
\end{figure*}

% From classic (Atari) to modern (StarCraft...) back to classic (Pokémon Red)
Video games have significantly advanced artificial intelligence, acting as both a catalyst for research and a benchmark for solving complex problems.
DRL and its adjacent fields have achieved remarkable results by enabling the training of agents to play games with extremely long horizons such as DotA~2 \cite{Berner2019Dota2}, NetHack \cite{kuettler2020nethack}, and Minecraft \cite{minedojo2022}. While these modern games were used to demonstrate strong advancements, older games remain an untapped resource, offering similar levels of complexity, unique challenges, and opportunities for research.
% What is Pokémon Red?
This work examines an older game, the 1996 Game Boy JRPG: Pokémon Red.

%To beat Pokemon, the player must catch, train, and carefully manage their Pokémon party navigate a 2D grid-based world, interact with user interfaces, and solve diverse puzzles.
% What are the unique traits of Pokémon Red as a research subject
Pokémon games have consistently gained widespread attention, both within and beyond the gaming community.
A 2023 YouTube video showcasing an early DRL agent playing Pokémon Red went viral, garnering over 7.5 million views \cite{pdubsYoutube}.
The video's viral success highlights the game's broad appeal and serves as the inspiration that we build upon. 
Pokémon and its many successors challenge the player with a variety of tasks with the ultimate goal of becoming a Pokémon Master by defeating the game's Champion and catching all 151 Pokémon species.
In this paper, we highlight Pokémon Red as a game that exposes players to multiple complex tasks:
\begin{itemize}
\item Multi-Task Challenge: Pokémon includes strategic battling, 2D navigation, UI navigation, resource management, and puzzle-solving (Figure \ref{fig:gameplay}).
\item Exploration Challenge: 
% Human players require tens of hours to beat Pokémon. 
Beating Pokémon requires careful planning and navigation. In our experiments, the trained DRL agents made tens of thousands of decisions per episode to complete only a small fraction of the game.
% \item Stochastic Elements: Many aspect of the game are randomized. Wild Pokémon encounters, battles, and Pokémon stats develop differently in every game making it resistant to memorization.
\item Large Policy Space: With 151 Pokémon species, diverse combinations of moves, and numerous usable items, Pokémon Red features vast gameplay possibilities and freedom to develop successful policies. 
\end{itemize}

% What are our key contributions? % What are the major insights?
To explore Pokémon Red’s potential, we contribute a minimal DRL environment and a simple, adaptable baseline agent trained with Proximal Policy Optimization (PPO) \cite{Schulman2017}.
Our approach lays the groundwork for more advanced exploration and hierarchical methods, while offering detailed insights into the performance and behavior of the trained baseline agent and various reward ablations.
% Structure of the paper
This paper is structured as follows: we establish the Pokémon baseline by introducing the gameplay and properties of the Pokémon Red environment; then, the training setup is detailed.
Section \ref{sec:experiments} presents results, discussing the key findings and potential future directions.
Before concluding, we refer to related work.
