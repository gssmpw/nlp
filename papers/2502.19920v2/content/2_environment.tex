%%%%%%%%%%%%%%%%%%%%%%%%%%%%%%%%%%%%%%%%%%%%%%%%%%%%%%%%%%%%
\section{\textbf{Pokémon Red Environment}}
%%%%%%%%%%%%%%%%%%%%%%%%%%%%%%%%%%%%%%%%%%%%%%%%%%%%%%%%%%%%

This section describes the core mechanics of Pokémon Red and our formulation as a Markov Decision Process (MDP) \cite{sutton2018reinforcement}, including the design of the observation space, action space, reward function, and terminal conditions.
The simulation speed of the environment was measured using random actions on an AMD Ryzen 7 2700X CPU. Pokémon runs at $\approx$9403 steps per second (SPS), which is faster than Atari Breakout (7117 SPS) and slower than Procgen (18530 SPS) \cite{pleines2023memory}.

%%%%%%%%%%%%%%%%%%%%%%%%%%%%%%%%%%%%%%%%%%%%%%%%%%%%%%%%%%%%
\subsection{Selected Game Objectives}
%%%%%%%%%%%%%%%%%%%%%%%%%%%%%%%%%%%%%%%%%%%%%%%%%%%%%%%%%%%%

In this work, we will focus on Pokémon Red's storyline objectives, which entail completing several quests and milestones.
For simplicity, we have reduced the objective from completing the entire game to completing only the main tasks up to the end of Cerulean City, the location of the second gym.
This constitutes approximately 20\% of the full game and is achieved by accomplishing the following milestones:

\begin{itemize}
\item Start in Pallet Town and pick a starter Pokémon from Professor Oak.
\item Travel to Viridian City and obtain Oak's Parcel.
\item Return the parcel to Oak's lab in Pallet Town.
\item Traverse Viridian Forest to reach Pewter City.
\item Defeat Pewter City's gym leader, Brock (Badge 1).
\item Traverse Mt. Moon to arrive at Cerulean City.
\item Defeat Cerulean City's gym leader, Misty (Badge 2).
% \item Defeat Rival in a mandatory trainer battle.
\item Transform Bill, a storyline non-playable character (NPC), back into a human to unlock the path to Vermilion City.
\end{itemize}

Because the first two objectives require backtracking, we have started the environment after the third objective, as the remaining tasks involve minimal or no backtracking.
Thus, the agent does not choose its starter.
Instead, we let it start with the water-type Pokémon Squirtle, which has an advantage against gym leader Brock's rock Pokémon.
After traversing Mt. Moon, the subsequent storyline objectives can be completed in a variety of orders, and the game transitions into a large, open-world setting with numerous non-linear tasks and puzzles.

%%%%%%%%%%%%%%%%%%%%%%%%%%%%%%%%%%%%%%%%%%%%%%%%%%%%%%%%%%%%
\subsection{Game Mechanics}
%%%%%%%%%%%%%%%%%%%%%%%%%%%%%%%%%%%%%%%%%%%%%%%%%%%%%%%%%%%%

Pokémon Red features turn-based combat, strategic team-building, and exploration of an expansive grid world.
The player navigates interconnected maps, overcoming obstacles such as trainer battles, environmental barriers, and puzzles.
Early on, battles are the primary challenge.

Battles initiate with the first Pokémon in each player's party (a wild Pokémon is treated as a party of one).
Battles are turn-based, with each side selecting moves that are executed in order based on the Pokémon's Speed stat, so that faster Pokémon act before slower ones.
On each turn, a player may choose a move (Fig.~\ref{fig:battle_geodude_2}), switch to another Pokémon in the party (Fig.~\ref{fig:party_menu_2}), use an item, or run from battle.
Escaping from battle is only possible during wild encounters.
Wild Pokémon appear randomly in grass, on water, or inside dungeons, and players can capture them using Poké Balls.
Each Pokémon can have up to four moves, each with a finite number of Power Points (PP) representing the number of times a move can be used in battle.
Using a move consumes one PP and may heal, deal damage, inflict status effects, or temporarily alter stats.
The effects of a move can vary based on the attacking and target Pokémon's types.
This type match-up system is analogous to rock-paper-scissors, where each type holds a natural advantage over another.
For example, water-type moves deal double damage against fire-type Pokémon.
Attacks can also randomly become Critical Hits that deal double damage.
When a Pokémon's Hit Points (HP) reach zero, it faints and can no longer be used in battle.
A Pokémon's HP and PP can be restored using items (Fig.~\ref{fig:item_menu}) or by visiting a Pokémon Center (Fig.~\ref{fig:poke_center_heal}).
Defeating a wild or trainer Pokémon grants experience points (XP) to all conscious Pokémon that participated in battle.
To gain XP, a Pokémon must remain conscious until the enemy Pokémon faints.
Gaining XP leads to level-ups that increase a Pokémon's stats and offer opportunities to learn new moves or evolve.
Evolution is triggered at certain levels and can be canceled with the B button; it raises a Pokémon's stats but may delay learning new moves, offering a strategic tradeoff.
Many Pokémon have evolved forms, and some moves are learned at lower levels in their base forms than after evolving.
If the player's entire party faints, it is considered a "blackout," and the player will respawn at the last visited Pokémon Center or at their home in Pallet Town if no Pokémon Centers have been visited.

% Pokémon Red features turn-based combat, strategic team-building, and exploration of an expansive grid world with diverse environments and NPCs.
% The player navigates interconnected maps, encountering obstacles such as trainer battles, environmental barriers, or puzzles that must be overcome to progress.
% During the first fifth of the story, battles serve as the primary challenge.

% The battle initiates with the first Pokémon in each player's party (a wild Pokémon is considered its own party of size 1) opposing each other.
% Battles are turn-based, with each side selecting moves that are executed in order based on the Pokémon's Speed stat; faster Pokémon act before slower ones.
% Each turn in battle, a player has the option to either choose a move (Fig.~\ref{fig:battle_geodude_2}), switch to another Pokémon in the player's party (Fig.~\ref{fig:party_menu_2}), use an item, or ``run" away from battle.
% Escaping battles is only possible against wild encounters.
% The player can encounter ``wild" Pokémon randomly in grass, on water, or inside dungeons.
% Players can capture wild Pokémon using Poké Balls.

% Defeating a wild or trainer Pokémon grants experience points (XP) to any Pokémon that was on the battlefield during the encounter.
% To gain XP from a battle, a Pokémon must not have fainted before the enemy Pokémon faints.
% XP leads to level-ups; level-ups increase stats and trigger the option to learn new moves or to evolve; evolving increases stats, and can be canceled with the B button.
% Each Pokémon has a base form, and many have one or more evolved forms. A base form of a Pokémon may learn certain moves at lower levels as compared to when those moves would be learned if evolved.

% Each Pokémon can have up to four moves and each move has a finite number of Power Points (PP).
% PP is the number of times that move can be used in battle.
% Using a move consumes one PP and can heal, deal damage, inflict status effects, or temporarily alter stats.
% The effects of a move can vary based on random number generation (RNG) and also depend on the Pokémon's type and the target Pokémon's type.
% For example, water-type moves deal double damage against ground-type Pokémon.
% When a Pokémon's Hit Points (HP) turns zero, the Pokémon ``faints" and can no longer be used by the player in battle.
% A Pokémon's HP and PP can be restored by items (Fig~\ref{fig:item_menu}) or by visiting a Pokémon Center (Fig.~\ref{fig:poke_center_heal}).

% If the player's entire party faints, the player is said to have ``blacked out," and will respawn at the last visited Pokémon Center, or at their home in Pallet Town if no Pokémon Centers were visited.

% The player can encounter ``wild" (uncaught) Pokémon randomly in grass, on water, or inside dungeons.
% In a wild Pokémon encounter, players can capture wild Pokémon using Poké Balls, run away, or defeat the wild Pokémon.
% The player can also encounter trainers (NPCs), that will battle the player with their own trained Pokémon.
% A trainer battle will not end until all Pokémon on one side have fainted.

%When damage is dealt to a Pokémon by an attacking move, that Pokémon's Hit Points (HP) are decreased by the amount of the damage. Status effects, such as Poison, and Burn, can also decrease a Pokémon's HP. Pokémon Centers provide a mechanism to fully restores the HP all party Pokémon, removes any status effects, and regenerates all moves' PP.

% Status effects - **talk about them here as they are important**

% Damage calculation for a move considers the attacking Pokémon's Attack power stat, for Physical-type moves, or its Special power stat, for Special-type moves, the move's base power, the Defense or the Special power stats of the defending Pokémon (depending again on the type of move), and multipliers for: using a move that matches the attacking Pokémon's type (Same Type Attack Bonus: STAB), attacking move type versus Pokémon type of the defending Pokémon, Critical Hits, and a random multiplier that scales the final damage to between approximately 0.8x and 1x. \cite{liikala2010advanced}

% Type effectiveness of an attack, in particular, plays a pivotal role in battle outcomes, with damage multipliers of either 0, 0.5, or 2 (Figure \ref{fig:Fig. 2: type_chart}). This chart is not provided to players: instead, a dialog appears stating that a move was "super effective," "not very effective," or had "no effect" when a type effectiveness multiplier is applied. Thus, for example, a player learns that Grass-type attacks are super effective against Water-type Pokémon, and that Water-type attacks are super effective against Fire, Rock, and Ground types of Pokémon, and, conversely, that Water-type attacks are not very effective against Water- and Grass-type Pokémon. It is also important to note that a move may miss its target altogether; a move's accuracy, which represents the chance the move will hit its target, is calculated using the move's accuracy value, the user's accuracy modifier, and the target's evade modifier \cite{gamefaqsEvadeAccuracy}.

%Most trainer battles initiate when the player first enters that trainer's line-of-sight, within a set distance specific to that trainer, or when the player approaches the trainer otherwise and presses the A button to interact. 

%%%%%%%%%%%%%%%%%%%%%%%%%%%%%%%%%%%%%%%%%%%%%%%%%%%%%%%%%%%%
\subsection{Observation Space}
%%%%%%%%%%%%%%%%%%%%%%%%%%%%%%%%%%%%%%%%%%%%%%%%%%%%%%%%%%%%

The agent receives two observation modalities: a vision input and a game state vector.

The vision input uses a grayscale representation of the Game Boy's screen downsampled by two for a resolution of 72×80 pixels.
We stack the current frame and two previously observed frames to provide minimal temporal awareness.
The agent additionally observes a separate 48×48 binary crop of the screen, centered on the player, that indicates visited coordinates.

The game state vector includes the current HP and level of each Pokémon in the party, as well as flags indicating the completion status of various in-game events.
Notably, this observation space is intentionally limited for simplicity and to mimick the information a first time human player would have.
The observation does not provide information on the species, stats, nor movesets of the Pokémon.

%%%%%%%%%%%%%%%%%%%%%%%%%%%%%%%%%%%%%%%%%%%%%%%%%%%%%%%%%%%%
\subsection{Action Space}
%%%%%%%%%%%%%%%%%%%%%%%%%%%%%%%%%%%%%%%%%%%%%%%%%%%%%%%%%%%%

The action space is discrete and consists of seven Game Boy buttons: A, B, Start, Up, Down, Left, and Right (Select is omitted).
An agent must select one of these actions for each environment step.
Navigation of both the overworld and in-game menus is performed using the arrow keys.
The A button confirms selections or initiates an interaction, while the B button cancels actions or exits menus.
The Start button opens the main menu.

For simplicity, actions are represented as full presses instead of having a press/unpress action for each button.
To reliably move one tile at a time in the grid world, each button is held for 8 frames, then released for 16 frames, resulting in one decision every 24 frames.
As a result, the simulation runs at a slower speed of approximately $\frac{9403}{24} \approx 392,\text{SPS}$.

%%%%%%%%%%%%%%%%%%%%%%%%%%%%%%%%%%%%%%%%%%%%%%%%%%%%%%%%%%%%
\subsection{Determinism and Terminal Conditions}
%%%%%%%%%%%%%%%%%%%%%%%%%%%%%%%%%%%%%%%%%%%%%%%%%%%%%%%%%%%%

Wild encounters, Pokémon stats, critical hits, and other mechanics appear stochastic.  
However, the random number generator (RNG) on the Game Boy is influenced by the player's input.  
As a result, if the player behaves deterministically, the environment will also behave deterministically. 
This property is commonly exploited by speedrunners and could potentially be leveraged by trained agents as well \cite{pdubsYoutube}.  
The agent could execute a specific sequence of inputs to encounter a strong Pokémon and catch it on the first try.
To counteract this determinism, a sequence of random button presses could be executed upon resetting the environment.

Regarding terminal conditions, time is the only constraint we define.
The agent begins with a budget of 10,240 steps, which increases by 2,048 steps for each completed event.  
Events include important trainer battles, distinct NPC interactions, and storyline progression.  
This dynamic step budget introduces variety in training samples, preventing environments from resetting at fixed intervals and mitigating the risk of catastrophic forgetting (Section \ref{sec:horizons}).  


%%%%%%%%%%%%%%%%%%%%%%%%%%%%%%%%%%%%%%%%%%%%%%%%%%%%%%%%%%%%
\subsection{Reward Function}
%%%%%%%%%%%%%%%%%%%%%%%%%%%%%%%%%%%%%%%%%%%%%%%%%%%%%%%%%%%%

Pokémon has an inherently long time horizon, requiring hours of real-world play time before major objectives are achieved. We use dense auxiliary rewards to reinforce intermediate progress towards and behaviors that support reaching these goals.

\textbf{Event Reward}: The agent receives a reward of $ R_{\text{event}} $ = +2 for every completed event (trainer battle, step in quest progression).

\textbf{Navigation Reward}: Without a navigation reward, the agent converges to a policy that does not progress much further than the player's origin in Pallet Town. To aid in exploring the overworld, the agent receives $ R_{\text{nav}} $= +0.005 for each new overworld coordinate visited within an episode.

\textbf{Healing Reward}: When HP is gained, the agent receives a reward proportional to the fractional sum of party HP:  
\begin{equation}
    R_{\text{heal}} = 2.5\sum_{i=1}^{6} \frac{\text{HP}_i^{\text{after}} - \text{HP}_i^{\text{before}}}{\text{HP}_i^{\text{max}}}
\end{equation}  
This reward is triggered when a Pokémon levels up, when a Pokémon Center is used, or when a potion is used on a Pokémon with less than full HP; each of these behaviors supports progression.

\textbf{Level Reward}: The agent receives a reward based on the levels the Pokémon in its party. This encourages early level progression and the capture of new Pokémon while discouraging wild-encounter grinding.
To balance this behavior, we define:
\begin{equation}
    R_{\text{lvl}} = 
    0.5\min\left(\sum_{i=1}^{6} \text{lvl}_i, \frac{\sum_{i=1}^{6} \text{lvl}_i - 22}{4} + 22 \right)
\end{equation}
The threshold of 22 signifies a reasonable level to battle the leader of Gym 2, Misty, in Cerulean City, with the marginal gain per additional level downscaled by a factor of 4.

Finally, all rewards are summed up to yield \( R = R_{\text{event}} + R_{\text{nav}} + R_{\text{heal}} + R_{\text{lvl}} \). Rewards are further discussed in Section \ref{sec:reward_shaping}.
