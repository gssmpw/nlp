\begin{table}[t]
\centering
\caption{\textbf{Comparison using CAT-Seg \cite{cho2024cat}, using ground truth classes as text embeddings at different stages}, where $T$ represents training and $I$ represents inference. The mIoU is reported on ADE-20K (A)\cite{zhou2019semantic}, Pascal Context (PC)\cite{mottaghi2014role}, and Pascal VOC (VOC) \cite{everingham2010pascal}.
}
\label{tab:gt_labels}
\resizebox{\columnwidth}{!}{%
\begin{tabular}{ccccccccc}
\toprule
\textbf{Method} & \multicolumn{2}{c}{\textbf{\begin{tabular}[c]{@{}c@{}}Only GT\\ Text Labels\end{tabular}}} & COCO & A-847 & PC-459 & A-150 & PC-59 & VOC-20 \\ \cmidrule{2-3}
 & T & I &  &  &  &  &  &  \\ \midrule
Base &  &  & 47.11 & 11.95 & 18.95 & 31.78 & 57.20 & 95.30 \\
L.Bound & \checkmark &  & 43.73 & 10.89 & 16.63 & 30.29 & 55.99 & 94.20 \\
U.Bound (I) &  & \checkmark & 56.15 & 12.38 & 18.38 & 45.53 & 69.77 & \textbf{95.87} \\
U.Bound & \checkmark & \checkmark & \textbf{64.03} & \textbf{13.98} & \textbf{24.04} & \textbf{51.21} & \textbf{72.79} & 94.38 \\
\bottomrule
\end{tabular}
}
\end{table}

\begin{table}[t]
\centering
\caption{All methods are based on \cite{cho2024cat}, changing textual descriptors, while performing inference on GT classes. (a)-(c) are trained using the predicted VLM information on COCO dataset.
}
\label{tab:attvsadj}
\resizebox{\columnwidth}{!}{%
\begin{tabular}{ccccccccc}
\toprule
\textbf{Method} & \multicolumn{2}{c}{\textbf{VLM input}} & COCO & A-847 & PC-459 & A-150 & PC-59 \\ \cmidrule{2-3}
& $Image$& $Text$ & & & \\ \midrule
Baseline \cite{cho2024cat} & & & 56.15 & 12.38 & 18.38 %
& 45.53 & 69.77 &\\ %
(a) Caption & \checkmark & & 58.17 & 12.71 & 17.07 %
& 47.09 & 71.04 &\\ %
(b) Class Adjectives & & \checkmark & 62.33 & 14.96 & 19.13 & 48.77 & 60.47 \\ %
(c) Instance Adjectives & \checkmark & \checkmark & \textbf{65.13} & \textbf{15.40} & \textbf{23.20} & \textbf{54.43} & \textbf{72.04} \\ %
\bottomrule
\end{tabular}
}
\end{table}

\begin{table*}[t]
\centering
\caption{Prompts for different algorithms for \cref{tab:attvsadj} results.}
\label{tab:prompt}
\resizebox{.9\textwidth}{!}{%
\begin{tabular}{cc}
\toprule
\textbf{\begin{tabular}[c]{@{}c@{}}Description\\ Level\end{tabular}} & \textbf{Prompts} \\ \midrule
Class & \begin{tabular}[c]{@{}c@{}}1. "Please group the classes in this list $<$dataset-class-list$>$ into groups of classes that are similar to each other \\ meaning they could be confused in an image. Every class should be in one group and only in one group. \\ Make sure there are no classes from the original list missing in your grouping. \\ This is an example of how the output should look: dog, cat, kitten, bird -- couch, desk, sofa, lamp -- knife, fork, plate"\\ 2. " The classes in the group are: $<$group$>$. Please generate a short list of adjectives for each class \\ that describe how the object looks in an image. The adjectives should be distinctive within each group meaning that \\ the same attribute should not appear for two classes in the same group. Generate at least one adjective for each class. \\ This is an example how the output should look. {giraffe: [tall, brown, spotted, yellow], tree: [tall, green], armchair: [comfortable]}\end{tabular}\\ \midrule
Instance & \begin{tabular}[c]{@{}c@{}}"The objects in the image are: $<$dataset-class-list$>$. Please generate a short list of adjectives\\ for each object that describes how the object looks in the image. \\ This is an example of how the output should look. \{giraffe: {[}tall, brown, spotted, interacting{]}, tree: {[}tall, green, leafy{]}\}"\end{tabular} \\ \bottomrule
\end{tabular}
}
\end{table*}
