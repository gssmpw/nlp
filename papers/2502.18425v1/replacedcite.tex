\section{Related Work}
%In this section, commercial as well as open-source educational tools for automated feedback and grading are covered. 
%Furthermore, we report on some of the latest developments in open-source large language models that can be hosted locally. % diese Einleitung und insb der letzte Satz gefallen mir nicht... Vllt einfach genz weglassen
\iffalse
Current solutions, ... => put into related work
- privacy issues: models on foreign cloud servers such as chatGPT by openai
- often rely on proprietary, closed source models
- tutors are put out of "the loop" / student solutions are stored in different files (unnecessary overhead for swapping between solutions) / feedback not immediate
- limited support to directly grade code / combine grading with unit tests
\fi

\subsubsection{Commercial Tools}
of numerous educational start-ups nowadays incorporate AI-based features in their products. For example, Khanmigo by Khan academy ____ offers students interactive learning experiences by chatting with historical figures or by giving hints on questions without giving away the answer. Personify ____ allows teachers to create tailored teaching assistants for their courses. myTAI ____ helps teachers create teaching materials, and tools like gotFeedback ____, Fellofish ____ or Fobizz ____ help with automatic grading and feedback for writing exercises. Cocalc ____ and Vocareum ____ integrate generative AI into collaborative notebooks to help students. 
% problems with privacy / costly
However, commercial tools usually outsource the AI to external providers such as OpenAI potentially leading to conflicts with privacy policies of universities. 
Furthermore, licenses can be quite expensive for lower-budget universities and students. %( wrsl schwaches argument... A100 ist auch teuer)
Finally, it is important to keep tutors in charge of the final grades since the quality of AI-only feedback without human oversight is often still insufficient ____.

%____
\iffalse
- AI in education: e.g. Khanmigo by Khan academy etc
    - most edu tools are based on ChatGPT
    - A few words about commercial viability of these tools?
    - And about why commercialization means less use for the education sector
- https://personifyai.app/
- CCC zu KI in der Lehre / automatische Korrektur: https://www.youtube.com/watch?v=o6DBGdnA1P4\&ab\_channel=media.ccc.de
    - fobizz => basiert auf Chat-GPT4 (zitiere evtl auch: Chatbots im Schulunterricht: Wir testen das Fobizz-Tool zur automatischen Bewertung von Hausaufgaben https://arxiv.org/abs/2412.06651 )
    - Fiete => Fellofish
    - myTAI
- Development and Evaluation of an AI-Enhanced Python Programming Education System (https://hsnarman.github.io/CONF/24-UEMCON-AIPython.pdf) => wen zitieren die noch so? (basieren auf OpenAI / Google)
- we want privacy => open source on premise LLM
    - local / open source LLMs:
- Mistral / LLama / 
- AI for coding: 
    - Codestral
    - qwen
    - ...
- latest developments:
    - deepseek => pretty large. quantized version didn't work that great. thinking process takes too long.
- unit testing
\fi


\subsubsection{Open-source tools} 
have been developed on multiple occasions to facilitate grading Jupyter Notebooks. Otter-Grader ____ as well as OKpy ____ were implemented at UC Berkley to serve in computer science and data science courses. UNCode ____ provides immediate feedback for students of an introductory lecture for "Intelligent Systems and Machine Learning" and NB grader ____ is widely used for creating and grading assignments in computer science. 
However, these tools primarily focus on testing code solutions with rigorous unit-tests and do not support grading text exercises (for example mathematical derivations). Grading such exercises requires a more flexible LLM as a backbone.

\iffalse
- NB grader ____
    => Problems: grades by tutors take time
    => no automation for "text exercises" / proofs
- https://notebookgrader.com/
    (see \url{https://www.youtube.com/watch?v=yvLWbpgnspM\&ab\_channel=LucadeAlfaro-InstructionalVideos})
    => based on nbgrader
    => grading with unit tests / no LLMs (AI feedback only at UCSC)
- Otter-grader ____ (https://discourse.jupyter.org/t/autograding-notebooks-with-otter-grader/4627?utm_source=chatgpt.com)
- UNCode ____
- OKpy ____
\fi

% Model vs. Service (ChatGPT / GPT-4)
\subsubsection{Large language models} like ChatGPT (OpenAI), Gemini (Google), Claude (Anthropic) have shown tremendous success in recent years and offer convenient APIs for easy access in the cloud. However, sending sensitive student data to external providers without explicit consent of the students might violate privacy policies of the university. 
Fortunately, more and more powerful open weight models become publicly available, such as Llama ____, Mistral ____, %TODO: Mistral Small, Mistral Large Citations? Is there a full paper about Nemo by now?
DeepSeek ____, and many more. Such models can run on local infrastructure to avoid any privacy concerns.


%But open alternatives have become increasingly powerful ...

% \subsubsection{Focus on Privacy / local open-source models} % => 
% Mention TGI and open WebUI?


% User privacy should have utmost importance for responsible use of LLM and is also required by laws like GDPR. Many universities in addition have strict regulations on how student data may be processed by external entities and may require prior approval of external systems before they can be used. This means that some of the popular service cannot readily be used without full consent of the students and the university.


%\subsubsection{Datasets}
%mostly: Q and A. not Q and A and feedback for A
% => datasets wären eher etwas für future work...