\section{RELATED WORKS}
\label{sec2}
\subsection{Subpixel edge localization}
Given the prevalence of subpixel edge detection in high-precision fields, a large number of related algorithms have been proposed. These algorithms may be broadly classified into three categories\cite{Fabijaska2015SubpixelED}: moment\cite{tabatabai1984edge,Ghosal1993OrthogonalMO,Bin2008SubpixelEL,Sun2014ARE}, interpolation\cite{Jensen,Qingli2003AIS,Chen201611,Single-Pixel-Multi-Point}, and fitting methods\cite{Nalwa1986699,Duan2018High,Liu2022ANS,2011Edge}.

Tabatabai and Mitchell \cite{tabatabai1984edge} developed the earliest moment-based method on subpixel edge localization, utilizing the relationship between the edge parameters and the gray moments (GMs). 
To obtain invariant under rotation, Ghosal et al.\cite{Ghosal1993OrthogonalMO} proposed Zernike orthogonal moments (ZOMs), but it cannot accurately detect small objects. The orthogonal Fourier–Mellin moments (OFMMs)\cite{Bin2008SubpixelEL} were proposed later to deal with this issue, and the testing accuracy was proven meeting the stringent requirements in medical image analysis or satellite remote sensing. 
Interpolation-based methods leverage interpolating the intensity or the corresponding derivative function of edge pixels to increase the edge information, thereby realizing subpixel edge detection. 
Jensen and Anastassiou\cite{Jensen} initially exploited non-linear interpolation to locate subpixel edges, their approach can produce noticeably sharper edges and exhibit a lower error than linear methods. 
Subsequent researchers designed different interpolation schemes\cite{Qingli2003AIS,Chen201611,Single-Pixel-Multi-Point}. 
The fitting-based methods usually assume that the curve of edge intensity or gradient variation follows to a functional model, after which subpixel coordinates are obtain from fitting curve using the least square method. 
Nalwa and Binford \cite{Nalwa1986699} proposed the first fitting-based method with the hyperbolic tangent function as the model. A blurred edge model \cite{YE2005453} was adopted to locate subpixel coordinates as well, and the experiments have demonstrated its good performance on both robustness and accuracy.
Similarly, Hagara and Kulla\cite{2011Edge} proposed the Erf function to approximate the true edge. Their method is considered to be the most accurate with the high computational cost. 
% Accounting for the blurring effects of the imaging process, a model\cite{Qing2023AUnified} for image gradient distribution around the control points is derived from theoretical analysis with a closed-form solution, this method is simple and is of high precision, and is not affected by the size of the fitting area.

Recently, researchers\cite{Gioi2017ASE,Seo2018SubpixelEL,TrujilloPino2013AccurateSE,Chu2020SubpixelDM} have made remarkable progresses on methods other than that from above three traditional categories.
The method based on the partial area effect\cite{TrujilloPino2013AccurateSE} has been applied in many different fields\cite{LI2022Visual,Lu2022ANS,poyraz2024sub}, which assumes a valid intensity equation by using the discrete character and regional correlation of edge pixels.
Gioi\cite{Gioi2017ASE} incorporated the classic Canny and Devernay, but using only three pixels makes this approach sensitive to noises. Besides, Seo\cite{Seo2018SubpixelEL} presented a subpixel edge localization method based on the adaptive weighting of gradients (AWG). It has less computational cost and more accuracy than the Erf-fitting method.
And the algorithm based on the intensity integration threshold (IIT)\cite{Chu2020SubpixelDM} locates the subpixel point where the intensity integration reaches the threshold. 
In contrast to these studies, our research delves deeply into the relationship between intensity mapping at subpixel and pixel levels, which leads us to propose a more efficient and streamlined formula.

\subsection{Edge linking}
The following literatures \cite{XIE1992647, FaragandDelp,Ghita2002479,Wang,Topal2012EdgeDA,Akinlar2016PELAP,Seo2019SubpixelLL} have presented the representative works so far on edge linking at pixel level, which inspired our related research at subpixel level.

As one of the earliest researchers presenting the methods of linking edge pixels, Xie\cite{XIE1992647} proposed the concepts of horizontal edge elements and causal neighborhood windows to realize edge linking. 
Farag and Delp\cite{FaragandDelp} used the path metric based on the linear model as well as the A* algorithm to construct a new linking algorithm. Wang and Zhang \cite{Wang} improved previous linking method by calculating the edge direction within a specific local neighborhood and measuring the geodesic distance. After that, Akinlar and Chome\cite{Topal2012EdgeDA,Akinlar2016PELAP} introduced smart routing (SR) and predictive edge linking (PEL) to link neighboring anchors and obtain continuous edges.

\begin{figure}[t]
    \centering
    \includegraphics[width=2in]{image_m/img_source.pdf} 
    \caption{Regions containing edges in the test image of a house.}
    \label{fig_s}
\end {figure}

\begin{figure}[h]
   \centering
    \subfigure{
    \includegraphics[width=0.7\linewidth]{image_m/img3.pdf}}
    \caption{The different intensity mappings in edge intensity curve. (a) Enlarged sub-image A form Fig. \ref{fig_s}. (b) Intensity distribution near an edge running vertically. (c) A sequence of $n$ pixels extending along the gradient orientation. 
    (d) Intensity sampling (direct mapping) used in the fitting method. (e) Integral mapping used in our method.}  
    \label{fig5}
\end{figure}

\begin{figure}[b]
\begin{minipage}[t!]{0.45\linewidth}
    \centering
    \includegraphics[width=2in]{image_m/img4.pdf} 
    \caption{The edge intensity curve and the curve from fitting pixel intensity. $c$ is the subpixel coordinate of the edge, while $g_c$ is its intensity. $g_a$ and $g_b$ are the intensities located on the either side of the edge.}
    \label{fig4}
\end{minipage}%
\hspace{5mm}
\begin{minipage}[t!]{0.45\linewidth}
    \centering
	\subfigure[]{
	\includegraphics[width=1.35in]{img3/img3.1.pdf} 
    \label{fig3.1}
    }
	%\hspace{0mm}
	\subfigure[]{
	\includegraphics[width=1.35in]{img3/img3.2.pdf} 
	\label{fig3.2}
    }	
	\caption{The integral in different functions. (a) The integral in the edge intensity curve. (b) The integral in the step function.}
    \label{fig3}
\end{minipage}

\end{figure}