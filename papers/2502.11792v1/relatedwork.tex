\section{Background and Related Work}
\label{sec:BGAndRW}
In this section, we provide an overview of the background and related work. In Section~\ref{sec:BGAndRW:BG}, we provide the background in terms of the OpenAPI Specification (OAS) used to design and realize the process ecosystem. Section~\ref{sec:BGAndRW:RW} discusses related work.

\subsection{Technical Background}
\label{sec:BGAndRW:BG}
The OpenAPI Specification (OAS; \cite{OpenAPI}) defines a standard, programming language-agnostic interface description for HTTP-based APIs, which enable both humans and computers to explore and comprehend the functionalities of a service without the need for having access to source code, extra documentation, or scrutiny of network traffic. When accurately specified through OpenAPI, a consumer can grasp and engage with the remote service with minimal implementation logic required. The API can be specified in \emph{Json} or \emph{Yaml} as machine-readable specification. With the help of \emph{SwaggerUI}, a specification can be visualized and used interactively. OpenAPI and the complementing software infrastructure lay the technical foundation for the research presented in the paper at hand.

\subsection{Related Work}
\label{sec:BGAndRW:RW}
Modeling the software process has been extensively researched over the years. For instance, Bendraou et~al.\ \cite{Bendraou:2010jq} studied six UML-based process modeling languages and evaluated their usability for various use cases. However, over the years, the relevant languages consolidated \cite{kuhrmann2013systematic} resulting in the SPEM \cite{OMG2005} and the V-Modell~XT metamodel \cite{kuhrmann2016use} as the final two actively researched and practically applied large metamodels. However, even though the basically metamodel-free agile methods gained popularity, especially for dependable and large-scale systems, research on software process lines (SPrL) and variability management in such process lines gained more attention \cite{Oliveira:1900cr,simmonds2013variability,de2014software,DBLP:journals/jss/KuhrmannTFRB16,costa2018software} alongside with opportunities to enact and execute software processes to directly support projects based on structured models \cite{Bendraou:2007ja,Oliveira:1900cr,Min:1997cq,kktw2010a,DBLP:journals/scp/KuhrmannKT14}. Costa et~al.\ \cite{costa2018software} provide the most recent state-of-the-art review on the definition of software processes using 26 publications and came to the conclusion that mapping- and rule-based approaches gained most attention. From their research, they derived a need for proper tool support. In \cite{costa2020evaluating}, Costa et~al.\ study the acceptance of such a tool called \emph{Odyssey} in Brazilian software-producing companies. However, Odyssee is, again, a specific, self-contained expert tool that requires users having a specific skillset to use it properly. Notably for complex software process lines, the simple access for end-users is crucial to avoid overheads due to improperly tailored software processes, as it is illustrated by the challenges of adopting process lines in industrial practice \cite{simmonds2013variability}.

The work by Kuhrmann et~al.\ \cite{kktw2010a,DBLP:journals/emisaij/KuhrmannKK13,DBLP:journals/scp/KuhrmannKT14} was specifically focused on defining domain-specific languages to allow for better connecting software process models with project-supporting tools to address the aforementioned issue. Specifically, the PDE and PET\footnote{PDE and PET have been developed in the experimental tools branch of the German V-Modell~XT (cf.\ Table~\ref{tab:CS:ProcessProfile}) and were primarily used to evaluate new features and to explore opportunities for next-generation process support tools.} tool set \cite{kktw2010a} used an \emph{intermediate model} to directly translate metamodel-based software process models into specific target-tool formats, e.g., document templates and project templates for Microsoft Visual Studio and SharePoint. For this, a dedicated meta-metamodel was designed \cite{DBLP:journals/emisaij/KuhrmannKK13}, which allowed for designing process metamodels in an enactable fashion. Source models, like SPEM-/EPF-based processes \cite{OMG2005,eclipse-process-framework} or V-Modell~XT variants \cite{DBLP:journals/jss/KuhrmannTFRB16} could be imported, edited, and exported to various formats, e.g., a Microsoft SharePoint team website including all document templates for a project of a specific kind. However, PDE and PET have been developed using the Visual Studio Language SDKs and followed a modular, yet, ``hard-coded'' approach in which all import and export filters were dedicated modules for expert-tools.

Most work on rich processes that is primarily focused on large-scale projects in regulated environments, such as aerospace, automotive, or public sector, aims at providing more flexibility or even more sophisticated tool support to ease the ``pain'' of using such large-scale models. However, easy access to specifically scoped information required for a specific task in a project was, so far, not in the spotlight. In this paper, we fill this gap by re-thinking complex SPrL-ready process infrastructures to improve the usability and accessibility of task-specific process content by describing a software process ecosystem \emph{as a service}.