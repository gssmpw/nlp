\section{Literature Review}
ML algorithms have demonstrated significant potential in fetal BW prediction, with recent studies showing varying degrees of success across different methodological approaches. Artificial Neural Networks (ANNs) and Naive Bayes (NB) achieved 70\% accuracy on balanced datasets (n=500) in predicting BW at six months of pregnancy (Adeeba, S., Kuhaneswaran, B., \& Kumara, B., 2022).\cite{5A}, while Gaussian Naïve Bayes implementations demonstrated 86\% accuracy in controlled studies with balanced datasets(n=445) when classifying BW into binary categories(Bekele W. T. (2022))\cite{6A}. In contrast, Decision Trees underperformed with accuracy rates below 60\% (Adeeba et al., 2022)\cite{5A}. The implementation of XGBoost showed particular promise in handling high-dimensional medical data through gradient tree boosting, especially when dealing with complex maternal health indicators (Chen \& Guestrin, 2016)\cite{7A}. Multiple Imputation by Chained Equations (MICE) demonstrated superior performance in handling missing values compared to K-Nearest Neighbors (KNN) imputation, particularly in preserving temporal consistency within longitudinal datasets, as evidenced by analysis of the Pune Maternal Nutrition Study (PMNS) dataset encompassing over 5000 variables (Varma et al., 2024)\cite{8A}.
 
The prior study, i.e., phase 1 study of birth cohort, which highlighted the effectiveness of data imputation technique by achieving superior performance in handling missing values, reducing imputation error by 23\% compared to conventional methods \cite{8A}. Their analysis of the PMNS dataset (n$>$5000) showed that MICE preserved temporal consistency in longitudinal data with 89\% accuracy, significantly outperforming K-Nearest Neighbors (KNN) imputation, which achieved 74\% accuracy in maintaining data relationships, where the present study (phase 2) employs the imputed PMNS dataset (of phase 1) to predict fetal BW. A study by (Luke Oluwaseye Joel..,2024) evaluated various imputation methods, including MICE, Mean, Median, Last Observation Carried Forward (LOCF), KNN, and Missforest imputation \cite{9A}. The findings indicated that MICE consistently outperformed other techniques in maintaining data integrity across different healthcare datasets, including those related to breast cancer and diabetes.  MICE operates under the assumption that missing values can be predicted based on observed data, thus reflecting the underlying distribution more accurately than simpler methods \cite{9A}. The Data Analytics Challenge on Missing Data Imputation (DACMI) highlighted advancements in clinical time series imputation, showcasing competitive ML models like LightGBM and XGBoost alongwith MICE, emphasizing the importance of temporal and cross-sectional features in achieving robust imputation results \cite{10A}.

The development of accurate fetal BW prediction models relies heavily on sophisticated feature selection methodologies and effective missing data imputation techniques. A comprehensive analysis by Gaillard et al. (2011)\cite{11A} used Principal Component Analysis (PCA) to identify critical maternal anthropometric measurements, revealing that gestational age, BMI, and blood pressure measurements exhibited the strongest correlations with fetal BW. The Generation R Study, which highlighted the critical role of maternal obesity indices and hypertensive disorders in influencing birth outcomes. Expanding upon this, D'souza et al. (2021) demonstrated the strong correlations between maternal vitamin B12 levels and neurodevelopmental outcomes, emphasizing the importance of incorporating nutritional factors into predictive models \cite{12A}. A more recent study by Esther Liu (2024) explored various feature selection methods, including forward selection, backward elimination, and stepwise selection, on a dataset of 1301 mother-child pairs \cite{13A}. The other feature selection approaches, such as filter methods using statistical tests like Pearson correlation, wrapper methods (which evaluate feature subsets based on model performance), and embedded methods (such as Lasso regularization and tree-based models)\cite{14A}, all of which are crucial for identifying the most relevant features for accurate predictions. 

Furthermore, the use of feature selection techniques like Boruta has been shown to optimize model outcomes, particularly in predicting LBW. Hybrid methods based on ensemble learning have also demonstrated effectiveness in predicting BW ranges, underscoring the importance of feature selection in improving prediction accuracy \cite{15A}. Additionally, studies like those of Moreira et al. (2019) and others focusing on high-risk pregnancies have reinforced the significance of robust feature selection in enhancing predictive performance \cite{16A}. Approaches like Mutual Information (MI), which captures both linear and nonlinear relationships\cite{17A}, and Kendall transformation, which preserves ranking in categorical data, offer versatile and robust methods for selecting relevant features \cite{18A}. Collectively, these studies emphasize the pivotal role of integrating advanced feature selection techniques with reliable imputation methods to develop robust and accurate predictive models for fetal BW, contributing to improved prenatal care outcomes.

Recent validation studies, by Hussain and Borah (2020), Adeeba et al. (2022) have utilized various ML evaluation techniques. In the study of Hussain and Borah's Random Forest model achieved coefficient of determination (R²) of 0.87, while Adeeba et al. employed Support Vector Machines, achieving R² of 0.83. These studies underscore the importance of using robust evaluation metrics like AUROC, R², Root Mean Squared Error (RMSE), precision, recall, and F1 score to assess model performance\cite{19A}. These studies highlight the need for multi-center validation across diverse healthcare settings and the development of standardized protocols for model deployment in clinical workflows to realize the full potential of these technologies in improving maternal-fetal health outcomes. Dataset size constraints ($n<500$ in multiple studies), geographic and demographic homogeneity in training data, and limited integration of real-time fetal monitoring data present significant methodological limitations \cite{20A}. Technical requirements for advanced implementation include standardized APIs for healthcare system integration and automated feature selection protocols for high-dimensional medical data. The integration of Electronic Health Record (EHR) systems presents specific challenges in data standardization across diverse healthcare systems and real-time prediction capabilities for immediate clinical decision support\cite{21A}. 

The gaps analyzed from the prior studies is– the research should focus on expanding dataset diversity, without advanced imaging data in prediction models, and developing clinical workflow integration protocols. The success of ML applications in fetal BW prediction ultimately depends on addressing these challenges while maintaining high prediction accuracy and clinical utility. As demonstrated by the reviewed studies, the field shows promising potential for enhancing prenatal care through accurate BW prediction, particularly when sophisticated imputation techniques and comprehensive feature selection methods are employed in conjunction with advanced ML algorithms.