\section{Related Work}
\label{sec:related work}
\subsection{Equality Saturation}
Equality saturation and its use as a code optimization tool was first introduced by \citeauthor{tateEqualitySaturationNew2009}~\cite{tateEqualitySaturationNew2009}, where it was used to optimize Java bytecode.
\citeauthor{willseyEggFastExtensible2021}~\cite{willseyEggFastExtensible2021} introduced algorithmic improvements for equality saturation providing asymptotic speed-ups.
This has led to an explosion of new work exploring different applications for equality saturation, from optimizing tensor programs~\cite{wangSPORESSumproductOptimization2020,MLSYS2021_cc427d93,smithPureTensorProgram2021}, to compiler verification~\cite{steppEqualityBasedTranslationValidator2011,kourtaCaviarEgraphBased2022}, optimizations for specialized hardware~\cite{ustunIMpressLargeInteger2022,vanhattumVectorizationDigitalSignal2021,matsumuraSymbolicEmulatorShuffle2023}, and other purposes~\cite{panchekhaAutomaticallyImprovingAccuracy2015,chandrakananandiRewriteRuleInference2021}.

Metatheory.jl~\cite{cheliMetatheoryjlFastElegant2021} provides a framework for equality saturation in Julia.
It has already been used for code optimization on symbolic representations of Julia code to speed up PDE solvers~\cite{gowdaHighperformanceSymbolicnumericsMultiple2022}. That work, however, only handles pure, symbolic programs, as it operates on a symbolic representation of Julia code obtained through manual expression building or tracing through heavily restricted, straight-line code. In contrast to our work, that existing use of Metatheory.jl does hence not allow the presence of general control flow or side effects.

To the best of our knowledge, by introducing type propagation in the e-graph, we are the first to apply equality saturation to a dynamic programming language.
This allows rewrites to coexist and complement Julia's multiple dispatch feature, as discussed in Section~\ref{sec:usecase}.

\subsection{Intermediate Representations and E-Graphs}
Cranelift, a compiler back-end for WebAssembly and Rust, is among the first to use e-graphs and equality saturation in a production-grade compiler~\cite{fallinAegraphsAcyclicEgraphs2023}. While we reuse ideas and methods first introduced in the Cranelift compiler, we implemented our framework as a standalone project, not reusing any Cranelift components. Our code is integrated in the Julia compiler using user-land compiler extensions.

The Cranelift compiler does not do full equality saturation, but only recognizes equalities the first time a node is inserted in the e-graph.
This implies that it is possible for equalities to remain undiscovered, but it obviates the fixed-point loop present in full equality saturation and ensures there are no cycles in the e-graph which allows for a more efficient representation of the graph in memory and leads to easier extraction.
We use the same e-graph representation of code as used in Cranelift, but by utilizing an extraction scheme that enforces acyclic extraction, the optimizer is able to carry out full equality saturation.
Whereas Cranelift and our work operate on IR that is complemented by a CFG, other work has sought to find alternative IR formats that allow representing control flow fully in the e-graph.
\citeauthor{tateEqualitySaturationNew2009}~\cite{tateEqualitySaturationNew2009} have introduced Program Expression Graphs (PEGs) which represent constructs such as loops as special expressions and show that PEGs can be used to do equality saturation.
Similarly, Regionalized Value State Dependency Graphs (RVSDGs)~\cite{bahmannPerfectReconstructabilityControl2015,reissmannRVSDGIntermediateRepresentation2020} represent control flow as (nested) expressions.
A research prototype already exists using RVSDGs in conjunction with equality saturation for a simple toy language~\cite{optir2022}.
Our approach remains closer to the original SSA-based IR of the source language, allowing less expensive conversion routines.

\citeauthor{willseyEggFastExtensible2021} introduced e-class analyses to associate additional metadata with each e-class~\cite{willseyEggFastExtensible2021}. The metadata is kept up to date throughout the equality saturation process by potential modifications whenever new information is available, for example, when e-classes are merged. An exemplary existing use case for an e-class analysis is constant propagation, where each e-class can optionally carry a constant value. To track types in the e-graph, we designed a novel e-class analysis, as discussed in Section~\ref{sec:types}


\subsection{Extraction}
We based our ILP optimal extraction formulation on the work of \citeauthor{heImprovingTermExtraction2017}~\cite{heImprovingTermExtraction2017}.
Recent work proposes other formulations, for example by looking at e-graphs through the lens of circuits~\cite{sunEgraphsCircuitsOptimal2024}, or finite state automata~\cite{y.wangEGraphsVSAsTree2022}.
These formulations can lead to faster extraction or provide termination guarantees under particular assumptions.
Cranelift operates in a Just-In-Time context where fast and reliable extraction is crucial.
For this reason, they use an extraction algorithm that greedily tries to minimize the total number of nodes, without taking into account node reuse in extracted expressions or from expressions that dominate them.
In contrast to Cranelift's greedy algorithm, we have introduced an ILP formulation that aims for node reuse from other nodes within an extracted expression, as well as from nodes in expressions that dominate them in the CFG. 

\subsection{Julia Code Optimization}
Other works exist that aim to improve the performance of Julia code with custom optimizations.
Symbolics.jl~\cite{gowdaHighperformanceSymbolicnumericsMultiple2022} can be used to transform scientific machine learning code into a symbolic representation at which point simplifications can be applied.
Finch.jl~\cite{ahrensFinchSparseStructured2024,ahrensLoopletsLanguageStructured2023} uses a custom IR to represent iteration patterns of loop nests over sparse or structured arrays.
We instead focus on optimizing code in Julia's own IR format, allowing users of our system to target any Julia code.

%\subsection{Rewriting Systems}
%\todo[inline]{If we want to, we can discuss other rewriting systems, such as suggested by reviewers at CC'25. This is not absolutely necessary, however.}