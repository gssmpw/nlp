\begin{abstract}
When Computer Science (CS) students try to use or extend open-source software (OSS) projects, they often encounter the common challenge of OSS failing to build on their local machines. Even though OSS often provides ready-to-build packages, subtle differences in local environment setups can lead to build issues, costing students tremendous time and effort in debugging. Despite the prevalence of build issues faced by CS students, there is a lack of studies exploring this topic. To investigate the build issues frequently encountered by CS students and explore methods to help them resolve these issues, we conducted a novel dual-phase study involving 330 build tasks among 55 CS students. Phase I characterized the build issues students faced, their resolution attempts, and the effectiveness of those attempts. Based on these findings, Phase II introduced an intervention method that emphasized key information (e.g., recommended programming language versions) to students. The study demonstrated the effectiveness of our intervention in improving build success rates. Our research will shed light on future directions in related areas, such as CS education on best practices for software builds and enhanced tool support to simplify the build process.
\end{abstract}