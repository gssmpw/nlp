\section{Introduction}
It is highly beneficial to integrate open-source software (OSS) projects into a software engineering (SE) curriculum for three reasons. First, OSS exposes students to real-world software development practices. By inspecting the software evolution and reading developers' discussions on technical issues, students can improve their understanding of industrial challenges and become better prepared for careers in the software industry~\cite{Gokhale2012}. Second, OSS offers opportunities for students to gain hands-on  experience when they use and modify the software based on local needs and preferences~\cite{Kotwani2011}, collaborate with a global community of developers, and adopt project management tools like version control as well as issue tracking. Third, the educational usage of OSS bridges the gap between theoretical knowledge and practical application, enabling students to see how theoretical knowledge is applied in real life~\cite{Pinto2017} and to solve real problems with book knowledge.

While the integration of OSS projects into SE curriculum brings great opportunities, it also poses challenges to Computer Science (CS) education. One big challenge is that students often fail to build the provided OSS on their local machines. Build failures can be very detrimental to students' learning outcomes. This is because as students encountered build failures at the beginning of OSS-based class activities, they might spend tremendous time and effort trying to resolve those issues and neglect teachers' further instructions on designated tasks. As a result, they might lose the valuable opportunities to read, modify, or execute the same software as other students do and become unavailable to collaborate with other students. 

In the research area of CS education, educators and researchers proposed various ways of integrating OSS into CS curriculum~\cite{nascimento2015open,Pinto2017,Kotwani2011,Gokhale2012,choi2021open}, discussed the benefits and challenges brought by the OSS-integration into courses~\cite{Salerno_de2023}, and measured students' contributions to OSS~\cite{Fang_Endres_Zimmermann_Ford_Weimer_Leach_Huang_2023}. In particular, Salerno et al.~\cite{Salerno_de2023} conducted a large-scale study on the impact of OSS-based courses, revealing that 83\% of the students faced technical challenges while contributing to OSS projects. While only 2\% of students anticipated hurdles in setting up the local environment, a notable 9\% of students eventually reported facing such issues. This discrepancy suggests that students tend to underestimate the challenge of addressing build issues. None of the existing work characterizes the build issues CS students frequently encounter in OSS-based courses, let alone suggest intervention to help address those issues.

In the research area of build issues, some researchers conducted user studies to explore the influencing factors (e.g., tool usage) for software build results~\cite{Kwan_Schroter_Damian_2011,downs2012ambient,phillips2014understanding,kerzazi2014automated,hilton2016continuous,vassallo2020every}. Other researchers examined relevant artifacts (e.g., build logs and bug fixes) to characterize various aspects of software build, including developers' efforts in resolving build issues~\cite{mcintosh2011empirical,mcintosh2015large}, root causes for build failures~\cite{xia2014empirical,barrak2021builds,wu2020empirical}, and fixes~\cite{zhao2014empirical,lou2020understanding}. However, these studies are only vaguely related to the build issues faced by CS students for two reasons. First, prior work studied build issues from the perspective of developers or project managers who owned the software-to-build, rather than from the perspective of CS students who are new to the software-to-build. Second, prior studies were based on either self-retrospection or artifacts of experienced software practitioners, instead of based on the experience of CS students in class. Thus, many of the research findings by prior work are unlikely to find evidence in the build issues faced by students.

We conducted a related pilot study~\cite{huang2024build} to investigate the symptoms and causes of build issues experienced by non-contributors (e.g., users, learners, and potential contributors). The findings highlight specific build issues that are challenging to resolve and emphasize the need for further study to understand non-contributors' behavior.

To overcome the limitations of prior work, in this paper, we introduce an exploratory study to investigate the build issues faced and resolved by CS students. The study has two phases, which tracked the behaviors of 55 CS students as they undertook 330 OSS build tasks in an advanced SE course. In Phase I, each student of the course in 2022 was assigned six build tasks. Phase I collected and analyzed the 198 build results from 33 students to characterize the symptoms of build issues, the students' resolution attempts, and the effectiveness of those attempts. Phase II introduced an intervention to help students build software. Namely, the intervention emphasized providing additional key information sources, so that students can refer to them initially when completing build tasks. As with Phase I, Phase II assigned each student of the course in 2023 with six build tasks. By gathering and analyzing the 132 build results of 22 students, Phase II compared students' results with those from Phase I, to assess the intervention effectiveness.

This paper makes the following major contributions:
\begin{itemize}[leftmargin=*]
    \item We presented a comprehensive analysis of the build issues encountered by 55 CS students, offering insights into the common challenges faced during students' OSS builds.

    \item Our study revealed the resolution strategies frequently applied by students, although some of these strategies often led to build issues that are challenging to resolve.

    \item We introduced an intervention method (i.e., providing key information sources initially to bridge the knowledge gap), which demonstrated effectiveness in helping students better build OSS.

   \item Our study offers practical recommendations for various stakeholders in the OSS and CS education communities, aiming to enhance the overall experience of integrating OSS into the SE curriculum.
\end{itemize}