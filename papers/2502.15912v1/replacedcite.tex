\section{Related Work}
\label{sec:related}

Our study is related to existing research efforts on the educational use of OSS projects, user studies of software build, empirical studies of build issues from the developer's perspective, and studies of novice programmers. We will discuss related works in each category in the following subsections.

\textbf{Educational Use of OSS Projects in CS Courses}
The integration of open-source software (OSS) projects into computer science (CS) education has been widely explored. 
Nascimento et al.____ reviewed how OSS projects facilitate software engineering (SE) learning. 
Choi et al.____ proposed models for integrating OSS into lower-level CS courses. 
Fang et al.____ compared student contributions to general OSS projects and those aimed at social good, offering insights into educational interventions. 
Salerno et al.____ examined the impact of OSS development courses on students' self-efficacy and the barriers they face. 
These studies collectively underscore the value of OSS projects in CS education, highlighting both the challenges and the potential for enhancing student learning and engagement. Unlike previous studies, our research specifically focuses on the build challenges encountered during the educational use of OSS projects in CS courses.

\textbf{User Studies of Software Build.} The most relevant research to ours involves user studies on software build tasks. 
Kwan et al.____ examined IBM developers to see if team composition and coordination affect build success. 
Dawns et al.____ evaluated a build management tool's impact on handling build failures.
Philips et al.____ studied Microsoft build teams, finding social challenges were significant.
Kerzazi et al.____ analyzed 3,214 builds, noting a 17.9\% failure rate and significant time costs.
Hilton et al.____ interviewed developers about continuous integration (CI) processes.
Vassallo et al.____ assessed a tool for summarizing build failures. 
Due to the challenges of performing such user studies, we can also see that there are not many existing studies, and all the studies have relatively small scales. 
Different from our research, all these studies are from the perspective of project managers or senior developers instead of CS students (OSS learners). 

\textbf{Empirical Studies on Build Issue}
Besides user studies, there have been many empirical studies on software build history and failures. McIntosh et al.____ studied version histories to estimate the effort required to revise and repair build scripts. Later, they correlated this effort with the type of build systems used____. Xia et al.____ summarized the characteristics of bugs in build systems. Zhao et al.____ found that build failures, though typically of lower severity, take more time to fix. Xia and Li____ investigated the feasibility of predicting build failures using the TravisTorrent dataset____. Shridhar et al.____ qualitatively analyzed changes in build scripts. Barrak et al.____ studied the correlation between build failures and code smells. Lou et al.____ identified patterns in build fixes. Zolfagharinia et al.____ found different distributions of build failures across environments. Licker and Rice____ used mutation testing to detect incorrect rules in build scripts. Wu et al.____ studied build failures in Docker environments. 
These existing studies focus on build failure logs and build failure fixes in the commit history. In contrast, our study monitors and analyzes the whole process of CS students finishing multiple building tasks and records all the system environment factors that affect the build process failures.

\textbf{Studies of Novice Programmers}
Another research area related to our research is user studies of novice programmers. 
Lahtinen et al.____ surveyed over 500 students and teachers in computer-related majors to identify challenges faced by novice developers. Warner et al.____ evaluated CodePilot with eight novice developers to assess its usability and educational benefits. Marques et al.____ monitored students over nine semesters to see if reflexive weekly monitoring aids in completing course projects and improving coding skills. Ardimento et al.____ analyzed 40 novice developers' IDE usage patterns across five tasks. Romano et al.____ examined 29 novice developers to determine the impact of test-driven development on positive affective states. Rehman et al.____ reviewed 208 novice contributions on GitHub to identify common contribution types. Oliveira et al.____ compared novice and experienced developers to study code smells.
Compared with these studies, our work focuses on the software build process instead of general programming.