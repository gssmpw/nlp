
\typeout{IJCAI--24 Instructions for Authors}


\documentclass{article}
\pdfpagewidth=8.5in
\pdfpageheight=11in

\usepackage{ijcai24}
\usepackage{arydshln}
\usepackage{amssymb}  
\usepackage{amsfonts}
\usepackage{graphicx}  
\usepackage[table,xcdraw]{xcolor} 

\definecolor{lightblue}{rgb}{0.88, 0.92, 0.97}

\usepackage{times}
\usepackage{soul}
\usepackage{url}
\usepackage[hidelinks]{hyperref}
\usepackage[utf8]{inputenc}
\usepackage[small]{caption}
\usepackage{graphicx}
\usepackage{amsmath}
\usepackage{amsthm}
\usepackage{booktabs}
\usepackage{algorithm}
\usepackage{algorithmic}
\usepackage[switch]{lineno}


\urlstyle{same}


\newtheorem{example}{Example}
\newtheorem{theorem}{Theorem}






\pdfinfo{
/TemplateVersion (IJCAI.2024.0)
}

\title{DOEI: Dual Optimization of Embedding Information for Attention-Enhanced Class Activation Maps}
 

\author{
Hongjie Zhu$^1$,~
Zeyu Zhang$^{2\dag}$,~ %
Guansong Pang$^3$,~
Xu Wang$^4$,~
Shimin Wen$^1$,~\\
Yu Bai$^1$,~
Daji Ergu$^1$,~ %
Ying Cai$^{1}$\thanks{Corresponding author (caiying@swun.edu.cn). $^\dag$Project lead.},~
Yang Zhao$^5$\\ %
\affiliations
$^1$Southwest Minzu University~
$^2$The Australian National University\\
$^3$Singapore Management University~
$^4$Sichuan University~
$^5$La Trobe University
}


\begin{document}

\maketitle

\begin{abstract}

    Weakly supervised semantic segmentation (WSSS) typically utilizes limited semantic annotations to obtain initial Class Activation Maps (CAMs). However, due to the inadequate coupling between class activation responses and semantic information in high-dimensional space, the CAM is prone to object co-occurrence or under-activation, resulting in inferior recognition accuracy. To tackle this issue, we propose \textbf{DOEI}, Dual Optimization of Embedding Information, a novel approach that reconstructs embedding representations through semantic-aware attention weight matrices to optimize the expression capability of embedding information. 
    Specifically, DOEI amplifies tokens with high confidence and suppresses those with low confidence during the class-to-patch interaction. This alignment of activation responses with semantic information strengthens the propagation and decoupling of target features, enabling the generated embeddings to more accurately represent target features in high-level semantic space. 
    In addition, we propose a hybrid-feature alignment module in DOEI that combines RGB values, embedding-guided features, and self-attention weights to increase the reliability of candidate tokens. 
    Comprehensive experiments show that DOEI is an effective plug-and-play module that empowers state-of-the-art visual transformer-based WSSS models to
    significantly improve the quality of CAMs and segmentation performance on popular benchmarks, including {\it PASCAl VOC} (+3.6\%, +1.5\%, +1.2\% mIoU) and {\it MS COCO} (+1.2\%, +1.6\% mIoU). 
    Code will be available at \url{https://github.com/AIGeeksGroup/DOEI}.
    
\end{abstract}

\section{Introduction}

    \begin{figure}[t]
    \centering
    \includegraphics[width=\columnwidth]{figures/IJCAI-fig1.pdf}  
    \caption{Visualization comparison between the baseline (MCTformer~\protect\cite{xu2022multi}) and our method. The baseline model results exhibit background noise (object co-occurrence) and less precise localization (under-activation), as highlighted by red arrows and white bounding boxes. In contrast, our proposed method effectively reduces background activation and enhances focus on target regions, as highlighted by yellow arrows.}
    \label{Figure 1}
    \end{figure}

    Semantic segmentation \cite{ge2024esa,zhang2025gamed,wu2023bhsd,zhang2023thinthick} aims to assign each pixel in an image a specific category label~\cite{li2022class,zhang2024segreg,tan2024segstitch,tan2024segkan}. Traditional methods often rely on extensive and precise pixel-level annotations to promote network performance. However, obtaining such annotations is notoriously time-consuming and resource-intensive~\cite{cheng2023out}. Consequently, researchers have increasingly adopted alternative forms of weak supervision, such as scribbles~\cite{vernaza2017learning}, bounding boxes~\cite{lee2021bbam}, point annotations~\cite{bearman2016s}, and image-level labels~\cite{lee2021anti}, to achieve pixel-level segmentation. This paper explores techniques based on image-level labels, which are particularly advantageous due to their ease of collection from internet sources and minimal annotation costs. 
    
    Current image-level WSSS techniques commonly include the following steps: (1) generating CAMs~\cite{zhou2016learning,wang2020self,xu2022multi} for specific categories leveraging a classification network to locate objects roughly; (2) refining them into pseudo-mask annotations~\cite{ahn2018learning}; (3) applying these pseudo-mask annotations and the original images to train a semantic segmentation network. Acquiring high-quality CAMs is vital for the subsequent processes~\cite{cheng2023out}. However, existing WSSS methods often suffer from inaccurate CAMs, such as the mistaken activation of non-target objects (object co-occurrence) and incomplete activation of target objects (under-activation), due to the limited semantic depth of image-level labels. These challenges hinder accurate localization and segmentation, especially in multi-target scenes. As shown in Figure~\ref{Figure 1}, this phenomenon occurs primarily due to the failure to fully capture complex interactions between deep structures and features while establishing a strong relationship between activation responses and image semantics. Prior knowledge of an object’s category can offer insights into its holistic features when interacting with the original image. Although this information may appear as simple words or numbers in low-dimensional space, it can be articulated in a more comprehensive and intricate manner in high-dimensional space.

    \begin{figure}[t]
    \centering
    \includegraphics[width=\columnwidth]{figures/IJCAI-fig2.pdf}  %
    \caption{(a) The image and query targets (\(\star\)). (b) The self-attention maps in the Transformer block capture semantic-level relationships at various granularities. The high-activation regions learned by each layer not only provide critical information that subsequent layers may miss but also generate CAMs that often focus on different regions. This broadens the coverage of target feature areas, effectively mitigating the tendency of activation maps to focus excessively on local salient regions.}
    \label{Figure 2}
    \end{figure}
    
    Existing works predominantly emphasize extracting feature information for classification from input images, neglecting the critical role of high-dimensional semantic space in generating accurate CAMs~\cite{wang2020self}. This oversight typically results in CAMs that either cover only a portion of objects’ distinctive features or erroneously include non-target objects. As application scenarios for WSSS tasks grow more complex, limitations in the model’s ability to fully recognize contours and accurately locate target objects have become increasingly apparent. Our observations reveal that, in WSSS tasks, the cascaded encoders in ViT~\cite{dosovitskiy2021image}-based classification or representation learning models facilitate hierarchical and long-range information modeling for class-specific activations, as illustrated in Figure~\ref{Figure 2}. Inspired by this, we incorporate feedback from attention mechanisms across cascaded encoders into the embedding process to strengthen the influence of embeddings with reliable semantic information during information interaction. This enables class-specific feature learning at each layer. Additionally, by mitigating the occurrence of false positives through reducing the influence of unreliable embeddings, the final embedding retains the maximal useful information about the target object, thereby amplifying the model’s discriminative capability in generating CAMs.

    Furthermore, to improve the accuracy of feature representations in the multi-dimensional space during optimization, we propose a hybrid-feature alignment module that integrates RGB information of the original image with the embedding's cosine similarity features. The incorporation of this module aims to address the limitations of embedding representation capability in low-dimensional space. By employing this strategy, we further refine the model’s comprehension and representation of image semantics, leading to higher-quality object localization maps. Our main contributions are summarized as follows:
    
    \begin{itemize}
    
    \item We propose a novel mechanism, namely Dual Optimization of Embedding Information (DOEI), which is plug-and-play and can be applied to each layer encoder of the ViT. This mechanism effectively boosts useful information in the embedding while suppressing irrelevant information, achieving a rich representation of image features and intra-class diversity, thereby improving the accuracy of CAMs and reducing activation noise.
    
    \item We introduce a novel feature alignment module as a complementary optimization to DOEI. This module integrates the RGB values of the original image, the spatial features of embeddings, and self-attention scores, making the candidate tokens more meaningful, thereby facilitating the accurate representation of semantic structures and the effective transmission of embedding information.
    
    \item We incorporate the proposed mechanism into ViT-based models, with experiments demonstrating that this mechanism advances baseline model performance across different datasets without adding additional learnable parameters. It effectively prevents incorrect target localization and significantly improves the integrity of target recognition.
    \end{itemize}
    
\section{Related Work}

    \begin{figure*}[t] 
    \centering
    \includegraphics[width=\textwidth]{figures/IJCAI-fig3.pdf}
    \caption{An overview of the proposed method. The RGB image undergoes transformation into class tokens and patch tokens via the embedding layer. The dot-product attention mechanism is then employed to compute the attention score matrix and generate output tokens. To further refine the attention score matrix, the cosine similarity between the RGB values of the original image and the tokens is used to adjust the distribution of attention weights. Furthermore, the CPDO and PPDO methods are customized to amplify high-confidence information and suppress the influence of low-confidence information. Finally, the optimized tokens are incorporated into the original tokens as residuals, producing refined output tokens for subsequent computations in the encoder.
    }
    \label{Figure 3} 
    \end{figure*}

\subsection{Weakly Supervised Semantic Segmentation}

    In WSSS, full pixel-level segmentation relies on limited supervisory signals. The development of WSSS has significantly alleviated the dependency on large amounts of pixel-level labels in traditional semantic segmentation models. Prevailing approaches mainly use the CAM technique introduced by Zhou et al.~[\citeyear{zhou2016learning}] as an initial step in identifying target object locations. The CAM combines the global average pooling layer with the classification layer to efficiently consolidate feature information for each pixel, generating an activation map that is aligned with the semantic representation of a specific category and serving as a key component of WSSS. However, CAMs typically display only the salient regions of the target object, hindering the capture of complete location information. Therefore, directly adopting CAM as full pixel-level segmentation labels is constrained by these shortcomings. Most of the research focuses on generating more precise and comprehensive CAMs and refining pseudo-labels to augment the fidelity and precision of segmentation results. By optimizing the CAM generation process and incorporating richer semantic information, researchers aim to mitigate issues of insufficient or overly concentrated activation areas, paving the way for more accurate and versatile WSSS methodologies.

    
    
\subsection{Generating High-quality CAMs}

    Convolutional Neural Networks (CNNs) \cite{qi2025medconv} and Vision Transformers (ViTs) \cite{wu2024xlip,ji2024sine,zhang2024jointvit} are two popular approaches for generating CAMs. CNN-based methods often face challenges with localized activation due to the limited receptive field and reliance on local features. To address this, strategies such as random occlusion~\cite{kumar2017hide} (e.g., "hide-and-seek") and adversarial training~\cite{kweon2023weakly} have been employed to expand activation areas. Techniques like dilated convolutions~\cite{huang2018weakly} and multi-layer feature integration~\cite{li2022weakly} further elevate segmentation accuracy by capturing more contextual information. Additionally, specialized loss functions, such as contrastive loss~\cite{zhu2024weakclip} and SEC loss~\cite{wu2022adaptive}, and supplementary data, including saliency maps and videos~\cite{wang2018weakly}, have been used to improve object localization.An alternative method for generating CAMs is to utilize the self-attention weight matrix in ViT~\cite{dosovitskiy2021image}. For example, Gao et al.~[\citeyear{gao2021ts}] combine semantic-aware annotations with semantically unrelated attention maps, providing a feasible approach for object localization by utilizing the semantic and localization information extracted by ViT. Ru et al.~[\citeyear{ru2022learning}] refine the initial pseudo-labels for segmentation by learning robust semantic affinities with the aid of a multi-head attention mechanism. Xu et al.~[\citeyear{xu2023learning}] transform simple class labels into high-dimensional semantic information through Contrastive Language-Image Pretraining (CLIP) to guide the ViT, forming a multimodal representation of text and images, thus generating more precise object localization maps.
    
\subsection{Refinement of CAMs}

     At present, existing refinement methods primarily focus on the second stage of multi-stage models. For instance, Ahn et al.~[\citeyear{ahn2018learning}] train AffinityNet by leveraging reliable foreground and background activation maps to predict affinities between neighboring pixels, which are then used as metrics for the random walk algorithm, thereby expanding the CAM. The IRN method~\cite{ahn2019weakly} further refines CAM by estimating object boundary information through the semantic affinities between the original pixels in the image. Wang et al.~[\citeyear{wang2020weakly}] propose a method that refines CAM by leveraging high-confidence pixels from segmentation results as inputs to a pairwise affinity network. Xu et al.~[\citeyear{xu2021leveraging}] highlight that affinities in saliency maps and segmented representations more effectively reflect the diversity in CAM representations. For WSSS tasks, such refinement is crucial for improving the final segmentation performance and helps generate more accurate and reliable pixel-level pseudo-labels.

\section{Methodology}

    
    
    \subsection{Preliminaries}

    For ViT-based multi-classification networks, an input image is first split into \( N \times N \) patches through a convolutional layer or a fully connected layer, which are then converted into \( N ^2 \) patch tokens \(\left\{ t_n \in \mathbb{R}^{1 \times D}, \, n = 1, 2, \dots, N^2 \right\}\), where \( D \) represents the embedding dimension of each token.  Inspired by MCTformer~\cite{xu2022multi}, among these patch tokens, \( C \) class tokens \( t_* \in \mathbb{R}^{C \times D} \) is concatenated, where \( C \) represents the number of classes, and \( t_* \) is a learnable parameter initialized randomly. After adding positional encoding, these tokens \( \Gamma \in \mathbb{R}^{ (N^2 + C) \times D} \) are fed into \( L \) cascaded encoders of the standard Transformer modules, where \( \Gamma \) represents the combination of \( t_n \) and \( t_* \). Specifically, tokens \( \Gamma \) are projected by employing three learnable parameter matrices \( W^Q \), \( W^K \), \( W^V \), which map them into query \( Q\in \mathbb{R}^{ (N^2 + C) \times d_q} \), key \( K \in \mathbb{R}^{ (N^2 + C) \times d_k} \), and value \( V \in \mathbb{R}^{ (N^2 + C) \times d_v} \). Here, \( d_q \) = \( d_k \), denotes the feature dimension for the query and key, while \( d_v \) indicates the feature dimension of the value. Subsequently, the scaled dot-product attention mechanism is applied to compute the attention matrix \( A \) between the query and key:

    \begin{equation}
    S_i = \Gamma_i W^{s}_i, \quad S \in \{Q, K, V\}, \text{and} \quad A_i = \frac{Q_i K_i^\top}{\sqrt{d_k}}
    \end{equation}

    \noindent where \( i \in \{1, 2, \dots, L\} \) denotes the \( i \)-th transformer block. The softmax function is applied to matrix \( A_i \), which is then multiplied by the key \( V_i \), after which a residual connection is added, followed by Layer Normalization and input into the MLP to produce output \( X_i \).

    \begin{equation}
    X_i = \text{MLP}(\text{LN}(\text{softmax}(A_i) V_i+ X_{i-1})) 
    \end{equation}
    
    \noindent where \( X_i \) denotes the output of the \( i \)-th transformer block. When \( i \) = \( L \), class tokens \( \Gamma_{cls} \in \mathbb{R}^{ C \times D } \) are extracted from \( X_i \) and averaged across channels to yield category scores \( y_{cls} \in \mathbb{R}^{ C } \). Subsequently, these category scores are matched with image-level ground truth labels via multi-label soft margin loss (\(MLSM \)), and the loss is calculated to enable supervised learning: 
    
    \begin{align}
    \mathcal{L}_{cls} &= \text{MLSM}(\mathbf{y}_{cls}, \mathbf{y}) \nonumber\\
    &= - \frac{1}{C} \sum_{i=1}^{C} y^i \log \sigma(y^i_{cls}) \\
    &\quad + (1 - y^i) \log (1 - \sigma(y^i_{cls})) \nonumber
    \end{align}
    
    \subsection{Overall Pipeline}
    
    Inspired by the modeling of long-range dependencies at different levels by the encoder layers in the standard ViT, we explore whether multi-scale inter-object coupling interactions can be utilized to optimize input embeddings during forward propagation, thereby extending the semantic diversity of anchor class activation features in the high-level semantic space. The overall architecture of DOEI is shown in Figure~\ref{Figure 3}. We employ TS-CAM~\cite{gao2021ts}, TransCAM~\cite{li2023transcam}, and MCTformer~\cite{xu2022multi} as baseline models to assess the effectiveness of the proposed method. In each baseline network, we adopt a standard multi-stage WSSS process: 1) Generating Class Activation Maps (CAMs); 2) Refining the CAMs; 3) Training the segmentation network.

    \textbf{Generating CAMs.} We generate CAMs by training a multi-class classifier on the baseline network. Specifically, the feature map output \( F \in \mathbb{R}^{ h \times w \times d} \) by the baseline network is weighted and summed with the weight matrix \( W \) of the classifier’s final layer to extract the corresponding class-specific feature map \( M^c \), followed by normalization. Finally, the CAM is generated by calculating the contrast threshold \( \beta \) \( (0 < \beta < 1) \) relative to the background.

    \begin{equation}
    M_{tmp}^c = \text{ReLu}\left( \sum_{i=1}^{d} W^{i,c} F^i \right) 
    \end{equation}
    
    \begin{equation}
    M^{i,j} = \begin{cases}
    \arg\max(M^{i,j,:}), & \text{if } \max(M_{tmp}^{i,j,:}) \geq \beta, \\
    0, & \text{if } \max(M_{tmp}^{i,j,:}) \leq \beta
    \end{cases}
    \end{equation}
    
    \noindent where ReLu is a nonlinear activation function used to filter out negative values in \( M_{tmp}^c \). Min-Max normalization is applied to scale \( M_{tmp}^c \) to [0, 1]. The pixel (\( i \), \( j \)) represents a specific location in the feature map.
    
    \textbf{Refining CAMs.} We employed the same method as the baseline network, utilizing the AffinityNet~\cite{ahn2018learning} introduced by Ahn et al. to refine the initial seeds. AffinityNet learns reliable affinities between pixels from the initial localization map as a supervisory signal and predicts an affinity matrix to enable stable expansion of seed regions, thereby generating pseudo-mask labels.
    
    \textbf{Training Segmentation Network.} Building on previous mainstream research~\cite{zhang2021complementary} and the baseline network configuration, we selected DeepLabv1~\cite{chen2014semantic} with a ResNet38~\cite{wu2019wider} backbone as the segmentation network. The segmentation network uses pseudo-mask labels as supervisory signals to perform regression prediction for each pixel. The loss function is given as: 
    
    \begin{equation}
    \mathcal{L}_{\text{seg}} = \frac{1}{N} \sum_{i=1}^{N} \sum_{j=1}^{C} -\hat{y}_i^j \log \left( \frac{\exp(z_i^j)}{\sum_{k=1}^{C} \exp(z_i^k)} \right) 
    \end{equation}
    
    \noindent where \( N \) represents the total number of pixels in the image; \( i \) denotes the index of each pixel, ranging from 1 to \( N \);and \( j \) represents the class index, ranging from 1 to \( C \). \(\hat{y}_i^j\) denotes the ground truth label for pixel \( i \), otherwise 0 (i.e., one-hot encoding).\( z_i^j \) denotes the logit value of the network output for pixel \( i \) belonging to class \( j \).
    
    \subsection{Dual Optimization of Embedding Information}

    Given the importance of high-quality CAMs for WSSS, this paper focuses on CAMs generation. Previous research has mainly focused on extracting key classification features from the image, frequently overlooking the importance of guiding information generated through the interaction between semantic content and image features. Therefore, we propose a dual optimization mechanism for embedding information (DOEI) in ViT to fully utilize the interaction between semantic content and image features. The proposed DOEI consists of two independent modules: Class-wise Progressive Decoupling Optimization (PPDO) and Patch-wise Progressive Decoupling Optimization (CPDO), which are used to optimize patch-to-class and class-to-class interactions, respectively.

    \textbf{Patch-wise Progressive Decoupling Optimization.} Traditional methods typically generate a weight matrix \( A_i \) through the dot-product self-attention mechanism of class tokens and patch tokens in the \( i \)-th layer, and multiply it by the key generated from the embedding of the input in the \( (i-1) \)-th layer to obtain the input embedding for the \( (i+1) \)-th layer. The \( S_{p2c}^i \), where \( S_{p2c}^{i,x} = A_i[1 : C, C + 1 : C + N^2] \in \mathbb{R}^{ N^2 \times C} \) represents the attention score of patch-to-class, and \( S_{p2c}^{i,y} = A_i[C + 1 : C + N^2, 1 : C + 1] \in \mathbb{R}^{ N^2 \times C} \) represents the attention score of class-to-patch, with \(x\) and \(y\) used here to denote the two different forms of \( S_{p2c}^{i}\). These scores reflects the degree of relevance of the image information in the patch to that class. In other words, the \( S_{p2c} \) indicates the most likely class affiliation for that patch. Unreliable \( S_{p2c} \) (e.g., when a patch lacks relevant positional information for the class) can result in the creation of numerous false-positive pixels in the CAM during subsequent embedding transmission, causing incorrect class activation. Furthermore, assigning greater weight to high-confidence \( S_{p2c} \) can amplify their influence in subsequent embeddings, thereby guiding the model to focus more on specific categories. We partitioned all patch-to-class attention scores \( S_{p2c} \) into Patch Candidate Confidence Score (\(PC_{cs}\)) and Patch Candidate Non-confidence Scores (\(PC_{ns}\)):

    \begin{equation}
    \mathcal{P} = S_{p2c} = S_{p2c}^{x} + (S_{p2c}^{y})^T
    \end{equation}

    \begin{equation}
    PC_{cs} = \bigcup_{i=1}^{m} \left\{ a_{i,j} \mid j \in \operatorname{TopIndices}\left( \mathcal{P}_{i,j}, t \right) \right\}
    \end{equation}

    \begin{equation}
    PC_{ns} = \bigcup_{i=1}^{m} \left\{ b_{i,j} \mid j \in \operatorname{BottomIndices}(\mathcal{P}_{i,j}, 1-t) \right\}
    \end{equation}

    \noindent denote \(m\) and \(n\) as the dimensions of \(\mathcal{P}\), where \(m\) indicates the number of classes and \(n\) indicates the number of patches. \(t\) is calculated by the following formula:

    \begin{equation}
    t = n*(L - i)*ST_{p2c}       
    \end{equation}

    \noindent where \( i \in \{1, 2, \dots, L\} \) represents the \( l \)-th transformer block and \(L\) represents the total number of transformer layers. The selective threhold (\(ST_{p2c}\)) represents a hyperparameter used to control the division threshold. We consider that each layer of the self-attention mechanism learns image information at different scales. Therefore, we employed a progressive strategy that selects different numbers of confidence and non-confidence scores based on the features of each layer.

    We apply enhancement and suppression operations to \(PC_{cs}\) and \(PC_{ns}\), respectively, calculate the optimized embedding vectors, and combine them with the original embedding vectors with residual connections:

    \begin{align}
    X^p_i &= X_i[m:,:] \nonumber + X_i[n:m,:] \times A_i[PC_{cs}] * AF_{p2c} \\
        &+ X_i[m:,:] \times A_i[PC_{ns}] * SF_{p2c}   
    \end{align}

    \noindent where the Augment Factor (\(AF_{p2c}\)) and Suppression Factor (\(SF_{p2c}\)) are controllable hyper-parameters.

    \textbf{Class-wise Progressive Decoupling Optimization.} Class-to-class attention scores capture the similarity among categories. In multi-object, complex scenes, where the model relies solely on image-level annotations, it often struggles to accurately capture target location information, making it difficult to align these locations with semantic data. To address this challenge, CPDO is introduced to establish complementary similarities between categories. Specifically, the location information of one category can be supplemented by the semantic information of other similar categories, thereby enhancing the model's ability to perceive target locations and reinforcing the correspondence between semantic content and spatial positions. Similarity, we partitioned all class-to-class attention scores \( S_{c2c} \) into Class Candidate Confidence Score \(CC_{cs}\) and Class Candidate Non-confidence Score \(CC_{us}\):Similarly, we divide all class-to-class attention scores \( S_{c2c} \) into candidate confidence scores \(CC_{cs}\) and non-confidence scores \(CC_{us}\),then apply enhancement and suppression operations to obtain the optimized embedding vectors \(X^c_i\):

    \begin{equation}
    \mathcal{C} = S_{c2c} = S_{c2c}^{x} + (S_{c2c}^{y})^T
    \end{equation}

    \begin{equation}
    C_{cs} = \bigcup_{i=1}^{n} \left\{ a_{i,j} \mid j \in \operatorname{TopIndices}\left( \mathcal{C}_{i,j}, t \right) \right\}
    \end{equation}

    \begin{equation}
    C_{ns} = \bigcup_{i=1}^{n} \left\{ b_{i,j} \mid j \in \operatorname{BottomIndices}(\mathcal{C}_{i,j}, 1-t) \right\}
    \end{equation}

    \begin{equation}
    t = n*(L - i)*ST_{c2c}       
    \end{equation}

    \begin{align}
    X^c_i &= X_i[n:m,:] \nonumber + X_i[n:m,:] \times A_i[CC_{cs}] * AF_{c2c} \\
        &+ X_i[n:m,:] \times A_i[CC_{ns}] * SF_{c2c}   
    \end{align}

    Finally, \(X^p_i\) and \(X^c_i\) are concatenated to form a new embedding vector \(X_{i+1}\), which is then fed as input to the next layer of the self-attention mechanism.
    
    \subsection{Hybrid Feature Alignment Module}

    The self-attention scores, which fully represent image feature information, provide a reliable foundation for embedding optimization in DOEI. Relying solely on the attention weight matrix generated by the ViT encoder as a source of potential information for embedding optimization is insufficient to fully capture the high-dimensional features of objects anchored in the image. Therefore, to comprehensively capture the latent semantic and structural features within the image, the process of selecting feature information needs to be re-evaluated to encompass a broader range of rich content. This paper introduces a novel feature information construction method that effectively extracts key information from images by integrating self-attention weights, raw RGB values, and features from the ViT patch embedding layer. The specific construction method is as follows:

    \begin{equation}
    A = (1 - \alpha) \cdot W_{\text{attn}} + \alpha \cdot d_{\text{rgb}}(x) \cdot d_{\text{emb}}(X) 
    \end{equation}

    \noindent here, \( W_{attn}\) represents the self-attention weight matrix generated by the ViT model. Denote \( d_{rgb}(x) \) and \( d_{emb}(X) \) as the normalized pure RGB features of the original image and the cosine similarity based on Transformer embedding features, respectively.
    
\section{Experiments}

    \subsection{Datasets and Evaluation Protocol} The datasets employed in the experiments -- PASCAL VOC 2012~\cite{everingham2010pascal} and MS COCO 2014~\cite{lin2014microsoft} -- along with a detailed description of the evaluation metrics, are provided in \textbf{Supplementary Section 1}. 


    
    \subsection{Implementation Details}

    We integrate the proposed method in three baseline networks (\(i.e\)., MCTformer~\cite{xu2022multi}, TS-CAM~\cite{gao2021ts}, TransCAM~\cite{li2023transcam}). All classification networks for generating CAMs use the parameter settings of the original baseline. Due to differences in network architecture, the Patch-wise Progressive Decoupling Optimization (PPDO) module is only applied in TransCAM and TS-CAM. We draw on previous work~\cite{xu2022multi} and apply PSA~\cite{ahn2018learning} to the generated initial CAMs to obtain pseudo-mask labels (Mask). The final semantic segmentation network uses DeepLabV1~\cite{chen2014semantic} based on ResNet38~\cite{wu2019wider}, with parameter settings consistent with the compared baseline methods. In testing, we evaluate the model performance with inputs at multiple scales (0.5, 0.75, 1.0, 1.25, 1.5) and apply DenseCRFs~\cite{chen2014semantic} post-processing to the output results.
    
    \subsection{Main Results}

    \textbf{Pascal VOC.} In Table ~\ref{Table 1}, we demonstrate the improvement in seed object localization maps after incorporating the DOEI mechanism into baseline methods, and compare these results with those of classical WSSS methods. As shown in Table ~\ref{Table 1}, the baseline models with DOEI achieved improvements of 6.8\%, 1.4\%, and 3.8\% on seed, and enhancements of 4.7\%, 1.1\%, and 3.7\% on pseudo-mask labels (mask) after PSA processing. Additionally, Table  ~\ref{Table 2} presents the performance of segmentation models trained with pseudo-labels generated by the baseline models incorporating DOEI on the Pascal VOC validation and test sets. The results show that the proposed mechanism significantly improved the baseline models, with performance increases of 4.1\%, 0.9\%, and 0.9\%, respectively.

    \begin{table}[H]
    \fontsize{8pt}{10pt}\selectfont
    \centering
    \begin{tabular}{lcc}
        \toprule
        Method  & Seed & Mask \\
        \midrule
        AdvCAM~[\citeyear{lee2021anti}] & 55.6 & 69.9 \\
        CDA~[\citeyear{su2021context}] &  55.4  & 63.4\\
        CPN~[\citeyear{zhang2021complementary}] & 57.4 & 67.8\\
        ReCAM~[\citeyear{chen2022class}] & 54.8 & 69.7 \\
        CLIMS~[\citeyear{xie2022clims}] & 56.6 & 70.5 \\
        FPR~[\citeyear{chen2023fpr}] & 60.3 & - \\
        SFC~[\citeyear{zhao2024sfc}] & 64.7 & 73.7 \\
        \midrule
        TS-CAM~[\citeyear{gao2021ts}] & 29.9 & $41.4^{\star}$ \\
        \rowcolor{lightblue} DOEI+TS-CAM & \textbf{36.7} \textcolor{red}{↑ \textbf{6.8}}& \textbf{45.5} \textcolor{red}{↑ \textbf{4.1}} \\
        \cdashline{1-3}
        TransCAM~[\citeyear{li2023transcam}]  & 64.9 & 70.2\\
        \rowcolor{lightblue} DOEI+TransCAM & \textbf{66.3} \textcolor{red}{↑ \textbf{1.4}}& \textbf{71.3} \textcolor{red}{↑ \textbf{1.1}}\\
        \cdashline{1-3}
        MCTformer~[\citeyear{xu2022multi}] & 61.7 & 69.1 \\
        \rowcolor{lightblue} DOEI+MCTformer  & \textbf{65.5} \textcolor{red}{↑ \textbf{3.8}}& \textbf{72.2} \textcolor{red}{↑ \textbf{3.7}}\\
        \bottomrule
    \end{tabular}
    \caption{Evaluation of the initial seed (Seed) and its corresponding pseudo segmentation ground-truth mask (Mask) is conducted through mIoU (\%) on the PASCAL VOC \(train\) set. \(\star\) denotes our reproduced result.}
    \label{Table 1}
    \end{table}
    
    \begin{table}[H]
    \fontsize{8pt}{10pt}\selectfont
    \centering
    \begin{tabular}{lccc}
        \toprule
        Method  & Backbone & Val & Test \\
        \midrule
        SEAM~[\citeyear{wang2020self}]  & ResNet101 & 64.5 & 65.7\\
        AuxSegNet~[\citeyear{xu2021leveraging}] & ResNet38 & 69.0 & 68.6\\
        EPS~[\citeyear{lee2021railroad}] & ResNet101 & 71.0 & 71.8\\
        ECS-Net~[\citeyear{sun2021ecs}] & ResNet38 & 64.3 & 65.3\\
        AdvCAM~[\citeyear{lee2021anti}] & ResNet101 & 68.1 & 68.0\\
        CDA~[\citeyear{sun2021ecs}] & ResNet38 & 66.1 & 66.8\\
        Spatial-BCE~[\citeyear{wu2022adaptive}] & ResNet38 & 70.0 & 71.3\\
        ReCAM~[\citeyear{chen2022class}] & ResNet101 & 69.5 & 69.6\\
        SIPE~[\citeyear{chen2022self}] & ResNet101 & 68.8 & 69.7 \\
        CLIMS~[\citeyear{xie2022clims}] & ResNet101 & 70.4 & 70.0\\
        LPCAM~[\citeyear{chen2023extracting}] & ResNet101 & 71.8 & 72.1\\
        CLIP-ES~[\citeyear{lin2023clip}] & ResNet101 & 73.8 & 73.9\\
        \midrule
        TS-CAM~[\citeyear{gao2021ts}] & ResNet38 & $43.2^{\star}$ & $42.9^{\star}$ \\
        \rowcolor{lightblue} DOEI+TS-CAM & ResNet38 & \textbf{46.8} \textcolor{red}{↑ \textbf{3.6}}& \textbf{47.0} \textcolor{red}{↑ \textbf{4.1}} \\
        \cdashline{1-3}
        TransCAM~[\citeyear{li2023transcam}]  & ResNet38 & 69.3 & 69.6\\
        \rowcolor{lightblue} DOEI+TransCAM & ResNet38 & \textbf{70.8} \textcolor{red}{↑ \textbf{1.5}}& \textbf{70.5} \textcolor{red}{↑ \textbf{0.9}}\\
        \cdashline{1-3}
        MCTformer~[\citeyear{xu2022multi}] & ResNet38 & 70.2\textcolor{black}{†} & 70.1\textcolor{black}{†} \\
        \rowcolor{lightblue} DOEI+MCTformer  & ResNet38 & \textbf{71.4} \textcolor{red}{↑ \textbf{1.2}}& \textbf{71.0} \textcolor{red}{↑ \textbf{0.9}}\\
        \bottomrule
    \end{tabular}
    \caption{Performance comparison of WSSS methods in terms of mIoU (\%) on the PASCAL VOC \(val\) and \(test\) split through different segmentation backbones. †: Our reimplemented results by means of official code. Note that TS-CAM doesn’t provide official evaluation results on PASCAL VOC dataset, the results of TS-CAM (\(\star\)) are derived by us.}
    \label{Table 2}
    \end{table}
    
    \textbf{MS COCO.} Table~\ref{Table 3} reports the results of our DOEI compared to previous methods on MS COCO \(val\) split. MS COCO is a complex dataset containing 80 classification categories, covering a wide range of objects from everyday life. Our method achieved performance improvements of 1.2\% and 1.6\% over the baseline model, respectively.

    \subsection{Ablation Study and Parameter Analysis}

    We perform ablation and parameter experiments on the PASCAL VOC \(train\) set to assess the effectiveness of the proposed method. All experiments are conducted via MCTformer~\cite{xu2022multi}, with DeiT-S serving as the backbone network.

    \begin{table}[H]
    \fontsize{8pt}{10pt}\selectfont
    \centering
    \begin{tabular}{lcc}
        \toprule
        Method  & Backbone & Val \\
        \midrule
        CONTA~[\citeyear{zhang2020causal}] & ResNet38 & 32.8\\
        AuxSegNet~[\citeyear{xu2021leveraging}] & ResNet38 & 33.9 \\
        CDA~[\citeyear{sun2021ecs}] & ResNet38  & 33.2\\
        EPS~[\citeyear{lee2021railroad}] & ResNet101 & 35.7\\
        ReCAM~[\citeyear{chen2022class}] & ResNet38  & 42.9\\
        SIPE~[\citeyear{chen2022self}] & ResNet101 & 43.6\\
        \midrule
        TransCAM~[\citeyear{li2023transcam}]  & ResNet38 & $43.5^{\star}$ \\
        \rowcolor{lightblue} DOEI+TransCAM & ResNet38 & \textbf{44.7} \textcolor{red}{↑ \textbf{1.2}}\\
        \cdashline{1-3}
        MCTformer~[\citeyear{xu2022multi}] & ResNet38 & 42.0 \\
        \rowcolor{lightblue} DOEI+MCTformer  & ResNet38 & \textbf{43.6} \textcolor{red}{\textbf{↑ 1.6}}\\
        \bottomrule
    \end{tabular}
    \caption{Performance comparison of WSSS methods in terms of mIoU (\%) on the MS COCO \(val\) set. Note that TransCAM doesn’t provide official evaluation results on MS COCO dataset, the results of TransCAM (\(\star\)) are implemented by us.}
    \label{Table 3}
    \end{table}

    \textbf{The Effectiveness of the Components.} Table~\ref{Table 4} presents the performance improvements contributed by each component of the Dual Optimization of Embedding Information (DOEI) mechanism and the Hybrid-Feature Alignment (HFA) module. Introducing either Patch-wise Progressive Decoupling Optimization (PPDO) or Class-wise Progressive Decoupling Optimization (CPDO) individually increases the model's seed accuracy from 61.7\% to 64.2\% and 64.3\%, respectively. Combining PPDO and CPDO to form the DOEI mechanism yields a significant performance improvement, achieving an mIoU of 65.1\%. Further integrating DOEI with HFA results in an even greater performance boost, with an mIoU of 65.5\%. These results highlight the DOEI method's ability to enhance the global representation of target objects while effectively suppressing interference from non-target objects. Additionally, the findings confirm the effectiveness of the HFA module in addressing the limitations of relying solely on attention scores as features, thereby enhancing overall model performance.

    \begin{table}[H]
    \fontsize{8pt}{10pt}\selectfont
    \centering
    \begin{tabular}{ccccc}
        \toprule
        Baseline & CPDO & PPDO & HFA & mIoU \\
        \midrule
        \checkmark &  &  &  & 61.7 \\
        \checkmark & \checkmark &  &  & 63.3 \\
        \checkmark &  & \checkmark &  & 63.2 \\
        \checkmark & \checkmark & \checkmark &  & 64.3 \\
        \checkmark & \checkmark &  & \checkmark & 64.6 \\
        \checkmark &  & \checkmark & \checkmark & 64.9 \\
        \checkmark & \checkmark & \checkmark & \checkmark & \textbf{65.5} \\
        \bottomrule
    \end{tabular}
    \caption{Performance improvements from different optimization mechanisms in the mIoU (\%) evaluation metric on the PASCAL VOC \(train\) set.}
    \label{Table 4}
    \end{table}
    
    \textbf{Hyper-parameters in PPDO and CPDO.} Figure~\ref{Figure 4} shows the results of analyzing each of the following parameters individually, while keeping the other hyper-parameters fixed: selection threshold, amplification factor, and suppress-factor for both patch-to-class and class-to-class (\( AF_{p2c} \), \( SF_{p2c} \) and \( ST_{c2c} \), \( AF_{c2c} \), \( ST_{p2c} \)). These hyper-parameters are used to generate the optimal CAM configuration. The experimental results show that when these parameter values fall within a certain range, the mIoU achieves better values. As shown in Figure ~\ref{Figure 4}(a) and (c), by amplifying the patch information most likely to belong to a specific class (as learned by the model), the model improves its attention to that class, expanding the activation range; conversely, by suppressing the information least likely to belong to that class, the model reduces its focus on irrelevant regions, effectively preventing false activations. Additionally, class-to-class attention scores reflect the similarity of different semantic features. By leveraging the commonalities of similar semantics within the image, the model can selectively activate complementary semantic information, even in the absence of explicit class guidance. When an appropriately selected value for \( ST_{c2c} \), \( AF_{c2c} \) and \( SF_{c2c} \) are used, the model's embedding information can be optimized toward the correct semantic direction, thereby improving its performance and accuracy, as shown in Figure~\ref{Figure 4}(b) and (d).

    
    
    
    

    
    \textbf{Additional Parameter Analysis.} To verify the effectiveness of the proposed method, we also conduct experiments to analyze the parameters of the Hybrid Feature Alignment weight \( \alpha \) and the selection of embedding optimization layers. When the \( \alpha \) value falls within a specific range, the mIoU value significantly improves. Furthermore, as the number of optimization layers increases, the mIoU value reaches its maximum when the layer count is at its maximum, achieving optimal performance. The detailed experimental results are provided in \textbf{Supplementary Section 2}.
    
    

    
    \begin{figure}[H]
    \centering
    \includegraphics[width=\columnwidth]{figures/IJCAI-fig4.pdf}  
    \caption{The impact of different hyper-parameter values in the CPDO and PPDO modules on mIoU.}
    \label{Figure 4}
    \end{figure}
    
    \subsection{Qualitative Analysis}

    We provide qualitative comparison results of the baseline network (i.e., TS-CAM, TransCAM and MCTformer) with DOEI on representative examples from the PASCAL VOC and MS COCO datasets in \textbf{Supplementary Figure 2}. These comparisons provide a clear visual demonstration of the substantial improvements achieved by our method, particularly in enhancing the target localization accuracy of the baseline models. By refining embedded information, DOEI surpasses baseline approaches, demonstrating its ability to deliver more precise localization across diverse, challenging datasets.


    
    
\section{Conclusion}


    In this paper, we investigate the optimization of input embeddings by coupling class tokens and patch tokens across multiple scales of self-attention, thus enriching the diversity of anchored class activation features. We propose a Dual Optimization of Embedding Information mechanism (DOEI) that integrates seamlessly with the ViT. DOEI leverages coupled attention within the self-attention mechanism to amplify the semantic information of specific categories and suppress noise during forward propagation, effectively reconstructing the semantic representation of embeddings. Additionally, we construct hybrid feature representations by combining the RGB values of the image, embedding features, and self-attention scores, thereby enhancing the reliability of candidate tokens within the DOEI mechanism. Experimental results show that the baseline models with DOEI successfully alleviate over-activation and under-activation issues on the Pascal VOC and MS COCO datasets, greatly improving the quality of class activation maps and boosting semantic segmentation performance.
    
\section*{Ethical Statement}

There are no ethical issues.
%%%% ijcai24.tex

\typeout{IJCAI--24 Instructions for Authors}

% These are the instructions for authors for IJCAI-24.

\documentclass{article}
\pdfpagewidth=8.5in
\pdfpageheight=11in

% The file ijcai24.sty is a copy from ijcai22.sty
% The file ijcai22.sty is NOT the same as previous years'
\usepackage{ijcai24}

% Use the postscript times font!
\usepackage{times}
\usepackage{soul}
\usepackage{url}
\usepackage[hidelinks]{hyperref}
\usepackage[utf8]{inputenc}
\usepackage[small]{caption}
\usepackage{graphicx}
\usepackage{amsmath}
\usepackage{amsthm}
\usepackage{booktabs}
% \usepackage{algorithm}
% \usepackage{algorithmic}
\usepackage[switch]{lineno}
\usepackage{xcolor}
\usepackage{amssymb}

\usepackage{bbding}
\usepackage{fontawesome}
% \usepackage{tablefootnote}

\usepackage[linesnumbered,ruled,vlined]{algorithm2e}
\SetKwComment{Comment}{$\rhd$}{}
\AtBeginDocument{%
 \providecommand\BibTeX{{%
  \normalfont B\kern-0.5em{\scshape i\kern-0.25em b}\kern-0.8em\TeX}}}

\usepackage[most]{tcolorbox}

\newcommand{\citet}[1]{\citeauthor{#1}~\shortcite{#1}}
\newcommand{\red}{\textcolor{red}}
\newcommand{\blue}{\textcolor{blue}}
\newcommand{\green}{\textcolor{green}}
\newcommand{\etal}{\textit{et} \textit{al}.}
\newcommand{\eg}{\textit{e.g.}}
\newcommand{\ie}{\textit{i.e.}}
\newcommand{\fstar}{\textsuperscript{\fontsize{7pt}{7pt}\selectfont \faStarO}}
\newcommand{\fmoon}{\textsuperscript{\fontsize{7pt}{7pt}\selectfont \faMoonO}}

% \newcommand{\box1}{\colorbox{red}}
% \newcommand{\box2}{\colorbox{blue}}
% \newcommand{\box3}{\colorbox{green}}
% Comment out this line in the camera-ready submission
% \linenumbers

\urlstyle{same}

% the following package is optional:
%\usepackage{latexsym}

% See https://www.overleaf.com/learn/latex/theorems_and_proofs
% for a nice explanation of how to define new theorems, but keep
% in mind that the amsthm package is already included in this
% template and that you must *not* alter the styling.
\newtheorem{example}{Example}
\newtheorem{theorem}{Theorem}

% Following comment is from ijcai97-submit.tex:
% The preparation of these files was supported by Schlumberger Palo Alto
% Research, AT\&T Bell Laboratories, and Morgan Kaufmann Publishers.
% Shirley Jowell, of Morgan Kaufmann Publishers, and Peter F.
% Patel-Schneider, of AT\&T Bell Laboratories collaborated on their
% preparation.

% These instructions can be modified and used in other conferences as long
% as credit to the authors and supporting agencies is retained, this notice
% is not changed, and further modification or reuse is not restricted.
% Neither Shirley Jowell nor Peter F. Patel-Schneider can be listed as
% contacts for providing assistance without their prior permission.

% To use for other conferences, change references to files and the
% conference appropriate and use other authors, contacts, publishers, and
% organizations.
% Also change the deadline and address for returning papers and the length and
% page charge instructions.
% Put where the files are available in the appropriate places.


% PDF Info Is REQUIRED.

% Please leave this \pdfinfo block untouched both for the submission and
% Camera Ready Copy. Do not include Title and Author information in the pdfinfo section
\pdfinfo{
/TemplateVersion (IJCAI.2024.0)
}

\title{
% (Graph) Discrete/Quantized Representation Learning: Survey.  Vector Quantization
A Survey of Quantized Graph Representation Learning: Connecting\\ Graph Structures with Large Language Models
% and Its Integration with\\ Language Models
}


% Single author syntax
\author{
    Qika Lin\textsuperscript{$\heartsuit$}, Zhen Peng\textsuperscript{$\diamondsuit$}, Kaize Shi\fstar, Kai He\textsuperscript{$\heartsuit$}, Yiming Xu\textsuperscript{$\diamondsuit$},
    \\Erik Cambria\fmoon,
    Mengling Feng\textsuperscript{$\heartsuit$}
    \affiliations
\textsuperscript{$\heartsuit$}Saw Swee Hock School of Public Health, National University of Singapore\\
\textsuperscript{$\diamondsuit$}School of Computer Science and Technology, Xi'an Jiaotong University \\
\fstar School of Computer Science, University of Technology Sydney\\
\fmoon College of Computing and Data Science, Nanyang Technological University
    \emails
    qikalin@foxmail.com, cambria@ntu.edu.sg,
    ephfm@nus.edu.sg
}

% Multiple author syntax (remove the single-author syntax above and the \iffalse ... \fi here)
\iffalse
\author{
First Author$^1$
\and
Second Author$^2$\and
Third Author$^{2,3}$\And
Fourth Author$^4$\\
\affiliations
$^1$First Affiliation\\
$^2$Second Affiliation\\
$^3$Third Affiliation\\
$^4$Fourth Affiliation\\
\emails
\{first, second\}@example.com,
third@other.example.com,
fourth@example.com
}
\fi

\begin{document}

\maketitle

\begin{abstract}
    %The past five years or so: never use expressions like this as they are no longer relevant after a couple of years
    Recent years have witnessed rapid advances in graph representation learning, with the continuous embedding approach emerging as the dominant paradigm.
    However, such methods encounter issues regarding parameter efficiency, interpretability, and robustness.
    % However, this type of method would face the problems of parameter efficiency, explainability, and robustness.
    %Owing to this, in the recent few years, 
    Thus, Quantized Graph Representation (QGR) learning has recently gained increasing interest, which represents the graph structure with discrete codes instead of conventional continuous embeddings.
    % Quantized Graph Representation (QGR) learning has received more and more attention because of its merits, which represent the graph structure into discrete codes rather than continuous embedding. 
    % , including embedding parameter efficiency, explainability and interoperability, robustness and generalization, and seamless integration with NLP models.
    Given its analogous representation form to natural language, QGR also possesses the capability to seamlessly integrate graph structures with large language models (LLMs).
    % Besides, QGR holds the potentials for seamlessly integrating graph structures with advanced large language models (LLMs), as it has the similar representation form with natural language.
    As this emerging paradigm is still in its infancy yet holds significant promise, we undertake this thorough survey to promote its rapid future prosperity.
    % , aiming to give a current comprehensive picture of it and inspire future research.
    % Considering this type of technique is very promising and in a primary stage, we make a comprehensive review of QGR learning, from the perspectives of general framework, quantized methods, distinctive designs, and applications, hoping give a current comprehensive picture of this new paradigm.
    We first present the background of the general quantization methods and their merits.
    Moreover, we provide an in-depth demonstration of current QGR studies from the perspectives of quantized strategies, training objectives, distinctive designs, knowledge graph quantization, and applications.
    We further explore the strategies for code dependence learning and integration with LLMs.
    At last, we give discussions and conclude future directions, aiming to provide a comprehensive picture of QGR and inspire future research.
    % This type of method proposes a promising potential of comprehensively modeling graphs with advanced techniques of LLMs.
    % The relevant studies are constantly updated online at: \blue{github.com/}.
\end{abstract}

\section{Introduction}

Graph representation usually involves transforming nodes, edges, or structures within graph data to low-dimensional dense embeddings, which is called continuous graph representation as shown in Figure~\ref{fig_intro} (a).
The learned representation is obligated to preserve both node attributes and topological structure~\cite{DBLP:journals/corr/abs-2006-04131}.
% into a form that can be processed by computers, commonly achieved through vector or matrix representation.
% The essential aspect of this process is to preserve the original graph's structural information while compacting it into a lower dimensional space.
% whose key is to maintain the structural information of the original graph in a low-dimensional space.
% In recent years, the field of graph representation learning has experienced significant growth and advancement.
% The field of graph representation learning
This field has seen substantial development and expansion in recent years.
The emergence of numerous advanced technologies has invariably drawn significant attention within the graph community, becoming focal points of extensive studies, \eg, the well-known node2vec~\cite{DBLP:conf/kdd/GroverL16} model, graph neural networks (GNNs)~\cite{DBLP:conf/iclr/KipfW17}, and self-supervised graph learning~\cite{liu2022graph}.
They have achieved great empirical success in graph-related domains, including bioinformatics, social networks, recommendation systems, and anomaly detection~\cite{xia2021graph,DBLP:journals/tkde/LiuJPZZXY23}.

% Currently, continuous graph representation is the mainstream direction of graph learning, which is a way to represent graph data in a continuous space, transforming nodes, edges, and graph structures into continuous and real value vector representations.
% Continuous graph representation can retain rich information about the original graph data and identify relationships within it, applicable to a wide range of machine learning and deep learning tasks.
Although continuous graph representation successfully extracts rich semantic information from the input graph to adapt to a wide range of machine learning and deep learning tasks, concerns may arise regarding the embedding efficiency, interpretability, and robustness~\cite{wang2024learning}.
% However, there would be some issues about embedding efficiency, interpretability, and robustness.
Specifically, with the great growth of graph scales, there will be a linear increase in embedding parameters, becoming a considerable challenge in graph learning contexts involving millions or even more nodes.
Besides, continuous embeddings are generated in a black-box fashion and the meaning of the overall representation or any one dimension is unknowable, leading to a lack of interpretability.
Moreover, these continuous representations are typically task-specific and show no significant abstraction at the representation level, making them not robust enough to generalize to a variety of practical tasks as effectively as LLMs.
% making them insufficiently robust to be effectively generalized to other tasks.

% For example, as shown in Figure, a person represented by embedding is general and ambiguous for human understanding.
% The features of a person, like age (\eg, 18) and gender (male or female) are usually discrete, 
% Furthermore, when facing graph-text integration scenarios, fusing continuous embedding directly into NLP model would be at the cost of losing semantic representation.

\begin{figure}[t]
\centering
\includegraphics[width=0.8\linewidth]{intro2.png}
% \vspace{-0.3cm}
\setlength{\abovecaptionskip}{-0.05cm}
\setlength{\belowcaptionskip}{-0.3cm}
\caption{Illustration of different strategies for graph representations. (a) is the continuous graph representation. (b) is the quantized graph representation which represents the graph structure with discrete codes instead of conventional continuous embeddings.}
\label{fig_intro}
\end{figure}

% In reality, it is usually preferred to use discrete features for representation.
% instead of continuous embeddings.
Interestingly, in real life, discrete features are often preferred for representation.
For example, as shown in Figure~\ref{fig_intro} (b), descriptions affiliated to an individual are often a combination of discrete and meaningful features, such as age (\eg, \emph{18}), gender (\emph{male} or \emph{female}), and place of residence (\eg, \emph{Beijing}).
This scheme is simple and easy to understand.
% a person represented by embedding is general and ambiguous for human understanding.
% The features of a person, like age (\eg, 18) and gender (male or female) are usually discrete, 
% Furthermore, when facing graph-text integration scenarios, fusing continuous embedding directly into NLP model would be at the cost of losing semantic representation.
Based on this intuition, quantized representation came into being, which is now a hot topic in the computer vision domain that can facilitate image generation by combining with Transfomer or LLMs~\cite{esser2021taming,DBLP:conf/iclr/0004KCY24}.
% The famous XXX model all uses this technique.
Inspired by general quantized representation techniques, quantized graph representation (QGR) learning has recently been proposed to address the aforementioned shortcomings of continuous graph learning.
The core idea is to learn discrete codes for nodes, subgraphs, or the overall graph structures, instead of employing continuous embeddings.
As shown in Figure~\ref{fig_intro} (b), a node can be represented as discrete codes of (\emph{male}, \emph{18}, \emph{Beijing}, \emph{Ph.D.}).
In QGR methods, codes can be learned by differentiable optimization with specific graph targets or self-supervised graph learning.
It has the capacity to learn more high-level representations due to a compact latent space, which would enhance the representation efficiency, interpretability, and robustness.
% Moreover, when facing graph-text scenarios, the incorporation of continuous embeddings into advanced LLMs may present difficulties, potentially leading to a loss of semantic information.
Recall that when faced with graph-text scenarios, incorporating continuous embeddings into advanced LLMs is inherently challenging and prone to introducing the loss of semantic information. In contrast, due to the consistency with the discrete nature of natural language, the output of QGR would be seamlessly integrated into LLMs, like early-fusion strategy~\cite{DBLP:journals/corr/abs-2405-09818}.
% which holds great potential .
Considering that this emerging paradigm is still in its initial stage but carries substantial potential in the era of LLMs, we undertake this thorough survey.
It encompasses various viewpoints, including different graph types, quantized strategies, training objectives, distinctive designs, applications and so forth.
We also provide discussions and conclude future directions, intending to motivate future investigation and contribute to the advancement of the graph community.
% The research is in its early stages.
% In this paper, we give a comprehensive survey of related studies from multiple perspectives, expecting to bring awareness of QGR to the community and promote its further development.

To start with, we first briefly give the background
% including the merits of QGR and general quantization methods 
in Section 2. Then, we provide a comprehensive introduction to QGR frameworks from multiple perspectives
% of QGR framework, training objectives, distinctive designs, knowledge graph quantization, and application scenarios 
in Section 3.
Next, we present the code dependence learning as well as the integration with LLMs in Section 4 and Section 5, respectively.
Finally, we give the discussions and future directions in Section 6.

\section{Background}

\subsection{Merits of Quantized Graph Representation}

Compared to continuous representation methods, QGR offers a series of distinct advantages that fertilize its applications.

% \red{
% (1) significantly reduced memory requirements, (2) improved inference efficiency, (3) allowing Transformers to focus on long-range dependencies rather than local information, and (4) the capacity to learn more high-level representations due to a compact latent space~\cite{wang2024learning}
% }

\paragraph{Embedding Parameter Efficiency.}
As large-scale graphs become more prevalent, there is a corresponding substantial increase in the representation parameters, which necessitates significant memory usage for storage and substantial computational expenses~\cite{DBLP:conf/iclr/0001DWH22}.
Supposing $n$ as the number of nodes and $d$ as the dimensions, the required embedding parameters would be $n\times d$.
In the quantized settings, the total count of parameters would reduce to $n\times m+M\times d$, where $M$ and $m$ denote the length of the codeword set and the number of codes assigned to each node.
For example, if there are $10^6$ nodes with 1024 feature dimensions, 1024 codewords, and 8 assigned codes,
it would be a 113-fold reduction in required embedding parameters.

% Parameters needed is usually large, especially as large-scale networks become more prevalent.
% \red{the trained embeddings often require a significant amount of space to store, making storage and processing a challenge, especially as large-scale networks become more prevalent.
% Although quantisation and hashing methods introduce additional information loss, they significantly reduce storage footprint and retrieval time.
% }
% $n\times d ->M\times d, n\times M, n>>M d>M$
% \red{
% $10^6$ nodes and a feature dimension of 1024 m=3, resulting in a 270-fold reduction in required memory
% }

\paragraph{Explainability and Interpretability.}
The discrete nature of QGR brings the explainability and interpretability for embeddings and reasoning, as shown in Figure\ref{fig_intro} (b).
Each code would be assigned to the practical item by specific designs for direct interpretability, \eg, Dr. E~\cite{liu2024dr} introduces a language vocabulary as the code set and each node in graphs can be represented as the permutation of explainable language tokens.


\paragraph{Robustness and Generalization.}
The discrete tokens of the QGR are more robust and generalizable, which usually utilizes the self-supervised strategies to learn both local-level and high-level graph structures (\eg, edge reconstruction and DGI~\cite{DBLP:conf/iclr/VelickovicFHLBH19}) as well as semantics (\eg, feature reconstruction).
% which is more robust for downstream applications.
Moreover, the acquired tokens can readily adapt to downstream applications, either directly or indirectly~\cite{yang2024vqgraph}.
% \red{The tokens should be robust and generalizable. To achieve this, we rely on graph self-supervised learning. Self-supervised representations have been shown to be more robust to class imbalance (Liu et al., 2022) and distribution shift (Shi et al., 2023), while also capturing better semantic information (Assran et al., 2023) compared to representations learned through supervised objectives~\cite{wang2024learning}}




\paragraph{Seamless Integration with NLP Models.}
The swift advancement in natural language processing (NLP) techniques, particularly LLMs, has sparked an increased interest in employing NLP models to resolve graph tasks.
However, the inherent representation gap between the typical graph structure and natural language poses a significant challenge to their seamless and effective integration. Graph quantization, by learning discrete codes that bear resemblance to natural language forms, can facilitate this integration directly and seamlessly~\cite{lin2024self,liu2024dr}.



\subsection{General Quantization Methods}
\label{sec_quantization}
The mainstream quantization methods can be roughly categorized into product quantization, vector quantization, finite scalar quantization, and anchor selection and assignment.

\paragraph{Product quantization (PQ).}
The core idea of PQ~\cite{jegou2010product} is to divide a high-dimensional space into multiple low-dimensional subspaces and then quantize them independently within each subspace. Specifically, it splits a high-dimensional vector into multiple smaller subvectors and quantizes each subvector separately. In the quantization process, each subvector is mapped to a set of a finite number of center points by minimizing the quantization error.


% \red{The VQ-VAE model~\cite{DBLP:conf/nips/OordVK17} is originally proposed for modeling continuous data distribution, such as images, audio and video. It encodes observations into a sequence of discrete latent variables, and reconstructs the observations from these discrete variables. Both encoder and decoder use a shared codebook.
% }
% Taming transformers~\cite{esser2021taming} VQGAN:





\paragraph{Vector Quantization (VQ).}

To effectively learn quantized representations in a differentiable manner,
the VQ strategy~\cite{DBLP:conf/nips/OordVK17,esser2021taming} is proposed and has emerged as the predominant method in the field.
The core idea involves assigning the learned continuous representations to the codebook's index and employing the Straight-Through Estimator~\cite{DBLP:journals/corr/BengioLC13} for effective optimization.
The formulation is as follows.
$f_e$, $f_d$, $f_q$ denote the encoder, decoder, and quantization process, respectively.
$\mathbf{C}\in \mathbb{R}^{M\times d}$ is the codebook representation with dimension $d$, which corresponds to $M$ discrete codewords $\mathcal{C}=\{1, 2, \cdots, M\}$.
$\mathcal{G}=\{\mathcal{V}, \mathcal{E}, \mathbf{X}, \mathbf{Y}\}$ is the graph, where $\mathcal{V}$ is the node set with size $n$ and $\mathcal{E}$ is the edge set.
The presence of $\mathbf{X}$ or $\mathbf{Y}$ is not guaranteed, where $\mathbf{X}\in \mathbb{R}^{n\times d}$ is the input feature of a graph and $\mathbf{Y} = \{y_1, y_2, \cdots\}$ is the label set of each node.
% $\mathbf{X}\in \mathbb{R}^{n\times d}$ is the input feature of a sample graph with $n$ nodes, and the feature dimension is $d$.
% $\mathbf{X}\in \mathbb{R}^{n\times d}$ is the input feature of a simple.
$\widetilde{\mathbf{X}}\in \mathbb{R}^{n\times d}$ is the latent representation after encoder and $\widehat{\mathbf{X}}\in \mathbb{R}^{n\times d}$ is the reconstruction feature after quantization and decoder, \ie,
$\widetilde{\mathbf{X}}=f_e(\mathbf{X})$,
$\widehat{\mathbf{X}}=f_d(f_q(\widetilde{\mathbf{X}}))$.

Specifically, the VQ strategy yields a nearest-neighbor lookup between latent representation and prototype vectors in the codebook for quantized representations:
\begin{equation}
\small
\label{eq_vq}
f_q(\widetilde{\mathbf{X}}_i)=\mathbf{C}_{m},\;m=\underset{j\in \mathcal{C}}{\mathop{\arg\min}}\| \widetilde{\mathbf{X}}_i-\mathbf{C}_{j}\|_2^2,
\end{equation}
indicating that node $i$ is assigned codeword $c_m$.
Further, the model can be optimized by the Straight-Through Estimator:
\begin{equation}
\small
  \mathcal{L}_{vq}=\frac{1}{n}\sum_{i=1}^{n}\underbrace{\|\text{sg}[\widetilde{\mathbf{X}}_i]-\mathbf{C}_{m}\|}_{codebook}+
  \underbrace{\beta\|\text{sg}[\mathbf{C}_{m}]-\widetilde{\mathbf{X}}_i\|}_{commitment},
\end{equation}
where \text{sg} represents the stop-gradient operator. $\beta$ is a hyperparameter to balance and is usually set to 0.25.
In this loss, the first codebook loss encourages the codeword vectors to align closely with the encoder's output.
The second term of the commitment loss aids in stabilizing the training process by encouraging the encoder's output to adhere closely to the codebook vectors, thereby preventing excessive deviation.

Using VQ techniques, a continuous vector can be quantized to the counterpart of a discrete code.
However, there would be distinctions and differences in the vector pre and post-quantization.
To address it, Residual Vector Quantization (RVQ)~\cite{DBLP:conf/cvpr/LeeKKCH22} is proposed to constantly fit residuals caused by the quantization process.
The whole process can be seen in Algorithm~\ref{algorithm_rvq}.
RVQ is actually a multi-stage quantization method and usually has multiple codebooks.
The codebook stores the residual values of the quantized vector and the original vector at each step, and the original vector can be approximated infinitely by quantizing the residual values at multiple steps.

% \paragraph{Residual Vector Quantization (RVQ).}
% \red{Residual quantization is actually a multi-stage quantization method, and the algorithm flow is very simple. As shown in the figure below, the codebook stores the residual values of the quantized vector and the original vector at each step, and the original vector can be approximated infinitely by quantizing the residual values at multiple steps.
% Is to choose a CodeBook vector to fit Residual, but this fitting will still have errors, then fit this "fitting error", gradually recursive down.}


\begin{algorithm}[t]
% \small
 % \DontPrintSemicolon
  \KwIn{$\widetilde{\mathbf{X}}_i$ of the encoder, codebook representation $\mathbf{C}$.
  }
  \KwOut{Quantized representation $Q(\widetilde{\mathbf{X}}_i)$.}
  Init $Q(\widetilde{\mathbf{X}}_i)=0$, $residual=\widetilde{\mathbf{X}}_i$\;
  \For{l=1 {\rm to} N$_q$ {\rm of residual iterations}}
    {  
    $Q(\widetilde{\mathbf{X}}_i)\,+\!\!=f_q(residual)$\Comment*[r]{\textrm{ Eq. (\ref{eq_vq})}}
    $residual\,-\!\!= f_q(residual)$\Comment*[r]{\textrm{ Eq. (\ref{eq_vq})}}
    }
  \textbf{Return} $Q(\widetilde{\mathbf{X}}_i)$.
 \caption{Residual Vector Quantization (RVQ).}
 \label{algorithm_rvq}
\end{algorithm}


\paragraph{Finite Scalar Quantization (FSQ).}  
Because of the introduction of the codebook,
there would be complex technique designs and codebook collapse in VQ techniques.
Thus, FSQ~\cite{DBLP:conf/iclr/MentzerMAT24} introduces a very simple quantized strategy, \ie, directly \emph{rounding} to integers rather than explicitly introducing parameters of the codebook.
FSQ projects the hidden representations into a few dimensions (usually fewer than 10) and each dimension is quantized into a limited selection of predetermined values, which inherently formulates a codebook as a result of these set combinations.
It is formulated as:
\begin{equation}
\small
\text{FSQ}(\widetilde{\mathbf{X}}_i)=\mathcal{R}[(L-1)\sigma(\widetilde{\mathbf{X}}_i)]\in \{0,1,\cdots,L-1\}^d,
\end{equation}
where $\mathcal{R}$ is the rounding operation and $L\in \mathbb{N}$ is the number of unique values.
The process of rounding simply adjusts a scalar to its nearest integer, thereby executing quantization in every dimension.
$\sigma$ is a sigmoid or tanh function.
Its loss is calculated by also using the Straight-Through Estimator:
\begin{equation}
\small
\mathcal{L}_{fsq}=\frac{1}{n}\sum_{i=1}^{n}(L-1)\sigma(\widetilde{\mathbf{X}}_i)+\text{sg}[\mathcal{R}[(L-1)\sigma(\widetilde{\mathbf{X}}_i)]-(L-1)\sigma(\widetilde{\mathbf{X}}_i)].
\end{equation}
FSQ has the advantages of better performance, faster convergence and more stable training,
particularly when dealing with large codebook sizes.



% \red{FSQ projects the latent representations down to a few dimensions (typically less than 10). Each dimension is quantized to a small set of fixed values, leading to an (implicit) codebook given by the product of these sets.}
% \begin{equation}
% \small
% \text{FSQ}(\widetilde{\mathbf{X}}_i)=\mathcal{R}[(L-1)\sigma(\widetilde{\mathbf{X}}_i)]\in \{0,1,\cdots,L-1\}^d.
% \end{equation}


\paragraph{Anchor Selection and Assignment (ASA).} It is typically unsupervised and often employs prior strategies to identify informative nodes within the graph, \eg, the Personalized PageRank~\cite{page1999pagerank} and node degree.
Consequently, each node can be attributed to a combination of anchors that exhibit structural or semantic relationships,
like using the shortest path.
This form of quantization designates the code as a real node, unlike the aforementioned three methods.

\section{Quantized Graph Learning Framework}

\begin{figure*}[t]
\centering
\includegraphics[width=0.8\linewidth]{arc1.png}
% \vspace{-0.3cm}
\setlength{\abovecaptionskip}{-0.1cm}
\setlength{\belowcaptionskip}{-0.3cm}
\caption{The general framework of the QGR studies, which mainly comprises an encoder, decoder, and quantization process.
Training objectives of different levels can be utilized. 
By combining a predictor, multiple applications can be realized.}
\label{fig_arc}
\end{figure*}

We summarize the main studies of QGR in Table~\ref{tab_all_studies} from the primary perspectives of graph types, quantization methods, training strategies, and applications.
Additionally, we also highlight if there are in pipeline and self-supervised manners, as well as whether they learn code dependence and integrate with language models.

The general process of QGR is illustrated in Figure~\ref{fig_arc}, which comprises the encoder $f_e$, decoder $f_d$, and quantization process $f_q$.
The encoder $f_e$ is to model the original graph structure into latent space, where MLPs~\cite{he2020sneq} and GNNs~\cite{DBLP:conf/iclr/XiaZHG0LLL23} are usually utilized.
The decoder $f_d$ also commonly utilizes GNNs for structure embedding and the quantization process $f_q$ implements the strategies outlined in Section~\ref{sec_quantization}.
In particular, Bio2Token~\cite{liu2024bio2token} utilizes Mamba~\cite{DBLP:journals/corr/abs-2312-00752} as both encoder and decoder.
UniMoT~\cite{DBLP:journals/corr/abs-2408-00863} leverages the pre-trained MoleculeSTM molecule encoder~\cite{DBLP:journals/natmi/LiuNWLQLTXA23} to connect the molecule graph encoder and causal Q-Former~\cite{DBLP:conf/icml/0008LSH23}.
In the following sections, we will delve into the details of the model design.

\begin{table*}[t]
\setlength\tabcolsep{3pt} 
% \small
\caption{The summary of cutting-edge studies for QGR learning.
Abbreviations are as follows: HOG: homogeneous graph that includes attributed graph (AG), text-attributed graph (TAG), and molecular graph (MG), HEG: heterogeneous graph.
% multiple relations?
For training and applications, there are STC: structure connection, SEC: semantic connection, NC: node classification, NR: node recommendation, FR: feature reconstruction, LP: Link Prediction, PP: path prediction, GC: graph classification, GR: graph regression, GG: graph generation, MCM: masked code modeling, GCL: graph contrastive learning, NTP: next-token prediction, IC: invariant learning.
\faToggleOff\;and \faToggleOn\; indicate for one-stage or multi-stage learning, respectively.
% code learning in one step or multi-steps.
% method includes the information for supervised. or unsupervised.
In the training column, ``/''split the training stages.
% Self-Sup means the self-supervised training target and is task-agnostic.
    }
    \centering
    \resizebox{1.0\textwidth}{!}{
    \begin{tabular}{l|ccllllcc}
        \toprule
        Model  &one/p&Self-Sup& Graph Type &  Quantization & Training & Application&Dependence&w/ LM\\
        \midrule
        SNEQ~\shortcite{he2020sneq}&\faToggleOff&\Checkmark&AG (HOG)&PQ&STC, SEC& NC, LP, NR&\XSolidBrush&\XSolidBrush \\
        d-SNEQ~\shortcite{DBLP:journals/tnn/HeGSL23}&\faToggleOff&\Checkmark&AG (HOG)&PQ&STC, SEC& NC, LP, NR, PP&\XSolidBrush&\XSolidBrush \\
        Mole-BERT~\shortcite{DBLP:conf/iclr/XiaZHG0LLL23}&\faToggleOn&\Checkmark&MG (HOG)&VQ&FR/MCM, GCL&GC, GR&\Checkmark&\XSolidBrush\\
        iMoLD~\shortcite{DBLP:conf/nips/ZhuangZDBWLCC23}&\faToggleOff&\Checkmark&MG (HOG)&VQ&IL, GC (GR)&GC, GR&\XSolidBrush&\XSolidBrush\\
        VQGraph~\shortcite{yang2024vqgraph} &\faToggleOn&\Checkmark& HOG & VQ&FR, LP/NC, Distill&  NC&\XSolidBrush&\XSolidBrush\\
        Dr.E~\shortcite{liu2024dr}&\faToggleOn& \Checkmark & TAG (HOG)&RVQ&FR, LP, NC/FR, NTP&  NC&\Checkmark&\Checkmark  \\
        DGAE~\shortcite{DBLP:journals/tmlr/BogetGK24}&\faToggleOn&\Checkmark& HEG &VQ&LP/NTP&  GG&\Checkmark&\XSolidBrush\\
        LLPS~\shortcite{DBLP:journals/corr/abs-2405-15840}&\faToggleOff&\Checkmark& MG (HOG)&FSQ&Reconstruction&  GG&\XSolidBrush&\XSolidBrush\\
        GLAD~\shortcite{boget2024glad}         &\faToggleOn&\Checkmark& MG (HOG) &FSQ&FR+LP/Diffusion Bridge&  GG&\Checkmark&\XSolidBrush\\
        Bio2Token~\shortcite{liu2024bio2token}&\faToggleOff&\Checkmark&MG (HOG)&FSQ&FR&GG&\XSolidBrush&\XSolidBrush\\
        GQT~\shortcite{wang2024learning}&\faToggleOn&\Checkmark&HOG&RVQ&DGI, GraphMAE2/NC&NC&\XSolidBrush&\XSolidBrush\\
        NID~\shortcite{DBLP:journals/corr/abs-2405-16435}&\faToggleOn&\Checkmark&HOG&RVQ&GraphMAE (GraphCL)&NC, LP, GC, GR, etc&\XSolidBrush&\XSolidBrush\\
        UniMoT~\shortcite{DBLP:journals/corr/abs-2408-00863}&\faToggleOn&\Checkmark&MG (HOG)&VQ&FR/NTP&GC, GR, Captioning, etc.&\XSolidBrush&\Checkmark\\
        % MSPmol~\shortcite{DBLP:conf/ijcnn/LuPZC24}&\faToggleOn&\Checkmark&MOG&VQ&\red{NC, edge type, etc}&GC&\XSolidBrush&\XSolidBrush\\
         NodePiece~\shortcite{DBLP:conf/iclr/0001DWH22}&\faToggleOn&\XSolidBrush&KG&ASA&--/LP&LP&\XSolidBrush&\XSolidBrush\\
         EARL~\shortcite{DBLP:conf/aaai/ChenZYZGPC23}&\faToggleOn&\XSolidBrush&KG&ASA&--/LP&LP&\XSolidBrush&\XSolidBrush\\
         RandomEQ~\shortcite{DBLP:conf/emnlp/LiWLZM23}&\faToggleOn&\XSolidBrush&KG&ASA&--/LP&LP&\XSolidBrush&\XSolidBrush\\
         SSQR~\shortcite{lin2024self}&\faToggleOn&\Checkmark&KG&VQ&LP, Semantic distilling/NTP&LP, Triple Classification&\XSolidBrush&\Checkmark\\
        \bottomrule
    \end{tabular}
    }
    \label{tab_all_studies}
\end{table*}

 
% encoder and decoder

% encoder, MLP (SNEQ)  GNN(Mole-BERT)

% \red{Homophilous graphs are characterized by nodes with similar classes being connected to each other, whereas heterophilous graphs exhibit connections between nodes with different classes.}


% Proprocessing:
% UniMoT
% \red{We connect the molecule encoder and Causal Q-Former, leveraging the pretrained MoleculeSTM molecule encoder~\cite{DBLP:journals/natmi/LiuNWLQLTXA23}. The molecule encoder remains frozen while only the Causal Q-Former is updated.
% }
% \red{We connect the Causal Q-Former with subsequent blocks and use the objective defined in Equation (2). We employ the pretrained ChemFormer~\cite{DBLP:journals/mlst/IrwinDHB22} as the generative model. Specifically, we leverage the SMILES encoder and the SMILES decoder provided by ChemFormer.}


\subsection{Quantized Strategies}

Except for SNEQ~\cite{he2020sneq} and d-SNEQ~\cite{DBLP:journals/tnn/HeGSL23}, the majority of research tends to employ VQ-related strategies.
Beyond the general techniques in VQ,
Dr.E~\cite{liu2024dr} utilizes RVQ in one specific layer, which is called intra-layer residue.
For more effective in encoding the structural information of the central node, it involves preserving the multi-view of a graph.
Specifically, between GCN layers, the inter-layer residue is carried out to enhance the representations:
\begin{equation}
\small
\mathbf{h}_v^{l+1}=\sigma\big(\mathbf{W}\cdot\mathrm{Con}(\mathbf{h}_v^l+\mathrm{Pool}(\{\mathbf{c}_{v,k}^l\}_{k=1}^K),\mathbf{h}_{\mathcal{N}(v)}^l)\big),
\end{equation}
where $\mathbf{h}$ and $\mathbf{c}$ denote the latent representations and quantized ones.
$K$ is the number of learned codes for each layer.
Using this process, the discrete codes of each node are generated in a gradual (auto-regressive) manner, following two specific orders: from previous to subsequent, and from the bottom layer to the top.
This approach guarantees comprehensive multi-scale graph structure modeling and efficient code learning.
iMoLD~\cite{DBLP:conf/nips/ZhuangZDBWLCC23} proposes another type of \emph{Redidual} VQ, incorporating both the continuous and discrete representations, \ie, $\widetilde{\mathbf{X}}_i+f_q(\widetilde{\mathbf{X}}_i)$, as the final node representation.

GLAD~\cite{boget2024glad} and Bio2Token~\cite{liu2024bio2token} utilize the straightforward approach, FSQ, to facilitate the learning of discrete codes.
For KG quantization, early methods such as NodePiece~\cite{DBLP:conf/iclr/0001DWH22}, EARL~\cite{DBLP:conf/aaai/ChenZYZGPC23}, and RandomEQ~\cite{DBLP:conf/emnlp/LiWLZM23} have adopted the unsupervised learning approach ASA to achieve effective learning outcomes.
For more condensed code representations, SSQR~\cite{lin2024self} employs the VQ strategy to capture both the structures and semantics of KGs.


\subsection{Training Objectives}

In this section, we will delve deeper into training details from the perspectives of node, edge, and graph levels.

\paragraph{Node Level.}
Feature reconstruction usually serves as a simple self-supervised method for QGR learning, which is to compare the original node features with reconstructed features based on the learned codes.
It mainly has the following two implementation types:
\begin{equation}
\small
    \mathcal{L}_{fr}=\frac{1}{n}\sum_{i=1}^n\big(1-\frac{\mathbf{X}_i^{\top}\widehat{\mathbf{X}}_i}{\|\mathbf{X}_i\|\|\widehat{\mathbf{X}}_i\|}\big)^{\gamma},\mathcal{L}_{fr}=\frac{1}{n}\sum_{i=1}^n\|\mathbf{X}_i-\widehat{\mathbf{X}}_i\|^2,
\end{equation}
The first involves employing cosine similarity with scaled parameter $\gamma\geq 1$, as implemented in Mole-BERT and VQGraph.
The second is the mean square error as in Dr.E.
NID~\cite{DBLP:journals/corr/abs-2405-16435} utilizes GraphMAE~\cite{DBLP:conf/kdd/HouLCDYW022} for self-supervised learning, which can be viewed as a masked feature reconstruction.
It involves selecting a subset of nodes, masking the node features, encoding by a message-passing network, and subsequently reconstructing the masked features with a decoder.
 
\paragraph{Edge Level.}
It is a primary strategy to acquire the graph structures in QGR codes using a self-supervised paradigm.
The direct link prediction, \ie, edge reconstruction, is a widely adopted technique to assess the presence or absence of each edge before and after the reconstruction process as:
\begin{equation}
\small
\begin{aligned}
    &\mathcal{L}_{lp}=\|\textbf{A}-\sigma(\widehat{\mathbf{X}}\cdot \widehat{\mathbf{X}}^{\top}) \|^2,\\
    \mathcal{L}_{lp}=-\frac{1}{|\mathcal{E}|}&\sum_{j=1}^{|\mathcal{E}|}\big[y_j\log(\hat{y}_j)+(1-y_j)\log(1-\hat{y}_j)\big],\\
\end{aligned}
\end{equation}
where the mean square error (\eg, VQGraph) and the binary cross-entropy are utilized to optimize (\eg, Dr.E), respectively.
$\hat{y}_j$ is the prediction of probability for the existence of the edge based on the learned representations.
$y_j\in\{0,1\}$ is the ground label.

Beyond direct edge prediction, \citet{he2020sneq} proposes high-level connection prediction between two nodes, \ie, structural connection, and semantic connection.
% Structural connection between two nodes, which 
The former aims to accurately maintain the shortest distances $\delta$  within the representations:
\begin{equation}
\small
    \mathcal{L}_{stc}=\frac{1}{n}\sum_{(i,j,k)}^n \text{max}\big(D_{i,j}-D_{i,k}+\delta_{i,j}-\delta_{i,k},0\big),
\end{equation}
where $D$ is to calculate the distance of two nodes based on the learned codes.
% in the embedding space.
% $\delta$ indicates the shortest distance of two nodes. 
Considering that nodes sharing identical labels ought to be situated closer together within the embedding space,
semantic connection is proposed:
\begin{equation}
\small
    \mathcal{L}_{sec}=\frac{1}{n\cdot T}\sum_{i}^{n}\sum_{i,j=1}^{T}(D_{i,j}-S_{i,j})^2,
\end{equation}
where $T$ is the sample size. $S$ presents the constant semantic margin, which is set to 0 if the labels of two nodes are totally distinct, otherwise, it is set to a constant.
\paragraph{Graph Level.}
For high-level representations, the graph modeling targets are proposed.
For instance, graph contrastive learning is a widely utilized method for modeling relationships between graph pairs:
\begin{equation}
\small
\mathcal{L}_{gcl}=-\frac{1}{|\mathcal{D}|}\sum_{\mathcal{G}\in\mathcal{D}}\log\frac{e^{sim(\mathbf{h}_{g1},\mathbf{h}_{g2})/\tau}}{\sum_{\mathcal{G}'\in\mathcal{B}}e^{sim(\mathbf{h}_{g1},\mathbf{h}_{\mathcal{G}'})/\tau}},
\end{equation}
where $\mathcal{D}$ is the dataset that contains all graphs and $\mathbf{h}_{g}$ is the graph representation based on $f_q(\widetilde{\mathbf{X}})$.
$g1$ and $g2$ can be seen as positive samples for the target graph, which are usually transformed by the target graph $\mathcal{G}$.
For example, Mole-BERT~\cite{DBLP:conf/iclr/XiaZHG0LLL23} utilizes different masking ratios (\eg,  15\% and 30\%) for molecular graphs to obtain $g1$ and $g2$.
$\mathcal{B}$ is contrastive samples,  typically derived from the sampled batch.
GraphCL~\cite{DBLP:conf/nips/YouCSCWS20} is also a well-known framework of this kind based on graph augmentations, including node dropping, edge perturbation, attribute masking, and subgraph, which is used by NID~\cite{DBLP:journals/corr/abs-2405-16435}.

Furthermore,
motivated by a straightforward self-supervised learning framework that identifies varied augmentation views as similar positive pairs, iMoLD~\cite{DBLP:conf/nips/ZhuangZDBWLCC23} considers the learned invariant and global aspects (invariant plus spurious) as positive pairs, aiming to enhance their similarity.
An MLP-based predictor is introduced to transform the output of one view and align it with the other.
The whole process can be interpreted as \emph{invariant learning}.
Deep Graph Infomax (DGI)~\cite{DBLP:conf/iclr/VelickovicFHLBH19} can also be used as a high-level QGR learning method, which is a contrastive approach that compares local (node) representations with global (graph or sub-graph) ones.
% whereas GMAE2 combines generative and distillation objectives to jointly reconstruct masked features and track teacher representations.
GQT~\cite{wang2024learning} utilize both high-level DGI~\cite{DBLP:conf/iclr/VelickovicFHLBH19} and node-level GraphMAE2~\cite{DBLP:conf/www/HouHCLDK023} as training targets.
% GMAE2 combines generative and distillation objectives to jointly reconstruct masked features and track teacher representations.
% \red{GraphCL: Such approaches usually consider each sample as its own class, that is, a positive pair consists of two different views of it; and all other samples in a batch are used as the negative pairs during training.  Specifically, a minibatch of N graphs is randomly sampled and subjected to contrastive learning. This process results in 2N augmented graphs, along with a corresponding contrastive loss to be optimized.}




\subsection{Distinctive Designs}

\paragraph{Codebook Design.}
Beyond randomly setting the codebook for updating, there are several specifically designed strategies for domain knowledge adaption and generalization.
For example, Dr.E~\cite{liu2024dr} sets the vocabulary of LLaMA as the codebook, realizing token-level alignment between GNNs and LLM.
In this way, each node of graphs can be quantized to a permutation of language tokens and then can be directly input to LLMs to make predictions.
When modeling molecular graphs, Mole-BERT~\cite{DBLP:conf/iclr/XiaZHG0LLL23} categorizes the codebook embeddings into various groups, each representing a distinct type of atom. For instance, the quantized codes of carbon, nitrogen and oxygen are confined to different groups.


\paragraph{Training Pipelines.}
Aside from the single-stage learning process applied to QGR and particular tasks, some methods incorporate multiple stages into the training framework.
GQT~\cite{wang2024learning} modulate the learned codes through hierarchical encoding and structural gating, which are subsequently fed into the Transformer network and aggregate the learned representations through an attention module.
Based on the learned quantized representation, NID~\cite{DBLP:journals/corr/abs-2405-16435} trains an MLP network for downstream tasks, such as node classification and link prediction.


\subsection{KG Quantization}

Recently, several KG quantization methods have been proposed for effective embedding, responding to the increasing demand for larger KGs.
Unlike the techniques employed in the HOG or HEG setting, these methods tend to leverage an unsupervised paradigm for quantization.
NodePiece~\cite{DBLP:conf/iclr/0001DWH22}, EARL~\cite{DBLP:conf/aaai/ChenZYZGPC23}, and random entity quantization (RandomEQ for short)~\cite{DBLP:conf/emnlp/LiWLZM23} are all in such a framework.
They first select a number of entities as anchors and then utilize the structural statistical strategy to match them for each entity.
Specifically, NodePiece employs metrics such as Personalized PageRank~\cite{page1999pagerank} and node degree to select certain anchor entities and subsequently view top-$k$ closest anchors as entity codes.
EARL and RandomEQ sample 10\% entities as anchors and then assign soft weights to each anchor according to the similarity between connected relation sets.
Given the limitations of these methods in thoroughly capturing the structure and semantics of KGs, SSQR presents a self-supervised approach that employs learned discrete codes to reconstruct structures and imply semantic text, offering more holistic modeling with only 16 quantized codes.


\subsection{Application Scenarios}



% \paragraph{Node Level}
% \paragraph{Edge Level.}
% link prediction (LP)
% \paragraph{Graph Level.}
% molecular property prediction in Mole-BERT, as graph classification task.


\paragraph{General Graph Tasks.}
Based on the learned quantized codes, many general graph tasks can be addressed, for example, node classification, link prediction, graph classification, graph regression, and graph generation.
In addition, SNEQ~\cite{he2020sneq} can be employed for node recommendation that ranks all nodes based on a specific distance metric and suggests the nearest node.
d-SNEQ~\cite{DBLP:journals/tnn/HeGSL23} is further utilized for path prediction that predicts the path (\eg, the shortest path) between two nodes.

Based on the learned structure-aware codes for nodes, VQGraph~\cite{yang2024vqgraph} enhances the GNN-to-MLP distillation by proposing a new distillation target, namely \emph{soft code assignments}, which utilizes the Kullback–Leibler divergence to make two distributions (codes distributions by GNN and MLP) be close together.
This can directly transfer the structural knowledge of each node from GNN to MLP.
The results show it can improve the expressiveness of existing graph representation space and facilitate structure-aware GNN-to-MLP distillation.
In the KG scenarios, NodePiece~\cite{DBLP:conf/iclr/0001DWH22}, EARL~\cite{DBLP:conf/aaai/ChenZYZGPC23}, RandomEQ~\cite{DBLP:conf/emnlp/LiWLZM23}, SSQR~\cite{lin2024self} all can be used for KG link prediction, which is a ranking task that to predict the object entity based the given subject entity and the relation, \ie, $(s,r,?)$.
Using the learned codes as features, SSQR can also fix the triple classification problem, predicting the validity of the given triple $(s,r,t)$.

\paragraph{Molecular Tasks \& AI for Science.}

Recently, QGR methods have been widely used for molecular tasks and AI for science scenarios, including molecular property prediction (classification and regression), molecule-text prediction, and protein \& RNA reconstruction.
Specifically, DGAE~\cite{DBLP:journals/tmlr/BogetGK24} conduct the graph generation.
It first iteratively samples discrete codes and then generates the graph structures by the pre-trained decoder, which can be used in the generation for molecular graphs and has the potential for drug design, material design and protein design.
% $P_\theta$
% \red{For generation, we iteratively sample from p$\theta$(ki,c), and get the corresponding zi,c. Sequence generation stops upon sampling the end-of-sequence token or reaching the maximum number of node embeddings nmax. So, we can generate graphs of various sizes, and we need at most nmaxC iterations to generate an instance.}
LLPS~\cite{DBLP:journals/corr/abs-2405-15840} shows the potential for the reconstruction of protein sequences by training a de novo generative model for protein structures using a vanilla decoder-only Transformer model.
% GLAD used for the molecular generation task.
Bio2Token~\cite{liu2024bio2token} conducts efficient representation of large 3D molecular structures with high fidelity for molecules, proteins, and RNA, holding the potential for the design of biomolecules and biomolecular complexes.
UniMoT~\cite{DBLP:journals/corr/abs-2408-00863} unifies the molecule-text applications, including molecular classification \& regression, molecule captioning, molecule-text retrieval, and caption-guided molecule generation.
It demonstrates the impressive and comprehensive potentials that arise from the combination of QGR and LLMs.





\section{Code Dependence Learning}

Inspired by the distribution learning of the discrete codes in computer vision field~\cite{esser2021taming} that predicts the next code by the auto-regressive Transformer,
QGR methods also implement this technique, facilitating comprehensive semantic modeling and supporting the generation tasks.
Similar to BERT~\cite{DBLP:conf/naacl/DevlinCLT19} pre-training style, MOLE-BERT adopts a strategy similar to Masked Language Modeling (MLM). 
It employs this method to pre-train the GNN encoder by randomly masking certain discrete codes. Subsequently, it pre-trains GNNs to predict these masked codes, a process known as Masked Code Modeling (MCM):
\begin{equation}
\small
\mathcal{L}_{mcm}=-\sum_{\mathcal{G}\in\mathcal{D}}\sum_{j\in\mathcal{M}}\log p(c_j|\mathcal{G}_{\mathcal{M}}),
\end{equation}
where $\mathcal{M}$ is the set of masked nodes in the graph and $c_j$ is the quantized codes.
Auto-regressive code dependence learning is another mainstream strategy (usually using Transformer) based on next token prediction:
\begin{equation}
\small
\mathcal{L}_{ntp}=-\frac{1}{N}\prod_{j=1}^{N} P_\theta(\textbf{\textit{c}}_j|\textbf{\textit{c}}_1,\textbf{\textit{c}}_2,\cdots, \textbf{\textit{c}}_{j-1}),
\end{equation}
where network $P_\theta$ is with parameter $\theta$.
$N$ is the length of the code sequence $\textbf{\textit{c}}$.
DGAE~\cite{DBLP:journals/tmlr/BogetGK24} splits the latent representation of each node into $C$ parts.
So after the quantization, each graph can be represented as $n\times C$ code permutation.
To learn the dependence of it, the 2D Transformer~\cite{DBLP:conf/nips/VaswaniSPUJGKP17} is introduced to auto-regressively generate the codes, \ie, the joint probability can be $\prod_{i=1}^{n}\prod_{j=1}^{C}P_\theta(\textbf{\textit{c}}_{i,j}|\textbf{\textit{c}}_{<i,1},\cdots, \textbf{\textit{c}}_{<i,C},\textbf{\textit{c}}_{i,<j})$.

Moreover, GLAD~\cite{boget2024glad} implements diffusion bridges~\cite{DBLP:conf/iclr/LiuW0l23} for codes' dependence, where Brownian motion is utilized as a non-conditional diffusion process defined by a stochastic differential equation (SDE).
Based on the learned model bridge, discrete practical codes can be acquired by the iterative denoising process from the random Gaussian noise.
Followed by the pre-trained decoder, these codes can be used for the graph generation.
Although Dr.E does not directly or explicitly learn the dependence of the codes, it introduces intra-layer and inter-layer residuals to gradually generate the codes, which enhances the representation of sequential information as the newly generated code depends on the previously obtained codes.
It can be viewed as an implicit auto-regressive manner.

% \subsection{Distribution Learning}

% \subsection{Incorporation with Language Models}

\section{Integration with LLMs}

\begin{figure}[t]
\centering
\includegraphics[width=0.99\linewidth]{llm2.png}
% \vspace{-0.3cm}
% \setlength{\abovecaptionskip}{-0.1cm}
% \setlength{\belowcaptionskip}{-0.5cm}
\caption{Illustration of integrating QGR with LLMs.}
\label{fig_llm}
\end{figure}

The quantized codes, being discrete, share a similar structure with natural language.
As such, QGR methods can be seamlessly incorporated with LLMs to facilitate robust modeling and generalization as shown in Figure~\ref{fig_llm}.
Table~\ref{tab_instruction} provides examples of tuning instructions for both Dr.E and SSQR, arranged for better understanding and intuitive interpretation.

\begin{table}[]
\small
    \centering
    \begin{tabular}{c}
                \begin{tcolorbox}[colback=gray!10,%gray background
                      colframe=black,% black frame colour
                      width=8.3cm,% Use 5cm total width,
                      boxrule=0.8pt,
                      arc=1mm, auto outer arc,
                      left = 1mm, %文字离线框左边的边距
                        right = 1mm,%同上
                        top = 1mm,%同上
                        bottom = 1mm,%同上
                     ]
                \footnotesize{
                \textbf{Input:} Given a node, you need to classify it among `Case Based', `Genetic Algorithms'.... With the node's 1-hop information being `\blue{amass}', `\blue{traverse}', `\blue{handle}'..., 2-hop information being `\blue{provable}', `\blue{revolution}', `\blue{creative}'..., 3-hop information being `\blue{nous}', `\blue{hypothesis}', `\blue{minus}', the node should be classified as:\\
                \textbf{Output:} Rule Learning
                }
                \end{tcolorbox} \\
                (a) Node classification, taken from Dr.E~\cite{liu2024dr}.\\
                \begin{tcolorbox}[colback=gray!10,%gray background
                      colframe=black,% black frame colour
                      width=8.3cm,% Use 5cm total width,
                      boxrule=0.8pt,
                      arc=1mm, auto outer arc,
                      left = 1mm, %文字离线框左边的边距
                        right = 1mm,%同上
                        top = 1mm,%同上
                        bottom = 1mm,%同上
                     ]
                \footnotesize{
                \textbf{Input:}  Given a triple in the knowledge graph, you need to predict its validity based on the triple itself and entities' quantized representations.\\
                The triple is: (\emph{h}, \emph{r}, \emph{t})\\
                The quantized representation of entity \emph{h} is: \blue{[Code(\emph{h})]}\\
                The quantized representation of entity \emph{t} is: \blue{[Code(\emph{t})]}\\
                Please determine the validity of the triple and respond True or False.\\
                \textbf{Output:} True/False
                }
                \end{tcolorbox} \\
                (b) KG triple classification, taken from SSQR~\cite{lin2024self}.\\
    \end{tabular}
    \caption{Instruction examples with learned the discrete codes. In (a), the codeword responds to the real word in the LLM's vocabulary. In (b), the learned codewords are virtual tokens, which need to expand the LLM's vocabulary for fine-tuning.}
    \label{tab_instruction}
\end{table}






Dr.E implements token-level alignment between GNNs and LLMs.
Based on the quantized codes through intra-layer and inter-layer residuals, the preservation of multiple perspectives in each convolution step is guaranteed, fostering reliable information transfer from the surrounding nodes to the central point.
Moreover, by utilizing the LLMs’ vocabulary as the codebook, Dr.E can represent each node with real words from different graph hops, as shown in Table~\ref{tab_instruction} (a).
By fine-tuning LLaMA-2-7B~\cite{touvron2023llama2} with the designed instruction data, the node label can be easily predicted.
Different from the fixed vocabulary in Dr.E, UniMoT~\cite{DBLP:journals/corr/abs-2408-00863} expands the original LLMs' vocabulary through the incorporation of learned molecular tokens from the codebook, forming a unified molecule-text vocabulary.
Based on the specific instructions for molecular property prediction, molecule captioning, molecule-text retrieval, caption-guided molecule generation,
LLaMA2 is fine-tuned with LoRA~\cite{DBLP:conf/iclr/HuSWALWWC22},
thereby bestowing upon it an impressive and comprehensive proficiency in molecule-text applications.
% and is endowed with an impressive and comprehensive capacity for molecule-text applications.
Similarly, in KG-related scenarios, SSRQ leverages the learned discrete entity codes as additional features to directly input LLMs, helping to make accurate predictions.
Both LLaMA-2-7B and LLaMA3.1-8B~\cite{DBLP:journals/corr/abs-2407-21783} are utilized for the KG link prediction task and triple classification task.
During the fine-tuning phase, the introduction of new tokens necessitates the expansion of the LLM's tokenizer vocabulary, which is in line with the size of the codebook, \eg, 1024 or 2048.
All other elements of the network remain unaltered.


\section{Discussions and Future Directions}


Beyond the above methods, there are also some methods that does not use the quantized code for the final representation, such as VQ-GNN~\cite{ding2021vq}, VQSynery~\cite{DBLP:journals/corr/abs-2403-03089}, and MSPmol~\cite{DBLP:conf/ijcnn/LuPZC24}.
For example,
instead of employing discrete motif features as the ultimate representation, MSPmol~\cite{DBLP:conf/ijcnn/LuPZC24} utilizes the Vector Quantization (VQ) method on motif nodes within specific layers of the molecular graph.
There are also benefits for the representation, which can be viewed as a paradigm of \emph{contiguous+quantized} with the network, rather than the outputs.
They differ from the studies in Table~\ref{tab_all_studies}, so we do not include them in this survey.

Despite some achievements of the QGR methods, there are also some shortcomings from both methodology and application perspectives.
First, the quantization process may result in the loss of some details of the original data, which could pose challenges in areas that require accurate modeling.
Second, the quantization process can require complex techniques to ensure effective and efficient optimization, while pinepine's schema and integration with LLMs add additional complexity to the overall framework.
Third, compared with the wide application and success of general LLMs, the research on the integration of QGR and LLMs is still lacking.
Thus, the direction of future development lies in the following four aspects.

\paragraph{Choice of Codeword.}
For QGR learning, one can incorporate certain domain knowledge into the construction of the codebook. This knowledge might encompass semantic concepts, structural formations, and textual language. Such an approach can augment the interpretability of the whole framework, thereby facilitating a better human understanding.

\paragraph{Advanced Quantized Methods.}
GQR can adopt the latest advanced quantized methods to enhance the efficiency and efficacy of learning graph codes, which may have already demonstrated success within the domain of computer vision.
For instance, the rotation trick~\cite{DBLP:journals/corr/abs-2410-06424} is a proven method to mitigate the issues of codebook collapse and underutilization in VQ.
SimVQ~\cite{zhu2024addressing} simply incorporates a linear transformation into the codebook, yielding substantial improvement.


\paragraph{Unified Graph Foundation Model.}

While the GQR methods have indeed made some notable strides, the current research primarily follows a similar approach that tackles various tasks using distinct training targets, resulting in multiple distinct models.
This could restrict its applicability, particularly in the age of LLMs.
Drawing inspiration from the remarkable success of unified LLMs and multimodal LLMs~\cite{DBLP:journals/corr/abs-2303-18223,lin2024has}, the unified graph foundation model could be realized to execute various tasks across diverse graphs through QGR, by learning unified codes for different graphs and then tuning with LLM techniques.


\paragraph{Graph RAG with QGR.}
Retrieval-Augmented Generation (RAG) has emerged as a significant focal point for improving the capabilities of LLMs within specific fields.
Graph RAG~\cite{DBLP:journals/corr/abs-2404-16130} would be beneficial for downstream tasks in the context of LLMs, where the QGR learning would provide an effective manner to retrieve relevant nodes or subgraphs by calculating the similarity of the codes.
It can be easily realized by the statistical metrics of the codes or the numerical calculations among embeddings.

% \newpage
% \newpage


% \section{Methodologies}



% 3 SNEQ~\cite{he2020sneq}
% \red{
% Semantic margin loss + Adaptive margin loss.
% Self-attention for Deep Quantisation.
% }

% 4 d-SNEQ~\cite{DBLP:journals/tnn/HeGSL23}\red{
% 2021/08 incorporates a rank loss to equip the learned quantization codes with rich high-order information and is able to substantially compress the size of trained embeddings, thus reducing storage footprint and accelerating retrieval speed.
% }
% only add the ranking loss for positive pair and negative pair.

% 5 TD-DNE~\cite{he2023transferable}

% 6 DNE~\cite{shen2018discrete} discrete representation +1/-1  no gcn  graph representation learning

% 7. DGAE~\cite{DBLP:journals/tmlr/BogetGK24}
% \red{quantized+transformer graph generation?}
% \red{auto-encoder to convert graphs into sets of node embeddings and reconstruct graphs from sets of node embeddings.
% graph to set. set ordering. distribution learning using 2D Transfomer.
% like raming Transformer.}

% 8. LLPS Learning the Language of Protein Structure~\cite{DBLP:journals/corr/abs-2405-15840}
% \red{
% This representation learning task is similar to the traditional task of mapping point-clouds to sequences.
% How to construct Graph?
% Protein reconstruction?
% Differently, utilize Finite Scalar Quantization (FSQ)~\cite{DBLP:conf/iclr/MentzerMAT24} framework.
% }

% 9. GLAD~\cite{boget2024glad} quantized, diffusion. molecular benchmark datasets.
% \red{
% Finite Scalar Quantization (FSQ)~\cite{DBLP:conf/iclr/MentzerMAT24}.
% two-stage.
% }

% 10. Bio2Token~\cite{liu2024bio2token}. FSQ with Mamba. reconstruction.
% \red{point cloud, not graph?}

% 11. Graph Quantized Tokenizer (GQT)~\cite{wang2024learning}.
% Deep Graph Infomax (DGI)~\cite{DBLP:conf/iclr/VelickovicFHLBH19} and Graph Masked Auto-Encoder 2 (GMAE2)~\cite{DBLP:conf/www/HouHCLDK023} as loss.   two stages? or one


% 12. NID~\cite{DBLP:journals/corr/abs-2405-16435}. graph MAE. RVQ. hierarchical $\times$

% 13. iMoLD~\cite{DBLP:conf/nips/ZhuangZDBWLCC23}. Task-agnostic Self-supervised Invariant Learning + Task Prediction. another type of Residual Vector Quantization (RVQ) module (continue+quantized)
% just commitment loss for VQ.

% 14. UniMoT~\cite{DBLP:journals/corr/abs-2408-00863} encoder - Q-Former - vq. Using LLMs for multiple tasks.

% 15. MSPmol~\cite{DBLP:conf/ijcnn/LuPZC24}. \red{that does not use discrete motif features as the final representation, but rather applies the VQ approach to motif nodes in certain layers of the molecular graph.}
% Quantized representation in Discrete Motifs Message Passing.

% 16. VQSynery~\cite{DBLP:journals/corr/abs-2403-03089}.
% drug synergy prediction. VQ in message-passing.

% 17. Mole-BERT~\cite{DBLP:conf/iclr/XiaZHG0LLL23}
% contextual code learning for each atom in the molecular graph.




% $\mathbf{h}$ is the GNN representation.
% L: GNN layers.
% 20 papers and details
% 11.16

% 1. VQ-GNN~\cite{ding2021vq} not the same to others. like the centroids in k-means. Dimensionality reduction.

% 2. VQGraph~\cite{yang2024vqgraph}. one node to one code. distilling to MLP using soft code assignments.

% 3. Dr. E~\cite{liu2024dr}:
% \red{Dr.E represents the first and a pioneering effort to achieve token-level alignment between GNNs and LLMs, setting a new standard for the integration of graph data in multi-modal learning environments.
% By implementing intra-layer and inter-layer residuals, we ensure the preservation of multilayer perspectives at each convolution step, allowing for robust information transmission from the surrounding nodes to the center.}
% generate codes in sequence. using LLM.
% codebook (4096*3*8)
% Intra-Layer Residual:
% \begin{equation}
% \mathbf{e}_{v,k}=\underset{\mathbf{z}_m\in \mathbf{Z}}{\mathop{\arg\min}}\|\mathbf{h}_{v,k-1}-\mathbf{e}_{v,k-1}-\mathbf{z}_m\|,
% \end{equation}
% \begin{equation}
% \mathbf{h}_{v,k}=\mathbf{h}_{v,k-1}-\mathbf{e}_{v,k}
% \end{equation}
% \red{using a single token per layer proves insufficient to capture information embedded within neighboring nodes.}
% The paper uses K codes in each layer.
% Inter-Layer Residual:
% \begin{equation}
% \mathbf{h}_v^{l+1}=\sigma\big(\mathbf{W}\cdot\mathrm{Con}(\mathbf{h}_v^l+\mathrm{Pool}(\{\mathbf{e}_{v,k}^l\}_{k=1}^K),\mathbf{h}_{\mathcal{N}(v)}^l)\big),
% \end{equation}
% Every layer have K codes.
% In this way, the codes are generated in sequence in each layer and from the low layers to the top layers.

% The equation of VQGraph~\cite{yang2024vqgraph} is:

% training:
% 1. feature reconstruction (FR) self-supervised Mean Squared Error (MSE) Loss
% 2. adjacency reconstruction (AR) self-supervised Cross-Entropy (CE) Loss
% 3. label prediction (LP), supervised Cross-Entropy (CE) Loss
% 4. LLM Next-token prediction(NTP)
% % \footnote{ddd}





% % \begin{table*}[t]
% % % \small
% % \caption{The overall studies of Quantized Graph Models. attributed graph (AG)   Text-Attributed Graphs (TAG), Homogeneous graphs (HOG) and heterogeneous graphs (HEG) KG node classification (NC) Link Prediction (LP) Node Recommendation (NR)
% %     node feature reconstruction (NFR) edge reconstruction (ER) Node label classification (NC)
% %     Training: structure loss (STL), semantics loss (SEL), QL (quantized Loss) GG (Graph Generation)
% %     }
% %     \centering
% %     \resizebox{1.0\textwidth}{!}{
    
% %     \begin{tabular}{lcccccc}
% %         \toprule
% %         Model  & Time & Graph Type &  Codebook&2 stage & Training & Application.\\
% %         \midrule
% %         SNEQ~\cite{he2020sneq}  & 2020/02&AG+HOG&16*8&&STL+SEL+QL& NC+LP+NR       \\
% %         VQGraph~\cite{yang2024vqgraph}    & 2023/08         & HOG & 2048, 8192&--&NFR+ER+NC&  NC (GNN-to-MLP distillation) \\
% %         Dr. E~\cite{liu2024dr}     & 2024/06         & TAG &subset of LLMs’ vocabulary (fixed)&LLaMA-2&FR+AR+LP/FR+NTP&  NC (fine-tuning and few-shot) \\
% %         DGAE~\cite{DBLP:journals/tmlr/BogetGK24}& 2024/03         & HEG &\red{256,1024,4096}&2D Transformer&LP/NTP&  GG\\
% %         LLPS~\cite{DBLP:journals/corr/abs-2405-15840}& 2024/05         &  &4096&&&  GG\\
% %         GLAD~\cite{boget2024glad}& 2024/03         & \red{HOG} &&Diffusion&NFR+LP/&  GG\\
% %         Bio2Token~\cite{liu2024bio2token}&24/10&&&&&\\
% %         &&&&&&\\
% %         \bottomrule
% %     \end{tabular}
% %     }
% %     \label{tab:booktabs}
% % \end{table*}






% 1. Anchor representation

% 2. Codebook Learning



% \subsection{Codes Evaluation/Distribution}
% perplexity~\cite{liu2024dr}


% \subsection{Codes Evaluation/Distribution}

% \subsection{Codes/Token Integration with downstream NNs.}

% Graph Quantized Tokenizer (GQT)~\cite{wang2024learning}.








% \section*{Acknowledgments}

% This project was funded by the National Research Foundation Singapore under the RIE2025 Industry Alignment Fund (I2101E0002-Cisco-NUS Accelerated Digital Economy Corporate Laboratory) and under AI Singapore Programme (Award Number: AISG2-TC-2022-004).


%% The file named.bst is a bibliography style file for BibTeX 0.99c
\bibliographystyle{named}
\bibliography{ijcai24}

\end{document}


\clearpage

%\title{Generating 3D \hl{Small} Binding Molecules Using Shape-Conditioned Diffusion Models with Guidance}
%\date{\vspace{-5ex}}

%\author{
%	Ziqi Chen\textsuperscript{\rm 1}, 
%	Bo Peng\textsuperscript{\rm 1}, 
%	Tianhua Zhai\textsuperscript{\rm 2},
%	Xia Ning\textsuperscript{\rm 1,3,4 \Letter}
%}
%\newcommand{\Address}{
%	\textsuperscript{\rm 1}Computer Science and Engineering, The Ohio Sate University, Columbus, OH 43210.
%	\textsuperscript{\rm 2}Perelman School of Medicine, University of Pennsylvania, Philadelphia, PA 19104.
%	\textsuperscript{\rm 3}Translational Data Analytics Institute, The Ohio Sate University, Columbus, OH 43210.
%	\textsuperscript{\rm 4}Biomedical Informatics, The Ohio Sate University, Columbus, OH 43210.
%	\textsuperscript{\Letter}ning.104@osu.edu
%}

%\newcommand\affiliation[1]{%
%	\begingroup
%	\renewcommand\thefootnote{}\footnote{#1}%
%	\addtocounter{footnote}{-1}%
%	\endgroup
%}



\setcounter{secnumdepth}{2} %May be changed to 1 or 2 if section numbers are desired.

\setcounter{section}{0}
\renewcommand{\thesection}{S\arabic{section}}

\setcounter{table}{0}
\renewcommand{\thetable}{S\arabic{table}}

\setcounter{figure}{0}
\renewcommand{\thefigure}{S\arabic{figure}}

\setcounter{algorithm}{0}
\renewcommand{\thealgorithm}{S\arabic{algorithm}}

\setcounter{equation}{0}
\renewcommand{\theequation}{S\arabic{equation}}


\begin{center}
	\begin{minipage}{0.95\linewidth}
		\centering
		\LARGE 
	Generating 3D Binding Molecules Using Shape-Conditioned Diffusion Models with Guidance (Supplementary Information)
	\end{minipage}
\end{center}
\vspace{10pt}

%%%%%%%%%%%%%%%%%%%%%%%%%%%%%%%%%%%%%%%%%%%%%
\section{Parameters for Reproducibility}
\label{supp:experiments:parameters}
%%%%%%%%%%%%%%%%%%%%%%%%%%%%%%%%%%%%%%%%%%%%%

We implemented both \SE and \methoddiff using Python-3.7.16, PyTorch-1.11.0, PyTorch-scatter-2.0.9, Numpy-1.21.5, Scikit-learn-1.0.2.
%
We trained the models using a Tesla V100 GPU with 32GB memory and a CPU with 80GB memory on Red Hat Enterprise 7.7.
%
%We released the code, data, and the trained model at Google Drive~\footnote{\url{https://drive.google.com/drive/folders/146cpjuwenKGTd6Zh4sYBy-Wv6BMfGwe4?usp=sharing}} (will release to the public on github once the manuscript is accepted).

%===================================================================
\subsection{Parameters of \SE}
%===================================================================


In \SE, we tuned the dimension of all the hidden layers including VN-DGCNN layers
(Eq.~\ref{eqn:shape_embed}), MLP layers (Eq.~\ref{eqn:se:decoder}) and
VN-In layer (Eq.~\ref{eqn:se:decoder}), and the dimension $d_p$ of generated shape latent embeddings $\shapehiddenmat$ with the grid-search algorithm in the 
parameter space presented in Table~\ref{tbl:hyper_se}.
%
We determined the optimal hyper-parameters according to the mean squared errors of the predictions of signed distances for 1,000 validation molecules that are selected as described in Section ``Data'' 
in the main manuscript.
%
The optimal dimension of all the hidden layers is 256, and the optimal dimension $d_p$ of shape latent embedding \shapehiddenmat is 128.
%
The optimal number of points $|\pc|$ in the point cloud \pc is 512.
%
We sampled 1,024 query points in $\mathcal{Z}$ for each molecule shape.
%
We constructed graphs from point clouds, which are employed to learn $\shapehiddenmat$ with VN-DGCNN layer (Eq.~\ref{eqn:shape_embed}), using the $k$-nearest neighbors based on Euclidean distance with $k=20$.
%
We set the number of VN-DGCNN layers as 4.
%
We set the number of MLP layers in the decoder (Eq.~\ref{eqn:se:decoder}) as 2.
%
We set the number of VN-In layers as 1.

%
We optimized the \SE model via Adam~\cite{adam} with its parameters (0.950, 0.999), %betas (0.95, 0.999), 
learning rate 0.001, and batch size 16.
%
We evaluated the validation loss every 2,000 training steps.
%
We scheduled to decay the learning rate with a factor of 0.6 and a minimum learning rate of 1e-6 if 
the validation loss does not decrease in 5 consecutive evaluations.
%
The optimal \SE model has 28.3K learnable parameters. 
%
We trained the \SE model %for at most 80 hours 
with $\sim$156,000 training steps.
%
The training took 80 hours with our GPUs.
%
The trained \SE model achieved the minimum validation loss at 152,000 steps.


\begin{table*}[!h]
  \centering
      \caption{{Hyper-Parameter Space for \SE Optimization}}
  \label{tbl:hyper_se}
  \begin{threeparttable}
 \begin{scriptsize}
      \begin{tabular}{
%	@{\hspace{2pt}}l@{\hspace{2pt}}
	@{\hspace{2pt}}l@{\hspace{5pt}} 
	@{\hspace{2pt}}r@{\hspace{2pt}}         
	}
        \toprule
        %Notation &
          Hyper-parameters &  Space\\
        \midrule
        %$t_a$    & 
         %hidden layer dimension         & \{16, 32, 64, 128\} \\
         %atom/node embedding dimension &  \{16, 32, 64, 128\} \\
         %$\latent^{\add}$/$\latent^{\delete}$ dimension        & \{8, 16, 32, 64\} \\
         hidden layer dimension            & \{128, 256\}\\
         dimension $d_p$ of \shapehiddenmat        &  \{64, 128\} \\
         \#points in \pc        & \{512, 1,024\} \\
         \#query points in $\mathcal{Z}$                & 1,024 \\%1024 \\%\bo{\{1024\}}\\
         \#nearest neighbors              & 20          \\
         \#VN-DGCNN layers (Eq~\ref{eqn:shape_embed})               & 4            \\
         \#MLP layers in Eq~\ref{eqn:se:decoder} & 4           \\
        \bottomrule
      \end{tabular}
%  	\begin{tablenotes}[normal,flushleft]
%  		\begin{footnotesize}
%  	
%  	\item In this table, hidden dimension represents the dimension of hidden layers and 
%  	atom/node embeddings; latent dimension represents the dimension of latent embedding \latent.
%  	\par
%  \end{footnotesize}
%  
%\end{tablenotes}
%      \begin{tablenotes}
%      \item 
%      \par
%      \end{tablenotes}
\end{scriptsize}
  \end{threeparttable}
\end{table*}

%
\begin{table*}[!h]
  \centering
      \caption{{Hyper-Parameter Space for \methoddiff Optimization}}
  \label{tbl:hyper_diff}
  \begin{threeparttable}
 \begin{scriptsize}
      \begin{tabular}{
%	@{\hspace{2pt}}l@{\hspace{2pt}}
	@{\hspace{2pt}}l@{\hspace{5pt}} 
	@{\hspace{2pt}}r@{\hspace{2pt}}         
	}
        \toprule
        %Notation &
          Hyper-parameters &  Space\\
        \midrule
        %$t_a$    & 
         %hidden layer dimension         & \{16, 32, 64, 128\} \\
         %atom/node embedding dimension &  \{16, 32, 64, 128\} \\
         %$\latent^{\add}$/$\latent^{\delete}$ dimension        & \{8, 16, 32, 64\} \\
         scalar hidden layer dimension         & 128 \\
         vector hidden layer dimension         & 32 \\
         weight of atom type loss $\xi$ (Eq.~\ref{eqn:loss})  & 100           \\
         threshold of step weight $\delta$ (Eq.~\ref{eqn:diff:obj:pos}) & 10 \\
         \#atom features $K$                   & 15 \\
         \#layers $L$ in \molpred             & 8 \\
         %\# \eqgnn/\invgnn layers     &  8 \\
         %\# heads {$n_h$} in $\text{MHA}^{\mathtt{x}}/\text{MHA}^{\mathtt{v}}$                               & 16 \\
         \#nearest neighbors {$N$}  (Eq.~\ref{eqn:geometric_embedding} and \ref{eqn:attention})            & 8          \\
         {\#diffusion steps $T$}                  & 1,000 \\
        \bottomrule
      \end{tabular}
%  	\begin{tablenotes}[normal,flushleft]
%  		\begin{footnotesize}
%  	
%  	\item In this table, hidden dimension represents the dimension of hidden layers and 
%  	atom/node embeddings; latent dimension represents the dimension of latent embedding \latent.
%  	\par
%  \end{footnotesize}
%  
%\end{tablenotes}
%      \begin{tablenotes}
%      \item 
%      \par
%      \end{tablenotes}
\end{scriptsize}
  \end{threeparttable}

\end{table*}


%===================================================================
\subsection{Parameters of \methoddiff}
%===================================================================

Table~\ref{tbl:hyper_diff} presents the parameters used to train \methoddiff.
%
In \methoddiff, we set the hidden dimensions of all the MLP layers and the scalar hidden layers in GVPs (Eq.~\ref{eqn:pred:gvp} and Eq.~\ref{eqn:mess:gvp}) as 128. %, including all the MLP layers in \methoddiff and the scalar dimension of GVP layers in Eq.~\ref{eqn:pred:gvp} and Eq.~\ref{eqn:mess:gvp}. %, and MLP layer (Eq.~\ref{eqn:diff:graph:atompred}) as 128.
%
We set the dimensions of all the vector hidden layers in GVPs as 32.
%
We set the number of layers $L$ in \molpred as 8.
%and the number of layers in graph neural networks as 8.
%
Both two GVP modules in Eq.~\ref{eqn:pred:gvp} and Eq.~\ref{eqn:mess:gvp} consist of three GVP layers. %, which consisa GVP modset the number of layer of GVP modules %is a multi-head attention layer ($\text{MHA}^{\mathtt{x}}$ or $\text{MHA}^{\mathtt{h}}$) with 16 heads.
% 
We set the number of VN-MLP layers in Eq.~\ref{eqn:shaper} as 1 and the number of MLP layers as 2 for all the involved MLP functions.
%

We constructed graphs from atoms in molecules, which are employed to learn the scalar embeddings and vector embeddings for atoms %predict atom coordinates and features  
(Eq.~\ref{eqn:geometric_embedding} and \ref{eqn:attention}), using the $N$-nearest neighbors based on Euclidean distance with $N=8$. 
%
We used $K=15$ atom features in total, indicating the atom types and its aromaticity.
%
These atom features include 10 non-aromatic atoms (i.e., ``H'', ``C'', ``N'', ``O'', ``F'', ``P'', ``S'', ``Cl'', ``Br'', ``I''), 
and 5 aromatic atoms (i.e., ``C'', ``N'', ``O'', ``P'', ``S'').
%
We set the number of diffusion steps $T$ as 1,000.
%
We set the weight $\xi$ of atom type loss (Eq.~\ref{eqn:loss}) as $100$ to balance the values of atom type loss and atom coordinate loss.
%
We set the threshold $\delta$ (Eq.~\ref{eqn:diff:obj:pos}) as 10.
%
The parameters $\beta_t^{\mathtt{x}}$ and $\beta_t^{\mathtt{v}}$ of variance scheduling in the forward diffusion process of \methoddiff are discussed in 
Supplementary Section~\ref{supp:forward:variance}.
%
%Please note that as in \squid, we did not perform extensive hyperparameter optimization for \methoddiff.
%
Following \squid, we did not perform extensive hyperparameter tunning for \methoddiff given that the used 
hyperparameters have enabled good performance.

%
We optimized the \methoddiff model via Adam~\cite{adam} with its parameters (0.950, 0.999), learning rate 0.001, and batch size 32.
%
We evaluated the validation loss every 2,000 training steps.
%
We scheduled to decay the learning rate with a factor of 0.6 and a minimum learning rate of 1e-5 if 
the validation loss does not decrease in 10 consecutive evaluations.
%
The \methoddiff model has 7.8M learnable parameters. 
%
We trained the \methoddiff model %for at most 60 hours 
with $\sim$770,000 training steps.
%
The training took 70 hours with our GPUs.
%
The trained \methoddiff achieved the minimum validation loss at 758,000 steps.

During inference, %the sampling, 
following Adams and Coley~\cite{adams2023equivariant}, we set the variance $\phi$ 
of atom-centered Gaussians as 0.049, which is used to build a set of points for shape guidance in Section ``\method with Shape Guidance'' 
in the main manuscript.
%
We determined the number of atoms in the generated molecule using the atom number distribution of training molecules that have surface shape sizes similar to the condition molecule.
%
The optimal distance threshold $\gamma$ is 0.2, and the optimal stop step $S$ for shape guidance is 300.
%
With shape guidance, each time we updated the atom position (Eq.~\ref{eqn:shape_guidance}), we randomly sampled the weight $\sigma$ from $[0.2, 0.8]$. %\bo{(XXX)}.
%
Moreover, when using pocket guidance as mentioned in Section ``\method with Pocket Guidance'' in the main manuscript, each time we updated the atom position (Eq.~\ref{eqn:pocket_guidance}), we randomly sampled the weight $\epsilon$ from $[0, 0.5]$. 
%
For each condition molecule, it took around 40 seconds on average to generate 50 molecule candidates with our GPUs.



%%%%%%%%%%%%%%%%%%%%%%%%%%%%%%%%%%%%%%%%%%%%%%
\section{Performance of \decompdiff with Protein Pocket Prior}
\label{supp:app:decompdiff}
%%%%%%%%%%%%%%%%%%%%%%%%%%%%%%%%%%%%%%%%%%%%%%

In this section, we demonstrate that \decompdiff with protein pocket prior, referred to as \decompdiffbeta, shows very limited performance in generating drug-like and synthesizable molecules compared to all the other methods, including \methodwithpguide and \methodwithsandpguide.
%
We evaluate the performance of \decompdiffbeta in terms of binding affinities, drug-likeness, and diversity.
%
We compare \decompdiffbeta with \methodwithpguide and \methodwithsandpguide and report the results in Table~\ref{tbl:comparison_results_decompdiff}.
%
Note that the results of \methodwithpguide and \methodwithsandpguide here are consistent with those in Table~\ref{tbl:overall_results_docking2} in the main manuscript.
%
As shown in Table~\ref{tbl:comparison_results_decompdiff}, while \decompdiffbeta achieves high binding affinities in Vina M and Vina D, it substantially underperforms \methodwithpguide and \methodwithsandpguide in QED and SA.
%
Particularly, \decompdiffbeta shows a QED score of 0.36, while \methodwithpguide substantially outperforms \decompdiffbeta in QED (0.77) with 113.9\% improvement.
%
\decompdiffbeta also substantially underperforms \methodwithpguide in terms of SA scores (0.55 vs 0.76).
%
These results demonstrate the limited capacity of \decompdiffbeta in generating drug-like and synthesizable molecules.
%
As a result, the generated molecules from \decompdiffbeta can have considerably lower utility compared to other methods.
%
Considering these limitations of \decompdiffbeta, we exclude it from the baselines for comparison.

\begin{table*}[!h]
	\centering
		\caption{Comparison on PMG among \methodwithpguide, \methodwithsandpguide and \decompdiffbeta}
	\label{tbl:comparison_results_decompdiff}
\begin{threeparttable}
	\begin{scriptsize}
	\begin{tabular}{
		@{\hspace{2pt}}l@{\hspace{2pt}}
		%
		%@{\hspace{2pt}}l@{\hspace{2pt}}
		%
		@{\hspace{2pt}}r@{\hspace{2pt}}
		@{\hspace{2pt}}r@{\hspace{2pt}}
		%
		@{\hspace{6pt}}r@{\hspace{6pt}}
		%
		@{\hspace{2pt}}r@{\hspace{2pt}}
		@{\hspace{2pt}}r@{\hspace{2pt}}
		%
		@{\hspace{5pt}}r@{\hspace{5pt}}
		%
		@{\hspace{2pt}}r@{\hspace{2pt}}
		@{\hspace{2pt}}r@{\hspace{2pt}}
		%
		@{\hspace{5pt}}r@{\hspace{5pt}}
		%
		@{\hspace{2pt}}r@{\hspace{2pt}}
	         @{\hspace{2pt}}r@{\hspace{2pt}}
		%
		@{\hspace{5pt}}r@{\hspace{5pt}}
		%
		@{\hspace{2pt}}r@{\hspace{2pt}}
		@{\hspace{2pt}}r@{\hspace{2pt}}
		%
		@{\hspace{5pt}}r@{\hspace{5pt}}
		%
		@{\hspace{2pt}}r@{\hspace{2pt}}
		@{\hspace{2pt}}r@{\hspace{2pt}}
		%
		@{\hspace{5pt}}r@{\hspace{5pt}}
		%
		@{\hspace{2pt}}r@{\hspace{2pt}}
		@{\hspace{2pt}}r@{\hspace{2pt}}
		%
		@{\hspace{5pt}}r@{\hspace{5pt}}
		%
		@{\hspace{2pt}}r@{\hspace{2pt}}
		%@{\hspace{2pt}}r@{\hspace{2pt}}
		%@{\hspace{2pt}}r@{\hspace{2pt}}
		}
		\toprule
		\multirow{2}{*}{method} & \multicolumn{2}{c}{Vina S$\downarrow$} & & \multicolumn{2}{c}{Vina M$\downarrow$} & & \multicolumn{2}{c}{Vina D$\downarrow$} & & \multicolumn{2}{c}{{HA\%$\uparrow$}}  & & \multicolumn{2}{c}{QED$\uparrow$} & & \multicolumn{2}{c}{SA$\uparrow$} & & \multicolumn{2}{c}{Div$\uparrow$} & %& \multirow{2}{*}{SR\%$\uparrow$} & 
		& \multirow{2}{*}{time$\downarrow$} \\
	    \cmidrule{2-3}\cmidrule{5-6} \cmidrule{8-9} \cmidrule{11-12} \cmidrule{14-15} \cmidrule{17-18} \cmidrule{20-21}
		& Avg. & Med. &  & Avg. & Med. &  & Avg. & Med. & & Avg. & Med.  & & Avg. & Med.  & & Avg. & Med.  & & Avg. & Med.  & & \\ %& & \\
		%\multirow{2}{*}{method} & \multirow{2}{*}{\#c\%} &  \multirow{2}{*}{\#u\%} &  \multirow{2}{*}{QED} & \multicolumn{3}{c}{$\nmax=50$} & & \multicolumn{2}{c}{$\nmax=1$}\\
		%\cmidrule(r){5-7} \cmidrule(r){8-10} 
		%& & & & \avgshapesim(std) & \avggraphsim(std  &  \diversity(std  & & \avgshapesim(std) & \avggraphsim(std \\
		\midrule
		%Reference                          & -5.32 & -5.66 & & -5.78 & -5.76 & & -6.63 & -6.67 & & - & - & & 0.53 & 0.49 & & 0.77 & 0.77 & & - & - & %& 23.1 & & - \\
		%\midrule
		%\multirow{4}{*}{PM} 
		%& \AR & -5.06 & -4.99 & &  -5.59 & -5.29 & &  -6.16 & -6.05 & &  37.69 & 31.00 & &  0.50 & 0.49 & &  0.66 & 0.65 & & - & - & %& 7.0 & 
		%& 7,789 \\
		%& \pockettwomol   & -4.50 & -4.21 & &  -5.70 & -5.27 & &  -6.43 & -6.25 & &  48.00 & 51.00 & &  0.58 & 0.58 & &  \textbf{0.77} & \textbf{0.78} & &  0.69 & 0.71 &  %& 24.9 & 
		%& 2,544 \\
		%& \targetdiff     & -4.88 & \underline{-5.82} & &  -6.20 & \underline{-6.36} & &  \textbf{-7.37} & \underline{-7.51} & &  57.57 & 58.27 & &  0.50 & 0.51 & &  0.60 & 0.59 & &  0.72 & 0.71 & % & 10.4 & 
		%& 1,252 \\
		 \decompdiffbeta             & -4.72 & -4.86 & & \textbf{-6.84} & \textbf{-6.91} & & \textbf{-8.85} & \textbf{-8.90} & &  {72.16} & {72.16} & &  0.36 & 0.36 & &  0.55 & 0.55 & & 0.59 & 0.59 & & 3,549 \\ 
		%-4.76 & -6.18 & &  \textbf{-6.86} & \textbf{-7.52} & &  \textbf{-8.85} & \textbf{-8.96} & &  \textbf{72.7} & \textbf{89.8} & &  0.36 & 0.34 & &  0.55 & 0.57 & & 0.59 & 0.59 & & 15.4 \\
		%& \decompdiffref  & -4.58 & -4.77 & &  -5.47 & -5.51 & &  -6.43 & -6.56 & &  47.76 & 48.66 & &  0.56 & 0.56 & &  0.70 & 0.69  & &  0.72 & 0.72 &  %& 15.2 & 
		%& 1,859 \\
		%\midrule
		%\multirow{2}{*}{PC}
		\methodwithpguide       &  \underline{-5.53} & \underline{-5.64} & & {-6.37} & -6.33 & &  \underline{-7.19} & \underline{-7.52} & &  \underline{78.75} & \textbf{94.00} & &  \textbf{0.77} & \textbf{0.80} & &  \textbf{0.76} & \textbf{0.76} & & 0.63 & 0.66 & & 462 \\
		\methodwithsandpguide   & \textbf{-5.81} & \textbf{-5.96} & &  \underline{-6.50} & \underline{-6.58} & & -7.16 & {-7.51} & &  \textbf{79.92} & \underline{93.00} & &  \underline{0.76} & \underline{0.79} & &  \underline{0.75} & \underline{0.74} & & 0.64 & 0.66 & & 561\\
		\bottomrule
	\end{tabular}%
	\begin{tablenotes}[normal,flushleft]
		\begin{footnotesize}
	\item 
\!\!Columns represent: {``Vina S'': the binding affinities between the initially generated poses of molecules and the protein pockets; 
		``Vina M'': the binding affinities between the poses after local structure minimization and the protein pockets;
		``Vina D'': the binding affinities between the poses determined by AutoDock Vina~\cite{Eberhardt2021} and the protein targets;
		``QED'': the drug-likeness score;
		``SA'': the synthesizability score;
		``Div'': the diversity among generated molecules;
		``time'': the time cost to generate molecules.}
		
		\par
		\par
		\end{footnotesize}
	\end{tablenotes}
	\end{scriptsize}
\end{threeparttable}
  \vspace{-10pt}    
\end{table*}



%===================================================================
\section{{Additional Experimental Results on SMG}}
\label{supp:app:results}
%===================================================================

%-------------------------------------------------------------------------------------------------------------------------------------
\subsection{Comparison on Shape and Graph Similarity}
\label{supp:app:results:overall_shape}
%-------------------------------------------------------------------------------------------------------------------------------------

%\ziqi{Outline for this section:
%	\begin{itemize}
%		\item \method can consistently generate molecules with novel structures (low graph similarity) and similar shapes (high shape similarity), such that these molecules have comparable binding capacity with the condition molecules, and potentially better properties as will be shown in Table~\ref{tbl:overall_results_quality_10}.
%	\end{itemize}
%}

\begin{table*}[!h]
	\centering
		\caption{Similarity Comparison on SMG}
	\label{tbl:overall_sim}
\begin{threeparttable}
	\begin{scriptsize}
	\begin{tabular}{
		@{\hspace{0pt}}l@{\hspace{8pt}}
		%
		@{\hspace{8pt}}l@{\hspace{8pt}}
		%
		@{\hspace{8pt}}c@{\hspace{8pt}}
		@{\hspace{8pt}}c@{\hspace{8pt}}
		%
	    	@{\hspace{0pt}}c@{\hspace{0pt}}
		%
		@{\hspace{8pt}}c@{\hspace{8pt}}
		@{\hspace{8pt}}c@{\hspace{8pt}}
		%
		%@{\hspace{8pt}}r@{\hspace{8pt}}
		}
		\toprule
		$\delta_g$  & method          & \avgshapesim$\uparrow$(std) & \avggraphsim$\downarrow$(std) & & \maxshapesim$\uparrow$(std) & \maxgraphsim$\downarrow$(std)       \\ %& \#n\%$\uparrow$  \\ 
		\midrule
		%\multirow{5}{0.079\linewidth}%{\hspace{0pt}0.1} & \dataset   & 0.0             & 0.628(0.139)          & 0.567(0.068)          & 0.078(0.010)          &  & 0.588(0.086)          & 0.081(0.013)          & 4.7              \\
		%&  \squid($\lambda$=0.3) & 0.0             & 0.320(0.000)          & 0.420(0.163)          & \textbf{0.056}(0.032) &  & 0.461(0.170)          & \textbf{0.065}(0.033) & 1.4              \\
		%& \squid($\lambda$=1.0) & 0.0             & 0.414(0.177)          & 0.483(0.184)          & \underline{0.064}(0.030)  &  & 0.531(0.182)          & \underline{0.073}(0.029)  & 2.4              \\
		%& \method               & \underline{1.6}     & \textbf{0.857}(0.034) & \underline{0.773}(0.045)  & 0.086(0.011)          &  & \underline{0.791}(0.053)  & 0.087(0.012)          & \underline{5.1}      \\
		%& \methodwithsguide      & \textbf{3.7}    & \underline{0.833}(0.062)  & \textbf{0.812}(0.037) & 0.088(0.009)          &  & \textbf{0.835}(0.047) & 0.089(0.010)          & \textbf{6.2}     \\ 
		%\cmidrule{2-10}
		%& improv\% & - & 36.5 & 43.2 & -53.6 &  & 42.0 & -33.8 & 31.9  \\
		%\midrule
		\multirow{6}{0.059\linewidth}{\hspace{0pt}0.3} & \dataset             & 0.745(0.037)          & \textbf{0.211}(0.026) &  & 0.815(0.039)          & \textbf{0.215}(0.047)      \\ %    & \textbf{100.0}   \\
			& \squid($\lambda$=0.3) & 0.709(0.076)          & 0.237(0.033)          &  & 0.841(0.070)          & 0.253(0.038)        \\ %  & 45.5             \\
		    & \squid($\lambda$=1.0) & 0.695(0.064)          & \underline{0.216}(0.034)  &  & 0.841(0.056)          & 0.231(0.047)        \\ %  & 84.3             \\
			& \method               & \underline{0.770}(0.039)  & 0.217(0.031)          &  & \underline{0.858}(0.038)  & \underline{0.220}(0.046)  \\ %& \underline{87.1}     \\
			& \methodwithsguide     & \textbf{0.823}(0.029) & 0.217(0.032)          &  & \textbf{0.900}(0.028) & 0.223(0.048)  \\ % & 86.0             \\ 
		%\cmidrule{2-7}
		%& improv\% & 10.5 & -2.8 &  & 7.0 & -2.3  \\ % & %-12.9  \\
		\midrule
		\multirow{6}{0.059\linewidth}{\hspace{0pt}0.5} & \dataset & 0.750(0.037)          & \textbf{0.225}(0.037) &  & 0.819(0.039)          & \textbf{0.236}(0.070)          \\ %& \textbf{100.0}   \\
			& \squid($\lambda$=0.3)  & 0.728(0.072)          & 0.301(0.054)          &  & \underline{0.888}(0.061)  & 0.355(0.088)          \\ %& 85.9             \\
			& \squid($\lambda$=1.0)  & 0.699(0.063)          & 0.233(0.043)          &  & 0.850(0.057)          & 0.263(0.080)          \\ %& \underline{99.5}     \\
			& \method               & \underline{0.771}(0.039)  & \underline{0.229}(0.043)  &  & 0.862(0.036)          & \textbf{0.236}(0.065) \\ %& 99.2             \\
			& \methodwithsguide    & \textbf{0.824}(0.029) & \underline{0.229}(0.044)  &  & \textbf{0.903}(0.027) & \underline{0.242}(0.069)  \\ %& 99.0             \\ 
		%\cmidrule{2-7}
		%& improv\% & 9.9 & -1.8 &  & 1.7 & 0.0 \\ %& -0.8  \\
		\midrule
		\multirow{6}{0.059\linewidth}{\hspace{0pt}0.7} 
		& \dataset &  0.750(0.037) & \textbf{0.226}(0.038) & & 0.819(0.039) & \underline{0.240}(0.081) \\ %& \textbf{100.0} \\
		%& \dataset & 12.3            & 0.736(0.076)          & 0.768(0.037)          & \textbf{0.228}(0.042) &  & 0.819(0.039)          & \underline{0.242}(0.085)  & \textbf{100.0}   \\
			& \squid($\lambda$=0.3) &  0.735(0.074)          & 0.328(0.070)          &  & \underline{0.900}(0.062)  & 0.435(0.143)          \\ %& 95.4             \\
			& \squid($\lambda$=1.0) &  0.699(0.064)          & 0.234(0.045)          &  & 0.851(0.057)          & 0.268(0.090)          \\ %& \underline{99.9}     \\
			& \method               &  \underline{0.771}(0.039)  & \underline{0.229}(0.043)  &  & 0.862(0.036)          & \textbf{0.237}(0.066) \\ %& 99.3             \\
			& \methodwithsguide     &  \textbf{0.824}(0.029) & 0.230(0.045)          &  & \textbf{0.903}(0.027) & 0.244(0.074)          \\ %& 99.2             \\ 
		%\cmidrule{2-7}
		%& improv\% & 9.9 & -1.3 &  & 0.3 & 1.3 \\%& -0.7  \\
		\midrule
		\multirow{6}{0.059\linewidth}{\hspace{0pt}1.0} 
		& \dataset & 0.750(0.037)          & \textbf{0.226}(0.038) &  & 0.819(0.039)          & \underline{0.242}(0.085)  \\%& \textbf{100.0}  \\
		& \squid($\lambda$=0.3) & 0.740(0.076)          & 0.349(0.088)          &  & \textbf{0.909}(0.065) & 0.547(0.245)       \\ %   & \textbf{100.0}  \\
		& \squid($\lambda$=1.0) & 0.699(0.064)          & 0.235(0.045)          &  & 0.851(0.057)          & 0.271(0.097)          \\ %& \textbf{100.0}   \\
		& \method               & \underline{0.771}(0.039)  & \underline{0.229}(0.043)  &  & 0.862(0.036)          & \textbf{0.237}(0.066) \\ %& \underline{99.3}  \\
		& \methodwithsguide      & \textbf{0.824}(0.029) & 0.230(0.045)          &  & \underline{0.903}(0.027)  & 0.244(0.076)          \\ %& 99.2            \\
		%\cmidrule{2-7}
		%& improv\% &  9.9               & -1.3              &  & -0.7              & -2.1           \\ %       & -0.7 \\
		\bottomrule
	\end{tabular}%
	\begin{tablenotes}[normal,flushleft]
		\begin{footnotesize}
	\item 
\!\!Columns represent: ``$\delta_g$'': the graph similarity constraint; 
%``\#d\%'': the percentage of molecules that satisfy the graph similarity constraint and are with high \shapesim ($\shapesim>=0.8$);
%``\diversity'': the diversity among the generated molecules;
``\avgshapesim/\avggraphsim'': the average of shape or graph similarities between the condition molecules and generated molecules with $\graphsim<=\delta_g$;
``\maxshapesim'': the maximum of shape similarities between the condition molecules and generated molecules with $\graphsim<=\delta_g$;
``\maxgraphsim'': the graph similarities between the condition molecules and the molecules with the maximum shape similarities and $\graphsim<=\delta_g$;
%``\#n\%'': the percentage of molecules that satisfy the graph similarity constraint ($\graphsim<=\delta_g$).
%
``$\uparrow$'' represents higher values are better, and ``$\downarrow$'' represents lower values are better.
%
 Best values are in \textbf{bold}, and second-best values are \underline{underlined}. 
\par
		\par
		\end{footnotesize}
	\end{tablenotes}
\end{scriptsize}
\end{threeparttable}
  \vspace{-10pt}    
\end{table*}
%\label{tbl:overall_sim}


{We evaluate the shape similarity \shapesim and graph similarity \graphsim of molecules generated from}
%Table~\ref{tbl:overall_sim} presents the comparison of shape-conditioned molecule generation among 
\dataset, \squid, \method and \methodwithsguide under different graph similarity constraints  ($\delta_g$=1.0, 0.7, 0.5, 0.3). 
%
%During the evaluation, for each molecule in the test set, all the methods are employed to generate or identify 50 molecules with similar shapes.
%
We calculate evaluation metrics using all the generated molecules satisfying the graph similarity constraints.
%
Particularly, when $\delta_g$=1.0, we do not filter out any molecules based on the constraints and directly calculate metrics on all the generated molecules.
%
When $\delta_g$=0.7, 0.5 or 0.3, we consider only generated molecules with similarities lower than $\delta_g$.
%
Based on \shapesim and \graphsim as described in Section ``Evaluation Metrics'' in the main manuscript,
we calculate the following metrics using the subset of molecules with \graphsim lower than $\delta_g$, from a set of 50 generated molecules for each test molecule and report the average of  these metrics across all test molecules:
%
(1) \avgshapesim\ measures the average \shapesim across each subset of generated molecules with $\graphsim$ lower than $\delta_g$; %per test molecule, with the overall average calculated across all test molecules; }%the 50 generated molecules for each test molecule, averaged across all test molecules;
(2) \avggraphsim\ calculates the average \graphsim for each set; %, with these means averaged across all test molecules}; %} 50 molecules, %\bo{@Ziqi rephrase}, with results averaged on the test set;\ziqi{with the average computed over the test set; }
(3) \maxshapesim\ determines the maximum \shapesim within each set; %, with these maxima averaged across all test molecules; }%\hl{among 50 molecules}, averaged across all test molecules;
(4) \maxgraphsim\ measures the \graphsim of the molecule with maximum \shapesim in each set. %, averaged across all test molecules; }%\hl{among 50 molecules}, averaged across all test molecules;

%
As shown in Table~\ref{tbl:overall_sim}, \method and \methodwithsguide demonstrate outstanding performance in terms of the average shape similarities (\avgshapesim) and the average graph similarities (\avggraphsim) among generated molecules.
%
%\ziqi{
%Table~\ref{tbl:overall} also shows that \method and \methodwithsguide consistently outperform all the baseline methods in average shape similarities (\avgshapesim) and only slightly underperform 
%the best baseline \dataset in average graph similarities (\avggraphsim).
%}
%
Specifically, when $\delta_g$=0.3, \methodwithsguide achieves a substantial 10.5\% improvement in \avgshapesim\ over the best baseline \dataset. 
%
In terms of \avggraphsim, \methodwithsguide also achieves highly comparable performance with \dataset (0.217 vs 0.211, in \avggraphsim, lower values indicate better performance).
%
%This trend remains consistent across various $\delta_g$ values.
This trend remains consistent when applying various similarity constraints (i.e., $\delta_g$) as shown in Table~\ref{tbl:overall_sim}.


Similarly, \method and \methodwithsguide demonstrate superior performance in terms of the average maximum shape similarity across generated molecules for all test molecules (\maxshapesim), as well as the average graph similarity of the molecules with the maximum shape similarities (\maxgraphsim). %maximum shape similarities of generated molecules (\maxshapesim) and the average graph similarities of molecules with the maximum shape similarities (\maxgraphsim). %\bo{\maxgraphsim is misleading... how about $\text{avgMSim}_\text{g}$}
%
%\bo{
%in terms of the maximum shape similarities (\maxshapesim) and the maximum graph similarities (\maxgraphsim) among all the generated molecules.
%@Ziqi are the metrics maximum values or the average of maximum values?
%}
%
Specifically, at \maxshapesim, Table~\ref{tbl:overall_sim} shows that \methodwithsguide outperforms the best baseline \squid ($\lambda$=0.3) when $\delta_g$=0.3, 0.5, and 0.7, and only underperforms
it by 0.7\% when $\delta$=1.0.
%
We also note that the molecules generated by {\methodwithsguide} with the maximum shape similarities have substantially lower graph similarities ({\maxgraphsim}) compared to those generated by {\squid} ({$\lambda$}=0.3).
%\hl{We also note that the molecules with the maximum shape similarities generated by {\methodwithsguide} are with significantly lower graph similarities ({\maxgraphsim}) than those generated by {\squid} ({$\lambda$}=0.3).}
%
%\bo{@Ziqi please rephrase the language}
%
%\bo{
%@Ziqi the conclusion is not obvious. You may want to remind the meaning of \maxshapesim and \maxgraphsim here, and based on what performance you say this.
%}
%
%\bo{\st{This also underscores the ability of {\methodwithsguide} in generating molecules with similar shapes to condition molecules and novel graph structures.}}
%
As evidenced by these results, \methodwithsguide features strong capacities of generating molecules with similar shapes yet novel graph structures compared to the condition molecule, facilitating the discovery of promising drug candidates.
%

\begin{comment}
\ziqi{replace \#n\% with the percentage of novel molecules that do not exist in the dataset and update the discussion accordingly}
%\ziqi{
Table~\ref{tbl:overall_sim} also presents \bo{\#n\%}, the percentage of molecules generated by each method %\st{(\#n\%)} 
with graph similarities lower than the constraint $\delta_g$. 
%
%\bo{
%Table~\ref{tbl:overall_sim} also presents \#n\%, the percentage of generated molecules with graph similarities lower than the constraint $\delta_g$, of different methods. 
%}
%
As shown in Table~\ref{tbl:overall_sim},  when a restricted constraint (i.e., $\delta_g$=0.3) is applied, \method and \methodwithsguide could still generate a sufficient number of molecules satisfying the constraint.
%
Particularly, when $\delta_g$=0.3, \method outperforms \squid with $\lambda$=0.3 by XXX and \squid with $\lambda$=1.0 by XXX.
% achieve the second and the third in \#n\% and only underperform the best baseline \dataset.
%
This demonstrates the ability of \method in generating molecules with novel structures. 
%
When $\delta_g$=0.5, 0.7 and 1.0, both methods generate over 99.0\% of molecules satisfying the similarity constraint $\delta_g$.
%
%Note that \dataset is guaranteed to identify at least 50 molecules satisfying the $\delta_g$ by searching within a training dataset of diverse molecules.
%
Note that \dataset is a search algorithm that always first identifies the molecules satisfying $\delta_g$ and then selects the top-50 molecules of the highest shape similarities among them. 
%
Due to the diverse molecules in %\hl{the subset} \bo{@Ziqi why do you want to stress subset?} of 
the training set, \dataset can always identify at least 50 molecules under different $\delta_g$ and thus achieve 100\% in \#n\%.
%
%\bo{
%Note that \dataset is a search algorithm that always generate molecules XXX
%@Ziqi
%We need to discuss here. For \dataset, \#n\% in this table does not look aligned with that in Fig 1 if the highlighted defination is correct...
%}
%
%Thus, \dataset achieves 100.0\% in \#n\% under different $\delta_g$.
%
It is also worth noting that when $\delta_g$=1.0, \#n\% reflects the validity among all the generated molecules. 
%
As shown in Table~\ref{tbl:overall_sim}, \method and \methodwithsguide are able to generate 99.3\% and 99.2\% valid molecules.
%
This demonstrates their ability to effectively capture the underlying chemical rules in a purely data-driven manner without relying on any prior knowledge (e.g., fragments) as \squid does.
%
%\bo{
%@Ziqi I feel this metric is redundant with the avg graph similarity when constraint is 1.0. Generally, if the avg similarity is small. You have more mols satisfying the requirement right?
%}
\end{comment}

Table~\ref{tbl:overall_sim} also shows that by incorporating shape guidance, \methodwithsguide
%\bo{
%@Ziqi where does this come from...
%}
substantially outperforms \method in both \avgshapesim and \maxshapesim, while maintaining comparable graph similarities (i.e., \avggraphsim\ and \maxgraphsim).
%
Particularly, when $\delta_g$=0.3, \methodwithsguide 
establishes a considerable improvement of 6.9\% and 4.9\%
%\bo{\st{achieves 6.9\% and 4.9\% improvements}} 
over \method in \avgshapesim and \maxshapesim, respectively. 
%
%\hl{In the meanwhile}, 
%\bo{@Ziqi it is not the right word...}
Meanwhile, \methodwithsguide achieves the same \avggraphsim with \method and only slightly underperforms \method in \maxgraphsim (0.223 vs 0.220).
%\bo{
%XXX also achieves XXX
%}
%it maintains the same \avggraphsim\ with \method and only slightly underperforms \method in \maxgraphsim (0.223 vs 0.220).
%
%Compared with \method, \methodwithsguide consistently generates molecules with higher shape similarities while maintaining comparable graph similarities.
%
%\bo{
%@Ziqi you may want to highlight the utility of "generating molecules with higher shape similarities while maintaining comparable graph similarities" in real drug discovery applications.
%
%
%\bo{
%@Ziqi You did not present the details of method yet...
%}
%
%\methodwithsguide leverages additional shape guidance to push the predicted atoms to the shape of condition molecules \bo{and XXX (@Ziqi boosts the shape similarities XXX)} , as will be discussed in Section ``\method with Shape Guidance'' later.
%
The superior performance of \methodwithsguide suggests that the incorporation of shape guidance effectively boosts the shape similarities of generated molecules without compromising graph similarities.
%
%This capability could be crucial in drug discovery, 
%\bo{@Ziqi it is a strong statement. Need citations here}, 
%as it enables the discovery of drug candidates that are both more potentially effective due to the improved shape similarities and novel induced by low graph similarities.
%as it could enable the identification of candidates with similar binding patterns %with the condition molecule (i.e., high shape similarities) 
%(i.e., high shape similarities) and graph structures distinct from the condition molecules (i.e., low graph similarities).
%\bo{\st{and enjoys novel structures (i.e., low graph similarities) with potentially better properties. } \ziqi{change enjoys}}
%\bo{
%and enjoys potentially better properties (i.e., low graph similarities). \ziqi{this looks weird to me... need to discuss}
%}
%\st{potentially better properties (i.e., low graph similarities).}}

%-------------------------------------------------------------------------------------------------------------------------------------
\subsection{Comparison on Validity and Novelty}
\label{supp:app:results:valid_novel}
%-------------------------------------------------------------------------------------------------------------------------------------

We evaluate the ability of \method and \squid to generate molecules with valid and novel 2D molecular graphs.
%
We calculate the percentages of the valid and novel molecules among all the generated molecules.
%
As shown in Table~\ref{tbl:validity_novelty}, both \method and \methodwithsguide outperform \squid with $\lambda$=0.3 and $\lambda$=1.0 in generating novel molecules.
%
Particularly, almost all valid molecules generated by \method and \methodwithsguide are novel (99.8\% and 99.9\% at \#n\%), while the best baseline \squid with $\lambda$=0.3 achieves 98.4\% in novelty.
%
In terms of the percentage of valid and novel molecules among all the generated ones (\#v\&n\%), \method and \methodwithsguide again outperform \squid with $\lambda$=0.3 and $\lambda$=1.0.
%
We also note that at \#v\%,  \method (99.1\%) and \methodwithsguide (99.2\%) slightly underperform \squid with $\lambda$=0.3 and $\lambda$=1.0 (100.0\%) in generating valid molecules.
%
\squid guarantees the validity of generated molecules by incorporating valence rules into the generation process and ensuring it to avoid fragments that violate these rules.
%
Conversely, \method and \methodwithsguide use a purely data-driven approach to learn the generation of valid molecules.
%
These results suggest that, even without integrating valence rules, \method and \methodwithsguide can still achieve a remarkably high percentage of valid and novel generated molecules.

\begin{table*}
	\centering
		\caption{Comparison on Validity and Novelty between \method and \squid}
	\label{tbl:validity_novelty}
	\begin{scriptsize}
\begin{threeparttable}
%	\setlength\tabcolsep{0pt}
	\begin{tabular}{
		@{\hspace{3pt}}l@{\hspace{10pt}}
		%
		@{\hspace{10pt}}r@{\hspace{10pt}}
		%
		@{\hspace{10pt}}r@{\hspace{10pt}}
		%
		@{\hspace{10pt}}r@{\hspace{3pt}}
		}
		\toprule
		method & \#v\% & \#n\% & \#v\&n\% \\
		\midrule
		\squid ($\lambda$=0.3) & \textbf{100.0} & 96.7 & 96.7 \\
		\squid ($\lambda$=1.0) & \textbf{100.0} & 98.4 & 98.4 \\
		\method & 99.1 & 99.8 & 98.9 \\
		\methodwithsguide & 99.2 & \textbf{99.9} & \textbf{99.1} \\
		\bottomrule
	\end{tabular}%
	%
	\begin{tablenotes}[normal,flushleft]
		\begin{footnotesize}
	\item 
\!\!Columns represent: ``\#v\%'': the percentage of generated molecules that are valid;
		``\#n\%'': the percentage of valid molecules that are novel;
		``\#v\&n\%'': the percentage of generated molecules that are valid and novel.
		Best values are in \textbf{bold}. 
		\par
		\end{footnotesize}
	\end{tablenotes}
\end{threeparttable}
\end{scriptsize}
\end{table*}


%-------------------------------------------------------------------------------------------------------------------------------------
\subsection{Additional Quality Comparison between Desirable Molecules Generated by \method and \squid}
\label{supp:app:results:quality_desirable}
%-------------------------------------------------------------------------------------------------------------------------------------

\begin{table*}[!h]
	\centering
		\caption{Comparison on Quality of Generated Desirable Molecules between \method and \squid ($\delta_g$=0.5)}
	\label{tbl:overall_results_quality_05}
	\begin{scriptsize}
\begin{threeparttable}
	\begin{tabular}{
		@{\hspace{0pt}}l@{\hspace{16pt}}
		@{\hspace{0pt}}l@{\hspace{2pt}}
		%
		@{\hspace{6pt}}c@{\hspace{6pt}}
		%
		%@{\hspace{3pt}}c@{\hspace{3pt}}
		@{\hspace{3pt}}c@{\hspace{3pt}}
		@{\hspace{3pt}}c@{\hspace{3pt}}
		@{\hspace{3pt}}c@{\hspace{3pt}}
		@{\hspace{3pt}}c@{\hspace{3pt}}
		%
		%
		}
		\toprule
		group & metric & 
        %& \dataset 
        & \squid ($\lambda$=0.3) & \squid ($\lambda$=1.0)  &  \method & \methodwithsguide  \\
		%\multirow{2}{*}{method} & \multirow{2}{*}{\#c\%} &  \multirow{2}{*}{\#u\%} &  \multirow{2}{*}{QED} & \multicolumn{3}{c}{$\nmax=50$} & & \multicolumn{2}{c}{$\nmax=1$}\\
		%\cmidrule(r){5-7} \cmidrule(r){8-10} 
		%& & & & \avgshapesim(std) & \avggraphsim(std  &  \diversity(std  & & \avgshapesim(std) & \avggraphsim(std \\
		\midrule
		\multirow{2}{*}{stability}
		& atom stability ($\uparrow$) & 
        %& 0.990 
        & \textbf{0.996} & 0.995 & 0.992 & 0.989     \\
		& mol stability ($\uparrow$) & 
        %& 0.875 
        & \textbf{0.948} & 0.947 & 0.886 & 0.839    \\
		%\midrule
		%\multirow{3}{*}{Drug-likeness} 
		%& QED ($\uparrow$) & 
        %& \textbf{0.805} 
        %& 0.766 & 0.760 & 0.755 & 0.751    \\
	%	& SA ($\uparrow$) & 
        %& \textbf{0.874} 
        %& 0.814 & 0.813 & 0.699 & 0.692    \\
	%	& Lipinski ($\uparrow$) & 
        %& \textbf{4.999} 
        %& 4.979 & 4.980 & 4.967 & 4.975    \\
		\midrule
		\multirow{4}{*}{3D structures} 
		& RMSD ($\downarrow$) & 
        %& \textbf{0.419} 
        & 0.907 & 0.906 & 0.897 & \textbf{0.881}    \\
		& JS. bond lengths ($\downarrow$) & 
        %& \textbf{0.286} 
        & 0.457 & 0.477 & 0.436 & \textbf{0.428}    \\
		& JS. bond angles ($\downarrow$) & 
        %& \textbf{0.078} 
        & 0.269 & 0.289 & \textbf{0.186} & 0.200    \\
		& JS. dihedral angles ($\downarrow$) & 
        %& \textbf{0.151} 
        & 0.199 & 0.209 & \textbf{0.168} & 0.170    \\
		\midrule
		\multirow{5}{*}{2D structures} 
		& JS. \#bonds per atoms ($\downarrow$) & 
        %& 0.325 
        & 0.291 & 0.331 & \textbf{0.176} & 0.181    \\
		& JS. basic bond types ($\downarrow$) & 
        %& \textbf{0.055} 
        & \textbf{0.071} & 0.083 & 0.181 & 0.191    \\
		%& JS. freq. bond types ($\downarrow$) & 
        %& \textbf{0.089} 
        %& 0.123 & 0.130 & 0.245 & 0.254    \\
		%& JS. freq. bond pairs ($\downarrow$) & 
        %& \textbf{0.078} 
        %& 0.085 & 0.089 & 0.209 & 0.221    \\
		%& JS. freq. bond triplets ($\downarrow$) & 
        %& \textbf{0.089} 
        %& 0.097 & 0.114 & 0.211 & 0.223    \\
		%\midrule
		%\multirow{3}{*}{Rings} 
		& JS. \#rings ($\downarrow$) & 
        %& 0.142 
        & 0.280 & 0.330 & \textbf{0.043} & 0.049    \\
		& JS. \#n-sized rings ($\downarrow$) & 
        %& \textbf{0.055} 
        & \textbf{0.077} & 0.091 & 0.099 & 0.112    \\
		& \#Intersecting rings ($\uparrow$) & 
        %& \textbf{6} 
        & \textbf{6} & 5 & 4 & 5    \\
		%\method (+bt)            & 100.0 & 98.0 & 100.0 & 0.742 & 0.772 (0.040) & 0.211 (0.033) & & 0.862 (0.036) & 0.211 (0.033) & 0.743 (0.043) \\
		%\methodwithguide (+bt)    & 99.8 & 98.0 & 100.0 & 0.736 & 0.814 (0.031) & 0.193 (0.042) & & 0.895 (0.029) & 0.193 (0.042) & 0.745 (0.045) \\
		%
		\bottomrule
	\end{tabular}%
	\begin{tablenotes}[normal,flushleft]
		\begin{footnotesize}
	\item 
\!\!Rows represent:  {``atom stability'': the proportion of stable atoms that have the correct valency; 
		``molecule stability'': the proportion of generated molecules with all atoms stable;
		%``QED'': the drug-likeness score;
		%``SA'': the synthesizability score;
		%``Lipinski'': the Lipinski 
		``RMSD'': the root mean square deviation (RMSD) between the generated 3D structures of molecules and their optimal conformations; % identified via energy minimization;
		``JS. bond lengths/bond angles/dihedral angles'': the Jensen-Shannon (JS) divergences of bond lengths, bond angles and dihedral angles;
		``JS. \#bonds per atom/basic bond types/\#rings/\#n-sized rings'': the JS divergences of bond counts per atom, basic bond types, counts of all rings, and counts of n-sized rings;
		%``JS. \#rings/\#n-sized rings'': the JS divergences of the total counts of rings and the counts of n-sized rings;
		``\#Intersecting rings'': the number of rings observed in the top-10 frequent rings of both generated and real molecules. } \par
		\par
		\end{footnotesize}
	\end{tablenotes}
\end{threeparttable}
\end{scriptsize}
\end{table*}

%\label{tbl:overall_quality05}

\begin{table*}[!h]
	\centering
		\caption{Comparison on Quality of Generated Desirable Molecules between \method and \squid ($\delta_g$=0.7)}
	\label{tbl:overall_results_quality_07}
	\begin{scriptsize}
\begin{threeparttable}
	\begin{tabular}{
		@{\hspace{0pt}}l@{\hspace{14pt}}
		@{\hspace{0pt}}l@{\hspace{2pt}}
		%
		@{\hspace{4pt}}c@{\hspace{4pt}}
		%
		%@{\hspace{3pt}}c@{\hspace{3pt}}
		@{\hspace{3pt}}c@{\hspace{3pt}}
		@{\hspace{3pt}}c@{\hspace{3pt}}
		@{\hspace{3pt}}c@{\hspace{3pt}}
		@{\hspace{3pt}}c@{\hspace{3pt}}
		%
		%
		}
		\toprule
		group & metric & 
        %& \dataset 
        & \squid ($\lambda$=0.3) & \squid ($\lambda$=1.0)  &  \method & \methodwithsguide  \\
		%\multirow{2}{*}{method} & \multirow{2}{*}{\#c\%} &  \multirow{2}{*}{\#u\%} &  \multirow{2}{*}{QED} & \multicolumn{3}{c}{$\nmax=50$} & & \multicolumn{2}{c}{$\nmax=1$}\\
		%\cmidrule(r){5-7} \cmidrule(r){8-10} 
		%& & & & \avgshapesim(std) & \avggraphsim(std  &  \diversity(std  & & \avgshapesim(std) & \avggraphsim(std \\
		\midrule
		\multirow{2}{*}{stability} 
		& atom stability ($\uparrow$) & 
        %&  0.990 
        & \textbf{0.995} & 0.995 & 0.992 & 0.988 \\
		& molecule stability ($\uparrow$) & 
        %& 0.876 
        & 0.944 & \textbf{0.947} & 0.885 & 0.839 \\
		\midrule
		%\multirow{3}{*}{Drug-likeness} 
		%& QED ($\uparrow$) & 
        %& \textbf{0.805} 
        %& 0.766 & 0.760 & 0.755 & 0.751    \\
	%	& SA ($\uparrow$) & 
        %& \textbf{0.874} 
        %& 0.814 & 0.813 & 0.699 & 0.692    \\
	%	& Lipinski ($\uparrow$) & 
        %& \textbf{4.999} 
        %& 4.979 & 4.980 & 4.967 & 4.975    \\
	%	\midrule
		\multirow{4}{*}{3D structures} 
		& RMSD ($\downarrow$) & 
        %& \textbf{0.420} 
        & 0.897 & 0.906 & 0.897 & \textbf{0.881}    \\
		& JS. bond lengths ($\downarrow$) & 
        %& \textbf{0.286} 
        & 0.457 & 0.477 & 0.436 & \textbf{0.428}    \\
		& JS. bond angles ($\downarrow$) & 
        %& \textbf{0.078} 
        & 0.269 & 0.289 & \textbf{0.186} & 0.200    \\
		& JS. dihedral angles ($\downarrow$) & 
        %& \textbf{0.151} 
        & 0.199 & 0.209 & \textbf{0.168} & 0.170    \\
		\midrule
		\multirow{5}{*}{2D structures} 
		& JS. \#bonds per atoms ($\downarrow$) & 
        %& 0.325 
        & 0.285 & 0.329 & \textbf{0.176} & 0.181    \\
		& JS. basic bond types ($\downarrow$) & 
        %& \textbf{0.055} 
        & \textbf{0.067} & 0.083 & 0.181 & 0.191    \\
	%	& JS. freq. bond types ($\downarrow$) & 
        %& \textbf{0.089} 
        %& 0.123 & 0.130 & 0.245 & 0.254    \\
	%	& JS. freq. bond pairs ($\downarrow$) & 
        %& \textbf{0.078} 
        %& 0.085 & 0.089 & 0.209 & 0.221    \\
	%	& JS. freq. bond triplets ($\downarrow$) & 
        %& \textbf{0.089} 
        %& 0.097 & 0.114 & 0.211 & 0.223    \\
	%	\midrule
	%	\multirow{3}{*}{Rings} 
		& JS. \#rings ($\downarrow$) & 
        %& 0.143 
        & 0.273 & 0.328 & \textbf{0.043} & 0.049    \\
		& JS. \#n-sized rings ($\downarrow$) & 
        %& \textbf{0.055} 
        & \textbf{0.076} & 0.091 & 0.099 & 0.112    \\
		& \#Intersecting rings ($\uparrow$) & 
        %& \textbf{6} 
        & \textbf{6} & 5 & 4 & 5    \\
		%\method (+bt)            & 100.0 & 98.0 & 100.0 & 0.742 & 0.772 (0.040) & 0.211 (0.033) & & 0.862 (0.036) & 0.211 (0.033) & 0.743 (0.043) \\
		%\methodwithguide (+bt)    & 99.8 & 98.0 & 100.0 & 0.736 & 0.814 (0.031) & 0.193 (0.042) & & 0.895 (0.029) & 0.193 (0.042) & 0.745 (0.045) \\
		%
		\bottomrule
	\end{tabular}%
	\begin{tablenotes}[normal,flushleft]
		\begin{footnotesize}
	\item 
\!\!Rows represent:  {``atom stability'': the proportion of stable atoms that have the correct valency; 
		``molecule stability'': the proportion of generated molecules with all atoms stable;
		%``QED'': the drug-likeness score;
		%``SA'': the synthesizability score;
		%``Lipinski'': the Lipinski 
		``RMSD'': the root mean square deviation (RMSD) between the generated 3D structures of molecules and their optimal conformations; % identified via energy minimization;
		``JS. bond lengths/bond angles/dihedral angles'': the Jensen-Shannon (JS) divergences of bond lengths, bond angles and dihedral angles;
		``JS. \#bonds per atom/basic bond types/\#rings/\#n-sized rings'': the JS divergences of bond counts per atom, basic bond types, counts of all rings, and counts of n-sized rings;
		%``JS. \#rings/\#n-sized rings'': the JS divergences of the total counts of rings and the counts of n-sized rings;
		``\#Intersecting rings'': the number of rings observed in the top-10 frequent rings of both generated and real molecules. } \par
		\par
		\end{footnotesize}
	\end{tablenotes}
\end{threeparttable}
\end{scriptsize}
\end{table*}

%\label{tbl:overall_quality07}

\begin{table*}[!h]
	\centering
		\caption{Comparison on Quality of Generated Desirable Molecules between \method and \squid ($\delta_g$=1.0)}
	\label{tbl:overall_results_quality_10}
	\begin{scriptsize}
\begin{threeparttable}
	\begin{tabular}{
		@{\hspace{0pt}}l@{\hspace{14pt}}
		@{\hspace{0pt}}l@{\hspace{2pt}}
		%
		@{\hspace{4pt}}c@{\hspace{4pt}}
		%
		%@{\hspace{3pt}}c@{\hspace{3pt}}
		@{\hspace{3pt}}c@{\hspace{3pt}}
		@{\hspace{3pt}}c@{\hspace{3pt}}
		@{\hspace{3pt}}c@{\hspace{3pt}}
		@{\hspace{3pt}}c@{\hspace{3pt}}
		%
		%
		}
		\toprule
		group & metric & 
        %& \dataset 
        & \squid ($\lambda$=0.3) & \squid ($\lambda$=1.0)  &  \method & \methodwithsguide \\
		%\multirow{2}{*}{method} & \multirow{2}{*}{\#c\%} &  \multirow{2}{*}{\#u\%} &  \multirow{2}{*}{QED} & \multicolumn{3}{c}{$\nmax=50$} & & \multicolumn{2}{c}{$\nmax=1$}\\
		%\cmidrule(r){5-7} \cmidrule(r){8-10} 
		%& & & & \avgshapesim(std) & \avggraphsim(std  &  \diversity(std  & & \avgshapesim(std) & \avggraphsim(std \\
		\midrule
		\multirow{2}{*}{stability}
		& atom stability ($\uparrow$) & 
        %& 0.990 
        & \textbf{0.995} & \textbf{0.995} & 0.992 & 0.988     \\
		& mol stability ($\uparrow$) & 
        %& 0.876 
        & 0.942 & \textbf{0.947} & 0.885 & 0.839    \\
		\midrule
	%	\multirow{3}{*}{Drug-likeness} 
	%	& QED ($\uparrow$) & 
        %& \textbf{0.805} 
        %& \textbf{0.766} & 0.760 & 0.755 & 0.751    \\
	%	& SA ($\uparrow$) & 
        %& \textbf{0.874} 
        %& \textbf{0.813} & \textbf{0.813} & 0.699 & 0.692    \\
	%	& Lipinski ($\uparrow$) & 
        %& \textbf{4.999} 
        %& 4.979 & \textbf{4.980} & 4.967 & 4.975    \\
	%	\midrule
		\multirow{4}{*}{3D structures} 
		& RMSD ($\downarrow$) & 
        %& \textbf{0.420} 
        & 0.898 & 0.906 & 0.897 & \textbf{0.881}    \\
		& JS. bond lengths ($\downarrow$) & 
        %& \textbf{0.286} 
        & 0.457 & 0.477 & 0.436 & \textbf{0.428}    \\
		& JS. bond angles ($\downarrow$) & 
        %& \textbf{0.078} 
        & 0.269 & 0.289 & \textbf{0.186} & 0.200   \\
		& JS. dihedral angles ($\downarrow$) & 
        %& \textbf{0.151} 
        & 0.199 & 0.209 & \textbf{0.168} & 0.170    \\
		\midrule
		\multirow{5}{*}{2D structures} 
		& JS. \#bonds per atoms ($\downarrow$) & 
        %& 0.325 
        & 0.280 & 0.330 & \textbf{0.176} & 0.181    \\
		& JS. basic bond types ($\downarrow$) & 
        %& \textbf{0.055} 
        & \textbf{0.066} & 0.083 & 0.181 & 0.191   \\
	%	& JS. freq. bond types ($\downarrow$) & 
        %& \textbf{0.089} 
        %& \textbf{0.123} & 0.130 & 0.245 & 0.254    \\
	%	& JS. freq. bond pairs ($\downarrow$) & 
        %& \textbf{0.078} 
        %& \textbf{0.085} & 0.089 & 0.209 & 0.221    \\
	%	& JS. freq. bond triplets ($\downarrow$) & 
        %& \textbf{0.089} 
        %& \textbf{0.097} & 0.114 & 0.211 & 0.223    \\
		%\midrule
		%\multirow{3}{*}{Rings} 
		& JS. \#rings ($\downarrow$) & 
        %& 0.143 
        & 0.269 & 0.328 & \textbf{0.043} & 0.049    \\
		& JS. \#n-sized rings ($\downarrow$) & 
        %& \textbf{0.055} 
        & \textbf{0.075} & 0.091 & 0.099 & 0.112    \\
		& \#Intersecting rings ($\uparrow$) & 
        %& \textbf{6} 
        & \textbf{6} & 5 & 4 & 5    \\
		%\method (+bt)            & 100.0 & 98.0 & 100.0 & 0.742 & 0.772 (0.040) & 0.211 (0.033) & & 0.862 (0.036) & 0.211 (0.033) & 0.743 (0.043) \\
		%\methodwithguide (+bt)    & 99.8 & 98.0 & 100.0 & 0.736 & 0.814 (0.031) & 0.193 (0.042) & & 0.895 (0.029) & 0.193 (0.042) & 0.745 (0.045) \\
		%
		\bottomrule
	\end{tabular}%
	\begin{tablenotes}[normal,flushleft]
		\begin{footnotesize}
	\item 
\!\!Rows represent:  {``atom stability'': the proportion of stable atoms that have the correct valency; 
		``molecule stability'': the proportion of generated molecules with all atoms stable;
		%``QED'': the drug-likeness score;
		%``SA'': the synthesizability score;
		%``Lipinski'': the Lipinski 
		``RMSD'': the root mean square deviation (RMSD) between the generated 3D structures of molecules and their optimal conformations; % identified via energy minimization;
		``JS. bond lengths/bond angles/dihedral angles'': the Jensen-Shannon (JS) divergences of bond lengths, bond angles and dihedral angles;
		``JS. \#bonds per atom/basic bond types/\#rings/\#n-sized rings'': the JS divergences of bond counts per atom, basic bond types, counts of all rings, and counts of n-sized rings;
		%``JS. \#rings/\#n-sized rings'': the JS divergences of the total counts of rings and the counts of n-sized rings;
		``\#Intersecting rings'': the number of rings observed in the top-10 frequent rings of both generated and real molecules. } \par
		\par
		\end{footnotesize}
	\end{tablenotes}
\end{threeparttable}
\end{scriptsize}
\end{table*}

%\label{tbl:overall_quality10}

Similar to Table~\ref{tbl:overall_results_quality_desired} in the main manuscript, we present the performance comparison on the quality of desirable molecules generated by different methods under different graph similarity constraints $\delta_g$=0.5, 0.7 and 1.0, as detailed in Table~\ref{tbl:overall_results_quality_05}, Table~\ref{tbl:overall_results_quality_07}, and Table~\ref{tbl:overall_results_quality_10}, respectively.
%
Overall, these tables show that under varying graph similarity constraints, \method and \methodwithsguide can always generate desirable molecules with comparable quality to baselines in terms of stability, 3D structures, and 2D structures.
%
These results demonstrate the strong effectiveness of \method and \methodwithsguide in generating high-quality desirable molecules with stable and realistic structures in both 2D and 3D.
%
This enables the high utility of \method and \methodwithsguide in discovering promising drug candidates.


\begin{comment}
The results across these tables demonstrate similar observations with those under $\delta_g$=0.3 in Table~\ref{tbl:overall_results_quality_desired}.
%
For stability, when $\delta_g$=0.5, 0.7 or 1.0, \method and \methodwithsguide achieve comparable performance or fall slightly behind \squid ($\lambda$=0.3) and \squid ($\lambda$=1.0) in atom stability and molecule stability.
%
For example, when $\delta_g$=0.5, as shown in Table~\ref{tbl:overall_results_quality_05}, \method achieves similar performance with the best baseline \squid ($\lambda$=0.3) in atom stability (0.992 for \method vs 0.996 for \squid with $\lambda$=0.3).
%
\method underperforms \squid ($\lambda$=0.3) in terms of molecule stability.
%
For 3D structures, \method and \methodwithsguide also consistently generate molecules with more realistic 3D structures compared to \squid.
%
Particularly, \methodwithsguide achieves the best performance in RMSD and JS of bond lengths across $\delta_g$=0.5, 0.7 and 1.0.
%
In JS of dihedral angles, \method achieves the best performance among all the methods.
%
\method and \methodwithsguide underperform \squid in JS of bond angles, primarily because \squid constrains the bond angles in the generated molecules.
%
For 2D structures, \method and \methodwithsguide again achieve the best performance 
\end{comment}

%===================================================================
\section{Additional Experimental Results on PMG}
\label{supp:app:results_PMG}
%===================================================================

%\label{tbl:comparison_results_decompdiff}


%-------------------------------------------------------------------------------------------------------------------------------------
%\subsection{{Additional Comparison for PMG}}
%\label{supp:app:results:docking}
%-------------------------------------------------------------------------------------------------------------------------------------

In this section, we present the results of \methodwithpguide and \methodwithsandpguide when generating 100 molecules. 
%
Please note that both \methodwithpguide and \methodwithsandpguide show remarkable efficiency over the PMG baselines.
%
\methodwithpguide and \methodwithsandpguide generate 100 molecules in 48 and 58 seconds on average, respectively, while the most efficient baseline \targetdiff requires 1,252 seconds.
%
We report the performance of \methodwithpguide and \methodwithsandpguide against state-of-the-art PMG baselines in Table~\ref{tbl:overall_results_docking_100}.


%
According to Table~\ref{tbl:overall_results_docking_100}, \methodwithpguide and \methodwithsandpguide achieve comparable performance with the PMG baselines in generating molecules with high binding affinities.
%
Particularly, in terms of Vina S, \methodwithsandpguide achieves very comparable performance (-4.56 kcal/mol) to the third-best baseline \decompdiff (-4.58 kcal/mol) in average Vina S; it also achieves the third-best performance (-4.82 kcal/mol) among all the methods and slightly underperforms the second-best baseline \AR (-4.99 kcal/mol) in median Vina S
%
\methodwithsandpguide also achieves very close average Vina M (-5.53 kcal/mol) with the third-best baseline \AR (-5.59 kcal/mol) and the third-best performance (-5.47 kcal/mol) in median Vina M.
%
Notably, for Vina D, \methodwithpguide and \methodwithsandpguide achieve the second and third performance among all the methods.
%
In terms of the average percentage of generated molecules with Vina D higher than those of known ligands (i.e., HA), \methodwithpguide (58.52\%) and \methodwithsandpguide (58.28\%) outperform the best baseline \targetdiff (57.57\%).
%
These results signify the high utility of \methodwithpguide and \methodwithsandpguide in generating molecules that effectively bind with protein targets and have better binding affinities than known ligands.

In addition to binding affinities, \methodwithpguide and \methodwithsandpguide also demonstrate similar performance compared to the baselines in metrics related to drug-likeness and diversity.
%
For drug-likeness, both \methodwithpguide and \methodwithsandpguide achieve the best (0.67) and the second-best (0.66) QED scores.
%
They also achieve the third and fourth performance in SA scores.
%
In terms of the diversity among generated molecules,  \methodwithpguide and \methodwithsandpguide slightly underperform the baselines, possibly due to the design that generates molecules with similar shapes to the ligands.
%
These results highlight the strong ability of \methodwithpguide and \methodwithsandpguide in efficiently generating effective binding molecules with favorable drug-likeness and diversity.
%
This ability enables them to potentially serve as promising tools to facilitate effective and efficient drug development.

\begin{table*}[!h]
	\centering
		\caption{Additional Comparison on PMG When All Methods Generate 100 Molecules}
	\label{tbl:overall_results_docking_100}
\begin{threeparttable}
	\begin{scriptsize}
	\begin{tabular}{
		@{\hspace{2pt}}l@{\hspace{2pt}}
		%
		@{\hspace{2pt}}r@{\hspace{2pt}}
		%
		@{\hspace{2pt}}r@{\hspace{2pt}}
		@{\hspace{2pt}}r@{\hspace{2pt}}
		%
		@{\hspace{6pt}}r@{\hspace{6pt}}
		%
		@{\hspace{2pt}}r@{\hspace{2pt}}
		@{\hspace{2pt}}r@{\hspace{2pt}}
		%
		@{\hspace{5pt}}r@{\hspace{5pt}}
		%
		@{\hspace{2pt}}r@{\hspace{2pt}}
		@{\hspace{2pt}}r@{\hspace{2pt}}
		%
		@{\hspace{5pt}}r@{\hspace{5pt}}
		%
		@{\hspace{2pt}}r@{\hspace{2pt}}
	         @{\hspace{2pt}}r@{\hspace{2pt}}
		%
		@{\hspace{5pt}}r@{\hspace{5pt}}
		%
		@{\hspace{2pt}}r@{\hspace{2pt}}
		@{\hspace{2pt}}r@{\hspace{2pt}}
		%
		@{\hspace{5pt}}r@{\hspace{5pt}}
		%
		@{\hspace{2pt}}r@{\hspace{2pt}}
		@{\hspace{2pt}}r@{\hspace{2pt}}
		%
		@{\hspace{5pt}}r@{\hspace{5pt}}
		%
		@{\hspace{2pt}}r@{\hspace{2pt}}
		@{\hspace{2pt}}r@{\hspace{2pt}}
		%
		@{\hspace{5pt}}r@{\hspace{5pt}}
		%
		@{\hspace{2pt}}r@{\hspace{2pt}}
		%@{\hspace{2pt}}r@{\hspace{2pt}}
		%@{\hspace{2pt}}r@{\hspace{2pt}}
		}
		\toprule
		\multirow{2}{*}{method} & \multicolumn{2}{c}{Vina S$\downarrow$} & & \multicolumn{2}{c}{Vina M$\downarrow$} & & \multicolumn{2}{c}{Vina D$\downarrow$} & & \multicolumn{2}{c}{{HA\%$\uparrow$}}  & & \multicolumn{2}{c}{QED$\uparrow$} & & \multicolumn{2}{c}{SA$\uparrow$} & & \multicolumn{2}{c}{Div$\uparrow$} & %& \multirow{2}{*}{SR\%$\uparrow$} & 
		& \multirow{2}{*}{time$\downarrow$} \\
	    \cmidrule{2-3}\cmidrule{5-6} \cmidrule{8-9} \cmidrule{11-12} \cmidrule{14-15} \cmidrule{17-18} \cmidrule{20-21}
		 & Avg. & Med. &  & Avg. & Med. &  & Avg. & Med. & & Avg. & Med.  & & Avg. & Med.  & & Avg. & Med.  & & Avg. & Med.  & & \\ %& & \\
		%\multirow{2}{*}{method} & \multirow{2}{*}{\#c\%} &  \multirow{2}{*}{\#u\%} &  \multirow{2}{*}{QED} & \multicolumn{3}{c}{$\nmax=50$} & & \multicolumn{2}{c}{$\nmax=1$}\\
		%\cmidrule(r){5-7} \cmidrule(r){8-10} 
		%& & & & \avgshapesim(std) & \avggraphsim(std  &  \diversity(std  & & \avgshapesim(std) & \avggraphsim(std \\
		\midrule
		Reference                          & -5.32 & -5.66 & & -5.78 & -5.76 & & -6.63 & -6.67 & & - & - & & 0.53 & 0.49 & & 0.77 & 0.77 & & - & - & %& 23.1 & 
		& - \\
		\midrule
		\AR & \textbf{-5.06} & -4.99 & &  -5.59 & -5.29 & &  -6.16 & -6.05 & &  37.69 & 31.00 & &  0.50 & 0.49 & &  0.66 & 0.65 & & 0.70 & 0.70 & %& 7.0 & 
		& 7,789 \\
		\pockettwomol   & -4.50 & -4.21 & &  -5.70 & -5.27 & &  -6.43 & -6.25 & &  48.00 & 51.00 & &  0.58 & 0.58 & &  \textbf{0.77} & \textbf{0.78} & &  0.69 & 0.71 &  %& 24.9 & 
		& 2,150 \\
		\targetdiff     & -4.88 & \textbf{-5.82} & &  \textbf{-6.20} & \textbf{-6.36} & &  \textbf{-7.37} & \textbf{-7.51} & &  57.57 & 58.27 & &  0.50 & 0.51 & &  0.60 & 0.59 & &  \textbf{0.72} & 0.71 & % & 10.4 & 
		& 1,252 \\
		%& \decompdiffbeta                    & 63.03 & %-4.72 & -4.86 & & \textbf{-6.84} & \textbf{-6.91} & & \textbf{-8.85} & \textbf{-8.90} & &  \textbf{72.16} & \textbf{72.16} & &  0.36 & 0.36 & &  0.55 & 0.55 & & 0.59 & 0.59 & & 14.9 \\ 
		%-4.76 & -6.18 & &  \textbf{-6.86} & \textbf{-7.52} & &  \textbf{-8.85} & \textbf{-8.96} & &  \textbf{72.7} & \textbf{89.8} & &  0.36 & 0.34 & &  0.55 & 0.57 & & 0.59 & 0.59 & & 15.4 \\
		\decompdiffref  & -4.58 & -4.77 & &  -5.47 & -5.51 & &  -6.43 & -6.56 & &  47.76 & 48.66 & &  0.56 & 0.56 & &  0.70 & 0.69  & &  \textbf{0.72} & \textbf{0.72} &  %& 15.2 & 
		& 1,859 \\
		\midrule
		%\method & 14.04 & 9.74 & &  -2.80 & -3.87 & &  -6.32 & -6.41 & &  42.37 & 40.40 & &  0.70 & 0.71 & &  0.73 & 0.72 & & 0.71 & 0.74 & & 42 \\
		%\methodwithsguide & 1.04 & -0.33 & &  -4.23 & -4.39 & &  -6.31 & -6.46 & &  46.18 & 44.00 & &  0.69 & 0.71 & &  0.72 & 0.71 & & 0.70 & 0.73 & 53 \\
		\methodwithpguide      & -4.15 & -4.59 & &  -5.41 & -5.34 & &  -6.49 & -6.74 & &  \textbf{58.52} & 59.00 & &  \textbf{0.67} & \textbf{0.69} & &  0.68 & 0.68 & & 0.67 & 0.70 & %& 28.0 & 
		& 48 \\
		\methodwithsandpguide  & -4.56 & -4.82 & &  -5.53 & -5.47 & &  -6.60 & -6.78 & &  58.28 & \textbf{60.00} & &  0.66 & 0.68 & &  0.67 & 0.66 & & 0.68 & 0.71 &
		& 58 \\
		\bottomrule
	\end{tabular}%
	\begin{tablenotes}[normal,flushleft]
		\begin{footnotesize}
	\item 
\!\!Columns represent: {``Vina S'': the binding affinities between the initially generated poses of molecules and the protein pockets; 
		``Vina M'': the binding affinities between the poses after local structure minimization and the protein pockets;
		``Vina D'': the binding affinities between the poses determined by AutoDock Vina~\cite{Eberhardt2021} and the protein pockets;
		``HA'': the percentage of generated molecules with Vina D higher than those of condition molecules;
		``QED'': the drug-likeness score;
		``SA'': the synthesizability score;
		``Div'': the diversity among generated molecules;
		``time'': the time cost to generate molecules.}
		\par
		\par
		\end{footnotesize}
	\end{tablenotes}
	\end{scriptsize}
\end{threeparttable}
\end{table*}


%\label{tbl:overall_results_docking_100}

%-------------------------------------------------------------------------------------------------------------------------------------
%\subsection{{Comparison of Pocket Guidance}}
%\label{supp:app:results:docking}
%-------------------------------------------------------------------------------------------------------------------------------------


\begin{comment}
%-------------------------------------------------------------------------------------------------------------------------------------
\subsection{\ziqi{Simiarity Comparison for Pocket-based Molecule Generation}}
%-------------------------------------------------------------------------------------------------------------------------------------


\begin{table*}[t!]
	\centering
	\caption{{Overall Comparison on Similarity for Pocket-based Molecule Generation}}
	\label{tbl:docking_results_similarity}
	\begin{small}
		\begin{threeparttable}
			\begin{tabular}{
					@{\hspace{0pt}}l@{\hspace{5pt}}
					%
					@{\hspace{3pt}}l@{\hspace{3pt}}
					%
					@{\hspace{3pt}}r@{\hspace{8pt}}
					@{\hspace{3pt}}c@{\hspace{3pt}}
					%
					@{\hspace{3pt}}c@{\hspace{3pt}}
					@{\hspace{3pt}}c@{\hspace{3pt}}
					%
					@{\hspace{0pt}}c@{\hspace{0pt}}
					%
					@{\hspace{3pt}}c@{\hspace{3pt}}
					@{\hspace{3pt}}c@{\hspace{3pt}}
					%
					@{\hspace{3pt}}r@{\hspace{3pt}}
				}
				\toprule
				$\delta_g$  & method          & \#d\%$\uparrow$ & $\diversity_d$$\uparrow$(std) & \avgshapesim$\uparrow$(std) & \avggraphsim$\downarrow$(std) & & \maxshapesim$\uparrow$(std) & \maxgraphsim$\downarrow$(std)       & \#n\%$\uparrow$  \\ 
				\midrule
				%\multirow{6}{0.059\linewidth}{\hspace{0pt}0.1} 
				%& \AR   & 4.4 & 0.781(0.076) & 0.511(0.197) & \textbf{0.056}(0.020) & & 0.619(0.222) & 0.074(0.024) & 21.4  \\
				%& \pockettwomol & 6.6 & 0.795(0.099) & 0.519(0.216) & 0.063(0.020) & & 0.608(0.236) & 0.076(0.022) & \textbf{24.1}  \\
				%& \targetdiff & 2.0 & 0.872(0.041) & 0.619(0.110) & 0.068(0.018) & & 0.721(0.146) & 0.075(0.023) & 17.7  \\
				%& \decompdiffbeta & 0.0 & - & 0.374(0.138) & 0.059(0.031) & & 0.414(0.141) & \textbf{0.058}(0.032) & 9.8  \\
				%& \decompdiffref & 8.5 & 0.805(0.096) & 0.810(0.070) & 0.076(0.018) & & 0.861(0.085) & 0.076(0.020) & 11.3  \\
				%& \methodwithpguide   &  9.9 & \textbf{0.876}(0.041) & 0.795(0.058) & 0.073(0.015) & & 0.869(0.073) & 0.076(0.020) & 17.7  \\
				%& \methodwithsandpguide & \textbf{11.9} & 0.872(0.036) & \textbf{0.813}(0.052) & 0.075(0.014) & & \textbf{0.874}(0.069) & 0.080(0.014) & 17.0  \\
				%\cmidrule{2-10}
				%& improv\% & 40.4$^*$ & 8.8$^*$ & 0.4 & -30.4$^*$ &  & 1.6 & -30.0$^*$ & -26.3$^*$  \\
				%\midrule
				\multirow{7}{0.059\linewidth}{\hspace{0pt}1.0} 
				& \AR & 14.6 & 0.681(0.163) & 0.644(0.119) & 0.236(0.123) & & 0.780(0.110) & 0.284(0.177) & 95.8  \\
				& \pockettwomol & 18.6 & 0.711(0.152) & 0.654(0.131) &   \textbf{0.217}(0.129) & & 0.778(0.121) &   \textbf{0.243}(0.137) &  \textbf{98.3}  \\
				& \targetdiff & 7.1 &  \textbf{0.785}(0.085) & 0.622(0.083) & 0.238(0.122) & & 0.790(0.102) & 0.274(0.158) & 90.4  \\
				%& \decompdiffbeta & 0.1 & 0.589(0.030) & 0.494(0.124) & 0.263(0.143) & & 0.567(0.143) & 0.275(0.162) & 67.7  \\
				& \decompdiffref & 37.3 & 0.721(0.108) & 0.770(0.087) & 0.282(0.130) & & \textbf{0.878}(0.059) & 0.343(0.174) & 83.7  \\
				& \methodwithpguide   &  27.4 & 0.757(0.134) & 0.747(0.078) & 0.265(0.165) & & 0.841(0.081) & 0.272(0.168) & 98.1  \\
				& \methodwithsandpguide &\textbf{45.2} & 0.724(0.142) &   \textbf{0.789}(0.063) & 0.265(0.162) & & 0.876(0.062) & 0.264(0.159) & 97.8  \\
				\cmidrule{2-10}
				& Improv\%  & 21.2$^*$ & -3.6 & 2.5$^*$ & -21.7$^*$ &  & -0.1 & -8.4$^*$ & -0.2  \\
				\midrule
				\multirow{7}{0.059\linewidth}{\hspace{0pt}0.7} 
				& \AR   & 14.5 & 0.692(0.151) & 0.644(0.119) & 0.233(0.116) & & 0.779(0.110) & 0.266(0.140) & 94.9  \\
				& \pockettwomol & 18.6 & 0.711(0.152) & 0.654(0.131) & \textbf{0.217}(0.129) & & 0.778(0.121) & \textbf{0.243}(0.137) & \textbf{98.2}  \\
				& \targetdiff & 7.1 & \textbf{0.786}(0.084) & 0.622(0.083) & 0.238(0.121) & & 0.790(0.101) & 0.270(0.151) & 90.3  \\
				%& \decompdiffbeta & 0.1 & 0.589(0.030) & 0.494(0.124) & 0.263(0.142) & &0.567(0.143) & 0.273(0.156) & 67.6  \\
				& \decompdiffref & 36.2 & 0.721(0.113) & 0.770(0.086) & 0.273(0.123) & & \textbf{0.876}(0.059) & 0.325(0.139) & 82.3  \\
				& \methodwithpguide   &  27.4 & 0.757(0.134) & 0.746(0.078) & 0.263(0.160) & & 0.841(0.081) & 0.271(0.164) & 96.8  \\
				& \methodwithsandpguide      & \textbf{45.0} & 0.732(0.129) & \textbf{0.789}(0.063) & 0.262(0.157) & & \textbf{0.876}(0.063) & 0.262(0.153) & 96.2  \\
				\cmidrule{2-10}
				& Improv\%  & 24.3$^*$ & -3.6 & 2.5$^*$ & -20.8$^*$ &  & 0.0 & -7.6$^*$ & -1.5  \\
				\midrule
				\multirow{7}{0.059\linewidth}{\hspace{0pt}0.5} 
				& \AR   & 14.1 & 0.687(0.160) & 0.639(0.124) & 0.218(0.097) & & 0.778(0.110) & 0.260(0.130) & 89.8  \\
				& \pockettwomol & 18.5 & 0.711(0.152) & 0.649(0.134) & \textbf{0.209}(0.114) & & 0.777(0.121) & \textbf{0.240}(0.131) & \textbf{93.2}  \\
				& \targetdiff & 7.1 & \textbf{0.786}(0.084) & 0.621(0.083) & 0.230(0.111) & & 0.788(0.105) & 0.254(0.127) & 86.5  \\
				%&\decompdiffbeta & 0.1 & 0.595(0.025) & 0.494(0.124) & 0.254(0.129) & & 0.565(0.142) & 0.259(0.138) & 63.9  \\
				& \decompdiffref & 34.7 & 0.730(0.105) & 0.769(0.086) & 0.261(0.109) & & 0.874(0.080) & 0.301(0.117) & 77.3   \\
				& \methodwithpguide  &  27.2 & 0.765(0.123) & 0.749(0.075) & 0.245(0.135) & & 0.840(0.082) & 0.252(0.137) & 88.6  \\
				& \methodwithsandpguide & \textbf{44.3} & 0.738(0.122) & \textbf{0.791}(0.059) & 0.247(0.132) &  & \textbf{0.875}(0.065) & 0.249(0.130) & 88.8  \\
				\cmidrule{2-10}
				& Improv\%   & 27.8$^*$ & -2.7 & 2.9$^*$ & -17.6$^*$ &  & 0.2 & -3.4 & -4.7$^*$  \\
				\midrule
				\multirow{7}{0.059\linewidth}{\hspace{0pt}0.3} 
				& \AR   & 12.2 & 0.704(0.146) & 0.614(0.146) & 0.164(0.059) & & 0.751(0.138) & 0.206(0.059) & 66.4  \\
				& \pockettwomol & 17.1 & 0.731(0.129) & 0.617(0.163) & \textbf{0.155}(0.056) & & 0.740(0.159) & \textbf{0.190}(0.076) & \textbf{71.0}  \\
				& \targetdiff & 6.2 & \textbf{0.809}(0.061) & 0.619(0.087) & 0.181(0.068) & & 0.768(0.119) & 0.196(0.076) & 61.7  \\				
                %& \decompdiffbeta & 0.0 & - & 0.489(0.124) & 0.195(0.080) & & 0.547(0.139) & 0.203(0.087) & 42.0  \\
				& \decompdiffref & 27.7 & 0.775(0.081) & 0.767(0.086) & 0.202(0.062) & & 0.854(0.093) & 0.216(0.068) & 52.6  \\
				& \methodwithpguide   &  24.4 & 0.805(0.084) & 0.763(0.066) & 0.180(0.074) & & 0.847(0.080) & \textbf{0.190}(0.059) & 61.4  \\
				& \methodwithsandpguide & \textbf{36.3} & 0.789(0.081) & \textbf{0.800}(0.056) & 0.181(0.071) & &\textbf{0.878}(0.067) & \textbf{0.190}(0.078) & 61.8  \\
				\cmidrule{2-10}
				& improv\% & 31.1$^*$ & 3.9$^*$ & 4.3$^*$ & -16.5$^*$ &  & 2.8$^*$ & 0.0 & -12.9$^*$  \\
				\bottomrule
			\end{tabular}%
			\begin{tablenotes}[normal,flushleft]
				\begin{footnotesize}
					\item 
					\!\!Columns represent: \ziqi{``$\delta_g$'': the graph similarity constraint; ``\#n\%'': the percentage of molecules that satisfy the graph similarity constraint ($\graphsim<=\delta_g$);
						``\#d\%'': the percentage of molecules that satisfy the graph similarity constraint and are with high \shapesim ($\shapesim>=0.8$);
						``\avgshapesim/\avggraphsim'': the average of shape or graph similarities between the condition molecules and generated molecules with $\graphsim<=\delta_g$;
						``\maxshapesim'': the maximum of shape similarities between the condition molecules and generated molecules with $\graphsim<=\delta_g$;
						``\maxgraphsim'': the graph similarities between the condition molecules and the molecules with the maximum shape similarities and $\graphsim<=\delta_g$;
						``\diversity'': the diversity among the generated molecules.
						%
						``$\uparrow$'' represents higher values are better, and ``$\downarrow$'' represents lower values are better.
						%
						Best values are in \textbf{bold}, and second-best values are \underline{underlined}. 
					} 
					%\todo{double-check the significance value}
					\par
					\par
				\end{footnotesize}
			\end{tablenotes}
		\end{threeparttable}
	\end{small}
	\vspace{-10pt}    
\end{table*}
%\label{tbl:docking_results_similarity}

\bo{@Ziqi you may want to check my edits for the discussion in Table 1 first.
%
If the pocket if known, do you still care about the shape similarity in real applications?
}

\ziqi{Table~\ref{tbl:docking_results_similarity} presents the overall comparison on similarity-based metrics between \methodwithpguide, \methodwithsandpguide and other baselines under different graph similarity constraints  ($\delta_g$=1.0, 0.7, 0.5, 0.3), similar to Table~\ref{tbl:overall}. 
%
As shown in Table~\ref{tbl:docking_results_similarity}, regarding desirable molecules,  \methodwithsandpguide consistently outperforms all the baseline methods in the likelihood of generating desirable molecules (i.e., $\#d\%$).
%
For example, when $\delta_g$=1.0, at $\#d\%$, \methodwithsandpguide (45.2\%) demonstrates significant improvement of $21.2\%$ compared to the best baseline \decompdiff (37.3\%).
%
In terms of $\diversity_d$, \methodwithpguide and \methodwithsandpguide also achieve the second and the third best performance. 
%
Note that the best baseline \targetdiff in $\diversity_d$ achieves the least percentage of desirable molecules (7.1\%), substantially lower than \methodwithpguide and \methodwithsandpguide.
%
This makes its diversity among desirable molecules incomparable with other methods. 
%
When $\delta_g$=0.7, 0.5, and 0.3, \methodwithsandpguide also establishes a significant improvement of 24.3\%, 27.8\%, and 31.1\% compared to the best baseline method \decompdiff.
%
It is also worth noting that the state-of-the-art baseline \decompdiff underperforms \methodwithpguide and \methodwithsandpguide in binding affinities as shown in Table~\ref{tbl:overall_results_docking}, even though it outperforms \methodwithpguide in \#d\%.
%
\methodwithpguide and \methodwithsandpguide also achieve the second and the third best performance in $\diversity_d$ at $\delta_g$=0.7, 0.5, and 0.3. 
%
The superior performance of \methodwithpguide and \methodwithsandpguide in $\#d\%$ at small $\delta_g$ indicates their strong capacity in generating desirable molecules of novel graph structures, thereby facilitating the discovery of novel drug candidates.
%
}

\ziqi{Apart from the desirable molecules, \methodwithpguide and \methodwithsandpguide also demonstrate outstanding performance in terms of the average shape similarities (\avgshapesim) and the average graph similarities (\avggraphsim).
%
Specifically, when $\delta_g$=1.0, \methodwithsandpguide achieves a significant 2.5\% improvement in \avgshapesim\ over the best baseline \decompdiff. 
%
In terms of \avggraphsim, \methodwithsandpguide also achieves higher performance than the baseline \decompdiff of the highest \avgshapesim (0.265 vs 0.282).
%
Please note that all the baseline methods except \decompdiff achieve substantially lower performance in \avgshapesim than \methodwithpguide and \methodwithsandpguide, even though these methods achieve higher \avggraphsim values.
%
This trend remains consistent when applying various similarity constraints (i.e., $\delta_g$) as shown in Table~\ref{tbl:overall_results_docking}.
}

\ziqi{Similarly, \methodwithpguide and \methodwithsandpguide also achieve superior performance in \maxshapesim and \maxgraphsim.
%
Specifically, when $\delta_g$=1.0, for \maxshapesim, \methodwithsandpguide achieves highly comparable performance in \maxshapesim\ compared to the best baseline \decompdiff (0.876 vs 0.878).
%
We also note that \methodwithsandpguide achieves lower \maxgraphsim\ than the \decompdiff with 23.0\% difference. 
%
When $\delta_g$ gets smaller from 0.7 to 0.3, \methodwithsandpguide maintains a high \maxshapesim value around 0.876, while the best baseline \decompdiff has \maxshapesim decreased from 0.878 to 0.854.
%
This demonstrates the superior ability of \methodwithsandpguide in generating molecules with similar shapes and novel structures.
%
}

\ziqi{
In terms of \#n\%, when $\delta_g$=1.0, the percentage of molecules with \graphsim below $\delta_g$ can be interpreted as the percentage of valid molecules among all the generated molecules. 
%
As shown in Table~\ref{tbl:docking_results_similarity}, \methodwithpguide and \methodwithsandpguide are able to generate 98.1\% and 97.8\% of valid molecules, slightly below the best baseline \pockettwomol (98.3\%). 
%
When $\delta_g$=0.7, 0.5, or 0.3, all the methods, including \methodwithpguide and \methodwithsandpguide, can consistently find a sufficient number of novel molecules that meet the graph similarity constraints.
%
The only exception is \decompdiff, which substantially underperforms all the other methods in \#n\%.
}
\end{comment}

%%%%%%%%%%%%%%%%%%%%%%%%%%%%%%%%%%%%%%%%%%%%%
\section{Properties of Molecules in Case Studies for Targets}
\label{supp:app:results:properties}
%%%%%%%%%%%%%%%%%%%%%%%%%%%%%%%%%%%%%%%%%%%%%

%-------------------------------------------------------------------------------------------------------------------------------------
\subsection{Drug Properties of Generated Molecules}
\label{supp:app:results:properties:drug}
%-------------------------------------------------------------------------------------------------------------------------------------

Table~\ref{tbl:drug_property} presents the drug properties of three generated molecules: NL-001, NL-002, and NL-003.
%
As shown in Table~\ref{tbl:drug_property}, each of these molecules has a favorable profile, making them promising drug candidates. 
%
{As discussed in Section ``Case Studies for Targets'' in the main manuscript, all three molecules have high binding affinities in terms of Vina S, Vina M and Vina D, and favorable QED and SA values.
%
In addition, all of them meet the Lipinski's rule of five criteria~\cite{Lipinski1997}.}
%
In terms of physicochemical properties, all these properties of NL-001, NL-002 and NL-003, including number of rotatable bonds, molecule weight, LogP value, number of hydrogen bond doners and acceptors, and molecule charges, fall within the desired range of drug molecules. 
%
This indicates that these molecules could potentially have good solubility and membrane permeability, essential qualities for effective drug absorption.

These generated molecules also demonstrate promising safety profiles based on the predictions from ICM~\cite{Neves2012}.
%
In terms of drug-induced liver injury prediction scores, all three molecules have low scores (0.188 to 0.376), indicating a minimal risk of hepatotoxicity. 
%
NL-001 and NL-002 fall under `Ambiguous/Less concern' for liver injury, while NL-003 is categorized under 'No concern' for liver injury. 
%
Moreover, all these molecules have low toxicity scores (0.000 to 0.236). 
%
NL-002 and NL-003 do not have any known toxicity-inducing functional groups. 
%
NL-001 and NL-003 are also predicted not to include any known bad groups that lead to inappropriate features.
%
These attributes highlight the potential of NL-001, NL-002, and NL-003 as promising treatments for cancers and Alzheimer’s disease.

%\begin{table*}
	\centering
		\caption{Drug Properties of Generated Molecules}
	\label{tbl:binding_drug_mols}
	\begin{scriptsize}
\begin{threeparttable}
	\begin{tabular}{
		@{\hspace{6pt}}r@{\hspace{6pt}}
		@{\hspace{6pt}}r@{\hspace{6pt}}
		@{\hspace{6pt}}r@{\hspace{6pt}}
		@{\hspace{6pt}}r@{\hspace{6pt}}
		@{\hspace{6pt}}r@{\hspace{6pt}}
		@{\hspace{6pt}}r@{\hspace{6pt}}
		@{\hspace{6pt}}r@{\hspace{6pt}}
		@{\hspace{6pt}}r@{\hspace{6pt}}
		@{\hspace{6pt}}r@{\hspace{6pt}}
		%
		}
		\toprule
Target & Molecule & Vina S & Vina M & Vina D & QED   & SA   & Logp  & Lipinski \\
\midrule
\multirow{3}{*}{CDK6} & NL-001 & -6.817      & -7.251    & -8.319     & 0.834 & 0.72 & 1.313 & 5        \\
& NL-002 & -6.970       & -7.605    & -8.986     & 0.851 & 0.74 & 3.196 & 5        \\
\cmidrule{2-9}
& 4AU & 0.736       & -5.939    & -7.592     & 0.773 & 0.79 & 2.104 & 5        \\
\midrule
\multirow{2}{*}{NEP} & NL-003 & -11.953     & -12.165   & -12.308    & 0.772 & 0.57 & 2.944 & 5        \\
\cmidrule{2-9}
& BIR & -9.399      & -9.505    & -9.561     & 0.463 & 0.73 & 2.677 & 5        \\
		\bottomrule
	\end{tabular}%
	\begin{tablenotes}[normal,flushleft]
		\begin{footnotesize}
	\item Columns represent: {``Target'': the names of protein targets;
		``Molecule'': the names of generated molecules and known ligands;
		``Vina S'': the binding affinities between the initially generated poses of molecules and the protein pockets; 
		``Vina M'': the binding affinities between the poses after local structure minimization and the protein pockets;
		``Vina D'': the binding affinities between the poses determined by AutoDock Vina~\cite{Eberhardt2021} and the protein pockets;
		``HA'': the percentage of generated molecules with Vina D higher than those of condition molecules;
		``QED'': the drug-likeness score;
		``SA'': the synthesizability score;
		``Div'': the diversity among generated molecules;
		``time'': the time cost to generate molecules.}
\!\! \par
		\par
		\end{footnotesize}
	\end{tablenotes}
\end{threeparttable}
\end{scriptsize}
  \vspace{-10pt}    
\end{table*}

%\label{tbl:binding_drug_mols}

\begin{table*}
	\centering
		\caption{Drug Properties of Generated Molecules}
	\label{tbl:drug_property}
	\begin{scriptsize}
\begin{threeparttable}
	\begin{tabular}{
		@{\hspace{0pt}}p{0.23\linewidth}@{\hspace{5pt}}
		%
		@{\hspace{1pt}}r@{\hspace{2pt}}
		@{\hspace{2pt}}r@{\hspace{6pt}}
		@{\hspace{6pt}}r@{\hspace{6pt}}
		%
		}
		\toprule
		Property Name & NL-001 & NL-002 & NL-003 \\
		\midrule
Vina S & -6.817 &  -6.970 & -11.953 \\
Vina M & -7.251 & -7.605 & -12.165 \\
Vina D & -8.319 & -8.986 & -12.308 \\
QED    & 0.834  & 0.851  & 0.772 \\
SA       & 0.72    & 0.74    & 0.57    \\
Lipinski & 5 & 5 & 5 \\
%bbbScore          & 3.386                                                                                        & 4.240                                                                                        & 3.892      \\
%drugLikeness      & -0.081                                                                                       & -0.442                                                                                       & -0.325     \\
%molLogP1          & 1.698                                                                                        & 2.685                                                                                        & 2.382      \\
\#rotatable bonds          & 3                                                                                        & 2                                                                                        & 2      \\
molecule weight         & 267.112                                                                                      & 270.117                                                                                      & 390.206    \\
molecule LogP           & 1.698                                                                                        & 2.685                                                                                        & 2.382     \\
\#hydrogen bond doners           & 1                                                                                        & 1                                                                                        & 2      \\
\#hydrogen bond acceptors           & 5                                                                                       & 3                                                                                        & 5      \\
\#molecule charges   & 1                                                                                        & 0                                                                                        & 0      \\
drug-induced liver injury predScore    & 0.227                                                                                        & 0.376                                                                                        & 0.188      \\
drug-induced liver injury predConcern  & Ambiguous/Less concern                                                                       & Ambiguous/Less concern                                                                       & No concern \\
drug-induced liver injury predLabel    & Warnings/Precautions/Adverse reactions & Warnings/Precautions/Adverse reactions & No match   \\
drug-induced liver injury predSeverity & 2                                                                                        & 3                                                                                        & 2      \\
%molSynth1         & 0.254                                                                                        & 0.220                                                                                        & 0.201      \\
%toxicity class         & 0.480                                                                                        & 0.480                                                                                        & 0.450      \\
toxicity names         & hydrazone                                                                                    &   -                                                                                           &   -         \\
toxicity score         & 0.236                                                                                        & 0.000                                                                                        & 0.000      \\
bad groups         & -                                                                                             & Tetrahydroisoquinoline:   allergies                                                          &   -         \\
%MolCovalent       &                                                                                              &                                                                                              &            \\
%MolProdrug        &                                                                                              &                                                                                              &            \\
		\bottomrule
	\end{tabular}%
	\begin{tablenotes}[normal,flushleft]
		\begin{footnotesize}
	\item ``-'': no results found by algorithms
\!\! \par
		\par
		\end{footnotesize}
	\end{tablenotes}
\end{threeparttable}
\end{scriptsize}
  \vspace{-10pt}    
\end{table*}

%\label{tbl:drug_property}

%-------------------------------------------------------------------------------------------------------------------------------------
\subsection{Comparison on ADMET Profiles between Generated Molecules and Approved Drugs}
\label{supp:app:results:properties:admet}
%-------------------------------------------------------------------------------------------------------------------------------------

\paragraph{Generated Molecules for CDK6}
%
Table~\ref{tbl:admet_cdk6} presents the comparison on ADMET profiles between two generated molecules for CDK6 and the approved CDK6 inhibitors, including Abemaciclib~\cite{Patnaik2016}, Palbociclib~\cite{Lu2015}, and Ribociclib~\cite{Tripathy2017}.
%
As shown in Table~\ref{tbl:admet_cdk6}, the generated molecules, NL-001 and NL-002, exhibit comparable ADMET profiles with those of the approved CDK6 inhibitors. 
%
Importantly, both molecules demonstrate good potential in most crucial properties, including Ames mutagenesis, favorable oral toxicity, carcinogenicity, estrogen receptor binding, high intestinal absorption and favorable oral bioavailability.
%
Although the generated molecules are predicted as positive in hepatotoxicity and mitochondrial toxicity, all the approved drugs are also predicted as positive in these two toxicity.
%
This result suggests that these issues might stem from the limited prediction accuracy rather than being specific to our generated molecules.
%
Notably, NL-001 displays a potentially better plasma protein binding score compared to other molecules, which may improve its distribution within the body. 
%
Overall, these results indicate that NL-001 and NL-002 could be promising candidates for further drug development.


\begin{table*}
	\centering
		\caption{Comparison on ADMET Profiles among Generated Molecules and Approved Drugs Targeting CDK6}
	\label{tbl:admet_cdk6}
	\begin{scriptsize}
\begin{threeparttable}
	\begin{tabular}{
		%@{\hspace{0pt}}p{0.23\linewidth}@{\hspace{5pt}}
		%
		@{\hspace{6pt}}l@{\hspace{5pt}}
		@{\hspace{6pt}}r@{\hspace{6pt}}
		@{\hspace{6pt}}r@{\hspace{6pt}}
		@{\hspace{6pt}}r@{\hspace{6pt}}
		@{\hspace{6pt}}r@{\hspace{6pt}}
		@{\hspace{6pt}}r@{\hspace{6pt}}
		%
		%
		@{\hspace{6pt}}r@{\hspace{6pt}}
		%@{\hspace{6pt}}r@{\hspace{6pt}}
		%
		}
		\toprule
		\multirow{2}{*}{Property name} & \multicolumn{2}{c}{Generated molecules} & & \multicolumn{3}{c}{FDA-approved drugs} \\
		\cmidrule{2-3}\cmidrule{5-7}
		 & NL--001 & NL--002 & & Abemaciclib & Palbociclib & Ribociclib \\
		\midrule
\rowcolor[HTML]{D2EAD9}Ames   mutagenesis                             & --   &  --  & & + &  --  & --  \\
\rowcolor[HTML]{D2EAD9}Acute oral toxicity (c)                           & III & III & &  III          & III          & III         \\
Androgen receptor binding                         & +                          & +            &              & +            & +            & +             \\
Aromatase binding                                 & +                          & +            &              & +            & +            & +            \\
Avian toxicity                                    & --                          & --          &                & --            & --            & --            \\
Blood brain barrier                               & +                          & +            &              & +            & +            & +            \\
BRCP inhibitior                                   & --                          & --          &                & --            & --            & --            \\
Biodegradation                                    & --                          & --          &                & --            & --            & --           \\
BSEP inhibitior            & +                          & +            &              & +            & +            & +        \\
Caco-2                                            & +                          & +            &              & --            & --            & --            \\
\rowcolor[HTML]{D2EAD9}Carcinogenicity (binary)                          & --                          & --             &             & --            & --            & --          \\
\rowcolor[HTML]{D2EAD9}Carcinogenicity (trinary)                         & Non-required               & Non-required   &            & Non-required & Non-required & Non-required  \\
Crustacea aquatic toxicity & --                          & --            &              & --            & --            & --            \\
 CYP1A2 inhibition                                 & +                          & +            &              & --            & --            & +             \\
CYP2C19 inhibition                                & --                          & +            &              & +            & --            & +            \\
CYP2C8 inhibition                                 & --                          & --           &               & +            & +            & +            \\
CYP2C9 inhibition                                 & --                          & --           &               & --            & --            & +             \\
CYP2C9 substrate                                  & --                          & --           &               & --            & --            & --            \\
CYP2D6 inhibition                                 & --                          & --           &               & --            & --            & --            \\
CYP2D6 substrate                                  & --                          & --           &               & --            & --            & --            \\
CYP3A4 inhibition                                 & --                          & +            &              & --            & --            & --            \\
CYP3A4 substrate                                  & +                          & --            &              & +            & +            & +            \\
\rowcolor[HTML]{D2EAD9}CYP inhibitory promiscuity                        & +                          & +                    &      & +            & --            & +            \\
Eye corrosion                                     & --                          & --           &               & --            & --            & --            \\
Eye irritation                                    & --                          & --           &               & --            & --            & --             \\
\rowcolor[HTML]{D8E7FF}Estrogen receptor binding                         & +                          & +                    &      & +            & +            & +            \\
Fish aquatic toxicity                             & --                          & +            &              & +            & --            & --            \\
Glucocorticoid receptor   binding                 & +                          & +             &             & +            & +            & +            \\
Honey bee toxicity                                & --                          & --           &               & --            & --            & --            \\
\rowcolor[HTML]{D2EAD9}Hepatotoxicity                                    & +                          & +            &              & +            & +            & +             \\
Human ether-a-go-go-related gene inhibition     & +                          & +               &           & +            & --            & --           \\
\rowcolor[HTML]{D8E7FF}Human intestinal absorption                       & +                          & +             &             & +            & +            & +    \\
\rowcolor[HTML]{D8E7FF}Human oral bioavailability                        & +                          & +              &            & +            & +            & +     \\
\rowcolor[HTML]{D2EAD9}MATE1 inhibitior                                  & --                          & --              &            & --            & --            & --    \\
\rowcolor[HTML]{D2EAD9}Mitochondrial toxicity                            & +                          & +                &          & +            & +            & +    \\
Micronuclear                                      & +                          & +                          & +            & +            & +           \\
\rowcolor[HTML]{D2EAD9}Nephrotoxicity                                    & --                          & --             &             & --            & --            & --             \\
Acute oral toxicity                               & 2.325                      & 1.874    &     & 1.870        & 3.072        & 3.138        \\
\rowcolor[HTML]{D8E7FF}OATP1B1 inhibitior                                & +                          & +              &            & +            & +            & +             \\
\rowcolor[HTML]{D8E7FF}OATP1B3 inhibitior                                & +                          & +              &            & +            & +            & +             \\
\rowcolor[HTML]{D2EAD9}OATP2B1 inhibitior                                & --                          & --             &             & --            & --            & --             \\
OCT1 inhibitior                                   & --                          & --        &                  & +            & --            & +             \\
OCT2 inhibitior                                   & --                          & --        &                  & --            & --            & +             \\
P-glycoprotein inhibitior                         & --                          & --        &                  & +            & +            & +     \\
P-glycoprotein substrate                          & --                          & --        &                  & +            & +            & +     \\
PPAR gamma                                        & +                          & +          &                & +            & +            & +      \\
\rowcolor[HTML]{D8E7FF}Plasma protein binding                            & 0.359        & 0.745     &    & 0.865        & 0.872        & 0.636       \\
Reproductive toxicity                             & +                          & +          &                & +            & +            & +           \\
Respiratory toxicity                              & +                          & +          &                & +            & +            & +         \\
Skin corrosion                                    & --                          & --        &                  & --            & --            & --           \\
Skin irritation                                   & --                          & --        &                  & --            & --            & --         \\
Skin sensitisation                                & --                          & --        &                  & --            & --            & --          \\
Subcellular localzation                           & Mitochondria               & Mitochondria  &             & Lysosomes    & Mitochondria & Mitochondria \\
Tetrahymena pyriformis                            & 0.398                      & 0.903         &             & 1.033        & 1.958        & 1.606         \\
Thyroid receptor binding                          & +                          & +             &             & +            & +            & +           \\
UGT catelyzed                                     & --                          & --           &               & --            & --            & --           \\
\rowcolor[HTML]{D8E7FF}Water solubility                                  & -3.050                     & -3.078              &       & -3.942       & -3.288       & -2.673     \\
		\bottomrule
	\end{tabular}%
	\begin{tablenotes}[normal,flushleft]
		\begin{footnotesize}
	\item Blue cells highlight crucial properties where a negative outcome (``--'') is desired; for acute oral toxicity (c), a higher category (e.g., ``III'') is desired; and for carcinogenicity (trinary), ``Non-required'' is desired.
	%
	Green cells highlight crucial properties where a positive result (``+'') is desired; for plasma protein binding, a lower value is desired; and for water solubility, values higher than -4 are desired~\cite{logs}.
\!\! \par
		\par
		\end{footnotesize}
	\end{tablenotes}
\end{threeparttable}
\end{scriptsize}
  \vspace{--10pt}    
\end{table*}

%\label{tbl:admet_cdk6}

\paragraph{Generated Molecule for NEP}
%
Table~\ref{tbl:admet_nep} presents the comparison on ADMET profiles between a generated molecule for NEP targeting Alzheimer's disease and three approved drugs, Donepezil, Galantamine, and Rivastigmine, for Alzheimer's disease~\cite{Hansen2008}.
%
Overall, NL-003 exhibits a comparable ADMET profile with the three approved drugs.  
%
Notably, same as other approved drugs, NL-003 is predicted to be able to penetrate the blood brain barrier, a crucial property for Alzheimer's disease.
%  
In addition, it demonstrates a promising safety profile in terms of Ames mutagenesis, favorable oral toxicity, carcinogenicity, estrogen receptor binding, high intestinal absorption, nephrotoxicity and so on.
%
These results suggest that NL-003 could be promising candidates for the drug development of Alzheimer's disease.

\begin{table*}
	\centering
		\caption{Comparison on ADMET Profiles among Generated Molecule Targeting NEP and Approved Drugs for Alzhimer's Disease}
	\label{tbl:admet_nep}
	\begin{scriptsize}
\begin{threeparttable}
	\begin{tabular}{
		@{\hspace{6pt}}l@{\hspace{5pt}}
		%
		@{\hspace{6pt}}r@{\hspace{6pt}}
		@{\hspace{6pt}}r@{\hspace{6pt}}
		@{\hspace{6pt}}r@{\hspace{6pt}}
		@{\hspace{6pt}}r@{\hspace{6pt}}
		@{\hspace{6pt}}r@{\hspace{6pt}}
		%
		%
		%@{\hspace{6pt}}r@{\hspace{6pt}}
		%
		}
		\toprule
		\multirow{2}{*}{Property name} & Generated molecule & & \multicolumn{3}{c}{FDA-approved drugs} \\
\cmidrule{2-2}\cmidrule{4-6}
			& NL--003 & & Donepezil	& Galantamine & Rivastigmine \\
		\midrule
\rowcolor[HTML]{D2EAD9} 
Ames   mutagenesis                            & --                      &              & --                                    & --                                 & --                     \\
\rowcolor[HTML]{D2EAD9}Acute oral toxicity (c)                       & III           &                       & III                                  & III                               & II                      \\
Androgen receptor binding                     & +      &      & +            & --         & --         \\
Aromatase binding                             & --     &       & +            & --         & --        \\
Avian toxicity                                & --     &                               & --                                    & --                                 & --                        \\
\rowcolor[HTML]{D8E7FF} 
Blood brain barrier                           & +      &                              & +                                    & +                                 & +                        \\
BRCP inhibitior                               & --     &       & --            & --         & --         \\
Biodegradation                                & --     &                               & --                                    & --                                 & --                        \\
BSEP inhibitior                               & +      &      & +            & --         & --         \\
Caco-2                                        & +      &      & +            & +         & +         \\
\rowcolor[HTML]{D2EAD9} 
Carcinogenicity (binary)                      & --     &                               & --                                    & --                                 & --                        \\
\rowcolor[HTML]{D2EAD9} 
Carcinogenicity (trinary)                     & Non-required    &                     & Non-required                         & Non-required                      & Non-required             \\
Crustacea aquatic toxicity                    & +               &                     & +                                    & +                                 & --                        \\
CYP1A2 inhibition                             & +               &                     & +                                    & --                                 & --                        \\
CYP2C19 inhibition                            & +               &                     & --                                    & --                                 & --                        \\
CYP2C8 inhibition                             & +               &                     & --                                    & --                                 & --                        \\
CYP2C9 inhibition                             & --              &                      & --                                    & --                                 & --                        \\
CYP2C9 substrate                              & --              &                      & --                                    & --                                 & --                        \\
CYP2D6 inhibition                             & --              &                      & +                                    & --                                 & --                        \\
CYP2D6 substrate                              & --              &                      & +                                    & +                                 & +                        \\
CYP3A4 inhibition                             & --              &                      & --                                    & --                                 & --                        \\
CYP3A4 substrate                              & +               &                     & +                                    & +                                 & --                        \\
\rowcolor[HTML]{D2EAD9} 
CYP inhibitory promiscuity                    & +               &                     & +                                    & --                                 & --                        \\
Eye corrosion                                 & --     &       & --            & --         & --         \\
Eye irritation                                & --     &       & --            & --         & --         \\
Estrogen receptor binding                     & +      &      & +            & --         & --         \\
Fish aquatic toxicity                         & --     &                               & +                                    & +                                 & +                        \\
Glucocorticoid receptor binding             & --      &      & +            & --         & --         \\
Honey bee toxicity                            & --    &                                & --                                    & --                                 & --                        \\
\rowcolor[HTML]{D2EAD9} 
Hepatotoxicity                                & +     &                               & +                                    & --                                 & --                        \\
Human ether-a-go-go-related gene inhibition & +       &     & +            & --         & --         \\
\rowcolor[HTML]{D8E7FF} 
Human intestinal absorption                   & +     &                               & +                                    & +                                 & +                        \\
\rowcolor[HTML]{D8E7FF} 
Human oral bioavailability                    & --    &                                & +                                    & +                                 & +                        \\
\rowcolor[HTML]{D2EAD9} 
MATE1 inhibitior                              & --    &                                & --                                    & --                                 & --                        \\
\rowcolor[HTML]{D2EAD9} 
Mitochondrial toxicity                        & +     &                               & +                                    & +                                 & +                        \\
Micronuclear                                  & +     &       & --            & --         & +         \\
\rowcolor[HTML]{D2EAD9} 
Nephrotoxicity                                & --    &                                & --                                    & --                                 & --                        \\
Acute oral toxicity                           & 2.704  &      & 2.098        & 2.767     & 2.726     \\
\rowcolor[HTML]{D8E7FF} 
OATP1B1 inhibitior                            & +      &                              & +                                    & +                                 & +                        \\
\rowcolor[HTML]{D8E7FF} 
OATP1B3 inhibitior                            & +      &                              & +                                    & +                                 & +                        \\
\rowcolor[HTML]{D2EAD9} 
OATP2B1 inhibitior                            & --     &                               & --                                    & --                                 & --                        \\
OCT1 inhibitior                               & +      &      & +            & --         & --         \\
OCT2 inhibitior                               & --     &       & +            & --         & --         \\
P-glycoprotein inhibitior                     & +      &      & +            & --         & --         \\
\rowcolor[HTML]{D8E7FF} 
P-glycoprotein substrate                      & +      &                              & +                                    & +                                 & --                        \\
PPAR gamma                                    & +      &      & --            & --         & --         \\
\rowcolor[HTML]{D8E7FF} 
Plasma protein binding                        & 0.227   &                             & 0.883                                & 0.230                             & 0.606                    \\
Reproductive toxicity                         & +       &     & +            & +         & +         \\
Respiratory toxicity                          & +       &     & +            & +         & +         \\
Skin corrosion                                & --      &      & --            & --         & --         \\
Skin irritation                               & --      &      & --            & --         & --         \\
Skin sensitisation                            & --      &      & --            & --         & --         \\
Subcellular localzation                       & Mitochondria & &Mitochondria & Lysosomes & Mitochondria  \\
Tetrahymena pyriformis                        & 0.053           &                     & 0.979                                & 0.563                             & 0.702                        \\
Thyroid receptor binding                      & +       &     & +            & +         & --             \\
UGT catelyzed                                 & --      &      & --            & +         & --             \\
\rowcolor[HTML]{D8E7FF} 
Water solubility                              & -3.586   &                            & -2.425                               & -2.530                            & -3.062                       \\
		\bottomrule
	\end{tabular}%
	\begin{tablenotes}[normal,flushleft]
		\begin{footnotesize}
	\item Blue cells highlight crucial properties where a negative outcome (``--'') is desired; for acute oral toxicity (c), a higher category (e.g., ``III'') is desired; and for carcinogenicity (trinary), ``Non-required'' is desired.
	%
	Green cells highlight crucial properties where a positive result (``+'') is desired; for plasma protein binding, a lower value is desired; and for water solubility, values higher than -4 are desired~\cite{logs}.
\!\! \par
		\par
		\end{footnotesize}
	\end{tablenotes}
\end{threeparttable}
\end{scriptsize}
  \vspace{--10pt}    
\end{table*}

%\label{tbl:admet_nep}

\clearpage
%%%%%%%%%%%%%%%%%%%%%%%%%%%%%%%%%%%%%%%%%%%%%
\section{Algorithms}
\label{supp:algorithms}
%%%%%%%%%%%%%%%%%%%%%%%%%%%%%%%%%%%%%%%%%%%%%

Algorithm~\ref{alg:shapemol} describes the molecule generation process of \method.
%
Given a known ligand \molx, \method generates a novel molecule \moly that has a similar shape to \molx and thus potentially similar binding activity.
%
\method can also take the protein pocket \pocket as input and adjust the atoms of generated molecules for optimal fit and improved binding affinities.
%
Specifically, \method first calculates the shape embedding \shapehiddenmat for \molx using the shape encoder \SEE described in Algorithm~\ref{alg:see_shaperep}.
%
Based on \shapehiddenmat, \method then generates a novel molecule with a similar shape to \molx using the diffusion-based generative model \methoddiff as in Algorithm~\ref{alg:diffgen}.
%
During generation, \method can use shape guidance to directly modify the shape of \moly to closely resemble the shape of \molx.
%
When the protein pocket \pocket is available, \method can also use pocket guidance to ensure that \moly is specifically tailored to closely fit within \pocket.

\begin{algorithm}[!h]
    \caption{\method}
    \label{alg:shapemol}
         %\hspace*{\algorithmicindent} 
	\textbf{Required Input: $\molx$} \\
 	%\hspace*{\algorithmicindent} 
	\textbf{Optional Input: $\pocket$} 
    \begin{algorithmic}[1]
        \FullLineComment{calculate a shape embedding with Algorithm~\ref{alg:see_shaperep}}
        \State $\shapehiddenmat$, $\pc$ = $\SEE(\molx)$
        \FullLineComment{generate a molecule conditioned on the shape embedding with Algorithm~\ref{alg:diffgen}}
         \If{\pocket is not available}
        \State $\moly = \diffgenerative(\shapehiddenmat, \molx)$
        \Else
        \State $\moly = \diffgenerative(\shapehiddenmat, \molx, \pocket)$
        \EndIf
        \State \Return \moly
    \end{algorithmic}
\end{algorithm}
%\label{alg:shapemol}

\begin{algorithm}[!h]
    \caption{\SEE for shape embedding calculation}
    \label{alg:see_shaperep}
    \textbf{Required Input: $\molx$}
    \begin{algorithmic}[1]
        %\Require $\molx$
        \FullLineComment{sample a point cloud over the molecule surface shape}
        \State $\pc$ = $\text{samplePointCloud}(\molx)$
        \FullLineComment{encode the point cloud into a latent embedding (Equation~\ref{eqn:shape_embed})}
        \State $\shapehiddenmat = \SEE(\pc)$
        \FullLineComment{move the center of \pc to zero}
        \State $\pc = \pc - \text{center}(\pc)$
        \State \Return \shapehiddenmat, \pc
    \end{algorithmic}
\end{algorithm}
%\label{alg:see_shaperep}

\begin{algorithm}[!h]
    \caption{\diffgenerative for molecule generation}
    \label{alg:diffgen}
    	\textbf{Required Input: $\molx$, \shapehiddenmat} \\
 	%\hspace*{\algorithmicindent} 
	\textbf{Optional Input: $\pocket$} 
    \begin{algorithmic}[1]
        \FullLineComment{sample the number of atoms in the generated molecule}
        \State $n = \text{sampleAtomNum}(\molx)$
        \FullLineComment{sample initial positions and types of $n$ atoms}
        \State $\{\pos_T\}^n = \mathcal{N}(0, I)$
        \State $\{\atomfeat_T\}^n = C(K, \frac{1}{K})$
        \FullLineComment{generate a molecule by denoising $\{(\pos_T, \atomfeat_T)\}^n$ to $\{(\pos_0, \atomfeat_0)\}^n$}
        \For{$t = T$ to $1$}
            \IndentLineComment{predict the molecule without noise using the shape-conditioned molecule prediction module \molpred}{1.5}
            \State $(\tilde{\pos}_{0,t}, \tilde{\atomfeat}_{0,t}) = \molpred(\pos_t, \atomfeat_t, \shapehiddenmat)$
            \If{use shape guidance and $t > s$}
                \State $\tilde{\pos}_{0,t} = \shapeguide(\tilde{\pos}_{0,t}, \molx)$
                %\State $\tilde{\pos}_{0,t} = \pos^*_{0,t}$
            \EndIf
            \IndentLineComment{sample $(\pos_{t-1}, \atomfeat_{t-1})$ from $(\pos_t, \atomfeat_t)$ and $(\tilde{\pos}_{0,t}, \tilde{\atomfeat}_{0,t})$}{1.5}
            \State $\pos_{t-1} = P(\pos_{t-1}|\pos_t, \tilde{\pos}_{o,t})$
            \State $\atomfeat_{t-1} = P(\atomfeat_{t-1}|\atomfeat_t, \tilde{\atomfeat}_{o,t})$
            \If{use pocket guidance and $\pocket$ is available}
                \State $\pos_{t-1} = \pocketguide(\pos_{t-1}, \pocket)$
                %\State $\pos_{t-1} = \pos_{t-1}^*$
            \EndIf  
        \EndFor
        \State \Return $\moly = (\pos_0, \atomfeat_0)$
    \end{algorithmic}
\end{algorithm}
%\label{alg:diffgen}

%\input{algorithms/train_SE}
%\label{alg:train_se}

%\begin{algorithm}[!h]
    \caption{Training Procedure of \methoddiff}
    \label{alg:diffgen}
    \begin{algorithmic}[1]
        \Require $\shapehiddenmat, \molx, \pocket$
        \FullLineComment{sample the number of atoms in the generated molecule}
    \end{algorithmic}
\end{algorithm}
%\label{alg:train_diff}

%---------------------------------------------------------------------------------------------------------------------
\section{{Equivariance and Invariance}}
\label{supp:ei}
%---------------------------------------------------------------------------------------------------------------------

%.................................................................................................
\subsection{Equivariance}
\label{supp:ei:equivariance}
%.................................................................................................

{Equivariance refers to the property of a function $f(\pos)$ %\bo{is it the property of the function or embedding (x)?} 
that any translation and rotation transformation from the special Euclidean group SE(3)~\cite{Atz2021} applied to a geometric object
$\pos\in\mathbb{R}^3$ is mirrored in the output of $f(\pos)$, accordingly.
%
This property ensures $f(\pos)$ to learn a consistent representation of an object's geometric information, regardless of its orientation or location in 3D space.
%
%As a result, it provides $f(\pos)$ better generalization capabilities~\cite{Jonas20a}.
%
Formally, given any translation transformation $\mathbf{t}\in\mathbb{R}^3$ and rotation transformation $\mathbf{R}\in\mathbb{R}^{3\times3}$ ($\mathbf{R}^{\mathsf{T}}\mathbf{R}=\mathbb{I}$), %\xia{change the font types for $^{\mathsf{T}}$ and $\mathbb{I}$ in the entire manuscript}), 
$f(\pos)$ is equivariant with respect to these transformations %$g$ (\bo{where is $g$...})
if it satisfies
\begin{equation}
f(\mathbf{R}\pos+\mathbf{t}) = \mathbf{R}f(\pos) + \mathbf{t}. %\ \text{where}\ \hiddenpos = f(\pos).
\end{equation}
%
%where $\hiddenpos=f(\pos)$ is the output of $\pos$. 
%
In \method, both \SE and \methoddiff are developed to guarantee equivariance in capturing the geometric features of objects regardless of any translation or rotation transformations, as will be detailed in the following sections.
}

%.................................................................................................
\subsection{Invariance}
\label{supp:ei:invariance}
%.................................................................................................

%In contrast to equivariance, 
Invariance refers to the property of a function that its output {$f(\pos)$} remains constant under any translation and rotation transformations of the input $\pos$. %a geometric object's feature $\pos$.
%
This property enables $f(\pos)$ to accurately capture %a geometric object's 
the inherent features (e.g., atom features for 3D molecules) that are invariant of its orientation or position in 3D space.
%
Formally, $f(\pos)$ is invariant under any translation $\mathbf{t}$ and  rotation $\mathbf{R}$ if it satisfies
%
\begin{equation}
f(\mathbf{R}\pos+\mathbf{t}) = f(\pos).
\end{equation}
%
In \method, both \SE and \methoddiff capture the inherent features of objects in an invariant way, regardless of any translation or rotation transformations, as will be detailed in the following sections.

%%%%%%%%%%%%%%%%%%%%%%%%%%%%%%%%%%%%%%%%%%%%%
\section{Point Cloud Construction}
\label{supp:point_clouds}
%%%%%%%%%%%%%%%%%%%%%%%%%%%%%%%%%%%%%%%%%%%%%

In \method, we represented molecular surface shapes using point clouds (\pc).
%
$\pc$
serves as input to \SE, from which we derive shape latent embeddings.
%
To generate $\pc$, %\bo{\st{create this}}, \bo{generate $\pc$}
we initially generated a molecular surface mesh using the algorithm from the Open Drug Discovery Toolkit~\cite{Wjcikowski2015oddt}.
%
Following this, we uniformly sampled points on the mesh surface with probability proportional to the face area, %\xia{how to uniformly?}, ensuring the sampling is done proportionally to the face area, with
using the algorithm from PyTorch3D~\cite{ravi2020pytorch3d}.
%
This point cloud $\pc$ is then centralized by setting the center of its points to zero.
%
%

%%%%%%%%%%%%%%%%%%%%%%%%%%%%%%%%%%%%%%%%%%%%%
\section{Query Point Sampling}
\label{supp:training:shapeemb}
%%%%%%%%%%%%%%%%%%%%%%%%%%%%%%%%%%%%%%%%%%%%%

As described in Section ``Shape Decoder (\SED)'', the signed distances of query points $z_q$ to molecule surface shape $\pc$ are used to optimize \SE.
%
In this section, we present how to sample these points $z_q$ in 3D space.
%
Particularly, we first determined the bounding box around the molecular surface shape, using the maximum and minimum \mbox{($x$, $y$, $z$)-axis} coordinates for points in our point cloud \pc,
denoted as $(x_\text{min}, y_\text{min}, z_\text{min})$ and $(x_\text{max}, y_\text{max}, z_\text{max})$.
%
We extended this box slightly by defining its corners as \mbox{$(x_\text{min}-1, y_\text{min}-1, z_\text{min}-1)$} and \mbox{$(x_\text{max}+1, y_\text{max}+1, z_\text{max}+1)$}.
%
For sampling $|\mathcal{Z}|$ query points, we wanted an even distribution of points inside and outside the molecule surface shape.
%
%\ziqi{Typically, within this bounding box, molecules occupy only a small portion of volume, which makes it more likely to sample
%points outside the molecule surface shape.}
%
When a bounding box is defined around the molecule surface shape, there could be a lot of empty spaces within the box that the molecule does not occupy due to 
its complex and irregular shape.
%
This could lead to that fewer points within the molecule surface shape could be sampled within the box.
%
Therefore, we started by randomly sampling $3k$ points within our bounding box to ensure that there are sufficient points within the surface.
%
We then determined whether each point lies within the molecular surface, using an algorithm from Trimesh~\footnote{https://trimsh.org/} based on the molecule surface mesh.
%
If there are $n_w$ points found within the surface, we selected $n=\min(n_w, k/2)$ points from these points, 
and randomly choose the remaining 
%\bo{what do you mean by remaining? If all the 3k sampled points are inside the surface, you get no points left.} 
$k-n$ points 
from those outside the surface.
%
For each query point, we determined its signed distance to the molecule surface by its closest distance to points in \pc with a sign indicating whether it is inside the surface.

%%%%%%%%%%%%%%%%%%%%%%%%%%%%%%%%%%%%%%%%%%%%%
\section{Forward Diffusion (\diffnoise)}
\label{supp:forward}
%%%%%%%%%%%%%%%%%%%%%%%%%%%%%%%%%%%%%%%%%%%%%

%===================================================================
\subsection{{Forward Process}}
\label{supp:forward:forward}
%===================================================================

Formally, for atom positions, the probability of $\pos_t$ sampled given $\pos_{t-1}$, denoted as $q(\pos_t|\pos_{t-1})$, is defined as follows,
%\xia{revise the representation, should be $\beta^x_t$ -- note the space} as follows,
%
\begin{equation}
q(\pos_t|\pos_{t-1}) = \mathcal{N}(\pos_t|\sqrt{1-\beta^{\mathtt{x}}_t}\pos_{t-1}, \beta^{\mathtt{x}}_t\mathbb{I}), 
\label{eqn:noiseposinter}
\end{equation}
%
%\xia{should be a comma after the equation. you also missed it. }
%\st{in which} 
where %\hl{$\pos_0$ denotes the original atom position;} \xia{no $\pos_0$ in the equation...}
%$\mathbf{I}$ denotes the identity matrix;
$\mathcal{N}(\cdot)$ is a Gaussian distribution of $\pos_t$ with mean $\sqrt{1-\beta_t^{\mathtt{x}}}\pos_{t-1}$ and covariance $\beta_t^{\mathtt{x}}\mathbf{I}$.
%\xia{what is $\mathcal{N}$? what is $q$? you abused $q$. need to be crystal clear... }
%\bo{Should be $\sim$ not $=$ in the equation}
%
Following Hoogeboom \etal~\cite{hoogeboom2021catdiff}, 
%the forward process for the discrete atom feature $\atomfeat_t\in\mathbb{R}^K$ adds 
%categorical noise into $\atomfeat_{t-1}$ according to a variance schedule $\beta_t^v\in (0, 1)$. %as follows, %\hl{$\betav_t\in (0, 1)$} as follows,
%\xia{presentation...check across the entire manuscript... } as follows,
%
%\ziqi{Formally, 
for atom features, the probability of $\atomfeat_t$ across $K$ classes given $\atomfeat_{t-1}$ is defined as follows,
%
\begin{equation}
q(\atomfeat_t|\atomfeat_{t-1}) = \mathcal{C}(\atomfeat_t|(1-\beta^{\mathtt{v}}_t) \atomfeat_{t-1}+\beta^{\mathtt{v}}_t\mathbf{1}/K),
\label{eqn:noisetypeinter}
\end{equation}
%
where %\hl{$\atomfeat_0$ denotes the original atom positions}; 
$\mathcal{C}$ is a categorical distribution of $\atomfeat_t$ derived from the %by 
noising $\atomfeat_{t-1}$ with a uniform noise $\beta^{\mathtt{v}}_t\mathbf{1}/K$ across $K$ classes.
%adding an uniform noise $\beta^v_t$ to $\atomfeat_{t-1}$ across K classes.
%\xia{there is always a comma or period after the equations. Equations are part of a sentence. you always missed it. }
%\xia{what is $\mathcal{C}$? what does $q$ mean? it is abused. }

Since the above distributions form Markov chains, %} \xia{grammar!}, 
the probability of any $\pos_t$ or $\atomfeat_t$ can be derived from $\pos_0$ or $\atomfeat_0$:
%samples $\mol_0$ as follows,
%
\begin{eqnarray}
%\begin{aligned}
& q(\pos_t|\pos_{0}) & = \mathcal{N}(\pos_t|\sqrt{\cumalpha^{\mathtt{x}}_t}\pos_0, (1-\cumalpha^{\mathtt{x}}_t)\mathbb{I}), \label{eqn:noisepos}\\
& q(\atomfeat_t|\atomfeat_0)  & = \mathcal{C}(\atomfeat_t|\cumalpha^{\mathtt{v}}_t\atomfeat_0 + (1-\cumalpha^{\mathtt{v}}_t)\mathbf{1}/K), \label{eqn:noisetype}\\
& \text{where }\cumalpha^{\mathtt{u}}_t & = \prod\nolimits_{\tau=1}^{t}\alpha^{\mathtt{u}}_\tau, \ \alpha^{\mathtt{u}}_\tau=1 - \beta^{\mathtt{u}}_\tau, \ {\mathtt{u}}={\mathtt{x}} \text{ or } {\mathtt{v}}.\;\;\;\label{eqn:noiseschedule}
%\end{aligned}
\label{eqn:pos_prior}
\end{eqnarray}
%\xia{always punctuations after equations!!! also use ``eqnarray" instead of ``equation" + ``aligned" for multiple equations, each
%with a separate reference numbering...}
%\st{in which}, 
%where \ziqi{$\cumalpha^u_t = \prod_{\tau=1}^{t}\alpha^u_\tau$ and $\alpha^u_\tau=1 - \beta^u_\tau$ ($u$=$x$ or $v$)}.
%\xia{no such notations in the above equations; also subscript $s$ is abused with shape};
%$K$ is the number of categories for atom features.
%
%The details about noise schedules $\beta^x_t$ and $\beta^v_t$ are available in Supplementary Section \ref{XXX}. \ziqi{add trend}
%
Note that $\bar{\alpha}^{\mathtt{u}}_t$ ($\mathtt{u}={\mathtt{x}}\text{ or }{\mathtt{v}}$)
%($u$=$x$ or $v$) 
is monotonically decreasing from 1 to 0 over $t=[1,T]$. %\xia{=???}. 
%
As $t\rightarrow 1$, $\cumalpha^{\mathtt{x}}_t$ and $\cumalpha^{\mathtt{v}}_t$ are close to 1, leading to that $\pos_t$ or $\atomfeat_t$ approximates 
%the original data 
$\pos_0$ or $\atomfeat_0$.
%
Conversely, as  $t\rightarrow T$, $\cumalpha^{\mathtt{x}}_t$ and $\cumalpha^{\mathtt{v}}_t$ are close to 0,
leading to that $q(\pos_T|\pos_{0})$ %\st{$\rightarrow \mathcal{N}(\mathbf{0}, \mathbf{I})$} 
resembles  {$\mathcal{N}(\mathbf{0}, \mathbb{I})$} 
and $q(\atomfeat_T|\atomfeat_0)$ %\st{$\rightarrow \mathcal{C}(\mathbf{I}/K)$} 
resembles {$\mathcal{C}(\mathbf{1}/K)$}.

Using Bayes theorem, the ground-truth Normal posterior of atom positions $p(\pos_{t-1}|\pos_t, \pos_0)$ can be calculated in a
closed form~\cite{ho2020ddpm} as below,
%
\begin{eqnarray}
& p(\pos_{t-1}|\pos_t, \pos_0) = \mathcal{N}(\pos_{t-1}|\mu(\pos_t, \pos_0), \tilde{\beta}^\mathtt{x}_t\mathbb{I}), \label{eqn:gt_pos_posterior_1}\\
&\!\!\!\!\!\!\!\!\!\!\!\mu(\pos_t, \pos_0)\!=\!\frac{\sqrt{\bar{\alpha}^{\mathtt{x}}_{t-1}}\beta^{\mathtt{x}}_t}{1-\bar{\alpha}^{\mathtt{x}}_t}\pos_0\!+\!\frac{\sqrt{\alpha^{\mathtt{x}}_t}(1-\bar{\alpha}^{\mathtt{x}}_{t-1})}{1-\bar{\alpha}^{\mathtt{x}}_t}\pos_t, 
\tilde{\beta}^\mathtt{x}_t\!=\!\frac{1-\bar{\alpha}^{\mathtt{x}}_{t-1}}{1-\bar{\alpha}^{\mathtt{x}}_{t}}\beta^{\mathtt{x}}_t.\;\;\;
\end{eqnarray}
%
%\xia{Ziqi, please double check the above two equations!}
Similarly, the ground-truth categorical posterior of atom features $p(\atomfeat_{t-1}|\atomfeat_{t}, \atomfeat_0)$ can be calculated~\cite{hoogeboom2021catdiff} as below,
%
\begin{eqnarray}
& p(\atomfeat_{t-1}|\atomfeat_{t}, \atomfeat_0) = \mathcal{C}(\atomfeat_{t-1}|\mathbf{c}(\atomfeat_t, \atomfeat_0)), \label{eqn:gt_atomfeat_posterior_1}\\
& \mathbf{c}(\atomfeat_t, \atomfeat_0) = \tilde{\mathbf{c}}/{\sum_{k=1}^K \tilde{c}_k}, \label{eqn:gt_atomfeat_posterior_2} \\
& \tilde{\mathbf{c}} = [\alpha^{\mathtt{v}}_t\atomfeat_t + \frac{1 - \alpha^{\mathtt{v}}_t}{K}]\odot[\bar{\alpha}^{\mathtt{v}}_{t-1}\atomfeat_{0}+\frac{1-\bar{\alpha}^{\mathtt{v}}_{t-1}}{K}], 
\label{eqn:gt_atomfeat_posterior_3}
%\label{eqn:atomfeat_posterior}
\end{eqnarray}
%
%\xia{Ziqi: please double check the above equations!}
%
where $\tilde{c}_k$ denotes the likelihood of $k$-th class across $K$ classes in $\tilde{\mathbf{c}}$; 
$\odot$ denotes the element-wise product operation;
$\tilde{\mathbf{c}}$ is calculated using $\atomfeat_t$ and $\atomfeat_{0}$ and normalized into $\mathbf{c}(\atomfeat_t, \atomfeat_0)$ so as to represent
probabilities. %\xia{is this correct? is $\tilde{c}_k$ always greater than 0?}
%\xia{how is it calculated?}.
%\ziqi{the likelihood distribution $\tilde{c}$ is calculated by $p(\atomfeat_t|\atomfeat_{t-1})p(\atomfeat_{t-1}|\atomfeat_0)$, according to 
%Equation~\ref{eqn:noisetypeinter} and \ref{eqn:noisetype}.
%\xia{need to write the key idea of the above calculation...}
%
The proof of the above equations is available in Supplementary Section~\ref{supp:forward:proof}.

%===================================================================
\subsection{Variance Scheduling in \diffnoise}
\label{supp:forward:variance}
%===================================================================

Following Guan \etal~\cite{guan2023targetdiff}, we used a sigmoid $\beta$ schedule for the variance schedule $\beta_t^{\mathtt{x}}$ of atom coordinates as below,

\begin{equation}
\beta_t^{\mathtt{x}} = \text{sigmoid}(w_1(2 t / T - 1)) (w_2 - w_3) + w_3
\end{equation}
in which $w_i$($i$=1,2, or 3) are hyperparameters; $T$ is the maximum step.
%
We set $w_1=6$, $w_2=1.e-7$ and $w_3=0.01$.
%
For atom types, we used a cosine $\beta$ schedule~\cite{nichol2021} for $\beta_t^{\mathtt{v}}$ as below,

\begin{equation}
\begin{aligned}
& \bar{\alpha}_t^{\mathtt{v}} = \frac{f(t)}{f(0)}, f(t) = \cos(\frac{t/T+s}{1+s} \cdot \frac{\pi}{2})^2\\
& \beta_t^{\mathtt{v}} = 1 - \alpha_t^{\mathtt{v}} = 1 - \frac{\bar{\alpha}_t^{\mathtt{v}} }{\bar{\alpha}_{t-1}^{\mathtt{v}} }
\end{aligned}
\end{equation}
in which $s$ is a hyperparameter and set as 0.01.

As shown in Section ``Forward Diffusion Process'', the values of $\beta_t^{\mathtt{x}}$ and $\beta_t^{\mathtt{v}}$ should be 
sufficiently small to ensure the smoothness of forward diffusion process. In the meanwhile, their corresponding $\bar{\alpha}_t$
values should decrease from 1 to 0 over $t=[1,T]$.
%
Figure~\ref{fig:schedule} shows the values of $\beta_t$ and $\bar{\alpha}_t$ for atom coordinates and atom types with our hyperparameters.
%
Please note that the value of $\beta_{t}^{\mathtt{x}}$ is less than 0.1 for 990 out of 1,000 steps. %\bo{\st{, though it increases when $t$ is close to 1,000}}.
%
This guarantees the smoothness of the forward diffusion process.
%\bo{add $\beta_t^{\mathtt{x}}$ and $\beta_t^{\mathtt{v}}$ in the legend of the figure...}
%\bo{$\beta_t^{\mathtt{v}}$ does not look small when $t$ is close to 1000...}

\begin{figure}
	\begin{subfigure}[t]{.45\linewidth}
		\centering
		\includegraphics[width=.7\linewidth]{figures/var_schedule_beta.pdf}
	\end{subfigure}
	%
	\hfill
	\begin{subfigure}[t]{.45\linewidth}
		\centering
		\includegraphics[width=.7\linewidth]{figures/var_schedule_alpha.pdf}
	\end{subfigure}
	\caption{Schedule}
	\label{fig:schedule}
\end{figure}

%===================================================================
\subsection{Derivation of Forward Diffusion Process}
\label{supp:forward:proof}
%===================================================================

In \method, a Gaussian noise and a categorical noise are added to continuous atom position and discrete atom features, respectively.
%
Here, we briefly describe the derivation of posterior equations (i.e., Eq.~\ref{eqn:gt_pos_posterior_1}, and   \ref{eqn:gt_atomfeat_posterior_1}) for atom positions and atom types in our work.
%
We refer readers to Ho \etal~\cite{ho2020ddpm} and Kong \etal~\cite{kong2021diffwave} %\bo{add XXX~\etal here...} \cite{ho2020ddpm,kong2021diffwave} 	
for a detailed description of diffusion process for continuous variables and Hoogeboom \etal~\cite{hoogeboom2021catdiff} for
%\bo{add XXX~\etal here...} \cite{hoogeboom2021catdiff} for
the description of diffusion process for discrete variables.

For continuous atom positions, as shown in Kong \etal~\cite{kong2021diffwave}, according to Bayes theorem, given $q(\pos_t|\pos_{t-1})$ defined in Eq.~\ref{eqn:noiseposinter} and 
$q(\pos_t|\pos_0)$ defined in Eq.~\ref{eqn:noisepos}, the posterior $q(\pos_{t-1}|\pos_{t}, \pos_0)$ is derived as below (superscript $\mathtt{x}$ is omitted for brevity),

\begin{equation}
\begin{aligned}
& q(\pos_{t-1}|\pos_{t}, \pos_0)  = \frac{q(\pos_t|\pos_{t-1}, \pos_0)q(\pos_{t-1}|\pos_0)}{q(\pos_t|\pos_0)} \\
& =  \frac{\mathcal{N}(\pos_t|\sqrt{1-\beta_t}\pos_{t-1}, \beta_{t}\mathbf{I}) \mathcal{N}(\pos_{t-1}|\sqrt{\bar{\alpha}_{t-1}}\pos_{0}, (1-\bar{\alpha}_{t-1})\mathbf{I}) }{ \mathcal{N}(\pos_{t}|\sqrt{\bar{\alpha}_t}\pos_{0}, (1-\bar{\alpha}_t)\mathbf{I})}\\
& =  (2\pi{\beta_t})^{-\frac{3}{2}} (2\pi{(1-\bar{\alpha}_{t-1})})^{-\frac{3}{2}} (2\pi(1-\bar{\alpha}_t))^{\frac{3}{2}} \times \exp( \\
& -\frac{\|\pos_t - \sqrt{\alpha}_t\pos_{t-1}\|^2}{2\beta_t} -\frac{\|\pos_{t-1} - \sqrt{\bar{\alpha}}_{t-1}\pos_{0} \|^2}{2(1-\bar{\alpha}_{t-1})} \\
& + \frac{\|\pos_t - \sqrt{\bar{\alpha}_t}\pos_0\|^2}{2(1-\bar{\alpha}_t)}) \\
& = (2\pi\tilde{\beta}_t)^{-\frac{3}{2}} \exp(-\frac{1}{2\tilde{\beta}_t}\|\pos_{t-1}-\frac{\sqrt{\bar{\alpha}_{t-1}}\beta_t}{1-\bar{\alpha}_t}\pos_0 \\
& - \frac{\sqrt{\alpha_t}(1-\bar{\alpha}_{t-1})}{1-\bar{\alpha}_t}\pos_{t}\|^2) \\
& \text{where}\ \tilde{\beta}_t = \frac{1-\bar{\alpha}_{t-1}}{1-\bar{\alpha}_t}\beta_t.
\end{aligned}
\end{equation}
%\bo{marked part does not look right to me.}
%\bo{How to you derive from the second equation to the third one?}

Therefore, the posterior of atom positions is derived as below,

\begin{equation}
q(\pos_{t-1}|\pos_{t}, \pos_0)\!\!=\!\!\mathcal{N}(\pos_{t-1}|\frac{\sqrt{\bar{\alpha}_{t-1}}\beta_t}{1-\bar{\alpha}_t}\pos_0 + \frac{\sqrt{\alpha_t}(1-\bar{\alpha}_{t-1})}{1-\bar{\alpha}_t}\pos_{t}, \tilde{\beta}_t\mathbf{I}).
\end{equation}

For discrete atom features, as shown in Hoogeboom \etal~\cite{hoogeboom2021catdiff} and Guan \etal~\cite{guan2023targetdiff},
according to Bayes theorem, the posterior $q(\atomfeat_{t-1}|\atomfeat_{t}, \atomfeat_0)$ is derived as below (supperscript $\mathtt{v}$ is omitted for brevity),

\begin{equation}
\begin{aligned}
& q(\atomfeat_{t-1}|\atomfeat_{t}, \atomfeat_0) =  \frac{q(\atomfeat_t|\atomfeat_{t-1}, \atomfeat_0)q(\atomfeat_{t-1}|\atomfeat_0)}{\sum_{\scriptsize{\atomfeat}_{t-1}}q(\atomfeat_t|\atomfeat_{t-1}, \atomfeat_0)q(\atomfeat_{t-1}|\atomfeat_0)} \\
%& = \frac{\mathcal{C}(\atomfeat_t|(1-\beta_t)\atomfeat_{t-1} + \beta_t\frac{\mathbf{1}}{K}) \mathcal{C}(\atomfeat_{t-1}|\bar{\alpha}_{t-1}\atomfeat_0+(1-\bar{\alpha}_{t-1})\frac{\mathbf{1}}{K})} \\
\end{aligned}
\end{equation}

For $q(\atomfeat_t|\atomfeat_{t-1}, \atomfeat_0)$, we have % $\atomfeat_t=\atomfeat_{t-1}$ with probability $1-\beta_t+\beta_t / K$, and $\atomfeat_t \neq \atomfeat_{t-1}$
%with probability $\beta_t / K$.
%
%Therefore, this function can be symmetric, that is, 
%
\begin{equation}
\begin{aligned}
q(\atomfeat_t|\atomfeat_{t-1}, \atomfeat_0) & = \mathcal{C}(\atomfeat_t|(1-\beta_t)\atomfeat_{t-1} + \beta_t/{K})\\
& = \begin{cases}
1-\beta_t+\beta_t/K,\!&\text{when}\ \atomfeat_{t} = \atomfeat_{t-1},\\
\beta_t / K,\! &\text{when}\ \atomfeat_{t} \neq \atomfeat_{t-1},
\end{cases}\\
& = \mathcal{C}(\atomfeat_{t-1}|(1-\beta_t)\atomfeat_{t} + \beta_t/{K}).
\end{aligned}
%\mathcal{C}(\atomfeat_{t-1}|(1-\beta_{t})\atomfeat_{t} + \beta_t\frac{\mathbf{1}}{K}).
\end{equation}
%
Therefore, we have
%\bo{why it can be symmetric}
%
\begin{equation}
\begin{aligned}
& q(\atomfeat_t|\atomfeat_{t-1}, \atomfeat_0)q(\atomfeat_{t-1}|\atomfeat_0) \\
& = \mathcal{C}(\atomfeat_{t-1}|(1-\beta_t)\atomfeat_{t} + \beta_t\frac{\mathbf{1}}{K}) \mathcal{C}(\atomfeat_{t-1}|\bar{\alpha}_{t-1}\atomfeat_0+(1-\bar{\alpha}_{t-1})\frac{\mathbf{1}}{K}) \\
& = [\alpha_t\atomfeat_t + \frac{1 - \alpha_t}{K}]\odot[\bar{\alpha}_{t-1}\atomfeat_{0}+\frac{1-\bar{\alpha}_{t-1}}{K}].
\end{aligned}
\end{equation}
%
%\bo{what is $\tilde{\mathbf{c}}$...}
Therefore, with $q(\atomfeat_t|\atomfeat_{t-1}, \atomfeat_0)q(\atomfeat_{t-1}|\atomfeat_0) = \tilde{\mathbf{c}}$, the posterior is as below,

\begin{equation}
q(\atomfeat_{t-1}|\atomfeat_{t}, \atomfeat_0) = \mathcal{C}(\atomfeat_{t-1}|\mathbf{c}(\atomfeat_t, \atomfeat_0)) = \frac{\tilde{\mathbf{c}}}{\sum_{k}^K\tilde{c}_k}.
\end{equation}

%%%%%%%%%%%%%%%%%%%%%%%%%%%%%%%%%%%%%%%%%%%%%
\section{{Backward Generative Process} (\diffgenerative)}
\label{supp:backward}
%%%%%%%%%%%%%%%%%%%%%%%%%%%%%%%%%%%%%%%%%%%%%

Following Ho \etal~\cite{ho2020ddpm}, with $\tilde{\pos}_{0,t}$, the probability of $\pos_{t-1}$ denoised from $\pos_t$, denoted as $p(\pos_{t-1}|\pos_t)$,
can be estimated %\hl{parameterized} \xia{???} 
by the approximated posterior $p_{\boldsymbol{\Theta}}(\pos_{t-1}|\pos_t, \tilde{\pos}_{0,t})$ as below,
%
\begin{equation}
\begin{aligned}
p(\pos_{t-1}|\pos_t) & \approx p_{\boldsymbol{\Theta}}(\pos_{t-1}|\pos_t, \tilde{\pos}_{0,t}) \\
& = \mathcal{N}(\pos_{t-1}|\mu_{\boldsymbol{\Theta}}(\pos_t, \tilde{\pos}_{0,t}),\tilde{\beta}_t^{\mathtt{x}}\mathbb{I}),
\end{aligned}
\label{eqn:aprox_pos_posterior}
\end{equation}
%
where ${\boldsymbol{\Theta}}$ is the learnable parameter; $\mu_{\boldsymbol{\Theta}}(\pos_t, \tilde{\pos}_{0,t})$ is an estimate %estimation
of $\mu(\pos_t, \pos_{0})$ by replacing $\pos_0$ with its estimate $\tilde{\pos}_{0,t}$ 
in Equation~{\ref{eqn:gt_pos_posterior_1}}.
%
Similarly, with $\tilde{\atomfeat}_{0,t}$, the probability of $\atomfeat_{t-1}$ denoised from $\atomfeat_t$, denoted as $p(\atomfeat_{t-1}|\atomfeat_t)$, 
can be estimated %\hl{parameterized} 
by the approximated posterior $p_{\boldsymbol{\Theta}}(\atomfeat_{t-1}|\atomfeat_t, \tilde{\atomfeat}_{0,t})$ as below,
%
\begin{equation}
\begin{aligned}
p(\atomfeat_{t-1}|\atomfeat_t)\approx p_{\boldsymbol{\Theta}}(\atomfeat_{t-1}|\atomfeat_{t}, \tilde{\atomfeat}_{0,t}) 
=\mathcal{C}(\atomfeat_{t-1}|\mathbf{c}_{\boldsymbol{\Theta}}(\atomfeat_t, \tilde{\atomfeat}_{0,t})),\!\!\!\!
\end{aligned}
\label{eqn:aprox_atomfeat_posterior}
\end{equation}
%
where $\mathbf{c}_{\boldsymbol{\Theta}}(\atomfeat_t, \tilde{\atomfeat}_{0,t})$ is an estimate of $\mathbf{c}(\atomfeat_t, \atomfeat_0)$
by replacing $\atomfeat_0$  
with its estimate $\tilde{\atomfeat}_{0,t}$ in Equation~\ref{eqn:gt_atomfeat_posterior_1}.



%===================================================================
\section{\method Loss Function Derivation}
\label{supp:training:loss}
%===================================================================

In this section, we demonstrate that a step weight $w_t^{\mathtt{x}}$ based on the signal-to-noise ratio $\lambda_t$ should be 
included into the atom position loss (Eq.~\ref{eqn:diff:obj:pos}).
%
In the diffusion process for continuous variables, the optimization problem is defined 
as below~\cite{ho2020ddpm},
%
\begin{equation*}
\begin{aligned}
& \arg\min_{\boldsymbol{\Theta}}KL(q(\pos_{t-1}|\pos_t, \pos_0)|p_{\boldsymbol{\Theta}}(\pos_{t-1}|\pos_t, \tilde{\pos}_{0,t})) \\
& = \arg\min_{\boldsymbol{\Theta}} \frac{\bar{\alpha}_{t-1}(1-\alpha_t)}{2(1-\bar{\alpha}_{t-1})(1-\bar{\alpha}_{t})}\|\tilde{\pos}_{0, t}-\pos_0\|^2 \\
& = \arg\min_{\boldsymbol{\Theta}} \frac{1-\alpha_t}{2(1-\bar{\alpha}_{t-1})\alpha_{t}} \|\tilde{\boldsymbol{\epsilon}}_{0,t}-\boldsymbol{\epsilon}_0\|^2,
\end{aligned}
\end{equation*}
where $\boldsymbol{\epsilon}_0 = \frac{\pos_t - \sqrt{\bar{\alpha}_t}\pos_0}{\sqrt{1-\bar{\alpha}_t}}$ is the ground-truth noise variable sampled from $\mathcal{N}(\mathbf{0}, \mathbf{1})$ and is used to sample $\pos_t$ from $\mathcal{N}(\pos_t|\sqrt{\cumalpha_t}\pos_0, (1-\cumalpha_t)\mathbf{I})$ in Eq.~\ref{eqn:noisetype};
$\tilde{\boldsymbol{\epsilon}}_0 = \frac{\pos_t - \sqrt{\bar{\alpha}_t}\tilde{\pos}_{0, t}}{\sqrt{1-\bar{\alpha}_t}}$ is the predicted noise variable. 

%A simplified training objective is proposed by Ho \etal~\cite{ho2020ddpm} as below,
Ho \etal~\cite{ho2020ddpm} further simplified the above objective as below and
demonstrated that the simplified one can achieve better performance:
%
\begin{equation}
\begin{aligned}
& \arg\min_{\boldsymbol{\Theta}} \|\tilde{\boldsymbol{\epsilon}}_{0,t}-\boldsymbol{\epsilon}_0\|^2 \\
& = \arg\min_{\boldsymbol{\Theta}} \frac{\bar{\alpha}_t}{1-\bar{\alpha}_t}\|\tilde{\pos}_{0,t}-\pos_0\|^2,
\end{aligned}
\label{eqn:supp:losspos}
\end{equation}
%
where $\lambda_t=\frac{\bar{\alpha}_t}{1-\bar{\alpha}_t}$ is the signal-to-noise ratio.
%
While previous work~\cite{guan2023targetdiff} applies uniform step weights across
different steps, we demonstrate that a step weight should be included into the atom position loss according to Eq.~\ref{eqn:supp:losspos}.
%
However, the value of $\lambda_t$ could be very large when $\bar{\alpha}_t$ is close to 1 as $t$ approaches 1.
%
Therefore, we clip the value of $\lambda_t$ with threshold $\delta$ in Eq.~\ref{eqn:diff:obj:pos}.




\end{document}
