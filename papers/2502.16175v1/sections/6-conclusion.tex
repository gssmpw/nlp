\section{Discussions and Conclusions}
We have introduced Mojito, an innovative system for real-time human motion capture and online motion analysis, via jitter-reduced inertial tokens. By integrating a novel jitter-reduced and motion-aware IMU tokenizer with a large language model, Mojito establishes an interaction-friendly framework for motion description and instructor application. Experimental results demonstrate that our method achieves robust motion capture, effectively addressing various noisy input conditions that pose challenges for traditional RNN-based and physics solver-based approaches. Additionally, our user study highlights the professionalism, naturalism, precision and brevity of textual feedback generated by Mojito, underscoring its practical utility in real-world applications such as fitness training, rehabilitation and AR/VR.

\vspace{-4mm}
\paragraph{Limitations and Future Work}
Despite its contributions, Mojito still has several limitations. First, our jitter-reduced IMU tokenizer cannot operate in a per-frame inference manner, as it requires input signals to be segmented into data chunks. This leads to relatively high latency and discontinuous motion reconstruction results. Additionally, Mojito is constrained to structured IMU sensor placements on the human body, which may limit its applicability in more unstructured scenarios, such as general IMU signal understanding in robotics, autonomous driving, and object tracking.
In the future, we aim to explore autoregressive motion capture in real time through next-token prediction to mitigate inference latency and discontinuity. 
Furthermore, extend our multimodal system to general and unstructured IMU sensor signal understanding holds significant promise, as it could substantially broaden the scope of practical applications.