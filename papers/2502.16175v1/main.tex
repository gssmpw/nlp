% \PassOptionsToPackage{prologue,dvipsnames}{xcolor}
\documentclass[10pt,twocolumn,letterpaper]{article}

\usepackage{iccv}
\usepackage{times}
\usepackage{epsfig}
\usepackage{graphicx}
\usepackage{cuted}
\usepackage{capt-of}
\usepackage{caption}
\usepackage{subcaption}
\usepackage{amsmath}
\usepackage{amssymb}
\usepackage{bm}
\usepackage{bbm}
\usepackage{mathtools}
\usepackage{booktabs} % For formal tables
\usepackage{lipsum}
\usepackage{xcolor}
\usepackage{bbding}
\usepackage{colortbl}
% \usepackage[svgnames]{xcolor}
% \usepackage[dvipsnames]{xcolor}
\usepackage{enumitem}
\setenumerate[1]{itemsep=0pt,partopsep=0pt,parsep=\parskip,topsep=5pt}
\setitemize[1]{itemsep=0pt,partopsep=0pt,parsep=\parskip,topsep=5pt}
\setdescription{itemsep=0pt,partopsep=0pt,parsep=\parskip,topsep=5pt}
\DeclareMathOperator*{\argmax}{\arg\!\max}
\DeclareMathOperator*{\argmin}{\arg\!\min}
\DeclarePairedDelimiterX{\js}[2]{}{}{%
  #1\;\delimsize\|\;#2%
}
\DeclarePairedDelimiter{\norm}{\lVert}{\rVert}

\newcommand\blfootnote[1]{%
  \begingroup
  \renewcommand\thefootnote{}\footnote{#1}%
  \addtocounter{footnote}{-1}%
  \endgroup
}

% Include other packages here, before hyperref.

% If you comment hyperref and then uncomment it, you should delete
% egpaper.aux before re-running latex.  (Or just hit 'q' on the first latex
% run, let it finish, and you should be clear).
\usepackage[pagebackref,breaklinks,colorlinks]{hyperref}

\iccvfinalcopy % *** Uncomment this line for the final submission

\def\iccvPaperID{****} % *** Enter the ICCV Paper ID here
\def\httilde{\mbox{\tt\raisebox{-.5ex}{\symbol{126}}}}

% Pages are numbered in submission mode, and unnumbered in camera-ready
\ificcvfinal\pagestyle{empty}\fi

\begin{document}

%%%%%%%%% TITLE
\title{Mojito: LLM-Aided Motion Instructor with Jitter-Reduced Inertial Tokens}

\author{
    Ziwei Shan\textsuperscript{1,*}\qquad
    Yaoyu He\textsuperscript{1,*}\qquad
    Chengfeng Zhao\textsuperscript{1,*,\textdagger}\qquad
    Jiashen Du\textsuperscript{1}\qquad
    Jingyan Zhang\textsuperscript{1}\\
    Qixuan Zhang\textsuperscript{1,2}\qquad
    Jingyi Yu\textsuperscript{1,\textdaggerdbl}\qquad
    Lan Xu\textsuperscript{1,\textdaggerdbl}\\
    \textsuperscript{1}ShanghaiTech University\qquad
    \textsuperscript{2}Deemos Technology\\
    {\tt\footnotesize \{shanzw2022,heyy2022,zhaochf2022,dujsh2022,zhangjy7,zhangqx1,yujingyi,xulan1\}@shanghaitech.edu.cn}
}

\maketitle
% Remove page # from the first page of camera-ready.
\ificcvfinal\thispagestyle{empty}\fi

%%%%%%%%% ABSTRACT
\begin{strip}\centering
    \vspace{-45px}
    \captionsetup{type=figure}
    \includegraphics[width=\textwidth]{figures/teaser.png}
    \vspace{-15px}
    \caption{Our \textit{Mojito} produces real-time human motion capture and online motion analysis, from six inertial measurement units (IMUs). (a) Human who performs exercise wearing IMU sensors in reality. (b) Digitalized human motion from six IMU sensor signals. (c) Motion recognition, analysis and instruction feedback.}
\label{fig:teaser}
\vspace{-10px}
\end{strip}

\blfootnote{\textsuperscript{*}Equal contributions}
\blfootnote{\textsuperscript{\textdagger}Project lead}
\blfootnote{\textsuperscript{\textdaggerdbl}Corresponding author}

\begin{abstract}
Human bodily movements convey critical insights into action intentions and cognitive processes, yet existing multimodal systems primarily focused on understanding human motion via language, vision, and audio, which struggle to capture the dynamic forces and torques inherent in 3D motion. Inertial measurement units (IMUs) present a promising alternative, offering lightweight, wearable, and privacy-conscious motion sensing. However, processing of streaming IMU data faces challenges such as instable wireless transmission, sensor noise, and drift, limiting their utility for long-term real-time motion capture (MoCap), and more importantly, online motion analysis. 
%
To address these challenges, we introduce Mojito, an intelligent motion agent that integrates inertial sensing with large language models (LLMs) for interactive motion capture and behavioral analysis. The core innovation of Mojito lies in a jitter-reduced inertial token representation with a novel IMU signal encoding framework, and an extended language model involving inertial tokens. By employing VQVAE, Mojito learns a discrete latent space of continuous IMU signals, mitigating sensor noise and drift through quantization. The inertial tokens are then aligned with inductive bias of natural language and mapped to textual semantics to enhance compatibility with LLMs, enabling efficient sequence modeling. 
%
To support domain-specific applications, Mojito further incorporates tunable LoRA adapters, facilitating personalized feedback tailored to roles such as fitness trainers or rehabilitation therapists. 
%
Extensive experiments demonstrate that Mojito outperforms existing IMU-based methods in motion capture under noisy conditions, and achieves comparable behavior analysis capability compared to large vision-language models. The user study further highlights its practical effectiveness in various scenarios as a versatile tool for intelligent human-agent interaction. Our code and data will be released at \href{https://koyui.github.io/mojito/}{our project page}.
\end{abstract}

%%%%%%%%% BODY TEXT
%!TEX root = gcn.tex
\section{Introduction}
Graphs, representing structural data and topology, are widely used across various domains, such as social networks and merchandising transactions.
Graph convolutional networks (GCN)~\cite{iclr/KipfW17} have significantly enhanced model training on these interconnected nodes.
However, these graphs often contain sensitive information that should not be leaked to untrusted parties.
For example, companies may analyze sensitive demographic and behavioral data about users for applications ranging from targeted advertising to personalized medicine.
Given the data-centric nature and analytical power of GCN training, addressing these privacy concerns is imperative.

Secure multi-party computation (MPC)~\cite{crypto/ChaumDG87,crypto/ChenC06,eurocrypt/CiampiRSW22} is a critical tool for privacy-preserving machine learning, enabling mutually distrustful parties to collaboratively train models with privacy protection over inputs and (intermediate) computations.
While research advances (\eg,~\cite{ccs/RatheeRKCGRS20,uss/NgC21,sp21/TanKTW,uss/WatsonWP22,icml/Keller022,ccs/ABY318,folkerts2023redsec}) support secure training on convolutional neural networks (CNNs) efficiently, private GCN training with MPC over graphs remains challenging.

Graph convolutional layers in GCNs involve multiplications with a (normalized) adjacency matrix containing $\numedge$ non-zero values in a $\numnode \times \numnode$ matrix for a graph with $\numnode$ nodes and $\numedge$ edges.
The graphs are typically sparse but large.
One could use the standard Beaver-triple-based protocol to securely perform these sparse matrix multiplications by treating graph convolution as ordinary dense matrix multiplication.
However, this approach incurs $O(\numnode^2)$ communication and memory costs due to computations on irrelevant nodes.
%
Integrating existing cryptographic advances, the initial effort of SecGNN~\cite{tsc/WangZJ23,nips/RanXLWQW23} requires heavy communication or computational overhead.
Recently, CoGNN~\cite{ccs/ZouLSLXX24} optimizes the overhead in terms of  horizontal data partitioning, proposing a semi-honest secure framework.
Research for secure GCN over vertical data  remains nascent.

Current MPC studies, for GCN or not, have primarily targeted settings where participants own different data samples, \ie, horizontally partitioned data~\cite{ccs/ZouLSLXX24}.
MPC specialized for scenarios where parties hold different types of features~\cite{tkde/LiuKZPHYOZY24,icml/CastigliaZ0KBP23,nips/Wang0ZLWL23} is rare.
This paper studies $2$-party secure GCN training for these vertical partition cases, where one party holds private graph topology (\eg, edges) while the other owns private node features.
For instance, LinkedIn holds private social relationships between users, while banks own users' private bank statements.
Such real-world graph structures underpin the relevance of our focus.
To our knowledge, no prior work tackles secure GCN training in this context, which is crucial for cross-silo collaboration.


To realize secure GCN over vertically split data, we tailor MPC protocols for sparse graph convolution, which fundamentally involves sparse (adjacency) matrix multiplication.
Recent studies have begun exploring MPC protocols for sparse matrix multiplication (SMM).
ROOM~\cite{ccs/SchoppmannG0P19}, a seminal work on SMM, requires foreknowledge of sparsity types: whether the input matrices are row-sparse or column-sparse.
Unfortunately, GCN typically trains on graphs with arbitrary sparsity, where nodes have varying degrees and no specific sparsity constraints.
Moreover, the adjacency matrix in GCN often contains a self-loop operation represented by adding the identity matrix, which is neither row- nor column-sparse.
Araki~\etal~\cite{ccs/Araki0OPRT21} avoid this limitation in their scalable, secure graph analysis work, yet it does not cover vertical partition.

% and related primitives
To bridge this gap, we propose a secure sparse matrix multiplication protocol, \osmm, achieving \emph{accurate, efficient, and secure GCN training over vertical data} for the first time.

\subsection{New Techniques for Sparse Matrices}
The cost of evaluating a GCN layer is dominated by SMM in the form of $\adjmat\feamat$, where $\adjmat$ is a sparse adjacency matrix of a (directed) graph $\graph$ and $\feamat$ is a dense matrix of node features.
For unrelated nodes, which often constitute a substantial portion, the element-wise products $0\cdot x$ are always zero.
Our efficient MPC design 
avoids unnecessary secure computation over unrelated nodes by focusing on computing non-zero results while concealing the sparse topology.
We achieve this~by:
1) decomposing the sparse matrix $\adjmat$ into a product of matrices (\S\ref{sec::sgc}), including permutation and binary diagonal matrices, that can \emph{faithfully} represent the original graph topology;
2) devising specialized protocols (\S\ref{sec::smm_protocol}) for efficiently multiplying the structured matrices while hiding sparsity topology.


 
\subsubsection{Sparse Matrix Decomposition}
We decompose adjacency matrix $\adjmat$ of $\graph$ into two bipartite graphs: one represented by sparse matrix $\adjout$, linking the out-degree nodes to edges, the other 
by sparse matrix $\adjin$,
linking edges to in-degree nodes.

%\ie, we decompose $\adjmat$ into $\adjout \adjin$, where $\adjout$ and $\adjin$ are sparse matrices representing these connections.
%linking out-degree nodes to edges and edges to in-degree nodes of $\graph$, respectively.

We then permute the columns of $\adjout$ and the rows of $\adjin$ so that the permuted matrices $\adjout'$ and $\adjin'$ have non-zero positions with \emph{monotonically non-decreasing} row and column indices.
A permutation $\sigma$ is used to preserve the edge topology, leading to an initial decomposition of $\adjmat = \adjout'\sigma \adjin'$.
This is further refined into a sequence of \emph{linear transformations}, 
which can be efficiently computed by our MPC protocols for 
\emph{oblivious permutation}
%($\Pi_{\ssp}$) 
and \emph{oblivious selection-multiplication}.
% ($\Pi_\SM$)
\iffalse
Our approach leverages bipartite graph representation and the monotonicity of non-zero positions to decompose a general sparse matrix into linear transformations, enhancing the efficiency of our MPC protocols.
\fi
Our decomposition approach is not limited to GCNs but also general~SMM 
by 
%simply 
treating them 
as adjacency matrices.
%of a graph.
%Since any sparse matrix can be viewed 

%allowing the same technique to be applied.

 
\subsubsection{New Protocols for Linear Transformations}
\emph{Oblivious permutation} (OP) is a two-party protocol taking a private permutation $\sigma$ and a private vector $\xvec$ from the two parties, respectively, and generating a secret share $\l\sigma \xvec\r$ between them.
Our OP protocol employs correlated randomnesses generated in an input-independent offline phase to mask $\sigma$ and $\xvec$ for secure computations on intermediate results, requiring only $1$ round in the online phase (\cf, $\ge 2$ in previous works~\cite{ccs/AsharovHIKNPTT22, ccs/Araki0OPRT21}).

Another crucial two-party protocol in our work is \emph{oblivious selection-multiplication} (OSM).
It takes a private bit~$s$ from a party and secret share $\l x\r$ of an arithmetic number~$x$ owned by the two parties as input and generates secret share $\l sx\r$.
%between them.
%Like our OP protocol, o
Our $1$-round OSM protocol also uses pre-computed randomnesses to mask $s$ and $x$.
%for secure computations.
Compared to the Beaver-triple-based~\cite{crypto/Beaver91a} and oblivious-transfer (OT)-based approaches~\cite{pkc/Tzeng02}, our protocol saves ${\sim}50\%$ of online communication while having the same offline communication and round complexities.

By decomposing the sparse matrix into linear transformations and applying our specialized protocols, our \osmm protocol
%($\prosmm$) 
reduces the complexity of evaluating $\numnode \times \numnode$ sparse matrices with $\numedge$ non-zero values from $O(\numnode^2)$ to $O(\numedge)$.

%(\S\ref{sec::secgcn})
\subsection{\cgnn: Secure GCN made Efficient}
Supported by our new sparsity techniques, we build \cgnn, 
a two-party computation (2PC) framework for GCN inference and training over vertical
%ly split
data.
Our contributions include:

1) We are the first to explore sparsity over vertically split, secret-shared data in MPC, enabling decompositions of sparse matrices with arbitrary sparsity and isolating computations that can be performed in plaintext without sacrificing privacy.

2) We propose two efficient $2$PC primitives for OP and OSM, both optimally single-round.
Combined with our sparse matrix decomposition approach, our \osmm protocol ($\prosmm$) achieves constant-round communication costs of $O(\numedge)$, reducing memory requirements and avoiding out-of-memory errors for large matrices.
In practice, it saves $99\%+$ communication
%(Table~\ref{table:comm_smm}) 
and reduces ${\sim}72\%$ memory usage over large $(5000\times5000)$ matrices compared with using Beaver triples.
%(Table~\ref{table:mem_smm_sparse}) ${\sim}16\%$-

3) We build an end-to-end secure GCN framework for inference and training over vertically split data, maintaining accuracy on par with plaintext computations.
We will open-source our evaluation code for research and deployment.

To evaluate the performance of $\cgnn$, we conducted extensive experiments over three standard graph datasets (Cora~\cite{aim/SenNBGGE08}, Citeseer~\cite{dl/GilesBL98}, and Pubmed~\cite{ijcnlp/DernoncourtL17}),
reporting communication, memory usage, accuracy, and running time under varying network conditions, along with an ablation study with or without \osmm.
Below, we highlight our key achievements.

\textit{Communication (\S\ref{sec::comm_compare_gcn}).}
$\cgnn$ saves communication by $50$-$80\%$.
(\cf,~CoGNN~\cite{ccs/KotiKPG24}, OblivGNN~\cite{uss/XuL0AYY24}).

\textit{Memory usage (\S\ref{sec::smmmemory}).}
\cgnn alleviates out-of-memory problems of using %the standard 
Beaver-triples~\cite{crypto/Beaver91a} for large datasets.

\textit{Accuracy (\S\ref{sec::acc_compare_gcn}).}
$\cgnn$ achieves inference and training accuracy comparable to plaintext counterparts.
%training accuracy $\{76\%$, $65.1\%$, $75.2\%\}$ comparable to $\{75.7\%$, $65.4\%$, $74.5\%\}$ in plaintext.

{\textit{Computational efficiency (\S\ref{sec::time_net}).}} 
%If the network is worse in bandwidth and better in latency, $\cgnn$ shows more benefits.
$\cgnn$ is faster by $6$-$45\%$ in inference and $28$-$95\%$ in training across various networks and excels in narrow-bandwidth and low-latency~ones.

{\textit{Impact of \osmm (\S\ref{sec:ablation}).}}
Our \osmm protocol shows a $10$-$42\times$ speed-up for $5000\times 5000$ matrices and saves $10$-2$1\%$ memory for ``small'' datasets and up to $90\%$+ for larger ones.

\section{Related Work}

\subsection{Personalization and Role-Playing}
Recent works have introduced benchmark datasets for personalizing LLM outputs in tasks like email, abstract, and news writing, focusing on shorter outputs (e.g., 300 tokens for product reviews \citep{kumar2024longlamp} and 850 for news writing \citep{shashidhar-etal-2024-unsupervised}). These methods infer user traits from history for task-specific personalization \citep{sun-etal-2024-revealing, sun-etal-2025-persona, pal2024beyond, li2023teach, salemi2025reasoning}. In contrast, we tackle the more subjective problem of long-form story writing, with author stories averaging 1500 tokens. Unlike prior role-playing approaches that use predefined personas (e.g., Tony Stark, Confucius) \citep{wang-etal-2024-rolellm, sadeq-etal-2024-mitigating, tu2023characterchat, xu2023expertprompting}, we propose a novel method to infer story-writing personas from an author’s history to guide role-playing.


\subsection{Story Understanding and Generation}  
Prior work on persona-aware story generation \citep{yunusov-etal-2024-mirrorstories, bae-kim-2024-collective, zhang-etal-2022-persona, chandu-etal-2019-way} defines personas using discrete attributes like personality traits, demographics, or hobbies. Similarly, \citep{zhu-etal-2023-storytrans} explore story style transfer across pre-defined domains (e.g., fairy tales, martial arts, Shakespearean plays). In contrast, we mimic an individual author's writing style based on their history. Our approach differs by (1) inferring long-form author personas—descriptions of an author’s style from their past works, rather than relying on demographics, and (2) handling long-form story generation, averaging 1500 tokens per output, exceeding typical story lengths in prior research.
\section{Jitter-reduced IMU Tokenizer}
\label{sec:method1}
In practical application of long-sequence motion capture and online motion analysis using IMU sensors, device connection, signal transmission and wearing fashion can significantly influence the quality of motion capture and the convenience of user experience. However, it is challenging due to the inherent defects of IMUs such as data drifting and jittery signals. Therefore, we start by proposing a jitter-reduced and motion-aware IMU tokenizer to represent sparse inertial signals by discrete tokens, which can compress continuous IMU signals into a fixed collection of latent codes shared with motion latent space. Consequently, the discrete inertial tokens can be integrated into the vocabulary of LLMs, while also support high-quality motion reconstruction. The IMU tokenizer is built upon the standard VQ-VAE framework \cite{van2017neural}, with a novel distribution matching strategy to learn an approximate latent space of corresponding human motion. Additionally, linguistic properties are assigned to the learned inertial tokens through regularization under Zipf's law \cite{zipf2013psycho}, which facilitates following semantic alignment with natural language.

\subsection{Motion VQ-VAE}
\label{sec:motion_vq}
We first follow MotionGPT \cite{jiang2023motiongpt} to learn a VQ-VAE for human motion. Differently, we represent human motion with a complete state of root joint and foot-ground contacts to suit IMU sensor characteristics. In addition, we involve regularization terms on foot-ground contacts to eliminate jittery results and sliding artifacts in decoded motion.

\vspace{-4mm}
\paragraph{Motion Representation.} While HumanML3D \cite{Guo_2022_CVPR} establishes an effective motion representation for text-to-motion generation tasks, it is limited to incomplete global dynamics and missing foot-ground contacts. Inspired by HuMoR \cite{rempe2021humor}, we incorporate root translation and angular velocity along all three axes into our representation to improve expressiveness. Specifically, we represent a motion sequence as
\begin{equation}
    \mathbf M^{1:T} = \left[
        \mathbf r
        \quad \dot{\mathbf{r}}
        \quad \mathbf \Phi 
        \quad \dot{\mathbf{\Phi}}
        \quad \mathbf{j}^r
        \quad \mathbf{j}^p
        \quad \mathbf{j}^v
        \quad \mathbf{p}
    \right] \in \mathbb{R}^{T\times d_m}\text{,}
\end{equation}
where $T$ is the sequence length. Within the representation, we first include the root translation $\mathbf r\in\mathbb R^{T\times3}$, linear velocity $\dot{\mathbf{r}} \in\mathbb R^{T\times3}$, orientation $\mathbf \Phi\in\mathbb R^{T\times6}$, and angular velocity $\dot \Phi\in\mathbb R^{T\times 3}$. Then, we use $\mathbf {j}^r\in \mathbb R^{T\times 6J}$, $\mathbf {j}^p\in\mathbb R^{T\times 3J}$, $\mathbf {j}^v\in\mathbb R^{T\times3J}$ to represent local joint rotations, positions, and velocities, respectively. Finally, $\mathbf p\in \mathbb R^{T\times 4}$ records the binary contact labels of toes and heels. Here, $d_m=271$ is the dimension of our motion representation, and $J=21$ is the number of local joints. All the rotational parts are in 6D rotation convention \cite{zhou2019continuity}.

\vspace{-4mm}
\paragraph{Training of Motion VQ-VAE} Given a motion sequence $\mathbf{M}^{1:T}$, we first encode it into discrete latent codes $\mathbf{Z}^{\text{motion}} \in \mathbb{R}^{S\times d_z}$ using 1D convolution layers, where $d_z=512$ is the dimension of latent code, and $S$ is the number of the resulting latent codes. We define the hyperparameter $l=\lfloor T/S\rfloor$ as the compression rate for discretization. Following the encoding process, each latent code $\mathbf{z}_s^{\text{motion}}$ is quantized to a learned codebook $\mathbf{C}^{\text{motion}}\in \mathbb{R}^{K\times d_z}$, where $K$ is the codebook size. The quantization process runs as follows
\begin{equation}
    \mathbf{b}_{s}^{\text{motion}} = \underset{\mathbf{c}_k^{\text{motion}}}{\argmin}\left\Vert \mathbf{z}_{s}^{\text{motion}}-\mathbf{c}_k^{\text{motion}} \right\Vert_2^2\text{,}
\end{equation}
which selects the nearest codebook entry according to Euclidean metric, and results in the motion token sequence $\mathbf B^{\text{motion}}\in \mathbb{R}^{S\times d_z}$. Subsequently, $\mathbf{B}^{\text{motion}}$ is fed into the decoder to reconstruct original motion sequence $\hat{\mathbf M}^{1:T'}$ with possible truncation $T'=lS$. To train the motion VQ-VAE, we utilize the discrete representation learning objective \cite{van2017neural} to supervise our network
\begin{equation}
    \mathcal{L}_{\text{vq}} = \lambda_{\text{recon}}\mathcal{L}_{\text{recon}} + \lambda_{\text{commit}}\mathcal{L}_{\text{commit}}\text{.}
\end{equation}
Specifically, the reconstruction loss is defined as
\begin{equation}
    \mathcal L_{\textrm{recon}} = \frac{1}{T'}\left\Vert \hat{\mathbf M}^{1:T'} - \mathbf M^{1:T'} \right\Vert_2^2\text{,}
    \label{eq:recon_loss}
\end{equation}
and the commit loss with gradient pass-through is
\begin{equation}
    \mathcal L_{\textrm{commit}} = \frac{1}{S} \left\Vert\mathbf{Z}^{\textrm{motion}} - \mathbf{B}^{\textrm{motion}}\right\Vert_2^2\text{.}
    \label{eq:commit_loss}
\end{equation}
\begin{figure}[t]
  \centering
  \includegraphics[width=\linewidth]{figures/imu_tokenization.png}
  \vspace{-15px}
  \captionof{figure}{\textbf{IMU Tokenizing Process.} The rotation, acceleration, and angular velocity components of the IMU signal are first flattened and concatenated. The resulting sequence is then processed by an encoder comprising multiple 1D convolutional layers and subsequently passed through a quantizer to generate the jitter reduced inertial tokens.}
\label{fig:imu_tokenizer}
\vspace{-10px}
\end{figure}
Additionally, to improve the fidelity of reconstructed motion and eliminate the foot-ground sliding artifacts, we follow HuMoR \cite{rempe2021humor} to constrain the foot-ground interactions using
\begin{equation}
    \mathcal{L}_{\text{foot}} = \lambda_{\textrm{contact}}\mathcal L_{\textrm{contact}} + \lambda_{\textrm{slide}}\mathcal L_{\textrm{slide}}\text{,}
\end{equation}
where the contact discrimination loss is
\begin{footnotesize}
\begin{equation}
    \mathcal L_{\textrm{contact}} = \frac{1}{T'}\sum_{i\in\left\{1,2,3,4\right\}} \left[-\mathbf{p}_i\log\hat{\mathbf{p}}_i-\left(1-\mathbf{p}_i\right)\log\left(1-\hat{\mathbf{p}}_i\right)\right] \text{,}
    \label{eq:contat_loss}
\end{equation}
\end{footnotesize}
and the sliding penalty is
\begin{equation}
    \mathcal L_{\textrm{slide}} = \frac{1}{T'}\sum_{i\in\{1,2,3,4\}}\hat{\mathbf{p}}_i\left\Vert \mathbf j^v_{\operatorname{foot}\left(i\right)}\right\Vert_2^2\text{.}
    \label{eq:velocity_loss}
\end{equation}
Overall, the total training loss of our motion VQ-VAE is
\begin{equation}
    \mathcal{L}_{\text{motion}} = \mathcal{L}_\text{vq} + \mathcal{L}_\text{foot}\text{.}
\end{equation}
For following distribution matching, we maintain the frequency distribution of the motion codebook within each training batch
\begin{footnotesize}
\begin{equation}
  \mathbf{F}^{\textrm{motion}} = \frac{1}{S}\pi\left(\sum_{s=1}^S \mathcal{G}\left(\left\{-\left\Vert\mathbf{z}_s^\textrm{motion}-\mathbf{c}_k^\textrm{motion}\right\Vert_2^2\right\}_{k=1}^K\right)\right)\text{,}
\label{eq:gumbel_softmax}
\end{equation}
\end{footnotesize}
where $\mathcal{G}(\cdot):\mathbb{R}^{K}\mapsto\mathbb{R}^{K}$ is the differentiable sampling procedure using Gumbel-Softmax trick \cite{jang2016categorical}, and $\pi(\cdot)$ operates sorting on the token frequencies in descending order.

\subsection{IMU Tokenizer}
\label{sec:imu_vq}
In this subsection, we introduce the jitter-reduced and motion-aware IMU tokenizer. To facilitate the integration of continuous inertial signals with natural language in a manner compatible with large language models (LLMs), we propose a novel approach that encodes inertial signals into discrete tokens, as shown in Fig.~\ref{fig:imu_tokenizer}. These tokens are designed to align seamlessly with the LLM vocabulary, enabling direct incorporation into the language modeling framework. Meanwhile, to empower inertial tokens with the capability of reproducing high-quality 3D motion, we devise a novel distribution matching strategy to approximate the corresponding motion latent space. Therefore, the learned IMU codebook can be utilized for motion reconstruction and analysis.

\vspace{-4mm}
\paragraph{Inertia Representation} In prior works \cite{huang2018DIP,TransPoseSIGGRAPH2021,TIP22,PIPCVPR2022,yi2024pnp}, inertial signals are considered as the composition of orientation and linear acceleration. However, to fully utilize the sensor measurements of the accelerometer, gyroscope, and magnetometer, we represent an inertia sequence as follows
\begin{equation}
    \mathbf I^{1:T} = [\mathbf{q} \quad \mathbf{a} \quad \bm{\omega}] \in\mathbb R^{T\times d_u}\text{,}
\label{eq:inertia_representation}
\end{equation}
which includes orientation $\mathbf q\in\mathbb R^{T\times 6N}$, free acceleration $\mathbf a\in\mathbb R^{T\times 3N}$ and angular velocity $\bm{\omega} \in \mathbb{R}^{T\times 3N}$. In this work, we utilize a configuration of $N=6$ IMU sensors. The collected inertial data is represented in the feature space with a dimensionality of $d_u=72$, capturing comprehensive motion characteristics.

\vspace{-4mm}
\paragraph{Data Pre-processing} Due to the scarcity of MoCap data paired with real IMU readings \cite{huang2018DIP,trumble2017total,dai2024hmd}, we simulate synthetic IMU signals on extensive motion data \cite{huang2018DIP,TransPoseSIGGRAPH2021}. To model the characteristics of IMU sensors, such as data drift, we follow PNP \cite{yi2024pnp} to use random walk variables to mimic cumulative error. Since acceleration data can fluctuate violently within a wide range, we normalize it to a standard normal distribution using the mean and variance determined on the training dataset. This preprocessing procedure mitigates the impact of high-frequency noise spikes and irregular waves while preserving the drifting feature, improving the learning stability of the tokenizer.

\vspace{-4mm}
\paragraph{Training of IMU Tokenizer} Given an inertia sequence $\mathbf{I}^{1:T}$, we learn to construct a codebook $\mathbf{C}^{\text{imu}} \in \mathbb{R}^{K\times d_z}$. To be specific, each codebook entry $\mathbf{c}_k^{\text{imu}}$ is updated through exponential moving average (EMA) according to \cite{razavi2019generating}
\begin{align}
    \mathbf{c}_k^{\textrm{imu}} &\leftarrow \frac{\bm{\sigma}_k}{\delta_k} \notag\\
    \bm{\sigma}_k &\leftarrow \gamma\bm{\sigma}_k + \left(1-\gamma\right)\sum_{s=1}^S
    \mathbbm{1}\left(\mathbf{b}_s^{\textrm{imu}}=\mathbf{c}_k^{\textrm{imu}}\right)\mathbf{z}_s^{\textrm{imu}} \notag\\
    \delta_k &\leftarrow \gamma\delta_k + \left(1-\gamma\right)\sum_{s=1}^S\mathbbm{1}\left(\mathbf{b}_s^{\textrm{imu}}=\mathbf{c}_k^{\textrm{imu}}\right)\text{,}
\label{eq:codebook_update}
\end{align}
where the summation $\sum_{s=1}^S\mathbbm{1}\left(\mathbf{b}_s^{\textrm{imu}}=\mathbf{c}_k^{\textrm{imu}}\right)$ records the count that $\mathbf{c}_k^{\textrm{imu}}$ is selected. Similar to Eq.\ref{eq:gumbel_softmax}, we also maintain the frequency distribution of IMU codebook $\mathbf{F}^{\text{imu}}$ within each training batch. To inject motion dynamics and inductive bias of natural language into inertial tokens, we propose to learn by unsupervised distribution matching, inspired by CM \cite{starke2024categorical}. Specifically, the training objective of our motion-aware IMU tokenizer is
\begin{equation}
    \mathcal{L}_{\text{imu}} = \lambda_{\text{code}}\mathcal{L}_{\text{code}} + \lambda_{\text{dist}}\mathcal{L}_{\text{dist}}\text{,}
\label{eq:loss_imu_tokenizer}
\end{equation}
where the code matching loss enforces the quantized token from the IMU tokenizer close to that from the motion tokenizer
\begin{equation}
    \mathcal L_{\text{code}} = \frac{1}{S}\left\Vert \mathbf{B}^{\text{imu}}-\mathbf{B}^{\text{motion}}\right\Vert_2^2\text{.}
\label{eq:token_match_mse}
\end{equation}
We also incorporate Zipf's law \cite{zipf2013psycho,piantadosi2014zipf}, a principle about the word frequency distribution in natural language, as a regularization term to enhance the linguistic properties of the inertial tokens \cite{qu2024llms,papadimitriou2023pretrain}. Formally, the Zipfian distribution $\mathbf{F}^{\text{zipf}}$ is defined as
\begin{equation}
    \mathbf{F}^{\text{zipf}} \propto \left\{\frac{1}{(k+\beta)^\alpha} \;\middle|\; k\in\left\{1\dots K\right\}\right\}\text{,}
\label{eq:zipf_law}
\end{equation}
where $\alpha\approx 1$, $\beta\approx2.7$, and the distribution matching loss tries to minimize the Jensen-Shannon (JS) divergence between the categorical frequency distribution of IMU and motion codebook
\begin{small}
\begin{equation}
    \mathcal L_{\text{dist}} = \operatorname{JS}\left({\mathbf{F}^\text{imu}} \;\middle|\middle|\; {\mathbf{F}^{\text{motion}}}\right) + \lambda_{\text{zipf}}\operatorname{JS}\left({\mathbf{F}^\text{motion}} \;\middle|\middle|\; {\mathbf{F}^{\text{zipf}}}\right)\text{.}
\label{eq:distribution_matching}
\end{equation}
\end{small}

\subsection{Implementation Details}
\label{sec:implementation_detail_part1}
To accommodate the high frame rates typical of IMU sensors, we standardize motion data across various datasets to 50 and 60 frames per second (fps). Consequently, we adopt a codebook size of $K=1024$, which exceeds that used in MotionGPT \cite{jiang2023motiongpt}, and trained on 20 fps data to better capture the increased temporal resolution. We observed that higher compression rates $l$ can introduce square-wave-like artifacts in the encoded IMU signals in our experiment. To address this, we set $l=4$, achieving a balanced trade-off between the compactness and the expressiveness of the discrete token representation. During training, the EMA coefficient is $\gamma=0.99$, and the loss weights are configured as: $\lambda_{\textrm{recon}}=1.0$, $\lambda_{\textrm{commit}}=0.02$, $\lambda_{\textrm{contact}}=0.01$, $\lambda_{\textrm{slide}}=0.01$, $\lambda_{\textrm{dist}}=1.0$, $\lambda_{\textrm{code}}=1.0$, and $\lambda_{\textrm{zipf}}=0.2$.
We utilize the AdamW optimizer \cite{loshchilov2017decoupled} with learning rate $\textrm{lr}=2\times10^{-4}$ and cosine annealing scheduler \cite{loshchilov2016sgdr}. The training batch size is set to $512$ for both motion and IMU tokenizer.

\section{Language Model with Inertial Tokens}
\label{sec:method2}

% With the aforementioned IMU tokenizer, our method discretizes continuous and jittery IMU signals to sequential jitter-reduced tokens. However, compared to high-dimensional embedding space of LLMs, our learned inertial tokens are situated in a compact latent space with low dimension. Therefore, it's necessary to pre-align inertial tokens with language embeddings for following multimodal understanding. In this section, we first introduce our method for preparing curated textual annotations paired with inertia and motion sequences (Sec. \ref{sec:data_preparation}). Subsequently, we present our method for projecting the inertial tokens onto the vocabulary space of Qwen2-7B-Instruct language model \cite{yang2024qwen2} in Sec.~\ref{sec:pretrain}, and our LoRA model adapters to further improve the professionalism, reasonability and stylization of our system in Sec.~\ref{sec:finetuning_lora}.

Using the aforementioned IMU tokenizer, our method discretizes continuous and jittery IMU signals into sequential jitter-reduced tokens. However, unlike the high-dimensional embedding space of LLMs, the learned inertial tokens reside in a compact, low-dimensional latent space. Consequently, it is essential to pre-align these inertial tokens with language embeddings to facilitate subsequent multimodal understanding. In this section, we first introduce our method for generating curated textual annotations paired with inertial and motion sequences. (Sec.~\ref{sec:data_preparation}). Following this, we detail our method for projecting the inertial tokens into the vocabulary space of Qwen2-7B-Instruct language model \cite{yang2024qwen2} in Sec.~\ref{sec:pretrain}, and introduce our LoRA model adapters, which enhance the system's professionalism, rationality, and stylization in Sec.~\ref{sec:finetuning_lora}

\begin{figure}[t!]
  \centering
  \includegraphics[width=\linewidth]{figures/data_preparation.png}
  \vspace{-20px}
  \captionof{figure}{\textbf{Data Generation Pipeline.} The corresponding motion label is first extracted and expanded into a descriptive sentence using the LLM. Subsequently, a prompt is employed to generate a more refined and professional description or instructional output.}
\label{fig:data}
\vspace{-10px}
\end{figure}

\subsection{Data Preparation}
\label{sec:data_preparation}
% In order to prepare extensive training data, we instruct GPT-4o-mini using specifically devised prompts to automatically rephrase the raw textual annotations in original datasets or generate interactive dialogues based on concise action labels. Given the rich diversity of human motion, as depicted in Fig.~\ref{fig:data}, we split our collected datasets into two broad categories: ``daily motion” and ``professional exercise”, and annotate with captions and instructions respectively.

To prepare extensive training data, we instruct GPT-4o-mini with carefully designed prompts to automatically rephrase raw textual annotations from original datasets or generate interactive dialogues based on concise action labels. Given the rich diversity of human motion, as illustrated in Fig.~\ref{fig:data}, we categorize the collected datasets into two broad groups: ``daily motion'' and ``professional exercise'', and annotate them with descriptions and instructions respectively.

\vspace{-4mm}
\paragraph{Descriptive and Instructive Text Data Generation} 
% For daily motion, we induce GPT-4o-mini to output responses composed of two structural parts: objective description of the given action, and subjective reasoning of the motivation or intention of the acting human. For instructive split, we ask for detailed motion analysis, accompanied with assessment and instruction, including encouragements or critiques with clear attitudes. To ensure the human-like feedback, we enforce the generated texts to be in second-person narrative style.

For the daily motion category, we prompt GPT-4o-mini to generate responses structured into two parts: an objective description of the given action and a subjective analysis of the motivation or intention behind the human actor's behavior. For the instructive category, we request detailed motion analysis, supplemented with assessments and instructions, including encouragements or critiques expressed with clear attitudes. To ensure the feedback resembles human-like communication, we constrain the generated texts to adopt a second-person narrative style.

\vspace{-4mm}
\paragraph{Stylized Feedback Generation} 
% To further diversify and enrich the dataset, we introduce stylized roles that incorporate distinct personalities and tones. We first characterize each role by specifying its personality, typical wording, and overarching stylistic features. Next, we either select an existing role utterance, or virtually construct one as an example. This yields vivid, character-driven dialogues that encompass various styles for our system interactions.

To further diversify and enrich the dataset, we introduce stylized roles that incorporate distinct personalities and linguistic tones. First, we define each role by specifying its personality traits, typical phrasing and overarching stylistic characteristics. Subsequently, we either select an existing role-specific utterance or virtually construct one as an exemplar. This approach generates vivid, character-driven dialogues that encompass a wide range of stylistic variations, enhancing the diversity of interactions within our system.

\begin{figure}[t!]
  \centering
  \includegraphics[width=\linewidth]{figures/inference.png}
  \vspace{-20px}
  % \captionof{figure}{\textbf{Inference Pipeline.} Jittery IMU signals are initially tokenized into jitter-reduced inertial tokens, which are simultaneously decoded by the learned motion decoder to reconstruct human motion, and mapped to the language semantic space via the pretrained projection module for motion analysis.}
  \captionof{figure}{\textbf{Inference Pipeline.}
    Jittery IMU signals are first tokenized into jitter-reduced inertial tokens. These tokens are concurrently processed in two ways: (1) they are decoded by the learned motion decoder to reconstruct human motion, and (2) they are projected into the language semantic space via the pretrained projection module for motion analysis.  
  }
\label{fig:inference_pipeline}
\vspace{-10px}
\end{figure}


\subsection{Projecting Inertial Token to Text Embedding}
\label{sec:pretrain}
% Seamless translating and understanding between IMU and textual modalities require a shared and well-aligned embedding space. However, it is challenging to achieve this since the text embedding dimension of well-known LLMs is usually high, for example the Qwen2-7B-Instruct~\cite{yang2024qwen2} defines its embedding dimension as $d_h=3584$, which is far higher than the dimension $d_z=512$ of our inertial tokens. To resolve this challenge, we draw inspiration from OneLLM~\cite{han2024onellm} by introducing a projection module $\mathcal{P}_{\theta}$, which projects inertial tokens $\mathbf{b}_s^{\textrm{imu}}$ onto the text embedding space. 

Seamless translation and understanding between IMU and textual modalities necessitate a shared and well-aligned embedding space. However, achieving this is challenging due to the significant disparity in embedding dimensions between well-known LLMs and inertial tokens. For instance, the Qwen2-7B-Instruct~\cite{yang2024qwen2} employs a text embedding dimension $d_h=3584$, which is substantially larger than $d_z=512$ dimension of our inertial tokens. To address this issue, we adopt a strategy inspired by OneLLM~\cite{han2024onellm}, introducing a projection module $\mathcal{P}_{\theta}$ that maps inertial tokens $\mathbf{b}_s^{\textrm{imu}}$ into the text embedding space.


\vspace{-4mm}
\paragraph{Training of Projection Module} 
% As illustrated in Fig.~\ref{fig:train}, our projection module consists of 8 transformer blocks followed by a linear layer. In each transformer block, we follow Llama3~\cite{dubey2024llama} to incorporate self-attention layer, feed-forward network, and skip connections to ensure gradient flow. For each inertial token $\mathbf{b}_s^{\textrm{imu}}$, our projection module maps it onto the text embedding space of Qwen2-7B-Instruct~\cite{yang2024qwen2}
% \[
% \mathbf{e}_s = \mathcal{P}_{\theta}(\mathbf{b}_s^{\textrm{imu}}) \in \mathbb{R}^{d_h}\text{.}
% \]
% Simultaneously, the optional user textual prompts are tokenized and embedded into the identical embedding space, which are subsequently concatenated with projected IMU tokens, formulating the input of our language model. In current stage, we freeze the Qwen2-7B-Instruct~\cite{yang2024qwen2} model and only train our projection layers $\mathcal{P}_{\theta}$ using the cross-entropy loss. In order to build the semantic association between inertial tokens and texts, rather than the inner causality of inertial token sequence, we apply the training strategy of chatting language models, which augments mask to all the input tokens and excludes them from the training loss computation.

As illustrated in Fig.~\ref{fig:train}, our projection module comprises eight transformer blocks followed by a linear layer. Each transformer block incorporates a self-attention layer, a feed forward network, and skip connections, following the architecture of Llama3~\cite{dubey2024llama}, to ensure effective gradient flow. For each inertial token $\mathbf{b}_s^{\textrm{imu}}$, the projection module maps it to the text embedding space of Qwen2-7B-Instruct~\cite{yang2024qwen2}:
\[
\mathbf{e}_s = \mathcal{P}_{\theta}(\mathbf{b}_s^{\textrm{imu}}) \in \mathbb{R}^{d_h}\text{.}
\]
Concurrently, optional user-provided textual prompts are tokenized and embedded into the same space, after which they are concatenated with projected inertial tokens to form the input for the language model. At this stage, we keep the Qwen2-7B-Instruct~\cite{yang2024qwen2} model frozen and train only the projection layer $\mathcal{P}_{\theta}$ using cross-entropy loss. To establish semantic associations between inertial tokens and text, rather than focusing on the intrinsic causality within the inertial token sequence, we employ a training strategy inspired by chatting language models. This involves augmenting mask to all input tokens and excluding them from the loss computation during training.

\subsection{Fine-tuning Language Model Adapters}
\label{sec:finetuning_lora}

% To further empower our language model with more flexibilities and customization capabilities, we finetune 4 Low-Rank Adaptation (LoRA)~\cite{hu2022lora} adapters, which enable stylized feedback with character-specific tones in customized responsive roles. During fine-tuning, we freeze both the projection module $\mathcal{P}_{\theta}$ and the pretrained weights of Qwen2-7B-Instruct~\cite{yang2024qwen2} language model, solely adjusting the LoRA adapters. Our finetuning process provides our model diverse linguistic understanding on same inertial sequences, enabling personalized and flexible usage.

To further empower our language model with greater flexibility and customization capabilities, we finetune 4 Low-Rank Adaptation (LoRA)~\cite{hu2022lora} adapters. These adapters enable the generation of stylized feedback with character-specific tones, allowing for tailored responses in customized roles. During fine-tuning, both the projection module $\mathcal{P}_{\theta}$ and the pretrained weights of the Qwen2-7B-Instruct~\cite{yang2024qwen2} language model remain frozen, with updates applied exclusively to the LoRA adapters. This fine-tuning process enriches the model's linguistic understanding of the same inertial sequences, facilitating personalized and adaptable usage scenarios.

\section{Experiments}

\subsection{Datasets}

\textbf{MSMARCO}.
We utilized the MS MARCO Passage Ranking dataset as the data source to evaluate the ability of our method to improve document rankings under more challenging topic-query tasks. Specifically, we assessed whether our method could significantly enhance the ranking of documents by the retrieval model within a RAG system.

To construct topic-lists for evaluation, we applied a K-means clustering algorithm to group similar queries, forming topics that each contained a series of related queries. To further evaluate the performance of our method under extreme topic-query scenarios, we applied an intra-topic similarity filtering process. Only topics with queries exhibiting high semantic diversity and containing a sufficient number of queries were retained.

This process resulted in 29 topics, with each topic containing an average of 22.28 queries. The average similarity score within each topic was approximately 0.5, indicating sufficient diversity among queries to ensure a rigorous evaluation. This curated dataset enabled us to test the robustness of our method in handling highly diverse and challenging topic-query tasks within a RAG system.

\textbf{PROCON}.
To conduct our experiments, we utilized controversial topic data scraped from the PROCON.ORG website. This dataset includes over 80 topics covering various domains, such as society, health, government, and education. Each topic is discussed from two stance labels \{\textit{PRO (support), CON (oppose)}\}, with passages arguing from these perspectives.

To simulate real-world user interactions with a RAG system, we instructed a large language model (GPT-4o) to act as a user and generate 40 potential sub-queries for each topic. These sub-queries were designed to reflect the diverse questions and concerns users might raise when exploring a specific controversial topic. 

After generating the sub-queries, we applied a similarity filtering process to ensure diversity by retaining only those with a similarity score below approximately 0.85. The filtering step effectively removed redundant queries while preserving a wide range of perspectives. As a result, the final set of topic-queries achieved an average similarity score of approximately 0.7, ensuring that the queries were sufficiently diverse yet semantically relevant. This process provided a robust and balanced sub-queries set for evaluation.


\subsection{Experiment Details}
The specific setting details for the Topic-queries RAG manipulation experiment are as follows:

(1) Black-box RAG. We represent the black-box RAG process as \( \text{RAG}_{\text{black}} \). The RAG framework is Conversational RAG from LangChain. The LLMs adopted in RAG are the open-source models Meta-Llama-3.1-8B-Instruct (Llama3.1), Qwen-2.5-7B-Instruct(Qwen2.5). The system prompt and additional detailed descriptions are provided in Appendix~\ref{exp-detail}.

(2) Retrieval Model Specification. We benchmark three dominant dense retrievers—Contriever \cite{gao2021unsupervised}, DPR \cite{karpukhin-etal-2020-dense}, and ANCE \cite{xiong2020approximate}.By convention, we use dot product between the embedding vectors of questions(queries) and candidate documents as their similarity score \(R\) in the ranking. 


\label{opinion-classfication}
(3) Opinion classification. We use Qwen2.5-Instruct-72B as the opinion classifier. Qwen2.5-Instruct-72B, due to its large parameter size, is capable of accurately identifying and classifying opinions within text.

(4) Experimental parameters. In knowledge-guided attack process, we set the maximum editing distance $\epsilon$ to 0.2, the semantic similarity threshold $\lambda$ to 0.85, and the number of iterations $N$ to 5. For adversarial trigger generation, we used a beam size of 3, top-$k$ of 10, a batch size of 32, a temperature of 1.0, a learning rate of 0.005, and a sequence length of 10. In RAG\textsubscript{black}, $k$ (the number of retrieved documents) is set to 3, with the LLM temperature also fixed at 1.0 to mirror real-world conditions.

(5) Poisoned Target. For the PROCON dataset, to investigate the manipulation performance under more challenging conditions, we performed relevance ranking for the documents with respect to each topic-query set $Q$ and the target stance $S_t$ . From the ranked list, we selected the last five documents (i.e., those with the lowest relevance) as the target poisoned documents.
For the MS MARCO dataset, we utilized the top-1000 relevance-ranked passage list for each topic-query set. From this list, we selected the passage with the lowest average rank as the target passage. This approach ensures that the evaluation focuses on passages that are least relevant to the target queries, thus providing a more rigorous benchmark for the proposed method.

(6) Experimental environment. All our experiments are conducted in Python 3.8 environment and run on a NVIDIA DGX H100 GPU. 

\subsection{Research Questions}

We propose four research questions to evaluate the effectiveness of our method in the topic-queries task, focusing on black-box NRM attacks and opinion manipulation to RAGs.

\textbf{RQ1}: Can Topic-FlipRAG significantly enhance the rankings of target documents in the RAG retriever for topic-queries?

\textbf{RQ2}: To what extent does Topic-FlipRAG affect the answers generated by the target RAG systems?

\textbf{RQ3}: Does topic-oriented opinion manipulation significantly impact users' perceptions of controversial topics?

\textbf{RQ4}: How robust does Topic-FlipRAG against exisiting mitigation mechanism?

\subsection{Baseline Settings}
To assess the effectiveness of our proposed method, we compare it against adversarial attack baselines designed for black-box, topic-oriented RAG scenarios, ensuring minimal modifications to the original documents. We exclude BadRAG\cite{xue2024badrag}, a backdoor RAG attack limited to white-box scenarios, and topic-IR-attack\cite{liu2023topic}, as its incomplete implementation prevents reliable replication.
For the selected baseline methods, we adapted them to meet the requirements of our task while preserving their core components. A brief overview of the baseline methods is provided below, with detailed descriptions available in Appendix~\ref{baselines-details}.

\textbf{PoisonedRAG.}
Zou et al.\cite{zou2024poisonedrag} propose an approach adaptable to both black-box and white-box settings. For our task, we employ its black-box strategy by inserting a randomly chosen query from the topic-queries set \( Q \) at the beginning of each document.

\textbf{PAT.}
This gradient-based adversarial retrieval attack uses a pairwise loss function to generate triggers that meet fluency and coherence constraints. We adapt PAT to produce triggers \( T_{\text{pat}} \) for target documents within the topic-queries set, evaluating their effectiveness under black-box conditions.


\textbf{Collision.}
This method generates adversarial paragraphs (collisions) via gradient-based optimization to produce content semantically aligned with the target query. In a topic-queries context, we create collisions for the entire topic-queries set and examine their transfer performance on black-box RAG retrievers.

These baseline methods provide benchmarks for comparing the efficacy of our approach in a fully black-box, topic-oriented RAG attack scenario.

\subsection{Evaluation Metrics}

For \textbf{RQ1}, we focus on ranking manipulation. We measure the average proportion of target opinions in top-3 rankings before and after manipulation (\(\text{Top3}_{\text{ori}}, \text{Top3}_{\text{att}}\)) and define top3-v as their difference. We also employ the Ranking Attack Success Rate (RASR), reflecting how often target documents are successfully boosted, and Boost Rank (BRank), denoting the average rank improvement for all target documents. Lastly, we report the proportion of target documents in the Top-50 and Top-500 positions to indicate how effectively they are pushed toward higher rankings.

\textbf{top3-v.} Computed by subtracting \(\text{Top3}_{\text{ori}}\) from \(\text{Top3}_{\text{att}}\), top3-v ranges from -1 to 1. A positive value signifies a successful increase of the target opinion in top-3 results, while a negative value indicates a detrimental effect.

\textbf{Ranking Attack Success Rate (RASR).} RASR captures how frequently target documents are successfully boosted in each query’s ranking. Higher values indicate greater attack effectiveness.

\textbf{Boost Rank (BRank).} BRank is the average rank improvement for all target documents under each query. A target document contributes negatively if its rank is unintentionally lowered.

\textbf{Top-50, Top-500.} These metrics represent the percentage of target documents that move into specific ranking thresholds in the MS MARCO Dataset after manipulation. Higher percentages imply more effective promotion of target documents. 


For \textbf{RQ2}, we employ Average Stance Variation (ASV) to assess how significantly our opinion manipulation influences the LLM’s responses in a black-box RAG. To address the natural variability of controversial topics and the inherent instability of large language models, we also propose Real Adjusted ASV (\(\Delta\)-ASV).

\textbf{Average Stance Variation (ASV).}
ASV is defined as the absolute difference between the manipulated opinion score and the original opinion score assigned to an LLM response (0 = opposing, 1 = neutral, 2 = supporting). A higher ASV signifies a more pronounced shift in polarity and hence greater manipulation effectiveness.

\textbf{Real Adjusted ASV ($\Delta$-ASV)}. To account for the inherent variability of controversial topics and the instability of large language models, we measure the baseline ASV in a clean state, denoted as ASV\textsubscript{clean} (calculated without adversarial manipulation). $\Delta$-ASV is then obtained by subtracting ASV\textsubscript{clean} from the manipulated ASV, i.e., \( \text{$\Delta$-ASV} = \text{ASV} - \text{ASV\textsubscript{clean}} \). This adjustment ensures that $\Delta$-ASV reflects the true impact of adversarial manipulation by eliminating the influence of natural stance variation. It reflects the extent to which the polarity of the RAG-system outputs is affected by the manipulation.  A positive $\Delta$-ASV indicates a significant shift in the opinion polarity due to manipulation, with larger values representing greater manipulation effectiveness.

\section{Conclusion }
This paper introduces the Latent Radiance Field (LRF), which to our knowledge, is the first work to construct radiance field representations directly in the 2D latent space for 3D reconstruction. We present a novel framework for incorporating 3D awareness into 2D representation learning, featuring a correspondence-aware autoencoding method and a VAE-Radiance Field (VAE-RF) alignment strategy to bridge the domain gap between the 2D latent space and the natural 3D space, thereby significantly enhancing the visual quality of our LRF.
Future work will focus on incorporating our method with more compact 3D representations, efficient NVS, few-shot NVS in latent space, as well as exploring its application with potential 3D latent diffusion models.


{\small
\bibliographystyle{ieee_fullname}
\bibliography{main}
}

\end{document}
