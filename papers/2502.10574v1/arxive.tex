%%%%%%%% ICML 2024 EXAMPLE LATEX SUBMISSION FILE %%%%%%%%%%%%%%%%%

\documentclass{article}

% Recommended, but optional, packages for figures and better typesetting:
\usepackage{microtype}
\usepackage{graphicx}
\usepackage{subfigure}
\usepackage{booktabs} % for professional tables

% hyperref makes hyperlinks in the resulting PDF.
% If your build breaks (sometimes temporarily if a hyperlink spans a page)
% please comment out the following usepackage line and replace
% \usepackage{icml2024} with \usepackage[nohyperref]{icml2024} above.
\usepackage{hyperref}


% Attempt to make hyperref and algorithmic work together better:

% \usepackage{algorithm} 
% \usepackage{algorithm}
%\usepackage{algpseudocode}

%\newcommand{\theHalgorithm}{\arabic{algorithm}}

% Use the following line for the initial blind version submitted for review:
% \usepackage{icml2024}

% If accepted, instead use the following line for the camera-ready submission:
\usepackage[accepted]{icml2025}
% \usepackage{icml2025}
% For theorems and such
\usepackage{amsmath}
\usepackage{amssymb}
\usepackage{mathtools}
\usepackage{amsthm}


% \definecolor{cvprblue}{rgb}{0.21,0.49,0.74}
% \usepackage[pagebackref,breaklinks,colorlinks,allcolors=cvprblue]{hyperref}


% \usepackage{cancel}
% \usepackage{makecell}


\usepackage{multirow}

\def\e{\varepsilon{}}
\def\N{\mathcal{N}}
\def\E{\mathbb{E}}
\def\cost{\mathcal{L}}

\def\R{\mathbb{R}}

\newcommand\jacek[1]{{\color{blue} \bf #1}}
\newcommand\red[1]{{\color{red} [ PS: #1]}}
\newcommand\greenc{\color[HTML]{1469D9}}

\definecolor{ourgreen}{RGB}{20, 165, 105}

% \def\our{AdaScale}

\def\our{$\beta$-CFG}

% if you use cleveref..
\usepackage[capitalize,noabbrev]{cleveref}

%%%%%%%%%%%%%%%%%%%%%%%%%%%%%%%%
% THEOREMS
%%%%%%%%%%%%%%%%%%%%%%%%%%%%%%%%
\theoremstyle{plain}
\newtheorem{theorem}{Theorem}[section]
\newtheorem{proposition}[theorem]{Proposition}
\newtheorem{lemma}[theorem]{Lemma}
\newtheorem{corollary}[theorem]{Corollary}
\theoremstyle{definition}
\newtheorem{definition}[theorem]{Definition}
\newtheorem{assumption}[theorem]{Assumption}
\theoremstyle{remark}
\newtheorem{remark}[theorem]{Remark}

% Todonotes is useful during development; simply uncomment the next line
%    and comment out the line below the next line to turn off comments
%\usepackage[disable,textsize=tiny]{todonotes}
\usepackage[textsize=tiny]{todonotes}


% The \icmltitle you define below is probably too long as a header.
% Therefore, a short form for the running title is supplied here:
\icmltitlerunning{Classifier-free Guidance with Adaptive Scaling}

\begin{document}

\twocolumn[
\icmltitle{Classifier-free Guidance with Adaptive Scaling}

% It is OKAY to include author information, even for blind
% submissions: the style file will automatically remove it for you
% unless you've provided the [accepted] option to the icml2024
% package.

% List of affiliations: The first argument should be a (short)
% identifier you will use later to specify author affiliations
% Academic affiliations should list Department, University, City, Region, Country
% Industry affiliations should list Company, City, Region, Country

% You can specify symbols, otherwise they are numbered in order.
% Ideally, you should not use this facility. Affiliations will be numbered
% in order of appearance and this is the preferred way.
\icmlsetsymbol{equal}{*}

\begin{icmlauthorlist}
\icmlauthor{Dawid Malarz}{equal,yyy}
\icmlauthor{Artur Kasymov}{equal,yyy}
\icmlauthor{Maciej Zieba}{comp}
\icmlauthor{ Jacek Tabor}{yyy}
\icmlauthor{Przemys\l{}aw Spurek}{yyy}
%\icmlauthor{}{sch}
% \icmlauthor{Firstname8 Lastname8}{sch}
% \icmlauthor{Firstname8 Lastname8}{yyy,comp}
%\icmlauthor{}{sch}
%\icmlauthor{}{sch}
\end{icmlauthorlist}

\icmlaffiliation{yyy}{Jagiellonian University}
\icmlaffiliation{comp}{ University of Science and Technology Wrocław}
% \icmlaffiliation{sch}{School of ZZZ, Institute of WWW, Location, Country}

\icmlcorrespondingauthor{Przemys\l{}aw Spurek}{przemyslaw.spurek@uj.edu.pl}

% You may provide any keywords that you
% find helpful for describing your paper; these are used to populate
% the "keywords" metadata in the PDF but will not be shown in the document
\icmlkeywords{Machine Learning, ICML}

\vskip 0.3in
]

% this must go after the closing bracket ] following \twocolumn[ ...

% This command actually creates the footnote in the first column
% listing the affiliations and the copyright notice.
% The command takes one argument, which is text to display at the start of the footnote.
% The \icmlEqualContribution command is standard text for equal contribution.
% Remove it (just {}) if you do not need this facility.

%\printAffiliationsAndNotice{}  % leave blank if no need to mention equal contribution
\printAffiliationsAndNotice{\icmlEqualContribution} % otherwise use the standard text.

% The other important mechanism one can observe, is that CFG behaves differently on high, medium, and low noise levels.
% In the initial phase, substantial guidance restricts sampling to a few average (template) images. The middle stage is crucial, where guidance modifies important high-order features. In the end, we only denoise images.
\begin{abstract}
Classifier-free guidance (CFG) is an essential mechanism in contemporary text-driven diffusion models. In practice, in controlling the impact of guidance we can see the trade-off between the quality of the generated images and correspondence to the prompt. When we use strong guidance, generated images fit the conditioned text perfectly but at the cost of their quality. Dually, we can use small guidance to generate high-quality results, but the generated images do not suit our prompt. 
In this paper, we present \our{} ($\beta$-adaptive scaling in Classifier-Free Guidance), which controls the impact of guidance during generation to solve the above trade-off. First, \our{} stabilizes the effects of guiding by gradient-based adaptive normalization. Second, \our{} uses the family of single-modal ($\beta$-distribution), time-dependent curves to dynamically adapt the trade-off between prompt matching and the quality of samples during the diffusion denoising process. Our model obtained better FID scores, maintaining the text-to-image CLIP similarity scores at a level similar to that of the reference CFG.
\end{abstract}

\begin{figure}[h]
    \centering
        % \includegraphics[width=\linewidth]{img/synth/g/cfg4.0_comp.png}

    \begin{tabular}{cc}
        \includegraphics[width=0.43\linewidth]{img/synth/1.jpg} & 
        \includegraphics[width=0.43\linewidth]{img/synth/2.jpg} \\ 
        (a) Ground truth & (b) No guidance \\
        \includegraphics[width=0.43\linewidth]{img/synth/4.jpg} & 
        \includegraphics[width=0.43\linewidth]{img/synth/3.jpg} \\
        (c) CFG & (d) \our{}
    \end{tabular}
    
        
    \caption{
    % We present simple 2D example visualizing the behaviors of \our{}. As we can see, no guidance strategy generates data that is much out of data distribution. CFG partially solves the problem, but we do not sample from all datasets (we do not see points on the bottom right branch). \our{} is able to generate data from all branches and produce fewer points from out of distribution.
    A two-dimensional distribution featuring two classes represented by gray and orange regions. {\bf(a)} Ground truth samples from the orange class. {\bf(b)} Conditional sampling with no additional guidance techniques. {\bf(c)} Classifier-free guidance decreases sample diversity to achieve outlier removal {\bf (d) \our{}} preserves the diversity of the samples while still achieving the objective of outlier removal.
    }
    \vspace{-0.5cm}
    \label{fig:toy-example}    
\end{figure}


\section{Introduction}
\label{sec:intro}

Diffusion models \cite{dhariwal2021diffusion,rombach2022high,croitoru2023diffusion} are regarded as one of the leading techniques for image generation, especially due to their ability to be easily conditioned with text prompts. Classifier-free guidance (CFG) \citep{ho2022classifier} is a crucial component in modern diffusion models used for generating content based on text prompts. This method aims to balance diversity and consistency relative to the conditioning factor by employing a mix of constrained and unconstrained diffusion models. In practice, a trade-off \cite{kynkaanniemi2024applying,chung2024cfg++} must be made between the quality of generated elements and their alignment with the prompt. Employing strong guidance results in images that match the conditioned text but of compromised quality. Conversely, using limited guidance yields high-quality results at the expense of alignment with the prompt. 
\begin{figure}
    \centering
        \includegraphics[width=0.95\linewidth, trim=0 10 0 0 , clip]{img/images/teaser.png}
        \vspace{-0.5cm}
    \caption{  Norm values of the modification factor applied at each iteration of the classifier-free guided diffusion sampling backward process. We compare classical CFG and our solution \our{}. We model such trajectory by $\beta $-distribution and parameter $\gamma$. $\beta $-distribution gives the general trend of a diffusion process. For $\gamma=1$ we have a pure Gamma curve while by going with Gamma to zero, add local perturbation from pure CFG. Thanks to the $\beta$-distribution, we have no guidance at the beginning and at the end of trajectory.}.
    \vspace{-0.9cm}
    \label{fig:curves}
\end{figure}


Using the same guidance for every sampling step isn't optimal because CFG functions uniquely at high, medium, and low noise levels. In \cite{kynkaanniemi2024applying} the authors analyze such three phases. Strong guidance restricts sampling to a few average (template) images in the initial steps. The middle stage is crucial, where guidance modifies important high-order features. In such a part, the guidance can change the sampling trajectory without significantly losing the quality of the renders. In CFG++\cite{chung2024cfg++}, authors introduce a simple modification of CFG, which keeps the trajectory closer to the data manifold.  The last part of diffusion sampling is only denoising, and conditioning can only destroy this process~\cite{poleski2024geoguide}.






This paper introduces \our{}\footnote{The
source code is available at \url{https://github.com/gmum/beta-CFG}} ($\beta$-distribution Classifier-Free Guidance), which controls the impact of guidance during generation to solve the above trade-off. \our{} use a family of single-modal curve families ($\beta$-distribution) to model the strength of guidance. Instead of scores between a conditional and an unconditional diffusion model we normalize such value. A similar strategy was used in Classifier Guided Diffusion~\cite{poleski2024geoguide} where fixes constant classification guidance weight was used.
In Classifier-Free Guidance, we need to control the impact of conditioning dynamically, so we use the additional parametric function. Thanks to this approach, we can dynamically change the trade-off between prompt matching and sample quality. 
The single modal $\beta$-distribution allows the data manifolds to remain at the beginning and end of the sampling trajectory. 
Furthermore, we use additional $\beta$ parameters that control the middle stage of the diffusion process. 

Due to this adjustment, we can more accurately represent the data distribution. This is demonstrated in a 2D illustration; refer to Fig.~\ref{fig:toy-example}. As observed, the traditional CFG fails to draw samples from the data distribution, evidenced by the bottom right branch in Fig.~\ref{fig:toy-example}~(c). Conversely, \our{} aligns more closely with the training data distribution, avoiding outlier generation, see Fig.~\ref{fig:toy-example}~(c).

Concluding, the main contributions of the paper are the following:
\begin{itemize}
%\vspace{-0.3cm}
    \item we propose \our{} a model which solves the tartrate-of between prompt fitting and quality of generated objects
\vspace{-0.3cm}    
    \item \our{} is easy to implement and controls the norm of the guidance (see~Fig.~\ref{fig:curves}),
\vspace{-0.3cm}    
    \item \our{} surpasses the traditional CFG in terms of FID score while maintaining a constant CLIP value.  
\vspace{-0.3cm}    
\end{itemize}

\section{Related works}


\paragraph{Diffusion models}

The idea of diffusion models was first presented in \cite{sohl2015deep}. These models leverage Stochastic Differential Equations (SDEs) to progressively transform a simple initial distribution (e.g., a normal distribution) into a more complex target distribution through a series of manageable diffusion steps. The evolving advances, including the decrease in the trajectory steps~\cite{bordes2017learning}, have created more efficient diffusion models.

\begin{figure*}[!ht]
    \centering
    % \renewcommand{\arraystretch}{0} % Usunięcie odstępów między wierszami
    \setlength{\tabcolsep}{0.8pt}    % Usunięcie odstępów między kolumnami
    \centering
    \begin{tabular}{@{}c@{}c@{}c@{}c@{}c@{}c@{}}
        \rotatebox{90}{ \qquad CFG} &
        \includegraphics[width=0.19\linewidth]{img/images/2.0_3.0_9.0/100/000050/000050_ddim_9.0.jpg} & 
        \includegraphics[width=0.19\linewidth]{img/images/2.0_3.0_9.0/100/000052/000052_ddim_9.0.jpg} & 
        \includegraphics[width=0.19\linewidth]{img/images/2.0_3.0_9.0/100/000070/000070_ddim_9.0.jpg} & 
        \includegraphics[width=0.19\linewidth]{img/images/2.0_3.0_9.0/100/000084/000084_ddim_9.0.jpg} & 
        \includegraphics[width=0.19\linewidth]{img/images/2.0_3.0_9.0/100/000091/000091_ddim_9.0.jpg} 
        % \includegraphics[width=0.15\linewidth]{img/images/2.0_3.0_9.0/100/000020/000020_ddim_9.0.jpg} 
        \\
        \rotatebox{90}{ \qquad \our{}} &
        \includegraphics[width=0.19\linewidth]{img/images/2.0_3.0_9.0/100/000050/2.0_3.0_ddim_beta_9.0.jpg} & 
        \includegraphics[width=0.19\linewidth]{img/images/2.0_3.0_9.0/100/000052/2.0_3.0_ddim_beta_9.0.jpg} & 
        \includegraphics[width=0.19\linewidth]{img/images/2.0_3.0_9.0/100/000070/2.0_3.0_ddim_beta_9.0.jpg} &  
        \includegraphics[width=0.19\linewidth]{img/images/2.0_3.0_9.0/100/000084/2.0_3.0_ddim_beta_9.0.jpg} & 
        \includegraphics[width=0.19\linewidth]{img/images/2.0_3.0_9.0/100/000091/2.0_3.0_ddim_beta_9.0.jpg}  
        % \includegraphics[width=0.15\linewidth]{img/images/2.0_3.0_9.0/100/000020/2.0_3.0_ddim_beta_9.0.jpg} 
        \\[0.08cm]
        &
        \multicolumn{1}{p{0.16\linewidth}}{\centering \small  A traffic light sitting next to a building with its red blurred.
        }& 
        \multicolumn{1}{p{0.16\linewidth}}{\centering \small 
        There are bananas, pineapples, oranges, sandwiches, and drinks at the stand.
        }& 
        \multicolumn{1}{p{0.16\linewidth}}{\centering \small 
        a green train is coming down the tracks
        }& 
        \multicolumn{1}{p{0.16\linewidth}}{\centering \small 
        Small birds are walking along the waters edge
        }& 
        \multicolumn{1}{p{0.16\linewidth}}{\centering \small 
        A boy is jumping a hurdle while on a skateboard.
        }
        
        % \multicolumn{1}{p{0.16\linewidth}}{\centering \small A snowboarder hitting a trick off a giant ramp. 
        % }
        \\[1.2cm]
        \rotatebox{90}{ \qquad CFG} &
        \includegraphics[width=0.19\linewidth]{img/images/2.0_2.5_9.0/custom/000001/000001_ddim_9.0.jpg} & 
        \includegraphics[width=0.19\linewidth]{img/images/2.0_2.5_9.0/custom/000002/000002_ddim_9.0.jpg} & 
        \includegraphics[width=0.19\linewidth]{img/images/2.0_2.5_9.0/custom/000006/000006_ddim_9.0.jpg} & 
        \includegraphics[width=0.19\linewidth]{img/images/2.0_2.5_9.0/custom/000008/000008_ddim_9.0.jpg} & 
        \includegraphics[width=0.19\linewidth]{img/images/2.0_3.0_9.0/100/000020/000020_ddim_9.0.jpg} 
        % \includegraphics[width=0.15\linewidth]{img/images/2.0_2.5_9.0/custom/000010/000010_ddim_9.0.jpg} &
        % \includegraphics[width=0.15\linewidth]{img/images/2.0_2.5_9.0/custom/000012/000012_ddim_9.0.jpg} 
        \\
        \rotatebox{90}{ \qquad \our{}} &
        \includegraphics[width=0.19\linewidth]{img/images/2.0_2.5_9.0/custom/000001/2.0_2.5_ddim_beta_9.0.jpg} & 
        \includegraphics[width=0.19\linewidth]{img/images/2.0_2.5_9.0/custom/000002/2.0_2.5_ddim_beta_9.0.jpg} & 
        \includegraphics[width=0.19\linewidth]{img/images/2.0_2.5_9.0/custom/000006/2.0_2.5_ddim_beta_9.0.jpg} &  
        \includegraphics[width=0.19\linewidth]{img/images/2.0_2.5_9.0/custom/000008/2.0_2.5_ddim_beta_9.0.jpg} & 
        \includegraphics[width=0.19\linewidth]{img/images/2.0_3.0_9.0/100/000020/2.0_3.0_ddim_beta_9.0.jpg}
        % \includegraphics[width=0.15\linewidth]{img/images/2.0_2.5_9.0/custom/000010/2.0_2.5_ddim_beta_9.0.jpg} &  
        % \includegraphics[width=0.15\linewidth]{img/images/2.0_2.5_9.0/custom/000012/2.0_2.5_ddim_beta_9.0.jpg} 
        \\[0.08cm]
                &
        \multicolumn{1}{p{0.16\linewidth}}{\centering \small kayak in the water, optical color, aerial view, rainbow} & 
        \multicolumn{1}{p{0.16\linewidth}}{\centering \small A small cactus with a happy face in the sahara desert}& 
        \multicolumn{1}{p{0.16\linewidth}}{\centering \small woman sniper, wearing soviet army uniform, in snow ground}& 
        \multicolumn{1}{p{0.16\linewidth}}{\centering \small A man wearing a suit is taking a self portrait with a camera}& 
        \multicolumn{1}{p{0.16\linewidth}}{\centering \small A snowboarder hitting a trick off a giant ramp.}
        % \multicolumn{1}{p{0.16\linewidth}}{\centering \small Spectacular Tiny World in the Transparent Jar..}
        % & 
        % \multicolumn{1}{p{0.16\linewidth}}{\centering \small A blue jay standing on a large basket of rainbow macarons}
       
        
    \end{tabular}
    \vspace{-0.4cm}
    \caption{Comparison of CFG and \our{}. As we can see, our model produces more realistic images, which is consistent with the numerical results from Tab.~\ref{tab:all}.} 
    \vspace{-0.4cm}
    \label{fig:sampling}
\end{figure*}

Significant progress was made in developing diffusion probabilistic denoising models (DDPM)~\citep{ho2020denoising,dhariwal2021diffusion}. DDPMs utilize a weighted variational bound objective by integrating probabilistic diffusion models with denoising score matching~\citep{song2019generative}. Despite demonstrating excellent generative capabilities and producing high-quality samples, the substantial computational expense of these models presented a significant drawback. Denoising Diffusion Implicit Models (DDIMs)~\citep{song2020denoising} improve scalability, particularly sample efficiency.


Ultimately, Latent Diffusion Models~\citep{rombach2022high} reduced the significant computational demands associated with applying diffusion models to high-dimensional scenarios by suggesting the implementation of diffusion within the low-dimensional latent space of an autoencoder. The practical application of this method is shown in Stable Diffusion~\citep{rombach2022high}. Scalability was enhanced further in models such as SDXL~\citep{podell2023sdxl}, which expanded the potential of latent diffusion models to tackle larger and more intricate tasks.

\paragraph{Guidance of diffusion models}
The Stochastic Differential Equation (SDE) framework is vital in diffusion models. While it facilitates exceptional generative capabilities, better scalability, and expedited training compared to models utilizing Ordinary Differential Equations (ODEs)~\citep{dinh2014nice,rezende2015variational, grathwohl2018ffjord}, the process of stochastic inference still needs \textit{guidance} to generate satisfactory samples.

Various techniques have been designed to guide the generation process in a specific direction. These can largely be classified into different strategies: guidance through classifiers, Langevin dynamics, Markov Chain Monte Carlo (MCMC), external guiding signals, architecture-specific features, etc. Despite their variations, these techniques generally steer the diffusion process toward areas of minimal energy, as inferred by different proxies.



A widely recognized approach is Classifier Guidance~\citep{dhariwal2021diffusion,poleski2024geoguide}, which incorporates an external classifier to infer the class from intermediate noisy diffusion steps. On the other hand, Classifier-Free Guidance~\citep{ho2022classifier} removes the requirement for an external classifier by using a unified model trained in both conditioned and unconditioned modes. Langevin dynamics is frequently used for off-policy guidance, where during each step along a trajectory, the model aligns with the scaled gradient norm toward areas of minimal energy (the maximum \textit{log probability}) \cite{zhang2021path} and \cite{sendera2024diffusion}. Alternatively, MCMC sampling strategies are directly used in diffusion processes \citep{song2023loss,chung2023diffusion}.

Specific techniques integrate external guiding functions to refine the generation path towards the targeted results \cite{bansal2023universal}. Alternatively, some exploit the inherent features of diffusion models, such as leveraging intermediate self-attention maps \citep{hong2023improving} or employing an externally trained discriminator network \citep{kim2022refining}. AutoGuidance \citep{karras2024guiding} enhances classifier-free guidance by substituting the unconditional model with a more compact, less sophisticated version to direct the conditional model.

% \begin{figure}[ht]
%     \centering
%     \renewcommand{\arraystretch}{0} % Usunięcie odstępów między wierszami
%     \setlength{\tabcolsep}{0.8pt}    % Usunięcie odstępów między kolumnami
%     \begin{tabular}{cccc}
%         \rotatebox{90}{ \qquad CFG}&
%         \includegraphics[width=0.30\linewidth]{img/images/fig5/cfg/img1/000043_ddim_7.5.png} & 
%         \includegraphics[width=0.30\linewidth]{example-image-a.png} & 
%         \includegraphics[width=0.30\linewidth]{img/images/fig5/cfg/img3/000040_ddim_7.5.png}  
%         \\
%         \rotatebox{90}{ \qquad \our{}}&
%         \includegraphics[width=0.30\linewidth]{img/images/fig5/cfg/img1/2.0_2.0_ddim_beta_7.5.png} & 
%         \includegraphics[width=0.30\linewidth]{example-image-a.png} & 
%         \includegraphics[width=0.30\linewidth]{img/images/fig5/cfg/img3/2.0_2.0_ddim_beta_7.5.png}  
%         \\[10pt]
%         & \multicolumn{1}{p{0.30\linewidth}}{\centering Prompt: A bed and a mirror in a small room.} & 
%         \multicolumn{1}{p{0.30\linewidth}}{\centering Prompt: a shoe rack with some shoes and a dog sleeping on them} & 
%         \multicolumn{1}{p{0.30\linewidth}}{\centering Prompt: A cat is sitting on a leather couch next to two remotes.}
%     \end{tabular}
%     \caption{T2I using SDXL, 50 NFE, CFG vs \our{}. }
%     \label{fig:grid_search_beta}
% \end{figure}



\section{Background}

\paragraph{Diffusion models}

% For the reader's convenience, we present the main idea behind the guidance of diffusion processes. 
% \paragraph{Introduction to diffusion models}
% \paragraph{Diffusion models.}
Diffusion models are generative algorithms that produce new samples through a gradual denoising process. This process begins with an initial Gaussian noise sample, denoted as $x_T$, and progressively refines it through steps that reduce noise, producing samples $x_{T-1}, x_{T-2}, \ldots, x_0$. The end result, $x_0$, lies on the data manifold. At each step $t$, there is a designated noise level, where $x_t$ combines the underlying signal $x_0$ and Gaussian noise $\epsilon$. The parameter $t$ controls the intensity of noise at each step. Training diffusion models involves randomizing over noise levels and time steps to produce a denoised $x_{t-1}$ from $x_t$. This denoising process is often modeled by U-Net ~\cite{ho2020denoising}.

Diffusion models involve two key processes: the forward and the reverse diffusion process. Consider $q(x_0)$ to be the data distribution such that $x_0 \sim q(x_0)$. This forward process introduces small Gaussian noise to the sample over $T$ steps, resulting in a sequence $x_{0}, \ldots, x_{T}$. The process is controlled by parameter $\{ \beta_{t} \in (0,1) \}_{t=1}^T$:
\begin{equation} \label{eq:0}
q(x_t|x_{t-1}) := \N(x_t ; \sqrt{1-\beta_t} x_{t-1},\beta_{t}I).
\end{equation}
Using this formulation, $x_t \sim q(x_t|x_0)$ can be calculated in one step:
\begin{equation}
\label{eq:epsilon}
\begin{split}
q(x_t|x_0) &  = \N(x_t ; \sqrt{ \bar \alpha_t } x_0, (1-\bar \alpha_t) I) \\
& = \sqrt{ \bar \alpha_t } x_0 + \e \sqrt{ 1-\bar \alpha_t }, \mbox{ } \e \sim \N(0,I),
\end{split}
\end{equation}
where $\alpha_t = 1-\beta_t$ and $\bar \alpha_t = \prod_{s=0}^t \alpha_t$.

% Typically we choose the schedule $\beta_t$ in such a way that $\bar \alpha_T$ is close to $0$, which implies that $q(x_T|x_0)$ is close to the distribution $\N(0,I)$. A simplest schedule for $\bar \alpha_t$ which satisfies this condition can be given by
% \begin{equation}
%     \bar \alpha_t=1-t/T.
% \end{equation}
% Additionally, we usually assume that the $\beta_t$ is increasing so that we have more denoising steps at the end of the backward process, which yields better quality of generated images (while we simultaneously allow larger denoising steps at the beginning of the backward process) \cite{ho2020denoising,nichol2021improved}.  

The calculation of the backward process is more challenging and requires access to the posterior distribution $q(x_{t-1}|x_t,x_0)$, which is Gaussian, with the mean given by $\tilde \mu_t (x_t, x_0)$ and the variance represented by $\tilde \beta_t$:
\begin{equation}
 q(x_{t-1}|x_t,x_0) = \N( x_{t-1}; \tilde \mu_t(x_t,x_0),\tilde \beta_t I ),   
\end{equation}
where $\tilde \mu_t(x_t,x_0) := \frac{\sqrt{\bar \alpha_{t-1}}\beta_t}{1- \bar \alpha_t} x_0 + \frac{\sqrt{\alpha_t}(1-\bar \alpha_{t-1})}{1-\bar \alpha_{t}}x_t $ and $\tilde \beta_t := \frac{1-\bar \alpha_{t-1}}{1-\bar \alpha_{t}} \beta_t$.

Practically, a neural network is applied to approximate conditional probabilities  $q(x_{t-1}|x_t)$.  In \cite{sohl2015deep} the authors show that as $T \to \infty$, $q(x_{t-1}|x_t)$ converges towards a diagonal Gaussian distribution and $\beta_t$ approaches zero. In this context, a neural network is trained to predict both the mean $\mu_{\theta}$ and a diagonal covariance matrix $\gamma_t I$ for the reverse diffusion process:
\begin{equation}
p(x_{t-1}|x_{t}):= \N(x_{t-1};\mu_{\theta}(x_t,t), \gamma_t I).  
\end{equation}
To ensure $p(x_0)$ effectively represents the true data distribution, $q(x_0)$ variational lower bound should be optimized in the training process. In practice, \cite{ho2020denoising} suggests training a model $\e_{\theta}(x_t, t)$ to approximate $\e$ from equation (\ref{eq:epsilon}) instead of modeling mean $\mu_{\theta}$ directly. The training objective is formulated as follows:
\begin{equation}
\cost{}:= \E_{t\sim [1,T],x_t \sim q(x_t|x_0), \e \sim \N (0,I)}  \| \e - \e_{\theta}(x_t,t) \|^2 ,
\end{equation}
where $\gamma_t$ is usually a fixed value, such as $\beta_t$ or $\tilde \beta_t$ representing the maximum and minimum boundaries for the true reverse-step variance, respectively.

For sampling purposes the mean $\mu_{\theta}(x_t, t)$ from $\e_{\theta}(x_t, t)$ can be calculated using the following formula:
\begin{equation}
\mu_{\theta}(x_t,t) = \frac{1}{\sqrt{\alpha_t}} \left( x_t - \frac{1-\alpha_t}{\sqrt{ 1-\bar \alpha_t }} \e_{\theta}(x_t,t)  \right).
\end{equation}
For clarity, we further use $\e(x_t):=\e_{\theta}(x_t,t)$. 

% \subsection{Guided Diffusion}

\paragraph{Classifier Guidance.} The process of generating samples from unconditioned, trained model $\e(x_t, t)$ can be guided using classification model $p(y|x_t)$, that predicts class $y$ from intermediate noisy samples $\mathbf{x}_t$ \citep{dhariwal2021diffusion}. The sampling procedure is performed according to modified updates considering the formula below:
\begin{equation}
    \hat{\epsilon}(x_t) = \epsilon(x_t) - \sqrt{1 - \bar{\alpha}_t} w \nabla_{x_t} \log p(y \vert x_t),
\end{equation}   
where $w$ is a scaling factor that adjusts the strength of the classifier’s influence, and $\bar{\alpha}_t = \prod_{i=1}^t 1 - \beta_i$. While effective, classifier-based guidance introduces several downsides, such as added complexity, the need of training an additional classifier and potential inaccuracies due to classifier errors.

\begin{figure*}[t]
    \centering
    \renewcommand{\arraystretch}{0}
    \setlength{\tabcolsep}{0.4pt}
    \begin{tabular}{c@{}c@{}c@{}c@{}c@{}c@{}}
        % \our{} & 
        
        \includegraphics[width=0.15\linewidth]{img/images/beta/beta_distribution_single_2.0_2.0.jpg} &
        \includegraphics[width=0.15\linewidth]{img/images/grid/0/0_0.jpg} & 
        \includegraphics[width=0.15\linewidth]{img/images/grid/0/0_1.jpg} & 
        \includegraphics[width=0.15\linewidth]{img/images/grid/0/0_2.jpg} & 
        \includegraphics[width=0.15\linewidth]{img/images/grid/0/0_3.jpg} &
        \includegraphics[width=0.15\linewidth]{img/images/grid/0/0_4.jpg} \\
        \includegraphics[width=0.15\linewidth]{img/images/beta/beta_distribution_single_2.0_2.5.jpg} & 
        \includegraphics[width=0.15\linewidth]{img/images/grid/0/1_0.jpg} & 
        \includegraphics[width=0.15\linewidth]{img/images/grid/0/1_1.jpg} & 
        \includegraphics[width=0.15\linewidth]{img/images/grid/0/1_2.jpg} & 
        \includegraphics[width=0.15\linewidth]{img/images/grid/0/1_3.jpg} &
        \includegraphics[width=0.15\linewidth]{img/images/grid/0/1_4.jpg} \\
        \includegraphics[width=0.15\linewidth]{img/images/beta/beta_distribution_single_2.0_3.0.jpg} & 
        \includegraphics[width=0.15\linewidth]{img/images/grid/0/2_0.jpg} & 
        \includegraphics[width=0.15\linewidth]{img/images/grid/0/2_1.jpg} & 
        \includegraphics[width=0.15\linewidth]{img/images/grid/0/2_2.jpg} & 
        \includegraphics[width=0.15\linewidth]{img/images/grid/0/2_3.jpg} &
        \includegraphics[width=0.15\linewidth]{img/images/grid/0/2_4.jpg} \\

        \includegraphics[width=0.15\linewidth]{img/images/beta/beta_distribution_single_2.0_4.0.jpg} & 
        \includegraphics[width=0.15\linewidth]{img/images/grid/0/3_0.jpg} & 
        \includegraphics[width=0.15\linewidth]{img/images/grid/0/3_1.jpg} & 
        \includegraphics[width=0.15\linewidth]{img/images/grid/0/3_2.jpg} & 
        \includegraphics[width=0.15\linewidth]{img/images/grid/0/3_3.jpg} &
        \includegraphics[width=0.15\linewidth]{img/images/grid/0/3_4.jpg} \\[0.2cm]
        \our{}&
        $\omega = 2.0$ & 
        $\omega = 5.0$ &
        $\omega = 7.5$ &
        $\omega = 9.0$ &
        $\omega = 12.5$ \\
        
    \end{tabular}
    \caption{Ablation study of our models on data generated for the prompt: "beautiful lady, freckles, big smile, blue eyes, short ginger hair, wearing a floral blue vest top, soft light, dark gray background." Thanks to the $\beta$-distribution, we can model how the diffusion trajectory behaves near data manifolds. }
    \label{fig:sampling}
\end{figure*}

\paragraph{Classifier GeoGuide.} In \cite{poleski2024geoguide}, the authors propose to modify the classifier guidance with gradient-based normalization to control updates. The process of guiding the diffusion model uses fixed-length updates to force the denoising process to be as close as possible to the data manifold:
\begin{equation}
    \hat{\epsilon}(x_t) = \epsilon(x_t) - w \frac{\sqrt{D}}{T} \frac{\nabla_{x_t} p(y \vert x_t)}{||\nabla_{x_t} p(y \vert x_t)||},
\end{equation}  
where $D$ is data dimensionality and $T$ is the number of diffusion steps.  

\paragraph{Classifier-Free Guidance.}
Classifier-Free Guidance (CFG) \citep{ho2022classifier} is a technique employed in diffusion models to enhance control over the generative process without needing external classifiers. It has shown significant effectiveness in boosting the quality of generated outputs across tasks like image and text generation.

CFG requires access to the additional conditional generative model, for which the conditioning factor $c$ is incorporated as additional input to the denoising component, $\e_c(x_t)$. CFG guides generation by combining conditional and unconditional predictions. For a noisy sample $x_t$, this guidance is implemented by interpolating between these conditional and unconditional predictions as follows:
\begin{align}
    \label{eq:cfg}
    \hat{\epsilon}_c^{w}(x_t) = \epsilon_{\o}(x_t) + w \left( \epsilon_c(x_t) - \epsilon_{ \o}(x_t) \right),    
\end{align}
where $\epsilon_{\o}(x_t)$ represents the model’s prediction of the noise for $x_t$ in unconditional case, while $\epsilon_c(x_t)$ denotes the noise prediction when conditioned on $c$. The parameter $w$ serves as the guidance scale, adjusting the extent to which the conditional information $y$ influences the generated output. The procedure of the reverse diffusion sampling process is given by Algorithm \ref{alg:cfg}. 

Adjusting $w$ allows control over the balance between sample diversity and consistency to the conditioning $y$. Setting $w = 1$ results in standard conditional generation. When $w > 1$, the influence of the conditioning information is amplified, encouraging the model to generate samples that align more closely with $y$, though this may reduce diversity.



\begin{algorithm}[t]
    \caption{Reverse Diffusion with CFG}
	\includegraphics[width=1.0\linewidth]{img/cod_1.png}
%        \begin{algorithmic}[1]
%        \Require $x_T \sim \mathcal{N}(0, \mathbf{I}_d)$, $0 \leq \omega \in \mathbb{R}$
%            \For{$i=T$ to 1}		
%                \State $\hat{\epsilon}_{c}^{\omega}(x_t) = \epsilon_{\emptyset}(x_t) + \omega [\epsilon_{c}(x_t) - \epsilon_{\emptyset}(x_t)]$
%                \State $\hat{x}^{\omega}_{c}(x_t) \leftarrow \left( x_t - \sqrt{1 - \bar{\alpha}_{t}}\hat{\epsilon}_{c}^{\omega}(x_t)\right)/ \sqrt{\bar{\alpha}_t}$
%                \State $x_{t-1} = \sqrt{{\bar\alpha}_{t-1}}\hat{x}^{\omega}_c(x_t) + \sqrt{1 - {\bar\alpha}_{t-1}}\hat{\epsilon}^{\omega}_c(x_t)$
%            \EndFor
%            \State \Return $x_0$
%        \end{algorithmic} 
        \label{alg:cfg}
\end{algorithm}

\begin{algorithm}[t]
    \caption{Reverse Diffusion with CFG++ }
        \label{alg_our}
\includegraphics[width=1.0\linewidth]{img/cod_2.png}
%        \begin{algorithmic}[1]
%        \Require $x_T \sim \mathcal{N}(0, \mathbf{I}_d)$, $0 \leq \lambda \leq 1$
%            \For{$i=T$ to 1}		
%                \State $\hat{\epsilon}^{\lambda}_{c}(x_t) = \epsilon_{\o}(x_t) 
%                +  \lambda  
%                [\epsilon_{c}(x_t) - \epsilon_{\emptyset}(x_t)]$
%                \State $\hat{x}^{\lambda}_{c}(x_t) \leftarrow \left( x_t - \sqrt{1 - \bar{\alpha}_{t}}\hat{\epsilon}_{c}^{\lambda}(x_t)\right)/ \sqrt{\bar{\alpha}_t}$
%                \State $x_{t-1} = \sqrt{{\bar\alpha}_{t-1}}\hat{x}^{\lambda}_c(x_t) + \sqrt{1 - {\bar\alpha}_{t-1}}\hat{\epsilon}_{\emptyset}(x_t)$
%            \EndFor
%            \State \Return $x_0$
%        \end{algorithmic} 
        \label{alg:cfgplus}
\end{algorithm}
\begin{figure*}[ht]
    \centering
    \renewcommand{\arraystretch}{0}
    \setlength{\tabcolsep}{0pt}
    \begin{tabular}{c@{\;}ccccccc}
         &901 & 801 & 701 & 601 & 501 & 401 & 0\\[5pt]
        \rotatebox{90}{ \qquad CFG} &
        \includegraphics[width=0.13\linewidth]{img/images/trajectory/example1/ddim/x0_901.jpg} & 
        \includegraphics[width=0.13\linewidth]{img/images/trajectory/example1/ddim/x0_801.jpg} & 
        \includegraphics[width=0.13\linewidth]{img/images/trajectory/example1/ddim/x0_701.jpg} & 
        \includegraphics[width=0.13\linewidth]{img/images/trajectory/example1/ddim/x0_601.jpg} & 
        \includegraphics[width=0.13\linewidth]{img/images/trajectory/example1/ddim/x0_501.jpg} & 
        \includegraphics[width=0.13\linewidth]{img/images/trajectory/example1/ddim/x0_401.jpg} & 
        \includegraphics[width=0.13\linewidth]{img/images/trajectory/example1/ddim/x0_1.jpg} \\
        \rotatebox{90}{ \qquad \our{}} &
        \includegraphics[width=0.13\linewidth]{img/images/trajectory/example1/our/x0_901.jpg} & 
        \includegraphics[width=0.13\linewidth]{img/images/trajectory/example1/our/x0_801.jpg} & 
        \includegraphics[width=0.13\linewidth]{img/images/trajectory/example1/our/x0_701.jpg} & 
        \includegraphics[width=0.13\linewidth]{img/images/trajectory/example1/our/x0_601.jpg} & 
        \includegraphics[width=0.13\linewidth]{img/images/trajectory/example1/our/x0_501.jpg} & 
        \includegraphics[width=0.13\linewidth]{img/images/trajectory/example1/our/x0_401.jpg} &
        \includegraphics[width=0.13\linewidth]{img/images/trajectory/example1/our/x0_1.jpg}  
   
    \end{tabular}
    \caption{The the evolution of denoised estimates differs between CFG and \our{}. Both methods behave in a similar way at the beginning of the trajectory. However, \our{} converges faster to the data manifold to produce an image that is more consistent with the prompt: "a shoe rack with some shoes and a dog sleeping on them".}
    \label{fig:sampling}
\end{figure*}
\paragraph{CFG++} represents the simple extension of CFG, that utilizes a small guidance scale, typically $0 \leq \lambda \leq 1$, that enables smooth interpolation between unconditional and conditional sampling. The reverse diffusion process that utilizes CFG++ is provided by Algorithm \ref{alg:cfgplus}.

% \begin{algorithm}[t]
%     \caption{Reverse Diffusion with CFG++}
%         \label{alg_cfgpp}
%         \begin{algorithmic}[1]
%         \Require $x_T \sim \mathcal{N}(0, \mathbf{I}_d), \lambda \in \left[ 0, 1 \right]$
%             \For{$i=T$ to 1}		
%                 \State $\hat{\epsilon}_{c}^{\lambda}(x_t) = \hat{\epsilon}_{\emptyset}(x_t) + \lambda [\hat{\epsilon}_{c}(x_t) - \hat{\epsilon}_{\emptyset}(x_t)]$
%                 \State $\hat{x}^{\lambda}_{c}(x_t) \leftarrow \left( x_t - \sqrt{1 - \bar{\alpha}_{t}}\hat{\epsilon}_{c}^{\lambda}(x_t)\right)/ \sqrt{\bar{\alpha}_t}$
%                 \State $x_{t-1} = \sqrt{{\bar\alpha}_{t-1}}\hat{x}^{\lambda}_c(x_t) + \sqrt{1 - {\bar\alpha}_{t-1}}\hat{\epsilon}_{\emptyset}(x_t)$
%             \EndFor
%             \State \Return $x_0$
%         \end{algorithmic} 
% \end{algorithm}
% \begin{algorithm}[t]
%     \caption{Reverse Diffusion with CFG-GEO} 
%     \label{alg_geo}
%         \begin{algorithmic}[1]
%             \For {$i=T$ to $1$}	
%                 \State $\hat{\epsilon}^{w}(x_t, y) = \epsilon(x_t, \o) + w \frac{\sqrt{D}}{T}\frac{[\epsilon(x_t, y) - \epsilon(x_t, \o)]}{\ ||[\epsilon(x_t, y) - \epsilon(x_t, \o)] || }$
%                 \State $\hat{x}^{\omega}_c(x_t) \leftarrow (x_t - \sqrt{1 - \bar{\alpha}_t} \hat{\epsilon}^{w}(x_t,  y) / \sqrt{\bar{\alpha}_t}$
%                 \State $x_{t-1} = \sqrt{\bar{\alpha}_{t-1}} \hat{x}^{\omega}_c(x_t) + \sqrt{1 - \bar{\alpha}_{t-1}} \hat{\epsilon}^{w}(x_t, y)$
%             \EndFor
%             \State \Return $x_0$
%         \end{algorithmic} 
% \end{algorithm}
% \begin{algorithm}
% \caption{\textbf{Reverse Diffusion with CFG}}
% \begin{algorithmic}[1]
% \Require $x_T \sim \mathcal{N}(0, \mathbf{I}_d)$, $0 \leq \omega \in \mathbb{R}$
% \For{$i = T$ to $1$}
%     \State $\hat{\epsilon}^{\omega}_c(x_t) = \hat{\epsilon}_{\varnothing}(x_t) + \omega [\hat{\epsilon}_c(x_t) - \hat{\epsilon}_{\varnothing}(x_t)]$
%     \State $\hat{x}^{\omega}_c(x_t) \gets (x_t - \sqrt{1 - \bar{\alpha}_t} \hat{\epsilon}^{\omega}_c(x_t)) / \sqrt{\bar{\alpha}_t}$
%     \State $x_{t-1} = \sqrt{\bar{\alpha}_{t-1}} \hat{x}^{\omega}_c(x_t) + \sqrt{1 - \bar{\alpha}_{t-1}} \hat{\epsilon}^{\omega}_c(x_t)$
% \EndFor
% \State \Return $x_0$
% \end{algorithmic}
% \end{algorithm}

% \begin{center}
% $f(x) = ax^{2} - ax + c$
% \end{center}

% \begin{algorithm}[t]
%     \caption{Reverse Diffusion with \our{}} 
    
%         \begin{algorithmic}[1]
%             % f(x) = ax^2 - ax + c, where a = b
%             \For {$i=T$ to $1$}	
%                 \State $\hat{\epsilon}^{w}(x_t, y) = \epsilon(x_t, \o) + \beta(t) w \frac{[\hat{\epsilon}(x_t, y) - \hat{\epsilon}(x_t, \o)]}{\ ||[\hat{\epsilon}(x_t, y) - \hat{\epsilon}(x_t, \o)] || }$
%                 \State $\hat{x}^{\omega}_c(x_t) \leftarrow (x_t - \sqrt{1 - \bar{\alpha}_t} \hat{\epsilon}^{w}(x_t,  y) / \sqrt{\bar{\alpha}_t}$
%                 \State $x_{t-1} = \sqrt{\bar{\alpha}_{t-1}} \hat{x}^{\omega}_c(x_t) + \sqrt{1 - \bar{\alpha}_{t-1}} \hat{\epsilon}^{w}(x_t, y)$
%             \EndFor
%             \State \Return $x_0$
%         \end{algorithmic} 
% \end{algorithm}



%%%%%%%%%%%%%%%%%%%%%%%%%%%%%%%%%%%%%%%%
% \section{Theoretical backgound}

% Our model can be described as the iterative model
% $$
% x_{k+1}=F(x_k),
% $$
% $$
% x_{k-1}=G(x_k),
% $$
% where $f$ and $g$ are invertible (numerically close to invertible). 

% Starting from the data manifold we arrive at the gaussian, while in the reverse process we arrive at the data manifold.

% The guidance is denoted by $g$ -- it increases the desired property of the function.

% Moreover, the data is lying on the space given by $\sqrt{\bar\alpha_t}x_M+\sqrt{1-\bar \alpha_t}x_G$, where $x_M \in M$ and $X_G \sim N(0,I)$.

% Additionally, at the manifold we have the attraction, i.e. for points $x \in M$, we have 
% $$
% G()
% $$




% \begin{figure*}[t]
%     \centering
%     % \begin{tabular}{p{1cm}p{3cm}p{3cm}p{3cm}p{3cm}p{3cm}}
%     %     CFG & 2.0 & 5.0 & 7.5 & 9.0 & 12.5 \\
%     %      & 
%     %     \multicolumn{5}{c}{
%         \includegraphics[width=0.9\linewidth]{img/images/fig7/000000.jpg}
%     %     }
%     % \end{array}    
%      \caption{Quantitative evaluation (FID, CLIP-similarity) of T2I with SD v1.5}
%     \label{fig:grid_1}
% \end{figure*}
%%%%%%%%%%%%%%%%%%%%%%%%%%%%%%%%%%%%%%%%
\section{\our{}}

In this section, we introduce \our{}, the novel approach for stabilizing the guidance process with the normalized, dynamic control of the impact of CFG in the denoising process. The section is organized as follows. First, we motivate dynamic scaling by analyzing the impact of CFG on various stages of diffusion sampling. Second, we introduce the general procedure for stabilizing the CFG with the $\beta$ function scaling and gradient normalization.   

% First, we show how the impact can be balanced along the diffusion steps. Second, we introduce the approach where we dynamically control the scaling parameter considering the stage of the reverse diffusion process.  

%\subsection{Derivation of general \our{} model}

\begin{algorithm}[t] 
    \caption{General Reverse Diffusion with CFG}
    % \label{alg_cfg}
    \label{al:1}
\includegraphics[width=1.0\linewidth]{img/cod_3.png}
%    \begin{algorithmic}[1]
%        \Require $x_T \sim \mathcal{N}(0, \mathbf{I}_d)$
%        %and control function $E(t,x)$
%        \For{$i=T$ to 1}       
%            \State $\hat{\epsilon}_{c}(x_t) = \epsilon_{\emptyset}(x_t) + \color{blue}  E_c(t,x_t)$
%            \State $\hat{x}_{c}(x_t) \leftarrow \left( x_t - \sqrt{1 - \bar{\alpha}_{t}}\hat{\epsilon}_{c}(x_t)\right)/ \sqrt{\bar{\alpha}_t}$
%            \State $x_{t-1} = \sqrt{{\bar\alpha}_{t-1}}\hat{x}_c(x_t) + \sqrt{1 - {\bar\alpha}_{t-1}}\hat{\epsilon}_c(x_t)$
%        \EndFor
%        \State \Return $x_0$
%    \end{algorithmic} 
\end{algorithm}

\subsection{Motivation}

The guided sampling for diffusion models can be generally written as provided in Algorithm \ref{al:1},  where $E_c(t,x_t)$ is the general correction (drift) term, which aims to guide the trajectory towards the region satisfying the desired properties. In the special case where $E_c(t,x_t)=0$, the model serves as the standard DDIM model without any guidance. When the diffusion model is adequately trained, it transforms the example from the data manifold $M$ at $t=0$ to the Gaussian distribution at $t=T$. Empirically, the sample from a Gaussian distribution is located in a small neighborhood of the sphere $S=\{x: \|x\|=\sqrt{d}\}$, where $d$ is the dimensionality of the data. This implies that for values of $t$ close to zero, the trajectories are near the data manifold $M$, whereas for values of $t$ close to $T$, the trajectories are near the sphere $S$.

To investigate the role of the adjustment term (which can be interpreted as a drift) $E_c(t,x_t)$, let us assume that $E_c(t,x_t)=0$ and denote $P(t,x_t) = x_{t-1}$ as the DDIM dynamic process in the following way:
\begin{equation} 
\begin{split}
P(t,x_t)&=\tfrac{\sqrt{{\bar\alpha}_{t-1}}}{\sqrt{\bar{\alpha}_t}}\left( x_t - \sqrt{1 - \bar{\alpha}_{t}}\hat{\epsilon}_{\emptyset}(x_t)\right)  \\
&+\sqrt{1 - {\bar\alpha}_{t-1}}\hat{\epsilon}_{\emptyset}(x_t).
\end{split}
\end{equation}

As a consequence, adding guidance component $E_c(t,\tilde{x}_t)$ modifies the model's dynamics in the following way: 

\begin{equation}
\tilde{x}_{t-1}=P(t,\tilde{x}_t)+e(t) \cdot E_c(t,\tilde{x}_t),
\end{equation}
where $e(t)$ is defined as:
\begin{equation}
e(t)=\sqrt{1-\bar \alpha_{t-1}}-\sqrt{\frac{1}{\alpha_t}-\bar \alpha_{t-1}}.    
\end{equation}

\begin{table*}[h!]
    \centering
    \begin{tabular}{ccccccccccc}
        \hline
        \multirow{2}{*}{Method} & \multicolumn{2}{c}{$\omega = 2.0, \lambda = 0.2$} & \multicolumn{2}{c}{$\omega = 5.0, \lambda = 0.4$} & \multicolumn{2}{c}{$\omega = 7.5, \lambda = 0.6$} & \multicolumn{2}{c}{$\omega = 9.0, \lambda = 0.8$} & \multicolumn{2}{c}{$\omega = 12.5, \lambda = 1.0$} \\
        & FID $\downarrow$ & CLIP $\uparrow$ & FID $\downarrow$ & CLIP $\uparrow$ & FID $\downarrow$ & CLIP $\uparrow$ & FID $\downarrow$ & CLIP $\uparrow$ & FID $\downarrow$ & CLIP $\uparrow$ \\
        \hline
        CFG & 13.94 & 0.306 & 16.16 & \bf 0.318 & 18.98 & \bf 0.319 & 20.16 & \bf 0.320 & 22.32 & \bf 0.320 \\
        CFG++ & \bf 13.36 & \bf 0.311 & 16.02 & \bf 0.318 & 18.57 & \bf 0.319 & 20.48 & \bf 0.320 & 21.97 & \bf 0.320 \\
        \our{} & 15.02 & 0.306
        & \bf 15.76 & 0.317
        & \bf 17.99 & \bf 0.319
        & \bf 18.94 & 0.319
        & \bf 20.97 & \bf 0.320
        \\
        \hline

    \end{tabular}
    \caption{Quantitative evaluation (FID, CLIP-similarity) of 50NFE DDIM T2I with SD v1.5 on COCO 10k.
    }
    \label{tab:all}
\end{table*}

Since $\alpha_t=1-\beta_t$, and $\beta_t \neq 0$, we conclude that the function $e(t)$ is generally nonzero.

The correct DDIM training ensures that manifolds $M$ and $S$ act as attractors of the model at times $t=0$ and $t=T$. That is, every trajectory starting on the manifold $S$ is drawn towards $M$ and arrives at $M$ at $t=0$. Moreover, adding a sufficiently small noise at some point $x_t$ (for some $t>0$) will be compensated for by the model, ensuring convergence to $M$. Conversely, when reversing time, a trajectory starting in a small neighborhood of $M$ at $t=0$ will arrive in a neighborhood of $S$ at $t=T$.

However, this property is no longer assured if we incorporate a correction term $e(t) \cdot  E_c(t,x_t)$. Specifically, the trajectory at $t=0$ will usually diverge from the desired data manifold $M$. To illustrate this, consider the final step at~$t=1$:
\begin{equation}
\begin{array}{c}
\tilde{x}_0=P(1,\tilde{x}_1)+e(1) \cdot E_c(1,\tilde{x}_1).
\end{array}
\end{equation}
Since the correct iterative procedure ensures $m=x_0=P(1,\tilde{x}_1) \in M$, adding the correction $e(1) \cdot E_c(1,\tilde{x}_1)$ results in:
\begin{equation}
\tilde{x}_0=m+e(1) \cdot E_c(1,\tilde{x}_1), \quad \text{for some } m \in M.
\end{equation}







This implies that, in general, $\tilde{x}_0$ will not lie in the data manifold $M$.
This could be a significant drawback since we strongly prefer the images generated by the diffusion model to remain in the data manifold. This property is preserved if the function $E_c(1,\tilde{x}_1)$ satisfies the following boundary conditions:
\begin{equation}
\lim_{t \to 0}E_c(t,x)=0 \text{ and } \lim_{t \to T}E_c(t,x)=0.
\label{eq:conditions}
\end{equation}




To enforce these conditions, we propose multiplying the correction term by a continuous function that vanishes at the time limits. In our paper, we implement this by applying the beta distribution:
\begin{equation} 
 \beta(t)=\frac{t^{a - 1} (1 - t)^{b - 1}}{B(a, b)}, 
 \label{eq:beta}
\end{equation}
where $B(a, b)$ is \emph{Beta} function, and $a$ and $b$ are the hyperparameters that control the curvature of the density function. The function is defined for $ t \in [0, 1]$, so the integer indexing should be rescaled to this interval. We propose this kind of function due to the flexibility of modeling and shifting function with one mode, assuming $a>1$ and $b>1$, which guarantee that $\beta(0)=\beta(1)=0$. 

Thus the general model for an arbitrary is given by
Algorithm \ref{al:1}, with function $E$ multiplied by 
$\beta_{a,b}(t/T)$. In the next subsection we present the algorithm devised for CFG.

% We provide a theoretical motivation behind our proposed model. Observe that the CFG of diffusion models can generally be written as in Algorithm \ref{al:1}:
% \begin{algorithm}[t] 
%     \caption{General Reverse Diffusion with CFG}
%     % \label{alg_cfg}
%     \label{al:1}
%     \begin{algorithmic}[1]
%         \Require $x_T \sim \mathcal{N}(0, \mathbf{I}_d)$, $0 \leq \omega \in \mathbb{R}$
%         %and control function $E(t,x)$
%         \For{$i=T$ to 1}       
%             \State $\hat{\epsilon}_{c}(x_t) = \hat{\epsilon}_{\emptyset}(x_t) + \color{red} E_c(t,x_t)$
%             \State $\hat{x}_{c}(x_t) \leftarrow \left( x_t - \sqrt{1 - \bar{\alpha}_{t}}\hat{\epsilon}_{c}(x_t)\right)/ \sqrt{\bar{\alpha}_t}$
%             \State $x_{t-1} = \sqrt{{\bar\alpha}_{t-1}}\hat{x}_c(x_t) + \sqrt{1 - {\bar\alpha}_{t-1}}\hat{\epsilon}_c(x_t)$
%         \EndFor
%         \State \Return $x_0$
%     \end{algorithmic} 
% \end{algorithm}
% where $E(t,x_t)$ is the general correction (drift) term in our model, which aims to guide the trajectory towards the region satisfying the desired properties. In the special case where $E=0$, we recover the standard DDIM model. When properly trained, it transitions from the data manifold $M$ at $t=0$ to the Gaussian distribution at $t=T$. Empirically, the data sampled from a Gaussian distribution lies in a small neighborhood of the sphere $S=\{x: \|x\|=\sqrt{d}\}$, where $d$ is the dimensionality of the data. This implies that for values of $t$ close to zero, the trajectories are near the data manifold $M$, whereas for values of $t$ close to $T$, the trajectories are near the sphere $S$.

% To investigate the role of the adjustment term (which can be interpreted as a drift) $E$, let $P$ denote the right-hand side for $E=0$. In other words, the pure DDIM dynamical process is given by
% \[
% x_{t-1}=P(t,x_t)
% \]
% where 
% \[
% P(t,x_t)=\tfrac{\sqrt{{\bar\alpha}_{t-1}}}{\sqrt{\bar{\alpha}_t}}\left( x_t - \sqrt{1 - \bar{\alpha}_{t}}\hat{\epsilon}_{\emptyset}(x_t)\right) + \sqrt{1 - {\bar\alpha}_{t-1}}\hat{\epsilon}_{\emptyset}(x_t).
% \]
% By straightforward calculations, we see that adding $E$ modifies the model's dynamics into
% $$
% \tilde{x}_{t-1}=P(t,\tilde{x}_t)+e(t) \cdot E(t,\tilde{x}_t),
% $$
% where
% $$
% e(t)=\sqrt{1-\bar \alpha_{t-1}}-\sqrt{\frac{1}{\alpha_t}-\bar \alpha_{t-1}}.
% $$

% Since $\alpha_t=1-\beta_t$, and $\beta_t \neq 0$, we conclude that the function $e(t)$ is generally nonzero.

% The correct DDIM training ensures that manifolds $M$ and $S$ act as attractors of the model at times $t=0$ and $t=T$. That is, every trajectory starting on the manifold $S$ is drawn towards $M$ and arrives at $M$ at $t=0$. Moreover, adding a sufficiently small noise at some point $x_t$ (for some $t>0$) will be compensated for by the model, ensuring convergence to $M$. Conversely, when reversing time, a trajectory starting in a small neighborhood of $M$ at $t=0$ will arrive in a neighborhood of $S$ at $t=T$.

% However, if we introduce a correction term $e(t) \cdot E(t,x_t)$, this property is no longer guaranteed. Specifically, the trajectory at $t=0$ will typically deviate from the desired data manifold $M$. To observe this, consider the last iterative step at $t=1$:
% \[
% \tilde{x}_0=P(1,\tilde{x}_1)+e(1) \cdot E(1,\tilde{x}_1).
% \]
% Since the correct iterative procedure ensures $m=x_0=P(1,\tilde{x}_1) \in M$, adding the correction $e(1) \cdot E(1,\tilde{x}_1)$ results in
% \[
% \tilde{x}_0=m+e(1) \cdot E(1,\tilde{x}_1), \quad \text{for some } m \in M.
% \]
% This implies that, in general, $\tilde{x}_0$ will not lie in the data manifold $M$.

% This could be a significant drawback since we strongly prefer the images generated by the diffusion model to remain in the data manifold. This property is preserved if the function $E$ satisfies the following boundary conditions:
% $$
% \lim_{t \to 0}E(t,x)=0 \text{ and } \lim_{t \to T}E(t,x)=0.
% $$
% To enforce these conditions, we propose multiplying the correction term by a continuous function that vanishes at the time limits. In our paper, we implement this by applying the beta distribution:
% \begin{equation} 
%  \beta(t)=\frac{t^{a - 1} (1 - t)^{b - 1}}{B(a, b)}, 
% \end{equation}
% where $B(a, b)$ is \emph{Beta} function, and $a$ and $b$ are the hyperparameters that control the curvature of the density function. The function is defined for $ t \in [0, 1]$, so the integer indexing should be rescaled to this interval. We propose this kind of function due to the flexibility of modeling and shifting function with one mode, assuming $a>1$ and $b>1$, which guarantee that $\beta(0)=\beta(1)=0$. 

% Thus the general model for an arbitrary is given by
% Algorithm \ref{al:1}, with function $E$ multiplied by 
% $\beta_{a,b}(t/T)$. In the next subsection we present the algorithm devised for CFG.





% \paragraph{Gradient Normalization} \red{może do bacground}

% \begin{algorithm}[t]
%     \caption{Reverse Diffusion with \our{}$_{\gamma}$ }
%         \label{alg_our}
%         \begin{algorithmic}[1]
%         \Require $x_T \sim \mathcal{N}(0, \mathbf{I}_d)$, $0 \leq \lambda \in \mathbb{R}$
%         \Require $\gamma$
%             \For{$i=T$ to 1}		
%                 \State $\hat{\epsilon}^{\lambda}_{c}(x_t) = \epsilon_{\o}(x_t) 
%                 + \color{red} \beta(t) \lambda  
%                 \frac{[\epsilon_c(x_t) - \epsilon_{\o}(x_t)]}{\ ||[\epsilon_c(x_t) - \epsilon_{\o}(x_t)] ||^{\gamma} }$
%                 \State $\hat{x}^{\lambda}_{c}(x_t) \leftarrow \left( x_t - \sqrt{1 - \bar{\alpha}_{t}}\hat{\epsilon}_{c}^{\lambda}(x_t)\right)/ \sqrt{\bar{\alpha}_t}$
%                 \State $x_{t-1} = \sqrt{{\bar\alpha}_{t-1}}\hat{x}^{\lambda}_c(x_t) + \sqrt{1 - {\bar\alpha}_{t-1}}\hat{\epsilon}_{\emptyset}(x_t)$
%             \EndFor
%             \State \Return $x_0$
%         \end{algorithmic} 
%         \label{alg:our}
% \end{algorithm}

\begin{algorithm}[t]
    \caption{Reverse Diffusion with \our{}$_{\gamma}$ }
        \label{alg_our}
\includegraphics[width=1.0\linewidth]{img/cod_4.png}
%        \begin{algorithmic}[1]
%        \Require $x_T \sim \mathcal{N}(0, \mathbf{I}_d)$, $0 \leq \omega \in \mathbb{R}$, $\gamma \in \mathbb{R}_{+}$
%            \For{$i=T$ to 1}		
%                \State $\hat{\epsilon}^{\beta}_{c}(x_t) = \epsilon_{\o}(x_t) 
%                + \color{blue} \beta(t) \cdot \omega  
%                \frac{[\epsilon_c(x_t) - \epsilon_{\o}(x_t)]}{\ ||[\epsilon_c(x_t) - \epsilon_{\o}(x_t)] ||^{\gamma} }$
%                \State $\hat{x}^{\beta}_{c}(x_t) \leftarrow \left( x_t - \sqrt{1 - \bar{\alpha}_{t}}\hat{\epsilon}_{c}^{\beta}(x_t)\right)/ \sqrt{\bar{\alpha}_t}$
%                \State $x_{t-1} = \sqrt{{\bar\alpha}_{t-1}}\hat{x}^{\beta}_c(x_t) + \sqrt{1 - {\bar\alpha}_{t-1}}\hat{\epsilon}^{\beta}_c(x_t)$
%            \EndFor
%            \State \Return $x_0$
%        \end{algorithmic} 
        \label{alg:our}
        % \vspace{-0.5cm}
\end{algorithm}


% Inspired by GeoGuide for classifier guidance, we propose to adapt this method to the CFG approach:

% \begin{equation} 
% \hat{\epsilon}^{w}(x_t, y) = \epsilon(x_t, \o) + w \frac{\sqrt{D}}{T}\frac{[\epsilon(x_t, y) - \epsilon(x_t, \o)]}{\ ||[\epsilon(x_t, y) - \epsilon(x_t, \o)] || }.
% \end{equation}

% Similar to the basic classification variant, the normalization of the guidance step enables the process of generating samples to be more consistent with the data manifold, and the impact of the updates is resistant to the randomness in the generation process. The pseudocode for this approach is given by Algorithm \ref{alg_geo}. 

\begin{figure*}[t]
    \centering
    \renewcommand{\arraystretch}{0}
    \setlength{\tabcolsep}{0.4pt}
    \begin{tabular}{c@{}c@{}c@{}c@{}c@{}c@{}}
        
        \includegraphics[width=0.15\linewidth]{img/images/beta/beta_distribution_single_2.0_2.0.jpg} &
        \includegraphics[width=0.15\linewidth]{img/images/gammav2/0.0.jpg} &
        \includegraphics[width=0.15\linewidth]{img/images/gammav2/0.25.jpg} &
        \includegraphics[width=0.15\linewidth]{img/images/gammav2/0.5.jpg} &
        \includegraphics[width=0.15\linewidth]{img/images/gammav2/1.0.jpg} &
        \includegraphics[width=0.15\linewidth]{img/images/gammav2/2.0.jpg} \\[0.2cm]
        \our{} & $\gamma = 0.0 $ & $\gamma = 0.25 $ & $\gamma = 0.5 $  & $\gamma = 1.0 $ & $\gamma = 2.0 $ \\
    \end{tabular}
    \caption{Example of sampled element according to $\gamma$ parameters. Prompt: "A man holding a phone while sanding next to a street.".}
    \label{fig:sampling_example2}
\end{figure*}

\subsection{Sampling with \our{}} 

As shown in the previous subsection, the CFG may track the generated sample in the regions outside the data manifold. As a consequence, the impact of CFG should be different for some particular stages of sample generation. The initial sampling stage should focus on general templates of images, so the impact of the conditional model should be minor. During the intermediate stage, the model should follow the path determined by the conditioning factor $c$, increasing the importance of the CFG component. During the final stage of generating, the impact of CFG should be minor in order to locate the generated sample in the data manifold. To incorporate this, we propose to modify the CFG by simply scaling this term with the dynamic function:
\begin{equation} 
\hat{\epsilon}^{\beta}_{c}(x_t) = \epsilon_{\o}(x_t) 
+ \beta(t) \cdot \omega  
\frac{[\epsilon_c(x_t) - \epsilon_{\o}(x_t)]}{\ ||[\epsilon_c(x_t) - \epsilon_{\o}(x_t)] ||^{\gamma} },
\end{equation}
where $\beta(t)$ is the function that controls the impact of normalized CFG during the various training stages, and $\omega$ hyperparameter controls the magnitude of the scaling function. Moreover, drawing inspiration from the GeGuide approach, we propose normalizing the guidance term using the norm to the power of $\gamma \in \mathbb{R}_{+}$. Consequently, we ensure that the scaled updates remain independent of the dimensionality of the data.




In general, any $\beta(t) \geq 0$ that enforces conditions given by \eqref{eq:conditions} can be used to model the dynamics for scaling CFG. In this work, we postulate to utilize the density function for $\beta$ distribution given by the equation \eqref{eq:beta} that has desired properties and preserves the volume. The modified procedure of sampling with our approach is provided by Algorithm \ref{alg:our}.

\our{} can be also easily adopted to the CFG++ process, where the guidance step 2 from Algorithm \ref{alg:cfgplus} is replaced by the following update:
\begin{equation} 
\hat{\epsilon}^{\beta++}_{c}(x_t) = \epsilon_{\o}(x_t) 
+ \beta(t) \cdot \lambda  
\frac{[\epsilon_c(x_t) - \epsilon_{\o}(x_t)]}{\ ||[\epsilon_c(x_t) - \epsilon_{\o}(x_t)] ||^{\gamma} }.
\end{equation}

\section{Experiments}

This section presents a series of experiments designed to evaluate the performance of our method in comparison to reference CFG and CFG++. We start with a simple 2D example to visually demonstrate our model's behavior. Then, we conduct both quantitative and qualitative comparisons against for both SD v1.5 and SDXL models, using 50 NFE DDIM sampling. Finally, we report the results of our ablation studies.
% The $\lambda$ values for CFG++ and the $\omega$ values for CFG were selected based on the parameters proposed by the authors of \ref{}, where $\lambda$ was set to 0.2, 0.4, 0.6, 0.8, 1.0 and $\omega$ to 2.0, 5.0, 7.5, 9.0, 12.5. The hyperparameters of our approach, $\alpha$, $\beta$, and $\gamma$, were determined by a grid search, optimizing for the FID, CLIP and ImageReward metrics, as reported in Table \ref{}.


\begin{figure}
    \centering
        \includegraphics[width=\linewidth]{img/images/cfg_beta.png}
        \vspace{-0.3cm}
    \caption{Ablation study of \our{} models according to $\beta$-distribution parameters. We present the FID and CLIP score relation when the $\omega$ parameter is changed.  }
    \label{fig:ablation_cfg_beta}
    \vspace{-0.3cm}
\end{figure}

% \subsection{More details}
\paragraph{Toy example 2D}
% To illustrate the reason unguided diffusion models often produce inadequate images and how CFG resolves this problem, as discussed in \cite{karras2024guiding}, the authors present a 2D toy example. In this example, a simple denoiser network is trained to carry out conditional diffusion on a synthetic dataset (refer to Fig.~\ref{fig:toy-example}). The dataset is designed to show low local dimensionality, characterized by a highly anisotropic structure with limited support, and the gradual unveiling of local details as noise is reduced. These features are expected to mimic those present in the actual manifold of real-world images \cite{brown2022verifying}.
To illustrate why unguided diffusion models often produce poor images and how CFG mitigates this, as discussed in \cite{karras2024guiding}, the authors present a 2D toy example. A simple denoiser is trained for conditional diffusion on a synthetic dataset (Fig.~\ref{fig:toy-example}), designed with low local dimensionality and anisotropic structure. As noise decreases, local details emerge, mimicking real-world image manifolds \cite{brown2022verifying}.

In contrast to direct sampling from the original distribution (as depicted in Fig.~\ref{fig:toy-example} (a)), the unguided diffusion approach illustrated in Fig. 1b yields a significant quantity of highly improbable samples that lie beyond the main part of the distribution. In the context of generating images, these would equate to distorted or inadequate images Fig.~\ref{fig:toy-example} (a) and (b) display the learned score field and implied density in our illustrative example for two models with different capacities at a mid-level of noise. The classical CFG model encapsulates the data more closely, whereas the weaker model without guidance exhibits a more dispersed density.
\our{} model fits the target distribution more precisely than CDF. Additionally, it generates fewer outlier elements, as depicted in Fig.~\ref{fig:toy-example} (c).

\paragraph{Text to image generation}
In this experiment, we evaluate the quality of generated images (FID score) and match the prompt (CLIP score).
Utilizing specific scales for $\omega$ and $\lambda$, we directly compare the T2I task performance of SD v1.5. Tab.~\ref{tab:all} provides quantitative metrics based on 10,000 images created with COCO captions \cite{lin2014microsoft}. In practical application, \our{} achieves an improved FID score, as shown in Tab.~\ref{tab:all}, with a similar CLIP score.  Fig.~\ref{fig:sampling}) present samples generated from SDXL model.
% displays visual comparisons between CFG and \our{}.

\paragraph{Ablation study}
In \our{}, two significant parameters are employed. The first is the $\beta$-distribution utilized in the experiment. Fig.~\ref{fig:sampling} show relation between FID and CLIP score. The model with $\beta(2,2)$ parameters achieves the highest score. 
% Fig.~\ref{fig:sampling} provides a visualization of these parameters, illustrating that $\beta(2,2)$ yields the most realistic effects without exaggerated colors.
In Fig. \ref{fig:sampling_example2}, we illustrate the impact of the $\beta$ parameters on the sampling process. When $\gamma$ equals 1, the trajectory aligns precisely with the $\beta$-distribution. When $\gamma$ lies between 0 and 1, it modifies the intermediate phase of the diffusion process. For $\gamma$ values exceeding 1, it becomes evident that the guidance is overly strong.



% \begin{table}[h]
%     \centering
%     \begin{tabular}{llc}
%         \toprule
%         Model & Metric & CFG \\
%         \midrule
%         \multirow{3}{*}{\our{}(2.0, 2.0)} 
%         & FID$\downarrow$ & ?? \\
%         & CLIP$\uparrow$ & ?? \\
%         & ImageReward$\uparrow$ & ?? \\
%         \midrule
%         \multirow{3}{*}{\our{}(2.0, 2.5)} 
%         & FID$\downarrow$ & ?? \\
%         & CLIP$\uparrow$ & ?? \\
%         & ImageReward$\uparrow$ & ?? \\
%         \midrule
%         \multirow{3}{*}{\our{}(2.0, 3.0)} 
%         & FID$\downarrow$ & ?? \\
%         & CLIP$\uparrow$ & ?? \\
%         & ImageReward$\uparrow$ & ?? \\
%         \midrule
%         \multirow{3}{*}{\our{}(2.5, 2.0)} 
%         & FID$\downarrow$ & ?? \\
%         & CLIP$\uparrow$ & ?? \\
%         & ImageReward$\uparrow$ & ?? \\
%         \midrule
%         \multirow{3}{*}{\our{}(3.0, 2.0)} 
%         & FID$\downarrow$ & ?? \\
%         & CLIP$\uparrow$ & ?? \\
%         & ImageReward$\uparrow$ & ?? \\
%         \bottomrule
%     \end{tabular}
% \end{table}



\section{Conclusions}

In this paper, we explored the impact of classifier-free guidance (CFG) on text-driven diffusion models, highlighting its trade-off between image quality and prompt adherence. We analyzed how CFG behaves across different noise levels, influencing the sampling process at various stages. To address the inherent limitations of CFG, we introduced \our{} ($\beta$-adaptive scaling in Classifier-Free Guidance), which dynamically adjusts guidance strength throughout the generation process. By employing time-dependent $\beta$-distribution scaling, \our{} effectively balances prompt alignment and image fidelity. Experimental results demonstrated that our approach achieves improved FID scores while maintaining text-to-image CLIP similarity comparable to standard CFG. 
% These findings underscore the potential of adaptive guidance strategies in enhancing diffusion model performance.
\paragraph{Limitations} The primary drawback is the introduction of three extra meta-parameters. For future work, we intend to develop a mechanism for automatic parameter matching.


\section*{Impact Statement}

This paper presents work whose goal is to advance the field of Machine Learning. There are many potential societal consequences of our work, none which we feel must be specifically highlighted here.









% \section{Introduction}
Backdoor attacks pose a concealed yet profound security risk to machine learning (ML) models, for which the adversaries can inject a stealth backdoor into the model during training, enabling them to illicitly control the model's output upon encountering predefined inputs. These attacks can even occur without the knowledge of developers or end-users, thereby undermining the trust in ML systems. As ML becomes more deeply embedded in critical sectors like finance, healthcare, and autonomous driving \citep{he2016deep, liu2020computing, tournier2019mrtrix3, adjabi2020past}, the potential damage from backdoor attacks grows, underscoring the emergency for developing robust defense mechanisms against backdoor attacks.

To address the threat of backdoor attacks, researchers have developed a variety of strategies \cite{liu2018fine,wu2021adversarial,wang2019neural,zeng2022adversarial,zhu2023neural,Zhu_2023_ICCV, wei2024shared,wei2024d3}, aimed at purifying backdoors within victim models. These methods are designed to integrate with current deployment workflows seamlessly and have demonstrated significant success in mitigating the effects of backdoor triggers \cite{wubackdoorbench, wu2023defenses, wu2024backdoorbench,dunnett2024countering}.  However, most state-of-the-art (SOTA) backdoor purification methods operate under the assumption that a small clean dataset, often referred to as \textbf{auxiliary dataset}, is available for purification. Such an assumption poses practical challenges, especially in scenarios where data is scarce. To tackle this challenge, efforts have been made to reduce the size of the required auxiliary dataset~\cite{chai2022oneshot,li2023reconstructive, Zhu_2023_ICCV} and even explore dataset-free purification techniques~\cite{zheng2022data,hong2023revisiting,lin2024fusing}. Although these approaches offer some improvements, recent evaluations \cite{dunnett2024countering, wu2024backdoorbench} continue to highlight the importance of sufficient auxiliary data for achieving robust defenses against backdoor attacks.

While significant progress has been made in reducing the size of auxiliary datasets, an equally critical yet underexplored question remains: \emph{how does the nature of the auxiliary dataset affect purification effectiveness?} In  real-world  applications, auxiliary datasets can vary widely, encompassing in-distribution data, synthetic data, or external data from different sources. Understanding how each type of auxiliary dataset influences the purification effectiveness is vital for selecting or constructing the most suitable auxiliary dataset and the corresponding technique. For instance, when multiple datasets are available, understanding how different datasets contribute to purification can guide defenders in selecting or crafting the most appropriate dataset. Conversely, when only limited auxiliary data is accessible, knowing which purification technique works best under those constraints is critical. Therefore, there is an urgent need for a thorough investigation into the impact of auxiliary datasets on purification effectiveness to guide defenders in  enhancing the security of ML systems. 

In this paper, we systematically investigate the critical role of auxiliary datasets in backdoor purification, aiming to bridge the gap between idealized and practical purification scenarios.  Specifically, we first construct a diverse set of auxiliary datasets to emulate real-world conditions, as summarized in Table~\ref{overall}. These datasets include in-distribution data, synthetic data, and external data from other sources. Through an evaluation of SOTA backdoor purification methods across these datasets, we uncover several critical insights: \textbf{1)} In-distribution datasets, particularly those carefully filtered from the original training data of the victim model, effectively preserve the model’s utility for its intended tasks but may fall short in eliminating backdoors. \textbf{2)} Incorporating OOD datasets can help the model forget backdoors but also bring the risk of forgetting critical learned knowledge, significantly degrading its overall performance. Building on these findings, we propose Guided Input Calibration (GIC), a novel technique that enhances backdoor purification by adaptively transforming auxiliary data to better align with the victim model’s learned representations. By leveraging the victim model itself to guide this transformation, GIC optimizes the purification process, striking a balance between preserving model utility and mitigating backdoor threats. Extensive experiments demonstrate that GIC significantly improves the effectiveness of backdoor purification across diverse auxiliary datasets, providing a practical and robust defense solution.

Our main contributions are threefold:
\textbf{1) Impact analysis of auxiliary datasets:} We take the \textbf{first step}  in systematically investigating how different types of auxiliary datasets influence backdoor purification effectiveness. Our findings provide novel insights and serve as a foundation for future research on optimizing dataset selection and construction for enhanced backdoor defense.
%
\textbf{2) Compilation and evaluation of diverse auxiliary datasets:}  We have compiled and rigorously evaluated a diverse set of auxiliary datasets using SOTA purification methods, making our datasets and code publicly available to facilitate and support future research on practical backdoor defense strategies.
%
\textbf{3) Introduction of GIC:} We introduce GIC, the \textbf{first} dedicated solution designed to align auxiliary datasets with the model’s learned representations, significantly enhancing backdoor mitigation across various dataset types. Our approach sets a new benchmark for practical and effective backdoor defense.



% \section{Related work}
\label{sec:formatting}

\subsection{Text-to-Video Generation}

T2V generation has made notable progress, evolving from early GAN-based models \cite{saito2017temporal,tulyakov2018mocogan,fu2023tell,li2018video,wu2022nuwa,yu2022generating} to newer transformer \cite{yan2021videogpt,arnab2021vivit,esser2021taming,ramesh2021zero,yu2022scaling} and diffusion models \cite{kirkpatrick2017overcoming,sohl2015deep,song2020denoising,zhang2022gddim}. Early efforts like MoCoGAN~\cite{tulyakov2018mocogan} focused on short video clips but faced issues with stability and coherence. The introduction of transformers improved sequential data handling, enhancing video generation, while diffusion models further improved video quality by progressively denoising the input. 
Despite these advances, T2V models still struggle to reflect human preferences, with the generated videos generally lacking aesthetic quality. Additionally, the scarcity of paired video preference data hinders effective model training and may lead to insufficient flexibility and poor quality in the generated videos.


\subsection{RLHF}

\iffalse
Aligning LLMs \cite{dai1901transformer,radford2019language,zhang2023opt} typically involves two steps: supervised fine-tuning followed by Reinforcement Learning with Human Feedback (RLHF) \cite{gao2023scaling,stiennon2020learning,rafailov2024direct}. Although effective, RLHF is computationally expensive and can lead to issues like reward hacking. Methods like DPO have streamlined alignment by leveraging feedback data directly, improving efficiency.

In contrast, diffusion model alignment is still evolving, focusing mainly on enhancing visual quality through curated datasets. Techniques like DOODL \cite{wallace2023end} and AlignProp \cite{prabhudesai2023aligning} target aesthetic improvements but face challenges with complex tasks such as text-image alignment. Reinforcement learning methods like DPOK \cite{fan2024reinforcement} and DDPO \cite{black2023training} improve reward optimization but struggle with scalability. DPO-SDXL integrates DPO into T2I generation, boosting both alignment and aesthetics.

However, aligning video generation remains a largely unaddressed challenge, especially when dealing with motion consistency and semantic coherence across frames.
\fi

RLHF \cite{gao2023scaling,stiennon2020learning,rafailov2024direct} is a method that utilizes human feedback to guide machine learning models. Early RLHF algorithms, such as DDPG~\cite{lillicrap2015continuous} and PPO~\cite{schulman2017proximal}, typically relied on complex reward models to quantify human feedback. These reward models require a large amount of annotated data and face challenges during tuning. As research has progressed, more efficient preference learning methods have emerged, among which DPO has become a new framework. DPO does not depend on a separate reward model; instead, it obtains human preferences through pairwise comparisons and directly optimizes these preferences. This shift not only simplifies the application of RLHF but also enhances the alignment of models with human values. Furthermore, DPO has been successfully introduced into T2I tasks~\cite{wallace2024diffusion,yang2024using}, providing new insights for generative models in addressing the alignment of human preferences and showcasing DPO's potential in the field of AIGC~\cite{shi2024instantbooth,
qing2024hierarchical,menapace2024snap,koley2024s}. However, there remains a gap in current research regarding the application of DPO strategies to T2V tasks. Effectively integrating DPO into T2V tasks presents a challenging endeavor.


% \section{Preliminary}
\label{sec:preliminary}
In this section, we first introduce the mathematical formulation of flow-based text-to-image generative models~\cite{Xingchao_2022,Lipman_2022}, which forms the foundation of modern T2I systems~\cite{sd3,sdxl,imagen3,imagen}. We then describe classifier-free guidance~\cite{ho2022classifier}, a key technique to control the generation process through text conditioning.

\subsection{Flow-based text-to-image generative models}
In state-of-the-art T2I models~\cite{sd3}, the image generation process is modeled by learning, through a neural network, a flow $\psi$ that generates a probability path $(p_t)_{0\le t\le 1}$ bridging the source distribution $p_0$ and the target distribution $p_1$ ~\cite{Xingchao_2022,Lipman_2022}. This framework encompasses diffusion models~\cite{sohl2015deep,ddpm} as a special case. In particular, a commonly used formulation sets a Gaussian distribution as the source: $p_0 = \mathcal{N}(\mathbf{0}, \mathbf{I})$ and a delta distribution centered on a sample $\mathbf{x}_1$ from the data distribution $q$ as the target: $p_1 = \delta_{\mathbf{x}_1}$.
Then, it incorporates an affine conditional flow $\psi_t(\mathbf{x} | \mathbf{x}_1) = a_t \mathbf{x}_1 + b_t \mathbf{x}$ with the boundary condition $(a_0, b_0) = (0, 1),\ (a_1, b_1) = (1, 0)$ to bridge them. The neural network typically approximates quantities such as velocity fields, $x_0$ prediction or $x_1$ prediction. In this modeling, these quantities can be viewed as affine transformations of the marginal probability path score $\nabla_{\mathbf{x}} \log p_t(\mathbf{x})$.

\subsection{Classifier-free guidance in flow-based models}
Classifier-free guidance~\cite{ho2022classifier} is a method for sampling from a model conditioned by a text input $\mathbf{y}$ by guiding an unconditional image generation model modeled using the score $\nabla_{\mathbf{x}} \log p_t(\mathbf{x})$. This enables the sampling from
\[
q_w(\mathbf{x}, \mathbf{y}) \propto q(\mathbf{x})q(\mathbf{y}|\mathbf{x})^w \propto q(\mathbf{x})^{1-w}q(\mathbf{x}|\mathbf{y})^w
\]
where $w \in \mathbb{R}$ is the guidance scale typically used with $w > 1$. The score satisfies
\[
\nabla_{\mathbf{x}} \log q_w(\mathbf{x}, \mathbf{y}) = (1-w)\nabla_{\mathbf{x}} \log q(\mathbf{x}) + w\nabla_{\mathbf{x}} \log q(\mathbf{x}|\mathbf{y})
\]
so by training the network to learn both the unconditional score $\nabla_{\mathbf{x}} \log q(\mathbf{x})$ and conditional score $\nabla_{\mathbf{x}} \log q(\mathbf{x}|\mathbf{y})$, flexible sampling from the conditional distribution can be achieved through a weighted sum of the network outputs.

% \begin{table*}[h!]
%     \centering
%     \begin{tabular}{|c|cc|cc|cc|cc|cc|}
%         \hline
%         \multirow{2}{*}{Method} & \multicolumn{2}{c|}{$\omega = 2.0, \lambda = 0.2$} & \multicolumn{2}{c|}{$\omega = 5.0, \lambda = 0.4$} & \multicolumn{2}{c|}{$\omega = 7.5, \lambda = 0.6$} & \multicolumn{2}{c|}{$\omega = 9.0, \lambda = 0.8$} & \multicolumn{2}{c|}{$\omega = 12.5, \lambda = 1.0$} \\
%         & FID $\downarrow$ & CLIP $\uparrow$ & FID $\downarrow$ & CLIP $\uparrow$ & FID $\downarrow$ & CLIP $\uparrow$ & FID $\downarrow$ & CLIP $\uparrow$ & FID $\downarrow$ & CLIP $\uparrow$ \\
%         \hline
%         CFG & 13.38 & 0.301 & 15.86 & \bf0.313 & 18.90 & 0.314 & 20.15 & 0.315 & 22.39 & \bf{0.315} \\
%         CFG++ & \bf12.78 & 0.306 & 15.68 & \bf0.313 & 18.48 & \bf{0.315} & 20.46 & 0.314 & {21.97} & \bf{0.315} \\
%         \hline
%         CFG-GEO & {13.29} & \bf{0.308} & \bf15.63 & \bf0.313 & \bf18.04 & \bf{0.315} & \bf{18.93} & {0.313} & \bf{21.32} & \bf{0.315} \\
%         \hline

%         \hline
%         CFG* & 13.95 & 0.306 & 16.17 & 0.318 & 18.99 & 0.319 & 20.16 & 0.320 & 22.33 & 0.320 \\
%         CFG++* & 13.36 & 0.311 & 16.02 & 0.318 & 18.58 & 0.319 & 20.49 & 0.320 & 21.97 & 0.320 \\
%         \our{} & 13.98 & 0.314 & 15.68 & 0.317 & 18.50 & 0.319 & 19.90 & 0.319 & 21.11 & 0.319

        

        
%     \end{tabular}
%     \caption{Quantitative evaluation (FID, CLIP-similarity) of T2I with SD v1.5}
% \end{table*}






% \begin{figure*}[t]
%     \centering
%     \includegraphics[width=\linewidth]{img/cfg_ablation.png}
%     \caption{PLACEHOLDER: \przemek{tutaj przyklad 3-4 trajektorii a oryginalnego modelu dla CFG, SFG++. Nasz ze stałą i nasz z $\beta$-distribution}.}
%     \label{fig:trajectorid}
% \end{figure*}



% \begin{figure}
%     \centering
%    \begin{tabular}{cccc}  
%     \multicolumn{4}{c}{\bf CFG} \\
    
%     \rotatebox{90}{ \qquad \bf CFG}   
%     &
%     \includegraphics[width=0.3\linewidth]{example-image-a.png}
%     &
%     \includegraphics[width=0.3\linewidth]{example-image-b.png}
%     &
%     \includegraphics[width=0.3\linewidth]{example-image-c.png} \\
%     \rotatebox{90}{ \qquad \bf \our{}}   
%     &
%     \includegraphics[width=0.3\linewidth]{example-image-a.png}
%     &
%     \includegraphics[width=0.3\linewidth]{example-image-b.png}
%     &
%     \includegraphics[width=0.3\linewidth]{example-image-c.png}
%     \\
%     \multicolumn{4}{c}{\bf CFG++} \\
%     \rotatebox{90}{ \quad \bf CFG++}   
%     &
%     \includegraphics[width=0.3\linewidth]{example-image-a.png}
%     &
%     \includegraphics[width=0.3\linewidth]{example-image-b.png}
%     &
%     \includegraphics[width=0.3\linewidth]{example-image-c.png} \\
%     \rotatebox{90}{ \ \bf \our{}}   
%     &
%     \includegraphics[width=0.3\linewidth]{example-image-a.png}
%     &
%     \includegraphics[width=0.3\linewidth]{example-image-b.png}
%     &
%     \includegraphics[width=0.3\linewidth]{example-image-c.png} \\
% \end{tabular}
%     \caption{Caption}
%     \label{fig:enter-label}
% \end{figure}



% \begin{figure*}[ht]
%     \centering
%     \renewcommand{\arraystretch}{0}
%     \setlength{\tabcolsep}{0pt}
%     \begin{tabular}{ccccc}
%         \includegraphics[width=0.19\linewidth]{example-image-a} & 
%         \includegraphics[width=0.19\linewidth]{example-image-a} & 
%         \includegraphics[width=0.19\linewidth]{example-image-a} & 
%         \includegraphics[width=0.19\linewidth]{example-image-a} & 
%         \includegraphics[width=0.19\linewidth]{example-image-a} & \\
%         \includegraphics[width=0.19\linewidth]{example-image-a} & 
%         \includegraphics[width=0.19\linewidth]{example-image-a} & 
%         \includegraphics[width=0.19\linewidth]{example-image-a} & 
%         \includegraphics[width=0.19\linewidth]{example-image-a} & 
%         \includegraphics[width=0.19\linewidth]{example-image-a} & \\
%         \includegraphics[width=0.19\linewidth]{example-image-a} & 
%         \includegraphics[width=0.19\linewidth]{example-image-a} & 
%         \includegraphics[width=0.19\linewidth]{example-image-a} & 
%         \includegraphics[width=0.19\linewidth]{example-image-a} & 
%         \includegraphics[width=0.19\linewidth]{example-image-a} & \\
%         \includegraphics[width=0.19\linewidth]{example-image-a} & 
%         \includegraphics[width=0.19\linewidth]{example-image-a} & 
%         \includegraphics[width=0.19\linewidth]{example-image-a} & 
%         \includegraphics[width=0.19\linewidth]{example-image-a} & 
%         \includegraphics[width=0.19\linewidth]{example-image-a} & \\
%     \end{tabular}
%     \caption{PLACEHOLDER: Wizualne porównanie SFG, SFG++ i naszej metody.}
%     \label{fig:sampling}
% \end{figure*}




% \begin{table*}[h!]
%     \centering
%     \begin{tabular}{|c|cc|cc|cc|cc|cc|}
%         \hline
%         \multirow{2}{*}{Method} & \multicolumn{2}{c|}{$\omega = 2.0, \lambda = 0.2$} & \multicolumn{2}{c|}{$\omega = 5.0, \lambda = 0.4$} & \multicolumn{2}{c|}{$\omega = 7.5, \lambda = 0.6$} & \multicolumn{2}{c|}{$\omega = 9.0, \lambda = 0.8$} & \multicolumn{2}{c|}{$\omega = 12.5, \lambda = 1.0$} \\
%         & FID $\downarrow$ & CLIP $\uparrow$ & FID $\downarrow$ & CLIP $\uparrow$ & FID $\downarrow$ & CLIP $\uparrow$ & FID $\downarrow$ & CLIP $\uparrow$ & FID $\downarrow$ & CLIP $\uparrow$ \\
%         \hline
%         CFG & 13.84 & 0.298 & 15.08 & 0.310 & 17.71 & 0.312 & 20.01 & 0.312 & 21.23 & 0.313 \\
%         CFG++ & \textbf{12.75} & \textbf{0.303} & \textbf{14.95} & \textbf{0.310} & \textbf{17.47} & \textbf{0.312} & \textbf{19.34} & \textbf{0.313} & \textbf{20.88} & \textbf{0.313} \\
%         \hline
%         CFG-GEO & \textbf{12.75} & \textbf{0.303} & \textbf{14.95} & \textbf{0.310} & \textbf{17.47} & \textbf{0.312} & \textbf{19.34} & \textbf{0.313} & \textbf{20.88} & \textbf{0.313} \\
%         \hline
%     \end{tabular}
%     \caption{Quantitative evaluation (FID, CLIP-similarity) of T2I with SD v1.5}
% \end{table*}



% In the unusual situation where you want a paper to appear in the
% references without citing it in the main text, use \nocite
%\nocite{langley00}

%\bibliography{ref}
\bibliographystyle{icml2025}

\begin{thebibliography}{26}
\providecommand{\natexlab}[1]{#1}
\providecommand{\url}[1]{\texttt{#1}}
\expandafter\ifx\csname urlstyle\endcsname\relax
  \providecommand{\doi}[1]{doi: #1}\else
  \providecommand{\doi}{doi: \begingroup \urlstyle{rm}\Url}\fi

\bibitem[Bansal et~al.(2023)Bansal, Chu, Schwarzschild, Sengupta, Goldblum,
  Geiping, and Goldstein]{bansal2023universal}
Bansal, A., Chu, H.-M., Schwarzschild, A., Sengupta, S., Goldblum, M., Geiping,
  J., and Goldstein, T.
\newblock Universal guidance for diffusion models.
\newblock In \emph{Proceedings of the IEEE/CVF Conference on Computer Vision
  and Pattern Recognition}, pp.\  843--852, 2023.

\bibitem[Bordes et~al.(2017)Bordes, Honari, and Vincent]{bordes2017learning}
Bordes, F., Honari, S., and Vincent, P.
\newblock Learning to generate samples from noise through infusion training.
\newblock \emph{arXiv preprint arXiv:1703.06975}, 2017.

\bibitem[Brown et~al.(2022)Brown, Caterini, Ross, Cresswell, and
  Loaiza-Ganem]{brown2022verifying}
Brown, B.~C., Caterini, A.~L., Ross, B.~L., Cresswell, J.~C., and Loaiza-Ganem,
  G.
\newblock Verifying the union of manifolds hypothesis for image data.
\newblock \emph{arXiv preprint arXiv:2207.02862}, 2022.

\bibitem[Chung et~al.(2023)Chung, Kim, Mccann, Klasky, and
  Ye]{chung2023diffusion}
Chung, H., Kim, J., Mccann, M.~T., Klasky, M.~L., and Ye, J.~C.
\newblock Diffusion posterior sampling for general noisy inverse problems.
\newblock In \emph{The Eleventh International Conference on Learning
  Representations}, 2023.
\newblock URL \url{https://openreview.net/forum?id=OnD9zGAGT0k}.

\bibitem[Chung et~al.(2024)Chung, Kim, Park, Nam, and Ye]{chung2024cfg++}
Chung, H., Kim, J., Park, G.~Y., Nam, H., and Ye, J.~C.
\newblock Cfg++: Manifold-constrained classifier free guidance for diffusion
  models.
\newblock \emph{arXiv preprint arXiv:2406.08070}, 2024.

\bibitem[Croitoru et~al.(2023)Croitoru, Hondru, Ionescu, and
  Shah]{croitoru2023diffusion}
Croitoru, F.-A., Hondru, V., Ionescu, R.~T., and Shah, M.
\newblock Diffusion models in vision: A survey.
\newblock \emph{IEEE Transactions on Pattern Analysis and Machine
  Intelligence}, 45\penalty0 (9):\penalty0 10850--10869, 2023.

\bibitem[Dhariwal \& Nichol(2021)Dhariwal and Nichol]{dhariwal2021diffusion}
Dhariwal, P. and Nichol, A.
\newblock Diffusion models beat gans on image synthesis.
\newblock \emph{Advances in neural information processing systems},
  34:\penalty0 8780--8794, 2021.

\bibitem[Dinh et~al.(2014)Dinh, Krueger, and Bengio]{dinh2014nice}
Dinh, L., Krueger, D., and Bengio, Y.
\newblock Nice: Non-linear independent components estimation.
\newblock \emph{arXiv preprint arXiv:1410.8516}, 2014.

\bibitem[Grathwohl et~al.(2018)Grathwohl, Chen, Bettencourt, Sutskever, and
  Duvenaud]{grathwohl2018ffjord}
Grathwohl, W., Chen, R.~T., Bettencourt, J., Sutskever, I., and Duvenaud, D.
\newblock Ffjord: Free-form continuous dynamics for scalable reversible
  generative models.
\newblock \emph{arXiv preprint arXiv:1810.01367}, 2018.

\bibitem[Ho \& Salimans(2022)Ho and Salimans]{ho2022classifier}
Ho, J. and Salimans, T.
\newblock Classifier-free diffusion guidance.
\newblock \emph{arXiv preprint arXiv:2207.12598}, 2022.

\bibitem[Ho et~al.(2020)Ho, Jain, and Abbeel]{ho2020denoising}
Ho, J., Jain, A., and Abbeel, P.
\newblock Denoising diffusion probabilistic models.
\newblock \emph{Advances in neural information processing systems},
  33:\penalty0 6840--6851, 2020.

\bibitem[Hong et~al.(2023)Hong, Lee, Jang, and Kim]{hong2023improving}
Hong, S., Lee, G., Jang, W., and Kim, S.
\newblock Improving sample quality of diffusion models using self-attention
  guidance.
\newblock In \emph{Proceedings of the IEEE/CVF International Conference on
  Computer Vision}, pp.\  7462--7471, 2023.

\bibitem[Karras et~al.(2024)Karras, Aittala, Kynk{\"a}{\"a}nniemi, Lehtinen,
  Aila, and Laine]{karras2024guiding}
Karras, T., Aittala, M., Kynk{\"a}{\"a}nniemi, T., Lehtinen, J., Aila, T., and
  Laine, S.
\newblock Guiding a diffusion model with a bad version of itself.
\newblock \emph{arXiv preprint arXiv:2406.02507}, 2024.

\bibitem[Kim et~al.(2022)Kim, Kim, Kwon, Kang, and Moon]{kim2022refining}
Kim, D., Kim, Y., Kwon, S.~J., Kang, W., and Moon, I.-C.
\newblock Refining generative process with discriminator guidance in
  score-based diffusion models.
\newblock \emph{arXiv preprint arXiv:2211.17091}, 2022.

\bibitem[Kynk{\"a}{\"a}nniemi et~al.(2024)Kynk{\"a}{\"a}nniemi, Aittala,
  Karras, Laine, Aila, and Lehtinen]{kynkaanniemi2024applying}
Kynk{\"a}{\"a}nniemi, T., Aittala, M., Karras, T., Laine, S., Aila, T., and
  Lehtinen, J.
\newblock Applying guidance in a limited interval improves sample and
  distribution quality in diffusion models.
\newblock \emph{arXiv preprint arXiv:2404.07724}, 2024.

\bibitem[Lin et~al.(2014)Lin, Maire, Belongie, Hays, Perona, Ramanan,
  Doll{\'a}r, and Zitnick]{lin2014microsoft}
Lin, T.-Y., Maire, M., Belongie, S., Hays, J., Perona, P., Ramanan, D.,
  Doll{\'a}r, P., and Zitnick, C.~L.
\newblock Microsoft coco: Common objects in context.
\newblock In \emph{Computer Vision--ECCV 2014: 13th European Conference,
  Zurich, Switzerland, September 6-12, 2014, Proceedings, Part V 13}, pp.\
  740--755. Springer, 2014.

\bibitem[Podell et~al.(2023)Podell, English, Lacey, Blattmann, Dockhorn,
  M{\"u}ller, Penna, and Rombach]{podell2023sdxl}
Podell, D., English, Z., Lacey, K., Blattmann, A., Dockhorn, T., M{\"u}ller,
  J., Penna, J., and Rombach, R.
\newblock Sdxl: Improving latent diffusion models for high-resolution image
  synthesis.
\newblock \emph{arXiv preprint arXiv:2307.01952}, 2023.

\bibitem[Poleski et~al.(2024)Poleski, Tabor, and Spurek]{poleski2024geoguide}
Poleski, M., Tabor, J., and Spurek, P.
\newblock Geoguide: Geometric guidance of diffusion models.
\newblock \emph{arXiv preprint arXiv:2407.12889}, 2024.

\bibitem[Rezende \& Mohamed(2015)Rezende and Mohamed]{rezende2015variational}
Rezende, D. and Mohamed, S.
\newblock Variational inference with normalizing flows.
\newblock In \emph{International conference on machine learning}, pp.\
  1530--1538. PMLR, 2015.

\bibitem[Rombach et~al.(2022)Rombach, Blattmann, Lorenz, Esser, and
  Ommer]{rombach2022high}
Rombach, R., Blattmann, A., Lorenz, D., Esser, P., and Ommer, B.
\newblock High-resolution image synthesis with latent diffusion models.
\newblock In \emph{Proceedings of the IEEE/CVF conference on computer vision
  and pattern recognition}, pp.\  10684--10695, 2022.

\bibitem[Sendera et~al.(2024)Sendera, Kim, Mittal, Lemos, Scimeca,
  Rector-Brooks, Adam, Bengio, and Malkin]{sendera2024diffusion}
Sendera, M., Kim, M., Mittal, S., Lemos, P., Scimeca, L., Rector-Brooks, J.,
  Adam, A., Bengio, Y., and Malkin, N.
\newblock On diffusion models for amortized inference: Benchmarking and
  improving stochastic control and sampling.
\newblock \emph{arXiv preprint arXiv:2402.05098}, 2024.

\bibitem[Sohl-Dickstein et~al.(2015)Sohl-Dickstein, Weiss, Maheswaranathan, and
  Ganguli]{sohl2015deep}
Sohl-Dickstein, J., Weiss, E., Maheswaranathan, N., and Ganguli, S.
\newblock Deep unsupervised learning using nonequilibrium thermodynamics.
\newblock In \emph{International conference on machine learning}, pp.\
  2256--2265. PMLR, 2015.

\bibitem[Song et~al.(2020)Song, Meng, and Ermon]{song2020denoising}
Song, J., Meng, C., and Ermon, S.
\newblock Denoising diffusion implicit models.
\newblock \emph{arXiv preprint arXiv:2010.02502}, 2020.

\bibitem[Song et~al.(2023)Song, Zhang, Yin, Mardani, Liu, Kautz, Chen, and
  Vahdat]{song2023loss}
Song, J., Zhang, Q., Yin, H., Mardani, M., Liu, M.-Y., Kautz, J., Chen, Y., and
  Vahdat, A.
\newblock Loss-guided diffusion models for plug-and-play controllable
  generation.
\newblock In \emph{International Conference on Machine Learning}, pp.\
  32483--32498. PMLR, 2023.

\bibitem[Song \& Ermon(2019)Song and Ermon]{song2019generative}
Song, Y. and Ermon, S.
\newblock Generative modeling by estimating gradients of the data distribution.
\newblock \emph{Advances in neural information processing systems}, 32, 2019.

\bibitem[Zhang \& Chen(2021)Zhang and Chen]{zhang2021path}
Zhang, Q. and Chen, Y.
\newblock Path integral sampler: a stochastic control approach for sampling.
\newblock \emph{arXiv preprint arXiv:2111.15141}, 2021.

\end{thebibliography}


%%%%%%%%%%%%%%%%%%%%%%%%%%%%%%%%%%%%%%%%%%%%%%%%%%%%%%%%%%%%%%%%%%%%%%%%%%%%%%%
%%%%%%%%%%%%%%%%%%%%%%%%%%%%%%%%%%%%%%%%%%%%%%%%%%%%%%%%%%%%%%%%%%%%%%%%%%%%%%%
% APPENDIX
%%%%%%%%%%%%%%%%%%%%%%%%%%%%%%%%%%%%%%%%%%%%%%%%%%%%%%%%%%%%%%%%%%%%%%%%%%%%%%%
%%%%%%%%%%%%%%%%%%%%%%%%%%%%%%%%%%%%%%%%%%%%%%%%%%%%%%%%%%%%%%%%%%%%%%%%%%%%%%%
\newpage
\appendix
\onecolumn
\section{Ablation study.}

This section presents an ablation study based on the parameters of the $\beta$-distribution. The quantitative results are listed in Tab.~\ref{tab:able} and are illustrated in Fig.~\ref{fig:ablation_cfg_beta}. Fig.~\ref{fig:ap1}, \ref{fig:ap2}, \ref{fig:ap3}, \ref{fig:ap4}.
Tab~\ref{tab:able2} displays the ablation studies' results on the $\gamma$ parameter. Figure~\ref{fig:sampling_gamma} provides a visual comparison of various $\gamma$ parameter values.

\begin{table*}[h!]
    \centering
    \begin{tabular}{ccccccccccc}
        \hline
        \multirow{2}{*}{Method} & \multicolumn{2}{c}{$\omega = 2.0, \lambda = 0.2$} & \multicolumn{2}{c}{$\omega = 5.0, \lambda = 0.4$} & \multicolumn{2}{c}{$\omega = 7.5, \lambda = 0.6$} & \multicolumn{2}{c}{$\omega = 9.0, \lambda = 0.8$} & \multicolumn{2}{c}{$\omega = 12.5, \lambda = 1.0$} \\
        & FID $\downarrow$ & CLIP $\uparrow$ & FID $\downarrow$ & CLIP $\uparrow$ & FID $\downarrow$ & CLIP $\uparrow$ & FID $\downarrow$ & CLIP $\uparrow$ & FID $\downarrow$ & CLIP $\uparrow$ \\
        \hline
        \our (2.0, 2.0)& 15.02 & 0.306
& 15.76 & 0.317
& 17.99 & 0.319
& 18.94 & 0.319
& 20.97 & 0.320
\\
\our (2.0, 2.5)& 14.18 & 0.308
& 16.80 & 0.318
& 18.99 & 0.319
& 20.24 & 0.319
& 22.24 & 0.320
\\
\our (2.0, 3.0)& 13.85 & 0.309
& 17.51 & 0.317
& 19.90 & 0.319
& 21.03 & 0.319
& 23.26 & 0.319
\\
\our (2.5, 2.0)& 17.26 & 0.303
& 15.18 & 0.316
& 16.73 & 0.318
& 17.71 & 0.319
& 19.46 & 0.320
\\
\our (2.5, 2.5)& 15.31 & 0.306
& 15.62 & 0.317
& 17.80 & 0.319
& 18.76 & 0.319
& 20.62 & 0.319
\\
\our (2.5, 3.0)& 14.58 & 0.307
& 16.36 & 0.317
& 18.63 & 0.319
& 19.67 & 0.319
& 21.84 & 0.319
\\
\our (3.0, 2.0)& 20.11 & 0.300
& 15.16 & 0.314
& 16.38 & 0.317
& 17.00 & 0.318
& 18.62 & 0.319
\\
\our (3.0, 2.5)& 17.35 & 0.303
& 15.39 & 0.316
& 16.89 & 0.318
& 17.89 & 0.318
& 19.73 & 0.319
\\
\our (3.0, 3.0)& 15.70 & 0.305
& 15.77 & 0.316
& 17.73 & 0.318
& 18.75 & 0.319
& 20.72 & 0.319
\\
        \hline
    \end{tabular}
    \caption{Ablation study of $\beta$-distribution parameters of T2I with SD v1.5}
    \label{tab:able}
\end{table*}


\begin{table*}[h!]
    \centering
    \begin{tabular}{ccccccccccc}
        \hline
        \multirow{2}{*}{Method} & \multicolumn{2}{c}{$\omega = 2.0, \lambda = 0.2$} & \multicolumn{2}{c}{$\omega = 5.0, \lambda = 0.4$} & \multicolumn{2}{c}{$\omega = 7.5, \lambda = 0.6$} & \multicolumn{2}{c}{$\omega = 9.0, \lambda = 0.8$} & \multicolumn{2}{c}{$\omega = 12.5, \lambda = 1.0$} \\
        & FID $\downarrow$ & CLIP $\uparrow$ & FID $\downarrow$ & CLIP $\uparrow$ & FID $\downarrow$ & CLIP $\uparrow$ & FID $\downarrow$ & CLIP $\uparrow$ & FID $\downarrow$ & CLIP $\uparrow$ \\
        \hline
        $\gamma= 0.0$ & 195.72 & 0.308
        & 185.55 & 0.318
        & 185.27 & 0.320
        & 185.09 & 0.320
        & 185.76 & 0.321 \\
        $\gamma= 0.25$
        & 201.32 & 0.303
        & 185.81 & 0.317
        & 185.05 & 0.319
        & 185.36 & 0.320
        & 185.15 & 0.320 \\
        $\gamma= 0.50$ 
        & 210.74 & 0.293
        & 186.57 & 0.316
        & 185.91 & 0.319
        & 184.97 & 0.319
        & 184.67 & 0.320 \\
        $\gamma= 0.75$ 
        & 229.62 & 0.272
        & 188.86 & 0.313
        & 186.19 & 0.318
        & 185.08 & 0.319
        & 184.37 & 0.320 \\
        $\gamma= 1.0$ 
        & 248.45 & 0.238
        & 192.94 & 0.309
        & 186.95 & 0.316
        & 185.12 & 0.318
        & 184.39 & 0.319
\\
        \hline
    \end{tabular}
    \caption{Ablation study of $\gamma$ of T2I with SD v1.5. The metrics were computed based on 1k prompts.}
    \label{tab:able2}
\end{table*}

\begin{figure*}[!h]
    \centering
    \renewcommand{\arraystretch}{0}
    \setlength{\tabcolsep}{0.4pt}
    \begin{tabular}{c@{}c@{}c@{}c@{}c@{}c@{}}
        
        \includegraphics[width=0.15\linewidth]{img/images/beta/beta_distribution_single_2.0_2.0.jpg} &
        \includegraphics[width=0.15\linewidth]{img/images/gammav3/0.0.jpg} &
        \includegraphics[width=0.15\linewidth]{img/images/gammav3/0.25.jpg} &
        \includegraphics[width=0.15\linewidth]{img/images/gammav3/0.5.jpg} &
        \includegraphics[width=0.15\linewidth]{img/images/gammav3/1.0.jpg} &
        \includegraphics[width=0.15\linewidth]{img/images/gammav3/2.0.jpg} \\
        \includegraphics[width=0.15\linewidth]{img/images/beta/beta_distribution_single_2.0_2.0.jpg} &
        \includegraphics[width=0.15\linewidth]{img/images/gammav4/0.0.jpg} &
        \includegraphics[width=0.15\linewidth]{img/images/gammav4/0.25.jpg} &
        \includegraphics[width=0.15\linewidth]{img/images/gammav4/0.5.jpg} &
        \includegraphics[width=0.15\linewidth]{img/images/gammav4/1.0.jpg} &
        \includegraphics[width=0.15\linewidth]{img/images/gammav4/2.0.jpg} \\
        \includegraphics[width=0.15\linewidth]{img/images/beta/beta_distribution_single_2.0_2.0.jpg} &
        \includegraphics[width=0.15\linewidth]{img/images/gammav5/0.0.jpg} &
        \includegraphics[width=0.15\linewidth]{img/images/gammav5/0.25.jpg} &
        \includegraphics[width=0.15\linewidth]{img/images/gammav5/0.5.jpg} &
        \includegraphics[width=0.15\linewidth]{img/images/gammav5/1.0.jpg} &
        \includegraphics[width=0.15\linewidth]{img/images/gammav5/2.0.jpg} \\[0.2cm]
        \our{} & $\gamma = 0.0 $ & $\gamma = 0.25 $ & $\gamma = 0.5 $  & $\gamma = 1.0 $ & $\gamma = 2.0 $ \\
    \end{tabular}
    \caption{Example of sampled element according to $\gamma$ parameters. Prompts: "A boat is parked ashore without a passenger.", "A man sticking his head out of a doorway into a rainy city street.", "A kitten on a desk with an open sandwich and apple.".}
    \label{fig:sampling_gamma}
\end{figure*}



\begin{figure*}[!h]
    \centering
    \renewcommand{\arraystretch}{0}
    \setlength{\tabcolsep}{0pt}
\includegraphics[width=0.9\linewidth]{img/img_4.png}
%    \begin{tabular}{cccccc}
%        \includegraphics[width=0.15\linewidth]{img/images/beta/beta_distribution_single_2.0_2.0.jpg} &
%        \includegraphics[width=0.15\linewidth]{img/images/grid/1/0_0.jpg} & 
%        \includegraphics[width=0.15\linewidth]{img/images/grid/1/0_1.jpg} & 
%        \includegraphics[width=0.15\linewidth]{img/images/grid/1/0_2.jpg} & 
%        \includegraphics[width=0.15\linewidth]{img/images/grid/1/0_3.jpg} &
%        \includegraphics[width=0.15\linewidth]{img/images/grid/1/0_4.jpg} \\
%        \includegraphics[width=0.15\linewidth]{img/images/beta/beta_distribution_single_2.0_2.5.jpg} & 
%        \includegraphics[width=0.15\linewidth]{img/images/grid/1/1_0.jpg} & 
%        \includegraphics[width=0.15\linewidth]{img/images/grid/1/1_1.jpg} & 
%        \includegraphics[width=0.15\linewidth]{img/images/grid/1/1_2.jpg} & 
%        \includegraphics[width=0.15\linewidth]{img/images/grid/1/1_3.jpg} &
%        \includegraphics[width=0.15\linewidth]{img/images/grid/1/1_4.jpg} \\
%        \includegraphics[width=0.15\linewidth]{img/images/beta/beta_distribution_single_2.0_3.0.jpg} & 
%        \includegraphics[width=0.15\linewidth]{img/images/grid/1/2_0.jpg} & 
%        \includegraphics[width=0.15\linewidth]{img/images/grid/1/2_1.jpg} & 
%        \includegraphics[width=0.15\linewidth]{img/images/grid/1/2_2.jpg} & 
%        \includegraphics[width=0.15\linewidth]{img/images/grid/1/2_3.jpg} &
%        \includegraphics[width=0.15\linewidth]{img/images/grid/1/2_4.jpg} \\
%
%        \includegraphics[width=0.15\linewidth]{img/images/beta/beta_distribution_single_2.0_4.0.jpg} & 
%        \includegraphics[width=0.15\linewidth]{img/images/grid/1/3_0.jpg} & 
%        \includegraphics[width=0.15\linewidth]{img/images/grid/1/3_1.jpg} & 
%        \includegraphics[width=0.15\linewidth]{img/images/grid/1/3_2.jpg} & 
%        \includegraphics[width=0.15\linewidth]{img/images/grid/1/3_3.jpg} &
%        \includegraphics[width=0.15\linewidth]{img/images/grid/1/3_4.jpg} \\[0.05cm]
%        \our{} & 
%        $\omega = 2.0$ & 
%        $\omega = 5.0$ &
%        $\omega = 7.5$ &
%        $\omega = 9.0$ &
%        $\omega = 12.5$ \\
%        
%    \end{tabular}
    \vspace{-0.2cm}
    \caption{Prompt: "kayak in the water, optical color, aerial view, rainbow"}
    \label{fig:ap1}
\end{figure*}


\begin{figure*}[!h]
    \centering
    \renewcommand{\arraystretch}{0}
    \setlength{\tabcolsep}{0pt}
\includegraphics[width=0.9\linewidth]{img/img_3.png}
%    \begin{tabular}{cccccc}
%        
%        \includegraphics[width=0.15\linewidth]{img/images/beta/beta_distribution_single_2.0_2.0.jpg} &
%        \includegraphics[width=0.15\linewidth]{img/images/grid/2/0_0.jpg} & 
%        \includegraphics[width=0.15\linewidth]{img/images/grid/2/0_1.jpg} & 
%        \includegraphics[width=0.15\linewidth]{img/images/grid/2/0_2.jpg} & 
%        \includegraphics[width=0.15\linewidth]{img/images/grid/2/0_3.jpg} &
%        \includegraphics[width=0.15\linewidth]{img/images/grid/2/0_4.jpg} \\
%        \includegraphics[width=0.15\linewidth]{img/images/beta/beta_distribution_single_2.0_2.5.jpg} & 
%        \includegraphics[width=0.15\linewidth]{img/images/grid/2/1_0.jpg} & 
%        \includegraphics[width=0.15\linewidth]{img/images/grid/2/1_1.jpg} & 
%        \includegraphics[width=0.15\linewidth]{img/images/grid/2/1_2.jpg} & 
%        \includegraphics[width=0.15\linewidth]{img/images/grid/2/1_3.jpg} &
%        \includegraphics[width=0.15\linewidth]{img/images/grid/2/1_4.jpg} \\
%        \includegraphics[width=0.15\linewidth]{img/images/beta/beta_distribution_single_2.0_3.0.jpg} & 
%        \includegraphics[width=0.15\linewidth]{img/images/grid/2/2_0.jpg} & 
%        \includegraphics[width=0.15\linewidth]{img/images/grid/2/2_1.jpg} & 
%        \includegraphics[width=0.15\linewidth]{img/images/grid/2/2_2.jpg} & 
%        \includegraphics[width=0.15\linewidth]{img/images/grid/2/2_3.jpg} &
%        \includegraphics[width=0.15\linewidth]{img/images/grid/2/2_4.jpg} \\
%
%        \includegraphics[width=0.15\linewidth]{img/images/beta/beta_distribution_single_2.0_4.0.jpg} & 
%        \includegraphics[width=0.15\linewidth]{img/images/grid/2/3_0.jpg} & 
%        \includegraphics[width=0.15\linewidth]{img/images/grid/2/3_1.jpg} & 
%        \includegraphics[width=0.15\linewidth]{img/images/grid/2/3_2.jpg} & 
%        \includegraphics[width=0.15\linewidth]{img/images/grid/2/3_3.jpg} &
%        \includegraphics[width=0.15\linewidth]{img/images/grid/2/3_4.jpg} 
%        \\[0.05cm]
%        \our{} & 
%        $\omega = 2.0$ & 
%        $\omega = 5.0$ &
%        $\omega = 7.5$ &
%        $\omega = 9.0$ &
%        $\omega = 12.5$ \\
%    \end{tabular}
    \vspace{-0.2cm}
    \caption{Prompt: "A small cactus with a happy face in the sahara desert"}
    \label{fig:ap2}
\end{figure*}


\begin{figure*}[!h]
    \centering
    \renewcommand{\arraystretch}{0}
    \setlength{\tabcolsep}{0pt}
\includegraphics[width=0.9\linewidth]{img/img_2.png}
%    \begin{tabular}{cccccc}
%        % \our{} & 
%        % $\omega = 2.0$ & 
%        % $\omega = 5.0$ &
%        % $\omega = 7.5$ &
%        % $\omega = 9.0$ &
%        % $\omega = 12.5$ \\
%        \includegraphics[width=0.15\linewidth]{img/images/beta/beta_distribution_single_2.0_2.0.jpg} &
%        \includegraphics[width=0.15\linewidth]{img/images/grid/5/0_0.jpg} & 
%        \includegraphics[width=0.15\linewidth]{img/images/grid/5/0_1.jpg} & 
%        \includegraphics[width=0.15\linewidth]{img/images/grid/5/0_2.jpg} & 
%        \includegraphics[width=0.15\linewidth]{img/images/grid/5/0_3.jpg} &
%        \includegraphics[width=0.15\linewidth]{img/images/grid/5/0_4.jpg} \\
%        \includegraphics[width=0.15\linewidth]{img/images/beta/beta_distribution_single_2.0_2.5.jpg} & 
%        \includegraphics[width=0.15\linewidth]{img/images/grid/5/1_0.jpg} & 
%        \includegraphics[width=0.15\linewidth]{img/images/grid/5/1_1.jpg} & 
%        \includegraphics[width=0.15\linewidth]{img/images/grid/5/1_2.jpg} & 
%        \includegraphics[width=0.15\linewidth]{img/images/grid/5/1_3.jpg} &
%        \includegraphics[width=0.15\linewidth]{img/images/grid/5/1_4.jpg} \\
%        \includegraphics[width=0.15\linewidth]{img/images/beta/beta_distribution_single_2.0_3.0.jpg} & 
%        \includegraphics[width=0.15\linewidth]{img/images/grid/5/2_0.jpg} & 
%        \includegraphics[width=0.15\linewidth]{img/images/grid/5/2_1.jpg} & 
%        \includegraphics[width=0.15\linewidth]{img/images/grid/5/2_2.jpg} & 
%        \includegraphics[width=0.15\linewidth]{img/images/grid/5/2_3.jpg} &
%        \includegraphics[width=0.15\linewidth]{img/images/grid/5/2_4.jpg} \\
%
%        \includegraphics[width=0.15\linewidth]{img/images/beta/beta_distribution_single_2.0_4.0.jpg} & 
%        \includegraphics[width=0.15\linewidth]{img/images/grid/5/3_0.jpg} & 
%        \includegraphics[width=0.15\linewidth]{img/images/grid/5/3_1.jpg} & 
%        \includegraphics[width=0.15\linewidth]{img/images/grid/5/3_2.jpg} & 
%        \includegraphics[width=0.15\linewidth]{img/images/grid/5/3_3.jpg} &
%        \includegraphics[width=0.15\linewidth]{img/images/grid/5/3_4.jpg} 
%        \\[0.05cm]
%        \our{} & 
%        $\omega = 2.0$ & 
%        $\omega = 5.0$ &
%        $\omega = 7.5$ &
%        $\omega = 9.0$ &
%        $\omega = 12.5$ \\
%    \end{tabular}
    \vspace{-0.2cm}
    \caption{Prompt: "selfie of a woman and her lion cub on the plains"}
    \label{fig:ap3}
\end{figure*}


\begin{figure*}[!h]
    \centering
    \renewcommand{\arraystretch}{0}
    \setlength{\tabcolsep}{0pt}
\includegraphics[width=0.9\linewidth]{img/img_1.png}
%    \begin{tabular}{cccccc}
%        % \our{} & 
%        % $\omega = 2.0$ & 
%        % $\omega = 5.0$ &
%        % $\omega = 7.5$ &
%        % $\omega = 9.0$ &
%        % $\omega = 12.5$ \\
%        \includegraphics[width=0.15\linewidth]{img/images/beta/beta_distribution_single_2.0_2.0.jpg} &
%        \includegraphics[width=0.15\linewidth]{img/images/grid/4/0_0.jpg} & 
%        \includegraphics[width=0.15\linewidth]{img/images/grid/4/0_1.jpg} & 
%        \includegraphics[width=0.15\linewidth]{img/images/grid/4/0_2.jpg} & 
%        \includegraphics[width=0.15\linewidth]{img/images/grid/4/0_3.jpg} &
%        \includegraphics[width=0.15\linewidth]{img/images/grid/4/0_4.jpg} \\
%        \includegraphics[width=0.15\linewidth]{img/images/beta/beta_distribution_single_2.0_2.5.jpg} & 
%        \includegraphics[width=0.15\linewidth]{img/images/grid/4/1_0.jpg} & 
%        \includegraphics[width=0.15\linewidth]{img/images/grid/4/1_1.jpg} & 
%        \includegraphics[width=0.15\linewidth]{img/images/grid/4/1_2.jpg} & 
%        \includegraphics[width=0.15\linewidth]{img/images/grid/4/1_3.jpg} &
%        \includegraphics[width=0.15\linewidth]{img/images/grid/4/1_4.jpg} \\
%        \includegraphics[width=0.15\linewidth]{img/images/beta/beta_distribution_single_2.0_3.0.jpg} & 
%        \includegraphics[width=0.15\linewidth]{img/images/grid/4/2_0.jpg} & 
%        \includegraphics[width=0.15\linewidth]{img/images/grid/4/2_1.jpg} & 
%        \includegraphics[width=0.15\linewidth]{img/images/grid/4/2_2.jpg} & 
%        \includegraphics[width=0.15\linewidth]{img/images/grid/4/2_3.jpg} &
%        \includegraphics[width=0.15\linewidth]{img/images/grid/4/2_4.jpg} \\
%
%        \includegraphics[width=0.15\linewidth]{img/images/beta/beta_distribution_single_2.0_4.0.jpg} & 
%        \includegraphics[width=0.15\linewidth]{img/images/grid/4/3_0.jpg} & 
%        \includegraphics[width=0.15\linewidth]{img/images/grid/4/3_1.jpg} & 
%        \includegraphics[width=0.15\linewidth]{img/images/grid/4/3_2.jpg} & 
%        \includegraphics[width=0.15\linewidth]{img/images/grid/4/3_3.jpg} &
%        \includegraphics[width=0.15\linewidth]{img/images/grid/4/3_4.jpg} \\
%        \\[0.05cm]
%        \our{} & 
%        $\omega = 2.0$ & 
%        $\omega = 5.0$ &
%        $\omega = 7.5$ &
%        $\omega = 9.0$ &
%        $\omega = 12.5$ \\
%    \end{tabular}
    \vspace{-0.2cm}
    \caption{Prompt: "An illustration of a human heart made of translucent glass."}
    \label{fig:ap4}
\end{figure*}



% \begin{figure}
%     \centering
%         \includegraphics[width=\linewidth]{img/images/fig7/000002.jpg}
%     \label{fig:grid_1}
% \end{figure}

% \begin{figure}
%     \centering
%         \includegraphics[width=\linewidth]{img/images/fig7/000003.jpg}
%     \label{fig:grid_1}
% \end{figure}

% \begin{figure}
%     \centering
%         \includegraphics[width=\linewidth]{img/images/fig7/000004.jpg}
%     \label{fig:grid_1}
% \end{figure}

% \begin{figure}
%     \centering
%         \includegraphics[width=\linewidth]{img/images/fig7/000005.jpg}
%     \label{fig:grid_1}
% \end{figure}

% \begin{figure}
%     \centering
%         \includegraphics[width=\linewidth]{img/images/fig7/000006.jpg}
%     \label{fig:grid_1}
% \end{figure}




% You can have as much text here as you want. The main body must be at most $8$ pages long.
% For the final version, one more page can be added.
% If you want, you can use an appendix like this one.  

% The $\mathtt{\backslash onecolumn}$ command above can be kept in place if you prefer a one-column appendix, or can be removed if you prefer a two-column appendix.  Apart from this possible change, the style (font size, spacing, margins, page numbering, etc.) should be kept the same as the main body.
%%%%%%%%%%%%%%%%%%%%%%%%%%%%%%%%%%%%%%%%%%%%%%%%%%%%%%%%%%%%%%%%%%%%%%%%%%%%%%%
%%%%%%%%%%%%%%%%%%%%%%%%%%%%%%%%%%%%%%%%%%%%%%%%%%%%%%%%%%%%%%%%%%%%%%%%%%%%%%%



\end{document}


% This document was modified from the file originally made available by
% Pat Langley and Andrea Danyluk for ICML-2K. This version was created
% by Iain Murray in 2018, and modified by Alexandre Bouchard in
% 2019 and 2021 and by Csaba Szepesvari, Gang Niu and Sivan Sabato in 2022.
% Modified again in 2023 and 2024 by Sivan Sabato and Jonathan Scarlett.
% Previous contributors include Dan Roy, Lise Getoor and Tobias
% Scheffer, which was slightly modified from the 2010 version by
% Thorsten Joachims & Johannes Fuernkranz, slightly modified from the
% 2009 version by Kiri Wagstaff and Sam Roweis's 2008 version, which is
% slightly modified from Prasad Tadepalli's 2007 version which is a
% lightly changed version of the previous year's version by Andrew
% Moore, which was in turn edited from those of Kristian Kersting and
% Codrina Lauth. Alex Smola contributed to the algorithmic style files.
