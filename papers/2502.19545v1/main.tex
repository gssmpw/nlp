% This must be in the first 5 lines to tell arXiv to use pdfLaTeX, which is strongly recommended.
\pdfoutput=1
% In particular, the hyperref package requires pdfLaTeX in order to break URLs across lines.

\documentclass[11pt]{article}

% Change "review" to "final" to generate the final (sometimes called camera-ready) version.
% Change to "preprint" to generate a non-anonymous version with page numbers.
% \usepackage[review]{acl}
% \usepackage{acl}
\usepackage[preprint]{acl}

% Standard package includes
\usepackage{times}
\usepackage{latexsym}

% For proper rendering and hyphenation of words containing Latin characters (including in bib files)
\usepackage[T1]{fontenc}
% For Vietnamese characters
% \usepackage[T5]{fontenc}
% See https://www.latex-project.org/help/documentation/encguide.pdf for other character sets

% This assumes your files are encoded as UTF8
\usepackage[utf8]{inputenc}

% This is not strictly necessary, and may be commented out,
% but it will improve the layout of the manuscript,
% and will typically save some space.
\usepackage{microtype}

% This is also not strictly necessary, and may be commented out.
% However, it will improve the aesthetics of text in
% the typewriter font.
\usepackage{inconsolata}

%Including images in your LaTeX document requires adding
%additional package(s)
\usepackage{graphicx}
\usepackage{amsmath}
\usepackage{booktabs}
\usepackage{float} 


% If the title and author information does not fit in the area allocated, uncomment the following
%
%\setlength\titlebox{<dim>}
%
% and set <dim> to something 5cm or larger.


\title{Winning Big with Small Models: \\Knowledge Distillation vs.\ Self-Training\\for Reducing Hallucination in QA Agents}

\author{
 \textbf{Ashley Lewis\textsuperscript{1}}
 \textbf{Michael White\textsuperscript{1}}
 \textbf{Jing Liu\textsuperscript{2}}
 \textbf{Toshiaki Koike-Akino\textsuperscript{2}}
 \textbf{Kieran Parsons\textsuperscript{2}}
 \textbf{Ye Wang\textsuperscript{2}} \\
 \textsuperscript{1}The Ohio State University,  
 \textsuperscript{2}Mitsubishi Electric Research Laboratories \\
 \small{
 \{\href{mailto:lewis.2799@osu.edu}{lewis.2799}, 
 \href{mailto:white.1240@osu.edu}{white.1240}\}@osu.edu,  
 \{\href{mailto:jiliu@merl.com}{jiliu}, 
 \href{mailto:koike@merl.com}{koike}, 
 \href{mailto:parsons@merl.com}{parsons}, 
 \href{mailto:yewang@merl.com}{yewang}\}@merl.com}
}



\begin{document}
\maketitle
\begin{abstract}
The deployment of Large Language Models (LLMs) in customer support is constrained by hallucination—generating false information—and the high cost of proprietary models. To address these challenges, we propose a retrieval-augmented question-answering (QA) pipeline and explore how to balance human input and automation. Using a dataset of questions about a Samsung Smart TV user manual, we demonstrate that synthetic data generated by LLMs outperforms crowdsourced data in reducing hallucination in finetuned models. We also compare self-training (fine-tuning models on their own outputs) and knowledge distillation (fine-tuning on stronger models' outputs, e.g., GPT-4o), and find that self-training achieves comparable hallucination reduction.  We conjecture that this surprising finding can be attributed to increased exposure bias issues in the knowledge distillation case and support this conjecture with post hoc analysis. We also improve robustness to unanswerable questions and retrieval failures with contextualized “I don’t know” responses. These findings show that scalable, cost-efficient QA systems can be built using synthetic data and self-training with open-source models, reducing reliance on proprietary tools or costly human annotations.
\footnote{This work was conducted while Ashley Lewis was interning at Mitsubishi Electric Research Laboratories.}



\end{abstract}


\section{Introduction}
\label{sec:introduction}
The business processes of organizations are experiencing ever-increasing complexity due to the large amount of data, high number of users, and high-tech devices involved \cite{martin2021pmopportunitieschallenges, beerepoot2023biggestbpmproblems}. This complexity may cause business processes to deviate from normal control flow due to unforeseen and disruptive anomalies \cite{adams2023proceddsriftdetection}. These control-flow anomalies manifest as unknown, skipped, and wrongly-ordered activities in the traces of event logs monitored from the execution of business processes \cite{ko2023adsystematicreview}. For the sake of clarity, let us consider an illustrative example of such anomalies. Figure \ref{FP_ANOMALIES} shows a so-called event log footprint, which captures the control flow relations of four activities of a hypothetical event log. In particular, this footprint captures the control-flow relations between activities \texttt{a}, \texttt{b}, \texttt{c} and \texttt{d}. These are the causal ($\rightarrow$) relation, concurrent ($\parallel$) relation, and other ($\#$) relations such as exclusivity or non-local dependency \cite{aalst2022pmhandbook}. In addition, on the right are six traces, of which five exhibit skipped, wrongly-ordered and unknown control-flow anomalies. For example, $\langle$\texttt{a b d}$\rangle$ has a skipped activity, which is \texttt{c}. Because of this skipped activity, the control-flow relation \texttt{b}$\,\#\,$\texttt{d} is violated, since \texttt{d} directly follows \texttt{b} in the anomalous trace.
\begin{figure}[!t]
\centering
\includegraphics[width=0.9\columnwidth]{images/FP_ANOMALIES.png}
\caption{An example event log footprint with six traces, of which five exhibit control-flow anomalies.}
\label{FP_ANOMALIES}
\end{figure}

\subsection{Control-flow anomaly detection}
Control-flow anomaly detection techniques aim to characterize the normal control flow from event logs and verify whether these deviations occur in new event logs \cite{ko2023adsystematicreview}. To develop control-flow anomaly detection techniques, \revision{process mining} has seen widespread adoption owing to process discovery and \revision{conformance checking}. On the one hand, process discovery is a set of algorithms that encode control-flow relations as a set of model elements and constraints according to a given modeling formalism \cite{aalst2022pmhandbook}; hereafter, we refer to the Petri net, a widespread modeling formalism. On the other hand, \revision{conformance checking} is an explainable set of algorithms that allows linking any deviations with the reference Petri net and providing the fitness measure, namely a measure of how much the Petri net fits the new event log \cite{aalst2022pmhandbook}. Many control-flow anomaly detection techniques based on \revision{conformance checking} (hereafter, \revision{conformance checking}-based techniques) use the fitness measure to determine whether an event log is anomalous \cite{bezerra2009pmad, bezerra2013adlogspais, myers2018icsadpm, pecchia2020applicationfailuresanalysispm}. 

The scientific literature also includes many \revision{conformance checking}-independent techniques for control-flow anomaly detection that combine specific types of trace encodings with machine/deep learning \cite{ko2023adsystematicreview, tavares2023pmtraceencoding}. Whereas these techniques are very effective, their explainability is challenging due to both the type of trace encoding employed and the machine/deep learning model used \cite{rawal2022trustworthyaiadvances,li2023explainablead}. Hence, in the following, we focus on the shortcomings of \revision{conformance checking}-based techniques to investigate whether it is possible to support the development of competitive control-flow anomaly detection techniques while maintaining the explainable nature of \revision{conformance checking}.
\begin{figure}[!t]
\centering
\includegraphics[width=\columnwidth]{images/HIGH_LEVEL_VIEW.png}
\caption{A high-level view of the proposed framework for combining \revision{process mining}-based feature extraction with dimensionality reduction for control-flow anomaly detection.}
\label{HIGH_LEVEL_VIEW}
\end{figure}

\subsection{Shortcomings of \revision{conformance checking}-based techniques}
Unfortunately, the detection effectiveness of \revision{conformance checking}-based techniques is affected by noisy data and low-quality Petri nets, which may be due to human errors in the modeling process or representational bias of process discovery algorithms \cite{bezerra2013adlogspais, pecchia2020applicationfailuresanalysispm, aalst2016pm}. Specifically, on the one hand, noisy data may introduce infrequent and deceptive control-flow relations that may result in inconsistent fitness measures, whereas, on the other hand, checking event logs against a low-quality Petri net could lead to an unreliable distribution of fitness measures. Nonetheless, such Petri nets can still be used as references to obtain insightful information for \revision{process mining}-based feature extraction, supporting the development of competitive and explainable \revision{conformance checking}-based techniques for control-flow anomaly detection despite the problems above. For example, a few works outline that token-based \revision{conformance checking} can be used for \revision{process mining}-based feature extraction to build tabular data and develop effective \revision{conformance checking}-based techniques for control-flow anomaly detection \cite{singh2022lapmsh, debenedictis2023dtadiiot}. However, to the best of our knowledge, the scientific literature lacks a structured proposal for \revision{process mining}-based feature extraction using the state-of-the-art \revision{conformance checking} variant, namely alignment-based \revision{conformance checking}.

\subsection{Contributions}
We propose a novel \revision{process mining}-based feature extraction approach with alignment-based \revision{conformance checking}. This variant aligns the deviating control flow with a reference Petri net; the resulting alignment can be inspected to extract additional statistics such as the number of times a given activity caused mismatches \cite{aalst2022pmhandbook}. We integrate this approach into a flexible and explainable framework for developing techniques for control-flow anomaly detection. The framework combines \revision{process mining}-based feature extraction and dimensionality reduction to handle high-dimensional feature sets, achieve detection effectiveness, and support explainability. Notably, in addition to our proposed \revision{process mining}-based feature extraction approach, the framework allows employing other approaches, enabling a fair comparison of multiple \revision{conformance checking}-based and \revision{conformance checking}-independent techniques for control-flow anomaly detection. Figure \ref{HIGH_LEVEL_VIEW} shows a high-level view of the framework. Business processes are monitored, and event logs obtained from the database of information systems. Subsequently, \revision{process mining}-based feature extraction is applied to these event logs and tabular data input to dimensionality reduction to identify control-flow anomalies. We apply several \revision{conformance checking}-based and \revision{conformance checking}-independent framework techniques to publicly available datasets, simulated data of a case study from railways, and real-world data of a case study from healthcare. We show that the framework techniques implementing our approach outperform the baseline \revision{conformance checking}-based techniques while maintaining the explainable nature of \revision{conformance checking}.

In summary, the contributions of this paper are as follows.
\begin{itemize}
    \item{
        A novel \revision{process mining}-based feature extraction approach to support the development of competitive and explainable \revision{conformance checking}-based techniques for control-flow anomaly detection.
    }
    \item{
        A flexible and explainable framework for developing techniques for control-flow anomaly detection using \revision{process mining}-based feature extraction and dimensionality reduction.
    }
    \item{
        Application to synthetic and real-world datasets of several \revision{conformance checking}-based and \revision{conformance checking}-independent framework techniques, evaluating their detection effectiveness and explainability.
    }
\end{itemize}

The rest of the paper is organized as follows.
\begin{itemize}
    \item Section \ref{sec:related_work} reviews the existing techniques for control-flow anomaly detection, categorizing them into \revision{conformance checking}-based and \revision{conformance checking}-independent techniques.
    \item Section \ref{sec:abccfe} provides the preliminaries of \revision{process mining} to establish the notation used throughout the paper, and delves into the details of the proposed \revision{process mining}-based feature extraction approach with alignment-based \revision{conformance checking}.
    \item Section \ref{sec:framework} describes the framework for developing \revision{conformance checking}-based and \revision{conformance checking}-independent techniques for control-flow anomaly detection that combine \revision{process mining}-based feature extraction and dimensionality reduction.
    \item Section \ref{sec:evaluation} presents the experiments conducted with multiple framework and baseline techniques using data from publicly available datasets and case studies.
    \item Section \ref{sec:conclusions} draws the conclusions and presents future work.
\end{itemize}

\section{RELATED WORK}
\label{sec:relatedwork}
In this section, we describe the previous works related to our proposal, which are divided into two parts. In Section~\ref{sec:relatedwork_exoplanet}, we present a review of approaches based on machine learning techniques for the detection of planetary transit signals. Section~\ref{sec:relatedwork_attention} provides an account of the approaches based on attention mechanisms applied in Astronomy.\par

\subsection{Exoplanet detection}
\label{sec:relatedwork_exoplanet}
Machine learning methods have achieved great performance for the automatic selection of exoplanet transit signals. One of the earliest applications of machine learning is a model named Autovetter \citep{MCcauliff}, which is a random forest (RF) model based on characteristics derived from Kepler pipeline statistics to classify exoplanet and false positive signals. Then, other studies emerged that also used supervised learning. \cite{mislis2016sidra} also used a RF, but unlike the work by \citet{MCcauliff}, they used simulated light curves and a box least square \citep[BLS;][]{kovacs2002box}-based periodogram to search for transiting exoplanets. \citet{thompson2015machine} proposed a k-nearest neighbors model for Kepler data to determine if a given signal has similarity to known transits. Unsupervised learning techniques were also applied, such as self-organizing maps (SOM), proposed \citet{armstrong2016transit}; which implements an architecture to segment similar light curves. In the same way, \citet{armstrong2018automatic} developed a combination of supervised and unsupervised learning, including RF and SOM models. In general, these approaches require a previous phase of feature engineering for each light curve. \par

%DL is a modern data-driven technology that automatically extracts characteristics, and that has been successful in classification problems from a variety of application domains. The architecture relies on several layers of NNs of simple interconnected units and uses layers to build increasingly complex and useful features by means of linear and non-linear transformation. This family of models is capable of generating increasingly high-level representations \citep{lecun2015deep}.

The application of DL for exoplanetary signal detection has evolved rapidly in recent years and has become very popular in planetary science.  \citet{pearson2018} and \citet{zucker2018shallow} developed CNN-based algorithms that learn from synthetic data to search for exoplanets. Perhaps one of the most successful applications of the DL models in transit detection was that of \citet{Shallue_2018}; who, in collaboration with Google, proposed a CNN named AstroNet that recognizes exoplanet signals in real data from Kepler. AstroNet uses the training set of labelled TCEs from the Autovetter planet candidate catalog of Q1–Q17 data release 24 (DR24) of the Kepler mission \citep{catanzarite2015autovetter}. AstroNet analyses the data in two views: a ``global view'', and ``local view'' \citep{Shallue_2018}. \par


% The global view shows the characteristics of the light curve over an orbital period, and a local view shows the moment at occurring the transit in detail

%different = space-based

Based on AstroNet, researchers have modified the original AstroNet model to rank candidates from different surveys, specifically for Kepler and TESS missions. \citet{ansdell2018scientific} developed a CNN trained on Kepler data, and included for the first time the information on the centroids, showing that the model improves performance considerably. Then, \citet{osborn2020rapid} and \citet{yu2019identifying} also included the centroids information, but in addition, \citet{osborn2020rapid} included information of the stellar and transit parameters. Finally, \citet{rao2021nigraha} proposed a pipeline that includes a new ``half-phase'' view of the transit signal. This half-phase view represents a transit view with a different time and phase. The purpose of this view is to recover any possible secondary eclipse (the object hiding behind the disk of the primary star).


%last pipeline applies a procedure after the prediction of the model to obtain new candidates, this process is carried out through a series of steps that include the evaluation with Discovery and Validation of Exoplanets (DAVE) \citet{kostov2019discovery} that was adapted for the TESS telescope.\par
%



\subsection{Attention mechanisms in astronomy}
\label{sec:relatedwork_attention}
Despite the remarkable success of attention mechanisms in sequential data, few papers have exploited their advantages in astronomy. In particular, there are no models based on attention mechanisms for detecting planets. Below we present a summary of the main applications of this modeling approach to astronomy, based on two points of view; performance and interpretability of the model.\par
%Attention mechanisms have not yet been explored in all sub-areas of astronomy. However, recent works show a successful application of the mechanism.
%performance

The application of attention mechanisms has shown improvements in the performance of some regression and classification tasks compared to previous approaches. One of the first implementations of the attention mechanism was to find gravitational lenses proposed by \citet{thuruthipilly2021finding}. They designed 21 self-attention-based encoder models, where each model was trained separately with 18,000 simulated images, demonstrating that the model based on the Transformer has a better performance and uses fewer trainable parameters compared to CNN. A novel application was proposed by \citet{lin2021galaxy} for the morphological classification of galaxies, who used an architecture derived from the Transformer, named Vision Transformer (VIT) \citep{dosovitskiy2020image}. \citet{lin2021galaxy} demonstrated competitive results compared to CNNs. Another application with successful results was proposed by \citet{zerveas2021transformer}; which first proposed a transformer-based framework for learning unsupervised representations of multivariate time series. Their methodology takes advantage of unlabeled data to train an encoder and extract dense vector representations of time series. Subsequently, they evaluate the model for regression and classification tasks, demonstrating better performance than other state-of-the-art supervised methods, even with data sets with limited samples.

%interpretation
Regarding the interpretability of the model, a recent contribution that analyses the attention maps was presented by \citet{bowles20212}, which explored the use of group-equivariant self-attention for radio astronomy classification. Compared to other approaches, this model analysed the attention maps of the predictions and showed that the mechanism extracts the brightest spots and jets of the radio source more clearly. This indicates that attention maps for prediction interpretation could help experts see patterns that the human eye often misses. \par

In the field of variable stars, \citet{allam2021paying} employed the mechanism for classifying multivariate time series in variable stars. And additionally, \citet{allam2021paying} showed that the activation weights are accommodated according to the variation in brightness of the star, achieving a more interpretable model. And finally, related to the TESS telescope, \citet{morvan2022don} proposed a model that removes the noise from the light curves through the distribution of attention weights. \citet{morvan2022don} showed that the use of the attention mechanism is excellent for removing noise and outliers in time series datasets compared with other approaches. In addition, the use of attention maps allowed them to show the representations learned from the model. \par

Recent attention mechanism approaches in astronomy demonstrate comparable results with earlier approaches, such as CNNs. At the same time, they offer interpretability of their results, which allows a post-prediction analysis. \par


\section{Data and Experimental Setup}
\label{sec:data}

\subsection{Datasets}

The primary dataset consists of 684 crowdsourced questions paired with retrieved passages from the manual \cite{nandy-etal-2021-question-answering}. We split the dataset into 534 training, 100 development, and 50 test questions (our ``regular test set''). Dataset preprocessing details can be found in Appendix \ref{app:data_preprocessing}. We focused on this dataset because many existing QA datasets either lack grounding documents or prioritize open-domain QA, which does not align with the controlled, retrieval-augmented QA setting we aimed to study. This approach also allowed us to conduct a deep-dive analysis into the trade-offs between self-training, knowledge distillation, and synthetic data generation in mitigating hallucinations within a well-defined context.

As mentioned, the dataset also contains a collection of 3,000 questions sourced from community forums. We create challenge sets by randomly selecting 100 development and 100 test questions from this set. These questions are noisier and less than half are answerable, which allows us to evaluate how well models handle particularly challenging cases. Examples from both types of questions can be found in Appendix \ref{app:example_questions}.

\subsection{Training Data}

\begin{table}[t]
    \centering
    \renewcommand{\arraystretch}{1.2} % Adjust row spacing
    \setlength{\tabcolsep}{6pt} % Adjust column spacing
    \begin{tabular}{l|c}
        \toprule
        \textbf{Model} & \textbf{FactScore} \\
        \midrule
        Llama-3      & 0.9077 \\
        GPT-4o         & 0.9323 \\
        \hline
        Uncleaned       & 0.8798 \\
        Manual cleaned  & 0.8810 \\
        Autocleaned\textsubscript{L} & 0.8202 \\
        Autocleaned\textsubscript{G}     & 0.8966 \\
        \hline
        SynthGPT   & 0.9116 \\
        SynthLlama & 0.9211 \\
        SynthLlama+ & \textbf{0.9461} \\
        \bottomrule
    \end{tabular}
    \caption{FactScore results for the test set. Pretrained base models: Llama-3 and GPT-4o. Finetuned Llama-3-B models on the \citet{nandy-etal-2021-question-answering} dataset: Uncleaned (no data cleaning performed), Manual cleaned (cleaning done by the first author), Autocleaned\textsubscript{L} and Autocleaned\textsubscript{G} (cleaning done by Llama-3-70B and GPT-4o, respectively). Finetuned Llama-3-B models on synthetic data: SynthGPT (trained on data generated by GPT-4o), SynthLlama (trained on data generated by Llama-3-8B), and SynthLlama+ (same as SynthLlama, with additional negative examples).}
    \label{tab:factscore_test_set}
\end{table}

\paragraph{Regular Training Data}

We use the pretrained Llama-3-8B-Instruct \cite{dubey2024llama3herdmodels} to generate answers for the 534 training questions. Three datasets are created:
(1) a manually cleaned version where responses were reviewed and corrected by the first author, and
(2)--(3) automatically cleaned versions using GPT-4o and Llama-3-70B, respectively. This allows a systematic evaluation of the trade-offs between human effort and automated cleaning. As shown in Table \ref{tab:factscore_test_set}, cleaning with Llama-3 was largely unsuccessful. Thus in the remaining experiments GPT-4o was used for the cleaning task. We anticipate that improvements in open-source models like Llama-3 may reduce reliance on proprietary alternatives in the future. Prompts for both data generation and cleaning can be found in Appendix \ref{app:prompts}.

\paragraph{Synthetic Data} 

In addition to crowdsourced training questions, we generate fully synthetic QA data using LLMs. Specifically, we prompt Llama-3 and GPT-4o to generate new QA pairs based on passages from the Samsung Smart TV manual. To ensure that these datasets have comparable information coverage to the crowdsourced dataset and to prevent retrieval quality from being a confounding factor, we select passages systematically rather than randomly. We identify all 208 unique sections in the manual that are referenced in the crowdsourced training data. From these passages, we generate two synthetic QA pairs per passage, two from Llama-3 and two from GPT-4o. This approach ensures that the synthetic datasets are no larger than the crowdsourced dataset and cover similar content while maintaining consistency in passage selection. In a real-world application, this limitation does not exist, as synthetic training data can be generated from any number of passages. Thus, coverage is not inherently a bottleneck when using synthetic data in practical settings.

\subsection{Baseline and Experimental Models}

To evaluate the impact of data cleaning type and synthetic training data on hallucination reduction, we experiment with both pretrained models and finetuned models trained on different datasets.

\paragraph{Baseline Models}  
\begin{itemize}
    \item \textbf{Pretrained Llama-3-8B-Instruct (Llama-3)}: An open-source model that serves as a strong starting point for retrieval-augmented generation (RAG) without task-specific adaptation \cite{dubey2024llama3herdmodels}. The model is run with few-shot prompting.
    \item \textbf{GPT-4o}: A state-of-the-art proprietary model, included as a benchmark to assess how well finetuned open-source models compare to a highly optimized general-purpose system \cite{openai2024gpt4technicalreport}. The model is run with few-shot prompting.
\end{itemize}

\paragraph{Finetuned Models}  

We finetune Llama-3 on different variations of training data to analyze the effects of data source, cleaning method, and exposure bias on hallucination rates. Specifically, we train models on the following datasets using the \citet{zheng-etal-2024-llamafactory} finetuning framework and parameters:
\begin{itemize}
    \item \textbf{Manually Cleaned Training Data}: A dataset where the first author reviewed and corrected Llama-3-generated answers to the \citet{nandy-etal-2021-question-answering} 534 crowdsourced training questions.
    \item \textbf{Automatically Cleaned Training Data}: A version of the training set where errors in Llama-3-generated answers were identified and repaired using GPT-4o.
    \item \textbf{Synthetic Data (Llama vs.\ GPT)}: Two datasets where  416 QA pairs were generated by either Llama-3 or GPT-4o based on passages from the Samsung Smart TV manual. All synthetic data was cleaned using GPT-4o.
    \item \textbf{Synth Llama+}: Trained on the synthetic Llama data, and augmented with 100 negative examples (see section \ref{synth_llama+} for more details).
\end{itemize}

\subsection{Metrics for Evaluation}

We evaluate model performance using two methods: FactScore \cite{min-etal-2023-factscore}, an automated metric for factual accuracy, and human evaluation by trained annotators. These complementary approaches measure factual consistency and response quality.

\paragraph{FactScore}

FactScore evaluates whether a model's response aligns with a reference document. It works by decomposing a response into sentences, breaking each sentence into discrete factual claims, and verifying their alignment with the reference text. FactScore measures the proportion of supported claims while penalizing hallucinated content. However, responses from GPT-4o and SynthGPT, which often use structured formatting (e.g., lists, topic headers), cause FactScore to produce fragmented or nonsensical claims, unfairly penalizing these models. To address this, we removed the sentence-splitting preprocessing and instead generated atomic facts directly from the full response.

FactScore, which we computed using GPT-4o-mini, has been shown to be a reliable proxy for factuality, correlating well with human judgments \cite{min-etal-2023-factscore}. However, we find that it is unsuitable for evaluating \textit{I don’t know} responses. Thus, we applied FactScore only to the regular test set (mostly answerable questions), excluding the challenge set (many unanswerable questions). We also used it to evaluate human-written training questions for synthetic models, as they do not see these at training time and it provides a more robust evaluation. Further information in Appendix \ref{app:factscore}.


\begin{table}[t]
    \centering
    \renewcommand{\arraystretch}{1.3} % Adjust row spacing
    \begin{tabular}{p{2cm}p{4.75cm}}  % 10cm width for descriptions
        \toprule
        \textbf{Category} & \textbf{Description} \\
        \midrule
        \textbf{Hallucination} & The response contains information not present in the manual. \\
        \textbf{Non-Answer} & The response does not answer the question. \\
        \textbf{Partial answer} & The response does not fully answer the question, or omits important information. \\
        \textbf{IDK - Bad} & The manual section has the information required to answer the question, but the response is mistakenly ``I don’t know''. \\
        \textbf{Disfluent} & The response contains grammatical or fluency problems. \\
        \textbf{Other} & The response contains some other type of error. \\
        \textbf{IDK - Good} & The manual section does not contain the information required to answer the question and the response is appropriately ``I don’t know''. \\
        \textbf{Good} & There are no errors. \\

        \bottomrule
    \end{tabular}
    \caption{Response error categories and their descriptions. Examples can be found in Appendix \ref{app:error_category_examples}.}
    \label{tab:error_categories}
\end{table}

\paragraph{Human Evaluation} 
To obtain a more nuanced assessment of response quality, we conducted a human evaluation with three fluent English speaking, Linguistics PhD students (instructions in Appendix \ref{app:human_eval}), who annotated each model-generated response for the regular test set (50 items) and 50 items from the challenge set. They assigned to each response one of the categories listed in Table \ref{tab:error_categories} (examples in Appendix \ref{app:error_category_examples}), which were determined by an author analysis of the dev set. Three-way agreement occurred between annotators 63.14\% of the time and two-way agreement occurred 36.43\% of the time. Krippendorff's Alpha was $\alpha$ = 0.625, indicating substantial agreement.

Each response was labeled independently by all three annotators. The final assigned label was determined by a majority vote. In the few cases where annotators provided three different labels, the response was assigned the most severe error based on the following predefined ranking:  Hallucination > Non-Answer > Partial Answer > IDK - Bad > Disfluent > Other. The purpose of this ranking is to prioritize hallucination and content errors. For example, if a response is labeled as ``Hallucination,'' ``Good,'' and ``Partial Answer,'' it is assigned the final label of ``Hallucination'' due to its higher severity in the ranking.

By combining automated and human evaluation, we ensure a comprehensive analysis of both quality and factual consistency in model-generated responses. The aggregated results can be found in Table \ref{tab:humaneval} and the separate results on the regular and challenge test sets can be found in Appendix \ref{app:human_eval_breakdown}.

\begin{table}[t]
    \centering
    \small % Reduce font size for better fit
    \renewcommand{\arraystretch}{1.2} % Adjust row spacing
    \setlength{\tabcolsep}{4pt} % Reduce column padding
    \begin{tabular}{lccc}
        \toprule
        \textbf{Model} & \textbf{Chall. (100)} & \textbf{Reg. (50)} & \textbf{Total (150)} \\
        \midrule
        Pretrain & 26.56 & 28.74 & 27.29 \\
        GPT-4o & 22.23 & 31.56 & 25.34 \\
        \hline
        Manual & 21.74 & 28.54 & 24.01 \\
        Auto-cleaned & 26.33 & 31.00 & 27.89 \\
        \hline
        SynthLlama & 36.06 & 44.56 & 38.89 \\
        SynthGPT & 40.40 & 47.34 & 42.71 \\
        SynthLlama+ & 21.92 & 42.06 & 28.63 \\
        \bottomrule
    \end{tabular}
    \caption{Average response lengths for different models across challenge and regular test sets.}
    \label{tab:lengths}
\end{table}



\section{Results and Analysis}
\begin{table}[t]
\centering
\resizebox{\linewidth}{!}{%
\begin{tabular}{lccccc}
\toprule
\textbf{Model}                  & \textbf{Precision} & \textbf{Recall} & \textbf{F1-Score} & \textbf{Accuracy} & \textbf{MCC} \\ 
\midrule
MTL                             & 0.59               & 0.40            & 0.37              & 0.41              & \textbf{0.78}            \\ 
GNN                             & 0.64               & 0.50            & 0.54              & \textbf{0.64}              & 0.40         \\ 
Role-Aware                      & 0.21               & 0.20            & 0.14              & 0.50              & 0.04            \\ 
ToInLegalBERT                   & 0.67               & 0.60            & 0.62              & \textbf{0.64}              & 0.52         \\ 
LLaMA-2 (Quantized)   & 0.17                  & 0.16               & 0.09                 & 0.20                 & 0.3            \\ 
LLaMA-2 (Unquantized)   & 0.19                  & 0.15               & 0.08                 & 0.25                 & 0.05            \\
\texttt{RhetoricLLaMA}                   & 0.19               & 0.15            & 0.09              & 0.39              & 0.02         \\ 
InLegalBERT(i)                  & 0.57               & 0.45            & 0.49              & 0.53              & 0.45         \\ 
InLegalBERT(i-1, i)             & 0.60               & 0.53            & 0.55              & 0.57              & 0.50         \\ 
InLegalBERT(i-2, i-1, i)        & 0.62               & 0.56            & 0.58              & 0.59              & 0.52         \\ 
InLegalBERT(i-1, i, i+1)        & 0.61               & 0.56            & 0.58              & 0.59              & 0.52         \\ 
InLegalBERT(i-1, label\_t, i)   & 0.63                  & 0.32               & 0.34                 & 0.45                 & 0.22            \\ 
InLegalBERT(i-1, label\_p, i)   & 0.54               & 0.46            & 0.48              & 0.52              & 0.35         \\ 
Hier\_BiLSTM CRF                & \textbf{0.78}               & \textbf{0.77}            & \textbf{0.77}              & 0.62              & 0.68         \\ 

\bottomrule
\end{tabular}
}
\caption{Performance Comparison of Models on Rhetorical Role Classification. In the Model column, \(i\) indicates the current sentence, \(i-1\) means the previous sentence, and \(i+1\) means the next sentence. \texttt{label\_t} and \texttt{label\_p} refer to the true and predicted labels of the previous sentences. The best results are in bold.}
\label{tab:model_performance}
\end{table}
%%%%%%%%%%%%%%%%%%%%%%%%%%%%%%%%%%%%%%%%%%%%%

In this section, we present the results of our experiments on rhetorical role classification and analyze the performance of different models. Table \ref{tab:model_performance} summarizes the evaluation metrics for each model.

\subsection{Model Performance}
Among the evaluated models, the hierarchical BiLSTM-CRF achieves the highest overall performance. The sequential nature of BiLSTM allows the model to capture dependencies between sentences, while the CRF layer explicitly models label transitions, refining predictions by enforcing structural coherence. This ability to learn the transition relationships between rhetorical roles plays a crucial role in classification, as labels in legal documents follow a structured sequence. For example, an issue is likely to be followed by supporting arguments and eventually a decision. The ability to maintain coherence in predictions by capturing dependencies between consecutive sentences makes the BiLSTM-CRF model more effective in comparison to models that classify each sentence independently. Prior studies in structured text classification have similarly observed the benefits of explicit modeling of transition relationships between labels, as seen in \citet{bhattacharya2019identification, modi-etal-2023-semeval, santosh2024hiculr}.

In contrast, transformer-based models such as ToInLegalBERT, InLegalBERT, and Role-Aware Transformers process sentences independently, limiting their ability to model long-range dependencies within legal documents. These models rely primarily on self-attention mechanisms, which work well for general NLP tasks but struggle to capture structured rhetorical transitions without explicit sequential modeling. ToInLegalBERT, which integrates sentence-level positional encodings and hierarchical structuring, performs better than standard BERT-based models, highlighting the benefit of incorporating document structure into transformers.

The Graph Neural Network model performs competitively by effectively propagating contextual information across sentence nodes, capturing both local and global dependencies within legal documents. Among the InLegalBERT variants, the model trained using the current sentence along with two preceding sentences achieves the best performance, reinforcing the importance of sentence context in improving classification accuracy.

The Multi-Task Learning model, which incorporates label shift prediction as an auxiliary task, achieves moderate performance. While this method aims to capture role transitions, the additional complexity may have introduced challenges in optimization. Despite this, multitask learning remains a promising approach, particularly when combined with stronger baseline models.

The \texttt{RhetoricLLaMA} model, despite being instruction-tuned, did not perform as strongly as expected. While large language models like LLaMA-2-7B have achieved success in NLP, their effectiveness in specialized tasks such as rhetorical role classification remains limited without extensive domain-specific fine-tuning. Further research is needed to optimize large language models for structured legal NLP tasks.

\subsection{Impact of Transition Relationships in Classification}
Our experiments highlight the critical role of transition relationships between rhetorical roles in improving classification performance. Models such as the BiLSTM-CRF explicitly model these transitions, allowing them to maintain coherence in predictions by capturing dependencies between consecutive sentences. This is particularly advantageous because legal documents are highly structured, with rhetorical roles appearing in predictable sequences. In contrast, models that classify each sentence in isolation struggle to maintain contextual consistency, leading to higher misclassification rates.

For instance, when a sentence is labeled as an issue, the subsequent sentences are highly likely to be arguments or facts rather than a decision. CRF layers enforce these structural constraints, making BiLSTM-CRF more effective than independent sentence classifiers. This aligns with previous findings in rhetorical role classification, where modeling dependencies between sequential labels significantly improved performance in structured text classification tasks.

\subsection{Justification for Predicted Labels Showing Higher Performance}
An interesting observation from Table \ref{tab:model_performance} is that models using predicted labels for previous sentences sometimes outperform those using true labels. This initially appears counterintuitive, but a plausible explanation is that during training, both true labels and predicted labels were provided to the model, allowing it to learn effective dependencies. However, during testing, true labels are not available, meaning models trained exclusively with true labels may not learn to handle missing labels during inference. In contrast, models using predicted labels during training are already exposed to prediction noise, making them better adapted to real-world inference conditions where true labels are not available.

This suggests that training models to rely on predicted labels during both training and inference improves robustness, as the model learns to correct potential errors in label predictions over multiple steps. However, further research is needed to analyze whether explicitly modeling label uncertainty could further enhance performance.

\subsection{Impact of Instruction-Tuning in \texttt{RhetoricLLaMA}}
We conducted extensive experiments to analyze the impact of instruction-tuning in \texttt{RhetoricLLaMA} by comparing it against Vanilla LLaMA models in both quantized (4-bits) and unquantized forms. Despite leveraging large-scale pre-trained models, the instruction-tuned \texttt{RhetoricLLaMA} did not achieve the expected performance, suggesting that rhetorical role classification in legal texts requires more specialized adaptations.

The comparison revealed that the instruction-tuned model performed slightly better than the Vanilla LLaMA model but still lagged behind traditional transformer-based models like \texttt{ToInLegalBERT} and BiLSTM-CRF. While instruction-tuning provides explicit task-specific guidance, our results indicate that for highly specialized domains such as legal NLP, additional domain-specific pre-training and refined instruction sets are necessary to enhance model performance.

\subsection{Error Analysis}
Our error analysis revealed that the models struggled primarily with distinguishing between closely related rhetorical roles, such as Facts and Reasoning, due to the overlap in their language and structure within legal documents. This challenge is clearly illustrated in the confusion matrix of the Hierarchical BiLSTM-CRF model Figure~\ref{fig:hier_bilstm_crf}, which shows frequent misclassifications between these roles. Similarly, confusion between Arguments of Petitioner and Arguments of Respondent was prevalent, as both often exhibit similar language patterns, further complicating accurate classification. Models that incorporated contextual information from preceding or following sentences demonstrated some improvement in reducing these errors, particularly for roles requiring a clear transition, such as Issue and Decision. However, despite this improvement, the context-aware models still encountered difficulties, suggesting that the rhetorical role boundaries within these transitions are not always well-defined. Another critical issue identified was class imbalance. More frequent labels like None and Facts were consistently overpredicted, leading to lower precision for less frequent labels such as Issue and Decision. This imbalance skewed the performance, resulting in models favoring high-frequency roles at the expense of accuracy for underrepresented roles.
% Furthermore, models utilizing predicted labels from previous sentences experienced error propagation. When earlier sentences were misclassified, these errors compounded, negatively affecting the subsequent predictions, particularly in sequential models that rely heavily on the accuracy of previous classifications. 
Figures~\ref{fig:mtl},~\ref{fig:gnn},~\ref{fig:toinlegalbert},~\ref{fig:rhetoricllama},~\ref{fig:inlegalbert_i},~\ref{fig:inlegalbert_i-1_i},~\ref{fig:inlegalbert_i-2_i-1_i},~\ref{fig:inlegalbert_i-1_i_i+1},~\ref{fig:inlegalbert_label_t},~\ref{fig:inlegalbert_label_p} illustrating the confusion matrices for other models, are provided in the Appendix due to space constraints. These figures further highlight the patterns of misclassification and the impact of various model architectures on error distribution. Addressing these issues, particularly through improved handling of context, mitigating class imbalance, and minimizing the propagation of sequential errors, remains a critical area for future research and model refinement.
% %%%%%%%%%%%%%%%%%%%%%%%%%%%%%%%%%%%
\begin{figure}[t]
    \centering
    \includegraphics[width=\linewidth]{figures/Hier_BiLSTM_CRF.pdf}
    \caption{Confusion matrix for rhetorical role classification using Hierarchical BiLSTM-CRF model.}
    \label{fig:hier_bilstm_crf}
\end{figure}
% %%%%%%%%%%%%%%%%%%%%%%%%%%%%%%%%%%%%
\section{Discussion of Assumptions}\label{sec:discussion}
In this paper, we have made several assumptions for the sake of clarity and simplicity. In this section, we discuss the rationale behind these assumptions, the extent to which these assumptions hold in practice, and the consequences for our protocol when these assumptions hold.

\subsection{Assumptions on the Demand}

There are two simplifying assumptions we make about the demand. First, we assume the demand at any time is relatively small compared to the channel capacities. Second, we take the demand to be constant over time. We elaborate upon both these points below.

\paragraph{Small demands} The assumption that demands are small relative to channel capacities is made precise in \eqref{eq:large_capacity_assumption}. This assumption simplifies two major aspects of our protocol. First, it largely removes congestion from consideration. In \eqref{eq:primal_problem}, there is no constraint ensuring that total flow in both directions stays below capacity--this is always met. Consequently, there is no Lagrange multiplier for congestion and no congestion pricing; only imbalance penalties apply. In contrast, protocols in \cite{sivaraman2020high, varma2021throughput, wang2024fence} include congestion fees due to explicit congestion constraints. Second, the bound \eqref{eq:large_capacity_assumption} ensures that as long as channels remain balanced, the network can always meet demand, no matter how the demand is routed. Since channels can rebalance when necessary, they never drop transactions. This allows prices and flows to adjust as per the equations in \eqref{eq:algorithm}, which makes it easier to prove the protocol's convergence guarantees. This also preserves the key property that a channel's price remains proportional to net money flow through it.

In practice, payment channel networks are used most often for micro-payments, for which on-chain transactions are prohibitively expensive; large transactions typically take place directly on the blockchain. For example, according to \cite{river2023lightning}, the average channel capacity is roughly $0.1$ BTC ($5,000$ BTC distributed over $50,000$ channels), while the average transaction amount is less than $0.0004$ BTC ($44.7k$ satoshis). Thus, the small demand assumption is not too unrealistic. Additionally, the occasional large transaction can be treated as a sequence of smaller transactions by breaking it into packets and executing each packet serially (as done by \cite{sivaraman2020high}).
Lastly, a good path discovery process that favors large capacity channels over small capacity ones can help ensure that the bound in \eqref{eq:large_capacity_assumption} holds.

\paragraph{Constant demands} 
In this work, we assume that any transacting pair of nodes have a steady transaction demand between them (see Section \ref{sec:transaction_requests}). Making this assumption is necessary to obtain the kind of guarantees that we have presented in this paper. Unless the demand is steady, it is unreasonable to expect that the flows converge to a steady value. Weaker assumptions on the demand lead to weaker guarantees. For example, with the more general setting of stochastic, but i.i.d. demand between any two nodes, \cite{varma2021throughput} shows that the channel queue lengths are bounded in expectation. If the demand can be arbitrary, then it is very hard to get any meaningful performance guarantees; \cite{wang2024fence} shows that even for a single bidirectional channel, the competitive ratio is infinite. Indeed, because a PCN is a decentralized system and decisions must be made based on local information alone, it is difficult for the network to find the optimal detailed balance flow at every time step with a time-varying demand.  With a steady demand, the network can discover the optimal flows in a reasonably short time, as our work shows.

We view the constant demand assumption as an approximation for a more general demand process that could be piece-wise constant, stochastic, or both (see simulations in Figure \ref{fig:five_nodes_variable_demand}).
We believe it should be possible to merge ideas from our work and \cite{varma2021throughput} to provide guarantees in a setting with random demands with arbitrary means. We leave this for future work. In addition, our work suggests that a reasonable method of handling stochastic demands is to queue the transaction requests \textit{at the source node} itself. This queuing action should be viewed in conjunction with flow-control. Indeed, a temporarily high unidirectional demand would raise prices for the sender, incentivizing the sender to stop sending the transactions. If the sender queues the transactions, they can send them later when prices drop. This form of queuing does not require any overhaul of the basic PCN infrastructure and is therefore simpler to implement than per-channel queues as suggested by \cite{sivaraman2020high} and \cite{varma2021throughput}.

\subsection{The Incentive of Channels}
The actions of the channels as prescribed by the DEBT control protocol can be summarized as follows. Channels adjust their prices in proportion to the net flow through them. They rebalance themselves whenever necessary and execute any transaction request that has been made of them. We discuss both these aspects below.

\paragraph{On Prices}
In this work, the exclusive role of channel prices is to ensure that the flows through each channel remains balanced. In practice, it would be important to include other components in a channel's price/fee as well: a congestion price  and an incentive price. The congestion price, as suggested by \cite{varma2021throughput}, would depend on the total flow of transactions through the channel, and would incentivize nodes to balance the load over different paths. The incentive price, which is commonly used in practice \cite{river2023lightning}, is necessary to provide channels with an incentive to serve as an intermediary for different channels. In practice, we expect both these components to be smaller than the imbalance price. Consequently, we expect the behavior of our protocol to be similar to our theoretical results even with these additional prices.

A key aspect of our protocol is that channel fees are allowed to be negative. Although the original Lightning network whitepaper \cite{poon2016bitcoin} suggests that negative channel prices may be a good solution to promote rebalancing, the idea of negative prices in not very popular in the literature. To our knowledge, the only prior work with this feature is \cite{varma2021throughput}. Indeed, in papers such as \cite{van2021merchant} and \cite{wang2024fence}, the price function is explicitly modified such that the channel price is never negative. The results of our paper show the benefits of negative prices. For one, in steady state, equal flows in both directions ensure that a channel doesn't loose any money (the other price components mentioned above ensure that the channel will only gain money). More importantly, negative prices are important to ensure that the protocol selectively stifles acyclic flows while allowing circulations to flow. Indeed, in the example of Section \ref{sec:flow_control_example}, the flows between nodes $A$ and $C$ are left on only because the large positive price over one channel is canceled by the corresponding negative price over the other channel, leading to a net zero price.

Lastly, observe that in the DEBT control protocol, the price charged by a channel does not depend on its capacity. This is a natural consequence of the price being the Lagrange multiplier for the net-zero flow constraint, which also does not depend on the channel capacity. In contrast, in many other works, the imbalance price is normalized by the channel capacity \cite{ren2018optimal, lin2020funds, wang2024fence}; this is shown to work well in practice. The rationale for such a price structure is explained well in \cite{wang2024fence}, where this fee is derived with the aim of always maintaining some balance (liquidity) at each end of every channel. This is a reasonable aim if a channel is to never rebalance itself; the experiments of the aforementioned papers are conducted in such a regime. In this work, however, we allow the channels to rebalance themselves a few times in order to settle on a detailed balance flow. This is because our focus is on the long-term steady state performance of the protocol. This difference in perspective also shows up in how the price depends on the channel imbalance. \cite{lin2020funds} and \cite{wang2024fence} advocate for strictly convex prices whereas this work and \cite{varma2021throughput} propose linear prices.

\paragraph{On Rebalancing} 
Recall that the DEBT control protocol ensures that the flows in the network converge to a detailed balance flow, which can be sustained perpetually without any rebalancing. However, during the transient phase (before convergence), channels may have to perform on-chain rebalancing a few times. Since rebalancing is an expensive operation, it is worthwhile discussing methods by which channels can reduce the extent of rebalancing. One option for the channels to reduce the extent of rebalancing is to increase their capacity; however, this comes at the cost of locking in more capital. Each channel can decide for itself the optimum amount of capital to lock in. Another option, which we discuss in Section \ref{sec:five_node}, is for channels to increase the rate $\gamma$ at which they adjust prices. 

Ultimately, whether or not it is beneficial for a channel to rebalance depends on the time-horizon under consideration. Our protocol is based on the assumption that the demand remains steady for a long period of time. If this is indeed the case, it would be worthwhile for a channel to rebalance itself as it can make up this cost through the incentive fees gained from the flow of transactions through it in steady state. If a channel chooses not to rebalance itself, however, there is a risk of being trapped in a deadlock, which is suboptimal for not only the nodes but also the channel.

\section{Conclusion}
This work presents DEBT control: a protocol for payment channel networks that uses source routing and flow control based on channel prices. The protocol is derived by posing a network utility maximization problem and analyzing its dual minimization. It is shown that under steady demands, the protocol guides the network to an optimal, sustainable point. Simulations show its robustness to demand variations. The work demonstrates that simple protocols with strong theoretical guarantees are possible for PCNs and we hope it inspires further theoretical research in this direction.
\section{Conclusion}
In this work, we propose a simple yet effective approach, called SMILE, for graph few-shot learning with fewer tasks. Specifically, we introduce a novel dual-level mixup strategy, including within-task and across-task mixup, for enriching the diversity of nodes within each task and the diversity of tasks. Also, we incorporate the degree-based prior information to learn expressive node embeddings. Theoretically, we prove that SMILE effectively enhances the model's generalization performance. Empirically, we conduct extensive experiments on multiple benchmarks and the results suggest that SMILE significantly outperforms other baselines, including both in-domain and cross-domain few-shot settings.
\section{Limitation}
The use of 3D-printed PLA for structural components improves improving ease of assembly and reduces weight and cost, yet it causes deformation under heavy load, which can diminish end-effector precision. Using metal, such as aluminum, would remedy this problem. Additionally, \robot relies on integrated joint relative encoders, requiring manual initialization in a fixed joint configuration each time the system is powered on. Using absolute joint encoders could significantly improve accuracy and ease of use, although it would increase the overall cost. 

%Reliance on commercially available actuators simplifies integration but imposes constraints on control frequency and customization, further limiting the potential for tailored performance improvements.

% The 6 DoF configuration provides sufficient mobility for most tasks; however, certain bimanual operations could benefit from an additional degree of freedom to handle complex joint constraints more effectively. Furthermore, the limited torque density of commercially available proprioceptive actuators restricts the payload and torque output, making the system less suitability for handling heavier loads or high-torque applications. 

The 6 DoF configuration of the arm provides sufficient mobility for single-arm manipulation tasks, yet it shows a limitation in certain bimanual manipulation problems. Specifically, when \robot holds onto a rigid object with both hands, each arm loses 1 DoF because the hands are fixed to the object during grasping. This leads to an underactuated kinematic chain which has a limited mobility in 3D space. We can achieve more mobility by letting the object slip inside the grippers, yet this renders the grasp less robust and simulation difficult. Therefore, we anticipate that designing a lightweight 3 DoF wrist in place of the current 2 DoF wrist allows a more diverse repertoire of manipulation in bimanual tasks.

Finally, the limited torque density of commercially available proprioceptive actuators restricts the performance. Currently, all of our actuators feature a 1:10 gear ratio, so \robot can handle up to 2.5 kg of payload. To handle a heavier object and manipulate it with higher torque, we expect the actuator to have 1:20$\sim$30 gear ratio, but it is difficult to find an off-the-shelf product that meets our requirements. Customizing the actuator to increase the torque density while minimizing the weight will enable \robot to move faster and handle more diverse objects.

%These constraints highlight opportunities for improvement in future iterations, including alternative materials for enhanced rigidity, custom actuator designs for higher control precision and torque density, the adoption of absolute joint encoders, and optimized configurations to balance dexterity and weight.


Our study was approved by the IRB of our institution.
Participants electronically signed a consent form describing the nature of our study and the data we would collect: their answers to the questionnaires, their demographic information provided by the platform, and their interactions with the study platform. All data was stored pseudonymously.
While our initial study description did not explicitly mention participants they would be exposed to phishing, this is a commonly used method in most phishing studies~\cite{resnik2018ethics,thomopoulos2023methodologies} to avoid excessive priming.
The participants were debriefed after completing the study with the full description, and is confirmed to incur only minimal risks~\cite{finn2007designing}, also confirmed by our IRB classifying our study as minimal risk.
Participants were appropriately remunerated for their time with a payment matching the highest minimum wage in their country.

We took further countermeasures to ensure participants' safety: the discomfort of being exposed to phishing emails was mitigated by the roleplay setting and their assigned fictitious identity.
Furthermore, their task was limited to clicking on links---there was no interaction with simulated phishing websites or other potentially harmful content.
Additionally, the phishing URLs we provided did not offer an easy way for participants to actually visit them (as our environment was preventing navigation); however, to protect participants that might transcribe or copy-paste them into their browsers, we constantly monitored all URLs to ensure they were offline during the duration of the study.



% Bibliography entries for the entire Anthology, followed by custom entries
\bibliography{anthology, custom}
% Custom bibliography entries only
% \bibliography{custom}

\appendix

% \section{Example Appendix}
% \label{sec:appendix}

\paragraph{Step 1: Data Extraction} We obtained the ASLLRP dataset from the project web interface\footnote{DAI 2: \url{https://dai.cs.rutgers.edu/dai/s/cart}, login required.}. The dataset includes ASL sentence-level signed videos and XML files\footnote{These XML files are generated from the SignStream annotation tool. More details about these files can be found here: \url{https://dai.cs.rutgers.edu/downloads/XML-Export-format.pdf}.} containing corresponding English translations and annotations. For the translation task in Module 1, we focused on extracting manual information from the textual annotations to capture the primary meaning of the English translations. Specifically, we extracted existing English sentences from the XML files and systematically spliced English-based annotations, including vocabulary and compound symbols, fingerspelling, name signs, classifiers, locative words, and gestures, in chronological order. In total, we extracted 2,119 English sentences with corresponding English-based glosses. Additionally, we trimmed the signing videos based on the XML data so that each English sentence corresponds to a specific sign language video (utterance) for our subsequent tasks.

\paragraph{Step 2: Data Cleaning} Following a similar approach to prior work~\cite{amin_sign_2021}, we removed gloss annotations that did not alter the overall meaning of the sentences when omitted, such as repetition (annotated as a single or multiple ``+'' signs), number of signing hands (annotated as ``(1h)'' and ``(2h)''), and signs indicator that both hands move in an alternating manner (annotated as ``alt.''). To reduce translation errors, we standardized all fingerspelling-related glosses from fs-XXX to fs-X-X-X (\eg from ``fs-JOHN'' to ``fs-J-O-H-N'') and unified annotations for spatial locations (\eg ``i:GIVE:j'' and ``i:GIVE:k'' were standardized to ``i:GIVE:j''). While classifiers play a crucial role in ASL, we excluded them from this work because they typically appear only once or very few times in the datasets, so there was insufficient data for effective model prompting. After data cleaning, we retained 1,843 English sentences with corresponding English-based glosses for the remaining experiments. 

\paragraph{Step 3: Text-to-Gloss Dictionary Generation} To improve consistency in sign representations across different sentences and datasets, we constructed a text-to-gloss dictionary using the ASLLRP Sign Bank\footnote{\url{https://dai.cs.rutgers.edu/dai/s/signbank}}, which contains isolated signs along with their corresponding English-based glosses and translations. We then systematically unified the glosses based on step 2 to ensure consistency between the dictionary and the gloss annotations for the sentences. During the dictionary generation, we observed that some words may have variants of glosses depending on the context (\eg ``ask, inquire, query, question'' can be annotated as ``ASK'', ``ASK:i'', or ``i:ASK:j'', depending on whether the previous and following words are signed in a neutral location). Therefore, our dictionary employs a one-to-multiple mapping, accommodating the variability in gloss annotations. In total, the dictionary contains 3,915 text-to-gloss pairs. Notably, we identified 43 words that do not have corresponding glosses (\ie out-of-vocab words). For these words, which lack corresponding videos, fingerspelling is used as an alternative. 

\paragraph{Step 4: Ground True Correction} During the process of extracting ground truth from XML files to determine whether a sentence is a yes/no question, wh- question, conditional statement, and/or contains negation, we discovered that the ground truth labels were based on the signing rather than the English text, leading to some misalignments between the English text and the linguistic labels. For example, ``I guarantee that the parents will be mad if the children dye their hair orange'' was originally labeled as a negation statement, because the signing of it contains negation, although the English sentence does not. To address these issues, four of our researchers iteratively re-labeled and discussed the test set sentence categories, refining the labels to better reflect the text content. These revised labels were then used as the ground truth, allowing us to calculate precision and recall for each sentence type predictions and to identify patterns in the model's errors.
\section{Examples of Questions from the Dataset}
\label{app:example_questions}


The following are two examples of questions from the crowdsourced dataset:

\begin{enumerate}
    \item  \textbf{Question}: How do I get better audio quality. What are the connections guidelines for it?

    \textbf{Retrieved Document}:

    For better audio quality, it is a good idea to use an AV receiver.
    
    If you connect an external audio device using an optical cable, the Sound Output setting is automatically changed to the connected device. However, to make this happen, you must turn on the external audio device before connecting the optical cable. To manually change the Sound Output setting, do one of the following:
    
    - Use the Quick Settings screen to change to the connected device: Use the Select button to select Audio Out/Optical on the Sound Output menu. ([HOME] > [SETTINGS] Settings > up directional button > Sound Output).
    
    - Use the Settings screen to change to the connected device: Select Audio Out/Optical on the Sound Output menu. ([HOME] > [SETTINGS] Settings > Sound > Sound Output).
    
    An unusual noise coming from a connected audio device while you are using it may indicate a problem with the audio device itself. If this occurs, ask for assistance from the audio device's manufacturer.
    
    Digital audio is only available with 5.1 channel broadcasts.

    \item  \textbf{Question}: How do I access the main accessibility menu to change Voice Guide settings?

    \textbf{Retrieved Document}:

    You can also go to an accessibility menu from the TV settings menu. This provides more options, for example, to change the speed of Voice Guide.
    
    The TV will not verbalize this menu unless Voice Guide is already turned on.
    
    1. Press the HOME button.
    
    2. Press the left directional button until you reach Settings.
    
    3. Press Select and a menu will open.
    
    4. Press the down directional button to reach General, and then press Select to open this menu.
    
    5. Use the directional buttons to go to the Accessibility menu, and then press Select to open this menu.
    
    6. The menu will appear with Voice Guide Settings being the first menu. Highlight Voice Guide Settings, and then press Select.
    
    7. A menu appears with the options to change Voice Guide and Volume, Speed, Pitch.
    
    8. Select the menu using the directional buttons, and then press Select.
    
\end{enumerate}


The following are two examples of questions from the challenge set (from community forums):

\begin{enumerate}
    \item \textbf{Question}: Does this tv allow me to play contents from my ipad or iphone?
    
    \textbf{Retrieved Document}:
    
    English > Connections > Connecting Your Mobile Device > Text

    You can install the SmartThings app from App Store or Google Play Store.

    \textbf{Answer}: Yes.

    \item \textbf{Question}: What is the return policy if I don't like it?

    \textbf{Retrieved Document}:
    
    English > Troubleshooting > Getting Support > Requesting service

    [HOME] > Settings > Support > Request Support
    
    You can request service when you encounter a problem with the TV. Select the item matching the problem that you encountered, and then select Request Now or Schedule Appointment > Send. Your service request will be registered. The Samsung Contact Center will contact you to set up or confirm your service appointment.

    [NOTE] You must agree to the terms and conditions for the service request.
    
    [NOTE] This function may not be supported depending on the geographical area.
    
    [NOTE] This function requires an Internet connection.

    \textbf{Answer}: You won't want to return it as it's the best in its 32 inch class.
\end{enumerate}
\begin{lstlisting}[title={Sampling Responses During Training/Inference}]
Please reason step by step, and put your final answer within 
\boxed{}. 
Problem: {problem} 
\end{lstlisting}

\begin{lstlisting}[title={Verification Refinement}]
You are a math teacher. I will give you a math problem and an answer. 
Verify the answer's correctness without step-by-step solving. Use alternative verification methods. 
Question: {problem}
Answer: {answer}
Verification:
\end{lstlisting}

\begin{lstlisting}[title={Verification Collection}]
Refine this verification text to read as a natural self-check within a solution. Maintain logical flow and professionalism.
Key Requirements:
1. Avoid phrases like "without solving step-by-step" or "as a math teacher".
2. Treat the answer as your own prior solution.
3. Conclude with EXACTLY one of:
Therefore, the answer is correct.
Therefore, the answer is incorrect.
Therefore, the answer cannot be verified.
Original text: {verification}
\end{lstlisting}

\section{FactScore}
\label{app:factscore}


\subsection{Sentence Splitting}
The FactScore methodology consists of breaking generated responses first into sentences, and then takes each of those sentences and generates atomic facts from them. These atomic facts are then individually verified as True or False.

As mentioned, we found that FactScore often unfairly penalized responses that contained structured formatting (e.g., lists, topic headers) which are most common in the outputs of GPT-4o and SynthGPT. This seems to occur due to the sentence splitting not handling such formatting well. For example, this generated response from GPT:

\begin{quote}
    To get Samsung Remote Support for your TV, follow these steps:

    1. Call the Samsung Contact Center and ask for remote support.
    
    2. On your TV, go to the menu and select [HOME] > Settings > Support.
    
    3. In the Support section, select Remote Management, then read and agree to the service agreements.
    
    4.. When the PIN screen appears, provide the PIN number to the agent.
    
    5. The agent will then access your TV to provide remote support.

\end{quote}

When we calculate FactScore with the original preprocessing step (using sentence splitting), we get these atomic facts with their annotations:

%%%% ENDED UP SPLITTING THIS TABLE INTO TWO SO IT WOULD FIT WELL 
\begin{table}[H]
\centering
\begin{tabular}{cp{5.5cm}c}
1 & Samsung Remote Support is available for your TV. & True \\ 
2 & There are specific steps to follow to access Samsung Remote Support. & True \\ 
3 & \textbf{The steps for accessing support are not provided in the sentence.} & False \\ 
4 & You should call the Samsung Contact Center. & True \\ 
5 & \textbf{The number for the contact center is 1.} & False \\ 
6 & You should ask for remote support. & True \\ 
7 & You should go to the menu on your TV. & True \\ 
8 & Select [HOME]. & False \\ 
\end{tabular}
\end{table}


\begin{table}[H]
\centering
\begin{tabular}{cp{5.5cm}c}
9 & Choose Settings. & False \\ 
10 & Select Support. & False \\ 
11 & There is a Support section. & True \\ 
12 & Users should select Remote Management in the Support section. & True \\ 
13 & Users need to read the service agreements. & True \\ 
14 & Users are required to agree to the service agreements. & True \\ 
15 & A PIN screen appears. & True \\ 
16 & The user should provide the PIN number to the agent. & True \\ 
17 & \textbf{The PIN number is 5.} & False \\ 
18 & The agent will access your TV. & True \\ 
19 & The purpose of accessing the TV is to provide remote support. & False \\ 
\end{tabular}
\end{table}


In this example you can see that the deconstruction of the list makes the numbers confusing to the model (facts 5 and 17) and that the model is confused by not having access to the remainder of the response in fact 3.

In contrast, without the sentence splitting, the following facts are generated from this response:

\begin{table}[H]
\centering
\begin{tabular}{cp{5.5cm}c}

1 & To get Samsung Remote Support for your TV, you need to call the Samsung Contact Center. & True \\ 
2 & You should ask for remote support when you call. & True \\ 
3 & On your TV, you need to go to the menu. & True \\ 
4 & You should select [HOME] > Settings > Support. & True \\ 
5 & In the Support section, you need to select Remote Management. & True \\ 
6 & You must read and agree to the service agreements. & True \\ 
7 & When the PIN screen appears, you need to provide the PIN number to the agent. & True \\ 
8 & The agent will access your TV to provide remote support. & True \\ 
\end{tabular}
\end{table}


As you can see, these facts are much more sensible and better reflect the content of the response.



\subsection{\textit{I Don't Know} Responses}

As mentioned, FactScore turns out to be unhelpful in assessing ``I don't know'' responses. For example, the generated response is:

\begin{quote}
    Unfortunately, the provided section does not mention turning on the TV using voice. It only provides information on turning the TV on using the [POWER] button.
\end{quote}

And the decomposed atomic facts are:

\begin{table}[H]
\centering
\begin{tabular}{cp{5.5cm}c}
1 & The provided section does not mention turning on the TV using voice. & False \\
2 & It provides information on turning the TV on using the [POWER] button. & True \\
\end{tabular}
\end{table}

The resulting FactScore for this response is .5 (quite low) despite the response being appropriate. Because of this undesired penalty, we do not use FactScore to evaluate the challenge set, as it consists of mostly \textit{I don't know} responses.

\section{Human Evaluation Tutorial}
\label{app:human_eval}

Human evaluators were instructed to review the following slide deck prior to beginning the evaluation. The slides provide instructions for how to annotate items and examples of errors (from the dev set) -- see Appendix \ref{app:error_category_examples}.

\begin{figure}[H]
    \centering
    \includegraphics[width=0.5\textwidth]{figures/eval_tutorial_images/EvaluationTutorial1.jpg}
    \caption*{Slide 1}
    \label{fig:eval2}
\end{figure}

\vspace{-1cm}

\begin{figure}[H]
    \centering
    \includegraphics[width=0.5\textwidth]{figures/eval_tutorial_images/EvaluationTutorial2.jpg}
    \caption*{Slide 2}
    \label{fig:eval3}
\end{figure}

\vspace{-1cm}

\begin{figure}[H]
    \centering
    \includegraphics[width=0.5\textwidth]{figures/eval_tutorial_images/EvaluationTutorial3.jpg}
    \caption*{Slide 3}
    \label{fig:eval4}
\end{figure}


\noindent Slides 4 - 7 show the same examples as Appendix \ref{app:error_category_examples} and thus have been omitted here.


\begin{figure}[H]
    \centering
    \includegraphics[width=0.5\textwidth]{figures/eval_tutorial_images/EvaluationTutorial8.jpg}
    \caption*{Slide 8}
    \label{fig:eval9}
\end{figure}

\section{Error Category Examples}
\label{app:error_category_examples}

The following examples come from the regular and challenge dev sets and were included in the tutorial for human evaluation in Appendix \ref{app:human_eval}. The error categories are as follows:

\begin{itemize}
    \item \textbf{Hallucination}: The response contains information not present in the manual.
    \item \textbf{Non-Answer}: The response does not answer the question.
    \item \textbf{Partial Answer}: The response does not fully answer the question, or omits important information.
    \item \textbf{IDK - Bad}: The manual section has the information required to answer the question, but the response is mistakenly ``I don’t know''.
    \item \textbf{Disfluent}: The response contains grammatical or fluency problems.
    \item \textbf{Other}: The response contains some other type of error.
    \item \textbf{IDK - Good}: The manual section does not contain the information required to answer the question and the response is appropriately ``I don’t know''.
    \item \textbf{Good}: There are no errors.
\end{itemize}


\subsection{Example Set 1}

\textbf{Question}: Why the TV smells of plastic?

\vspace{.25cm}

\noindent\textbf{Retrieved Passage}:

\noindent Troubleshooting > Other Issues > The TV smells of plastic.

\noindent This smell is normal and will dissipate over time.

\vspace{.25cm}

\noindent \textbf{Responses:}

\begin{enumerate}
    \item \textbf{Response:} The smell of plastic is normal and will dissipate over time. \\
          \textbf{Evaluation:} Good \\
          \textbf{Explanation:} The answer is correct.
    \item \textbf{Response:} If the TV smells of plastic, you should contact customer service. \\
          \textbf{Evaluation:} Hallucination \\
          \textbf{Explanation:} This response contradicts the information in the retrieved passage.  
    \item \textbf{Response:} The TV smells of plastic is normal and will go away over time. \\
          \textbf{Evaluation:} Disfluent \\
          \textbf{Explanation:} The answer is not grammatically correct.               
    \item \textbf{Response:} I’m sorry, I’m not sure of the answer to that. You can check the user manual. \\
          \textbf{Evaluation:} IDK – Bad \\
          \textbf{Explanation:} The response is “I don’t know” but the retrieved passage DOES contain the answer, so the response should have answered the question.
\end{enumerate}


\subsection{Example Set 2}

\textbf{Question}: how many inches does it stick out off the wall?

\vspace{.25cm}

\noindent\textbf{Retrieved Passage}:

\noindent English > Precautions and Notes > Read After Installing the TV > Installing an Anti-theft Lock

\noindent An anti-theft lock is a physical device that can be used to protect the TV against theft. Look for the lock slot on the back of the TV. The slot has a [LOCK] icon next to it. To use the lock, wrap the lock cable around an object that is too heavy to carry and then thread it through the TV's lock slot. The lock is sold separately. The method of using an anti-theft lock may differ for each TV model. Refer to the lock's user manual for more information.

\noindent [NOTE] This function may not be supported depending on the model or geographical area.


\vspace{.25cm}

\noindent \textbf{Responses:}

\begin{enumerate}

    \item \textbf{Response:} The provided document does not mention the inches of protrusion off the wall. It only discusses the anti-theft lock installation. \\
          \textbf{Evaluation:}  IDK – Good \\
          \textbf{Explanation:} The answer cannot be found in the retrieved passage, so the response is appropriate.

\end{enumerate}



\subsection{Example Set 3}

\textbf{Question}: Does this tv have an antenna output and does it have the vesa hole pattern for wall mounting?

\vspace{.25cm}

\noindent\textbf{Retrieved Passage}:

\noindent English > Connections > Connecting an Antenna

\noindent You can connect an antenna cable to your TV.

\noindent [NOTE] An antenna connection is not necessary if you connect a cable box or satellite box.

\noindent [image\_7.png]

\vspace{.25cm}

\noindent \textbf{Responses:}

\begin{enumerate}
    \item \textbf{Response:} Yes, the TV has an antenna output, and you can connect an antenna cable to it. Additionally, the TV has a VESA hole pattern for wall mounting. \\
          \textbf{Evaluation:} Hallucination \\
          \textbf{Explanation:} While the answer to the antenna question is correct, the response makes up information about the VESA hole pattern.

    \item \textbf{Response:} Yes, this TV has an antenna output. \\
          \textbf{Evaluation:} Partial answer \\
          \textbf{Explanation:} The answer only contains a response to the first part of the question.

    \item \textbf{Response:} Yes, this TV has an antenna output, but the user manual passage doesn’t mention the VESA hole pattern for wall mounting, so I cannot answer that question. \\
          \textbf{Evaluation:} IDK – good \\
          \textbf{Explanation:} Even though the answer to the first question is correct, it still contains an “IDK” response to the second question
          
\end{enumerate}


\subsection{Example Set 4}

\textbf{Question}: Is this tv hdr 10?

\vspace{.25cm}

\noindent\textbf{Retrieved Passage}:

\noindent English > Quick Guides > Connecting the Samsung Smart Remote to the TV

\noindent Connect the Samsung Smart Remote to your TV to operate the TV.
When you turn on the TV for the first time, the Samsung Smart Remote pairs to the TV automatically. If the Samsung Smart Remote does not pair to the TV automatically, point it at the remote control sensor of the TV, and then press and hold the [GO BACK] and [PLAY] buttons simultaneously for 3 seconds or more.

\noindent [image\_1.png]

\noindent [NOTE] The images, buttons, and functions of the Samsung Smart Remote may differ depending on the model.


\noindent [NOTE] For more information about the Samsung Smart Remote that comes with the QLED TV, refer to "About the Samsung Smart Remote (QLED TV)."

\noindent [NOTE] For more information about the Samsung Smart Remote that comes with the UHD TV, refer to "About the Samsung Smart Remote (UHD TV)."

\noindent [NOTE] The Samsung Smart Remote may not be supported depending on the model or geographical area.

\vspace{.25cm}

\noindent \textbf{Responses:}

\begin{enumerate}
    \item \textbf{Response:} No, this document does not mention HDR 10 or any other HDR format. \\
          \textbf{Evaluation:}  Other \\
          \textbf{Explanation}: This response has an “other” error because it correctly says “IDK” but it starts the response with “No, …” which makes it seem like the answer to the user’s question is no, which is misleading.
\end{enumerate}
\clearpage

\onecolumn

\section{Human Evaluation: Regular and Challenge Test Sets}
\label{app:human_eval_breakdown}

\subsection{Regular Test Set Results}

\begin{table}[H]
    \centering
    \renewcommand{\arraystretch}{1.2} % Adjust row spacing for better readability
    \resizebox{\textwidth}{!}{ % Automatically resizes to fit within text width
    \begin{tabular}{lcccccccc|c}
        \toprule
        \textbf{Model} & \textbf{Halluc.} & \textbf{Non-Ans} & \textbf{Partial} & \textbf{IDK - Bad} & \textbf{Disfl.} & \textbf{Other} & \textbf{IDK - Good} & \textbf{Good} & \textbf{Total Good} \\
        \midrule
        Pretrained  & 4  & 0  & 4  & 0  & 0  & 0 & 1  & 41  & 42 \\
        GPT-4o  & 2   & 0  & 1  & 0  & 0  & 0 & 1  & 46  & 47 \\
        \hline
        Manual  & 4  & 0  & 5  & 0  & 1 & 0 & 1  & 39  & 40 \\
        Autocleaned\textsubscript{G}  & 4  & 0  & 4  & 0  & 2  & 0 & 0  & 40  & 40 \\
        \hline
        SynthGPT  & 2  & 0  & 0  & 0  & 2  & 0 & 1  & 45  & 46 \\
        SynthLlama  & 2  & 0  & 1  & 0  & 1  & 0 & 1  & 45  & 46 \\
        SynthLlama+  & 2  & 0  & 0  & 0  & 1  & 0 & 1  & 46  & 47 \\
        \bottomrule
    \end{tabular}
    }
    \caption{Human evaluation results on the Regular Test set, assessing response quality across various error categories. Majority vote determined the final category for each item.}
    \label{tab:humaneval_updated_2}
\end{table}

\subsection{Challenge Test Set Results}

\begin{table}[H] % Use table* for spanning both columns
    \centering
    \renewcommand{\arraystretch}{1.2} % Adjust row spacing for better readability
    \resizebox{\textwidth}{!}{ % Automatically resizes to fit within text width
    \begin{tabular}{lcccccccc|c}
        \toprule
        \textbf{Model} & \textbf{Halluc.} & \textbf{Non-Ans} & \textbf{Partial} & \textbf{IDK - Bad} & \textbf{Disfl.} & \textbf{Other} & \textbf{IDK - Good} & \textbf{Good} & \textbf{Total Good} \\
        \midrule
        Pretrained  & 9  & 0  & 2  & 0  & 1  & 5 & 23  & 10  & 33 \\
        GPT-4o  & 7   & 0  & 1  & 1  & 0  & 0 & 28  & 13  & 41 \\
        \hline
        Manual  & 10  & 2  & 2  & 0  & 2 & 5 & 20  & 9  & 29 \\
        Autocleaned\textsubscript{G}   & 9  & 0  & 2  & 0  & 0  & 9 & 19  & 11  & 30 \\
        \hline
        SynthGPT  & 7  & 1  & 0  & 2  & 1  & 8  & 21  & 11  & 32 \\
        SynthLlama  & 5  & 0  & 1  & 0  & 1  & 7  & 25  & 11  & 36 \\
        SynthLlama+  & 4  & 0  & 0  & 0  & 0  & 2  & 30  & 14  & 44 \\
        \bottomrule
    \end{tabular}
    }
    \caption{Human evaluation results on the Challenge Test Set, assessing response quality across various error categories. Majority vote decided the final category for each item.}
    \label{tab:humaneval_updated}
\end{table}
\twocolumn

\section{Human vs. Synthetic Data Analysis}
\label{app:bertscores}

In order to get a better sense of the differences between the datasets, we plot the distribution of BERTScores for each. As you can see, the human-written questions cluster lower, meaning that fewer questions are very similar to each other. Both sets of synthetic questions cluster higher and more evenly, suggesting less variety.

\begin{figure}[H]
    \centering
    \includegraphics[width=0.4\textwidth]{figures/f1_distribution_human.png}
    \caption{Distribution of the BERTScores for every combination of two questions in the crowdsourced dataset.}
    \includegraphics[width=0.4\textwidth]{figures/f1_distribution_synth_gpt.png}
    \caption{Distribution of the BERTScores for every combination of two questions in the SynthGPT dataset.}
    \includegraphics[width=0.4\textwidth]{figures/f1_distribution_synth_llama.png}
    \caption{Distribution of the BERTScores for every combination of two questions in the SynthLlama dataset.}
    \label{fig:bertscore_plots}
\end{figure}


Further, we utilize a t-distributed Stochastic Neighbor Embedding (t-SNE) plot to visualize the embedding space of three datasets: human-generated questions, synthetic questions generated by LLaMA, and synthetic questions generated by GPT. The embeddings are extracted from Llama-3-8B-Instruct (the model we finetune in all our experiments), and the t-SNE method reduces the high-dimensional embeddings into a two-dimensional space for visual interpretation.

This visualization allows us to compare the semantic distributions of the datasets and assess how closely the synthetic datasets align with the human-generated questions. Distinct clustering of the datasets in the t-SNE space suggest meaningful differences in their semantic representations. It seems that the two synthetic questions overlap a great deal and have a fair amount of overlap with the crowdsourced questions. However, the crowdsource (human) questions cluster distinctly to the right, outside the space of the synthetic questions. This also suggests greater variety in the crowdsourced questions.

\begin{figure}[H]
    \centering
    \includegraphics[width=0.5\textwidth]{figures/tsne_plot_all_sets.png}
    \caption{Distribution of the BERTScores for every combination of two questions in the crowdsourced dataset.}
    \label{fig:tSNE}
\end{figure}


\end{document}
