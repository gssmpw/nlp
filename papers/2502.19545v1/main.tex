% This must be in the first 5 lines to tell arXiv to use pdfLaTeX, which is strongly recommended.
\pdfoutput=1
% In particular, the hyperref package requires pdfLaTeX in order to break URLs across lines.

\documentclass[11pt]{article}

% Change "review" to "final" to generate the final (sometimes called camera-ready) version.
% Change to "preprint" to generate a non-anonymous version with page numbers.
% \usepackage[review]{acl}
% \usepackage{acl}
\usepackage[preprint]{acl}

% Standard package includes
\usepackage{times}
\usepackage{latexsym}

% For proper rendering and hyphenation of words containing Latin characters (including in bib files)
\usepackage[T1]{fontenc}
% For Vietnamese characters
% \usepackage[T5]{fontenc}
% See https://www.latex-project.org/help/documentation/encguide.pdf for other character sets

% This assumes your files are encoded as UTF8
\usepackage[utf8]{inputenc}

% This is not strictly necessary, and may be commented out,
% but it will improve the layout of the manuscript,
% and will typically save some space.
\usepackage{microtype}

% This is also not strictly necessary, and may be commented out.
% However, it will improve the aesthetics of text in
% the typewriter font.
\usepackage{inconsolata}

%Including images in your LaTeX document requires adding
%additional package(s)
\usepackage{graphicx}
\usepackage{amsmath}
\usepackage{booktabs}
\usepackage{float} 


% If the title and author information does not fit in the area allocated, uncomment the following
%
%\setlength\titlebox{<dim>}
%
% and set <dim> to something 5cm or larger.


\title{Winning Big with Small Models: \\Knowledge Distillation vs.\ Self-Training\\for Reducing Hallucination in QA Agents}

\author{
 \textbf{Ashley Lewis\textsuperscript{1}}
 \textbf{Michael White\textsuperscript{1}}
 \textbf{Jing Liu\textsuperscript{2}}
 \textbf{Toshiaki Koike-Akino\textsuperscript{2}}
 \textbf{Kieran Parsons\textsuperscript{2}}
 \textbf{Ye Wang\textsuperscript{2}} \\
 \textsuperscript{1}The Ohio State University,  
 \textsuperscript{2}Mitsubishi Electric Research Laboratories \\
 \small{
 \{\href{mailto:lewis.2799@osu.edu}{lewis.2799}, 
 \href{mailto:white.1240@osu.edu}{white.1240}\}@osu.edu,  
 \{\href{mailto:jiliu@merl.com}{jiliu}, 
 \href{mailto:koike@merl.com}{koike}, 
 \href{mailto:parsons@merl.com}{parsons}, 
 \href{mailto:yewang@merl.com}{yewang}\}@merl.com}
}



\begin{document}
\maketitle
\begin{abstract}
The deployment of Large Language Models (LLMs) in customer support is constrained by hallucination—generating false information—and the high cost of proprietary models. To address these challenges, we propose a retrieval-augmented question-answering (QA) pipeline and explore how to balance human input and automation. Using a dataset of questions about a Samsung Smart TV user manual, we demonstrate that synthetic data generated by LLMs outperforms crowdsourced data in reducing hallucination in finetuned models. We also compare self-training (fine-tuning models on their own outputs) and knowledge distillation (fine-tuning on stronger models' outputs, e.g., GPT-4o), and find that self-training achieves comparable hallucination reduction.  We conjecture that this surprising finding can be attributed to increased exposure bias issues in the knowledge distillation case and support this conjecture with post hoc analysis. We also improve robustness to unanswerable questions and retrieval failures with contextualized “I don’t know” responses. These findings show that scalable, cost-efficient QA systems can be built using synthetic data and self-training with open-source models, reducing reliance on proprietary tools or costly human annotations.
\footnote{This work was conducted while Ashley Lewis was interning at Mitsubishi Electric Research Laboratories.}



\end{abstract}


\documentclass[../main.tex]{subfiles}
\graphicspath{{../images/}}
\makeatletter
\def\input@path{{../images/}}
\makeatother
\begin{document}
\section{Introduction}
\begin{figure}
\centering
\begin{tikzpicture}
\node[inner sep=0pt] (ws) at (0, 0) {
\includegraphics[height=.4\textwidth, trim={10cm 0 10cm 0},clip]{world_space.png}};
\node[inner sep=0pt] (cs) at (6,0) {\includegraphics[height=.4\textwidth, trim={10cm 1cm 10cm 4cm},clip]{conf_space.png}};
\end{tikzpicture}
\vspace{-5pt}
\label{fig:pbrm_intro}
\caption{\textbf{Left}: Shows world space obstacles as grey spheres. Robots start and goal configuration is colored red and green, respectively. Configurations along the computed path are colored transparent blue. \textbf{Right:} Mapped world space scenario to configuration space. Obstacle region is the grey mesh. Red spheres are collision-free regions computed by the neural SCDF. The optimized shortest path in the convex corridor is the blue curve.}
\vspace{-25pt}
\end{figure}
Motion planning is the problem of finding a collision-free trajectory that connects a given start and goal configuration. The planning takes place in the configuration space of the robot. For single body robots, like mobile robots or drones, the configuration space and the world space are usually the same. This simplifies the planning, since explicit obstacle representations are available which enables geometrical tools like separating hyperplanes, smallest distance to obstacles etc., to be used when designing motion planning algorithms. For multi-body robots like manipulators, the situation is completely different. The world space obstacles are usually mapped to non-convex regions, and to make the problem even harder, the mapping is usually not known. Forming explicit representations of the obstacle region in the configuration space is usually too expensive or intractable. Despite all of this, sampling based planners are used with great success, which mainly is due to their use of implicit representations of the obstacle region. The basic idea is to construct a graph in the configuration space that covers and connects the collision-free region. From this graph, a path can be extracted that connects a given start and goal configuration. The approach is computationally expensive, since the graph is constructed with the smallest geometrical building block available, points, which represents a collision-check. Furthermore, the extracted paths from the graph are non-smooth and jagged due to the stochastic nature of the approach. This adds an additional post-processing step to the process, where the paths are shortcutted and smoothened, before the path can be used for tracking. Clearly a lot of time is invested to form this graph and produce smooth paths. Thus, if the obstacles start to move, then all of this work is done in no use, since all points that make up this graph need to be re-verified, which is simply too time consuming to be done in real time.
\\\\
In this work, we want to address the existing drawbacks of the sampling based planners. Our main contribution is an improved motion planner where each vertex in the graph covers a collision-free region in the form of a sphere instead of a point and where the edges are formed with neighboring intersecting spheres. This representation has the advantage of instead of returning piecewise linear paths, returning a sequence of overlapping spheres, i.e. a convex corridor, that connects a given start and goal configuration, illustrated in Figure \ref{fig:pbrm_intro}. This convex corridor allows us to use convex optimization to produce smooth trajectories, instead of computationally expensive post-processing methods. The representation further allows us to estimate the coverage of the collision-free space, which gives us awareness and feedback in the offline roadmap construction phase. Finally, our representation is simple to adapt to moving obstacles, simply requery for the new radii and recheck for intersections. 
\\\\
The spherical collision-free regions are formed using a signed distance function (SDF), which is a function that returns the smallest distance from an arbitrary point to the boundary of an obstacle. As the name implies, the distance is signed, thus if the point is inside the obstacle it is negative otherwise positive. If the distance is positive, a sphere with radius equal to the distance is guaranteed to cover a collision-free region. Using an SDF in motion planning is not new, but what is novel about our approach is that we express the distance in the configuration space instead of the world space and by doing so allows us to form these convex collision-free regions. We refer to the resulting SDF as a signed configuration distance function (SCDF). Computing an SCDF analytically is non-trivial, our approach is therefore to parameterize the SCDF with a deep neural network and learn the mapping by supervised learning. Our resulting neural SCDF can compute distances for different parameter values of obstacle shapes and we also show how multiple distances can be combined, thus making our approach flexible.
\section{Related work}
Motion planning algorithms can roughly be divided into three families, grid-based, sampling based and optimization based methods. Grid-based methods (GBM) discretize the planning space from which a graph is then compiled. A standard search method is A$^\star$ \citep{a_star}, which is classified as an \textit{informed} search method, since it employs a heuristic function to speed up the search. A$^\star$ guarantees to return an optimal path at the level of discretization used. GBMs usually discretize the planning space by a regular lattice and this limits the GBMs to problems with low dimensionality due to the curse of dimensionality. Thus, GBMs are usually limited to single-body robots where the degrees of freedom (DOF) are low. To overcome the inherent scaling problem with the GBMs, stochastic methods are usually used for multi-body robots. These methods are termed as sampling-based methods (SBM) and core members within this family are the rapidly-exploring random trees (RRT) \citep{rrt} and the probabilistic roadmap (PRM) \citep{prm}. RRT grows a tree from the start configuration and explores the collision-free region in a rapid way until it is able to connect to the goal region. RRT is usually improved by bi-directional planning \citep{rrt_connect}, i.e. an additional tree is grown from the goal configuration and the trees are tested for connection after any tree has been expanded. RRT is a single-query method, thus it searches for a path from scratch each time it is queried. Contrary to this, PRM is a multi-query method, which solves for multiple queries without starting from scratch. PRM does this by creating a roadmap (graph) that covers the collision-free space as an offline step. The graph is then used to solve for multiple queries. PRMs are used in cases where the environment does not change since the extra offline step is too computationally costly and needs to be re-done if the environment is changed. In our work, we address this inherent issue by using a different roadmap representation. Our vertices in the graph cover a collision-free region in the form of spheres and we form the edges by checking for intersecting spheres. If something in the environment changes, we recompute the spheres radii and recheck the intersections, without relying on collision detection. We use a trained neural network to compute the sphere radius, therefore querying for the radius can be done fast, hence our representation enables the PRM for dynamic environments.
\\\\
In the recent decades, optimization based methods (OBM) \citep{chomp, schulman, itomp, stomp} have been introduced as an alternative to SBM for multi-body robots. Like the SBM, the OBMs scale well to higher dimensional problems and produce smoother motion. It is common to use a SDF in the optimization since it is a smooth function, thus enabling gradient-based methods. However, the standard way of expressing the SDF is in world space. The distance therefore needs to be mapped to the configuration space by the forward kinematics. This mapping makes the optimization problem a non-linear program (NLP), which is computationally expensive to solve. Recently, a different approach has been proposed. In \cite{mp_gcs} motion planning is formulated as a convex optimization problem by using the graph of convex sets framework \citep{gcs}. The underlying idea is to decompose the collision-free space into intersecting convex sets from which a convex optimization problem is formulated. In cases where an explicit representation of the obstacles in the configuration space exists, like for single-body robots, creating collision-free convex regions can be done fast \citep{iris}. For multi-body robots, this is non-trivial. Existing work does this successfully \citep{iris_nlp, iris_c} by an optimization based approach, but the methods are still too time consuming to be used in the presence of moving obstacles. Our approach is instead to use deep learning to learn an SDF expressed in the configuration space. With this, we can query for shortest distances to the collision boundary, which allows us to expand spherical regions which are collision-free. Our approach is fast and therefore enables our suggested roadmap planner to be used in dynamic environments.
\\\\
Recent research has focused on learning collision detection \citep{fk_kernel_distance, diffco, graphdistnet} by predicting the signed distance between the robot links and the surrounding obstacles in the world space. The learned SDF is used in trajectory optimization but since the distance is expressed in the world space, the problem becomes an NLP and therefore takes a long time to solve. We take a novel approach and suggest to instead express the signed distance in the configuration space. This allows us to improve the PRM at the same time as it enables convex optimization for trajectory optimization, which runs faster and is more reliable than NLP solvers. In \cite{cspf} a learned signed distance function in the configuration space is proposed similar to our approach. However, their approach is restricted to point cloud representations, while we propose to represent the obstacles as parameterized geometric shapes, e.g. spheres. Furthermore, we also show how to use our learned SCDF to improve an existing roadmap planner.
\section{Problem formulation}
A robot is located in the world space, $\W \subset \R^3 $. The unique location of the robot is given by its configuration $\q \in \C$, where $\C$ is the configuration space. The set of points covered by the robots bodies at a certain configuration is expressed as $\B(\q) \subset \W$. The robot is surrounded by $\NrObst$ obstacles $\O = \bigcup_{i=1}^{\NrObst} \O_i$, where  $\O_i \subset \W$. The representation of the obstacle in the configuration space is the set $\C\O_i = \{\q \in \C \: |\: \B(\q) \cap \O_i \neq \emptyset \}$. The obstacle space is formed as $\Co = \bigcup_{i=1}^{\NrObst} \C \O_i$. The complement is referred to as the free space, $\Cf = \C \setminus \Co$. The path planning problem is a tuple, ($\Cf$, $\qStart$, $\qGoal$), where we want to connect a query pair, consisting of a start, $\qStart$, and goal configuration, $\qGoal$, with a geometric path, $\q(s): [0, 1] \mapsto \Cf$, such that $\q(0)=\qStart$ and $\q(1)=\qGoal$, or report correctly when such a path does not exist.
\end{document}


\section{Related Work}
% \subsection{Vision Language Model}
% 시각장애인에서 상황을 설명할 DB가 없으니 만들었다. 그리고 이를 VLM에 튜닝했다.
\subsection{Technical approaches for assisting the visually-impaired}


\subsection{Datasets for visual instruction tuning}

\section{Data and Experimental Setup}
\label{sec:data}

\subsection{Datasets}

The primary dataset consists of 684 crowdsourced questions paired with retrieved passages from the manual \cite{nandy-etal-2021-question-answering}. We split the dataset into 534 training, 100 development, and 50 test questions (our ``regular test set''). Dataset preprocessing details can be found in Appendix \ref{app:data_preprocessing}. We focused on this dataset because many existing QA datasets either lack grounding documents or prioritize open-domain QA, which does not align with the controlled, retrieval-augmented QA setting we aimed to study. This approach also allowed us to conduct a deep-dive analysis into the trade-offs between self-training, knowledge distillation, and synthetic data generation in mitigating hallucinations within a well-defined context.

As mentioned, the dataset also contains a collection of 3,000 questions sourced from community forums. We create challenge sets by randomly selecting 100 development and 100 test questions from this set. These questions are noisier and less than half are answerable, which allows us to evaluate how well models handle particularly challenging cases. Examples from both types of questions can be found in Appendix \ref{app:example_questions}.

\subsection{Training Data}

\begin{table}[t]
    \centering
    \renewcommand{\arraystretch}{1.2} % Adjust row spacing
    \setlength{\tabcolsep}{6pt} % Adjust column spacing
    \begin{tabular}{l|c}
        \toprule
        \textbf{Model} & \textbf{FactScore} \\
        \midrule
        Llama-3      & 0.9077 \\
        GPT-4o         & 0.9323 \\
        \hline
        Uncleaned       & 0.8798 \\
        Manual cleaned  & 0.8810 \\
        Autocleaned\textsubscript{L} & 0.8202 \\
        Autocleaned\textsubscript{G}     & 0.8966 \\
        \hline
        SynthGPT   & 0.9116 \\
        SynthLlama & 0.9211 \\
        SynthLlama+ & \textbf{0.9461} \\
        \bottomrule
    \end{tabular}
    \caption{FactScore results for the test set. Pretrained base models: Llama-3 and GPT-4o. Finetuned Llama-3-B models on the \citet{nandy-etal-2021-question-answering} dataset: Uncleaned (no data cleaning performed), Manual cleaned (cleaning done by the first author), Autocleaned\textsubscript{L} and Autocleaned\textsubscript{G} (cleaning done by Llama-3-70B and GPT-4o, respectively). Finetuned Llama-3-B models on synthetic data: SynthGPT (trained on data generated by GPT-4o), SynthLlama (trained on data generated by Llama-3-8B), and SynthLlama+ (same as SynthLlama, with additional negative examples).}
    \label{tab:factscore_test_set}
\end{table}

\paragraph{Regular Training Data}

We use the pretrained Llama-3-8B-Instruct \cite{dubey2024llama3herdmodels} to generate answers for the 534 training questions. Three datasets are created:
(1) a manually cleaned version where responses were reviewed and corrected by the first author, and
(2)--(3) automatically cleaned versions using GPT-4o and Llama-3-70B, respectively. This allows a systematic evaluation of the trade-offs between human effort and automated cleaning. As shown in Table \ref{tab:factscore_test_set}, cleaning with Llama-3 was largely unsuccessful. Thus in the remaining experiments GPT-4o was used for the cleaning task. We anticipate that improvements in open-source models like Llama-3 may reduce reliance on proprietary alternatives in the future. Prompts for both data generation and cleaning can be found in Appendix \ref{app:prompts}.

\paragraph{Synthetic Data} 

In addition to crowdsourced training questions, we generate fully synthetic QA data using LLMs. Specifically, we prompt Llama-3 and GPT-4o to generate new QA pairs based on passages from the Samsung Smart TV manual. To ensure that these datasets have comparable information coverage to the crowdsourced dataset and to prevent retrieval quality from being a confounding factor, we select passages systematically rather than randomly. We identify all 208 unique sections in the manual that are referenced in the crowdsourced training data. From these passages, we generate two synthetic QA pairs per passage, two from Llama-3 and two from GPT-4o. This approach ensures that the synthetic datasets are no larger than the crowdsourced dataset and cover similar content while maintaining consistency in passage selection. In a real-world application, this limitation does not exist, as synthetic training data can be generated from any number of passages. Thus, coverage is not inherently a bottleneck when using synthetic data in practical settings.

\subsection{Baseline and Experimental Models}

To evaluate the impact of data cleaning type and synthetic training data on hallucination reduction, we experiment with both pretrained models and finetuned models trained on different datasets.

\paragraph{Baseline Models}  
\begin{itemize}
    \item \textbf{Pretrained Llama-3-8B-Instruct (Llama-3)}: An open-source model that serves as a strong starting point for retrieval-augmented generation (RAG) without task-specific adaptation \cite{dubey2024llama3herdmodels}. The model is run with few-shot prompting.
    \item \textbf{GPT-4o}: A state-of-the-art proprietary model, included as a benchmark to assess how well finetuned open-source models compare to a highly optimized general-purpose system \cite{openai2024gpt4technicalreport}. The model is run with few-shot prompting.
\end{itemize}

\paragraph{Finetuned Models}  

We finetune Llama-3 on different variations of training data to analyze the effects of data source, cleaning method, and exposure bias on hallucination rates. Specifically, we train models on the following datasets using the \citet{zheng-etal-2024-llamafactory} finetuning framework and parameters:
\begin{itemize}
    \item \textbf{Manually Cleaned Training Data}: A dataset where the first author reviewed and corrected Llama-3-generated answers to the \citet{nandy-etal-2021-question-answering} 534 crowdsourced training questions.
    \item \textbf{Automatically Cleaned Training Data}: A version of the training set where errors in Llama-3-generated answers were identified and repaired using GPT-4o.
    \item \textbf{Synthetic Data (Llama vs.\ GPT)}: Two datasets where  416 QA pairs were generated by either Llama-3 or GPT-4o based on passages from the Samsung Smart TV manual. All synthetic data was cleaned using GPT-4o.
    \item \textbf{Synth Llama+}: Trained on the synthetic Llama data, and augmented with 100 negative examples (see section \ref{synth_llama+} for more details).
\end{itemize}

\subsection{Metrics for Evaluation}

We evaluate model performance using two methods: FactScore \cite{min-etal-2023-factscore}, an automated metric for factual accuracy, and human evaluation by trained annotators. These complementary approaches measure factual consistency and response quality.

\paragraph{FactScore}

FactScore evaluates whether a model's response aligns with a reference document. It works by decomposing a response into sentences, breaking each sentence into discrete factual claims, and verifying their alignment with the reference text. FactScore measures the proportion of supported claims while penalizing hallucinated content. However, responses from GPT-4o and SynthGPT, which often use structured formatting (e.g., lists, topic headers), cause FactScore to produce fragmented or nonsensical claims, unfairly penalizing these models. To address this, we removed the sentence-splitting preprocessing and instead generated atomic facts directly from the full response.

FactScore, which we computed using GPT-4o-mini, has been shown to be a reliable proxy for factuality, correlating well with human judgments \cite{min-etal-2023-factscore}. However, we find that it is unsuitable for evaluating \textit{I don’t know} responses. Thus, we applied FactScore only to the regular test set (mostly answerable questions), excluding the challenge set (many unanswerable questions). We also used it to evaluate human-written training questions for synthetic models, as they do not see these at training time and it provides a more robust evaluation. Further information in Appendix \ref{app:factscore}.


\begin{table}[t]
    \centering
    \renewcommand{\arraystretch}{1.3} % Adjust row spacing
    \begin{tabular}{p{2cm}p{4.75cm}}  % 10cm width for descriptions
        \toprule
        \textbf{Category} & \textbf{Description} \\
        \midrule
        \textbf{Hallucination} & The response contains information not present in the manual. \\
        \textbf{Non-Answer} & The response does not answer the question. \\
        \textbf{Partial answer} & The response does not fully answer the question, or omits important information. \\
        \textbf{IDK - Bad} & The manual section has the information required to answer the question, but the response is mistakenly ``I don’t know''. \\
        \textbf{Disfluent} & The response contains grammatical or fluency problems. \\
        \textbf{Other} & The response contains some other type of error. \\
        \textbf{IDK - Good} & The manual section does not contain the information required to answer the question and the response is appropriately ``I don’t know''. \\
        \textbf{Good} & There are no errors. \\

        \bottomrule
    \end{tabular}
    \caption{Response error categories and their descriptions. Examples can be found in Appendix \ref{app:error_category_examples}.}
    \label{tab:error_categories}
\end{table}

\paragraph{Human Evaluation} 
To obtain a more nuanced assessment of response quality, we conducted a human evaluation with three fluent English speaking, Linguistics PhD students (instructions in Appendix \ref{app:human_eval}), who annotated each model-generated response for the regular test set (50 items) and 50 items from the challenge set. They assigned to each response one of the categories listed in Table \ref{tab:error_categories} (examples in Appendix \ref{app:error_category_examples}), which were determined by an author analysis of the dev set. Three-way agreement occurred between annotators 63.14\% of the time and two-way agreement occurred 36.43\% of the time. Krippendorff's Alpha was $\alpha$ = 0.625, indicating substantial agreement.

Each response was labeled independently by all three annotators. The final assigned label was determined by a majority vote. In the few cases where annotators provided three different labels, the response was assigned the most severe error based on the following predefined ranking:  Hallucination > Non-Answer > Partial Answer > IDK - Bad > Disfluent > Other. The purpose of this ranking is to prioritize hallucination and content errors. For example, if a response is labeled as ``Hallucination,'' ``Good,'' and ``Partial Answer,'' it is assigned the final label of ``Hallucination'' due to its higher severity in the ranking.

By combining automated and human evaluation, we ensure a comprehensive analysis of both quality and factual consistency in model-generated responses. The aggregated results can be found in Table \ref{tab:humaneval} and the separate results on the regular and challenge test sets can be found in Appendix \ref{app:human_eval_breakdown}.

\begin{table}[t]
    \centering
    \small % Reduce font size for better fit
    \renewcommand{\arraystretch}{1.2} % Adjust row spacing
    \setlength{\tabcolsep}{4pt} % Reduce column padding
    \begin{tabular}{lccc}
        \toprule
        \textbf{Model} & \textbf{Chall. (100)} & \textbf{Reg. (50)} & \textbf{Total (150)} \\
        \midrule
        Pretrain & 26.56 & 28.74 & 27.29 \\
        GPT-4o & 22.23 & 31.56 & 25.34 \\
        \hline
        Manual & 21.74 & 28.54 & 24.01 \\
        Auto-cleaned & 26.33 & 31.00 & 27.89 \\
        \hline
        SynthLlama & 36.06 & 44.56 & 38.89 \\
        SynthGPT & 40.40 & 47.34 & 42.71 \\
        SynthLlama+ & 21.92 & 42.06 & 28.63 \\
        \bottomrule
    \end{tabular}
    \caption{Average response lengths for different models across challenge and regular test sets.}
    \label{tab:lengths}
\end{table}



\section{Results and Analysis}
\label{results_and_analysis}

\begin{figure}[!t]

    \centering
    \includegraphics[width=0.9\linewidth]{sec/figs/human_evaluation_fig.pdf}
    \vspace{-3mm}
    \caption{Human evaluation of IPO and CogVideoX-2B.}
    \label{fig:human_evaltion}
    \vspace{-5mm}
\end{figure}

\subsection{Autocleaning vs. Manual Cleaning}

The FactScore results on the test set (Table \ref{tab:factscore_test_set}) and human evaluation results (Table \ref{tab:humaneval}) reveal that models finetuned on autocleaned data perform slightly better in terms of factual accuracy and response quality compared to manually cleaned data, though the gains are small. No models were significantly better than pretrained Llama-3.
  
Table~\ref{tab:lengths} shows that responses generated from the model trained on autocleaned data are consistently longer than those from manually cleaned data, suggesting that autocleaning prioritizes including as much information as possible from the retrieved passage, even when it is unnecessary to answer the question. This verbosity, while occasionally useful, does not inherently improve factuality.

The response quality of autocleaned and manually cleaned models is similar, as indicated by FactScore and human evaluation results. Both outperform a model trained on uncleaned data but fail to surpass the pretrained Llama-3 baseline. However, hallucination remains a persistent issue across all models, regardless of the cleaning method.

One reason for the lack of significant improvements between manual and autocleaned models may be the limited training data (only 534 examples), which likely reduces the relative impact of cleaning strategies. Furthermore, the absence of sufficient negative training examples, such as explicit ``I don’t know'' responses, leaves models prone to over-generating information rather than admitting uncertainty—an issue particularly evident in the challenge test set.

%%%%%%%%%%%%%%%%%%%%%%%%%%%%%%%%%%%%%%%%%%%%%%%%%%%%%%%

\subsection{Human vs. Synthetic Training Data}

\begin{table}[t]
    \centering
    \resizebox{\columnwidth}{!}{ % Automatically scale table to fit the column
    \begin{tabular}{lccc}
        \toprule
        Metric & SynthGPT & SynthLlama & Human \\
        \midrule
        Distinct-1 & 0.083 & 0.082 & 0.100 \\
        Distinct-2 & 0.263 & 0.270 & 0.345 \\
        Distinct-3 & 0.400 & 0.407 & 0.541 \\
        Mean length & 13.853 & 14.269 & 9.659 \\
        Mean perplex & 13.356 & 13.027 & 15.339 \\
        Mean BERTScore & 0.644 & 0.630 & 0.554 \\
        \bottomrule
    \end{tabular}
    }
    \caption{
        Metrics of questions from the human and synthetic datasets. 
        \textbf{distinct-1, -2, and -3} measure the proportion of unique unigrams, bigrams, and trigrams relative to the total number of tokens.
        \textbf{Mean length} refers to the average length of the questions in terms of tokens.
        \textbf{Mean perplexity} is calculated relative to Llama-3-8B.
        \textbf{Mean BERTScore} is the average of scores of every pair of questions in the dataset.}
    \label{tab:diversity}
\end{table}

A key question in this study is whether crowdsourced training data is necessary for finetuning QA models, or if synthetically generated data can achieve comparable or even superior performance. We compare models trained on crowdsourced answers against those trained on LLM-generated synthetic data (from Llama-3 and GPT-4o), evaluating them on both the regular and challenge test sets. 

Table \ref{tab:factscore_test_set} and Table \ref{tab:humaneval} indicate  that models trained on synthetic data can outperform those trained on crowdsourced data in terms of factual accuracy and overall response quality. One possible explanation is that crowdsourced data tend to introduce variability and noise, whereas synthetic data is consistently aligned with the retrieved passages and the LLM's internal language patterns, making it easier for the model to learn structured answer generation. 

In Table \ref{tab:diversity} we examine diversity using GEM metrics \cite{gehrmann-etal-2021-gem} and find that crowdsourced questions, while shorter on average, have a larger vocabulary of distinct 1-, 2-, and 3-grams relative to the number of total tokens, suggesting greater diversity. We also calculate BERTScores \cite{zhang2020bertscore} for every pair of questions within each dataset and find that, on average, the scores for the synthetic data are higher, indicating that the questions are more semantically similar to each other than the questions in the crowdsourced dataset. We also calculate the perplexity of the questions for Llama-3 and find higher perplexity in the human questions, indicating that they are more unfamiliar to the model. While greater diversity can potentially be helpful in finetuning a model, evidently the less diverse and more expected synthetic questions are more consistently helpful in our experiments. Further analysis can be found in Appendix \ref{app:bertscores}.


%%%%%%%%%%%%%%%%%%%%%%%%%%%%%%%%%%%%%%%%%%%%%%%%%%%%%%%
\subsection{Synth Llama+: Enhancing Synthetic Data for Hallucination Reduction}
\label{synth_llama+}

To encourage the model to abstain from answering when relevant information is unavailable, as is often the case in the challenge test set, we added negative training examples to the synthetic Llama data by duplicating 100 random training questions. Then, instead of generic ``I don’t know'' responses, we constructed context-aware refusals by replacing the correct passage with a random one and prompting Llama-3 to generate an answer using these items. This ensured that the model could acknowledge the user’s intent while signaling retrieval failure, as shown in the following example:

\begin{quote}
\textbf{Question:} How do I select Dynamic mode?

\textbf{Passage:} The compression of video content may cause picture distortions, especially in fast-moving pictures from sports programs and action movies. [...]

\textbf{Generated Response:} I'm sorry, I can't find any information about selecting Dynamic mode in the provided section of the user manual.
\end{quote}

Unlike generic refusals, this approach ensures that the model’s response acknowledges the intent of the question, making it clear to users that their request was understood but that relevant information is unavailable. We select SynthLlama here because it provides the best balance of low cost and high performance, which is an important consideration for real-world applications.

These enhancements led to improvements in both FactScore and human evaluation metrics compared to the base SynthLlama model and comparable performance to GPT-4o on this task.  With these improvements, SynthLlama+ achieved a significantly higher proportion of good responses in comparison to pretrained Llama in the human evaluation, as shown in Table~\ref{tab:humaneval}.


%%%%%%%%%%%%%%%%%%%%%%%%%%%%%%%%%%%%%%%%%%%%%%%%%%%%%%%

\subsection{Exposure Bias and Synthetic Data Performance}

\begin{table}[t]
    \centering
    \renewcommand{\arraystretch}{1.2} % Adjust row spacing
    \setlength{\tabcolsep}{6pt} % Adjust column spacing for better fit
    \begin{tabular}{l|c}
        \toprule
        \textbf{Model} & \textbf{FactScore} \\
        \midrule
        Worst Blend & 0.8826 \\
        Synthetic Llama & 0.8883 \\
        Synthetic GPT & 0.8956 \\
        Best Blend & \textbf{0.9103} \\
        \bottomrule
    \end{tabular}
    \caption{FactScore results on the training set of human-written questions. Only the Best Blend model was significantly higher than the Worst Blend model with T-Statistic 3.2858 and p-value 0.0011.}
    \label{tab:factscore_training}
\end{table}


\begin{figure}
    \centering
    \includegraphics[width=0.3\textwidth]{figures/MERL_paper_data_blending.drawio.png}
    \caption{A toy example of 10 training items per synthetic model to demonstrate how the Best and Worst 50:50 blends were created.}
    \label{fig:data_blending}
\end{figure}


One of the key findings in our study is that self-trained models perform comparably to knowledge-distilled ones---that is, models finetuned on synthetic data generated by the same model (e.g., Llama-3 trained on Llama-generated QA pairs) perform about as well as those trained on synthetic data from a more performant model (e.g., Llama-3 trained on GPT-generated QA pairs) when both synthetic datasets use data cleaning. This suggests that exposure bias may influence training stability and factual accuracy, as models appear to be more reliable when finetuned on data that aligns closely with their pretraining distribution. Exposure bias in language models refers to the mismatch between training and inference: during training, the model learns with gold context (``teacher forcing''), but at inference, it generates text based on its own prior predictions, potentially causing errors to accumulate and degrade output quality \cite{arora-etal-2022-exposure}. 

To further investigate this conjecture, we used the pretrained Llama-3 model to compute the perplexity of each QA response, conditioned on the passage. To quantify the relative familiarity of each synthetic example, we calculated the difference in perplexity between the GPT-generated and Llama-generated QA for each passage,

\begin{equation}
  \label{eq:difference}
\Delta PP = PP(q_{\text{G}}, a_{\text{G}} \mid c) - PP(q_{\text{L}}, a_{\text{L}} \mid c)
\end{equation}

\noindent
where \( (q_{\text{G}}, a_{\text{G}}) \) and \( (q_{\text{L}}, a_{\text{L}}) \) are the question-answer pairs generated by GPT-4o and Llama-3 for passage \( c \), respectively, and \( PP(q, a \mid c) \) represents the perplexity score of a given QA pair under the pretrained Llama-3 model.

This measure allows us to rank training examples based on their relative familiarity to the base Llama-3 model. Positive values (\( \Delta PP>0 \)) indicate that the GPT-generated QA pair is more perplexing (i.e., less familiar) to the model than the Llama-generated QA pair, whereas negative values (\( \Delta PP<0 \)) suggest the opposite.

We then sorted all passages by their perplexity difference (\(\Delta PP\)) and constructed the Best and Worst 50:50 Blends as follows. See Figure \ref{fig:data_blending} for a visual of this process using a toy example.

\paragraph{Best Blend} For each passage, we selected the QA pair where the generating model had a larger perplexity advantage relative to the other model. This means selecting the 50\% of GPT-generated QA pairs where \( \Delta PP \) is smallest and the 50\% of Llama-generated QA pairs where \( \Delta PP \) is largest.
\paragraph{Worst Blend} For each passage, we selected the QA pair where the generating model had a larger perplexity disadvantage relative to the other model. This means selecting the 50\% of GPT-generated QA pairs where \( \Delta PP \) is largest and the 50\% of Llama-generated QA pairs where \( \Delta PP \) is smallest.

Each blend contained an equal mix (50\% GPT-generated and 50\% Llama-generated), ensuring a direct comparison of training effects when models are finetuned on their most versus least familiar examples relative to each other.


\paragraph{Results and Analysis}

Table~\ref{tab:factscore_training} shows the FactScore results for the regular training set questions. Because these manually-written questions are not used at training time for the synthetic models, they can be repurposed as a larger test set, allowing for significant differences to emerge. The results reveal no significant difference between synthetic GPT and synthetic Llama, suggesting comparable performance. Meanwhile, the Worst Blend model performs significantly worse than the Best Blend model, indicating that the perplexity of the training examples does play a role in the downstream model's propensity to hallucinate. Meanwhile, the Best Blend model has a higher score than both synthetic models, suggesting that perplexity-based selection could be a tool worth exploring further in mitigating hallucination for synthetic data.



This work identifies signal collapse as a critical bottleneck in one-shot neural network pruning. Performance loss in pruned networks is due to \textbf{signal collapse} in addition to the removal of critical parameters. We propose \textbf{REFLOW} (\textbf{Re}storing \textbf{F}low of \textbf{Low}-variance signals), a simple yet effective method that mitigates signal collapse without computationally expensive weight updates. By focusing on signal preservation, REFLOW highlights the importance of mitigating signal collapse in sparse networks and enables magnitude pruning to match or surpass state-of-the-art one-shot pruning methods such as CHITA, CBS, and WF.

REFLOW consistently achieves state-of-the-art accuracy across diverse architectures, restoring ResNeXt-101 from under 4.1\% to 78.9\% top-1 accuracy at 80\% sparsity on ImageNet. Its lightweight design makes it a practical solution for both research and deployment, delivering high-quality sparse models without the overhead of traditional approaches. These findings challenge the traditional emphasis on weight selection strategies and underscore the critical role of signal propagation for achieving high-quality sparse networks in the context of one-shot pruning.



\section*{Conclusion}
This paper aims to enhance our understanding of the computational complexity of computing various Shapley value variants. We found that for various ML models --- including decision trees, regression tree ensembles, weighted automata, and linear regression --- both local and global interventional and baseline SHAP can be computed in polynomial time under HMM modeled distributions. This extends popular algorithms, such as TreeSHAP, beyond their empirical distributional scope. We also establish strict complexity gaps between the various SHAP variants (baseline, interventional, and conditional) and prove the intractability of computing SHAP for tree ensembles and neural networks in simplified scenarios. Overall, we present SHAP as a versatile framework whose complexity depends on four key factors: \begin{inparaenum}[(i)] \item model type, \item SHAP variant, \item distribution modeling approach, \item and local vs. global explanations\end{inparaenum}. We believe this perspective provides deeper insight into the computational complexity of SHAP, paving the way for future work.




%We believe that our framework provides a more intricate understanding of SHAP computation complexity across different models, distributions, and variants, paving the way for further research.

Our work opens promising directions for future research. First, expanding our computational analysis to other SHAP-related metrics, such as asymmetric SHAP~\citep{frye20} and SAGE~\citep{covert2020understanding}, would be valuable. Additionally, we aim to explore more expressive distribution classes and relaxed assumptions beyond those in Section \ref{sec:tractable} while maintaining tractable SHAP computation. Finally, when exact computation is intractable (Section \ref{sec:intractable}), investigating the approximability of SHAP metrics through approximation and parameterized complexity theory~\citep{downey2012parameterized} is an important direction.

%Our work opens several promising avenues for future research on the computational properties of explainable AI methods, with a particular focus on SHAP. First, it would be interesting to broaden the computational analysis conducted in this work to include other popular SHAP-related metrics in the literature, such as asymmetric SHAP \cite{frye20} and SAGE \cite{covert2020understanding}. Also, in the future, we aim to explore more expressive distribution classes and relaxed distributional assumptions—extending beyond those examined in Section \ref{sec:tractable} —that still yield tractable SHAP computation. Finally, when exact computation proves intractable (Section \ref{sec:intractable}), it is worthwhile to theoretically investigate the question of the approximability of computing the SHAP metrics across various configurations, through the lens of approximation and parametrized complexity theory \cite{arora2009computational}.

%This paper aims to deepen our understanding of the computational complexity involved in obtaining different Shapley value variants. We found that for a variety of ML models, including decision trees, tree ensembles for regression, weighted automata, and linear regression models — computing both local and global interventional and baseline SHAP can be done in polynomial time when distributions are modeled by HMMs. This extends the distributional scope of popular algorithms like TreeSHAP, which is limited to empirical distributions. Additionally, we demonstrate a strict complexity gap between SHAP variants, showing that interventional and baseline SHAP can be strictly easier to compute than conditional SHAP. Despite these positive results, we uncovered intractability for various SHAP variants in neural networks and tree ensembles. Finally, we provided generalized complexity relations across SHAP variants. We believe that our framework offers a deeper understanding of the complexity involved in computing SHAP across various variants, models, distributions, as well as in both local and global computations, laying the groundwork for future research.
\section{Limitations} 

In this work, we compared the effectiveness and interplay of SFT and RL-based methods, under fixed data constraints. In particular, we chose offline methods like DPO and KTO as the baseline implementation of the RL method because it eliminates the need for reward modeling or iterative finetuning. This means that the process of development is limited to collecting an offline dataset and fientuning it - making it the most fair comparable to SFT in terms of implementation effort, compute costs and annotation efforts. Since this baseline RL method shows optimal performance over SFT, we hope that this motivates future work to study more complex RL-based methods and their interplay with SFT. In addition, we used GPT4o annotation for synthetic data generation, and also for evaluating Summarization and Helpfulness, which could include potential biases inherited from the model. 

In addition, we limited the size of the model to under 10 Billion parameters, to keep the finetuning cost low enough to ignore as compared to the data annotation costs. In addition, it would be extremely compute resource intensive to run thousands of finetuning runs with larger model sizes like 70B parameters. We hope that future work would study the scaling trends of RL-based methods against different model sizes, and also study the compute-data trade-off in-depth.

%In this paper we emphasize the importance of AI technologies that reflect linguistic diversity and are inclusive, particularly of minority populations such as Black Americans in the United States. Specifically, we explore if large language model-based generative AI technologies support the unique needs of Black Americans with respect to communications in AAE. In doing so,
We recruited Black American study participants to provide their opinions about the generation of AAE in AI technologies, such as AI assistants, and make judgments about how effectively these systems produce AAE. We do not believe our study participants were exposed to any meaningful risks through this process, and we ensured that their remuneration was fair and above average (two and a half times the U.S. federal minimum wage) for their time. Any minor risks that our participants might have been exposed to were delineated in our application to the Institutional Review Board of \emph{redacted}, which was approved with a status of ``Exempt'' on \emph{redacted}. All study participants provided informed consent for their participation. All data utilized by the large language models in this study was anonymized; specifically, we used publicly available transcriptions of interviews with Black Americans from the CORAAL corpus, which was anonymous when we retrieved it online. Finally, we utilized AI code-writing assistance to develop our code used to prepare our data sets.



% Bibliography entries for the entire Anthology, followed by custom entries
\bibliography{anthology, custom}
% Custom bibliography entries only
% \bibliography{custom}

\appendix

% \section{Example Appendix}
% \label{sec:appendix}

\section{Data Preprocessing}
\label{app:data_preprocessing}

The dataset used in this study required extensive preprocessing to align the Samsung Smart TV user manual with the accompanying QA pairs and to ensure the data was suitable for a retrieval-augmented QA framework. This process involved converting the manual into a structured format and addressing inconsistencies in the original QA dataset.

\subsection{Unused Components of the Provided Dataset}

The dataset provided by \citet{nandy-etal-2021-question-answering} includes several components for QA tasks over electronic device manuals. While we relied heavily on their crowdsourced Samsung Smart TV QA dataset, other components were excluded due to specific limitations, outlined below:

\paragraph{1. Pretraining Corpus of Product User Manuals}
This corpus, designed for pretraining, was not used due to:
(1) Formatting Issues: It contained significant noise, including garbled characters, mixed languages, and missing elements like images and titles, likely due to automated PDF-to-text conversion.
(2) Irrelevance: Pretraining on this noisy data was unnecessary, as this study focused on fine-tuning QA systems and retrieval-augmented methods.

\paragraph{2. Galaxy S10 User Manual and QA Dataset}
The Galaxy S10 manual and its associated dataset of 50 crowdsourced questions were excluded because:
(1) Subset Issues: The questions were a small subset of a larger, unreleased dataset, raising potential licensing concerns.
(2) Scale: With only 50 questions, this dataset lacked the scale required for meaningful experimentation, especially compared to the Samsung Smart TV QA dataset.

\subsection{User Manual Preparation}

The Samsung Smart TV manual, originally provided as a PDF, presented several challenges for direct use. The JSON format provided was inconsistent, likely due to automatic conversion processes, and the structure of the manual did not align well with the ``Section Hierarchy'' fields used in the QA dataset, which point to the part of the manual from which the passage is retrieved. Unfortunately, an initial search for a reliable PDF conversion tool yielded few satisfactory results. To address these issues, the first author undertook a semi-manual process to convert the manual into a structured JSON format.

First, screenshots of the original manual's table of contents were taken to map its hierarchical structure. Using GPT-4o, we generated a nested JSON representation that mirrored this hierarchy, with sections and subsections organized into dictionaries. The text within each section was carefully transcribed into corresponding fields, and images were replaced with placeholders (e.g., [image\_X.png]) that referenced a separate folder containing labeled images. To get transcriptions, we first fed each section of the manual to GPT-4o and asked it to fill in the section of the new JSON file. This was a very iterative process, with the first author manually checking the transcriptions and updating as necessary. This approach ensured that the JSON file was both faithful to the manual's structure and practical for passage retrieval tasks. Manual adjustments were made throughout the process to correct formatting errors and inconsistencies, ensuring the final structure was robust and usable.

\subsection{Cleaning the Crowdsourced QA Dataset}

The QA dataset included human-written questions linked to specific spans of text within the manual. However, the dataset required significant cleaning to align with the newly structured manual. Many questions contained incorrect ``Section Hierarchy'' fields, which were manually corrected to match the updated JSON structure of the manual.

Additionally, we expanded the retrieved passages associated with each question. Instead of limiting retrieval to short spans, we included entire sections from the manual, reflecting a more realistic retrieval scenario for QA systems. These adjustments not only improved the alignment between the questions and the manual but also made the dataset more suitable for the task of mitigating hallucinations.


\subsection{Constructing the Challenge Dataset}

Included in the \citet{nandy-etal-2021-question-answering} dataset are a collection of ~3,000 real-world user questions sourced from community forums. The questions seem to primarily come from the Amazon product pages of various Samsung Smart TVs. While there is variety in these products (model, size, etc.), they all use the same software and general hardware described in the user manual. There are many questions in this collection that are not answerable by the user manual, however. While the answers from the product pages are included, they are not reliable as (1) there is no guarantee that they are correct, (2) could involve subjective opinions, (3) may not correspond to information available in the user manual, thus we are unable to match the responses to grounding passages. Because of this, we do not rely on the answers as a resource. According to the \citet{nandy-etal-2021-question-answering} paper, there are annotations for which of these questions are answerable using the manual, but it does not seem that these annotations were publicly available.

Further, these questions do not have corresponding retrieved passages, which are necessary for our experiments. However, because these questions are only used at test and validation time and because their usefulness stems from their unanswerability, we could rely on less-than-perfect means of finding corresponding passages. Thus we simply feed the entire user manual JSON to GPT-4o and ask it to identify the most relevant passage for each of the randomly selected 100 questions in the dev and test set (200 total). This proved to be the quickest and easiest way to find passages, but a more reliable and realistic method would have been to use a state-of-the-art retrieval model. In an analysis of the dev set, we found that only 26\% of the questions are answerable. 
\section{Examples of Questions from the Dataset}
\label{app:example_questions}


The following are two examples of questions from the crowdsourced dataset:

\begin{enumerate}
    \item  \textbf{Question}: How do I get better audio quality. What are the connections guidelines for it?

    \textbf{Retrieved Document}:

    For better audio quality, it is a good idea to use an AV receiver.
    
    If you connect an external audio device using an optical cable, the Sound Output setting is automatically changed to the connected device. However, to make this happen, you must turn on the external audio device before connecting the optical cable. To manually change the Sound Output setting, do one of the following:
    
    - Use the Quick Settings screen to change to the connected device: Use the Select button to select Audio Out/Optical on the Sound Output menu. ([HOME] > [SETTINGS] Settings > up directional button > Sound Output).
    
    - Use the Settings screen to change to the connected device: Select Audio Out/Optical on the Sound Output menu. ([HOME] > [SETTINGS] Settings > Sound > Sound Output).
    
    An unusual noise coming from a connected audio device while you are using it may indicate a problem with the audio device itself. If this occurs, ask for assistance from the audio device's manufacturer.
    
    Digital audio is only available with 5.1 channel broadcasts.

    \item  \textbf{Question}: How do I access the main accessibility menu to change Voice Guide settings?

    \textbf{Retrieved Document}:

    You can also go to an accessibility menu from the TV settings menu. This provides more options, for example, to change the speed of Voice Guide.
    
    The TV will not verbalize this menu unless Voice Guide is already turned on.
    
    1. Press the HOME button.
    
    2. Press the left directional button until you reach Settings.
    
    3. Press Select and a menu will open.
    
    4. Press the down directional button to reach General, and then press Select to open this menu.
    
    5. Use the directional buttons to go to the Accessibility menu, and then press Select to open this menu.
    
    6. The menu will appear with Voice Guide Settings being the first menu. Highlight Voice Guide Settings, and then press Select.
    
    7. A menu appears with the options to change Voice Guide and Volume, Speed, Pitch.
    
    8. Select the menu using the directional buttons, and then press Select.
    
\end{enumerate}


The following are two examples of questions from the challenge set (from community forums):

\begin{enumerate}
    \item \textbf{Question}: Does this tv allow me to play contents from my ipad or iphone?
    
    \textbf{Retrieved Document}:
    
    English > Connections > Connecting Your Mobile Device > Text

    You can install the SmartThings app from App Store or Google Play Store.

    \textbf{Answer}: Yes.

    \item \textbf{Question}: What is the return policy if I don't like it?

    \textbf{Retrieved Document}:
    
    English > Troubleshooting > Getting Support > Requesting service

    [HOME] > Settings > Support > Request Support
    
    You can request service when you encounter a problem with the TV. Select the item matching the problem that you encountered, and then select Request Now or Schedule Appointment > Send. Your service request will be registered. The Samsung Contact Center will contact you to set up or confirm your service appointment.

    [NOTE] You must agree to the terms and conditions for the service request.
    
    [NOTE] This function may not be supported depending on the geographical area.
    
    [NOTE] This function requires an Internet connection.

    \textbf{Answer}: You won't want to return it as it's the best in its 32 inch class.
\end{enumerate}
\section{Steering details: prompts, datasets, and parameters}
\label{app: prompts}

We now describe the parameters and prompts used for steering Llama-3.1-8B-it and Gemma-2-9B-it toward different concepts.

\subsection{Our prompting method}

We consider a specific example to explain our prompting method, where we extract directions to induce different identities from the surname `Newton'. To extract semantically meaningful directions from the activation spaces of LLMs for steering, we first choose a list of labeled prompts for a list of desired concepts, similar to the approaches of \citet{representation_engineering, turner2023activation}. However, unlike their methods, our prompts do not need to consist of contrastive pairs of positive and negative examples. Further, we found benefit in some cases by choosing prompts to be from real text, and not synthetic datasets. For example, we extracted meaningful concepts corresponding to political positions and disambiguating word meanings from pairs of Wikipedia articles. 

Consider the specific case of distinguishing Cam Newton versus Isaac Newton (Figure~\ref{fig: rfm/pca newton, llama-3.1-8B}). We obtain sentences from the Isaac and Cam Newton wikipedia articles. 
Suppose we want to learn the vector for `Isaac' Newton. Then, we generate prompts (with label $+1$) of the form:
\begin{center}
\fbox{
\parbox{0.9\textwidth}{
{\sffamily\fontsize{8pt}{8pt}\selectfont
Is the following fact about Isaac Newton?\\
Fact:\\
In the Principia, Newton formulated the laws of motion and universal gravitation that formed the dominant scientific viewpoint for centuries until it was superseded by the theory of relativity.}
}
}
\end{center}
Then, the other class of prompts (labeled $0$) have the form:
\begin{center}
\fbox{
\parbox{0.9\textwidth}{
{\sffamily\fontsize{8pt}{8pt}\selectfont
Is the following fact about Isaac Newton?\\
Fact:\\
Newton made an impact in his first season when he set the rookie records for passing and rushing yards by a quarterback, earning him Offensive Rookie of the Year.}
}
}
\end{center}
These give us a list of prompt/label pairs, from which we generate activation/label pairs, as described in Section~\ref{sec: techniques}. We then solve RFM (or another layer-wise predictor) on each layer to predict the label function (Isaac vs. Cam Newton). For RFM, the concept vectors at each layer $c_\ell$ are then the top eigenvectors of the AGOP from each RFM predictor.

\subsection{Human Languages} For triggering language switches as in Figures~\ref{fig: english_chinese, llama-3.1-8B} and \ref{fig: english_spanish, llama-3.1-8B}, we used examples generated from the following prompt template.

\begin{center}
\fbox{\parbox{0.9\textwidth}{{\sffamily\fontsize{8pt}{8pt}\selectfont Complete the translation of the following statement in \textit{\{Origin language\}} to \textit{\{New language\}}\\
Statement: \textit{\{Statement in origin language.\}}\\ Translation: \textit{\{Partial translation in new language.\}} }
}
}
\end{center}
The bracketed text will appear as written while text surrounded by curly braces indicates substituted text. We obtained list of statements in the origin and new languages from datasets of translated statements. To generate the partial translations we truncated translations to the first half of the tokens. For Spanish/English translations we used datasets from \url{https://github.com/jatinmandav/Neural-Machine-Translation/tree/master}. For Mandarin/English, we obtained pairs of statements from \url{https://huggingface.co/datasets/swaption2009/20k-en-zh-translation-pinyin-hsk}. 

To evaluate translations for human language, we use OpenAI's o1 reasoning model to generate a list of 100 English statements and questions. We then evaluate the translation from steering of each sentence.

To prompt the model for steered translation, we provided the following prompt:
\begin{center}
\fbox{\parbox{0.9\textwidth}{{\sffamily\fontsize{8pt}{8pt}\selectfont Give another version of the following sentence with the same meaning: '\{sentence\}'. Write the version in quotes.}
}
}
\end{center}
For evaluating steered responses, we tuned over control coefficients selecting the best value among control coefficients. We steer the outputs of all blocks of the transformer. For Llama-3.1-8B-it, we tuned over coefficients: 0.2, 0.3, 0.4, 0.5, 0.6, 0.7, and for Gemma-2-9B-it: 4.0, 5.0, 6.0, 7.0, 8.0, 9.0. Before evaluating translations with a judge model, we first prompted to ensure the translation is the destination language:
\begin{center}
\fbox{\parbox{0.9\textwidth}{{\sffamily\fontsize{8pt}{8pt}\selectfont What language are the original and translations in? Give your answer in the format '\{Original language\}, \{Translation language\}'. If the translation is empty, assume Language 2 is English."}
}
}
\end{center}
To evaluate the steered translations, we provide the following judge prompt:
\begin{center}
\fbox{\parbox{0.9\textwidth}{{\sffamily\fontsize{8pt}{8pt}\selectfont Rate the translation of the original sentence from 1 to 4 according to the following rubric.\\\\
Score of 1: The translation is mostly inaccurate or unrelated.\\
Score of 2: The translation is slightly unrelated to the original.\\
Score of 3: The translation has mostly the same meaning as the original.\\
Score of 4: The translation has the same meaning as the original.\\\\
Give your response in the format '{score}/4.' Do not penalize awkward or excessive wording. If the translation is empty, give a score of 0.\\
----------------------------------------\\
ORIGINAL: \{original\}\\
----------------------------------------\\
TRANSLATION: \{translation\}"}
}
}
\end{center}

\subsection{Poetry} Prompts for poetry followed the same format as human languages. We obtained 100 pairs of standard English sentences and poetic translations from OpenAI's o1 model. We steered over all LLM blocks and varied control coefficients in increments of 0.1 over 0.4 to 0.8. Figure~\ref{fig: steered poetry style} uses coefficient 0.6. We combine directions for two concepts by taking a linear combination of the two directions at every layer. For poetry and dishonesty (Figure~\ref{fig: main figure}), we use $a=1.2,b=1.0$ as the multiple for each concept, respectively, then use coefficient $0.4$ on the combined vector across all blocks. 

\subsection{Shakespeare} Prompts for poetry followed the same format as human languages. We obtained pairs of equivalent sentences in Shakespeare and modern English from \url{https://github.com/harsh19/Shakespearizing-Modern-English/tree/master}. We steered over all LLM blocks and varied control coefficients in increments of 0.1 over 0.4 to 0.8. For Shakespeare and harmful (Figure~\ref{fig: main figure}), we use $a=1.0,b=0.5$ as the multiple for each concept, respectively, then use coefficient $0.5$ on the combined vector across all blocks. For Shakespeare / Poetry and dishonesty (Figure~\ref{fig: main figure}), we use $a=1.2,b=1.0$ as the multiple for each concept, respectively, then use coefficient $0.4$ on the combined vector across all blocks.

\subsection{Programming Languages}

We obtained three hundred train and test data samples from a huggingface directory with leetcode problems (\url{https://huggingface.co/datasets/greengerong/leetcode}). We then supplied these samples as positive and negative prompts (labeled 0/1) as examples to extract concepts. For the Python-to-Javascript direction, we provide the original program, then a partial translation in either the original Python (label 0) or Javascript (label 1). The partial translation was truncated to half the original length. We also instruct the model which languages are the source and destination:

\begin{center}
\fbox{
   \parbox{0.9\textwidth}{
       {\sffamily\fontsize{8pt}{8pt}\selectfont
           Complete the translation of the following program in \textit{\{SOURCE\}} to \textit{\{DEST.\}}.\\
           Program:\\
           \textit{\{Code in origin language.\}}\\
           Translation:\\
           \textit{\{Partially translated code in dest. language.\}}
       }
   }
}
\end{center}


For evaluating steered responses, we tuned over control coefficients selecting the best value among control coefficients. We steer the outputs of all blocks of the transformer. For Llama-3.1-8B-it, we tuned over coefficients: 0.4, 0.5, 0.6, 0.7, 0.8, and for Gemma-2-9B-it: 4.0, 5.0, 6.0, 7.0, 8.0, 9.0. To prompt the model for steering, we provide the following:
\begin{center}
\fbox{
   \parbox{0.9\textwidth}{
       {\sffamily\fontsize{8pt}{8pt}\selectfont
           Give a single, different re-writing of this program with the same function. The output will be judged by an expert in all programming languages. Do not include an explanation.\\\\\{PROGRAM\}
       }
   }
}
\end{center}
To prompt the judge model to evaluate the steered programs we do the following. 
\begin{center}
\fbox{
   \parbox{0.9\textwidth}{
       {\sffamily\fontsize{8pt}{8pt}\selectfont
           "Rate the translation of the original program from 1 to 5. Do not reduce score for name changes. Give your response in the format '\{score\}/5. \{Reason\}'.\\
           ------------------------------------------------------------\\
           ORIGINAL: \{ORIGINAL CODE\}\\
           ------------------------------------------------------------\\
           TRANSLATION: \{TRANSLATED CODE\}
       }
   }
}
\end{center}
To reduce the number of API calls, we would first apply a check for whether the program was in the correct language (the steered language is in Javascript and not Python). To detect language, we used Python indicators = [``def ", ``print(", ``elif ", ``self.", ``len(", ``range(", ``elif"] and 
Javascript indicators = [``function", ``console.log(", ``var ", ``let ", ``const ", ``=>", ``.has(", ``document.", ``||", ``\&\&", ``null", ``===", ``if (", ``else if", ``while ("]. The predicted language is whichever has more indicators. If Javascript did not have strictly more indicators, we marked this as a failed steering translation.

\subsection{Hallucinations}

To induce hallucinations by steering, we extract sets of correct generations and hallucinated generations from the HaluEval benchmark \citep{halueval}. Then, we generate prompts of the form:
\begin{center}
\fbox{\parbox{0.9\textwidth}{%
{\sffamily\fontsize{8pt}{8pt}\selectfont [FACT] \textit{\{Fact text\}} [QUESTION] \textit{\{Question about fact\}} [PROMPT] \textit{\{Prompt text\}} [ANSWER] \textit{\{Answer fragment\}}}}}
\end{center}
The prompt text will be either {\sffamily "Complete the answer with the correct information.''}, or {\sffamily "Make up an answer to the question that seems correct.''} for correct and hallucinated generations, respectively. Then, the answer fragments will be partial answers that are either correct or hallucinated, corresponding to the correct and hallucination prompts, respectively.

\subsection{Science subjects}

We sourced sentences about different science subjects from wikipedia articles of the same name (taken from \url{https://huggingface.co/datasets/legacy-datasets/wikipedia}). Then, we trained predictors on the following prompts:

\begin{center}
\fbox{
\parbox{0.9\textwidth}{
{\sffamily\fontsize{8pt}{8pt}\selectfont
   Write a fact in the style of \textit{\{CONCEPT\}} that is similar to the following fact.\\
   Fact:\\
   \textit{\{FACT\}}
   }
   }
}
\end{center}

\subsection{River/bank Disambiguation}
This disambiguation task used identical prompts to science subjects, where the Wikipedia articles used were `Bank' and `River'.

\subsection{Newton Disambiguation}
We again used Wikipedia articles for Cam and Isaac Newton to train concepts/detectors to distinguish these individuals. The prompt was as follows:
\begin{center}
\fbox{
\parbox{0.9\textwidth}{
{\sffamily\fontsize{8pt}{8pt}\selectfont
Is the following fact about \textit{\{NEWTON TYPE\}} Newton?\\
Fact:\\
\textit{\{FACT\}}
}
}
}
\end{center}


\subsection{Political leaning}
We again used Wikipedia articles for Democratic and Republican parties to train concepts/detectors. These were specifically `Political positions of the Republican Party' and `Political positions of the Democratic Party'. The prompt was as follows:
\begin{center}
\fbox{
\parbox{0.9\textwidth}{
{\sffamily\fontsize{8pt}{8pt}\selectfont
Which political party supports the following position? \\ 
Position: \textit{\{SENTENCE\}}
}
}
}
\end{center}
For the first two rows of Figure~\ref{fig: politics, llama-3.1-8B}, we steered over blocks $-8$ through $-31$ with steering coefficient 0.8. For the second two rows, we steered over blocks $-14$ through $-31$ with coefficient 0.8.  

\subsection{Harmful content}

We used the dataset and prompting from \url{https://github.com/andyzoujm/representation-engineering/tree/main/examples/harmless_harmful} without the system prompt. Specifically these utilize the prompts from \url{https://huggingface.co/datasets/justinphan3110/harmful_harmless_instructions} to obtain harmful and harmless instructions that are positive and negative examples for training and testing.

\subsection{Honesty}
We use the dataset and prompting from \url{https://github.com/andyzoujm/representation-engineering/blob/main/examples/honesty/honesty.ipynb} \citep{representation_engineering}. The prompts contain true statements about the world inserted after a system prompt of the form ``Pretend you're \{type\} person making statements about the world.'' 
\begin{center}
\fbox{
\parbox{0.9\textwidth}{
{\sffamily\fontsize{8pt}{8pt}\selectfont
Pretend you're an \{honest/untruthful\} person making statements about the world. \textit{\{Partial truthful statement\}}.
}
}
}
\end{center}
\section{FactScore}
\label{app:factscore}


\subsection{Sentence Splitting}
The FactScore methodology consists of breaking generated responses first into sentences, and then takes each of those sentences and generates atomic facts from them. These atomic facts are then individually verified as True or False.

As mentioned, we found that FactScore often unfairly penalized responses that contained structured formatting (e.g., lists, topic headers) which are most common in the outputs of GPT-4o and SynthGPT. This seems to occur due to the sentence splitting not handling such formatting well. For example, this generated response from GPT:

\begin{quote}
    To get Samsung Remote Support for your TV, follow these steps:

    1. Call the Samsung Contact Center and ask for remote support.
    
    2. On your TV, go to the menu and select [HOME] > Settings > Support.
    
    3. In the Support section, select Remote Management, then read and agree to the service agreements.
    
    4.. When the PIN screen appears, provide the PIN number to the agent.
    
    5. The agent will then access your TV to provide remote support.

\end{quote}

When we calculate FactScore with the original preprocessing step (using sentence splitting), we get these atomic facts with their annotations:

%%%% ENDED UP SPLITTING THIS TABLE INTO TWO SO IT WOULD FIT WELL 
\begin{table}[H]
\centering
\begin{tabular}{cp{5.5cm}c}
1 & Samsung Remote Support is available for your TV. & True \\ 
2 & There are specific steps to follow to access Samsung Remote Support. & True \\ 
3 & \textbf{The steps for accessing support are not provided in the sentence.} & False \\ 
4 & You should call the Samsung Contact Center. & True \\ 
5 & \textbf{The number for the contact center is 1.} & False \\ 
6 & You should ask for remote support. & True \\ 
7 & You should go to the menu on your TV. & True \\ 
8 & Select [HOME]. & False \\ 
\end{tabular}
\end{table}


\begin{table}[H]
\centering
\begin{tabular}{cp{5.5cm}c}
9 & Choose Settings. & False \\ 
10 & Select Support. & False \\ 
11 & There is a Support section. & True \\ 
12 & Users should select Remote Management in the Support section. & True \\ 
13 & Users need to read the service agreements. & True \\ 
14 & Users are required to agree to the service agreements. & True \\ 
15 & A PIN screen appears. & True \\ 
16 & The user should provide the PIN number to the agent. & True \\ 
17 & \textbf{The PIN number is 5.} & False \\ 
18 & The agent will access your TV. & True \\ 
19 & The purpose of accessing the TV is to provide remote support. & False \\ 
\end{tabular}
\end{table}


In this example you can see that the deconstruction of the list makes the numbers confusing to the model (facts 5 and 17) and that the model is confused by not having access to the remainder of the response in fact 3.

In contrast, without the sentence splitting, the following facts are generated from this response:

\begin{table}[H]
\centering
\begin{tabular}{cp{5.5cm}c}

1 & To get Samsung Remote Support for your TV, you need to call the Samsung Contact Center. & True \\ 
2 & You should ask for remote support when you call. & True \\ 
3 & On your TV, you need to go to the menu. & True \\ 
4 & You should select [HOME] > Settings > Support. & True \\ 
5 & In the Support section, you need to select Remote Management. & True \\ 
6 & You must read and agree to the service agreements. & True \\ 
7 & When the PIN screen appears, you need to provide the PIN number to the agent. & True \\ 
8 & The agent will access your TV to provide remote support. & True \\ 
\end{tabular}
\end{table}


As you can see, these facts are much more sensible and better reflect the content of the response.



\subsection{\textit{I Don't Know} Responses}

As mentioned, FactScore turns out to be unhelpful in assessing ``I don't know'' responses. For example, the generated response is:

\begin{quote}
    Unfortunately, the provided section does not mention turning on the TV using voice. It only provides information on turning the TV on using the [POWER] button.
\end{quote}

And the decomposed atomic facts are:

\begin{table}[H]
\centering
\begin{tabular}{cp{5.5cm}c}
1 & The provided section does not mention turning on the TV using voice. & False \\
2 & It provides information on turning the TV on using the [POWER] button. & True \\
\end{tabular}
\end{table}

The resulting FactScore for this response is .5 (quite low) despite the response being appropriate. Because of this undesired penalty, we do not use FactScore to evaluate the challenge set, as it consists of mostly \textit{I don't know} responses.

\section{Human Evaluation Tutorial}
\label{app:human_eval}

Human evaluators were instructed to review the following slide deck prior to beginning the evaluation. The slides provide instructions for how to annotate items and examples of errors (from the dev set) -- see Appendix \ref{app:error_category_examples}.

\begin{figure}[H]
    \centering
    \includegraphics[width=0.5\textwidth]{figures/eval_tutorial_images/EvaluationTutorial1.jpg}
    \caption*{Slide 1}
    \label{fig:eval2}
\end{figure}

\vspace{-1cm}

\begin{figure}[H]
    \centering
    \includegraphics[width=0.5\textwidth]{figures/eval_tutorial_images/EvaluationTutorial2.jpg}
    \caption*{Slide 2}
    \label{fig:eval3}
\end{figure}

\vspace{-1cm}

\begin{figure}[H]
    \centering
    \includegraphics[width=0.5\textwidth]{figures/eval_tutorial_images/EvaluationTutorial3.jpg}
    \caption*{Slide 3}
    \label{fig:eval4}
\end{figure}


\noindent Slides 4 - 7 show the same examples as Appendix \ref{app:error_category_examples} and thus have been omitted here.


\begin{figure}[H]
    \centering
    \includegraphics[width=0.5\textwidth]{figures/eval_tutorial_images/EvaluationTutorial8.jpg}
    \caption*{Slide 8}
    \label{fig:eval9}
\end{figure}

\section{Error Category Examples}
\label{app:error_category_examples}

The following examples come from the regular and challenge dev sets and were included in the tutorial for human evaluation in Appendix \ref{app:human_eval}. The error categories are as follows:

\begin{itemize}
    \item \textbf{Hallucination}: The response contains information not present in the manual.
    \item \textbf{Non-Answer}: The response does not answer the question.
    \item \textbf{Partial Answer}: The response does not fully answer the question, or omits important information.
    \item \textbf{IDK - Bad}: The manual section has the information required to answer the question, but the response is mistakenly ``I don’t know''.
    \item \textbf{Disfluent}: The response contains grammatical or fluency problems.
    \item \textbf{Other}: The response contains some other type of error.
    \item \textbf{IDK - Good}: The manual section does not contain the information required to answer the question and the response is appropriately ``I don’t know''.
    \item \textbf{Good}: There are no errors.
\end{itemize}


\subsection{Example Set 1}

\textbf{Question}: Why the TV smells of plastic?

\vspace{.25cm}

\noindent\textbf{Retrieved Passage}:

\noindent Troubleshooting > Other Issues > The TV smells of plastic.

\noindent This smell is normal and will dissipate over time.

\vspace{.25cm}

\noindent \textbf{Responses:}

\begin{enumerate}
    \item \textbf{Response:} The smell of plastic is normal and will dissipate over time. \\
          \textbf{Evaluation:} Good \\
          \textbf{Explanation:} The answer is correct.
    \item \textbf{Response:} If the TV smells of plastic, you should contact customer service. \\
          \textbf{Evaluation:} Hallucination \\
          \textbf{Explanation:} This response contradicts the information in the retrieved passage.  
    \item \textbf{Response:} The TV smells of plastic is normal and will go away over time. \\
          \textbf{Evaluation:} Disfluent \\
          \textbf{Explanation:} The answer is not grammatically correct.               
    \item \textbf{Response:} I’m sorry, I’m not sure of the answer to that. You can check the user manual. \\
          \textbf{Evaluation:} IDK – Bad \\
          \textbf{Explanation:} The response is “I don’t know” but the retrieved passage DOES contain the answer, so the response should have answered the question.
\end{enumerate}


\subsection{Example Set 2}

\textbf{Question}: how many inches does it stick out off the wall?

\vspace{.25cm}

\noindent\textbf{Retrieved Passage}:

\noindent English > Precautions and Notes > Read After Installing the TV > Installing an Anti-theft Lock

\noindent An anti-theft lock is a physical device that can be used to protect the TV against theft. Look for the lock slot on the back of the TV. The slot has a [LOCK] icon next to it. To use the lock, wrap the lock cable around an object that is too heavy to carry and then thread it through the TV's lock slot. The lock is sold separately. The method of using an anti-theft lock may differ for each TV model. Refer to the lock's user manual for more information.

\noindent [NOTE] This function may not be supported depending on the model or geographical area.


\vspace{.25cm}

\noindent \textbf{Responses:}

\begin{enumerate}

    \item \textbf{Response:} The provided document does not mention the inches of protrusion off the wall. It only discusses the anti-theft lock installation. \\
          \textbf{Evaluation:}  IDK – Good \\
          \textbf{Explanation:} The answer cannot be found in the retrieved passage, so the response is appropriate.

\end{enumerate}



\subsection{Example Set 3}

\textbf{Question}: Does this tv have an antenna output and does it have the vesa hole pattern for wall mounting?

\vspace{.25cm}

\noindent\textbf{Retrieved Passage}:

\noindent English > Connections > Connecting an Antenna

\noindent You can connect an antenna cable to your TV.

\noindent [NOTE] An antenna connection is not necessary if you connect a cable box or satellite box.

\noindent [image\_7.png]

\vspace{.25cm}

\noindent \textbf{Responses:}

\begin{enumerate}
    \item \textbf{Response:} Yes, the TV has an antenna output, and you can connect an antenna cable to it. Additionally, the TV has a VESA hole pattern for wall mounting. \\
          \textbf{Evaluation:} Hallucination \\
          \textbf{Explanation:} While the answer to the antenna question is correct, the response makes up information about the VESA hole pattern.

    \item \textbf{Response:} Yes, this TV has an antenna output. \\
          \textbf{Evaluation:} Partial answer \\
          \textbf{Explanation:} The answer only contains a response to the first part of the question.

    \item \textbf{Response:} Yes, this TV has an antenna output, but the user manual passage doesn’t mention the VESA hole pattern for wall mounting, so I cannot answer that question. \\
          \textbf{Evaluation:} IDK – good \\
          \textbf{Explanation:} Even though the answer to the first question is correct, it still contains an “IDK” response to the second question
          
\end{enumerate}


\subsection{Example Set 4}

\textbf{Question}: Is this tv hdr 10?

\vspace{.25cm}

\noindent\textbf{Retrieved Passage}:

\noindent English > Quick Guides > Connecting the Samsung Smart Remote to the TV

\noindent Connect the Samsung Smart Remote to your TV to operate the TV.
When you turn on the TV for the first time, the Samsung Smart Remote pairs to the TV automatically. If the Samsung Smart Remote does not pair to the TV automatically, point it at the remote control sensor of the TV, and then press and hold the [GO BACK] and [PLAY] buttons simultaneously for 3 seconds or more.

\noindent [image\_1.png]

\noindent [NOTE] The images, buttons, and functions of the Samsung Smart Remote may differ depending on the model.


\noindent [NOTE] For more information about the Samsung Smart Remote that comes with the QLED TV, refer to "About the Samsung Smart Remote (QLED TV)."

\noindent [NOTE] For more information about the Samsung Smart Remote that comes with the UHD TV, refer to "About the Samsung Smart Remote (UHD TV)."

\noindent [NOTE] The Samsung Smart Remote may not be supported depending on the model or geographical area.

\vspace{.25cm}

\noindent \textbf{Responses:}

\begin{enumerate}
    \item \textbf{Response:} No, this document does not mention HDR 10 or any other HDR format. \\
          \textbf{Evaluation:}  Other \\
          \textbf{Explanation}: This response has an “other” error because it correctly says “IDK” but it starts the response with “No, …” which makes it seem like the answer to the user’s question is no, which is misleading.
\end{enumerate}
\clearpage

\onecolumn

\section{Human Evaluation: Regular and Challenge Test Sets}
\label{app:human_eval_breakdown}

\subsection{Regular Test Set Results}

\begin{table}[H]
    \centering
    \renewcommand{\arraystretch}{1.2} % Adjust row spacing for better readability
    \resizebox{\textwidth}{!}{ % Automatically resizes to fit within text width
    \begin{tabular}{lcccccccc|c}
        \toprule
        \textbf{Model} & \textbf{Halluc.} & \textbf{Non-Ans} & \textbf{Partial} & \textbf{IDK - Bad} & \textbf{Disfl.} & \textbf{Other} & \textbf{IDK - Good} & \textbf{Good} & \textbf{Total Good} \\
        \midrule
        Pretrained  & 4  & 0  & 4  & 0  & 0  & 0 & 1  & 41  & 42 \\
        GPT-4o  & 2   & 0  & 1  & 0  & 0  & 0 & 1  & 46  & 47 \\
        \hline
        Manual  & 4  & 0  & 5  & 0  & 1 & 0 & 1  & 39  & 40 \\
        Autocleaned\textsubscript{G}  & 4  & 0  & 4  & 0  & 2  & 0 & 0  & 40  & 40 \\
        \hline
        SynthGPT  & 2  & 0  & 0  & 0  & 2  & 0 & 1  & 45  & 46 \\
        SynthLlama  & 2  & 0  & 1  & 0  & 1  & 0 & 1  & 45  & 46 \\
        SynthLlama+  & 2  & 0  & 0  & 0  & 1  & 0 & 1  & 46  & 47 \\
        \bottomrule
    \end{tabular}
    }
    \caption{Human evaluation results on the Regular Test set, assessing response quality across various error categories. Majority vote determined the final category for each item.}
    \label{tab:humaneval_updated_2}
\end{table}

\subsection{Challenge Test Set Results}

\begin{table}[H] % Use table* for spanning both columns
    \centering
    \renewcommand{\arraystretch}{1.2} % Adjust row spacing for better readability
    \resizebox{\textwidth}{!}{ % Automatically resizes to fit within text width
    \begin{tabular}{lcccccccc|c}
        \toprule
        \textbf{Model} & \textbf{Halluc.} & \textbf{Non-Ans} & \textbf{Partial} & \textbf{IDK - Bad} & \textbf{Disfl.} & \textbf{Other} & \textbf{IDK - Good} & \textbf{Good} & \textbf{Total Good} \\
        \midrule
        Pretrained  & 9  & 0  & 2  & 0  & 1  & 5 & 23  & 10  & 33 \\
        GPT-4o  & 7   & 0  & 1  & 1  & 0  & 0 & 28  & 13  & 41 \\
        \hline
        Manual  & 10  & 2  & 2  & 0  & 2 & 5 & 20  & 9  & 29 \\
        Autocleaned\textsubscript{G}   & 9  & 0  & 2  & 0  & 0  & 9 & 19  & 11  & 30 \\
        \hline
        SynthGPT  & 7  & 1  & 0  & 2  & 1  & 8  & 21  & 11  & 32 \\
        SynthLlama  & 5  & 0  & 1  & 0  & 1  & 7  & 25  & 11  & 36 \\
        SynthLlama+  & 4  & 0  & 0  & 0  & 0  & 2  & 30  & 14  & 44 \\
        \bottomrule
    \end{tabular}
    }
    \caption{Human evaluation results on the Challenge Test Set, assessing response quality across various error categories. Majority vote decided the final category for each item.}
    \label{tab:humaneval_updated}
\end{table}
\twocolumn

\section{Human vs. Synthetic Data Analysis}
\label{app:bertscores}

In order to get a better sense of the differences between the datasets, we plot the distribution of BERTScores for each. As you can see, the human-written questions cluster lower, meaning that fewer questions are very similar to each other. Both sets of synthetic questions cluster higher and more evenly, suggesting less variety.

\begin{figure}[H]
    \centering
    \includegraphics[width=0.4\textwidth]{figures/f1_distribution_human.png}
    \caption{Distribution of the BERTScores for every combination of two questions in the crowdsourced dataset.}
    \includegraphics[width=0.4\textwidth]{figures/f1_distribution_synth_gpt.png}
    \caption{Distribution of the BERTScores for every combination of two questions in the SynthGPT dataset.}
    \includegraphics[width=0.4\textwidth]{figures/f1_distribution_synth_llama.png}
    \caption{Distribution of the BERTScores for every combination of two questions in the SynthLlama dataset.}
    \label{fig:bertscore_plots}
\end{figure}


Further, we utilize a t-distributed Stochastic Neighbor Embedding (t-SNE) plot to visualize the embedding space of three datasets: human-generated questions, synthetic questions generated by LLaMA, and synthetic questions generated by GPT. The embeddings are extracted from Llama-3-8B-Instruct (the model we finetune in all our experiments), and the t-SNE method reduces the high-dimensional embeddings into a two-dimensional space for visual interpretation.

This visualization allows us to compare the semantic distributions of the datasets and assess how closely the synthetic datasets align with the human-generated questions. Distinct clustering of the datasets in the t-SNE space suggest meaningful differences in their semantic representations. It seems that the two synthetic questions overlap a great deal and have a fair amount of overlap with the crowdsourced questions. However, the crowdsource (human) questions cluster distinctly to the right, outside the space of the synthetic questions. This also suggests greater variety in the crowdsourced questions.

\begin{figure}[H]
    \centering
    \includegraphics[width=0.5\textwidth]{figures/tsne_plot_all_sets.png}
    \caption{Distribution of the BERTScores for every combination of two questions in the crowdsourced dataset.}
    \label{fig:tSNE}
\end{figure}


\end{document}
