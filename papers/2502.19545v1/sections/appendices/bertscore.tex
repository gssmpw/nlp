\section{Human vs. Synthetic Data Analysis}
\label{app:bertscores}

In order to get a better sense of the differences between the datasets, we plot the distribution of BERTScores for each. As you can see, the human-written questions cluster lower, meaning that fewer questions are very similar to each other. Both sets of synthetic questions cluster higher and more evenly, suggesting less variety.

\begin{figure}[H]
    \centering
    \includegraphics[width=0.4\textwidth]{figures/f1_distribution_human.png}
    \caption{Distribution of the BERTScores for every combination of two questions in the crowdsourced dataset.}
    \includegraphics[width=0.4\textwidth]{figures/f1_distribution_synth_gpt.png}
    \caption{Distribution of the BERTScores for every combination of two questions in the SynthGPT dataset.}
    \includegraphics[width=0.4\textwidth]{figures/f1_distribution_synth_llama.png}
    \caption{Distribution of the BERTScores for every combination of two questions in the SynthLlama dataset.}
    \label{fig:bertscore_plots}
\end{figure}


Further, we utilize a t-distributed Stochastic Neighbor Embedding (t-SNE) plot to visualize the embedding space of three datasets: human-generated questions, synthetic questions generated by LLaMA, and synthetic questions generated by GPT. The embeddings are extracted from Llama-3-8B-Instruct (the model we finetune in all our experiments), and the t-SNE method reduces the high-dimensional embeddings into a two-dimensional space for visual interpretation.

This visualization allows us to compare the semantic distributions of the datasets and assess how closely the synthetic datasets align with the human-generated questions. Distinct clustering of the datasets in the t-SNE space suggest meaningful differences in their semantic representations. It seems that the two synthetic questions overlap a great deal and have a fair amount of overlap with the crowdsourced questions. However, the crowdsource (human) questions cluster distinctly to the right, outside the space of the synthetic questions. This also suggests greater variety in the crowdsourced questions.

\begin{figure}[H]
    \centering
    \includegraphics[width=0.5\textwidth]{figures/tsne_plot_all_sets.png}
    \caption{Distribution of the BERTScores for every combination of two questions in the crowdsourced dataset.}
    \label{fig:tSNE}
\end{figure}