\section{CONCLUSIONS}
\label{sec:conclusions}

This paper proposed RT-FMT, which is a real-time path planning algorithm that is capable of quickly generating low-cost paths in environments with dynamic obstacle avoidance with unknown trajectories. The real-time capability of the algorithm comes from the fact that it generates local paths for the robot to follow while the global path is not available. All these characteristics indicate that RT-FMT is ideal for time-critical applications. To show the capabilities of RT-FMT, we compare it against RT-RRT* on simulated environments with dynamic obstacles. The results show that RT-FMT has higher success rates, lower traveled distances, and smaller arrival times in almost all cases. The greater performance of RT-FMT over RT-RRT* is mostly related to the fact that, in cluttered spaces, the probability of randomly connecting a node to the tree is very low for RT-RRT*, which requires many attempts to grow the tree. On the other hand, RT-FMT starts the search with all nodes already laid out in the environment, which reduces the number of attempts to expand the tree.



%combines the real-time capabilities of RT-RRT* with the advantages of FMT* over RRT* which are quicker planning times and higher success rates. Our approach, RT-FMT, is capable of quickly generating low-cost paths, and it also features dynamic obstacle avoidance with unknown trajectories and multiple-query planning. %This is possible by applying two deterministic rewiring processes that removes branches of the tree that are blocked by any dynamic obstacle and by updating the tree root whenever the robot moves. 
%The real-time capability of the algorithm comes from the fact that it generates local paths for the robot to follow while the global path is not available. All these characteristics indicate that RT-FMT is ideal for time-critical applications.


Future work on the proposed algorithm will involve an implementation in the Open Motion Planning Library (OMPL) \cite{sucan2012the-open-motion-planning-library} to facilitate comparisons with other state-of-the-art approaches, mainly on higher dimensional problems, and to understand the effects of the real-time variation on the complexity of the algorithm and its computing time. The OMPL implementation might also simplify tests on real robots. In addition, a variation of the approach that works for constrained robots can also be implemented.

%Future work on the proposed algorithm will involve an implementation in the Open Motion Planning Library (OMPL) \cite{sucan2012the-open-motion-planning-library} to facilitate bench-marking RT-FMT with other more recent approaches such as RRT*FN-Replan \cite{tong2019rrt}, mainly on higher dimensional problems, which is an important advantage of probabilistic sampling-based planners. It will also facilitate comparisons with other state-of-the-art approaches, mainly to understand the effects of the real-time variation on the complexity of the algorithm and its computing time. The OMPL implementation might also simplify tests on real robots. In addition, a variation of the approach that works for constrained robots can also be implemented.

