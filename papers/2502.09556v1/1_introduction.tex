\section{INTRODUCTION}

Planning optimal paths for mobile robots in environments filled with obstacles is a complex task that may require considerable computation time.  This can cause issues in time-sensitive applications, such as mining \cite{tian2021trajectory} because the longer it takes to execute a task, the lower the efficiency and the higher the expenses. This is also true for search and rescue applications\cite{hayat2020multi} since any saved time is used to look for survivors. These are applications where it is possible for a robot to encounter dynamic obstacles such as humans or other robots. To deal with such scenarios, this paper proposes a new path planning algorithm that works in real-time, thus reducing unnecessary waiting time while still avoiding dynamic obstacles.


% \begin{figure}
% \centerline{\subfigure[Case I]{
% \includegraphics[width=0.475\linewidth]{Figures/mine_a_up.eps}
% \label{fig_first_case}}
% \subfigure[Case II]{
% \includegraphics[width=0.475\linewidth]{Figures/mine_b_up.eps}
% \label{fig_second_case}}}
% \caption{Simulation results}
% \label{fig_sim}
% \end{figure}



\begin{figure}
\centerline{
\subfloat[]{
\includegraphics[width=0.45\linewidth]{Figures/mine_a_up.eps}
\label{fig:mine_a}}\\
\subfloat[]{
\includegraphics[width=0.45\linewidth]{Figures/mine_b_up.eps}
\label{fig:mine_b}}
}
\caption{Illustration of a mine-like space. The start and goal configurations are shown as $x_s$ and $x_g$, respectively. The solution is highlighted in red. The pink squares are the dynamic obstacles representing mining trucks and the dark solid obstacles are the mine pillars. The goal is to traverse the space with the least amount of time to maximize profit.}
\label{fig:mine}
\end{figure}

In dynamic environments, new paths must be generated whenever a path becomes unfeasible. It is possible to save planning time by incorporating a re-planning approach that leverages the current search structure (e.g., a graph) that has been created previously instead of starting a new instance from scratch. One common approach to reducing planning time for time-sensitive applications %, regardless of the existence of dynamic obstacles,
is to generate a sub-optimal but complete path and start the path execution as soon as possible while it continues to optimize the path. Algorithms that fall into this category are, for example, the anytime planners \cite{karaman2011anytime, xu2020informed}. A different approach that can be applied when the application has hard time constraints is to expand the search until it is time to perform an action. In this case, actions are taken before the planner finds the complete trajectory. Algorithms that fall into this category are called \emph{real-time planners} \cite{naderi2015rt}.

While real-time planners provide the fastest trajectory execution, which may allow the robot to arrive at the destination sooner, there are two problems associated with these techniques. At the time to perform an action, the planner applies a heuristic function to decide the best local path to follow with the currently available information. Local minima in the heuristic function can cause the robot to visit unnecessary states. In addition, the planner may lead a robot to a state of unavoidable collision if there is not enough look-ahead \cite{cserna2016anytime}.

These problems can be greatly reduced if the planner is able to quickly search through the space to provide sufficient look-ahead for the real-time planner to select low-cost local paths while it searches for the optimal global path. Fast search is a characteristic of probabilistic sampling-based planners such as RRT* \cite{karaman2011sampling} and FMT* \cite{fmt}, which have been successfully applied in robotic problems for high-dimensional configuration spaces due to the speed at which these algorithms are able to search over the configuration space \cite{karaman2011sampling}. These types of planners are clearly good candidates for real-time applications. To fill this gap, the real-time variant RT-RRT* was proposed by Naderi et al. \cite{naderi2015rt}. This variant is capable of leveraging the generated tree to avoid dynamic obstacles by rewiring the tree around them as the robot executes the generated path. In addition, it generates local paths for execution before the complete path is found for real-time execution. 

Unfortunately, RRT variants have a common drawback. In environments with many obstacles, there is a low probability that a sample will be added to the tree due to collisions, requiring more iterations for the planner to find a solution. In contrast, the FMT* algorithm starts the search with all samples already distributed in the environment. As a result, FMT* provides solutions in less time and with lower costs than RRT*. This effect is discussed in more detail in \cite{fmt}. 

Inspired by the advantages of FMT* over RRT*, this paper proposes RT-FMT\footnote{We decided to drop the ``*'' in FMT* when naming our approach since real-time execution on local paths does not guarantee optimality}, a real-time variant of the FMT* that features local path generation, multiple-query planning, and dynamic obstacles avoidance. During the search, our approach quickly looks for the global solution and, in the meantime, generates local paths that can be used by the robot to start execution faster. In addition, our algorithm constantly updates the tree to avoid dynamic obstacles, and to maintain the tree root near the robot, which allows the tree to be reused multiple times for different goals. Fig.~\ref{fig:mine} shows the RT-FMT in a layout similar to an idealized underground mine.  Fig.~\ref{fig:mine_a} shows the initial stage with the solution highlighted in red, and Fig.~\ref{fig:mine_b} shows the environment some time later. We can see that the tree root is near the robot in both images and that the path has been updated to avoid the nearest truck (pink rectangle) in Fig.~\ref{fig:mine_b}. 

The performance of RT-FMT was evaluated against RT-RRT* in two dynamic environments. The results (see Section \ref{sec:results}) show that robots applying RT-FMT consistently arrived at the goal state before RT-RRT*. In addition, and they also show that when the robots start an early execution, risking moving towards sub-optimal paths, they consistently arrived at the goal state faster when compared to waiting for the complete optimal path.

The remainder of this paper is structured as follows. Section \ref{sec:related} presents some related works on planning in dynamic environments with online execution. Section \ref{sec:algorithm} presents the description of the proposed algorithm. Section \ref{sec:methodology} presents the evaluation methodology. Section \ref{sec:results} presents and discusses the results of the simulated experiments, and Section \ref{sec:conclusions} presents some final remarks and future work.
