%%%%%%%%%%%%%%%%%%%%%%%%%%%%%%%%%%%%%%%%%%%%%%%%%%%%%%%%%%%%%%%%%%%%%%%%%%%%%%%%
%2345678901234567890123456789012345678901234567890123456789012345678901234567890
%        1         2         3         4         5         6         7         8

\documentclass[letterpaper, 10 pt, conference]{ieeeconf}  % Comment this line out if you need a4paper

%\documentclass[a4paper, 10pt, conference]{ieeeconf}      % Use this line for a4 paper

\IEEEoverridecommandlockouts                              % This command is only needed if 
                                                          % you want to use the \thanks command

\overrideIEEEmargins                                      % Needed to meet printer requirements.

%In case you encounter the following error:
%Error 1010 The PDF file may be corrupt (unable to open PDF file) OR
%Error 1000 An error occurred while parsing a contents stream. Unable to analyze the PDF file.
%This is a known problem with pdfLaTeX conversion filter. The file cannot be opened with acrobat reader
%Please use one of the alternatives below to circumvent this error by uncommenting one or the other
%\pdfobjcompresslevel=0
%\pdfminorversion=4

% See the \addtolength command later in the file to balance the column lengths
% on the last page of the document

% The following packages can be found on http:\\www.ctan.org
%\usepackage{graphics} % for pdf, bitmapped graphics files
%\usepackage{epsfig} % for postscript graphics files
%\usepackage{mathptmx} % assumes new font selection scheme installed
%\usepackage{times} % assumes new font selection scheme installed
%\usepackage{amsmath} % assumes amsmath package installed
%\usepackage{amssymb}  % assumes amsmath package installed
\usepackage{graphicx}
\usepackage{cite}
\usepackage{amsmath,amssymb,amsfonts}
\usepackage{algorithmic}
\usepackage{graphicx}
\usepackage{textcomp}
\usepackage{xcolor}
\usepackage[ruled,vlined,linesnumbered]{algorithm2e}
\usepackage{bm}
\usepackage{subfig}
\usepackage{url}
\newcommand\figsize{0.46}

\title{\LARGE \bf Real-Time Fast Marching Tree for Mobile Robot Motion Planning in Dynamic Environments*
\thanks{*This research was supported in part by the Natural Science and Engineering Research Council of Canada (NSERC) under grant number DNDPJ 533392-18, General Dynamics Land Systems (Canada), and Defence R\&D Canada (DRDC).}
}

\author{Jefferson~Silveira$^{1}$\thanks{$^{1}$ J.\ Silveira is with the Department of Electrical \& Computer Engineering and the Ingenuity Labs Research Institute, Queen's University, Kingston, ON K7L 3N6 Canada \texttt{jefferson.silveira@queensu.ca}}, %~\IEEEmembership{Member,~IEEE,}
        Kleber~Cabral$^{2}$\thanks{$^{2}$K.\ Cabral is with the School of Computing, Queen's University, Kingston, ON K7L 3N6 Canada \texttt{kleber.cabral@queensu.ca}}, %~\IEEEmembership{Fellow,~OSA,}
        Sidney~Givigi$^{3}$\thanks{$^{3}$S.\ Givigi is with the School of Computing and the Ingenuity Labs Research Institute, Queen's University, Kingston, ON K7L 3N6 Canada \texttt{sidney.givigi@queensu.ca}}
        and~Joshua~A.~Marshall$^{4}$\thanks{$^{4}$J.\ Marshall is with the Department of Electrical \& Computer Engineering and the Ingenuity Labs Research Institute, Queen's University, Kingston, ON K7L 3N6 Canada \texttt{joshua.marshall@queensu.ca}}%,~\IEEEmembership{Senior Member,~IEEE}% <-this % stops a space
        }
        
% \author{Albert Author$^{1}$ and Bernard D. Researcher$^{2}$% <-this % stops a space
% \thanks{*This work was not supported by any organization}% <-this % stops a space
% \thanks{$^{1}$Albert Author is with Faculty of Electrical Engineering, Mathematics and Computer Science,
%         University of Twente, 7500 AE Enschede, The Netherlands
%         {\tt\small albert.author@papercept.net}}%
% \thanks{$^{2}$Bernard D. Researcheris with the Department of Electrical Engineering, Wright State University,
%         Dayton, OH 45435, USA
%         {\tt\small b.d.researcher@ieee.org}}%
% }

\urlstyle{same}
\begin{document}



\maketitle
\thispagestyle{empty}
\pagestyle{empty}


%%%%%%%%%%%%%%%%%%%%%%%%%%%%%%%%%%%%%%%%%%%%%%%%%%%%%%%%%%%%%%%%%%%%%%%%%%%%%%%%
\begin{abstract}

This paper proposes the Real-Time Fast Marching Tree (RT-FMT), a real-time planning algorithm that features local and global path generation, multiple-query planning, and dynamic obstacle avoidance. During the search, RT-FMT quickly looks for the global solution and, in the meantime, generates local paths that can be used by the robot to start execution faster. In addition, our algorithm constantly rewires the tree to keep branches from forming inside the dynamic obstacles and to maintain the tree root near the robot, which allows the tree to be reused multiple times for different goals. Our algorithm is based on the planners Fast Marching Tree (FMT*) and Real-time Rapidly-Exploring Random Tree (RT-RRT*). We show via simulations that RT-FMT outperforms RT-RRT* in both execution cost and arrival time, in most cases. Moreover, we also demonstrate via simulation that it is worthwhile taking the local path before the global path is available in order to reduce arrival time, even though there is a small possibility of taking an inferior path. 

\end{abstract}


%%%%%%%%%%%%%%%%%%%%%%%%%%%%%%%%%%%%%%%%%%%%%%%%%%%%%%%%%%%%%%%%%%%%%%%%%%%%%%%%



\section{Introduction}\label{sec:intro}

In computational finance, Monte Carlo simulations are used extensively to estimate the expected value of financial payoffs based on the solution of stochastic differential equations (SDEs) which model the evolution of stock prices, interest rates, exchange rates and other quantities \cite{glasserman04}.  Monte Carlo methods are very general and flexible, but for high accuracy it requires generating a large number of costly SDE path approximations, which has motivated research into a number of variance reduction or, equivalently, cost reduction techniques. One such method is
Multilevel Monte Carlo (MLMC), which was proposed in \cite{GILES2008} and was adapted for various applications that are summarised in \cite{Giles_overview17} and successfully combined with other methods such as quasi-Monte Carlo methods. The main idea of MLMC is to approximate the payoff using different time stepping resolutions when numerically solving the underlying SDE and to generate an optimal number of samples on each level, such that the overall computational cost is minimised subject to the desired bound on the variance. %, such that the total computational cost is minimised. 
The computational savings come from the fact that most samples are computed on the coarser levels and hence are less expensive while only a few samples from the finest levels are required \cite{GILES2008}.


Among the directions in which the computational cost 
of MLMC methods could further be reduced, an important avenue is the use of lower precision calculations, especially for the first Monte Carlo levels where the targeted accuracy is relatively low. 
 An overview of the research on mixed precision for the standard Monte Carlo (MC) framework is provided in \cite{ChowMixedPrecisionStandardMC} but only a few references study the potential of low precision computation in the MLMC framework \cite{Rounding_error_oliver}. To the best of our knowledge, the only MLMC framework with customised precision in the literature is \cite{brugger2014mixed}, but they use a uniform precision for all operations on each Monte Carlo level instead of optimising 
 the precision of each intermediary variable to reduce as much as possible the cost of path generation.
 
An important motivation for an MLMC framework with variable precision would be performing the low precision computations on reconfigurable hardware devices such as Field Programmable Gate Arrays (FPGAs). FPGAs contain customizable logic blocks and connectors that make it easy to adapt the digital circuit architecture for a specific application, leading to a highly parallel and optimised implementation. Therefore they are successfully exploited in applications that require high speed and have high computational workload, such as signal processing \cite{woods2008fpga}, and real time applications like high frequency trading \cite{HFT1,HFT2}. That is why a number of previous works in hardware architecture design implemented the MLMC algorithm to price financial options using FPGAs as accelerators, which resulted in improved speed and power efficiency compared to full CPU architectures \cite{Schryver2013AMM}. The paper \cite{lindsey2016domain} also proposed 
a Domain Specific Language to automate the configuration of FPGAs for this specific application. However, only \cite{brugger2014mixed} proposed a heuristic to reduce the precision in calculations.

In addition, all aforementioned works considered that the random number generation (RNG) is performed in single or double precision. Yet in most cases an important portion of the workload in the overall MLMC simulation comes from the RNG and in \cite{brugger2014mixed} this limited the total computational savings.
To reduce the cost of MLMC simulations in particular those based on the Geometric Brownian Motion (GBM), \cite{approximateICDF_Oliver, NestedOliver} have proposed to use approximate random numbers that are generated by applying an approximation of the inverse CDF to uniform random numbers. In \cite{NestedOliver}, the authors proposed a way to integrate these lower precision random variables into a \textit{nested} MLMC framework and completed a numerical analysis to bound the resulting error at each MC level by a product of the time step and the error in the random number approximation. The same authors show in \cite{approximateICDF_Oliver} that using approximate random variables reduces the cost of path generation by a factor 7.


In this paper we propose a nested MLMC framework that combines the use of approximate random normal variables and lower precision calculations to reduce the computational cost of MLMC even further than \cite{brugger2014mixed,NestedOliver}. We illustrate the efficiency of our framework in Matlab, after making several assumptions on the cost of operations and size of the errors that we carefully justify. We focus on the case of GBM and use the approximate RNG methods presented in \cite{approximateICDF_Oliver} as well as a new slightly modified method that combines CDF inversion and the central limit theorem. To choose the precision of the variables in the low precision path generation, we introduce a novel method to optimise the bit-widths. This optimisation is performed before the main path generation loop is executed and is based on a linear model of the payoff error  
due to rounding when computing in low precision. The error model relies on algorithmic differentiation in a similar manner to \cite{unifying-bwoptim,bitwidth-AD,ADAPT}. The bit-width optimisation procedure can be performed off-line, so this stage can be excluded from the on-line time complexity of our framework. The user specified desired accuracy is then enforced by calculating on-line the number of samples that need to be generated.

In terms of hardware design, we suggest implementing the low precision path generation on FPGAs and the full-precision ones on a CPU or GPU. 
The FPGA offers enough flexibility to define a separate bit-width for every variable in the low precision path generation, and can be reconfigured periodically to update the bit-widths when the market parameters have changed considerably. 


The paper is organized as follows : \Cref{sec:MLMC} introduces MLMC and nested MLMC to make clear the estimator that is implemented in our framework. Then in \Cref{sec:RNG} we detail the methods that could be used to obtain approximate random normally distributed numbers very cheaply for the low precision path generation. In \Cref{sec:error_model} and \Cref{sec:costModel} we propose an error model and a cost model (resp.) that we then use to formulate the optimisation problem that is solved to obtain the optimal bit-widths of fixed point variables in \Cref{sec:optimisation}. Finally we summarise our results and future directions in \Cref{sec:conclusion}.




\section{Related Work}

\paragraph{LLMs for Agent tasks.}

Our research is related to deploying large language models (LLMs) as agents for decision-making tasks in interactive environments~\citep{liu2023agentbench,zhou2023webarena,shridhar2020alfred,toyama2021androidenv}. Earlier works, such as~\citep{yao2023webshopscalablerealworldweb}, fine-tuned models like BERT~\citep{devlin2019bertpretrainingdeepbidirectional} for decision-making in simplified environments, such as online shopping or mobile phone manipulation. With the advent of large language models~\citep{brown2020languagemodelsfewshotlearners,openai2024gpt4technicalreport}, it became feasible to perform decision-making tasks through zero-shot or few-shot in-context learning. To better assess the capabilities of LLMs as agents, several models have been developed~\citep{deng2024mind2web,xiong2024watch,hong2023cogagent,yan2023gpt}. Most approaches~\citep{zheng2024seeact,deng2024mind2web} provide the agent with observation and action history, and the language model predicts the next action via in-context learning. Additionally, some methods~\citep{zhang2023building,li2023camel,song2024trial} attempt to distill trajectories from state-of-the-art language models to train more effective policy models. In contrast, our paper introduces a novel framework that automatically learns a reward model from LLM agent navigation, using it to guide the agents in making more effective plans.

\textbf{LLM Planning.} Our paper is also related to planning with large language models. Early researchers~\citep{brown2020languagemodelsfewshotlearners} often prompted large language models to directly perform agent tasks. Later, \citet{yao2022react} proposed ReAct, which combined LLMs for action prediction with chain-of-thought prompting~\citep{wei2022chain}. Several other works~\citep{yao2023treethoughtsdeliberateproblem,hao2023reasoning,zhao2023large,qiao2024agentplanningworldknowledge} have focused on enhancing multi-step reasoning capabilities by integrating LLMs with tree search methods. Our model differs from these previous studies in several significant ways. First, rather than solely focusing on text generation tasks, our pipeline addresses multi-step action planning tasks in interactive environments, where we must consider not only historical input but also multimodal feedback from the environment. Additionally, our pipeline involves automatic learning of the reward model from the environment without relying on human-annotated data, whereas previous works rely on prompting-based frameworks that require large commercial LLMs like GPT-4~\citep{openai2024gpt4technicalreport} to learn action prediction. Furthermore, \Model supports a variety of planning algorithms beyond tree search.

\textbf{Learning from AI Feedback.} In contrast to prior work on LLM planning, our approach also draws on recent advances in learning from AI feedback~\citep{bai2022constitutional,lee2023rlaif,yuan2024self,sharma2024critical,pan2024autonomous,koh2024tree}. These studies initially prompt state-of-the-art large language models to generate text responses that adhere to predefined principles and then potentially fine-tune the LLMs with reinforcement learning. Like previous studies, we also prompt large language models to generate synthetic data. However, unlike them, we focus not on fine-tuning a better generative model but on developing a classification model that evaluates how well action trajectories fulfill the intended instructions. This approach is simpler, requires no reliance on state-of-the-art LLMs, and is more efficient. We also demonstrate that our learned reward model can integrate with various LLMs and planning algorithms, consistently improving their performance.

\textbf{Inference-Time Scaling.} ~\citet{snell2024scaling} validates the efficacy of inference-time scaling for language models. Based on inference-time scaling, various methods have been proposed, such as random sampling~\citep{wang2022self} and tree-search methods~\citep{hao2023reasoning, zhang2024accessing, guan2025rstar}. Concurrently, several works have also leveraged inference-time scaling to improve the performance of agentic tasks. ~\citet{koh2024tree} adopts a training-free approach, employing MCTS to enhance policy model performance during inference and prompting the LLM to return the reward. ~\citet{gu2024your} introduces a novel speculative reasoning approach to bypass irreversible actions by leveraging LLMs or VLMs. It also employs tree search to improve performance and prompts an LLM to output rewards. ~\citet{yu2024exact} proposes Reflective-MCTS to perform tree search and fine-tune the GPT model, leading to improvements in ~\citet{koh2024visualwebarena}. ~\citet{putta2024agent} also utilizes MCTS to enhance performance on web-based tasks such as ~\citet{yao2023webshopscalablerealworldweb} and real-world booking environments. ~\cite{lin2025qlass} utilizes the stepwise reward to give effective intermediate guidance across different agentic tasks. Our work differs from previous efforts in two key aspects: (1) Broader Application Domain. Unlike prior studies that primarily focus on tasks from a single domain, our method demonstrates strong generalizability across web agents, mathematical reasoning, and scientific discovery domains, further proving its effectiveness. (2) Flexible and Effective Reward Modeling. Instead of simply prompting an LLM as a reward model, we finetune a small scale VLM~\citep{lin2023vila} to evaluate input trajectories. %Our reward scores range continuously between 0 and 1, in contrast to existing methods that rely on discrete scoring (e.g., 0 and 1, or 0, 0.5, and 1) through direct LLM prompting.

% Concurrently, several works have also leveraged inference-time scaling to improve the performance of agentic tasks. ~\citet{pan2024autonomous} demonstrates that LLMs and VLMs, such as the GPT series, can function as evaluators or reward models to provide guidance for fine-tuning or reflection, thereby enhancing digital agents. This lays the groundwork for subsequent studies that directly prompt LLMs as reward models. ~\citet{koh2024tree} adopts a training-free approach, employing MCTS to enhance policy model performance during inference. However, it is limited to web environments~\citep{koh2024visualwebarena}. Moreover, its value function relies on prompting an LLM, which is less effective than our proposed method. We validate our approach through ablation studies, demonstrating that our fine-tuned reward model is more effective. ~\citet{gu2024your} introduces a novel speculative reasoning approach to bypass irreversible actions, such as purchasing a product, by leveraging LLMs or VLMs. It also employs tree search to improve performance, but it remains restricted to the web domain~\citep{koh2024visualwebarena, deng2024mind2web}. Additionally, it lacks reward modeling and instead prompts an LLM to output rewards. ~\citet{yu2024exact} proposes Reflective-MCTS to perform tree search and fine-tune the GPT model, leading to improvements in ~\citep{koh2024visualwebarena}. However, this work focuses solely on a single web agent task, and its reward modeling is derived from multi-agent debate, differing from our more effective and efficient reward modeling approach. ~\citet{putta2024agent} also utilizes MCTS to enhance performance, but it is limited to web-based tasks such as ~\citep{yao2023webshopscalablerealworldweb} and real-world booking environments.

%\section{Problem Formulation and Notations}

\section{THE REAL-TIME FAST MARCHING TREE ALGORITHM}
\label{sec:algorithm}
In general, RT-FMT works by expanding a tree, similarly to how FMT* does, while it also checks for dynamic obstacles and rewires the tree around them. During the search, the algorithm also searches for the local paths with the least costs. These paths are used for the robot to start moving before planning is finished. When the robot reaches a new waypoint in the path, the root of the tree is updated. This event triggers a complete rewire of the tree to update the costs of all nodes. In a real-time application, the robot sends velocity commands at specific intervals. Therefore, we only allow the tree expansion and rewiring to run for a defined number of iterations ($N_e$) to expand and rewire the tree. This method can also be easily adapted to run for a desired time interval or planning frequency. 

The method is shown in Algorithm \ref{alg:main}, which starts by sampling $N$ configurations in free space (Line \ref{alg:samplefree}) considering only the fixed obstacles. This function also calculates the neighborhood radius $r_n$ based on the number of samples and the dimensionality of the problem according to
\begin{equation}
\label{eq:rn}
    r_n = \gamma_s 2\bigg(1 +\frac{1}{d}\bigg)^{\frac{1}{d}}\bigg(\frac{\mu(\mathcal{X}_{\rm free})}{\zeta_d} \bigg)^\frac{1}{d} \bigg(\frac{\log(N)}{N} \bigg)^\frac{1}{d},
\end{equation}
where $\gamma_s > 1$ is a tuning parameter, $d$ is the dimension of the problem, $\mu(\mathcal{X}_{\rm free})$ is the Lebesgue measure of the free space, and $\zeta_d$ is the volume of a unit ball in $\mathbb{R}^d$. Although FMT* provides an equation for $r_n$, we use the equation defined in \cite{karaman2011sampling} for PRM* since it computes an $r_n$ slightly bigger than the FMT* equation for $r_n$.

\begin{algorithm}[tb]
\SetKwFunction{SampleFree}{SampleFree}
\SetKwFunction{UpdateContext}{UpdateContext}
\SetKwFunction{ExpandFMT}{ExpandFMT}
\SetKwFunction{RewireFromObstacles}{RewireFromObstacles}
\SetKwFunction{RewireFromRoot}{RewireFromRoot}
\SetKwFunction{GeneratePath}{GeneratePath}
\SetKwFunction{UpdateRoot}{UpdateRoot}

 $\mathcal{T} \leftarrow \bm x_{s}$;  $\bm x_{\rm root} \leftarrow \bm x_{s}$\;
 $\mathcal{S} \leftarrow \SampleFree(N) \cup \bm x_{s} \cup \bm x_{g}$\; \label{alg:samplefree}
 $\mathcal{V}_{b} \leftarrow \emptyset$; $\mathcal{Q}_{o} \leftarrow \emptyset$;  $\mathcal{Q}_{r} \leftarrow \emptyset$\;
 $\mathcal{V}_{unv} \leftarrow \mathcal{S}\backslash\{\bm x_{s}\}$; $\mathcal{V}_{\rm open} \leftarrow \{\bm x_{s}\}$; $\mathcal{V}_{\rm closed} \leftarrow \emptyset$\;
 $\bm z \leftarrow \bm x_{s}$\;
 \While{True}{
 ($\bm x_{\rm robot} ,\bm x_{g}, \mathcal{N}_{b}) \leftarrow \UpdateContext(\mathcal{T}, \mathcal{X}_{\rm Dobs})$\; \label{alg:updateContext}
 \For{$i=1$ \KwTo $N_{e}$}{
  $\ExpandFMT(\mathcal{T})$\; \label{alg:expandAndRewire}
  $\RewireFromObstacles(\mathcal{T})$\; \label{alg:rewireObs}
  $\RewireFromRoot(\mathcal{T})$\; \label{alg:rewireRoot}
}
 $(\bm x_{\rm root}, \bm x_1, ..., \bm x_k) \leftarrow \GeneratePath(\mathcal{T}, \bm x_{\rm root})$\; \label{alg:generatepath}
 \If{$\bm x_{\rm robot}$ \upshape{is near} $\bm x_{\rm root}$}{
    $\bm x_{\rm root}\leftarrow \bm x_{1}$\;
    $\UpdateRoot(\mathcal{T},\bm x_{\rm root})$\; \label{alg:updateroot}
 }
 Steer robot towards  $\bm x_{\rm root}$\; \label{alg:steer}
 Perform other tasks\;
}
\caption{RT-FMT($\bm x_{s}, \bm x_{g}, \mathcal{X}_{Fobs}, \mathcal{X}_{\rm Dobs}, N_s, N_e $)}
\label{alg:main}
\end{algorithm}

Inside the infinite loop, the algorithm updates the context of the problem by returning the current position of the robot and goal $x_{g}$, and by finding the nodes in the tree that have been blocked or unblocked by the dynamic obstacles. When a configuration $\bm x_q$ in the tree is blocked, its cost $\textup{c}(\bm x_q)$ is set to infinity. When a node is unblocked, its cost is updated to
\begin{equation}
    \textup{c}(\bm x_q) = \textup{c}(\bm x_{\rm parent}) + \textup{Cost}(\bm x_{\rm parent}, \bm x_{q}), \label{eq:cost}
\end{equation}
where $\bm x_{\rm parent}$ is the configuration of the parent of $\bm x_q$, and
\begin{equation}
    \textup{Cost}(\bm u, \bm v) = ||\bm v- \bm u||.
\end{equation}
If the updated node has children, its children's costs are recursively updated. Lines \ref{alg:expandAndRewire}\textendash\ref{alg:rewireRoot} expand the tree according to Algorithm \ref{alg:ExpandFMT}, and rewire the tree based on Algorithm \ref{alg:RewireFromObstacle}. Line \ref{alg:generatepath} returns a global path if $\bm x_g$ is in the tree or a local path otherwise. The local path is found by computing 
\begin{equation}
    \bm x_k \leftarrow \arg\min_{\bm x \in \mathcal{T}}\textup{c}(\bm x) +  ||\bm x -\bm x_{goal}||,
\end{equation}
and then a path starting at $x_{\rm root}$ and ending at $\bm x_k$ is generated. While we do not limit $k$, RT-RRT* generates a path up to a specific $k$. More details on the local path generation can be found in Algorithm 6 in Naderi et al.~\cite{naderi2015rt}. The approach resembles the A* search. 

In Line \ref{alg:updateroot}, the root of the tree is set to the next configuration in the path, which is always the second element since the path always starts at the old root.  Finally, the algorithm steers the robot towards the new root of the tree on Line \ref{alg:steer}. If there are other tasks to perform such as mapping, and localization, they can be called in the main function as well. 


\subsection{Expanding the Tree}

The tree expansion is described  in Algorithm \ref{alg:ExpandFMT}. Most of the algorithm (Lines \ref{alg:x_near}\textendash\ref{alg:findz}) was inspired by FMT*, proposed in \cite{fmt}, but our implementation has two major differences. 

First, the loops in the original implementation were substituted for conditional statements. These statements ensure that only one node can be added in the tree per call. If multiple nodes are added to the tree at once, there is a possibility of delaying other tasks since the processor will spend too much time expanding many nodes at once. In addition, Line \ref{alg:checkdynamic} not only checks for fixed obstacles but also checks whether $\bm y_{\rm min}$ is not being blocked by a dynamic obstacle according to (\ref{eq:cost}). 

Second, in the original approach, once a node has checked all possible connections with its neighbors, it is closed and never checked again. In our approach, we do not spend time expanding the nodes that are inside $\mathcal{X}_{\rm Dobs}$. As a consequence, when a dynamic obstacle moves, there will be unvisited nodes around closed nodes. To add these nodes to the tree, our algorithm must be able to reopen closed nodes nearby. This is done by adding all $\bm z$ that are near unvisited nodes at closing time to $\mathcal{V}_{\rm toOpen}$ (Line \ref{alg:ztoopen}). Then, when the regular expansion is finished (Line \ref{alg:znull}), the algorithm reopens these nodes (Line \ref{alg:reopen}) to continue the expansion in case a dynamic obstacle moves.

\begin{algorithm}[tb]
\SetKwFunction{Near}{Near}
\SetKwFunction{PopLast}{PopLast}
\SetKwFunction{Cost}{Cost}
\SetKwFunction{c}{c}
\SetKwFunction{CollisionFree}{CollisionFree}
\SetKwFunction{Open}{Open}
\SetKwFunction{Close}{Close}
\lIf{$\mathcal{X}_{near}$ = $\emptyset$ $\&$ $\bm z \neq \emptyset$}{$\mathcal{X}_{near} \leftarrow$ \Near($\bm z$,$\mathcal{V}_{unv}$)} \label{alg:x_near}
\Else{
$\bm x = \PopLast(\mathcal{X}_{near})$\; 
$\mathcal{Y}_{near} \leftarrow$ \Near($\bm x$,$\mathcal{V}_{\rm open}$)\;
\bm $y_{\rm min} \leftarrow \arg\min_{\bm y \in \mathcal{Y}_{near} }(\c(\bm y) + \Cost(\bm x, \bm y))$\;

\If{$\CollisionFree(\bm y_{\rm min}, \bm x) \& \c(\bm y_{\rm min}) < \infty$}{ \label{alg:checkdynamic}
$\mathcal{V}_{open, new} \leftarrow  \mathcal{V}_{open, new} \cup \{\bm x\}$\;
$\mathcal{V}_{unv} \leftarrow \mathcal{V}_{unv}\backslash\{\bm x\}$\;
$\c(\bm x) \leftarrow \c(\bm y_{\rm min}) + \Cost(\bm y_{\rm min}, \bm x)$\;
$\mathcal{T} \leftarrow \mathcal{T} \cup \{\bm x, (\bm y_{\rm min}, \bm x) \}$
}
\If{$\mathcal{X}_{near}$ = $\emptyset$ $\&$ $\bm z \neq \emptyset$}{
\Close($\bm z$)\;
$\mathcal{Z}_{near} \leftarrow$ \Near($\bm z$,$\mathcal{V}_{unv}$)\;
$\bm z \leftarrow \arg\min_{\bm y \in \mathcal{V}_{\rm open} }\c(\bm y)$\; \label{alg:findz}
\ForEach{$\bm x \in \mathcal{Z}_{near}$ }{ \label{alg:ztoopen}
\lIf{\CollisionFree($\bm z, \bm x$)}{$\mathcal{V}_{\rm toOpen} \leftarrow \mathcal{V}_{\rm toOpen} \cup \bm z  $} 
}

}
\If{$\bm z = \emptyset$}{ \label{alg:znull}
$\Open(\mathcal{V}_{\rm toOpen})$;~$\mathcal{V}_{\rm toOpen} = \emptyset$\; \label{alg:reopen}
$\bm z \leftarrow \arg\min_{\bm y \in \mathcal{V}_{\rm open} }\c(\bm y)$\; }
}

\SetKwProg{Def}{def}{:}{}
\Def{\Close($\mathcal{V}$)}{
$\mathcal{V}_{\rm open} \leftarrow  \mathcal{V}_{\rm open} \cup \mathcal{V}_{open, new}\backslash\mathcal{V}$\;
$\mathcal{V}_{\rm closed} \leftarrow  \mathcal{V}_{\rm closed} \cup \mathcal{V}$\;
}

\SetKwProg{Def}{def}{:}{}
\Def{\Open($\mathcal{V}$)}{
$\mathcal{V}_{\rm open} \leftarrow  \mathcal{V}_{\rm open} \cup  \mathcal{V}$\;
$\mathcal{V}_{\rm closed} \leftarrow  \mathcal{V}_{\rm closed} \backslash \mathcal{V}$\;
}
\caption{ExpandFMT($\mathcal{T}$)}
\label{alg:ExpandFMT}
\end{algorithm}

\subsection{Rewiring the Tree}

As the dynamic obstacles move around the environment, the tree nodes are constantly being blocked and unblocked by Line \ref{alg:updateContext} in Algorithm \ref{alg:main}. When a node is blocked or unblocked, its cost is changed and all its children are recursively updated. The task of the {\tt RewireFromObstacle} method in Algorithm \ref{alg:RewireFromObstacle} is to find the connections with lower cost in the neighborhood of nodes that have recently been blocked or unblocked. This function only rewires the nodes that have recently been affected by a dynamic obstacle. The rewiring process starts by adding all blocked nodes to $\mathcal{Q}_o$. Then, nodes are iteratively removed from the list and the algorithm tries to find parents nearby with a lower cost. If there is a connection with a lower cost that is also collision-free, the children of the updated nodes are also added to $\mathcal{Q}_o$. 

\begin{algorithm}[tb]
\SetKwFunction{PopFirst}{PopFirst}
\SetKwFunction{UpdateParentChild}{UpdateParentChild}
\SetKwFunction{RecalculateChildrenCost}{RecalculateChildrenCost}
\lIf{$\mathcal{Q}_o$ = $\emptyset$}{$\mathcal{Q}_o \leftarrow \mathcal{N}_{b}$} \label{alg:addblockedtolist}
\Else{
$\bm x_b \leftarrow \PopFirst(\mathcal{Q}_o)$\;
   \If{$\bm x_b \notin \mathcal{X}_{\rm Dobs}$}{
        $\mathcal{Y}_{near} \leftarrow \Near(\bm x_b,\mathcal{V}_{\rm open}\cup \mathcal{V}_{\rm closed}$)\;
        \bm $y_{\rm min} \leftarrow \arg\min_{\bm y \in \mathcal{Y}_{near} }(\c(\bm y) + \Cost(\bm x_b, \bm y))$\;
        \If{$\CollisionFree(\bm y_{\rm min}, \bm x) \& \c(\bm y_{\rm min}) < \infty$}{
        $\UpdateParentChild(\mathcal{T}, \bm y_{\rm min}, \bm x_b)$\;
        $\RecalculateChildrenCost(\bm x_b)$
        }
   }

}
\caption{RewireFromObstacles($\mathcal{T}$)}
\label{alg:RewireFromObstacle}
\end{algorithm}

As the robot moves around the environment, the tree root is updated by Algorithm \ref{alg:main}, Line \ref{alg:updateroot}. When this happens, the algorithm triggers the {\tt RewireFromRoot} function (Line \ref{alg:rewireRoot}) to rewire all nodes in the tree by inserting the new root into $\mathcal{Q}_{r}$. The {\tt RewireFromRoot} function is very similar to Algorithm \ref{alg:RewireFromObstacle}. The algorithm removes a node from $\mathcal{Q}_{r}$ and tries to find better connections in the tree. When a new connection is made, the children of the updated node are also added to $\mathcal{Q}_{r}$. This causes a chain reaction that starts from the root and updates all nodes in the tree. This method has also been implemented without any loops to only update a single node per call. Our implementation triggers this chain reaction whenever the root is updated. However, this can easily be modified to happen at a desired frequency.


%\subsection{Real-time considerations}




\section{Methodology}
\label{section:methodology}

Table~\ref{tab:Hardware configurations} lists the system configurations of \att{} and our GPU baseline.
The GPU system contains 4 NVIDIA A100 80GB GPUs equipped with the NVLink 3.0 interconnect.
\att{} has 32 CXL devices, resulting in a similar average power to the GPU system, as further explained in Section~\ref{subsec:power_results}. 

\begin{table}[h]
\footnotesize
\renewcommand\arraystretch{1.1}
\centering
    \caption{Evaluated system configurations}
    \label{tab:Hardware configurations}    
    \scalebox{1}{
        \begin{tabular}{|c||c|c|}
            \hline
            \textbf{System} & \textbf{\att{}} & \textbf{GPU} \\
            \hline
            \hline
            Hardware & 32 CXL devices & 4 NVIDIA A100 \\
            \hline
            Process & 1Y nm (14-16nm) & 7nm \\
            \hline
            Memory & 512GB, GDDR6 & 320GB, HBM2E \\
            \hline
            \multirow{2}{*}{\shortstack{Compute \\ Throughput}} & 512 TFLOPS (PIM) & \multirow{2}{*}{\shortstack{1248 TFLOPS}} \\
            \cline{2-2}
            & 96 TFLOPS (PNM) & \\
            \hline
            Peak Bandwidth & 512 TB/s (Internal) & 8 TB/s (External) \\
            \hline
            3-Year Owned TCO & 0.73\$/hour & 1.76\$/hour \\ 
            \hline
            3-Year Rental TCO & 1.05\$/hour & 5.45\$/hour \\ 
            \hline
            \hline
            GDDR6-PIM & \multicolumn{2}{|c|}{$t_{RCDRD}$=18ns, $t_{RAS}$=27ns, $t_{CL}$=25ns} \\
            Timing Constraints & \multicolumn{2}{|c|}{$t_{RCDWR}$=14ns, $t_{CCDS}$=1ns, $t_{RP}$=16ns} \\
            \hline
        \end{tabular}
    }
\end{table}

We benchmark Llama2 7B, 13B, and 70B models~\cite{touvron2023llama}.
Each evaluated query contains 512 tokens in the prefill stage and 3584 tokens in the decoding stage, adding up to a context length of 4K, \textit{i.e.,} the maximum supported by the Llama2 models.
For a fair comparison between \att{} and the GPU baseline, we deploy these models using different configurations for different parameter sizes: 1, 2, and 4 GPUs, and 8, 20 and 32 CXL devices.
We use vLLM~\cite{vLLM}, the \sota{} inference serving framework on GPUs with a batch size of 128, where the inference throughput saturates (Figure~\ref{fig:Context_Length}).



We generate \att{} instruction traces for a single block and verify the correctness using a functional simulator.
We modify Ramulator2~\cite{luo2023ramulator} to model a CXL device containing 32 GDDR6-PIM memory channels with timing constraints in Table~\ref{tab:Hardware configurations}. 
The inter-device communication through the CXL 3.0 protocol is modeled by an analytical model based on the CXL latency~\cite{li2023pond} and PCIe 6.0 bandwidth. 
To model a CXL switch supporting multicast, we use half of the bandwidth and double the latency of the baseline switch. 
We use Intel Xeon Gold 6430L CPU~\cite{intelxeon} as the host machine in \att{}.

We use Micron DRAM Power Calculator~\cite{micron-power-calculator} to evaluate DRAM core power using current and voltage specifications of Samsung's 8Gb GDDR6 SGRAM C-die~\cite{samsung-8gb-gddr6}.
The MAC operation power is modeled assuming 3$\times$ more current than a typical gapless read~\cite{aim2}.
We assume that each GDDR6 memory controller for two channels consumes 314.6 mW~\cite{dram-controller-power} and each BOOM RISC-V core consumes 250 mW~\cite{boom-pdf}.
We implement the RTL of the remaining components in the CXL controller and synthesize it using a TSMC 28nm technology library and the Synopsys Design Compiler~\cite{synopsis_dc}.
We find the critical path delay as 1ns at 28nm and project the CXL controller clock frequency to be 2.0 GHz at 7nm~\cite{scaling-technology}.

We estimate the die area of CXL controller in two parts. First, we synthesize the custom logic in 28nm~(See Table~\ref{tab:Area_and_power}) and scale it down to 7nm~\cite{scaling-technology}. Then, we add measurements of the memory controller, PCIe controller, and PHY from the NVIDIA GPU die shots~\cite{TU104, A100-die-shot}, which are also scaled down to 7nm. 
This results in an estimated area of 19.0$mm^2$ in 7nm.


\begin{table}[h]
\footnotesize
\centering
\caption{CXL Controller Custom Logic Area\&Power in 28nm}
    \label{tab:Area_and_power}    
    \scalebox{1}{
        \begin{tabular}{|c||c|c|c|}
            \hline
            \multicolumn{2}{|c|}{Components} & Area (mm$^2$) & Power (W) \\
            \hline
            \hline
            \multirow{2}{*}{SRAM} & Instruction Buffer & 3.33 & 0.61 \\
            \cline{2-4}
            % \hline
            & Shared Buffer & 0.11 & 0.03 \\
            \hline
            \multirow{3}{*}{Logics} & Accelerators & 1.34 & 0.18 \\
            \cline{2-4}
            % \hline
            & RISC-V Cores & 2.94 & 0.19 \\
            \cline{2-4}
            % \hline
            & Others & 0.12 & 0.05 \\
            \hline
            \hline
            \multicolumn{2}{|c|}{\textbf{Total}} & \textbf{7.85} & \textbf{1.06} \\
            \hline
        \end{tabular}
    }
\end{table}





\label{TCO}

Table~\ref{tab:Hardware configurations} presents the 3-year Total Cost of Ownership (TCO) for both owned and rental hardware. (a) \textit{Own TCO:} We model a local server by accounting for hardware and operational costs. (b) \textit{Rental TCO:} The cost for host CPU in \att{} and GPU are estimated based on the Microsoft Azure prices ~\cite{azure-price}. The CXL devices in \att{} are evaluated using the owned TCO methodology, as there are no available references for rental costs. To calculate operational cost, we use \$$0.139/KWh$~\cite{electricity-price} and average power consumption. Hardware costs are listed in Table~\ref{tab:hardware_cost}. While the lowest available price for A100 80GB is close to \$20,000, we instead use only \$10,000 by conservatively deducting 50\% margin~\cite{gpu-price}. The PIM module cost is estimated as 10$\times$ the cost of standard DRAM modules~\cite{pim-price, dram-price}.

\begin{table}[h]
    \footnotesize
    \centering
    \caption{Hardware Costs}
    \begin{tabular}{|c||c|c|}
        \hline
        \textbf{System} & \textbf{Hardware} & \textbf{Cost (\$)} \\
        \hline
        \hline
        \multirow{3}{*}{GPU} & Xeon Gold 6430 CPU~\cite{CPU-price} & 2,128 \\
        & 4 NVIDIA A100 80GB GPU~\cite{gpu-price} & 40,000 \\
        \cline{2-3}
        & \textbf{Total Cost} & \textbf{42,128}  \\
        \hline
        \hline
        \multirow{5}{*}{\shortstack{\att{} \\ 32 devices}} & Xeon Gold 6430 CPU~\cite{CPU-price} & 2,128 \\
        & 512GB GDDR6-PIM~\cite{pim-price, dram-price} & 11,873 \\
        & 32 CXL Controllers & 381.3 \\
        & 96-lane 48-port switch~\cite{switch-price} & 490 \\
        \cline{2-3}
        & \textbf{Total Cost} & \textbf{14,873} \\
        \hline
    \end{tabular}
    \label{tab:hardware_cost}
\end{table}

Figure~\ref{fig:TCO} illustrates the breakdown of CXL controller cost per \att{} CXL device (Figure~\ref{fig:CXL_device}). The CXL controller costs are broken down into die, packaging and Non Recurring Engineering (NRE) cost components~\cite{ning2023supply, moonwalk}. Die cost is derived from the wafer cost, considering the CXL controller die area (19.0$mm^2$ in 7nm) and yield rate. A 300mm diameter 7nm wafer costs \$9,346 with a defect density of 0.0015 per $mm^2$~\cite{ning2023supply}. Cost of 2D packaging is assumed to be 29\% of chip cost~\cite{packaging-cost}, while the 2.5D packaging cost is calculated based on interposer, die placement and substrate assembly~\cite{palesko2014cost}.
NRE cost is influenced by chip production volumes, which we estimate at 3 million units based on the following assumptions.
NVIDIA shipped $3.76M$ datacenter GPUs in 2023~\cite{GPU-volume}.
We assume that $10\%$ of datacenter GPUs (around $370K$) are used for LLM inference.
Since each GPU consumes ${\sim}8\times$ more power compared to a CENT device (explained in Section~\ref{subsec:power_results}), we project ${\sim}3M$ volume for \att{} devices.

\begin{figure}[t]
    \centering
    \includegraphics[width=8cm]{Figure/TCO_NEW.pdf}
    \caption{CXL Controller Cost Breakdown}
    \label{fig:TCO}
\end{figure}








\section{Evaluation Results} \label{section:restuls}

\subsection{Interference-Free Analysis}
\noindent
\textbf{Performance of the \texttt{Exchange} primitive.}
Figure~\ref{fig:io-bandwidth} illustrates a comparison of the IO throughput achieved by our optimized \texttt{Exchange} and the baseline solution, which solely relies on the GPU runtime. 
We vary the total amount of data transferred from 2GB to 16GB and adjust the packet size from 10MB to 80MB. 
The combination of data size and packet size determines the total number of packets, which in turn affects the number of pipeline stages required for data transfer. 
Too few pipeline stages can lead to significant overhead in the prologue and epilogue phases of the pipeline. 
Conversely, utilizing excessively small packets is also inefficient, as each memory copy incurs a fixed overhead from the runtime, regardless of the transferred data volume. 
Therefore, small packet sizes exacerbate this overhead, making it disproportionately large.

Our solution achieves up to 140GB/s throughput when transferring 8GB or more of data. 
When the total amount of data transferred is small, we observe a decrease in throughput due to the reduced number of pipeline stages. 
As previously explained, this issue cannot be alleviated by simply reducing the packet size. 
For instance, while a packet size of 10MB provides better performance compared to an 80MB packet size when transferring 2GB of data, it delivers lower throughput when the data size exceeds 8GB. 
Empirically, we find that a packet size of 20MB strikes a balance, achieving desirable performance for small and large data transfers.
Consequently, we use a packet size of 20MB for all the applications evaluated below.

Compared to the baseline, which fully relies on the GPU runtime, our solution is not only more performant but also more stable. 
Such a baseline fails to fully exploit the full-duplex capabilities of PCIe links, achieving only about 110-130GB/s throughput when transferring data bidirectionally. 
Additionally, its performance is highly unstable due to the irregular PCIe bandwidth, especially when the CPU DRAM bandwidth becomes saturated.

%%%%%%%%%%%% OLD TEXT START %%%%%%%%%%%%
\begin{comment}

Figure \ref{fig:io-bandwidth} compares the IO throughput achieved by our optimized \texttt{Exchange} with the one achieved by the baseline solution only relying on the GPU runtime.
We vary the total amount of data transferred from 2GB to 16GB, and the packet size from 10MB to 80MB.
The amount of data and packet size determines the total number of packets, which consequently determines the number of pipeline stages for the data transfer.
If there are too few pipeline stages, the overhead in the prologue and epilogue of the pipeline becomes considerable.
On the other hand, using tiny packets is also unacceptable.
Each memory copy pays a fixed overhead for the runtime regardless of the amount of data being transferred.
Tiny packets make such overhead significant.

Our solution achieves up to around 140GB/s throughput when transferring 8GB or more data.
When the total amount of data transferred is less, we observe decreased throughput due to fewer pipeline stages.
As explained, this can not be relieved by reducing the packet size.
While the case of 10 MB packet size achieves a better performance than the case of 80MB packet size when the total amount of data being transferred is 2GB, it delivers less throughput when the data size is larger than 8GB.
Empirically, we find 20MB is a sweet point that achieves desirable performance in transferring small and large amounts of data.
Thus, we use 20MB for all the applications evaluated below.

Compared to the baseline that fully depends on the GPU runtime, our solution is not only more performant but also more stable.
The baseline solution fails to take advantage of the full-duplex property of PCIe links properly, thus it only achieves around 110-130GB/s throughput when transferring the traffic in both directions. 
Besides, its performance is highly unstable due to the irregular PCIe bandwidth when the CPU DRAM bandwidth is saturating.
\end{comment}
%%%%%%%%%%%% OLD TEXT END %%%%%%%%%%%%

% \begin{figure}
%     \centering
%     \includegraphics[width=0.8\linewidth]{figures/sort-result.pdf}
%     \caption{Results for Sort. (a) the throughput achieved by different solutions, (b) the time breakdown for the \THISWORK\ sort, and (c) the time taken by on-GPU kernel execution of a typical pipeline stage.}
%     \label{fig:sort-perf}
% \end{figure}

\noindent
\textbf{Performance of Sort.}
We compare our sort implementation with CPU and GPU baselines in Figure~\ref{fig:sort-perf}(a). 
Our implementation achieves a throughput of 2.7 billion elements per second, which is 27.9$\times$ faster than TBB, 6.3$\times$ faster than PARADIS, and 1.7$\times$ faster than the configuration using only one GPU's IO resources. 
Figure~\ref{fig:sort-perf}(b) provides a breakdown of the sort operation, revealing that 65.1\% of the time is consumed by the \texttt{SortExKernel}. 
This occurs because, after enhancing the IO throughput, the sorting operation becomes bounded by the GPU processing throughput, as illustrated in Figure~\ref{fig:sort-perf}(c). 
While it takes the GPU approximately 208ms to sort a partition of 500 million 8-byte integers, transferring that partition to the GPU using four GPUs' IO resources requires only about 113ms. 
This limitation explains why we do not achieve nearly a 4$\times$ speedup compared to the single GPU IO solution. 
Conversely, the \texttt{MergeExKernel} remains IO-bound, with the on-GPU kernel completing in approximately 67ms.

%%%%%%%%%%%% OLD TEXT START %%%%%%%%%%%%
\begin{comment}
We compare our sort implementation with the CPU and GPU baselines in Figure~\ref{fig:sort-perf}(a).
Our sort implementation achieves 2.67B elements per second throughput, which is 27.9$\times$ compared to TBB, 6.3$\times$ compared to PARADIS, and 1.7$\times$ compared to the case using only one GPU's IO.
Figure~\ref{fig:sort-perf}(b) is the time breakdown of the sort operation, where 65.1\% of time is spent on the \texttt{SortExOperation}.
The reason is that after we enhance the IO throughput, sorting the array by partition is bounded by GPU-processing throughput, which is showcased in Figure~\ref{fig:sort-perf}(c).
It takes the GPU ~208ms to sort a partition of 500M 8-byte integers, but only ~113ms to transfer that partition to GPU using 4 GBUs' IO resources.
This is why we do not achieve close to 4$\times$ speedup compared to the single GPU IO solution.
On the other hand, \texttt{MergeExOperation} is still IO-bound, which finishes the on-GPU kernel in ~67ms.
\end{comment}
%%%%%%%%%%%% OLD TEXT END %%%%%%%%%%%%

% \begin{figure}
%     \centering
%     \includegraphics[width=0.8\linewidth]{figures/join-result.pdf}
%     \caption{Results for Hash Join. (a) the throughput achieved by different solutions, (b) the time breakdown for the \THISWORK\ hash join, and (c) the time taken by on-GPU kernel execution of a typical pipeline stage.}
%     \label{fig:hash-join-perf}
% \end{figure}

\begin{figure*}[t]
\centerline{\includegraphics[width=\linewidth]{figures/ssb-result.pdf}}
\caption{Star Schema Benchmark execution time and speedup.}
\label{fig:ssb-perf}
\end{figure*}
\begin{figure}
    \centering
    \includegraphics[width=0.86\linewidth]{figures/interference.pdf}
    \caption{Interference between \THISWORK\ on the target GPU and the deep learning applications on the forwarding GPUs. 
    % (a) The slowdown for the deep learning applications (x-axis) when the IO traffic (y-axis) runs in the background. 
    % (b) The slowdown for the \THISWORK\ applications (y-axis) when the deep learning applications (x-axis) run in the background.
    }
    \label{fig:interference}
\end{figure}
\noindent
\textbf{Performance of Hash Join.}
In contrast to sorting, hash join remains an IO-bound kernel even with our IO optimization technique. 
As shown in Figure~\ref{fig:sort-perf}(d), our solution achieves a throughput of 2.3 billion tuples per second. 
This is 24.1$\times$ faster than DuckDB, 2.4$\times$ faster than Triton Join (CPU), 1.3$\times$ faster than the CPU-GPU-NVLink-based Triton Join (GPU), and 3.2$\times$ faster than the single GPU solution using a standard PCIe link.
The speedup over the single GPU IO solution is more pronounced because all phases of hash join are IO-bound. 
This is evident in Figure~\ref{fig:sort-perf}(f). 
The \texttt{HashJoinExKer} requires only 34ms to complete the on-GPU join kernel, which is significantly less than the 61ms required for data transfer.
Similarly, it takes 90ms to partition a chunk of data, which is transferred in around 113ms. 
All phases scale uniformly with the improvement of IO throughput, as depicted in the time breakdown in Figure~\ref{fig:sort-perf}(e), where they consume a comparable amount of time. 
Notably, \THISWORK\ outperforms Triton Join without using proprietary CPU-GPU interconnects by exploiting untapped PCIe bandwidth.
% Notably, while surpassing the performance of Triton Join, our solution relies solely on commodity PCIe links, without utilizing any proprietary CPU-GPU connections to enhance IO throughput.


%%%%%%%%%%%% OLD TEXT START %%%%%%%%%%%%
\begin{comment}
Unlike sort, hash join is still an IO-bound kernel even with our IO-redistribution technique.
As shown in Figure~\ref{fig:hash-join-perf}(a), our solution achieves 2.3 billion tuples per second throughput.
This is around 24.1$\times$ over DuckDB, 2.4$\times$ over the CPU implementation of Triton Join, 1.3$\times$ over the CPU-GPU-NVlink based GPU Triton Join~\cite{triton-join}, and 3.2$\times$ over the single GPU solution with a common PCIe link.
The speedup over the single GPU IO solution is more significant because all hash join phases are IO-bound.
This can be observed from Figure~\ref{fig:hash-join-perf}(c).
The \texttt{HashJoinExOp} takes only ~34ms to finish the on-GPU join kernel, which is much lower than the ~61ms data transfer time.
Similarly, it only takes ~90ms to partition a chunk of data transferred in ~113ms.
All phases scale uniformly with the improvement of IO throughput, thus the time breakdown in Figure~\ref{fig:hash-join-perf}(b) shows that they take a similar amount of time.
While achieving better results than Triton Join, we do not use any proprietary CPU-GPU links to improve the IO through, but solely based on commodity PCIe links.
\end{comment}
%%%%%%%%%%%% OLD TEXT END %%%%%%%%%%%%

\noindent
\textbf{Performance of SSB queries.}
Figure~\ref{fig:ssb-perf} illustrates the comparison of SSB query performance between our solution and the baseline approaches.
On average, our solution achieves a 3.4$\times$ speedup over DuckDB, with all data dynamically fetched from CPU DRAM.
When examining individual query flights, the speedup is 2.4$\times$ for Q1.*, 3.6$\times$ for Q2.*, 3.9$\times$ for Q3.*, and 3.7$\times$ for Q4.*. 
The higher speedup observed in Q2.*, Q3.*, and Q4.* is attributed to their inclusion of more complex multi-way joins.
The more complex multi-way join demands higher memory throughput for hash table probing, thus favoring GPU-based solutions more as they can operate in high-bandwidth GPU memory.
The CPU-based solution has to use the limited DRAM bandwidth on hash table probing and fact table reading, while our solution only uses DRAM bandwidth for the latter.
Lightweight queries like Q11 only filter the fact table based on some predicates, whose only DRAM traffic is reading the fact table once.
Thus, the benefit of high-bandwidth GPU memory is minimized, and we observe less speedup. 


By comparing the bars of \texttt{navie} and \texttt{Proteus-GPU} with \texttt{DuckDB}, it becomes evident that GPU-based solutions struggle to achieve performance comparable to the CPU-based DuckDB without utilizing our IO optimization technique. 
However, this technique alone is insufficient, as indicated by the comparison between the \THISWORK\ and \texttt{DuckDB} bars. 
It only achieves a 1.6$\times$ speedup against \texttt{DuckDB} because it transfers unused data to the GPU without considering column selectivity. 
While zero-copy can exploit selectivity, it falls short of maximizing throughput because it relies on a single PCIe link. 
Notably, using zero-copy alone results in worse performance than \THISWORK\ .
Our final solution dynamically switches between SDMA-based data transfer for columns with selectivity greater than a threshold \(TH\) and zero-copy data transfer for columns with selectivity less than \(TH\).
Our solution also achieves 5.7$\times$ speedup over \texttt{Proteus-Hybrid}, despite that it uses both CPU and GPU.
It is difficult for such a hybrid solution to divide work between CPU and GPU and efficiently utilize the CPU DRAM bandwidth.
Our solution achieves 6.2$\times$ speedup over \texttt{Proteus-Lazy}, which enhances \texttt{Proteus-GPU} with late materialization techniques.
After we resolve the IO bottleneck and fully utilize CPU-side DRAM, a pure GPU-based solution can achieve highly competitive results.


%%%%%%%%%%%% OLD TEXT START %%%%%%%%%%%%
\begin{comment}
Figure~\ref{fig:ssb-perf} shows the comparison of SSB query performance between our solution and the baselines.
On average, our solution achieves 3.4$\times$ speedup over DuckDB, with all the data fetched from CPU DRAM on the fly.
Broken down into each query flight, the speedup is 2.4$\times$ for Q1.*, 3.6$\times$ for Q2.*, 3.9$\times$ for Q3.* and 3.7$\times$ for Q4.*.
More speedup is observed in Q2.*, Q3.*, and Q4.* because they include more complicated multi-way joins.

By comparing the bars of \texttt{navie} and \texttt{Proteus-GPU} with \texttt{DuckDB}, note that GPU-based solutions fail to achieve comparable performance to CPU-based DuckDB without using our IO redistribution technique.
However, this technique only is not enough, as we can see by comparing the bars of \texttt{GPU-IO} with \texttt{DuckDB}. 
It only achieves a 1.6$\times$ speedup against DuckDB, as it transfers unused data to GPU ignoring columns' selectivity.
While we can use zero-copy to exploit selectivity, it fails short in maximum throughput because it only uses one PCIe link.
We can notice that using zero-copy alone only delivers worse performance than \texttt{GPU-IO}.
Our final solution switches between SDMA-based data transfer for columns with selectivity larger than a threshold $TH$ and zero-copy data transfer for columns with selectivity lower than $TH$.
\end{comment}
%%%%%%%%%%%% OLD TEXT END %%%%%%%%%%%%

% \begin{figure}
%     \centering
%     \includegraphics[width=\linewidth]{figures/zero-copy-vs-gpu-io.pdf}
%     \caption{Zero copy vs GPU IO}
%     \label{fig:selectivity-perf}
% \end{figure}

% In our study, we set the threshold \(TH = 64\) based on the formula outlined in \S\ref{sec:design-ssb}. 
% To ensure the accuracy and effectiveness of this threshold, we developed the following micro-benchmark specifically designed for validation purposes.
% \begin{verbatim}
% for i in range(16e9)
%   sum += pred[i % 2e9] % SEL == 0 ? v[i] : 0
% \end{verbatim}
% The \texttt{pred} array resides in GPU memory, and \texttt{SEL} is a hyperparameter that is inversely related to selectivity. 
% We implement this micro-benchmark using both GPU-IO and zero-copy data transfer techniques, varying \texttt{SEL} from 1 to 128. 
% The results are presented in Figure~\ref{fig:selectivity-perf}. 
% Notably, when \texttt{SEL} \(> 64\), zero-copy becomes more efficient. 
% This aligns with the threshold \(TH < \frac{1}{64}\), corroborating the results derived from our formula.

%%%%%%%%%%%% OLD TEXT START %%%%%%%%%%%%
\begin{comment}
The \texttt{pred} array on GPU memory, and \texttt{SEL} is a hyperparameter that is inverse to the selectivity.
We implement this micro-benchmark using both GPU-IO and zero-copy data transfer and varies \texttt{SEL} from 1 to 128.
The result is presented in Figure~\ref{fig:selectivity-perf}.
We notice when \texttt{SEL} $>64$ zero-copy becomes more efficient. 
This corresponds to $TH < \frac{1}{64}$ and matches the result from our formula.
\end{comment}
%%%%%%%%%%%% OLD TEXT END %%%%%%%%%%%%

\subsection{Interference Analysis}
\label{sec:interference}
\noindent
While \THISWORK\ utilizes additional GPUs and their IO resources to forward data to a target GPU, running AI workloads on these auxiliary GPUs can lead to a slowdown of these workloads.
Figure~\ref{fig:interference}(a) presents the slowdown for the AI applications (x-axis) when the IO traffic (y-axis) runs in the background, and (b) shows the slowdown for the \THISWORK\ applications (y-axis) when the deep learning applications (x-axis) run in the background.
(1) Compared to single-direction IO traffic, bidirectional IO traffic has a more significant impact on the performance of foreground applications. This is likely due to the increased stress placed on the memory subsystems of the forwarding GPUs.
(2) Memory-intensive workloads are more susceptible to interference from data forwarding activities, as their performance is constrained by the memory bandwidth available on the GPUs. 
Background data forwarding consumes a portion of the memory bandwidth, leading to an average slowdown of 6.8\%.
Compared to SD3, text embedding generation, and LLM prefilling, LLM decoding experiences a greater degree of slowdown.

Figure~\ref{fig:interference} illustrates that current hardware may not optimize for our IO optimization techniques due to two key observations.
First, although the memory subsystem is theoretically stressed to the same degree in both scenarios, forwarding IO traffic from the device to the host results in a more significant slowdown compared to traffic from the host to the device.
Second, to support the 140GB/s IO throughput we achieved, each GPU incurs an additional memory bandwidth cost of $\frac{140 \times 2}{4} = 70$GB/s, which constitutes only $\frac{70}{1200} \approx 5.8\%$ of the MI100's total bandwidth.
However, empirical observations reveal slowdowns of 7.2\%, 13.4\%, and 16.9\% for \texttt{SD3}, \texttt{Llama3} decoding with a batch size of 32, and \texttt{Llama3} decoding with a batch size of 1, respectively.
We hypothesize that this discrepancy arises because our programming model generates atypical memory traffic that hinders the GPU memory controller's ability to fully utilize bandwidth for the foreground application.

We analyze the slowdown of data analytics applications caused by DL applications on forwarding GPUs. 
As shown in Figure~\ref{fig:interference}, the target GPU experiences less slowdown, with a maximum of 10.4\%. 
However, the slowdown patterns are more irregular compared to forwarding GPUs. 
Text embedding generation and \texttt{Llama3} prefilling cause more interference than \texttt{SD3}, despite all being compute-bound workloads. 
Interestingly, the memory-bound \texttt{Llama3} decoding shows less interference on the target GPU, contrasting with the significant interference on the forwarding GPUs.

%%%%%%%%%%%% OLD TEXT START %%%%%%%%%%%%
\begin{comment}
Besides, Figure~\ref{fig:interference} also shows current hardware may not be able to handle our novel use cases efficiently.
(1) While stressing the memory subsystem to the same degree theoretically, the forwarding IO traffic from device to host causes a greater slowdown than from host to device.
(2) To support the ~140GB/s IO throughput we achieved, each GPU only needs to pay $\frac{140 \times 2}{4} = 70$ GB/s additional memory bandwidth, which is only $\frac{70}{1200} \approx 5.8\%$ of MI100's total bandwidth.
However, we observe 7.2\%, 13.4\%, and 16.9\% slowdown for \texttt{SD3}, \texttt{Llama3} decoding with batch size 32, and \texttt{Llama3} decoding with batch size 1.
We speculate that this is because our new way of programming generates uncommon memory traffic to the GPU memory controller, and prevents it from fully utilizing the maximum memory bandwidth.

Next, we also analyze the slowdown of data analytics applications influenced by DL applications running on the forwarding GPUs.
Less slowdown is observed on the target GPU as shown in Figure~\ref{fig:interference}, where the maximum slowdown is 10.4\%.
However, the slowdown numbers become more irregular compared to the case of forwarding GPUs.
We observe that text embedding generation and \texttt{Llama3} prefilling cause more interference than \texttt{SD3}, although all of them are compute-bound workloads.
Surprisingly, memory-bound \texttt{Llama3} decoding shows less interference on the target GPU, in contrast to the high degree of interference on the forwarding GPUs.
\end{comment}
%%%%%%%%%%%% OLD TEXT END %%%%%%%%%%%%

% \noindent
% \textbf{How are \THISWORK\ applications influenced by the DL applications on the forwarding GPUs?}


\noindent
\textbf{Overall system efficiency.}
Given that our technique can accelerate heavily IO-bound applications by 3 to 4 times, we argue that the system is still more efficient even with a slowdown of up to 16.9\% on the other GPUs.
The improvement of overall system efficiency in a 4-GPU system can be quantified as shown below.
% We discuss how our technique enhances the overall efficiency of a 4-GPU system, as quantified by the following formula.
% Given that our technique can speed up the heavily IO-bounded applications by 3~4$\times$, up to 16.9\% slowdown on the other three GPUs is acceptable. 
% We discuss how much our technique improves the 4-GPU system's efficiency as a whole.
% The improvement of the whole 4-GPU system's efficiency is given by the following formula
$$
\text{speedup}_\text{sys} = \frac{\text{speedup}_\text{t} * \text{slowdown}_\text{t} + 3 * \text{slowdown}_\text{f}}{4}
$$
The subscripts `t' and `f' denote the target GPU and forwarding GPUs, respectively. 
Consider the scenario where \texttt{SD3} and hash join, both with primarily bidirectional IO traffic, are collocated.
The overall system speedup is $\frac{3.2 * (1 - 0.051) + 3 * (1 - 0.072)}{4} \approx 1.45$.
In our setup, the least favorable combination is running \texttt{Llama3} decoding without batching alongside sort. 
Despite this, we still achieve a modest speedup of$\frac{1.7 * (1 - 0.032) + 3 * (1 - 0.169)}{4} \approx 1.03$ speedup.
Note that these speedup values refer to the entire 4-GPU system. 
For a single GPU, they correspond to speedups of 2.8$\times$ and 1.12$\times$, respectively.

%%%%%%%%%%%% OLD TEXT START %%%%%%%%%%%%
\begin{comment}
where the subscript ``t'' and ``f'' mean the target GPU and the forwarding GPUs.
Consider the case of collocating \texttt{SD3} and hash join, whose IO traffic is mainly bidirectional.
The whose system speedup is $\frac{3.2 * (1 - 0.051) + 3 * (1 - 0.072)}{4} \approx 1.45$.
In our setup, the worst combination is running \texttt{Llama3} decoding without batching with sort, but we still achieve a minor $\frac{1.7 * (1 - 0.032) + 3 * (1 - 0.169)}{4} \approx 1.03$ speedup.
Note that the speedup here is in terms of all 4 GPUs, and the speedup above translates to 2.8$\times$ and 1.12$\times$ in terms of a single GPU.
\end{comment}
%%%%%%%%%%%% OLD TEXT END %%%%%%%%%%%%

\section{CONCLUSIONS}
\label{sec:conclusions}

This paper proposed RT-FMT, which is a real-time path planning algorithm that is capable of quickly generating low-cost paths in environments with dynamic obstacle avoidance with unknown trajectories. The real-time capability of the algorithm comes from the fact that it generates local paths for the robot to follow while the global path is not available. All these characteristics indicate that RT-FMT is ideal for time-critical applications. To show the capabilities of RT-FMT, we compare it against RT-RRT* on simulated environments with dynamic obstacles. The results show that RT-FMT has higher success rates, lower traveled distances, and smaller arrival times in almost all cases. The greater performance of RT-FMT over RT-RRT* is mostly related to the fact that, in cluttered spaces, the probability of randomly connecting a node to the tree is very low for RT-RRT*, which requires many attempts to grow the tree. On the other hand, RT-FMT starts the search with all nodes already laid out in the environment, which reduces the number of attempts to expand the tree.



%combines the real-time capabilities of RT-RRT* with the advantages of FMT* over RRT* which are quicker planning times and higher success rates. Our approach, RT-FMT, is capable of quickly generating low-cost paths, and it also features dynamic obstacle avoidance with unknown trajectories and multiple-query planning. %This is possible by applying two deterministic rewiring processes that removes branches of the tree that are blocked by any dynamic obstacle and by updating the tree root whenever the robot moves. 
%The real-time capability of the algorithm comes from the fact that it generates local paths for the robot to follow while the global path is not available. All these characteristics indicate that RT-FMT is ideal for time-critical applications.


Future work on the proposed algorithm will involve an implementation in the Open Motion Planning Library (OMPL) \cite{sucan2012the-open-motion-planning-library} to facilitate comparisons with other state-of-the-art approaches, mainly on higher dimensional problems, and to understand the effects of the real-time variation on the complexity of the algorithm and its computing time. The OMPL implementation might also simplify tests on real robots. In addition, a variation of the approach that works for constrained robots can also be implemented.

%Future work on the proposed algorithm will involve an implementation in the Open Motion Planning Library (OMPL) \cite{sucan2012the-open-motion-planning-library} to facilitate bench-marking RT-FMT with other more recent approaches such as RRT*FN-Replan \cite{tong2019rrt}, mainly on higher dimensional problems, which is an important advantage of probabilistic sampling-based planners. It will also facilitate comparisons with other state-of-the-art approaches, mainly to understand the effects of the real-time variation on the complexity of the algorithm and its computing time. The OMPL implementation might also simplify tests on real robots. In addition, a variation of the approach that works for constrained robots can also be implemented.








%%%%%%%%%%%%%%%%%%%%%%%%%%%%%%%%%%%%%%%%%%%%%%%%%%%%%%%%%%%%%%%%%%%%%%%%%%%%%%%%

\addtolength{\textheight}{-15cm}   % This command serves to balance the column lengths
                                  % on the last page of the document manually. It shortens
                                  % the textheight of the last page by a suitable amount.
                                  % This command does not take effect until the next page
                                  % so it should come on the page before the last. Make
                                  % sure that you do not shorten the textheight too much.

%%%%%%%%%%%%%%%%%%%%%%%%%%%%%%%%%%%%%%%%%%%%%%%%%%%%%%%%%%%%%%%%%%%%%%%%%%%%%%%%



%%%%%%%%%%%%%%%%%%%%%%%%%%%%%%%%%%%%%%%%%%%%%%%%%%%%%%%%%%%%%%%%%%%%%%%%%%%%%%%%
% Generated by IEEEtran.bst, version: 1.14 (2015/08/26)
\begin{thebibliography}{10}
\providecommand{\url}[1]{#1}
\csname url@samestyle\endcsname
\providecommand{\newblock}{\relax}
\providecommand{\bibinfo}[2]{#2}
\providecommand{\BIBentrySTDinterwordspacing}{\spaceskip=0pt\relax}
\providecommand{\BIBentryALTinterwordstretchfactor}{4}
\providecommand{\BIBentryALTinterwordspacing}{\spaceskip=\fontdimen2\font plus
\BIBentryALTinterwordstretchfactor\fontdimen3\font minus
  \fontdimen4\font\relax}
\providecommand{\BIBforeignlanguage}[2]{{%
\expandafter\ifx\csname l@#1\endcsname\relax
\typeout{** WARNING: IEEEtran.bst: No hyphenation pattern has been}%
\typeout{** loaded for the language `#1'. Using the pattern for}%
\typeout{** the default language instead.}%
\else
\language=\csname l@#1\endcsname
\fi
#2}}
\providecommand{\BIBdecl}{\relax}
\BIBdecl

\bibitem{tian2021trajectory}
F.~Tian, R.~Zhou, Z.~Li, L.~Li, Y.~Gao, D.~Cao, and L.~Chen, ``Trajectory
  planning for autonomous mining trucks considering terrain constraints,''
  \emph{IEEE Transactions on Intelligent Vehicles}, vol.~6, no.~4, pp.
  772--786, 2021.

\bibitem{hayat2020multi}
S.~Hayat, E.~Yanmaz, C.~Bettstetter, and T.~X. Brown, ``Multi-objective drone
  path planning for search and rescue with quality-of-service requirements,''
  \emph{Autonomous Robots}, vol.~44, no.~7, pp. 1183--1198, 2020.

\bibitem{karaman2011anytime}
S.~Karaman, M.~R. Walter, A.~Perez, E.~Frazzoli, and S.~Teller, ``Anytime
  motion planning using the rrt,'' in \emph{Proceedings of the 2011 IEEE
  International Conference on Robotics and Automation (ICRA)}.\hskip 1em plus
  0.5em minus 0.4em\relax IEEE, 2011, pp. 1478--1483.

\bibitem{xu2020informed}
J.~Xu, K.~Song, D.~Zhang, H.~Dong, Y.~Yan, and Q.~Meng, ``Informed anytime fast
  marching tree for asymptotically optimal motion planning,'' \emph{IEEE
  Transactions on Industrial Electronics}, vol.~68, no.~6, pp. 5068--5077,
  2020.

\bibitem{naderi2015rt}
K.~Naderi, J.~Rajam{\"a}ki, and P.~H{\"a}m{\"a}l{\"a}inen, ``{RT-RRT*:} {a}
  real-time path planning algorithm based on {RRT},'' in \emph{Proceedings of
  the 8th ACM SIGGRAPH Conference on Motion in Games}, 2015, pp. 113--118.

\bibitem{cserna2016anytime}
B.~Cserna, M.~Bogochow, S.~Chambers, M.~Tremblay, S.~Katt, and W.~Ruml,
  ``Anytime versus real-time heuristic search for on-line planning,'' in
  \emph{Proceedings of the International Symposium on Combinatorial Search},
  vol.~7, no.~1, 2016.

\bibitem{karaman2011sampling}
S.~Karaman and E.~Frazzoli, ``Sampling-based algorithms for optimal motion
  planning,'' \emph{The International Journal of Robotics Research}, vol.~30,
  no.~7, pp. 846--894, 2011.

\bibitem{fmt}
L.~Janson, E.~Schmerling, A.~Clark, and M.~Pavone, ``Fast marching tree: A fast
  marching sampling-based method for optimal motion planning in many
  dimensions,'' \emph{The International Journal of Robotics Research}, vol.~34,
  no.~7, pp. 883--921, 2015.

\bibitem{grothe2022st}
F.~Grothe, V.~N. Hartmann, A.~Orthey, and M.~Toussaint, ``St-rrt*:
  Asymptotically-optimal bidirectional motion planning through space-time,'' in
  \emph{Proceedings of the 2022 International Conference on Robotics and
  Automation (ICRA)}, 2022, pp. 3314--3320.

\bibitem{luders2010bounds}
B.~D. Luders, S.~Karaman, E.~Frazzoli, and J.~P. How, ``Bounds on tracking
  error using closed-loop rapidly-exploring random trees,'' in
  \emph{Proceedings of the 2010 American Control Conference}.\hskip 1em plus
  0.5em minus 0.4em\relax IEEE, 2010, pp. 5406--5412.

\bibitem{chandler2017online}
B.~Chandler and M.~A. Goodrich, ``Online rrt and online fmt: Rapid replanning
  with dynamic cost,'' in \emph{Proceedings of the 2017 IEEE/RSJ International
  Conference on Intelligent Robots and Systems (IROS)}, 2017, pp. 6313--6318.

\bibitem{tong2019rrt}
B.~Tong, Q.~Liu, and C.~Dai, ``A rrt* fn based path replanning algorithm,'' in
  \emph{Proceedings of the 2019 IEEE 4th Advanced Information Technology,
  Electronic and Automation Control Conference (IAEAC)}, vol.~1.\hskip 1em plus
  0.5em minus 0.4em\relax IEEE, 2019, pp. 1435--1445.

\bibitem{sucan2012the-open-motion-planning-library}
I.~A. Sucan, M.~Moll, and L.~E. Kavraki, ``The {O}pen {M}otion {P}lanning
  {L}ibrary,'' \emph{{IEEE} Robotics \& Automation Magazine}, vol.~19, no.~4,
  pp. 72--82, December 2012.

\end{thebibliography}

\end{document}
