%%%%%%%%%%%%%%%%%%%%%%%%%%%%%%%%%%%%%%%%%%%%%%%%%%%%%%%%%%%%%%%%%%%%%%%%%%%%%%%%
%2345678901234567890123456789012345678901234567890123456789012345678901234567890
%        1         2         3         4         5         6         7         8

\documentclass[letterpaper, 10 pt, conference]{ieeeconf}  % Comment this line out if you need a4paper

%\documentclass[a4paper, 10pt, conference]{ieeeconf}      % Use this line for a4 paper

\IEEEoverridecommandlockouts                              % This command is only needed if 
                                                          % you want to use the \thanks command

\overrideIEEEmargins                                      % Needed to meet printer requirements.

%In case you encounter the following error:
%Error 1010 The PDF file may be corrupt (unable to open PDF file) OR
%Error 1000 An error occurred while parsing a contents stream. Unable to analyze the PDF file.
%This is a known problem with pdfLaTeX conversion filter. The file cannot be opened with acrobat reader
%Please use one of the alternatives below to circumvent this error by uncommenting one or the other
%\pdfobjcompresslevel=0
%\pdfminorversion=4

% See the \addtolength command later in the file to balance the column lengths
% on the last page of the document

% The following packages can be found on http:\\www.ctan.org
%\usepackage{graphics} % for pdf, bitmapped graphics files
%\usepackage{epsfig} % for postscript graphics files
%\usepackage{mathptmx} % assumes new font selection scheme installed
%\usepackage{times} % assumes new font selection scheme installed
%\usepackage{amsmath} % assumes amsmath package installed
%\usepackage{amssymb}  % assumes amsmath package installed
\usepackage{graphicx}
\usepackage{cite}
\usepackage{amsmath,amssymb,amsfonts}
\usepackage{algorithmic}
\usepackage{graphicx}
\usepackage{textcomp}
\usepackage{xcolor}
\usepackage[ruled,vlined,linesnumbered]{algorithm2e}
\usepackage{bm}
\usepackage{subfig}
\usepackage{url}
\newcommand\figsize{0.46}

\title{\LARGE \bf Real-Time Fast Marching Tree for Mobile Robot Motion Planning in Dynamic Environments*
\thanks{*This research was supported in part by the Natural Science and Engineering Research Council of Canada (NSERC) under grant number DNDPJ 533392-18, General Dynamics Land Systems (Canada), and Defence R\&D Canada (DRDC).}
}

\author{Jefferson~Silveira$^{1}$\thanks{$^{1}$ J.\ Silveira is with the Department of Electrical \& Computer Engineering and the Ingenuity Labs Research Institute, Queen's University, Kingston, ON K7L 3N6 Canada \texttt{jefferson.silveira@queensu.ca}}, %~\IEEEmembership{Member,~IEEE,}
        Kleber~Cabral$^{2}$\thanks{$^{2}$K.\ Cabral is with the School of Computing, Queen's University, Kingston, ON K7L 3N6 Canada \texttt{kleber.cabral@queensu.ca}}, %~\IEEEmembership{Fellow,~OSA,}
        Sidney~Givigi$^{3}$\thanks{$^{3}$S.\ Givigi is with the School of Computing and the Ingenuity Labs Research Institute, Queen's University, Kingston, ON K7L 3N6 Canada \texttt{sidney.givigi@queensu.ca}}
        and~Joshua~A.~Marshall$^{4}$\thanks{$^{4}$J.\ Marshall is with the Department of Electrical \& Computer Engineering and the Ingenuity Labs Research Institute, Queen's University, Kingston, ON K7L 3N6 Canada \texttt{joshua.marshall@queensu.ca}}%,~\IEEEmembership{Senior Member,~IEEE}% <-this % stops a space
        }
        
% \author{Albert Author$^{1}$ and Bernard D. Researcher$^{2}$% <-this % stops a space
% \thanks{*This work was not supported by any organization}% <-this % stops a space
% \thanks{$^{1}$Albert Author is with Faculty of Electrical Engineering, Mathematics and Computer Science,
%         University of Twente, 7500 AE Enschede, The Netherlands
%         {\tt\small albert.author@papercept.net}}%
% \thanks{$^{2}$Bernard D. Researcheris with the Department of Electrical Engineering, Wright State University,
%         Dayton, OH 45435, USA
%         {\tt\small b.d.researcher@ieee.org}}%
% }

\urlstyle{same}
\begin{document}



\maketitle
\thispagestyle{empty}
\pagestyle{empty}


%%%%%%%%%%%%%%%%%%%%%%%%%%%%%%%%%%%%%%%%%%%%%%%%%%%%%%%%%%%%%%%%%%%%%%%%%%%%%%%%
\begin{abstract}

This paper proposes the Real-Time Fast Marching Tree (RT-FMT), a real-time planning algorithm that features local and global path generation, multiple-query planning, and dynamic obstacle avoidance. During the search, RT-FMT quickly looks for the global solution and, in the meantime, generates local paths that can be used by the robot to start execution faster. In addition, our algorithm constantly rewires the tree to keep branches from forming inside the dynamic obstacles and to maintain the tree root near the robot, which allows the tree to be reused multiple times for different goals. Our algorithm is based on the planners Fast Marching Tree (FMT*) and Real-time Rapidly-Exploring Random Tree (RT-RRT*). We show via simulations that RT-FMT outperforms RT-RRT* in both execution cost and arrival time, in most cases. Moreover, we also demonstrate via simulation that it is worthwhile taking the local path before the global path is available in order to reduce arrival time, even though there is a small possibility of taking an inferior path. 

\end{abstract}


%%%%%%%%%%%%%%%%%%%%%%%%%%%%%%%%%%%%%%%%%%%%%%%%%%%%%%%%%%%%%%%%%%%%%%%%%%%%%%%%



\section{Introduction}

Video generation has garnered significant attention owing to its transformative potential across a wide range of applications, such media content creation~\citep{polyak2024movie}, advertising~\citep{zhang2024virbo,bacher2021advert}, video games~\citep{yang2024playable,valevski2024diffusion, oasis2024}, and world model simulators~\citep{ha2018world, videoworldsimulators2024, agarwal2025cosmos}. Benefiting from advanced generative algorithms~\citep{goodfellow2014generative, ho2020denoising, liu2023flow, lipman2023flow}, scalable model architectures~\citep{vaswani2017attention, peebles2023scalable}, vast amounts of internet-sourced data~\citep{chen2024panda, nan2024openvid, ju2024miradata}, and ongoing expansion of computing capabilities~\citep{nvidia2022h100, nvidia2023dgxgh200, nvidia2024h200nvl}, remarkable advancements have been achieved in the field of video generation~\citep{ho2022video, ho2022imagen, singer2023makeavideo, blattmann2023align, videoworldsimulators2024, kuaishou2024klingai, yang2024cogvideox, jin2024pyramidal, polyak2024movie, kong2024hunyuanvideo, ji2024prompt}.


In this work, we present \textbf{\ours}, a family of rectified flow~\citep{lipman2023flow, liu2023flow} transformer models designed for joint image and video generation, establishing a pathway toward industry-grade performance. This report centers on four key components: data curation, model architecture design, flow formulation, and training infrastructure optimization—each rigorously refined to meet the demands of high-quality, large-scale video generation.


\begin{figure}[ht]
    \centering
    \begin{subfigure}[b]{0.82\linewidth}
        \centering
        \includegraphics[width=\linewidth]{figures/t2i_1024.pdf}
        \caption{Text-to-Image Samples}\label{fig:main-demo-t2i}
    \end{subfigure}
    \vfill
    \begin{subfigure}[b]{0.82\linewidth}
        \centering
        \includegraphics[width=\linewidth]{figures/t2v_samples.pdf}
        \caption{Text-to-Video Samples}\label{fig:main-demo-t2v}
    \end{subfigure}
\caption{\textbf{Generated samples from \ours.} Key components are highlighted in \textcolor{red}{\textbf{RED}}.}\label{fig:main-demo}
\end{figure}


First, we present a comprehensive data processing pipeline designed to construct large-scale, high-quality image and video-text datasets. The pipeline integrates multiple advanced techniques, including video and image filtering based on aesthetic scores, OCR-driven content analysis, and subjective evaluations, to ensure exceptional visual and contextual quality. Furthermore, we employ multimodal large language models~(MLLMs)~\citep{yuan2025tarsier2} to generate dense and contextually aligned captions, which are subsequently refined using an additional large language model~(LLM)~\citep{yang2024qwen2} to enhance their accuracy, fluency, and descriptive richness. As a result, we have curated a robust training dataset comprising approximately 36M video-text pairs and 160M image-text pairs, which are proven sufficient for training industry-level generative models.

Secondly, we take a pioneering step by applying rectified flow formulation~\citep{lipman2023flow} for joint image and video generation, implemented through the \ours model family, which comprises Transformer architectures with 2B and 8B parameters. At its core, the \ours framework employs a 3D joint image-video variational autoencoder (VAE) to compress image and video inputs into a shared latent space, facilitating unified representation. This shared latent space is coupled with a full-attention~\citep{vaswani2017attention} mechanism, enabling seamless joint training of image and video. This architecture delivers high-quality, coherent outputs across both images and videos, establishing a unified framework for visual generation tasks.


Furthermore, to support the training of \ours at scale, we have developed a robust infrastructure tailored for large-scale model training. Our approach incorporates advanced parallelism strategies~\citep{jacobs2023deepspeed, pytorch_fsdp} to manage memory efficiently during long-context training. Additionally, we employ ByteCheckpoint~\citep{wan2024bytecheckpoint} for high-performance checkpointing and integrate fault-tolerant mechanisms from MegaScale~\citep{jiang2024megascale} to ensure stability and scalability across large GPU clusters. These optimizations enable \ours to handle the computational and data challenges of generative modeling with exceptional efficiency and reliability.


We evaluate \ours on both text-to-image and text-to-video benchmarks to highlight its competitive advantages. For text-to-image generation, \ours-T2I demonstrates strong performance across multiple benchmarks, including T2I-CompBench~\citep{huang2023t2i-compbench}, GenEval~\citep{ghosh2024geneval}, and DPG-Bench~\citep{hu2024ella_dbgbench}, excelling in both visual quality and text-image alignment. In text-to-video benchmarks, \ours-T2V achieves state-of-the-art performance on the UCF-101~\citep{ucf101} zero-shot generation task. Additionally, \ours-T2V attains an impressive score of \textbf{84.85} on VBench~\citep{huang2024vbench}, securing the top position on the leaderboard (as of 2025-01-25) and surpassing several leading commercial text-to-video models. Qualitative results, illustrated in \Cref{fig:main-demo}, further demonstrate the superior quality of the generated media samples. These findings underscore \ours's effectiveness in multi-modal generation and its potential as a high-performing solution for both research and commercial applications.

\section{Related Work}
\label{sec:related-works}
\subsection{Novel View Synthesis}
Novel view synthesis is a foundational task in the computer vision and graphics, which aims to generate unseen views of a scene from a given set of images.
% Many methods have been designed to solve this problem by posing it as 3D geometry based rendering, where point clouds~\cite{point_differentiable,point_nfs}, mesh~\cite{worldsheet,FVS,SVS}, planes~\cite{automatci_photo_pop_up,tour_into_the_picture} and multi-plane images~\cite{MINE,single_view_mpi,stereo_magnification}, \etal
Numerous methods have been developed to address this problem by approaching it as 3D geometry-based rendering, such as using meshes~\cite{worldsheet,FVS,SVS}, MPI~\cite{MINE,single_view_mpi,stereo_magnification}, point clouds~\cite{point_differentiable,point_nfs}, etc.
% planes~\cite{automatci_photo_pop_up,tour_into_the_picture}, 


\begin{figure*}[!t]
    \centering
    \includegraphics[width=1.0\linewidth]{figures/overview-v7.png}
    %\caption{\textbf{Overview.} Given a set of images, our method obtains both camera intrinsics and extrinsics, as well as a 3DGS model. First, we obtain the initial camera parameters, global track points from image correspondences and monodepth with reprojection loss. Then we incorporate the global track information and select Gaussian kernels associated with track points. We jointly optimize the parameters $K$, $T_{cw}$, 3DGS through multi-view geometric consistency $L_{t2d}$, $L_{t3d}$, $L_{scale}$ and photometric consistency $L_1$, $L_{D-SSIM}$.}
    \caption{\textbf{Overview.} Given a set of images, our method obtains both camera intrinsics and extrinsics, as well as a 3DGS model. During the initialization, we extract the global tracks, and initialize camera parameters and Gaussians from image correspondences and monodepth with reprojection loss. We determine Gaussian kernels with recovered 3D track points, and then jointly optimize the parameters $K$, $T_{cw}$, 3DGS through the proposed global track constraints (i.e., $L_{t2d}$, $L_{t3d}$, and $L_{scale}$) and original photometric losses (i.e., $L_1$ and $L_{D-SSIM}$).}
    \label{fig:overview}
\end{figure*}

Recently, Neural Radiance Fields (NeRF)~\cite{2020NeRF} provide a novel solution to this problem by representing scenes as implicit radiance fields using neural networks, achieving photo-realistic rendering quality. Although having some works in improving efficiency~\cite{instant_nerf2022, lin2022enerf}, the time-consuming training and rendering still limit its practicality.
Alternatively, 3D Gaussian Splatting (3DGS)~\cite{3DGS2023} models the scene as explicit Gaussian kernels, with differentiable splatting for rendering. Its improved real-time rendering performance, lower storage and efficiency, quickly attract more attentions.
% Different from NeRF-based methods which need MLPs to model the scene and huge computational cost for rendering, 3DGS has stronger real-time performance, higher storage and computational efficiency, benefits from its explicit representation and gradient backpropagation.

\subsection{Optimizing Camera Poses in NeRFs and 3DGS}
Although NeRF and 3DGS can provide impressive scene representation, these methods all need accurate camera parameters (both intrinsic and extrinsic) as additional inputs, which are mostly obtained by COLMAP~\cite{colmap2016}.
% This strong reliance on COLMAP significantly limits their use in real-world applications, so optimizing the camera parameters during the scene training becomes crucial.
When the prior is inaccurate or unknown, accurately estimating camera parameters and scene representations becomes crucial.

% In early works, only photometric constraints are used for scene training and camera pose estimation. 
% iNeRF~\cite{iNerf2021} optimizes the camera poses based on a pre-trained NeRF model.
% NeRFmm~\cite{wang2021nerfmm} introduce a joint optimization process, which estimates the camera poses and trains NeRF model jointly.
% BARF~\cite{barf2021} and GARF~\cite{2022GARF} provide new positional encoding strategy to handle with the gradient inconsistency issue of positional embedding and yield promising results.
% However, they achieve satisfactory optimization results when only the pose initialization is quite closed to the ground-truth, as the photometric constrains can only improve the quality of camera estimation within a small range.
% Later, more prior information of geometry and correspondence, \ie monocular depth and feature matching, are introduced into joint optimisation to enhance the capability of camera poses estimation.
% SC-NeRF~\cite{SCNeRF2021} minimizes a projected ray distance loss based on correspondence of adjacent frames.
% NoPe-NeRF~\cite{bian2022nopenerf} chooses monocular depth maps as geometric priors, and defines undistorted depth loss and relative pose constraints for joint optimization.
In earlier studies, scene training and camera pose estimation relied solely on photometric constraints. iNeRF~\cite{iNerf2021} refines the camera poses using a pre-trained NeRF model. NeRFmm~\cite{wang2021nerfmm} introduces a joint optimization approach that simultaneously estimates camera poses and trains the NeRF model. BARF~\cite{barf2021} and GARF~\cite{2022GARF} propose a new positional encoding strategy to address the gradient inconsistency issues in positional embedding, achieving promising results. However, these methods only yield satisfactory optimization when the initial pose is very close to the ground truth, as photometric constraints alone can only enhance camera estimation quality within a limited range. Subsequently, 
% additional prior information on geometry and correspondence, such as monocular depth and feature matching, has been incorporated into joint optimization to improve the accuracy of camera pose estimation. 
SC-NeRF~\cite{SCNeRF2021} minimizes a projected ray distance loss based on correspondence between adjacent frames. NoPe-NeRF~\cite{bian2022nopenerf} utilizes monocular depth maps as geometric priors and defines undistorted depth loss and relative pose constraints.

% With regard to 3D Gaussian Splatting, CF-3DGS~\cite{CF-3DGS-2024} also leverages mono-depth information to constrain the optimization of local 3DGS for relative pose estimation and later learn a global 3DGS progressively in a sequential manner.
% InstantSplat~\cite{fan2024instantsplat} focus on sparse view scenes, first use DUSt3R~\cite{dust3r2024cvpr} to generate a set of densely covered and pixel-aligned points for 3D Gaussian initialization, then introduce a parallel grid partitioning strategy in joint optimization to speed up.
% % Jiang et al.~\cite{Jiang_2024sig} proposed to build the scene continuously and progressively, to next unregistered frame, they use registration and adjustment to adjust the previous registered camera poses and align unregistered monocular depths, later refine the joint model by matching detected correspondences in screen-space coordinates.
% \gjh{Jiang et al.~\cite{Jiang_2024sig} also implemented an incremental approach for reconstructing camera poses and scenes. Initially, they perform feature matching between the current image and the image rendered by a differentiable surface renderer. They then construct matching point errors, depth errors, and photometric errors to achieve the registration and adjustment of the current image. Finally, based on the depth map, the pixels of the current image are projected as new 3D Gaussians. However, this method still exhibits limitations when dealing with complex scenes and unordered images.}
% % CG-3DGS~\cite{sun2024correspondenceguidedsfmfree3dgaussian} follows CF-3DGS, first construct a coarse point cloud from mono-depth maps to train a 3DGS model, then progressively estimate camera poses based on this pre-trained model by constraining the correspondences between rendering view and ground-truth.
% \gjh{Similarly, CG-3DGS~\cite{sun2024correspondenceguidedsfmfree3dgaussian} first utilizes monocular depth estimation and the camera parameters from the first frame to initialize a set of 3D Gaussians. It then progressively estimates camera poses based on this pre-trained model by constraining the correspondences between the rendered views and the ground truth.}
% % Free-SurGS~\cite{freesurgs2024} matches the projection flow derived from 3D Gaussians with optical flow to estimate the poses, to compensate for the limitations of photometric loss.
% \gjh{Free-SurGS~\cite{freesurgs2024} introduces the first SfM-free 3DGS approach for surgical scene reconstruction. Due to the challenges posed by weak textures and photometric inconsistencies in surgical scenes, Free-SurGS achieves pose estimation by minimizing the flow loss between the projection flow and the optical flow. Subsequently, it keeps the camera pose fixed and optimizes the scene representation by minimizing the photometric loss, depth loss and flow loss.}
% \gjh{However, most current works assume camera intrinsics are known and primarily focus on optimizing camera poses. Additionally, these methods typically rely on sequentially ordered image inputs and incrementally optimize camera parameters and scene representation. This inevitably leads to drift errors, preventing the achievement of globally consistent results. Our work aims to address these issues.}

Regarding 3D Gaussian Splatting, CF-3DGS~\cite{CF-3DGS-2024} utilizes mono-depth information to refine the optimization of local 3DGS for relative pose estimation and subsequently learns a global 3DGS in a sequential manner. InstantSplat~\cite{fan2024instantsplat} targets sparse view scenes, initially employing DUSt3R~\cite{dust3r2024cvpr} to create a densely covered, pixel-aligned point set for initializing 3D Gaussian models, and then implements a parallel grid partitioning strategy to accelerate joint optimization. Jiang \etal~\cite{Jiang_2024sig} develops an incremental method for reconstructing camera poses and scenes, but it struggles with complex scenes and unordered images. 
% Similarly, CG-3DGS~\cite{sun2024correspondenceguidedsfmfree3dgaussian} progressively estimates camera poses using a pre-trained model by aligning the correspondences between rendered views and actual scenes. Free-SurGS~\cite{freesurgs2024} pioneers an SfM-free 3DGS method for reconstructing surgical scenes, overcoming challenges such as weak textures and photometric inconsistencies by minimizing the discrepancy between projection flow and optical flow.
%\pb{SF-3DGS-HT~\cite{ji2024sfmfree3dgaussiansplatting} introduced VFI into training as additional photometric constraints. They separated the whole scene into several local 3DGS models and then merged them hierarchically, which leads to a significant improvement on simple and dense view scenes.}
HT-3DGS~\cite{ji2024sfmfree3dgaussiansplatting} interpolates frames for training and splits the scene into local clips, using a hierarchical strategy to build 3DGS model. It works well for simple scenes, but fails with dramatic motions due to unstable interpolation and low efficiency.
% {While effective for simple scenes, it struggles with dramatic motion due to unstable view interpolation and suffers from low computational efficiency.}

However, most existing methods generally depend on sequentially ordered image inputs and incrementally optimize camera parameters and 3DGS, which often leads to drift errors and hinders achieving globally consistent results. Our work seeks to overcome these limitations.


%\section{Problem Formulation and Notations}

\section{THE REAL-TIME FAST MARCHING TREE ALGORITHM}
\label{sec:algorithm}
In general, RT-FMT works by expanding a tree, similarly to how FMT* does, while it also checks for dynamic obstacles and rewires the tree around them. During the search, the algorithm also searches for the local paths with the least costs. These paths are used for the robot to start moving before planning is finished. When the robot reaches a new waypoint in the path, the root of the tree is updated. This event triggers a complete rewire of the tree to update the costs of all nodes. In a real-time application, the robot sends velocity commands at specific intervals. Therefore, we only allow the tree expansion and rewiring to run for a defined number of iterations ($N_e$) to expand and rewire the tree. This method can also be easily adapted to run for a desired time interval or planning frequency. 

The method is shown in Algorithm \ref{alg:main}, which starts by sampling $N$ configurations in free space (Line \ref{alg:samplefree}) considering only the fixed obstacles. This function also calculates the neighborhood radius $r_n$ based on the number of samples and the dimensionality of the problem according to
\begin{equation}
\label{eq:rn}
    r_n = \gamma_s 2\bigg(1 +\frac{1}{d}\bigg)^{\frac{1}{d}}\bigg(\frac{\mu(\mathcal{X}_{\rm free})}{\zeta_d} \bigg)^\frac{1}{d} \bigg(\frac{\log(N)}{N} \bigg)^\frac{1}{d},
\end{equation}
where $\gamma_s > 1$ is a tuning parameter, $d$ is the dimension of the problem, $\mu(\mathcal{X}_{\rm free})$ is the Lebesgue measure of the free space, and $\zeta_d$ is the volume of a unit ball in $\mathbb{R}^d$. Although FMT* provides an equation for $r_n$, we use the equation defined in \cite{karaman2011sampling} for PRM* since it computes an $r_n$ slightly bigger than the FMT* equation for $r_n$.

\begin{algorithm}[tb]
\SetKwFunction{SampleFree}{SampleFree}
\SetKwFunction{UpdateContext}{UpdateContext}
\SetKwFunction{ExpandFMT}{ExpandFMT}
\SetKwFunction{RewireFromObstacles}{RewireFromObstacles}
\SetKwFunction{RewireFromRoot}{RewireFromRoot}
\SetKwFunction{GeneratePath}{GeneratePath}
\SetKwFunction{UpdateRoot}{UpdateRoot}

 $\mathcal{T} \leftarrow \bm x_{s}$;  $\bm x_{\rm root} \leftarrow \bm x_{s}$\;
 $\mathcal{S} \leftarrow \SampleFree(N) \cup \bm x_{s} \cup \bm x_{g}$\; \label{alg:samplefree}
 $\mathcal{V}_{b} \leftarrow \emptyset$; $\mathcal{Q}_{o} \leftarrow \emptyset$;  $\mathcal{Q}_{r} \leftarrow \emptyset$\;
 $\mathcal{V}_{unv} \leftarrow \mathcal{S}\backslash\{\bm x_{s}\}$; $\mathcal{V}_{\rm open} \leftarrow \{\bm x_{s}\}$; $\mathcal{V}_{\rm closed} \leftarrow \emptyset$\;
 $\bm z \leftarrow \bm x_{s}$\;
 \While{True}{
 ($\bm x_{\rm robot} ,\bm x_{g}, \mathcal{N}_{b}) \leftarrow \UpdateContext(\mathcal{T}, \mathcal{X}_{\rm Dobs})$\; \label{alg:updateContext}
 \For{$i=1$ \KwTo $N_{e}$}{
  $\ExpandFMT(\mathcal{T})$\; \label{alg:expandAndRewire}
  $\RewireFromObstacles(\mathcal{T})$\; \label{alg:rewireObs}
  $\RewireFromRoot(\mathcal{T})$\; \label{alg:rewireRoot}
}
 $(\bm x_{\rm root}, \bm x_1, ..., \bm x_k) \leftarrow \GeneratePath(\mathcal{T}, \bm x_{\rm root})$\; \label{alg:generatepath}
 \If{$\bm x_{\rm robot}$ \upshape{is near} $\bm x_{\rm root}$}{
    $\bm x_{\rm root}\leftarrow \bm x_{1}$\;
    $\UpdateRoot(\mathcal{T},\bm x_{\rm root})$\; \label{alg:updateroot}
 }
 Steer robot towards  $\bm x_{\rm root}$\; \label{alg:steer}
 Perform other tasks\;
}
\caption{RT-FMT($\bm x_{s}, \bm x_{g}, \mathcal{X}_{Fobs}, \mathcal{X}_{\rm Dobs}, N_s, N_e $)}
\label{alg:main}
\end{algorithm}

Inside the infinite loop, the algorithm updates the context of the problem by returning the current position of the robot and goal $x_{g}$, and by finding the nodes in the tree that have been blocked or unblocked by the dynamic obstacles. When a configuration $\bm x_q$ in the tree is blocked, its cost $\textup{c}(\bm x_q)$ is set to infinity. When a node is unblocked, its cost is updated to
\begin{equation}
    \textup{c}(\bm x_q) = \textup{c}(\bm x_{\rm parent}) + \textup{Cost}(\bm x_{\rm parent}, \bm x_{q}), \label{eq:cost}
\end{equation}
where $\bm x_{\rm parent}$ is the configuration of the parent of $\bm x_q$, and
\begin{equation}
    \textup{Cost}(\bm u, \bm v) = ||\bm v- \bm u||.
\end{equation}
If the updated node has children, its children's costs are recursively updated. Lines \ref{alg:expandAndRewire}\textendash\ref{alg:rewireRoot} expand the tree according to Algorithm \ref{alg:ExpandFMT}, and rewire the tree based on Algorithm \ref{alg:RewireFromObstacle}. Line \ref{alg:generatepath} returns a global path if $\bm x_g$ is in the tree or a local path otherwise. The local path is found by computing 
\begin{equation}
    \bm x_k \leftarrow \arg\min_{\bm x \in \mathcal{T}}\textup{c}(\bm x) +  ||\bm x -\bm x_{goal}||,
\end{equation}
and then a path starting at $x_{\rm root}$ and ending at $\bm x_k$ is generated. While we do not limit $k$, RT-RRT* generates a path up to a specific $k$. More details on the local path generation can be found in Algorithm 6 in Naderi et al.~\cite{naderi2015rt}. The approach resembles the A* search. 

In Line \ref{alg:updateroot}, the root of the tree is set to the next configuration in the path, which is always the second element since the path always starts at the old root.  Finally, the algorithm steers the robot towards the new root of the tree on Line \ref{alg:steer}. If there are other tasks to perform such as mapping, and localization, they can be called in the main function as well. 


\subsection{Expanding the Tree}

The tree expansion is described  in Algorithm \ref{alg:ExpandFMT}. Most of the algorithm (Lines \ref{alg:x_near}\textendash\ref{alg:findz}) was inspired by FMT*, proposed in \cite{fmt}, but our implementation has two major differences. 

First, the loops in the original implementation were substituted for conditional statements. These statements ensure that only one node can be added in the tree per call. If multiple nodes are added to the tree at once, there is a possibility of delaying other tasks since the processor will spend too much time expanding many nodes at once. In addition, Line \ref{alg:checkdynamic} not only checks for fixed obstacles but also checks whether $\bm y_{\rm min}$ is not being blocked by a dynamic obstacle according to (\ref{eq:cost}). 

Second, in the original approach, once a node has checked all possible connections with its neighbors, it is closed and never checked again. In our approach, we do not spend time expanding the nodes that are inside $\mathcal{X}_{\rm Dobs}$. As a consequence, when a dynamic obstacle moves, there will be unvisited nodes around closed nodes. To add these nodes to the tree, our algorithm must be able to reopen closed nodes nearby. This is done by adding all $\bm z$ that are near unvisited nodes at closing time to $\mathcal{V}_{\rm toOpen}$ (Line \ref{alg:ztoopen}). Then, when the regular expansion is finished (Line \ref{alg:znull}), the algorithm reopens these nodes (Line \ref{alg:reopen}) to continue the expansion in case a dynamic obstacle moves.

\begin{algorithm}[tb]
\SetKwFunction{Near}{Near}
\SetKwFunction{PopLast}{PopLast}
\SetKwFunction{Cost}{Cost}
\SetKwFunction{c}{c}
\SetKwFunction{CollisionFree}{CollisionFree}
\SetKwFunction{Open}{Open}
\SetKwFunction{Close}{Close}
\lIf{$\mathcal{X}_{near}$ = $\emptyset$ $\&$ $\bm z \neq \emptyset$}{$\mathcal{X}_{near} \leftarrow$ \Near($\bm z$,$\mathcal{V}_{unv}$)} \label{alg:x_near}
\Else{
$\bm x = \PopLast(\mathcal{X}_{near})$\; 
$\mathcal{Y}_{near} \leftarrow$ \Near($\bm x$,$\mathcal{V}_{\rm open}$)\;
\bm $y_{\rm min} \leftarrow \arg\min_{\bm y \in \mathcal{Y}_{near} }(\c(\bm y) + \Cost(\bm x, \bm y))$\;

\If{$\CollisionFree(\bm y_{\rm min}, \bm x) \& \c(\bm y_{\rm min}) < \infty$}{ \label{alg:checkdynamic}
$\mathcal{V}_{open, new} \leftarrow  \mathcal{V}_{open, new} \cup \{\bm x\}$\;
$\mathcal{V}_{unv} \leftarrow \mathcal{V}_{unv}\backslash\{\bm x\}$\;
$\c(\bm x) \leftarrow \c(\bm y_{\rm min}) + \Cost(\bm y_{\rm min}, \bm x)$\;
$\mathcal{T} \leftarrow \mathcal{T} \cup \{\bm x, (\bm y_{\rm min}, \bm x) \}$
}
\If{$\mathcal{X}_{near}$ = $\emptyset$ $\&$ $\bm z \neq \emptyset$}{
\Close($\bm z$)\;
$\mathcal{Z}_{near} \leftarrow$ \Near($\bm z$,$\mathcal{V}_{unv}$)\;
$\bm z \leftarrow \arg\min_{\bm y \in \mathcal{V}_{\rm open} }\c(\bm y)$\; \label{alg:findz}
\ForEach{$\bm x \in \mathcal{Z}_{near}$ }{ \label{alg:ztoopen}
\lIf{\CollisionFree($\bm z, \bm x$)}{$\mathcal{V}_{\rm toOpen} \leftarrow \mathcal{V}_{\rm toOpen} \cup \bm z  $} 
}

}
\If{$\bm z = \emptyset$}{ \label{alg:znull}
$\Open(\mathcal{V}_{\rm toOpen})$;~$\mathcal{V}_{\rm toOpen} = \emptyset$\; \label{alg:reopen}
$\bm z \leftarrow \arg\min_{\bm y \in \mathcal{V}_{\rm open} }\c(\bm y)$\; }
}

\SetKwProg{Def}{def}{:}{}
\Def{\Close($\mathcal{V}$)}{
$\mathcal{V}_{\rm open} \leftarrow  \mathcal{V}_{\rm open} \cup \mathcal{V}_{open, new}\backslash\mathcal{V}$\;
$\mathcal{V}_{\rm closed} \leftarrow  \mathcal{V}_{\rm closed} \cup \mathcal{V}$\;
}

\SetKwProg{Def}{def}{:}{}
\Def{\Open($\mathcal{V}$)}{
$\mathcal{V}_{\rm open} \leftarrow  \mathcal{V}_{\rm open} \cup  \mathcal{V}$\;
$\mathcal{V}_{\rm closed} \leftarrow  \mathcal{V}_{\rm closed} \backslash \mathcal{V}$\;
}
\caption{ExpandFMT($\mathcal{T}$)}
\label{alg:ExpandFMT}
\end{algorithm}

\subsection{Rewiring the Tree}

As the dynamic obstacles move around the environment, the tree nodes are constantly being blocked and unblocked by Line \ref{alg:updateContext} in Algorithm \ref{alg:main}. When a node is blocked or unblocked, its cost is changed and all its children are recursively updated. The task of the {\tt RewireFromObstacle} method in Algorithm \ref{alg:RewireFromObstacle} is to find the connections with lower cost in the neighborhood of nodes that have recently been blocked or unblocked. This function only rewires the nodes that have recently been affected by a dynamic obstacle. The rewiring process starts by adding all blocked nodes to $\mathcal{Q}_o$. Then, nodes are iteratively removed from the list and the algorithm tries to find parents nearby with a lower cost. If there is a connection with a lower cost that is also collision-free, the children of the updated nodes are also added to $\mathcal{Q}_o$. 

\begin{algorithm}[tb]
\SetKwFunction{PopFirst}{PopFirst}
\SetKwFunction{UpdateParentChild}{UpdateParentChild}
\SetKwFunction{RecalculateChildrenCost}{RecalculateChildrenCost}
\lIf{$\mathcal{Q}_o$ = $\emptyset$}{$\mathcal{Q}_o \leftarrow \mathcal{N}_{b}$} \label{alg:addblockedtolist}
\Else{
$\bm x_b \leftarrow \PopFirst(\mathcal{Q}_o)$\;
   \If{$\bm x_b \notin \mathcal{X}_{\rm Dobs}$}{
        $\mathcal{Y}_{near} \leftarrow \Near(\bm x_b,\mathcal{V}_{\rm open}\cup \mathcal{V}_{\rm closed}$)\;
        \bm $y_{\rm min} \leftarrow \arg\min_{\bm y \in \mathcal{Y}_{near} }(\c(\bm y) + \Cost(\bm x_b, \bm y))$\;
        \If{$\CollisionFree(\bm y_{\rm min}, \bm x) \& \c(\bm y_{\rm min}) < \infty$}{
        $\UpdateParentChild(\mathcal{T}, \bm y_{\rm min}, \bm x_b)$\;
        $\RecalculateChildrenCost(\bm x_b)$
        }
   }

}
\caption{RewireFromObstacles($\mathcal{T}$)}
\label{alg:RewireFromObstacle}
\end{algorithm}

As the robot moves around the environment, the tree root is updated by Algorithm \ref{alg:main}, Line \ref{alg:updateroot}. When this happens, the algorithm triggers the {\tt RewireFromRoot} function (Line \ref{alg:rewireRoot}) to rewire all nodes in the tree by inserting the new root into $\mathcal{Q}_{r}$. The {\tt RewireFromRoot} function is very similar to Algorithm \ref{alg:RewireFromObstacle}. The algorithm removes a node from $\mathcal{Q}_{r}$ and tries to find better connections in the tree. When a new connection is made, the children of the updated node are also added to $\mathcal{Q}_{r}$. This causes a chain reaction that starts from the root and updates all nodes in the tree. This method has also been implemented without any loops to only update a single node per call. Our implementation triggers this chain reaction whenever the root is updated. However, this can easily be modified to happen at a desired frequency.


%\subsection{Real-time considerations}




\section{Methodology}

\begin{figure*}[t]
\begin{minipage}{0.63\textwidth}
\centering
\includegraphics[width=\linewidth]{imgs/architecture.pdf}
\label{fig:edeline_architecture}
\end{minipage}
\hfill
\begin{minipage}{0.35\textwidth}
\vspace{-2em}
\caption{\textbf{Framework Overview of EDELINE.}} 
      The model integrates three principal components: 
      (1) An U-Net-like \textit{Next-Frame Predictor} enhanced by adaptive group normalization and cross-attention mechanisms,
      % A \textit{Next-Frame Predictor} constructed with a U-Net architecture, enhanced by adaptive group normalization and cross-attention mechanisms, 
      (2) \textcolor{black}{A \textit{Recurrent Embedding Module} built on Mamba architecture for temporal sequence processing, and}
      % A \textit{Recurrent Embedding Module} built on Mamba architecture for temporal sequence processing through observation encoding and embedding layers, and
      (3) A \textit{Reward/Termination Predictor} implemented through linear layers. The EDELINE framework uses shared hidden representations across the components for efficient world model learning.
\end{minipage}\vspace{-1em}
\end{figure*}

Conventional diffusion-based world models \cite{alonso2024diamond} demonstrate promise in learning environment dynamics yet face fundamental limitations in memory capacity and horizon prediction consistency. To address these challenges, this paper presents EDELINE, as illustrated in Fig~\ref{fig:edeline_architecture}, a unified architecture that integrates state space models (SSMs) with diffusion-based world models. EDELINE's core innovation lies in its integration of SSMs for encoding sequential observations and actions into hidden embeddings, which a diffusion model then processes for future frame prediction. This hybrid design maintains temporal consistency while generating high-quality visual predictions. A Convolutional Neural Network based actor processes these predicted frames to determine actions, thus enabling autoregressive generation of imagined trajectories for policy optimization.

% This section presents EDELINE's world model architecture and training methodology, with emphasis on the SSM integration capabilities for enhanced memory capacity and imagination consistency in diffusion-based world models. The subsequent discussion details the mechanisms of actor-critic utilization of imagined trajectories in pixel space for efficient and effective policy learning.

\subsection{World Model Learning}

The core architecture of EDELINE consists of a \textit{Recurrent Embedding Module {(REM)}} $f_\phi$ that processes the history of observations and actions $(o_0, a_0, o_1, a_1, ..., o_t, a_t)$ to generate a hidden embedding $h_t$ through recursive computation. This embedding enables the \textit{Next-Frame Predictor} $p_\phi$ to generate predictions of the subsequent observation $\hat{o}_{t+1}$. The architecture further incorporates dedicated \textit{Reward and Termination Predictors} to estimate the reward $\hat{r}_t$ and episode termination signal $\hat{d}_t$ respectively. The trainable components of EDELINE's world model are formalized as: \vspace{-2em}
\begin{itemize} [itemsep=3pt, parsep=0pt]
    \item Recurrent Embedding Module: $h_t = f_\phi(h_{t-1}, o_t, a_t)$
    \item Next-Frame Predictor: $\hat{o}_{t+1} \sim p_\phi(\hat{o}_{t+1}|h_t)$
    \item Reward Predictor: $\hat{r}_t \sim p_\phi(\hat{r}_t|h_t)$
    \item Termination Predictor: $\hat{d}_t \sim p_\phi(\hat{d}_t|h_t)$
\end{itemize}

\subsubsection{Recurrent Embedding Module 
% \josout{(REM)}
}

While DIAMOND, the current state-of-the-art in diffusion-based world models, relies on a fixed context window of four previous observations and actions sequence, the proposed EDELINE architecture advances beyond this limitation through a recurrent architecture for extended temporal sequence processing. Specifically, we provide theoretical evidence in Theorem~\ref{the:information_retention_superiority} to support that recurrent embedding can preserve more information compared with stacked frames inputs. At each timestep $t$, the Recurrent Embedding Module processes the current observation-action pair $(o_t, a_t)$ to update a hidden state $h_t = f_\phi(h_{t-1}, o_t, a_t)$.
% , expressed as follows:
% \begin{equation}
% h_t = f_\phi(h_{t-1}, o_t, a_t).
% \end{equation}
The implementation of REM utilizes Mamba~\cite{gu2024mamba}, an SSM architecture that offers distinct advantages for world modeling. This architectural selection is motivated by the limitations of current sequence processing methods in deep learning. Self-attention-based Transformer architectures, despite their strong modeling capabilities, suffer from quadratic computational complexity which impairs efficiency. Traditional recurrent architectures including Long Short-Term Memory (LSTM)~\cite{HochSchm97} and Gated Recurrent Unit (GRU)~\cite{69e088c8129341ac89810907fe6b1bfe} experience gradient instability issues that affect dependency learning. In contrast, SSMs provide an effective alternative through linear-time sequence processing coupled with robust memorization capabilities via their state-space formulation. The adoption of Mamba emerges as a promising choice due to its demonstrated effectiveness in modeling temporal patterns across various sequence modeling tasks. Section~\ref{subsec:ablation_studies} presents a comprehensive ablation study that evaluates different architectural choices for the REM.

\subsubsection{Next-Frame Predictor}

While motivated by DIAMOND's success in diffusion-based world modeling, EDELINE introduces significant architectural innovations in its Next-Frame Predictor to enhance temporal consistency and feature integration. At time step $t$, the model conditions on both the last $L$ frames and the hidden embedding $h_t$ from the Recurrent Embedding Module to predict the next frame $\hat{o}_{t+1}$. The predictive distribution $p_\phi(o^0_{t+1}|h_t)$ is implemented through a denoising diffusion process, where $D_\phi$ functions as the denoising network. Let $y_t^{\tau} = (\tau, o^0_{t-L+1}, ..., o^0_t, h_t)$ represent the conditioning information, where $\tau$ represents the diffusion time. The denoising process can be formulated as $o^0_{t+1} = D_\phi(o^{\tau}_{t+1}, y_t^{\tau}).$
% then be formulated as follows:
% \begin{equation}
%     o^0_{t+1} = D_\phi(o^{\tau}_{t+1}, y_t^{\tau}).
% \end{equation}
To effectively integrate both visual and hidden information, $D_\phi$ employs two complementary conditioning mechanisms. First, the architecture incorporates \cite{AGN} layers within each residual block to condition normalization parameters on the hidden embedding $h_t$ and diffusion time $\tau$, which establishes context-aware feature normalization \cite{AGN}. This design significantly extends DIAMOND's implementation, which limits AGN conditioning to $\tau$ and action embeddings only. The second key innovation introduces cross-attention blocks inspired by Latent Diffusion Models (LDMs), which utilize $h_t$ and $\tau$ as context vectors. The UNet's feature maps generate the query, while $h_t$ and $\tau$ project to keys and values. This novel attention mechanism, which is absent in DIAMOND, facilitates the fusion of spatial-temporal features with abstract dynamics encoded in $h_t$. The observation modeling loss $\mathcal{L}_{\text{obs}}(\phi)$ is defined based on Eq.~(\ref{eq:d_loss}), and can be formulated as follows:
\begin{equation}
\mathcal{L}_{\text{obs}}(\phi) = \mathbb{E}\left[\|D_\phi(o^{\tau}_{t+1}, y_t^{\tau}) - o^0_{t+1}\|^2\right].
\end{equation}
\subsubsection{Reward / Termination Predictor}
EDELINE advances beyond DIAMOND's architectural limitations through an integrated approach to reward and termination prediction. Rather than employing separate neural networks, EDELINE leverages the rich representations from its REM. The reward and termination predictors are implemented as multilayer perceptrons (MLPs) that utilize the deterministic hidden embedding $h_t$ as their conditioning input. This architectural unification enables efficient representation sharing across all predictive tasks. EDELINE processes both reward and termination signals as probability distributions conditioned on the hidden embedding: $p_\phi(\hat{r}_t|h_t)$ and $p_\phi(\hat{d}_t|h_t)$ respectively. The predictors are optimized via negative log-likelihood losses, expressed as:
\begin{equation}
\mathcal{L}_{\text{rew}}(\phi) = -\ln p_\phi(r_t|h_t),
% \end{equation}
% \begin{equation}
\mathcal{L}_{\text{end}}(\phi) = -\ln p_\phi(d_t|h_t).
\end{equation}
This unified architectural design represents an improvement over DIAMOND's separate network approach, where reward and termination predictions require independent representation learning from the world model. The integration of these predictive tasks with shared representations enables REM to learn dynamics that encompass all relevant aspects of the environment. The architectural efficiency facilitates enhanced learning effectiveness and better performance.
\vspace{-0.5em}

\begin{table*}[ht!]  
  \vspace{-1em}
  \caption{Game scores and overall human-normalized scores on the $26$ games in the Atari $100$k benchmark. Results are averaged over 3 seeds, with bold numbers indicating the best performing method for each metric.}
  \label{table:atari_100k}
  \vspace{0.1cm}
  \centering
  \resizebox{\textwidth}{!}{\begin{tabular}{lrrrrrrrrrr}
Game                &  Random    &  Human     &  SimPLe    &  TWM                &  IRIS              &  STORM              &  DreamerV3   &  Drama     &  DIAMOND           &  EDELINE (ours)      \\
\midrule
Alien               &  227.8     &  7127.7    &  616.9     &  674.6              &  420.0             &  \textbf{983.6}     &  959.4       &  820      &  744.1             &  974.6               \\
Amidar              &  5.8       &  1719.5    &  74.3      &  121.8              &  143.0             &  204.8              &  139.1       &  131      &  225.8             &  \textbf{299.5}      \\
Assault             &  222.4     &  742.0     &  527.2     &  682.6              &  1524.4            &  801.0              &  705.6       &  539      &  \textbf{1526.4}   &  1225.8              \\
Asterix             &  210.0     &  8503.3    &  1128.3    &  1116.6             &  853.6             &  1028.0             &  932.5       &  1632     &  3698.5            &  \textbf{4224.5}     \\
BankHeist           &  14.2      &  753.1     &  34.2      &  466.7              &  53.1              &  641.2              &  648.7       &  137      &  19.7              &  \textbf{854.0}      \\
BattleZone          &  2360.0    &  37187.5   &  4031.2    &  5068.0             &  13074.0           &  \textbf{13540.0}   &  12250.0     &  10860    &  4702.0            &  5683.3              \\
Boxing              &  0.1       &  12.1      &  7.8       &  77.5               &  70.1              &  79.7               &  78.0        &  78       &  86.9              &  \textbf{88.1}       \\
Breakout            &  1.7       &  30.5      &  16.4      &  20.0               &  83.7              &  15.9               &  31.1        &  7        &  132.5             &  \textbf{250.5}      \\
ChopperCommand      &  811.0     &  7387.8    &  979.4     &  1697.4             &  1565.0            &  1888.0             &  410.0       &  1642     &  1369.8            &  \textbf{2047.3}     \\
CrazyClimber        &  10780.5   &  35829.4   &  62583.6   &  71820.4            &  59324.2           &  66776.0            &  97190.0     &  52242    &  99167.8           &  \textbf{101781.0}   \\
DemonAttack         &  152.1     &  1971.0    &  208.1     &  350.2              &  \textbf{2034.4}   &  164.6              &  303.3       &  201      &  288.1             &  1016.1              \\
Freeway             &  0.0       &  29.6      &  16.7      &  24.3               &  31.1              &  33.5               &  0.0         &  15       &  33.3              &  \textbf{33.8}       \\
Frostbite           &  65.2      &  4334.7    &  236.9     &  \textbf{1475.6}    &  259.1             &  1316.0             &  909.4       &  785      &  274.1             &  286.8               \\
Gopher              &  257.6     &  2412.5    &  596.8     &  1674.8             &  2236.1            &  \textbf{8239.6}    &  3730.0      &  2757     &  5897.9            &  6102.3              \\
Hero                &  1027.0    &  30826.4   &  2656.6    &  7254.0             &  7037.4            &  11044.3            &  11160.5     &  7946     &  5621.8            &  \textbf{12780.8}    \\
Jamesbond           &  29.0      &  302.8     &  100.5     &  362.4              &  462.7             &  509.0              &  444.6       &  372      &  427.4             &  \textbf{784.3}      \\
Kangaroo            &  52.0      &  3035.0    &  51.2      &  1240.0             &  838.2             &  4208.0             &  4098.3      &  1384     &  \textbf{5382.2}   &  5270.0              \\
Krull               &  1598.0    &  2665.5    &  2204.8    &  6349.2             &  6616.4            &  8412.6             &  7781.5      &  9693     &  8610.1            &  \textbf{9748.8}     \\
KungFuMaster        &  258.5     &  22736.3   &  14862.5   &  24554.6            &  21759.8           &  26182.0            &  21420.0     &  17236    &  18713.6           &  \textbf{31448.0}    \\
MsPacman            &  307.3     &  6951.6    &  1480.0    &  1588.4             &  999.1             &  \textbf{2673.5}    &  1326.9      &  2270     &  1958.2            &  1849.3              \\
Pong                &  -20.7     &  14.6      &  12.8      &  18.8               &  14.6              &  11.3               &  18.4        &  15       &  20.4              &  \textbf{20.5}       \\
PrivateEye          &  24.9      &  69571.3   &  35.0      &  86.6               &  100.0             &  \textbf{7781.0}    &  881.6       &  90       &  114.3             &  99.5                \\
Qbert               &  163.9     &  13455.0   &  1288.8    &  3330.8             &  745.7             &  4522.5             &  3405.1      &  796      &  4499.3            &  \textbf{6776.2}     \\
RoadRunner          &  11.5      &  7845.0    &  5640.6    &  9109.0             &  9614.6            &  17564.0            &  15565.0     &  14020    &  20673.2           &  \textbf{32020.0}    \\
Seaquest            &  68.4      &  42054.7   &  683.3     &  774.4              &  661.3             &  525.2              &  618.0       &  497      &  551.2             &  \textbf{2140.1}     \\
UpNDown             &  533.4     &  11693.2   &  3350.3    &  \textbf{15981.7}   &  3546.2            &  7985.0             &  7567.1      &  7387     &  3856.3            &  5650.3              \\
\midrule
\#Superhuman (↑)    &  0         &  N/A       &  1         &  8                  &  10                &  10                 &  9           &  7         &  11                &  \textbf{13}         \\
Mean (↑)            &  0.000     &  1.000     &  0.332     &  0.956              &  1.046             &  1.266              &  1.124       &  0.989     &  1.459             &  \textbf{1.866}      \\
Median (↑)          &  0.000     &  1.000     &  0.134     &  0.505              &  0.289             &  0.580              &  0.485       &  0.270     &  0.373             &  \textbf{0.817}      \\
IQM (↑)             &  0.000     &  1.000     &  0.130     &  0.459              &  0.501             &  0.636              &  0.487       &  -         &  0.641             &  \textbf{0.940}      \\
Optimality Gap (↓)  &  1.000     &  0.000     &  0.729     &  0.513              &  0.512             &  0.433              &  0.510       &  -         &  0.480             &  \textbf{0.387}      \\

  \end{tabular}}
  \vspace{-1.5em}
\end{table*}

\subsubsection{EDELINE World Model Training}
The world model integrates an innovative end-to-end training strategy with a self-supervised approach. EDELINE extends the harmonization technique from HarmonyDream \cite{ma2024harmonydream} through the adoption of harmonizers $w_o$ and $w_r$, which dynamically balance the observation modeling loss $\mathcal{L}_{\text{obs}}(\phi)$ and reward modeling loss $\mathcal{L}_{\text{rew}}(\phi)$. This adaptive mechanism results in the total loss function $\mathcal{L}(\phi)$:
\begin{equation}
\label{eq:total_loss}
\begin{split}
\mathcal{L}(\phi) = w_0\mathcal{L}_{\text{obs}}(\phi) &+ w_r\mathcal{L}_{\text{rew}}(\phi) + \mathcal{L}_{\text{end}}(\phi) \\
    &+ \log(w_o^{-1}) + \log(w_r^{-1})
\end{split}
\end{equation}
To optimize computational efficiency while ensuring robust learning, the Next-Frame Predictor learns to utilize hidden embeddings from any timestep through strategic random sampling for $\mathcal{L}_{\text{obs}}(\phi)$. For a sequence of length $T$, 
% this computation 
it follows:
\begin{equation}
\mathcal{L}_{\text{obs}}(\phi) = \|\hat{o}^0_{t+1} - o_{t+1}\|^2,
\end{equation}
where $i \sim \text{Uniform}\{1,2,\ldots,T-1\}, \quad \hat{o}^0_{t+1} \sim p(\hat{o}^0_{t+1} \mid h_t).$ For reward and termination prediction, EDELINE utilizes cross-entropy losses averaged over the sequence, which can be formulated as:
% \begin{equation}
    $\mathcal{L}_{\text{rew}}(\phi) = \frac{1}{T}\sum_{t=1}^T \text{CrossEnt}(\hat{r}_t, r_t),$
% \end{equation}
% \begin{equation}
    $\mathcal{L}_{\text{end}}(\phi) = \frac{1}{T}\sum_{t=1}^T \text{CrossEnt}(\hat{d}_t, d_t).$
% \end{equation}
This unified training approach, combining random sampling strategies with dynamic loss harmonization, demonstrates superior efficiency compared to DIAMOND's separate network methodology, as validated in our results presented in Section 6. Moreover, the quantitative analysis presented in Appendix~\ref{appendix:training_time_profile} reveals substantial reductions in world model training duration.
\vspace{-0.5em}

\subsection{Agent Behavior Learning}
To enable fair comparison and demonstrate the effectiveness of EDELINE's world model architecture, the agent architecture adopts the same optimization framework as DIAMOND. Specifically, the agent integrates policy $\pi_\theta$ and value $V_\theta$ networks with REINFORCE value baseline and Bellman error optimization using $\lambda$-returns~\cite{alonso2024diamond}. The training framework executes a procedure with three key phases: experience collection, world model updates, and policy optimization. This method, as formalized in Algorithm~\ref{alg:edeline}, follows the established paradigms in model-based RL literature~\cite{Kaiser2020SimPLe,Hafner2020Dreamer,micheli2023iris,alonso2024diamond}. To ensure reproducibility, we provide extensive details in the Appendix, with documentation of objective functions, and the hyperparameter configurations in Appendices~\ref{appendix:rl_objectives}, \ref{appendix:hyper}, respectively. \vspace{-2em}


\begin{table*}[!t]
    \centering
    \resizebox{\textwidth}{!}{%
        \begin{tabular}{lccccc|c}
            \toprule
            & \textbf{Microbiology} & \textbf{Chemistry} & \textbf{Economics} & \textbf{Sociology} & \textbf{US History} & \textbf{Average} \\
            \midrule
            Base score & 0.46 & 0.09 & 0.00 & 0.61 & 0.03 & 0.24 \\
            \midrule
            Zero-shot & 0.62 (+0.16) & 0.40 (+0.31) & 0.40 (+0.40) & 0.61 (+0.00) & 0.19 (+0.16) & 0.44 (+0.20) \\
            Few-shot & 0.62 (+0.16) & \underline{0.45} (+0.36) & \underline{0.47} (+0.47) & 0.62 (+0.01) & 0.16 (+0.13) & 0.46 (+0.22) \\
            Chain-of-thought & 0.61 (+0.15) & \underline{0.45} (+0.36) & 0.46 (+0.46) & 0.61 (+0.00) & 0.19 (+0.16) & 0.46 (+0.22) \\
            Bloom-based & 0.57 (+0.11) & 0.37 (+0.28) & 0.29 (+0.29) & 0.62 (+0.01) & 0.22 (+0.19) & 0.41 (+0.17) \\
            \midrule
            SFT (Subject-Specific) & 0.65 (+0.19) & 0.24 (+0.15) & 0.46 (+0.46) & \underline{0.64} (+0.03) & 0.20 (+0.17) & 0.44 (+0.20) \\
            SFT (Cross-Subject) & 0.59 (+0.13) & 0.21 (+0.12) & \underline{0.47 (+0.47)} & 0.63 (+0.02) & \underline{0.26} (+0.23) & 0.43 (+0.19) \\
            \midrule
            \textsc{QUEST} (Subject-Specific) & \bf 0.76 (+0.30) & \bf 0.46 (+0.37) & \bf 0.58 (+0.58) & \bf 0.65 (+0.04) & \bf 0.31 (+0.28) & \bf 0.55 (+0.31) \\
            \textsc{QUEST} (Cross-Subject) & \underline{0.73} (+0.27) & \underline{0.41} (+0.32) & \underline{0.47} (+0.47) & \bf 0.65 (+0.04) & 0.25 (+0.22) & \underline{0.50} (+0.26) \\
            \bottomrule
        \end{tabular}
    }
    \caption{\textbf{End-of-chapter exam score} results of different question generation approaches across various subjects. 
    \ours produces models that generate questions that lead to the highest scores on all subjects.  
    Gain values (in parentheses) are calculated as the increase from the base score. The highest and second highest scores per column are \textbf{bolded} and \underline{underlined}, respectively.}
    \label{tab:question-gen-results}
\end{table*}


\section{Experimental Results}
In this section, we first compare the overall performance of all question generation baselines based on the learner's exam score (\secref{ssec:overall-performance}).  
Next, we analyze evaluation metrics by examining their correlations with utility and assessing the impact of optimizing models on high-scoring questions for each metric (\secref{ssec:evaluation-metrics}).
We then conduct a qualitative analysis of high-utility questions to understand their characteristics (\secref{ssec:high-utility-question}). 
Finally, we perform ablation studies on the framework by varying the criteria for selecting high-utility questions for training and replacing \texttt{gpt-4o-mini} to \texttt{gpt-4o} (\secref{ssec:rs-analysis}-\secref{ssec:model-variants}).



\subsection{Overall Performance}
\label{ssec:overall-performance}
Table~\ref{tab:question-gen-results} presents the learner's exam performance of different question generators, measured by exam scores using all generated question-answer pairs (\secref{ssec:quest-evaluation}).
Here are findings:
(1) \textbf{Prompting techniques (Few-shot, CoT, Bloom-based)} offer only marginal performance gains. While advanced prompting enhances reasoning and task accuracy~\cite{brown2020language, wei2022chain, zhou2024self}, it does not directly optimize utility, which reflects real-world impact—how well generated questions enhance learning. Without explicit selection or optimization, prompting cannot systematically improve this measure;
(2) \textbf{SFT} shows no performance gains, indicating that while it learns the style of exam questions, it fails to generate questions that enhance learner understanding.
This highlights the key distinction between producing syntactically valid questions and generating those that effectively promote learning;
(3) \textbf{QUEST} achieves the highest performance gain, improving by approximately 20\% on average.
Performance gap between subject-specific and cross-subject rejection sampling suggests that the definition of a ``high-utility'' question varies by domain.
The results indicate that outcome-based learning is most effective when applied within a specific domain.


\subsection{Evaluation Metrics Analysis}
\label{ssec:evaluation-metrics}
\paragraph{Correlation.}
\begin{table}[!t]
    \centering
    \resizebox{\columnwidth}{!}{%
        \begin{tabular}{ll
        cc}
            \toprule
            Metric 1 & Metric 2 & \textbf{Spearman correlation} & \textbf{p-value} \\
            \midrule
            \textbf{Utility} & Saliency & 0.097 & 0.003 \\
            \textbf{Utility} & EIG  & -0.022 & 0.512 \\
            Saliency & EIG  & 0.030 & 0.363 \\
            \bottomrule
        \end{tabular}
    }
    \caption{\textbf{Spearman correlation between metrics.} Utility shows a weak correlation with saliency and EIG, showing that it is independent of these indirect metrics.}
    \label{tab:correlation_results}
\end{table}
To analyze the relationship between \textit{utility} and existing metrics (\textit{saliency}, \textit{EIG}), we estimate all three metrics on generated questions from the training set.
Table~\ref{tab:correlation_results} shows that both saliency and EIG have weak correlations with utility.
While saliency has a weak but statistically significant correlation with utility, EIG shows no meaningful relationship.
This indicates that existing indirect metrics may not accurately reflect a question’s impact on learning outcomes.

\paragraph{Optimization on Indirect Metrics.}
\begin{table*}[!t]
    \centering
    \small
    \resizebox{\textwidth}{!}{%
        \begin{tabular}{lccc ccc ccc ccc ccc}
            \toprule
            & \multicolumn{3}{c}{\textbf{Microbiology}} & \multicolumn{3}{c}{\textbf{Chemistry}} & \multicolumn{3}{c}{\textbf{Economics}} & \multicolumn{3}{c}{\textbf{Sociology}} & \multicolumn{3}{c}{\textbf{US History}} \\
            \cmidrule(lr){2-4} \cmidrule(lr){5-7} \cmidrule(lr){8-10} \cmidrule(lr){11-13} \cmidrule(lr){14-16}
            \textbf{Train Metric} & \textbf{Utility} & \textbf{Saliency} & \textbf{EIG} & \textbf{Utility} & \textbf{Saliency} & \textbf{EIG} & \textbf{Utility} & \textbf{Saliency} & \textbf{EIG} & \textbf{Utility} & \textbf{Saliency} & \textbf{EIG} & \textbf{Utility} & \textbf{Saliency} & \textbf{EIG} \\
            \midrule
            $utility > 0.1$ & \textbf{0.76} & 4.27 & -0.18 & \textbf{0.46} & 4.65 & -0.20 & \textbf{0.58} & \textbf{4.70} & -0.04 & \textbf{0.65} & \textbf{4.49} & -0.02 & \textbf{0.31} & 4.65 & \bf -0.01 \\
            $saliency = 5$  & 0.73 & \textbf{4.42} & -0.24 & 0.39 & \textbf{4.46} & -0.22 & 0.46 & 4.66 & -0.08 & 0.64 & \textbf{4.49} & -0.03 & 0.23 & \textbf{4.68} & -0.02 \\
            $EIG > 0$      & 0.61 & 4.21 & \textbf{-0.17} & 0.32 & 4.40 & \textbf{-0.09} & 0.47 & 4.65 & \textbf{0.01} & 0.62 & 4.46 & \textbf{0.01} & 0.21 & 4.65 & \bf -0.01 \\
            \bottomrule
        \end{tabular}
    }
    \caption{\textbf{End-of-chapter exam scores (utility), average saliency, and average EIG of generated questions} for different \textsc{QUEST}-optimized models trained on datasets filtered by different selection criteria.}
    \label{tab:quest-indirect}
\end{table*}
To further investigate the impact of indirect metrics on question generation performance, we compare the results of \textsc{QUEST} when trained using different selection criteria: (1) only questions with a \textit{utility} score greater than 0.1 (Ours), (2) only questions with a \textit{saliency} score of 5, and (3) only questions with an \textit{expected information gain (EIG)} greater than 0.
We evaluate the generated questions based on their overall utility (\textit{i.e.,} end-of-chapter exam scores), as well as their average saliency and EIG, to assess the question generator’s performance across different quality metrics.
Table~\ref{tab:quest-indirect} shows that while saliency- and EIG-based training improves their respective scores, it does not enhance utility. 
In contrast, the utility-trained model consistently achieves the highest utility across all subjects and even improves some indirect metrics (\textit{e.g.,} it matches or outperforms saliency-based training in Economics and Sociology).
Furthermore, utility-based training outperforms EIG-based training on saliency and saliency-based training on EIG, demonstrating its broader effectiveness. 
These results emphasize that optimizing for indirect metrics does not improve real-world learning, whereas utility-driven training yields the best overall performance.




\subsection{High Utility Questions Analysis}
\label{ssec:high-utility-question}
\paragraph{Overlap with exam questions.}
To evaluate the relationship between generated high-utility questions and exam questions, we measured their semantic and lexical similarity.
For each generated question, we computed embedding similarity using \texttt{text-3-embedding-small}\footnote{\href{https://platform.openai.com/docs/guides/embeddings/}{https://platform.openai.com/docs/guides/embeddings/}} for semantic overlap and the ROUGE score for lexical overlap with all exam questions in the same chapter.
We then assess the correlation between utility and the most similar exam question based on these measures.
The correlation between utility and semantic similarity is 0.25 (p < 0.001), indicating a weak positive relationship, while the correlation with ROUGE is nearly zero at 0.04 (p < 0.01).
These findings suggest that high-utility questions are not simple rephrasings of exam questions but introduce novel concepts that enhance learning beyond surface-level similarity.

\paragraph{Qualitative analysis.}
Qualitative question examples do not exhibit clear patterns in question style (see Appendix~\ref{appendix:qualitative-examples}).
An interesting observation is that Bloom's taxonomy, which categorizes cognitive depth based on question type—where "what" questions typically involve simple recall, while "why" and "how" questions require deeper processing—does not strongly correlate with utility.
Using Bloom's taxonomy as a cognitive depth scale (Likert 1-6), the correlation between utility and cognitive depth is 0.12 (p < 0.001), indicating a weak positive relationship.


\subsection{Rejection Sampling Analysis}
\label{ssec:rs-analysis}
Filtering for high-utility questions through rejection sampling is crucial for improving question generation. 
As shown in Figure~\ref{fig:rs_analysis}, increasing the utility threshold enhances question quality, leading to higher exam scores.
However, stricter filtering reduces the available training data, posing challenges for model training. 
These results suggest that increasing the dataset size while applying a higher threshold could further boost performance.
\begin{figure}[!t]
    \centering
    \begin{minipage}{\columnwidth}
    \centering
    \includegraphics[width=\columnwidth]{figures/rs_analysis.pdf}
    \end{minipage}
    \caption{\textbf{Impact of threshold} in \ours on end-of-chapter exam scores for Chemistry.}
    \label{fig:rs_analysis}
    \vspace{-0.5cm}
\end{figure}

\subsection{Model Variants Analysis}
\label{ssec:model-variants}
\begin{table*}[!t]
    \centering
    \resizebox{\textwidth}{!}{%
    \begin{tabular}{c|ccc|ccccc}
        \toprule
        & \textbf{QG ($M_q$)} & \textbf{AG ($M_a$)} & \textbf{RS ($M_l$)} & \textbf{Microbiology} & \textbf{Chemistry} & \textbf{Economics} & \textbf{Sociology} & \textbf{US History} \\
        \midrule
        \multicolumn{1}{c|}{\multirow{4}{*}{Zero-Shot}} 
        & \texttt{gpt-4o-mini} & \texttt{gpt-4o-mini} & \texttt{gpt-4o-mini} & 0.620 & 0.414 & 0.398 & 0.609 & 0.233 \\
        & \texttt{gpt-4o} & \texttt{gpt-4o-mini} & \texttt{gpt-4o-mini} & 0.681 & 0.457 & 0.466 & 0.634 & 0.180 \\
        & \texttt{gpt-4o-mini} & \texttt{gpt-4o} & \texttt{gpt-4o-mini} & 0.682 & 0.422 & 0.480 & 0.634 & 0.232 \\
        & \texttt{gpt-4o-mini} & \texttt{gpt-4o-mini} & \texttt{gpt-4o} & 0.710 & 0.173 & 0.476 & 0.564 & 0.263 \\
        \midrule
        \textsc{QUEST} & \texttt{gpt-4o-mini} & \texttt{gpt-4o-mini} & \texttt{gpt-4o-mini} & \bf 0.756 &	\bf 0.457 &	\bf 0.582 &	\bf 0.649 &	\bf 0.311 \\
        \bottomrule
    \end{tabular}
    }
    \vspace{-0.2cm}
    \caption{\textbf{End-of-chapter exam scores} for different model sizes across various subjects and modules.}
        \vspace{-0.2cm}

    \label{tab:model-variants}
\end{table*}

To evaluate the robustness of our framework and the impact of model size on different components, we conduct experiments to analyze how a larger model affects each module.
Table~\ref{tab:model-variants} presents results for different configurations of the question generator ($M_q$), answer generator ($M_a$), and reader simulator (\textit{i.e.,} learner $M_l$). 
The baseline corresponds to the \texttt{zero-shot} setting in Table~\ref{tab:question-gen-results}. 
In the $M_a$ experiment, we use the same questions from the \texttt{zero-shot} setting but generate new answers.
For the reader simulator experiment, we keep the same questions and answers from \texttt{zero-shot} and re-run the simulation only.

Our main findings are the following:
(1) \textbf{Question Generator}: A larger model (\texttt{gpt-4o}) improves the utility score by 5.7\%. However, it still underperforms compared to the smaller, utility-optimized model (\texttt{gpt-4o-mini}) by 12.3\%;
(2) \textbf{Answer Generator}: A larger model improves performance by 7.1\%, suggesting that higher answer quality provides additional information to the QA pair. 
However, it remains 11\% behind the optimized \texttt{gpt-4o-mini} in utility.
(3) \textbf{Reader Simulator}: Using a larger model (\texttt{gpt-4o}) as the reader simulator leads to mixed results, with performance gains in some subjects but a sharp decline in Chemistry. 
This suggests that larger models may introduce different reasoning strategies or evaluation biases, leading to inconsistencies in scoring. 
Additionally, since our framework is optimized for \texttt{gpt-4o-mini}, the larger model may not align well with the training dynamics. 
These results highlight the importance of consistency in simulation for reliable utility estimation.

% equipped with a larger model (\texttt{gpt-4o}) is expected to make more accurate simulation of how much generated questions really impact to the student understanding on exams.
% There is consistent increasing gap on microbiology, economics, sociology, us history while there is a huge gap in decreasing gap on chemistry, compared to smaller model (\texttt{gpt-4o-mini})



% \begin{itemize}
%     \item \textbf{Stronger question generators improve performance in some subjects but not all.} Upgrading $M_q$ from GPT-4o-mini to GPT-4o enhances performance in Microbiology and Economics but degrades US History, suggesting that question quality improvement is domain-dependent.
%     \item \textbf{Answer generator quality has a moderate effect on utility.} Stronger answer generators help in certain subjects (e.g., Economics) but have minimal impact in others (e.g., Chemistry).
%     \item \textbf{Evaluator model strength influences overall results but can introduce variability.} Using a stronger evaluator (GPT-4o) improves utility estimates in Microbiology and Economics but drastically reduces performance in Chemistry. This suggests that evaluators need to be carefully tuned to ensure robust and reliable assessments across different domains.
% \end{itemize}




\section{Conclusions}
In this paper, we propose semantic orientation, a language-grounded representation that defines object orientations via intuitive descriptors (\eg, ``plug-in direction''), bridging geometric reasoning and functional semantics. To enable this, we introduce OrienText300K, a large-scale dataset of 3D models annotated with semantic orientation. Through PointSO and the integrated \sofar system, we significantly enhance robotic manipulation capabilities, as demonstrated by strong performance in both simulated and real-world experiments.






%%%%%%%%%%%%%%%%%%%%%%%%%%%%%%%%%%%%%%%%%%%%%%%%%%%%%%%%%%%%%%%%%%%%%%%%%%%%%%%%

\addtolength{\textheight}{-15cm}   % This command serves to balance the column lengths
                                  % on the last page of the document manually. It shortens
                                  % the textheight of the last page by a suitable amount.
                                  % This command does not take effect until the next page
                                  % so it should come on the page before the last. Make
                                  % sure that you do not shorten the textheight too much.

%%%%%%%%%%%%%%%%%%%%%%%%%%%%%%%%%%%%%%%%%%%%%%%%%%%%%%%%%%%%%%%%%%%%%%%%%%%%%%%%



%%%%%%%%%%%%%%%%%%%%%%%%%%%%%%%%%%%%%%%%%%%%%%%%%%%%%%%%%%%%%%%%%%%%%%%%%%%%%%%%
% Generated by IEEEtran.bst, version: 1.14 (2015/08/26)
\begin{thebibliography}{10}
\providecommand{\url}[1]{#1}
\csname url@samestyle\endcsname
\providecommand{\newblock}{\relax}
\providecommand{\bibinfo}[2]{#2}
\providecommand{\BIBentrySTDinterwordspacing}{\spaceskip=0pt\relax}
\providecommand{\BIBentryALTinterwordstretchfactor}{4}
\providecommand{\BIBentryALTinterwordspacing}{\spaceskip=\fontdimen2\font plus
\BIBentryALTinterwordstretchfactor\fontdimen3\font minus
  \fontdimen4\font\relax}
\providecommand{\BIBforeignlanguage}[2]{{%
\expandafter\ifx\csname l@#1\endcsname\relax
\typeout{** WARNING: IEEEtran.bst: No hyphenation pattern has been}%
\typeout{** loaded for the language `#1'. Using the pattern for}%
\typeout{** the default language instead.}%
\else
\language=\csname l@#1\endcsname
\fi
#2}}
\providecommand{\BIBdecl}{\relax}
\BIBdecl

\bibitem{tian2021trajectory}
F.~Tian, R.~Zhou, Z.~Li, L.~Li, Y.~Gao, D.~Cao, and L.~Chen, ``Trajectory
  planning for autonomous mining trucks considering terrain constraints,''
  \emph{IEEE Transactions on Intelligent Vehicles}, vol.~6, no.~4, pp.
  772--786, 2021.

\bibitem{hayat2020multi}
S.~Hayat, E.~Yanmaz, C.~Bettstetter, and T.~X. Brown, ``Multi-objective drone
  path planning for search and rescue with quality-of-service requirements,''
  \emph{Autonomous Robots}, vol.~44, no.~7, pp. 1183--1198, 2020.

\bibitem{karaman2011anytime}
S.~Karaman, M.~R. Walter, A.~Perez, E.~Frazzoli, and S.~Teller, ``Anytime
  motion planning using the rrt,'' in \emph{Proceedings of the 2011 IEEE
  International Conference on Robotics and Automation (ICRA)}.\hskip 1em plus
  0.5em minus 0.4em\relax IEEE, 2011, pp. 1478--1483.

\bibitem{xu2020informed}
J.~Xu, K.~Song, D.~Zhang, H.~Dong, Y.~Yan, and Q.~Meng, ``Informed anytime fast
  marching tree for asymptotically optimal motion planning,'' \emph{IEEE
  Transactions on Industrial Electronics}, vol.~68, no.~6, pp. 5068--5077,
  2020.

\bibitem{naderi2015rt}
K.~Naderi, J.~Rajam{\"a}ki, and P.~H{\"a}m{\"a}l{\"a}inen, ``{RT-RRT*:} {a}
  real-time path planning algorithm based on {RRT},'' in \emph{Proceedings of
  the 8th ACM SIGGRAPH Conference on Motion in Games}, 2015, pp. 113--118.

\bibitem{cserna2016anytime}
B.~Cserna, M.~Bogochow, S.~Chambers, M.~Tremblay, S.~Katt, and W.~Ruml,
  ``Anytime versus real-time heuristic search for on-line planning,'' in
  \emph{Proceedings of the International Symposium on Combinatorial Search},
  vol.~7, no.~1, 2016.

\bibitem{karaman2011sampling}
S.~Karaman and E.~Frazzoli, ``Sampling-based algorithms for optimal motion
  planning,'' \emph{The International Journal of Robotics Research}, vol.~30,
  no.~7, pp. 846--894, 2011.

\bibitem{fmt}
L.~Janson, E.~Schmerling, A.~Clark, and M.~Pavone, ``Fast marching tree: A fast
  marching sampling-based method for optimal motion planning in many
  dimensions,'' \emph{The International Journal of Robotics Research}, vol.~34,
  no.~7, pp. 883--921, 2015.

\bibitem{grothe2022st}
F.~Grothe, V.~N. Hartmann, A.~Orthey, and M.~Toussaint, ``St-rrt*:
  Asymptotically-optimal bidirectional motion planning through space-time,'' in
  \emph{Proceedings of the 2022 International Conference on Robotics and
  Automation (ICRA)}, 2022, pp. 3314--3320.

\bibitem{luders2010bounds}
B.~D. Luders, S.~Karaman, E.~Frazzoli, and J.~P. How, ``Bounds on tracking
  error using closed-loop rapidly-exploring random trees,'' in
  \emph{Proceedings of the 2010 American Control Conference}.\hskip 1em plus
  0.5em minus 0.4em\relax IEEE, 2010, pp. 5406--5412.

\bibitem{chandler2017online}
B.~Chandler and M.~A. Goodrich, ``Online rrt and online fmt: Rapid replanning
  with dynamic cost,'' in \emph{Proceedings of the 2017 IEEE/RSJ International
  Conference on Intelligent Robots and Systems (IROS)}, 2017, pp. 6313--6318.

\bibitem{tong2019rrt}
B.~Tong, Q.~Liu, and C.~Dai, ``A rrt* fn based path replanning algorithm,'' in
  \emph{Proceedings of the 2019 IEEE 4th Advanced Information Technology,
  Electronic and Automation Control Conference (IAEAC)}, vol.~1.\hskip 1em plus
  0.5em minus 0.4em\relax IEEE, 2019, pp. 1435--1445.

\bibitem{sucan2012the-open-motion-planning-library}
I.~A. Sucan, M.~Moll, and L.~E. Kavraki, ``The {O}pen {M}otion {P}lanning
  {L}ibrary,'' \emph{{IEEE} Robotics \& Automation Magazine}, vol.~19, no.~4,
  pp. 72--82, December 2012.

\end{thebibliography}

\end{document}
