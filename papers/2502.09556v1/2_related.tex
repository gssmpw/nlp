\section{RELATED WORK}
\label{sec:related}
In this section, we review the literature on online and real-time sampling-based planners for dynamic environments. Although there are algorithms that plan with prior information on the trajectories of the obstacles such as \cite{grothe2022st}, we focus on the techniques that only require the current position of the dynamic obstacles because it is easier to obtain such information from sensors or communication.

The CL-RRT \cite{luders2010bounds} is an approach that integrates an online controller with RRT in order to plan in the presence of uncertainty and still satisfy bounds on tracking error. When the robot moves along the generated path, the algorithm prunes infeasible branches. However, it does not update the root of the tree as RT-RRT* does. In addition, it has been shown in \cite{naderi2015rt}, that RT-RRT* is able to find shorter paths with fewer iterations in simpler systems without motion constraints. Online versions of the RRT* (ORRT*) and FMT* (OFMT*) have been proposed by Chandler and Goodrich \cite{chandler2017online}. Their approach is similar to RT-RRT* in the sense that multiple goals can be assigned without having to re-plan from scratch, but they apply a different rewiring approach. While RT-RRT* rewires around obstacles by checking for collisions, ORRT* and OFMT* apply a time-varying cost that increases near obstacles, causing nodes in the tree to prefer connections far from the obstacles. Another difference between these algorithms is the local path generation process that is part of RT-RRT*, which is not present in ORRT* and OFMT*. More recently, Tong et al.\ \cite{tong2019rrt} proposed RRT*FN-Replan, which is an online algorithm that maintains a fixed number of nodes in the tree and that takes advantage of previously generated tree branches to update the current plan. They have shown that their approach outperforms not only RT-RRT* but also ORRT*. 

Based on the current state of the art, we decided to implement RT-FMT and compare it against RT-RRT* for a few reasons. First, it has been shown to outperform CL-RRT. ORRT* and OFMT* do not include local path generation, and the obstacle avoidance module (based on cost) guarantees safe execution at the expense of higher cost trajectories. Even though it has been shown that RRT*FN-Replan outperforms RT-RRT*, these experiments involved testing only the rewire and re-plan processes, disregarding the first stage of tree expansion. Because the arrival time is directly affected as well by the tree expansion, it was deemed important to also consider this information.