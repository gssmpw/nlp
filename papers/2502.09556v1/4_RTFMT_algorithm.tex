\section{THE REAL-TIME FAST MARCHING TREE ALGORITHM}
\label{sec:algorithm}
In general, RT-FMT works by expanding a tree, similarly to how FMT* does, while it also checks for dynamic obstacles and rewires the tree around them. During the search, the algorithm also searches for the local paths with the least costs. These paths are used for the robot to start moving before planning is finished. When the robot reaches a new waypoint in the path, the root of the tree is updated. This event triggers a complete rewire of the tree to update the costs of all nodes. In a real-time application, the robot sends velocity commands at specific intervals. Therefore, we only allow the tree expansion and rewiring to run for a defined number of iterations ($N_e$) to expand and rewire the tree. This method can also be easily adapted to run for a desired time interval or planning frequency. 

The method is shown in Algorithm \ref{alg:main}, which starts by sampling $N$ configurations in free space (Line \ref{alg:samplefree}) considering only the fixed obstacles. This function also calculates the neighborhood radius $r_n$ based on the number of samples and the dimensionality of the problem according to
\begin{equation}
\label{eq:rn}
    r_n = \gamma_s 2\bigg(1 +\frac{1}{d}\bigg)^{\frac{1}{d}}\bigg(\frac{\mu(\mathcal{X}_{\rm free})}{\zeta_d} \bigg)^\frac{1}{d} \bigg(\frac{\log(N)}{N} \bigg)^\frac{1}{d},
\end{equation}
where $\gamma_s > 1$ is a tuning parameter, $d$ is the dimension of the problem, $\mu(\mathcal{X}_{\rm free})$ is the Lebesgue measure of the free space, and $\zeta_d$ is the volume of a unit ball in $\mathbb{R}^d$. Although FMT* provides an equation for $r_n$, we use the equation defined in \cite{karaman2011sampling} for PRM* since it computes an $r_n$ slightly bigger than the FMT* equation for $r_n$.

\begin{algorithm}[tb]
\SetKwFunction{SampleFree}{SampleFree}
\SetKwFunction{UpdateContext}{UpdateContext}
\SetKwFunction{ExpandFMT}{ExpandFMT}
\SetKwFunction{RewireFromObstacles}{RewireFromObstacles}
\SetKwFunction{RewireFromRoot}{RewireFromRoot}
\SetKwFunction{GeneratePath}{GeneratePath}
\SetKwFunction{UpdateRoot}{UpdateRoot}

 $\mathcal{T} \leftarrow \bm x_{s}$;  $\bm x_{\rm root} \leftarrow \bm x_{s}$\;
 $\mathcal{S} \leftarrow \SampleFree(N) \cup \bm x_{s} \cup \bm x_{g}$\; \label{alg:samplefree}
 $\mathcal{V}_{b} \leftarrow \emptyset$; $\mathcal{Q}_{o} \leftarrow \emptyset$;  $\mathcal{Q}_{r} \leftarrow \emptyset$\;
 $\mathcal{V}_{unv} \leftarrow \mathcal{S}\backslash\{\bm x_{s}\}$; $\mathcal{V}_{\rm open} \leftarrow \{\bm x_{s}\}$; $\mathcal{V}_{\rm closed} \leftarrow \emptyset$\;
 $\bm z \leftarrow \bm x_{s}$\;
 \While{True}{
 ($\bm x_{\rm robot} ,\bm x_{g}, \mathcal{N}_{b}) \leftarrow \UpdateContext(\mathcal{T}, \mathcal{X}_{\rm Dobs})$\; \label{alg:updateContext}
 \For{$i=1$ \KwTo $N_{e}$}{
  $\ExpandFMT(\mathcal{T})$\; \label{alg:expandAndRewire}
  $\RewireFromObstacles(\mathcal{T})$\; \label{alg:rewireObs}
  $\RewireFromRoot(\mathcal{T})$\; \label{alg:rewireRoot}
}
 $(\bm x_{\rm root}, \bm x_1, ..., \bm x_k) \leftarrow \GeneratePath(\mathcal{T}, \bm x_{\rm root})$\; \label{alg:generatepath}
 \If{$\bm x_{\rm robot}$ \upshape{is near} $\bm x_{\rm root}$}{
    $\bm x_{\rm root}\leftarrow \bm x_{1}$\;
    $\UpdateRoot(\mathcal{T},\bm x_{\rm root})$\; \label{alg:updateroot}
 }
 Steer robot towards  $\bm x_{\rm root}$\; \label{alg:steer}
 Perform other tasks\;
}
\caption{RT-FMT($\bm x_{s}, \bm x_{g}, \mathcal{X}_{Fobs}, \mathcal{X}_{\rm Dobs}, N_s, N_e $)}
\label{alg:main}
\end{algorithm}

Inside the infinite loop, the algorithm updates the context of the problem by returning the current position of the robot and goal $x_{g}$, and by finding the nodes in the tree that have been blocked or unblocked by the dynamic obstacles. When a configuration $\bm x_q$ in the tree is blocked, its cost $\textup{c}(\bm x_q)$ is set to infinity. When a node is unblocked, its cost is updated to
\begin{equation}
    \textup{c}(\bm x_q) = \textup{c}(\bm x_{\rm parent}) + \textup{Cost}(\bm x_{\rm parent}, \bm x_{q}), \label{eq:cost}
\end{equation}
where $\bm x_{\rm parent}$ is the configuration of the parent of $\bm x_q$, and
\begin{equation}
    \textup{Cost}(\bm u, \bm v) = ||\bm v- \bm u||.
\end{equation}
If the updated node has children, its children's costs are recursively updated. Lines \ref{alg:expandAndRewire}\textendash\ref{alg:rewireRoot} expand the tree according to Algorithm \ref{alg:ExpandFMT}, and rewire the tree based on Algorithm \ref{alg:RewireFromObstacle}. Line \ref{alg:generatepath} returns a global path if $\bm x_g$ is in the tree or a local path otherwise. The local path is found by computing 
\begin{equation}
    \bm x_k \leftarrow \arg\min_{\bm x \in \mathcal{T}}\textup{c}(\bm x) +  ||\bm x -\bm x_{goal}||,
\end{equation}
and then a path starting at $x_{\rm root}$ and ending at $\bm x_k$ is generated. While we do not limit $k$, RT-RRT* generates a path up to a specific $k$. More details on the local path generation can be found in Algorithm 6 in Naderi et al.~\cite{naderi2015rt}. The approach resembles the A* search. 

In Line \ref{alg:updateroot}, the root of the tree is set to the next configuration in the path, which is always the second element since the path always starts at the old root.  Finally, the algorithm steers the robot towards the new root of the tree on Line \ref{alg:steer}. If there are other tasks to perform such as mapping, and localization, they can be called in the main function as well. 


\subsection{Expanding the Tree}

The tree expansion is described  in Algorithm \ref{alg:ExpandFMT}. Most of the algorithm (Lines \ref{alg:x_near}\textendash\ref{alg:findz}) was inspired by FMT*, proposed in \cite{fmt}, but our implementation has two major differences. 

First, the loops in the original implementation were substituted for conditional statements. These statements ensure that only one node can be added in the tree per call. If multiple nodes are added to the tree at once, there is a possibility of delaying other tasks since the processor will spend too much time expanding many nodes at once. In addition, Line \ref{alg:checkdynamic} not only checks for fixed obstacles but also checks whether $\bm y_{\rm min}$ is not being blocked by a dynamic obstacle according to (\ref{eq:cost}). 

Second, in the original approach, once a node has checked all possible connections with its neighbors, it is closed and never checked again. In our approach, we do not spend time expanding the nodes that are inside $\mathcal{X}_{\rm Dobs}$. As a consequence, when a dynamic obstacle moves, there will be unvisited nodes around closed nodes. To add these nodes to the tree, our algorithm must be able to reopen closed nodes nearby. This is done by adding all $\bm z$ that are near unvisited nodes at closing time to $\mathcal{V}_{\rm toOpen}$ (Line \ref{alg:ztoopen}). Then, when the regular expansion is finished (Line \ref{alg:znull}), the algorithm reopens these nodes (Line \ref{alg:reopen}) to continue the expansion in case a dynamic obstacle moves.

\begin{algorithm}[tb]
\SetKwFunction{Near}{Near}
\SetKwFunction{PopLast}{PopLast}
\SetKwFunction{Cost}{Cost}
\SetKwFunction{c}{c}
\SetKwFunction{CollisionFree}{CollisionFree}
\SetKwFunction{Open}{Open}
\SetKwFunction{Close}{Close}
\lIf{$\mathcal{X}_{near}$ = $\emptyset$ $\&$ $\bm z \neq \emptyset$}{$\mathcal{X}_{near} \leftarrow$ \Near($\bm z$,$\mathcal{V}_{unv}$)} \label{alg:x_near}
\Else{
$\bm x = \PopLast(\mathcal{X}_{near})$\; 
$\mathcal{Y}_{near} \leftarrow$ \Near($\bm x$,$\mathcal{V}_{\rm open}$)\;
\bm $y_{\rm min} \leftarrow \arg\min_{\bm y \in \mathcal{Y}_{near} }(\c(\bm y) + \Cost(\bm x, \bm y))$\;

\If{$\CollisionFree(\bm y_{\rm min}, \bm x) \& \c(\bm y_{\rm min}) < \infty$}{ \label{alg:checkdynamic}
$\mathcal{V}_{open, new} \leftarrow  \mathcal{V}_{open, new} \cup \{\bm x\}$\;
$\mathcal{V}_{unv} \leftarrow \mathcal{V}_{unv}\backslash\{\bm x\}$\;
$\c(\bm x) \leftarrow \c(\bm y_{\rm min}) + \Cost(\bm y_{\rm min}, \bm x)$\;
$\mathcal{T} \leftarrow \mathcal{T} \cup \{\bm x, (\bm y_{\rm min}, \bm x) \}$
}
\If{$\mathcal{X}_{near}$ = $\emptyset$ $\&$ $\bm z \neq \emptyset$}{
\Close($\bm z$)\;
$\mathcal{Z}_{near} \leftarrow$ \Near($\bm z$,$\mathcal{V}_{unv}$)\;
$\bm z \leftarrow \arg\min_{\bm y \in \mathcal{V}_{\rm open} }\c(\bm y)$\; \label{alg:findz}
\ForEach{$\bm x \in \mathcal{Z}_{near}$ }{ \label{alg:ztoopen}
\lIf{\CollisionFree($\bm z, \bm x$)}{$\mathcal{V}_{\rm toOpen} \leftarrow \mathcal{V}_{\rm toOpen} \cup \bm z  $} 
}

}
\If{$\bm z = \emptyset$}{ \label{alg:znull}
$\Open(\mathcal{V}_{\rm toOpen})$;~$\mathcal{V}_{\rm toOpen} = \emptyset$\; \label{alg:reopen}
$\bm z \leftarrow \arg\min_{\bm y \in \mathcal{V}_{\rm open} }\c(\bm y)$\; }
}

\SetKwProg{Def}{def}{:}{}
\Def{\Close($\mathcal{V}$)}{
$\mathcal{V}_{\rm open} \leftarrow  \mathcal{V}_{\rm open} \cup \mathcal{V}_{open, new}\backslash\mathcal{V}$\;
$\mathcal{V}_{\rm closed} \leftarrow  \mathcal{V}_{\rm closed} \cup \mathcal{V}$\;
}

\SetKwProg{Def}{def}{:}{}
\Def{\Open($\mathcal{V}$)}{
$\mathcal{V}_{\rm open} \leftarrow  \mathcal{V}_{\rm open} \cup  \mathcal{V}$\;
$\mathcal{V}_{\rm closed} \leftarrow  \mathcal{V}_{\rm closed} \backslash \mathcal{V}$\;
}
\caption{ExpandFMT($\mathcal{T}$)}
\label{alg:ExpandFMT}
\end{algorithm}

\subsection{Rewiring the Tree}

As the dynamic obstacles move around the environment, the tree nodes are constantly being blocked and unblocked by Line \ref{alg:updateContext} in Algorithm \ref{alg:main}. When a node is blocked or unblocked, its cost is changed and all its children are recursively updated. The task of the {\tt RewireFromObstacle} method in Algorithm \ref{alg:RewireFromObstacle} is to find the connections with lower cost in the neighborhood of nodes that have recently been blocked or unblocked. This function only rewires the nodes that have recently been affected by a dynamic obstacle. The rewiring process starts by adding all blocked nodes to $\mathcal{Q}_o$. Then, nodes are iteratively removed from the list and the algorithm tries to find parents nearby with a lower cost. If there is a connection with a lower cost that is also collision-free, the children of the updated nodes are also added to $\mathcal{Q}_o$. 

\begin{algorithm}[tb]
\SetKwFunction{PopFirst}{PopFirst}
\SetKwFunction{UpdateParentChild}{UpdateParentChild}
\SetKwFunction{RecalculateChildrenCost}{RecalculateChildrenCost}
\lIf{$\mathcal{Q}_o$ = $\emptyset$}{$\mathcal{Q}_o \leftarrow \mathcal{N}_{b}$} \label{alg:addblockedtolist}
\Else{
$\bm x_b \leftarrow \PopFirst(\mathcal{Q}_o)$\;
   \If{$\bm x_b \notin \mathcal{X}_{\rm Dobs}$}{
        $\mathcal{Y}_{near} \leftarrow \Near(\bm x_b,\mathcal{V}_{\rm open}\cup \mathcal{V}_{\rm closed}$)\;
        \bm $y_{\rm min} \leftarrow \arg\min_{\bm y \in \mathcal{Y}_{near} }(\c(\bm y) + \Cost(\bm x_b, \bm y))$\;
        \If{$\CollisionFree(\bm y_{\rm min}, \bm x) \& \c(\bm y_{\rm min}) < \infty$}{
        $\UpdateParentChild(\mathcal{T}, \bm y_{\rm min}, \bm x_b)$\;
        $\RecalculateChildrenCost(\bm x_b)$
        }
   }

}
\caption{RewireFromObstacles($\mathcal{T}$)}
\label{alg:RewireFromObstacle}
\end{algorithm}

As the robot moves around the environment, the tree root is updated by Algorithm \ref{alg:main}, Line \ref{alg:updateroot}. When this happens, the algorithm triggers the {\tt RewireFromRoot} function (Line \ref{alg:rewireRoot}) to rewire all nodes in the tree by inserting the new root into $\mathcal{Q}_{r}$. The {\tt RewireFromRoot} function is very similar to Algorithm \ref{alg:RewireFromObstacle}. The algorithm removes a node from $\mathcal{Q}_{r}$ and tries to find better connections in the tree. When a new connection is made, the children of the updated node are also added to $\mathcal{Q}_{r}$. This causes a chain reaction that starts from the root and updates all nodes in the tree. This method has also been implemented without any loops to only update a single node per call. Our implementation triggers this chain reaction whenever the root is updated. However, this can easily be modified to happen at a desired frequency.


%\subsection{Real-time considerations}


