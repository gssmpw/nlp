\section{EVALUATION METHODOLOGY}
\label{sec:methodology}
The main goal of our evaluation process was to study the advantages of real-time planning and execution using FMT over RRT in certain environments, as well as to verify whether real-time execution and planning can result in faster arrival times. To properly achieve our evaluation goals, we implemented and tested our algorithm in Unity 3D using a similar environment to the one provided by \cite{naderi2015rt}. Then, we prepared two different scenarios. A Maze space, shown in Fig.~\ref{fig:maze}, which was designed as a toy problem to highlight one shortcoming of RRT-based algorithms in maze-like spaces. This shortcoming is caused by the lower probability of successfully connecting a node to the tree without hitting a wall. The second environment is a Mine-like space illustrated in Fig.~\ref{fig:mine}, as an application problem that was designed following the room and pillar method of creating underground mines.

\begin{figure}
\centerline{
\subfloat[RT-FMT]{
\includegraphics[width=0.4\linewidth]{Figures/maze_info_rtfmt_500.eps}
\label{fig:maze_a}}\\
\subfloat[RT-RRT*]{
\includegraphics[width=0.4\linewidth]{Figures/maze_rtrrt_500_up.eps}
\label{fig:maze_b}}
}
\caption{Simulation environment with the Maze space for (a) RT-FMT and (b) RT-RRT*. The parameter $r_o$ represents the sensing range of the robot, and the distance $r_b$ represents the safety radius in which tree nodes are considered blocked if a dynamic obstacle is within $r_o$ from the robot.}
\label{fig:maze}
\end{figure}

The experiments in the Maze space were executed with the robot traveling at 2 m$/$s, with $r_b = 2$ m, $r_o = 10$ m, and a robot radius of 0.5 m. For the Mine space, the robot traveled at 4 m$/$s, with $r_b = 14$ m, $r_o = 50$ m, and a robot radius of 1.5 m. For both environments, the dynamic obstacles traveled at half the robot's speed. All remaining parameters of RT-RRT* were kept the same as provided by \cite{naderi2015rt}. F or example, the number of tentative expansions per action ($N_e$) was set to 32. For RT-FMT, we also maintained  $N_e = 32$, and $\gamma_s$, defined in \eqref{eq:rn}, was kept at 1.1 for all experiments.

We split the evaluation methodology into three experiments with each one repeated 50 times for a different number of sample attempts. They ranged from 500 to 4500 samples with an increment of 1000 samples. It is important to emphasize that sample attempts is not the number of nodes in the tree. For RT-FMT, it indicates the free space sample count, and for RT-RRT*, the sample attempts represent the number of times the algorithm sampled a point and attempted to expand the tree. Since both approaches are very different in building the tree, care must be taken when analyzing the results. This same approach was applied by \cite{fmt} to compare FMT* against RRT*.


Experiment 1 involved obtaining the planning times, executed costs, and arrival times by running the planners in a non real-time manner; i.e., allowing the robot to move only after trying the defined number of samples and finding the complete path. These metrics were used as a baseline to judge the performance of the real-time execution for Experiments 2 and 3. Experiment 2 allowed the robots to execute the local paths in real-time without the presence of dynamic obstacles. With this experiment, it is possible to evaluate whether the robots can arrive earlier at the goal, considering that there is a risk of finding a costlier trajectory when compared to Experiment 1. Experiment 3 is similar to 2, but with dynamic obstacles present in the space. In the Maze space, the dynamic obstacles maintained a fixed initial position and they moved in random directions. In the Mine space, their vertical positions were randomly selected at the start of the simulation, and they would move down if they appeared in the top part of the environment, and vice-versa.

The RT-FMT implementation used for the evaluation procedure is available at \url{github.com/offroad-robotics/rt-fmt-icra.git}.



