\section{SIMULATED EXPERIMENTS}
\label{sec:results}

This section presents the results obtained from the simulations of the different experiments and discusses the results. During the experiments, both algorithms reached quickly 100~\% success rate on both spaces during experiments 1 and 2 when increasing the number of sample attempts. During Experiment 3, which is the more challenging one, only RT-FMT reached 100~\% at 4500 samples in the Maze space, while RT-RRT* achieved only 46~\% at the same number of samples. For the Mine space, RT-FMT achieved 84~\% and RT-RRT only 50~\% success rate. The low success rate of RT-RRT* is caused by node connections with longer edges than RT-FMT, as shown in Fig. \ref{fig:maze}. 

During the execution of both algorithms, the robot does not react to changes in the environment when it is moving in between nodes. While it is possible to limit the edge lengths of RT-RRT* to improve the success rate of the algorithm, it worsens other important characteristics such as planning time and executed cost that are analyzed in this section. For such reasons, we did not modify this behavior in RT-RRT*.


%both algorithms reached a 100~\% success rate on both spaces for experiments 1 and 2. However, in Experiment 3, only RT-FMT reached 100~\% at 4500 samples in the Maze space, while RT-RRT* achieved only 46~\% at the same number of samples. For the Mine space, RT-FMT achieved 84~\% and RT-RRT only 50~\% success rate. The low success rate of RT-RRT* was related to the high probability of the node expansion ending up with a collision.

\subsection{Experiment 1}

For this experiment, we show the solution cost for all runs in Fig.~\ref{fig:experiment_1}. It is clear that RT-FMT outperforms RT-RRT* in planning time and solution cost for both environments. The solution was found faster with the blue curve located to the left side of the graph (shorter planning time) and below the orange line (shorter executed path). Also, the solution found by the RT-FMT yielded a shorter path for the same sample count. This is because, for the same number of sample count (i.e., attempts to connect points to the tree), RT-FMT will have many more nodes added to the tree than RT-RRT*, which is a direct consequence of how the samples of RRT* and FMT* are computed.

Furthermore, the planning time is much more predictable in RT-FMT than RT-RRT* due to its smaller deviation, which indicates that RT-FMT is better suited for real-time applications due to the hard real-time bounds that are present in these systems~\cite{cserna2016anytime}.


%Furthermore, the planning time is much more predictable in RT-FMT than RT-RRT* due to its smaller deviation, which indicates that RT-FMT is better suited for applications with hard real-time bounds~\cite{cserna2016anytime}.
\begin{figure}
\centering
\subfloat[Maze space]{
\label{fig:experiment_1_a}
\includegraphics[width=\figsize\linewidth]{Figures/maze_solution_cost_planned_time_exp_1.eps}
}
\subfloat[Mine space]{
\label{fig:experiment_b}
\includegraphics[width=\figsize\linewidth]{Figures/mine_solution_cost_planned_time_exp_1.eps}
}
\caption{Simulation results for Experiment 1. Hollow squares represent a single simulation, solid circles the average on both axes for different sample counts, and the ellipses 90~\% confidence level.}
\label{fig:experiment_1}
\end{figure}

\subsection{Experiment 2}

For the results of this experiment and Experiment 3, instead of displaying the planning time as in Fig.~\ref{fig:experiment_1}, we display the sample count in the $x$-axis because it is fixed for each simulation group. The results of this experiment are shown in Fig.~\ref{fig:maze_2} and Fig.~\ref{fig:mine_2} for the Maze and Mine spaces, respectively. From these images, we can draw a few relevant conclusions. First, the executed cost and arrival time of RT-FMT was also smaller than RT-RRT* for both environments. And, interestingly, RT-FMT was able to match the ideal cost (from Experiment 1) almost perfectly, as can be seen in Fig.~\ref{fig:maze_2_a} and Fig.~\ref{fig:mine_2_a}. RT-RRT* also performed relatively well.

Another relevant observation to mention in Fig.~\ref{fig:maze_2_b} and Fig.~\ref{fig:mine_2_b} is the fact that the arrival time for Experiment 1 starts to monotonically increase after 1500 samples. This is caused by the extra time the algorithm takes to expand the tree with more samples. As a consequence of this fact and of both algorithms being able to closely match the executed costs to Experiment 1, both approaches had smaller arrival times for higher sample counts (dashed lines in  Fig.~\ref{fig:maze_2_b} and Fig.~\ref{fig:mine_2_b}). This result indicates that it is indeed worthwhile risking taking a sub-optimal path, in the beginning, to save planning time since the difference in executed cost was small for RT-RRT* and negligible for RT-FMT.
\begin{figure}
\centering
\subfloat[]{
\label{fig:maze_2_a}
\includegraphics[width=\figsize\linewidth]{Figures/maze_cost_exp_2.eps}
}
\subfloat[]{
\label{fig:maze_2_b}
\includegraphics[width=\figsize\linewidth]{Figures/maze_arrival_time_exp_2.eps}
}
\caption{Simulation results for the Maze in Experiment 2.}
\label{fig:maze_2}
\centering
\subfloat[]{
\label{fig:mine_2_a}
\includegraphics[width=\figsize\linewidth]{Figures/mine_cost_exp_2.eps}
}
\subfloat[]{
\label{fig:mine_2_b}
\includegraphics[width=\figsize\linewidth]{Figures/mine_arrival_time_exp_2.eps}
}
\caption{Simulation results for the Mine in Experiment 2.}
\label{fig:mine_2}
\end{figure}

\subsection{Experiment 3}

After adding the dynamic obstacles, the cost of the executed path, unsurprisingly, increased with respect to the execution without obstacles (Experiment 2). This is shown in both  Fig.~\ref{fig:maze_3} and Fig.~\ref{fig:mine_3}. However, that was the only common characteristic between the results of the Maze and Mine spaces for Experiment 3. 

If we focus on the arrival time of the maze space for RT-FMT (Fig.~\ref{fig:maze_3_b}), we see that as the number of samples increased, the difference in arrival time of Experiment 3 and Experiment 1 became smaller up to the point that the real-time execution arrived faster at the goal state, even with the extra task of avoiding dynamic obstacles. This is an excellent result since it is very common to select a very high number of nodes to guarantee that a solution will be found.

Regarding the Mine space, the big difference in executed cost of Experiment 3 and Experiment 1 in Fig.~\ref{fig:mine_3} can be explained by looking at the layout of the mine in Fig.~\ref{fig:mine}. When a dynamic obstacle blocks the robot in the middle of a hallway, the robot has to go around the block to avoid it, which greatly increases the executed cost. That is also the reason why RT-FMT and RT-RRT* performed similarly, they both had to go around the same number of blocks, on average, to avoid the obstacles.

Another behavior that is important to discuss is the upward trend of the arrival cost as the number of samples increased in Fig.~\ref{fig:mine_3_b} for RT-FMT. This was a consequence of the Mine space layout, the sample count definition in Section \ref{sec:methodology}, and how RT-FMT and RT-RRT* sample points in the environment. The Mine space contains more occupied space than free space, which caused the RT-FMT samples to be densely packed since they are guaranteed to lie in the small free space. In contrast, most sampled points for RT-RRT* failed to connect to the tree because of the likelihood of collisions. As a result, RT-FMT had many more nodes in the tree than RT-RRT*. With many nodes densely packed, the rewiring process of RT-FMT became too slow to respond to the dynamic obstacles in advance, causing the robot to react to dynamic obstacles as if it was short-sighted, which, consequently, also forced the robot to take longer routes. The same effect would have happened to RT-RRT* if the tree had a similar number of nodes to RT-FMT. Still, RT-FMT had an 84~\% success rate while RT-RRT* had only 50~\%.

\begin{figure}
\centering
\subfloat[]{
\label{fig:maze_3_a}
\includegraphics[width=\figsize\linewidth]{Figures/maze_cost_exp_3.eps}
}
\subfloat[]{
\label{fig:maze_3_b}
\includegraphics[width=\figsize\linewidth]{Figures/maze_arrival_time_exp_3.eps}
}
\caption{Simulation results for the Maze in Experiment 3.}
\label{fig:maze_3}
\centering
\subfloat[]{
\label{fig:mine_3_a}
\includegraphics[width=\figsize\linewidth]{Figures/mine_cost_exp_3.eps}
}
\subfloat[]{
\label{fig:mine_3_b}
\includegraphics[width=\figsize\linewidth]{Figures/mine_arrival_time_exp_3.eps}
}
\caption{Simulation results for the Mine in Experiment 3.}
\label{fig:mine_3}
\end{figure}

