% This is samplepaper.tex, a sample chapter demonstrating the
% LLNCS macro package for Springer Computer Science proceedings;

\documentclass[runningheads]{llncs}
\usepackage{framed}
\usepackage{mdframed}
\usepackage{xcolor}
\usepackage{svg}
\usepackage{booktabs}
\usepackage{enumitem}
%\usepackage{algorithmicx}
\usepackage{algpseudocode}
\usepackage{multirow}
\usepackage{url}
\usepackage{framed}
\usepackage{graphicx}
\usepackage{caption}
\usepackage{subcaption}
\usepackage{balance}
\usepackage{makecell}
\usepackage[T1]{fontenc}
% T1 fonts will be used to generate the final print and online PDFs,
% so please use T1 fonts in your manuscript whenever possible.
% Other font encondings may result in incorrect characters.
%
\usepackage{graphicx}
% Used for displaying a sample figure. If possible, figure files should
% be included in EPS format.
%
% If you use the hyperref package, please uncomment the following two lines
% to display URLs in blue roman font according to Springer's eBook style:
%\usepackage{color}
%\renewcommand\UrlFont{\color{blue}\rmfamily}
%\urlstyle{rm}
%
\usepackage{marvosym}
\usepackage[colorlinks,
            linkcolor=blue,
            anchorcolor=black,
            citecolor=blue]{hyperref}

\definecolor{shadecolor}{gray}{0.9}

\newcommand{\sq}[1]{\textcolor{cyan}{\textbf{Sihang: #1}}}

\begin{document}
%
\title{AutoS$^2$earch: Unlocking the Reasoning Potential of Large Models for Web-based Source Search}
%
\titlerunning{Large Models for Web-based Source Search}
% If the paper title is too long for the running head, you can set
% an abbreviated paper title here
%
% \author{Anonymous Author}
\author{Zhengqiu Zhu\inst{1} \and
Yatai Ji\inst{1} \and
Jiaheng Huang\inst{1} \and
Yong Zhao\inst{1} \and
Sihang Qiu\inst{1} \and 
Rusheng Ju\inst{1}
}
%
\authorrunning{Z. ZHU et al.}
% First names are abbreviated in the running head.
% If there are more than two authors, 'et al.' is used.
%
\institute{National University of Defense Technology, Changsha, China \email{zhuzhengqiu@nudt.edu.cn,jiyatai\_1209@nudt.edu.cn,12254745966@qq.com,\ zhaoyong15@nudt.edu.cn,sihangq@acm.org,jrscy@sina.com}}
%
\maketitle              % typeset the header of the contribution
%
\begin{abstract}

Web-based management systems have been widely used in risk control and industrial safety. However, effectively integrating source search capabilities into these systems, to enable decision-makers to locate and address the hazard (e.g., gas leak detection) remains a challenge. While prior efforts have explored using web crowdsourcing and AI algorithms for source search decision support, these approaches suffer from overheads in recruiting human participants and slow response times in time-sensitive situations.
%Web crowdsourcing has emerged as a transformative paradigm for tackling critical source search issues in complex environments, spanning applications from industrial safety (e.g., gas leak detection) to ecological monitoring (e.g., biological signal tracking). By integrating human intelligence, it effectively overcomes purely algorithmic limitations such as local optima and deadlocks. However, this approach also entails substantial costs and places considerable burdens on human workers. 
To address this, we introduce AutoS$^2$earch, a novel framework leveraging large models for zero-shot source search in web applications. AutoS$^2$earch operates on a simplified visual environment projected through a web-based display, utilizing a chain-of-thought prompt designed to emulate human reasoning. The multi-modal large language model (MLLMs) dynamically converts visual observations into language descriptions, enabling the LLM to perform linguistic reasoning on four directional choices. Extensive experiments demonstrate that AutoS$^2$earch achieves performance nearly equivalent to human-AI collaborative source search while eliminating dependency on crowdsourced labor. Our work offers valuable insights in using web engineering to design such autonomous systems in other industrial applications.






\keywords{Web Crowdsourcing \and Source Search \and Multi-modal Large Language Model \and Human-AI Collaboration.}
\end{abstract}


\section{Introduction}\label{sec:introduction}
% -- Outline
% ---- LLMs are popular
% ---- There're many stakeholders in the training and inference loop
% ---- Adversaries in the training loop are a problem -- malpractice, poisoning
% ---- Also, showing compliance
% ---- Need a framework to prove the integrity of the pipeline
% ---- Enter Atlas

% ---- LLMs are popular
In recent years, machine learning (ML) models, have become increasingly popular.
The pervasive use of large language models (LLMs), in particular, and multi-stakeholder
involvement in model creation and deployment exacerbate security and privacy risks.
These considerations are emphasized by the global nature and the complexity of
large-scale ML deployments with different lifecycle stages:
%gathering and sanitizing the data from different sources,
%training and inferencing across many data centers,
%compliance with local laws or corporate policies.

% ---- There're many stakeholders in the training and inference loop
%Additionally, different stages of the ML development pipeline come with their own stakeholders:
\begin{enumerate}[label=\arabic*)]
    \item Collection and sanitation of a \emph{training} dataset from several public and proprietary sources.
    %\item Solicitation and facilitation of training.
    \item Provisioning of the training environment (hardware and software).
    \item Execution of training across many data centers.
    \item Construction of a \emph{testing} dataset from several sources, and the evaluation.
    \item Deployment and use of the model for inference that is compliant with local laws or corporate policies.
    %\item Use of the model in compliance with local laws or corporate policies.
\end{enumerate}

% ---- Adversaries in the training loop are a problem -- malpractice, poisoning
Each of these stages is vulnerable to malicious or dishonest parties.
For example, data can be poisoned~\cite{biggio2012poisoning,carlini2024poisoning} during collection or training.
Service providers executing outsourced training can shorten or omit critical steps to reduce their cost.
Model providers can serve smaller models in SaaS, or even distribute malicious ones.

% ---- Also, showing compliance
On the other hand, responsible model builders and other stakeholders may be incentivised or required to provide security and trust guarantees.
They may want to prove low bias in their training data, offer easily verifiable performance claims, or guarantee end-to-end integrity of the model creation in high risk domains.

% ---- Need a framework to prove the integrity of the pipeline
To address these challenges, it is necessary to guarantee the integrity of the entire ML lifecycle --
beginning with the data, through the training, and finally, the evaluation and deployment.
Was the data modified?
Did the hardware and software environment adhere to the specification?
Did the contractor follow the specified training procedure?
Can I trust the evaluation?
How can I guarantee that I am interacting with the intended model?
These are example questions that showcase the breadth of the involved challenges that must be tackled to provide end-to-end security.

% --- Enter Atlas
In this work, we introduce \atlas, a framework for enhancing the security and transparency of the lifecycle of ML models.
\atlas establishes the baseline of fundamental components and capabilities needed for comprehensive provenance tracking
at each stage of the ML lifecycle.
Subsequently, \atlas defines the core integrity requirements for verifiable ML lifecycle transparency.
We provide a reference implementation that instantiates \atlas using hardware-based security mechanisms -- with trusted execution environment (TEE),
including attestations.% , and comprehensive metadata-based provenance tracking.
%Our implementation satisfies all \atlas requirements.

We claim the following contributions:
\begin{enumerate}[label=\arabic*.]\label{sec:introduction:contributions}
    \item We introduce \atlas, a framework designed for end-to-end ML lifecycle transparency.
    \item We instantiate \atlas using TEEs and metadata-based provenance tracking.
    \item We evaluate our \atlas prototype through two case studies:
        \begin{enumerate*}[label=\arabic*)]
            \item fine-tuning of a BERT model~\cite{lin2023metabert, lin2023metabertimpl};
            \item fine-tuning of a bge-reranker model~\cite{chen2023bge}
        \end{enumerate*}.
\end{enumerate}

%\msm{revise: Integrate this motivation into intro}
%Organizations frequently leverage pre-trained models, outsource training processes, and integrate components from multiple sources,
%making it difficult to verify the authenticity and trustworthiness of their ML systems. This complexity is further compounded
%by the potential for malicious modifications at various stages of the model lifecycle, from data preparation through deployment.
%The involvement of various third parties in ML model development and deployment
%creates critical challenges in ensuring supply chain integrity.
%
%While Software Bills of Materials (SBOMs) and AI Bills of Materials (AI BOMs) provide basic inventory tracking for model components,
%they fall short in addressing the dynamic nature of ML pipelines. These approaches typically offer point-in-time snapshots but
%fail to capture the complex transformations, fine-tuning operations, and runtime modifications that characterize modern ML workflows.
%Additionally, they lack cryptographic guarantees about the integrity of recorded information and cannot effectively track the provenance
% of model weights and training data.
%
% These approaches demonstrate the growing importance of ML supply chain security.
% However, they are typically applied in an ad-hoc fashion, highlighting the need
% for a more integrated approach that combines comprehensive lineage tracking,
% strong cryptographic properties, and practical integration capabilities with existing ML development and deployment pipelines.
%
%A comprehensive solution requires not just documentation of components, but verifiable evidence of their origins,
%transformations, and integrity throughout the entire model lifecycle. This need has driven interest in more robust
%provenance tracking mechanisms that can:
%
%\begin{itemize}
%\item Provide cryptographic proof of model lineage
%\item Track and verify all pipeline transformations
%\item Maintain tamper-evident records of training processes
%\item Ensure integrity of model artifacts across organizational boundaries
%\end{itemize}
%
%Several existing tools and frameworks
%commonly focusing on different components of the model lifecycle and provenance tracking.
%While these solutions offer valuable capabilities, they often address only specific parts of the end-to-end ML
%supply chain rather than providing comprehensive coverage.
%\msm{end-revise}
%
%\todo{add discussion of EU-CRA AI Act requirements for model documentation and FDA guidelines for AI/ML in healthcare}

%The remainder of this paper is organized as follows:
%in Section~\ref{sec:background-related} we provide an overview of the necessary background, and the related work;
%Section~\ref{sec:problem} presents the challenge of providing integrity in the ML pipeline, the threat model, and the system assumptions;
%in Section~\ref{sec:framework} we present \atlas -- our framework for providing ML integrity;
%Section~\ref{sec:implementation} covers implementation details;
%in Section~\ref{sec:eval}, we show that \atlas is effective across three dimensions: training overhead $<8\%$, the verification time increases linearly with the size of the model, and it is compatible with PyTorch and Tensorflow;
%in Section~\ref{sec:casestudies} we present the case studies;
%in Section~\ref{sec:discussion} we discuss additional considerations for \atlas,
%and Section~\ref{sec:conclusion} concludes the paper and provides directions for future work.

\section{Background and Motivation}
In this section, we review related works across three key areas and then outline the motivation behind this study.

\subsection{Web Crowdsourcing and Human-AI Collaboration Empowerment}
% Crowdsourcing~\cite{howe2006rise} refers to the practice of acquiring ideas, services, or content by soliciting contributions from a large group of people. 
With the advancement of web technologies, crowdsourcing activities have increasingly migrated to web and mobile internet platforms, namely web crowdsourcing~\cite{doan2011crowdsourcing}. An exponential rise in its applications has witnessed, such as ride-hailing and software development.
To tackle complex web-based tasks, scientists at Microsoft introduced human-AI interaction guidelines to assist researchers and practitioners in designing studies utilizing AI technologies~\cite{amershi2019guidelines}. Following this, numerous studies have integrated human intelligence with AI methods to address challenges such as conversational agent learning for intent detection and text classification~\cite{yang2018leveraging,arous2021marta}. A recent study, for example, engaged online users from crowdsourcing platforms and implemented advanced computer vision techniques to generate city maps~\cite{qiu2019crowd}. Given the growing significance of AI-in-the-loop systems in human-intervened tasks, the concept and principles of human-AI decision-making within the context of web crowdsourcing were provided~\cite{green2019principles}. 

% Thus, we have witnessed an exponential rise in applications built around the concept of crowdsourcing~\cite{howe2006rise}--from ride-hailing~\cite{seng2023ridesharing} and food delivery~\cite{liu2018foodnet} to software development~\cite{abd2021use} and urban governance ~\cite{qiu2019crowd}. The "algorithmic crowdsourcing" paradigm is exemplified where web architectures coordinate human-machine interactions at scale. In light of the practical effectiveness in these domains, web crowdsourcing is now expanding into more complex task domains such as source search (e.g., locating gas leaks or biological signals)~\cite{zhao2024user}.

\subsection{Source Search and Crowd-powered Practices}
Source search is a critical problem for both nature and mankind~\cite{jing2021recent} focusing on determining the location of a source (of gas or signal) in the shortest possible time. Existing source search approaches can generally be classified into three categories: information-theoretic~\cite{jang2023improved}, biologically-inspired~\cite{al2021distributed}, and gradient-based methods~\cite{jiang2019source}. Among these, information-theoretic algorithms, especially those grounded in the Bayesian framework~\cite{ojeda2024robotic}, stand out for their distinct advantages. To further enhance the performance (i.e., success rate and efficiency) of a searching algorithm, multi-robot collaboration mechanisms~\cite{tang2020multirobot} have been designed and adopted. However, when source search takes place in complex environments, the search process always encounters fatal problems, resulting in wrong outcomes. Thus, researchers started to explore effective ways leveraging human intelligence to improve AI-based search algorithms through web platforms~\cite{zhao2024user}. However, this approach also entails substantial costs and imposes considerable burdens on human workers.

\subsection{Large Models for Scene Understanding and Reasoning}
MLLMs integrate multimodal encoders/decoders with traditional LLMs, enabling cross-modal understanding that overcomes text-only limitations. While these models demonstrate remarkable capabilities across diverse tasks including image-text understanding~\cite{liu2024visual}, video-text understanding~\cite{li2023videochat}, and even multimodal generation~\cite{peng2023kosmos}, their effectiveness in handling complex tasks remains constrained by predominant single-step reasoning approaches. To this end, CoT prompts are utilized to enhance problem-solving abilities by guiding LLMs through structured multi-step reasoning. Recent work explores CoT adaptations for multimodal problems, for instance, Shikra~\cite{chen2023shikra} pioneers CoT application in visual grounding tasks, while SoM~\cite{yang2023set} introduces structural image annotations like segmentation maps and spatial grids to provide spatial reasoning anchors. However, CoT has not been comprehensively explored for fine-grain reasoning in source search tasks.

\subsection{Motivation}
Building on the demonstrated scene understanding and reasoning capabilities of large models across various tasks, as well as addressing the limitations of human-AI collaborative source search, our work seeks to explore concrete methods for leveraging large models in zero-shot source search tasks within a top-down view of web-based search environments.



\section{System Design}
%说明web-based系统开发设计
%\sq{To answer \textbf{RQ1}, we designed the AutoS$^2$earch framework based on web platforms.} To support the AutoS$^2$earch framework, we developed a prototype system utilizing advanced web technologies. 
To answer \textbf{\textit{RQ1}}, we designed the AutoS$^2$earch framework based on web platforms. This involved migrating our previously developed crowd-powered source search prototype system~\cite{zhao2022crowd} from a desktop application to a web platform. The primary goals of this implementation were to achieve cross-platform accessibility, real-time interaction, and dynamic visualization.

(1) \textbf{Back-End Implementation:} we selected the lightweight and scalable Flask framework, and initialized the application using Flask and Socket.IO. The functions are handled by defining routes and Socket.IO events.

% % design 1
% \begin{center}
% \begin{minipage}{0.92\linewidth}
% \begin{shaded}
% % \centering
% \textit{Python:}
% app = Flask(name),
% socketio = SocketIO(app)
% \end{shaded}
% \end{minipage}
% \end{center}

(2) \textbf{Real-Time Communication:} Socket.IO was used to support WebSocket and polling to ensure low-latency communication. Data is sent to the front-end using `socketio.emit', and on the front-end, events sent by the back-end are received using `socket.on'.

(3) \textbf{Map Drawing and Updating:} we first determined the map drawing logic (initializing the map and updating it based on changes), and then converted the map data into a format recognizable by the front-end.

(4) \textbf{Front-End Rendering:} we utilized HTML5's Canvas -- an ideal choice for dynamic map displays—to achieve efficient graphic rendering. We defined the Canvas element and implemented the drawing logic using JavaScript. To facilitate user interaction, we added control buttons on the interface and bound click events to them. Additionally, a click event is bound to the Canvas to send the user's click coordinates to the back-end.

% % design 2
% \begin{center}
% \begin{minipage}{0.92\linewidth}
% \begin{shaded}
% % \centering
% \textit{HTML:}
% <canvas id="maze-canvas"></canvas>
% \end{shaded}
% \end{minipage}
% \end{center}

The user interface of this system, shown in Fig.~\ref{fig:system}, uses graphical elements to illustrate the source search task and the problem. It displays the robotic searcher, search environment, current state, posterior probability distribution of the source location, and four directional choices. When a problem is detected, the system automatically generates a task for large models, and then large models analyze the scene and reason step by step to plan a path for the robot. Once a deadlock is resolved, the system resumes automatic search, continuing until the source is found. This user interface allows decision-makers to observe the source search process in real-time. It provides the capability to: 1) enable decision-makers to interrupt the search process as needed, 2) facilitate crowdsourcing during the search, and 3) integrate large models for handling the detected problems.

\begin{figure}[htbp]
    \centering
    \includegraphics[width=\linewidth]{fig1.jpg}
    \caption{A screenshot of the web-based source search prototype system.}
    \label{fig:system}
\end{figure}





\section{ \uqeval: A framework for Assessing Confidence Estimation}\label{sec:method}
At a high level, existing evaluation frameworks
% the evaluation pipeline described in \cref{sec:prelim:old_eval} 
for $C_{\mathcal{M}}$ includes three main steps (blue path in \cref{fig:pipeline}):
\begin{enumerate}[nosep]
    \item Generate $\predSeq$ from $\mathcal{M}$ given the input $\xInput_i$.
    \item Determine the correctness label of $\predSeq$ using the function $\acc(\cdot,\xInput)$.
    \item Compute evaluation metrics such as AUROC. A higher metric value indicates that $C_{\mathcal{M}}$ is a ``better'' confidence estimation.
\end{enumerate}
The main limitation of this general pipeline lies in $\acc$ in step 2. 
Existing evaluation frameworks all implicitly assume step 1---that the confidence measure $C_{\mathcal{M}}$ must apply to generated sequences $\predSeq$. 
While this might hold for consistency-based uncertainty measures, where response divergence indicates uncertainty, it does not extend to confidence measures. 
In other words, we could relax step 1 in order to improve step 2.
% Consider, for instance, Eccentricity~\cite{lin2024generating}: The uncertainty measure $U_{ecc}$ computes the average distance between sampled generation embeddings and their centroid, while the confidence measure evaluates a specific generation. 
% Consequently, we could simply use the sampled generations to construct the embedding space, yet we can still measure the distance of \textit{any} sequence to the center of embeddings.


%The previous discussion suggests that existing evaluation frameworks all bear an implicit assumption: The confidence measure $C_{\mathcal{M}}$ must be applied to the generations $\predSeq$.
%While this might be true for the case of most existing consistency-based uncertainty measures, where a high degree of divergence of the sampled responses is a strong indicator of high uncertainty, this is \textit{not} the case for confidence measures.
%Take Eccentricity~\cite{lin2024generating} as an example: The uncertainty measure $U_{ecc}$ is based on the average distance of the embeddings of the sampled generations to the center of all embeddings and the confidence measure is that of a particular generation. 
%Consequently, we could simply use the sampled generations to construct the embedding space, yet we can still measure the distance of \textit{any} sequence to the center of embeddings. 

\textit{Our main proposal in this paper is to adapt multiple-choice datasets to evaluate confidence measures designed for free-form NLG.}
Unlike free-form NLG datasets, multiple-choice datasets provide inherent correctness values for options, eliminating the need for an explicit correctness function. 
If we simply ``pretend'' that these options are free-form generations from the base LM, we can directly evaluate the confidence measure quality. 
As \cref{fig:pipeline} shows, the approach differs from existing evaluation pipelines only in applying confidence estimation methods to multiple-choice options.




Consider the QASC~\cite{khot2020qasc} dataset as an example,
each problem comes with a question $\xInput$ and a few choices, $o_1,\ldots,o_K$. 
Unlike what such datasets were designed for, we re-format the input prompt as a free-form NLG question, as illustrated in \cref{fig:qasc_example}, as if the base LLM generated each option itself, in different runs.
In what follows, we first explain explain slight nuances in applying internal state-based white-box confidence measures as well as consistency-based black-box ones. 
%shows a reformatted question from the QASC dataset.

\begin{figure}[t]
  \includegraphics[width=\columnwidth]{figures/qasc_example.pdf}
  % \caption{A reformatted question example from the QASC dataset. The Question and Choices are directly from the original dataset, while our prompt is specifically designed for LLM input to generate open-form responses.}
  \caption{
  We reformat each option from the multiple-choice question (left), by injecting the \smash{\colorbox{yellow!40}{{{\color{blue}option}}}} to a free-form QA \smash{\colorbox{green!40}{prompt}}.
  One could typically apply any confidence estimation method by treating this \smash{\colorbox{yellow!40}{{{\color{blue}option}}}} as if it was generated by the base LM.
  For black-box confidence measures that require additional responses, we only feed the \smash{\colorbox{green!40}{prompt}} to the base LM.
  }
  \label{fig:qasc_example}
\vspace{-3mm}
\end{figure}



\textbf{Logit or Internal State-Based Measures} typically examine the internals of a LM when it generates a particular response.
The nature of the free-form generation task allows us to simply plug-in the option $o_i$ into the corresponding location of the prompt, and extract similar information that allows us to evaluate the confidence\footnote{In fact, this was the practice to compute \baselineSL for actual generations. For example, \url{https://github.com/lorenzkuhn/semantic_uncertainty/blob/main/code/get_likelihoods.py} and \url{https://huggingface.co/docs/transformers/perplexity}.}.
% One concern is whether these options are too ``different'' from what the LM would otherwise generate itself.
% As exemplified in \cref{fig:true_distribution}, $C(o_i)$ in general shares a similar distribution to $C(\predSeq_i)$. 

% Taking CommonsenseQA~\footnote{A multiple-choice dataset for our experiment is described in Section~\ref{sec:experiments}} as an example, we compare the logit distribution of the correct answer choices with the distributions of other LLM-generated responses. 
% As shown in \cref{fig:true_distribution}, the distributions exhibit notable similarities, indicating that logit-based confidence estimation can capture underlying patterns shared between correct answer choices and free-form generations.

% \begin{figure}[t]
%   \includegraphics[width=\columnwidth]{figures/true_distribution_new.png}
%   \caption{Confidence score distribution for the \baselinePTrue method on the CQA dataset. The blue distribution represents 20 open-form responses, while the red distribution corresponds to the correct option. }
%   \label{fig:true_distribution}
% \end{figure}


\textbf{Consistency-based Confidence Measures}
Unlike logit-based or internal-state-based measures, consistency-based confidence measures typically rely on an estimate of the predictive distribution, denoted as $\PredDist$, and any response that is closer to the center of the distribution (in the ``semantic space'') is considered to be of higher confidence. 
Consider methods from~\citet{lin2024generating} as an example. To preserve the integrity of the predictive distribution, we first sample $n$ responses from $\PredDist$ as usual, and then iteratively include one option $o_i$ at a time to compute its associated confidence score~\cite{rivera-etal-2024-combining,manakul-etal-2023-selfcheckgpt}. 
\cref{alg:confidence_score} outlines this process. 


\renewcommand{\algorithmicrequire}{\textbf{Input:}}
\renewcommand{\algorithmicensure}{\textbf{Output:}}

\begin{algorithm}[t]
\small
% \caption{Confidence Score Computation in the Black-Box Method}
\caption{Consistency-based Confidence Estimation for Any Sequences}
\label{alg:confidence_score}
\begin{algorithmic}[1]
    \Require $\xInput$, $\mathcal{M}$, candidate sequences $A = \{a_1, \dots, a_K\}$
    \Ensure $\{C_{\mathcal{M}}(\xInput,a_1), \dots,C_{\mathcal{M}}(\xInput,a_K)\}$ 
    
    \State Generate $S = \{\predSeq_1, \dots, \predSeq_{n}\}$ using $\mathcal{M}$ for question $\xInput$
    \State Compute pairwise similarity matrix $M$ of $S$.% \in \mathbb{R}^{|S'| \times |S'|}$
    % \State Construct full response set $S' = S \cup A$, where $|S'| = n+K$
    % \State Compute pairwise similarity matrix $M_{sim} \in \mathbb{R}^{|S'| \times |S'|}$
    
    \For{each $a_i \in A$}
        \State Compute a new similarity matrix $M_i$ of $S\cup\{a_i\}$, reusing $M$. % \in 
        % \State Form subset $S_i = S \cup \{a_i\}$, where $|S_i| = n+1$
        % \State Extract pairwise similarity matrix $M_{sim}^{(i)} \in \mathbb{R}^{|S_i| \times |S_i|}$
        \State Compute confidence score $C_{\mathcal{M}}(\xInput,a_i)$ using $M_i$. %degree matrix or eccentricity of $M_{sim}^{(i)}$
    \EndFor

    \State \Return $\{C_{\mathcal{M}}(\xInput,a_1), \dots,C_{\mathcal{M}}(\xInput,a_K)\}$ 
\end{algorithmic}
\end{algorithm}

% Since computing the similarity matrix is the most computationally expensive step, the subsequent 5 confidence score calculations reuse precomputed similarity values. As a result, the additional computations take less than 1 minute in total, ensuring efficiency.

% Consistency-based confidence measures are a little different from .


\paragraph{Remarks}
Our proposal relaxes step 1 at the beginning of this section, allowing for $\predSeq^*=o_i$ not sampled from $\PredDist$.
This is not to be misunderstood as a proposal to \textit{replace} the current pipeline (\cref{sec:prelim:old_eval})---rather, it is \textit{complementary}.
The rationale is that if a good confidence measure predicts the correctness well, it should perform well in \textit{both} evaluation frameworks.
In fact, any $o_i\in\Sigma^*$ that does not violate the generation configuration, has a non-zero probability to be sampled from $\PredDist$, and a robust confidence measure should be expected to model it well.
% \fontred{
% In fact, any $o_i$, as long as it does not violate the generation config, could be sampled from $\PredDist$ given enough time.
% % As we will see in \cref{sec:exp}, even though $o_i$ are not sampled from $\PredDist$, we do not observe a big distribution shift in terms of the confidence values as well.
% }
% Note that we do not advise \textit{replacing} the existing valuation 

% It is important to note that our method only obtains correctness labels for $o_i$. Consequently, when computing AUROC, AUARC, and other evaluation metrics, we only consider confidence values associated with these options.\textcolor{red}{already mention it in \cref{sec:metrics} }

\section{Experiments}\label{case}

In this section, we introduce experimental setup, baseline algorithms, and evaluation metrics. The project code can be found here\footnote[1]{https://gitee.com/parallelsimlab/autos2earch}.

\subsection{Experimental Setup}
The source search activities are performed by a virtual robot within a simulated 2D environment measuring $20m\times20m$. The search area is divided into a $20\times20$ grid of cells. Each cell has a probability $P_o$ of containing an obstacle, with $P_o$ set to 0.75 to introduce a relatively high difficulty (more obstacles). This higher complexity is chosen because simpler environments (with fewer obstacles) do not require external assistance. In this study, we did not consider the specific types or shapes of obstacles. If a cell contains an obstacle, it is considered completely obstructed, meaning the robot cannot enter or traverse it.

\subsection{Baseline Algorithms}
As detailed in the published work~\cite{zhao2023leveraging} on human-collaborative source search, the baselines adopted in this study naturally follow from that setup. Baseline 1 employs the Infotaxis algorithm directly, while Baseline 2 incorporates our proposed automatic problem detection method, navigating the robot to a random location to escape problematic scenarios. For consistency, we adopt an aided control interaction model of human-AI collaboration in this comparative analysis.
Furthermore, we introduce Baseline 3, where the robot navigates to a randomly chosen direction from four possible options (mentioned in Section 4.2) upon detecting a problem. It is worth noting that both Baseline 2 and Baseline 3 represent state-of-the-art improvements over traditional source search algorithms.

\subsection{Evaluation Metrics}
In this study, we evaluate the effectiveness and efficiency of the source search process and its outcomes. Effectiveness is measured by the success rate, defined as the robot successfully locating the source within 400 steps (where a step represents one iteration of updating search states). If the robot fails to find the source within 400 steps, regardless of whether large models are involved, the task is considered unsuccessful. Efficiency is assessed by the number of steps the robot takes to find the source, with failed attempts excluded from the calculation. Additionally, we measure the execution time of large models per task to see whether they hold an advantage over human workers in time-sensitive tasks.


% \subsection{Procedures}


\section{Results}
In this section, we present the results of (1) an illustrative run, (2) the comparison study, and (3) the ablation study. 

\begin{figure}[htbp]
    \centering
    \includegraphics[width=.7\linewidth]{fig4.jpg}
    \caption{An illustrative run of the proposed framework at different time steps. (a) step=0; (b) step=63; (c) step=151; (d) step=203}
    \label{fig:Illus}
\end{figure}

\subsection{Illustrative Run} % (\textit{RQ1})} 

We conducted an experiment using one scenario from a set of 20 benchmark scenarios to illustrate a successful search process. The illustrative run of AutoS$^2$earch is shown in Fig.~\ref{fig:Illus}. The process includes the initiation of the search, the progression of the algorithm-driven search, the involvement of large models when a problem is detected, and ultimately resolving the issue to successfully locate the source. As we can see in Fig.~\ref{fig:Illus}(c), large models (both GPT-4o\footnote[2]{https://github.com/topics/gpt-4o}) are activated at search step=151. Based on the current visual inputs and the provided prompt for MLLMs, the language description of this scene is presented as follows. 

% Output of MLLM
\begin{center}
\begin{minipage}{\linewidth}
\begin{shaded}
% \centering
\textit{\textbf{Output of the MLLM}}

Region A:
Distance to dense area of small green dots: close;
Density of surrounding black exploration area: medium

Region B:
Distance to dense area of small green dots: medium;
Density of surrounding black exploration area: medium

Region C:
Distance to dense area of small green dots: medium;
Density of surrounding black exploration area: medium

Region D:
Distance to dense area of small green dots: far;
Density of surrounding black exploration area: low

\end{shaded}
\end{minipage}
\end{center}

Using the language descriptions of the four directional choices, the LLM initiates CoT reasoning, guided by the prompt for LLMs. The LLM's final choice is \textbf{A}, and its explanation for this selection is also provided. The decision-making process closely mirrors that of humans. 

To explore this further, we recruited three human raters to independently evaluate the LLM's reasoning process and compare the rationale behind the optimal choices made by the LLM and human intelligence. Our findings reveal that the LLM generally adheres strictly to a predefined priority order, whereas humans display an ability to flexibly adjust priorities based on the relative magnitudes of differences between the choices. For instance, when two regions are similarly distant from a dense cluster of green dots (indicating minimal differences in second-level priority), but the surrounding black area densities differ significantly (indicating greater differences in third-level priority), humans tend to prioritize the third-level criteria over the second. In contrast, the LLM struggle with such adaptive decision-making, even when additional rules are introduced.
% The total time taken by large models for the entire search was approximately 22 seconds, which is faster than the average execution time of the human-aided mode (over 30 seconds). 
% \sq{needs some analysis on AI explanations. If we want to make this part good, it is suggested to have ~3 human raters independently evaluate all the AI's reasoning processes. This depends on how much time we have before the deadline, otherwise we can choose some typical examples and analyze their qualities.}

% Explanation of LLM
\begin{center}
\begin{minipage}{\linewidth}
\begin{shaded}
% \centering
\textit{\textbf{Reasoning Process of the LLM}}

Based on the description and priority rules, the analysis is conducted to identify the highest-priority region that meets the criteria:

1.First, check if any regions are missing:
Regions A, B, C, and D all exist.

2.Next, consider the closest distance to the dense area of green dots:
Region A is "close", regions B and C are "medium", and region D is "far"; thus, region A has the highest priority.

3.Then, consider the density of unexplored black areas surrounding each region:
Regions A, B, and C have "medium" density, Region D has "low" density.

\textbf{According to the priority rules, the region with the highest priority is A. Therefore, my selection is: A.}
\end{shaded}
\end{minipage}
\end{center}

\subsection{Comparative Study} % (\textit{RQ2})}

In this subsection, we evaluated the effectiveness of large models-assisted source search over 20  benchmark scenarios (each scenario was run ten times) by measuring the effectiveness (success rate), the efficiency (the number of steps taken to find the source), the execution time of large models. The experimental results, averaged over Monte Carlo simulations across 20 scenarios, are presented in Table ~\ref{tab:results-1}. Clearly, large models-assisted source search has proved to be effective, achieving a success rate of 97\% in most cases. This represents an improvement of approximately 18.5\% over Baseline 1, 9\% over Baseline 2, and 7\% over Baseline 3, while being only 3\% lower than the success rate of human-AI collaborative search. Note that Baseline 2 and 3 are improvements based on the original algorithm (Baseline 1) since automatic problem detection and rule-based problem-solving strategies are used. Furthermore, we observe that the efficiency of AutoS$^2$earch (in terms of steps taken) is comparable to that of human-AI collaborative search, while the average execution time of large models is even shorter. For details on how human workers complete the crowdsourcing task, interested readers can refer to the previous work~\cite{zhao2022crowd}.

\begin{table}[!ht]
    \centering
    \caption{Results of the comparisons over various baselines.}
    \label{tab:results-1}
    \resizebox{0.9\textwidth}{!}{
    \begin{tabular}{llccc}
    \toprule
         \textbf{\emph{Methods}} & \textbf{Expertise} & \makecell[c]{\textbf{Effectiveness}\\(\% success rate)} & \makecell[c]{\textbf{Efficiency}\\(\# steps per task)} & \makecell[c]{\textbf{Human/MLLM+LLM execution time}\\(seconds per task)} \\
    \midrule
         \multirow{2}{*}{\emph{Human Aided}}
         & Expert   & 100  & 175.10 $\pm$ 67.67  & 33.58 $\pm$ 27.87\\
         & Non-expert  & 100  & 165.67 $\pm$ 80.60  & 29.01 $\pm$ 29.51\\
         \midrule
         \emph{Baseline 1} & -  & 78.5 & 154.04 $\pm$ 91.32  & - \\
         \midrule
         \emph{Baseline 2} & -  & 88  & 179.64 $\pm$ 96.45  & - \\
         \midrule
         \emph{Baseline 3} & -  & 90  & 179.76 $\pm$ 97.40  & - \\
         \midrule
         \emph{Ours} & -  & 97  & 170.97 $\pm$ 89.57  & 25.95 $\pm$ 38.20 \\
    \bottomrule
    \end{tabular}
    }
\end{table}

To further explore whether the impressive performance is solely due to GPT-4o's strong capabilities, we evaluated various combinations of MLLMs and LLMs from different companies. The results, presented in Table~\ref{tab:results-2}, reveal that our proposed framework is highly robust, consistently achieving success rates above 95\%. Notably, the Qwen model\footnote[3]{https://github.com/JMaiGC/ComfyUI-Qwen-VL-API} from the Chinese company Alibaba achieves the highest success rate at 98\%.

\vspace{-5mm}

\begin{table}[!ht]
    \centering
    \caption{Results of the comparisons over various large models.}
    \label{tab:results-2}
    \resizebox{0.9\textwidth}{!}{
    \begin{tabular}{llccc}
    \toprule
         \textbf{\emph{LLMs}} & \makecell[c]{\textbf{Effectiveness}\\(\% success rate)} & \makecell[c]{\textbf{Efficiency}\\(\# steps per task)} & \makecell[c]{\textbf{MLLM+LLM execution time}\\(seconds per task)} \\
    \midrule
         \emph{GLM-4v-plus + GLM-4-plus}  & 95 & 171.13 $\pm$ 92.64  & 26.39 $\pm$ 32.75 \\
         \midrule
         \emph{Qwen-VL-plus + Qwen-max}  & 98  & 172.85 $\pm$ 91.08  & 26.74 $\pm$ 36.39 \\
         \midrule
         \emph{GPT-4o + GPT-4o}  & 97  & 170.97 $\pm$ 89.57  & 25.95 $\pm$ 38.20 \\
    \bottomrule
    \end{tabular}
    }
\end{table}

\vspace{-5mm}

\subsection{Ablation Study}

We further ablation studies to validate the importance of main elements designed in our framework: the Chain-of-Thought prompt for the LLM and the size of directional choices A, B, C, and D (which determine the number of candidate cells for each option). We designated the model without CoT reasoning as Our-A and the model with reduced block sizes as Our-B. The average results across 20 scenarios are presented in Table \ref{tab:results-3}. As we can see, both the removal of CoT reasoning and the reduction in block sizes significantly decrease the success rate by approximately 6\% and 7\%, respectively. Notably, while removing CoT reasoning compromises the effectiveness performance, it does lead to improved efficiency and shorter execution time due to fewer reasoning steps.

\begin{table}[!ht]
    \centering
    \caption{Results of the ablation study.}
    \label{tab:results-3}
    \resizebox{0.8\textwidth}{!}{
    \begin{tabular}{llccc}
    \toprule
         \textbf{\emph{Methods}} & \makecell[c]{\textbf{Effectiveness}\\(\% success rate)} & \makecell[c]{\textbf{Efficiency}\\(\# steps per task)} & \makecell[c]{\textbf{MLLM+LLM execution time}\\(seconds per task)} \\
    \midrule
         \emph{Ours-A}  & 91 & 157.74 $\pm$ 85.33  & 23.82 $\pm$ 36.45 \\
         \midrule
         \emph{Ours-B}  & 90  & 170.28 $\pm$ 93.35  & 24.49 $\pm$ 34.19 \\
         \midrule
         \emph{Ours}  & 97  & 170.97 $\pm$ 89.57  & 25.95 $\pm$ 38.20 \\
    \bottomrule
    \end{tabular}
    }
\end{table}












\section{Discussions}

The source search results convey three main messages: (1) By incorporating carefully designed prompts that enable large language models with scene comprehension and multi-step reasoning capabilities, autonomous source search capabilities can be integrated into web-based systems to support decision-making in time-sensitive scenarios. (2) The large models-assisted method is effective and efficient for improving source search, approaching the performance of human-AI collaborative approaches while reducing execution time by approximately 25\%. (3) Whether in scene element presentation, problem detection mechanisms, or CoT prompt design, each component reflects human intelligence, highlighting that complex task solving fundamentally relies on human-AI hybrid intelligence.

\noindent\textbf{\textit{Drawbacks.}} Despite the strengths, this work has several limitations. (1) \textit{Environmental Complexity Gap:} The simplified $20 \times 20$ grid with static obstacles fail to capture real-world dynamics (e.g., moving obstructions, multi-source scenarios). The visual environment used here is insufficient to test whether large models truly possess robust scene understanding and multi-step reasoning capabilities in complex settings. (2) \textit{Limited Task Understanding:} While simple scene elements were designed to help the large model understand tasks, the lack of domain-specific knowledge makes it difficult for the model to balance exploration and exploitation during the search, sometimes leading to hallucinations by selecting irrelevant areas. (3) \textit{Underutilization of MLLM Potential}: In this work, MLLMs were mainly used to convert visual observations into textual descriptions, with large language models handling subsequent reasoning. This separation of visual understanding and language reasoning may limit the integrated capabilities MLLMs are designed to offer.

\noindent\textbf{\textit{Potential Avenues.}} To address these limitations, we propose to explore: (1) \textit{Dynamic Environment Adaptation:} Design LLM-empowered search agent and develop online prompt tuning mechanisms where LLMs could adjust decision rules according to the environment variations. (2) \textit{Visual Thinking Augmentation:} Integrate graph-based scene representations and reflection mechanisms to help MLLMs directly reason on the visual inputs without hallucinations. (3) \textit{Human-AI Value Alignment:} Implement human-in-the-loop feedback mechanisms in complex and high-risk scenarios and ensure alignment of decision objectives between humans and AI.

\noindent\textbf{\textit{Implications.}} The implications of AutoS$^2$earch extend far beyond the technical achievements in web-based autonomous systems. Its design reflects a broader trend in human-AI collaborative systems, where the goal is to harness the cognitive strengths of both entities in tandem.  Moreover, it may redefine the role of humans in web crowdsourcing systems—from task executors to validators of AI rationality in the future. 



\section{Conclusions}
\label{sec:conclusion}
As industrial environments continue to evolve, the integration of OT and IT systems has become a cornerstone of modern industrial operations. However, this convergence has also introduced unprecedented cybersecurity challenges, exposing critical infrastructure to sophisticated threats such as malware, ransomware, and advanced persistent threats (APTs). The unique vulnerabilities of OT networks, stemming from legacy systems, real-time operational demands, and the high cost of downtime, necessitate a proactive and comprehensive approach to cybersecurity.

This study has highlighted the critical importance of understanding the key components of OT systems, including SCADA, PLCs, and RTUs, and the security risks that arise from their integration with IT networks. By examining recent cyberattacks on OT environments, we have underscored the devastating potential of these threats, not only to data but also to physical infrastructure and human safety. Analysis of attack vectors, such as phishing, malware injection, and supply chain compromises, emphasizes the need for robust defense mechanisms tailored to the unique requirements of OT networks.

Emerging trends in OT cybersecurity, such as the adoption of AI-driven threat detection, Zero Trust Architecture, blockchain for secure event logging, and digital twins for proactive security, offer promising solutions to these challenges. These technologies enable real-time threat detection, improve incident response efficiency, and improve system resilience, ensuring the continuity and reliability of critical industrial operations. Furthermore, regulatory frameworks such as the NIST Cybersecurity Framework and IEC 62443 provide essential guidelines for securing OT environments, promoting a structured and adaptive approach to risk management.

As the cybersecurity landscape continues to evolve, the importance of cyber resilience cannot be overstated. Moving beyond traditional intrusion prevention, resilience frameworks focus on real-time threat detection, autonomous recovery, and adaptive system reinforcement. By integrating these advanced strategies, industries can not only mitigate risks but also ensure the long-term safety and reliability of their critical infrastructure.

In conclusion, the cybersecurity of OT networks is not just a technical challenge but a strategic imperative. As industrial systems become increasingly interconnected, the adoption of proactive cybersecurity measures, coupled with a focus on resilience, will be essential in safeguarding critical infrastructure from the ever-evolving threat landscape. The lessons learned from past attacks and the innovations in cybersecurity technologies provide a roadmap for building a secure and resilient industrial future.

% \begin{credits}
% \subsubsection{\ackname} This study is funded by Youth Independent Innovation Foundation of NUDT (ZK-2023-21) and National Natural Science Foundation of China (no. 62202477, 62173337).

% % \subsubsection{\discintname}
% % The authors have no competing interests to declare that are relevant to the content of this article.
% \end{credits}


% ---- Bibliography ----

\bibliographystyle{splncs04}
\bibliography{ref}

\end{document}
