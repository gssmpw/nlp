% This is samplepaper.tex, a sample chapter demonstrating the
% LLNCS macro package for Springer Computer Science proceedings;

\documentclass[runningheads]{llncs}
\usepackage{framed}
\usepackage{mdframed}
\usepackage{xcolor}
\usepackage{svg}
\usepackage{booktabs}
\usepackage{enumitem}
%\usepackage{algorithmicx}
\usepackage{algpseudocode}
\usepackage{multirow}
\usepackage{url}
\usepackage{framed}
\usepackage{graphicx}
\usepackage{caption}
\usepackage{subcaption}
\usepackage{balance}
\usepackage{makecell}
\usepackage[T1]{fontenc}
% T1 fonts will be used to generate the final print and online PDFs,
% so please use T1 fonts in your manuscript whenever possible.
% Other font encondings may result in incorrect characters.
%
\usepackage{graphicx}
% Used for displaying a sample figure. If possible, figure files should
% be included in EPS format.
%
% If you use the hyperref package, please uncomment the following two lines
% to display URLs in blue roman font according to Springer's eBook style:
%\usepackage{color}
%\renewcommand\UrlFont{\color{blue}\rmfamily}
%\urlstyle{rm}
%
\usepackage{marvosym}
\usepackage[colorlinks,
            linkcolor=blue,
            anchorcolor=black,
            citecolor=blue]{hyperref}

\definecolor{shadecolor}{gray}{0.9}

\newcommand{\sq}[1]{\textcolor{cyan}{\textbf{Sihang: #1}}}

\begin{document}
%
\title{AutoS$^2$earch: Unlocking the Reasoning Potential of Large Models for Web-based Source Search}
%
\titlerunning{Large Models for Web-based Source Search}
% If the paper title is too long for the running head, you can set
% an abbreviated paper title here
%
% \author{Anonymous Author}
\author{Zhengqiu Zhu\inst{1} \and
Yatai Ji\inst{1} \and
Jiaheng Huang\inst{1} \and
Yong Zhao\inst{1} \and
Sihang Qiu\inst{1} \and 
Rusheng Ju\inst{1}
}
%
\authorrunning{Z. ZHU et al.}
% First names are abbreviated in the running head.
% If there are more than two authors, 'et al.' is used.
%
\institute{National University of Defense Technology, Changsha, China \email{zhuzhengqiu@nudt.edu.cn,jiyatai\_1209@nudt.edu.cn,12254745966@qq.com,\ zhaoyong15@nudt.edu.cn,sihangq@acm.org,jrscy@sina.com}}
%
\maketitle              % typeset the header of the contribution
%
\begin{abstract}

Web-based management systems have been widely used in risk control and industrial safety. However, effectively integrating source search capabilities into these systems, to enable decision-makers to locate and address the hazard (e.g., gas leak detection) remains a challenge. While prior efforts have explored using web crowdsourcing and AI algorithms for source search decision support, these approaches suffer from overheads in recruiting human participants and slow response times in time-sensitive situations.
%Web crowdsourcing has emerged as a transformative paradigm for tackling critical source search issues in complex environments, spanning applications from industrial safety (e.g., gas leak detection) to ecological monitoring (e.g., biological signal tracking). By integrating human intelligence, it effectively overcomes purely algorithmic limitations such as local optima and deadlocks. However, this approach also entails substantial costs and places considerable burdens on human workers. 
To address this, we introduce AutoS$^2$earch, a novel framework leveraging large models for zero-shot source search in web applications. AutoS$^2$earch operates on a simplified visual environment projected through a web-based display, utilizing a chain-of-thought prompt designed to emulate human reasoning. The multi-modal large language model (MLLMs) dynamically converts visual observations into language descriptions, enabling the LLM to perform linguistic reasoning on four directional choices. Extensive experiments demonstrate that AutoS$^2$earch achieves performance nearly equivalent to human-AI collaborative source search while eliminating dependency on crowdsourced labor. Our work offers valuable insights in using web engineering to design such autonomous systems in other industrial applications.






\keywords{Web Crowdsourcing \and Source Search \and Multi-modal Large Language Model \and Human-AI Collaboration.}
\end{abstract}


\section{Introduction}

In today’s rapidly evolving digital landscape, the transformative power of web technologies has redefined not only how services are delivered but also how complex tasks are approached. Web-based systems have become increasingly prevalent in risk control across various domains. This widespread adoption is due their accessibility, scalability, and ability to remotely connect various types of users. For example, these systems are used for process safety management in industry~\cite{kannan2016web}, safety risk early warning in urban construction~\cite{ding2013development}, and safe monitoring of infrastructural systems~\cite{repetto2018web}. Within these web-based risk management systems, the source search problem presents a huge challenge. Source search refers to the task of identifying the origin of a risky event, such as a gas leak and the emission point of toxic substances. This source search capability is crucial for effective risk management and decision-making.

Traditional approaches to implementing source search capabilities into the web systems often rely on solely algorithmic solutions~\cite{ristic2016study}. These methods, while relatively straightforward to implement, often struggle to achieve acceptable performances due to algorithmic local optima and complex unknown environments~\cite{zhao2020searching}. More recently, web crowdsourcing has emerged as a promising alternative for tackling the source search problem by incorporating human efforts in these web systems on-the-fly~\cite{zhao2024user}. This approach outsources the task of addressing issues encountered during the source search process to human workers, leveraging their capabilities to enhance system performance.

These solutions often employ a human-AI collaborative way~\cite{zhao2023leveraging} where algorithms handle exploration-exploitation and report the encountered problems while human workers resolve complex decision-making bottlenecks to help the algorithms getting rid of local deadlocks~\cite{zhao2022crowd}. Although effective, this paradigm suffers from two inherent limitations: increased operational costs from continuous human intervention, and slow response times of human workers due to sequential decision-making. These challenges motivate our investigation into developing autonomous systems that preserve human-like reasoning capabilities while reducing dependency on massive crowdsourced labor.

Furthermore, recent advancements in large language models (LLMs)~\cite{chang2024survey} and multi-modal LLMs (MLLMs)~\cite{huang2023chatgpt} have unveiled promising avenues for addressing these challenges. One clear opportunity involves the seamless integration of visual understanding and linguistic reasoning for robust decision-making in search tasks. However, whether large models-assisted source search is really effective and efficient for improving the current source search algorithms~\cite{ji2022source} remains unknown. \textit{To address the research gap, we are particularly interested in answering the following two research questions in this work:}

\textbf{\textit{RQ1: }}How can source search capabilities be integrated into web-based systems to support decision-making in time-sensitive risk management scenarios? 
% \sq{I mention ``time-sensitive'' here because I feel like we shall say something about the response time -- LLM has to be faster than humans}

\textbf{\textit{RQ2: }}How can MLLMs and LLMs enhance the effectiveness and efficiency of existing source search algorithms? 

% \textit{\textbf{RQ2:}} To what extent does the performance of large models-assisted search align with or approach the effectiveness of human-AI collaborative search? 

To answer the research questions, we propose a novel framework called Auto-\
S$^2$earch (\textbf{Auto}nomous \textbf{S}ource \textbf{Search}) and implement a prototype system that leverages advanced web technologies to simulate real-world conditions for zero-shot source search. Unlike traditional methods that rely on pre-defined heuristics or extensive human intervention, AutoS$^2$earch employs a carefully designed prompt that encapsulates human rationales, thereby guiding the MLLM to generate coherent and accurate scene descriptions from visual inputs about four directional choices. Based on these language-based descriptions, the LLM is enabled to determine the optimal directional choice through chain-of-thought (CoT) reasoning. Comprehensive empirical validation demonstrates that AutoS$^2$-\ 
earch achieves a success rate of 95–98\%, closely approaching the performance of human-AI collaborative search across 20 benchmark scenarios~\cite{zhao2023leveraging}. 

Our work indicates that the role of humans in future web crowdsourcing tasks may evolve from executors to validators or supervisors. Furthermore, incorporating explanations of LLM decisions into web-based system interfaces has the potential to help humans enhance task performance in risk control.






\section{Notation and Background}
We denote the derivative of a function $f: \gU \to \gV$
at $v = f(u)$ as the linear map $\frac{\partial f(u)}{\partial u}: \gU \to \gV$,
or equivalently as $\frac{\partial v}{\partial u}$.
In general, $u$ can represent a high dimensional tensor,
such as a feature map or the filter parameters in a convolutional neural network (CNN).
Similarly, we denote the transpose of the derivative as
$\frac{\partial v}{\partial u}^T: \gV \to \gU$.
Throughout this work we'll operate in vector spaces;
when $f$ is a scalar function, we interchangeably denote
the gradient vector at $u$, $\nabla f(u) \in \gU$, as the transpose of its derivative,
$\frac{\partial v}{\partial u}^T  \in \gU$.

\subsection{Preconditioned Gradient Descent}
\label{sec:pcgd}
We denote $\ell_{(x,y)}: \Theta \to \sR$ as the pointwise loss function for a labeled sample,
where $x$ and $y$ are the inputs and labels, respectively, and the 
samples are drawn from a data distribution, $\gD$.
Minimizing the expected loss, 
$\bar{\ell}(\theta) = \E_{(x,y) \sim \gD} [\ell_{(x,y)}(\theta)]$,
via gradient descent leads to the update rule
$\theta \leftarrow \theta - \alpha^{(t)} \nabla \bar{\ell}(\theta)$,
where $\alpha^{(t)}$ is the desired step size at iteration $t$.
Depending on the form of $\ell$, convergence can be improved by preconditioning
the gradient vector by a matrix $P$, \eg, using the inverse of the Hessian of $\bar\ell(\theta)$ when performing
Newton's method.
In general, when the preconditioner can be factored as $P=WW^T$, the update rule
of following the preconditioned gradient vector, $WW^T \nabla \bar{\ell}(\theta)$, is equivalent to
performing gradient descent on $\bar\ell(\acute\theta)$ with the reparameterized model, $\acute\theta = W\theta$.

%

\subsection{Natural Gradient Descent}
\label{sec:cgrad}
One way to view natural gradient descent \cite{amari1998} is as
a trust-region optimization problem to find a small step, \wrt some norm,
that is aligned in the direction of $\nabla\bar\ell(\theta)$:
%
\begin{align}
\argmin_{\delta \theta} & ~ \bar{\ell}(\theta) + \nabla \bar{\ell}(\theta) \cdot \delta\theta \nonumber\\
\text{s.t.} & ~ \frac{1}{2} \langle \delta\theta, \delta\theta\rangle_{M_{\theta}}  = \eps, \label{eq:trust}
\end{align}
%
where $\langle u, v\rangle_{M_{\theta}} > 0$ is the inner product
defined by the (Riemannian) metric $M_{\theta}$ in the local tangent space at $\theta$.
%
Solving for the stationary point of its Lagrange function results in an update step
$\delta \theta \propto M_{\theta}^{-1} \nabla \bar{\ell}(\theta)$.
When defining $M_\theta$ as the Fisher Information Matrix (FIM), the resulting
vector is referred to as the natural gradient vector; see \citet{kunster2019} for an informative review and discussion.
However, in general, we are free to choose any positive-definite metric, $M_{\theta} \succ 0$,
and we denote the resulting vector under the metric as $\nabla_{M} \ell(\theta) = M_{\theta}^{-1} \nabla \ell(\theta)$.
%
%
Note that taking a step solely proportional $\nabla_{M}\bar{\ell}(\theta)$
ignores the trust region constraint of \Eqref{eq:trust};
solving for it obtains the appropriate step size, $\alpha_{\theta}$,
in the original parameter space that induces an  $\eps$-sized step under the chosen metric:
$\alpha_{\theta} = \sqrt{\frac{\eps}{z_\theta}}$, where
%
\begin{equation}
\label{eq:metricnorm}
z_\theta
= \langle \nabla_{M} \bar{\ell}(\theta), \nabla_{M} \bar{\ell}(\theta) \rangle_{M_{\theta}}
= \nabla_{M} \bar{\ell}(\theta) \cdot \nabla \bar{\ell}(\theta).
\end{equation}
%
%
%
%
%

%
%
%
%
%

\subsection{Functional Gradient Descent}
\label{sec:fgrad}
In contrast to optimizing a function in the space of parameters,
an alternative is to optimize in the space of functions, $\gF$, which
can also be viewed as boosting \citep{mason2000, friedman2001}.
The following summarizes \S2 of \citet{grubbthesis} to introduce notation on this topic.
We denote $\bar{\gL}[f] = \E_{(x,y) \sim \gD}[l_{y}(f(x))]$ to be the loss functional
that computes the expected loss for a given function $f: \gX \to \gV$,
where $l_y: \gV \to \sR$ is loss function (using label $y$) in the image (output space) of $f$,
and $\gV$ is an application-specific vector
space\footnote{In general, each element in $\gV$ need not be a 1-D vector and each could
be a multi-dimensional tensor. 
For example, in $k$-class logistic regression, the image of $f$ is the predicted logits vector
($\gV = \sR^k$) and $l$ is pointwise log-loss.
Whereas for $k$-class semantic image segmentation, each the image of $f$ is a tensor
($\gV = \sR^{h \times w \times k}$) whose outer component is the per-pixel logits 
and $l$ is the per-pixel log-loss summed over the inner spatial components.}.
Letting $v_x = f(x)$, the functional gradient ``vector'' of $\bar{\gL}$ at $f$ is defined as the
\emph{function},
%
\begin{equation*}
\nabla_F \bar{\gL}[f] = \E_{(x,y) \sim \gD} [\nabla_F l_y(v_x)] = \E_{(x,y) \sim \gD} [ \lambda_{(x,y)} \mathbbm{1}_x],
\end{equation*}
%
where each $\lambda_{(x,y)} = \frac{\partial l_y(v_x)}{\partial v_x}^T \in \gV$
is a gradient vector, and $\mathbbm{1}_x: \gX \to \{0, 1\}$ is the Dirac delta function centered at $x$.
For brevity, let $\triangle_f = \nabla_F \tilde{\gL}[f]$.

Instead of the defining the strong-learner, $f$, additively in $\triangle_f \in \gF$,
each $\triangle_f$ is projected onto a smaller hypothesis space, $\gH \subseteq \gF$,
such as small decision trees or MLPs, in order to generalize.
The projection of $\triangle_f$ onto a hypothesis $h \in \gH$ is analogous to vector projection in Euclidean space:
$\frac{\langle \triangle_f, h \rangle_F}{\Vert h \Vert_F}\frac{h}{\Vert h \Vert_F}$,
where $\langle f,g \rangle_F = \mathbb{E}_x[f(x) \cdot g(x)]$ is the inner product in function
space with norm $\Vert h \Vert_F = \sqrt{\langle h,h \rangle_F}$.

When $\gH$ are regressors, the hypothesis, $\hat{h}$, that maximizes the scalar projection term is
equivalent \citep{friedman2001} to minimizing a least squares problem,
%
\begin{equation}
\label{eq:scalarproj}
\hat{h} = \argmax_{h \in \gH} \frac{\langle \triangle_f, h \rangle_F}{\Vert h \Vert_F}
= \argmin_{h \in \gH} \E_{x} [ \Vert h(x) -\triangle_f(x) \Vert^2 ].
\end{equation}
%
In practice, this translates to training a (vector-output) regressor
over the dataset $\{(x,\lambda_{(x,y)})\}$.
Finally, this leads to the update rule
$f \leftarrow f - \eps^{(t)} \frac{\hat{h}}{\Vert \hat{h} \Vert_F}$,
and we refer to each $\hat{h}$ as weak-learners.

\subsubsection{Regularization}
\label{sec:reg}
In addition to minimizing the incurred loss of $f$, we may also want to include
a regularization term in the objective $\tilde{\gL}[f] + \frac{\rho}{2} \Vert f \Vert_F^2$, where 
$\rho \geq 0$. The update rule with the regularized objective is
%
\begin{equation}
\label{eq:reg}
f \gets (1-\eps^{(t)}\rho) f -  \eps^{(t)} \frac{\hat{h}}{\Vert \hat{h} \Vert_F}.
\end{equation}
%
In practice, this is implemented by shrinking the existing weak-learner coefficients $\{\eps^{(j)}|j<t\}$
by the factor $0 \leq (1-\eps^{(t)}\rho) \leq 1$.

\subsection{Boosted Backpropagation}
\label{sec:bbp}

For simplicity in explanation, we begin with the problem of training a feed-forward
network (FFN), as done in \citet{grub2010}, and note that general architectures
will be discussed in the next section.
We represent a FFN as a composition of $m$ differentiable functions
$F = \{f_i:\gX_{i-1} \to \gX_i| 1 \leq i \leq m \}$, where a subset $F_B \subseteq F$
are neurons with parameters $\theta_i$ that we want to train, \eg,
$f_i(\cdot;\theta_i) \in F_B$ could be convolutional layer with filters $\theta_i$.
We denote $\theta_B$ as all the trainable parameters in the network.
The complement set of functions, $F \setminus F_B$, are fixed, \eg, activation functions, resizing layers, etc.
We denote the loss incurred for sample $(x,y)$ for the given network parameters
as $\ell_{(x,y)}(\theta_B) = (l_y \circ f_m \circ \ldots \circ f_1)(x)$,
where $l_y: \gX_m \to \sR$ computes the loss of the last layer's prediction vector with label $y$.
Backpropagation can be used to compute the gradient vector of the
expected loss, $\nabla \bar\ell(\theta_B)$,
and we denote $g_i$ to be the component of $\nabla \bar\ell(\theta_B)$
that corresponds to $\theta_i$, i.e., $g_i = \frac{\partial\bar\ell(\theta_B)}{\partial \theta_i}^T$.

Forgoing parameterizing the FFN neurons with weights,
\citet{grub2010} optimize the loss \wrt each neuron's \emph{function}
by using the adjoint state method for
the equivalent constrained minimization problem,
%
%
%
\begin{align}
\argmin_{F_B} ~& \E_{(x,y) \sim \gD} [l_y(x_{m})] \nonumber \\
\begin{split}
\label{eq:forward}
\text{s.t.} ~& x_0 = x, x_{i} = f_{i}(x_{i-1}), i \in [1, \ldots, m].
\end{split}
\end{align}
%
Using the necessary conditions for a stationary point of its Lagrangian,
they describe an algorithm to recursively compute the functional gradient vector for each neuron.
For $f_i \in F_B$ in a FFN, its functional gradient vector is $\lambda_i \mathbbm{1}_{x_{i-1}}$,
where
%
\begin{align}
\lambda_{m+1} &= \frac{\partial l_y(x_m)}{\partial x_m}^T, \nonumber \\
\lambda_{i} &= \frac{\partial x_i}{\partial x_{i-1}}^T \lambda_{i+1},  i \in [1, \ldots, m],
\label{eq:costates}
\end{align}
%
are the errors backpropagated from the loss layer via vector-Jacobian products (VJPs).
That it is, each training step is analogous to performing normal backpropagation
with the key difference of training a weak-learner using the dataset 
$\{(x_{i-1}, \lambda_i)\}$ for each $f_i \in F_B$,
instead of pulling-back the targets, $\lambda_i$, through respective neuron's local derivative.
That is, $f_i$'s component of $\nabla \bar\ell(\theta_B)$ is
$g_i = \frac{\partial x_i}{\partial \theta_i}^T \lambda_i$.

As presented, the time complexity for computing the unprojected
functional gradient vectors is the same backpropagation; however,
the storage complexity increases linearly with the dataset and dimensionality
due to aggregating the regression targets.
The projection operation then adds significant compute as it requires solving a
large vector regression problem for each neuron\footnote{\eg,
for a convolutional layer, each $\lambda_i$ represents a
$\sR^{h \times w \times d}$ feature map and there are $|\gD|$ of them.}.
Lastly, the presented algorithm is specific to a FFN and it is left
as an exercise to construct the corresponding recurrence relation of \Eqref{eq:costates}
for different network architectures, which can be error prone.
In the next section we address these concerns and provide a simple and efficient boosting algorithm
for any network architecture compatible with autodifferentiation.

\section{System Design}
%说明web-based系统开发设计
%\sq{To answer \textbf{RQ1}, we designed the AutoS$^2$earch framework based on web platforms.} To support the AutoS$^2$earch framework, we developed a prototype system utilizing advanced web technologies. 
To answer \textbf{\textit{RQ1}}, we designed the AutoS$^2$earch framework based on web platforms. This involved migrating our previously developed crowd-powered source search prototype system~\cite{zhao2022crowd} from a desktop application to a web platform. The primary goals of this implementation were to achieve cross-platform accessibility, real-time interaction, and dynamic visualization.

(1) \textbf{Back-End Implementation:} we selected the lightweight and scalable Flask framework, and initialized the application using Flask and Socket.IO. The functions are handled by defining routes and Socket.IO events.

% % design 1
% \begin{center}
% \begin{minipage}{0.92\linewidth}
% \begin{shaded}
% % \centering
% \textit{Python:}
% app = Flask(name),
% socketio = SocketIO(app)
% \end{shaded}
% \end{minipage}
% \end{center}

(2) \textbf{Real-Time Communication:} Socket.IO was used to support WebSocket and polling to ensure low-latency communication. Data is sent to the front-end using `socketio.emit', and on the front-end, events sent by the back-end are received using `socket.on'.

(3) \textbf{Map Drawing and Updating:} we first determined the map drawing logic (initializing the map and updating it based on changes), and then converted the map data into a format recognizable by the front-end.

(4) \textbf{Front-End Rendering:} we utilized HTML5's Canvas -- an ideal choice for dynamic map displays—to achieve efficient graphic rendering. We defined the Canvas element and implemented the drawing logic using JavaScript. To facilitate user interaction, we added control buttons on the interface and bound click events to them. Additionally, a click event is bound to the Canvas to send the user's click coordinates to the back-end.

% % design 2
% \begin{center}
% \begin{minipage}{0.92\linewidth}
% \begin{shaded}
% % \centering
% \textit{HTML:}
% <canvas id="maze-canvas"></canvas>
% \end{shaded}
% \end{minipage}
% \end{center}

The user interface of this system, shown in Fig.~\ref{fig:system}, uses graphical elements to illustrate the source search task and the problem. It displays the robotic searcher, search environment, current state, posterior probability distribution of the source location, and four directional choices. When a problem is detected, the system automatically generates a task for large models, and then large models analyze the scene and reason step by step to plan a path for the robot. Once a deadlock is resolved, the system resumes automatic search, continuing until the source is found. This user interface allows decision-makers to observe the source search process in real-time. It provides the capability to: 1) enable decision-makers to interrupt the search process as needed, 2) facilitate crowdsourcing during the search, and 3) integrate large models for handling the detected problems.

\begin{figure}[htbp]
    \centering
    \includegraphics[width=\linewidth]{fig1.jpg}
    \caption{A screenshot of the web-based source search prototype system.}
    \label{fig:system}
\end{figure}





\section{ \uqeval: A framework for Assessing Confidence Estimation}\label{sec:method}
At a high level, existing evaluation frameworks
% the evaluation pipeline described in \cref{sec:prelim:old_eval} 
for $C_{\mathcal{M}}$ includes three main steps (blue path in \cref{fig:pipeline}):
\begin{enumerate}[nosep]
    \item Generate $\predSeq$ from $\mathcal{M}$ given the input $\xInput_i$.
    \item Determine the correctness label of $\predSeq$ using the function $\acc(\cdot,\xInput)$.
    \item Compute evaluation metrics such as AUROC. A higher metric value indicates that $C_{\mathcal{M}}$ is a ``better'' confidence estimation.
\end{enumerate}
The main limitation of this general pipeline lies in $\acc$ in step 2. 
Existing evaluation frameworks all implicitly assume step 1---that the confidence measure $C_{\mathcal{M}}$ must apply to generated sequences $\predSeq$. 
While this might hold for consistency-based uncertainty measures, where response divergence indicates uncertainty, it does not extend to confidence measures. 
In other words, we could relax step 1 in order to improve step 2.
% Consider, for instance, Eccentricity~\cite{lin2024generating}: The uncertainty measure $U_{ecc}$ computes the average distance between sampled generation embeddings and their centroid, while the confidence measure evaluates a specific generation. 
% Consequently, we could simply use the sampled generations to construct the embedding space, yet we can still measure the distance of \textit{any} sequence to the center of embeddings.


%The previous discussion suggests that existing evaluation frameworks all bear an implicit assumption: The confidence measure $C_{\mathcal{M}}$ must be applied to the generations $\predSeq$.
%While this might be true for the case of most existing consistency-based uncertainty measures, where a high degree of divergence of the sampled responses is a strong indicator of high uncertainty, this is \textit{not} the case for confidence measures.
%Take Eccentricity~\cite{lin2024generating} as an example: The uncertainty measure $U_{ecc}$ is based on the average distance of the embeddings of the sampled generations to the center of all embeddings and the confidence measure is that of a particular generation. 
%Consequently, we could simply use the sampled generations to construct the embedding space, yet we can still measure the distance of \textit{any} sequence to the center of embeddings. 

\textit{Our main proposal in this paper is to adapt multiple-choice datasets to evaluate confidence measures designed for free-form NLG.}
Unlike free-form NLG datasets, multiple-choice datasets provide inherent correctness values for options, eliminating the need for an explicit correctness function. 
If we simply ``pretend'' that these options are free-form generations from the base LM, we can directly evaluate the confidence measure quality. 
As \cref{fig:pipeline} shows, the approach differs from existing evaluation pipelines only in applying confidence estimation methods to multiple-choice options.




Consider the QASC~\cite{khot2020qasc} dataset as an example,
each problem comes with a question $\xInput$ and a few choices, $o_1,\ldots,o_K$. 
Unlike what such datasets were designed for, we re-format the input prompt as a free-form NLG question, as illustrated in \cref{fig:qasc_example}, as if the base LLM generated each option itself, in different runs.
In what follows, we first explain explain slight nuances in applying internal state-based white-box confidence measures as well as consistency-based black-box ones. 
%shows a reformatted question from the QASC dataset.

\begin{figure}[t]
  \includegraphics[width=\columnwidth]{figures/qasc_example.pdf}
  % \caption{A reformatted question example from the QASC dataset. The Question and Choices are directly from the original dataset, while our prompt is specifically designed for LLM input to generate open-form responses.}
  \caption{
  We reformat each option from the multiple-choice question (left), by injecting the \smash{\colorbox{yellow!40}{{{\color{blue}option}}}} to a free-form QA \smash{\colorbox{green!40}{prompt}}.
  One could typically apply any confidence estimation method by treating this \smash{\colorbox{yellow!40}{{{\color{blue}option}}}} as if it was generated by the base LM.
  For black-box confidence measures that require additional responses, we only feed the \smash{\colorbox{green!40}{prompt}} to the base LM.
  }
  \label{fig:qasc_example}
\vspace{-3mm}
\end{figure}



\textbf{Logit or Internal State-Based Measures} typically examine the internals of a LM when it generates a particular response.
The nature of the free-form generation task allows us to simply plug-in the option $o_i$ into the corresponding location of the prompt, and extract similar information that allows us to evaluate the confidence\footnote{In fact, this was the practice to compute \baselineSL for actual generations. For example, \url{https://github.com/lorenzkuhn/semantic_uncertainty/blob/main/code/get_likelihoods.py} and \url{https://huggingface.co/docs/transformers/perplexity}.}.
% One concern is whether these options are too ``different'' from what the LM would otherwise generate itself.
% As exemplified in \cref{fig:true_distribution}, $C(o_i)$ in general shares a similar distribution to $C(\predSeq_i)$. 

% Taking CommonsenseQA~\footnote{A multiple-choice dataset for our experiment is described in Section~\ref{sec:experiments}} as an example, we compare the logit distribution of the correct answer choices with the distributions of other LLM-generated responses. 
% As shown in \cref{fig:true_distribution}, the distributions exhibit notable similarities, indicating that logit-based confidence estimation can capture underlying patterns shared between correct answer choices and free-form generations.

% \begin{figure}[t]
%   \includegraphics[width=\columnwidth]{figures/true_distribution_new.png}
%   \caption{Confidence score distribution for the \baselinePTrue method on the CQA dataset. The blue distribution represents 20 open-form responses, while the red distribution corresponds to the correct option. }
%   \label{fig:true_distribution}
% \end{figure}


\textbf{Consistency-based Confidence Measures}
Unlike logit-based or internal-state-based measures, consistency-based confidence measures typically rely on an estimate of the predictive distribution, denoted as $\PredDist$, and any response that is closer to the center of the distribution (in the ``semantic space'') is considered to be of higher confidence. 
Consider methods from~\citet{lin2024generating} as an example. To preserve the integrity of the predictive distribution, we first sample $n$ responses from $\PredDist$ as usual, and then iteratively include one option $o_i$ at a time to compute its associated confidence score~\cite{rivera-etal-2024-combining,manakul-etal-2023-selfcheckgpt}. 
\cref{alg:confidence_score} outlines this process. 


\renewcommand{\algorithmicrequire}{\textbf{Input:}}
\renewcommand{\algorithmicensure}{\textbf{Output:}}

\begin{algorithm}[t]
\small
% \caption{Confidence Score Computation in the Black-Box Method}
\caption{Consistency-based Confidence Estimation for Any Sequences}
\label{alg:confidence_score}
\begin{algorithmic}[1]
    \Require $\xInput$, $\mathcal{M}$, candidate sequences $A = \{a_1, \dots, a_K\}$
    \Ensure $\{C_{\mathcal{M}}(\xInput,a_1), \dots,C_{\mathcal{M}}(\xInput,a_K)\}$ 
    
    \State Generate $S = \{\predSeq_1, \dots, \predSeq_{n}\}$ using $\mathcal{M}$ for question $\xInput$
    \State Compute pairwise similarity matrix $M$ of $S$.% \in \mathbb{R}^{|S'| \times |S'|}$
    % \State Construct full response set $S' = S \cup A$, where $|S'| = n+K$
    % \State Compute pairwise similarity matrix $M_{sim} \in \mathbb{R}^{|S'| \times |S'|}$
    
    \For{each $a_i \in A$}
        \State Compute a new similarity matrix $M_i$ of $S\cup\{a_i\}$, reusing $M$. % \in 
        % \State Form subset $S_i = S \cup \{a_i\}$, where $|S_i| = n+1$
        % \State Extract pairwise similarity matrix $M_{sim}^{(i)} \in \mathbb{R}^{|S_i| \times |S_i|}$
        \State Compute confidence score $C_{\mathcal{M}}(\xInput,a_i)$ using $M_i$. %degree matrix or eccentricity of $M_{sim}^{(i)}$
    \EndFor

    \State \Return $\{C_{\mathcal{M}}(\xInput,a_1), \dots,C_{\mathcal{M}}(\xInput,a_K)\}$ 
\end{algorithmic}
\end{algorithm}

% Since computing the similarity matrix is the most computationally expensive step, the subsequent 5 confidence score calculations reuse precomputed similarity values. As a result, the additional computations take less than 1 minute in total, ensuring efficiency.

% Consistency-based confidence measures are a little different from .


\paragraph{Remarks}
Our proposal relaxes step 1 at the beginning of this section, allowing for $\predSeq^*=o_i$ not sampled from $\PredDist$.
This is not to be misunderstood as a proposal to \textit{replace} the current pipeline (\cref{sec:prelim:old_eval})---rather, it is \textit{complementary}.
The rationale is that if a good confidence measure predicts the correctness well, it should perform well in \textit{both} evaluation frameworks.
In fact, any $o_i\in\Sigma^*$ that does not violate the generation configuration, has a non-zero probability to be sampled from $\PredDist$, and a robust confidence measure should be expected to model it well.
% \fontred{
% In fact, any $o_i$, as long as it does not violate the generation config, could be sampled from $\PredDist$ given enough time.
% % As we will see in \cref{sec:exp}, even though $o_i$ are not sampled from $\PredDist$, we do not observe a big distribution shift in terms of the confidence values as well.
% }
% Note that we do not advise \textit{replacing} the existing valuation 

% It is important to note that our method only obtains correctness labels for $o_i$. Consequently, when computing AUROC, AUARC, and other evaluation metrics, we only consider confidence values associated with these options.\textcolor{red}{already mention it in \cref{sec:metrics} }

\section{Experiments}\label{case}

In this section, we introduce experimental setup, baseline algorithms, and evaluation metrics. The project code can be found here\footnote[1]{https://gitee.com/parallelsimlab/autos2earch}.

\subsection{Experimental Setup}
The source search activities are performed by a virtual robot within a simulated 2D environment measuring $20m\times20m$. The search area is divided into a $20\times20$ grid of cells. Each cell has a probability $P_o$ of containing an obstacle, with $P_o$ set to 0.75 to introduce a relatively high difficulty (more obstacles). This higher complexity is chosen because simpler environments (with fewer obstacles) do not require external assistance. In this study, we did not consider the specific types or shapes of obstacles. If a cell contains an obstacle, it is considered completely obstructed, meaning the robot cannot enter or traverse it.

\subsection{Baseline Algorithms}
As detailed in the published work~\cite{zhao2023leveraging} on human-collaborative source search, the baselines adopted in this study naturally follow from that setup. Baseline 1 employs the Infotaxis algorithm directly, while Baseline 2 incorporates our proposed automatic problem detection method, navigating the robot to a random location to escape problematic scenarios. For consistency, we adopt an aided control interaction model of human-AI collaboration in this comparative analysis.
Furthermore, we introduce Baseline 3, where the robot navigates to a randomly chosen direction from four possible options (mentioned in Section 4.2) upon detecting a problem. It is worth noting that both Baseline 2 and Baseline 3 represent state-of-the-art improvements over traditional source search algorithms.

\subsection{Evaluation Metrics}
In this study, we evaluate the effectiveness and efficiency of the source search process and its outcomes. Effectiveness is measured by the success rate, defined as the robot successfully locating the source within 400 steps (where a step represents one iteration of updating search states). If the robot fails to find the source within 400 steps, regardless of whether large models are involved, the task is considered unsuccessful. Efficiency is assessed by the number of steps the robot takes to find the source, with failed attempts excluded from the calculation. Additionally, we measure the execution time of large models per task to see whether they hold an advantage over human workers in time-sensitive tasks.


% \subsection{Procedures}


\section{Results}
In this section, we present the results of (1) an illustrative run, (2) the comparison study, and (3) the ablation study. 

\begin{figure}[htbp]
    \centering
    \includegraphics[width=.7\linewidth]{fig4.jpg}
    \caption{An illustrative run of the proposed framework at different time steps. (a) step=0; (b) step=63; (c) step=151; (d) step=203}
    \label{fig:Illus}
\end{figure}

\subsection{Illustrative Run} % (\textit{RQ1})} 

We conducted an experiment using one scenario from a set of 20 benchmark scenarios to illustrate a successful search process. The illustrative run of AutoS$^2$earch is shown in Fig.~\ref{fig:Illus}. The process includes the initiation of the search, the progression of the algorithm-driven search, the involvement of large models when a problem is detected, and ultimately resolving the issue to successfully locate the source. As we can see in Fig.~\ref{fig:Illus}(c), large models (both GPT-4o\footnote[2]{https://github.com/topics/gpt-4o}) are activated at search step=151. Based on the current visual inputs and the provided prompt for MLLMs, the language description of this scene is presented as follows. 

% Output of MLLM
\begin{center}
\begin{minipage}{\linewidth}
\begin{shaded}
% \centering
\textit{\textbf{Output of the MLLM}}

Region A:
Distance to dense area of small green dots: close;
Density of surrounding black exploration area: medium

Region B:
Distance to dense area of small green dots: medium;
Density of surrounding black exploration area: medium

Region C:
Distance to dense area of small green dots: medium;
Density of surrounding black exploration area: medium

Region D:
Distance to dense area of small green dots: far;
Density of surrounding black exploration area: low

\end{shaded}
\end{minipage}
\end{center}

Using the language descriptions of the four directional choices, the LLM initiates CoT reasoning, guided by the prompt for LLMs. The LLM's final choice is \textbf{A}, and its explanation for this selection is also provided. The decision-making process closely mirrors that of humans. 

To explore this further, we recruited three human raters to independently evaluate the LLM's reasoning process and compare the rationale behind the optimal choices made by the LLM and human intelligence. Our findings reveal that the LLM generally adheres strictly to a predefined priority order, whereas humans display an ability to flexibly adjust priorities based on the relative magnitudes of differences between the choices. For instance, when two regions are similarly distant from a dense cluster of green dots (indicating minimal differences in second-level priority), but the surrounding black area densities differ significantly (indicating greater differences in third-level priority), humans tend to prioritize the third-level criteria over the second. In contrast, the LLM struggle with such adaptive decision-making, even when additional rules are introduced.
% The total time taken by large models for the entire search was approximately 22 seconds, which is faster than the average execution time of the human-aided mode (over 30 seconds). 
% \sq{needs some analysis on AI explanations. If we want to make this part good, it is suggested to have ~3 human raters independently evaluate all the AI's reasoning processes. This depends on how much time we have before the deadline, otherwise we can choose some typical examples and analyze their qualities.}

% Explanation of LLM
\begin{center}
\begin{minipage}{\linewidth}
\begin{shaded}
% \centering
\textit{\textbf{Reasoning Process of the LLM}}

Based on the description and priority rules, the analysis is conducted to identify the highest-priority region that meets the criteria:

1.First, check if any regions are missing:
Regions A, B, C, and D all exist.

2.Next, consider the closest distance to the dense area of green dots:
Region A is "close", regions B and C are "medium", and region D is "far"; thus, region A has the highest priority.

3.Then, consider the density of unexplored black areas surrounding each region:
Regions A, B, and C have "medium" density, Region D has "low" density.

\textbf{According to the priority rules, the region with the highest priority is A. Therefore, my selection is: A.}
\end{shaded}
\end{minipage}
\end{center}

\subsection{Comparative Study} % (\textit{RQ2})}

In this subsection, we evaluated the effectiveness of large models-assisted source search over 20  benchmark scenarios (each scenario was run ten times) by measuring the effectiveness (success rate), the efficiency (the number of steps taken to find the source), the execution time of large models. The experimental results, averaged over Monte Carlo simulations across 20 scenarios, are presented in Table ~\ref{tab:results-1}. Clearly, large models-assisted source search has proved to be effective, achieving a success rate of 97\% in most cases. This represents an improvement of approximately 18.5\% over Baseline 1, 9\% over Baseline 2, and 7\% over Baseline 3, while being only 3\% lower than the success rate of human-AI collaborative search. Note that Baseline 2 and 3 are improvements based on the original algorithm (Baseline 1) since automatic problem detection and rule-based problem-solving strategies are used. Furthermore, we observe that the efficiency of AutoS$^2$earch (in terms of steps taken) is comparable to that of human-AI collaborative search, while the average execution time of large models is even shorter. For details on how human workers complete the crowdsourcing task, interested readers can refer to the previous work~\cite{zhao2022crowd}.

\begin{table}[!ht]
    \centering
    \caption{Results of the comparisons over various baselines.}
    \label{tab:results-1}
    \resizebox{0.9\textwidth}{!}{
    \begin{tabular}{llccc}
    \toprule
         \textbf{\emph{Methods}} & \textbf{Expertise} & \makecell[c]{\textbf{Effectiveness}\\(\% success rate)} & \makecell[c]{\textbf{Efficiency}\\(\# steps per task)} & \makecell[c]{\textbf{Human/MLLM+LLM execution time}\\(seconds per task)} \\
    \midrule
         \multirow{2}{*}{\emph{Human Aided}}
         & Expert   & 100  & 175.10 $\pm$ 67.67  & 33.58 $\pm$ 27.87\\
         & Non-expert  & 100  & 165.67 $\pm$ 80.60  & 29.01 $\pm$ 29.51\\
         \midrule
         \emph{Baseline 1} & -  & 78.5 & 154.04 $\pm$ 91.32  & - \\
         \midrule
         \emph{Baseline 2} & -  & 88  & 179.64 $\pm$ 96.45  & - \\
         \midrule
         \emph{Baseline 3} & -  & 90  & 179.76 $\pm$ 97.40  & - \\
         \midrule
         \emph{Ours} & -  & 97  & 170.97 $\pm$ 89.57  & 25.95 $\pm$ 38.20 \\
    \bottomrule
    \end{tabular}
    }
\end{table}

To further explore whether the impressive performance is solely due to GPT-4o's strong capabilities, we evaluated various combinations of MLLMs and LLMs from different companies. The results, presented in Table~\ref{tab:results-2}, reveal that our proposed framework is highly robust, consistently achieving success rates above 95\%. Notably, the Qwen model\footnote[3]{https://github.com/JMaiGC/ComfyUI-Qwen-VL-API} from the Chinese company Alibaba achieves the highest success rate at 98\%.

\vspace{-5mm}

\begin{table}[!ht]
    \centering
    \caption{Results of the comparisons over various large models.}
    \label{tab:results-2}
    \resizebox{0.9\textwidth}{!}{
    \begin{tabular}{llccc}
    \toprule
         \textbf{\emph{LLMs}} & \makecell[c]{\textbf{Effectiveness}\\(\% success rate)} & \makecell[c]{\textbf{Efficiency}\\(\# steps per task)} & \makecell[c]{\textbf{MLLM+LLM execution time}\\(seconds per task)} \\
    \midrule
         \emph{GLM-4v-plus + GLM-4-plus}  & 95 & 171.13 $\pm$ 92.64  & 26.39 $\pm$ 32.75 \\
         \midrule
         \emph{Qwen-VL-plus + Qwen-max}  & 98  & 172.85 $\pm$ 91.08  & 26.74 $\pm$ 36.39 \\
         \midrule
         \emph{GPT-4o + GPT-4o}  & 97  & 170.97 $\pm$ 89.57  & 25.95 $\pm$ 38.20 \\
    \bottomrule
    \end{tabular}
    }
\end{table}

\vspace{-5mm}

\subsection{Ablation Study}

We further ablation studies to validate the importance of main elements designed in our framework: the Chain-of-Thought prompt for the LLM and the size of directional choices A, B, C, and D (which determine the number of candidate cells for each option). We designated the model without CoT reasoning as Our-A and the model with reduced block sizes as Our-B. The average results across 20 scenarios are presented in Table \ref{tab:results-3}. As we can see, both the removal of CoT reasoning and the reduction in block sizes significantly decrease the success rate by approximately 6\% and 7\%, respectively. Notably, while removing CoT reasoning compromises the effectiveness performance, it does lead to improved efficiency and shorter execution time due to fewer reasoning steps.

\begin{table}[!ht]
    \centering
    \caption{Results of the ablation study.}
    \label{tab:results-3}
    \resizebox{0.8\textwidth}{!}{
    \begin{tabular}{llccc}
    \toprule
         \textbf{\emph{Methods}} & \makecell[c]{\textbf{Effectiveness}\\(\% success rate)} & \makecell[c]{\textbf{Efficiency}\\(\# steps per task)} & \makecell[c]{\textbf{MLLM+LLM execution time}\\(seconds per task)} \\
    \midrule
         \emph{Ours-A}  & 91 & 157.74 $\pm$ 85.33  & 23.82 $\pm$ 36.45 \\
         \midrule
         \emph{Ours-B}  & 90  & 170.28 $\pm$ 93.35  & 24.49 $\pm$ 34.19 \\
         \midrule
         \emph{Ours}  & 97  & 170.97 $\pm$ 89.57  & 25.95 $\pm$ 38.20 \\
    \bottomrule
    \end{tabular}
    }
\end{table}












\section{Discussions}

The source search results convey three main messages: (1) By incorporating carefully designed prompts that enable large language models with scene comprehension and multi-step reasoning capabilities, autonomous source search capabilities can be integrated into web-based systems to support decision-making in time-sensitive scenarios. (2) The large models-assisted method is effective and efficient for improving source search, approaching the performance of human-AI collaborative approaches while reducing execution time by approximately 25\%. (3) Whether in scene element presentation, problem detection mechanisms, or CoT prompt design, each component reflects human intelligence, highlighting that complex task solving fundamentally relies on human-AI hybrid intelligence.

\noindent\textbf{\textit{Drawbacks.}} Despite the strengths, this work has several limitations. (1) \textit{Environmental Complexity Gap:} The simplified $20 \times 20$ grid with static obstacles fail to capture real-world dynamics (e.g., moving obstructions, multi-source scenarios). The visual environment used here is insufficient to test whether large models truly possess robust scene understanding and multi-step reasoning capabilities in complex settings. (2) \textit{Limited Task Understanding:} While simple scene elements were designed to help the large model understand tasks, the lack of domain-specific knowledge makes it difficult for the model to balance exploration and exploitation during the search, sometimes leading to hallucinations by selecting irrelevant areas. (3) \textit{Underutilization of MLLM Potential}: In this work, MLLMs were mainly used to convert visual observations into textual descriptions, with large language models handling subsequent reasoning. This separation of visual understanding and language reasoning may limit the integrated capabilities MLLMs are designed to offer.

\noindent\textbf{\textit{Potential Avenues.}} To address these limitations, we propose to explore: (1) \textit{Dynamic Environment Adaptation:} Design LLM-empowered search agent and develop online prompt tuning mechanisms where LLMs could adjust decision rules according to the environment variations. (2) \textit{Visual Thinking Augmentation:} Integrate graph-based scene representations and reflection mechanisms to help MLLMs directly reason on the visual inputs without hallucinations. (3) \textit{Human-AI Value Alignment:} Implement human-in-the-loop feedback mechanisms in complex and high-risk scenarios and ensure alignment of decision objectives between humans and AI.

\noindent\textbf{\textit{Implications.}} The implications of AutoS$^2$earch extend far beyond the technical achievements in web-based autonomous systems. Its design reflects a broader trend in human-AI collaborative systems, where the goal is to harness the cognitive strengths of both entities in tandem.  Moreover, it may redefine the role of humans in web crowdsourcing systems—from task executors to validators of AI rationality in the future. 



\section{Conclusion}
\label{sec:conclusion}

In summary, \dataset and \method collectively advance open-source LLM-based autonomous agents by addressing critical gaps in pre-training corpora. 
Through exhaustive scaling law experiments, we identify an empirically optimal data mix ratio of approximately 1:1:1 for agent, code, and text data, maximizing the fundamental and generalization capabilities of LLM agents. 
Empirical evaluations underscore the efficacy and validity of \dataset in fostering enhanced fundamental agentic capabilities and superior generalization in LLM-based autonomous agents.



% This is probably the first work discussing the agent capability in pretraining? for these we study the recipe, scaling law, and train the models to verify the effectiveness
% insights
% I think it should be agent pre-training recipe not a single checkpoint. The recipe includes data and algorithm. There is no much to say about the algorithm right? So I feel the data is more important, all other points are like supporting the effectiveness and validness of the data. Sources -> how we get the data originally. Scaling law -> how we get the mixing ratios. Experiments and model checkpoints -> why the data is effective.


% In summary, \dataset and \method collectively advance the state-of-the-art in LLM-based autonomous agents by addressing critical gaps in pre-training data and training methodologies. This advancement paves the way for more adaptable, intelligent, and robust autonomous agents capable of performing effectively across diverse and dynamic real-world environments.

\section*{Limitations}
\noindent \textbf{Data Composition.} 
While knowledge of the composition of pre-training or instruction fine-tuning data would further enhance the effectiveness of \method, most prominent open-source LLMs (\eg, \texttt{LLaMA-3-8B-Instruct}) do not disclose detailed data information. Nevertheless, our continual pre-training experiments with \texttt{LLaMA-3-8B} demonstrate that significant improvements are achievable even without this knowledge. 

\noindent \textbf{Model Scalability.} Computational constraints currently restrict our ability to extend these experiments to larger models. In future work, we aim to validate our findings and methodologies on more expansive LLM architectures, pending access to increased computational resources.

% \noindent \textbf{Training Objective and Agent Frameworks.} The primary focus of this work is on building an effective dataset creation recipe to scale up retrieval models for the biomedical domain.
% However, other biomedical LLMs can also be used as the backbones~\citep{luo2022biogpt,bolton2024biomedlm,chen2023meditron}. 
% In our early attempts, we did not observe significant performance gains with some of these backbones (e.g., BioGPT-Large, BioMedLM-2.7B, Meditron-7B), potentially due to the loose connection between the objective of sequence contrastive learning and causal language modeling. 
% Besides, since we utilize standard contrastive learning objectives with hard negative mining without proposing new learning objectives, we \emph{do not claim} this as our core contribution. 
% Exploring ways to leverage domain-specific LLMs with improved training techniques to enhance performance is an interesting topic for future research.
% How to further leverage domain-specific LLMs with improved training techniques to bolster the performance is an interesting topic. 

\section*{Ethical Statement}
\noindent \textbf{Data Contamination.} 
A potential concern in our evaluations is test set contamination, which occurs when some task-specific examples overlap with data used during continual pre-training~\cite{oren2024proving}. To mitigate this issue, we follow~\citet{wang-etal-2024-improving-text} and conduct a string-matching analysis, which indicates no overlap between our training data and the datasets of the target tasks. Moreover, we intentionally exclude all evaluation benchmark data from both our pre-training and fine-tuning datasets to ensure a fair comparison.

\noindent \textbf{Reproducibility.}
To promote transparency, reproducibility, and generalizability in our research, we include all details of the dataset construction (\eg, data collection, processing, retrieving, filtering, scaling law, \etc) of \dataset in \cref{sec:dataset} and the training procedures for \method in \cref{sec:method}. 
Experimental setups and results are presented in \cref {sec:exp}.
Additionally, we detail the pre-training, instruction fine-tuning, and testing tasks and datasets in \cref{app:data-pretrain,app:data-ift,app:data}, respectively. 
% All code, data, and models will be made publicly available on GitHub and HuggingFace upon acceptance.


% \begin{credits}
% \subsubsection{\ackname} This study is funded by Youth Independent Innovation Foundation of NUDT (ZK-2023-21) and National Natural Science Foundation of China (no. 62202477, 62173337).

% % \subsubsection{\discintname}
% % The authors have no competing interests to declare that are relevant to the content of this article.
% \end{credits}


% ---- Bibliography ----

\bibliographystyle{splncs04}
\bibliography{ref}

\end{document}
