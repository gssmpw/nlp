\section{Background and Motivation}
In this section, we review related works across three key areas and then outline the motivation behind this study.

\subsection{Web Crowdsourcing and Human-AI Collaboration Empowerment}
% Crowdsourcing~\cite{howe2006rise} refers to the practice of acquiring ideas, services, or content by soliciting contributions from a large group of people. 
With the advancement of web technologies, crowdsourcing activities have increasingly migrated to web and mobile internet platforms, namely web crowdsourcing~\cite{doan2011crowdsourcing}. An exponential rise in its applications has witnessed, such as ride-hailing and software development.
To tackle complex web-based tasks, scientists at Microsoft introduced human-AI interaction guidelines to assist researchers and practitioners in designing studies utilizing AI technologies~\cite{amershi2019guidelines}. Following this, numerous studies have integrated human intelligence with AI methods to address challenges such as conversational agent learning for intent detection and text classification~\cite{yang2018leveraging,arous2021marta}. A recent study, for example, engaged online users from crowdsourcing platforms and implemented advanced computer vision techniques to generate city maps~\cite{qiu2019crowd}. Given the growing significance of AI-in-the-loop systems in human-intervened tasks, the concept and principles of human-AI decision-making within the context of web crowdsourcing were provided~\cite{green2019principles}. 

% Thus, we have witnessed an exponential rise in applications built around the concept of crowdsourcing~\cite{howe2006rise}--from ride-hailing~\cite{seng2023ridesharing} and food delivery~\cite{liu2018foodnet} to software development~\cite{abd2021use} and urban governance ~\cite{qiu2019crowd}. The "algorithmic crowdsourcing" paradigm is exemplified where web architectures coordinate human-machine interactions at scale. In light of the practical effectiveness in these domains, web crowdsourcing is now expanding into more complex task domains such as source search (e.g., locating gas leaks or biological signals)~\cite{zhao2024user}.

\subsection{Source Search and Crowd-powered Practices}
Source search is a critical problem for both nature and mankind~\cite{jing2021recent} focusing on determining the location of a source (of gas or signal) in the shortest possible time. Existing source search approaches can generally be classified into three categories: information-theoretic~\cite{jang2023improved}, biologically-inspired~\cite{al2021distributed}, and gradient-based methods~\cite{jiang2019source}. Among these, information-theoretic algorithms, especially those grounded in the Bayesian framework~\cite{ojeda2024robotic}, stand out for their distinct advantages. To further enhance the performance (i.e., success rate and efficiency) of a searching algorithm, multi-robot collaboration mechanisms~\cite{tang2020multirobot} have been designed and adopted. However, when source search takes place in complex environments, the search process always encounters fatal problems, resulting in wrong outcomes. Thus, researchers started to explore effective ways leveraging human intelligence to improve AI-based search algorithms through web platforms~\cite{zhao2024user}. However, this approach also entails substantial costs and imposes considerable burdens on human workers.

\subsection{Large Models for Scene Understanding and Reasoning}
MLLMs integrate multimodal encoders/decoders with traditional LLMs, enabling cross-modal understanding that overcomes text-only limitations. While these models demonstrate remarkable capabilities across diverse tasks including image-text understanding~\cite{liu2024visual}, video-text understanding~\cite{li2023videochat}, and even multimodal generation~\cite{peng2023kosmos}, their effectiveness in handling complex tasks remains constrained by predominant single-step reasoning approaches. To this end, CoT prompts are utilized to enhance problem-solving abilities by guiding LLMs through structured multi-step reasoning. Recent work explores CoT adaptations for multimodal problems, for instance, Shikra~\cite{chen2023shikra} pioneers CoT application in visual grounding tasks, while SoM~\cite{yang2023set} introduces structural image annotations like segmentation maps and spatial grids to provide spatial reasoning anchors. However, CoT has not been comprehensively explored for fine-grain reasoning in source search tasks.

\subsection{Motivation}
Building on the demonstrated scene understanding and reasoning capabilities of large models across various tasks, as well as addressing the limitations of human-AI collaborative source search, our work seeks to explore concrete methods for leveraging large models in zero-shot source search tasks within a top-down view of web-based search environments.


