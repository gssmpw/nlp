\section{Background and Related Work}
The incentive incompatibility of disclosure is a classic economic dilemma. And this incompatibility imposes serious costs: inventors cannot trust that their disclosures will remain protected, and so many potential breakthroughs may remain hidden or never be developed in the first place. And, since new innovations themselves are often spurred through recombining of ideas \citep{fleming2001recombinant}, this effect can have long-lasting stifling implications.

Under incomplete contracts, parties must balance the need to disclose private information to realize gains from trade against the risk that disclosed information may be appropriated \citep{arrow1972economic}. This trade-off is especially stark for innovators or entrepreneurs, who often incur large fixed costs to develop a new product or idea but cannot attract funding without revealing it \citep{nelson1959simple}. 

Our analysis builds on several literature streams. First, we extend work on information disclosure in contracting \citep{crawford1982strategic, okuno1991incentive} by explicitly modeling disclosure as a continuous choice under uncertainty. Second, we connect to research on hold-up problems and incomplete contracts \citep{hart1988incomplete, aghion1992innovation, grossman1986disclosure, bernheim1987sequential, anton1994expropriation}, showing how technological solutions like trusted execution environments can mitigate appropriation risks.  AI agents in our treatment are not quite \emph{automata} in a machine game, as in \citet{rubinstein1986finite} but could modeled as such. Our treatment of TEE-resident AI agents is closer to the principle agent setting of \citet{aghion1997formal}, where agents have some \emph{congruence parameter} measuring how closely its objectives match the principal’s.

By introducing cryptographic and hardware-based solutions, our framework departs from traditional reliance on legal instruments such as patents or NDAs, offering a technologically enforced approach to secure collaboration. This complements the partial disclosure focus in \citet{anton1994expropriation, anton2002sale} by suggesting that if inventors can reliably limit expropriation through secure hardware, they may opt for more complete disclosure earlier in the R\&D timeline, thereby accelerating cumulative innovation in the spirit of \citet{Scotchmer1999}.

Finally, we also note that this problem mirrors the time-priority exploitation common in blockchain MEV \citep{daian2020flash}, where e.g. order flow information necessary for market function enables front-running. This is a major inspiration for our solution as well as connecting this work to the literature on MEV and mechanism design in decentralized systems \citep{roughgarden2020transaction, capponi2023adoption}.

\paragraph{Contributions.}
We highlight our key contributions as follows:
\begin{enumerate}[noitemsep]
\item \textbf{Formalizing the disclosure–expropriation trade‐off.} We develop a simple game‐theoretic model in which a seller (inventor) chooses how much to disclose to a buyer (investor) before a potential transaction, under the threat of expropriation. Absent any protective mechanism, full or partial disclosure is thwarted by hold‐up.

\item \textbf{Introducing a TEE‐based mechanism to resolve hold‐up.} 
We show that delegating decisions to AI agents operating inside a trusted execution environment (TEE) can render disclosure incentive‐compatible. Under sufficient security, full disclosure and investment can be achieved, raising total surplus and yielding Pareto improvements.

\item \textbf{Modeling TEE security risk for high‐value secrets.}
To address the concern that real‐world TEEs are not perfectly secure, we formalize the expected gain from malfeasance using threshold encryption and positive detection probabilities from the TEE. This yields a “scope condition” under which even high‐value ideas can be partially or fully protected, bridging theory and practical adoption.

\item \textbf{Extending the analysis to imperfect (“noisy”) agents.} 
We relax the assumption of perfectly congruent agents to allow for random errors in payments or disclosure. We show that a simple budget cap for the buyer’s agent and a reject‐option for the seller’s agent can contain the risk of overpayments or underpayments, preserving most gains from trade even with high error rates.

\item \textbf{Implications for policy and mechanism design.} 
Our results illustrate how cryptographic or hardware‐based safeguards can substitute for—rather than merely supplement—costly legal instruments like NDAs. This has broad ramifications for protecting intellectual property, incentivizing R\&D, and promoting collaborative innovation across firms and industries.
\end{enumerate}