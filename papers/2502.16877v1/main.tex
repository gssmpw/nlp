%%
%% This is file `sample-sigconf.tex',
%% generated with the docstrip utility.
%%
%% The original source files were:
%%
%% samples.dtx  (with options: `all,proceedings,bibtex,sigconf')
%% 
%% IMPORTANT NOTICE:
%% 
%% For the copyright see the source file.
%% 
%% Any modified versions of this file must be renamed
%% with new filenames distinct from sample-sigconf.tex.
%% 
%% For distribution of the original source see the terms
%% for copying and modification in the file samples.dtx.
%% 
%% This generated file may be distributed as long as the
%% original source files, as listed above, are part of the
%% same distribution. (The sources need not necessarily be
%% in the same archive or directory.)
%%
%%
%% Commands for TeXCount
%TC:macro \cite [option:text,text]
%TC:macro \citep [option:text,text]
%TC:macro \citet [option:text,text]
%TC:envir table 0 1
%TC:envir table* 0 1
%TC:envir tabular [ignore] word
%TC:envir displaymath 0 word
%TC:envir math 0 word
%TC:envir comment 0 0
%%
%%
%% The first command in your LaTeX source must be the \documentclass
%% command.
%%
%% For submission and review of your manuscript please change the
%% command to \documentclass[manuscript, screen, review]{acmart}.
%%
%% When submitting camera ready or to TAPS, please change the command
%% to \documentclass[sigconf]{acmart} or whichever template is required
%% for your publication.
%%
%%
\documentclass[sigconf]{acmart}

\newcommand{\sysname}{APINT }
\usepackage{algorithm}
\usepackage{algpseudocode}
\usepackage{amsmath}
%\usepackage{amssymb}
\usepackage{indentfirst}
\usepackage{graphicx}
\usepackage{caption}
\usepackage{tikz}
\usepackage{enumitem}
\usepackage{titlesec}



\newcommand*\circled[1]{\tikz[baseline=(char.base)]{
            \node[shape=circle,draw,inner sep=1pt] (char) {\scriptsize #1};}}

\captionsetup[figure]{skip=5pt} % 그림과 캡션 사이 간격 조정
% \setlength{\textfloatsep}{5pt plus 1.0pt minus 1.0pt} % 캡션과 글 사이 간격 조정

%%
%% \BibTeX command to typeset BibTeX logo in the docs
\AtBeginDocument{%
  \providecommand\BibTeX{{%
    Bib\TeX}}}

%% Rights management information.  This information is sent to you
%% when you complete the rights form.  These commands have SAMPLE
%% values in them; it is your responsibility as an author to replace
%% the commands and values with those provided to you when you
%% complete the rights form.


\copyrightyear{2024}
\acmYear{2024}
\setcopyright{rightsretained}
\acmConference[ICCAD '24]{IEEE/ACM International Conference on
Computer-Aided Design}{October 27--31, 2024}{New York, NY, USA}
\acmBooktitle{IEEE/ACM International Conference on Computer-Aided Design
(ICCAD '24), October 27--31, 2024, New York, NY, USA}
\acmDOI{10.1145/3676536.3676786}
\acmISBN{979-8-4007-1077-3/24/10}

% 주석 제거하는코드
% \settopmatter{printacmref=false}
% \renewcommand\footnotetextcopyrightpermission[1]{}
\pagestyle{plain}
%%%%%%%%%%


%%
%% Submission ID.
%% Use this when submitting an article to a sponsored event. You'll
%% receive a unique submission ID from the organizers
%% of the event, and this ID should be used as the parameter to this command.
\acmSubmissionID{1237}

%%
%% For managing citations, it is recommended to use bibliography
%% files in BibTeX format.
%%
%% You can then either use BibTeX with the ACM-Reference-Format style,
%% or BibLaTeX with the acmnumeric or acmauthoryear sytles, that include
%% support for advanced citation of software artefact from the
%% biblatex-software package, also separately available on CTAN.
%%
%% Look at the sample-*-biblatex.tex files for templates showcasing
%% the biblatex styles.
%%

%%
%% The majority of ACM publications use numbered citations and
%% references.  The command \citestyle{authoryear} switches to the
%% "author year" style.
%%
%% If you are preparing content for an event
%% sponsored by ACM SIGGRAPH, you must use the "author year" style of
%% citations and references.
%% Uncommenting
%% the next command will enable that style.
%%\citestyle{acmauthoryear}


%%
%% end of the preamble, start of the body of the document source.
\begin{document}

%%
%% The "title" command has an optional parameter,
%% allowing the author to define a "short title" to be used in page headers.
\title{APINT: A Full-Stack Framework for Acceleration of Privacy-Preserving Inference of Transformers based on Garbled Circuits}

%%
%% The "author" command and its associated commands are used to define
%% the authors and their affiliations.
%% Of note is the shared affiliation of the first two authors, and the
%% "authornote" and "authornotemark" commands
%% used to denote shared contribution to the research.


% \numberofauthors{4}
% % Three authors sharing the same affiliation.
%     \author{
%       \alignauthor Hyunjun Cho\\      
%       \email{h.cho@kaist.ac.kr}
% %
%       \alignauthor Jaeho Jeon\\     
%       \email{math15738@kaist.ac.kr}
% %
%       \alignauthor Jaehoon Heo\\    
%       \email{kd01050@kaist.ac.kr}
      
%       \alignauthor Joo-Young Kim\\    
%       \email{jooyoung1203@kaist.ac.kr}
% %
%       \sharedaffiliation
%       % \affaddr{Department of Electrical Engineering and Computer Science}  \\
%       \affaddr{KAIST, Daejeon }   \\
%           }

% \author{Hyunjun Cho, Jaeho Jeon, Jaehoon Heo, Joo-Young Kim}
% \affiliation{%
%  \institution{School of Electrical Engineering, KAIST}
% }
% \affiliation{%
%   \institution{\{h.cho, math15738, kd01050, jooyoung1203\}@kaist.ac.kr}
% }

\settopmatter{authorsperrow=4}
\author{Hyunjun Cho}
% \authornote{Both authors contributed equally to this research.}
\email{h.cho@kaist.ac.kr}
\orcid{0009-0008-2378-852X}
\affiliation{%
  \institution{KAIST}
  \city{Daejeon}
  \country{South Korea}
}

\author{Jaeho Jeon}
% \authornote{Both authors contributed equally to this research.}
\email{math15738@kaist.ac.kr}
\orcid{0009-0008-2250-6068}
\affiliation{%
  \institution{KAIST}
  \city{Daejeon}
  \country{South Korea}
}

\author{Jaehoon Heo}
% \authornote{Both authors contributed equally to this research.}
\email{kd01050@kaist.ac.kr}
\orcid{0000-0003-1742-4275}
\affiliation{%
  \institution{KAIST}
  \city{Daejeon}
  \country{South Korea}
}

\author{Joo-Young Kim}
% \authornote{Both authors contributed equally to this research.}
\email{jooyoung1203@kaist.ac.kr}
\orcid{0000-0003-1099-1496}
\affiliation{%
  \institution{KAIST}
  \city{Daejeon}
  \country{South Korea}
}


\begin{abstract}
As the importance of Privacy-Preserving Inference of Transformers (PiT) increases, a hybrid protocol that integrates Garbled Circuits (GC) and Homomorphic Encryption (HE) is emerging for its implementation. While this protocol is preferred for its ability to maintain accuracy, it has a severe drawback of excessive latency. 
%To address this, the existing protocol decreased its latency by primarily focusing on HE latency but not on the latency of GC. Consequently, GC has become the primary bottleneck of the protocol, remaining an unresolved challenge. Furthermore, despite efforts in various sectors to reduce its latency, previous studies have not optimally minimized its overhead, nor have they provided a comprehensive solution.
To address this, existing protocols primarily focused on reducing HE latency, thus making GC the new latency bottleneck. Furthermore, previous studies only focused on individual computing layers, such as protocol or hardware accelerator, lacking a comprehensive solution at the system level.
%Furthermore, previous studies only worked on particular aspects such as reducing the GC workload and employing a GC hardware accelerator, unable to provide a comprehensive solution with holistic optimization.

% However, previous efforts to reduce GC workload by decreasing the number of AND gates in circuits have not achieved optimal reductions. Additionally, prior attempts to mitigate GC latency through hardware accelerators have encountered severe memory bottlenecks and data dependencies when processing PiT. These issues highlight the urgent need for innovative approaches to accelerate the PiT by addressing GC latency.

% The proliferation of cloud-based services using pre-trained Transformer models for NLP tasks has highlighted critical privacy issues, as sensitive data such as financial and health information are at risk of exposure. This challenge underscores the growing necessity for Privacy-preserving inference on Transformers (PiT). However, PiT faces significant latency challenges, primarily due to the overhead associated with Garbled Circuits (GC). Existing approaches have not sufficiently addressed the reduction of GC overhead, nor have they effectively minimized the computational workload, especially in reducing the number of AND gates in circuit designs. Additionally, despite advancements in accelerator technology for GC, they lack a practical solution for managing nonlinear functions within transformer models.

This paper presents APINT, a full-stack framework designed to reduce PiT's overall latency by addressing the latency problem of GC through both software and hardware solutions. APINT features a novel protocol that reallocates possible GC workloads to alternative methods (i.e., HE or standard matrix operation), substantially decreasing the GC workload. It also suggests GC-friendly circuit generation that reduces the number of AND gates at the most, which is the expensive operator in GC. Furthermore, APINT proposes an innovative netlist scheduling that combines coarse-grained operation mapping and fine-grained scheduling for maximal data reuse and minimal dependency. Finally, APINT's hardware accelerator, combined with its compiler speculation, effectively resolves the memory stall issue. Putting it all together, APINT achieves a remarkable end-to-end reduction in latency, outperforming the existing protocol on CPU platform by 12.2$\times$ online and 2.2$\times$ offline. Meanwhile, the APINT accelerator not only reduces its latency by 3.3$\times$ but also saves energy consumption by 4.6$\times$ while operating PiT compared to the state-of-the-art GC accelerator.

% \vspace{-0.04in}

% The rising importance of Privacy-preserving inference on Transformers (PiT) is driven by the need to protect sensitive client data, like financial and health information, which is potentially exposed when using cloud-based, pre-trained Transformer models for NLP tasks. However, it suffers from significant latency issues, particularly since Garbled Circuits (GC) account for the majority of the online latency.
% Prior studies have identified several challenges in accelerating Privacy-preserving inference on Transformers (PiT) due to the latency issues primarily associated with Garbled Circuits (GC), which contribute significantly to online latency. Existing protocols have not addressed the reduction of GC overhead. Additionally, prior methods have not adequately minimized the workload, particularly in terms of reducing the number of AND gates. Furthermore, while advancements have been made in the development of accelerators for GC, they have not proposed a practical solution for nonlinear functions in transformer models. 
% This paper introduces APINT, a full-stack framework designed to enhance the efficiency of PiT across both the software and hardware side. First, it employs a novel protocol that mitigates online latency by redistributing GC workloads between alternative computational methods and the preprocessing phase. Second, it suggests GC-Friendly netlist generation to significantly cut down the number of AND gates, effectively decreasing both GC computation and communication latency. Third, APINT utilizes coarse-grained and fine-grained scheduling to reduce data dependencies and enhance data reuse. Fourth, the APINT hardware accelerator with compiler speculation optimizes data reusability and minimize unnecessary DRAM accesses. Collectively, APINT achieves significant latency reductions---58.5\% offline and 89.9\% online---resulting in a total latency decrease of 66.6\% compared to the conventional method. Specifically, the APINT accelerator reduces latency by 23.4\% and system energy consumption by 78.1\% compared to the SOTA GC accelerator.


\end{abstract}


\maketitle

\section{Introduction}

Video generation has garnered significant attention owing to its transformative potential across a wide range of applications, such media content creation~\citep{polyak2024movie}, advertising~\citep{zhang2024virbo,bacher2021advert}, video games~\citep{yang2024playable,valevski2024diffusion, oasis2024}, and world model simulators~\citep{ha2018world, videoworldsimulators2024, agarwal2025cosmos}. Benefiting from advanced generative algorithms~\citep{goodfellow2014generative, ho2020denoising, liu2023flow, lipman2023flow}, scalable model architectures~\citep{vaswani2017attention, peebles2023scalable}, vast amounts of internet-sourced data~\citep{chen2024panda, nan2024openvid, ju2024miradata}, and ongoing expansion of computing capabilities~\citep{nvidia2022h100, nvidia2023dgxgh200, nvidia2024h200nvl}, remarkable advancements have been achieved in the field of video generation~\citep{ho2022video, ho2022imagen, singer2023makeavideo, blattmann2023align, videoworldsimulators2024, kuaishou2024klingai, yang2024cogvideox, jin2024pyramidal, polyak2024movie, kong2024hunyuanvideo, ji2024prompt}.


In this work, we present \textbf{\ours}, a family of rectified flow~\citep{lipman2023flow, liu2023flow} transformer models designed for joint image and video generation, establishing a pathway toward industry-grade performance. This report centers on four key components: data curation, model architecture design, flow formulation, and training infrastructure optimization—each rigorously refined to meet the demands of high-quality, large-scale video generation.


\begin{figure}[ht]
    \centering
    \begin{subfigure}[b]{0.82\linewidth}
        \centering
        \includegraphics[width=\linewidth]{figures/t2i_1024.pdf}
        \caption{Text-to-Image Samples}\label{fig:main-demo-t2i}
    \end{subfigure}
    \vfill
    \begin{subfigure}[b]{0.82\linewidth}
        \centering
        \includegraphics[width=\linewidth]{figures/t2v_samples.pdf}
        \caption{Text-to-Video Samples}\label{fig:main-demo-t2v}
    \end{subfigure}
\caption{\textbf{Generated samples from \ours.} Key components are highlighted in \textcolor{red}{\textbf{RED}}.}\label{fig:main-demo}
\end{figure}


First, we present a comprehensive data processing pipeline designed to construct large-scale, high-quality image and video-text datasets. The pipeline integrates multiple advanced techniques, including video and image filtering based on aesthetic scores, OCR-driven content analysis, and subjective evaluations, to ensure exceptional visual and contextual quality. Furthermore, we employ multimodal large language models~(MLLMs)~\citep{yuan2025tarsier2} to generate dense and contextually aligned captions, which are subsequently refined using an additional large language model~(LLM)~\citep{yang2024qwen2} to enhance their accuracy, fluency, and descriptive richness. As a result, we have curated a robust training dataset comprising approximately 36M video-text pairs and 160M image-text pairs, which are proven sufficient for training industry-level generative models.

Secondly, we take a pioneering step by applying rectified flow formulation~\citep{lipman2023flow} for joint image and video generation, implemented through the \ours model family, which comprises Transformer architectures with 2B and 8B parameters. At its core, the \ours framework employs a 3D joint image-video variational autoencoder (VAE) to compress image and video inputs into a shared latent space, facilitating unified representation. This shared latent space is coupled with a full-attention~\citep{vaswani2017attention} mechanism, enabling seamless joint training of image and video. This architecture delivers high-quality, coherent outputs across both images and videos, establishing a unified framework for visual generation tasks.


Furthermore, to support the training of \ours at scale, we have developed a robust infrastructure tailored for large-scale model training. Our approach incorporates advanced parallelism strategies~\citep{jacobs2023deepspeed, pytorch_fsdp} to manage memory efficiently during long-context training. Additionally, we employ ByteCheckpoint~\citep{wan2024bytecheckpoint} for high-performance checkpointing and integrate fault-tolerant mechanisms from MegaScale~\citep{jiang2024megascale} to ensure stability and scalability across large GPU clusters. These optimizations enable \ours to handle the computational and data challenges of generative modeling with exceptional efficiency and reliability.


We evaluate \ours on both text-to-image and text-to-video benchmarks to highlight its competitive advantages. For text-to-image generation, \ours-T2I demonstrates strong performance across multiple benchmarks, including T2I-CompBench~\citep{huang2023t2i-compbench}, GenEval~\citep{ghosh2024geneval}, and DPG-Bench~\citep{hu2024ella_dbgbench}, excelling in both visual quality and text-image alignment. In text-to-video benchmarks, \ours-T2V achieves state-of-the-art performance on the UCF-101~\citep{ucf101} zero-shot generation task. Additionally, \ours-T2V attains an impressive score of \textbf{84.85} on VBench~\citep{huang2024vbench}, securing the top position on the leaderboard (as of 2025-01-25) and surpassing several leading commercial text-to-video models. Qualitative results, illustrated in \Cref{fig:main-demo}, further demonstrate the superior quality of the generated media samples. These findings underscore \ours's effectiveness in multi-modal generation and its potential as a high-performing solution for both research and commercial applications.
\section{Background and Motivation}

\begin{figure}[t]
    \vspace{-0.2in}
    \centering
    \includegraphics[width=1\linewidth]{Figures/gc_overview_ver3.pdf}
    \caption{Garbled Circuit Protocol Overview}
    \vspace{-0.2in}
    \label{fig:garbled_circuit}
\end{figure}

\subsection{Protocols for Privacy-Preserving Inference}
\subsubsection{\textbf{Homomorphic Encryption}}
HE is a cryptographic method allowing operations on encrypted data (ciphertext) that produce the same results as if performed on unencrypted data (plaintext). It supports operations like addition and multiplication between plaintext and ciphertext ($Enc(X)+Y = Enc(X+Y)$, $Enc(X)\times Y = Enc(X\times Y)$), and among ciphertexts ($Enc(X)+Enc(Y) = Enc(X+Y)$, $Enc(X)\times Enc(Y) = Enc(X\times Y)$).

% Multiplying ciphertexts is more resource-intensive than operating between plaintext and ciphertext. To optimize efficiency, the upcoming protocol restricts operations to plaintext-ciphertext interactions.

% ($Enc$ refers to the process of encrypting plaintext to convert it into ciphertext.)\\
% \textit{1) ciphertext-plaintext addition:} $Enc(X)+Y = Enc(X+Y)$\\
% \textit{2) ciphertext-plaintext multiplication:} $Enc(X)Y = Enc(XY)$

\subsubsection{\textbf{Garbled Circuits}}
% Garbled Circuit detailed
GC involves two parties---the Garbler and the Evaluator---jointly compute without disclosing their inputs. The GC process involves four steps, as shown in Figure~\ref{fig:garbled_circuit}.\\
\textit{1) Circuit and Netlist Generation: } 
The function to be computed is represented as a circuit of 2-input logic gates, and it is converted to a netlist format, which contains information about each gate's inputs (input wires), output (output wire), and type (ex. AND).\\
\textit{2) GC Garbling: } 
For each wire \textit{i} in the netlist, the garbler randomly generates a 128b label ($W_i^0$) corresponding to 0. The label ($W_i^1$) corresponding to 1 is then produced by performing an XOR operation on ($W_i^0$) with a random 128b value R, which is fixed and public. After assigning the labels, the garbler constructs a truth table for each logic gate and encrypts it to create the garbled table. Finally, the garbler selects the corresponding label for its input wire ($a \in (W_a^0, W_a^1)$), which will be used in GC evaluation.\\
\textit{3) Garbler-Evaluator Communication:}
The garbler transmits the garbled tables and the selected label to the evaluator. Additionally, the garbler sends the label, corresponding to the evaluator's input wire ($b \in (W_b^0, W_b^1)$), to the evaluator without knowing the wire's value. This can be achieved through Oblivious Transfer (OT) protocol ~\cite{ishai2003extending}.\\
\textit{4) GC Evaluation:}
The evaluator calculates the output for each gate using the garbled tables and the labels provided by both the garbler and evaluator. The output becomes the new label of the input wire of the next gates, and the evaluations proceed sequentially for all gates in the netlist.

Recent enhancements in GC---Half-Gate~\cite{zahur2015two} for AND gates and FreeXOR~\cite{kolesnikov2008improved} for XOR gates---have reduced computational demands and memory footprints. To facilitate this, the netlist employs solely AND, XOR, and INV gates. It should be noted that INV gates, unlike AND and XOR gates, can be implemented at no cost by simply removing them and inverting the correspondence between the values and labels of the wires. In the garbling phase, contrary to the original GC that assigns labels of output wires randomly, the Half-Gate operation for AND gates produces an output wire's label and two garbled tables through four AES computations of the input wires' labels. Conversely, the FreeXOR operation for XOR gates generates an output wire's label by XOR computation of the input wires' labels without creating any garbled tables.
During the evaluation phase, a Half-Gate operation produces an output wire's label by garbled tables and two AES computations of input wires' labels, and a FreeXOR operation computes an output wire's label directly through an XOR operation of input wires' labels.




\begin{figure}[t]
    \vspace{-0.2in}
    \centering
    \includegraphics[width=1\linewidth]{Figures/Camera-ready/Motivation_Reduced.pdf}
    % \includegraphics[width=1\linewidth]{Figures/motivation.pdf}
    \caption{Latency Analysis of Prior Works (a) PRIMER Protocol and (b) HAAC}
    \vspace{-0.2in}
    \label{fig:APINT_motivation}
\end{figure}
    
\subsection{Motivation for \sysname}

Previous studies have several issues in effectively processing PiT by speeding up GC, the primary bottleneck.
First, in terms of protocol, PRIMER~\cite{zheng2023primer} has solely focused on optimizing HE and has not introduced any methods to reduce the latency of GC. As depicted in Figure~\ref{fig:APINT_motivation} (a), which presents the latency result for a single inference of the BERT-base model~\cite{devlin2018bert} with 128 tokens using the PRIMER protocol on a CPU, GC evaluation accounts for 94.7\% of the online latency. Moreover, GC garbling, along with the transfer of labels and garbled tables, contributes to 77.9\% of the offline latency. Therefore, there is a significant need for a new protocol that substantially reduces the latency of GC.

Second, regarding circuit generation, prior works~\cite{testa2019reducing, testa2020logic, liu2022don} haven't proposed the optimal way to reduce the number of AND gates. Given that Half-Gate operations are more complex than FreeXOR operations, reducing the number of AND gates can significantly decrease the GC workload. To achieve this, previous works have optimized the XOR-AND-Graph (XAG) in which the circuit is converted into a DAG. However, these approaches failed to realize a more optimal method by overlooking modifications to the fundamental implementation of the circuit.

Third, with respect to the accelerator, previous works~\cite{hussain2019fase, mo2023haac} mitigated the latency issue of GC, but none of them proposed practical solutions to operate the nonlinear functions of transformers. FASE~\cite{hussain2019fase}, which benchmarked with a small dataset, assumed that the on-chip itself could cover all the wires required during the operation.
However, this approach is infeasible since the nonlinear functions have many more wires than the on-chip can physically support. Although HAAC~\cite{mo2023haac} partially solved this problem with an additional scheme utilizing off-chip memory, it suffers from memory stalls and pipeline stalls as described in Figure~\ref{fig:APINT_motivation} (b) due to suboptimal scheduling, inefficient on-chip memory policies, and hardware structures not considering wire reusability.
% However, it suffers from memory bottlenecks when operating nonlinear functions of transformers due to suboptimal scheduling, as well as on-chip memory management policies and hardware structures not mainly designed to consider wire reusability.

\sysname aims to overcome these deficiencies by providing the full stack solution while fulfilling four key requirements. First, it must adhere to a new protocol that reduces the GC computations, the main bottleneck of PiT. Second, it should generate the circuit in a GC-friendly way that ensures a reduction of GC workloads to decrease latency and memory footprint. Third, it requires appropriate scheduling to reduce memory stalls and pipeline stalls. Fourth, it necessitates an accelerator equipped with the compiler, which resolves memory bottlenecks and redundant DRAM traffic.

% Third, it requires appropriate scheduling and accelerator architecture to resolve latency overhead caused by memory bottlenecks and wire dependencies.
\section{\sysname Framework}

\begin{figure}[t]
    \vspace{-0.2in}
    \centering
    \includegraphics[width=1\linewidth]{Figures/apint_framework.pdf}
    \caption{Overall \sysname Framework}
    \vspace{-0.2in}
    \label{fig:APINT_framework}
\end{figure}

In this section, we propose APINT, a full-stack framework designed to accelerate PiT by reducing the overhead of GC, the primary bottleneck of PiT. The overall workflow is illustrated in Figure~\ref{fig:APINT_framework}, distinguishing between compile-time and runtime processes.
In the initial compile time stage, \sysname begins by extracting nonlinear function operations in the process of computing the transformer model through the \sysname protocol. Next, through the GC-friendly circuit generation, the extracted function is implemented as a circuit consisting of a 2-input gate, and it is converted to netlist in Bristol format~\cite{tillich2016circuits}.
This step significantly alleviates the computational load on GC in subsequent stages. Following this, \sysname adopts a scheduling strategy that combines coarse-grained and fine-grained scheduling. The strategy enables full utilization of DRAM bandwidth and decrement in wire dependency. Furthermore, \sysname incorporates compiler speculation to generate instructions that capitalize on wire reusability and are executed on hardware accelerators at runtime. These accelerators, designed to further reduce memory stalls by eliminating redundant DRAM accesses, are deployed on both the server and client, allowing them to perform GC evaluation or GC garbling.
% These accelerators are deployed on both the server and client sides within the framework to further reduce memory stalls by eliminating redundant DRAM accesses.

\subsection{\sysname Protocol}

\begin{figure}[t]
    \vspace{-0.2in}
    \centering
    \includegraphics[width=1\linewidth]{Figures/Camera-ready/Fig4_Protocol.pdf}
    % \includegraphics[width=1\linewidth]{Figures/Protocol_Offload.pdf}
    \caption{\sysname Protocol}
    \vspace{-0.2in}
    \label{fig:APINT_protocol}
\end{figure}

\sysname protocol is based on the PiT protocol in PRIMER, but it reduces the circuit in GC operation by offloading its partial calculations to HE and standard operations, thereby significantly decreasing the workload of GC. As illustrated in Figure~\ref{fig:APINT_protocol}, the basic concept of the protocol is the combination of HE for linear operations and GC for nonlinear operations. To maintain confidentiality, each party adds or subtracts a random matrix ($R_i$ of the client and $S_i$ of the server) before sending the data to each other.
At the offline phase, HE is utilized to compute linear function for the client's random matrix $R_{1}$, which has the same matrix size as the input matrix $X_1$ \circled{1}. Simultaneously, the client garbles the circuit $\Tilde{C_1}$, which integrates adding the secret shares from both parties, processing the nonlinear function, and subtracting a random matrix to ensure confidentiality \circled{2}. The client then transmits labels of $R_2, R_3$ to the server \circled{3}. During the online phase, the server calculates the linear function of $(X_{1}-R_{1})$ using standard matrix operations. This intermediate result is merged with data from process \circled{1} to complete the computation of the linear function, yielding $(X_{2}-R_{2})$ \circled{4}. After that, the labels of $(X_{2}-R_{2})$ are sent from the client via the OT protocol~\cite{ishai2003extending} \circled{5}, and the server proceeds the GC evaluation for the garbled $\Tilde{C_1}$ \circled{6}.




However, in contrast to other functions, the reduced circuit $\Tilde{C_2}$ is employed when operating LayerNorm. This circuit specifically excludes calculations of mean and variance, as well as operations involving the parameters $\beta$ and $\gamma$. The excluded calculations are offloaded and computed using standard operations and HE, thereby reducing the workload of GC.
During the offline phase, the client garbles the circuit $\Tilde{C_2}$, transmitting the labels for $\sum{R_4/N}, R_5, R_6$, and also sends $Enc(R_2')$.
During the online phase, the mean of $(X_{2}-R_{2})$ is initially calculated using standard operations. This mean is then subtracted from $(X_{2}-R_{2})$, resulting in $(X_{2}'-R_{2}')$ \circled{7}. Subsequently, to prepare for variance calculations, this result is multiplied by two times $Enc(R_2')$ \circled{8}. The results from equation \circled{7} and \circled{8} are then used to compute the variance \circled{9}. Third, multiplying with the parameter $\beta$ can be processed by utilizing HE with $(X_{2}'-R_{2}')$, $Enc(R_2')$, and $\beta$ \circled{10}, \circled{11}. Then, the labels of the data from the process \circled{9} and \circled{11}, which are obtained from the client via OT protocol, and the labels from the client are computed through GC evaluation \circled{12}. Finally, a straightforward addition of the parameter $\gamma$ is processed \circled{13}.

Although this protocol incurs additional overhead due to HE and communication of two parties, it brings a substantial reduction of GC latency, offsetting these increased costs perfectly. This reduction marks a significant improvement over the baseline protocol, which merely utilized GC for processing the nonlinear functions. Ultimately, the \sysname protocol achieves a significant reduction in the online latency of GC operations, reducing it by 47.3\% during the LayerNorm computation.

\subsection{GC-friendly Circuit Generation}

To further minimize the workload of GC, \sysname proposes GC-friendly circuit generation of the nonlinear functions. It involves implementing each function to a circuit with 2-input AND, XOR, and INV gates while preserving the accuracy of computations. The process unfolds in two main steps.

The initial step focuses on minimizing the total number of gates in the circuit. Since GC processes the gates of the circuit sequentially, reducing the gate count directly lowers the overall computational load. For instance, in the implementation of Softmax, the method from i-BERT~\cite{kim2021bert} is adopted, which scales inputs by \textit{ln2}, thereby reducing the range of values and the number of required gates for exponential operations. The exponential operations are performed through combinational logic, which performs as a Look-Up Table (LUT) interpolation. For the GeLU function, LUT interpolation is utilized after clipping the input values within a range (-4, 4)~\cite{gupta2023sigma}. In LayerNorm, the conventional approach is employed without any approximation, as it doesn't incur any accuracy drop.

The second step aims to decrease the number of AND gates further. \sysname proposes a method employing XOR-Friendly Binary Quantization (XFBQ)~\cite{jian2020fast} to implement multiplication with fewer AND gates compared to the conventional method. This approach is motivated by the observation that the multiplication process accounts for a significant portion of each nonlinear function's implementation. 
Figure~\ref{fig:circuit_generation} (a) summarizes XFBQ and its multiplication process. XFBQ modifies the binary representation, wherein 1 represents 1 and 0 represents -1, exploiting the correspondence between the result patterns of XOR operations and the product of 1 and -1. The way to XFBQ is straightforward: it involves a right shift and changing the MSB to 1, introducing only a minimal quantization error (Q error) as small as the INV of Least Significant Bit (LSB) of the original number. For instance, \textit{1000}=8 turns into \textit{1100}=8+4-2-1=9 after XFBQ with Q error as \textit{INV(0)=1}.  When expressing conventional multiplication ($A\times B$) as XFBQ multiplication ($\hat{A} \times \hat{B}$) along with additional terms due to Q errors, the AND operations of the conventional method are all replaced by XOR, thereby a high reduction of AND gates occurs. Moreover, given that the Q error is INV of LSB value, its negligible impact on multiplication results and PiT warrants disregarding the additional terms, leading to a further decrease in AND gates.

\begin{figure}[t]
    \vspace{-0.2in}
    \centering
    \includegraphics[width=1\linewidth]{Figures/Camera-ready/xfbq_correct.pdf}
    \caption{(a) Reduction of ANDs via XBFQ Multiplication (b) Comparison of ANDs for 64b Multiplication}
    \vspace{-0.2in}
    \label{fig:circuit_generation}
\end{figure}

Figure~\ref{fig:circuit_generation} (b) shows the effects of the multiplication using XFBQ while operating 64b multiplication. It reduces the number of AND gates 38.9-45.5\% compared to prior work~\cite{liu2022don}, depending on the inclusion of Q error adjustments. In addition, GC-friendly circuit generation employed methods from the work~\cite{testa2020logic} for operations other than multiplication, which has been proven to perform as an open-source. Finally, it reduces the workload of GC for nonlinear functions by an average of 42.5\%.



   
\begin{figure*}[t]
    \vspace{-0.2in}
    \centering
    \includegraphics[width=1\linewidth]{Figures/Camera-ready/Scheduling_Reduced.pdf}
    \caption{(a) Methods and (b) Effects of Coarse-Grained and Fine-Grained Scheduling}
    \vspace{-0.2in}
    \label{fig:scheduling}
\end{figure*}
\subsection{Netlist Scheduling}
% \subsection{Compiler and Hardware Accelerator}
Despite the reductions in GC overhead facilitated by the \sysname protocol and GC-friendly circuit generation, GC still accounts for a notable portion of the latency. This implies that reducing GC overhead necessitates the integration of hardware accelerators beyond software solutions.
However, since a GC accelerator takes a netlist, converted from the circuit, as input and processes gates in the netlist sequentially, efficient netlist scheduling is crucial for hardware acceleration. Therefore, we introduce coarse-grained and fine-grained scheduling, maximizing DRAM bandwidth utilization and minimizing computational dependencies to accelerate the GC operation of nonlinear functions. 

% For instance, HAAC~\cite{mo2023haac} introduced an accelerator showcasing superior performance compared to CPUs and GPUs by utilizing multi-core designs operating concurrently and leveraging off-chip memory when on-chip memory resources are insufficient. However, this approach encounters a significant memory bottleneck when processing nonlinear functions of transformers due to inadequate scheduling schemes and an accelerator design overlooking wire reusability. Consequently, as a final step, \sysname introduces a compiler-integrated accelerator, addressing memory bottleneck by enhancing wire reuse and maximizing DRAM bandwidth utilization through compiler-driven strategies and hardware structure.

\subsubsection{\textbf{Coarse-Grained Scheduling}}

Due to the complex implementation of the circuit of the nonlinear function, the netlist exhibits highly irregular patterns in input and output wires, leading to irregular DRAM accesses that hinder optimal DRAM bandwidth utilization.
To tackle this issue, \sysname adopts coarse-grained scheduling, which maps each independent operation onto each core, allowing cores to function independently yet synchronously. This approach leverages the characteristic of nonlinear functions composed of independent unit operations, such as rows in Softmax, to be computed separately. 
Assuming there are two Processing Engines (PEs) and the need to compute two Softmax rows, the red box of Figure~\ref{fig:scheduling} (a) illustrates a DAG of two rows ordered in a depth-first manner without any scheduling, where node number corresponds to the order of the gates in the netlist. In this case, two PEs concurrently process the netlists of two rows in a dependent manner. In contrast, as illustrated in the green box, each PE exclusively handles an independent row with coarse-grained scheduling. Hence, contrary to the red box in Figure~\ref{fig:scheduling} (b), the green box demonstrates that coarse-grained scheduling allows all PEs to operate synchronously, ensuring they request DRAM data simultaneously. Therefore, while the intra-core DRAM access pattern is irregular, the inter-core DRAM access pattern becomes the same. As a result, by enabling cores to share the DRAM data bus, coarse-grained scheduling ensures maximal utilization of DRAM bandwidth.

Moreover, coarse-grained scheduling offers the additional benefit of resolving wire dependencies. Unlike the scenario without coarse-grained scheduling, the scheduling enables each PE to independently operate on a distinct row without dependencies. Thus, coarse-grained scheduling not only maximizes the utilization of DRAM bandwidth but also reduces the pipeline stalls.
        
\subsubsection{\textbf{Fine-Grained Scheduling}}
In addition to the coarse-grained scheduling, \sysname applies fine-grained scheduling that further diminishes GC latency compared to Full Reorder (FR) and Segment Reorder (SR), the scheduling method by the SOTA GC accelerator, HAAC~\cite{mo2023haac}. The FR transforms the netlist into a DAG and establishes processing order by traversing the graph in a breadth-first manner, reducing wire dependencies. However, due to limited on-chip memory, it can cause off-chip traffic by spilling wires over to DRAM, especially in applications where the DAG has a wide breadth, such as the nonlinear function of transformers. To tackle this problem, HAAC proposed the SR that segments the netlist, ordered in a depth-first manner, to enhance wire reuse and then applies FR within each segment to reduce wire dependencies. 
% \sysname applies fine-grained scheduling in addition to the coarse-grained scheduling. This further diminishes GC latency beyond the scheduling method proposed by HAAC, such as Full Reorder (FR) and Segment Reorder (SR).
% FR transforms the netlist into a DAG, where each node represents a gate. It establishes processing order by traversing the graph in a breadth-first manner, reducing wire dependencies but decreasing wire reuse in limited on-chip memory. Consequently, it can cause a memory bottleneck by spilling data over to DRAM, especially in applications where the DAG has a wide breadth, such as the nonlinear function of transformers. HAAC additionally proposed SR to tackle this problem. As shown in Figure~\ref{fig:scheduling}, SR firstly segments the netlist after depth-first scheduling to enhance wire reuse and applies FR within each segment to reduce dependencies.
% However, FR is not the optimal way to reduce dependencies within each segment. Recent research by S.Zhao~\cite{zhao2020dag, zhao2022dag} has highlighted the efficiency of priority scheduling based on Critical-Path-First-Execution (CPFE) in reducing dependencies of DAG. Therefore, \sysname proposes a fine-grained scheduling by combining SR and CPFE.
Despite these advancements, we identified opportunities for further latency reduction since FR does not optimally eliminate wire dependency within segments. Therefore, \sysname introduces a fine-grained scheduling strategy that combines segmentation and Critical-Path-First-Execution (CPFE)~\cite{zhao2020dag, zhao2022dag} instead of FR, achieving enhanced performance by effectively minimizing wire dependencies.

The fine-grained scheduling begins by segmenting the netlist, with each segment half the size of on-chip memory. A DAG is then constructed for each segment, with assigning weights reflecting the cycle latency of each gate. Nodes without children are linked to \textit{v\_{src}}, and those without parents are connected to \textit{v\_{sink}}. After establishing the DAG, it finds a critical path from \textit{v\_{src}} to \textit{v\_{sink}} and prioritizes nodes along this path, starting from the lowest depth. Subsequently, for each node on the path, a sub-DAG is formed comprising unprioritized descendants, and the process of identifying the critical path and assigning priorities repeats recursively.

The blue box in Figure~\ref{fig:scheduling} (a) shows how the fine-grained scheduling works. First, in step \circled{1}, it finds a critical path and prioritizes from the lowest depth. Then, from step \circled{2}-\circled{6}, it creates sub-DAG with unprioritized descendants for each node of the path and operates recursively. 
% As shown in Figure~\ref{fig:scheduling}, DAG \circled{1} shows finding a critical path and prioritizing from the lowest depth. Then, from DAG \circled{2} to DAG \circled{6}, it creates sub-DAG with unprioritized descendants for each node of the path and operates recursively.
For example, in step \circled{6}, nodes \textit{4} and \textit{6} compose the sub-DAG, and then the process of identifying the critical path and prioritizing is recursively executed at the sub-DAG. After assigning priorities to all nodes, the scheduling order is determined by the cycle-accurate simulation. The simulation selects the operable node with the highest priority in each cycle. The "operable" refers to the condition where both input wires of a DAG node have been produced. As a result, step \circled{6} is reordered as $2\rightarrow1\rightarrow4\rightarrow5\rightarrow6\rightarrow3\rightarrow8\rightarrow7$. 
Hence, as depicted in Figure~\ref{fig:scheduling}, fine-grained scheduling significantly reduces pipeline stalls by wire dependencies within each segment, enhancing the computation speed of nonlinear functions by an average of 30.2\% compared to the SR of HAAC.

\begin{figure*}[t]
    \vspace{-0.2in}
    \centering
    \includegraphics[width=1\linewidth]{Figures/Camera-ready/Hardware_Reduced.pdf}
    \caption{APINT hardware, Compiler Speculation Flow, and Runtime Flow Descriptions}
    \vspace{-0.2in}
    \label{fig:compiler_and_hardware}
\end{figure*}
\subsection{Accelerator with Compiler Speculation}
% \subsubsection{\textbf{Compiler Speculation and Hardware Accelerator}}
HAAC introduced an accelerator showcasing superior performance compared to CPUs and GPUs by utilizing pipelined multi-core designs operating concurrently and leveraging off-chip memory when on-chip memory resources are insufficient. However, it encounters a significant memory bottleneck when processing nonlinear functions of transformers due to inadequate on-chip memory policy and hardware structure.
% A further cause of the memory bottleneck is the on-chip memory policy and hardware structure of HAAC.
HAAC's approach of sequentially writing output wires in on-chip memory doesn't consider the wire reusability. Moreover, the hardware structure that involves directly fetching wires from DRAM to a PE via a queue structure limits the wire's usage to a single time, thereby restricting its potential for reuse. To counter these issues, \sysname suggests the accelerator alongside compiler speculation techniques, aiming to reduce unnecessary DRAM accesses and improve wire reuse.

\subsubsection{\textbf{APINT Accelerator}}
% \textit{\textbf{APINT Accelerator} \ }
Figure~\ref{fig:compiler_and_hardware} illustrates the architecture of \sysname accelerator, which features 16 independent cores operating synchronously under coarse-grained scheduling. This eliminates the need for inter-core communication, allowing for a shared unified Instruction Memory (16KB). Each core includes a Wire Memory (128KB), a Table Memory (2KB), an Out-of-Range-Wire (OoRW) Prefetch Buffer (1KB), and a PE, all of which are pipelined.
Wire Memory stores the wire's label (value of the wire) and special flag bits, which are a block bit and an Out-of-Range (OoR) bit, for each address. The block bit prevents other wires from accessing the address, while the OoR bit indicates that an OoRW, a wire that is fetched from DRAM, is being fetched to that address. Also, the Table Memory stores garble tables required for Half-Gate operations, and the OoRW Prefetch Buffer temporarily stores OoRWs fetched from DRAM. They are then transferred to the Wire Memory, which allows multiple reuses within the Wire Memory in contrast to HAAC, where OoRWs are used only once per fetch.

The execution of the accelerator is structured into four stages. First, upon receiving an instruction, the Write Address Preemption stage the write address in the Wire Memory, activating the block bit. Next, the Read stage reads two input wires from the memory or forwarding path over three cycles. Third, the input wires are processed in the Half-gate unit (taking 18 cycles for evaluation and 21 for garbling) or FreeXOR unit (taking one cycle) in PE, and OoRWs are transferred from the Prefetch Buffer to Wire Memory if required. Finally, the output wire generated in the PE is written back to Wire Memory over two cycles and, if needed, also to DRAM.

% The execution of the \sysname accelerator is structured into four stages: Write Address Preemption, Read, OoRW Transfer and PE Execution, and Write. Upon receiving an instruction, the accelerator preempts the write address in the Wire Memory, setting the block bit alive. It then reads input wires over three cycles and processes them in the PE, which includes a Half-Gate unit (taking 18 cycles for evaluation and 21 for garbling) and a FreeXOR unit (taking one cycle). Garble tables are transferred from the Table Memory if Half-Gate is operated. Simultaneously, OoRWs are transferred from the Prefetch Buffer to the Wire Memory if required. Finally, after the PE execution, output wires are written back to Wire Memory over two cycles and, if needed, also to DRAM.

\subsubsection{\textbf{Compiler Speculation Flow}}
Before running the accelerator, compiler speculation is initially processed with a netlist as input. Its purpose is to generate instructions for the accelerator, which implements a memory policy that enhances wire reuse. It proceeds through the following two phases, as depicted in Figure~\ref{fig:compiler_and_hardware}. During the first phase, it assigns read and write addresses in Wire Memory and an OP bit for each gate in the netlist through a cycle-accurate simulation. After filling Wire Memory as much as possible with operable input wires, the speculation begins with the Write Address Preemption stage, allocating a write address either to a blank space or to the Last-to-Be-Used Wire (LBUW), if Wire Memory is full. The LBUW is the wire that will be used last among wires within the memory. This demonstrates that APINT employs a memory policy considering the reusability of wires. After assigning the write address, the block bit is activated for the preemption.

% Compiler speculation begins by using a netlist as input, producing instructions generated to enhance wire reuse. This process proceeds through the following two phases after filling Wire Memory as much as possible with operable input wires. During the first phase, it assigns read and write addresses in Wire Memory and an OP bit for each gate through a cycle-accurate simulation. First, during the Write Address Preemption stage, if the Wire Memory is not full, the write address is assigned to a blank space, if not, to the address of the Last-to-Be-Used Wire (LBUW), designated to be used last among wires in Wire Memory, and a block bit for this address is activated. Therefore, this process enables the on-chip memory policy to consider the reusability of wires.

Next, the Read stage assigns the read address based on whether an input wire is present in Wire Memory. If present, the read address corresponds to its location. If not, indicating it is likely to become an OoRW at runtime, the address of the LBUW with inactive block bit is assigned, which also contributes to the memory policy that considers the reusability of wires. The input wire is then replaced with the LBUW and added to the OoRW list. Subsequently, the PE execution stage begins by setting the OP bit based on the gate type. This is followed by the Write stage, which writes the output wire at the assigned address and resets the block bit. This cycle-accurate simulation is repeated until every wire and gate in the netlist has been allocated the instructions with addresses and OP bits.

After completing the first phase, the second phase involves assigning the Live bit, two OoRW-fetch bits, and the Write Enable Not (WEN) bit, which are determined by analyzing the interrelationships among instructions. The Live bit is assigned to instructions that output an OoRW, designating that the wire should be written to DRAM for later use. For example, in Figure~\ref{fig:compiler_and_hardware}, instruction \circled{1} outputs OoRW \textit{30} and is marked with a Live bit of 1.
Each OoRW-fetch bit is assigned to ensure the timely transfer of an OoRW from the Prefetch Buffer to Wire Memory, based on instruction sequence and read dependencies.
For instance, instruction \circled{2}, which reads address \textit{0} immediately before instruction \circled{4} reads OoRW \textit{30} from the same address, is assigned an OoRW-fetch bit to ensure that OoRW \textit{30} is transferred right after instruction \circled{2} reads the address \textit{0} to prevent stalls due to non-arrival.
The WEN bit is assigned to prevent premature overwriting in Wire Memory.
For example, if OoRW \textit{30} is transferred to memory before instruction \circled{3} writes to address \textit{0}, it could be overwritten before it is read by instruction \circled{4}. Therefore, a WEN bit is assigned to instruction \circled{3} to prevent it from overwriting OoRW \textit{30}, and wire \textit{35} is written only to DRAM as dictated by the Live bit.
% For example, to avoid overwriting OoRW \textit{30} needed by instruction \circled{4}, instruction \circled{3} receives a WEN bit of 1.


% During the speculation process, the sequence for using instructions and garbled tables is predetermined and mapped sequentially in DRAM. Similarly, the DRAM read and write orders for OoRWs are predecided, with DRAM write addresses being assigned incrementally. Hence, if an OoRW's DRAM read address exceeds the current increment value, indicating the wire has yet to be written to DRAM, the accelerator stalls until the increment value matches the address. This approach removes the need to handle DRAM addresses during runtime, enabling the accelerator to fetch instructions, garbled tables, and OoRWs from DRAM to on-chip in a predetermined order while executing the instructions.

\subsubsection{\textbf{Runtime Flow}}

During the speculation process, the DRAM addresses for instructions, garbled tables, and OoRWs are predetermined, removing the need to handle the addresses during runtime. While fetching the data from DRAM to each corresponding memory, the runtime process is executed in the following four stages, as depicted in Figure~\ref{fig:compiler_and_hardware}.
% During runtime, the four stages are executed with the \sysname accelerator, as depicted in Figure~\ref{fig:compiler_and_hardware}. 
After an instruction is decoded, the Write Address Preemption stage activates the block bit at the write address, and the Read stage operates based on the statuses of the block and OoR bits. Depending on these bits, the accelerator either performs a normal read or stall until the necessary wire is transferred from the Prefetch Buffer or the forwarding path. The OoRW Transfer and PE Execution stage then commences. The OoR bit is assigned with the OoRW-fetch bit, indicating whether an OoRW transfer is started. If the OoRW-fetch bit is 1, the address is preempted by activating the block bit, and an OoRW begins to be transferred to the address. After the completion of the transfer, the block bit is deactivated. Concurrently, the PE processes Half-gate or FreeXOR operation based on the OP bit. After the output wire is generated in the PE, the Write stage begins, writing the wire to Wire Memory and DRAM depending on the WEN and Live bits. After these stages are executed across all instructions, the runtime process is completed. Overall, through APINT accelerator and compiler speculation, it achieves a reduction in memory stall times by 86.1\% to 99.4\% compared to HAAC when operating nonlinear functions.

% After an instruction is decoded, the Write Address Preemption stage sets the block bit at the write address, and the Read stage proceeds based on the block and OoR bit statuses. If the block bit is 0, a normal read is performed. Otherwise, the OoR bit determines if an OoRW or non-OoRW wire has preempted the address. An OoR bit of 1 indicates a scheduled OoRW write at that address, causing the accelerator to stall until the OoRW is transferred from the Prefetch Buffer to Wire Memory. If the OoR bit is 0, the accelerator is stalled until the wire is retrieved from the forwarding path.

% Once the Read stage is completed, the OoRW-fetch and PE execution stage begins. The OoR bit is assigned with the OoRW-fetch bit, preempting the address for an OoRW transfer. If the OoRW-fetch bit is 1, the block bit is activated, and the OoRW begins to be transferred from the Prefetch Buffer to the address. The block bit is deactivated after the transfer. Concurrently, the PE processes operations based on the OP bit, and the garbled table is transferred from the Table Memory for Half-Gate operation.

% Lastly, Write stage begins. Unless the WEN bit is active, the generated output wires are written to Wire Memory, and the block bit of the write address is deactivated. If the Live bit is active, the wires are also written to DRAM. After these stages are executed across all instructions, the runtime process is completed. Overall, through compiler speculation and the accelerator, APINT achieves a reduction in memory stall times by 74.3\% to 99.9\% compared to HAAC for nonlinear functions of transformers.


\section{Experiments}
\label{sec:experiments}



\begin{figure*}[t]
    \centering
    \includegraphics[width=1\linewidth]{images/Environments.pdf} 
    % \vspace{-20pt}
    \captionsetup{
    width=\textwidth,
    font=Smallfont,
    labelfont=Smallfont,
    textfont=Smallfont
    }
    \captionsetup{
    width=\textwidth,
    font=Smallfont,
    labelfont=Smallfont,
    textfont=Smallfont
    }
    \caption{Four different real-world experiment environments.}
    \label{fig:environments}
    % \vspace{-6pt}
\end{figure*}

\begin{figure*}[t]
    \centering
    \captionsetup{
    width=\textwidth,
    font=Smallfont,
    labelfont=Smallfont,
    textfont=Smallfont
    }
    % Top-left subfigure
    \begin{subfigure}[b]{0.45\textwidth}
        \centering
        \includegraphics[width=\textwidth]{images/fig_office.pdf}
        \caption{Office}
        \label{fig:subfig1}
    \end{subfigure}
    \hspace{0.02\textwidth}
    % Top-right subfigure
    \begin{subfigure}[b]{0.45\textwidth}
        \centering
        \includegraphics[width=\textwidth]{images/fig_apt.pdf}
        \caption{Apartment}
        \label{fig:subfig2}
    \end{subfigure}

    \vskip\baselineskip

    % Bottom-left subfigure
    \begin{subfigure}[b]{0.45\textwidth}
        \centering
        \includegraphics[width=\textwidth]{images/fig_outdoor.pdf}
        \caption{Outdoor}
        \label{fig:subfig3}
    \end{subfigure}
    \hspace{0.02\textwidth}
    % Bottom-right subfigure
    \begin{subfigure}[b]{0.45\textwidth}
        \centering
        \includegraphics[width=\textwidth]{images/fig_hallway.pdf}
        \caption{Hallway}
        \label{fig:subfig4}
    \end{subfigure}

    \caption{Top-down view of the trajectories comparison on the value maps with the detection results across the four different environments.}
    \label{fig:value_map}
\end{figure*}


\begin{table*}[ht]
\captionsetup{
    width=\textwidth,
    font=Smallfont,
    labelfont=Smallfont,
    textfont=Smallfont
    }
\caption{Vision-language navigation performance in 4 unseen environments (SR and SPL).}
\label{SOTAResults}
\centering
\begin{tabular}{lcccc|cccc}
\toprule
\multirow{2}{*}{\textbf{Method}} & \multicolumn{4}{c}{\textbf{SR (\%)}} & \multicolumn{4}{c}{\textbf{SPL}} \\
\cmidrule(lr){2-5} \cmidrule(lr){6-9}
 & Hallway & Office & Apartment & Outdoor & Hallway & Office & Apartment & Outdoor \\
\midrule
\textbf{Frontier Exploration}  
  & 40.0 & 41.7 & 55.6 & 33.3  
  & 0.239 & 0.317 & 0.363 & 0.189 \\

\textbf{VLFM} \cite{yokoyama2024vlfm}                 
  & 53.3 & 75.0 & 66.7 & 44.4  
  & 0.366 & 0.556 & 0.412 & 0.308 \\

\textbf{VL-Nav w/o IBTP}      
  & 66.7 & 83.3 & \underline{70.2} & \underline{55.6}  
  & 0.593 & 0.738 & 0.615 & \underline{0.573} \\

\textbf{VL-Nav w/o curiosity}      
  & \underline{73.3} & \underline{86.3} & 66.7 & \underline{55.6}  
  & \underline{0.612} & \underline{0.743} & \underline{0.631} & 0.498 \\

\textbf{VL-Nav}               
  & \textbf{86.7} & \textbf{91.7} & \textbf{88.9} & \textbf{77.8}  
  & \textbf{0.672} & \textbf{0.812} & \textbf{0.733} & \textbf{0.637} \\

\bottomrule
\end{tabular}
\end{table*}








\subsection{Experimental Setting}
\label{sec:experimental_setting}

We evaluate our approach in real-robot experiments against five methods: (1) classical frontier-based exploration, (2) VLFM \cite{yokoyama2024vlfm}, (3) VLNav without instance-based target points, (4) VLNav without curiosity terms, and (5) the full VLNav configuration. Because the original VLFM relies on BLIP-2 \cite{li2023blip}, which is too computationally heavy for real-time edge deployment, we use the YOLO-World \cite{cheng2024yolo} model instead to generate per-observation similarity scores for VLFM. Each method is tested under the same conditions to ensure a fair comparison of performance.

\paragraph{Environments:}
We consider four distinct environments (shown in \fref{fig:environments}), each with a specific combination of semantic complexity (\textit{High}, \textit{Medium}, or \textit{Low}) and size (\textit{Big}, \textit{Mid}, or \textit{Small}). Concretely, we use a Hallway (\textit{Medium \& Big}), an Office (\textit{High \& Mid}), an Outdoor area (\textit{Low \& Big}), and an Apartment (\textit{High \& Small}). In each environment, we evaluate five methods using three language prompts, yielding a diverse range of spatial layouts and semantic challenges. This setup provides a rigorous assessment of each method’s adaptability.

\paragraph{Language-Described Instance:}
We define nine distinct, uncommon human-described instances to serve as target objects or persons during navigation. Examples include phrases such as “tall white trash bin,” “there seems to be a man in white,” “find a man in gray,” “there seems to be a black chair,” “tall white board,” and “there seems to be a fold chair.” The variety in these descriptions ensures that the robot must rely on vision-language understanding to accurately locate these targets.

\noindent\textbf{Robots and Sensor Setup:} 
All experiments are conducted using a four-wheel Rover equipped with a Livox Mid-360 LiDAR. The LiDAR is tilted by approximately 23 degrees to the front to achieve a $\pm 30$ degrees vertical FOV coverage closely aligned with the forward camera’s view. An Intel RealSense D455 RGB-D camera, tilted upward by 7 degrees to detect taller objects, provides visual observation, though its depth data are not used for positioning or mapping. LiDAR measurements are a primary source of mapping and localization due to their higher accuracy. The whole VL-Nav system runs on an NVIDIA Jetson Orin NX on-board computer.




\subsection{Main Results}
\label{sec:main_results}

We validate the proposed VL-Nav system in real-robot experiments across four distinct environments (\textit{Hallway}, \textit{Office}, \textit{Apartment}, and \textit{Outdoor}), each featuring different semantic levels and sizes. Building on the motivation articulated in~\sref{sec:intro}, we focus on evaluating VL-Nav’s ability to (1) interpret fine-grained vision-language features and conduct robust VLN, (2) explore efficiently in unfamiliar spaces across various environments, and (3) run in real-time on resource-constrained platforms. \fref{fig:value_map} presents a top-down comparison of trajectories and detection results on the value map.

\begin{figure*}[t]
    \centering
    \includegraphics[width=1\linewidth]{images/result_plot.pdf} 
    % \vspace{-20pt}
    \captionsetup{
    width=\textwidth,
    font=Smallfont,
    labelfont=Smallfont,
    textfont=Smallfont
    }
    \caption{Plots of performance in different environments sizes and semantic comlexities.}
    \label{fig:results}
    % \vspace{-6pt}
\end{figure*}

\paragraph{Overall Performance:}
As reported in Table~\ref{SOTAResults}, our full \textbf{VL-Nav} consistently obtains the highest Success Rate (SR) and Success weighted by Path Length (SPL) across all four environments. In particular, VL-Nav outperforms classical exploration by a large margin, confirming the advantage of integrating CVL spatial reasoning with partial frontier-based search rather than relying solely on geometric exploration.

\paragraph{Effect of Instance-Based Target Points (IBTP):}
We note a marked improvement when enabling IBTP: the variant without IBTP lags behind, particularly in complex domains like the \textit{Apartment} and \textit{Office}. As discussed in \sref{sec:method}, IBTP allows VL-Nav to pursue and verify tentative detections with confidence above a threshold, mirroring human search behavior. This pragmatic mechanism prevents ignoring possible matches to the target description and reduces overall travel distance to confirm or discard candidate objects.

\paragraph{Curiosity Contributions:}
The \emph{curiosity Score} is also significant to VL-Nav’s performance. It merges two key components:
\begin{itemize}
    \item \textbf{Distance Weighting}: Preventing easily select very far way goals to reduce travel time and energy consumption which is extremely important for the efficiency (metrics SPL) in the large-size environments.
    \item \textbf{Unknown-Area Weighting}: Rewards navigation toward regions that yield more information.
\end{itemize}
Our ablations reveal that removing the distance-scoring element (\textit{VL-Nav w/o curiosity}) degrades both SR and SPL, particularly in the more cluttered environments. Meanwhile, dropping the instance-based target points (IBTP) similarly lowers performance, reflecting how each piece of CVL addresses a complementary aspect of semantic navigation.

\paragraph{Comparison to VLFM:}
Although the VLFM approach \cite{yokoyama2024vlfm} harnesses vision-language similarity value, it lacks the pixel-wise vision-language features, instance-based target points verification mechanism, and CVL-based spatial reasoning. Consequently, VL-Nav surpasses VLFM in both SR and SPL by effectively combining the pixel-wise vision language features and the curiosity cues via the CVL spatial reasoning. These gains are especially pronounced in semantic complex (\textit{Apartment}) and open-area (\textit{Outdoor}) environments, underscoring how our CVL spatial reasoning enhance vision-language navigation in complex settings and scenarios.



\paragraph{Summary of Findings:}
In conclusion, the experimental results confirm that VL-Nav delivers superior vision-language navigation across diverse, unseen real-world environments. By fusing frontier-based target points detection, instance-based target points, and the CVL spatial reasoning for goal selection, VL-Nav balances semantic awareness and exploration efficiency. The system’s robust performance, even in large or cluttered domains, highlights its potential as a practical solution for zero-shot vision-language navigation on low-power robots.

\section{Conclusion}
We introduce a novel approach, \algo, to reduce human feedback requirements in preference-based reinforcement learning by leveraging vision-language models. While VLMs encode rich world knowledge, their direct application as reward models is hindered by alignment issues and noisy predictions. To address this, we develop a synergistic framework where limited human feedback is used to adapt VLMs, improving their reliability in preference labeling. Further, we incorporate a selective sampling strategy to mitigate noise and prioritize informative human annotations.

Our experiments demonstrate that this method significantly improves feedback efficiency, achieving comparable or superior task performance with up to 50\% fewer human annotations. Moreover, we show that an adapted VLM can generalize across similar tasks, further reducing the need for new human feedback by 75\%. These results highlight the potential of integrating VLMs into preference-based RL, offering a scalable solution to reducing human supervision while maintaining high task success rates. 

\section*{Impact Statement}
This work advances embodied AI by significantly reducing the human feedback required for training agents. This reduction is particularly valuable in robotic applications where obtaining human demonstrations and feedback is challenging or impractical, such as assistive robotic arms for individuals with mobility impairments. By minimizing the feedback requirements, our approach enables users to more efficiently customize and teach new skills to robotic agents based on their specific needs and preferences. The broader impact of this work extends to healthcare, assistive technology, and human-robot interaction. One possible risk is that the bias from human feedback can propagate to the VLM and subsequently to the policy. This can be mitigated by personalization of agents in case of household application or standardization of feedback for industrial applications. 

\bibliographystyle{ACM-Reference-Format}
\bibliography{ref}
% \bibliography{main}

\end{document}
\endinput
%%
%% End of file `sample-sigconf.tex'.
