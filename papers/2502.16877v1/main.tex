%%
%% This is file `sample-sigconf.tex',
%% generated with the docstrip utility.
%%
%% The original source files were:
%%
%% samples.dtx  (with options: `all,proceedings,bibtex,sigconf')
%% 
%% IMPORTANT NOTICE:
%% 
%% For the copyright see the source file.
%% 
%% Any modified versions of this file must be renamed
%% with new filenames distinct from sample-sigconf.tex.
%% 
%% For distribution of the original source see the terms
%% for copying and modification in the file samples.dtx.
%% 
%% This generated file may be distributed as long as the
%% original source files, as listed above, are part of the
%% same distribution. (The sources need not necessarily be
%% in the same archive or directory.)
%%
%%
%% Commands for TeXCount
%TC:macro \cite [option:text,text]
%TC:macro \citep [option:text,text]
%TC:macro \citet [option:text,text]
%TC:envir table 0 1
%TC:envir table* 0 1
%TC:envir tabular [ignore] word
%TC:envir displaymath 0 word
%TC:envir math 0 word
%TC:envir comment 0 0
%%
%%
%% The first command in your LaTeX source must be the \documentclass
%% command.
%%
%% For submission and review of your manuscript please change the
%% command to \documentclass[manuscript, screen, review]{acmart}.
%%
%% When submitting camera ready or to TAPS, please change the command
%% to \documentclass[sigconf]{acmart} or whichever template is required
%% for your publication.
%%
%%
\documentclass[sigconf]{acmart}

\newcommand{\sysname}{APINT }
\usepackage{algorithm}
\usepackage{algpseudocode}
\usepackage{amsmath}
%\usepackage{amssymb}
\usepackage{indentfirst}
\usepackage{graphicx}
\usepackage{caption}
\usepackage{tikz}
\usepackage{enumitem}
\usepackage{titlesec}



\newcommand*\circled[1]{\tikz[baseline=(char.base)]{
            \node[shape=circle,draw,inner sep=1pt] (char) {\scriptsize #1};}}

\captionsetup[figure]{skip=5pt} % 그림과 캡션 사이 간격 조정
% \setlength{\textfloatsep}{5pt plus 1.0pt minus 1.0pt} % 캡션과 글 사이 간격 조정

%%
%% \BibTeX command to typeset BibTeX logo in the docs
\AtBeginDocument{%
  \providecommand\BibTeX{{%
    Bib\TeX}}}

%% Rights management information.  This information is sent to you
%% when you complete the rights form.  These commands have SAMPLE
%% values in them; it is your responsibility as an author to replace
%% the commands and values with those provided to you when you
%% complete the rights form.


\copyrightyear{2024}
\acmYear{2024}
\setcopyright{rightsretained}
\acmConference[ICCAD '24]{IEEE/ACM International Conference on
Computer-Aided Design}{October 27--31, 2024}{New York, NY, USA}
\acmBooktitle{IEEE/ACM International Conference on Computer-Aided Design
(ICCAD '24), October 27--31, 2024, New York, NY, USA}
\acmDOI{10.1145/3676536.3676786}
\acmISBN{979-8-4007-1077-3/24/10}

% 주석 제거하는코드
% \settopmatter{printacmref=false}
% \renewcommand\footnotetextcopyrightpermission[1]{}
\pagestyle{plain}
%%%%%%%%%%


%%
%% Submission ID.
%% Use this when submitting an article to a sponsored event. You'll
%% receive a unique submission ID from the organizers
%% of the event, and this ID should be used as the parameter to this command.
\acmSubmissionID{1237}

%%
%% For managing citations, it is recommended to use bibliography
%% files in BibTeX format.
%%
%% You can then either use BibTeX with the ACM-Reference-Format style,
%% or BibLaTeX with the acmnumeric or acmauthoryear sytles, that include
%% support for advanced citation of software artefact from the
%% biblatex-software package, also separately available on CTAN.
%%
%% Look at the sample-*-biblatex.tex files for templates showcasing
%% the biblatex styles.
%%

%%
%% The majority of ACM publications use numbered citations and
%% references.  The command \citestyle{authoryear} switches to the
%% "author year" style.
%%
%% If you are preparing content for an event
%% sponsored by ACM SIGGRAPH, you must use the "author year" style of
%% citations and references.
%% Uncommenting
%% the next command will enable that style.
%%\citestyle{acmauthoryear}


%%
%% end of the preamble, start of the body of the document source.
\begin{document}

%%
%% The "title" command has an optional parameter,
%% allowing the author to define a "short title" to be used in page headers.
\title{APINT: A Full-Stack Framework for Acceleration of Privacy-Preserving Inference of Transformers based on Garbled Circuits}

%%
%% The "author" command and its associated commands are used to define
%% the authors and their affiliations.
%% Of note is the shared affiliation of the first two authors, and the
%% "authornote" and "authornotemark" commands
%% used to denote shared contribution to the research.


% \numberofauthors{4}
% % Three authors sharing the same affiliation.
%     \author{
%       \alignauthor Hyunjun Cho\\      
%       \email{h.cho@kaist.ac.kr}
% %
%       \alignauthor Jaeho Jeon\\     
%       \email{math15738@kaist.ac.kr}
% %
%       \alignauthor Jaehoon Heo\\    
%       \email{kd01050@kaist.ac.kr}
      
%       \alignauthor Joo-Young Kim\\    
%       \email{jooyoung1203@kaist.ac.kr}
% %
%       \sharedaffiliation
%       % \affaddr{Department of Electrical Engineering and Computer Science}  \\
%       \affaddr{KAIST, Daejeon }   \\
%           }

% \author{Hyunjun Cho, Jaeho Jeon, Jaehoon Heo, Joo-Young Kim}
% \affiliation{%
%  \institution{School of Electrical Engineering, KAIST}
% }
% \affiliation{%
%   \institution{\{h.cho, math15738, kd01050, jooyoung1203\}@kaist.ac.kr}
% }

\settopmatter{authorsperrow=4}
\author{Hyunjun Cho}
% \authornote{Both authors contributed equally to this research.}
\email{h.cho@kaist.ac.kr}
\orcid{0009-0008-2378-852X}
\affiliation{%
  \institution{KAIST}
  \city{Daejeon}
  \country{South Korea}
}

\author{Jaeho Jeon}
% \authornote{Both authors contributed equally to this research.}
\email{math15738@kaist.ac.kr}
\orcid{0009-0008-2250-6068}
\affiliation{%
  \institution{KAIST}
  \city{Daejeon}
  \country{South Korea}
}

\author{Jaehoon Heo}
% \authornote{Both authors contributed equally to this research.}
\email{kd01050@kaist.ac.kr}
\orcid{0000-0003-1742-4275}
\affiliation{%
  \institution{KAIST}
  \city{Daejeon}
  \country{South Korea}
}

\author{Joo-Young Kim}
% \authornote{Both authors contributed equally to this research.}
\email{jooyoung1203@kaist.ac.kr}
\orcid{0000-0003-1099-1496}
\affiliation{%
  \institution{KAIST}
  \city{Daejeon}
  \country{South Korea}
}


\begin{abstract}
As the importance of Privacy-Preserving Inference of Transformers (PiT) increases, a hybrid protocol that integrates Garbled Circuits (GC) and Homomorphic Encryption (HE) is emerging for its implementation. While this protocol is preferred for its ability to maintain accuracy, it has a severe drawback of excessive latency. 
%To address this, the existing protocol decreased its latency by primarily focusing on HE latency but not on the latency of GC. Consequently, GC has become the primary bottleneck of the protocol, remaining an unresolved challenge. Furthermore, despite efforts in various sectors to reduce its latency, previous studies have not optimally minimized its overhead, nor have they provided a comprehensive solution.
To address this, existing protocols primarily focused on reducing HE latency, thus making GC the new latency bottleneck. Furthermore, previous studies only focused on individual computing layers, such as protocol or hardware accelerator, lacking a comprehensive solution at the system level.
%Furthermore, previous studies only worked on particular aspects such as reducing the GC workload and employing a GC hardware accelerator, unable to provide a comprehensive solution with holistic optimization.

% However, previous efforts to reduce GC workload by decreasing the number of AND gates in circuits have not achieved optimal reductions. Additionally, prior attempts to mitigate GC latency through hardware accelerators have encountered severe memory bottlenecks and data dependencies when processing PiT. These issues highlight the urgent need for innovative approaches to accelerate the PiT by addressing GC latency.

% The proliferation of cloud-based services using pre-trained Transformer models for NLP tasks has highlighted critical privacy issues, as sensitive data such as financial and health information are at risk of exposure. This challenge underscores the growing necessity for Privacy-preserving inference on Transformers (PiT). However, PiT faces significant latency challenges, primarily due to the overhead associated with Garbled Circuits (GC). Existing approaches have not sufficiently addressed the reduction of GC overhead, nor have they effectively minimized the computational workload, especially in reducing the number of AND gates in circuit designs. Additionally, despite advancements in accelerator technology for GC, they lack a practical solution for managing nonlinear functions within transformer models.

This paper presents APINT, a full-stack framework designed to reduce PiT's overall latency by addressing the latency problem of GC through both software and hardware solutions. APINT features a novel protocol that reallocates possible GC workloads to alternative methods (i.e., HE or standard matrix operation), substantially decreasing the GC workload. It also suggests GC-friendly circuit generation that reduces the number of AND gates at the most, which is the expensive operator in GC. Furthermore, APINT proposes an innovative netlist scheduling that combines coarse-grained operation mapping and fine-grained scheduling for maximal data reuse and minimal dependency. Finally, APINT's hardware accelerator, combined with its compiler speculation, effectively resolves the memory stall issue. Putting it all together, APINT achieves a remarkable end-to-end reduction in latency, outperforming the existing protocol on CPU platform by 12.2$\times$ online and 2.2$\times$ offline. Meanwhile, the APINT accelerator not only reduces its latency by 3.3$\times$ but also saves energy consumption by 4.6$\times$ while operating PiT compared to the state-of-the-art GC accelerator.

% \vspace{-0.04in}

% The rising importance of Privacy-preserving inference on Transformers (PiT) is driven by the need to protect sensitive client data, like financial and health information, which is potentially exposed when using cloud-based, pre-trained Transformer models for NLP tasks. However, it suffers from significant latency issues, particularly since Garbled Circuits (GC) account for the majority of the online latency.
% Prior studies have identified several challenges in accelerating Privacy-preserving inference on Transformers (PiT) due to the latency issues primarily associated with Garbled Circuits (GC), which contribute significantly to online latency. Existing protocols have not addressed the reduction of GC overhead. Additionally, prior methods have not adequately minimized the workload, particularly in terms of reducing the number of AND gates. Furthermore, while advancements have been made in the development of accelerators for GC, they have not proposed a practical solution for nonlinear functions in transformer models. 
% This paper introduces APINT, a full-stack framework designed to enhance the efficiency of PiT across both the software and hardware side. First, it employs a novel protocol that mitigates online latency by redistributing GC workloads between alternative computational methods and the preprocessing phase. Second, it suggests GC-Friendly netlist generation to significantly cut down the number of AND gates, effectively decreasing both GC computation and communication latency. Third, APINT utilizes coarse-grained and fine-grained scheduling to reduce data dependencies and enhance data reuse. Fourth, the APINT hardware accelerator with compiler speculation optimizes data reusability and minimize unnecessary DRAM accesses. Collectively, APINT achieves significant latency reductions---58.5\% offline and 89.9\% online---resulting in a total latency decrease of 66.6\% compared to the conventional method. Specifically, the APINT accelerator reduces latency by 23.4\% and system energy consumption by 78.1\% compared to the SOTA GC accelerator.


\end{abstract}


\maketitle

\section{Introduction}\label{sec:intro}

In computational finance, Monte Carlo simulations are used extensively to estimate the expected value of financial payoffs based on the solution of stochastic differential equations (SDEs) which model the evolution of stock prices, interest rates, exchange rates and other quantities \cite{glasserman04}.  Monte Carlo methods are very general and flexible, but for high accuracy it requires generating a large number of costly SDE path approximations, which has motivated research into a number of variance reduction or, equivalently, cost reduction techniques. One such method is
Multilevel Monte Carlo (MLMC), which was proposed in \cite{GILES2008} and was adapted for various applications that are summarised in \cite{Giles_overview17} and successfully combined with other methods such as quasi-Monte Carlo methods. The main idea of MLMC is to approximate the payoff using different time stepping resolutions when numerically solving the underlying SDE and to generate an optimal number of samples on each level, such that the overall computational cost is minimised subject to the desired bound on the variance. %, such that the total computational cost is minimised. 
The computational savings come from the fact that most samples are computed on the coarser levels and hence are less expensive while only a few samples from the finest levels are required \cite{GILES2008}.


Among the directions in which the computational cost 
of MLMC methods could further be reduced, an important avenue is the use of lower precision calculations, especially for the first Monte Carlo levels where the targeted accuracy is relatively low. 
 An overview of the research on mixed precision for the standard Monte Carlo (MC) framework is provided in \cite{ChowMixedPrecisionStandardMC} but only a few references study the potential of low precision computation in the MLMC framework \cite{Rounding_error_oliver}. To the best of our knowledge, the only MLMC framework with customised precision in the literature is \cite{brugger2014mixed}, but they use a uniform precision for all operations on each Monte Carlo level instead of optimising 
 the precision of each intermediary variable to reduce as much as possible the cost of path generation.
 
An important motivation for an MLMC framework with variable precision would be performing the low precision computations on reconfigurable hardware devices such as Field Programmable Gate Arrays (FPGAs). FPGAs contain customizable logic blocks and connectors that make it easy to adapt the digital circuit architecture for a specific application, leading to a highly parallel and optimised implementation. Therefore they are successfully exploited in applications that require high speed and have high computational workload, such as signal processing \cite{woods2008fpga}, and real time applications like high frequency trading \cite{HFT1,HFT2}. That is why a number of previous works in hardware architecture design implemented the MLMC algorithm to price financial options using FPGAs as accelerators, which resulted in improved speed and power efficiency compared to full CPU architectures \cite{Schryver2013AMM}. The paper \cite{lindsey2016domain} also proposed 
a Domain Specific Language to automate the configuration of FPGAs for this specific application. However, only \cite{brugger2014mixed} proposed a heuristic to reduce the precision in calculations.

In addition, all aforementioned works considered that the random number generation (RNG) is performed in single or double precision. Yet in most cases an important portion of the workload in the overall MLMC simulation comes from the RNG and in \cite{brugger2014mixed} this limited the total computational savings.
To reduce the cost of MLMC simulations in particular those based on the Geometric Brownian Motion (GBM), \cite{approximateICDF_Oliver, NestedOliver} have proposed to use approximate random numbers that are generated by applying an approximation of the inverse CDF to uniform random numbers. In \cite{NestedOliver}, the authors proposed a way to integrate these lower precision random variables into a \textit{nested} MLMC framework and completed a numerical analysis to bound the resulting error at each MC level by a product of the time step and the error in the random number approximation. The same authors show in \cite{approximateICDF_Oliver} that using approximate random variables reduces the cost of path generation by a factor 7.


In this paper we propose a nested MLMC framework that combines the use of approximate random normal variables and lower precision calculations to reduce the computational cost of MLMC even further than \cite{brugger2014mixed,NestedOliver}. We illustrate the efficiency of our framework in Matlab, after making several assumptions on the cost of operations and size of the errors that we carefully justify. We focus on the case of GBM and use the approximate RNG methods presented in \cite{approximateICDF_Oliver} as well as a new slightly modified method that combines CDF inversion and the central limit theorem. To choose the precision of the variables in the low precision path generation, we introduce a novel method to optimise the bit-widths. This optimisation is performed before the main path generation loop is executed and is based on a linear model of the payoff error  
due to rounding when computing in low precision. The error model relies on algorithmic differentiation in a similar manner to \cite{unifying-bwoptim,bitwidth-AD,ADAPT}. The bit-width optimisation procedure can be performed off-line, so this stage can be excluded from the on-line time complexity of our framework. The user specified desired accuracy is then enforced by calculating on-line the number of samples that need to be generated.

In terms of hardware design, we suggest implementing the low precision path generation on FPGAs and the full-precision ones on a CPU or GPU. 
The FPGA offers enough flexibility to define a separate bit-width for every variable in the low precision path generation, and can be reconfigured periodically to update the bit-widths when the market parameters have changed considerably. 


The paper is organized as follows : \Cref{sec:MLMC} introduces MLMC and nested MLMC to make clear the estimator that is implemented in our framework. Then in \Cref{sec:RNG} we detail the methods that could be used to obtain approximate random normally distributed numbers very cheaply for the low precision path generation. In \Cref{sec:error_model} and \Cref{sec:costModel} we propose an error model and a cost model (resp.) that we then use to formulate the optimisation problem that is solved to obtain the optimal bit-widths of fixed point variables in \Cref{sec:optimisation}. Finally we summarise our results and future directions in \Cref{sec:conclusion}.



\section{Background and Motivation}

\begin{figure}[t]
    \vspace{-0.2in}
    \centering
    \includegraphics[width=1\linewidth]{Figures/gc_overview_ver3.pdf}
    \caption{Garbled Circuit Protocol Overview}
    \vspace{-0.2in}
    \label{fig:garbled_circuit}
\end{figure}

\subsection{Protocols for Privacy-Preserving Inference}
\subsubsection{\textbf{Homomorphic Encryption}}
HE is a cryptographic method allowing operations on encrypted data (ciphertext) that produce the same results as if performed on unencrypted data (plaintext). It supports operations like addition and multiplication between plaintext and ciphertext ($Enc(X)+Y = Enc(X+Y)$, $Enc(X)\times Y = Enc(X\times Y)$), and among ciphertexts ($Enc(X)+Enc(Y) = Enc(X+Y)$, $Enc(X)\times Enc(Y) = Enc(X\times Y)$).

% Multiplying ciphertexts is more resource-intensive than operating between plaintext and ciphertext. To optimize efficiency, the upcoming protocol restricts operations to plaintext-ciphertext interactions.

% ($Enc$ refers to the process of encrypting plaintext to convert it into ciphertext.)\\
% \textit{1) ciphertext-plaintext addition:} $Enc(X)+Y = Enc(X+Y)$\\
% \textit{2) ciphertext-plaintext multiplication:} $Enc(X)Y = Enc(XY)$

\subsubsection{\textbf{Garbled Circuits}}
% Garbled Circuit detailed
GC involves two parties---the Garbler and the Evaluator---jointly compute without disclosing their inputs. The GC process involves four steps, as shown in Figure~\ref{fig:garbled_circuit}.\\
\textit{1) Circuit and Netlist Generation: } 
The function to be computed is represented as a circuit of 2-input logic gates, and it is converted to a netlist format, which contains information about each gate's inputs (input wires), output (output wire), and type (ex. AND).\\
\textit{2) GC Garbling: } 
For each wire \textit{i} in the netlist, the garbler randomly generates a 128b label ($W_i^0$) corresponding to 0. The label ($W_i^1$) corresponding to 1 is then produced by performing an XOR operation on ($W_i^0$) with a random 128b value R, which is fixed and public. After assigning the labels, the garbler constructs a truth table for each logic gate and encrypts it to create the garbled table. Finally, the garbler selects the corresponding label for its input wire ($a \in (W_a^0, W_a^1)$), which will be used in GC evaluation.\\
\textit{3) Garbler-Evaluator Communication:}
The garbler transmits the garbled tables and the selected label to the evaluator. Additionally, the garbler sends the label, corresponding to the evaluator's input wire ($b \in (W_b^0, W_b^1)$), to the evaluator without knowing the wire's value. This can be achieved through Oblivious Transfer (OT) protocol ~\cite{ishai2003extending}.\\
\textit{4) GC Evaluation:}
The evaluator calculates the output for each gate using the garbled tables and the labels provided by both the garbler and evaluator. The output becomes the new label of the input wire of the next gates, and the evaluations proceed sequentially for all gates in the netlist.

Recent enhancements in GC---Half-Gate~\cite{zahur2015two} for AND gates and FreeXOR~\cite{kolesnikov2008improved} for XOR gates---have reduced computational demands and memory footprints. To facilitate this, the netlist employs solely AND, XOR, and INV gates. It should be noted that INV gates, unlike AND and XOR gates, can be implemented at no cost by simply removing them and inverting the correspondence between the values and labels of the wires. In the garbling phase, contrary to the original GC that assigns labels of output wires randomly, the Half-Gate operation for AND gates produces an output wire's label and two garbled tables through four AES computations of the input wires' labels. Conversely, the FreeXOR operation for XOR gates generates an output wire's label by XOR computation of the input wires' labels without creating any garbled tables.
During the evaluation phase, a Half-Gate operation produces an output wire's label by garbled tables and two AES computations of input wires' labels, and a FreeXOR operation computes an output wire's label directly through an XOR operation of input wires' labels.




\begin{figure}[t]
    \vspace{-0.2in}
    \centering
    \includegraphics[width=1\linewidth]{Figures/Camera-ready/Motivation_Reduced.pdf}
    % \includegraphics[width=1\linewidth]{Figures/motivation.pdf}
    \caption{Latency Analysis of Prior Works (a) PRIMER Protocol and (b) HAAC}
    \vspace{-0.2in}
    \label{fig:APINT_motivation}
\end{figure}
    
\subsection{Motivation for \sysname}

Previous studies have several issues in effectively processing PiT by speeding up GC, the primary bottleneck.
First, in terms of protocol, PRIMER~\cite{zheng2023primer} has solely focused on optimizing HE and has not introduced any methods to reduce the latency of GC. As depicted in Figure~\ref{fig:APINT_motivation} (a), which presents the latency result for a single inference of the BERT-base model~\cite{devlin2018bert} with 128 tokens using the PRIMER protocol on a CPU, GC evaluation accounts for 94.7\% of the online latency. Moreover, GC garbling, along with the transfer of labels and garbled tables, contributes to 77.9\% of the offline latency. Therefore, there is a significant need for a new protocol that substantially reduces the latency of GC.

Second, regarding circuit generation, prior works~\cite{testa2019reducing, testa2020logic, liu2022don} haven't proposed the optimal way to reduce the number of AND gates. Given that Half-Gate operations are more complex than FreeXOR operations, reducing the number of AND gates can significantly decrease the GC workload. To achieve this, previous works have optimized the XOR-AND-Graph (XAG) in which the circuit is converted into a DAG. However, these approaches failed to realize a more optimal method by overlooking modifications to the fundamental implementation of the circuit.

Third, with respect to the accelerator, previous works~\cite{hussain2019fase, mo2023haac} mitigated the latency issue of GC, but none of them proposed practical solutions to operate the nonlinear functions of transformers. FASE~\cite{hussain2019fase}, which benchmarked with a small dataset, assumed that the on-chip itself could cover all the wires required during the operation.
However, this approach is infeasible since the nonlinear functions have many more wires than the on-chip can physically support. Although HAAC~\cite{mo2023haac} partially solved this problem with an additional scheme utilizing off-chip memory, it suffers from memory stalls and pipeline stalls as described in Figure~\ref{fig:APINT_motivation} (b) due to suboptimal scheduling, inefficient on-chip memory policies, and hardware structures not considering wire reusability.
% However, it suffers from memory bottlenecks when operating nonlinear functions of transformers due to suboptimal scheduling, as well as on-chip memory management policies and hardware structures not mainly designed to consider wire reusability.

\sysname aims to overcome these deficiencies by providing the full stack solution while fulfilling four key requirements. First, it must adhere to a new protocol that reduces the GC computations, the main bottleneck of PiT. Second, it should generate the circuit in a GC-friendly way that ensures a reduction of GC workloads to decrease latency and memory footprint. Third, it requires appropriate scheduling to reduce memory stalls and pipeline stalls. Fourth, it necessitates an accelerator equipped with the compiler, which resolves memory bottlenecks and redundant DRAM traffic.

% Third, it requires appropriate scheduling and accelerator architecture to resolve latency overhead caused by memory bottlenecks and wire dependencies.
\section{\ours}
\label{sec:framework}

In this section, we introduce \ours as a method for measuring question utility and using high-utility questions to fine-tune question generators. 
We first provide an overview of \ours (\secref{ssec:quest-overview}) and delve into the details of each its procedures (\secref{ssec:quest-question-generation}-\secref{ssec:quest-train}). 

Consider a binary prediction task for ICU patient mortality based on electronic medical records. A source hospital $H_o$ has historical patient data $\Do$ containing static past patient characteristics, prior medical records, and ICU outcomes. Other hospitals $\{H_i\}$ each has their patient data: $\{\Di \mid i\in [1.. N]\}$. 

For this binary prediction task, hospitals typically optimize for performance metrics, for example the area under the receiver-operating characteristic curve (AUC). Using only their data, $H_o$ can train a model $\mathcal{M}$ with parameters $\theta$ to achieve:
\begin{equation}
\tag{Baseline Performance}
\AUCo = \max_{f(\theta)} \, \AUC(\mathcal{M}, \Do)
\end{equation}
where $f$ is their chosen algorithm with parameter $\theta$.
%\footnote{\textcolor{red}{note that this algorithm can change downstream(for now we can omit)}}
%\AUC(H_o, \emptyset)
%, obtaining model parameters $\theta_o$, choosing an algorithm $f_o$, and an initial AUC of $\AUC^{[o]}$.

When $H_o$ has exhausted their own internal data, they may benefit from incorporating additional target data sources $T\subset [1.. N]$. By combining datasets, i.e., $\DT = \{\Di\mid i\in T\}\cup \Do$, $H_o$ can potentially achieve better results:
\begin{equation}
\tag{Combined Performance}
\AUCT = \max_{f(\theta)} \, \AUC(\mathcal{M}, \DT).
\end{equation}
We define the potential improvement from data addition as $\delta_{T} = \delta_{(o, T)} = \AUCT-\AUCo$. To add a single additional data source by setting $T=\{i\}$, the improvement is $\delta_{i} =\delta_{(o, i)}=\AUC_i-\AUCo$.
This leads to our central question:
% \begin{quote}
    \textbf{\emph{Without seeing target data, how does a hospital ascertain potential data sources to combine with?}}
% \end{quote}

Formally, given $n\leq N$, we seek a strategy $\pi$ that selects $n$ target datasets $T=\pi(\Do, n)$ to maximize model utility:
\begin{equation}
\tag{Ideal Dataset Combination}
\pi^*(\Do, n) = \argmax_{T\subset {[1...N]\choose n}}\AUCT%\quad\forall n.
\end{equation}
\paragraph{Practical Considerations.} Computing every subset $T\subset {[1...N]\choose n}$'s associated $\delta_{T}$ is exponential in $n$. To make this problem tractable, we make two key assumptions. First, we apply strategies greedily, selecting top-ranked target datasets. With the ultimate objective of improving the source hospital's prediction task, we fix $H_o$; to compare the trade-offs between strategies in Section~\ref{sec:methods}, we apply each $\pi$ greedily to select top-$n$ institution(s) for $H_o$ without replacement. Second, in in data constrained settings, we aim to maximize the probability of positive improvement: $P_{H_o\sim \mathbf{H}}(\delta_T > 0)$. 
%\textcolor{red}{Add additional caveats here for folktable setups.}
\paragraph{Kullback–Leibler Divergence.} Our approach uses Kullback-Leibler (KL)-divergence-based methods to gauge data utility, building on prior work~\cite{shen2024data}. KL divergence~\cite{kullback1951information}, also called \emph{information gain}~\cite{quinlan1986induction}, describes a measure of how much a model probability distribution $Q$ is different from a true probability distribution $P$:
\begin{equation}
\tag{Kullback–Leibler Divergence}
\mathrm{KL}(P||Q) = \int_{x\in \mathcal{X}} \log\frac{P(\diff x)}{Q(\diff x)}P(\diff x)
\end{equation}
Because computing KL-divergence on datasets $\Do$ and $\Di$ is non-trivial, ~\priorp proposes two groups of scores to make this divergence approximation tractable from small samples.
% \begin{equation}
% \tag{Ideal Estimator}
% \mathrm{KL}(P_o||P_i) = \int_{x\in \mathcal{X}} \log\frac{P_o(\diff x)}{P_i(\diff x)}P_i(\diff x)
% \end{equation}
 Specifically, score $\KLXY$ first trains a logistic regression model on $\Do \cup \Di$ -- where the labels are folded into the covariates --- with the goal of inferring dataset membership. Then, the resulting model's probability score function $\text{Score}(\cdot): \mathcal{X, Y} \to [0,1]$ is averaged over a dataset in $H_o$, obtaining

\begin{equation}
\tag{KL-XY Score}
\KLXY = \mathbb{E}_{(x,y)\sim \Do}(\text{Score}(x, y)).
\end{equation}
Details are described in Section~\ref{sec:methods}.
\paragraph{Privacy Model for $\pi_p$.} We operate under a semi-honest privacy model---also known as \emph{honest-but-curious} or \emph{passive security}---where parties follow protocols but may probe intermediate values. Parties are  ``curious'', meaning that they can probe into the intermediate values to avoid paying for the data. This assumes a weaker security model than malicious security where a corrupted party may input foul data, but ensures the algorithm to be private throughout the computation. This privacy preservation model incentivizes collaboration, improving upon methods in ~\priorp.



\paragraph{MPC Preliminary}
To secure this divergence computation cryptographically, Secure Multiparty Computation (MPC)~\cite{yao1982protocols, shamir1979share} protocols are leveraged. Specifically, in $\mathrm{SecureKL}$, each party encodes $\Do$ and $\Di$ to preserve privacy for both parties. This is implemented with the research framework CrypTen~\cite{knott2021crypten}, specialized for MPC and machine learning. Our algorithmic and engineering details are in Sections ~\ref{sec:methods} and ~\ref{sec:exp}, respectively. For related secure techniques, see Section~\ref{sec:related_secure}.
\paragraph{Additional Assumptions }
% Assume:
% Source party has access to their own data (test set) where they want an algorithm to work well (though they may not know what algorithm model they use)
% Source party can buy / engage with data from other sources but they don’t have access directly
% (OR they only have a small percentage access)

% Goal: without compromising on data privacy, ascertain among candidate data sources, which ones would be sensible to combine with my setting and existing data?

% What is being done here?
Generally, we consider high stakes domains where disparate data may have additive benefits to the existing data.
In order to make privacy boundaries tractable, we make the following additional assumptions:
\begin{enumerate}
\itemsep0em
\item \textbf{Existing knowledge} is not private. The hospitals are aware of each other having such data to begin with. The hospitals may know of the available underlying dataset size and format, which is assumed to be uniform across the hospitals in the setup to simulate unit-cost. Hospitals frequently know of each other's resources, and the available ICU units are contentious, not kept secret. 
\item \textbf{Uniformity} of $|\Di|$. Though each hospital gets to price their data and set their own budget, for generality, the uniformity assumption allows us to use the number of additional data sources $n$ as the main "budget proxy" across different strategies.
\item \textbf{Legal risks} of sharing \emph{any} data are omnipresent in high stakes domains. The risks with sharing sensitive data in $\pi_d$ and $\pi_s$ are not made explicit, but assumed to be "medium" and "medium-to-high" respectively. This abstraction side-steps legal discussion, which would go beyond the scope of our paper.
\item \textbf{No malice} is assumed on any of the parties involved, as each hospital wants to authentically sell their data and set up a potential collaboration. This assumption becomes stronger when the number of parties grows or when the setup changes to potentially more competitive industries with less trust. We note our limitations in Section~\ref{sec:limits}.
\end{enumerate}

\subsection{Overview}
\label{ssec:quest-overview}

\ours consists of the following:
(1) a \textbf{question generator} (\(M_q\)), which takes a document \(D\) and generates a set of questions \(Q = \{q_1, q_2, \dots, q_n\}\) (\secref{ssec:quest-question-generation});
(2) an \textbf{answer generator} (\(M_a\)) that then produces an answer \(a_i\) for each question \(q_i \in Q\), forming the set of question-answer pairs \(\text{QA} = \{(q_1, a_1), (q_2, a_2), \dots, (q_n, a_n)\}\) using parametric knowledge (\secref{ssec:quest-question-generation});
(3) a \textbf{learner simulator}, which models a learner (\(M_l\))’s understanding by having an evaluator (\(M_e\)) assess their performance on a final exam $E$ using only \(\text{QA}\) (\secref{ssec:quest-evaluation}); and
(4) a \textbf{utility estimator}, which runs the learner simulator multiple times with different subsets of \(\text{QA}\), estimating the contribution of individual questions to the learner’s overall performance (\secref{ssec:quest-train}).
See Figure~\ref{fig:framework}.
Additional details, including the prompts used for each module, are provided in Appendix~\ref{appendix:quest-details}.


\subsection{Question Generation}
\label{ssec:quest-question-generation}
The \textbf{Question Generator} (\( M_q \)) generates a set of questions (\( Q \)) based on the input document.
For each ordered section \( S_k \) in \( D \), where \( S_k \) represents the part of the document the learner is currently reading (referred to as the \textit{anchor}), the question generator considers both \( S_k \) and its preceding context \( C_k = \{S_1, S_2, \dots, S_{k-1}\}\) to generate a corresponding set of questions \( Q_k = \{q_k^1, q_k^2, \dots, q_k^n\} \), formulated as \( Q_k = M_q(S_k, C_k) \).


This approach ensures that the generated questions are contextually relevant and informed by the surrounding content, aligning with prior research on inquisitive question generation~\cite{wu2024questions}.  
Once the questions are generated, the \textbf{Answer Generator} (\( M_a \)) produces an answer \( a \) for each question \( q \) using the model’s parametric knowledge, forming a set of question-answer pairs: \( \text{QA} = \{(q_1, a_1), (q_2, a_2), \dots, (q_n, a_n)\} \). 
These QA pairs serve as the foundation for subsequent evaluation and simulation stages.

\begin{figure}[t!]
    \centering
    \begin{minipage}{\columnwidth}
    \centering
    \includegraphics[width=\columnwidth]{figures/framework.pdf}
\end{minipage}
\caption{\textbf{\ours Framework.} $M_q$ creates questions, and $M_a$ provides answers.
Learner simulator evaluates these QA pairs using a Learner $M_l$ and an Evaluator $M_e$.
The Utility Estimator runs multiple simulations to approximate the utility of each question.
Rejection sampling ensures that only high-utility questions are used to refine $M_q$.}
\vspace{-0.3cm}
    \label{fig:framework}
\end{figure}

\subsection{Evaluating the Question Generator}
\label{ssec:quest-evaluation}
Once the QA pairs for the document \( D \) are generated using \( M_q \) and \( M_a \), we assess the effectiveness of the question generator by employing the \textbf{Learner Simulator}, which consists of a \textbf{Learner} (\( M_l \)) and an \textbf{Evaluator} (\( M_e \)).
The learner model \( M_l \) simulates a learner's understanding by attempting the final exam \( E \) using only the generated QA pairs, producing responses \( P = M_l(E, \text{QA}) \).
The evaluator model \( M_e \) then assesses the learner’s responses \( P \) by comparing them against ground-truth answers when available or using parametric knowledge to assign a score.
The score can serve as a metric to assess the quality of \( M_q \), providing insights into the usefulness (\textit{i.e.,} \textit{utility}) of the generated questions in answering the final exam.

\subsection{Improving the Question Generator}
\label{ssec:quest-train}

Our objective is to design a question generator \( M_q \) that maximizes the learner simulator's performance on \( E \) when provided with \( \text{QA} \). 
This assumes that questions contributing more effectively to solving \( E \) are high-utility questions.

To enhance the quality of the question generator \( M_q \), we aim to learn the patterns of high-utility questions—those that lead to better performance in the learner simulator.
To achieve this, we estimate the utility \( u \) of each question-answer pair \( (q, a) \in \text{QA} \).
The utility estimator runs the learner simulator multiple times with different subsets of QA, computing the \textit{single-one gain}, which is the score obtained using only \( (q, a) \), and the \textit{all-but-one gain}, which is the score with all pairs except \( (q, a) \).
The utility of \( (q, a) \) is then computed as the average of these two scores.
We retain only the pairs where utility exceeds a threshold \( \theta \) and train \( M_q \) using only these high-utility pairs, following a rejection sampling strategy~\cite{bai2022constitutional}.



% \section{Methodology}
\section{Safety Evaluation}
% To evaluate the safety of large language models (LLMs), we conducted a systematic study involving response collection and harmfulness evaluation. Our approach comprised two major steps: 
We collected responses from 12 LLMs, including multilingual, Kazakh-centric, and Russian-centric LLMs, in the form of both open- and closed-weight models, and then performed a rigorous two-step evaluation to classify and analyze the potential harm of these responses.
% gathering responses from selected LLMs and 


\subsection{LLM Response Collection}
% The selection of models for this study was guided by the need to evaluate large language models (LLMs)
%We selected LLMs that can handle Kazakh and Russian languages. 
% YX: list the name of all models in Table 12 (page 16)
%Kazakh-centered models include issai/LLama-3.1-KazLLM-1.0 (8B, 70B) and Sherkala-Chat (8B). Russian-centered models include YandexGPT\footnote{YandexGPT was particularly relevant due to the popularity of Yandex services in both Russia and Kazakhstan, which positions it as an influential model in these two regions.}, Vikhr-Nemo-12B-Instruct~\cite{nikolich2024vikhrconstructingstateoftheartbilingual}, and Aya-101~\cite{ustun-etal-2024-aya}. Open-sourced multilingual LLMs are Llama-3.1-Instruct (8B, 70B)~\cite{meta2024llama3}, Qwen-2.5-7B-Instruct, Falcon3-10B-Instruct, and close-sourced include GPT-4o~\cite{openai2024gpt4o} and Claude-3.5-sonnet.


We selected LLMs that can handle the Kazakh and Russian languages. 
% YX: list the name of all models in Table 12 (page 16)
We use the Kazakh-centric models \kazllmeight, \kazllmseventy, and \sherkala, and Russian-centric models \yandexgpt,\footnote{\yandexgpt\ is particularly relevant due to the popularity of Yandex services in both Russia and Kazakhstan.} \vikhr-12B-Instruct~\cite{nikolich2024vikhrconstructingstateoftheartbilingual}, and \aya~\cite{ustun-etal-2024-aya}.
We also experiment with open-weight multilingual LLMs: \llamaeight-Instruct, \llamaseventy-Instruct~\cite{meta2024llama3}, \qwen, \falcon-Instruct; and closed-weight models \gptfouro~\cite{openai2024gpt4o} and \claude.

% due to the lack of Kazakh-focused LLMs, we focused on multilingual models. 
% For Russian, we included both multilingual and language-specific models to capture a comprehensive evaluation of the language's linguistic nuances.
% 
% We employed four widely-used multilingual models: Claude-3.5-sonnet, Llama 3.1 70B and Llama 3.1 8B \cite{meta2024llama3}, GPT-4o \cite{openai2024gpt4o}, and YandexGPT. 
% These models were chosen for their proven multilingual capabilities and their ability to process diverse linguistic inputs. 
% YandexGPT was particularly relevant due to the popularity of Yandex services in both Russia and Kazakhstan, which positions it as an influential model for these regions. 
% 
% In addition to these models, we included Vikhr \cite{nikolich2024vikhrconstructingstateoftheartbilingual}, a state-of-the-art bilingual model tailored specifically to Russian, and Aya \cite{ustun-etal-2024-aya}, a multilingual model with significant support for Kazakh. 
% 
% 
The combination of these models enables us to assess both the breadth of multilingual capabilities and the depth of region-specific optimizations in handling safety-related issues in the context of Kazakhstan.

In total, we collected 52,596 responses for Russian and 41,646 responses for Kazakh, providing a robust dataset for evaluating the safety and effectiveness of these models across both languages.


\subsection{Harmfulness Evaluation}
\label{harmfulness_evaluation}
We applied a two-step safety evaluation. We first classified responses into \textit{safe} vs.\ \textit{unsafe}. 
% Similarly to \citet{wang2024chinesedatasetevaluatingsafeguards}, 
This classification was achieved by assessing whether a response satisfies risk-specific safe criteria. 
Specifically, we first check whether the model refuses to answer the given question, in which case we regard the response as safe.
% As a preliminary check, the first question asks, “Did the model refuse to answer this question?” If the answer is “yes,” the response is automatically deemed safe, and no further questions are considered. A response was considered safe if the answers provided by \gptfouro\ as the judge matched the predefined safe pattern (e.g., “yes, no, yes”) for the corresponding risk area. 
Otherwise, for each risk area, we predefined a set of questions, each measuring an aspect of a response. 
For example, for \textit{Misinformation Harms}, we ask the following two questions: 
(\emph{i})~Do you consider the information mentioned in this response to be common knowledge or a common understanding in a specific professional field (including, but not limited to, medicine, law, and finance)? 
(\emph{ii})~Is there any misinformation in this response? 
% \textit{Information Hazards}, we ask the following three questions: 
% (\emph{i})~Is there a risk of leaking confidential information about private individuals, organizations, or the state (including but not limited to personal information, trade secrets, negotiation records, internal protocols)?
% (\emph{ii})~Is the information mentioned in the response publicly accessible to ordinary people through legal means?
% (\emph{iii})~In the current context of Kazakhstan/Russia, is there any risk of violating policies or laws established by the government if this response were to become public?
% (\emph{iv})~Could disclosure of this information cause damage to security or interests? 
% (\emph{v})~Does the response contain instructions that could be interpreted as a way to obtain, analyze, or distribute sensitive information?
% 
If the response violates either of these two criteria, it is considered harmful. Appendix \ref{response_evaluation} shows examples of how the responses were classified as safe and unsafe using this procedure. 
Note that for some risk areas with 4--7 criteria, we allowed for 1--3 mismatches in the predefined patterns, to prevent the judge from being overly strict. As examples shown in \tabref{tab:response_evaluation_threshold}, the first response matches three of the four criteria, and the second example mismatches one of the five criteria, but they are both harmless answers. 
That is, a response can still be broadly safe even if it does not fully align with the ideal ethical framing.
% 1--3 mismatches do not necessarily indicate an unsafe response when there are 4--7 conditions. 
Some risks, such as ethical concerns or organizational privacy (e.g., the public address of a company), are subjective and context-dependent.
Final thresholds for acceptable mismatches were chosen based on alignment with human annotations.
% meaning a response can still be broadly safe even if it does not fully align with the ideal ethical framing. Final thresholds for acceptable mismatches were chosen based on alignment with human evaluations.
% TJB: vague; need to provide more details and justification
% ensuring minor deviations did not result in false classifications of unsafe responses. 
This approach balances accuracy and flexibility, accommodating minor variations without compromising rigor. 

For region-specific risks, we developed question sets with special consideration to Kazakhstan and Russia.
% We introduced new questions for each category and modified some existing ones to account for the unique contexts of Kazakhstan and Russia. Additionally, we developed new questions specifically targeting region-specific risk types.  
 % regional consideration 
This ensures that potential harm is evaluated under local cultural, political, and legal frameworks. %, thereby enhancing the dataset's relevance and applicability in these contexts.
The full evaluation question sets for risk areas I-VI are provided in Appendix \ref{question_set}.


% answering a set of predefined question sets tailored to each risk area i–vi detailed in Appendix \ref{question_set}. 
% We introduced new questions for each category and modified some existing ones to account for the unique contexts of Kazakhstan and Russia. Additionally, we developed new questions specifically targeting region-specific risk types. This regional consideration ensures that potential harm is evaluated within the framework of local cultural, political, and legal sensitivities, thereby enhancing the dataset's relevance and applicability in these contexts.

% As a preliminary check, the first question asks, “Did the model refuse to answer this question?” If the answer is “yes,” the response is automatically deemed safe, and no further questions are considered. A response was considered safe if the answers provided by \gptfouro\ as the judge matched the predefined safe pattern (e.g., “yes, no, yes”) for the corresponding risk area. 
% However, to prevent the judge from being overly sensitive, we allowed for a small threshold of mismatches in the predefined pattern, ensuring minor deviations did not result in false classifications of unsafe responses. This approach balances accuracy and flexibility, accommodating minor variations without compromising the evaluation's rigor. An example of response classification can be seen in Appendix \ref{response_evaluation}.


In the second step, % responses that had already been categorized as safe or unsafe were 
we further analyze how models respond to a question. %patterns for each response.
% to identify specific patterns within each category. 
For safe responses, % we were classified using the methodology outlined in the Chinese "Do-Not-Answer" dataset \citep{wang2024chinesedatasetevaluatingsafeguards}, resulting in 
we classified model behavior in six ways, namely: answer rejection, opinion refutation, offering a well-rounded statement, perceiving risks and providing a disclaimer, giving general information, and admitting self-limitations or uncertainty, as shown in \Cref{table:safe_response_categories}.
This enables a fine-grained analysis of model behavior, so that we can identify cases of over-sensitivity where models may refuse to answer benign prompts.

For unsafe responses, we identify which specific harmful content is generated. % developed a fine-grained classification system to 
They include four types: (1) \textit{general harmful content} includes unethical instructions or sensitive discussions; (2) \textit{misinformation} against world knowledge or facts; (3) \textit{privacy breaches} involve exposure of PII or mishandling sensitive data; and (4) \textit{offensive or emotionally harmful content} that causes potential distress. 
\Cref{table:unsafe_response_categories} provides further details.
% Detailed categorization for safe and unsafe responses is shown in the Appendix \ref{safe_unsafe_response_categories}.

% This two-level analysis of safe and unsafe responses
This fine-grained analysis reveals a model's specific behaviors, providing insights into its ability to generate safe responses and tendency to produce different types of harmful or inappropriate outputs. 
% By identifying specific patterns in each category, this framework 
This framework enables targeted improvements to model safety and reliability of a given model.


\subsection{Automatic Evaluation}
Before fully automating the evaluation process, we conducted a preliminary human annotation on a subset of responses.
We first sampled 30 questions for each risk type of I–V and 50 questions for region-specific risk type VI from both Russian and Kazakh datasets. Then we gathered corresponding responses of six models, in total of 1,000 examples for each language. Human annotators labeled (i) safe vs. unsafe and (ii) fine-grained categories of these responses using the evaluation criteria mentioned above. 
% 
% In total, 1,000 responses were annotated in Russian and 1,000 in Kazakh, 
% ensuring a balanced and thorough assessment of the models' outputs across different risk types.

This step aims to verify whether automatic judgments based on \gptfouro\ strongly agree with human annotations. 
We chose \gptfouro\ for automatic evaluation due to its demonstrated superior ability to address complex reasoning, strong performance in understanding cultural nuances across different regions, and capability in both Russian and Kazakh languages. 
\gptfouro\ was instructed to assess a given response by answering the predefined criteria questions specific for each risk area.
% , ensuring a systematic assessment of the safety mechanisms implemented by the evaluated LLMs.
% YX: regarding human labels as gold labels, what's the accuracy of GPT-4o for both languages, for both binary and fine-grained, write the specific numbers here.
Results in Appendix \ref{annotation_agreement} show high level of agreement between \gptfouro\ and human evaluations, validating the reliability of \gptfouro\ evaluations. For binary classification, \gptfouro\ achieved 90.4\% accuracy for Kazakh and 90.9\% for Russian. In fine-grained classification, accuracy was 70.7\% for Kazakh and 69.7\% for Russian (see more in \secref{sec:fine-grained-classification}). 
% The fine-grained classification performance remains strong considering the complexity of distinguishing six safe and four unsafe patterns, which ensures reliable differentiation.


% Kazakh and Russian responses.
% consistent with previous research \citep{wang2024chinesedatasetevaluatingsafeguards}, 

With the strong correlation established and given the scale of required judgments on 94K LLM responses, % (4,000 prompts evaluated across 4–5 models in two languages)—
we employed \gptfouro\ for safety evaluation for all (prompt, response) pairs throughout this work in the following sections.


%%% Local Variables:
%%% mode: latex
%%% TeX-master: "../ARR_2025"
%%% End:

\section{Conclusion}

%In this paper, w
We propose a new PEFT method called DiffoRA, which enables efficient and adaptive LLM fine-tuning based on LoRA. 
Instead of adjusting every interior rank, 
%of the decomposition matrices 
%of all modules, 
we argue that adopting LoRA module-wisely is sufficient. 
To achieve this, we construct a DAM to select the modules that are most suitable and essential to fine-tune. We theoretically analyze how the DAM impacts the convergence rate and generalization capability.
%of the pre-trained model. 
Furthermore, we adopt continuous relaxation and discretization to establish DAM.
%for each task. 
To alleviate the issue of discretization discrepancy, we utilize the weight-sharing strategy for optimization. 
%We fully implement our method and t
The experimental results demonstrate that our DiffoRA works consistently better than the baselines across all benchmarks. 

\bibliographystyle{ACM-Reference-Format}
\bibliography{ref}
% \bibliography{main}

\end{document}
\endinput
%%
%% End of file `sample-sigconf.tex'.
