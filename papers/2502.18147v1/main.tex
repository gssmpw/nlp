\documentclass{article}

\newif\ifarxiv
\arxivtrue % Uncomment for arXiv version
% \arxivfalse % Uncomment for ICML version
\newcommand{\onlyarxiv}[1]{\ifarxiv#1\fi}
\newcommand{\onlyicml}[1]{\ifarxiv\else#1\fi}

% commands
\newcommand{\red}[1]{\textcolor{red}{#1}}
% \newcommand{\red}[1]{}
\newcommand{\cn}{\red{citation needed }}

\usepackage[T1]{fontenc}
\usepackage{microtype}
\usepackage{graphicx}
\usepackage{subcaption}
\usepackage{booktabs}
\usepackage{siunitx}
\usepackage[dvipsnames]{xcolor}
\usepackage{placeins}
\usepackage{colortbl}
\usepackage{pgfplots}
\usepackage{pgfplotstable}

\usepackage{array}
\newcommand{\EscapeBackslash}[1]{\let\temp=\\#1\let\\=\temp}
\newcolumntype{C}[1]{>{\EscapeBackslash\centering}p{#1}}
\newcolumntype{R}[1]{>{\EscapeBackslash\raggedleft}p{#1}}
\newcolumntype{L}[1]{>{\EscapeBackslash\raggedright}p{#1}}

\usepackage{hyperref}
\newcommand{\theHalgorithm}{\arabic{algorithm}}

\onlyicml{\usepackage{icml2025}}
\onlyarxiv{\usepackage[accepted]{icml2025_arxiv}}

\usepackage{amsmath}
\usepackage{amssymb}
\usepackage{mathtools}
\usepackage{amsthm}

\usepackage[capitalize,noabbrev]{cleveref}

\renewcommand{\vec}[1]{\mathbf{#1}}
\newcommand{\mat}[1]{\mathbf{#1}}
\newcommand{\Reals}{\mathbb{R}}
\DeclareMathOperator{\ReLU}{ReLU}
\DeclareMathOperator{\TopK}{TopK}
\newcommand{\dense}{\text{de}}
\newcommand{\sparse}{\text{sp}}
\newcommand{\post}{\text{post}}
\newcommand{\pre}{\text{pre}}
\newcommand{\dec}{\text{dec}}
\newcommand{\enc}{\text{enc}}
\newcommand{\FFN}{\text{FFN}}
\newcommand{\SAE}{\text{SAE}}

\newcommand{\W}{\mat{W}}

\newcommand{\Wenc}{\W_\text{\!enc}}
\newcommand{\Wdec}{\W_\text{\!dec}}
\newcommand{\benc}{\vec{b}_\text{enc}}
\newcommand{\bdec}{\vec{b}_\text{dec}}

\newcommand{\Win}{\W_\text{\!1}}
\newcommand{\Wout}{\W_\text{\!2}}

\newcommand{\setx}{\mathcal{X}}
\newcommand{\sety}{\mathcal{Y}}
\newcommand{\x}{\vec{x}}
\newcommand{\y}{\vec{y}}

\newcommand{\hatx}{\vec{\hat{x}}}
\newcommand{\haty}{\vec{\hat{y}}}

\newcommand{\setsx}{\mathcal{S}_\text{x}}
\newcommand{\setsy}{\mathcal{S}_\text{y}}
\newcommand{\sx}{\vec{s}_\text{x}}
\newcommand{\sy}{\vec{s}_\text{y}}

\newcommand{\sxj}{s_{\text{x},j}}
\newcommand{\syi}{s_{\text{y},i}}
\newcommand{\bars}{\bar{\vec{s}}}
\newcommand{\barsx}{\bar{\vec{s}}_\text{x}}
\newcommand{\barsxj}{\bar{s}_{\text{x},j}}
\newcommand{\syprime}{\vec{s}'_\text{y}}

\newcommand{\encx}{e_\text{x}}
\newcommand{\ency}{e_\text{y}}
\newcommand{\decx}{d_\text{x}}
\newcommand{\decy}{d_\text{y}}

\newcommand{\Wencx}{\mathbf{W}_\text{x}^\text{enc}}
\newcommand{\bencx}{\mathbf{b}_\text{x}^\text{enc}}
\newcommand{\Wdecx}{\mathbf{W}_\text{x}^\text{dec}}
\newcommand{\bdecx}{\mathbf{b}_\text{x}^\text{dec}}

\newcommand{\Wency}{\mathbf{W}_\text{y}^\text{enc}}
\newcommand{\bency}{\mathbf{b}_\text{y}^\text{enc}}
\newcommand{\Wdecy}{\mathbf{W}_\text{y}^\text{dec}}
\newcommand{\bdecy}{\mathbf{b}_\text{y}^\text{dec}}

\newcommand{\Wdecxmj}{W_{\text{x},mj}^\text{dec}}
\newcommand{\Wencyik}{W_{\text{y},ik}^\text{enc}}
\newcommand{\bencyi}{b_{\text{y},i}^\text{enc}}
\newcommand{\Wencyactive}{\mathbf{W}_\text{y}^\text{enc(active)}}
\newcommand{\Wdecxactive}{\mathbf{W}_\text{x}^\text{dec(active)}}

\newcommand{\dimx}{m_\text{x}}
\newcommand{\dimsx}{n_\text{x}}
\newcommand{\dimy}{m_\text{y}}
\newcommand{\dimsy}{n_\text{y}}

\newcommand{\relu}{\phi}
\newcommand{\topk}{\tau_k}

\newcommand{\fsij}{f_{s,(i,j)}|_{\sx}}

\theoremstyle{plain}
\newtheorem{theorem}{Theorem}[section]
\newtheorem{proposition}[theorem]{Proposition}
\newtheorem{lemma}[theorem]{Lemma}
\newtheorem{corollary}[theorem]{Corollary}
\theoremstyle{definition}
\newtheorem{definition}[theorem]{Definition}
\newtheorem{assumption}[theorem]{Assumption}
\theoremstyle{remark}
\newtheorem{remark}[theorem]{Remark}

% \usepackage[disable,textsize=tiny]{todonotes}
\usepackage[textsize=tiny]{todonotes}

\icmltitlerunning{Jacobian Sparse Autoencoders: Sparsify Computations, Not Just Activations}

\begin{document}

\twocolumn[
\icmltitle{Jacobian Sparse Autoencoders: Sparsify Computations, Not Just Activations}

\icmlsetsymbol{equal}{*}

\begin{icmlauthorlist}
\icmlauthor{Lucy Farnik}{bris}
\icmlauthor{Tim Lawson}{bris}
\icmlauthor{Conor Houghton}{bris}
\icmlauthor{Laurence Aitchison}{bris}
\end{icmlauthorlist}

\icmlaffiliation{bris}{School of Engineering Mathematics and Technology, University of Bristol, Bristol, UK}

\icmlcorrespondingauthor{Lucy Farnik}{lucyfarnik@gmail.com}

\icmlkeywords{Mechanistic Interpretability, Sparse Autoencoders, Superposition, Circuits, AI Safety}

\vskip 0.3in
]

\printAffiliationsAndNotice{}
% \printAffiliationsAndNotice{\icmlEqualContribution}

% notes on formatting

\begin{abstract}
Sparse autoencoders (SAEs) have been successfully used to discover sparse and human-interpretable representations of the latent activations of LLMs.
However, we would ultimately like to understand the computations performed by LLMs and not just their representations.
The extent to which SAEs can help us understand computations is unclear because they are not designed to ``sparsify'' computations in any sense, only latent activations.
%However, existing work has focused on using SAEs to understand the representations present at a particular point in the model.
%This means that SAEs give us little insight into the computation performed by specific model components.
To solve this, we propose Jacobian SAEs (JSAEs), which yield not only sparsity in the input and output activations of a given model component but also sparsity in the computation (formally, the Jacobian) connecting them.
With a na\"ive implementation, the Jacobians in LLMs would be computationally intractable due to their size.
One key technical contribution is thus finding an efficient way of computing Jacobians in this setup.
We find that JSAEs extract a relatively large degree of computational sparsity while preserving downstream LLM performance approximately as well as traditional SAEs.
% We also find that the degree of interpretability and monosemanticity of JSAE latents is roughly on par with traditional SAEs but with the added benefit of interpretable computational connections from input latents to output latents.
We also show that Jacobians are a reasonable proxy for computational sparsity because MLPs are approximately linear when rewritten in the JSAE basis.
Lastly, we show that JSAEs achieve a greater degree of computational sparsity on pre-trained LLMs than on the equivalent randomized LLM. This shows that the sparsity of the computational graph appears to be a property that LLMs learn through training, and suggests that JSAEs might be more suitable for understanding learned transformer computations than standard SAEs.
%We also use JSAEs to demonstrate that pretrained LLMs have a greater degree of Jacobian sparsity than randomly re-initialized LLMs, which suggests Jacobian sparsity may be a more useful metric than sparsity of SAE latents for understanding the computation learned by LLMs.
%This suggests JSAEs may be better suited for discovering and analyzing computational structures.
\end{abstract}

\section{Introduction}

Sparse autoencoders (SAEs) have emerged as a powerful tool for understanding the internal representations of large language models \citep{bricken_monosemanticity_2023,cunningham_sparse_2023,gao_scaling_2024,rajamanoharan_jumping_2024,lieberum_gemma_2024,lawson_residual_2024,braun_identifying_2024,kissane2024interpretingattentionlayeroutputs,rajamanoharan_improving_2024}.
By decomposing neural network activations into sparse, interpretable components, SAEs have helped researchers gain significant insights into how these models process information \citep{marks_sparse_2024,lieberum_gemma_2024,templeton_scaling_2024,obrien_steering_2024,farrell2024applyingsparseautoencodersunlearn,paulo_automatically_2024,balcells2024evolutionsaefeatureslayers,lan2024sparseautoencodersrevealuniversal,brinkmann2025largelanguagemodelsshare, spies2024transformersusecausalworld}.

When trained on the activation vectors from neural network layers, SAEs learn to reconstruct the inputs using a dictionary of sparse `features', where there are many more features than basis dimensions of the inputs, and each feature tends to capture a specific, interpretable concept.
% This sparsity is crucial: while the original neural activations typically involve complex interactions between many dimensions, SAE features are mostly inactive, firing strongly only when the corresponding specific concept is present in the input activations.
However, the goal of this paper is to improve understanding of \emph{computations} in transformers.
While SAEs are designed to disentangle the representations of concepts in the LLM, they are not designed to help us understand the computations performed with those representations.
Indeed, SAEs have been shown to exhibit pathological behaviors such as feature absorption, which seem unlikely to be properties of the actual LLM computation \citep{absorption}.

One approach to understanding computation would be to train two SAEs, one at the input and one at the output of an MLP in a transformer. 
We can then ask how the MLP maps sparse latent features at the inputs to sparse features in the outputs.
For this mapping to be interpretable, it would be desirable that it is sparse, in the sense that each latent in the SAE trained on the output depends on a small number of latents of the SAE trained on the input.
These dependencies can be understood as a computation graph or `circuit' \citep{olah_zoom_2020,cammarata_thread_2020}.
SAEs are not designed to encourage this computation graph to be sparse.
To address this, we develop Jacobian SAEs (JSAEs), where we include a term in the objective to encourage SAE bases with sparse computational graphs, not just sparse activations.
Specifically, we treat the mapping between the latent activations of the input and output SAEs as a function and encourage its Jacobian to be sparse by including an $L^1$ penalty term in the loss function.

%SAEs disentangle superposition, but they only focus on sparsity in the latent activations and do not necessarily encourage sparsity in the computation graph which connects the activations at different points in the model. This makes them less than ideal for finding circuits

\begin{figure*}
    \centering
    \includegraphics{figures/schematic.pdf}
    \caption{A diagram illustrating our setup.
    We have two SAEs: one trained on the MLP inputs and the other trained on the MLP outputs.
    We then consider the function $f_s$, which takes the latent activations of the first SAE and returns the latent activations of the second SAE, i.e., $f_s(\vec{s}_\text{x})=\vec{s}_\text{y}$.
    The function $f_s$ is described by the function composition of the TopK activation function of the first (input) SAE $\topk$, the decoder of the first SAE $\decx$, the MLP $f$, and the encoder of the second (output) SAE $\ency$.
    We note that the activation function $\topk$ is included for computational efficiency only; see Section~\ref{sec:jacobian_tractable} for details.
    JSAEs optimize for $f_s$ having a sparse Jacobian matrix, which we illustrate by reducing the number of edges in the computational graph that corresponds to $f_s$.
    Traditional SAEs have sparse SAE latents on either side of the MLP but a dense computational graph between them; JSAEs have both sparse SAE latents \textit{and} a sparse computational graph.
    Importantly, Jacobian sparsity approximates the computational graph notion, but, as we discuss in Section~\ref{sec:mostly_linear} and Appendix~\ref{app:not_local}, this approximation is highly accurate due to the fact that $f_s$ is a mostly linear function.}
    \label{fig:schematic}
\end{figure*}

%To address this, we train SAEs with the explicit criterion of finding sparse computation.
%We do this by training an SAE pair, with one SAE on the MLP's inputs and the other one on the MLP's outputs.
%We then take the function from one SAE acts to the other and optimize for it to have a sparse Jacobian by applying an L1 penalty.
%We demonstrate that Jacobians are a solid, if imperfect, metric of sparsity, and discuss the limitations of using them as a proxy.
%
%CHART: conceptual chart visualizing what kind of connections we're looking for, maybe also contrasting against 
%normal SAEs and transcoders

With a na\"ive implementation, it is intractable to compute Jacobian matrices because each matrix would have on the order of a trillion elements, even for modestly sized language models and SAEs.
% Naively computing this Jacobian would be intractible, as it would have over a trillion elements even for modestly-sized LLMs and SAEs (when including the batch dimension).
Therefore, one of our core contributions is to develop an efficient means to compute Jacobian matrices in this context.
The approach we develop makes it possible to train a pair of Jacobian SAEs with only approximately double the computational requirements of training a single standard SAE (Section~\ref{sec:jacobian_tractable}).
% One of our core contributions is therefore developing an efficient way to compute the Jacobian in this setup, which allows us to train JSAEs with roughly the same amount of computational resources as traditional SAEs.
These methods enabled us to make three downstream findings.

First, we find that Jacobian SAEs successfully induce sparsity in the Jacobian matrices between input and output SAE latents relative to standard SAEs without a Jacobian term (Section~\ref{sec:jac_sparsity}).
%Moreover, the distribution of absolute values of elements of the Jacobian matrix is extremely heavy-tailed, particularly when compared to standard SAEs trained without our Jacobian term in the objective (Section~\ref{sec:jac_sparsity}).
%The distribution being heavy-tailed means that a relatively small number of elements are responsible for a large amount of the computation, while a large number of elements are near-zero and can effectively be ignored; this implies that there is a relatively simple structure.
% The distribution of Jacobian elements is very heavy-tailed, much more than with traditional SAEs (i.e. SAEs trained without the Jacobian sparsity objective).
We find that JSAEs achieve the desired increase in the sparsity of the Jacobian with only a slight decrease in reconstruction quality and model performance preservation, which remain roughly on par with standard SAEs.
% JSAEs achieve this without sacrificing on reconstruction quality which is on par with traditional SAEs.
We also find that the input and output latents learned by Jacobian SAEs are approximately as interpretable as standard SAEs, as quantified by auto-interpretability scores.
% We also show that the features of these SAEs are roughly as interpretable as those of standard SAEs
% \todo{we don't have these results yet and need to get them}
% Notably, we further show that the causal connections represented by non-zero elements of our Jacobian matrices may have clearly interpretable semantics. \todo{we don't have these results yet}

Second, inspired by \citet{heap_sparse_2025}, we investigated the behavior of Jacobian SAEs when applied to random transformers, i.e., where the parameters have been reinitialized.
We find that the degree of Jacobian sparsity that can be achieved when JSAEs are applied to a pre-trained transformer is much greater than the sparsity achieved for a random transformer (Section~\ref{sec:random}).
% This preliminary finding appears to confirm that pre-trained LLMs exhibit a greater degree of computational structure than randomized LLMs.
This preliminary finding suggests that Jacobian sparsity may be a useful tool for discovering learned computational structure.

Lastly, we find that Jacobians accurately approximate computational sparsity in this context because the function we are analyzing (i.e., the combination of JSAEs and MLP) is approximately linear (Section~\ref{sec:mostly_linear}).

Our source code can be found 
\onlyicml{in the supplementary material.}
\onlyarxiv{at \href{https://github.com/lucyfarnik/jacobian-saes}{https://github.com/lucyfarnik/jacobian-saes}.}

\section{Related work}
% \todo{make this more concise}

\subsection{Sparse autoencoders}

SAEs have been widely applied to `disentangle' the representations learned by transformer language models into a very large number of concepts, a.k.a. sparse latents, features, or dictionary elements
% There is a very large body of work introducing SAEs as a method for understanding ``concepts'', or sparse features in transformer activations 
\citep{sharkey_taking_2022,cunningham_sparse_2023,bricken_monosemanticity_2023,gao_scaling_2024,rajamanoharan_jumping_2024,lieberum_gemma_2024}.
Human experiments and quantitative proxies apparently confirm that SAE latents are much more likely to correspond to human-interpretable concepts than raw language-model neurons, i.e., the basis dimensions of their activation vectors \citep{cunningham_sparse_2023,bricken_monosemanticity_2023,rajamanoharan_improving_2024}.
SAEs have been successfully applied to modifying the behavior of LLMs by using a direction discovered by an SAE to ``steer'' the model towards a certain concept \citep{makelov_sparse_2024,obrien_steering_2024,templeton_scaling_2024}.

% The most common application for SAEs is to `intervene' in the activations of the underlying model in terms of the dictionary elements (decoder weight vectors) learned by SAEs \citep{makelov_sparse_2024,obrien_steering_2024,templeton_scaling_2024}
% This is expected, given that SAEs find a basis for the underlying activations in terms of relatively interpretable latent concepts, each of which is active when the text input to the model expresses the corresponding concepts.
% By `steering' the activations, i.e., modifying them to artificially express the corresponding concepts to a greater or lesser degree, we can usefully alter the behavior of the language model, but these approaches provide limited insight into the form of computations that take place in the model.

% There is a body of work showing that it is possible to intervene on neural networks by exploiting the sparse basis learned by an SAE \todo{refs}.
% The success of these methods is perhaps not surprising: SAEs give a basis for network activations in terms of human-interpretable concepts, in the sense that directions in that basis are active when the input text contains the corresponding concept.
% Thus, if you intervene on the activations to push them in the direction corresponding to a particular concept, then it makes sense that the network will start to behave as if that concept was present in the input.
% While this approach clearly facilitates useful interventions on transformer activations, and validates the usefulness of SAE bases in practice, the depth of insights it can provide into the underlying computation is not clear.

Our work is based on SAEs but has a very different aim: standard SAEs only sparsify activations, while JSAEs also sparsify the computation graph between them (Figure~\ref{fig:schematic}).

\subsection{Transcoders}

In this paper, we focus on MLPs.
\citet{dunefsky_transcoders_2024,templeton_predicting_2024} developed \emph{transcoders}, an alternative SAE-like method to understand MLPs. 
However, JSAEs and transcoders take radically different approaches and solve radically different problems.
This is perhaps easiest to see if we look at what transcoders and JSAEs sparsify.
JSAEs are fundamentally an extension of standard SAEs: they train SAEs at the input and output of the MLP and add an extra term to the objective such that these sparse latents are also appropriate for interpreting the MLP (Figure~\ref{fig:schematic}).
In contrast, transcoders do not sparsify the inputs and outputs; they work with dense inputs and outputs.
Instead, transcoders, in essence, sparsify the MLP hidden states.
Specifically, a transcoder is an MLP that you train to match (using a mean squared error objective) the input-to-output mapping of the underlying MLP from the transformer.
The key difference between the transcoder MLP and the underlying MLP is that the transcoder MLP is much wider, and its hidden layer is trained to be sparse.

Thus, transcoders and JSAEs take fundamentally different approaches.
Each transcoder latent tells us `there is computation in the MLP related to [concept].'
By comparison, JSAEs learn a pair of SAEs (which have mostly interpretable latents) and sparse connections between them.
At a conceptual level, JSAEs tell us that `this feature in the MLP's output was computed using only these few input features'.
Ultimately, we believe that the JSAE approach, grounded in understanding how the SAE basis at one layer is mapped to the SAE basis at another layer, is potentially powerful and worth thoroughly exploring.

Importantly, it is worth emphasizing that JSAEs and transcoders are asking fundamentally different questions, as can be seen in terms of e.g., differences in what they sparsify.  
As such, it is not, to our knowledge, possible to design meaningful quantitative comparisons, at least not without extensive future work to develop very general auto-interpretability methods for evaluating methods of understanding MLP circuits.

%Of course, JSAEs also take an SAE-based approach to understand MLPs.
%
%These differences turn up at two levels: what transcoders / JSAEs sparsify, and what objective they use.
%First, JSAEs find a sparse basis for the input and output of the MLP, and abstract away the hidden layer. 
%In contrast, transcoders do not find a sparse basis for the input and output; instead, they find a sparse basis for the hidden layer.
%Second, transcoders train this sparse basis so that the transcoder MLP matches the MLP in the underlying transformer; they do not e.g. train the sparse basis to reconstruct the hiddens themselves.
%In contrast, JSAEs do not in any sense training anything to match the transcoder MLP.
%Instead, they train sparse bases for the input and output exactly as in a traditional SAE setup.
%In fact, if you take a JSAE, and set the coefficient for the Jacobian loss term to zero, that is precisely what you get: two SAEs trained on the input and output of the MLP.
%
%
%Ultimately, it only makes sense to compare JSAEs and transcoders in terms of how we should be trying to interpret the MLP.

\subsection{Automated circuit discovery}

In ``automated circuit discovery'', the goal is to isolate the causally relevant intermediate variables and connections between them necessary for a neural network to perform a given task \citep{olah_zoom_2020}.
In this context, a circuit is defined as a computational subgraph with an interpretable function.
The causal connections between elements are determined via activation patching, i.e., modifying or replacing the activations at a particular site of the model \citep{meng_locating_2022,zhang_best_2023,wang_interpretability_2022,hanna_how_2023}.
% For example, \citet{wang_interpretability_2022} identified the elements of GPT-2 small needed to perform indirect object identification, and \citet{hanna_how_2023} analyzed the mathematical `greater-than' operation.
In some cases, researchers have identified sub-components of transformer language models with simple algorithmic roles that appear to generalize across models \citep{olsson_context_2022}.
% While progress has been made on this front, the complexity of even small transformers makes circuit analysis onerous.

\citet{conmy_automated_2023} proposed a means to automatically prune the connections between the sub-components of a neural network to the most relevant for a given task using activation patching.
Given a choice of task (i.e., a dataset and evaluation metric), this approach to automated circuit discovery (ACDC) returns a minimal computational subgraph needed to implement the task, e.g., previously identified `circuits' like \citet{hanna_how_2023}.
Naturally, this is computationally expensive, leading other authors to explore using linear approximations to activation patching \citep{nanda_attribution_2023,syed_attribution_2024,atpstar}.
% \citet{nanda_attribution_2023} introduced the use of gradient information to make linear approximations to activation patching, termed attribution patching, which is significantly cheaper to compute.
% \citet{syed_attribution_2024} applied this technique to circuit discovery, calling it edge attribution patching (EAP) and recommending that it precedes ACDC \citep{conmy_automated_2023}.
\citet{marks_sparse_2024} later improved on this technique by using SAE latents as the nodes in the computational graph.

In a sense, these methods are supervised because they require the user to specify a task.
Naturally, it is not feasible to manually iterate over all tasks an LLM can perform, so a fully unsupervised approach is desirable.
With JSAEs, we take a step towards resolving this problem, although the architecture introduced in this paper initially only applies to a single MLP layer and not an entire model.
Additionally, to the best of our knowledge, no automated circuit discovery algorithm sparsifies the computations inside of MLPs.
% \todo{ACDC-style things typically work with attn heads because MLPs are a superposed mess, JSAEs kinda address this}

%\section{Related work (old)}
%
%\subsection{Sparse autoencoders and transcoders}
%
%The basic intuition behind applying sparse dictionary learning to language models is that models somehow represent many more `features' or concepts than their layers have basis dimensions (on the order of thousands).
%%In neuroscience, \citet{olshausen_emergence_1996} showed that the receptive fields of simple cells in the mammalian visual cortex are explained by sparse coding, where a relatively large number of signals are represented by a smaller number of elements.
%Sparse dictionary learning (SDL) approximates a set of input vectors by linear combinations of a relatively small number of basis vectors.
%The best-known classical algorithm is Independent Component Analysis \citep{bell_informationmaximization_1995,hyvarinen_independent_2000}.
%Sparse autoencoders are a simple kind of neural network whose parameters can be optimized by gradient descent over batches of input vectors \citep{lee_efficient_2006,ng_sparse_2011,makhzani_ksparse_2014}.
%
%Importantly, \citet{dunefsky_transcoders_2024} introduced \emph{transcoders}, which take the activations immediately before an MLP component as inputs and predict the activations immediately after the MLP.
%We discuss transcoders and the key differences of our proposed architecture in Appendix~\ref{app:transcoders}.
%% Pros: not task-dependent, like most circuit analysis methods?
%% Cons, especially relative to our proposal?
%
%SAEs on random LLMs (Thomas)
%% Zhang and Andreas, algorithmic capabilities of random transformers.
%
%\subsection{Automated circuit discovery}
%
%Mechanistic interpretability research often involves \emph{circuit} analysis: isolating the causally relevant intermediate variables and connections between them necessary for a neural network to perform a given task \citep{olah_zoom_2020}.
%In this context, a circuit is defined as a computational subgraph with an interpretable function, and causal connections are determined via activation patching \citep{meng_locating_2022,zhang_best_2023}.
%For example, \citet{wang_interpretability_2022} identified the elements of GPT-2 small needed to perform indirect object identification, and \citet{hanna_how_2023} analyzed the mathematical `greater-than' operation.
%In some cases, researchers have identified sub-components of transformer language models with simple algorithmic roles that appear to generalize across models \citep{olsson_context_2022}.
%While progress has been made on this front, the complexity of even small transformers makes circuit analysis onerous.
%
%\citet{conmy_automated_2023} proposed a means to automatically prune the connections between the sub-components of a neural network to the most relevant for a given task using activation patching.
%Given a choice of task (i.e., a dataset and evaluation metric), this approach to automated circuit discovery (ACDC) returns a minimal computational subgraph needed to implement the task, e.g., previously identified `circuits' like \citet{hanna_how_2023}.
%\citet{nanda_attribution_2023} introduced the use of gradient information to make linear approximations to activation patching, termed attribution patching, which is significantly cheaper to compute.
%\citet{syed_attribution_2024} applied this technique to circuit discovery, calling it edge attribution patching (EAP) and recommending that it precedes ACDC \citep{conmy_automated_2023}.
%
%\todo{add Marks et al}
%
%\subsection{Automatic neuron description}
%
%`Interpretability' ultimately depends on human judgment, but quantitative proxies are practically necessary.
%To this end, \citet{bills_language_2023} introduced auto-interpretability (as in self-interpreting).
%They generate an explanation for the per-token activation patterns of a given language-model neuron over samples from a text dataset, simulate the patterns based on the generated explanation, and score the explanation by comparing the observed and simulated activations.
%
%More recently, \citet{paulo_automatically_2024} introduced classification-based measures of the fidelity of automatic latent descriptions that are inexpensive to compute relative to simulating activation patterns.
%\citet{choi_scaling_2024} generate multiple explanations for MLP neurons based on varying subsets of the maximally activating examples.
%They seek to reduce the computational expense by distilling 8-billion parameter models to generate explanations and simulate activations based on the explanations with the highest scores.
%
%The approaches of inspecting per-token activation patterns and/or prompting a language model to describe them have been widely adopted to interpret language-model neurons and latents learned by SAEs \citep[e.g.][]{cunningham_sparse_2023,bricken_monosemanticity_2023,gao_scaling_2024,templeton_scaling_2024,lieberum_gemma_2024}.

\section{Background}

\subsection{Sparse autoencoders}
%
In an SAE, we have input vectors, $\x \in \setx = \mathbb{R}^{\dimx}$. 
We want to approximate each vector $\x$ by a sparse linear combination of vectors, $\sx \in \setsx = \mathbb{R}^{\dimsx}$.
The dimension of the sparse vector, $\dimsx$, is typically much larger than the dimension of the input vectors $\dimx$ (i.e.\ the basis is overcomplete).

In the case of SAEs, we treat the vectors as inputs to an autoencoder with an encoder $\encx: \setx \rightarrow \setsx$ and a decoder $\decx: \setsx \rightarrow \setx$ defined by,
\begin{align}
  \sx &= \encx(\x) = \phi(\Wencx \x + \bencx)\\
  \hatx &= \decx(\sx) = \Wdecx \sx + \bdecx
\end{align}
Here, the parameters are the encoder weights $\Wenc \in \mathbb{R}^{\dimsx \times \dimx}$, decoder weights $\Wdec \in \mathbb{R}^{\dimx \times \dimsx}$, encoder bias $\bencx\in\mathbb{R}^{\dimsx}$, and decoder bias $\bdecx\in\mathbb{R}^{\dimx}$.
The non-linearity $\phi$ can be, for instance, ReLU.
%The encoder is then defined as $\vec{h} = e(\vec{x})=\phi(\Wenc \vec{x} + \benc)$ where $\phi$ is a nonlinear activation function such as ReLU.
%The decoder is defined as $\vec{\hat{x}} = d(\vec{h})=\Wdec \vec{h} + \bdec$.
These parameters are then optimized to minimize the difference between $\x$ and $\hatx$, typically measured in terms of the mean squared error (MSE), while imposing an $L^1$ penalty on the latent activations $\sx$ to incentivize sparsity.

\subsection{Automatic interpretability of SAE latents}

In order to compare the quality of different SAEs, it is desirable to be able to quantify how interpretable its latents are.
A popular approach to quantifying interpretability at scale is to collect the examples that maximally activate a given latent, prompt an LLM to generate an explanation of the concept the examples have in common, and then prompt an LLM to predict whether a given prompt activates the SAE latent given the generated explanation.
We can then score the accuracy of the predicted activations relative to the ground truth.
There are several variants of this approach \citep[e.g.,][]{bills_language_2023,choi_scaling_2024}; in this paper, we use ``fuzzing'' where the scoring model classifies whether the highlighted tokens in prompts activate an SAE latent given an explanation of that latent \citep{paulo_automatically_2024}.
% \todo{expand on this}

\section{Methods}
\label{sec:methods}

The key idea with a Jacobian SAE is to train a pair of SAEs on the inputs and outputs of a neural network layer while additionally optimizing the sparsity of the Jacobian of the function that relates the input and output SAE latent activations (Figure~\ref{fig:schematic}).
In this paper, we apply Jacobian SAEs to multi-layer perceptrons (MLPs) of the kind commonly found in transformer language models \citep{radford_language_2019,biderman_pythia_2023}.

\subsection{Setup}
\label{sec:methods_setup}

Consider an MLP mapping from $\x \in \setx$ to $\y \in\sety$, i.e., $f: \setx \rightarrow \sety$ or $\y = f(\x)$.
%Suppose we have two sets of activation vectors $X$ and $Y$, where each vector $\vec{y}^{(i)} = f(\vec{x}^{(i)})$ and $f$ represents some neural network layer (in our case an MLP).
We can then train two $k$-sparse SAEs, one on $\x$ and the other on $\y$.
The resulting SAEs map from each of $\x$ and $\y$ to corresponding sparse latents $\sx \in \setsx$ and $\sy \in \setsy$, i.e., $\sx = \encx(\x)$ and $\sy=\ency(\y)$, where $\encx$ is the encoder of the first SAE and $\ency$ is the encoder of the second SAE. 
%\todo{change this, $S_X$ needs to be the pre-activation func activations}
Each of these SAEs also has a decoder that maps from the sparse latents back to an approximation of the original vector: $\hatx = \decx(\sx)$ and $\haty = \decy(\sy)$.

We may now consider the function $f_s:S_X \to S_Y$, which intuitively represents the function, $f$, but written in terms of the sparse bases learned by the SAE pair for the original vectors $\x$ and $\y$.
Specifically, we define $f_s$ by
\begin{align}
f_s=\ency \circ f\circ \decx \circ \topk
\end{align}
where $\circ$ denotes function composition.
Here, $\decx: \setsx \rightarrow \setx$ maps the sparse latents given as input to $f_s$ to ``dense'' inputs. Then, $f: \setx \rightarrow \sety$ maps the dense inputs to dense outputs. 
Finally, $\ency: \sety \rightarrow \setsy$ maps the dense outputs to sparse outputs.
Note that $f_s$ first applies the TopK activation function $\topk$ to the sparse inputs, $\sx$.
Critically, with $k$-sparse SAEs, we produce the sparse inputs by $\sx = \encx(\x)$, implying that $\sx$ only has $k$ non-zero elements. 
In that setting, TopK does not change the inputs, i.e.\ $\sx = \topk(\sx)$, but it does affect the Jacobian and, in particular, allows us to compute it much more efficiently (Section~\ref{sec:jacobian_tractable}).
%That raises the question of why we apply $\phi_k(\sx)$ first.
%The answer is that while $\phi_k(\sx)$ does not modify any of the $\sx$ that we use in practice, it does modify the Jacobian.
%Specifically, the 
%we use a TopK nonlinearity in , then $\sx$ only has $k$ non-zero elements, as 
%The inputs to $f_s$ are sparse vectors, $\sx$ that arise from applying $\encx$ to $\x$.
%This is the identity function our k-sparse SAEs, assuming that the input, $\sx$, arises from $\encx$, as $\encx$ has a TopK nonlinearity, so only returns vectors with at most $k$ nonzero elements.
%Thus, while $\phi_k$ 
%In order to make later mathematical derivations easier, we set this function's inputs to be the pre-activation-function latent activations, in other words $f_s=e^\post\circ f\circ d^\pre\circ\phi^\pre$.
%\todo{explain what $\phi$ is properly}

At a high level, we want the function $f_s$ to be `sparse', in the sense that each of its input dimensions (i.e. SAE latent activations) only affects a small number of its output dimensions, and each of its output dimensions only depends on a small number of its input dimensions.
We quantify the sparsity of $f_s$ in terms of its Jacobian matrix.
The Jacobian of $f_s$ is, in index notation:
\begin{align}
  J_{f_s,i,j} &= \frac{\partial f_s(s_{\text{x},i})}{\partial s_{\text{x},j}}.
\end{align}
Intuitively, we can consider maximizing the sparsity of the Jacobian as minimizing the number of edges in the computational graph connecting the input and output nodes (Figure~\ref{fig:schematic}), i.e.\ maximizing the number of near-zero elements in the Jacobian matrix.
We note that the Jacobian is not a perfect measure of the sparsity of the computational graph, but it is an accurate proxy (see Section~\ref{sec:mostly_linear} and Appendix \ref{app:not_local}) while being computationally tractable.

We simultaneously train two separate SAEs on the input and output of a transformer MLP with the objectives of low reconstruction error and sparse relations between the separate SAE latents (via the Jacobian).
We do not need to optimize for the sparsity of the latent activations via a penalty term in the loss function because we use $k$-sparse autoencoders, which keep only the $k$ largest latent activations per token position.
Hence, our loss function is
% \begin{multline}
%     \mathcal{L} = \text{MSE}(\x,\hatx)
%     + \text{MSE}(\y,\haty) \\
%     + \lambda \frac{1}{k^2}\sum_{i=1}^{\dimsy} \sum_{j=1}^{\dimsx} \left(|\mat{J}_{f_s,i,j}^{(l)}|\right)
% \end{multline}
\begin{equation}
   \mathcal{L} = \text{MSE}(\x,\hatx)
   + \text{MSE}(\y,\haty) 
   + \frac{\lambda}{k^2}\sum_{i=1}^{\dimsy} \sum_{j=1}^{\dimsx} |J_{f_s,i,j}|
\end{equation}
Here, $k$ is the number of non-zero elements in the TopK activation function, $\dimsx,\dimsy$ are the dimensionalities of the latent spaces of the input and output SAEs, respectively, and $\lambda$ is the coefficient of the Jacobian loss term.
We divide by $k^2$ because, as we will see later, there are at most $k^2$ non-zero elements in the Jacobian.
Finally, note that if we set $\lambda=0$, then our objective effectively trains traditional SAEs for each of $\x$ and $\y$ independently.
%Note that because both the pre-MLP SAE and the post-MLP SAE are needed for computing the Jacobian loss term (and are both subsequently optimized by it), we must train both SAEs simultaneously as if they were a single model.

\subsection{Making the Jacobian calculation tractable}
\label{sec:jacobian_tractable}

% Figure 2: Jacobian heatmap...
% But the Jacobian heatmaps are super oversimplified because there's a coefficient we can sweep!
\begin{figure}[t]
    \centering
    \includegraphics{figures/jac_sparsity_410m.pdf}
    \caption{
    JSAEs induce a much greater degree of sparsity in the elements of the Jacobian of $f_s$ than traditional SAEs.
    % (a) A representative example of Jacobians with traditional SAEs vs JSAEs. For each column, the heatmaps represent the same token position and prompt. Note that we only show the $k\times k$ slice of the Jacobian corresponding to active SAE latents; all other elements of the Jacobian are guaranteed to be zero. % because of the TopK activation functions of the input and output SAEs.
    % All examples are L2 normalized, see Figure~\ref{fig:jac_examples_normed} for the unnormalized and L1 normalized versions.
    % \red{should we swap this for the un-normalized version again?}
    % (b) 
    The bars show the average proportion of Jacobian elements with absolute values above certain thresholds. At most $k\times k$ elements can be nonzero, so we take 100\% on the y-axis to mean $k\times k$.
    The average was taken across 10 million tokens.
    This example is from layer 15 of Pythia-410m. For layer 3 of Pythia-70m and layer 7 of Pythia-160m, see Figure~\ref{fig:jac_hist_multiple_models}, for more quantitative information on Jacobian sparsity across model sizes, layers, and hyperparameters see Figures~\ref{fig:expansion_factor}, \ref{fig:k}, and \ref{fig:jacobian_coef_70m}.
    We present further discussion of the sparsity of the Jacobian in Appendix~\ref{app:jac_sparsity}.
    }
    \label{fig:jac_sparsity}
\end{figure}

Computing the Jacobian naively (e.g., using an automatic differentiation package) is computationally intractable, as the full Jacobian has size $B\times \dimsy\times \dimsx$ where $B$ is the number of tokens in a training batch $\dimsx$ is the number of SAE latents for the input, and $\dimsy$ is the number of SAE latents for the output.
Unfortunately, typical values are around $1,000$ for $B$ and around $32,000$ for $\dimsx$ and $\dimsy$ (taking as an example a model dimension of $1,000$ and an expansion factor of 32).
Combined, this gives a Jacobian with around 1 trillion elements.
This is obviously far too large to work with in practice, and our key technical contribution is to develop an efficient approach to working with this huge Jacobian.

Our first insight is that for each element of the batch, we have a $\dimsy \times \dimsx$ Jacobian, where $\dimsx$ and $\dimsy$ are around $32,000$.
This is obviously far too large.
However, remember that we are interested in the Jacobian of $f_s$, so the input is the sparse SAE latent vector, $\sx$ and the output is the sparse SAE latent vector, $\sy$.
Importantly, as we are using $k$-sparse SAEs, only $k$ elements of the input and output are ``on'' for any given token.
As such, we really only care about the $k \times k$ elements of the Jacobian of $f_s$, corresponding to the inputs and outputs that are ``on''.
This reduces the size of the Jacobian by around six orders of magnitude, and renders the computation tractable.
However, to make this work formally, we need all elements of the Jacobian corresponding to ``off'' elements of the input and output to be zero.
This is where the $\topk$ in the definition of $f_s$ becomes important.
Specifically, the $\topk$ ensures that the gradient of $f_s$ wrt any of the inputs that are ``off'' is zero.
Without $\topk$, the Jacobian could be non-zero for any of the inputs, even if changing those inputs would not make sense, as it would give more than $k$ elements being ``on'' in the input, and thus could not be produced by the k-sparse SAE.

% Our second insight was that while we could use autodiff to compute the Jacobian, this would typically still be relatively inefficient (e.g.\ requiring $k$ backward passes).
% Instead, for standard GPT-2-style MLPs, we noticed that an extremely efficient Jacobian formula can be derived by hand, consisting of only 3 matrix multiplications (along with a few pointwise operations).
% Details are presented in Appendix \ref{app:jac-comp}.

Our second insight was that computing the Jacobian by automatic differentiation would still be relatively inefficient, e.g., requiring $k$ backward passes.
Instead, for standard GPT-2-style MLPs, we noticed that an extremely efficient Jacobian formula can be derived by hand, requiring only three matrix multiplications and along with a few pointwise operations.
We present this derivation in Appendix~\ref{app:jac-comp}.



%To address this, we took several optimization steps in order to make this computation tractable as part of our loss function:
%\begin{enumerate}
%    \item \textbf{All Jacobian elements corresponding to non-active SAE latents must be zero} -- this is because both the very first and the very last steps when evaluating $f_s$ is an activation function whose derivative at zero is always equal to zero. We can therefore only consider the sliced Jacobian where we only include the rows and columns corresponding to SAE latents which are active at the specific token position.
%    \item \textbf{The TopK activation function guarantees a square sliced Jacobian of size $k\times k$} -- this enables more efficient batching of the Jacobians from different training samples, making the size of the relevant portion of the Jacobian $B\times k\times k$ rather than $B\times n \times n$. Since $k$ is typically 10–100 while $n$ is typically in the tens of thousands, this reduces the size of the Jacobian by about 6 orders of magnitude.
%    \item \textbf{Minimizing the number of matmul operations} -- because we know the exact structure of $f_s$, we can derive its Jacobian mathematically and use this to minimize the amount of computation necessary. Specifically, for a standard GPT-2-style MLP, we only need 3 matrix multiplications to compute the Jacobian. Details are presented in Appendix \ref{app:jac-comp}.
%\end{enumerate}

With these optimizations in place, training a pair of JSAEs takes about twice as long as training a single standard SAE.
% \footnote{
We measured this by training ten of each model on Pythia-70m with an expansion factor of 32 for 100 million tokens on an RTX 3090. The average training durations were 72mins for a pair of JSAEs and 33 mins for a traditional SAE, with standard deviations below 30 seconds for both.
% }
%\todo{also talk about the fact that x post went through the original function rather than the SAEs reconstructed thingy; this means that setting lambda to 0 gives us two standard SAEs}

\section{Results}

Our experiments were performed on LLMs from the Pythia suite \citep{biderman_pythia_2023}, the figures in the main text contain results from Pythia-410m unless otherwise specified.
We trained on 300 million tokens with $k=32$ and an expansion factor of $64$ for Pythia-410m and $32$ for smaller models.
We reproduced all our experiments on multiple models and found the same qualitative results (see Appendix~\ref{app:evaluation}).

\subsection{Jacobian sparsity, reconstruction quality, and auto-interpretability scores}\label{sec:jac_sparsity}

% Figure 3: Jacobian loss coefficient vs. stuff. But as you sweep the coefficient, SAEs get worse as Jacobian sparsity gets better. So, to get good Jacobians, do we need terrible SAEs?  No!  There is a sweet spot where we get good SAEs and good Jacobians!
\begin{figure*}
    \centering
    \includegraphics{figures/jacobian_coef_160m.pdf}
    \caption{
        Reconstruction quality, model performance preservation, and sparsity metrics against the Jacobian loss coefficient.
        JSAEs trained on layer 7 of Pythia-160m with expansion factor $64$ and $k=32$; see Figure~\ref{fig:jacobian_coef_70m} for layer 3 of Pythia-70m.
        Recall that the maximum number of non-zero Jacobian values is $k^2=1024$.
        In accordance with Figure~\ref{fig:tradeoff_reconstr_sparsity_410m}, all evaluation metrics degrade for values of the coefficient above 1.
        See Appendix~\ref{app:evaluation} for details of the evaluation metrics.
    }
    \label{fig:jacobian_coef_160m}
\end{figure*}

First, we compared the Jacobian sparsity for standard SAEs and JSAEs.
Note that, unlike with SAE latent activations, there is no mechanism for producing exact zeros in the Jacobian elements corresponding to active latents.
Hence, we consider the number of near-zero elements rather than the number of exact zeros.
% We plotted the $k \times k$ elements of the Jacobian corresponding to the ``on'' elements in each of the input and output SAEs (Figure~\ref{fig:jac_sparsity}a) for very small samples of four tokens.
% We found that traditional SAEs had seemingly dense Jacobians (Figure~\ref{fig:jac_sparsity}a; top), with many elements with relatively large absolute values.
% In contrast, JSAEs had apparently sparser Jacobians, with most elements close to zero and only a few large values (Figure~\ref{fig:jac_sparsity}a; bottom).
% While Figure~\ref{fig:jac_sparsity}a suggests that the sparsity might be higher in the JSAE setting, it is hardly conclusive.
To quantify the difference in sparsity between the two, we looked at the proportion of the elements of the Jacobian above a particular threshold when aggregating over 10 million tokens (Figure~\ref{fig:jac_sparsity}).
Here, we found that JSAEs dramatically reduced the number of large elements of the Jacobian relative to traditional SAEs.

Importantly, the degree of sparsity depends on our choice of the coefficient $\lambda$ of the Jacobian loss term.
Therefore, we trained multiple JSAEs with different values of this parameter.
As we might expect, for small values of $\lambda$, i.e., little incentive to sparsify the Jacobian, the input and output SAEs perform similarly to standard SAEs (Figure~\ref{fig:jacobian_coef_160m} blue lines), including in terms of the variance explained by the reconstructed activation vectors and the increase in the cross-entropy loss when the input activations are replaced by their reconstructions.
Unsurprisingly, as $\lambda$ grows larger and the Jacobian loss term starts to dominate, our evaluation metrics degrade.
Interestingly, this degradation happens almost entirely in the output SAE rather than the input SAE --- we leave it to future work to investigate this phenomenon further.
% SAE trained on the MLP's input appears to be "fixed", in the sense that changing the value of $\lambda$ does not appear to effect it (Figure~\ref{fig:jacobian_coef_160m}).
% We refrain from speculating as to what may be causing this phenomenon and leave it to future work to investigate.
% For large values of $\lambda$, we found that the Jacobians were very sparse (Figure~\ref{fig:jacobian_coef_160m} bottom right), while as $\lambda$ decreased, the Jacobian became far denser.


Critically, Figure~\ref{fig:jacobian_coef_160m} suggests there is a `sweet spot' of the $\lambda$ hyperparameter where the SAE quality metrics remain reasonable, but the Jacobian is much sparser than for standard SAEs.
% \onlyarxiv{
% Based on data seen in Figure~\ref{}, we speculate that the location of the sweet spot is a function of $k$ and the dimensionality of the SAE's inputs $\x,\y$, specifically that the general sweet spot is $\lambda=\frac{k^2}{2\dimx}$ where $\dimx$ is the dimensionality of the residual stream, but we leave it to future work to investigate this further.
% \todo{TODO(Lucy): do sweeps, justify that equation}}
To further investigate this trade-off, we plotted a measure of Jacobian sparsity (the proportion of elements of the Jacobian above 0.01) against the average cross-entropy (Figures ~\ref{fig:jacobian_coef_160m}, \ref{fig:tradeoff_reconstr_sparsity_410m}, and \ref{fig:tradeoff_reconstr_sparsity_70m}).
We found that there is indeed a sweet spot where the average cross-entropy is only slightly worse than a traditional SAE, while the Jacobian is far sparser.
For Pythia 410m (Figure~\ref{fig:tradeoff_reconstr_sparsity_410m}) this value is around $\lambda=0.5$, whereas for Pythia-70m, it is around $\lambda=1$ (Figure~\ref{fig:tradeoff_reconstr_sparsity_70m}).
We choose this value of the Jacobian coefficient (i.e.\ $\lambda=0.5$ for Pythia-410m in the main text, and $\lambda=1$ for Pythia-160m in the Appendix) in other experiments.

% Figure 4: A bunch of Pareto fronts (e.g., you would have a grid with a couple of measures of sparsity for the rows and three or four measures of SAE sparsity in the columns).
\begin{figure}[t]
    \centering
    \includegraphics[width=\linewidth]{figures/tradeoff_reconst_jac_410m.pdf}
    \caption{The trade-off between reconstruction quality and Jacobian sparsity as we vary the Jacobian loss coefficient. Each dot represents a pair of JSAEs trained with a specific Jacobian coefficient.
    % Reconstruction quality here is measured by the CE loss score, which is 1 when the reconstruction is perfect and 0 when the reconstruction degrades the cross-entropy loss equally to ablating the activations (setting them to zero). The Jacobian sparsity is measured by the number of elements whose absolute value is at least 0.01.
    The value of $\lambda$ is included for some points.
    We can see that a coefficient of roughly $\lambda=0.5$ is optimal for Pythia-410m with $k=32$.
    % \red{swap for 160m; note: computational reasons, can't do 410}
    Note that the CE loss score is the average of the CE loss scores of the pre-MLP JSAE and the post-mlp JSAE.
    Measured on layer~15 of Pythia-410m, similar charts with a wider range of models and metrics can be found in Figures \ref{fig:pareto_70m}, \ref{fig:pareto_160m}, and \ref{fig:tradeoff_reconstr_sparsity_70m}.}
    \label{fig:tradeoff_reconstr_sparsity_410m}
\end{figure}

We also measure the interpretability of JSAE latents using the automatic interpretability pipeline developed by \citet{paulo_automatically_2024} and compare this to traditional SAEs.
We find that JSAEs achieve similar interpretability scores
% In fact, JSAEs seem to slightly outperform traditional SAEs, though there is too much noise in the data to conclude that this increase is statistically significant
(Figure~\ref{fig:autointerp_410m}).

Lastly, we attempted to interpret the pairs of JSAE latents corresponsing to the largest Jacobian elements by "max-activating" examples.
Though the pairs were generally interpretable, we believe that the problem of interpreting these pairs properly is very subtle and complex (see Appendix \ref{sec:pairs_max_act}) and leave it to future work to investigate this further.

\begin{figure}[t]
    \centering
    \includegraphics{figures/autointerp_410m.pdf}
    \caption{Automatic interpretability scores of JSAEs are very similar to traditional SAEs. Measured on all odd-numbered layers of Pythia-410m using the ``fuzzing'' scorer from \citet{paulo_automatically_2024}. For all layers of Pythia-70m see Figure~\ref{fig:autointerp_70m}.}
    \label{fig:autointerp_410m}
\end{figure}

\subsection{Performance on re-initialized transformers}\label{sec:random}~
% \todo{edit the wording in this section, I don't want more dead salmon blood on my hands}
To confirm that JSAEs are extracting information about the complex learned computation, we considered a form of control analysis inspired by \citet{heap_sparse_2025}.
Specifically, we would expect that trained transformers have carefully learned specific, structured computations while randomly initialized transformers do not.
Thus, a possible desideratum for tools in mechanistic interpretability is that they ought to work substantially better when analyzing the complex computations in trained LLMs than when applied to LLMs with randomly re-initialized weights.
% \onlyarxiv{\todo{explain this more thoroughly, namely that the point is that SAE latent sparsity fails the desideratum of being linked to learned computation while Jacobian sparsity doesn't}}
%\onlyicml{\red{have a think about how to better pitch this for icml}}
%\citet{heap_sparse_2025} found that the auto-interpretability scores were comparable when applying SAEs to trained and randomly initialized transformers.
% Whatever conclusions one draws from \citet{heap_sparse_2025}, it is clear that if our method purports to understand the complex computations learned in trained transformers, then it should work better in trained than randomly initialized transformers.
This is precisely what we find.
Specifically, we find that the Jacobians for trained networks are always substantially sparser than the corresponding random trained network, and this holds for both traditional SAEs and JSAEs (Figure~\ref{fig:randomized_410m}).
Further, the relative improvement in sparsity from the traditional SAE to the JSAE is much larger for trained than random LLMs, again indicating that JSAEs are extracting structure that only exists in the trained network.
Note that we also see that for traditional SAEs, there is a somewhat more sparse Jacobian for the trained than randomly initialized transformer.
This makes sense: we would hope that the traditional SAE basis is somewhat more aligned with the computation (as expressed by a sparse Jacobian) than we would expect by chance.
However, it turns out that without a ``helping hand'' from the Jacobian sparsity term, the alignment in a traditional SAE is relatively small.
% Thus, Jacobian sparsity is a property that results at least to some extent from training the underlying model and thus may be a more useful optimization target than activation sparsity alone when trying to discover computational structure.
Thus, Jacobian sparsity is a property related to the complex computations LLMs learn during training, which should make it substantially useful for discovering the learned structures of LLMs.

\begin{figure}[t]
    \centering
    \includegraphics{figures/randomized_410m.pdf}
    \caption{
    Jacobians are substantially more sparse in pre-trained LLMs than in randomly initialzied transformers.
    This holds both when you actively optimize for Jacobian sparsity with JSAEs, and when you don't optimize for it and use traditional SAEs.
    The figure shows the proportion of Jacobian elements with absolute values above certain thresholds. At most $k^2$ elements can be nonzero, we therefore take $k^2$ to be 100\% on the y-axis.
    Jacobians are significantly more sparse in pre-trained transformers than in randomly re-initialized transformers.
    This shows that Jacobian sparsity is, at least to some extent, connected to the structures that LLMs learn during training.
    This stands in contrast to recent work by \citet{heap_sparse_2025} showing that traditional SAEs achieve roughly equal auto-interpretability scores on randomly initialized transformers as they do on pre-trained LLMs.
    Measured on layer 15 of Pythia-410m, for layer 3 of Pythia-70m see Figure~\ref{fig:randomized_70m}.
    % In essence, sparsity of SAE latents does not appear to detect learned computation while Jacobian sparsity does.
    % Measured on layer~3 of Pythia-70m.
    Averaged across 10 million tokens.}
    \label{fig:randomized_410m}
\end{figure}

\subsection{$f_s$ is mostly linear}\label{sec:mostly_linear}

\begin{figure*}[t]
    \centering
    \includegraphics{figures/mostly_linear_410m.pdf}
    \caption{The function $f_s$, which combines the decoder of the first SAE, the MLP, and the encoder of the second SAE, is mostly linear.
    Specifically, the vast majority of scalar functions going from $\sxj$ to $\syi$ are linear.
    (a)~Examples of linear, JumpReLU, and other functions relating individual input SAE latents and output SAE latents. See Figure~\ref{fig:func_examples} for more examples.
    (b)~For the empirically observed $\sx$ and randomly selected $i,j$ (of those corresponding to active SAE latents), the vast majority of scalar functions from $\sxj$ to $\syi$ are linear. For details see Appendix~\ref{app:not_local}.
    The proportion of linear function also noticeably increases with JSAEs compared to traditional SAEs, meaning that JSAEs induce additional linearity in $f_s$.
    % To see how these proportions change with different model sizes and hyperparameters, see Figures~\ref{fig:linear_layer}, \ref{fig:linear_jacobian_coef}, and \ref{fig:linear_expansion_factor_k}.
    (c)~Because the vast majority of functions are linear, the Jacobian usually precisely predicts the change observed in the output SAE latent when we make a large change to the input SAE latent's value (namely subtracting 1, note that the empirical median value of $\sxj$ is $2.5$).
    Each dot corresponds to an $(\sxj,\syi)$ pair.
    For 97.7\% of pairs (across a sample size of 10 million) their Jacobian value nearly exactly predicts the change we see in the output SAE latent when making large changes to the input SAE latent's activation, i.e. $|\Delta \syi|\approx|J_{{f_s},ij}|$.
    The scatter plot shows a randomly selected subset of 1,000 $(\sxj,\syi)$ pairs. 
    For further details see Appendix~\ref{app:not_local}.
    Measured on layer 15 of Pythia-410m, for layer 3 of Pythia-70m see Figure~\ref{fig:mostly_linear_70m}, for the linearity results on other models and hyperparameters see Figures~\ref{fig:linear_layer}, \ref{fig:linear_jacobian_coef}, and \ref{fig:linear_expansion_factor_k}.
    }
    \label{fig:mostly_linear_410m}
\end{figure*}

Importantly, the Jacobian is a local measure.
Thus, strictly speaking, a near-zero element of the Jacobian matrix implies only that a small change to the input SAE latent does not affect the corresponding output SAE latent.
It may, however, still be the case that a large change to the input SAE latent would change the output SAE latent.
We investigated this question and found that $f_s$ is usually approximately linear in a wide range and is often close to linear.
Specifically, of the scalar functions relating individual input SAE latents $\sxj$ to individual output SAE latents $\syi$, the vast majority are linear (Figure~\ref{fig:mostly_linear_410m}b).
This is important because, for any linear function, its local slope is completely predictive of its global shape, and therefore, a near-zero Jacobian element implies a near-zero causal relationship.
For the scalar functions which are not linear, we frequently observed they have a JumpReLU structure\footnote{By JumpReLU, we mean any function of the form $f(x)=a\text{JumpReLU}(bx+c)$. Recall that $\text{JumpReLU}(x)=x$ if $x>d$ and $0$ otherwise. $a,b,c,d \in\Reals$ are constants.} \citep{erichson2019jumpreluretrofitdefensestrategy}.
Notably, a JumpReLU is linear in a subset of its input space, so even for these scalar functions the first derivative is still an accurate measure within some range of $\sxj$ values.
It is also worth noting that with JSAEs, the proportion of linear functions is noticeably higher than with traditional SAEs, so at least to a certain extent, JSAEs induce additional linearity in the MLP.
To confirm these results, we plotted the Jacobian against the change of output SAE latent $\syi$ as we change the input SAE latent $\sxj$ by subtracting $1$ (Figure~\ref{fig:mostly_linear_410m}c)\footnote{For reference, the median value of $\sxj$ without any interventions is $2.5$.}.
We found that 97.7\% of the time, $|\Delta \syi|\approx|J_{{f_s},ij}|$.
For details see Appendix~\ref{app:not_local}.
% \todo{talk about which model this is with (which will change before the deadline)}
% For sparsity of the Jacobian to accurately represent computational sparsity of the kind depicted in Figure~\ref{fig:schematic}, we required these functions to be approximately linear, i.e., to be able to determine the behavior of the function over a wide range of inputs from only its gradient at a given point.

% Finally, we investigated the degree to which Jacobians accurately capture computational sparsity.
% Some readers may object that a near-zero partial derivative merely implies that a very small change to the input SAE's latent will not affect the output SAE's latent, but that a larger change may in fact change the output SAE's latent substantially.
% We investigated this question and found that $f_s$ is approximately linear.
% This is important because for any linear function, its local slope is completely predictive of its global shape.
% As it turns out, this is typically not the case -- the absolute value of the Jacobian element usually predicts the change observed in the output SAE's latent when the input SAE's latent is modified by a large amount (in our case by subtracting 1 from it, note that SAE latent activations are usually well below 5).
%This can be seen in Figure \ref{fig:not_local}.



%This is because we found that $f_s$ is approximately linear.
% Specifically, we collected activation vectors from the underlying language model over a sample of tokens, encoded them by the first (input) SAE, and then varied the $j$-th input latent over a relatively wide range to record the resulting change in the $i$-th output latent.
% By inspecting the $i$-th output latent against the $j$-th input latent (Appendix~\ref{app:not_local}), we were able to confirm that the majority of samples behaved very close to linearly (Figure~\ref{fig:mostly_linear_410m}).


% Then we plotted $i$-th output latent as a function of the $j$-th input latent (see Appendix~\ref{app:not_local} for details).
% We found that many of the resulting functions are linear (Figure~\ref{fig:mostly_linear_410m}a), while some appear to have a ``jump-relu'' like structure, and a few appear to be curved.
% Importantly, if the function is linear, then its global structure is entirely described by the slope (Jacobian) and intercept, i.e.\ its local properties.
% As such, if we have a linear function, with a Jacobian zero, then the input does not affect the output.
% Importantly, for the vast majority of LLM activations and $(i,j)$ pairs, the function is linear (Figure~\ref{fig:mostly_linear_410m}).

% \onlyarxiv{
% \subsection{Interpreting SAE latents by examples}\label{sec:pairs_max_act}
% \todo{write this up; add some screenshots to the appendix}
% }



% \section{Results}

% Overall, we find that JSAEs identify sparse bases for the input and output MLP activations that produce much sparser Jacobian matrices than between a pair of traditional SAEs (see Section \ref{sec:jac_sparsity}).
% We can achieve this with JSAEs without sacrificing on reconstruction quality or interpretability of latents, both of which are roughly equal to traditional SAEs (Section \ref{sec:on_par}).
% We also find that the latent pairs
% % (i.e. a latent of the input SAE and a latent of the output SAE)
% which are found by JSAEs tend to be interpretable using maximally activating examples (Section \ref{sec:pairs_max_act}).
% trained LLMs than on randomly re-initialized ones, which indicates both that there is indeed computational structure in LLMs and that JSAEs are better-suited to discovering it than standard SAEs (Section \ref{sec:random}).
% Finally, we investigate the accuracy of defining computational structure in terms of the Jacobian by investigating the linearity of the overall function $f_s$, finding it to be close to linear in the vast majority of instances (Section \ref{sec:not_local}).

% Our experiments were performed with the Pythia suite of models \citep{biderman_pythia_2023}, training details (including hyperparameters) can be found in Appendix \ref{app:training}, and evaluation details can be found in Appendix \ref{app:evaluation}.

% \todo{talk about which model to use, which layers, etc.}

% \subsection{Jacobian sparsity}\label{sec:jac_sparsity}

% The first result is that JSAEs do in fact find a solution (i.e. a pair of overcomplete bases) which has a highly sparse Jacobian, while also minimizing MSE loss and having at most $k$ latents active at a time.
% It is of course also possible to compute $\vec{J}_{f_s}$ with a pair of traditional SAEs trained on $\x$ and $\y$.
% This gives us a natural point of comparison, and we use this to show that the Jacobian of JSAEs is significantly more sparse than the Jacobian of traditional SAEs (see Figure \ref{fig:jac_sparsity}).

% CHART: Same hist as above but for averaged -- here we can't really see a difference between traditional SAEs and JSAEs with my setup

% Naturally, this changes as we vary the coefficient of the Jacobian loss term $\lambda$.
% While at extreme values of $\lambda$ there is a strong trade-off between Jacobian sparsity and other SAE desiderata (Figure \ref{fig:jacobian_coef_160m}), we show that there exists a `sweet spot' around $\lambda=1$ where both reconstruction quality and Jacobian sparsity are approximately ideal (Figure \ref{fig:tradeoff_reconstr_sparsity_410m}).
% Note that we do not need to measure the sparsity (e.g. $L^0$) of SAE latents since it is constant due to the TopK activation function.

% \subsection{Reconstruction and latent interpretability are approximately on par with traditional SAEs}\label{sec:on_par}
% The reconstruction quality of JSAEs is comparable to traditional SAEs when the Jacobian loss coefficient $\lambda=1$.
% The reconstruction quality of the first SAE (i.e. the one operating on the inputs to the MLP) is indistinguishable between JSAEs and traditional SAEs (e.g. CE loss score 0.91 vs 0.91\footnote{\label{foot:reconstr}Examples measured on layer 3 of Pythia-70m with an expansion factor of 32. A thorough exploration of the reconstruction quality for different LLMs and JSAE hyperparameters is presented in Figures \ref{fig:jacobian_coef_160m}, \ref{fig:tradeoff_reconstr_sparsity_410m}, \ref{fig:reconstruction_quality}, \ref{fig:ce_loss_score}, \ref{fig:expansion_factor}, \ref{fig:k}, \ref{fig:jacobian_coef_70m}, \ref{fig:pareto_70m}, and \ref{fig:pareto_160m}.}).
% The second SAE (i.e. the one operating on the MLP's outputs) does see a slight decrease in reconstruction quality (e.g. CE loss score 0.92 vs 0.88\textsuperscript{\ref{foot:reconstr}}), but the decrease is not significant.
% See Figures \ref{fig:jacobian_coef_160m} and \ref{fig:tradeoff_reconstr_sparsity_410m} for more detail.

% TABLE: reconstruction and autointerp scores of JSAEs vs SAEs (I still need to pull out the numbers for those across a large sample size)

% \FloatBarrier
% \todo{consider adding a float barrier so that people can follow the argument more easily}
% \subsection{Performance on random transformers}\label{sec:random}

% A natural desideratum for tools in mechanistic interpretability is that they ought to work substantially better on trained LLMs than on LLMs with randomly re-initialized weights.
% Traditional SAEs fail to meet this desideratum.
% They work roughly equally well on randomly re-initialized LLMs as they do on trained LLMs, both in terms of reconstruction quality and in terms of having interpretable latents \citep{heap_sparse_2025}.
% This indicates that, while they may help us in locating "concepts" in the activations, they do not tell us much about the structure of the learned computation itself.
% By comparison, JSAEs find solutions with substantially more Jacobian sparsity in trained LLMs than in randomly re-initialized LLMs (see Figure \ref{fig:randomized}).
% This means that, unlike sparsity of SAE latents, Jacobian sparsity is a property which results at least to some extent from the model's training, and therefore may be a more useful optimization target when trying to discover computational structures.

% \subsection{$f_s$ is mostly linear}\label{sec:not_local}
% Some readers may object that a near-zero partial derivative merely implies that a very small change to the input SAE's latent will not affect the output SAE's latent, but that a larger change may in fact change the output SAE's latent substantially.
% As it turns out, this is typically not the case -- the absolute value of the Jacobian element usually predicts the change observed in the output SAE's latent when the input SAE's latent is modified by a large amount (in our case by subtracting 1 from it, note that SAE latent activations are usually well below 5).
%This can be seen in Figure \ref{fig:not_local}.

%
%This is because the function $f_s$ is mostly linear.
%Consider the scalar function $\fsij:\Reals\to\Reals$ which takes as input the $i$-th latent activation of the first SAE (i.e. $\vec{s}_{\text{x},i}$) and returns as output the $j$-th latent activation of the second SAE (i.e. $\vec{s}_{\text{y},j}$), while keeping the other elements of the input vector fixed at the same values as $\sx$.
%In other words, this function captures the relationship between the $i$-th input SAE latent and the $j$-th output SAE latent in the context of $\sx$.
%Geometrically, we start off at the point $\sx$, and we move from it through the input spaces in parallel to the $i$-th basis vector, and then we observe how the output of $f_s$ projects onto the $j$-th basis vector.
%Formally,
%\begin{align}
%    \fsij(x)&=f_s\left(\psi(\sx,i,x)\right)[j] \label{eq:def_fsij}\\
%    \psi(\sx,i,x)[k]&=\begin{cases}
%        x&\text{if }i=k\\
%        \sx[k]&\text{otherwise}
%    \end{cases}
%\end{align}
%where square brackets denote indexing in a vector, e.g. $\x[i]$ is the $i$-th element of $\x$.

% The key observation is that for the vast majority of empirically observed $\sx$ and randomly chosen $i,j$ (out of the ones corresponding to active SAE latents), the function $\fsij$ is linear.
% This is important because for a linear function, the derivative (and therefore the Jacobian) perfectly captures the relationship between the two figures across the entire domain.
% In the cases where it isn't linear, it is nearly always a JumpReLU, which is still linear in a subset of its domain.
% See Figure \ref{fig:mostly_linear_410m} for the distribution and Appendix \ref{app:not_local} for details.

\onlyarxiv{
\section{Discussion}
\label{sec:discussion}
% \todo{maybe write up an appendix with the theory behind this approach (TLDR: if mech interp isn't doomed then computational sparsity must exist + initial evidence from circuit-style analysis)}
We believe JSAEs are a promising approach for discovering computational sparsity and understanding the reasoning of LLMs.
We would also argue that an approach like the one we introduced is in some sense necessary if we want to `reverse-engineer' or `decompile' LLMs into readable source code.
It is not enough that our variables (e.g., SAE features) are interpretable; they must also be connected in a relatively sparse way.
To illustrate this point, imagine a Python function that takes as input 5 arguments and returns a single variable, and compare this to a Python function that takes 32,000 arguments.
Naturally, the latter would be nearly impossible to reason about.
Discovering computational sparsity thus appears to be a prerequisite for solving interpretability.
It is also important that the mechanisms for discovering computational sparsity be fully unsupervised rather than requiring the user to manually specify the task being analyzed.
There are existing methods for taking a specific task and finding the circuit responsible for implementing it, but these require the user to specify the task first (e.g. as a small dataset of task-relevant prompts and a metric of success).
They are thus `supervised' in the sense that they need a clear direction from the user.
Naturally, it is not feasible to manually iterate over all tasks an LLM may be performing, so a fully unsupervised approach is needed.
JSAEs are the first step in this direction.

Naturally, JSAEs in their current form still have important limitations.
They currently only work on MLPs, and for now, they only operate on a single layer at a time rather than discovering circuits throughout the entire model.
Our initial implementation also works on GPT-2-style MLPs, while most LLMs from the last few years tend to use GLUs \cite{dauphin2017languagemodelinggatedconvolutional,shazeer2020gluvariantsimprovetransformer}, though we expect it to be fairly easy to extend our setup to GLUs.
Additionally, our current implementation relies on the TopK activation function for efficient batching; TopK SAEs can sometimes encourage high-density features, so it may be desirable to generalize our implementation to work with other activation functions.
These are, however, problems that can be addressed relatively straightforwardly in future work, and we would welcome correspondence from researchers interested in addressing them.

A pessimist may argue that partial derivatives (and, therefore, Jacobians) are merely local measures.
A small partial derivative tells you that if you slightly tweak the input latent's activation, you will see no change to the output latent's activation, but it may well be the case that a large change to the input latent's activation will lead to a large change in the output latent.
Thankfully, at least in MLPs, this is not quite the case.
As we show in Section \ref{sec:mostly_linear}, $f_s$ is approximately linear, and the size of the elements of the Jacobian nearly perfectly predicts the change you see in the output latent when you make a large change to the input latent.
For a linear function, a first-order derivative at any point is perfectly predictive of the relationship between the input and the output, and thus, at least for the fraction of $f_s$ that is linear, Jacobians perfectly measure the computational relationship between input and output variables.
We further discuss this in Appendix \ref{app:not_local}.
Additionally, as we showed in Section \ref{sec:random}, Jacobian sparsity is much more present in trained LLMs than in randomly initialized ones, which indicates that it does correspond in some way to structures that were learned during training.
At a high level, a sparse computational graph necessarily implies a sparse Jacobian, but a sparse Jacobian does not in and of itself imply a sparse computational graph.
But all of these results make it seem likely that Jacobian sparsity is a good approximation of computational sparsity, and when combined with the fact that we have now developed efficient ways of computing them at scale, this leads us to believe that JSAEs are a highly useful approach.
We would, however, still invite future work to further investigate the degree to which Jacobians, and by extension JSAEs, capture the structure we care about when analyzing LLMs.

% \todo{talk about problems w transcoders that we solve}
}

\section{Conclusion}

We introduced Jacobian sparse autoencoders (JSAEs), a new approach for discovering sparse computation in LLMs in a fully unsupervised way.
We found that JSAEs induce sparsity in the Jacobian matrix of the function that represents an MLP layer in the sparse basis found by JSAEs, with minimal degradation in the reconstruction quality and downstream performance of the underlying model and no degradation in the interpretability of latents.
We also found that Jacobian sparsity is substantially greater in pre-trained LLMs than in randomly initialized ones suggesting that Jacobian sparsity is indeed a proxy for learned computational structure.
Lastly, we found that Jacobians are a highly accurate measure of computational sparsity due to the fact that the MLP in the JSAE basis consists mostly of linear functions relating input to output JSAE latents.

% TODO for arxiv version
% \todo{Lucy and Callum both think it would be great to discuss directions for future work and high-level goals}

% Acknowledgements should only appear in the accepted version.
\onlyarxiv{
\section*{Acknowledgements}
The authors wish to thank Callum McDougall and Euan Ong for helpful discussions.
% \todo{continuously update this list}
We also thank the contributors to the open-source mechanistic interpretability tooling ecosystem, in particular the authors of SAELens \citep{bloom_jbloomaus_2023}, which formed the backbone of our codebase.
The authors wish to acknowledge and thank the financial support of the UK Research and Innovation (UKRI) [Grant ref EP/S022937/1] and the University of Bristol.
This work was carried out using the computational facilities of the Advanced Computing Research Centre, University of Bristol - http://www.bristol.ac.uk/acrc/.
We would like to thank Dr.~Stewart for funding for GPU resources.
}

% \textbf{Do not} include acknowledgements in the initial version of
% the paper submitted for blind review.

\section*{Impact Statement}
The work presented in this paper advances the field of mechanistic interpretability.
Our hope is that interpretability will prove beneficial in making LLMs safer and more robust in ways ranging from better detection of model misuse to editing LLMs to remove dangerous capabilities.
% Authors are \textbf{required} to include a statement of the potential 
% broader impact of their work, including its ethical aspects and future 
% societal consequences. This statement should be in an unnumbered 
% section at the end of the paper (co-located with Acknowledgements -- 
% the two may appear in either order, but both must be before References), 
% and does not count toward the paper page limit. In many cases, where 
% the ethical impacts and expected societal implications are those that 
% are well established when advancing the field of Machine Learning, 
% substantial discussion is not required, and a simple statement such 
% as the following will suffice:

% ``This paper presents work whose goal is to advance the field of 
% Machine Learning. There are many potential societal consequences 
% of our work, none which we feel must be specifically highlighted here.''

% The above statement can be used verbatim in such cases, but we 
% encourage authors to think about whether there is content which does 
% warrant further discussion, as this statement will be apparent if the 
% paper is later flagged for ethics review.
\onlyarxiv{
\section*{Author contribution statement}
Conceptualization was done by LF and LA.
Derivation of an efficient way to compute the Jacobian was done by LF and LA.
Implementation of the training codebase was done by LF.
The experiments in Jacobian sparsity, auto-interpretability, reconstruction quality, and approximate linearity of $f_s$ were done by LF.
The experiments interpreting feature pairs were done by TL.
LA and CH provided supervision and guidance throughout the project.
The text was written by LF, LA, TL, and CH.
Figures were created by LF and TL with advice from LA and CH.
% \todo{update contributions}
}

% In the unusual situation where you want a paper to appear in the
% references without citing it in the main text, use \nocite
% \nocite{langley00}

\bibliography{refs}
\bibliographystyle{icml2025}

\newpage
\appendix
\onecolumn

\section{Efficiently computing the Jacobian}\label{app:jac-comp}

A simple form for the Jacobian of the function $f_s = \ency \circ f \circ \decx \circ \topk$, which describes the action of an MLP layer $f$ in the sparse input and output bases, follows from applying the chain rule.
Note that here, the subscripts $f_s$, $e_\text{y}$, etc. denote the function in question rather than vector or matrix indices.
For the GPT-2-style MLPs that we study, the components of $f_s$ are:
% We derive the Jacobian of the sparse basis function $f_s = \ency \circ f \circ \decx \circ \topk$ using the chain rule. With a GPT-2-style MLP, our components are:

\begin{enumerate}
    \item \textbf{TopK}. This function takes sparse latents $\sx$ and outputs sparse latents $\barsx$. Importantly, $\sx=\barsx$. This step makes the backward pass of the Jacobian computation more efficient but does not affect the forward pass.
    \begin{align}
        \barsx &= \topk(\sx)
    \end{align}

    \item \textbf{Input SAE Decoder}. This function takes sparse latents $\barsx$ and outputs dense MLP inputs $\hatx$:
    \begin{align}
        \hatx &= \decx(\barsx) = \Wdecx \barsx + \bdecx
    \end{align}
    
    \item \textbf{MLP}. This function takes dense inputs $\hatx$ and outputs dense outputs $\y$:
    \begin{align}
        \vec{z} = \Win \hatx + \vec{b}_\text{1}
        \ ,\quad
        \y = \Wout \phi_{\text{MLP}}(\vec{z}) + \vec{b}_\text{2}
    \end{align}
    where $\phi_{\text{MLP}}$ is the activation function of the MLP (e.g., GeLU in the case of Pythia models).
    
    \item \textbf{Output SAE Encoder}. This function takes dense outputs $\y$ and outputs sparse latents $\sy$:
    \begin{align}
        \sy &= \ency(\y) = \topk\left( \Wency \y + \bency \right)
    \end{align}
\end{enumerate}

The Jacobian $\mat{J}_{f_s} \in \mathbb{R}^{\dimsy \times \dimsx}$ for a single input activation vector has the following elements, in index notation:
\begin{align}
    \label{eq:jacobian}
    J_{f_s,ij} &=
    %
    \frac{\partial s_{\text{y},i}}{\partial s_{\text{x},j}} =
    %
    \sum_{k \ell m n}
    \frac{\partial s_{\text{y},i}}{\partial y_k}
    \frac{\partial y_k}{\partial z_\ell}
    \frac{\partial z_\ell}{\partial \hat{x}_m}
    \frac{\partial \hat{x}_m}{\partial \bar{s}_{\text{x},n}}
    \frac{\partial \bar{s}_{\text{x},n}}{\partial s_{\text{x},j}}
\end{align}
We compute each term like so:
\begin{enumerate}
    \item \textbf{Output SAE Encoder derivative}:
    \begin{align}
        \frac{\partial s_{\text{y},i}}{\partial y_k} &=
        \tau_k'\left(
            \sum_j W_{ij}^\text{enc} y_j + b_{\text{enc},i}
        \right) \Wencyik =
        \begin{cases}
            \Wencyik & \text{if } i\in\mathcal{K}_\text{2} \\
            0        & \text{otherwise}
        \end{cases}
    \end{align}
    where $\mathcal{K}_\text{2}$ is the set of indices selected by the TopK activation function $\tau_k$ of the second (output) SAE.
    Importantly, the subscript $k$ \emph{does not} indicate the $k$-th element of $\tau_k$, whereas it \emph{does} indicate the $k$-th column of $\Wencyik$.
    
    \item \textbf{MLP derivatives}:
    \begin{align}
        \frac{\partial y_k}{\partial z_\ell} =
        W_{\text{2},k\ell} \, \phi_{\text{MLP}}'(z_\ell)
        \ ,\quad
        \frac{\partial z_\ell}{\partial \hat{x}_m} =
        W_{\text{1},\ell m}
    \end{align}
    
    \item \textbf{Input SAE Decoder derivative}:
    \begin{align}
        \frac{\partial \hat{x}_m}{\partial \bar{s}_{\text{x},n}} =
        W_{\text{x},mn}^\text{dec}
    \end{align}

    \item \textbf{TopK derivative}:
    \begin{align}
        \frac{\partial \bar{s}_{\text{x},n}}{\partial \sxj} &= \begin{cases}
            1&\text{if } j\in\mathcal{K}_\text{1}\\
            0&\text{otherwise}
        \end{cases}
    \end{align}
    where $\mathcal{K}_\text{1}$ is the set of indices (corresponding to SAE latents) that were selected by the TopK activation function $\tau_k$ of the first (input) SAE, which we explicitly included in the definition of $f_s$ above.
\end{enumerate}

When we combine all the terms:
\begin{align}
    J_{f_s,ij} &=
    \begin{cases}
        \sum_{k\ell m}
        \Wencyik \,
        W_{\text{2},k\ell} \,
        \phi_{\text{MLP}}'(z_\ell) \,
        W_{\text{1},\ell m} \,
        \Wdecxmj
        & \text{if } i\in\mathcal{K}_\text{2}\land j \in \mathcal{K}_\text{1} \\
        0 & \text{otherwise}
    \end{cases}
\end{align}

% Let $\active \subseteq \{1, \dots, \dimsy\}$ index the TopK active neurons in $\sy$. For $i \in \active$, $\phi' = 1$; else $\phi' = 0$. Thus, only rows $i \in \active$ contribute.
Let $\Wencyactive \in \mathbb{R}^{k \times \dimy}$ and $\Wdecxactive \in \mathbb{R}^{\dimx \times k}$ contain the active rows and columns, i.e., the rows and columns corresponding to the $\mathcal{K}_\text{2}$ or $\mathcal{K}_\text{1}$ indices respectively. The Jacobian then simplifies to:

\begin{align}
    \mat{J}_{f_s}^\text{(active)} &= \underbrace{\Wencyactive\W_{\text{2}}}_{\mathbb{R}^{k \times d_{\text{MLP}}}} \cdot 
    \underbrace{\phi_{\text{MLP}}'(\vec{z})}_{\mathbb{R}^{d_{\text{MLP}} \times d_{\text{MLP}}}} \cdot
    \underbrace{\W_\text{1} \Wdecxactive}_{\mathbb{R}^{d_{\text{MLP}} \times k}}
\end{align}

where $d_{\text{MLP}}$ is the hidden size of the MLP.
Note that $\mat{J}_{f_s}^\text{(active)}$ is of size $k\times k$, while the full Jacobian matrix $\mat{J}_{f_s}$ is of size $\dimsy\times\dimsx$.
However, $\mat{J}_{f_s}^\text{(active)}$ contains all the nonzero elements of $\mat{J}_{f_s}$, so it is all we need to compute the loss function to train Jacobian SAEs (Section~\ref{sec:methods_setup}).

% \subsection{Tim's version}
% We have the following equations:
% \begin{enumerate}
%     \item SAE decoder: sparse inputs $\to$ dense inputs
%     \begin{align}
%         x_{i;\alpha}^{(\dense)}(\vec{x}_{\alpha}^{(\sparse)})
%         & \triangleq b_i^{(\post)} + W_{ij}^{(\dec)} x_{j;\alpha}^{(\sparse)}
%     \end{align}
%     \item FFN: dense inputs $\to$ dense outputs
%     \begin{align}
%         z_{i;\alpha}^{(1)}(\vec{x}_{\alpha}^{(\dense)})
%         & \triangleq b_i^{(1)} + W_{ij}^{(1)} x_{j;\alpha}^{(\dense)} \\
%         %
%         z_{i;\alpha}^{(2)}(\vec{z}_\alpha^{(1)})
%         & \triangleq b_i^{(2)} + W_{ij}^{(2)} \sigma_{\FFN}(z_{j;\alpha}^{(1)})
%     \end{align}
%     \item SAE encoder: dense outputs $\to$ sparse outputs
%     \begin{align}
%         z_{i;\alpha}^{(\sparse)}(\vec{z}_\alpha^{(2)})
%         & \triangleq \sigma_{\SAE}(b_i^{(\pre)} + W_{ij}^{(\enc)}z_{j;\alpha}^{(2)})
%     \end{align}
% \end{enumerate}
% We assume that the inputs and outputs of the FFN have the same dimension, i.e., the residual stream dimension is $n^{(\dense)}$.
% We also assume that the SAEs at both layers have the same dimension, i.e., the number of SAE latents is $n^{(\sparse)}$.

% The Jacobian for a single token is, by the chain rule:
% \begin{align}
%     J_{ij}(\vec{x}_{\alpha}^{(\sparse)})
%     %
%     = \frac{\partial z_{i;\alpha}^{(\sparse)}}{\partial x_{j;\alpha}^{(\sparse)}}
%     %
%     = \frac{\partial z_{i;\alpha}^{(\sparse)}}{\partial z_{k;\alpha}^{(2)}}
%     \frac{\partial z_{k;\alpha}^{(2)}}{\partial z_{\ell;\alpha}^{(1)}}
%     \frac{\partial z_{\ell;\alpha}^{(1)}}{\partial x_{m;\alpha}^{(\dense)}}
%     \frac{\partial x_{m;\alpha}^{(\dense)}}{\partial x_{j;\alpha}^{(\sparse)}}
%     \ \in \Reals^{n^{(\sparse)} \times n^{(\sparse)}}
% \end{align}
% The four terms are:
% \begin{align}
%     \frac{\partial z_{i;\alpha}^{(\sparse)}}{\partial z_{k;\alpha}^{(2)}}
%     & = \sigma_\SAE^\prime(b_i^{(\pre)} + W_{ij}^{(\enc)}z_{j;\alpha}^{(2)}) W_{ik}^{(\enc)} \\
%     %
%     \frac{\partial z_{k;\alpha}^{(2)}}{\partial z_{\ell;\alpha}^{(1)}}
%     & = W_{k\ell}^{(2)} \sigma_\FFN^\prime(z_{\ell;\alpha}^{(1)}) \\
%     %
%     \frac{\partial z_{\ell;\alpha}^{(1)}}{\partial x_{m;\alpha}^{(\dense)}}
%     & = W_{\ell m}^{(1)} \\
%     %
%     \frac{\partial x_{m;\alpha}^{(\dense)}}{\partial x_{j;\alpha}^{(\sparse)}}
%     & = W_{mj}^{(\dec)}
% \end{align}
% Hence:
% \begin{align}
%     J_{ij}(\vec{x}_{\alpha}^{(\sparse)}) =
%     \sigma_\SAE^\prime(b_i^{(\pre)} + W_{in}^{(\enc)}z_n^{(\dense)}) \cdot W_{ik}^{(\enc)} W_{k\ell}^{(2)} \cdot
%     \sigma_\FFN^\prime(\vec{z}_\ell^{(1)}) \cdot
%     W_{\ell m}^{(1)} W_{mj}^{(\dec)}
% \end{align}
% We can simplify this by defining:
% \begin{align}
%     W_{ij}^{(\enc;2)}
%     & = W_{ik}^{(\enc)} W_{kj}^{(2)}
%     \in \Reals^{n^{(\sparse)} \times n^{(1)}}
%     \\
%     W_{ij}^{(1;\dec)}
%     & = W_{ik}^{(1)} W_{kj}^{(\dec)}
%     \in \Reals^{n^{(1)} \times n^{(\sparse)}}
% \end{align}
% Then we have with dimensions:
% \begin{align}
%     J_{ij}(x_{\alpha}^{(\sparse)}) = 
%     \underbrace{\sigma_\SAE^\prime(b_i^{(\pre)} + W_{in}^{(\enc)}z_n^{(\dense)})}_{\in \Reals^k}
%     \ \cdot\ 
%     \underbrace{W_{i\ell}^{(\enc;2)}}_{\in \Reals^{k \times n^{(1)}}}
%     \ \cdot\ 
%     \underbrace{\sigma_\FFN^\prime(\vec{z}_\ell^{(1)})}_{\in \Reals^{n^{(1)}}}
%     \ \cdot\ 
%     \underbrace{W_{\ell j}^{(1;\dec)}}_{\in \Reals^{n^{(1)} \times k}}
% \end{align}

% For TopK SAEs, $\sigma_\SAE(\vec{x}) \triangleq \ReLU(\TopK(\vec{x}))$ and, practically, $\sigma^\prime_\SAE(\vec{x})$ is 1 because the top-$k$ latent activations are almost always positive for reasonable choices of $k$.
% We assume that $k$ is the same at every layer, i.e., that $\vec{x}_\alpha^{\,(\sparse)},\ \vec{z}_\alpha^{\,(\sparse)} \in \Reals^k$.
% Hence, we may choose $k$ columns from $\mat{W}^{(\dec)}$ and $k$ rows from $\mat{W}^{(\enc)}$, where the columns and rows depend on the token $\alpha$.

\section{$f_s$ is approximately linear}\label{app:not_local}

Consider the scalar function $\fsij:\Reals\to\Reals$ which takes as input the $j$-th latent activation of the first SAE (i.e. $\sxj$) and returns as output the $i$-th latent activation of the second SAE (i.e., $\syi$), while keeping the other elements of the input vector fixed at the same values as $\sx$.
In other words, this function captures the relationship between the $j$-th input SAE latent and the $i$-th output SAE latent in the context of $\sx$.
Geometrically, we start off at the point $\sx$, and we move from it through the input spaces in parallel to the $j$-th basis vector, and then we observe how the output of $f_s$ projects onto the $i$-th basis vector.
Formally,
\begin{align}
   \fsij(x)&=f_s\left(\psi(\sx,i,x)\right)_j \label{eq:def_fsij}\\
   \psi(\sx,i,x)_k &=\begin{cases}
       x&\text{if }i=k\\
       s_{\text{x},j} &\text{otherwise}
   \end{cases}
\end{align}
% where square brackets denote indexing in a vector, e.g., $\x[i]$ is the $i$-th element of $\x$.
These are the functions shown in Figure~\ref{fig:mostly_linear_410m}a, of which the vast majority are linear (Figure~\ref{fig:mostly_linear_410m}b).

As we showed in Figure \ref{fig:mostly_linear_410m}c, the absolute value of a Jacobian element nearly perfectly predicts the change we see in the output SAE latent activation value when we make a large intervention on the input SAE latent activation.
However, in the same figure, there is a small cluster of approximately 2.5\% of samples, where the Jacobian element is near zero, but the change observed in the downstream feature is quite large.
We proceed by exploring the cause behind this phenomenon.

Note that each point in Figure~\ref{fig:mostly_linear_410m} corresponds to a single scalar function $\fsij$ (a pair of latent indices).
An expanded version of Figure \ref{fig:mostly_linear_410m} is presented in Figure \ref{fig:not_local_expanded}.
Importantly, we show the `line', the top-left cluster, and outliers visible in Figure~\ref{fig:mostly_linear_410m} in different colors, which we re-use in the following charts (Figures~\ref{fig:not_local_functions} and \ref{fig:nonlinear_functions}).
It also includes 10K samples, compared to 1K in Figure \ref{fig:mostly_linear_410m}c: as above, most samples remain on the line, but the greater number of samples makes the behavior of the top-left cluster and outliers clearer.

Figure \ref{fig:not_local_functions} illustrates some examples of functions $\fsij$ taken from each category shown in Figure~\ref{fig:mostly_linear_410m}, i.e., the line, cluster, and outliers.
The vast majority of functions belong to the line category and are typically either linear or akin to JumpReLU activation functions (which include step functions as a special case).
By contrast, the minority of functions belonging to the cluster or outliers are typically also JumpReLU-like, except where the unmodified input latent activation is close to the point where the function `jumps', so when we subtract an activation value of 1 from the input (as in Figures~\ref{fig:mostly_linear_410m}c and \ref{fig:not_local_expanded}), this moves to the flat region where the output latent activation value is zero.

As we can see, the vast majority of these functions are either linear or JumpReLUs.
\samepage{Indeed, we verify this across the sample size of 10,000 functions and find that 88\% are linear, 10\% are JumpReLU (excl. linear, which is arguably a special case of JumpReLU), and only 2\% are neither\footnote{Note that we are testing whether functions are linear or JumpReLUs only in the region of input space within which SAE activations exist. In particular, this means that we are excluding negative numbers. More specifically, the domain within which we test the function's structure is $[0,\max(5, \vec{s}_{\text{x},i}^{(l)}+1)]$. In 92\% of cases, $\vec{s}_{\text{x},i}^{(l)}+1 < 5$; the median $\vec{s}_{\text{x},i}^{(l)}$ is 2.5.}. \label{footnote:fsij_domain}}
This result is encouraging -- for a linear function, the first-order derivative is constant, so its value (i.e., the corresponding element of the Jacobian) completely expresses the relationship between the input and output values (up to a constant intercept).
For the 88\% of these scalar functions that are linear, the Jacobian thus accurately captures the notion of computational sparsity that interests us, rather than serving only as a proxy.
And for the 10\% of JumpReLUs, the Jacobians still perfectly measure the computational change we observe when changing the input latent within some subset of the input space.

While we expect the remaining 2\% of scalar functions (Jacobian elements) to contribute only a small fraction of the computational structure of the underlying model, we preliminarily investigated their behavior.
Figure \ref{fig:nonlinear_functions} shows 12 randomly selected non-linear, non-JumpReLU $\fsij$ functions.
Even though these functions are nonlinear, they are still reasonably close to being linear, i.e., their first derivative is still predictive of the change we see throughout the input space.
Indeed, most of them are on the diagonal line in Figure \ref{fig:not_local_expanded}.

We can measure this more precisely by looking at the second-order derivative of $\fsij$.
A zero second-order derivative across the whole domain would imply a linear function and, therefore, perfect predictive power of the Jacobian, while the larger the absolute value of the second-order derivative, the less predictive the Jacobian will be.
This distribution is shown in Figure \ref{fig:second_derivative_distribution}.
The same distribution, which only includes the non-linear, non-JumpReLU functions, is shown in Figure \ref{fig:second_derivative_distribution_nonlinear}.
On average, the second derivative is extremely small for all features and effectively zero for the vast majority.

\begin{figure}
    \centering
    \includegraphics{figures/func_examples.pdf}
    \caption{Additional examples of scalar functions between $\sxj$ to $\syi$. The top row shows linear functions, the middle row shows JumpReLU functions, and the bottom row shows other functions. Recall that linear functions constitute a majority of the functions we observe empirically and that using JSAEs instead of traditional SAEs further increases the proportion of linear functions.}
    \label{fig:func_examples}
\end{figure}

\begin{figure}
    \centering
    \includegraphics{figures/not_local_expanded.pdf}
    \caption{An expanded version of Figure~\ref{fig:mostly_linear_410m}c, measured on layer 3 of Pythia-70m. A scatter plot showing that values of Jacobian elements tend to be approximately equal to the change we see in the downstream feature when we modify the value of the upstream feature, namely when we subtract 1 from it.
    Each dot corresponds to an (input SAE latent, output SAE latent) pair.
    Unlike Figure~\ref{fig:mostly_linear_410m}c, this figure colors in the dots depending on which cluster they belong to -- blue for "on the line", green for "in the cluster", red for "outlier".
    Additionally, this figure contains 10,000 samples (rather than 1,000 as in Figure~\ref{fig:mostly_linear_410m}c), which allows us to see more of the outliers and edge cases, though at the cost of visually obfuscating the fact that 97.5\% of the samples are on the diagonal line, 2.1\% are in the cluster, and 0.4\% are outliers.
    }
    \label{fig:not_local_expanded}
\end{figure}

\begin{figure}
    \centering
    \includegraphics{figures/not_local_functions.pdf}
    \caption{A handful of $\fsij$ functions corresponding to the points in Figure \ref{fig:not_local_expanded}.
    The color matches the group (and therefore the color) they were assigned in Figure \ref{fig:not_local_expanded}.
    The red dashed vertical line denotes $\vec{s}_{\text{x},i}^{(l)}$, i.e. the activation value of the SAE latent before we intervened on it.
    Note that the functions are not selected randomly but rather hand-selected to demonstrate the range of functions.
    We will quantitatively explore what proportion of $\fsij$ functions have which structure in other figures.}
    \label{fig:not_local_functions}
\end{figure}

\begin{figure}
    \centering
    \includegraphics{figures/nonlinear_functions.pdf}
    \caption{A random selection of the non-linear, non-JumpReLU $\fsij$ functions.
    Note that non-linear, non-JumpReLU functions only constitute about 2\% of $\fsij$ functions.
    Even though these functions are clearly somewhat non-linear, their slope does still change quite slowly for the most part, which means that a first-order derivative at any point in the function is still reasonably predictive of the function's behavior in at least some portion of the input space (though there are some rare exceptions).
    The color again matches the group (and therefore the color) they were assigned in Figure \ref{fig:not_local_expanded};
    the red dashed vertical line denotes $\vec{s}_{\text{x},i}^{(l)}$, i.e. the activation value of the SAE latent before we intervened on it.}
    \label{fig:nonlinear_functions}
\end{figure}

\begin{figure}
    \centering
    \includegraphics{figures/second_derivative_distribution.pdf}
    \caption{Distribution of second-order derivatives of functions $\fsij$. Includes all functions, regardless of whether they are linear, JumpReLU, or neither. For a version that only includes non-linear, non-JumpReLU functions, see Figure \ref{fig:second_derivative_distribution_nonlinear}.
        (a) The mean of the second-order derivative over the region of the input space.
        (b) The mean of the absolute value of the second-order derivative over the region of the input space.
        (c) The maximum value the second-order derivative takes in the region of the input space.
        Note that we are approximating the second derivative by looking at changes over a very small region (specifically $0.005$), i.e., we do not take the limit as the size of this small region goes to zero; this is important because derivatives which would otherwise be undefined or infinite become finite with this approximation and therefore can be shown on the histograms.
        Also, we note that the means and maxima are taken over the region of the input space in which SAE features exist; see the footnote on page \pageref{footnote:fsij_domain}.
    }
    \label{fig:second_derivative_distribution}
\end{figure}


\begin{figure}
    \centering
    \includegraphics{figures/second_derivative_distribution_nonlinear.pdf}
    \caption{Distribution of second-order derivatives of functions $\fsij$. Unlike Figure \ref{fig:second_derivative_distribution}, this figure only includes the subset of the functions that are neither linear nor JumpReLU=like.
        (a) The mean of the second-order derivative over the region of the input space.
        (b) The mean of the absolute value of the second-order derivative over the region of the input space.
        (c) The maximum value the second-order derivative takes in the region of the input space.
        Note that we are approximating the second derivative by looking at changes over a very small region (specifically $0.005$), i.e. we do not take the limit as the size of this small region goes to zero; this is important because derivatives which would otherwise be undefined or infinite become finite with this approximation and therefore can be shown on the histograms.
        Also, we note that the means and maxima are taken over the region of the input space in which SAE features exist; see the footnote on page \pageref{footnote:fsij_domain}.
    }
    \label{fig:second_derivative_distribution_nonlinear}
\end{figure}

\begin{figure}
    \centering
    \includegraphics{figures/linear_layer.pdf}
    \caption{
        The fractions of Jacobian elements that exhibit a linear relationship between the input and output SAE latent activations, a JumpReLU-like relationship, and an uncategorized relationship, as described in Section~\ref{sec:mostly_linear}.
        Here, we consider Jacobian SAEs trained on the feed-forward network at different layers of Pythia-70m, 160m, and 410m with fixed expansion factors $R=64$ and $k=32$.
        We computed the fractions over 1 million samples.
    }
    \label{fig:linear_layer}
\end{figure}

\begin{figure}
    \centering
    \includegraphics{figures/linear_jacobian_coef.pdf}
    \caption{
        The fractions of Jacobian elements that exhibit a linear relationship between the input and output SAE latent activations, a JumpReLU-like relationship, and an uncategorized relationship, as described in Section~\ref{sec:mostly_linear}.
        Here, we consider Jacobian SAEs trained on the feed-forward network at layer 3 of Pythia-70m (left) and layer 7 of Pythia-160m (right), with fixed expansion factors $R=64$ and $k=32$ and varying Jacobian loss coefficient (Section~\ref{sec:methods}).
        We computed the fractions over 1 million samples.
    }
    \label{fig:linear_jacobian_coef}
\end{figure}

\begin{figure}
    \centering
    \includegraphics{figures/linear_expansion_factor_k.pdf}
    \caption{
        The fractions of Jacobian elements that exhibit a linear relationship between the input and output SAE latent activations, a JumpReLU-like relationship, and an uncategorized relationship, as described in Section~\ref{sec:mostly_linear}.
        Here, we consider Jacobian SAEs trained on the feed-forward network at layer 3 of Pythia-70m with varying expansion factors (and hence numbers of latents; left) but fixed sparsities $k=32$, and varying sparsities but fixed expansion factors $R=64$ (Section~\ref{sec:methods}).
        We computed the fractions over 1 million samples.
    }
    \label{fig:linear_expansion_factor_k}
\end{figure}

\section{Training}\label{app:training}

Our training implementation is based on the open-source SAELens library \citep{bloom_jbloomaus_2023}.
We train each pair of SAEs on 300 million tokens from the Pile \citep{gao_pile_2020}, excluding the copyrighted Books3 dataset, for a single epoch.
Except where noted, we use a batch size of 4096 sequences, each with a context size of 2048.
At a given time, we maintain 32 such batches of activation vectors in a buffer that is shuffled before training, which reduces variance in the training signal.

We use the Adam optimizer \citep{kingma_adam_2017} with the default beta parameters and a constant learning-rate schedule with 1\% warm-up steps, 20\% decay steps, and a maximum value of \qty{5e-4}.
Additionally, we use 5\% warm-up steps for the coefficient of the Jacobian term in the training loss.
We initialize the decoder weight matrix to the transpose of the encoder, and we scale the decoder weight vectors to unit norm at initialization and after each training step \citep{gao_scaling_2024}.

Except where noted, we choose an expansion factor $R=32$, keep the $k=32$ largest latents in the TopK activation function of each of the input and output SAEs, and choose a coefficient of $\lambda=1$ for the Jacobian term in the training loss.

\subsection{Training signal stability}
We initially considered the following setup:
\begin{equation}
  \sx = \encx(\x)\ ,\quad
  \hatx = \decx(\sx)\ ,\quad
  \y = f(\hatx)\ ,\quad
  \sy = \ency(\y)\ ,\quad
  \haty = \decy(\sy)
\end{equation}
The problem with this arrangement is that the second SAE depends on an output from the first SAE.
Since both SAEs are trained simultaneously, we found that this compromised training signal stability -- whenever the first SAE changed, the training distribution of the second SAE changed with it.
Additionally, at the start of training, when the first SAE was not yet capable of outputting anything meaningful, the second SAE had no meaningful training data at all, which not only made it impossible for the second SAE to learn but also made the first SAE less stable via the Jacobian sparsity loss term.

To address this problem, we instead used this setup:
\begin{equation}
    \sx = \encx(\x)\ ,\quad
    \hatx = \decx(\sx)\ ,\quad
    \y = f(\x)\ ,\quad
    \sy = \ency(\x)\ ,\quad
    \haty = \decy(\sy)
\end{equation}
Importantly, we pass the actual pre-MLP activations $\x$ rather than the reconstructed activations $\hatx$ into the MLP $f$.
In addition to improving training stability, we believe this setup to be more faithful to the underlying model because both SAEs are trained on the unmodified activations that pass through the MLP.

% \section{Formalism of circuits}\label{app:form}

% Formalism of circuits

% Predictions about the levels of sparsity we'd expect to see (and which we do in fact, see)

\section{Evaluation}
\label{app:evaluation}

We evaluated each of the input and output SAEs during training on ten batches of eight sequences, where each sequence has a context size of 2048, i.e., approximately 160K tokens.
We computed the sparsity of the Jacobian, measured by the mean number of absolute values above $0.01$ for a single token, separately after training.
In this case, we collected statistics over 10 million tokens from the validation subset of the C4 text dataset.

For reconstruction quality, we report the mean cosine similarity between input activation vectors and their autoencoder reconstructions, the explained variance (MSE reconstruction error divided by the variance of the input activation vectors), and the MSE reconstruction error.

For model performance preservation, we report the cross-entropy loss score, which is the increase in the cross-entropy loss when the input activations are replaced by their autoencoder reconstruction divided by the increase in the loss when the input activations are ablated (set to zero).

For sparsity, we report the number of `dead' latents that have not been activated (i.e., appeared in the $k$ largest latents of the TopK activation function) within the preceding 10 million tokens during training and the number of latents that have activated fewer than once per 1 million tokens during training on average.

Given an expansion factor of $64$, $k=32$, and a Jacobian loss coefficient of $1$, i.e., fixed hyperparameters, we find that the reconstruction error and cross-entropy loss score are consistently better for the input SAE than the output SAE.
Additionally, we find that the performance is generally poorer for the intermediate layers than early and later layers.

We speculate that it is necessary to tune our hyperparameters for each layer individually to achieve improved performance; see, for example, Figures~\ref{fig:jacobian_coef_70m} and \ref{fig:jacobian_coef_160m} for the variation of our evaluation metrics against the coefficient of the Jacobian loss term for individual layers of Pythia-70m and 160m.

\begin{figure}
    \centering
    \includegraphics{figures/reconstruction_quality.pdf}
    \caption{Reconstruction quality metrics for Jacobian SAEs trained on the feed-forward networks at every layer (residual block) of Pythia transformers. The cosine similarity is taken between the input and reconstructed activation vectors, and the explained variance is the MSE reconstruction error divided by the variance of the input activations. For each SAE, the expansion factor is $R=64$ and $k=32$; the Jacobian loss coefficient is 1.}
    \label{fig:reconstruction_quality}
\end{figure}

\begin{figure}
    \centering
    \includegraphics{figures/ce_loss_score.pdf}
    \caption{Model performance preservation metrics for Jacobian SAEs trained on the feed-forward networks at every layer (residual block) of Pythia transformers. The cross-entropy loss score is the increase in the cross-entropy loss when the input activations are replaced by their autoencoder reconstruction divided by the increase when the input activations are ablated (set to zero). For each SAE, the expansion factor is $R=64$ and $k=32$; the Jacobian loss coefficient is 1.}
    \label{fig:ce_loss_score}
\end{figure}

\begin{figure}
    \centering
    \includegraphics{figures/sparsity.pdf}
    \caption{Sparsity metrics per layer for Jacobian SAEs trained on the feed-forward networks at every layer (residual block) of Pythia transformers. Recall that the $L^0$ norm per token for each of the input and output SAEs is fixed at $k$ by the TopK activation function. For each SAE, the expansion factor is $R=64$ and $k=32$; the Jacobian loss coefficient is 1.}
    \label{fig:sparsity}
\end{figure}

\begin{figure}
    \centering
    \includegraphics{figures/expansion_factor.pdf}
    \caption{
        Reconstruction quality, model performance preservation, and sparsity metrics against the number of latents.
        Here, we consider Jacobian SAEs trained on the feed-forward network at layer 3 of Pythia-70m (model dimension 512) with $k=32$.
        Recall that the maximum number of non-zero Jacobian values is $k^2=1024$.
        The reconstruction quality and cross-entropy loss score improve as the number of latents increases, and the number of dead features grows more quickly for the output SAE than the input SAE.
        See Appendix~\ref{app:evaluation} for details of the evaluation metrics.
    }
    \label{fig:expansion_factor}
\end{figure}

\begin{figure}
    \centering
    \includegraphics{figures/k.pdf}
    \caption{
        Reconstruction quality, model performance preservation, and sparsity metrics against the $k$ largest latents to keep in the TopK activation function.
        Here, we consider Jacobian SAEs trained on the feed-forward network at layer 3 of Pythia-70m with expansion factor $R=64$.
        Recall that the maximum number of non-zero Jacobian values is $k^2$.
        The reconstruction quality and cross-entropy loss score improve as $k$ increases, and the number of dead features decreases.
        See Appendix~\ref{app:evaluation} for details of the evaluation metrics.
    }
    \label{fig:k}
\end{figure}

\begin{figure*}
    \centering
    \includegraphics{figures/jacobian_coef_70m.pdf}
    \caption{
        Reconstruction quality, model performance preservation, and sparsity metrics against the Jacobian loss coefficient.
        Here, we consider Jacobian SAEs trained on the feed-forward network at layer 3 of Pythia-70m with expansion factor $R=64$ and $k=32$.
        Recall that the maximum number of non-zero Jacobian values is $k^2=1024$.
        In accordance with Figure~\ref{fig:tradeoff_reconstr_sparsity_410m}, all evaluation metrics degrade for values of the coefficient above 1. See Appendix~\ref{app:evaluation} for details of the evaluation metrics.
    }
    \label{fig:jacobian_coef_70m}
\end{figure*}

\begin{figure*}
    \centering
    \includegraphics{figures/pareto_70m.pdf}
    \caption{
        Pareto frontiers of the explained variance and cross-entropy loss score against different sparsity measures when varying the Jacobian loss coefficient.
        Here, we consider Jacobian SAEs trained on the feed-forward network at layer 3 of Pythia-70m with expansion factor $R=64$ and $k=32$.
        Recall that the maximum number of (dead) latents is $32768$ ($64$ times the model dimension $512$), and the maximum number of non-zero Jacobian values is $k^2=1024$.
        See Appendix~\ref{app:evaluation} for details of the evaluation metrics.
    }
    \label{fig:pareto_70m}
\end{figure*}

\begin{figure*}
    \centering
    \includegraphics{figures/pareto_160m.pdf}
    \caption{
        Pareto frontiers of the explained variance and cross-entropy loss score against different sparsity measures when varying the Jacobian loss coefficient.
        The coefficient has a relatively small impact on the reconstruction quality and sparsity of the input SAE, whereas it has a large effect on the sparsity of the output SAE and elements of the Jacobian matrix.
        Here, we consider Jacobian SAEs trained on the feed-forward network at layer 7 of Pythia-160m with expansion factor $R=64$ and $k=32$.
        Recall that the maximum number of (dead) latents is $49152$ ($64$ times the model dimension $768$), and the maximum number of non-zero Jacobian values is $k^2=1024$.
        See Appendix~\ref{app:evaluation} for details of the evaluation metrics.
    }
    \label{fig:pareto_160m}
\end{figure*}

\begin{figure*}
    \centering
    \includegraphics{figures/tradeoff_reconst_jac_70m.pdf}
    \caption{The trade-off between reconstruction quality and Jacobian sparsity as we vary the Jacobian loss coefficient. Each dot represents a pair of JSAEs trained with a specific Jacobian coefficient.
    Measured on layer 3 of Pythia-70m with $k=32$.}
    \label{fig:tradeoff_reconstr_sparsity_70m}
\end{figure*}

% \FloatBarrier
% \section{Conceptual differences between JSAEs and transcoders}\label{app:transcoders}
% \red{should we drop this?}

% At a very high level, it could be said that JSAEs and transcoders aim to solve the same problem, namely "sparsifying" the MLP.
% Some readers may therefore ask how our setup compares to transcoders and which one performs better on certain metrics.
% The answer is that, in fact, JSAEs aim to solve a substantially different task, which leads to us using a completely different structure.
% This makes it impossible to do an apples-to-apples comparison between the performance of JSAEs and transcoders -- most of the metrics relevant for one of them cannot be applied to the other.

% Transcoders work by learning a sparse 2-layer fully-connected network which takes as input $\x$ and is optimized to return an approximation $\y$, with a sparsity penalty applied to its hidden activations.
% Typically these hidden activations have a much larger dimensionality than the hidden layer of the MLP.
% This then leads to each transcoder latent corresponding to some aspect of the computation of the MLP, in a way that is hopefully monosemantic.
% At a conceptual level, each transcoder latent tells us "there is computation in the MLP related to [concept]".

% By comparison, JSAEs learn a pair of SAEs (which have mostly interpretable latents) which have sparse connections between them.
% At a conceptual level, JSAEs tell us "this feature in the MLP's output was computed using only these ~5 input features".
% They still do not quite tell us how that small handful of input features was combined to compute the output feature's value, but they do tell us the inputs to the function computing each concept in the output space.
% Crucially, they make sure that there is only a small handful of inputs, which is crucial for the long-term ambition of mechanistic interpretability to decompile LLMs into human-readable source code -- it is nearly impossible to reason about a function that takes in tens of thousands of arguments.

\section{More data on Jacobian sparsity}\label{app:jac_sparsity}
In Figure \ref{fig:sparsity} we showed that Jacobians are much more sparse with JSAEs than traditional SAEs.
To this end, we provided a representative example of what the Jacobians look like with JSAEs vs traditional SAEs.
Some readers may object that this is not an apples-to-apples comparison since JSAEs are optimizing for lower L1 on the Jacobian, so it may be the case that JSAEs merely induce Jacobians with smaller elements, but their distribution may still be the same.
To address this criticism, the examples are L2 normalized; we provide un-normalized versions as well as L1 normalized versions of the example Jacobians in Figure \ref{fig:jac_examples_normed}.
We also provide a histogram and a CDF of the distribution of absolute values of Jacobian elements in Figure \ref{fig:jac_hist}, which is taken across 10 million tokens.

\subsection{Jacobian norms}\label{app:jac_norms}
In this section, we address an objection we expect some readers will have to our measures of sparsity.
Our main metric for sparsity is the percentage of elements with absolute values above certain small thresholds (e.g. Figure~\ref{fig:jac_sparsity}).
However, one can imagine two distributions with the same degree of sparsity, but vastly different results on this metric due to a different standard deviation.
For instance, imagine two Gaussian distributions, both with $\mu=0$ but with significantly different standard deviations, $\sigma_1\gg\sigma_2$.
They would score very differently on our metric, but their degrees of sparsity would not be meaningfully different (since sparsity requires there to be a small handful of relatively large elements).
Since our $L_1$ penalty encourages the Jacobians to be smaller, it could be that they simply become more tightly clustered around 0.
However, this is not the case.
We can measure this by looking at the "norms" of the Jacobian, i.e. we flatten the Jacobian, treat it as a vector, and compute its $L_p$ norms.
If the Jacobian is merely becoming smaller, we would expect all of its $L_p$ norms to decrease at roughly the same rate.
On the other hand, if the Jacobian is becoming sparser, we would expect its $L_1,L_2$ norms to decrease while its $L_4,\dots,L_\infty$ norms, which depend more strongly on the presence or absence of a few large elements, should stay roughly the same.
We present these results in Figure~\ref{fig:jac_norms}, as we can see, the Jacobian does become slightly smaller, but most of the effect we see is indeed the Jacobian becoming significantly more sparse.

\begin{figure*}
    \centering
    \includegraphics[width=\linewidth]{figures/jac_examples_normed_410m.pdf}
    \caption{
    Comparison of Jacobians from traditional SAEs vs JSAEs, same as Figure \ref{fig:jac_sparsity} but with different normalization.
    (a) Not normalized.
    (b) L2 normalized.
    Measured on layer 15 of Pythia-410m.}
    \label{fig:jac_examples_normed}
\end{figure*}


\begin{figure*}
    \centering
    \includegraphics[width=\linewidth]{figures/jac_examples_normed_70m.pdf}
    \caption{
    Comparison of Jacobians from traditional SAEs vs JSAEs, same as Figure \ref{fig:jac_sparsity} but with different normalization.
    (a) L1 normalized.
    (b) L2 normalized.
    Measured on layer 3 of Pythia-70m.}
    \label{fig:jac_examples_normed_70m}
\end{figure*}

\begin{figure*}
    \centering
    \includegraphics[width=\linewidth]{figures/jac_hist_410m.pdf}
    \caption{
    Further data showing that JSAEs induce much greater Jacobian sparsity than traditional SAEs.
    (a) A histogram of the absolute values of Jacobian elements in JSAEs versus traditional SAEs.
    JSAEs induce significantly more sparse Jacobians than standard SAEs.
    This means that there is a relatively small number of input-output feature pairs which explain a very large fraction of the computation being performed.
    Note that only the $k\times k$ elements corresponding to active latents are included in the histogram -- the remaining $(\dimsy-k)\times(\dimsx-k)$ elements are zero by definition both for JSAEs and standard TopK SAEs.
    The histogram was collected over 10 million tokens from the validation subset of the C4 text dataset, which produced 10.24 billion feature pairs.
    (b) The cumulative distribution function of the absolute values of Jacobian elements, again demonstrating that JSAEs induce significantly more computational sparsity than traditional SAEs.
    Measured on layer 15 of Pythia-410m.}
    \label{fig:jac_hist}
\end{figure*}
% \todo{add the 70m fig}

\begin{figure*}[t]
    \centering
    \includegraphics{figures/jac_sparsity_70m.pdf}
    \caption{
    JSAEs induce a much greater degree of sparsity in the elements of the Jacobian than traditional SAEs.
    Identical to Figure~\ref{fig:jac_sparsity} but measured on layer 3 of Pythia-70m.
    }
    \label{fig:jac_sparsity_70m}
\end{figure*}

\begin{figure}
    \centering
    \begin{subfigure}{0.49\textwidth}
        \centering
        \includegraphics{figures/jacobian_sparsity_70m.pdf}
        \caption{Pythia-70m Layer 3}
    \end{subfigure}
    \begin{subfigure}{0.49\textwidth}
        \centering
        \includegraphics{figures/jacobian_sparsity_160m.pdf}
        \caption{Pythia-160m Layer 7}
    \end{subfigure}
    \caption{
        Histograms that show the frequency of absolute values of non-zero Jacobian elements for different values of the coefficient of the Jacobian loss term.
        As the coefficient increases, the frequency of larger values decreases, i.e., the Jacobian becomes sparser.
        We provide further details in Figure~\ref{fig:jac_hist}.
    }
    \label{fig:jac_hist_multiple_models}
\end{figure}


\begin{figure*}
    \centering
    \includegraphics[width=\linewidth]{figures/jac_norms.pdf}
    \caption{$L_p$ norms of the Jacobians. We measure these by flattening the Jacobians and treating them as a vector. These results imply that the Jacobians are in fact becoming much more sparse, as opposed to merely becoming smaller (see Section~\ref{app:jac_norms}). Averaged across 1 million tokens, measured on layer 4 of Pythia-70m.}
    \label{fig:jac_norms}
\end{figure*}


\begin{figure}
    \centering
    \includegraphics{figures/autointerp_70m.pdf}
    \caption{Automatic interpretability scores of JSAEs are very similar to traditional SAEs. Measured on all layers of Pythia-70m using the ``fuzzing'' scorer from \citet{paulo_automatically_2024}.}
    \label{fig:autointerp_70m}
\end{figure}



\begin{figure}
    \centering
    \includegraphics{figures/randomized_70m.pdf}
    \caption{
    Jacobians are substantially more sparse in pre-trained LLMs than in randomly initialized transformers.
    This holds both when you actively optimize for Jacobian sparsity with JSAEs, and when you don't optimize for it and use traditional SAEs.
    The proportion of Jacobian elements with absolute values above certain thresholds.
    The figure shows the proportion of Jacobian elements with absolute values above certain thresholds.
    Identical to Figure \ref{fig:randomized_410m} but measured on layer~3 of Pythia-70m.}
    \label{fig:randomized_70m}
\end{figure}

\begin{figure*}
    \centering
    \includegraphics{figures/mostly_linear_70m.pdf}
    \caption{The function $f_s$, which combines the decoder of the first SAE, the MLP, and the encoder of the second SAE, is mostly linear.
    Identical to Figure~\ref{fig:mostly_linear_410m} but measured on layer 3 of Pythia-70m.
    }
    \label{fig:mostly_linear_70m}
\end{figure*}

\section{Interpreting Jacobian elements by examples}
\label{sec:pairs_max_act}

A common approach to interpreting language-model components such as neurons is to collect token sequences and the corresponding activations over a text dataset \citep[e.g.,][]{yun_transformer_2021,bills_language_2023}.
The maximal latent activations may be retained, or activations from different quantiles of the distribution over the dataset \citep{bricken_monosemanticity_2023,choi_scaling_2024,paulo_automatically_2024}.
With Jacobian SAEs, there are several types of activations that we could collect and from which we could retain (e.g.) the maximum positive values to interpret SAE latents: the latent activations of the input SAE, the latent activations of the output SAE, and the elements of the Jacobian matrix between the input and output SAE latents.

% \begin{table}
%     \centering
%     \begin{tabular}{lR{0.75in}R{0.75in}R{0.75in}}
%         \toprule
%          & \multicolumn{3}{c}{Pearson correlation coefficient $r$} \\
%          \cmidrule(lr){2-4}
%          Layer & In v. Out & In v. $\mat{J}$ & Out v. $\mat{J}$ \\
%          \midrule
%          0 & -0.127 & 0.003 & -0.075 \\
%          1 & 0.001 & 0.009 & 0.390 \\
%          2 & 0.030 & 0.044 & 0.183 \\
%          3 & 0.010 & 0.091 & 0.088 \\
%          4 & 0.020 & 0.108 & 0.105 \\
%          5 & 0.003 & 0.111 & 0.092 \\
%          \bottomrule
%     \end{tabular}
%     \caption{The correlations between $i$-th input SAE latent activation, the $j$-th output SAE latent activation, and the $(i,j)$-th element of the Jacobian matrix, given Jacobian SAEs trained at every layer of Pythia-70m over 10K tokens from the C4 text dataset \citep{raffel_exploring_2020}.}
%     \label{tab:correlations}
% \end{table}

Intuitively, we expected to find correlations between the $j$-th input SAE latent activation, the $i$-th output SAE latent activation, and the $(i,j)$-th element of the Jacobian matrix.
However, for Pythia-70m and a small sample of 10K tokens, the Pearson correlation coefficients between these pairs of values were mostly small, on the order of 0.1.
% (\cref{tab:correlations}).
Hence, we chose to collect the input and output SAE latent activations alongside the elements of the Jacobian.
Specifically, for each model, we began by collecting summary statistics for each non-zero element of the Jacobian matrix and the corresponding input and output SAE latent activations (Table~\ref{tab:feature_pair_summary}), over the first 10K records of the English subset of the C4 text dataset \citep{raffel_exploring_2020}.
Given these summary statistics, we found the top $32$ pairs of input and output SAE latent indices when the statistics for each pair were sorted in descending order of the mean absolute value of non-zero Jacobian elements.
% the mean and standard deviation of Jacobian elements, mean and standard deviation in the absolute value of Jacobian elements, and count of non-zero Jacobian elements.

\begin{table}
    \catcode`\_=12 % category other
    \pgfplotstableset{
    	col sep=comma,
    	column type={l},
    	columns={stat,50,mean,std},
    	columns/stat/.style={
                column name={Statistic for Latent Index Pairs},
                string type,
            },
    	columns/mean/.style={
                column name={Mean},
                column type={r},
                sci,
                sci zerofill,
                precision=2,
            },
    	columns/std/.style={
                column name={Standard Deviation},
                column type={r},
                sci,
                sci zerofill,
                precision=2,
            },
    	columns/50/.style={
                column name={Median},
                column type={r},
                sci,
                sci zerofill,
                precision=2,
            },
    	every head row/.style={before row=\toprule,after row=\midrule},
    	every last row/.style={after row=\bottomrule},
    }
    \centering
    \begin{subtable}[t]{\linewidth}
        \centering
        \pgfplotstabletypeset{describe/pythia-70m-layer-3.csv}
        \caption{Pythia-70m, Layer 3}
    \end{subtable}
    \par\bigskip
    \begin{subtable}[t]{\linewidth}
        \centering
        \pgfplotstabletypeset{describe/pythia-160m-layer-7.csv}
        \caption{Pythia-160m, Layer 7}
    \end{subtable}
    \par\bigskip
    \begin{subtable}[t]{\linewidth}
        \centering
        \pgfplotstabletypeset{describe/pythia-410m-layer-15.csv}
        \caption{Pythia-410m, Layer 15}
    \end{subtable}
    \caption{The mean and standard deviation over pairs of input and output SAE latent indices, i.e., non-zero elements of the Jacobian matrix, for different summary statistics of each pair. The statistics were collected over the first 10K records of the English subset of the C4 text dataset \citep{raffel_exploring_2020}. The standard deviation in the count of non-zero Jacobian elements is very large, i.e., the frequency with which pairs of latent indices are non-zero varies widely. For each underlying transformer and MLP layer, the Jacobian SAE pair was trained with an expansion factor of $R=64$ and sparsity $k=32$.}
    \label{tab:feature_pair_summary}
\end{table}

For each of these pairs, we collected the input and output SAE latent activations and Jacobian elements over examples of context length $16$ from the same text dataset, retaining examples where at least one token produced a non-zero Jacobian element.
We chose a context length of $16$ to conveniently display examples and retained the top eight examples when sorted in descending order of the maximum Jacobian element over the example.
Each table of examples displays a list of at most $12$ examples, each comprising $16$ tokens; we exclude the end-of-sentence token for brevity.
The values of non-zero Jacobian elements and the activations of the corresponding input and output SAE latent indices are indicated by the opacity of the background color for each token.
We take the opacity to be the element or activation divided by the maximum value over the dataset, i.e., all the examples with a non-zero Jacobian element for a given pair of input and output SAE latent indices.
For clarity, we report the maximum element or activation alongside the colored tokens.
Tables~\ref{tab:feature_pairs_24720_33709}, \ref{tab:feature_pairs_314_31729}, \ref{tab:feature_pairs_48028_64386}, \ref{tab:feature_pairs_26438_54734}, and \ref{tab:feature_pairs_54846_30912} were chosen from among the pairs of latent indices with the top $32$ maximum mean absolute values of Jacobian elements over the dataset to broadly demonstrate the patterns we observed within latent activations and Jacobian elements.

\renewcommand{\arraystretch}{0.5}
\renewcommand{\fboxsep}{0pt}

% The top $12$ examples that produce the maximum absolute values of the Jacobian element with input SAE latent index $314$ and output latent index $31729$.
% % This pair of latent indices is one of the top $5$ pairs by the mean absolute value of non-zero Jacobian elements.
% The Jacobian SAE pair was trained on layer 15 of Pythia-410m with an expansion factor of $R=64$ and sparsity $k=32$.
% The examples were collected over the first 10K records of the English subset of the C4 text dataset \citep{raffel_exploring_2020}, with a context length of $16$ tokens.
% For each example, the first row shows the values of the Jacobian element, and the second and third show the corresponding activations of the input and output SAE latents.

% \input{feature_pairs_tex/pythia-410m-layer-15-abs-mean-in-39503-out-34455-b32-t16} % German text?
% \input{feature_pairs_tex/pythia-410m-layer-15-abs-mean-in-3387-out-34455-b32-t16} % German syllables?
\begin{table}
\centering
\begin{tabular}{lrl}
\toprule
Category & Max. abs. value & Example tokens \\
\midrule
Jacobian & \num{3.993e-01} & \colorbox{Cyan!0.000}{\strut .} \colorbox{Cyan!0.000}{\strut  I} \colorbox{Cyan!93.634}{\strut  didn} \colorbox{Cyan!0.000}{\strut \textquotesingle{}} \colorbox{Cyan!0.000}{\strut t} \colorbox{Cyan!0.000}{\strut  ask} \colorbox{Cyan!0.000}{\strut  for} \colorbox{Cyan!0.000}{\strut  them} \colorbox{Cyan!0.000}{\strut .} \colorbox{Cyan!0.000}{\strut  I} \colorbox{Cyan!100.000}{\strut  didn} \colorbox{Cyan!0.000}{\strut \textquotesingle{}} \colorbox{Cyan!0.000}{\strut t} \colorbox{Cyan!0.000}{\strut  see} \colorbox{Cyan!0.000}{\strut  them} \\
Input SAE & \num{1.937e+01} & \colorbox{Green!0.000}{\strut .} \colorbox{Green!0.000}{\strut  I} \colorbox{Green!83.946}{\strut  didn} \colorbox{Green!0.000}{\strut \textquotesingle{}} \colorbox{Green!0.000}{\strut t} \colorbox{Green!0.000}{\strut  ask} \colorbox{Green!0.000}{\strut  for} \colorbox{Green!0.000}{\strut  them} \colorbox{Green!0.000}{\strut .} \colorbox{Green!0.000}{\strut  I} \colorbox{Green!65.482}{\strut  didn} \colorbox{Green!0.000}{\strut \textquotesingle{}} \colorbox{Green!0.000}{\strut t} \colorbox{Green!0.000}{\strut  see} \colorbox{Green!0.000}{\strut  them} \\
Output SAE & \num{6.176e+00} & \colorbox{Magenta!0.000}{\strut .} \colorbox{Magenta!0.000}{\strut  I} \colorbox{Magenta!82.899}{\strut  didn} \colorbox{Magenta!0.000}{\strut \textquotesingle{}} \colorbox{Magenta!0.000}{\strut t} \colorbox{Magenta!0.000}{\strut  ask} \colorbox{Magenta!0.000}{\strut  for} \colorbox{Magenta!0.000}{\strut  them} \colorbox{Magenta!0.000}{\strut .} \colorbox{Magenta!0.000}{\strut  I} \colorbox{Magenta!70.773}{\strut  didn} \colorbox{Magenta!0.000}{\strut \textquotesingle{}} \colorbox{Magenta!0.000}{\strut t} \colorbox{Magenta!0.000}{\strut  see} \colorbox{Magenta!0.000}{\strut  them} \\
\midrule
Jacobian & \num{3.947e-01} & \colorbox{Cyan!0.000}{\strut  by} \colorbox{Cyan!0.000}{\strut  minute} \colorbox{Cyan!0.000}{\strut  basis} \colorbox{Cyan!0.000}{\strut .} \colorbox{Cyan!0.000}{\strut  It} \colorbox{Cyan!0.000}{\strut \textquotesingle{}} \colorbox{Cyan!0.000}{\strut s} \colorbox{Cyan!0.000}{\strut  easy} \colorbox{Cyan!0.000}{\strut  for} \colorbox{Cyan!0.000}{\strut  someone} \colorbox{Cyan!0.000}{\strut  to} \colorbox{Cyan!0.000}{\strut  come} \colorbox{Cyan!0.000}{\strut  along} \colorbox{Cyan!0.000}{\strut  who} \colorbox{Cyan!98.832}{\strut  isn} \\
Input SAE & \num{1.677e+01} & \colorbox{Green!0.000}{\strut  by} \colorbox{Green!0.000}{\strut  minute} \colorbox{Green!0.000}{\strut  basis} \colorbox{Green!0.000}{\strut .} \colorbox{Green!0.000}{\strut  It} \colorbox{Green!0.000}{\strut \textquotesingle{}} \colorbox{Green!0.000}{\strut s} \colorbox{Green!0.000}{\strut  easy} \colorbox{Green!0.000}{\strut  for} \colorbox{Green!0.000}{\strut  someone} \colorbox{Green!0.000}{\strut  to} \colorbox{Green!0.000}{\strut  come} \colorbox{Green!0.000}{\strut  along} \colorbox{Green!0.000}{\strut  who} \colorbox{Green!72.659}{\strut  isn} \\
Output SAE & \num{5.698e+00} & \colorbox{Magenta!0.000}{\strut  by} \colorbox{Magenta!0.000}{\strut  minute} \colorbox{Magenta!0.000}{\strut  basis} \colorbox{Magenta!0.000}{\strut .} \colorbox{Magenta!0.000}{\strut  It} \colorbox{Magenta!0.000}{\strut \textquotesingle{}} \colorbox{Magenta!0.000}{\strut s} \colorbox{Magenta!0.000}{\strut  easy} \colorbox{Magenta!0.000}{\strut  for} \colorbox{Magenta!0.000}{\strut  someone} \colorbox{Magenta!0.000}{\strut  to} \colorbox{Magenta!0.000}{\strut  come} \colorbox{Magenta!0.000}{\strut  along} \colorbox{Magenta!0.000}{\strut  who} \colorbox{Magenta!76.484}{\strut  isn} \\
\midrule
Jacobian & \num{3.937e-01} & \colorbox{Cyan!0.000}{\strut !} \colorbox{Cyan!0.000}{\strut I} \colorbox{Cyan!0.000}{\strut \textquotesingle{}} \colorbox{Cyan!0.000}{\strut m} \colorbox{Cyan!0.000}{\strut  glad} \colorbox{Cyan!0.000}{\strut  it} \colorbox{Cyan!0.000}{\strut  worked} \colorbox{Cyan!0.000}{\strut  partially} \colorbox{Cyan!0.000}{\strut  for} \colorbox{Cyan!0.000}{\strut  you} \colorbox{Cyan!0.000}{\strut  although} \colorbox{Cyan!0.000}{\strut  it} \colorbox{Cyan!98.581}{\strut  didn} \\
Input SAE & \num{1.806e+01} & \colorbox{Green!0.000}{\strut !} \colorbox{Green!0.000}{\strut I} \colorbox{Green!0.000}{\strut \textquotesingle{}} \colorbox{Green!0.000}{\strut m} \colorbox{Green!0.000}{\strut  glad} \colorbox{Green!0.000}{\strut  it} \colorbox{Green!0.000}{\strut  worked} \colorbox{Green!0.000}{\strut  partially} \colorbox{Green!0.000}{\strut  for} \colorbox{Green!0.000}{\strut  you} \colorbox{Green!0.000}{\strut  although} \colorbox{Green!0.000}{\strut  it} \colorbox{Green!78.264}{\strut  didn} \\
Output SAE & \num{6.157e+00} & \colorbox{Magenta!0.000}{\strut !} \colorbox{Magenta!0.000}{\strut I} \colorbox{Magenta!0.000}{\strut \textquotesingle{}} \colorbox{Magenta!0.000}{\strut m} \colorbox{Magenta!0.000}{\strut  glad} \colorbox{Magenta!0.000}{\strut  it} \colorbox{Magenta!0.000}{\strut  worked} \colorbox{Magenta!0.000}{\strut  partially} \colorbox{Magenta!0.000}{\strut  for} \colorbox{Magenta!0.000}{\strut  you} \colorbox{Magenta!0.000}{\strut  although} \colorbox{Magenta!8.529}{\strut  it} \colorbox{Magenta!82.649}{\strut  didn} \\
\midrule
Jacobian & \num{3.931e-01} & \colorbox{Cyan!0.000}{\strut  read} \colorbox{Cyan!0.000}{\strut  other} \colorbox{Cyan!0.000}{\strut  books} \colorbox{Cyan!0.000}{\strut  then} \colorbox{Cyan!0.000}{\strut  the} \colorbox{Cyan!0.000}{\strut  Bible} \colorbox{Cyan!0.000}{\strut ,} \colorbox{Cyan!0.000}{\strut  but} \colorbox{Cyan!0.000}{\strut  if} \colorbox{Cyan!0.000}{\strut  the} \colorbox{Cyan!0.000}{\strut  Bible} \colorbox{Cyan!98.446}{\strut  isn} \colorbox{Cyan!0.000}{\strut \textquotesingle{}} \colorbox{Cyan!0.000}{\strut t} \colorbox{Cyan!0.000}{\strut  being} \\
Input SAE & \num{2.153e+01} & \colorbox{Green!0.000}{\strut  read} \colorbox{Green!0.000}{\strut  other} \colorbox{Green!0.000}{\strut  books} \colorbox{Green!0.000}{\strut  then} \colorbox{Green!0.000}{\strut  the} \colorbox{Green!0.000}{\strut  Bible} \colorbox{Green!0.000}{\strut ,} \colorbox{Green!0.000}{\strut  but} \colorbox{Green!0.000}{\strut  if} \colorbox{Green!0.000}{\strut  the} \colorbox{Green!0.000}{\strut  Bible} \colorbox{Green!93.312}{\strut  isn} \colorbox{Green!0.000}{\strut \textquotesingle{}} \colorbox{Green!0.000}{\strut t} \colorbox{Green!0.000}{\strut  being} \\
Output SAE & \num{7.214e+00} & \colorbox{Magenta!0.000}{\strut  read} \colorbox{Magenta!0.000}{\strut  other} \colorbox{Magenta!0.000}{\strut  books} \colorbox{Magenta!0.000}{\strut  then} \colorbox{Magenta!0.000}{\strut  the} \colorbox{Magenta!0.000}{\strut  Bible} \colorbox{Magenta!0.000}{\strut ,} \colorbox{Magenta!0.000}{\strut  but} \colorbox{Magenta!0.000}{\strut  if} \colorbox{Magenta!0.000}{\strut  the} \colorbox{Magenta!0.000}{\strut  Bible} \colorbox{Magenta!96.836}{\strut  isn} \colorbox{Magenta!7.402}{\strut \textquotesingle{}} \colorbox{Magenta!0.000}{\strut t} \colorbox{Magenta!0.000}{\strut  being} \\
\midrule
Jacobian & \num{3.930e-01} & \colorbox{Cyan!0.000}{\strut  hide} \colorbox{Cyan!0.000}{\strut  her} \colorbox{Cyan!0.000}{\strut  pills} \colorbox{Cyan!0.000}{\strut  in} \colorbox{Cyan!0.000}{\strut ).} \colorbox{Cyan!0.000}{\strut  She} \colorbox{Cyan!93.276}{\strut  didn} \colorbox{Cyan!0.000}{\strut \textquotesingle{}} \colorbox{Cyan!0.000}{\strut t} \colorbox{Cyan!0.000}{\strut  like} \colorbox{Cyan!0.000}{\strut  it} \colorbox{Cyan!0.000}{\strut .} \colorbox{Cyan!0.000}{\strut  She} \colorbox{Cyan!0.000}{\strut  also} \colorbox{Cyan!98.412}{\strut  didn} \\
Input SAE & \num{1.971e+01} & \colorbox{Green!0.000}{\strut  hide} \colorbox{Green!0.000}{\strut  her} \colorbox{Green!0.000}{\strut  pills} \colorbox{Green!0.000}{\strut  in} \colorbox{Green!0.000}{\strut ).} \colorbox{Green!0.000}{\strut  She} \colorbox{Green!85.424}{\strut  didn} \colorbox{Green!0.000}{\strut \textquotesingle{}} \colorbox{Green!0.000}{\strut t} \colorbox{Green!0.000}{\strut  like} \colorbox{Green!0.000}{\strut  it} \colorbox{Green!0.000}{\strut .} \colorbox{Green!0.000}{\strut  She} \colorbox{Green!0.000}{\strut  also} \colorbox{Green!72.553}{\strut  didn} \\
Output SAE & \num{6.299e+00} & \colorbox{Magenta!0.000}{\strut  hide} \colorbox{Magenta!0.000}{\strut  her} \colorbox{Magenta!0.000}{\strut  pills} \colorbox{Magenta!0.000}{\strut  in} \colorbox{Magenta!0.000}{\strut ).} \colorbox{Magenta!0.000}{\strut  She} \colorbox{Magenta!84.549}{\strut  didn} \colorbox{Magenta!0.000}{\strut \textquotesingle{}} \colorbox{Magenta!0.000}{\strut t} \colorbox{Magenta!0.000}{\strut  like} \colorbox{Magenta!0.000}{\strut  it} \colorbox{Magenta!0.000}{\strut .} \colorbox{Magenta!0.000}{\strut  She} \colorbox{Magenta!0.000}{\strut  also} \colorbox{Magenta!76.924}{\strut  didn} \\
\midrule
Jacobian & \num{3.925e-01} & \colorbox{Cyan!0.000}{\strut  breakdown} \colorbox{Cyan!0.000}{\strut s} \colorbox{Cyan!0.000}{\strut  that} \colorbox{Cyan!0.000}{\strut  day} \colorbox{Cyan!0.000}{\strut  because} \colorbox{Cyan!0.000}{\strut  I} \colorbox{Cyan!96.135}{\strut  couldn} \colorbox{Cyan!0.000}{\strut \textquotesingle{}} \colorbox{Cyan!0.000}{\strut t} \colorbox{Cyan!0.000}{\strut  connect} \colorbox{Cyan!0.000}{\strut  with} \colorbox{Cyan!0.000}{\strut  him} \colorbox{Cyan!0.000}{\strut .} \colorbox{Cyan!0.000}{\strut  I} \colorbox{Cyan!98.297}{\strut  couldn} \\
Input SAE & \num{1.977e+01} & \colorbox{Green!0.000}{\strut  breakdown} \colorbox{Green!0.000}{\strut s} \colorbox{Green!0.000}{\strut  that} \colorbox{Green!0.000}{\strut  day} \colorbox{Green!0.000}{\strut  because} \colorbox{Green!0.000}{\strut  I} \colorbox{Green!85.694}{\strut  couldn} \colorbox{Green!0.000}{\strut \textquotesingle{}} \colorbox{Green!0.000}{\strut t} \colorbox{Green!0.000}{\strut  connect} \colorbox{Green!0.000}{\strut  with} \colorbox{Green!0.000}{\strut  him} \colorbox{Green!0.000}{\strut .} \colorbox{Green!0.000}{\strut  I} \colorbox{Green!69.860}{\strut  couldn} \\
Output SAE & \num{6.820e+00} & \colorbox{Magenta!0.000}{\strut  breakdown} \colorbox{Magenta!0.000}{\strut s} \colorbox{Magenta!0.000}{\strut  that} \colorbox{Magenta!0.000}{\strut  day} \colorbox{Magenta!0.000}{\strut  because} \colorbox{Magenta!0.000}{\strut  I} \colorbox{Magenta!91.548}{\strut  couldn} \colorbox{Magenta!0.000}{\strut \textquotesingle{}} \colorbox{Magenta!0.000}{\strut t} \colorbox{Magenta!0.000}{\strut  connect} \colorbox{Magenta!0.000}{\strut  with} \colorbox{Magenta!0.000}{\strut  him} \colorbox{Magenta!0.000}{\strut .} \colorbox{Magenta!0.000}{\strut  I} \colorbox{Magenta!77.859}{\strut  couldn} \\
\midrule
Jacobian & \num{3.907e-01} & \colorbox{Cyan!0.000}{\strut  you} \colorbox{Cyan!0.000}{\strut .} \colorbox{Cyan!0.000}{\strut  Kot} \colorbox{Cyan!0.000}{\strut lin} \colorbox{Cyan!0.000}{\strut  is} \colorbox{Cyan!0.000}{\strut  like} \colorbox{Cyan!0.000}{\strut  a} \colorbox{Cyan!0.000}{\strut  Java} \colorbox{Cyan!0.000}{\strut  cousin} \colorbox{Cyan!0.000}{\strut  with} \colorbox{Cyan!0.000}{\strut  better} \colorbox{Cyan!0.000}{\strut  looks} \colorbox{Cyan!0.000}{\strut  and} \colorbox{Cyan!0.000}{\strut  who} \colorbox{Cyan!97.841}{\strut  doesn} \\
Input SAE & \num{2.208e+01} & \colorbox{Green!0.000}{\strut  you} \colorbox{Green!0.000}{\strut .} \colorbox{Green!0.000}{\strut  Kot} \colorbox{Green!0.000}{\strut lin} \colorbox{Green!0.000}{\strut  is} \colorbox{Green!0.000}{\strut  like} \colorbox{Green!0.000}{\strut  a} \colorbox{Green!0.000}{\strut  Java} \colorbox{Green!0.000}{\strut  cousin} \colorbox{Green!0.000}{\strut  with} \colorbox{Green!0.000}{\strut  better} \colorbox{Green!0.000}{\strut  looks} \colorbox{Green!0.000}{\strut  and} \colorbox{Green!0.000}{\strut  who} \colorbox{Green!95.694}{\strut  doesn} \\
Output SAE & \num{7.360e+00} & \colorbox{Magenta!0.000}{\strut  you} \colorbox{Magenta!0.000}{\strut .} \colorbox{Magenta!0.000}{\strut  Kot} \colorbox{Magenta!0.000}{\strut lin} \colorbox{Magenta!0.000}{\strut  is} \colorbox{Magenta!0.000}{\strut  like} \colorbox{Magenta!0.000}{\strut  a} \colorbox{Magenta!0.000}{\strut  Java} \colorbox{Magenta!0.000}{\strut  cousin} \colorbox{Magenta!0.000}{\strut  with} \colorbox{Magenta!0.000}{\strut  better} \colorbox{Magenta!0.000}{\strut  looks} \colorbox{Magenta!0.000}{\strut  and} \colorbox{Magenta!0.000}{\strut  who} \colorbox{Magenta!98.803}{\strut  doesn} \\
\midrule
Jacobian & \num{3.906e-01} & \colorbox{Cyan!0.000}{\strut  found} \colorbox{Cyan!0.000}{\strut  a} \colorbox{Cyan!0.000}{\strut  nice} \colorbox{Cyan!0.000}{\strut  copy} \colorbox{Cyan!0.000}{\strut  used} \colorbox{Cyan!0.000}{\strut .} \colorbox{Cyan!0.000}{\strut  Unfortunately} \colorbox{Cyan!0.000}{\strut ,} \colorbox{Cyan!0.000}{\strut  I} \colorbox{Cyan!97.812}{\strut  didn} \colorbox{Cyan!0.000}{\strut \textquotesingle{}} \colorbox{Cyan!0.000}{\strut t} \colorbox{Cyan!0.000}{\strut  like} \colorbox{Cyan!0.000}{\strut  it} \colorbox{Cyan!0.000}{\strut  very} \\
Input SAE & \num{2.139e+01} & \colorbox{Green!0.000}{\strut  found} \colorbox{Green!0.000}{\strut  a} \colorbox{Green!0.000}{\strut  nice} \colorbox{Green!0.000}{\strut  copy} \colorbox{Green!0.000}{\strut  used} \colorbox{Green!0.000}{\strut .} \colorbox{Green!0.000}{\strut  Unfortunately} \colorbox{Green!0.000}{\strut ,} \colorbox{Green!0.000}{\strut  I} \colorbox{Green!92.685}{\strut  didn} \colorbox{Green!0.000}{\strut \textquotesingle{}} \colorbox{Green!0.000}{\strut t} \colorbox{Green!0.000}{\strut  like} \colorbox{Green!0.000}{\strut  it} \colorbox{Green!0.000}{\strut  very} \\
Output SAE & \num{7.267e+00} & \colorbox{Magenta!0.000}{\strut  found} \colorbox{Magenta!0.000}{\strut  a} \colorbox{Magenta!0.000}{\strut  nice} \colorbox{Magenta!0.000}{\strut  copy} \colorbox{Magenta!0.000}{\strut  used} \colorbox{Magenta!0.000}{\strut .} \colorbox{Magenta!0.000}{\strut  Unfortunately} \colorbox{Magenta!0.000}{\strut ,} \colorbox{Magenta!0.000}{\strut  I} \colorbox{Magenta!97.544}{\strut  didn} \colorbox{Magenta!0.000}{\strut \textquotesingle{}} \colorbox{Magenta!0.000}{\strut t} \colorbox{Magenta!0.000}{\strut  like} \colorbox{Magenta!0.000}{\strut  it} \colorbox{Magenta!0.000}{\strut  very} \\
\midrule
Jacobian & \num{3.902e-01} & \colorbox{Cyan!0.000}{\strut  the} \colorbox{Cyan!0.000}{\strut  story} \colorbox{Cyan!0.000}{\strut  they} \colorbox{Cyan!0.000}{\strut  told} \colorbox{Cyan!0.000}{\strut  themselves} \colorbox{Cyan!0.000}{\strut  about} \colorbox{Cyan!0.000}{\strut  their} \colorbox{Cyan!0.000}{\strut  failures} \colorbox{Cyan!0.000}{\strut  was} \colorbox{Cyan!0.000}{\strut ,} \colorbox{Cyan!0.000}{\strut  \textquotedbl{}} \colorbox{Cyan!0.000}{\strut well} \colorbox{Cyan!0.000}{\strut aG} \colorbox{Cyan!0.000}{\strut  I} \colorbox{Cyan!97.720}{\strut  couldn} \\
Input SAE & \num{2.136e+01} & \colorbox{Green!0.000}{\strut  the} \colorbox{Green!0.000}{\strut  story} \colorbox{Green!0.000}{\strut  they} \colorbox{Green!0.000}{\strut  told} \colorbox{Green!0.000}{\strut  themselves} \colorbox{Green!0.000}{\strut  about} \colorbox{Green!0.000}{\strut  their} \colorbox{Green!0.000}{\strut  failures} \colorbox{Green!0.000}{\strut  was} \colorbox{Green!0.000}{\strut ,} \colorbox{Green!0.000}{\strut  \textquotedbl{}} \colorbox{Green!0.000}{\strut well} \colorbox{Green!0.000}{\strut aG} \colorbox{Green!0.000}{\strut  I} \colorbox{Green!92.559}{\strut  couldn} \\
Output SAE & \num{7.384e+00} & \colorbox{Magenta!0.000}{\strut  the} \colorbox{Magenta!0.000}{\strut  story} \colorbox{Magenta!0.000}{\strut  they} \colorbox{Magenta!0.000}{\strut  told} \colorbox{Magenta!0.000}{\strut  themselves} \colorbox{Magenta!0.000}{\strut  about} \colorbox{Magenta!0.000}{\strut  their} \colorbox{Magenta!0.000}{\strut  failures} \colorbox{Magenta!0.000}{\strut  was} \colorbox{Magenta!0.000}{\strut ,} \colorbox{Magenta!0.000}{\strut  \textquotedbl{}} \colorbox{Magenta!0.000}{\strut well} \colorbox{Magenta!0.000}{\strut aG} \colorbox{Magenta!0.000}{\strut  I} \colorbox{Magenta!99.116}{\strut  couldn} \\
\midrule
Jacobian & \num{3.894e-01} & \colorbox{Cyan!0.000}{\strut .} \colorbox{Cyan!0.000}{\strut  But} \colorbox{Cyan!0.000}{\strut  the} \colorbox{Cyan!0.000}{\strut  last} \colorbox{Cyan!0.000}{\strut  50} \colorbox{Cyan!0.000}{\strut  pages} \colorbox{Cyan!0.000}{\strut  or} \colorbox{Cyan!0.000}{\strut  so} \colorbox{Cyan!0.000}{\strut  are} \colorbox{Cyan!0.000}{\strut  brilliant} \colorbox{Cyan!0.000}{\strut .} \colorbox{Cyan!0.000}{\strut  I} \colorbox{Cyan!97.508}{\strut  couldn} \colorbox{Cyan!0.000}{\strut \textquotesingle{}} \colorbox{Cyan!0.000}{\strut t} \\
Input SAE & \num{1.966e+01} & \colorbox{Green!0.000}{\strut .} \colorbox{Green!0.000}{\strut  But} \colorbox{Green!0.000}{\strut  the} \colorbox{Green!0.000}{\strut  last} \colorbox{Green!0.000}{\strut  50} \colorbox{Green!0.000}{\strut  pages} \colorbox{Green!0.000}{\strut  or} \colorbox{Green!0.000}{\strut  so} \colorbox{Green!0.000}{\strut  are} \colorbox{Green!0.000}{\strut  brilliant} \colorbox{Green!0.000}{\strut .} \colorbox{Green!0.000}{\strut  I} \colorbox{Green!85.182}{\strut  couldn} \colorbox{Green!0.000}{\strut \textquotesingle{}} \colorbox{Green!0.000}{\strut t} \\
Output SAE & \num{6.700e+00} & \colorbox{Magenta!0.000}{\strut .} \colorbox{Magenta!0.000}{\strut  But} \colorbox{Magenta!0.000}{\strut  the} \colorbox{Magenta!0.000}{\strut  last} \colorbox{Magenta!0.000}{\strut  50} \colorbox{Magenta!0.000}{\strut  pages} \colorbox{Magenta!0.000}{\strut  or} \colorbox{Magenta!0.000}{\strut  so} \colorbox{Magenta!0.000}{\strut  are} \colorbox{Magenta!0.000}{\strut  brilliant} \colorbox{Magenta!0.000}{\strut .} \colorbox{Magenta!0.000}{\strut  I} \colorbox{Magenta!89.940}{\strut  couldn} \colorbox{Magenta!7.683}{\strut \textquotesingle{}} \colorbox{Magenta!0.000}{\strut t} \\
\midrule
Jacobian & \num{3.890e-01} & \colorbox{Cyan!0.000}{\strut  reminding} \colorbox{Cyan!0.000}{\strut  me} \colorbox{Cyan!0.000}{\strut  that} \colorbox{Cyan!0.000}{\strut  this} \colorbox{Cyan!0.000}{\strut  is} \colorbox{Cyan!0.000}{\strut  technical} \colorbox{Cyan!0.000}{\strut  manual} \colorbox{Cyan!0.000}{\strut  and} \colorbox{Cyan!0.000}{\strut  that} \colorbox{Cyan!0.000}{\strut  it} \colorbox{Cyan!97.415}{\strut  didn} \colorbox{Cyan!0.000}{\strut \textquotesingle{}t} \colorbox{Cyan!0.000}{\strut  need} \colorbox{Cyan!0.000}{\strut  to} \colorbox{Cyan!0.000}{\strut  be} \\
Input SAE & \num{2.161e+01} & \colorbox{Green!0.000}{\strut  reminding} \colorbox{Green!0.000}{\strut  me} \colorbox{Green!0.000}{\strut  that} \colorbox{Green!0.000}{\strut  this} \colorbox{Green!0.000}{\strut  is} \colorbox{Green!0.000}{\strut  technical} \colorbox{Green!0.000}{\strut  manual} \colorbox{Green!0.000}{\strut  and} \colorbox{Green!0.000}{\strut  that} \colorbox{Green!0.000}{\strut  it} \colorbox{Green!93.634}{\strut  didn} \colorbox{Green!0.000}{\strut \textquotesingle{}t} \colorbox{Green!0.000}{\strut  need} \colorbox{Green!0.000}{\strut  to} \colorbox{Green!0.000}{\strut  be} \\
Output SAE & \num{7.011e+00} & \colorbox{Magenta!0.000}{\strut  reminding} \colorbox{Magenta!0.000}{\strut  me} \colorbox{Magenta!0.000}{\strut  that} \colorbox{Magenta!0.000}{\strut  this} \colorbox{Magenta!0.000}{\strut  is} \colorbox{Magenta!0.000}{\strut  technical} \colorbox{Magenta!0.000}{\strut  manual} \colorbox{Magenta!0.000}{\strut  and} \colorbox{Magenta!0.000}{\strut  that} \colorbox{Magenta!0.000}{\strut  it} \colorbox{Magenta!94.112}{\strut  didn} \colorbox{Magenta!0.000}{\strut \textquotesingle{}t} \colorbox{Magenta!0.000}{\strut  need} \colorbox{Magenta!0.000}{\strut  to} \colorbox{Magenta!0.000}{\strut  be} \\
\midrule
Jacobian & \num{3.889e-01} & \colorbox{Cyan!0.000}{\strut  and} \colorbox{Cyan!0.000}{\strut  look} \colorbox{Cyan!0.000}{\strut  forward} \colorbox{Cyan!0.000}{\strut  to} \colorbox{Cyan!0.000}{\strut  many} \colorbox{Cyan!0.000}{\strut  more} \colorbox{Cyan!0.000}{\strut .} \colorbox{Cyan!73.883}{\strut  Don} \colorbox{Cyan!0.000}{\strut \textquotesingle{}} \colorbox{Cyan!0.000}{\strut t} \colorbox{Cyan!0.000}{\strut  tell} \colorbox{Cyan!0.000}{\strut  me} \colorbox{Cyan!0.000}{\strut  that} \colorbox{Cyan!0.000}{\strut  diversity} \colorbox{Cyan!97.390}{\strut  doesn} \\
Input SAE & \num{1.910e+01} & \colorbox{Green!0.000}{\strut  and} \colorbox{Green!0.000}{\strut  look} \colorbox{Green!0.000}{\strut  forward} \colorbox{Green!0.000}{\strut  to} \colorbox{Green!0.000}{\strut  many} \colorbox{Green!0.000}{\strut  more} \colorbox{Green!0.000}{\strut .} \colorbox{Green!61.899}{\strut  Don} \colorbox{Green!0.000}{\strut \textquotesingle{}} \colorbox{Green!0.000}{\strut t} \colorbox{Green!0.000}{\strut  tell} \colorbox{Green!0.000}{\strut  me} \colorbox{Green!0.000}{\strut  that} \colorbox{Green!0.000}{\strut  diversity} \colorbox{Green!82.787}{\strut  doesn} \\
Output SAE & \num{6.600e+00} & \colorbox{Magenta!0.000}{\strut  and} \colorbox{Magenta!0.000}{\strut  look} \colorbox{Magenta!0.000}{\strut  forward} \colorbox{Magenta!0.000}{\strut  to} \colorbox{Magenta!0.000}{\strut  many} \colorbox{Magenta!0.000}{\strut  more} \colorbox{Magenta!0.000}{\strut .} \colorbox{Magenta!51.788}{\strut  Don} \colorbox{Magenta!0.000}{\strut \textquotesingle{}} \colorbox{Magenta!0.000}{\strut t} \colorbox{Magenta!0.000}{\strut  tell} \colorbox{Magenta!0.000}{\strut  me} \colorbox{Magenta!0.000}{\strut  that} \colorbox{Magenta!0.000}{\strut  diversity} \colorbox{Magenta!88.593}{\strut  doesn} \\
\bottomrule
\end{tabular}
% feature pairs/Layer15-65536-J1-LR5.0e-04-k32-T3.0e+08 abs mean/examples-24720-v-33709 stas c4-en-10k,train,batch size=32,ctx len=16.csv
\caption{
The top $12$ examples that produce the maximum absolute values of the Jacobian element with input SAE latent index $24720$ and output latent index $33709$.
% This pair of latent indices is one of the top $5$ pairs by the mean absolute value of non-zero Jacobian elements.
The Jacobian SAE pair was trained on layer 15 of Pythia-410m with an expansion factor of $R=64$ and sparsity $k=32$.
The examples were collected over the first 10K records of the English subset of the C4 text dataset \citep{raffel_exploring_2020}, with a context length of $16$ tokens.
For each example, the first row shows the values of the Jacobian element, and the second and third show the corresponding activations of the input and output SAE latents.
In this case, both SAE latents appear to activate for tokens immediately preceding \texttt{'t} that form negative contractions in English.
}
\label{tab:feature_pairs_24720_33709}
\end{table} % didn, isn, couldn
% \input{feature_pairs_tex/pythia-410m-layer-15-abs-mean-in-53225-out-53924-b32-t16} % at...
\begin{table}
\centering
\begin{tabular}{lrl}
\toprule
Category & Max. abs. value & Example tokens \\
\midrule
Jacobian & \num{3.349e-01} & \colorbox{Cyan!0.000}{\strut  confirm} \colorbox{Cyan!0.000}{\strut  that} \colorbox{Cyan!100.000}{\strut  there} \colorbox{Cyan!0.000}{\strut  was} \colorbox{Cyan!0.000}{\strut  nothing} \colorbox{Cyan!0.000}{\strut  else} \colorbox{Cyan!0.000}{\strut  it} \colorbox{Cyan!0.000}{\strut  needed} \colorbox{Cyan!0.000}{\strut  to} \colorbox{Cyan!0.000}{\strut  disclose} \colorbox{Cyan!0.000}{\strut .\textquotedbl{}} \colorbox{Cyan!0.000}{\strut  B} \colorbox{Cyan!0.000}{\strut LC} \colorbox{Cyan!0.000}{\strut  Press} \colorbox{Cyan!0.000}{\strut  Release} \\
Input SAE & \num{5.743e-01} & \colorbox{Green!0.000}{\strut  confirm} \colorbox{Green!0.000}{\strut  that} \colorbox{Green!6.284}{\strut  there} \colorbox{Green!0.000}{\strut  was} \colorbox{Green!0.000}{\strut  nothing} \colorbox{Green!0.000}{\strut  else} \colorbox{Green!0.000}{\strut  it} \colorbox{Green!0.000}{\strut  needed} \colorbox{Green!0.000}{\strut  to} \colorbox{Green!0.000}{\strut  disclose} \colorbox{Green!0.000}{\strut .\textquotedbl{}} \colorbox{Green!0.000}{\strut  B} \colorbox{Green!0.000}{\strut LC} \colorbox{Green!0.000}{\strut  Press} \colorbox{Green!0.000}{\strut  Release} \\
Output SAE & \num{4.708e+00} & \colorbox{Magenta!0.000}{\strut  confirm} \colorbox{Magenta!0.000}{\strut  that} \colorbox{Magenta!89.506}{\strut  there} \colorbox{Magenta!0.000}{\strut  was} \colorbox{Magenta!0.000}{\strut  nothing} \colorbox{Magenta!0.000}{\strut  else} \colorbox{Magenta!0.000}{\strut  it} \colorbox{Magenta!0.000}{\strut  needed} \colorbox{Magenta!0.000}{\strut  to} \colorbox{Magenta!0.000}{\strut  disclose} \colorbox{Magenta!0.000}{\strut .\textquotedbl{}} \colorbox{Magenta!0.000}{\strut  B} \colorbox{Magenta!0.000}{\strut LC} \colorbox{Magenta!0.000}{\strut  Press} \colorbox{Magenta!0.000}{\strut  Release} \\
\midrule
Jacobian & \num{3.348e-01} & \colorbox{Cyan!0.000}{\strut Example} \colorbox{Cyan!0.000}{\strut :} \colorbox{Cyan!0.000}{\strut  The} \colorbox{Cyan!0.000}{\strut  integral} \colorbox{Cyan!0.000}{\strut  electronic} \colorbox{Cyan!0.000}{\strut  control} \colorbox{Cyan!0.000}{\strut  unit} \colorbox{Cyan!0.000}{\strut  ver} \colorbox{Cyan!0.000}{\strut ifies} \colorbox{Cyan!0.000}{\strut  whether} \colorbox{Cyan!99.966}{\strut  there} \colorbox{Cyan!0.000}{\strut  is} \colorbox{Cyan!76.380}{\strut  indeed} \colorbox{Cyan!0.000}{\strut  a} \colorbox{Cyan!0.000}{\strut  sol} \\
Input SAE & \num{2.363e+00} & \colorbox{Green!0.000}{\strut Example} \colorbox{Green!0.000}{\strut :} \colorbox{Green!0.000}{\strut  The} \colorbox{Green!0.000}{\strut  integral} \colorbox{Green!0.000}{\strut  electronic} \colorbox{Green!0.000}{\strut  control} \colorbox{Green!0.000}{\strut  unit} \colorbox{Green!0.000}{\strut  ver} \colorbox{Green!0.000}{\strut ifies} \colorbox{Green!0.000}{\strut  whether} \colorbox{Green!8.733}{\strut  there} \colorbox{Green!0.000}{\strut  is} \colorbox{Green!25.856}{\strut  indeed} \colorbox{Green!0.000}{\strut  a} \colorbox{Green!0.000}{\strut  sol} \\
Output SAE & \num{4.998e+00} & \colorbox{Magenta!0.000}{\strut Example} \colorbox{Magenta!0.000}{\strut :} \colorbox{Magenta!0.000}{\strut  The} \colorbox{Magenta!0.000}{\strut  integral} \colorbox{Magenta!0.000}{\strut  electronic} \colorbox{Magenta!0.000}{\strut  control} \colorbox{Magenta!0.000}{\strut  unit} \colorbox{Magenta!0.000}{\strut  ver} \colorbox{Magenta!0.000}{\strut ifies} \colorbox{Magenta!0.000}{\strut  whether} \colorbox{Magenta!95.020}{\strut  there} \colorbox{Magenta!0.000}{\strut  is} \colorbox{Magenta!16.019}{\strut  indeed} \colorbox{Magenta!0.000}{\strut  a} \colorbox{Magenta!0.000}{\strut  sol} \\
\midrule
Jacobian & \num{3.342e-01} & \colorbox{Cyan!0.000}{\strut  see} \colorbox{Cyan!0.000}{\strut  that} \colorbox{Cyan!99.783}{\strut  there} \colorbox{Cyan!0.000}{\strut  is} \colorbox{Cyan!0.000}{\strut  a} \colorbox{Cyan!0.000}{\strut  ton} \colorbox{Cyan!0.000}{\strut  of} \colorbox{Cyan!0.000}{\strut  old} \colorbox{Cyan!0.000}{\strut  fishing} \colorbox{Cyan!0.000}{\strut  line} \colorbox{Cyan!0.000}{\strut  out} \colorbox{Cyan!0.000}{\strut  there} \colorbox{Cyan!0.000}{\strut  floating} \colorbox{Cyan!0.000}{\strut  around} \colorbox{Cyan!0.000}{\strut .} \\
Input SAE & \num{5.289e-01} & \colorbox{Green!0.000}{\strut  see} \colorbox{Green!0.000}{\strut  that} \colorbox{Green!5.787}{\strut  there} \colorbox{Green!0.000}{\strut  is} \colorbox{Green!0.000}{\strut  a} \colorbox{Green!0.000}{\strut  ton} \colorbox{Green!0.000}{\strut  of} \colorbox{Green!0.000}{\strut  old} \colorbox{Green!0.000}{\strut  fishing} \colorbox{Green!0.000}{\strut  line} \colorbox{Green!0.000}{\strut  out} \colorbox{Green!0.000}{\strut  there} \colorbox{Green!0.000}{\strut  floating} \colorbox{Green!0.000}{\strut  around} \colorbox{Green!0.000}{\strut .} \\
Output SAE & \num{4.801e+00} & \colorbox{Magenta!0.000}{\strut  see} \colorbox{Magenta!0.000}{\strut  that} \colorbox{Magenta!91.277}{\strut  there} \colorbox{Magenta!0.000}{\strut  is} \colorbox{Magenta!0.000}{\strut  a} \colorbox{Magenta!0.000}{\strut  ton} \colorbox{Magenta!0.000}{\strut  of} \colorbox{Magenta!0.000}{\strut  old} \colorbox{Magenta!0.000}{\strut  fishing} \colorbox{Magenta!0.000}{\strut  line} \colorbox{Magenta!0.000}{\strut  out} \colorbox{Magenta!0.000}{\strut  there} \colorbox{Magenta!0.000}{\strut  floating} \colorbox{Magenta!0.000}{\strut  around} \colorbox{Magenta!0.000}{\strut .} \\
\midrule
Jacobian & \num{3.335e-01} & \colorbox{Cyan!0.000}{\strut .} \colorbox{Cyan!0.000}{\strut Maybe} \colorbox{Cyan!0.000}{\strut  you} \colorbox{Cyan!0.000}{\strut  won} \colorbox{Cyan!0.000}{\strut \textquotesingle{}} \colorbox{Cyan!0.000}{\strut t} \colorbox{Cyan!0.000}{\strut  believe} \colorbox{Cyan!0.000}{\strut  it} \colorbox{Cyan!0.000}{\strut ,} \colorbox{Cyan!0.000}{\strut  but} \colorbox{Cyan!0.000}{\strut  in} \colorbox{Cyan!0.000}{\strut  present} \colorbox{Cyan!0.000}{\strut ,} \colorbox{Cyan!99.563}{\strut  there} \\
Input SAE & \num{8.741e-01} & \colorbox{Green!0.000}{\strut .} \colorbox{Green!0.000}{\strut Maybe} \colorbox{Green!0.000}{\strut  you} \colorbox{Green!0.000}{\strut  won} \colorbox{Green!0.000}{\strut \textquotesingle{}} \colorbox{Green!0.000}{\strut t} \colorbox{Green!0.000}{\strut  believe} \colorbox{Green!0.000}{\strut  it} \colorbox{Green!0.000}{\strut ,} \colorbox{Green!0.000}{\strut  but} \colorbox{Green!0.000}{\strut  in} \colorbox{Green!0.000}{\strut  present} \colorbox{Green!0.000}{\strut ,} \colorbox{Green!9.563}{\strut  there} \\
Output SAE & \num{4.742e+00} & \colorbox{Magenta!0.000}{\strut .} \colorbox{Magenta!0.000}{\strut Maybe} \colorbox{Magenta!0.000}{\strut  you} \colorbox{Magenta!0.000}{\strut  won} \colorbox{Magenta!0.000}{\strut \textquotesingle{}} \colorbox{Magenta!0.000}{\strut t} \colorbox{Magenta!0.000}{\strut  believe} \colorbox{Magenta!0.000}{\strut  it} \colorbox{Magenta!0.000}{\strut ,} \colorbox{Magenta!0.000}{\strut  but} \colorbox{Magenta!0.000}{\strut  in} \colorbox{Magenta!0.000}{\strut  present} \colorbox{Magenta!0.000}{\strut ,} \colorbox{Magenta!90.145}{\strut  there} \\
\midrule
Jacobian & \num{3.334e-01} & \colorbox{Cyan!0.000}{\strut  virus} \colorbox{Cyan!0.000}{\strut  indicates} \colorbox{Cyan!0.000}{\strut  that} \colorbox{Cyan!99.533}{\strut  there} \colorbox{Cyan!93.573}{\strut  may} \colorbox{Cyan!0.000}{\strut  be} \colorbox{Cyan!0.000}{\strut  important} \colorbox{Cyan!0.000}{\strut  amino} \colorbox{Cyan!0.000}{\strut  acid} \colorbox{Cyan!0.000}{\strut  co} \colorbox{Cyan!0.000}{\strut -} \colorbox{Cyan!0.000}{\strut sub} \colorbox{Cyan!0.000}{\strut stit} \colorbox{Cyan!0.000}{\strut utions} \colorbox{Cyan!0.000}{\strut  in} \\
Input SAE & \num{7.625e+00} & \colorbox{Green!0.000}{\strut  virus} \colorbox{Green!0.000}{\strut  indicates} \colorbox{Green!0.000}{\strut  that} \colorbox{Green!4.797}{\strut  there} \colorbox{Green!83.421}{\strut  may} \colorbox{Green!0.000}{\strut  be} \colorbox{Green!0.000}{\strut  important} \colorbox{Green!0.000}{\strut  amino} \colorbox{Green!0.000}{\strut  acid} \colorbox{Green!0.000}{\strut  co} \colorbox{Green!0.000}{\strut -} \colorbox{Green!0.000}{\strut sub} \colorbox{Green!0.000}{\strut stit} \colorbox{Green!0.000}{\strut utions} \colorbox{Green!0.000}{\strut  in} \\
Output SAE & \num{4.617e+00} & \colorbox{Magenta!0.000}{\strut  virus} \colorbox{Magenta!0.000}{\strut  indicates} \colorbox{Magenta!0.000}{\strut  that} \colorbox{Magenta!87.774}{\strut  there} \colorbox{Magenta!63.440}{\strut  may} \colorbox{Magenta!0.000}{\strut  be} \colorbox{Magenta!0.000}{\strut  important} \colorbox{Magenta!0.000}{\strut  amino} \colorbox{Magenta!0.000}{\strut  acid} \colorbox{Magenta!0.000}{\strut  co} \colorbox{Magenta!0.000}{\strut -} \colorbox{Magenta!0.000}{\strut sub} \colorbox{Magenta!0.000}{\strut stit} \colorbox{Magenta!0.000}{\strut utions} \colorbox{Magenta!0.000}{\strut  in} \\
\midrule
Jacobian & \num{3.333e-01} & \colorbox{Cyan!0.000}{\strut  ensure} \colorbox{Cyan!0.000}{\strut  that} \colorbox{Cyan!99.529}{\strut  there} \colorbox{Cyan!92.465}{\strut  would} \colorbox{Cyan!0.000}{\strut  be} \colorbox{Cyan!0.000}{\strut  a} \colorbox{Cyan!0.000}{\strut  sufficient} \colorbox{Cyan!0.000}{\strut  reserve} \colorbox{Cyan!0.000}{\strut  to} \colorbox{Cyan!0.000}{\strut  avoid} \colorbox{Cyan!0.000}{\strut  un} \colorbox{Cyan!0.000}{\strut -} \colorbox{Cyan!0.000}{\strut staff} \colorbox{Cyan!0.000}{\strut ed} \colorbox{Cyan!0.000}{\strut  routes} \\
Input SAE & \num{7.634e+00} & \colorbox{Green!0.000}{\strut  ensure} \colorbox{Green!0.000}{\strut  that} \colorbox{Green!8.930}{\strut  there} \colorbox{Green!83.525}{\strut  would} \colorbox{Green!0.000}{\strut  be} \colorbox{Green!0.000}{\strut  a} \colorbox{Green!0.000}{\strut  sufficient} \colorbox{Green!0.000}{\strut  reserve} \colorbox{Green!0.000}{\strut  to} \colorbox{Green!0.000}{\strut  avoid} \colorbox{Green!0.000}{\strut  un} \colorbox{Green!0.000}{\strut -} \colorbox{Green!0.000}{\strut staff} \colorbox{Green!0.000}{\strut ed} \colorbox{Green!0.000}{\strut  routes} \\
Output SAE & \num{4.676e+00} & \colorbox{Magenta!0.000}{\strut  ensure} \colorbox{Magenta!0.000}{\strut  that} \colorbox{Magenta!88.888}{\strut  there} \colorbox{Magenta!64.098}{\strut  would} \colorbox{Magenta!0.000}{\strut  be} \colorbox{Magenta!0.000}{\strut  a} \colorbox{Magenta!0.000}{\strut  sufficient} \colorbox{Magenta!0.000}{\strut  reserve} \colorbox{Magenta!0.000}{\strut  to} \colorbox{Magenta!0.000}{\strut  avoid} \colorbox{Magenta!0.000}{\strut  un} \colorbox{Magenta!0.000}{\strut -} \colorbox{Magenta!0.000}{\strut staff} \colorbox{Magenta!0.000}{\strut ed} \colorbox{Magenta!0.000}{\strut  routes} \\
\midrule
Jacobian & \num{3.327e-01} & \colorbox{Cyan!0.000}{\strut  wire} \colorbox{Cyan!0.000}{\strut  cage} \colorbox{Cyan!0.000}{\strut ,} \colorbox{Cyan!0.000}{\strut  ensure} \colorbox{Cyan!0.000}{\strut  that} \colorbox{Cyan!99.329}{\strut  there} \colorbox{Cyan!0.000}{\strut  are} \colorbox{Cyan!0.000}{\strut  no} \colorbox{Cyan!0.000}{\strut  bits} \colorbox{Cyan!0.000}{\strut  of} \colorbox{Cyan!0.000}{\strut  wire} \colorbox{Cyan!0.000}{\strut  p} \colorbox{Cyan!0.000}{\strut oking} \colorbox{Cyan!0.000}{\strut  out} \colorbox{Cyan!0.000}{\strut  that} \\
Input SAE & \num{7.182e-01} & \colorbox{Green!0.000}{\strut  wire} \colorbox{Green!0.000}{\strut  cage} \colorbox{Green!0.000}{\strut ,} \colorbox{Green!0.000}{\strut  ensure} \colorbox{Green!0.000}{\strut  that} \colorbox{Green!7.858}{\strut  there} \colorbox{Green!0.000}{\strut  are} \colorbox{Green!0.000}{\strut  no} \colorbox{Green!0.000}{\strut  bits} \colorbox{Green!0.000}{\strut  of} \colorbox{Green!0.000}{\strut  wire} \colorbox{Green!0.000}{\strut  p} \colorbox{Green!0.000}{\strut oking} \colorbox{Green!0.000}{\strut  out} \colorbox{Green!0.000}{\strut  that} \\
Output SAE & \num{4.662e+00} & \colorbox{Magenta!0.000}{\strut  wire} \colorbox{Magenta!0.000}{\strut  cage} \colorbox{Magenta!0.000}{\strut ,} \colorbox{Magenta!0.000}{\strut  ensure} \colorbox{Magenta!0.000}{\strut  that} \colorbox{Magenta!88.625}{\strut  there} \colorbox{Magenta!13.242}{\strut  are} \colorbox{Magenta!0.000}{\strut  no} \colorbox{Magenta!0.000}{\strut  bits} \colorbox{Magenta!0.000}{\strut  of} \colorbox{Magenta!0.000}{\strut  wire} \colorbox{Magenta!0.000}{\strut  p} \colorbox{Magenta!0.000}{\strut oking} \colorbox{Magenta!0.000}{\strut  out} \colorbox{Magenta!0.000}{\strut  that} \\
\midrule
Jacobian & \num{3.326e-01} & \colorbox{Cyan!0.000}{\strut  it} \colorbox{Cyan!0.000}{\strut  is} \colorbox{Cyan!0.000}{\strut  important} \colorbox{Cyan!0.000}{\strut  to} \colorbox{Cyan!0.000}{\strut  study} \colorbox{Cyan!0.000}{\strut  if} \colorbox{Cyan!99.300}{\strut  there} \colorbox{Cyan!0.000}{\strut  is} \colorbox{Cyan!0.000}{\strut  one} \colorbox{Cyan!0.000}{\strut  approach} \colorbox{Cyan!0.000}{\strut  that} \colorbox{Cyan!0.000}{\strut  you} \colorbox{Cyan!0.000}{\strut  use} \colorbox{Cyan!0.000}{\strut  more} \colorbox{Cyan!0.000}{\strut  often} \\
Input SAE & \num{6.788e-01} & \colorbox{Green!0.000}{\strut  it} \colorbox{Green!0.000}{\strut  is} \colorbox{Green!0.000}{\strut  important} \colorbox{Green!0.000}{\strut  to} \colorbox{Green!0.000}{\strut  study} \colorbox{Green!0.000}{\strut  if} \colorbox{Green!7.426}{\strut  there} \colorbox{Green!0.000}{\strut  is} \colorbox{Green!0.000}{\strut  one} \colorbox{Green!0.000}{\strut  approach} \colorbox{Green!0.000}{\strut  that} \colorbox{Green!0.000}{\strut  you} \colorbox{Green!0.000}{\strut  use} \colorbox{Green!0.000}{\strut  more} \colorbox{Green!0.000}{\strut  often} \\
Output SAE & \num{4.480e+00} & \colorbox{Magenta!0.000}{\strut  it} \colorbox{Magenta!0.000}{\strut  is} \colorbox{Magenta!0.000}{\strut  important} \colorbox{Magenta!0.000}{\strut  to} \colorbox{Magenta!0.000}{\strut  study} \colorbox{Magenta!0.000}{\strut  if} \colorbox{Magenta!85.177}{\strut  there} \colorbox{Magenta!0.000}{\strut  is} \colorbox{Magenta!0.000}{\strut  one} \colorbox{Magenta!0.000}{\strut  approach} \colorbox{Magenta!10.792}{\strut  that} \colorbox{Magenta!0.000}{\strut  you} \colorbox{Magenta!0.000}{\strut  use} \colorbox{Magenta!0.000}{\strut  more} \colorbox{Magenta!0.000}{\strut  often} \\
\midrule
Jacobian & \num{3.322e-01} & \colorbox{Cyan!0.000}{\strut ,} \colorbox{Cyan!0.000}{\strut  that} \colorbox{Cyan!99.185}{\strut  there} \colorbox{Cyan!0.000}{\strut  is} \colorbox{Cyan!0.000}{\strut  no} \colorbox{Cyan!0.000}{\strut  point} \colorbox{Cyan!0.000}{\strut  in} \colorbox{Cyan!0.000}{\strut  trying} \colorbox{Cyan!0.000}{\strut  to} \colorbox{Cyan!0.000}{\strut  understand} \colorbox{Cyan!0.000}{\strut  any} \colorbox{Cyan!0.000}{\strut  users} \colorbox{Cyan!0.000}{\strut  at} \colorbox{Cyan!0.000}{\strut  all} \colorbox{Cyan!0.000}{\strut .} \\
Input SAE & \num{4.728e-01} & \colorbox{Green!0.000}{\strut ,} \colorbox{Green!0.000}{\strut  that} \colorbox{Green!5.173}{\strut  there} \colorbox{Green!0.000}{\strut  is} \colorbox{Green!0.000}{\strut  no} \colorbox{Green!0.000}{\strut  point} \colorbox{Green!0.000}{\strut  in} \colorbox{Green!0.000}{\strut  trying} \colorbox{Green!0.000}{\strut  to} \colorbox{Green!0.000}{\strut  understand} \colorbox{Green!0.000}{\strut  any} \colorbox{Green!0.000}{\strut  users} \colorbox{Green!0.000}{\strut  at} \colorbox{Green!0.000}{\strut  all} \colorbox{Green!0.000}{\strut .} \\
Output SAE & \num{4.607e+00} & \colorbox{Magenta!0.000}{\strut ,} \colorbox{Magenta!0.000}{\strut  that} \colorbox{Magenta!87.576}{\strut  there} \colorbox{Magenta!0.000}{\strut  is} \colorbox{Magenta!0.000}{\strut  no} \colorbox{Magenta!0.000}{\strut  point} \colorbox{Magenta!0.000}{\strut  in} \colorbox{Magenta!0.000}{\strut  trying} \colorbox{Magenta!0.000}{\strut  to} \colorbox{Magenta!0.000}{\strut  understand} \colorbox{Magenta!0.000}{\strut  any} \colorbox{Magenta!0.000}{\strut  users} \colorbox{Magenta!0.000}{\strut  at} \colorbox{Magenta!0.000}{\strut  all} \colorbox{Magenta!0.000}{\strut .} \\
\midrule
Jacobian & \num{3.318e-01} & \colorbox{Cyan!0.000}{\strut Now} \colorbox{Cyan!0.000}{\strut  that} \colorbox{Cyan!99.075}{\strut  there} \colorbox{Cyan!0.000}{\strut  is} \colorbox{Cyan!0.000}{\strut  known} \colorbox{Cyan!0.000}{\strut  information} \colorbox{Cyan!0.000}{\strut ,} \colorbox{Cyan!0.000}{\strut  everything} \colorbox{Cyan!0.000}{\strut  feels} \colorbox{Cyan!0.000}{\strut  like} \colorbox{Cyan!0.000}{\strut  it} \colorbox{Cyan!0.000}{\strut  is} \colorbox{Cyan!0.000}{\strut  within} \colorbox{Cyan!0.000}{\strut  reach} \colorbox{Cyan!0.000}{\strut .} \\
Input SAE & \num{9.027e-01} & \colorbox{Green!0.000}{\strut Now} \colorbox{Green!0.000}{\strut  that} \colorbox{Green!9.876}{\strut  there} \colorbox{Green!0.000}{\strut  is} \colorbox{Green!0.000}{\strut  known} \colorbox{Green!0.000}{\strut  information} \colorbox{Green!0.000}{\strut ,} \colorbox{Green!0.000}{\strut  everything} \colorbox{Green!0.000}{\strut  feels} \colorbox{Green!0.000}{\strut  like} \colorbox{Green!0.000}{\strut  it} \colorbox{Green!0.000}{\strut  is} \colorbox{Green!0.000}{\strut  within} \colorbox{Green!0.000}{\strut  reach} \colorbox{Green!0.000}{\strut .} \\
Output SAE & \num{4.570e+00} & \colorbox{Magenta!0.000}{\strut Now} \colorbox{Magenta!0.000}{\strut  that} \colorbox{Magenta!86.878}{\strut  there} \colorbox{Magenta!0.000}{\strut  is} \colorbox{Magenta!0.000}{\strut  known} \colorbox{Magenta!0.000}{\strut  information} \colorbox{Magenta!0.000}{\strut ,} \colorbox{Magenta!0.000}{\strut  everything} \colorbox{Magenta!0.000}{\strut  feels} \colorbox{Magenta!0.000}{\strut  like} \colorbox{Magenta!0.000}{\strut  it} \colorbox{Magenta!0.000}{\strut  is} \colorbox{Magenta!0.000}{\strut  within} \colorbox{Magenta!0.000}{\strut  reach} \colorbox{Magenta!0.000}{\strut .} \\
\midrule
Jacobian & \num{3.315e-01} & \colorbox{Cyan!0.000}{\strut  seen} \colorbox{Cyan!0.000}{\strut  that} \colorbox{Cyan!98.980}{\strut  there} \colorbox{Cyan!0.000}{\strut  is} \colorbox{Cyan!0.000}{\strut  a} \colorbox{Cyan!0.000}{\strut  wealth} \colorbox{Cyan!0.000}{\strut  of} \colorbox{Cyan!0.000}{\strut  un} \colorbox{Cyan!0.000}{\strut ta} \colorbox{Cyan!0.000}{\strut pped} \colorbox{Cyan!0.000}{\strut  talent} \colorbox{Cyan!0.000}{\strut  here} \colorbox{Cyan!0.000}{\strut .} \colorbox{Cyan!0.000}{\strut Some} \\
Input SAE & \num{6.374e-01} & \colorbox{Green!0.000}{\strut  seen} \colorbox{Green!0.000}{\strut  that} \colorbox{Green!6.974}{\strut  there} \colorbox{Green!0.000}{\strut  is} \colorbox{Green!0.000}{\strut  a} \colorbox{Green!0.000}{\strut  wealth} \colorbox{Green!0.000}{\strut  of} \colorbox{Green!0.000}{\strut  un} \colorbox{Green!0.000}{\strut ta} \colorbox{Green!0.000}{\strut pped} \colorbox{Green!0.000}{\strut  talent} \colorbox{Green!0.000}{\strut  here} \colorbox{Green!0.000}{\strut .} \colorbox{Green!0.000}{\strut Some} \\
Output SAE & \num{4.643e+00} & \colorbox{Magenta!0.000}{\strut  seen} \colorbox{Magenta!0.000}{\strut  that} \colorbox{Magenta!88.276}{\strut  there} \colorbox{Magenta!0.000}{\strut  is} \colorbox{Magenta!0.000}{\strut  a} \colorbox{Magenta!0.000}{\strut  wealth} \colorbox{Magenta!0.000}{\strut  of} \colorbox{Magenta!0.000}{\strut  un} \colorbox{Magenta!0.000}{\strut ta} \colorbox{Magenta!0.000}{\strut pped} \colorbox{Magenta!0.000}{\strut  talent} \colorbox{Magenta!0.000}{\strut  here} \colorbox{Magenta!0.000}{\strut .} \colorbox{Magenta!0.000}{\strut Some} \\
\midrule
Jacobian & \num{3.315e-01} & \colorbox{Cyan!0.000}{\strut If} \colorbox{Cyan!98.976}{\strut  there} \colorbox{Cyan!0.000}{\strut \textquotesingle{}} \colorbox{Cyan!0.000}{\strut s} \colorbox{Cyan!0.000}{\strut  one} \colorbox{Cyan!0.000}{\strut  that} \colorbox{Cyan!0.000}{\strut  you} \colorbox{Cyan!0.000}{\strut  like} \colorbox{Cyan!0.000}{\strut ,} \colorbox{Cyan!0.000}{\strut  you} \colorbox{Cyan!0.000}{\strut  can} \colorbox{Cyan!0.000}{\strut  stick} \colorbox{Cyan!0.000}{\strut  your} \colorbox{Cyan!0.000}{\strut  face} \colorbox{Cyan!0.000}{\strut  through} \\
Input SAE & \num{5.826e-01} & \colorbox{Green!0.000}{\strut If} \colorbox{Green!6.374}{\strut  there} \colorbox{Green!0.000}{\strut \textquotesingle{}} \colorbox{Green!0.000}{\strut s} \colorbox{Green!0.000}{\strut  one} \colorbox{Green!0.000}{\strut  that} \colorbox{Green!0.000}{\strut  you} \colorbox{Green!0.000}{\strut  like} \colorbox{Green!0.000}{\strut ,} \colorbox{Green!0.000}{\strut  you} \colorbox{Green!0.000}{\strut  can} \colorbox{Green!0.000}{\strut  stick} \colorbox{Green!0.000}{\strut  your} \colorbox{Green!0.000}{\strut  face} \colorbox{Green!0.000}{\strut  through} \\
Output SAE & \num{4.453e+00} & \colorbox{Magenta!0.000}{\strut If} \colorbox{Magenta!84.665}{\strut  there} \colorbox{Magenta!0.000}{\strut \textquotesingle{}} \colorbox{Magenta!0.000}{\strut s} \colorbox{Magenta!0.000}{\strut  one} \colorbox{Magenta!0.000}{\strut  that} \colorbox{Magenta!0.000}{\strut  you} \colorbox{Magenta!0.000}{\strut  like} \colorbox{Magenta!0.000}{\strut ,} \colorbox{Magenta!0.000}{\strut  you} \colorbox{Magenta!0.000}{\strut  can} \colorbox{Magenta!0.000}{\strut  stick} \colorbox{Magenta!0.000}{\strut  your} \colorbox{Magenta!0.000}{\strut  face} \colorbox{Magenta!0.000}{\strut  through} \\
\bottomrule
\end{tabular}
% feature pairs/Layer15-65536-J1-LR5.0e-04-k32-T3.0e+08 abs mean/examples-314-v-31729 stas c4-en-10k,train,batch size=32,ctx len=16.csv
\caption{
The top $12$ examples that produce the maximum absolute values of the Jacobian element with input SAE latent index $314$ and output latent index $31729$.
% This pair of latent indices is one of the top $5$ pairs by the mean absolute value of non-zero Jacobian elements.
The Jacobian SAE pair was trained on layer 15 of Pythia-410m with an expansion factor of $R=64$ and sparsity $k=32$.
The examples were collected over the first 10K records of the English subset of the C4 text dataset \citep{raffel_exploring_2020}, with a context length of $16$ tokens.
For each example, the first row shows the values of the Jacobian element, and the second and third show the corresponding activations of the input and output SAE latents.
In this case, the input SAE latent appears to weakly activate for the token `there' and strongly activate for modal auxiliary verbs following `there' (i.e., `may' and `would'), whereas the output SAE latent appears to activate for both tokens.
}
\label{tab:feature_pairs_314_31729}
\end{table} % there
% \begin{table}
\centering
\begin{longtable}{lrl}
\toprule
Category & Max. abs. value & Example tokens \\
\midrule
Jacobian & \num{3.020e-01} & \colorbox{Cyan!0.000}{\strut  cond} \colorbox{Cyan!0.000}{\strut enses} \colorbox{Cyan!0.000}{\strut  on} \colorbox{Cyan!0.000}{\strut  the} \colorbox{Cyan!0.000}{\strut  walls} \colorbox{Cyan!0.000}{\strut  and} \colorbox{Cyan!0.000}{\strut  collects} \colorbox{Cyan!0.000}{\strut  on} \colorbox{Cyan!0.000}{\strut  the} \colorbox{Cyan!0.000}{\strut  floor} \colorbox{Cyan!0.000}{\strut  of} \colorbox{Cyan!0.000}{\strut  the} \colorbox{Cyan!0.000}{\strut  chamber} \colorbox{Cyan!0.000}{\strut .} \colorbox{Cyan!100.000}{\strut  There} \\
Input SAE & \num{1.553e+01} & \colorbox{Green!0.000}{\strut  cond} \colorbox{Green!0.000}{\strut enses} \colorbox{Green!0.000}{\strut  on} \colorbox{Green!0.000}{\strut  the} \colorbox{Green!0.000}{\strut  walls} \colorbox{Green!0.000}{\strut  and} \colorbox{Green!0.000}{\strut  collects} \colorbox{Green!0.000}{\strut  on} \colorbox{Green!0.000}{\strut  the} \colorbox{Green!0.000}{\strut  floor} \colorbox{Green!0.000}{\strut  of} \colorbox{Green!0.000}{\strut  the} \colorbox{Green!0.000}{\strut  chamber} \colorbox{Green!0.000}{\strut .} \colorbox{Green!73.410}{\strut  There} \\
Output SAE & \num{3.851e+00} & \colorbox{Magenta!0.000}{\strut  cond} \colorbox{Magenta!0.000}{\strut enses} \colorbox{Magenta!0.000}{\strut  on} \colorbox{Magenta!0.000}{\strut  the} \colorbox{Magenta!0.000}{\strut  walls} \colorbox{Magenta!0.000}{\strut  and} \colorbox{Magenta!0.000}{\strut  collects} \colorbox{Magenta!0.000}{\strut  on} \colorbox{Magenta!0.000}{\strut  the} \colorbox{Magenta!0.000}{\strut  floor} \colorbox{Magenta!0.000}{\strut  of} \colorbox{Magenta!0.000}{\strut  the} \colorbox{Magenta!0.000}{\strut  chamber} \colorbox{Magenta!0.000}{\strut .} \colorbox{Magenta!63.590}{\strut  There} \\
\midrule
Jacobian & \num{3.014e-01} & \colorbox{Cyan!0.000}{\strut  by} \colorbox{Cyan!0.000}{\strut  the} \colorbox{Cyan!0.000}{\strut  food} \colorbox{Cyan!0.000}{\strut  in} \colorbox{Cyan!0.000}{\strut  Zur} \colorbox{Cyan!0.000}{\strut ich} \colorbox{Cyan!0.000}{\strut .} \colorbox{Cyan!99.816}{\strut  There} \colorbox{Cyan!0.000}{\strut  are} \colorbox{Cyan!0.000}{\strut  plenty} \colorbox{Cyan!0.000}{\strut  of} \colorbox{Cyan!0.000}{\strut  vegan} \colorbox{Cyan!0.000}{\strut  and} \colorbox{Cyan!0.000}{\strut  vegg} \colorbox{Cyan!0.000}{\strut ie} \\
Input SAE & \num{1.450e+01} & \colorbox{Green!0.000}{\strut  by} \colorbox{Green!0.000}{\strut  the} \colorbox{Green!0.000}{\strut  food} \colorbox{Green!0.000}{\strut  in} \colorbox{Green!0.000}{\strut  Zur} \colorbox{Green!0.000}{\strut ich} \colorbox{Green!0.000}{\strut .} \colorbox{Green!68.540}{\strut  There} \colorbox{Green!0.000}{\strut  are} \colorbox{Green!0.000}{\strut  plenty} \colorbox{Green!0.000}{\strut  of} \colorbox{Green!0.000}{\strut  vegan} \colorbox{Green!0.000}{\strut  and} \colorbox{Green!0.000}{\strut  vegg} \colorbox{Green!0.000}{\strut ie} \\
Output SAE & \num{3.551e+00} & \colorbox{Magenta!0.000}{\strut  by} \colorbox{Magenta!0.000}{\strut  the} \colorbox{Magenta!0.000}{\strut  food} \colorbox{Magenta!0.000}{\strut  in} \colorbox{Magenta!0.000}{\strut  Zur} \colorbox{Magenta!0.000}{\strut ich} \colorbox{Magenta!0.000}{\strut .} \colorbox{Magenta!58.634}{\strut  There} \colorbox{Magenta!9.982}{\strut  are} \colorbox{Magenta!0.000}{\strut  plenty} \colorbox{Magenta!0.000}{\strut  of} \colorbox{Magenta!0.000}{\strut  vegan} \colorbox{Magenta!0.000}{\strut  and} \colorbox{Magenta!0.000}{\strut  vegg} \colorbox{Magenta!0.000}{\strut ie} \\
\midrule
Jacobian & \num{3.002e-01} & \colorbox{Cyan!0.000}{\strut  brought} \colorbox{Cyan!0.000}{\strut  New} \colorbox{Cyan!0.000}{\strut  York} \colorbox{Cyan!0.000}{\strut  Eli} \colorbox{Cyan!0.000}{\strut  Manning} \colorbox{Cyan!0.000}{\strut -} \colorbox{Cyan!0.000}{\strut  Oxford} \colorbox{Cyan!0.000}{\strut ,} \colorbox{Cyan!0.000}{\strut  Mississippi} \colorbox{Cyan!0.000}{\strut .} \colorbox{Cyan!99.404}{\strut  There} \colorbox{Cyan!0.000}{\strut  are} \colorbox{Cyan!0.000}{\strut  many} \colorbox{Cyan!0.000}{\strut  issues} \colorbox{Cyan!0.000}{\strut  common} \\
Input SAE & \num{1.432e+01} & \colorbox{Green!0.000}{\strut  brought} \colorbox{Green!0.000}{\strut  New} \colorbox{Green!0.000}{\strut  York} \colorbox{Green!0.000}{\strut  Eli} \colorbox{Green!0.000}{\strut  Manning} \colorbox{Green!0.000}{\strut -} \colorbox{Green!0.000}{\strut  Oxford} \colorbox{Green!0.000}{\strut ,} \colorbox{Green!0.000}{\strut  Mississippi} \colorbox{Green!0.000}{\strut .} \colorbox{Green!67.706}{\strut  There} \colorbox{Green!0.000}{\strut  are} \colorbox{Green!0.000}{\strut  many} \colorbox{Green!0.000}{\strut  issues} \colorbox{Green!0.000}{\strut  common} \\
Output SAE & \num{3.292e+00} & \colorbox{Magenta!0.000}{\strut  brought} \colorbox{Magenta!0.000}{\strut  New} \colorbox{Magenta!0.000}{\strut  York} \colorbox{Magenta!0.000}{\strut  Eli} \colorbox{Magenta!0.000}{\strut  Manning} \colorbox{Magenta!0.000}{\strut -} \colorbox{Magenta!0.000}{\strut  Oxford} \colorbox{Magenta!0.000}{\strut ,} \colorbox{Magenta!0.000}{\strut  Mississippi} \colorbox{Magenta!0.000}{\strut .} \colorbox{Magenta!54.354}{\strut  There} \colorbox{Magenta!0.000}{\strut  are} \colorbox{Magenta!0.000}{\strut  many} \colorbox{Magenta!0.000}{\strut  issues} \colorbox{Magenta!0.000}{\strut  common} \\
\midrule
Jacobian & \num{3.001e-01} & \colorbox{Cyan!0.000}{\strut For} \colorbox{Cyan!0.000}{\strut  several} \colorbox{Cyan!0.000}{\strut  seconds} \colorbox{Cyan!0.000}{\strut ,} \colorbox{Cyan!99.370}{\strut  there} \colorbox{Cyan!0.000}{\strut  was} \colorbox{Cyan!0.000}{\strut  an} \colorbox{Cyan!0.000}{\strut  awkward} \colorbox{Cyan!0.000}{\strut  moment} \colorbox{Cyan!0.000}{\strut  of} \colorbox{Cyan!0.000}{\strut  silence} \colorbox{Cyan!0.000}{\strut ,} \colorbox{Cyan!0.000}{\strut  and} \colorbox{Cyan!0.000}{\strut  J} \\
Input SAE & \num{1.588e+01} & \colorbox{Green!0.000}{\strut For} \colorbox{Green!0.000}{\strut  several} \colorbox{Green!0.000}{\strut  seconds} \colorbox{Green!0.000}{\strut ,} \colorbox{Green!75.068}{\strut  there} \colorbox{Green!0.000}{\strut  was} \colorbox{Green!0.000}{\strut  an} \colorbox{Green!0.000}{\strut  awkward} \colorbox{Green!0.000}{\strut  moment} \colorbox{Green!0.000}{\strut  of} \colorbox{Green!0.000}{\strut  silence} \colorbox{Green!0.000}{\strut ,} \colorbox{Green!0.000}{\strut  and} \colorbox{Green!0.000}{\strut  J} \\
Output SAE & \num{3.431e+00} & \colorbox{Magenta!0.000}{\strut For} \colorbox{Magenta!0.000}{\strut  several} \colorbox{Magenta!0.000}{\strut  seconds} \colorbox{Magenta!0.000}{\strut ,} \colorbox{Magenta!56.663}{\strut  there} \colorbox{Magenta!0.000}{\strut  was} \colorbox{Magenta!0.000}{\strut  an} \colorbox{Magenta!0.000}{\strut  awkward} \colorbox{Magenta!0.000}{\strut  moment} \colorbox{Magenta!0.000}{\strut  of} \colorbox{Magenta!0.000}{\strut  silence} \colorbox{Magenta!0.000}{\strut ,} \colorbox{Magenta!0.000}{\strut  and} \colorbox{Magenta!0.000}{\strut  J} \\
\midrule
Jacobian & \num{3.000e-01} & \colorbox{Cyan!0.000}{\strut  the} \colorbox{Cyan!0.000}{\strut  quality} \colorbox{Cyan!0.000}{\strut  is} \colorbox{Cyan!0.000}{\strut  very} \colorbox{Cyan!0.000}{\strut  high} \colorbox{Cyan!0.000}{\strut  because} \colorbox{Cyan!99.341}{\strut  there} \colorbox{Cyan!0.000}{\strut  are} \colorbox{Cyan!0.000}{\strut  more} \colorbox{Cyan!0.000}{\strut  quality} \colorbox{Cyan!0.000}{\strut  checks} \colorbox{Cyan!0.000}{\strut  for} \colorbox{Cyan!0.000}{\strut  drivers} \colorbox{Cyan!0.000}{\strut  ed} \colorbox{Cyan!0.000}{\strut  programs} \\
Input SAE & \num{1.749e+01} & \colorbox{Green!0.000}{\strut  the} \colorbox{Green!0.000}{\strut  quality} \colorbox{Green!0.000}{\strut  is} \colorbox{Green!0.000}{\strut  very} \colorbox{Green!0.000}{\strut  high} \colorbox{Green!0.000}{\strut  because} \colorbox{Green!82.697}{\strut  there} \colorbox{Green!0.000}{\strut  are} \colorbox{Green!0.000}{\strut  more} \colorbox{Green!0.000}{\strut  quality} \colorbox{Green!0.000}{\strut  checks} \colorbox{Green!0.000}{\strut  for} \colorbox{Green!0.000}{\strut  drivers} \colorbox{Green!0.000}{\strut  ed} \colorbox{Green!0.000}{\strut  programs} \\
Output SAE & \num{4.185e+00} & \colorbox{Magenta!0.000}{\strut  the} \colorbox{Magenta!0.000}{\strut  quality} \colorbox{Magenta!0.000}{\strut  is} \colorbox{Magenta!0.000}{\strut  very} \colorbox{Magenta!0.000}{\strut  high} \colorbox{Magenta!0.000}{\strut  because} \colorbox{Magenta!69.105}{\strut  there} \colorbox{Magenta!9.397}{\strut  are} \colorbox{Magenta!0.000}{\strut  more} \colorbox{Magenta!0.000}{\strut  quality} \colorbox{Magenta!0.000}{\strut  checks} \colorbox{Magenta!0.000}{\strut  for} \colorbox{Magenta!0.000}{\strut  drivers} \colorbox{Magenta!0.000}{\strut  ed} \colorbox{Magenta!0.000}{\strut  programs} \\
\midrule
Jacobian & \num{2.995e-01} & \colorbox{Cyan!0.000}{\strut  a} \colorbox{Cyan!0.000}{\strut  monastery} \colorbox{Cyan!0.000}{\strut .} \colorbox{Cyan!99.166}{\strut  There} \colorbox{Cyan!0.000}{\strut  is} \colorbox{Cyan!0.000}{\strut  almost} \colorbox{Cyan!0.000}{\strut  always} \colorbox{Cyan!0.000}{\strut  sooner} \colorbox{Cyan!0.000}{\strut  or} \colorbox{Cyan!0.000}{\strut  later} \colorbox{Cyan!0.000}{\strut  a} \colorbox{Cyan!0.000}{\strut  gain} \colorbox{Cyan!0.000}{\strut  for} \colorbox{Cyan!0.000}{\strut  others} \colorbox{Cyan!0.000}{\strut  in} \\
Input SAE & \num{1.273e+01} & \colorbox{Green!0.000}{\strut  a} \colorbox{Green!0.000}{\strut  monastery} \colorbox{Green!0.000}{\strut .} \colorbox{Green!60.170}{\strut  There} \colorbox{Green!0.000}{\strut  is} \colorbox{Green!0.000}{\strut  almost} \colorbox{Green!0.000}{\strut  always} \colorbox{Green!0.000}{\strut  sooner} \colorbox{Green!0.000}{\strut  or} \colorbox{Green!0.000}{\strut  later} \colorbox{Green!0.000}{\strut  a} \colorbox{Green!0.000}{\strut  gain} \colorbox{Green!0.000}{\strut  for} \colorbox{Green!0.000}{\strut  others} \colorbox{Green!0.000}{\strut  in} \\
Output SAE & \num{3.100e+00} & \colorbox{Magenta!0.000}{\strut  a} \colorbox{Magenta!0.000}{\strut  monastery} \colorbox{Magenta!0.000}{\strut .} \colorbox{Magenta!51.187}{\strut  There} \colorbox{Magenta!0.000}{\strut  is} \colorbox{Magenta!0.000}{\strut  almost} \colorbox{Magenta!0.000}{\strut  always} \colorbox{Magenta!0.000}{\strut  sooner} \colorbox{Magenta!0.000}{\strut  or} \colorbox{Magenta!0.000}{\strut  later} \colorbox{Magenta!0.000}{\strut  a} \colorbox{Magenta!0.000}{\strut  gain} \colorbox{Magenta!0.000}{\strut  for} \colorbox{Magenta!0.000}{\strut  others} \colorbox{Magenta!0.000}{\strut  in} \\
\midrule
Jacobian & \num{2.992e-01} & \colorbox{Cyan!0.000}{\strut  see} \colorbox{Cyan!0.000}{\strut  the} \colorbox{Cyan!0.000}{\strut  level} \colorbox{Cyan!0.000}{\strut  of} \colorbox{Cyan!0.000}{\strut  which} \colorbox{Cyan!0.000}{\strut  the} \colorbox{Cyan!0.000}{\strut  liquid} \colorbox{Cyan!0.000}{\strut  is} \colorbox{Cyan!0.000}{\strut  at} \colorbox{Cyan!0.000}{\strut ,} \colorbox{Cyan!0.000}{\strut  and} \colorbox{Cyan!99.078}{\strut  there} \colorbox{Cyan!0.000}{\strut  are} \colorbox{Cyan!0.000}{\strut  gaug} \colorbox{Cyan!0.000}{\strut es} \\
Input SAE & \num{1.479e+01} & \colorbox{Green!0.000}{\strut  see} \colorbox{Green!0.000}{\strut  the} \colorbox{Green!0.000}{\strut  level} \colorbox{Green!0.000}{\strut  of} \colorbox{Green!0.000}{\strut  which} \colorbox{Green!0.000}{\strut  the} \colorbox{Green!0.000}{\strut  liquid} \colorbox{Green!0.000}{\strut  is} \colorbox{Green!0.000}{\strut  at} \colorbox{Green!0.000}{\strut ,} \colorbox{Green!0.000}{\strut  and} \colorbox{Green!69.923}{\strut  there} \colorbox{Green!0.000}{\strut  are} \colorbox{Green!0.000}{\strut  gaug} \colorbox{Green!0.000}{\strut es} \\
Output SAE & \num{3.428e+00} & \colorbox{Magenta!0.000}{\strut  see} \colorbox{Magenta!0.000}{\strut  the} \colorbox{Magenta!0.000}{\strut  level} \colorbox{Magenta!0.000}{\strut  of} \colorbox{Magenta!0.000}{\strut  which} \colorbox{Magenta!0.000}{\strut  the} \colorbox{Magenta!0.000}{\strut  liquid} \colorbox{Magenta!0.000}{\strut  is} \colorbox{Magenta!0.000}{\strut  at} \colorbox{Magenta!0.000}{\strut ,} \colorbox{Magenta!0.000}{\strut  and} \colorbox{Magenta!56.600}{\strut  there} \colorbox{Magenta!0.000}{\strut  are} \colorbox{Magenta!0.000}{\strut  gaug} \colorbox{Magenta!0.000}{\strut es} \\
\midrule
Jacobian & \num{2.989e-01} & \colorbox{Cyan!0.000}{\strut  This} \colorbox{Cyan!0.000}{\strut  method} \colorbox{Cyan!0.000}{\strut  is} \colorbox{Cyan!0.000}{\strut  much} \colorbox{Cyan!0.000}{\strut  similar} \colorbox{Cyan!0.000}{\strut  to} \colorbox{Cyan!0.000}{\strut  VPN} \colorbox{Cyan!0.000}{\strut  method} \colorbox{Cyan!0.000}{\strut ,} \colorbox{Cyan!0.000}{\strut  but} \colorbox{Cyan!0.000}{\strut  in} \colorbox{Cyan!0.000}{\strut  this} \colorbox{Cyan!0.000}{\strut  method} \colorbox{Cyan!0.000}{\strut ,} \colorbox{Cyan!98.994}{\strut  there} \\
Input SAE & \num{1.548e+01} & \colorbox{Green!0.000}{\strut  This} \colorbox{Green!0.000}{\strut  method} \colorbox{Green!0.000}{\strut  is} \colorbox{Green!0.000}{\strut  much} \colorbox{Green!0.000}{\strut  similar} \colorbox{Green!0.000}{\strut  to} \colorbox{Green!0.000}{\strut  VPN} \colorbox{Green!0.000}{\strut  method} \colorbox{Green!0.000}{\strut ,} \colorbox{Green!0.000}{\strut  but} \colorbox{Green!0.000}{\strut  in} \colorbox{Green!0.000}{\strut  this} \colorbox{Green!0.000}{\strut  method} \colorbox{Green!0.000}{\strut ,} \colorbox{Green!73.161}{\strut  there} \\
Output SAE & \num{4.041e+00} & \colorbox{Magenta!0.000}{\strut  This} \colorbox{Magenta!0.000}{\strut  method} \colorbox{Magenta!0.000}{\strut  is} \colorbox{Magenta!0.000}{\strut  much} \colorbox{Magenta!0.000}{\strut  similar} \colorbox{Magenta!0.000}{\strut  to} \colorbox{Magenta!0.000}{\strut  VPN} \colorbox{Magenta!0.000}{\strut  method} \colorbox{Magenta!0.000}{\strut ,} \colorbox{Magenta!0.000}{\strut  but} \colorbox{Magenta!0.000}{\strut  in} \colorbox{Magenta!0.000}{\strut  this} \colorbox{Magenta!0.000}{\strut  method} \colorbox{Magenta!0.000}{\strut ,} \colorbox{Magenta!66.738}{\strut  there} \\
\midrule
Jacobian & \num{2.987e-01} & \colorbox{Cyan!0.000}{\strut  vehicle} \colorbox{Cyan!0.000}{\strut  is} \colorbox{Cyan!0.000}{\strut  parked} \colorbox{Cyan!0.000}{\strut  or} \colorbox{Cyan!0.000}{\strut  stuck} \colorbox{Cyan!0.000}{\strut  on} \colorbox{Cyan!0.000}{\strut  the} \colorbox{Cyan!0.000}{\strut  tracks} \colorbox{Cyan!0.000}{\strut  at} \colorbox{Cyan!0.000}{\strut  a} \colorbox{Cyan!0.000}{\strut  rail} \colorbox{Cyan!0.000}{\strut  crossing} \colorbox{Cyan!0.000}{\strut .} \colorbox{Cyan!98.921}{\strut  There} \colorbox{Cyan!0.000}{\strut  have} \\
Input SAE & \num{1.767e+01} & \colorbox{Green!0.000}{\strut  vehicle} \colorbox{Green!0.000}{\strut  is} \colorbox{Green!0.000}{\strut  parked} \colorbox{Green!0.000}{\strut  or} \colorbox{Green!0.000}{\strut  stuck} \colorbox{Green!0.000}{\strut  on} \colorbox{Green!0.000}{\strut  the} \colorbox{Green!0.000}{\strut  tracks} \colorbox{Green!0.000}{\strut  at} \colorbox{Green!0.000}{\strut  a} \colorbox{Green!0.000}{\strut  rail} \colorbox{Green!0.000}{\strut  crossing} \colorbox{Green!0.000}{\strut .} \colorbox{Green!83.549}{\strut  There} \colorbox{Green!0.000}{\strut  have} \\
Output SAE & \num{4.281e+00} & \colorbox{Magenta!0.000}{\strut  vehicle} \colorbox{Magenta!0.000}{\strut  is} \colorbox{Magenta!0.000}{\strut  parked} \colorbox{Magenta!0.000}{\strut  or} \colorbox{Magenta!0.000}{\strut  stuck} \colorbox{Magenta!0.000}{\strut  on} \colorbox{Magenta!0.000}{\strut  the} \colorbox{Magenta!0.000}{\strut  tracks} \colorbox{Magenta!0.000}{\strut  at} \colorbox{Magenta!0.000}{\strut  a} \colorbox{Magenta!0.000}{\strut  rail} \colorbox{Magenta!0.000}{\strut  crossing} \colorbox{Magenta!0.000}{\strut .} \colorbox{Magenta!69.582}{\strut  There} \colorbox{Magenta!70.695}{\strut  have} \\
\midrule
Jacobian & \num{2.985e-01} & \colorbox{Cyan!0.000}{\strut  vast} \colorbox{Cyan!0.000}{\strut  area} \colorbox{Cyan!0.000}{\strut .} \colorbox{Cyan!0.000}{\strut  Also} \colorbox{Cyan!98.859}{\strut  there} \colorbox{Cyan!0.000}{\strut  are} \colorbox{Cyan!0.000}{\strut  new} \colorbox{Cyan!0.000}{\strut  sed} \colorbox{Cyan!0.000}{\strut iments} \colorbox{Cyan!0.000}{\strut  that} \colorbox{Cyan!0.000}{\strut  lie} \colorbox{Cyan!0.000}{\strut  on} \colorbox{Cyan!0.000}{\strut  top} \colorbox{Cyan!0.000}{\strut  of} \colorbox{Cyan!0.000}{\strut  this} \\
Input SAE & \num{1.542e+01} & \colorbox{Green!0.000}{\strut  vast} \colorbox{Green!0.000}{\strut  area} \colorbox{Green!0.000}{\strut .} \colorbox{Green!0.000}{\strut  Also} \colorbox{Green!72.875}{\strut  there} \colorbox{Green!0.000}{\strut  are} \colorbox{Green!0.000}{\strut  new} \colorbox{Green!0.000}{\strut  sed} \colorbox{Green!0.000}{\strut iments} \colorbox{Green!0.000}{\strut  that} \colorbox{Green!0.000}{\strut  lie} \colorbox{Green!0.000}{\strut  on} \colorbox{Green!0.000}{\strut  top} \colorbox{Green!0.000}{\strut  of} \colorbox{Green!0.000}{\strut  this} \\
Output SAE & \num{4.022e+00} & \colorbox{Magenta!0.000}{\strut  vast} \colorbox{Magenta!0.000}{\strut  area} \colorbox{Magenta!0.000}{\strut .} \colorbox{Magenta!0.000}{\strut  Also} \colorbox{Magenta!66.424}{\strut  there} \colorbox{Magenta!0.000}{\strut  are} \colorbox{Magenta!0.000}{\strut  new} \colorbox{Magenta!0.000}{\strut  sed} \colorbox{Magenta!0.000}{\strut iments} \colorbox{Magenta!0.000}{\strut  that} \colorbox{Magenta!0.000}{\strut  lie} \colorbox{Magenta!0.000}{\strut  on} \colorbox{Magenta!0.000}{\strut  top} \colorbox{Magenta!0.000}{\strut  of} \colorbox{Magenta!0.000}{\strut  this} \\
\midrule
Jacobian & \num{2.984e-01} & \colorbox{Cyan!0.000}{\strut  formations} \colorbox{Cyan!0.000}{\strut ;} \colorbox{Cyan!0.000}{\strut  and} \colorbox{Cyan!0.000}{\strut  in} \colorbox{Cyan!0.000}{\strut  these} \colorbox{Cyan!0.000}{\strut  intervals} \colorbox{Cyan!98.797}{\strut  there} \colorbox{Cyan!0.000}{\strut  may} \colorbox{Cyan!0.000}{\strut  have} \colorbox{Cyan!0.000}{\strut  been} \colorbox{Cyan!0.000}{\strut  much} \colorbox{Cyan!0.000}{\strut  slow} \colorbox{Cyan!0.000}{\strut  ex} \colorbox{Cyan!0.000}{\strut termination} \colorbox{Cyan!0.000}{\strut .} \\
Input SAE & \num{1.354e+01} & \colorbox{Green!0.000}{\strut  formations} \colorbox{Green!0.000}{\strut ;} \colorbox{Green!0.000}{\strut  and} \colorbox{Green!0.000}{\strut  in} \colorbox{Green!0.000}{\strut  these} \colorbox{Green!0.000}{\strut  intervals} \colorbox{Green!63.997}{\strut  there} \colorbox{Green!0.000}{\strut  may} \colorbox{Green!0.000}{\strut  have} \colorbox{Green!0.000}{\strut  been} \colorbox{Green!0.000}{\strut  much} \colorbox{Green!0.000}{\strut  slow} \colorbox{Green!0.000}{\strut  ex} \colorbox{Green!0.000}{\strut termination} \colorbox{Green!0.000}{\strut .} \\
Output SAE & \num{3.308e+00} & \colorbox{Magenta!0.000}{\strut  formations} \colorbox{Magenta!0.000}{\strut ;} \colorbox{Magenta!0.000}{\strut  and} \colorbox{Magenta!0.000}{\strut  in} \colorbox{Magenta!0.000}{\strut  these} \colorbox{Magenta!0.000}{\strut  intervals} \colorbox{Magenta!54.626}{\strut  there} \colorbox{Magenta!47.459}{\strut  may} \colorbox{Magenta!46.242}{\strut  have} \colorbox{Magenta!0.000}{\strut  been} \colorbox{Magenta!0.000}{\strut  much} \colorbox{Magenta!0.000}{\strut  slow} \colorbox{Magenta!0.000}{\strut  ex} \colorbox{Magenta!0.000}{\strut termination} \colorbox{Magenta!0.000}{\strut .} \\
\midrule
Jacobian & \num{2.983e-01} & \colorbox{Cyan!0.000}{\strut  or} \colorbox{Cyan!0.000}{\strut  selection} \colorbox{Cyan!0.000}{\strut  process} \colorbox{Cyan!0.000}{\strut  (} \colorbox{Cyan!0.000}{\strut all} \colorbox{Cyan!0.000}{\strut  bodies} \colorbox{Cyan!0.000}{\strut  submitted} \colorbox{Cyan!0.000}{\strut  are} \colorbox{Cyan!0.000}{\strut  drawn} \colorbox{Cyan!0.000}{\strut ),} \colorbox{Cyan!0.000}{\strut  and} \colorbox{Cyan!98.795}{\strut  there} \colorbox{Cyan!0.000}{\strut  is} \colorbox{Cyan!0.000}{\strut  no} \colorbox{Cyan!0.000}{\strut  monetary} \\
Input SAE & \num{1.813e+01} & \colorbox{Green!0.000}{\strut  or} \colorbox{Green!0.000}{\strut  selection} \colorbox{Green!0.000}{\strut  process} \colorbox{Green!0.000}{\strut  (} \colorbox{Green!0.000}{\strut all} \colorbox{Green!0.000}{\strut  bodies} \colorbox{Green!0.000}{\strut  submitted} \colorbox{Green!0.000}{\strut  are} \colorbox{Green!0.000}{\strut  drawn} \colorbox{Green!0.000}{\strut ),} \colorbox{Green!0.000}{\strut  and} \colorbox{Green!85.711}{\strut  there} \colorbox{Green!0.000}{\strut  is} \colorbox{Green!0.000}{\strut  no} \colorbox{Green!0.000}{\strut  monetary} \\
Output SAE & \num{4.328e+00} & \colorbox{Magenta!0.000}{\strut  or} \colorbox{Magenta!0.000}{\strut  selection} \colorbox{Magenta!0.000}{\strut  process} \colorbox{Magenta!0.000}{\strut  (} \colorbox{Magenta!0.000}{\strut all} \colorbox{Magenta!0.000}{\strut  bodies} \colorbox{Magenta!0.000}{\strut  submitted} \colorbox{Magenta!0.000}{\strut  are} \colorbox{Magenta!0.000}{\strut  drawn} \colorbox{Magenta!0.000}{\strut ),} \colorbox{Magenta!0.000}{\strut  and} \colorbox{Magenta!71.465}{\strut  there} \colorbox{Magenta!0.000}{\strut  is} \colorbox{Magenta!0.000}{\strut  no} \colorbox{Magenta!0.000}{\strut  monetary} \\
\bottomrule
\end{longtable}
\caption{feature pairs/Layer15-65536-J1-LR5.0e-04-k32-T3.0e+08 abs mean/examples-13810-v-31729 stas c4-en-10k,train,batch size=32,ctx len=16.csv}
\end{table} % also there
% \begin{table}
\centering
\begin{longtable}{lrl}
\toprule
Category & Max. abs. value & Example tokens \\
\midrule
Jacobian & \num{3.109e-01} & \colorbox{Cyan!0.000}{\strut \textquotesingle{}} \colorbox{Cyan!0.000}{\strut s} \colorbox{Cyan!0.000}{\strut  Court} \colorbox{Cyan!0.000}{\strut  Theatre} \colorbox{Cyan!0.000}{\strut ),} \colorbox{Cyan!0.000}{\strut  What} \colorbox{Cyan!100.000}{\strut  It} \colorbox{Cyan!0.000}{\strut  Fe} \colorbox{Cyan!0.000}{\strut els} \colorbox{Cyan!0.000}{\strut  Like} \colorbox{Cyan!0.000}{\strut  (} \colorbox{Cyan!0.000}{\strut C} \colorbox{Cyan!0.000}{\strut  Ven} \colorbox{Cyan!0.000}{\strut ues} \colorbox{Cyan!0.000}{\strut  in} \\
Input SAE & \num{1.743e+00} & \colorbox{Green!0.000}{\strut \textquotesingle{}} \colorbox{Green!0.000}{\strut s} \colorbox{Green!0.000}{\strut  Court} \colorbox{Green!0.000}{\strut  Theatre} \colorbox{Green!0.000}{\strut ),} \colorbox{Green!0.000}{\strut  What} \colorbox{Green!15.309}{\strut  It} \colorbox{Green!0.000}{\strut  Fe} \colorbox{Green!0.000}{\strut els} \colorbox{Green!0.000}{\strut  Like} \colorbox{Green!0.000}{\strut  (} \colorbox{Green!0.000}{\strut C} \colorbox{Green!0.000}{\strut  Ven} \colorbox{Green!0.000}{\strut ues} \colorbox{Green!0.000}{\strut  in} \\
Output SAE & \num{1.077e+00} & \colorbox{Magenta!0.000}{\strut \textquotesingle{}} \colorbox{Magenta!0.000}{\strut s} \colorbox{Magenta!0.000}{\strut  Court} \colorbox{Magenta!0.000}{\strut  Theatre} \colorbox{Magenta!0.000}{\strut ),} \colorbox{Magenta!0.000}{\strut  What} \colorbox{Magenta!17.793}{\strut  It} \colorbox{Magenta!0.000}{\strut  Fe} \colorbox{Magenta!0.000}{\strut els} \colorbox{Magenta!0.000}{\strut  Like} \colorbox{Magenta!0.000}{\strut  (} \colorbox{Magenta!0.000}{\strut C} \colorbox{Magenta!0.000}{\strut  Ven} \colorbox{Magenta!0.000}{\strut ues} \colorbox{Magenta!0.000}{\strut  in} \\
\midrule
Jacobian & \num{3.106e-01} & \colorbox{Cyan!0.000}{\strut  so} \colorbox{Cyan!0.000}{\strut  now} \colorbox{Cyan!0.000}{\strut  I} \colorbox{Cyan!0.000}{\strut \textquotesingle{}ll} \colorbox{Cyan!0.000}{\strut  show} \colorbox{Cyan!0.000}{\strut  you} \colorbox{Cyan!0.000}{\strut  what} \colorbox{Cyan!99.875}{\strut  that} \colorbox{Cyan!0.000}{\strut  looks} \colorbox{Cyan!0.000}{\strut  like} \colorbox{Cyan!0.000}{\strut .} \colorbox{Cyan!0.000}{\strut  The} \colorbox{Cyan!0.000}{\strut  deck} \colorbox{Cyan!0.000}{\strut  is} \colorbox{Cyan!0.000}{\strut  really} \\
Input SAE & \num{1.478e+00} & \colorbox{Green!0.000}{\strut  so} \colorbox{Green!0.000}{\strut  now} \colorbox{Green!0.000}{\strut  I} \colorbox{Green!0.000}{\strut \textquotesingle{}ll} \colorbox{Green!0.000}{\strut  show} \colorbox{Green!0.000}{\strut  you} \colorbox{Green!0.000}{\strut  what} \colorbox{Green!12.980}{\strut  that} \colorbox{Green!0.000}{\strut  looks} \colorbox{Green!0.000}{\strut  like} \colorbox{Green!0.000}{\strut .} \colorbox{Green!0.000}{\strut  The} \colorbox{Green!0.000}{\strut  deck} \colorbox{Green!0.000}{\strut  is} \colorbox{Green!0.000}{\strut  really} \\
Output SAE & \num{1.158e+00} & \colorbox{Magenta!0.000}{\strut  so} \colorbox{Magenta!0.000}{\strut  now} \colorbox{Magenta!0.000}{\strut  I} \colorbox{Magenta!0.000}{\strut \textquotesingle{}ll} \colorbox{Magenta!0.000}{\strut  show} \colorbox{Magenta!0.000}{\strut  you} \colorbox{Magenta!0.000}{\strut  what} \colorbox{Magenta!19.115}{\strut  that} \colorbox{Magenta!0.000}{\strut  looks} \colorbox{Magenta!0.000}{\strut  like} \colorbox{Magenta!0.000}{\strut .} \colorbox{Magenta!0.000}{\strut  The} \colorbox{Magenta!0.000}{\strut  deck} \colorbox{Magenta!0.000}{\strut  is} \colorbox{Magenta!0.000}{\strut  really} \\
\midrule
Jacobian & \num{3.093e-01} & \colorbox{Cyan!0.000}{\strut  for} \colorbox{Cyan!0.000}{\strut  signs} \colorbox{Cyan!0.000}{\strut !} \colorbox{Cyan!0.000}{\strut I} \colorbox{Cyan!0.000}{\strut  am} \colorbox{Cyan!0.000}{\strut  a} \colorbox{Cyan!0.000}{\strut  food} \colorbox{Cyan!0.000}{\strut ie} \colorbox{Cyan!0.000}{\strut  as} \colorbox{Cyan!0.000}{\strut  much} \colorbox{Cyan!0.000}{\strut  as} \colorbox{Cyan!99.475}{\strut  I} \colorbox{Cyan!0.000}{\strut  am} \colorbox{Cyan!0.000}{\strut  a} \\
Input SAE & \num{1.533e+00} & \colorbox{Green!0.000}{\strut  for} \colorbox{Green!0.000}{\strut  signs} \colorbox{Green!0.000}{\strut !} \colorbox{Green!0.000}{\strut I} \colorbox{Green!0.000}{\strut  am} \colorbox{Green!0.000}{\strut  a} \colorbox{Green!0.000}{\strut  food} \colorbox{Green!0.000}{\strut ie} \colorbox{Green!0.000}{\strut  as} \colorbox{Green!0.000}{\strut  much} \colorbox{Green!0.000}{\strut  as} \colorbox{Green!13.460}{\strut  I} \colorbox{Green!0.000}{\strut  am} \colorbox{Green!0.000}{\strut  a} \\
Output SAE & \num{1.148e+00} & \colorbox{Magenta!0.000}{\strut  for} \colorbox{Magenta!0.000}{\strut  signs} \colorbox{Magenta!0.000}{\strut !} \colorbox{Magenta!0.000}{\strut I} \colorbox{Magenta!0.000}{\strut  am} \colorbox{Magenta!0.000}{\strut  a} \colorbox{Magenta!0.000}{\strut  food} \colorbox{Magenta!0.000}{\strut ie} \colorbox{Magenta!17.626}{\strut  as} \colorbox{Magenta!0.000}{\strut  much} \colorbox{Magenta!0.000}{\strut  as} \colorbox{Magenta!18.953}{\strut  I} \colorbox{Magenta!0.000}{\strut  am} \colorbox{Magenta!0.000}{\strut  a} \\
\midrule
Jacobian & \num{3.072e-01} & \colorbox{Cyan!0.000}{\strut \textquotedbl{}} \colorbox{Cyan!0.000}{\strut  run} \colorbox{Cyan!0.000}{\strut  in} \colorbox{Cyan!0.000}{\strut  Louisville} \colorbox{Cyan!0.000}{\strut  would} \colorbox{Cyan!0.000}{\strut  be} \colorbox{Cyan!0.000}{\strut  nice} \colorbox{Cyan!0.000}{\strut .} \colorbox{Cyan!0.000}{\strut  As} \colorbox{Cyan!98.811}{\strut  it} \colorbox{Cyan!0.000}{\strut  turned} \colorbox{Cyan!0.000}{\strut  out} \colorbox{Cyan!0.000}{\strut  the} \colorbox{Cyan!0.000}{\strut  group} \colorbox{Cyan!0.000}{\strut  she} \\
Input SAE & \num{8.963e-01} & \colorbox{Green!0.000}{\strut \textquotedbl{}} \colorbox{Green!0.000}{\strut  run} \colorbox{Green!0.000}{\strut  in} \colorbox{Green!0.000}{\strut  Louisville} \colorbox{Green!0.000}{\strut  would} \colorbox{Green!0.000}{\strut  be} \colorbox{Green!0.000}{\strut  nice} \colorbox{Green!0.000}{\strut .} \colorbox{Green!0.000}{\strut  As} \colorbox{Green!7.871}{\strut  it} \colorbox{Green!0.000}{\strut  turned} \colorbox{Green!0.000}{\strut  out} \colorbox{Green!0.000}{\strut  the} \colorbox{Green!0.000}{\strut  group} \colorbox{Green!0.000}{\strut  she} \\
Output SAE & \num{1.474e+00} & \colorbox{Magenta!0.000}{\strut \textquotedbl{}} \colorbox{Magenta!0.000}{\strut  run} \colorbox{Magenta!0.000}{\strut  in} \colorbox{Magenta!0.000}{\strut  Louisville} \colorbox{Magenta!0.000}{\strut  would} \colorbox{Magenta!0.000}{\strut  be} \colorbox{Magenta!0.000}{\strut  nice} \colorbox{Magenta!0.000}{\strut .} \colorbox{Magenta!12.975}{\strut  As} \colorbox{Magenta!24.333}{\strut  it} \colorbox{Magenta!0.000}{\strut  turned} \colorbox{Magenta!0.000}{\strut  out} \colorbox{Magenta!0.000}{\strut  the} \colorbox{Magenta!0.000}{\strut  group} \colorbox{Magenta!0.000}{\strut  she} \\
\midrule
Jacobian & \num{3.068e-01} & \colorbox{Cyan!0.000}{\strut  share} \colorbox{Cyan!0.000}{\strut  the} \colorbox{Cyan!0.000}{\strut  posts} \colorbox{Cyan!0.000}{\strut  as} \colorbox{Cyan!0.000}{\strut  if} \colorbox{Cyan!0.000}{\strut  your} \colorbox{Cyan!0.000}{\strut  life} \colorbox{Cyan!0.000}{\strut  depended} \colorbox{Cyan!0.000}{\strut  on} \colorbox{Cyan!0.000}{\strut  it} \colorbox{Cyan!0.000}{\strut .} \colorbox{Cyan!0.000}{\strut  Which} \colorbox{Cyan!98.680}{\strut  it} \colorbox{Cyan!0.000}{\strut  doesn} \colorbox{Cyan!0.000}{\strut \textquotesingle{}t} \\
Input SAE & \num{3.135e+00} & \colorbox{Green!0.000}{\strut  share} \colorbox{Green!0.000}{\strut  the} \colorbox{Green!0.000}{\strut  posts} \colorbox{Green!0.000}{\strut  as} \colorbox{Green!0.000}{\strut  if} \colorbox{Green!0.000}{\strut  your} \colorbox{Green!0.000}{\strut  life} \colorbox{Green!0.000}{\strut  depended} \colorbox{Green!0.000}{\strut  on} \colorbox{Green!0.000}{\strut  it} \colorbox{Green!0.000}{\strut .} \colorbox{Green!0.000}{\strut  Which} \colorbox{Green!27.528}{\strut  it} \colorbox{Green!0.000}{\strut  doesn} \colorbox{Green!0.000}{\strut \textquotesingle{}t} \\
Output SAE & \num{1.217e+00} & \colorbox{Magenta!0.000}{\strut  share} \colorbox{Magenta!0.000}{\strut  the} \colorbox{Magenta!0.000}{\strut  posts} \colorbox{Magenta!0.000}{\strut  as} \colorbox{Magenta!0.000}{\strut  if} \colorbox{Magenta!0.000}{\strut  your} \colorbox{Magenta!0.000}{\strut  life} \colorbox{Magenta!0.000}{\strut  depended} \colorbox{Magenta!0.000}{\strut  on} \colorbox{Magenta!0.000}{\strut  it} \colorbox{Magenta!0.000}{\strut .} \colorbox{Magenta!0.000}{\strut  Which} \colorbox{Magenta!20.102}{\strut  it} \colorbox{Magenta!0.000}{\strut  doesn} \colorbox{Magenta!0.000}{\strut \textquotesingle{}t} \\
\midrule
Jacobian & \num{3.066e-01} & \colorbox{Cyan!0.000}{\strut  color} \colorbox{Cyan!0.000}{\strut  red} \colorbox{Cyan!0.000}{\strut ,} \colorbox{Cyan!0.000}{\strut  the} \colorbox{Cyan!0.000}{\strut  higher} \colorbox{Cyan!0.000}{\strut  confidence} \colorbox{Cyan!0.000}{\strut  a} \colorbox{Cyan!98.596}{\strut  relationship} \colorbox{Cyan!84.681}{\strut  should} \colorbox{Cyan!0.000}{\strut  not} \colorbox{Cyan!0.000}{\strut  be} \colorbox{Cyan!0.000}{\strut  present} \colorbox{Cyan!0.000}{\strut .} \colorbox{Cyan!0.000}{\strut In} \\
Input SAE & \num{4.553e+00} & \colorbox{Green!0.000}{\strut  color} \colorbox{Green!0.000}{\strut  red} \colorbox{Green!0.000}{\strut ,} \colorbox{Green!0.000}{\strut  the} \colorbox{Green!0.000}{\strut  higher} \colorbox{Green!0.000}{\strut  confidence} \colorbox{Green!0.000}{\strut  a} \colorbox{Green!17.475}{\strut  relationship} \colorbox{Green!39.984}{\strut  should} \colorbox{Green!11.887}{\strut  not} \colorbox{Green!14.466}{\strut  be} \colorbox{Green!0.000}{\strut  present} \colorbox{Green!0.000}{\strut .} \colorbox{Green!0.000}{\strut In} \\
Output SAE & \num{1.376e+00} & \colorbox{Magenta!0.000}{\strut  color} \colorbox{Magenta!0.000}{\strut  red} \colorbox{Magenta!0.000}{\strut ,} \colorbox{Magenta!0.000}{\strut  the} \colorbox{Magenta!0.000}{\strut  higher} \colorbox{Magenta!0.000}{\strut  confidence} \colorbox{Magenta!0.000}{\strut  a} \colorbox{Magenta!22.726}{\strut  relationship} \colorbox{Magenta!17.279}{\strut  should} \colorbox{Magenta!0.000}{\strut  not} \colorbox{Magenta!0.000}{\strut  be} \colorbox{Magenta!0.000}{\strut  present} \colorbox{Magenta!0.000}{\strut .} \colorbox{Magenta!0.000}{\strut In} \\
\midrule
Jacobian & \num{3.066e-01} & \colorbox{Cyan!0.000}{\strut  color} \colorbox{Cyan!0.000}{\strut  red} \colorbox{Cyan!0.000}{\strut ,} \colorbox{Cyan!0.000}{\strut  the} \colorbox{Cyan!0.000}{\strut  higher} \colorbox{Cyan!0.000}{\strut  confidence} \colorbox{Cyan!0.000}{\strut  a} \colorbox{Cyan!98.596}{\strut  relationship} \colorbox{Cyan!84.681}{\strut  should} \colorbox{Cyan!0.000}{\strut  not} \colorbox{Cyan!0.000}{\strut  be} \colorbox{Cyan!0.000}{\strut  present} \colorbox{Cyan!0.000}{\strut .} \colorbox{Cyan!0.000}{\strut For} \colorbox{Cyan!0.000}{\strut  3000} \\
Input SAE & \num{4.553e+00} & \colorbox{Green!0.000}{\strut  color} \colorbox{Green!0.000}{\strut  red} \colorbox{Green!0.000}{\strut ,} \colorbox{Green!0.000}{\strut  the} \colorbox{Green!0.000}{\strut  higher} \colorbox{Green!0.000}{\strut  confidence} \colorbox{Green!0.000}{\strut  a} \colorbox{Green!17.475}{\strut  relationship} \colorbox{Green!39.984}{\strut  should} \colorbox{Green!11.887}{\strut  not} \colorbox{Green!14.466}{\strut  be} \colorbox{Green!0.000}{\strut  present} \colorbox{Green!0.000}{\strut .} \colorbox{Green!0.000}{\strut For} \colorbox{Green!0.000}{\strut  3000} \\
Output SAE & \num{1.376e+00} & \colorbox{Magenta!0.000}{\strut  color} \colorbox{Magenta!0.000}{\strut  red} \colorbox{Magenta!0.000}{\strut ,} \colorbox{Magenta!0.000}{\strut  the} \colorbox{Magenta!0.000}{\strut  higher} \colorbox{Magenta!0.000}{\strut  confidence} \colorbox{Magenta!0.000}{\strut  a} \colorbox{Magenta!22.726}{\strut  relationship} \colorbox{Magenta!17.279}{\strut  should} \colorbox{Magenta!0.000}{\strut  not} \colorbox{Magenta!0.000}{\strut  be} \colorbox{Magenta!0.000}{\strut  present} \colorbox{Magenta!0.000}{\strut .} \colorbox{Magenta!0.000}{\strut For} \colorbox{Magenta!0.000}{\strut  3000} \\
\midrule
Jacobian & \num{3.063e-01} & \colorbox{Cyan!0.000}{\strut  what} \colorbox{Cyan!0.000}{\strut  an} \colorbox{Cyan!0.000}{\strut  after} \colorbox{Cyan!0.000}{\strut sh} \colorbox{Cyan!0.000}{\strut ave} \colorbox{Cyan!98.494}{\strut  even} \colorbox{Cyan!0.000}{\strut  is} \colorbox{Cyan!0.000}{\strut ,} \colorbox{Cyan!0.000}{\strut  don} \colorbox{Cyan!0.000}{\strut \textquotesingle{}} \colorbox{Cyan!0.000}{\strut t} \colorbox{Cyan!0.000}{\strut  worry} \colorbox{Cyan!0.000}{\strut  cave} \colorbox{Cyan!0.000}{\strut men} \colorbox{Cyan!0.000}{\strut  weren} \\
Input SAE & \num{2.327e+00} & \colorbox{Green!0.000}{\strut  what} \colorbox{Green!0.000}{\strut  an} \colorbox{Green!0.000}{\strut  after} \colorbox{Green!0.000}{\strut sh} \colorbox{Green!0.000}{\strut ave} \colorbox{Green!20.430}{\strut  even} \colorbox{Green!0.000}{\strut  is} \colorbox{Green!0.000}{\strut ,} \colorbox{Green!0.000}{\strut  don} \colorbox{Green!0.000}{\strut \textquotesingle{}} \colorbox{Green!0.000}{\strut t} \colorbox{Green!0.000}{\strut  worry} \colorbox{Green!0.000}{\strut  cave} \colorbox{Green!0.000}{\strut men} \colorbox{Green!0.000}{\strut  weren} \\
Output SAE & \num{1.587e+00} & \colorbox{Magenta!0.000}{\strut  what} \colorbox{Magenta!0.000}{\strut  an} \colorbox{Magenta!0.000}{\strut  after} \colorbox{Magenta!0.000}{\strut sh} \colorbox{Magenta!0.000}{\strut ave} \colorbox{Magenta!26.203}{\strut  even} \colorbox{Magenta!0.000}{\strut  is} \colorbox{Magenta!0.000}{\strut ,} \colorbox{Magenta!0.000}{\strut  don} \colorbox{Magenta!0.000}{\strut \textquotesingle{}} \colorbox{Magenta!0.000}{\strut t} \colorbox{Magenta!0.000}{\strut  worry} \colorbox{Magenta!0.000}{\strut  cave} \colorbox{Magenta!0.000}{\strut men} \colorbox{Magenta!0.000}{\strut  weren} \\
\midrule
Jacobian & \num{3.062e-01} & \colorbox{Cyan!0.000}{\strut  (} \colorbox{Cyan!0.000}{\strut see} \colorbox{Cyan!0.000}{\strut  Fasc} \colorbox{Cyan!0.000}{\strut ism} \colorbox{Cyan!0.000}{\strut  isn} \colorbox{Cyan!0.000}{\strut \textquotesingle{}} \colorbox{Cyan!0.000}{\strut t} \colorbox{Cyan!0.000}{\strut  what} \colorbox{Cyan!95.953}{\strut  it} \colorbox{Cyan!98.487}{\strut  used} \colorbox{Cyan!87.256}{\strut  to} \colorbox{Cyan!0.000}{\strut  be} \colorbox{Cyan!0.000}{\strut ,} \colorbox{Cyan!0.000}{\strut  by} \colorbox{Cyan!0.000}{\strut  Jean} \\
Input SAE & \num{3.354e+00} & \colorbox{Green!0.000}{\strut  (} \colorbox{Green!0.000}{\strut see} \colorbox{Green!0.000}{\strut  Fasc} \colorbox{Green!0.000}{\strut ism} \colorbox{Green!0.000}{\strut  isn} \colorbox{Green!0.000}{\strut \textquotesingle{}} \colorbox{Green!0.000}{\strut t} \colorbox{Green!0.000}{\strut  what} \colorbox{Green!22.521}{\strut  it} \colorbox{Green!19.881}{\strut  used} \colorbox{Green!29.456}{\strut  to} \colorbox{Green!0.000}{\strut  be} \colorbox{Green!0.000}{\strut ,} \colorbox{Green!0.000}{\strut  by} \colorbox{Green!0.000}{\strut  Jean} \\
Output SAE & \num{2.144e+00} & \colorbox{Magenta!0.000}{\strut  (} \colorbox{Magenta!0.000}{\strut see} \colorbox{Magenta!0.000}{\strut  Fasc} \colorbox{Magenta!0.000}{\strut ism} \colorbox{Magenta!0.000}{\strut  isn} \colorbox{Magenta!0.000}{\strut \textquotesingle{}} \colorbox{Magenta!0.000}{\strut t} \colorbox{Magenta!0.000}{\strut  what} \colorbox{Magenta!19.019}{\strut  it} \colorbox{Magenta!15.858}{\strut  used} \colorbox{Magenta!35.406}{\strut  to} \colorbox{Magenta!0.000}{\strut  be} \colorbox{Magenta!0.000}{\strut ,} \colorbox{Magenta!0.000}{\strut  by} \colorbox{Magenta!0.000}{\strut  Jean} \\
\midrule
Jacobian & \num{3.060e-01} & \colorbox{Cyan!0.000}{\strut  difficult} \colorbox{Cyan!0.000}{\strut  to} \colorbox{Cyan!0.000}{\strut  nap} \colorbox{Cyan!0.000}{\strut  enough} \colorbox{Cyan!0.000}{\strut  during} \colorbox{Cyan!0.000}{\strut  the} \colorbox{Cyan!0.000}{\strut  day} \colorbox{Cyan!0.000}{\strut .} \colorbox{Cyan!0.000}{\strut  If} \colorbox{Cyan!0.000}{\strut  your} \colorbox{Cyan!0.000}{\strut  baby} \colorbox{Cyan!0.000}{\strut  is} \colorbox{Cyan!0.000}{\strut  anything} \colorbox{Cyan!0.000}{\strut  like} \colorbox{Cyan!98.397}{\strut  mine} \\
Input SAE & \num{1.563e+00} & \colorbox{Green!0.000}{\strut  difficult} \colorbox{Green!0.000}{\strut  to} \colorbox{Green!0.000}{\strut  nap} \colorbox{Green!0.000}{\strut  enough} \colorbox{Green!0.000}{\strut  during} \colorbox{Green!0.000}{\strut  the} \colorbox{Green!0.000}{\strut  day} \colorbox{Green!0.000}{\strut .} \colorbox{Green!0.000}{\strut  If} \colorbox{Green!0.000}{\strut  your} \colorbox{Green!0.000}{\strut  baby} \colorbox{Green!0.000}{\strut  is} \colorbox{Green!0.000}{\strut  anything} \colorbox{Green!0.000}{\strut  like} \colorbox{Green!13.721}{\strut  mine} \\
Output SAE & \num{1.311e+00} & \colorbox{Magenta!0.000}{\strut  difficult} \colorbox{Magenta!0.000}{\strut  to} \colorbox{Magenta!0.000}{\strut  nap} \colorbox{Magenta!0.000}{\strut  enough} \colorbox{Magenta!0.000}{\strut  during} \colorbox{Magenta!0.000}{\strut  the} \colorbox{Magenta!0.000}{\strut  day} \colorbox{Magenta!0.000}{\strut .} \colorbox{Magenta!0.000}{\strut  If} \colorbox{Magenta!0.000}{\strut  your} \colorbox{Magenta!0.000}{\strut  baby} \colorbox{Magenta!0.000}{\strut  is} \colorbox{Magenta!0.000}{\strut  anything} \colorbox{Magenta!0.000}{\strut  like} \colorbox{Magenta!21.654}{\strut  mine} \\
\midrule
Jacobian & \num{3.059e-01} & \colorbox{Cyan!0.000}{\strut  I} \colorbox{Cyan!0.000}{\strut  enjoy} \colorbox{Cyan!0.000}{\strut  my} \colorbox{Cyan!0.000}{\strut  first} \colorbox{Cyan!0.000}{\strut  B} \colorbox{Cyan!0.000}{\strut HY} \colorbox{Cyan!0.000}{\strut G} \colorbox{Cyan!0.000}{\strut  meal} \colorbox{Cyan!0.000}{\strut .} \colorbox{Cyan!0.000}{\strut And} \colorbox{Cyan!0.000}{\strut  most} \colorbox{Cyan!0.000}{\strut  excellent} \colorbox{Cyan!98.376}{\strut  it} \colorbox{Cyan!0.000}{\strut  is} \\
Input SAE & \num{3.350e+00} & \colorbox{Green!0.000}{\strut  I} \colorbox{Green!0.000}{\strut  enjoy} \colorbox{Green!0.000}{\strut  my} \colorbox{Green!0.000}{\strut  first} \colorbox{Green!0.000}{\strut  B} \colorbox{Green!0.000}{\strut HY} \colorbox{Green!0.000}{\strut G} \colorbox{Green!0.000}{\strut  meal} \colorbox{Green!0.000}{\strut .} \colorbox{Green!0.000}{\strut And} \colorbox{Green!0.000}{\strut  most} \colorbox{Green!0.000}{\strut  excellent} \colorbox{Green!29.418}{\strut  it} \colorbox{Green!18.270}{\strut  is} \\
Output SAE & \num{1.171e+00} & \colorbox{Magenta!0.000}{\strut  I} \colorbox{Magenta!0.000}{\strut  enjoy} \colorbox{Magenta!0.000}{\strut  my} \colorbox{Magenta!0.000}{\strut  first} \colorbox{Magenta!0.000}{\strut  B} \colorbox{Magenta!0.000}{\strut HY} \colorbox{Magenta!0.000}{\strut G} \colorbox{Magenta!0.000}{\strut  meal} \colorbox{Magenta!0.000}{\strut .} \colorbox{Magenta!0.000}{\strut And} \colorbox{Magenta!0.000}{\strut  most} \colorbox{Magenta!0.000}{\strut  excellent} \colorbox{Magenta!19.334}{\strut  it} \colorbox{Magenta!0.000}{\strut  is} \\
\midrule
Jacobian & \num{3.053e-01} & \colorbox{Cyan!0.000}{\strut  he} \colorbox{Cyan!0.000}{\strut  did} \colorbox{Cyan!0.000}{\strut ,} \colorbox{Cyan!0.000}{\strut  shows} \colorbox{Cyan!0.000}{\strut  how} \colorbox{Cyan!0.000}{\strut  important} \colorbox{Cyan!0.000}{\strut  Cost} \colorbox{Cyan!98.194}{\strut ello} \colorbox{Cyan!0.000}{\strut \textquotesingle{}} \colorbox{Cyan!0.000}{\strut s} \colorbox{Cyan!95.468}{\strut  music} \colorbox{Cyan!85.791}{\strut  has} \colorbox{Cyan!0.000}{\strut  been} \colorbox{Cyan!0.000}{\strut ,} \colorbox{Cyan!0.000}{\strut  and} \\
Input SAE & \num{5.740e+00} & \colorbox{Green!0.000}{\strut  he} \colorbox{Green!0.000}{\strut  did} \colorbox{Green!0.000}{\strut ,} \colorbox{Green!0.000}{\strut  shows} \colorbox{Green!0.000}{\strut  how} \colorbox{Green!0.000}{\strut  important} \colorbox{Green!0.000}{\strut  Cost} \colorbox{Green!14.510}{\strut ello} \colorbox{Green!0.000}{\strut \textquotesingle{}} \colorbox{Green!0.000}{\strut s} \colorbox{Green!26.128}{\strut  music} \colorbox{Green!50.409}{\strut  has} \colorbox{Green!21.607}{\strut  been} \colorbox{Green!0.000}{\strut ,} \colorbox{Green!0.000}{\strut  and} \\
Output SAE & \num{2.403e+00} & \colorbox{Magenta!0.000}{\strut  he} \colorbox{Magenta!0.000}{\strut  did} \colorbox{Magenta!0.000}{\strut ,} \colorbox{Magenta!0.000}{\strut  shows} \colorbox{Magenta!0.000}{\strut  how} \colorbox{Magenta!24.698}{\strut  important} \colorbox{Magenta!0.000}{\strut  Cost} \colorbox{Magenta!10.377}{\strut ello} \colorbox{Magenta!0.000}{\strut \textquotesingle{}} \colorbox{Magenta!0.000}{\strut s} \colorbox{Magenta!20.621}{\strut  music} \colorbox{Magenta!39.681}{\strut  has} \colorbox{Magenta!0.000}{\strut  been} \colorbox{Magenta!0.000}{\strut ,} \colorbox{Magenta!0.000}{\strut  and} \\
\bottomrule
\end{longtable}
\caption{feature pairs/Layer15-65536-J1-LR5.0e-04-k32-T3.0e+08 abs mean/examples-41606-v-31729 stas c4-en-10k,train,batch size=32,ctx len=16.csv}
\end{table} % unclear
% \begin{table}
\centering
\begin{longtable}{lrl}
\toprule
Category & Max. abs. value & Example tokens \\
\midrule
Jacobian & \num{3.069e-01} & \colorbox{Cyan!90.884}{\strut k} \colorbox{Cyan!83.897}{\strut ering} \colorbox{Cyan!100.000}{\strut  and} \colorbox{Cyan!0.000}{\strut  commercial} \colorbox{Cyan!0.000}{\strut  operations} \colorbox{Cyan!0.000}{\strut  gaining} \colorbox{Cyan!0.000}{\strut  invaluable} \colorbox{Cyan!0.000}{\strut  experience} \colorbox{Cyan!0.000}{\strut  in} \colorbox{Cyan!0.000}{\strut  dealing} \colorbox{Cyan!0.000}{\strut  with} \colorbox{Cyan!0.000}{\strut  Contract} \colorbox{Cyan!0.000}{\strut s} \colorbox{Cyan!0.000}{\strut  and} \colorbox{Cyan!0.000}{\strut  Charter} \\
Input SAE & \num{7.746e+00} & \colorbox{Green!10.056}{\strut k} \colorbox{Green!63.620}{\strut ering} \colorbox{Green!19.727}{\strut  and} \colorbox{Green!0.000}{\strut  commercial} \colorbox{Green!0.000}{\strut  operations} \colorbox{Green!0.000}{\strut  gaining} \colorbox{Green!0.000}{\strut  invaluable} \colorbox{Green!0.000}{\strut  experience} \colorbox{Green!0.000}{\strut  in} \colorbox{Green!0.000}{\strut  dealing} \colorbox{Green!0.000}{\strut  with} \colorbox{Green!0.000}{\strut  Contract} \colorbox{Green!0.000}{\strut s} \colorbox{Green!0.000}{\strut  and} \colorbox{Green!0.000}{\strut  Charter} \\
Output SAE & \num{2.875e+00} & \colorbox{Magenta!19.695}{\strut k} \colorbox{Magenta!62.027}{\strut ering} \colorbox{Magenta!25.127}{\strut  and} \colorbox{Magenta!0.000}{\strut  commercial} \colorbox{Magenta!0.000}{\strut  operations} \colorbox{Magenta!0.000}{\strut  gaining} \colorbox{Magenta!0.000}{\strut  invaluable} \colorbox{Magenta!0.000}{\strut  experience} \colorbox{Magenta!0.000}{\strut  in} \colorbox{Magenta!0.000}{\strut  dealing} \colorbox{Magenta!0.000}{\strut  with} \colorbox{Magenta!0.000}{\strut  Contract} \colorbox{Magenta!0.000}{\strut s} \colorbox{Magenta!0.000}{\strut  and} \colorbox{Magenta!0.000}{\strut  Charter} \\
\midrule
Jacobian & \num{3.069e-01} & \colorbox{Cyan!0.000}{\strut  food} \colorbox{Cyan!0.000}{\strut  (} \colorbox{Cyan!0.000}{\strut usually} \colorbox{Cyan!0.000}{\strut  cookies} \colorbox{Cyan!0.000}{\strut )} \colorbox{Cyan!0.000}{\strut  out} \colorbox{Cyan!0.000}{\strut  for} \colorbox{Cyan!0.000}{\strut  Father} \colorbox{Cyan!0.000}{\strut  Christmas} \colorbox{Cyan!0.000}{\strut ,} \colorbox{Cyan!0.000}{\strut  who} \colorbox{Cyan!0.000}{\strut  in} \colorbox{Cyan!81.146}{\strut  Sweden} \colorbox{Cyan!0.000}{\strut  is} \colorbox{Cyan!99.993}{\strut  called} \\
Input SAE & \num{2.542e+00} & \colorbox{Green!0.000}{\strut  food} \colorbox{Green!0.000}{\strut  (} \colorbox{Green!0.000}{\strut usually} \colorbox{Green!0.000}{\strut  cookies} \colorbox{Green!0.000}{\strut )} \colorbox{Green!0.000}{\strut  out} \colorbox{Green!0.000}{\strut  for} \colorbox{Green!0.000}{\strut  Father} \colorbox{Green!0.000}{\strut  Christmas} \colorbox{Green!0.000}{\strut ,} \colorbox{Green!0.000}{\strut  who} \colorbox{Green!0.000}{\strut  in} \colorbox{Green!16.044}{\strut  Sweden} \colorbox{Green!0.000}{\strut  is} \colorbox{Green!20.880}{\strut  called} \\
Output SAE & \num{1.986e+00} & \colorbox{Magenta!0.000}{\strut  food} \colorbox{Magenta!0.000}{\strut  (} \colorbox{Magenta!0.000}{\strut usually} \colorbox{Magenta!0.000}{\strut  cookies} \colorbox{Magenta!0.000}{\strut )} \colorbox{Magenta!14.381}{\strut  out} \colorbox{Magenta!0.000}{\strut  for} \colorbox{Magenta!0.000}{\strut  Father} \colorbox{Magenta!0.000}{\strut  Christmas} \colorbox{Magenta!0.000}{\strut ,} \colorbox{Magenta!0.000}{\strut  who} \colorbox{Magenta!0.000}{\strut  in} \colorbox{Magenta!42.844}{\strut  Sweden} \colorbox{Magenta!0.000}{\strut  is} \colorbox{Magenta!20.818}{\strut  called} \\
\midrule
Jacobian & \num{3.047e-01} & \colorbox{Cyan!94.226}{\strut der} \colorbox{Cyan!99.278}{\strut  and} \colorbox{Cyan!87.620}{\strut  elev} \colorbox{Cyan!0.000}{\strut ators} \colorbox{Cyan!0.000}{\strut  to} \colorbox{Cyan!0.000}{\strut  change} \colorbox{Cyan!0.000}{\strut  the} \colorbox{Cyan!0.000}{\strut  flight} \colorbox{Cyan!0.000}{\strut  characteristics} \colorbox{Cyan!0.000}{\strut  of} \colorbox{Cyan!0.000}{\strut  the} \colorbox{Cyan!0.000}{\strut  dragon} \colorbox{Cyan!0.000}{\strut .} \colorbox{Cyan!0.000}{\strut  Note} \colorbox{Cyan!0.000}{\strut  that} \\
Input SAE & \num{5.520e+00} & \colorbox{Green!45.334}{\strut der} \colorbox{Green!10.877}{\strut  and} \colorbox{Green!14.502}{\strut  elev} \colorbox{Green!0.000}{\strut ators} \colorbox{Green!0.000}{\strut  to} \colorbox{Green!0.000}{\strut  change} \colorbox{Green!0.000}{\strut  the} \colorbox{Green!0.000}{\strut  flight} \colorbox{Green!0.000}{\strut  characteristics} \colorbox{Green!0.000}{\strut  of} \colorbox{Green!0.000}{\strut  the} \colorbox{Green!0.000}{\strut  dragon} \colorbox{Green!0.000}{\strut .} \colorbox{Green!0.000}{\strut  Note} \colorbox{Green!0.000}{\strut  that} \\
Output SAE & \num{1.649e+00} & \colorbox{Magenta!35.580}{\strut der} \colorbox{Magenta!19.491}{\strut  and} \colorbox{Magenta!14.240}{\strut  elev} \colorbox{Magenta!0.000}{\strut ators} \colorbox{Magenta!12.452}{\strut  to} \colorbox{Magenta!0.000}{\strut  change} \colorbox{Magenta!0.000}{\strut  the} \colorbox{Magenta!0.000}{\strut  flight} \colorbox{Magenta!0.000}{\strut  characteristics} \colorbox{Magenta!0.000}{\strut  of} \colorbox{Magenta!0.000}{\strut  the} \colorbox{Magenta!0.000}{\strut  dragon} \colorbox{Magenta!0.000}{\strut .} \colorbox{Magenta!0.000}{\strut  Note} \colorbox{Magenta!0.000}{\strut  that} \\
\midrule
Jacobian & \num{3.045e-01} & \colorbox{Cyan!90.849}{\strut er} \colorbox{Cyan!87.641}{\strut .} \colorbox{Cyan!99.208}{\strut All} \colorbox{Cyan!0.000}{\strut  software} \colorbox{Cyan!0.000}{\strut  listed} \colorbox{Cyan!0.000}{\strut  on} \colorbox{Cyan!0.000}{\strut  file} \colorbox{Cyan!0.000}{\strut .} \colorbox{Cyan!0.000}{\strut B} \colorbox{Cyan!0.000}{\strut row} \colorbox{Cyan!0.000}{\strut se} \colorbox{Cyan!0.000}{\strut  by} \colorbox{Cyan!0.000}{\strut  extension} \\
Input SAE & \num{2.710e+00} & \colorbox{Green!22.254}{\strut er} \colorbox{Green!19.290}{\strut .} \colorbox{Green!13.011}{\strut All} \colorbox{Green!0.000}{\strut  software} \colorbox{Green!0.000}{\strut  listed} \colorbox{Green!0.000}{\strut  on} \colorbox{Green!0.000}{\strut  file} \colorbox{Green!0.000}{\strut .} \colorbox{Green!0.000}{\strut B} \colorbox{Green!0.000}{\strut row} \colorbox{Green!0.000}{\strut se} \colorbox{Green!0.000}{\strut  by} \colorbox{Green!0.000}{\strut  extension} \\
Output SAE & \num{1.579e+00} & \colorbox{Magenta!34.064}{\strut er} \colorbox{Magenta!30.330}{\strut .} \colorbox{Magenta!21.846}{\strut All} \colorbox{Magenta!0.000}{\strut  software} \colorbox{Magenta!0.000}{\strut  listed} \colorbox{Magenta!0.000}{\strut  on} \colorbox{Magenta!0.000}{\strut  file} \colorbox{Magenta!0.000}{\strut .} \colorbox{Magenta!0.000}{\strut B} \colorbox{Magenta!0.000}{\strut row} \colorbox{Magenta!0.000}{\strut se} \colorbox{Magenta!0.000}{\strut  by} \colorbox{Magenta!0.000}{\strut  extension} \\
\midrule
Jacobian & \num{3.038e-01} & \colorbox{Cyan!0.000}{\strut .} \colorbox{Cyan!0.000}{\strut  Am} \colorbox{Cyan!0.000}{\strut id} \colorbox{Cyan!98.984}{\strut st} \colorbox{Cyan!0.000}{\strut  opposition} \colorbox{Cyan!0.000}{\strut  from} \colorbox{Cyan!0.000}{\strut  Chief} \colorbox{Cyan!0.000}{\strut  Min} \colorbox{Cyan!0.000}{\strut isters} \colorbox{Cyan!0.000}{\strut  and} \colorbox{Cyan!0.000}{\strut  officials} \colorbox{Cyan!0.000}{\strut  from} \colorbox{Cyan!0.000}{\strut  the} \colorbox{Cyan!0.000}{\strut  Central} \colorbox{Cyan!0.000}{\strut  Board} \\
Input SAE & \num{1.736e+00} & \colorbox{Green!0.000}{\strut .} \colorbox{Green!0.000}{\strut  Am} \colorbox{Green!0.000}{\strut id} \colorbox{Green!14.261}{\strut st} \colorbox{Green!0.000}{\strut  opposition} \colorbox{Green!0.000}{\strut  from} \colorbox{Green!0.000}{\strut  Chief} \colorbox{Green!0.000}{\strut  Min} \colorbox{Green!0.000}{\strut isters} \colorbox{Green!0.000}{\strut  and} \colorbox{Green!0.000}{\strut  officials} \colorbox{Green!0.000}{\strut  from} \colorbox{Green!0.000}{\strut  the} \colorbox{Green!0.000}{\strut  Central} \colorbox{Green!0.000}{\strut  Board} \\
Output SAE & \num{1.108e+00} & \colorbox{Magenta!0.000}{\strut .} \colorbox{Magenta!0.000}{\strut  Am} \colorbox{Magenta!0.000}{\strut id} \colorbox{Magenta!23.911}{\strut st} \colorbox{Magenta!12.107}{\strut  opposition} \colorbox{Magenta!0.000}{\strut  from} \colorbox{Magenta!0.000}{\strut  Chief} \colorbox{Magenta!0.000}{\strut  Min} \colorbox{Magenta!0.000}{\strut isters} \colorbox{Magenta!0.000}{\strut  and} \colorbox{Magenta!0.000}{\strut  officials} \colorbox{Magenta!0.000}{\strut  from} \colorbox{Magenta!0.000}{\strut  the} \colorbox{Magenta!0.000}{\strut  Central} \colorbox{Magenta!0.000}{\strut  Board} \\
\midrule
Jacobian & \num{3.023e-01} & \colorbox{Cyan!0.000}{\strut  F} \colorbox{Cyan!0.000}{\strut A14r} \colorbox{Cyan!0.000}{\strut th} \colorbox{Cyan!98.486}{\strut  und} \colorbox{Cyan!0.000}{\strut  we} \colorbox{Cyan!0.000}{\strut ite} \colorbox{Cyan!0.000}{\strut re} \colorbox{Cyan!0.000}{\strut .} \colorbox{Cyan!0.000}{\strut  Spiel} \colorbox{Cyan!0.000}{\strut hall} \colorbox{Cyan!0.000}{\strut en} \colorbox{Cyan!0.000}{\strut au} \colorbox{Cyan!0.000}{\strut fs} \colorbox{Cyan!0.000}{\strut icht} \colorbox{Cyan!0.000}{\strut en} \\
Input SAE & \num{1.768e+00} & \colorbox{Green!0.000}{\strut  F} \colorbox{Green!0.000}{\strut A14r} \colorbox{Green!0.000}{\strut th} \colorbox{Green!14.520}{\strut  und} \colorbox{Green!0.000}{\strut  we} \colorbox{Green!0.000}{\strut ite} \colorbox{Green!9.344}{\strut re} \colorbox{Green!7.450}{\strut .} \colorbox{Green!9.955}{\strut  Spiel} \colorbox{Green!9.806}{\strut hall} \colorbox{Green!0.000}{\strut en} \colorbox{Green!0.000}{\strut au} \colorbox{Green!0.000}{\strut fs} \colorbox{Green!0.000}{\strut icht} \colorbox{Green!0.000}{\strut en} \\
Output SAE & \num{9.198e-01} & \colorbox{Magenta!0.000}{\strut  F} \colorbox{Magenta!0.000}{\strut A14r} \colorbox{Magenta!15.421}{\strut th} \colorbox{Magenta!19.841}{\strut  und} \colorbox{Magenta!0.000}{\strut  we} \colorbox{Magenta!0.000}{\strut ite} \colorbox{Magenta!0.000}{\strut re} \colorbox{Magenta!0.000}{\strut .} \colorbox{Magenta!0.000}{\strut  Spiel} \colorbox{Magenta!0.000}{\strut hall} \colorbox{Magenta!0.000}{\strut en} \colorbox{Magenta!0.000}{\strut au} \colorbox{Magenta!0.000}{\strut fs} \colorbox{Magenta!0.000}{\strut icht} \colorbox{Magenta!0.000}{\strut en} \\
\midrule
Jacobian & \num{3.012e-01} & \colorbox{Cyan!0.000}{\strut ikk} \colorbox{Cyan!84.469}{\strut elsen} \colorbox{Cyan!79.938}{\strut  (} \colorbox{Cyan!82.905}{\strut Program} \colorbox{Cyan!85.413}{\strut mer} \colorbox{Cyan!98.150}{\strut ,} \colorbox{Cyan!0.000}{\strut  C} \colorbox{Cyan!0.000}{\strut PH} \colorbox{Cyan!0.000}{\strut :} \colorbox{Cyan!0.000}{\strut DO} \colorbox{Cyan!0.000}{\strut X} \colorbox{Cyan!96.701}{\strut ),} \colorbox{Cyan!82.301}{\strut  Dan} \colorbox{Cyan!0.000}{\strut  N} \colorbox{Cyan!0.000}{\strut ux} \\
Input SAE & \num{7.856e+00} & \colorbox{Green!0.000}{\strut ikk} \colorbox{Green!64.519}{\strut elsen} \colorbox{Green!13.312}{\strut  (} \colorbox{Green!35.585}{\strut Program} \colorbox{Green!54.509}{\strut mer} \colorbox{Green!20.873}{\strut ,} \colorbox{Green!12.171}{\strut  C} \colorbox{Green!0.000}{\strut PH} \colorbox{Green!0.000}{\strut :} \colorbox{Green!0.000}{\strut DO} \colorbox{Green!0.000}{\strut X} \colorbox{Green!15.664}{\strut ),} \colorbox{Green!10.720}{\strut  Dan} \colorbox{Green!0.000}{\strut  N} \colorbox{Green!0.000}{\strut ux} \\
Output SAE & \num{3.286e+00} & \colorbox{Magenta!0.000}{\strut ikk} \colorbox{Magenta!70.893}{\strut elsen} \colorbox{Magenta!12.201}{\strut  (} \colorbox{Magenta!28.997}{\strut Program} \colorbox{Magenta!55.977}{\strut mer} \colorbox{Magenta!20.247}{\strut ,} \colorbox{Magenta!0.000}{\strut  C} \colorbox{Magenta!0.000}{\strut PH} \colorbox{Magenta!0.000}{\strut :} \colorbox{Magenta!0.000}{\strut DO} \colorbox{Magenta!0.000}{\strut X} \colorbox{Magenta!22.909}{\strut ),} \colorbox{Magenta!11.633}{\strut  Dan} \colorbox{Magenta!0.000}{\strut  N} \colorbox{Magenta!0.000}{\strut ux} \\
\midrule
Jacobian & \num{3.002e-01} & \colorbox{Cyan!0.000}{\strut nd} \colorbox{Cyan!0.000}{\strut ig} \colorbox{Cyan!85.869}{\strut .} \colorbox{Cyan!93.153}{\strut  Ger} \colorbox{Cyan!96.571}{\strut ade} \colorbox{Cyan!0.000}{\strut  Exp} \colorbox{Cyan!89.400}{\strut ats} \colorbox{Cyan!97.809}{\strut ,} \colorbox{Cyan!74.406}{\strut  die} \colorbox{Cyan!0.000}{\strut  in} \colorbox{Cyan!0.000}{\strut  Thailand} \colorbox{Cyan!0.000}{\strut  le} \colorbox{Cyan!0.000}{\strut ben} \colorbox{Cyan!0.000}{\strut  und} \colorbox{Cyan!0.000}{\strut  den} \\
Input SAE & \num{3.519e+00} & \colorbox{Green!0.000}{\strut nd} \colorbox{Green!0.000}{\strut ig} \colorbox{Green!12.080}{\strut .} \colorbox{Green!17.396}{\strut  Ger} \colorbox{Green!28.904}{\strut ade} \colorbox{Green!5.789}{\strut  Exp} \colorbox{Green!12.948}{\strut ats} \colorbox{Green!23.841}{\strut ,} \colorbox{Green!12.691}{\strut  die} \colorbox{Green!0.000}{\strut  in} \colorbox{Green!0.000}{\strut  Thailand} \colorbox{Green!0.000}{\strut  le} \colorbox{Green!0.000}{\strut ben} \colorbox{Green!0.000}{\strut  und} \colorbox{Green!0.000}{\strut  den} \\
Output SAE & \num{1.310e+00} & \colorbox{Magenta!0.000}{\strut nd} \colorbox{Magenta!13.487}{\strut ig} \colorbox{Magenta!16.300}{\strut .} \colorbox{Magenta!22.722}{\strut  Ger} \colorbox{Magenta!28.269}{\strut ade} \colorbox{Magenta!0.000}{\strut  Exp} \colorbox{Magenta!17.824}{\strut ats} \colorbox{Magenta!25.996}{\strut ,} \colorbox{Magenta!14.012}{\strut  die} \colorbox{Magenta!0.000}{\strut  in} \colorbox{Magenta!0.000}{\strut  Thailand} \colorbox{Magenta!0.000}{\strut  le} \colorbox{Magenta!0.000}{\strut ben} \colorbox{Magenta!0.000}{\strut  und} \colorbox{Magenta!0.000}{\strut  den} \\
\midrule
Jacobian & \num{2.997e-01} & \colorbox{Cyan!0.000}{\strut ell} \colorbox{Cyan!97.647}{\strut  und} \colorbox{Cyan!0.000}{\strut  e} \colorbox{Cyan!0.000}{\strut inf} \colorbox{Cyan!0.000}{\strut ach} \colorbox{Cyan!0.000}{\strut  von} \colorbox{Cyan!0.000}{\strut  stat} \colorbox{Cyan!0.000}{\strut ten} \colorbox{Cyan!0.000}{\strut  ge} \colorbox{Cyan!0.000}{\strut ht} \colorbox{Cyan!0.000}{\strut .} \colorbox{Cyan!0.000}{\strut W} \colorbox{Cyan!0.000}{\strut ie} \colorbox{Cyan!0.000}{\strut  Sie} \\
Input SAE & \num{2.369e+00} & \colorbox{Green!0.000}{\strut ell} \colorbox{Green!19.460}{\strut  und} \colorbox{Green!12.191}{\strut  e} \colorbox{Green!0.000}{\strut inf} \colorbox{Green!0.000}{\strut ach} \colorbox{Green!0.000}{\strut  von} \colorbox{Green!0.000}{\strut  stat} \colorbox{Green!0.000}{\strut ten} \colorbox{Green!0.000}{\strut  ge} \colorbox{Green!0.000}{\strut ht} \colorbox{Green!0.000}{\strut .} \colorbox{Green!0.000}{\strut W} \colorbox{Green!0.000}{\strut ie} \colorbox{Green!0.000}{\strut  Sie} \\
Output SAE & \num{1.257e+00} & \colorbox{Magenta!15.582}{\strut ell} \colorbox{Magenta!27.125}{\strut  und} \colorbox{Magenta!0.000}{\strut  e} \colorbox{Magenta!0.000}{\strut inf} \colorbox{Magenta!0.000}{\strut ach} \colorbox{Magenta!0.000}{\strut  von} \colorbox{Magenta!0.000}{\strut  stat} \colorbox{Magenta!0.000}{\strut ten} \colorbox{Magenta!0.000}{\strut  ge} \colorbox{Magenta!0.000}{\strut ht} \colorbox{Magenta!0.000}{\strut .} \colorbox{Magenta!0.000}{\strut W} \colorbox{Magenta!0.000}{\strut ie} \colorbox{Magenta!0.000}{\strut  Sie} \\
\midrule
Jacobian & \num{2.989e-01} & \colorbox{Cyan!0.000}{\strut  Disney} \colorbox{Cyan!97.388}{\strut  und} \colorbox{Cyan!0.000}{\strut  Lucas} \colorbox{Cyan!90.775}{\strut film} \colorbox{Cyan!0.000}{\strut  ent} \colorbox{Cyan!0.000}{\strut stand} \colorbox{Cyan!0.000}{\strut  star} \colorbox{Cyan!0.000}{\strut  wars} \colorbox{Cyan!0.000}{\strut  at} \colorbox{Cyan!0.000}{\strut  Madame} \colorbox{Cyan!0.000}{\strut  T} \colorbox{Cyan!0.000}{\strut uss} \colorbox{Cyan!0.000}{\strut aud} \colorbox{Cyan!0.000}{\strut s} \colorbox{Cyan!0.000}{\strut  Berlin} \\
Input SAE & \num{1.941e+00} & \colorbox{Green!0.000}{\strut  Disney} \colorbox{Green!13.073}{\strut  und} \colorbox{Green!0.000}{\strut  Lucas} \colorbox{Green!15.943}{\strut film} \colorbox{Green!7.982}{\strut  ent} \colorbox{Green!0.000}{\strut stand} \colorbox{Green!13.147}{\strut  star} \colorbox{Green!11.939}{\strut  wars} \colorbox{Green!0.000}{\strut  at} \colorbox{Green!0.000}{\strut  Madame} \colorbox{Green!0.000}{\strut  T} \colorbox{Green!0.000}{\strut uss} \colorbox{Green!0.000}{\strut aud} \colorbox{Green!0.000}{\strut s} \colorbox{Green!0.000}{\strut  Berlin} \\
Output SAE & \num{1.007e+00} & \colorbox{Magenta!0.000}{\strut  Disney} \colorbox{Magenta!19.205}{\strut  und} \colorbox{Magenta!0.000}{\strut  Lucas} \colorbox{Magenta!21.717}{\strut film} \colorbox{Magenta!0.000}{\strut  ent} \colorbox{Magenta!0.000}{\strut stand} \colorbox{Magenta!0.000}{\strut  star} \colorbox{Magenta!0.000}{\strut  wars} \colorbox{Magenta!0.000}{\strut  at} \colorbox{Magenta!0.000}{\strut  Madame} \colorbox{Magenta!0.000}{\strut  T} \colorbox{Magenta!0.000}{\strut uss} \colorbox{Magenta!0.000}{\strut aud} \colorbox{Magenta!0.000}{\strut s} \colorbox{Magenta!0.000}{\strut  Berlin} \\
\midrule
Jacobian & \num{2.989e-01} & \colorbox{Cyan!97.385}{\strut  og} \colorbox{Cyan!84.366}{\strut  R} \colorbox{Cyan!77.319}{\strut oms} \colorbox{Cyan!84.960}{\strut dal} \colorbox{Cyan!90.139}{\strut ,} \colorbox{Cyan!88.044}{\strut  in} \colorbox{Cyan!80.885}{\strut  Norway} \colorbox{Cyan!86.643}{\strut .} \colorbox{Cyan!0.000}{\strut 484} \colorbox{Cyan!0.000}{\strut  Lav} \colorbox{Cyan!82.353}{\strut ang} \colorbox{Cyan!83.354}{\strut st} \colorbox{Cyan!85.359}{\strut ind} \colorbox{Cyan!85.236}{\strut en} \\
Input SAE & \num{8.642e+00} & \colorbox{Green!5.105}{\strut  og} \colorbox{Green!32.082}{\strut  R} \colorbox{Green!20.599}{\strut oms} \colorbox{Green!65.097}{\strut dal} \colorbox{Green!41.147}{\strut ,} \colorbox{Green!52.092}{\strut  in} \colorbox{Green!27.327}{\strut  Norway} \colorbox{Green!14.680}{\strut .} \colorbox{Green!0.000}{\strut 484} \colorbox{Green!14.389}{\strut  Lav} \colorbox{Green!49.291}{\strut ang} \colorbox{Green!58.541}{\strut st} \colorbox{Green!70.977}{\strut ind} \colorbox{Green!43.673}{\strut en} \\
Output SAE & \num{3.367e+00} & \colorbox{Magenta!23.412}{\strut  og} \colorbox{Magenta!32.275}{\strut  R} \colorbox{Magenta!37.615}{\strut oms} \colorbox{Magenta!72.639}{\strut dal} \colorbox{Magenta!41.172}{\strut ,} \colorbox{Magenta!44.571}{\strut  in} \colorbox{Magenta!45.665}{\strut  Norway} \colorbox{Magenta!16.142}{\strut .} \colorbox{Magenta!0.000}{\strut 484} \colorbox{Magenta!0.000}{\strut  Lav} \colorbox{Magenta!48.849}{\strut ang} \colorbox{Magenta!51.316}{\strut st} \colorbox{Magenta!71.986}{\strut ind} \colorbox{Magenta!52.186}{\strut en} \\
\midrule
Jacobian & \num{2.987e-01} & \colorbox{Cyan!0.000}{\strut fs} \colorbox{Cyan!0.000}{\strut icht} \colorbox{Cyan!0.000}{\strut  Jobs} \colorbox{Cyan!97.338}{\strut  und} \colorbox{Cyan!0.000}{\strut  St} \colorbox{Cyan!0.000}{\strut ellen} \colorbox{Cyan!0.000}{\strut ange} \colorbox{Cyan!0.000}{\strut b} \colorbox{Cyan!0.000}{\strut ote} \colorbox{Cyan!0.000}{\strut .} \colorbox{Cyan!0.000}{\strut  Hamburg} \colorbox{Cyan!0.000}{\strut C} \colorbox{Cyan!0.000}{\strut ob} \colorbox{Cyan!0.000}{\strut urg} \colorbox{Cyan!0.000}{\strut ,} \\
Input SAE & \num{1.584e+00} & \colorbox{Green!0.000}{\strut fs} \colorbox{Green!11.615}{\strut icht} \colorbox{Green!11.843}{\strut  Jobs} \colorbox{Green!13.009}{\strut  und} \colorbox{Green!9.818}{\strut  St} \colorbox{Green!0.000}{\strut ellen} \colorbox{Green!0.000}{\strut ange} \colorbox{Green!0.000}{\strut b} \colorbox{Green!0.000}{\strut ote} \colorbox{Green!12.709}{\strut .} \colorbox{Green!0.000}{\strut  Hamburg} \colorbox{Green!0.000}{\strut C} \colorbox{Green!0.000}{\strut ob} \colorbox{Green!0.000}{\strut urg} \colorbox{Green!0.000}{\strut ,} \\
Output SAE & \num{7.930e-01} & \colorbox{Magenta!15.989}{\strut fs} \colorbox{Magenta!0.000}{\strut icht} \colorbox{Magenta!0.000}{\strut  Jobs} \colorbox{Magenta!17.106}{\strut  und} \colorbox{Magenta!0.000}{\strut  St} \colorbox{Magenta!0.000}{\strut ellen} \colorbox{Magenta!0.000}{\strut ange} \colorbox{Magenta!0.000}{\strut b} \colorbox{Magenta!0.000}{\strut ote} \colorbox{Magenta!0.000}{\strut .} \colorbox{Magenta!0.000}{\strut  Hamburg} \colorbox{Magenta!0.000}{\strut C} \colorbox{Magenta!0.000}{\strut ob} \colorbox{Magenta!0.000}{\strut urg} \colorbox{Magenta!0.000}{\strut ,} \\
\bottomrule
\end{longtable}
\caption{feature pairs/Layer15-65536-J1-LR5.0e-04-k32-T3.0e+08 abs mean/examples-23581-v-64386 stas c4-en-10k,train,batch size=32,ctx len=16.csv}
\end{table}
% \input{feature_pairs_tex/pythia-410m-layer-15-abs-mean-in-61448-out-29281-b32-t16}
% \begin{table}
\centering
\begin{longtable}{lrl}
\toprule
Category & Max. abs. value & Example tokens \\
\midrule
Jacobian & \num{3.128e-01} & \colorbox{Cyan!0.000}{\strut  through} \colorbox{Cyan!0.000}{\strut  the} \colorbox{Cyan!0.000}{\strut  gallery} \colorbox{Cyan!0.000}{\strut .} \colorbox{Cyan!0.000}{\strut  Enjoy} \colorbox{Cyan!0.000}{\strut .} \colorbox{Cyan!0.000}{\strut For} \colorbox{Cyan!0.000}{\strut  a} \colorbox{Cyan!0.000}{\strut  birthday} \colorbox{Cyan!0.000}{\strut  treat} \colorbox{Cyan!0.000}{\strut  I} \colorbox{Cyan!0.000}{\strut  decided} \colorbox{Cyan!0.000}{\strut  to} \colorbox{Cyan!100.000}{\strut  go} \colorbox{Cyan!0.000}{\strut  to} \\
Input SAE & \num{1.720e+01} & \colorbox{Green!0.000}{\strut  through} \colorbox{Green!0.000}{\strut  the} \colorbox{Green!0.000}{\strut  gallery} \colorbox{Green!0.000}{\strut .} \colorbox{Green!0.000}{\strut  Enjoy} \colorbox{Green!0.000}{\strut .} \colorbox{Green!0.000}{\strut For} \colorbox{Green!0.000}{\strut  a} \colorbox{Green!0.000}{\strut  birthday} \colorbox{Green!0.000}{\strut  treat} \colorbox{Green!0.000}{\strut  I} \colorbox{Green!0.000}{\strut  decided} \colorbox{Green!0.000}{\strut  to} \colorbox{Green!81.737}{\strut  go} \colorbox{Green!0.000}{\strut  to} \\
Output SAE & \num{5.101e+00} & \colorbox{Magenta!0.000}{\strut  through} \colorbox{Magenta!0.000}{\strut  the} \colorbox{Magenta!0.000}{\strut  gallery} \colorbox{Magenta!0.000}{\strut .} \colorbox{Magenta!0.000}{\strut  Enjoy} \colorbox{Magenta!0.000}{\strut .} \colorbox{Magenta!0.000}{\strut For} \colorbox{Magenta!0.000}{\strut  a} \colorbox{Magenta!0.000}{\strut  birthday} \colorbox{Magenta!0.000}{\strut  treat} \colorbox{Magenta!0.000}{\strut  I} \colorbox{Magenta!0.000}{\strut  decided} \colorbox{Magenta!0.000}{\strut  to} \colorbox{Magenta!89.080}{\strut  go} \colorbox{Magenta!13.633}{\strut  to} \\
\midrule
Jacobian & \num{3.127e-01} & \colorbox{Cyan!0.000}{\strut  \$} \colorbox{Cyan!0.000}{\strut 20} \colorbox{Cyan!0.000}{\strut ,} \colorbox{Cyan!0.000}{\strut  and} \colorbox{Cyan!0.000}{\strut  looking} \colorbox{Cyan!0.000}{\strut  to} \colorbox{Cyan!99.970}{\strut  go} \colorbox{Cyan!0.000}{\strut  as} \colorbox{Cyan!0.000}{\strut  high} \colorbox{Cyan!0.000}{\strut  as} \colorbox{Cyan!0.000}{\strut  \$} \colorbox{Cyan!0.000}{\strut 40} \colorbox{Cyan!0.000}{\strut  in} \colorbox{Cyan!0.000}{\strut  the} \\
Input SAE & \num{1.736e+01} & \colorbox{Green!0.000}{\strut  \$} \colorbox{Green!0.000}{\strut 20} \colorbox{Green!0.000}{\strut ,} \colorbox{Green!0.000}{\strut  and} \colorbox{Green!0.000}{\strut  looking} \colorbox{Green!0.000}{\strut  to} \colorbox{Green!82.521}{\strut  go} \colorbox{Green!0.000}{\strut  as} \colorbox{Green!0.000}{\strut  high} \colorbox{Green!0.000}{\strut  as} \colorbox{Green!0.000}{\strut  \$} \colorbox{Green!0.000}{\strut 40} \colorbox{Green!0.000}{\strut  in} \colorbox{Green!0.000}{\strut  the} \\
Output SAE & \num{4.855e+00} & \colorbox{Magenta!0.000}{\strut  \$} \colorbox{Magenta!0.000}{\strut 20} \colorbox{Magenta!0.000}{\strut ,} \colorbox{Magenta!0.000}{\strut  and} \colorbox{Magenta!10.428}{\strut  looking} \colorbox{Magenta!0.000}{\strut  to} \colorbox{Magenta!84.774}{\strut  go} \colorbox{Magenta!11.610}{\strut  as} \colorbox{Magenta!0.000}{\strut  high} \colorbox{Magenta!0.000}{\strut  as} \colorbox{Magenta!0.000}{\strut  \$} \colorbox{Magenta!0.000}{\strut 40} \colorbox{Magenta!0.000}{\strut  in} \colorbox{Magenta!0.000}{\strut  the} \\
\midrule
Jacobian & \num{3.121e-01} & \colorbox{Cyan!0.000}{\strut  action} \colorbox{Cyan!0.000}{\strut .} \colorbox{Cyan!0.000}{\strut  I} \colorbox{Cyan!0.000}{\strut  am} \colorbox{Cyan!0.000}{\strut  making} \colorbox{Cyan!0.000}{\strut  a} \colorbox{Cyan!0.000}{\strut  promise} \colorbox{Cyan!0.000}{\strut  to} \colorbox{Cyan!0.000}{\strut  myself} \colorbox{Cyan!0.000}{\strut  to} \colorbox{Cyan!99.758}{\strut  go} \colorbox{Cyan!0.000}{\strut  confident} \colorbox{Cyan!0.000}{\strut ly} \colorbox{Cyan!0.000}{\strut  in} \colorbox{Cyan!0.000}{\strut  the} \\
Input SAE & \num{1.860e+01} & \colorbox{Green!0.000}{\strut  action} \colorbox{Green!0.000}{\strut .} \colorbox{Green!0.000}{\strut  I} \colorbox{Green!0.000}{\strut  am} \colorbox{Green!0.000}{\strut  making} \colorbox{Green!0.000}{\strut  a} \colorbox{Green!0.000}{\strut  promise} \colorbox{Green!0.000}{\strut  to} \colorbox{Green!0.000}{\strut  myself} \colorbox{Green!0.000}{\strut  to} \colorbox{Green!88.392}{\strut  go} \colorbox{Green!0.000}{\strut  confident} \colorbox{Green!0.000}{\strut ly} \colorbox{Green!0.000}{\strut  in} \colorbox{Green!0.000}{\strut  the} \\
Output SAE & \num{5.388e+00} & \colorbox{Magenta!0.000}{\strut  action} \colorbox{Magenta!0.000}{\strut .} \colorbox{Magenta!0.000}{\strut  I} \colorbox{Magenta!0.000}{\strut  am} \colorbox{Magenta!0.000}{\strut  making} \colorbox{Magenta!0.000}{\strut  a} \colorbox{Magenta!0.000}{\strut  promise} \colorbox{Magenta!0.000}{\strut  to} \colorbox{Magenta!0.000}{\strut  myself} \colorbox{Magenta!0.000}{\strut  to} \colorbox{Magenta!94.087}{\strut  go} \colorbox{Magenta!0.000}{\strut  confident} \colorbox{Magenta!18.679}{\strut ly} \colorbox{Magenta!13.439}{\strut  in} \colorbox{Magenta!0.000}{\strut  the} \\
\midrule
Jacobian & \num{3.118e-01} & \colorbox{Cyan!0.000}{\strut en} \colorbox{Cyan!0.000}{\strut  your} \colorbox{Cyan!0.000}{\strut  impact} \colorbox{Cyan!0.000}{\strut  on} \colorbox{Cyan!0.000}{\strut  the} \colorbox{Cyan!0.000}{\strut  environment} \colorbox{Cyan!0.000}{\strut .} \colorbox{Cyan!0.000}{\strut Dec} \colorbox{Cyan!0.000}{\strut iding} \colorbox{Cyan!0.000}{\strut  to} \colorbox{Cyan!99.674}{\strut  go} \colorbox{Cyan!0.000}{\strut  solar} \colorbox{Cyan!0.000}{\strut  is} \colorbox{Cyan!0.000}{\strut  something} \\
Input SAE & \num{1.836e+01} & \colorbox{Green!0.000}{\strut en} \colorbox{Green!0.000}{\strut  your} \colorbox{Green!0.000}{\strut  impact} \colorbox{Green!0.000}{\strut  on} \colorbox{Green!0.000}{\strut  the} \colorbox{Green!0.000}{\strut  environment} \colorbox{Green!0.000}{\strut .} \colorbox{Green!0.000}{\strut Dec} \colorbox{Green!0.000}{\strut iding} \colorbox{Green!0.000}{\strut  to} \colorbox{Green!87.250}{\strut  go} \colorbox{Green!0.000}{\strut  solar} \colorbox{Green!0.000}{\strut  is} \colorbox{Green!0.000}{\strut  something} \\
Output SAE & \num{5.076e+00} & \colorbox{Magenta!0.000}{\strut en} \colorbox{Magenta!0.000}{\strut  your} \colorbox{Magenta!0.000}{\strut  impact} \colorbox{Magenta!0.000}{\strut  on} \colorbox{Magenta!0.000}{\strut  the} \colorbox{Magenta!0.000}{\strut  environment} \colorbox{Magenta!0.000}{\strut .} \colorbox{Magenta!0.000}{\strut Dec} \colorbox{Magenta!0.000}{\strut iding} \colorbox{Magenta!0.000}{\strut  to} \colorbox{Magenta!88.636}{\strut  go} \colorbox{Magenta!0.000}{\strut  solar} \colorbox{Magenta!0.000}{\strut  is} \colorbox{Magenta!0.000}{\strut  something} \\
\midrule
Jacobian & \num{3.118e-01} & \colorbox{Cyan!0.000}{\strut And} \colorbox{Cyan!0.000}{\strut  for} \colorbox{Cyan!0.000}{\strut  those} \colorbox{Cyan!0.000}{\strut  of} \colorbox{Cyan!0.000}{\strut  you} \colorbox{Cyan!0.000}{\strut  who} \colorbox{Cyan!0.000}{\strut  want} \colorbox{Cyan!0.000}{\strut  to} \colorbox{Cyan!99.661}{\strut  go} \colorbox{Cyan!0.000}{\strut  above} \colorbox{Cyan!0.000}{\strut  and} \colorbox{Cyan!0.000}{\strut  beyond} \colorbox{Cyan!0.000}{\strut  your} \colorbox{Cyan!0.000}{\strut  ability} \\
Input SAE & \num{1.875e+01} & \colorbox{Green!0.000}{\strut And} \colorbox{Green!0.000}{\strut  for} \colorbox{Green!0.000}{\strut  those} \colorbox{Green!0.000}{\strut  of} \colorbox{Green!0.000}{\strut  you} \colorbox{Green!0.000}{\strut  who} \colorbox{Green!0.000}{\strut  want} \colorbox{Green!0.000}{\strut  to} \colorbox{Green!89.096}{\strut  go} \colorbox{Green!0.000}{\strut  above} \colorbox{Green!0.000}{\strut  and} \colorbox{Green!0.000}{\strut  beyond} \colorbox{Green!0.000}{\strut  your} \colorbox{Green!0.000}{\strut  ability} \\
Output SAE & \num{4.918e+00} & \colorbox{Magenta!0.000}{\strut And} \colorbox{Magenta!0.000}{\strut  for} \colorbox{Magenta!0.000}{\strut  those} \colorbox{Magenta!0.000}{\strut  of} \colorbox{Magenta!0.000}{\strut  you} \colorbox{Magenta!0.000}{\strut  who} \colorbox{Magenta!0.000}{\strut  want} \colorbox{Magenta!0.000}{\strut  to} \colorbox{Magenta!85.884}{\strut  go} \colorbox{Magenta!0.000}{\strut  above} \colorbox{Magenta!0.000}{\strut  and} \colorbox{Magenta!0.000}{\strut  beyond} \colorbox{Magenta!0.000}{\strut  your} \colorbox{Magenta!0.000}{\strut  ability} \\
\midrule
Jacobian & \num{3.117e-01} & \colorbox{Cyan!0.000}{\strut  The} \colorbox{Cyan!0.000}{\strut  cheap} \colorbox{Cyan!0.000}{\strut  flights} \colorbox{Cyan!0.000}{\strut  also} \colorbox{Cyan!0.000}{\strut  encouraged} \colorbox{Cyan!0.000}{\strut  us} \colorbox{Cyan!0.000}{\strut  to} \colorbox{Cyan!0.000}{\strut  finally} \colorbox{Cyan!99.638}{\strut  go} \colorbox{Cyan!0.000}{\strut .} \colorbox{Cyan!0.000}{\strut  I} \colorbox{Cyan!0.000}{\strut \textquotesingle{}} \colorbox{Cyan!0.000}{\strut m} \colorbox{Cyan!0.000}{\strut  so} \colorbox{Cyan!0.000}{\strut  glad} \\
Input SAE & \num{1.768e+01} & \colorbox{Green!0.000}{\strut  The} \colorbox{Green!0.000}{\strut  cheap} \colorbox{Green!0.000}{\strut  flights} \colorbox{Green!0.000}{\strut  also} \colorbox{Green!0.000}{\strut  encouraged} \colorbox{Green!0.000}{\strut  us} \colorbox{Green!0.000}{\strut  to} \colorbox{Green!0.000}{\strut  finally} \colorbox{Green!84.050}{\strut  go} \colorbox{Green!0.000}{\strut .} \colorbox{Green!0.000}{\strut  I} \colorbox{Green!0.000}{\strut \textquotesingle{}} \colorbox{Green!0.000}{\strut m} \colorbox{Green!0.000}{\strut  so} \colorbox{Green!0.000}{\strut  glad} \\
Output SAE & \num{5.091e+00} & \colorbox{Magenta!0.000}{\strut  The} \colorbox{Magenta!0.000}{\strut  cheap} \colorbox{Magenta!0.000}{\strut  flights} \colorbox{Magenta!0.000}{\strut  also} \colorbox{Magenta!0.000}{\strut  encouraged} \colorbox{Magenta!0.000}{\strut  us} \colorbox{Magenta!0.000}{\strut  to} \colorbox{Magenta!0.000}{\strut  finally} \colorbox{Magenta!88.904}{\strut  go} \colorbox{Magenta!0.000}{\strut .} \colorbox{Magenta!0.000}{\strut  I} \colorbox{Magenta!0.000}{\strut \textquotesingle{}} \colorbox{Magenta!0.000}{\strut m} \colorbox{Magenta!0.000}{\strut  so} \colorbox{Magenta!0.000}{\strut  glad} \\
\midrule
Jacobian & \num{3.115e-01} & \colorbox{Cyan!0.000}{\strut .} \colorbox{Cyan!0.000}{\strut  I} \colorbox{Cyan!0.000}{\strut  also} \colorbox{Cyan!0.000}{\strut  decided} \colorbox{Cyan!0.000}{\strut  to} \colorbox{Cyan!99.587}{\strut  go} \colorbox{Cyan!0.000}{\strut  searching} \colorbox{Cyan!0.000}{\strut  to} \colorbox{Cyan!0.000}{\strut  see} \colorbox{Cyan!0.000}{\strut  if} \colorbox{Cyan!0.000}{\strut  i} \colorbox{Cyan!0.000}{\strut  should} \colorbox{Cyan!0.000}{\strut  discover} \colorbox{Cyan!0.000}{\strut  producers} \colorbox{Cyan!0.000}{\strut  manual} \\
Input SAE & \num{1.825e+01} & \colorbox{Green!0.000}{\strut .} \colorbox{Green!0.000}{\strut  I} \colorbox{Green!0.000}{\strut  also} \colorbox{Green!0.000}{\strut  decided} \colorbox{Green!0.000}{\strut  to} \colorbox{Green!86.720}{\strut  go} \colorbox{Green!0.000}{\strut  searching} \colorbox{Green!0.000}{\strut  to} \colorbox{Green!0.000}{\strut  see} \colorbox{Green!0.000}{\strut  if} \colorbox{Green!0.000}{\strut  i} \colorbox{Green!0.000}{\strut  should} \colorbox{Green!0.000}{\strut  discover} \colorbox{Green!0.000}{\strut  producers} \colorbox{Green!0.000}{\strut  manual} \\
Output SAE & \num{5.236e+00} & \colorbox{Magenta!0.000}{\strut .} \colorbox{Magenta!0.000}{\strut  I} \colorbox{Magenta!0.000}{\strut  also} \colorbox{Magenta!0.000}{\strut  decided} \colorbox{Magenta!0.000}{\strut  to} \colorbox{Magenta!91.434}{\strut  go} \colorbox{Magenta!0.000}{\strut  searching} \colorbox{Magenta!0.000}{\strut  to} \colorbox{Magenta!0.000}{\strut  see} \colorbox{Magenta!0.000}{\strut  if} \colorbox{Magenta!0.000}{\strut  i} \colorbox{Magenta!0.000}{\strut  should} \colorbox{Magenta!0.000}{\strut  discover} \colorbox{Magenta!0.000}{\strut  producers} \colorbox{Magenta!0.000}{\strut  manual} \\
\midrule
Jacobian & \num{3.115e-01} & \colorbox{Cyan!0.000}{\strut  leading} \colorbox{Cyan!0.000}{\strut  insurers} \colorbox{Cyan!0.000}{\strut ,} \colorbox{Cyan!0.000}{\strut  we} \colorbox{Cyan!0.000}{\strut  are} \colorbox{Cyan!0.000}{\strut  price} \colorbox{Cyan!0.000}{\strut  competitive} \colorbox{Cyan!0.000}{\strut  and} \colorbox{Cyan!0.000}{\strut  look} \colorbox{Cyan!0.000}{\strut  to} \colorbox{Cyan!99.581}{\strut  go} \colorbox{Cyan!0.000}{\strut  over} \colorbox{Cyan!0.000}{\strut  and} \colorbox{Cyan!0.000}{\strut  above} \colorbox{Cyan!0.000}{\strut  client} \\
Input SAE & \num{1.830e+01} & \colorbox{Green!0.000}{\strut  leading} \colorbox{Green!0.000}{\strut  insurers} \colorbox{Green!0.000}{\strut ,} \colorbox{Green!0.000}{\strut  we} \colorbox{Green!0.000}{\strut  are} \colorbox{Green!0.000}{\strut  price} \colorbox{Green!0.000}{\strut  competitive} \colorbox{Green!0.000}{\strut  and} \colorbox{Green!0.000}{\strut  look} \colorbox{Green!0.000}{\strut  to} \colorbox{Green!86.969}{\strut  go} \colorbox{Green!0.000}{\strut  over} \colorbox{Green!0.000}{\strut  and} \colorbox{Green!0.000}{\strut  above} \colorbox{Green!0.000}{\strut  client} \\
Output SAE & \num{5.248e+00} & \colorbox{Magenta!0.000}{\strut  leading} \colorbox{Magenta!0.000}{\strut  insurers} \colorbox{Magenta!0.000}{\strut ,} \colorbox{Magenta!0.000}{\strut  we} \colorbox{Magenta!0.000}{\strut  are} \colorbox{Magenta!0.000}{\strut  price} \colorbox{Magenta!0.000}{\strut  competitive} \colorbox{Magenta!0.000}{\strut  and} \colorbox{Magenta!0.000}{\strut  look} \colorbox{Magenta!0.000}{\strut  to} \colorbox{Magenta!91.637}{\strut  go} \colorbox{Magenta!0.000}{\strut  over} \colorbox{Magenta!0.000}{\strut  and} \colorbox{Magenta!0.000}{\strut  above} \colorbox{Magenta!0.000}{\strut  client} \\
\midrule
Jacobian & \num{3.114e-01} & \colorbox{Cyan!0.000}{\strut  the} \colorbox{Cyan!0.000}{\strut  O} \colorbox{Cyan!0.000}{\strut HL} \colorbox{Cyan!0.000}{\strut .} \colorbox{Cyan!0.000}{\strut For} \colorbox{Cyan!0.000}{\strut  a} \colorbox{Cyan!0.000}{\strut  speedy} \colorbox{Cyan!0.000}{\strut  refres} \colorbox{Cyan!0.000}{\strut her} \colorbox{Cyan!0.000}{\strut  and} \colorbox{Cyan!0.000}{\strut  when} \colorbox{Cyan!0.000}{\strut  you} \colorbox{Cyan!0.000}{\strut  want} \colorbox{Cyan!0.000}{\strut  to} \colorbox{Cyan!99.546}{\strut  go} \\
Input SAE & \num{1.902e+01} & \colorbox{Green!0.000}{\strut  the} \colorbox{Green!0.000}{\strut  O} \colorbox{Green!0.000}{\strut HL} \colorbox{Green!0.000}{\strut .} \colorbox{Green!0.000}{\strut For} \colorbox{Green!0.000}{\strut  a} \colorbox{Green!0.000}{\strut  speedy} \colorbox{Green!0.000}{\strut  refres} \colorbox{Green!0.000}{\strut her} \colorbox{Green!0.000}{\strut  and} \colorbox{Green!0.000}{\strut  when} \colorbox{Green!0.000}{\strut  you} \colorbox{Green!0.000}{\strut  want} \colorbox{Green!0.000}{\strut  to} \colorbox{Green!90.423}{\strut  go} \\
Output SAE & \num{5.053e+00} & \colorbox{Magenta!0.000}{\strut  the} \colorbox{Magenta!0.000}{\strut  O} \colorbox{Magenta!0.000}{\strut HL} \colorbox{Magenta!0.000}{\strut .} \colorbox{Magenta!0.000}{\strut For} \colorbox{Magenta!0.000}{\strut  a} \colorbox{Magenta!0.000}{\strut  speedy} \colorbox{Magenta!0.000}{\strut  refres} \colorbox{Magenta!0.000}{\strut her} \colorbox{Magenta!0.000}{\strut  and} \colorbox{Magenta!0.000}{\strut  when} \colorbox{Magenta!0.000}{\strut  you} \colorbox{Magenta!0.000}{\strut  want} \colorbox{Magenta!0.000}{\strut  to} \colorbox{Magenta!88.238}{\strut  go} \\
\midrule
Jacobian & \num{3.109e-01} & \colorbox{Cyan!0.000}{\strut HL} \colorbox{Cyan!0.000}{\strut .} \colorbox{Cyan!0.000}{\strut For} \colorbox{Cyan!0.000}{\strut  a} \colorbox{Cyan!0.000}{\strut  speedy} \colorbox{Cyan!0.000}{\strut  refres} \colorbox{Cyan!0.000}{\strut her} \colorbox{Cyan!0.000}{\strut  and} \colorbox{Cyan!0.000}{\strut  when} \colorbox{Cyan!0.000}{\strut  you} \colorbox{Cyan!0.000}{\strut  want} \colorbox{Cyan!0.000}{\strut  to} \colorbox{Cyan!99.400}{\strut  go} \colorbox{Cyan!0.000}{\strut  deeper} \colorbox{Cyan!0.000}{\strut  and} \\
Input SAE & \num{1.904e+01} & \colorbox{Green!0.000}{\strut HL} \colorbox{Green!0.000}{\strut .} \colorbox{Green!0.000}{\strut For} \colorbox{Green!0.000}{\strut  a} \colorbox{Green!0.000}{\strut  speedy} \colorbox{Green!0.000}{\strut  refres} \colorbox{Green!0.000}{\strut her} \colorbox{Green!0.000}{\strut  and} \colorbox{Green!0.000}{\strut  when} \colorbox{Green!0.000}{\strut  you} \colorbox{Green!0.000}{\strut  want} \colorbox{Green!0.000}{\strut  to} \colorbox{Green!90.490}{\strut  go} \colorbox{Green!0.000}{\strut  deeper} \colorbox{Green!0.000}{\strut  and} \\
Output SAE & \num{5.044e+00} & \colorbox{Magenta!0.000}{\strut HL} \colorbox{Magenta!0.000}{\strut .} \colorbox{Magenta!0.000}{\strut For} \colorbox{Magenta!0.000}{\strut  a} \colorbox{Magenta!0.000}{\strut  speedy} \colorbox{Magenta!0.000}{\strut  refres} \colorbox{Magenta!0.000}{\strut her} \colorbox{Magenta!0.000}{\strut  and} \colorbox{Magenta!0.000}{\strut  when} \colorbox{Magenta!0.000}{\strut  you} \colorbox{Magenta!0.000}{\strut  want} \colorbox{Magenta!0.000}{\strut  to} \colorbox{Magenta!88.074}{\strut  go} \colorbox{Magenta!0.000}{\strut  deeper} \colorbox{Magenta!0.000}{\strut  and} \\
\midrule
Jacobian & \num{3.106e-01} & \colorbox{Cyan!0.000}{\strut  has} \colorbox{Cyan!0.000}{\strut  to} \colorbox{Cyan!0.000}{\strut  inspire} \colorbox{Cyan!0.000}{\strut  enough} \colorbox{Cyan!0.000}{\strut  confidence} \colorbox{Cyan!0.000}{\strut  from} \colorbox{Cyan!0.000}{\strut  the} \colorbox{Cyan!0.000}{\strut  writer} \colorbox{Cyan!0.000}{\strut  to} \colorbox{Cyan!99.285}{\strut  go} \colorbox{Cyan!66.808}{\strut  out} \colorbox{Cyan!0.000}{\strut  on} \colorbox{Cyan!0.000}{\strut  a} \colorbox{Cyan!0.000}{\strut  limb} \colorbox{Cyan!0.000}{\strut  for} \\
Input SAE & \num{2.005e+01} & \colorbox{Green!0.000}{\strut  has} \colorbox{Green!0.000}{\strut  to} \colorbox{Green!0.000}{\strut  inspire} \colorbox{Green!0.000}{\strut  enough} \colorbox{Green!0.000}{\strut  confidence} \colorbox{Green!0.000}{\strut  from} \colorbox{Green!0.000}{\strut  the} \colorbox{Green!0.000}{\strut  writer} \colorbox{Green!0.000}{\strut  to} \colorbox{Green!95.290}{\strut  go} \colorbox{Green!9.495}{\strut  out} \colorbox{Green!0.000}{\strut  on} \colorbox{Green!0.000}{\strut  a} \colorbox{Green!0.000}{\strut  limb} \colorbox{Green!0.000}{\strut  for} \\
Output SAE & \num{5.704e+00} & \colorbox{Magenta!0.000}{\strut  has} \colorbox{Magenta!0.000}{\strut  to} \colorbox{Magenta!0.000}{\strut  inspire} \colorbox{Magenta!0.000}{\strut  enough} \colorbox{Magenta!0.000}{\strut  confidence} \colorbox{Magenta!0.000}{\strut  from} \colorbox{Magenta!0.000}{\strut  the} \colorbox{Magenta!0.000}{\strut  writer} \colorbox{Magenta!0.000}{\strut  to} \colorbox{Magenta!99.600}{\strut  go} \colorbox{Magenta!26.550}{\strut  out} \colorbox{Magenta!0.000}{\strut  on} \colorbox{Magenta!0.000}{\strut  a} \colorbox{Magenta!0.000}{\strut  limb} \colorbox{Magenta!0.000}{\strut  for} \\
\midrule
Jacobian & \num{3.106e-01} & \colorbox{Cyan!0.000}{\strut  something} \colorbox{Cyan!0.000}{\strut  simple} \colorbox{Cyan!0.000}{\strut :} \colorbox{Cyan!0.000}{\strut  a} \colorbox{Cyan!0.000}{\strut  pillow} \colorbox{Cyan!0.000}{\strut .} \colorbox{Cyan!0.000}{\strut  I} \colorbox{Cyan!0.000}{\strut  also} \colorbox{Cyan!0.000}{\strut  decided} \colorbox{Cyan!0.000}{\strut  to} \colorbox{Cyan!99.282}{\strut  go} \colorbox{Cyan!0.000}{\strut  w} \colorbox{Cyan!0.000}{\strut /} \colorbox{Cyan!0.000}{\strut  a} \colorbox{Cyan!0.000}{\strut  design} \\
Input SAE & \num{1.884e+01} & \colorbox{Green!0.000}{\strut  something} \colorbox{Green!0.000}{\strut  simple} \colorbox{Green!0.000}{\strut :} \colorbox{Green!0.000}{\strut  a} \colorbox{Green!0.000}{\strut  pillow} \colorbox{Green!0.000}{\strut .} \colorbox{Green!0.000}{\strut  I} \colorbox{Green!0.000}{\strut  also} \colorbox{Green!0.000}{\strut  decided} \colorbox{Green!0.000}{\strut  to} \colorbox{Green!89.544}{\strut  go} \colorbox{Green!0.000}{\strut  w} \colorbox{Green!0.000}{\strut /} \colorbox{Green!0.000}{\strut  a} \colorbox{Green!0.000}{\strut  design} \\
Output SAE & \num{5.563e+00} & \colorbox{Magenta!0.000}{\strut  something} \colorbox{Magenta!0.000}{\strut  simple} \colorbox{Magenta!0.000}{\strut :} \colorbox{Magenta!0.000}{\strut  a} \colorbox{Magenta!0.000}{\strut  pillow} \colorbox{Magenta!0.000}{\strut .} \colorbox{Magenta!0.000}{\strut  I} \colorbox{Magenta!0.000}{\strut  also} \colorbox{Magenta!0.000}{\strut  decided} \colorbox{Magenta!0.000}{\strut  to} \colorbox{Magenta!97.135}{\strut  go} \colorbox{Magenta!0.000}{\strut  w} \colorbox{Magenta!0.000}{\strut /} \colorbox{Magenta!0.000}{\strut  a} \colorbox{Magenta!0.000}{\strut  design} \\
\bottomrule
\end{longtable}
\caption{feature pairs/Layer15-65536-J1-LR5.0e-04-k32-T3.0e+08 abs mean/examples-65013-v-52843 stas c4-en-10k,train,batch size=32,ctx len=16.csv}
\end{table} % go
% \begin{table}
\centering
\begin{longtable}{lrl}
\toprule
Category & Max. abs. value & Example tokens \\
\midrule
Jacobian & \num{3.572e-01} & \colorbox{Cyan!0.000}{\strut  Court} \colorbox{Cyan!0.000}{\strut  found} \colorbox{Cyan!0.000}{\strut  in} \colorbox{Cyan!0.000}{\strut  favour} \colorbox{Cyan!0.000}{\strut  of} \colorbox{Cyan!0.000}{\strut  Des} \colorbox{Cyan!0.000}{\strut ane} \colorbox{Cyan!0.000}{\strut ,} \colorbox{Cyan!0.000}{\strut  finding} \colorbox{Cyan!0.000}{\strut  that} \colorbox{Cyan!0.000}{\strut  the} \colorbox{Cyan!0.000}{\strut  proposed} \colorbox{Cyan!0.000}{\strut  acquisition} \colorbox{Cyan!0.000}{\strut  notice} \colorbox{Cyan!100.000}{\strut  did} \\
Input SAE & \num{1.618e+01} & \colorbox{Green!0.000}{\strut  Court} \colorbox{Green!0.000}{\strut  found} \colorbox{Green!0.000}{\strut  in} \colorbox{Green!0.000}{\strut  favour} \colorbox{Green!0.000}{\strut  of} \colorbox{Green!0.000}{\strut  Des} \colorbox{Green!0.000}{\strut ane} \colorbox{Green!0.000}{\strut ,} \colorbox{Green!0.000}{\strut  finding} \colorbox{Green!0.000}{\strut  that} \colorbox{Green!0.000}{\strut  the} \colorbox{Green!0.000}{\strut  proposed} \colorbox{Green!0.000}{\strut  acquisition} \colorbox{Green!0.000}{\strut  notice} \colorbox{Green!77.091}{\strut  did} \\
Output SAE & \num{5.688e+00} & \colorbox{Magenta!0.000}{\strut  Court} \colorbox{Magenta!0.000}{\strut  found} \colorbox{Magenta!0.000}{\strut  in} \colorbox{Magenta!0.000}{\strut  favour} \colorbox{Magenta!0.000}{\strut  of} \colorbox{Magenta!0.000}{\strut  Des} \colorbox{Magenta!0.000}{\strut ane} \colorbox{Magenta!0.000}{\strut ,} \colorbox{Magenta!0.000}{\strut  finding} \colorbox{Magenta!0.000}{\strut  that} \colorbox{Magenta!0.000}{\strut  the} \colorbox{Magenta!0.000}{\strut  proposed} \colorbox{Magenta!0.000}{\strut  acquisition} \colorbox{Magenta!0.000}{\strut  notice} \colorbox{Magenta!84.652}{\strut  did} \\
\midrule
Jacobian & \num{3.566e-01} & \colorbox{Cyan!0.000}{\strut  It} \colorbox{Cyan!0.000}{\strut  is} \colorbox{Cyan!0.000}{\strut  true} \colorbox{Cyan!0.000}{\strut  that} \colorbox{Cyan!0.000}{\strut  the} \colorbox{Cyan!0.000}{\strut  U} \colorbox{Cyan!0.000}{\strut AH} \colorbox{Cyan!0.000}{\strut  dataset} \colorbox{Cyan!99.833}{\strut  does} \colorbox{Cyan!45.704}{\strut ,} \colorbox{Cyan!0.000}{\strut  indeed} \colorbox{Cyan!66.636}{\strut ,} \colorbox{Cyan!0.000}{\strut  show} \colorbox{Cyan!0.000}{\strut  less} \colorbox{Cyan!0.000}{\strut  warming} \\
Input SAE & \num{1.647e+01} & \colorbox{Green!0.000}{\strut  It} \colorbox{Green!0.000}{\strut  is} \colorbox{Green!0.000}{\strut  true} \colorbox{Green!0.000}{\strut  that} \colorbox{Green!0.000}{\strut  the} \colorbox{Green!0.000}{\strut  U} \colorbox{Green!0.000}{\strut AH} \colorbox{Green!0.000}{\strut  dataset} \colorbox{Green!78.501}{\strut  does} \colorbox{Green!9.885}{\strut ,} \colorbox{Green!0.000}{\strut  indeed} \colorbox{Green!13.936}{\strut ,} \colorbox{Green!0.000}{\strut  show} \colorbox{Green!0.000}{\strut  less} \colorbox{Green!0.000}{\strut  warming} \\
Output SAE & \num{5.590e+00} & \colorbox{Magenta!0.000}{\strut  It} \colorbox{Magenta!0.000}{\strut  is} \colorbox{Magenta!0.000}{\strut  true} \colorbox{Magenta!0.000}{\strut  that} \colorbox{Magenta!0.000}{\strut  the} \colorbox{Magenta!0.000}{\strut  U} \colorbox{Magenta!0.000}{\strut AH} \colorbox{Magenta!0.000}{\strut  dataset} \colorbox{Magenta!83.194}{\strut  does} \colorbox{Magenta!13.596}{\strut ,} \colorbox{Magenta!0.000}{\strut  indeed} \colorbox{Magenta!16.448}{\strut ,} \colorbox{Magenta!0.000}{\strut  show} \colorbox{Magenta!0.000}{\strut  less} \colorbox{Magenta!0.000}{\strut  warming} \\
\midrule
Jacobian & \num{3.562e-01} & \colorbox{Cyan!0.000}{\strut  criminal} \colorbox{Cyan!0.000}{\strut  complaint} \colorbox{Cyan!0.000}{\strut .} \colorbox{Cyan!0.000}{\strut  Article} \colorbox{Cyan!0.000}{\strut  517} \colorbox{Cyan!0.000}{\strut ter} \colorbox{Cyan!0.000}{\strut  of} \colorbox{Cyan!0.000}{\strut  the} \colorbox{Cyan!0.000}{\strut  Criminal} \colorbox{Cyan!0.000}{\strut  Code} \colorbox{Cyan!0.000}{\strut  applies} \colorbox{Cyan!0.000}{\strut  where} \colorbox{Cyan!0.000}{\strut  the} \colorbox{Cyan!0.000}{\strut  infringement} \colorbox{Cyan!99.715}{\strut  does} \\
Input SAE & \num{1.616e+01} & \colorbox{Green!0.000}{\strut  criminal} \colorbox{Green!0.000}{\strut  complaint} \colorbox{Green!0.000}{\strut .} \colorbox{Green!0.000}{\strut  Article} \colorbox{Green!0.000}{\strut  517} \colorbox{Green!0.000}{\strut ter} \colorbox{Green!0.000}{\strut  of} \colorbox{Green!0.000}{\strut  the} \colorbox{Green!0.000}{\strut  Criminal} \colorbox{Green!0.000}{\strut  Code} \colorbox{Green!0.000}{\strut  applies} \colorbox{Green!0.000}{\strut  where} \colorbox{Green!0.000}{\strut  the} \colorbox{Green!0.000}{\strut  infringement} \colorbox{Green!77.028}{\strut  does} \\
Output SAE & \num{5.064e+00} & \colorbox{Magenta!0.000}{\strut  criminal} \colorbox{Magenta!0.000}{\strut  complaint} \colorbox{Magenta!0.000}{\strut .} \colorbox{Magenta!0.000}{\strut  Article} \colorbox{Magenta!0.000}{\strut  517} \colorbox{Magenta!0.000}{\strut ter} \colorbox{Magenta!0.000}{\strut  of} \colorbox{Magenta!0.000}{\strut  the} \colorbox{Magenta!0.000}{\strut  Criminal} \colorbox{Magenta!0.000}{\strut  Code} \colorbox{Magenta!0.000}{\strut  applies} \colorbox{Magenta!0.000}{\strut  where} \colorbox{Magenta!0.000}{\strut  the} \colorbox{Magenta!0.000}{\strut  infringement} \colorbox{Magenta!75.354}{\strut  does} \\
\midrule
Jacobian & \num{3.561e-01} & \colorbox{Cyan!0.000}{\strut learning} \colorbox{Cyan!0.000}{\strut \textquotesingle{}.} \colorbox{Cyan!0.000}{\strut  This} \colorbox{Cyan!0.000}{\strut  means} \colorbox{Cyan!0.000}{\strut  that} \colorbox{Cyan!0.000}{\strut  it} \colorbox{Cyan!99.691}{\strut  does} \colorbox{Cyan!50.433}{\strut  all} \colorbox{Cyan!0.000}{\strut  the} \colorbox{Cyan!0.000}{\strut  processing} \colorbox{Cyan!0.000}{\strut  while} \colorbox{Cyan!0.000}{\strut  no} \colorbox{Cyan!0.000}{\strut  m} \colorbox{Cyan!0.000}{\strut ails} \colorbox{Cyan!0.000}{\strut  are} \\
Input SAE & \num{1.758e+01} & \colorbox{Green!0.000}{\strut learning} \colorbox{Green!0.000}{\strut \textquotesingle{}.} \colorbox{Green!0.000}{\strut  This} \colorbox{Green!0.000}{\strut  means} \colorbox{Green!0.000}{\strut  that} \colorbox{Green!0.000}{\strut  it} \colorbox{Green!83.792}{\strut  does} \colorbox{Green!8.384}{\strut  all} \colorbox{Green!0.000}{\strut  the} \colorbox{Green!0.000}{\strut  processing} \colorbox{Green!0.000}{\strut  while} \colorbox{Green!0.000}{\strut  no} \colorbox{Green!0.000}{\strut  m} \colorbox{Green!0.000}{\strut ails} \colorbox{Green!0.000}{\strut  are} \\
Output SAE & \num{5.793e+00} & \colorbox{Magenta!0.000}{\strut learning} \colorbox{Magenta!0.000}{\strut \textquotesingle{}.} \colorbox{Magenta!0.000}{\strut  This} \colorbox{Magenta!0.000}{\strut  means} \colorbox{Magenta!0.000}{\strut  that} \colorbox{Magenta!0.000}{\strut  it} \colorbox{Magenta!86.213}{\strut  does} \colorbox{Magenta!17.062}{\strut  all} \colorbox{Magenta!9.444}{\strut  the} \colorbox{Magenta!0.000}{\strut  processing} \colorbox{Magenta!0.000}{\strut  while} \colorbox{Magenta!0.000}{\strut  no} \colorbox{Magenta!0.000}{\strut  m} \colorbox{Magenta!0.000}{\strut ails} \colorbox{Magenta!0.000}{\strut  are} \\
\midrule
Jacobian & \num{3.557e-01} & \colorbox{Cyan!0.000}{\strut  their} \colorbox{Cyan!0.000}{\strut  claims} \colorbox{Cyan!0.000}{\strut .} \colorbox{Cyan!0.000}{\strut  However} \colorbox{Cyan!0.000}{\strut ,} \colorbox{Cyan!0.000}{\strut  it} \colorbox{Cyan!0.000}{\strut  should} \colorbox{Cyan!0.000}{\strut  be} \colorbox{Cyan!0.000}{\strut  noted} \colorbox{Cyan!0.000}{\strut  that} \colorbox{Cyan!0.000}{\strut  the} \colorbox{Cyan!0.000}{\strut  Code} \colorbox{Cyan!99.570}{\strut  does} \colorbox{Cyan!0.000}{\strut  not} \colorbox{Cyan!0.000}{\strut  recognize} \\
Input SAE & \num{1.618e+01} & \colorbox{Green!0.000}{\strut  their} \colorbox{Green!0.000}{\strut  claims} \colorbox{Green!0.000}{\strut .} \colorbox{Green!0.000}{\strut  However} \colorbox{Green!0.000}{\strut ,} \colorbox{Green!0.000}{\strut  it} \colorbox{Green!0.000}{\strut  should} \colorbox{Green!0.000}{\strut  be} \colorbox{Green!0.000}{\strut  noted} \colorbox{Green!0.000}{\strut  that} \colorbox{Green!0.000}{\strut  the} \colorbox{Green!0.000}{\strut  Code} \colorbox{Green!77.099}{\strut  does} \colorbox{Green!0.000}{\strut  not} \colorbox{Green!0.000}{\strut  recognize} \\
Output SAE & \num{5.461e+00} & \colorbox{Magenta!0.000}{\strut  their} \colorbox{Magenta!0.000}{\strut  claims} \colorbox{Magenta!0.000}{\strut .} \colorbox{Magenta!0.000}{\strut  However} \colorbox{Magenta!0.000}{\strut ,} \colorbox{Magenta!0.000}{\strut  it} \colorbox{Magenta!0.000}{\strut  should} \colorbox{Magenta!0.000}{\strut  be} \colorbox{Magenta!0.000}{\strut  noted} \colorbox{Magenta!0.000}{\strut  that} \colorbox{Magenta!0.000}{\strut  the} \colorbox{Magenta!0.000}{\strut  Code} \colorbox{Magenta!81.273}{\strut  does} \colorbox{Magenta!0.000}{\strut  not} \colorbox{Magenta!0.000}{\strut  recognize} \\
\midrule
Jacobian & \num{3.538e-01} & \colorbox{Cyan!0.000}{\strut  of} \colorbox{Cyan!0.000}{\strut  2016} \colorbox{Cyan!0.000}{\strut  (} \colorbox{Cyan!0.000}{\strut 88} \colorbox{Cyan!0.000}{\strut  percent} \colorbox{Cyan!0.000}{\strut )} \colorbox{Cyan!0.000}{\strut  were} \colorbox{Cyan!0.000}{\strut  opportun} \colorbox{Cyan!0.000}{\strut istic} \colorbox{Cyan!0.000}{\strut  attacks} \colorbox{Cyan!0.000}{\strut  that} \colorbox{Cyan!99.050}{\strut  did} \colorbox{Cyan!0.000}{\strut  not} \colorbox{Cyan!0.000}{\strut  target} \colorbox{Cyan!0.000}{\strut  a} \\
Input SAE & \num{1.916e+01} & \colorbox{Green!0.000}{\strut  of} \colorbox{Green!0.000}{\strut  2016} \colorbox{Green!0.000}{\strut  (} \colorbox{Green!0.000}{\strut 88} \colorbox{Green!0.000}{\strut  percent} \colorbox{Green!0.000}{\strut )} \colorbox{Green!0.000}{\strut  were} \colorbox{Green!0.000}{\strut  opportun} \colorbox{Green!0.000}{\strut istic} \colorbox{Green!0.000}{\strut  attacks} \colorbox{Green!0.000}{\strut  that} \colorbox{Green!91.294}{\strut  did} \colorbox{Green!0.000}{\strut  not} \colorbox{Green!0.000}{\strut  target} \colorbox{Green!0.000}{\strut  a} \\
Output SAE & \num{6.467e+00} & \colorbox{Magenta!0.000}{\strut  of} \colorbox{Magenta!0.000}{\strut  2016} \colorbox{Magenta!0.000}{\strut  (} \colorbox{Magenta!0.000}{\strut 88} \colorbox{Magenta!0.000}{\strut  percent} \colorbox{Magenta!0.000}{\strut )} \colorbox{Magenta!0.000}{\strut  were} \colorbox{Magenta!0.000}{\strut  opportun} \colorbox{Magenta!0.000}{\strut istic} \colorbox{Magenta!0.000}{\strut  attacks} \colorbox{Magenta!0.000}{\strut  that} \colorbox{Magenta!96.242}{\strut  did} \colorbox{Magenta!0.000}{\strut  not} \colorbox{Magenta!0.000}{\strut  target} \colorbox{Magenta!0.000}{\strut  a} \\
\midrule
Jacobian & \num{3.533e-01} & \colorbox{Cyan!0.000}{\strut  a} \colorbox{Cyan!0.000}{\strut  court} \colorbox{Cyan!0.000}{\strut  order} \colorbox{Cyan!0.000}{\strut .} \colorbox{Cyan!0.000}{\strut  Pro} \colorbox{Cyan!0.000}{\strut pos} \colorbox{Cyan!0.000}{\strut als} \colorbox{Cyan!0.000}{\strut  relating} \colorbox{Cyan!0.000}{\strut  to} \colorbox{Cyan!0.000}{\strut  children} \colorbox{Cyan!0.000}{\strut  often} \colorbox{Cyan!98.889}{\strut  do} \colorbox{Cyan!0.000}{\strut  not} \colorbox{Cyan!0.000}{\strut  need} \colorbox{Cyan!0.000}{\strut  to} \\
Input SAE & \num{1.847e+01} & \colorbox{Green!0.000}{\strut  a} \colorbox{Green!0.000}{\strut  court} \colorbox{Green!0.000}{\strut  order} \colorbox{Green!0.000}{\strut .} \colorbox{Green!0.000}{\strut  Pro} \colorbox{Green!0.000}{\strut pos} \colorbox{Green!0.000}{\strut als} \colorbox{Green!0.000}{\strut  relating} \colorbox{Green!0.000}{\strut  to} \colorbox{Green!0.000}{\strut  children} \colorbox{Green!0.000}{\strut  often} \colorbox{Green!88.026}{\strut  do} \colorbox{Green!0.000}{\strut  not} \colorbox{Green!0.000}{\strut  need} \colorbox{Green!0.000}{\strut  to} \\
Output SAE & \num{6.264e+00} & \colorbox{Magenta!0.000}{\strut  a} \colorbox{Magenta!0.000}{\strut  court} \colorbox{Magenta!0.000}{\strut  order} \colorbox{Magenta!0.000}{\strut .} \colorbox{Magenta!0.000}{\strut  Pro} \colorbox{Magenta!0.000}{\strut pos} \colorbox{Magenta!0.000}{\strut als} \colorbox{Magenta!0.000}{\strut  relating} \colorbox{Magenta!0.000}{\strut  to} \colorbox{Magenta!0.000}{\strut  children} \colorbox{Magenta!0.000}{\strut  often} \colorbox{Magenta!93.222}{\strut  do} \colorbox{Magenta!0.000}{\strut  not} \colorbox{Magenta!0.000}{\strut  need} \colorbox{Magenta!0.000}{\strut  to} \\
\midrule
Jacobian & \num{3.527e-01} & \colorbox{Cyan!0.000}{\strut ).} \colorbox{Cyan!0.000}{\strut It} \colorbox{Cyan!0.000}{\strut  is} \colorbox{Cyan!0.000}{\strut  important} \colorbox{Cyan!0.000}{\strut  to} \colorbox{Cyan!0.000}{\strut  note} \colorbox{Cyan!0.000}{\strut  that} \colorbox{Cyan!0.000}{\strut  this} \colorbox{Cyan!0.000}{\strut  issue} \colorbox{Cyan!98.745}{\strut  does} \colorbox{Cyan!0.000}{\strut  not} \colorbox{Cyan!0.000}{\strut  arise} \colorbox{Cyan!0.000}{\strut  under} \colorbox{Cyan!0.000}{\strut  the} \\
Input SAE & \num{1.710e+01} & \colorbox{Green!0.000}{\strut ).} \colorbox{Green!0.000}{\strut It} \colorbox{Green!0.000}{\strut  is} \colorbox{Green!0.000}{\strut  important} \colorbox{Green!0.000}{\strut  to} \colorbox{Green!0.000}{\strut  note} \colorbox{Green!0.000}{\strut  that} \colorbox{Green!0.000}{\strut  this} \colorbox{Green!0.000}{\strut  issue} \colorbox{Green!81.491}{\strut  does} \colorbox{Green!0.000}{\strut  not} \colorbox{Green!0.000}{\strut  arise} \colorbox{Green!0.000}{\strut  under} \colorbox{Green!0.000}{\strut  the} \\
Output SAE & \num{5.704e+00} & \colorbox{Magenta!0.000}{\strut ).} \colorbox{Magenta!0.000}{\strut It} \colorbox{Magenta!0.000}{\strut  is} \colorbox{Magenta!0.000}{\strut  important} \colorbox{Magenta!0.000}{\strut  to} \colorbox{Magenta!0.000}{\strut  note} \colorbox{Magenta!0.000}{\strut  that} \colorbox{Magenta!0.000}{\strut  this} \colorbox{Magenta!0.000}{\strut  issue} \colorbox{Magenta!84.883}{\strut  does} \colorbox{Magenta!0.000}{\strut  not} \colorbox{Magenta!0.000}{\strut  arise} \colorbox{Magenta!0.000}{\strut  under} \colorbox{Magenta!0.000}{\strut  the} \\
\midrule
Jacobian & \num{3.523e-01} & \colorbox{Cyan!0.000}{\strut  for} \colorbox{Cyan!0.000}{\strut  the} \colorbox{Cyan!0.000}{\strut  same} \colorbox{Cyan!0.000}{\strut .} \colorbox{Cyan!0.000}{\strut  Ad} \colorbox{Cyan!0.000}{\strut mitted} \colorbox{Cyan!0.000}{\strut ly} \colorbox{Cyan!0.000}{\strut ,} \colorbox{Cyan!0.000}{\strut  section} \colorbox{Cyan!0.000}{\strut  39} \colorbox{Cyan!0.000}{\strut  of} \colorbox{Cyan!0.000}{\strut  the} \colorbox{Cyan!0.000}{\strut  Act} \colorbox{Cyan!98.630}{\strut  does} \colorbox{Cyan!0.000}{\strut  provide} \\
Input SAE & \num{1.535e+01} & \colorbox{Green!0.000}{\strut  for} \colorbox{Green!0.000}{\strut  the} \colorbox{Green!0.000}{\strut  same} \colorbox{Green!0.000}{\strut .} \colorbox{Green!0.000}{\strut  Ad} \colorbox{Green!0.000}{\strut mitted} \colorbox{Green!0.000}{\strut ly} \colorbox{Green!0.000}{\strut ,} \colorbox{Green!0.000}{\strut  section} \colorbox{Green!0.000}{\strut  39} \colorbox{Green!0.000}{\strut  of} \colorbox{Green!0.000}{\strut  the} \colorbox{Green!0.000}{\strut  Act} \colorbox{Green!73.158}{\strut  does} \colorbox{Green!0.000}{\strut  provide} \\
Output SAE & \num{5.445e+00} & \colorbox{Magenta!0.000}{\strut  for} \colorbox{Magenta!0.000}{\strut  the} \colorbox{Magenta!0.000}{\strut  same} \colorbox{Magenta!0.000}{\strut .} \colorbox{Magenta!0.000}{\strut  Ad} \colorbox{Magenta!0.000}{\strut mitted} \colorbox{Magenta!0.000}{\strut ly} \colorbox{Magenta!0.000}{\strut ,} \colorbox{Magenta!0.000}{\strut  section} \colorbox{Magenta!0.000}{\strut  39} \colorbox{Magenta!0.000}{\strut  of} \colorbox{Magenta!0.000}{\strut  the} \colorbox{Magenta!0.000}{\strut  Act} \colorbox{Magenta!81.025}{\strut  does} \colorbox{Magenta!11.227}{\strut  provide} \\
\midrule
Jacobian & \num{3.523e-01} & \colorbox{Cyan!0.000}{\strut .} \colorbox{Cyan!0.000}{\strut The} \colorbox{Cyan!0.000}{\strut  court} \colorbox{Cyan!0.000}{\strut  also} \colorbox{Cyan!0.000}{\strut  ruled} \colorbox{Cyan!0.000}{\strut  that} \colorbox{Cyan!0.000}{\strut  the} \colorbox{Cyan!0.000}{\strut  ordinance} \colorbox{Cyan!98.628}{\strut  does} \colorbox{Cyan!0.000}{\strut  not} \colorbox{Cyan!0.000}{\strut  violate} \colorbox{Cyan!0.000}{\strut  the} \colorbox{Cyan!0.000}{\strut  right} \colorbox{Cyan!0.000}{\strut  to} \\
Input SAE & \num{1.665e+01} & \colorbox{Green!0.000}{\strut .} \colorbox{Green!0.000}{\strut The} \colorbox{Green!0.000}{\strut  court} \colorbox{Green!0.000}{\strut  also} \colorbox{Green!0.000}{\strut  ruled} \colorbox{Green!0.000}{\strut  that} \colorbox{Green!0.000}{\strut  the} \colorbox{Green!0.000}{\strut  ordinance} \colorbox{Green!79.358}{\strut  does} \colorbox{Green!0.000}{\strut  not} \colorbox{Green!0.000}{\strut  violate} \colorbox{Green!0.000}{\strut  the} \colorbox{Green!0.000}{\strut  right} \colorbox{Green!0.000}{\strut  to} \\
Output SAE & \num{5.571e+00} & \colorbox{Magenta!0.000}{\strut .} \colorbox{Magenta!0.000}{\strut The} \colorbox{Magenta!0.000}{\strut  court} \colorbox{Magenta!0.000}{\strut  also} \colorbox{Magenta!0.000}{\strut  ruled} \colorbox{Magenta!0.000}{\strut  that} \colorbox{Magenta!0.000}{\strut  the} \colorbox{Magenta!0.000}{\strut  ordinance} \colorbox{Magenta!82.908}{\strut  does} \colorbox{Magenta!0.000}{\strut  not} \colorbox{Magenta!0.000}{\strut  violate} \colorbox{Magenta!0.000}{\strut  the} \colorbox{Magenta!0.000}{\strut  right} \colorbox{Magenta!0.000}{\strut  to} \\
\midrule
Jacobian & \num{3.520e-01} & \colorbox{Cyan!0.000}{\strut .} \colorbox{Cyan!0.000}{\strut  However} \colorbox{Cyan!0.000}{\strut ,} \colorbox{Cyan!0.000}{\strut  that} \colorbox{Cyan!0.000}{\strut  standard} \colorbox{Cyan!0.000}{\strut  explicitly} \colorbox{Cyan!0.000}{\strut  states} \colorbox{Cyan!0.000}{\strut  that} \colorbox{Cyan!0.000}{\strut  it} \colorbox{Cyan!98.524}{\strut  does} \colorbox{Cyan!0.000}{\strut  not} \colorbox{Cyan!0.000}{\strut  apply} \colorbox{Cyan!0.000}{\strut  to} \colorbox{Cyan!0.000}{\strut  general} \colorbox{Cyan!0.000}{\strut  purpose} \\
Input SAE & \num{1.766e+01} & \colorbox{Green!0.000}{\strut .} \colorbox{Green!0.000}{\strut  However} \colorbox{Green!0.000}{\strut ,} \colorbox{Green!0.000}{\strut  that} \colorbox{Green!0.000}{\strut  standard} \colorbox{Green!0.000}{\strut  explicitly} \colorbox{Green!0.000}{\strut  states} \colorbox{Green!0.000}{\strut  that} \colorbox{Green!0.000}{\strut  it} \colorbox{Green!84.138}{\strut  does} \colorbox{Green!0.000}{\strut  not} \colorbox{Green!0.000}{\strut  apply} \colorbox{Green!0.000}{\strut  to} \colorbox{Green!0.000}{\strut  general} \colorbox{Green!0.000}{\strut  purpose} \\
Output SAE & \num{5.921e+00} & \colorbox{Magenta!0.000}{\strut .} \colorbox{Magenta!0.000}{\strut  However} \colorbox{Magenta!0.000}{\strut ,} \colorbox{Magenta!0.000}{\strut  that} \colorbox{Magenta!0.000}{\strut  standard} \colorbox{Magenta!0.000}{\strut  explicitly} \colorbox{Magenta!0.000}{\strut  states} \colorbox{Magenta!0.000}{\strut  that} \colorbox{Magenta!0.000}{\strut  it} \colorbox{Magenta!88.117}{\strut  does} \colorbox{Magenta!8.619}{\strut  not} \colorbox{Magenta!0.000}{\strut  apply} \colorbox{Magenta!0.000}{\strut  to} \colorbox{Magenta!0.000}{\strut  general} \colorbox{Magenta!0.000}{\strut  purpose} \\
\midrule
Jacobian & \num{3.508e-01} & \colorbox{Cyan!0.000}{\strut  looks} \colorbox{Cyan!0.000}{\strut  as} \colorbox{Cyan!0.000}{\strut  though} \colorbox{Cyan!0.000}{\strut  the} \colorbox{Cyan!0.000}{\strut  AT} \colorbox{Cyan!0.000}{\strut O} \colorbox{Cyan!0.000}{\strut  is} \colorbox{Cyan!0.000}{\strut  correctly} \colorbox{Cyan!0.000}{\strut  identifying} \colorbox{Cyan!0.000}{\strut  that} \colorbox{Cyan!0.000}{\strut  the} \colorbox{Cyan!0.000}{\strut  law} \colorbox{Cyan!98.212}{\strut  does} \colorbox{Cyan!0.000}{\strut  not} \colorbox{Cyan!0.000}{\strut  necessarily} \\
Input SAE & \num{1.732e+01} & \colorbox{Green!0.000}{\strut  looks} \colorbox{Green!0.000}{\strut  as} \colorbox{Green!0.000}{\strut  though} \colorbox{Green!0.000}{\strut  the} \colorbox{Green!0.000}{\strut  AT} \colorbox{Green!0.000}{\strut O} \colorbox{Green!0.000}{\strut  is} \colorbox{Green!0.000}{\strut  correctly} \colorbox{Green!0.000}{\strut  identifying} \colorbox{Green!0.000}{\strut  that} \colorbox{Green!0.000}{\strut  the} \colorbox{Green!0.000}{\strut  law} \colorbox{Green!82.545}{\strut  does} \colorbox{Green!0.000}{\strut  not} \colorbox{Green!0.000}{\strut  necessarily} \\
Output SAE & \num{5.833e+00} & \colorbox{Magenta!0.000}{\strut  looks} \colorbox{Magenta!0.000}{\strut  as} \colorbox{Magenta!0.000}{\strut  though} \colorbox{Magenta!0.000}{\strut  the} \colorbox{Magenta!0.000}{\strut  AT} \colorbox{Magenta!0.000}{\strut O} \colorbox{Magenta!0.000}{\strut  is} \colorbox{Magenta!0.000}{\strut  correctly} \colorbox{Magenta!0.000}{\strut  identifying} \colorbox{Magenta!0.000}{\strut  that} \colorbox{Magenta!0.000}{\strut  the} \colorbox{Magenta!0.000}{\strut  law} \colorbox{Magenta!86.801}{\strut  does} \colorbox{Magenta!0.000}{\strut  not} \colorbox{Magenta!0.000}{\strut  necessarily} \\
\bottomrule
\end{longtable}
\caption{feature pairs/Layer15-65536-J1-LR5.0e-04-k32-T3.0e+08 abs mean/examples-21465-v-60542 stas c4-en-10k,train,batch size=32,ctx len=16.csv}
\end{table} % did, does, do
% \begin{table}
\centering
\begin{longtable}{lrl}
\toprule
Category & Max. abs. value & Example tokens \\
\midrule
Jacobian & \num{2.969e-01} & \colorbox{Cyan!0.000}{\strut  textual} \colorbox{Cyan!0.000}{\strut  part} \colorbox{Cyan!0.000}{\strut  contains} \colorbox{Cyan!0.000}{\strut  more} \colorbox{Cyan!0.000}{\strut  information} \colorbox{Cyan!0.000}{\strut  than} \colorbox{Cyan!0.000}{\strut  numerical} \colorbox{Cyan!0.000}{\strut  part} \colorbox{Cyan!0.000}{\strut  in} \colorbox{Cyan!0.000}{\strut  an} \colorbox{Cyan!0.000}{\strut  annual} \colorbox{Cyan!0.000}{\strut  report} \colorbox{Cyan!0.000}{\strut  (} \colorbox{Cyan!0.000}{\strut Chen} \colorbox{Cyan!100.000}{\strut  et} \\
Input SAE & \num{2.034e+01} & \colorbox{Green!0.000}{\strut  textual} \colorbox{Green!0.000}{\strut  part} \colorbox{Green!0.000}{\strut  contains} \colorbox{Green!0.000}{\strut  more} \colorbox{Green!0.000}{\strut  information} \colorbox{Green!0.000}{\strut  than} \colorbox{Green!0.000}{\strut  numerical} \colorbox{Green!0.000}{\strut  part} \colorbox{Green!0.000}{\strut  in} \colorbox{Green!0.000}{\strut  an} \colorbox{Green!0.000}{\strut  annual} \colorbox{Green!0.000}{\strut  report} \colorbox{Green!0.000}{\strut  (} \colorbox{Green!0.000}{\strut Chen} \colorbox{Green!89.943}{\strut  et} \\
Output SAE & \num{6.570e+00} & \colorbox{Magenta!0.000}{\strut  textual} \colorbox{Magenta!0.000}{\strut  part} \colorbox{Magenta!0.000}{\strut  contains} \colorbox{Magenta!0.000}{\strut  more} \colorbox{Magenta!0.000}{\strut  information} \colorbox{Magenta!0.000}{\strut  than} \colorbox{Magenta!0.000}{\strut  numerical} \colorbox{Magenta!0.000}{\strut  part} \colorbox{Magenta!0.000}{\strut  in} \colorbox{Magenta!0.000}{\strut  an} \colorbox{Magenta!0.000}{\strut  annual} \colorbox{Magenta!0.000}{\strut  report} \colorbox{Magenta!0.000}{\strut  (} \colorbox{Magenta!0.000}{\strut Chen} \colorbox{Magenta!98.993}{\strut  et} \\
\midrule
Jacobian & \num{2.969e-01} & \colorbox{Cyan!0.000}{\strut  passwords} \colorbox{Cyan!0.000}{\strut  have} \colorbox{Cyan!0.000}{\strut  found} \colorbox{Cyan!0.000}{\strut  that} \colorbox{Cyan!0.000}{\strut  users} \colorbox{Cyan!0.000}{\strut  select} \colorbox{Cyan!0.000}{\strut  these} \colorbox{Cyan!0.000}{\strut  passwords} \colorbox{Cyan!0.000}{\strut  according} \colorbox{Cyan!0.000}{\strut  to} \colorbox{Cyan!0.000}{\strut  predictable} \colorbox{Cyan!0.000}{\strut  patterns} \colorbox{Cyan!0.000}{\strut .} \colorbox{Cyan!0.000}{\strut  Davis} \colorbox{Cyan!99.995}{\strut  et} \\
Input SAE & \num{1.990e+01} & \colorbox{Green!0.000}{\strut  passwords} \colorbox{Green!0.000}{\strut  have} \colorbox{Green!0.000}{\strut  found} \colorbox{Green!0.000}{\strut  that} \colorbox{Green!0.000}{\strut  users} \colorbox{Green!0.000}{\strut  select} \colorbox{Green!0.000}{\strut  these} \colorbox{Green!0.000}{\strut  passwords} \colorbox{Green!0.000}{\strut  according} \colorbox{Green!0.000}{\strut  to} \colorbox{Green!0.000}{\strut  predictable} \colorbox{Green!0.000}{\strut  patterns} \colorbox{Green!0.000}{\strut .} \colorbox{Green!0.000}{\strut  Davis} \colorbox{Green!88.002}{\strut  et} \\
Output SAE & \num{6.505e+00} & \colorbox{Magenta!0.000}{\strut  passwords} \colorbox{Magenta!0.000}{\strut  have} \colorbox{Magenta!0.000}{\strut  found} \colorbox{Magenta!0.000}{\strut  that} \colorbox{Magenta!0.000}{\strut  users} \colorbox{Magenta!0.000}{\strut  select} \colorbox{Magenta!0.000}{\strut  these} \colorbox{Magenta!0.000}{\strut  passwords} \colorbox{Magenta!0.000}{\strut  according} \colorbox{Magenta!0.000}{\strut  to} \colorbox{Magenta!0.000}{\strut  predictable} \colorbox{Magenta!0.000}{\strut  patterns} \colorbox{Magenta!0.000}{\strut .} \colorbox{Magenta!0.000}{\strut  Davis} \colorbox{Magenta!98.014}{\strut  et} \\
\midrule
Jacobian & \num{2.969e-01} & \colorbox{Cyan!0.000}{\strut  implement} \colorbox{Cyan!0.000}{\strut  through} \colorbox{Cyan!0.000}{\strut  an} \colorbox{Cyan!0.000}{\strut  automated} \colorbox{Cyan!0.000}{\strut  process} \colorbox{Cyan!0.000}{\strut  or} \colorbox{Cyan!0.000}{\strut  a} \colorbox{Cyan!0.000}{\strut  default} \colorbox{Cyan!0.000}{\strut  setting} \colorbox{Cyan!0.000}{\strut  .} \colorbox{Cyan!0.000}{\strut  Edwards} \colorbox{Cyan!99.989}{\strut  et} \colorbox{Cyan!0.000}{\strut  al} \colorbox{Cyan!0.000}{\strut .} \colorbox{Cyan!0.000}{\strut  describe} \\
Input SAE & \num{1.962e+01} & \colorbox{Green!0.000}{\strut  implement} \colorbox{Green!0.000}{\strut  through} \colorbox{Green!0.000}{\strut  an} \colorbox{Green!0.000}{\strut  automated} \colorbox{Green!0.000}{\strut  process} \colorbox{Green!0.000}{\strut  or} \colorbox{Green!0.000}{\strut  a} \colorbox{Green!0.000}{\strut  default} \colorbox{Green!0.000}{\strut  setting} \colorbox{Green!0.000}{\strut  .} \colorbox{Green!0.000}{\strut  Edwards} \colorbox{Green!86.782}{\strut  et} \colorbox{Green!0.000}{\strut  al} \colorbox{Green!0.000}{\strut .} \colorbox{Green!0.000}{\strut  describe} \\
Output SAE & \num{6.493e+00} & \colorbox{Magenta!0.000}{\strut  implement} \colorbox{Magenta!0.000}{\strut  through} \colorbox{Magenta!0.000}{\strut  an} \colorbox{Magenta!0.000}{\strut  automated} \colorbox{Magenta!0.000}{\strut  process} \colorbox{Magenta!0.000}{\strut  or} \colorbox{Magenta!0.000}{\strut  a} \colorbox{Magenta!0.000}{\strut  default} \colorbox{Magenta!0.000}{\strut  setting} \colorbox{Magenta!0.000}{\strut  .} \colorbox{Magenta!0.000}{\strut  Edwards} \colorbox{Magenta!97.831}{\strut  et} \colorbox{Magenta!10.912}{\strut  al} \colorbox{Magenta!0.000}{\strut .} \colorbox{Magenta!0.000}{\strut  describe} \\
\midrule
Jacobian & \num{2.945e-01} & \colorbox{Cyan!0.000}{\strut  SNP} \colorbox{Cyan!0.000}{\strut -} \colorbox{Cyan!0.000}{\strut P} \colorbox{Cyan!0.000}{\strut hen} \colorbox{Cyan!0.000}{\strut otype} \colorbox{Cyan!0.000}{\strut  Associ} \colorbox{Cyan!0.000}{\strut ations} \colorbox{Cyan!0.000}{\strut  dataset} \colorbox{Cyan!0.000}{\strut .} \colorbox{Cyan!0.000}{\strut He} \colorbox{Cyan!0.000}{\strut iser} \colorbox{Cyan!99.184}{\strut  et} \colorbox{Cyan!0.000}{\strut  al} \colorbox{Cyan!0.000}{\strut .,} \\
Input SAE & \num{2.056e+01} & \colorbox{Green!0.000}{\strut  SNP} \colorbox{Green!0.000}{\strut -} \colorbox{Green!0.000}{\strut P} \colorbox{Green!0.000}{\strut hen} \colorbox{Green!0.000}{\strut otype} \colorbox{Green!0.000}{\strut  Associ} \colorbox{Green!0.000}{\strut ations} \colorbox{Green!0.000}{\strut  dataset} \colorbox{Green!0.000}{\strut .} \colorbox{Green!0.000}{\strut He} \colorbox{Green!0.000}{\strut iser} \colorbox{Green!90.918}{\strut  et} \colorbox{Green!0.000}{\strut  al} \colorbox{Green!0.000}{\strut .,} \\
Output SAE & \num{6.449e+00} & \colorbox{Magenta!0.000}{\strut  SNP} \colorbox{Magenta!0.000}{\strut -} \colorbox{Magenta!0.000}{\strut P} \colorbox{Magenta!0.000}{\strut hen} \colorbox{Magenta!0.000}{\strut otype} \colorbox{Magenta!0.000}{\strut  Associ} \colorbox{Magenta!0.000}{\strut ations} \colorbox{Magenta!0.000}{\strut  dataset} \colorbox{Magenta!0.000}{\strut .} \colorbox{Magenta!0.000}{\strut He} \colorbox{Magenta!0.000}{\strut iser} \colorbox{Magenta!97.169}{\strut  et} \colorbox{Magenta!9.113}{\strut  al} \colorbox{Magenta!0.000}{\strut .,} \\
\midrule
Jacobian & \num{2.942e-01} & \colorbox{Cyan!0.000}{\strut  of} \colorbox{Cyan!0.000}{\strut  following} \colorbox{Cyan!0.000}{\strut  typical} \colorbox{Cyan!0.000}{\strut  password} \colorbox{Cyan!0.000}{\strut  security} \colorbox{Cyan!0.000}{\strut  guidance} \colorbox{Cyan!0.000}{\strut  to} \colorbox{Cyan!0.000}{\strut  create} \colorbox{Cyan!0.000}{\strut  compliant} \colorbox{Cyan!0.000}{\strut  passwords} \colorbox{Cyan!0.000}{\strut  [} \colorbox{Cyan!0.000}{\strut K} \colorbox{Cyan!0.000}{\strut uo} \colorbox{Cyan!99.082}{\strut  et} \colorbox{Cyan!0.000}{\strut  al} \\
Input SAE & \num{1.949e+01} & \colorbox{Green!0.000}{\strut  of} \colorbox{Green!0.000}{\strut  following} \colorbox{Green!0.000}{\strut  typical} \colorbox{Green!0.000}{\strut  password} \colorbox{Green!0.000}{\strut  security} \colorbox{Green!0.000}{\strut  guidance} \colorbox{Green!0.000}{\strut  to} \colorbox{Green!0.000}{\strut  create} \colorbox{Green!0.000}{\strut  compliant} \colorbox{Green!0.000}{\strut  passwords} \colorbox{Green!0.000}{\strut  [} \colorbox{Green!0.000}{\strut K} \colorbox{Green!0.000}{\strut uo} \colorbox{Green!86.212}{\strut  et} \colorbox{Green!0.000}{\strut  al} \\
Output SAE & \num{6.479e+00} & \colorbox{Magenta!0.000}{\strut  of} \colorbox{Magenta!0.000}{\strut  following} \colorbox{Magenta!0.000}{\strut  typical} \colorbox{Magenta!0.000}{\strut  password} \colorbox{Magenta!0.000}{\strut  security} \colorbox{Magenta!0.000}{\strut  guidance} \colorbox{Magenta!0.000}{\strut  to} \colorbox{Magenta!0.000}{\strut  create} \colorbox{Magenta!0.000}{\strut  compliant} \colorbox{Magenta!0.000}{\strut  passwords} \colorbox{Magenta!0.000}{\strut  [} \colorbox{Magenta!0.000}{\strut K} \colorbox{Magenta!0.000}{\strut uo} \colorbox{Magenta!97.625}{\strut  et} \colorbox{Magenta!9.562}{\strut  al} \\
\midrule
Jacobian & \num{2.941e-01} & \colorbox{Cyan!0.000}{\strut  trends} \colorbox{Cyan!0.000}{\strut  in} \colorbox{Cyan!0.000}{\strut  winter} \colorbox{Cyan!0.000}{\strut  and} \colorbox{Cyan!0.000}{\strut  negative} \colorbox{Cyan!0.000}{\strut  trends} \colorbox{Cyan!0.000}{\strut  in} \colorbox{Cyan!0.000}{\strut  summer} \colorbox{Cyan!0.000}{\strut  precipitation} \colorbox{Cyan!0.000}{\strut .} \colorbox{Cyan!0.000}{\strut  No} \colorbox{Cyan!0.000}{\strut one} \colorbox{Cyan!99.061}{\strut  et} \colorbox{Cyan!0.000}{\strut  al} \colorbox{Cyan!0.000}{\strut .} \\
Input SAE & \num{1.943e+01} & \colorbox{Green!0.000}{\strut  trends} \colorbox{Green!0.000}{\strut  in} \colorbox{Green!0.000}{\strut  winter} \colorbox{Green!0.000}{\strut  and} \colorbox{Green!0.000}{\strut  negative} \colorbox{Green!0.000}{\strut  trends} \colorbox{Green!0.000}{\strut  in} \colorbox{Green!0.000}{\strut  summer} \colorbox{Green!0.000}{\strut  precipitation} \colorbox{Green!0.000}{\strut .} \colorbox{Green!0.000}{\strut  No} \colorbox{Green!0.000}{\strut one} \colorbox{Green!85.920}{\strut  et} \colorbox{Green!0.000}{\strut  al} \colorbox{Green!0.000}{\strut .} \\
Output SAE & \num{6.294e+00} & \colorbox{Magenta!0.000}{\strut  trends} \colorbox{Magenta!0.000}{\strut  in} \colorbox{Magenta!0.000}{\strut  winter} \colorbox{Magenta!0.000}{\strut  and} \colorbox{Magenta!0.000}{\strut  negative} \colorbox{Magenta!0.000}{\strut  trends} \colorbox{Magenta!0.000}{\strut  in} \colorbox{Magenta!0.000}{\strut  summer} \colorbox{Magenta!0.000}{\strut  precipitation} \colorbox{Magenta!0.000}{\strut .} \colorbox{Magenta!0.000}{\strut  No} \colorbox{Magenta!0.000}{\strut one} \colorbox{Magenta!94.826}{\strut  et} \colorbox{Magenta!9.400}{\strut  al} \colorbox{Magenta!0.000}{\strut .} \\
\midrule
Jacobian & \num{2.928e-01} & \colorbox{Cyan!0.000}{\strut  that} \colorbox{Cyan!0.000}{\strut  were} \colorbox{Cyan!0.000}{\strut  developed} \colorbox{Cyan!0.000}{\strut  by} \colorbox{Cyan!0.000}{\strut  Han} \colorbox{Cyan!0.000}{\strut using} \colorbox{Cyan!0.000}{\strut  the} \colorbox{Cyan!0.000}{\strut  method} \colorbox{Cyan!0.000}{\strut  described} \colorbox{Cyan!0.000}{\strut  by} \colorbox{Cyan!0.000}{\strut  L} \colorbox{Cyan!0.000}{\strut ian} \colorbox{Cyan!98.615}{\strut  et} \colorbox{Cyan!0.000}{\strut  al} \\
Input SAE & \num{2.056e+01} & \colorbox{Green!0.000}{\strut  that} \colorbox{Green!0.000}{\strut  were} \colorbox{Green!0.000}{\strut  developed} \colorbox{Green!0.000}{\strut  by} \colorbox{Green!0.000}{\strut  Han} \colorbox{Green!0.000}{\strut using} \colorbox{Green!0.000}{\strut  the} \colorbox{Green!0.000}{\strut  method} \colorbox{Green!0.000}{\strut  described} \colorbox{Green!0.000}{\strut  by} \colorbox{Green!0.000}{\strut  L} \colorbox{Green!0.000}{\strut ian} \colorbox{Green!90.916}{\strut  et} \colorbox{Green!0.000}{\strut  al} \\
Output SAE & \num{6.269e+00} & \colorbox{Magenta!0.000}{\strut  that} \colorbox{Magenta!0.000}{\strut  were} \colorbox{Magenta!0.000}{\strut  developed} \colorbox{Magenta!0.000}{\strut  by} \colorbox{Magenta!0.000}{\strut  Han} \colorbox{Magenta!0.000}{\strut using} \colorbox{Magenta!0.000}{\strut  the} \colorbox{Magenta!0.000}{\strut  method} \colorbox{Magenta!0.000}{\strut  described} \colorbox{Magenta!0.000}{\strut  by} \colorbox{Magenta!0.000}{\strut  L} \colorbox{Magenta!0.000}{\strut ian} \colorbox{Magenta!94.462}{\strut  et} \colorbox{Magenta!11.316}{\strut  al} \\
\midrule
Jacobian & \num{2.925e-01} & \colorbox{Cyan!0.000}{\strut  Col} \colorbox{Cyan!0.000}{\strut lier} \colorbox{Cyan!0.000}{\strut ,} \colorbox{Cyan!0.000}{\strut  2004} \colorbox{Cyan!0.000}{\strut ;} \colorbox{Cyan!0.000}{\strut  Whit} \colorbox{Cyan!0.000}{\strut el} \colorbox{Cyan!0.000}{\strut aw} \colorbox{Cyan!98.500}{\strut  et} \colorbox{Cyan!0.000}{\strut  al} \colorbox{Cyan!0.000}{\strut .,} \colorbox{Cyan!0.000}{\strut  2005} \colorbox{Cyan!0.000}{\strut ;} \colorbox{Cyan!0.000}{\strut  Ni} \colorbox{Cyan!94.102}{\strut  et} \\
Input SAE & \num{2.244e+01} & \colorbox{Green!0.000}{\strut  Col} \colorbox{Green!0.000}{\strut lier} \colorbox{Green!0.000}{\strut ,} \colorbox{Green!0.000}{\strut  2004} \colorbox{Green!0.000}{\strut ;} \colorbox{Green!0.000}{\strut  Whit} \colorbox{Green!0.000}{\strut el} \colorbox{Green!0.000}{\strut aw} \colorbox{Green!91.055}{\strut  et} \colorbox{Green!0.000}{\strut  al} \colorbox{Green!0.000}{\strut .,} \colorbox{Green!0.000}{\strut  2005} \colorbox{Green!0.000}{\strut ;} \colorbox{Green!0.000}{\strut  Ni} \colorbox{Green!99.228}{\strut  et} \\
Output SAE & \num{6.333e+00} & \colorbox{Magenta!0.000}{\strut  Col} \colorbox{Magenta!0.000}{\strut lier} \colorbox{Magenta!0.000}{\strut ,} \colorbox{Magenta!0.000}{\strut  2004} \colorbox{Magenta!0.000}{\strut ;} \colorbox{Magenta!0.000}{\strut  Whit} \colorbox{Magenta!0.000}{\strut el} \colorbox{Magenta!0.000}{\strut aw} \colorbox{Magenta!95.421}{\strut  et} \colorbox{Magenta!11.470}{\strut  al} \colorbox{Magenta!0.000}{\strut .,} \colorbox{Magenta!0.000}{\strut  2005} \colorbox{Magenta!0.000}{\strut ;} \colorbox{Magenta!0.000}{\strut  Ni} \colorbox{Magenta!89.301}{\strut  et} \\
\midrule
Jacobian & \num{2.922e-01} & \colorbox{Cyan!0.000}{\strut  of} \colorbox{Cyan!0.000}{\strut  these} \colorbox{Cyan!0.000}{\strut  four} \colorbox{Cyan!0.000}{\strut  DNA} \colorbox{Cyan!0.000}{\strut  regions} \colorbox{Cyan!0.000}{\strut  via} \colorbox{Cyan!0.000}{\strut  PCR} \colorbox{Cyan!0.000}{\strut  were} \colorbox{Cyan!0.000}{\strut  described} \colorbox{Cyan!0.000}{\strut  in} \colorbox{Cyan!0.000}{\strut  Tab} \colorbox{Cyan!0.000}{\strut er} \colorbox{Cyan!0.000}{\strut let} \colorbox{Cyan!98.424}{\strut  et} \colorbox{Cyan!0.000}{\strut  al} \\
Input SAE & \num{2.016e+01} & \colorbox{Green!0.000}{\strut  of} \colorbox{Green!0.000}{\strut  these} \colorbox{Green!0.000}{\strut  four} \colorbox{Green!0.000}{\strut  DNA} \colorbox{Green!0.000}{\strut  regions} \colorbox{Green!0.000}{\strut  via} \colorbox{Green!0.000}{\strut  PCR} \colorbox{Green!0.000}{\strut  were} \colorbox{Green!0.000}{\strut  described} \colorbox{Green!0.000}{\strut  in} \colorbox{Green!0.000}{\strut  Tab} \colorbox{Green!0.000}{\strut er} \colorbox{Green!0.000}{\strut let} \colorbox{Green!89.154}{\strut  et} \colorbox{Green!0.000}{\strut  al} \\
Output SAE & \num{6.504e+00} & \colorbox{Magenta!0.000}{\strut  of} \colorbox{Magenta!0.000}{\strut  these} \colorbox{Magenta!0.000}{\strut  four} \colorbox{Magenta!0.000}{\strut  DNA} \colorbox{Magenta!0.000}{\strut  regions} \colorbox{Magenta!0.000}{\strut  via} \colorbox{Magenta!0.000}{\strut  PCR} \colorbox{Magenta!0.000}{\strut  were} \colorbox{Magenta!0.000}{\strut  described} \colorbox{Magenta!0.000}{\strut  in} \colorbox{Magenta!0.000}{\strut  Tab} \colorbox{Magenta!0.000}{\strut er} \colorbox{Magenta!0.000}{\strut let} \colorbox{Magenta!97.999}{\strut  et} \colorbox{Magenta!12.064}{\strut  al} \\
\midrule
Jacobian & \num{2.922e-01} & \colorbox{Cyan!0.000}{\strut  maintain} \colorbox{Cyan!0.000}{\strut  or} \colorbox{Cyan!0.000}{\strut  improve} \colorbox{Cyan!0.000}{\strut  lung} \colorbox{Cyan!0.000}{\strut  and} \colorbox{Cyan!0.000}{\strut  chest} \colorbox{Cyan!0.000}{\strut  wall} \colorbox{Cyan!0.000}{\strut  compliance} \colorbox{Cyan!0.000}{\strut  (} \colorbox{Cyan!0.000}{\strut D} \colorbox{Cyan!0.000}{\strut ail} \colorbox{Cyan!98.423}{\strut  et} \colorbox{Cyan!0.000}{\strut  al} \colorbox{Cyan!0.000}{\strut  1955} \colorbox{Cyan!0.000}{\strut ).} \\
Input SAE & \num{1.913e+01} & \colorbox{Green!0.000}{\strut  maintain} \colorbox{Green!0.000}{\strut  or} \colorbox{Green!0.000}{\strut  improve} \colorbox{Green!0.000}{\strut  lung} \colorbox{Green!0.000}{\strut  and} \colorbox{Green!0.000}{\strut  chest} \colorbox{Green!0.000}{\strut  wall} \colorbox{Green!0.000}{\strut  compliance} \colorbox{Green!0.000}{\strut  (} \colorbox{Green!0.000}{\strut D} \colorbox{Green!0.000}{\strut ail} \colorbox{Green!84.599}{\strut  et} \colorbox{Green!0.000}{\strut  al} \colorbox{Green!0.000}{\strut  1955} \colorbox{Green!0.000}{\strut ).} \\
Output SAE & \num{6.179e+00} & \colorbox{Magenta!0.000}{\strut  maintain} \colorbox{Magenta!0.000}{\strut  or} \colorbox{Magenta!0.000}{\strut  improve} \colorbox{Magenta!0.000}{\strut  lung} \colorbox{Magenta!0.000}{\strut  and} \colorbox{Magenta!0.000}{\strut  chest} \colorbox{Magenta!0.000}{\strut  wall} \colorbox{Magenta!0.000}{\strut  compliance} \colorbox{Magenta!0.000}{\strut  (} \colorbox{Magenta!0.000}{\strut D} \colorbox{Magenta!0.000}{\strut ail} \colorbox{Magenta!93.101}{\strut  et} \colorbox{Magenta!11.683}{\strut  al} \colorbox{Magenta!0.000}{\strut  1955} \colorbox{Magenta!0.000}{\strut ).} \\
\midrule
Jacobian & \num{2.921e-01} & \colorbox{Cyan!0.000}{\strut Wang} \colorbox{Cyan!98.371}{\strut  et} \colorbox{Cyan!0.000}{\strut  al} \colorbox{Cyan!0.000}{\strut .} \colorbox{Cyan!0.000}{\strut  2016} \colorbox{Cyan!0.000}{\strut )} \colorbox{Cyan!0.000}{\strut  investigated} \colorbox{Cyan!0.000}{\strut  the} \colorbox{Cyan!0.000}{\strut  correlation} \colorbox{Cyan!0.000}{\strut  between} \colorbox{Cyan!0.000}{\strut  mutual} \colorbox{Cyan!0.000}{\strut  funds} \colorbox{Cyan!0.000}{\strut \textquotesingle{}} \colorbox{Cyan!0.000}{\strut  scale} \colorbox{Cyan!0.000}{\strut  and} \\
Input SAE & \num{1.674e+01} & \colorbox{Green!0.000}{\strut Wang} \colorbox{Green!74.029}{\strut  et} \colorbox{Green!0.000}{\strut  al} \colorbox{Green!0.000}{\strut .} \colorbox{Green!0.000}{\strut  2016} \colorbox{Green!0.000}{\strut )} \colorbox{Green!0.000}{\strut  investigated} \colorbox{Green!0.000}{\strut  the} \colorbox{Green!0.000}{\strut  correlation} \colorbox{Green!0.000}{\strut  between} \colorbox{Green!0.000}{\strut  mutual} \colorbox{Green!0.000}{\strut  funds} \colorbox{Green!0.000}{\strut \textquotesingle{}} \colorbox{Green!0.000}{\strut  scale} \colorbox{Green!0.000}{\strut  and} \\
Output SAE & \num{5.520e+00} & \colorbox{Magenta!0.000}{\strut Wang} \colorbox{Magenta!83.167}{\strut  et} \colorbox{Magenta!14.114}{\strut  al} \colorbox{Magenta!0.000}{\strut .} \colorbox{Magenta!0.000}{\strut  2016} \colorbox{Magenta!0.000}{\strut )} \colorbox{Magenta!0.000}{\strut  investigated} \colorbox{Magenta!0.000}{\strut  the} \colorbox{Magenta!0.000}{\strut  correlation} \colorbox{Magenta!0.000}{\strut  between} \colorbox{Magenta!0.000}{\strut  mutual} \colorbox{Magenta!0.000}{\strut  funds} \colorbox{Magenta!0.000}{\strut \textquotesingle{}} \colorbox{Magenta!0.000}{\strut  scale} \colorbox{Magenta!0.000}{\strut  and} \\
\midrule
Jacobian & \num{2.920e-01} & \colorbox{Cyan!0.000}{\strut ,} \colorbox{Cyan!0.000}{\strut  PO} \colorbox{Cyan!0.000}{\strut 2} \colorbox{Cyan!0.000}{\strut  remains} \colorbox{Cyan!0.000}{\strut  constant} \colorbox{Cyan!0.000}{\strut .} \colorbox{Cyan!0.000}{\strut Richard} \colorbox{Cyan!0.000}{\strut son} \colorbox{Cyan!98.354}{\strut  et} \colorbox{Cyan!0.000}{\strut  al} \colorbox{Cyan!0.000}{\strut .} \colorbox{Cyan!0.000}{\strut  J} \colorbox{Cyan!0.000}{\strut AP} \colorbox{Cyan!0.000}{\strut  85} \\
Input SAE & \num{1.974e+01} & \colorbox{Green!0.000}{\strut ,} \colorbox{Green!0.000}{\strut  PO} \colorbox{Green!0.000}{\strut 2} \colorbox{Green!0.000}{\strut  remains} \colorbox{Green!0.000}{\strut  constant} \colorbox{Green!0.000}{\strut .} \colorbox{Green!0.000}{\strut Richard} \colorbox{Green!0.000}{\strut son} \colorbox{Green!87.309}{\strut  et} \colorbox{Green!0.000}{\strut  al} \colorbox{Green!0.000}{\strut .} \colorbox{Green!0.000}{\strut  J} \colorbox{Green!0.000}{\strut AP} \colorbox{Green!0.000}{\strut  85} \\
Output SAE & \num{6.236e+00} & \colorbox{Magenta!0.000}{\strut ,} \colorbox{Magenta!0.000}{\strut  PO} \colorbox{Magenta!0.000}{\strut 2} \colorbox{Magenta!0.000}{\strut  remains} \colorbox{Magenta!0.000}{\strut  constant} \colorbox{Magenta!0.000}{\strut .} \colorbox{Magenta!0.000}{\strut Richard} \colorbox{Magenta!0.000}{\strut son} \colorbox{Magenta!93.955}{\strut  et} \colorbox{Magenta!8.620}{\strut  al} \colorbox{Magenta!0.000}{\strut .} \colorbox{Magenta!0.000}{\strut  J} \colorbox{Magenta!0.000}{\strut AP} \colorbox{Magenta!0.000}{\strut  85} \\
\bottomrule
\end{longtable}
\caption{feature pairs/Layer15-65536-J1-LR5.0e-04-k32-T3.0e+08 abs mean/examples-33555-v-60468 stas c4-en-10k,train,batch size=32,ctx len=16.csv}
\end{table} % et in et al.
\begin{table}
\centering
\begin{tabular}{lrl}
\toprule
Category & Max. abs. value & Example tokens \\
\midrule
Jacobian & \num{2.788e-01} & \colorbox{Cyan!0.000}{\strut  flood} \colorbox{Cyan!0.000}{\strut  barrier} \colorbox{Cyan!0.000}{\strut  (} \colorbox{Cyan!0.000}{\strut Gre} \colorbox{Cyan!0.000}{\strut if} \colorbox{Cyan!0.000}{\strut sw} \colorbox{Cyan!100.000}{\strut ald} \colorbox{Cyan!0.000}{\strut ,} \colorbox{Cyan!0.000}{\strut  Germany} \colorbox{Cyan!0.000}{\strut ).} \colorbox{Cyan!0.000}{\strut The} \colorbox{Cyan!0.000}{\strut  b} \colorbox{Cyan!0.000}{\strut MC} \colorbox{Cyan!0.000}{\strut  Team} \\
Input SAE & \num{1.121e+00} & \colorbox{Green!0.000}{\strut  flood} \colorbox{Green!0.000}{\strut  barrier} \colorbox{Green!0.000}{\strut  (} \colorbox{Green!0.000}{\strut Gre} \colorbox{Green!0.000}{\strut if} \colorbox{Green!0.000}{\strut sw} \colorbox{Green!15.154}{\strut ald} \colorbox{Green!0.000}{\strut ,} \colorbox{Green!0.000}{\strut  Germany} \colorbox{Green!0.000}{\strut ).} \colorbox{Green!0.000}{\strut The} \colorbox{Green!0.000}{\strut  b} \colorbox{Green!0.000}{\strut MC} \colorbox{Green!0.000}{\strut  Team} \\
Output SAE & \num{5.615e-01} & \colorbox{Magenta!0.000}{\strut  flood} \colorbox{Magenta!0.000}{\strut  barrier} \colorbox{Magenta!0.000}{\strut  (} \colorbox{Magenta!0.000}{\strut Gre} \colorbox{Magenta!0.000}{\strut if} \colorbox{Magenta!0.000}{\strut sw} \colorbox{Magenta!12.112}{\strut ald} \colorbox{Magenta!0.000}{\strut ,} \colorbox{Magenta!0.000}{\strut  Germany} \colorbox{Magenta!0.000}{\strut ).} \colorbox{Magenta!0.000}{\strut The} \colorbox{Magenta!0.000}{\strut  b} \colorbox{Magenta!0.000}{\strut MC} \colorbox{Magenta!0.000}{\strut  Team} \\
\midrule
Jacobian & \num{2.773e-01} & \colorbox{Cyan!0.000}{\strut  Center} \colorbox{Cyan!0.000}{\strut .} \colorbox{Cyan!0.000}{\strut  Her} \colorbox{Cyan!0.000}{\strut  works} \colorbox{Cyan!0.000}{\strut  have} \colorbox{Cyan!0.000}{\strut  also} \colorbox{Cyan!0.000}{\strut  been} \colorbox{Cyan!0.000}{\strut  produced} \colorbox{Cyan!0.000}{\strut  at} \colorbox{Cyan!0.000}{\strut  Ste} \colorbox{Cyan!99.462}{\strut ppen} \colorbox{Cyan!0.000}{\strut wolf} \colorbox{Cyan!0.000}{\strut  Theatre} \colorbox{Cyan!0.000}{\strut  Company} \colorbox{Cyan!0.000}{\strut ,} \\
Input SAE & \num{2.075e+00} & \colorbox{Green!0.000}{\strut  Center} \colorbox{Green!0.000}{\strut .} \colorbox{Green!0.000}{\strut  Her} \colorbox{Green!0.000}{\strut  works} \colorbox{Green!0.000}{\strut  have} \colorbox{Green!0.000}{\strut  also} \colorbox{Green!0.000}{\strut  been} \colorbox{Green!0.000}{\strut  produced} \colorbox{Green!0.000}{\strut  at} \colorbox{Green!0.000}{\strut  Ste} \colorbox{Green!28.046}{\strut ppen} \colorbox{Green!0.000}{\strut wolf} \colorbox{Green!0.000}{\strut  Theatre} \colorbox{Green!0.000}{\strut  Company} \colorbox{Green!0.000}{\strut ,} \\
Output SAE & \num{9.516e-01} & \colorbox{Magenta!0.000}{\strut  Center} \colorbox{Magenta!0.000}{\strut .} \colorbox{Magenta!0.000}{\strut  Her} \colorbox{Magenta!0.000}{\strut  works} \colorbox{Magenta!0.000}{\strut  have} \colorbox{Magenta!0.000}{\strut  also} \colorbox{Magenta!0.000}{\strut  been} \colorbox{Magenta!0.000}{\strut  produced} \colorbox{Magenta!0.000}{\strut  at} \colorbox{Magenta!0.000}{\strut  Ste} \colorbox{Magenta!20.528}{\strut ppen} \colorbox{Magenta!0.000}{\strut wolf} \colorbox{Magenta!0.000}{\strut  Theatre} \colorbox{Magenta!0.000}{\strut  Company} \colorbox{Magenta!0.000}{\strut ,} \\
\midrule
Jacobian & \num{2.773e-01} & \colorbox{Cyan!0.000}{\strut 102} \colorbox{Cyan!0.000}{\strut  S} \colorbox{Cyan!0.000}{\strut A} \colorbox{Cyan!98.213}{\strut nt} \colorbox{Cyan!99.447}{\strut is} \colorbox{Cyan!0.000}{\strut  250} \colorbox{Cyan!0.000}{\strut 3} \colorbox{Cyan!0.000}{\strut  m} \colorbox{Cyan!0.000}{\strut  2021} \colorbox{Cyan!0.000}{\strut  m} \colorbox{Cyan!0.000}{\strut  1} \colorbox{Cyan!0.000}{\strut 150} \colorbox{Cyan!0.000}{\strut  m} \colorbox{Cyan!0.000}{\strut  Switzerland} \\
Input SAE & \num{1.529e+00} & \colorbox{Green!0.000}{\strut 102} \colorbox{Green!0.000}{\strut  S} \colorbox{Green!0.000}{\strut A} \colorbox{Green!20.670}{\strut nt} \colorbox{Green!18.883}{\strut is} \colorbox{Green!0.000}{\strut  250} \colorbox{Green!0.000}{\strut 3} \colorbox{Green!0.000}{\strut  m} \colorbox{Green!0.000}{\strut  2021} \colorbox{Green!0.000}{\strut  m} \colorbox{Green!0.000}{\strut  1} \colorbox{Green!0.000}{\strut 150} \colorbox{Green!0.000}{\strut  m} \colorbox{Green!0.000}{\strut  Switzerland} \\
Output SAE & \num{1.039e+00} & \colorbox{Magenta!0.000}{\strut 102} \colorbox{Magenta!0.000}{\strut  S} \colorbox{Magenta!18.329}{\strut A} \colorbox{Magenta!22.412}{\strut nt} \colorbox{Magenta!19.036}{\strut is} \colorbox{Magenta!0.000}{\strut  250} \colorbox{Magenta!0.000}{\strut 3} \colorbox{Magenta!13.362}{\strut  m} \colorbox{Magenta!0.000}{\strut  2021} \colorbox{Magenta!0.000}{\strut  m} \colorbox{Magenta!0.000}{\strut  1} \colorbox{Magenta!0.000}{\strut 150} \colorbox{Magenta!0.000}{\strut  m} \colorbox{Magenta!0.000}{\strut  Switzerland} \\
\midrule
Jacobian & \num{2.772e-01} & \colorbox{Cyan!0.000}{\strut  K} \colorbox{Cyan!0.000}{\strut ont} \colorbox{Cyan!0.000}{\strut ak} \colorbox{Cyan!0.000}{\strut  B} \colorbox{Cyan!0.000}{\strut BM} \colorbox{Cyan!0.000}{\strut  Mur} \colorbox{Cyan!0.000}{\strut ah} \colorbox{Cyan!99.409}{\strut  Berg} \colorbox{Cyan!0.000}{\strut ar} \colorbox{Cyan!0.000}{\strut ans} \colorbox{Cyan!0.000}{\strut i} \colorbox{Cyan!0.000}{\strut  \textbar{}} \colorbox{Cyan!0.000}{\strut  How} \colorbox{Cyan!0.000}{\strut  to} \colorbox{Cyan!0.000}{\strut  communication} \\
Input SAE & \num{1.235e+00} & \colorbox{Green!0.000}{\strut  K} \colorbox{Green!0.000}{\strut ont} \colorbox{Green!0.000}{\strut ak} \colorbox{Green!0.000}{\strut  B} \colorbox{Green!0.000}{\strut BM} \colorbox{Green!0.000}{\strut  Mur} \colorbox{Green!0.000}{\strut ah} \colorbox{Green!16.697}{\strut  Berg} \colorbox{Green!0.000}{\strut ar} \colorbox{Green!0.000}{\strut ans} \colorbox{Green!0.000}{\strut i} \colorbox{Green!0.000}{\strut  \textbar{}} \colorbox{Green!0.000}{\strut  How} \colorbox{Green!0.000}{\strut  to} \colorbox{Green!0.000}{\strut  communication} \\
Output SAE & \num{8.479e-01} & \colorbox{Magenta!0.000}{\strut  K} \colorbox{Magenta!0.000}{\strut ont} \colorbox{Magenta!0.000}{\strut ak} \colorbox{Magenta!0.000}{\strut  B} \colorbox{Magenta!0.000}{\strut BM} \colorbox{Magenta!0.000}{\strut  Mur} \colorbox{Magenta!0.000}{\strut ah} \colorbox{Magenta!17.429}{\strut  Berg} \colorbox{Magenta!0.000}{\strut ar} \colorbox{Magenta!18.291}{\strut ans} \colorbox{Magenta!0.000}{\strut i} \colorbox{Magenta!0.000}{\strut  \textbar{}} \colorbox{Magenta!0.000}{\strut  How} \colorbox{Magenta!0.000}{\strut  to} \colorbox{Magenta!0.000}{\strut  communication} \\
\midrule
Jacobian & \num{2.767e-01} & \colorbox{Cyan!0.000}{\strut  get} \colorbox{Cyan!0.000}{\strut  quality} \colorbox{Cyan!0.000}{\strut  care} \colorbox{Cyan!0.000}{\strut  and} \colorbox{Cyan!0.000}{\strut  all} \colorbox{Cyan!0.000}{\strut  hospitals} \colorbox{Cyan!0.000}{\strut  in} \colorbox{Cyan!99.221}{\strut  Berg} \colorbox{Cyan!93.980}{\strut en} \colorbox{Cyan!0.000}{\strut  County} \colorbox{Cyan!0.000}{\strut  will} \colorbox{Cyan!0.000}{\strut  see} \colorbox{Cyan!0.000}{\strut  less} \colorbox{Cyan!0.000}{\strut  traffic} \colorbox{Cyan!0.000}{\strut  in} \\
Input SAE & \num{3.480e+00} & \colorbox{Green!0.000}{\strut  get} \colorbox{Green!0.000}{\strut  quality} \colorbox{Green!0.000}{\strut  care} \colorbox{Green!0.000}{\strut  and} \colorbox{Green!0.000}{\strut  all} \colorbox{Green!0.000}{\strut  hospitals} \colorbox{Green!0.000}{\strut  in} \colorbox{Green!14.230}{\strut  Berg} \colorbox{Green!47.032}{\strut en} \colorbox{Green!0.000}{\strut  County} \colorbox{Green!0.000}{\strut  will} \colorbox{Green!0.000}{\strut  see} \colorbox{Green!0.000}{\strut  less} \colorbox{Green!0.000}{\strut  traffic} \colorbox{Green!0.000}{\strut  in} \\
Output SAE & \num{8.445e-01} & \colorbox{Magenta!0.000}{\strut  get} \colorbox{Magenta!0.000}{\strut  quality} \colorbox{Magenta!0.000}{\strut  care} \colorbox{Magenta!0.000}{\strut  and} \colorbox{Magenta!0.000}{\strut  all} \colorbox{Magenta!0.000}{\strut  hospitals} \colorbox{Magenta!0.000}{\strut  in} \colorbox{Magenta!18.219}{\strut  Berg} \colorbox{Magenta!15.544}{\strut en} \colorbox{Magenta!0.000}{\strut  County} \colorbox{Magenta!0.000}{\strut  will} \colorbox{Magenta!0.000}{\strut  see} \colorbox{Magenta!0.000}{\strut  less} \colorbox{Magenta!0.000}{\strut  traffic} \colorbox{Magenta!0.000}{\strut  in} \\
\midrule
Jacobian & \num{2.753e-01} & \colorbox{Cyan!0.000}{\strut  Pay} \colorbox{Cyan!0.000}{\strut .} \colorbox{Cyan!0.000}{\strut Car} \colorbox{Cyan!0.000}{\strut amel} \colorbox{Cyan!0.000}{\strut  eggs} \colorbox{Cyan!0.000}{\strut  by} \colorbox{Cyan!0.000}{\strut  Cad} \colorbox{Cyan!0.000}{\strut bury} \colorbox{Cyan!0.000}{\strut  UK} \colorbox{Cyan!0.000}{\strut ,} \colorbox{Cyan!0.000}{\strut  available} \colorbox{Cyan!0.000}{\strut  at} \colorbox{Cyan!0.000}{\strut  Eng} \colorbox{Cyan!98.724}{\strut en} \\
Input SAE & \num{1.189e+00} & \colorbox{Green!0.000}{\strut  Pay} \colorbox{Green!0.000}{\strut .} \colorbox{Green!0.000}{\strut Car} \colorbox{Green!0.000}{\strut amel} \colorbox{Green!0.000}{\strut  eggs} \colorbox{Green!0.000}{\strut  by} \colorbox{Green!0.000}{\strut  Cad} \colorbox{Green!0.000}{\strut bury} \colorbox{Green!0.000}{\strut  UK} \colorbox{Green!0.000}{\strut ,} \colorbox{Green!0.000}{\strut  available} \colorbox{Green!0.000}{\strut  at} \colorbox{Green!0.000}{\strut  Eng} \colorbox{Green!16.077}{\strut en} \\
Output SAE & \num{6.593e-01} & \colorbox{Magenta!0.000}{\strut  Pay} \colorbox{Magenta!0.000}{\strut .} \colorbox{Magenta!0.000}{\strut Car} \colorbox{Magenta!0.000}{\strut amel} \colorbox{Magenta!0.000}{\strut  eggs} \colorbox{Magenta!0.000}{\strut  by} \colorbox{Magenta!0.000}{\strut  Cad} \colorbox{Magenta!0.000}{\strut bury} \colorbox{Magenta!0.000}{\strut  UK} \colorbox{Magenta!0.000}{\strut ,} \colorbox{Magenta!0.000}{\strut  available} \colorbox{Magenta!0.000}{\strut  at} \colorbox{Magenta!0.000}{\strut  Eng} \colorbox{Magenta!14.223}{\strut en} \\
\midrule
Jacobian & \num{2.749e-01} & \colorbox{Cyan!0.000}{\strut  of} \colorbox{Cyan!0.000}{\strut  the} \colorbox{Cyan!0.000}{\strut  ER} \colorbox{Cyan!0.000}{\strut  by} \colorbox{Cyan!0.000}{\strut  providing} \colorbox{Cyan!0.000}{\strut  free} \colorbox{Cyan!0.000}{\strut ,} \colorbox{Cyan!0.000}{\strut  ongoing} \colorbox{Cyan!0.000}{\strut  primary} \colorbox{Cyan!0.000}{\strut  care} \colorbox{Cyan!0.000}{\strut  to} \colorbox{Cyan!0.000}{\strut  residents} \colorbox{Cyan!0.000}{\strut  of} \colorbox{Cyan!98.574}{\strut  Berg} \colorbox{Cyan!95.006}{\strut en} \\
Input SAE & \num{3.194e+00} & \colorbox{Green!0.000}{\strut  of} \colorbox{Green!0.000}{\strut  the} \colorbox{Green!0.000}{\strut  ER} \colorbox{Green!0.000}{\strut  by} \colorbox{Green!0.000}{\strut  providing} \colorbox{Green!0.000}{\strut  free} \colorbox{Green!0.000}{\strut ,} \colorbox{Green!0.000}{\strut  ongoing} \colorbox{Green!0.000}{\strut  primary} \colorbox{Green!0.000}{\strut  care} \colorbox{Green!0.000}{\strut  to} \colorbox{Green!0.000}{\strut  residents} \colorbox{Green!0.000}{\strut  of} \colorbox{Green!12.879}{\strut  Berg} \colorbox{Green!43.169}{\strut en} \\
Output SAE & \num{1.024e+00} & \colorbox{Magenta!0.000}{\strut  of} \colorbox{Magenta!0.000}{\strut  the} \colorbox{Magenta!0.000}{\strut  ER} \colorbox{Magenta!0.000}{\strut  by} \colorbox{Magenta!0.000}{\strut  providing} \colorbox{Magenta!0.000}{\strut  free} \colorbox{Magenta!0.000}{\strut ,} \colorbox{Magenta!0.000}{\strut  ongoing} \colorbox{Magenta!0.000}{\strut  primary} \colorbox{Magenta!0.000}{\strut  care} \colorbox{Magenta!0.000}{\strut  to} \colorbox{Magenta!0.000}{\strut  residents} \colorbox{Magenta!0.000}{\strut  of} \colorbox{Magenta!15.575}{\strut  Berg} \colorbox{Magenta!22.089}{\strut en} \\
\midrule
Jacobian & \num{2.744e-01} & \colorbox{Cyan!0.000}{\strut .} \colorbox{Cyan!0.000}{\strut C} \colorbox{Cyan!0.000}{\strut reme} \colorbox{Cyan!0.000}{\strut  egg} \colorbox{Cyan!0.000}{\strut  by} \colorbox{Cyan!0.000}{\strut  Cad} \colorbox{Cyan!0.000}{\strut bury} \colorbox{Cyan!0.000}{\strut  UK} \colorbox{Cyan!0.000}{\strut ,} \colorbox{Cyan!0.000}{\strut  available} \colorbox{Cyan!0.000}{\strut  at} \colorbox{Cyan!0.000}{\strut  Eng} \colorbox{Cyan!98.416}{\strut en} \colorbox{Cyan!0.000}{\strut .} \\
Input SAE & \num{1.206e+00} & \colorbox{Green!0.000}{\strut .} \colorbox{Green!0.000}{\strut C} \colorbox{Green!0.000}{\strut reme} \colorbox{Green!0.000}{\strut  egg} \colorbox{Green!0.000}{\strut  by} \colorbox{Green!0.000}{\strut  Cad} \colorbox{Green!0.000}{\strut bury} \colorbox{Green!0.000}{\strut  UK} \colorbox{Green!0.000}{\strut ,} \colorbox{Green!0.000}{\strut  available} \colorbox{Green!0.000}{\strut  at} \colorbox{Green!0.000}{\strut  Eng} \colorbox{Green!16.306}{\strut en} \colorbox{Green!0.000}{\strut .} \\
Output SAE & \num{6.785e-01} & \colorbox{Magenta!0.000}{\strut .} \colorbox{Magenta!0.000}{\strut C} \colorbox{Magenta!0.000}{\strut reme} \colorbox{Magenta!0.000}{\strut  egg} \colorbox{Magenta!0.000}{\strut  by} \colorbox{Magenta!0.000}{\strut  Cad} \colorbox{Magenta!0.000}{\strut bury} \colorbox{Magenta!0.000}{\strut  UK} \colorbox{Magenta!0.000}{\strut ,} \colorbox{Magenta!0.000}{\strut  available} \colorbox{Magenta!0.000}{\strut  at} \colorbox{Magenta!0.000}{\strut  Eng} \colorbox{Magenta!14.636}{\strut en} \colorbox{Magenta!0.000}{\strut .} \\
\midrule
Jacobian & \num{2.736e-01} & \colorbox{Cyan!0.000}{\strut .} \colorbox{Cyan!0.000}{\strut O} \colorbox{Cyan!0.000}{\strut reo} \colorbox{Cyan!0.000}{\strut  filled} \colorbox{Cyan!0.000}{\strut  eggs} \colorbox{Cyan!0.000}{\strut  by} \colorbox{Cyan!0.000}{\strut  Cad} \colorbox{Cyan!0.000}{\strut bury} \colorbox{Cyan!0.000}{\strut  UK} \colorbox{Cyan!0.000}{\strut ,} \colorbox{Cyan!0.000}{\strut  available} \colorbox{Cyan!0.000}{\strut  at} \colorbox{Cyan!0.000}{\strut  Eng} \colorbox{Cyan!98.115}{\strut en} \\
Input SAE & \num{1.260e+00} & \colorbox{Green!0.000}{\strut .} \colorbox{Green!0.000}{\strut O} \colorbox{Green!0.000}{\strut reo} \colorbox{Green!0.000}{\strut  filled} \colorbox{Green!0.000}{\strut  eggs} \colorbox{Green!0.000}{\strut  by} \colorbox{Green!0.000}{\strut  Cad} \colorbox{Green!0.000}{\strut bury} \colorbox{Green!0.000}{\strut  UK} \colorbox{Green!0.000}{\strut ,} \colorbox{Green!0.000}{\strut  available} \colorbox{Green!0.000}{\strut  at} \colorbox{Green!0.000}{\strut  Eng} \colorbox{Green!17.029}{\strut en} \\
Output SAE & \num{6.371e-01} & \colorbox{Magenta!0.000}{\strut .} \colorbox{Magenta!0.000}{\strut O} \colorbox{Magenta!0.000}{\strut reo} \colorbox{Magenta!0.000}{\strut  filled} \colorbox{Magenta!0.000}{\strut  eggs} \colorbox{Magenta!0.000}{\strut  by} \colorbox{Magenta!0.000}{\strut  Cad} \colorbox{Magenta!0.000}{\strut bury} \colorbox{Magenta!0.000}{\strut  UK} \colorbox{Magenta!0.000}{\strut ,} \colorbox{Magenta!0.000}{\strut  available} \colorbox{Magenta!0.000}{\strut  at} \colorbox{Magenta!0.000}{\strut  Eng} \colorbox{Magenta!13.743}{\strut en} \\
\midrule
Jacobian & \num{2.735e-01} & \colorbox{Cyan!0.000}{\strut Car} \colorbox{Cyan!98.076}{\strut ls} \colorbox{Cyan!0.000}{\strut bad} \colorbox{Cyan!0.000}{\strut  horse} \colorbox{Cyan!0.000}{\strut  owner} \colorbox{Cyan!0.000}{\strut  and} \colorbox{Cyan!0.000}{\strut  handic} \colorbox{Cyan!0.000}{\strut apper} \colorbox{Cyan!0.000}{\strut  Jon} \colorbox{Cyan!89.776}{\strut  Lind} \colorbox{Cyan!0.000}{\strut o} \colorbox{Cyan!0.000}{\strut  reinforced} \colorbox{Cyan!0.000}{\strut  Santa} \colorbox{Cyan!0.000}{\strut  An} \\
Input SAE & \num{1.743e+00} & \colorbox{Green!0.000}{\strut Car} \colorbox{Green!16.005}{\strut ls} \colorbox{Green!0.000}{\strut bad} \colorbox{Green!0.000}{\strut  horse} \colorbox{Green!0.000}{\strut  owner} \colorbox{Green!0.000}{\strut  and} \colorbox{Green!0.000}{\strut  handic} \colorbox{Green!0.000}{\strut apper} \colorbox{Green!8.235}{\strut  Jon} \colorbox{Green!23.563}{\strut  Lind} \colorbox{Green!0.000}{\strut o} \colorbox{Green!0.000}{\strut  reinforced} \colorbox{Green!0.000}{\strut  Santa} \colorbox{Green!0.000}{\strut  An} \\
Output SAE & \num{1.431e+00} & \colorbox{Magenta!0.000}{\strut Car} \colorbox{Magenta!16.200}{\strut ls} \colorbox{Magenta!0.000}{\strut bad} \colorbox{Magenta!0.000}{\strut  horse} \colorbox{Magenta!0.000}{\strut  owner} \colorbox{Magenta!0.000}{\strut  and} \colorbox{Magenta!0.000}{\strut  handic} \colorbox{Magenta!0.000}{\strut apper} \colorbox{Magenta!0.000}{\strut  Jon} \colorbox{Magenta!30.860}{\strut  Lind} \colorbox{Magenta!0.000}{\strut o} \colorbox{Magenta!0.000}{\strut  reinforced} \colorbox{Magenta!0.000}{\strut  Santa} \colorbox{Magenta!0.000}{\strut  An} \\
\midrule
Jacobian & \num{2.732e-01} & \colorbox{Cyan!0.000}{\strut  Policy} \colorbox{Cyan!0.000}{\strut  faculty} \colorbox{Cyan!0.000}{\strut  members} \colorbox{Cyan!0.000}{\strut  Ter} \colorbox{Cyan!0.000}{\strut ri} \colorbox{Cyan!0.000}{\strut  Sab} \colorbox{Cyan!0.000}{\strut ol} \colorbox{Cyan!0.000}{\strut  and} \colorbox{Cyan!0.000}{\strut  Hann} \colorbox{Cyan!97.979}{\strut es} \colorbox{Cyan!0.000}{\strut  Schw} \colorbox{Cyan!0.000}{\strut and} \colorbox{Cyan!0.000}{\strut t} \colorbox{Cyan!0.000}{\strut  are} \colorbox{Cyan!0.000}{\strut  among} \\
Input SAE & \num{2.084e+00} & \colorbox{Green!0.000}{\strut  Policy} \colorbox{Green!0.000}{\strut  faculty} \colorbox{Green!0.000}{\strut  members} \colorbox{Green!0.000}{\strut  Ter} \colorbox{Green!0.000}{\strut ri} \colorbox{Green!0.000}{\strut  Sab} \colorbox{Green!0.000}{\strut ol} \colorbox{Green!0.000}{\strut  and} \colorbox{Green!0.000}{\strut  Hann} \colorbox{Green!28.167}{\strut es} \colorbox{Green!0.000}{\strut  Schw} \colorbox{Green!0.000}{\strut and} \colorbox{Green!0.000}{\strut t} \colorbox{Green!0.000}{\strut  are} \colorbox{Green!0.000}{\strut  among} \\
Output SAE & \num{1.030e+00} & \colorbox{Magenta!0.000}{\strut  Policy} \colorbox{Magenta!0.000}{\strut  faculty} \colorbox{Magenta!0.000}{\strut  members} \colorbox{Magenta!0.000}{\strut  Ter} \colorbox{Magenta!0.000}{\strut ri} \colorbox{Magenta!0.000}{\strut  Sab} \colorbox{Magenta!0.000}{\strut ol} \colorbox{Magenta!0.000}{\strut  and} \colorbox{Magenta!0.000}{\strut  Hann} \colorbox{Magenta!22.224}{\strut es} \colorbox{Magenta!0.000}{\strut  Schw} \colorbox{Magenta!0.000}{\strut and} \colorbox{Magenta!0.000}{\strut t} \colorbox{Magenta!0.000}{\strut  are} \colorbox{Magenta!0.000}{\strut  among} \\
\midrule
Jacobian & \num{2.731e-01} & \colorbox{Cyan!0.000}{\strut  catch} \colorbox{Cyan!0.000}{\strut  it} \colorbox{Cyan!0.000}{\strut .} \colorbox{Cyan!0.000}{\strut Y} \colorbox{Cyan!0.000}{\strut och} \colorbox{Cyan!0.000}{\strut  was} \colorbox{Cyan!0.000}{\strut  out} \colorbox{Cyan!0.000}{\strut  at} \colorbox{Cyan!0.000}{\strut  the} \colorbox{Cyan!0.000}{\strut  intersection} \colorbox{Cyan!0.000}{\strut  of} \colorbox{Cyan!0.000}{\strut  Ott} \colorbox{Cyan!97.937}{\strut en} \colorbox{Cyan!0.000}{\strut  and} \\
Input SAE & \num{2.490e+00} & \colorbox{Green!0.000}{\strut  catch} \colorbox{Green!0.000}{\strut  it} \colorbox{Green!0.000}{\strut .} \colorbox{Green!0.000}{\strut Y} \colorbox{Green!0.000}{\strut och} \colorbox{Green!0.000}{\strut  was} \colorbox{Green!0.000}{\strut  out} \colorbox{Green!0.000}{\strut  at} \colorbox{Green!0.000}{\strut  the} \colorbox{Green!0.000}{\strut  intersection} \colorbox{Green!0.000}{\strut  of} \colorbox{Green!0.000}{\strut  Ott} \colorbox{Green!33.651}{\strut en} \colorbox{Green!0.000}{\strut  and} \\
Output SAE & \num{9.981e-01} & \colorbox{Magenta!0.000}{\strut  catch} \colorbox{Magenta!0.000}{\strut  it} \colorbox{Magenta!0.000}{\strut .} \colorbox{Magenta!0.000}{\strut Y} \colorbox{Magenta!0.000}{\strut och} \colorbox{Magenta!0.000}{\strut  was} \colorbox{Magenta!0.000}{\strut  out} \colorbox{Magenta!0.000}{\strut  at} \colorbox{Magenta!0.000}{\strut  the} \colorbox{Magenta!0.000}{\strut  intersection} \colorbox{Magenta!0.000}{\strut  of} \colorbox{Magenta!0.000}{\strut  Ott} \colorbox{Magenta!21.531}{\strut en} \colorbox{Magenta!0.000}{\strut  and} \\
\bottomrule
\end{tabular}
% feature pairs/Layer15-65536-J1-LR5.0e-04-k32-T3.0e+08 abs mean/examples-48028-v-64386 stas c4-en-10k,train,batch size=32,ctx len=16.csv
\caption{
The top $12$ examples that produce the maximum absolute values of the Jacobian element with input SAE latent index $48028$ and output latent index $64386$.
% This pair of latent indices is one of the top $5$ pairs by the mean absolute value of non-zero Jacobian elements.
The Jacobian SAE pair was trained on layer 15 of Pythia-410m with an expansion factor of $R=64$ and sparsity $k=32$.
The examples were collected over the first 10K records of the English subset of the C4 text dataset \citep{raffel_exploring_2020}, with a context length of $16$ tokens.
For each example, the first row shows the values of the Jacobian element, and the second and third show the corresponding activations of the input and output SAE latents.
In this case, both SAE latents appear to weakly activate for tokens within proper nouns in German.
}
\label{tab:feature_pairs_48028_64386}
\end{table} % German syllables?
% \begin{table}
\centering
\begin{longtable}{lrl}
\toprule
Category & Max. abs. value & Example tokens \\
\midrule
Jacobian & \num{2.839e-01} & \colorbox{Cyan!0.000}{\strut  information} \colorbox{Cyan!0.000}{\strut  regarding} \colorbox{Cyan!0.000}{\strut  Local} \colorbox{Cyan!0.000}{\strut  9} \colorbox{Cyan!0.000}{\strut 80} \colorbox{Cyan!0.000}{\strut ,} \colorbox{Cyan!100.000}{\strut  as} \colorbox{Cyan!0.000}{\strut  well} \colorbox{Cyan!0.000}{\strut  as} \colorbox{Cyan!0.000}{\strut  information} \colorbox{Cyan!0.000}{\strut  pertaining} \colorbox{Cyan!0.000}{\strut  to} \colorbox{Cyan!0.000}{\strut  the} \colorbox{Cyan!0.000}{\strut  IA} \colorbox{Cyan!0.000}{\strut FF} \\
Input SAE & \num{1.670e+01} & \colorbox{Green!0.000}{\strut  information} \colorbox{Green!0.000}{\strut  regarding} \colorbox{Green!0.000}{\strut  Local} \colorbox{Green!0.000}{\strut  9} \colorbox{Green!0.000}{\strut 80} \colorbox{Green!0.000}{\strut ,} \colorbox{Green!87.496}{\strut  as} \colorbox{Green!0.000}{\strut  well} \colorbox{Green!0.000}{\strut  as} \colorbox{Green!0.000}{\strut  information} \colorbox{Green!0.000}{\strut  pertaining} \colorbox{Green!0.000}{\strut  to} \colorbox{Green!0.000}{\strut  the} \colorbox{Green!0.000}{\strut  IA} \colorbox{Green!0.000}{\strut FF} \\
Output SAE & \num{4.767e+00} & \colorbox{Magenta!0.000}{\strut  information} \colorbox{Magenta!0.000}{\strut  regarding} \colorbox{Magenta!0.000}{\strut  Local} \colorbox{Magenta!0.000}{\strut  9} \colorbox{Magenta!0.000}{\strut 80} \colorbox{Magenta!0.000}{\strut ,} \colorbox{Magenta!95.310}{\strut  as} \colorbox{Magenta!0.000}{\strut  well} \colorbox{Magenta!0.000}{\strut  as} \colorbox{Magenta!0.000}{\strut  information} \colorbox{Magenta!0.000}{\strut  pertaining} \colorbox{Magenta!0.000}{\strut  to} \colorbox{Magenta!0.000}{\strut  the} \colorbox{Magenta!0.000}{\strut  IA} \colorbox{Magenta!0.000}{\strut FF} \\
\midrule
Jacobian & \num{2.832e-01} & \colorbox{Cyan!0.000}{\strut  that} \colorbox{Cyan!0.000}{\strut  our} \colorbox{Cyan!0.000}{\strut  safety} \colorbox{Cyan!0.000}{\strut  program} \colorbox{Cyan!0.000}{\strut  and} \colorbox{Cyan!0.000}{\strut  the} \colorbox{Cyan!0.000}{\strut  safety} \colorbox{Cyan!0.000}{\strut  rules} \colorbox{Cyan!0.000}{\strut ,} \colorbox{Cyan!0.000}{\strut  instructions} \colorbox{Cyan!0.000}{\strut  and} \colorbox{Cyan!0.000}{\strut  procedures} \colorbox{Cyan!0.000}{\strut ,} \colorbox{Cyan!99.765}{\strut  as} \colorbox{Cyan!0.000}{\strut  well} \\
Input SAE & \num{1.735e+01} & \colorbox{Green!0.000}{\strut  that} \colorbox{Green!0.000}{\strut  our} \colorbox{Green!0.000}{\strut  safety} \colorbox{Green!0.000}{\strut  program} \colorbox{Green!0.000}{\strut  and} \colorbox{Green!0.000}{\strut  the} \colorbox{Green!0.000}{\strut  safety} \colorbox{Green!0.000}{\strut  rules} \colorbox{Green!0.000}{\strut ,} \colorbox{Green!0.000}{\strut  instructions} \colorbox{Green!0.000}{\strut  and} \colorbox{Green!0.000}{\strut  procedures} \colorbox{Green!0.000}{\strut ,} \colorbox{Green!90.887}{\strut  as} \colorbox{Green!0.000}{\strut  well} \\
Output SAE & \num{4.884e+00} & \colorbox{Magenta!0.000}{\strut  that} \colorbox{Magenta!0.000}{\strut  our} \colorbox{Magenta!0.000}{\strut  safety} \colorbox{Magenta!0.000}{\strut  program} \colorbox{Magenta!0.000}{\strut  and} \colorbox{Magenta!0.000}{\strut  the} \colorbox{Magenta!0.000}{\strut  safety} \colorbox{Magenta!0.000}{\strut  rules} \colorbox{Magenta!0.000}{\strut ,} \colorbox{Magenta!0.000}{\strut  instructions} \colorbox{Magenta!0.000}{\strut  and} \colorbox{Magenta!0.000}{\strut  procedures} \colorbox{Magenta!0.000}{\strut ,} \colorbox{Magenta!97.638}{\strut  as} \colorbox{Magenta!0.000}{\strut  well} \\
\midrule
Jacobian & \num{2.817e-01} & \colorbox{Cyan!0.000}{\strut  you} \colorbox{Cyan!0.000}{\strut  ought} \colorbox{Cyan!0.000}{\strut  to} \colorbox{Cyan!0.000}{\strut  note} \colorbox{Cyan!0.000}{\strut  skills} \colorbox{Cyan!0.000}{\strut  that} \colorbox{Cyan!0.000}{\strut  might} \colorbox{Cyan!0.000}{\strut  be} \colorbox{Cyan!0.000}{\strut  relevant} \colorbox{Cyan!99.231}{\strut  as} \colorbox{Cyan!0.000}{\strut  well} \colorbox{Cyan!0.000}{\strut  as} \colorbox{Cyan!0.000}{\strut  other} \colorbox{Cyan!0.000}{\strut  qualifications} \colorbox{Cyan!0.000}{\strut  which} \\
Input SAE & \num{1.666e+01} & \colorbox{Green!0.000}{\strut  you} \colorbox{Green!0.000}{\strut  ought} \colorbox{Green!0.000}{\strut  to} \colorbox{Green!0.000}{\strut  note} \colorbox{Green!0.000}{\strut  skills} \colorbox{Green!0.000}{\strut  that} \colorbox{Green!0.000}{\strut  might} \colorbox{Green!0.000}{\strut  be} \colorbox{Green!0.000}{\strut  relevant} \colorbox{Green!87.295}{\strut  as} \colorbox{Green!0.000}{\strut  well} \colorbox{Green!0.000}{\strut  as} \colorbox{Green!0.000}{\strut  other} \colorbox{Green!0.000}{\strut  qualifications} \colorbox{Green!0.000}{\strut  which} \\
Output SAE & \num{4.226e+00} & \colorbox{Magenta!0.000}{\strut  you} \colorbox{Magenta!0.000}{\strut  ought} \colorbox{Magenta!0.000}{\strut  to} \colorbox{Magenta!0.000}{\strut  note} \colorbox{Magenta!0.000}{\strut  skills} \colorbox{Magenta!0.000}{\strut  that} \colorbox{Magenta!0.000}{\strut  might} \colorbox{Magenta!0.000}{\strut  be} \colorbox{Magenta!0.000}{\strut  relevant} \colorbox{Magenta!84.486}{\strut  as} \colorbox{Magenta!0.000}{\strut  well} \colorbox{Magenta!0.000}{\strut  as} \colorbox{Magenta!0.000}{\strut  other} \colorbox{Magenta!0.000}{\strut  qualifications} \colorbox{Magenta!0.000}{\strut  which} \\
\midrule
Jacobian & \num{2.815e-01} & \colorbox{Cyan!0.000}{\strut  24} \colorbox{Cyan!0.000}{\strut  species} \colorbox{Cyan!0.000}{\strut  names} \colorbox{Cyan!0.000}{\strut  in} \colorbox{Cyan!0.000}{\strut  the} \colorbox{Cyan!0.000}{\strut  database} \colorbox{Cyan!0.000}{\strut  at} \colorbox{Cyan!0.000}{\strut  present} \colorbox{Cyan!0.000}{\strut ,} \colorbox{Cyan!99.162}{\strut  as} \colorbox{Cyan!0.000}{\strut  well} \colorbox{Cyan!0.000}{\strut  as} \colorbox{Cyan!0.000}{\strut  5} \colorbox{Cyan!0.000}{\strut  in} \colorbox{Cyan!0.000}{\strut fr} \\
Input SAE & \num{1.751e+01} & \colorbox{Green!0.000}{\strut  24} \colorbox{Green!0.000}{\strut  species} \colorbox{Green!0.000}{\strut  names} \colorbox{Green!0.000}{\strut  in} \colorbox{Green!0.000}{\strut  the} \colorbox{Green!0.000}{\strut  database} \colorbox{Green!0.000}{\strut  at} \colorbox{Green!0.000}{\strut  present} \colorbox{Green!0.000}{\strut ,} \colorbox{Green!91.698}{\strut  as} \colorbox{Green!0.000}{\strut  well} \colorbox{Green!0.000}{\strut  as} \colorbox{Green!0.000}{\strut  5} \colorbox{Green!0.000}{\strut  in} \colorbox{Green!0.000}{\strut fr} \\
Output SAE & \num{4.830e+00} & \colorbox{Magenta!0.000}{\strut  24} \colorbox{Magenta!0.000}{\strut  species} \colorbox{Magenta!0.000}{\strut  names} \colorbox{Magenta!0.000}{\strut  in} \colorbox{Magenta!0.000}{\strut  the} \colorbox{Magenta!0.000}{\strut  database} \colorbox{Magenta!0.000}{\strut  at} \colorbox{Magenta!0.000}{\strut  present} \colorbox{Magenta!0.000}{\strut ,} \colorbox{Magenta!96.576}{\strut  as} \colorbox{Magenta!0.000}{\strut  well} \colorbox{Magenta!0.000}{\strut  as} \colorbox{Magenta!0.000}{\strut  5} \colorbox{Magenta!0.000}{\strut  in} \colorbox{Magenta!0.000}{\strut fr} \\
\midrule
Jacobian & \num{2.811e-01} & \colorbox{Cyan!0.000}{\strut  track} \colorbox{Cyan!0.000}{\strut  of} \colorbox{Cyan!0.000}{\strut  all} \colorbox{Cyan!0.000}{\strut  expenses} \colorbox{Cyan!0.000}{\strut  related} \colorbox{Cyan!0.000}{\strut  to} \colorbox{Cyan!0.000}{\strut  travel} \colorbox{Cyan!0.000}{\strut  and} \colorbox{Cyan!0.000}{\strut  all} \colorbox{Cyan!0.000}{\strut  birth} \colorbox{Cyan!0.000}{\strut mother} \colorbox{Cyan!0.000}{\strut  expenses} \colorbox{Cyan!99.016}{\strut  as} \colorbox{Cyan!0.000}{\strut  well} \colorbox{Cyan!0.000}{\strut .} \\
Input SAE & \num{1.724e+01} & \colorbox{Green!0.000}{\strut  track} \colorbox{Green!0.000}{\strut  of} \colorbox{Green!0.000}{\strut  all} \colorbox{Green!0.000}{\strut  expenses} \colorbox{Green!0.000}{\strut  related} \colorbox{Green!0.000}{\strut  to} \colorbox{Green!0.000}{\strut  travel} \colorbox{Green!0.000}{\strut  and} \colorbox{Green!0.000}{\strut  all} \colorbox{Green!0.000}{\strut  birth} \colorbox{Green!0.000}{\strut mother} \colorbox{Green!0.000}{\strut  expenses} \colorbox{Green!90.282}{\strut  as} \colorbox{Green!0.000}{\strut  well} \colorbox{Green!0.000}{\strut .} \\
Output SAE & \num{4.823e+00} & \colorbox{Magenta!0.000}{\strut  track} \colorbox{Magenta!0.000}{\strut  of} \colorbox{Magenta!0.000}{\strut  all} \colorbox{Magenta!0.000}{\strut  expenses} \colorbox{Magenta!0.000}{\strut  related} \colorbox{Magenta!0.000}{\strut  to} \colorbox{Magenta!0.000}{\strut  travel} \colorbox{Magenta!0.000}{\strut  and} \colorbox{Magenta!0.000}{\strut  all} \colorbox{Magenta!0.000}{\strut  birth} \colorbox{Magenta!0.000}{\strut mother} \colorbox{Magenta!0.000}{\strut  expenses} \colorbox{Magenta!96.429}{\strut  as} \colorbox{Magenta!0.000}{\strut  well} \colorbox{Magenta!0.000}{\strut .} \\
\midrule
Jacobian & \num{2.810e-01} & \colorbox{Cyan!0.000}{\strut  agree} \colorbox{Cyan!0.000}{\strut  to} \colorbox{Cyan!0.000}{\strut  abide} \colorbox{Cyan!0.000}{\strut  by} \colorbox{Cyan!0.000}{\strut  all} \colorbox{Cyan!0.000}{\strut  applicable} \colorbox{Cyan!0.000}{\strut  intellectual} \colorbox{Cyan!0.000}{\strut  property} \colorbox{Cyan!0.000}{\strut  and} \colorbox{Cyan!0.000}{\strut  competition} \colorbox{Cyan!0.000}{\strut  laws} \colorbox{Cyan!0.000}{\strut ,} \colorbox{Cyan!98.976}{\strut  as} \colorbox{Cyan!0.000}{\strut  well} \colorbox{Cyan!0.000}{\strut  as} \\
Input SAE & \num{1.678e+01} & \colorbox{Green!0.000}{\strut  agree} \colorbox{Green!0.000}{\strut  to} \colorbox{Green!0.000}{\strut  abide} \colorbox{Green!0.000}{\strut  by} \colorbox{Green!0.000}{\strut  all} \colorbox{Green!0.000}{\strut  applicable} \colorbox{Green!0.000}{\strut  intellectual} \colorbox{Green!0.000}{\strut  property} \colorbox{Green!0.000}{\strut  and} \colorbox{Green!0.000}{\strut  competition} \colorbox{Green!0.000}{\strut  laws} \colorbox{Green!0.000}{\strut ,} \colorbox{Green!87.900}{\strut  as} \colorbox{Green!0.000}{\strut  well} \colorbox{Green!0.000}{\strut  as} \\
Output SAE & \num{4.759e+00} & \colorbox{Magenta!0.000}{\strut  agree} \colorbox{Magenta!0.000}{\strut  to} \colorbox{Magenta!0.000}{\strut  abide} \colorbox{Magenta!0.000}{\strut  by} \colorbox{Magenta!0.000}{\strut  all} \colorbox{Magenta!0.000}{\strut  applicable} \colorbox{Magenta!0.000}{\strut  intellectual} \colorbox{Magenta!0.000}{\strut  property} \colorbox{Magenta!0.000}{\strut  and} \colorbox{Magenta!0.000}{\strut  competition} \colorbox{Magenta!0.000}{\strut  laws} \colorbox{Magenta!0.000}{\strut ,} \colorbox{Magenta!95.138}{\strut  as} \colorbox{Magenta!0.000}{\strut  well} \colorbox{Magenta!0.000}{\strut  as} \\
\midrule
Jacobian & \num{2.805e-01} & \colorbox{Cyan!0.000}{\strut  work} \colorbox{Cyan!0.000}{\strut  by} \colorbox{Cyan!0.000}{\strut  using} \colorbox{Cyan!0.000}{\strut  my} \colorbox{Cyan!0.000}{\strut  own} \colorbox{Cyan!0.000}{\strut  dietary} \colorbox{Cyan!0.000}{\strut  and} \colorbox{Cyan!0.000}{\strut  activity} \colorbox{Cyan!0.000}{\strut  data} \colorbox{Cyan!0.000}{\strut ,} \colorbox{Cyan!98.819}{\strut  as} \colorbox{Cyan!0.000}{\strut  well} \colorbox{Cyan!0.000}{\strut  as} \colorbox{Cyan!0.000}{\strut  publicly} \colorbox{Cyan!0.000}{\strut  available} \\
Input SAE & \num{1.702e+01} & \colorbox{Green!0.000}{\strut  work} \colorbox{Green!0.000}{\strut  by} \colorbox{Green!0.000}{\strut  using} \colorbox{Green!0.000}{\strut  my} \colorbox{Green!0.000}{\strut  own} \colorbox{Green!0.000}{\strut  dietary} \colorbox{Green!0.000}{\strut  and} \colorbox{Green!0.000}{\strut  activity} \colorbox{Green!0.000}{\strut  data} \colorbox{Green!0.000}{\strut ,} \colorbox{Green!89.163}{\strut  as} \colorbox{Green!0.000}{\strut  well} \colorbox{Green!0.000}{\strut  as} \colorbox{Green!0.000}{\strut  publicly} \colorbox{Green!0.000}{\strut  available} \\
Output SAE & \num{4.889e+00} & \colorbox{Magenta!0.000}{\strut  work} \colorbox{Magenta!0.000}{\strut  by} \colorbox{Magenta!0.000}{\strut  using} \colorbox{Magenta!0.000}{\strut  my} \colorbox{Magenta!0.000}{\strut  own} \colorbox{Magenta!0.000}{\strut  dietary} \colorbox{Magenta!0.000}{\strut  and} \colorbox{Magenta!0.000}{\strut  activity} \colorbox{Magenta!0.000}{\strut  data} \colorbox{Magenta!0.000}{\strut ,} \colorbox{Magenta!97.747}{\strut  as} \colorbox{Magenta!0.000}{\strut  well} \colorbox{Magenta!0.000}{\strut  as} \colorbox{Magenta!0.000}{\strut  publicly} \colorbox{Magenta!0.000}{\strut  available} \\
\midrule
Jacobian & \num{2.795e-01} & \colorbox{Cyan!0.000}{\strut  will} \colorbox{Cyan!0.000}{\strut  be} \colorbox{Cyan!0.000}{\strut  used} \colorbox{Cyan!0.000}{\strut  in} \colorbox{Cyan!0.000}{\strut  all} \colorbox{Cyan!0.000}{\strut  returning} \colorbox{Cyan!0.000}{\strut  WHERE} \colorbox{Cyan!0.000}{\strut  clauses} \colorbox{Cyan!0.000}{\strut ,} \colorbox{Cyan!0.000}{\strut  so} \colorbox{Cyan!0.000}{\strut  that} \colorbox{Cyan!0.000}{\strut  needs} \colorbox{Cyan!0.000}{\strut  consideration} \colorbox{Cyan!98.448}{\strut  as} \colorbox{Cyan!0.000}{\strut  well} \\
Input SAE & \num{1.598e+01} & \colorbox{Green!0.000}{\strut  will} \colorbox{Green!0.000}{\strut  be} \colorbox{Green!0.000}{\strut  used} \colorbox{Green!0.000}{\strut  in} \colorbox{Green!0.000}{\strut  all} \colorbox{Green!0.000}{\strut  returning} \colorbox{Green!0.000}{\strut  WHERE} \colorbox{Green!0.000}{\strut  clauses} \colorbox{Green!0.000}{\strut ,} \colorbox{Green!0.000}{\strut  so} \colorbox{Green!0.000}{\strut  that} \colorbox{Green!0.000}{\strut  needs} \colorbox{Green!0.000}{\strut  consideration} \colorbox{Green!83.691}{\strut  as} \colorbox{Green!0.000}{\strut  well} \\
Output SAE & \num{4.117e+00} & \colorbox{Magenta!0.000}{\strut  will} \colorbox{Magenta!0.000}{\strut  be} \colorbox{Magenta!0.000}{\strut  used} \colorbox{Magenta!0.000}{\strut  in} \colorbox{Magenta!0.000}{\strut  all} \colorbox{Magenta!0.000}{\strut  returning} \colorbox{Magenta!0.000}{\strut  WHERE} \colorbox{Magenta!0.000}{\strut  clauses} \colorbox{Magenta!0.000}{\strut ,} \colorbox{Magenta!0.000}{\strut  so} \colorbox{Magenta!0.000}{\strut  that} \colorbox{Magenta!0.000}{\strut  needs} \colorbox{Magenta!0.000}{\strut  consideration} \colorbox{Magenta!82.320}{\strut  as} \colorbox{Magenta!0.000}{\strut  well} \\
\midrule
Jacobian & \num{2.792e-01} & \colorbox{Cyan!0.000}{\strut -} \colorbox{Cyan!0.000}{\strut ling} \colorbox{Cyan!0.000}{\strut ual} \colorbox{Cyan!0.000}{\strut  table} \colorbox{Cyan!0.000}{\strut  of} \colorbox{Cyan!0.000}{\strut  the} \colorbox{Cyan!0.000}{\strut  original} \colorbox{Cyan!0.000}{\strut  subt} \colorbox{Cyan!0.000}{\strut it} \colorbox{Cyan!0.000}{\strut les} \colorbox{Cyan!0.000}{\strut  and} \colorbox{Cyan!0.000}{\strut  translation} \colorbox{Cyan!0.000}{\strut ,} \colorbox{Cyan!98.340}{\strut  as} \colorbox{Cyan!0.000}{\strut  well} \\
Input SAE & \num{1.624e+01} & \colorbox{Green!0.000}{\strut -} \colorbox{Green!0.000}{\strut ling} \colorbox{Green!0.000}{\strut ual} \colorbox{Green!0.000}{\strut  table} \colorbox{Green!0.000}{\strut  of} \colorbox{Green!0.000}{\strut  the} \colorbox{Green!0.000}{\strut  original} \colorbox{Green!0.000}{\strut  subt} \colorbox{Green!0.000}{\strut it} \colorbox{Green!0.000}{\strut les} \colorbox{Green!0.000}{\strut  and} \colorbox{Green!0.000}{\strut  translation} \colorbox{Green!0.000}{\strut ,} \colorbox{Green!85.067}{\strut  as} \colorbox{Green!0.000}{\strut  well} \\
Output SAE & \num{4.669e+00} & \colorbox{Magenta!0.000}{\strut -} \colorbox{Magenta!0.000}{\strut ling} \colorbox{Magenta!0.000}{\strut ual} \colorbox{Magenta!0.000}{\strut  table} \colorbox{Magenta!0.000}{\strut  of} \colorbox{Magenta!0.000}{\strut  the} \colorbox{Magenta!0.000}{\strut  original} \colorbox{Magenta!0.000}{\strut  subt} \colorbox{Magenta!0.000}{\strut it} \colorbox{Magenta!0.000}{\strut les} \colorbox{Magenta!0.000}{\strut  and} \colorbox{Magenta!0.000}{\strut  translation} \colorbox{Magenta!0.000}{\strut ,} \colorbox{Magenta!93.353}{\strut  as} \colorbox{Magenta!0.000}{\strut  well} \\
\midrule
Jacobian & \num{2.790e-01} & \colorbox{Cyan!0.000}{\strut  and} \colorbox{Cyan!0.000}{\strut  its} \colorbox{Cyan!0.000}{\strut  concepts} \colorbox{Cyan!0.000}{\strut ,} \colorbox{Cyan!98.272}{\strut  as} \colorbox{Cyan!0.000}{\strut  well} \colorbox{Cyan!0.000}{\strut  as} \colorbox{Cyan!0.000}{\strut  internal} \colorbox{Cyan!0.000}{\strut  mental} \colorbox{Cyan!0.000}{\strut  representation} \colorbox{Cyan!0.000}{\strut  (} \colorbox{Cyan!0.000}{\strut what} \colorbox{Cyan!0.000}{\strut  he} \colorbox{Cyan!0.000}{\strut  calls} \colorbox{Cyan!0.000}{\strut  text} \\
Input SAE & \num{1.621e+01} & \colorbox{Green!0.000}{\strut  and} \colorbox{Green!0.000}{\strut  its} \colorbox{Green!0.000}{\strut  concepts} \colorbox{Green!0.000}{\strut ,} \colorbox{Green!84.895}{\strut  as} \colorbox{Green!0.000}{\strut  well} \colorbox{Green!0.000}{\strut  as} \colorbox{Green!0.000}{\strut  internal} \colorbox{Green!0.000}{\strut  mental} \colorbox{Green!0.000}{\strut  representation} \colorbox{Green!0.000}{\strut  (} \colorbox{Green!0.000}{\strut what} \colorbox{Green!0.000}{\strut  he} \colorbox{Green!0.000}{\strut  calls} \colorbox{Green!0.000}{\strut  text} \\
Output SAE & \num{4.477e+00} & \colorbox{Magenta!0.000}{\strut  and} \colorbox{Magenta!0.000}{\strut  its} \colorbox{Magenta!0.000}{\strut  concepts} \colorbox{Magenta!0.000}{\strut ,} \colorbox{Magenta!89.502}{\strut  as} \colorbox{Magenta!0.000}{\strut  well} \colorbox{Magenta!0.000}{\strut  as} \colorbox{Magenta!0.000}{\strut  internal} \colorbox{Magenta!0.000}{\strut  mental} \colorbox{Magenta!0.000}{\strut  representation} \colorbox{Magenta!0.000}{\strut  (} \colorbox{Magenta!0.000}{\strut what} \colorbox{Magenta!0.000}{\strut  he} \colorbox{Magenta!0.000}{\strut  calls} \colorbox{Magenta!0.000}{\strut  text} \\
\midrule
Jacobian & \num{2.789e-01} & \colorbox{Cyan!0.000}{\strut  the} \colorbox{Cyan!0.000}{\strut  application} \colorbox{Cyan!0.000}{\strut  and} \colorbox{Cyan!0.000}{\strut  implementation} \colorbox{Cyan!0.000}{\strut  of} \colorbox{Cyan!0.000}{\strut  quality} \colorbox{Cyan!0.000}{\strut  risk} \colorbox{Cyan!0.000}{\strut  management} \colorbox{Cyan!0.000}{\strut  principles} \colorbox{Cyan!0.000}{\strut ,} \colorbox{Cyan!98.256}{\strut  as} \colorbox{Cyan!0.000}{\strut  supported} \colorbox{Cyan!0.000}{\strut  by} \colorbox{Cyan!0.000}{\strut  the} \colorbox{Cyan!0.000}{\strut  pharmaceutical} \\
Input SAE & \num{1.717e+01} & \colorbox{Green!0.000}{\strut  the} \colorbox{Green!0.000}{\strut  application} \colorbox{Green!0.000}{\strut  and} \colorbox{Green!0.000}{\strut  implementation} \colorbox{Green!0.000}{\strut  of} \colorbox{Green!0.000}{\strut  quality} \colorbox{Green!0.000}{\strut  risk} \colorbox{Green!0.000}{\strut  management} \colorbox{Green!0.000}{\strut  principles} \colorbox{Green!0.000}{\strut ,} \colorbox{Green!89.955}{\strut  as} \colorbox{Green!0.000}{\strut  supported} \colorbox{Green!0.000}{\strut  by} \colorbox{Green!0.000}{\strut  the} \colorbox{Green!0.000}{\strut  pharmaceutical} \\
Output SAE & \num{4.754e+00} & \colorbox{Magenta!0.000}{\strut  the} \colorbox{Magenta!0.000}{\strut  application} \colorbox{Magenta!0.000}{\strut  and} \colorbox{Magenta!0.000}{\strut  implementation} \colorbox{Magenta!0.000}{\strut  of} \colorbox{Magenta!0.000}{\strut  quality} \colorbox{Magenta!0.000}{\strut  risk} \colorbox{Magenta!0.000}{\strut  management} \colorbox{Magenta!0.000}{\strut  principles} \colorbox{Magenta!0.000}{\strut ,} \colorbox{Magenta!95.051}{\strut  as} \colorbox{Magenta!0.000}{\strut  supported} \colorbox{Magenta!0.000}{\strut  by} \colorbox{Magenta!0.000}{\strut  the} \colorbox{Magenta!0.000}{\strut  pharmaceutical} \\
\midrule
Jacobian & \num{2.789e-01} & \colorbox{Cyan!0.000}{\strut  France} \colorbox{Cyan!0.000}{\strut  read} \colorbox{Cyan!0.000}{\strut ,} \colorbox{Cyan!0.000}{\strut  when} \colorbox{Cyan!0.000}{\strut  available} \colorbox{Cyan!0.000}{\strut ,} \colorbox{Cyan!0.000}{\strut  the} \colorbox{Cyan!0.000}{\strut  documentation} \colorbox{Cyan!0.000}{\strut  relating} \colorbox{Cyan!0.000}{\strut  to} \colorbox{Cyan!0.000}{\strut  the} \colorbox{Cyan!0.000}{\strut  tender} \colorbox{Cyan!0.000}{\strut  offer} \colorbox{Cyan!0.000}{\strut ,} \colorbox{Cyan!98.239}{\strut  as} \\
Input SAE & \num{1.772e+01} & \colorbox{Green!0.000}{\strut  France} \colorbox{Green!0.000}{\strut  read} \colorbox{Green!0.000}{\strut ,} \colorbox{Green!0.000}{\strut  when} \colorbox{Green!0.000}{\strut  available} \colorbox{Green!0.000}{\strut ,} \colorbox{Green!0.000}{\strut  the} \colorbox{Green!0.000}{\strut  documentation} \colorbox{Green!0.000}{\strut  relating} \colorbox{Green!0.000}{\strut  to} \colorbox{Green!0.000}{\strut  the} \colorbox{Green!0.000}{\strut  tender} \colorbox{Green!0.000}{\strut  offer} \colorbox{Green!0.000}{\strut ,} \colorbox{Green!92.815}{\strut  as} \\
Output SAE & \num{5.002e+00} & \colorbox{Magenta!0.000}{\strut  France} \colorbox{Magenta!0.000}{\strut  read} \colorbox{Magenta!0.000}{\strut ,} \colorbox{Magenta!0.000}{\strut  when} \colorbox{Magenta!0.000}{\strut  available} \colorbox{Magenta!0.000}{\strut ,} \colorbox{Magenta!0.000}{\strut  the} \colorbox{Magenta!0.000}{\strut  documentation} \colorbox{Magenta!0.000}{\strut  relating} \colorbox{Magenta!0.000}{\strut  to} \colorbox{Magenta!0.000}{\strut  the} \colorbox{Magenta!0.000}{\strut  tender} \colorbox{Magenta!0.000}{\strut  offer} \colorbox{Magenta!0.000}{\strut ,} \colorbox{Magenta!100.000}{\strut  as} \\
\bottomrule
\end{longtable}
\caption{feature pairs/Layer15-65536-J1-LR5.0e-04-k32-T3.0e+08 abs mean/examples-34006-v-45751 stas c4-en-10k,train,batch size=32,ctx len=16.csv}
\end{table}
% \begin{table}
\centering
\begin{longtable}{lrl}
\toprule
Category & Max. abs. value & Example tokens \\
\midrule
Jacobian & \num{2.662e-01} & \colorbox{Cyan!0.000}{\strut  very} \colorbox{Cyan!0.000}{\strut  poor} \colorbox{Cyan!0.000}{\strut  prognosis} \colorbox{Cyan!0.000}{\strut  and} \colorbox{Cyan!0.000}{\strut  may} \colorbox{Cyan!0.000}{\strut  be} \colorbox{Cyan!0.000}{\strut  considered} \colorbox{Cyan!0.000}{\strut  for} \colorbox{Cyan!0.000}{\strut  advanced} \colorbox{Cyan!0.000}{\strut  options} \colorbox{Cyan!0.000}{\strut ,} \colorbox{Cyan!100.000}{\strut  such} \colorbox{Cyan!0.000}{\strut  as} \colorbox{Cyan!0.000}{\strut  mechanical} \colorbox{Cyan!0.000}{\strut  circul} \\
Input SAE & \num{1.217e+01} & \colorbox{Green!0.000}{\strut  very} \colorbox{Green!0.000}{\strut  poor} \colorbox{Green!0.000}{\strut  prognosis} \colorbox{Green!0.000}{\strut  and} \colorbox{Green!0.000}{\strut  may} \colorbox{Green!0.000}{\strut  be} \colorbox{Green!0.000}{\strut  considered} \colorbox{Green!0.000}{\strut  for} \colorbox{Green!0.000}{\strut  advanced} \colorbox{Green!0.000}{\strut  options} \colorbox{Green!0.000}{\strut ,} \colorbox{Green!71.235}{\strut  such} \colorbox{Green!0.000}{\strut  as} \colorbox{Green!0.000}{\strut  mechanical} \colorbox{Green!0.000}{\strut  circul} \\
Output SAE & \num{5.449e+00} & \colorbox{Magenta!0.000}{\strut  very} \colorbox{Magenta!0.000}{\strut  poor} \colorbox{Magenta!0.000}{\strut  prognosis} \colorbox{Magenta!0.000}{\strut  and} \colorbox{Magenta!0.000}{\strut  may} \colorbox{Magenta!0.000}{\strut  be} \colorbox{Magenta!0.000}{\strut  considered} \colorbox{Magenta!0.000}{\strut  for} \colorbox{Magenta!0.000}{\strut  advanced} \colorbox{Magenta!0.000}{\strut  options} \colorbox{Magenta!0.000}{\strut ,} \colorbox{Magenta!94.374}{\strut  such} \colorbox{Magenta!0.000}{\strut  as} \colorbox{Magenta!0.000}{\strut  mechanical} \colorbox{Magenta!0.000}{\strut  circul} \\
\midrule
Jacobian & \num{2.633e-01} & \colorbox{Cyan!0.000}{\strut al} \colorbox{Cyan!0.000}{\strut  woven} \colorbox{Cyan!0.000}{\strut  carbon} \colorbox{Cyan!0.000}{\strut  fiber} \colorbox{Cyan!0.000}{\strut .} \colorbox{Cyan!0.000}{\strut  This} \colorbox{Cyan!0.000}{\strut  tubing} \colorbox{Cyan!0.000}{\strut  is} \colorbox{Cyan!0.000}{\strut  designed} \colorbox{Cyan!0.000}{\strut  for} \colorbox{Cyan!0.000}{\strut  large} \colorbox{Cyan!0.000}{\strut  applications} \colorbox{Cyan!0.000}{\strut ,} \colorbox{Cyan!98.882}{\strut  such} \colorbox{Cyan!0.000}{\strut  as} \\
Input SAE & \num{1.205e+01} & \colorbox{Green!0.000}{\strut al} \colorbox{Green!0.000}{\strut  woven} \colorbox{Green!0.000}{\strut  carbon} \colorbox{Green!0.000}{\strut  fiber} \colorbox{Green!0.000}{\strut .} \colorbox{Green!0.000}{\strut  This} \colorbox{Green!0.000}{\strut  tubing} \colorbox{Green!0.000}{\strut  is} \colorbox{Green!0.000}{\strut  designed} \colorbox{Green!0.000}{\strut  for} \colorbox{Green!0.000}{\strut  large} \colorbox{Green!0.000}{\strut  applications} \colorbox{Green!0.000}{\strut ,} \colorbox{Green!70.516}{\strut  such} \colorbox{Green!0.000}{\strut  as} \\
Output SAE & \num{5.396e+00} & \colorbox{Magenta!0.000}{\strut al} \colorbox{Magenta!0.000}{\strut  woven} \colorbox{Magenta!0.000}{\strut  carbon} \colorbox{Magenta!0.000}{\strut  fiber} \colorbox{Magenta!0.000}{\strut .} \colorbox{Magenta!9.357}{\strut  This} \colorbox{Magenta!0.000}{\strut  tubing} \colorbox{Magenta!0.000}{\strut  is} \colorbox{Magenta!0.000}{\strut  designed} \colorbox{Magenta!0.000}{\strut  for} \colorbox{Magenta!0.000}{\strut  large} \colorbox{Magenta!0.000}{\strut  applications} \colorbox{Magenta!0.000}{\strut ,} \colorbox{Magenta!93.449}{\strut  such} \colorbox{Magenta!0.000}{\strut  as} \\
\midrule
Jacobian & \num{2.631e-01} & \colorbox{Cyan!0.000}{\strut Force} \colorbox{Cyan!0.000}{\strut  instruments} \colorbox{Cyan!0.000}{\strut  to} \colorbox{Cyan!0.000}{\strut  perform} \colorbox{Cyan!0.000}{\strut  a} \colorbox{Cyan!0.000}{\strut  wide} \colorbox{Cyan!0.000}{\strut  range} \colorbox{Cyan!0.000}{\strut  of} \colorbox{Cyan!0.000}{\strut  tests} \colorbox{Cyan!0.000}{\strut ,} \colorbox{Cyan!98.837}{\strut  such} \colorbox{Cyan!0.000}{\strut  as} \colorbox{Cyan!0.000}{\strut  tension} \colorbox{Cyan!0.000}{\strut ,} \colorbox{Cyan!0.000}{\strut  compression} \\
Input SAE & \num{1.076e+01} & \colorbox{Green!0.000}{\strut Force} \colorbox{Green!0.000}{\strut  instruments} \colorbox{Green!0.000}{\strut  to} \colorbox{Green!0.000}{\strut  perform} \colorbox{Green!0.000}{\strut  a} \colorbox{Green!0.000}{\strut  wide} \colorbox{Green!0.000}{\strut  range} \colorbox{Green!0.000}{\strut  of} \colorbox{Green!0.000}{\strut  tests} \colorbox{Green!0.000}{\strut ,} \colorbox{Green!63.002}{\strut  such} \colorbox{Green!0.000}{\strut  as} \colorbox{Green!0.000}{\strut  tension} \colorbox{Green!0.000}{\strut ,} \colorbox{Green!0.000}{\strut  compression} \\
Output SAE & \num{5.317e+00} & \colorbox{Magenta!0.000}{\strut Force} \colorbox{Magenta!0.000}{\strut  instruments} \colorbox{Magenta!0.000}{\strut  to} \colorbox{Magenta!0.000}{\strut  perform} \colorbox{Magenta!0.000}{\strut  a} \colorbox{Magenta!0.000}{\strut  wide} \colorbox{Magenta!0.000}{\strut  range} \colorbox{Magenta!0.000}{\strut  of} \colorbox{Magenta!0.000}{\strut  tests} \colorbox{Magenta!0.000}{\strut ,} \colorbox{Magenta!92.079}{\strut  such} \colorbox{Magenta!0.000}{\strut  as} \colorbox{Magenta!0.000}{\strut  tension} \colorbox{Magenta!0.000}{\strut ,} \colorbox{Magenta!0.000}{\strut  compression} \\
\midrule
Jacobian & \num{2.630e-01} & \colorbox{Cyan!0.000}{\strut .} \colorbox{Cyan!0.000}{\strut  The} \colorbox{Cyan!0.000}{\strut  kitchen} \colorbox{Cyan!0.000}{\strut  includes} \colorbox{Cyan!0.000}{\strut  basic} \colorbox{Cyan!0.000}{\strut  modern} \colorbox{Cyan!0.000}{\strut  amenities} \colorbox{Cyan!0.000}{\strut ,} \colorbox{Cyan!98.797}{\strut  such} \colorbox{Cyan!0.000}{\strut  as} \colorbox{Cyan!0.000}{\strut  a} \colorbox{Cyan!0.000}{\strut  refrigerator} \colorbox{Cyan!0.000}{\strut  (} \colorbox{Cyan!0.000}{\strut with} \colorbox{Cyan!0.000}{\strut  ice} \\
Input SAE & \num{1.259e+01} & \colorbox{Green!0.000}{\strut .} \colorbox{Green!0.000}{\strut  The} \colorbox{Green!0.000}{\strut  kitchen} \colorbox{Green!0.000}{\strut  includes} \colorbox{Green!0.000}{\strut  basic} \colorbox{Green!0.000}{\strut  modern} \colorbox{Green!0.000}{\strut  amenities} \colorbox{Green!0.000}{\strut ,} \colorbox{Green!73.702}{\strut  such} \colorbox{Green!0.000}{\strut  as} \colorbox{Green!0.000}{\strut  a} \colorbox{Green!0.000}{\strut  refrigerator} \colorbox{Green!0.000}{\strut  (} \colorbox{Green!0.000}{\strut with} \colorbox{Green!0.000}{\strut  ice} \\
Output SAE & \num{5.410e+00} & \colorbox{Magenta!0.000}{\strut .} \colorbox{Magenta!0.000}{\strut  The} \colorbox{Magenta!0.000}{\strut  kitchen} \colorbox{Magenta!0.000}{\strut  includes} \colorbox{Magenta!0.000}{\strut  basic} \colorbox{Magenta!0.000}{\strut  modern} \colorbox{Magenta!0.000}{\strut  amenities} \colorbox{Magenta!0.000}{\strut ,} \colorbox{Magenta!93.702}{\strut  such} \colorbox{Magenta!0.000}{\strut  as} \colorbox{Magenta!0.000}{\strut  a} \colorbox{Magenta!0.000}{\strut  refrigerator} \colorbox{Magenta!0.000}{\strut  (} \colorbox{Magenta!0.000}{\strut with} \colorbox{Magenta!0.000}{\strut  ice} \\
\midrule
Jacobian & \num{2.628e-01} & \colorbox{Cyan!0.000}{\strut  among} \colorbox{Cyan!0.000}{\strut  young} \colorbox{Cyan!0.000}{\strut  children} \colorbox{Cyan!0.000}{\strut .} \colorbox{Cyan!0.000}{\strut Other} \colorbox{Cyan!0.000}{\strut  conditions} \colorbox{Cyan!0.000}{\strut ,} \colorbox{Cyan!98.694}{\strut  such} \colorbox{Cyan!0.000}{\strut  as} \colorbox{Cyan!0.000}{\strut  psoriasis} \colorbox{Cyan!0.000}{\strut  and} \colorbox{Cyan!0.000}{\strut  type} \colorbox{Cyan!0.000}{\strut  2} \colorbox{Cyan!0.000}{\strut  diabetes} \\
Input SAE & \num{1.284e+01} & \colorbox{Green!0.000}{\strut  among} \colorbox{Green!0.000}{\strut  young} \colorbox{Green!0.000}{\strut  children} \colorbox{Green!0.000}{\strut .} \colorbox{Green!0.000}{\strut Other} \colorbox{Green!0.000}{\strut  conditions} \colorbox{Green!0.000}{\strut ,} \colorbox{Green!75.184}{\strut  such} \colorbox{Green!0.000}{\strut  as} \colorbox{Green!0.000}{\strut  psoriasis} \colorbox{Green!0.000}{\strut  and} \colorbox{Green!0.000}{\strut  type} \colorbox{Green!0.000}{\strut  2} \colorbox{Green!0.000}{\strut  diabetes} \\
Output SAE & \num{5.774e+00} & \colorbox{Magenta!0.000}{\strut  among} \colorbox{Magenta!0.000}{\strut  young} \colorbox{Magenta!0.000}{\strut  children} \colorbox{Magenta!0.000}{\strut .} \colorbox{Magenta!0.000}{\strut Other} \colorbox{Magenta!0.000}{\strut  conditions} \colorbox{Magenta!0.000}{\strut ,} \colorbox{Magenta!100.000}{\strut  such} \colorbox{Magenta!0.000}{\strut  as} \colorbox{Magenta!0.000}{\strut  psoriasis} \colorbox{Magenta!0.000}{\strut  and} \colorbox{Magenta!0.000}{\strut  type} \colorbox{Magenta!0.000}{\strut  2} \colorbox{Magenta!0.000}{\strut  diabetes} \\
\midrule
Jacobian & \num{2.625e-01} & \colorbox{Cyan!0.000}{\strut ,} \colorbox{Cyan!0.000}{\strut  the} \colorbox{Cyan!0.000}{\strut  customer} \colorbox{Cyan!0.000}{\strut  will} \colorbox{Cyan!0.000}{\strut  assist} \colorbox{Cyan!0.000}{\strut  at} \colorbox{Cyan!0.000}{\strut  simple} \colorbox{Cyan!0.000}{\strut  configuration} \colorbox{Cyan!0.000}{\strut  changes} \colorbox{Cyan!0.000}{\strut ,} \colorbox{Cyan!98.604}{\strut  such} \colorbox{Cyan!0.000}{\strut  as} \colorbox{Cyan!0.000}{\strut  entering} \colorbox{Cyan!0.000}{\strut  the} \colorbox{Cyan!0.000}{\strut  login} \\
Input SAE & \num{1.066e+01} & \colorbox{Green!0.000}{\strut ,} \colorbox{Green!0.000}{\strut  the} \colorbox{Green!0.000}{\strut  customer} \colorbox{Green!0.000}{\strut  will} \colorbox{Green!0.000}{\strut  assist} \colorbox{Green!0.000}{\strut  at} \colorbox{Green!0.000}{\strut  simple} \colorbox{Green!0.000}{\strut  configuration} \colorbox{Green!0.000}{\strut  changes} \colorbox{Green!0.000}{\strut ,} \colorbox{Green!62.419}{\strut  such} \colorbox{Green!0.000}{\strut  as} \colorbox{Green!0.000}{\strut  entering} \colorbox{Green!0.000}{\strut  the} \colorbox{Green!0.000}{\strut  login} \\
Output SAE & \num{5.430e+00} & \colorbox{Magenta!0.000}{\strut ,} \colorbox{Magenta!0.000}{\strut  the} \colorbox{Magenta!0.000}{\strut  customer} \colorbox{Magenta!0.000}{\strut  will} \colorbox{Magenta!0.000}{\strut  assist} \colorbox{Magenta!0.000}{\strut  at} \colorbox{Magenta!0.000}{\strut  simple} \colorbox{Magenta!0.000}{\strut  configuration} \colorbox{Magenta!0.000}{\strut  changes} \colorbox{Magenta!0.000}{\strut ,} \colorbox{Magenta!94.041}{\strut  such} \colorbox{Magenta!0.000}{\strut  as} \colorbox{Magenta!0.000}{\strut  entering} \colorbox{Magenta!0.000}{\strut  the} \colorbox{Magenta!0.000}{\strut  login} \\
\midrule
Jacobian & \num{2.623e-01} & \colorbox{Cyan!0.000}{\strut  chemical} \colorbox{Cyan!0.000}{\strut  composition} \colorbox{Cyan!0.000}{\strut  provides} \colorbox{Cyan!0.000}{\strut  great} \colorbox{Cyan!0.000}{\strut  resistance} \colorbox{Cyan!0.000}{\strut  to} \colorbox{Cyan!0.000}{\strut  many} \colorbox{Cyan!0.000}{\strut  corros} \colorbox{Cyan!0.000}{\strut ive} \colorbox{Cyan!0.000}{\strut  environments} \colorbox{Cyan!0.000}{\strut ,} \colorbox{Cyan!98.516}{\strut  such} \colorbox{Cyan!0.000}{\strut  as} \colorbox{Cyan!0.000}{\strut  p} \colorbox{Cyan!0.000}{\strut itting} \\
Input SAE & \num{1.012e+01} & \colorbox{Green!0.000}{\strut  chemical} \colorbox{Green!0.000}{\strut  composition} \colorbox{Green!0.000}{\strut  provides} \colorbox{Green!0.000}{\strut  great} \colorbox{Green!0.000}{\strut  resistance} \colorbox{Green!0.000}{\strut  to} \colorbox{Green!0.000}{\strut  many} \colorbox{Green!0.000}{\strut  corros} \colorbox{Green!0.000}{\strut ive} \colorbox{Green!0.000}{\strut  environments} \colorbox{Green!0.000}{\strut ,} \colorbox{Green!59.238}{\strut  such} \colorbox{Green!0.000}{\strut  as} \colorbox{Green!0.000}{\strut  p} \colorbox{Green!0.000}{\strut itting} \\
Output SAE & \num{5.286e+00} & \colorbox{Magenta!0.000}{\strut  chemical} \colorbox{Magenta!0.000}{\strut  composition} \colorbox{Magenta!0.000}{\strut  provides} \colorbox{Magenta!0.000}{\strut  great} \colorbox{Magenta!0.000}{\strut  resistance} \colorbox{Magenta!0.000}{\strut  to} \colorbox{Magenta!0.000}{\strut  many} \colorbox{Magenta!0.000}{\strut  corros} \colorbox{Magenta!0.000}{\strut ive} \colorbox{Magenta!0.000}{\strut  environments} \colorbox{Magenta!0.000}{\strut ,} \colorbox{Magenta!91.545}{\strut  such} \colorbox{Magenta!0.000}{\strut  as} \colorbox{Magenta!0.000}{\strut  p} \colorbox{Magenta!0.000}{\strut itting} \\
\midrule
Jacobian & \num{2.617e-01} & \colorbox{Cyan!0.000}{\strut  maintaining} \colorbox{Cyan!0.000}{\strut  a} \colorbox{Cyan!0.000}{\strut  good} \colorbox{Cyan!0.000}{\strut  credit} \colorbox{Cyan!0.000}{\strut  score} \colorbox{Cyan!0.000}{\strut .} \colorbox{Cyan!0.000}{\strut  Credit} \colorbox{Cyan!0.000}{\strut  scores} \colorbox{Cyan!0.000}{\strut  dictate} \colorbox{Cyan!0.000}{\strut  basic} \colorbox{Cyan!0.000}{\strut  life} \colorbox{Cyan!0.000}{\strut  needs} \colorbox{Cyan!98.303}{\strut  such} \colorbox{Cyan!0.000}{\strut  as} \colorbox{Cyan!0.000}{\strut  securing} \\
Input SAE & \num{1.011e+01} & \colorbox{Green!0.000}{\strut  maintaining} \colorbox{Green!0.000}{\strut  a} \colorbox{Green!0.000}{\strut  good} \colorbox{Green!0.000}{\strut  credit} \colorbox{Green!0.000}{\strut  score} \colorbox{Green!0.000}{\strut .} \colorbox{Green!0.000}{\strut  Credit} \colorbox{Green!0.000}{\strut  scores} \colorbox{Green!0.000}{\strut  dictate} \colorbox{Green!0.000}{\strut  basic} \colorbox{Green!0.000}{\strut  life} \colorbox{Green!0.000}{\strut  needs} \colorbox{Green!59.159}{\strut  such} \colorbox{Green!0.000}{\strut  as} \colorbox{Green!0.000}{\strut  securing} \\
Output SAE & \num{4.764e+00} & \colorbox{Magenta!0.000}{\strut  maintaining} \colorbox{Magenta!0.000}{\strut  a} \colorbox{Magenta!0.000}{\strut  good} \colorbox{Magenta!0.000}{\strut  credit} \colorbox{Magenta!0.000}{\strut  score} \colorbox{Magenta!0.000}{\strut .} \colorbox{Magenta!0.000}{\strut  Credit} \colorbox{Magenta!0.000}{\strut  scores} \colorbox{Magenta!0.000}{\strut  dictate} \colorbox{Magenta!0.000}{\strut  basic} \colorbox{Magenta!0.000}{\strut  life} \colorbox{Magenta!0.000}{\strut  needs} \colorbox{Magenta!82.508}{\strut  such} \colorbox{Magenta!0.000}{\strut  as} \colorbox{Magenta!0.000}{\strut  securing} \\
\midrule
Jacobian & \num{2.616e-01} & \colorbox{Cyan!0.000}{\strut  severe} \colorbox{Cyan!0.000}{\strut  stress} \colorbox{Cyan!0.000}{\strut .} \colorbox{Cyan!0.000}{\strut Se} \colorbox{Cyan!0.000}{\strut ek} \colorbox{Cyan!0.000}{\strut  medical} \colorbox{Cyan!0.000}{\strut  attention} \colorbox{Cyan!0.000}{\strut  for} \colorbox{Cyan!0.000}{\strut  signs} \colorbox{Cyan!0.000}{\strut  of} \colorbox{Cyan!0.000}{\strut  infection} \colorbox{Cyan!0.000}{\strut ,} \colorbox{Cyan!98.274}{\strut  such} \colorbox{Cyan!0.000}{\strut  as} \\
Input SAE & \num{1.024e+01} & \colorbox{Green!0.000}{\strut  severe} \colorbox{Green!0.000}{\strut  stress} \colorbox{Green!0.000}{\strut .} \colorbox{Green!0.000}{\strut Se} \colorbox{Green!0.000}{\strut ek} \colorbox{Green!0.000}{\strut  medical} \colorbox{Green!0.000}{\strut  attention} \colorbox{Green!0.000}{\strut  for} \colorbox{Green!0.000}{\strut  signs} \colorbox{Green!0.000}{\strut  of} \colorbox{Green!0.000}{\strut  infection} \colorbox{Green!0.000}{\strut ,} \colorbox{Green!59.973}{\strut  such} \colorbox{Green!0.000}{\strut  as} \\
Output SAE & \num{5.359e+00} & \colorbox{Magenta!0.000}{\strut  severe} \colorbox{Magenta!0.000}{\strut  stress} \colorbox{Magenta!0.000}{\strut .} \colorbox{Magenta!0.000}{\strut Se} \colorbox{Magenta!0.000}{\strut ek} \colorbox{Magenta!0.000}{\strut  medical} \colorbox{Magenta!0.000}{\strut  attention} \colorbox{Magenta!0.000}{\strut  for} \colorbox{Magenta!0.000}{\strut  signs} \colorbox{Magenta!0.000}{\strut  of} \colorbox{Magenta!0.000}{\strut  infection} \colorbox{Magenta!0.000}{\strut ,} \colorbox{Magenta!92.815}{\strut  such} \colorbox{Magenta!0.000}{\strut  as} \\
\midrule
Jacobian & \num{2.608e-01} & \colorbox{Cyan!0.000}{\strut .} \colorbox{Cyan!0.000}{\strut  Each} \colorbox{Cyan!0.000}{\strut  of} \colorbox{Cyan!0.000}{\strut  Anderson} \colorbox{Cyan!0.000}{\strut \textquotesingle{}} \colorbox{Cyan!0.000}{\strut s} \colorbox{Cyan!0.000}{\strut  stories} \colorbox{Cyan!0.000}{\strut  has} \colorbox{Cyan!0.000}{\strut  similar} \colorbox{Cyan!0.000}{\strut  themes} \colorbox{Cyan!0.000}{\strut ,} \colorbox{Cyan!97.944}{\strut  such} \colorbox{Cyan!0.000}{\strut  as} \colorbox{Cyan!0.000}{\strut  death} \colorbox{Cyan!0.000}{\strut ,} \\
Input SAE & \num{1.090e+01} & \colorbox{Green!0.000}{\strut .} \colorbox{Green!0.000}{\strut  Each} \colorbox{Green!0.000}{\strut  of} \colorbox{Green!0.000}{\strut  Anderson} \colorbox{Green!0.000}{\strut \textquotesingle{}} \colorbox{Green!0.000}{\strut s} \colorbox{Green!0.000}{\strut  stories} \colorbox{Green!0.000}{\strut  has} \colorbox{Green!0.000}{\strut  similar} \colorbox{Green!0.000}{\strut  themes} \colorbox{Green!0.000}{\strut ,} \colorbox{Green!63.791}{\strut  such} \colorbox{Green!0.000}{\strut  as} \colorbox{Green!0.000}{\strut  death} \colorbox{Green!0.000}{\strut ,} \\
Output SAE & \num{5.386e+00} & \colorbox{Magenta!0.000}{\strut .} \colorbox{Magenta!0.000}{\strut  Each} \colorbox{Magenta!0.000}{\strut  of} \colorbox{Magenta!0.000}{\strut  Anderson} \colorbox{Magenta!0.000}{\strut \textquotesingle{}} \colorbox{Magenta!0.000}{\strut s} \colorbox{Magenta!0.000}{\strut  stories} \colorbox{Magenta!0.000}{\strut  has} \colorbox{Magenta!0.000}{\strut  similar} \colorbox{Magenta!0.000}{\strut  themes} \colorbox{Magenta!0.000}{\strut ,} \colorbox{Magenta!93.274}{\strut  such} \colorbox{Magenta!0.000}{\strut  as} \colorbox{Magenta!0.000}{\strut  death} \colorbox{Magenta!0.000}{\strut ,} \\
\midrule
Jacobian & \num{2.603e-01} & \colorbox{Cyan!0.000}{\strut  found} \colorbox{Cyan!0.000}{\strut  that} \colorbox{Cyan!0.000}{\strut  while} \colorbox{Cyan!0.000}{\strut  some} \colorbox{Cyan!0.000}{\strut  countries} \colorbox{Cyan!0.000}{\strut ,} \colorbox{Cyan!97.771}{\strut  such} \colorbox{Cyan!0.000}{\strut  as} \colorbox{Cyan!0.000}{\strut  the} \colorbox{Cyan!0.000}{\strut  Netherlands} \colorbox{Cyan!0.000}{\strut  and} \colorbox{Cyan!0.000}{\strut  Slov} \colorbox{Cyan!0.000}{\strut akia} \colorbox{Cyan!0.000}{\strut ,} \colorbox{Cyan!0.000}{\strut  had} \\
Input SAE & \num{1.387e+01} & \colorbox{Green!0.000}{\strut  found} \colorbox{Green!0.000}{\strut  that} \colorbox{Green!0.000}{\strut  while} \colorbox{Green!0.000}{\strut  some} \colorbox{Green!0.000}{\strut  countries} \colorbox{Green!0.000}{\strut ,} \colorbox{Green!81.186}{\strut  such} \colorbox{Green!0.000}{\strut  as} \colorbox{Green!0.000}{\strut  the} \colorbox{Green!0.000}{\strut  Netherlands} \colorbox{Green!0.000}{\strut  and} \colorbox{Green!0.000}{\strut  Slov} \colorbox{Green!0.000}{\strut akia} \colorbox{Green!0.000}{\strut ,} \colorbox{Green!0.000}{\strut  had} \\
Output SAE & \num{5.570e+00} & \colorbox{Magenta!0.000}{\strut  found} \colorbox{Magenta!0.000}{\strut  that} \colorbox{Magenta!0.000}{\strut  while} \colorbox{Magenta!0.000}{\strut  some} \colorbox{Magenta!0.000}{\strut  countries} \colorbox{Magenta!0.000}{\strut ,} \colorbox{Magenta!96.466}{\strut  such} \colorbox{Magenta!0.000}{\strut  as} \colorbox{Magenta!0.000}{\strut  the} \colorbox{Magenta!0.000}{\strut  Netherlands} \colorbox{Magenta!0.000}{\strut  and} \colorbox{Magenta!0.000}{\strut  Slov} \colorbox{Magenta!0.000}{\strut akia} \colorbox{Magenta!0.000}{\strut ,} \colorbox{Magenta!0.000}{\strut  had} \\
\midrule
Jacobian & \num{2.601e-01} & \colorbox{Cyan!0.000}{\strut s} \colorbox{Cyan!0.000}{\strut  (} \colorbox{Cyan!0.000}{\strut PP} \colorbox{Cyan!0.000}{\strut Ps} \colorbox{Cyan!0.000}{\strut )} \colorbox{Cyan!0.000}{\strut  that} \colorbox{Cyan!0.000}{\strut  employ} \colorbox{Cyan!0.000}{\strut  a} \colorbox{Cyan!0.000}{\strut  variety} \colorbox{Cyan!0.000}{\strut  of} \colorbox{Cyan!0.000}{\strut  mechanisms} \colorbox{Cyan!0.000}{\strut  and} \colorbox{Cyan!0.000}{\strut  schemes} \colorbox{Cyan!0.000}{\strut ,} \colorbox{Cyan!97.685}{\strut  such} \\
Input SAE & \num{1.085e+01} & \colorbox{Green!0.000}{\strut s} \colorbox{Green!0.000}{\strut  (} \colorbox{Green!0.000}{\strut PP} \colorbox{Green!0.000}{\strut Ps} \colorbox{Green!0.000}{\strut )} \colorbox{Green!0.000}{\strut  that} \colorbox{Green!0.000}{\strut  employ} \colorbox{Green!0.000}{\strut  a} \colorbox{Green!0.000}{\strut  variety} \colorbox{Green!0.000}{\strut  of} \colorbox{Green!0.000}{\strut  mechanisms} \colorbox{Green!0.000}{\strut  and} \colorbox{Green!0.000}{\strut  schemes} \colorbox{Green!0.000}{\strut ,} \colorbox{Green!63.544}{\strut  such} \\
Output SAE & \num{5.268e+00} & \colorbox{Magenta!0.000}{\strut s} \colorbox{Magenta!0.000}{\strut  (} \colorbox{Magenta!0.000}{\strut PP} \colorbox{Magenta!0.000}{\strut Ps} \colorbox{Magenta!0.000}{\strut )} \colorbox{Magenta!0.000}{\strut  that} \colorbox{Magenta!0.000}{\strut  employ} \colorbox{Magenta!0.000}{\strut  a} \colorbox{Magenta!0.000}{\strut  variety} \colorbox{Magenta!0.000}{\strut  of} \colorbox{Magenta!0.000}{\strut  mechanisms} \colorbox{Magenta!0.000}{\strut  and} \colorbox{Magenta!0.000}{\strut  schemes} \colorbox{Magenta!0.000}{\strut ,} \colorbox{Magenta!91.235}{\strut  such} \\
\bottomrule
\end{longtable}
\caption{feature pairs/Layer15-65536-J1-LR5.0e-04-k32-T3.0e+08 abs mean/examples-10654-v-39734 stas c4-en-10k,train,batch size=32,ctx len=16.csv}
\end{table}
% \begin{table}
\centering
\begin{longtable}{lrl}
\toprule
Category & Max. abs. value & Example tokens \\
\midrule
Jacobian & \num{2.757e-01} & \colorbox{Cyan!0.000}{\strut  is} \colorbox{Cyan!0.000}{\strut  that} \colorbox{Cyan!0.000}{\strut  a} \colorbox{Cyan!0.000}{\strut  debate} \colorbox{Cyan!0.000}{\strut  has} \colorbox{Cyan!100.000}{\strut  taken} \colorbox{Cyan!0.000}{\strut  place} \colorbox{Cyan!0.000}{\strut  about} \colorbox{Cyan!0.000}{\strut  the} \colorbox{Cyan!0.000}{\strut  potential} \colorbox{Cyan!0.000}{\strut  infringement} \colorbox{Cyan!0.000}{\strut  of} \colorbox{Cyan!0.000}{\strut  Art} \colorbox{Cyan!0.000}{\strut .} \colorbox{Cyan!0.000}{\strut  14} \\
Input SAE & \num{1.387e+01} & \colorbox{Green!0.000}{\strut  is} \colorbox{Green!0.000}{\strut  that} \colorbox{Green!0.000}{\strut  a} \colorbox{Green!0.000}{\strut  debate} \colorbox{Green!0.000}{\strut  has} \colorbox{Green!67.270}{\strut  taken} \colorbox{Green!0.000}{\strut  place} \colorbox{Green!0.000}{\strut  about} \colorbox{Green!0.000}{\strut  the} \colorbox{Green!0.000}{\strut  potential} \colorbox{Green!0.000}{\strut  infringement} \colorbox{Green!0.000}{\strut  of} \colorbox{Green!0.000}{\strut  Art} \colorbox{Green!0.000}{\strut .} \colorbox{Green!0.000}{\strut  14} \\
Output SAE & \num{4.067e+00} & \colorbox{Magenta!0.000}{\strut  is} \colorbox{Magenta!0.000}{\strut  that} \colorbox{Magenta!0.000}{\strut  a} \colorbox{Magenta!0.000}{\strut  debate} \colorbox{Magenta!0.000}{\strut  has} \colorbox{Magenta!79.726}{\strut  taken} \colorbox{Magenta!0.000}{\strut  place} \colorbox{Magenta!0.000}{\strut  about} \colorbox{Magenta!0.000}{\strut  the} \colorbox{Magenta!0.000}{\strut  potential} \colorbox{Magenta!0.000}{\strut  infringement} \colorbox{Magenta!0.000}{\strut  of} \colorbox{Magenta!0.000}{\strut  Art} \colorbox{Magenta!0.000}{\strut .} \colorbox{Magenta!0.000}{\strut  14} \\
\midrule
Jacobian & \num{2.721e-01} & \colorbox{Cyan!0.000}{\strut  community} \colorbox{Cyan!0.000}{\strut .} \colorbox{Cyan!0.000}{\strut  But} \colorbox{Cyan!0.000}{\strut  where} \colorbox{Cyan!0.000}{\strut  is} \colorbox{Cyan!0.000}{\strut  the} \colorbox{Cyan!0.000}{\strut  community} \colorbox{Cyan!0.000}{\strut  in} \colorbox{Cyan!0.000}{\strut  An} \colorbox{Cyan!0.000}{\strut ba} \colorbox{Cyan!0.000}{\strut  Lab} \colorbox{Cyan!0.000}{\strut omb} \colorbox{Cyan!0.000}{\strut a} \colorbox{Cyan!0.000}{\strut  who} \colorbox{Cyan!98.685}{\strut  takes} \\
Input SAE & \num{1.909e+01} & \colorbox{Green!0.000}{\strut  community} \colorbox{Green!0.000}{\strut .} \colorbox{Green!0.000}{\strut  But} \colorbox{Green!0.000}{\strut  where} \colorbox{Green!0.000}{\strut  is} \colorbox{Green!0.000}{\strut  the} \colorbox{Green!0.000}{\strut  community} \colorbox{Green!0.000}{\strut  in} \colorbox{Green!0.000}{\strut  An} \colorbox{Green!0.000}{\strut ba} \colorbox{Green!0.000}{\strut  Lab} \colorbox{Green!0.000}{\strut omb} \colorbox{Green!0.000}{\strut a} \colorbox{Green!0.000}{\strut  who} \colorbox{Green!92.541}{\strut  takes} \\
Output SAE & \num{4.900e+00} & \colorbox{Magenta!0.000}{\strut  community} \colorbox{Magenta!0.000}{\strut .} \colorbox{Magenta!0.000}{\strut  But} \colorbox{Magenta!0.000}{\strut  where} \colorbox{Magenta!0.000}{\strut  is} \colorbox{Magenta!0.000}{\strut  the} \colorbox{Magenta!0.000}{\strut  community} \colorbox{Magenta!0.000}{\strut  in} \colorbox{Magenta!0.000}{\strut  An} \colorbox{Magenta!0.000}{\strut ba} \colorbox{Magenta!0.000}{\strut  Lab} \colorbox{Magenta!0.000}{\strut omb} \colorbox{Magenta!0.000}{\strut a} \colorbox{Magenta!0.000}{\strut  who} \colorbox{Magenta!96.050}{\strut  takes} \\
\midrule
Jacobian & \num{2.712e-01} & \colorbox{Cyan!0.000}{\strut  musicians} \colorbox{Cyan!0.000}{\strut hip} \colorbox{Cyan!0.000}{\strut  of} \colorbox{Cyan!0.000}{\strut  X} \colorbox{Cyan!0.000}{\strut oc} \colorbox{Cyan!0.000}{\strut  is} \colorbox{Cyan!0.000}{\strut  on} \colorbox{Cyan!0.000}{\strut  full} \colorbox{Cyan!0.000}{\strut  display} \colorbox{Cyan!0.000}{\strut  as} \colorbox{Cyan!0.000}{\strut  he} \colorbox{Cyan!98.383}{\strut  takes} \colorbox{Cyan!0.000}{\strut  on} \colorbox{Cyan!0.000}{\strut  another} \colorbox{Cyan!0.000}{\strut  entire} \\
Input SAE & \num{1.831e+01} & \colorbox{Green!0.000}{\strut  musicians} \colorbox{Green!0.000}{\strut hip} \colorbox{Green!0.000}{\strut  of} \colorbox{Green!0.000}{\strut  X} \colorbox{Green!0.000}{\strut oc} \colorbox{Green!0.000}{\strut  is} \colorbox{Green!0.000}{\strut  on} \colorbox{Green!0.000}{\strut  full} \colorbox{Green!0.000}{\strut  display} \colorbox{Green!0.000}{\strut  as} \colorbox{Green!0.000}{\strut  he} \colorbox{Green!88.784}{\strut  takes} \colorbox{Green!0.000}{\strut  on} \colorbox{Green!0.000}{\strut  another} \colorbox{Green!0.000}{\strut  entire} \\
Output SAE & \num{4.575e+00} & \colorbox{Magenta!0.000}{\strut  musicians} \colorbox{Magenta!0.000}{\strut hip} \colorbox{Magenta!0.000}{\strut  of} \colorbox{Magenta!0.000}{\strut  X} \colorbox{Magenta!0.000}{\strut oc} \colorbox{Magenta!0.000}{\strut  is} \colorbox{Magenta!0.000}{\strut  on} \colorbox{Magenta!0.000}{\strut  full} \colorbox{Magenta!0.000}{\strut  display} \colorbox{Magenta!0.000}{\strut  as} \colorbox{Magenta!0.000}{\strut  he} \colorbox{Magenta!89.675}{\strut  takes} \colorbox{Magenta!11.959}{\strut  on} \colorbox{Magenta!0.000}{\strut  another} \colorbox{Magenta!0.000}{\strut  entire} \\
\midrule
Jacobian & \num{2.711e-01} & \colorbox{Cyan!0.000}{\strut  B} \colorbox{Cyan!0.000}{\strut ODY} \colorbox{Cyan!0.000}{\strut ARM} \colorbox{Cyan!0.000}{\strut OR} \colorbox{Cyan!0.000}{\strut  made} \colorbox{Cyan!0.000}{\strut  sure} \colorbox{Cyan!0.000}{\strut  to} \colorbox{Cyan!98.334}{\strut  take} \colorbox{Cyan!0.000}{\strut  advantage} \colorbox{Cyan!0.000}{\strut  not} \colorbox{Cyan!0.000}{\strut  just} \colorbox{Cyan!0.000}{\strut  on} \colorbox{Cyan!0.000}{\strut  TV} \colorbox{Cyan!0.000}{\strut  broadcasts} \colorbox{Cyan!0.000}{\strut ,} \\
Input SAE & \num{1.886e+01} & \colorbox{Green!0.000}{\strut  B} \colorbox{Green!0.000}{\strut ODY} \colorbox{Green!0.000}{\strut ARM} \colorbox{Green!0.000}{\strut OR} \colorbox{Green!0.000}{\strut  made} \colorbox{Green!0.000}{\strut  sure} \colorbox{Green!0.000}{\strut  to} \colorbox{Green!91.444}{\strut  take} \colorbox{Green!0.000}{\strut  advantage} \colorbox{Green!0.000}{\strut  not} \colorbox{Green!0.000}{\strut  just} \colorbox{Green!0.000}{\strut  on} \colorbox{Green!0.000}{\strut  TV} \colorbox{Green!0.000}{\strut  broadcasts} \colorbox{Green!0.000}{\strut ,} \\
Output SAE & \num{5.102e+00} & \colorbox{Magenta!0.000}{\strut  B} \colorbox{Magenta!0.000}{\strut ODY} \colorbox{Magenta!0.000}{\strut ARM} \colorbox{Magenta!0.000}{\strut OR} \colorbox{Magenta!0.000}{\strut  made} \colorbox{Magenta!0.000}{\strut  sure} \colorbox{Magenta!0.000}{\strut  to} \colorbox{Magenta!100.000}{\strut  take} \colorbox{Magenta!0.000}{\strut  advantage} \colorbox{Magenta!0.000}{\strut  not} \colorbox{Magenta!0.000}{\strut  just} \colorbox{Magenta!0.000}{\strut  on} \colorbox{Magenta!0.000}{\strut  TV} \colorbox{Magenta!0.000}{\strut  broadcasts} \colorbox{Magenta!0.000}{\strut ,} \\
\midrule
Jacobian & \num{2.709e-01} & \colorbox{Cyan!0.000}{\strut \textquotesingle{}} \colorbox{Cyan!0.000}{\strut t} \colorbox{Cyan!0.000}{\strut  topped} \colorbox{Cyan!0.000}{\strut  with} \colorbox{Cyan!0.000}{\strut  a} \colorbox{Cyan!0.000}{\strut  chim} \colorbox{Cyan!0.000}{\strut ney} \colorbox{Cyan!0.000}{\strut  cap} \colorbox{Cyan!0.000}{\strut ,} \colorbox{Cyan!0.000}{\strut  animals} \colorbox{Cyan!0.000}{\strut  of} \colorbox{Cyan!0.000}{\strut  various} \colorbox{Cyan!0.000}{\strut  kinds} \colorbox{Cyan!0.000}{\strut  will} \colorbox{Cyan!98.276}{\strut  take} \\
Input SAE & \num{1.776e+01} & \colorbox{Green!0.000}{\strut \textquotesingle{}} \colorbox{Green!0.000}{\strut t} \colorbox{Green!0.000}{\strut  topped} \colorbox{Green!0.000}{\strut  with} \colorbox{Green!0.000}{\strut  a} \colorbox{Green!0.000}{\strut  chim} \colorbox{Green!0.000}{\strut ney} \colorbox{Green!0.000}{\strut  cap} \colorbox{Green!0.000}{\strut ,} \colorbox{Green!0.000}{\strut  animals} \colorbox{Green!0.000}{\strut  of} \colorbox{Green!0.000}{\strut  various} \colorbox{Green!0.000}{\strut  kinds} \colorbox{Green!0.000}{\strut  will} \colorbox{Green!86.119}{\strut  take} \\
Output SAE & \num{4.749e+00} & \colorbox{Magenta!0.000}{\strut \textquotesingle{}} \colorbox{Magenta!0.000}{\strut t} \colorbox{Magenta!0.000}{\strut  topped} \colorbox{Magenta!0.000}{\strut  with} \colorbox{Magenta!0.000}{\strut  a} \colorbox{Magenta!0.000}{\strut  chim} \colorbox{Magenta!0.000}{\strut ney} \colorbox{Magenta!0.000}{\strut  cap} \colorbox{Magenta!0.000}{\strut ,} \colorbox{Magenta!0.000}{\strut  animals} \colorbox{Magenta!0.000}{\strut  of} \colorbox{Magenta!0.000}{\strut  various} \colorbox{Magenta!0.000}{\strut  kinds} \colorbox{Magenta!0.000}{\strut  will} \colorbox{Magenta!93.085}{\strut  take} \\
\midrule
Jacobian & \num{2.708e-01} & \colorbox{Cyan!0.000}{\strut  to} \colorbox{Cyan!0.000}{\strut  these} \colorbox{Cyan!0.000}{\strut  incredible} \colorbox{Cyan!0.000}{\strut  women} \colorbox{Cyan!0.000}{\strut  who} \colorbox{Cyan!0.000}{\strut  have} \colorbox{Cyan!98.238}{\strut  taken} \colorbox{Cyan!0.000}{\strut  the} \colorbox{Cyan!0.000}{\strut  brave} \colorbox{Cyan!0.000}{\strut  step} \colorbox{Cyan!0.000}{\strut  of} \colorbox{Cyan!0.000}{\strut  leaving} \colorbox{Cyan!0.000}{\strut  the} \colorbox{Cyan!0.000}{\strut  corporate} \colorbox{Cyan!0.000}{\strut  world} \\
Input SAE & \num{1.686e+01} & \colorbox{Green!0.000}{\strut  to} \colorbox{Green!0.000}{\strut  these} \colorbox{Green!0.000}{\strut  incredible} \colorbox{Green!0.000}{\strut  women} \colorbox{Green!0.000}{\strut  who} \colorbox{Green!0.000}{\strut  have} \colorbox{Green!81.729}{\strut  taken} \colorbox{Green!0.000}{\strut  the} \colorbox{Green!0.000}{\strut  brave} \colorbox{Green!0.000}{\strut  step} \colorbox{Green!0.000}{\strut  of} \colorbox{Green!0.000}{\strut  leaving} \colorbox{Green!0.000}{\strut  the} \colorbox{Green!0.000}{\strut  corporate} \colorbox{Green!0.000}{\strut  world} \\
Output SAE & \num{4.258e+00} & \colorbox{Magenta!0.000}{\strut  to} \colorbox{Magenta!0.000}{\strut  these} \colorbox{Magenta!0.000}{\strut  incredible} \colorbox{Magenta!0.000}{\strut  women} \colorbox{Magenta!0.000}{\strut  who} \colorbox{Magenta!0.000}{\strut  have} \colorbox{Magenta!83.462}{\strut  taken} \colorbox{Magenta!34.257}{\strut  the} \colorbox{Magenta!17.576}{\strut  brave} \colorbox{Magenta!11.497}{\strut  step} \colorbox{Magenta!0.000}{\strut  of} \colorbox{Magenta!0.000}{\strut  leaving} \colorbox{Magenta!0.000}{\strut  the} \colorbox{Magenta!0.000}{\strut  corporate} \colorbox{Magenta!0.000}{\strut  world} \\
\midrule
Jacobian & \num{2.708e-01} & \colorbox{Cyan!0.000}{\strut  questions} \colorbox{Cyan!0.000}{\strut  and} \colorbox{Cyan!0.000}{\strut  to} \colorbox{Cyan!98.210}{\strut  take} \colorbox{Cyan!70.410}{\strut  the} \colorbox{Cyan!0.000}{\strut  confusion} \colorbox{Cyan!0.000}{\strut  out} \colorbox{Cyan!0.000}{\strut  of} \colorbox{Cyan!0.000}{\strut  your} \colorbox{Cyan!0.000}{\strut  transaction} \colorbox{Cyan!0.000}{\strut .} \colorbox{Cyan!0.000}{\strut An} \colorbox{Cyan!0.000}{\strut  oil} \colorbox{Cyan!0.000}{\strut  and} \\
Input SAE & \num{1.884e+01} & \colorbox{Green!0.000}{\strut  questions} \colorbox{Green!0.000}{\strut  and} \colorbox{Green!0.000}{\strut  to} \colorbox{Green!91.365}{\strut  take} \colorbox{Green!7.127}{\strut  the} \colorbox{Green!0.000}{\strut  confusion} \colorbox{Green!0.000}{\strut  out} \colorbox{Green!0.000}{\strut  of} \colorbox{Green!0.000}{\strut  your} \colorbox{Green!0.000}{\strut  transaction} \colorbox{Green!0.000}{\strut .} \colorbox{Green!0.000}{\strut An} \colorbox{Green!0.000}{\strut  oil} \colorbox{Green!0.000}{\strut  and} \\
Output SAE & \num{4.810e+00} & \colorbox{Magenta!0.000}{\strut  questions} \colorbox{Magenta!0.000}{\strut  and} \colorbox{Magenta!0.000}{\strut  to} \colorbox{Magenta!94.283}{\strut  take} \colorbox{Magenta!34.101}{\strut  the} \colorbox{Magenta!0.000}{\strut  confusion} \colorbox{Magenta!0.000}{\strut  out} \colorbox{Magenta!0.000}{\strut  of} \colorbox{Magenta!0.000}{\strut  your} \colorbox{Magenta!0.000}{\strut  transaction} \colorbox{Magenta!0.000}{\strut .} \colorbox{Magenta!0.000}{\strut An} \colorbox{Magenta!0.000}{\strut  oil} \colorbox{Magenta!0.000}{\strut  and} \\
\midrule
Jacobian & \num{2.707e-01} & \colorbox{Cyan!0.000}{\strut  medical} \colorbox{Cyan!0.000}{\strut  assistants} \colorbox{Cyan!0.000}{\strut  and} \colorbox{Cyan!0.000}{\strut  office} \colorbox{Cyan!0.000}{\strut  personnel} \colorbox{Cyan!0.000}{\strut  may} \colorbox{Cyan!0.000}{\strut  actually} \colorbox{Cyan!98.204}{\strut  take} \colorbox{Cyan!73.527}{\strut  better} \colorbox{Cyan!0.000}{\strut  photos} \colorbox{Cyan!0.000}{\strut  using} \colorbox{Cyan!0.000}{\strut  their} \colorbox{Cyan!0.000}{\strut  phones} \colorbox{Cyan!0.000}{\strut .} \\
Input SAE & \num{1.744e+01} & \colorbox{Green!0.000}{\strut  medical} \colorbox{Green!0.000}{\strut  assistants} \colorbox{Green!0.000}{\strut  and} \colorbox{Green!0.000}{\strut  office} \colorbox{Green!0.000}{\strut  personnel} \colorbox{Green!0.000}{\strut  may} \colorbox{Green!0.000}{\strut  actually} \colorbox{Green!84.578}{\strut  take} \colorbox{Green!9.442}{\strut  better} \colorbox{Green!0.000}{\strut  photos} \colorbox{Green!0.000}{\strut  using} \colorbox{Green!0.000}{\strut  their} \colorbox{Green!0.000}{\strut  phones} \colorbox{Green!0.000}{\strut .} \\
Output SAE & \num{4.503e+00} & \colorbox{Magenta!0.000}{\strut  medical} \colorbox{Magenta!0.000}{\strut  assistants} \colorbox{Magenta!0.000}{\strut  and} \colorbox{Magenta!0.000}{\strut  office} \colorbox{Magenta!0.000}{\strut  personnel} \colorbox{Magenta!0.000}{\strut  may} \colorbox{Magenta!0.000}{\strut  actually} \colorbox{Magenta!88.269}{\strut  take} \colorbox{Magenta!27.698}{\strut  better} \colorbox{Magenta!0.000}{\strut  photos} \colorbox{Magenta!0.000}{\strut  using} \colorbox{Magenta!0.000}{\strut  their} \colorbox{Magenta!0.000}{\strut  phones} \colorbox{Magenta!0.000}{\strut .} \\
\midrule
Jacobian & \num{2.705e-01} & \colorbox{Cyan!0.000}{\strut  not} \colorbox{Cyan!0.000}{\strut  attempt} \colorbox{Cyan!0.000}{\strut  to} \colorbox{Cyan!0.000}{\strut  engage} \colorbox{Cyan!0.000}{\strut  in} \colorbox{Cyan!0.000}{\strut  the} \colorbox{Cyan!0.000}{\strut  process} \colorbox{Cyan!0.000}{\strut  hol} \colorbox{Cyan!0.000}{\strut istically} \colorbox{Cyan!0.000}{\strut  and} \colorbox{Cyan!98.128}{\strut  take} \colorbox{Cyan!0.000}{\strut  the} \colorbox{Cyan!0.000}{\strut  effort} \colorbox{Cyan!0.000}{\strut  to} \colorbox{Cyan!0.000}{\strut  build} \\
Input SAE & \num{1.772e+01} & \colorbox{Green!0.000}{\strut  not} \colorbox{Green!0.000}{\strut  attempt} \colorbox{Green!0.000}{\strut  to} \colorbox{Green!0.000}{\strut  engage} \colorbox{Green!0.000}{\strut  in} \colorbox{Green!0.000}{\strut  the} \colorbox{Green!0.000}{\strut  process} \colorbox{Green!0.000}{\strut  hol} \colorbox{Green!0.000}{\strut istically} \colorbox{Green!0.000}{\strut  and} \colorbox{Green!85.905}{\strut  take} \colorbox{Green!0.000}{\strut  the} \colorbox{Green!0.000}{\strut  effort} \colorbox{Green!0.000}{\strut  to} \colorbox{Green!0.000}{\strut  build} \\
Output SAE & \num{4.598e+00} & \colorbox{Magenta!0.000}{\strut  not} \colorbox{Magenta!0.000}{\strut  attempt} \colorbox{Magenta!0.000}{\strut  to} \colorbox{Magenta!0.000}{\strut  engage} \colorbox{Magenta!0.000}{\strut  in} \colorbox{Magenta!0.000}{\strut  the} \colorbox{Magenta!0.000}{\strut  process} \colorbox{Magenta!0.000}{\strut  hol} \colorbox{Magenta!0.000}{\strut istically} \colorbox{Magenta!0.000}{\strut  and} \colorbox{Magenta!90.136}{\strut  take} \colorbox{Magenta!26.374}{\strut  the} \colorbox{Magenta!0.000}{\strut  effort} \colorbox{Magenta!0.000}{\strut  to} \colorbox{Magenta!0.000}{\strut  build} \\
\midrule
Jacobian & \num{2.705e-01} & \colorbox{Cyan!0.000}{\strut  such} \colorbox{Cyan!0.000}{\strut  facilities} \colorbox{Cyan!0.000}{\strut ,} \colorbox{Cyan!0.000}{\strut  it} \colorbox{Cyan!0.000}{\strut  is} \colorbox{Cyan!0.000}{\strut  imperative} \colorbox{Cyan!0.000}{\strut  that} \colorbox{Cyan!0.000}{\strut  the} \colorbox{Cyan!0.000}{\strut  professional} \colorbox{Cyan!0.000}{\strut  staff} \colorbox{Cyan!98.127}{\strut  take} \colorbox{Cyan!0.000}{\strut  steps} \colorbox{Cyan!0.000}{\strut  to} \colorbox{Cyan!0.000}{\strut  maintain} \colorbox{Cyan!0.000}{\strut  a} \\
Input SAE & \num{1.655e+01} & \colorbox{Green!0.000}{\strut  such} \colorbox{Green!0.000}{\strut  facilities} \colorbox{Green!0.000}{\strut ,} \colorbox{Green!0.000}{\strut  it} \colorbox{Green!0.000}{\strut  is} \colorbox{Green!0.000}{\strut  imperative} \colorbox{Green!0.000}{\strut  that} \colorbox{Green!0.000}{\strut  the} \colorbox{Green!0.000}{\strut  professional} \colorbox{Green!0.000}{\strut  staff} \colorbox{Green!80.251}{\strut  take} \colorbox{Green!0.000}{\strut  steps} \colorbox{Green!0.000}{\strut  to} \colorbox{Green!0.000}{\strut  maintain} \colorbox{Green!0.000}{\strut  a} \\
Output SAE & \num{4.589e+00} & \colorbox{Magenta!0.000}{\strut  such} \colorbox{Magenta!0.000}{\strut  facilities} \colorbox{Magenta!0.000}{\strut ,} \colorbox{Magenta!0.000}{\strut  it} \colorbox{Magenta!0.000}{\strut  is} \colorbox{Magenta!0.000}{\strut  imperative} \colorbox{Magenta!0.000}{\strut  that} \colorbox{Magenta!0.000}{\strut  the} \colorbox{Magenta!0.000}{\strut  professional} \colorbox{Magenta!0.000}{\strut  staff} \colorbox{Magenta!89.943}{\strut  take} \colorbox{Magenta!0.000}{\strut  steps} \colorbox{Magenta!0.000}{\strut  to} \colorbox{Magenta!0.000}{\strut  maintain} \colorbox{Magenta!0.000}{\strut  a} \\
\midrule
Jacobian & \num{2.705e-01} & \colorbox{Cyan!0.000}{\strut s} \colorbox{Cyan!0.000}{\strut  an} \colorbox{Cyan!0.000}{\strut  art} \colorbox{Cyan!0.000}{\strut  form} \colorbox{Cyan!0.000}{\strut  that} \colorbox{Cyan!98.115}{\strut  takes} \colorbox{Cyan!0.000}{\strut ,} \colorbox{Cyan!0.000}{\strut  not} \colorbox{Cyan!0.000}{\strut  merely} \colorbox{Cyan!0.000}{\strut  gives} \colorbox{Cyan!0.000}{\strut .} \colorbox{Cyan!0.000}{\strut  It} \colorbox{Cyan!0.000}{\strut \textquotesingle{}} \colorbox{Cyan!0.000}{\strut s} \colorbox{Cyan!0.000}{\strut  an} \\
Input SAE & \num{1.490e+01} & \colorbox{Green!0.000}{\strut s} \colorbox{Green!0.000}{\strut  an} \colorbox{Green!0.000}{\strut  art} \colorbox{Green!0.000}{\strut  form} \colorbox{Green!0.000}{\strut  that} \colorbox{Green!72.229}{\strut  takes} \colorbox{Green!0.000}{\strut ,} \colorbox{Green!0.000}{\strut  not} \colorbox{Green!0.000}{\strut  merely} \colorbox{Green!0.000}{\strut  gives} \colorbox{Green!0.000}{\strut .} \colorbox{Green!0.000}{\strut  It} \colorbox{Green!0.000}{\strut \textquotesingle{}} \colorbox{Green!0.000}{\strut s} \colorbox{Green!0.000}{\strut  an} \\
Output SAE & \num{4.485e+00} & \colorbox{Magenta!0.000}{\strut s} \colorbox{Magenta!0.000}{\strut  an} \colorbox{Magenta!0.000}{\strut  art} \colorbox{Magenta!0.000}{\strut  form} \colorbox{Magenta!0.000}{\strut  that} \colorbox{Magenta!87.910}{\strut  takes} \colorbox{Magenta!0.000}{\strut ,} \colorbox{Magenta!0.000}{\strut  not} \colorbox{Magenta!0.000}{\strut  merely} \colorbox{Magenta!0.000}{\strut  gives} \colorbox{Magenta!0.000}{\strut .} \colorbox{Magenta!0.000}{\strut  It} \colorbox{Magenta!0.000}{\strut \textquotesingle{}} \colorbox{Magenta!0.000}{\strut s} \colorbox{Magenta!0.000}{\strut  an} \\
\midrule
Jacobian & \num{2.703e-01} & \colorbox{Cyan!0.000}{\strut  continue} \colorbox{Cyan!0.000}{\strut  to} \colorbox{Cyan!98.033}{\strut  take} \colorbox{Cyan!70.345}{\strut  the} \colorbox{Cyan!0.000}{\strut  initiative} \colorbox{Cyan!0.000}{\strut  to} \colorbox{Cyan!0.000}{\strut  communicate} \colorbox{Cyan!0.000}{\strut  with} \colorbox{Cyan!0.000}{\strut  content} \colorbox{Cyan!0.000}{\strut  owners} \colorbox{Cyan!0.000}{\strut  on} \colorbox{Cyan!0.000}{\strut  websites} \colorbox{Cyan!0.000}{\strut  such} \colorbox{Cyan!0.000}{\strut  as} \colorbox{Cyan!0.000}{\strut  Forbes} \\
Input SAE & \num{1.876e+01} & \colorbox{Green!0.000}{\strut  continue} \colorbox{Green!0.000}{\strut  to} \colorbox{Green!90.943}{\strut  take} \colorbox{Green!8.043}{\strut  the} \colorbox{Green!0.000}{\strut  initiative} \colorbox{Green!0.000}{\strut  to} \colorbox{Green!0.000}{\strut  communicate} \colorbox{Green!0.000}{\strut  with} \colorbox{Green!0.000}{\strut  content} \colorbox{Green!0.000}{\strut  owners} \colorbox{Green!0.000}{\strut  on} \colorbox{Green!0.000}{\strut  websites} \colorbox{Green!0.000}{\strut  such} \colorbox{Green!0.000}{\strut  as} \colorbox{Green!0.000}{\strut  Forbes} \\
Output SAE & \num{5.005e+00} & \colorbox{Magenta!0.000}{\strut  continue} \colorbox{Magenta!0.000}{\strut  to} \colorbox{Magenta!98.099}{\strut  take} \colorbox{Magenta!33.547}{\strut  the} \colorbox{Magenta!0.000}{\strut  initiative} \colorbox{Magenta!0.000}{\strut  to} \colorbox{Magenta!0.000}{\strut  communicate} \colorbox{Magenta!0.000}{\strut  with} \colorbox{Magenta!0.000}{\strut  content} \colorbox{Magenta!0.000}{\strut  owners} \colorbox{Magenta!0.000}{\strut  on} \colorbox{Magenta!0.000}{\strut  websites} \colorbox{Magenta!0.000}{\strut  such} \colorbox{Magenta!0.000}{\strut  as} \colorbox{Magenta!0.000}{\strut  Forbes} \\
\bottomrule
\end{longtable}
\caption{feature pairs/Layer15-65536-J1-LR5.0e-04-k32-T3.0e+08 abs mean/examples-50233-v-59002 stas c4-en-10k,train,batch size=32,ctx len=16.csv}
\end{table}
% \begin{table}
\centering
\begin{longtable}{lrl}
\toprule
Category & Max. abs. value & Example tokens \\
\midrule
Jacobian & \num{2.742e-01} & \colorbox{Cyan!0.000}{\strut  pumpkin} \colorbox{Cyan!0.000}{\strut  projects} \colorbox{Cyan!0.000}{\strut .} \colorbox{Cyan!0.000}{\strut  Go} \colorbox{Cyan!0.000}{\strut  figure} \colorbox{Cyan!0.000}{\strut !} \colorbox{Cyan!0.000}{\strut  I} \colorbox{Cyan!0.000}{\strut  thought} \colorbox{Cyan!0.000}{\strut  I} \colorbox{Cyan!0.000}{\strut \textquotesingle{}d} \colorbox{Cyan!0.000}{\strut  round} \colorbox{Cyan!0.000}{\strut  up} \colorbox{Cyan!0.000}{\strut  some} \colorbox{Cyan!100.000}{\strut  of} \colorbox{Cyan!0.000}{\strut  my} \\
Input SAE & \num{1.189e+01} & \colorbox{Green!0.000}{\strut  pumpkin} \colorbox{Green!0.000}{\strut  projects} \colorbox{Green!0.000}{\strut .} \colorbox{Green!0.000}{\strut  Go} \colorbox{Green!0.000}{\strut  figure} \colorbox{Green!0.000}{\strut !} \colorbox{Green!0.000}{\strut  I} \colorbox{Green!0.000}{\strut  thought} \colorbox{Green!0.000}{\strut  I} \colorbox{Green!0.000}{\strut \textquotesingle{}d} \colorbox{Green!0.000}{\strut  round} \colorbox{Green!0.000}{\strut  up} \colorbox{Green!0.000}{\strut  some} \colorbox{Green!70.231}{\strut  of} \colorbox{Green!0.000}{\strut  my} \\
Output SAE & \num{3.697e+00} & \colorbox{Magenta!0.000}{\strut  pumpkin} \colorbox{Magenta!0.000}{\strut  projects} \colorbox{Magenta!0.000}{\strut .} \colorbox{Magenta!0.000}{\strut  Go} \colorbox{Magenta!0.000}{\strut  figure} \colorbox{Magenta!0.000}{\strut !} \colorbox{Magenta!0.000}{\strut  I} \colorbox{Magenta!0.000}{\strut  thought} \colorbox{Magenta!0.000}{\strut  I} \colorbox{Magenta!0.000}{\strut \textquotesingle{}d} \colorbox{Magenta!0.000}{\strut  round} \colorbox{Magenta!0.000}{\strut  up} \colorbox{Magenta!0.000}{\strut  some} \colorbox{Magenta!73.741}{\strut  of} \colorbox{Magenta!0.000}{\strut  my} \\
\midrule
Jacobian & \num{2.732e-01} & \colorbox{Cyan!0.000}{\strut  As} \colorbox{Cyan!0.000}{\strut  the} \colorbox{Cyan!0.000}{\strut  Modi} \colorbox{Cyan!0.000}{\strut  government} \colorbox{Cyan!0.000}{\strut  defines} \colorbox{Cyan!0.000}{\strut  its} \colorbox{Cyan!0.000}{\strut  foreign} \colorbox{Cyan!0.000}{\strut  policy} \colorbox{Cyan!0.000}{\strut  priorities} \colorbox{Cyan!0.000}{\strut ,} \colorbox{Cyan!0.000}{\strut  one} \colorbox{Cyan!99.648}{\strut  of} \colorbox{Cyan!0.000}{\strut  the} \colorbox{Cyan!0.000}{\strut  issues} \colorbox{Cyan!0.000}{\strut  that} \\
Input SAE & \num{8.835e+00} & \colorbox{Green!0.000}{\strut  As} \colorbox{Green!0.000}{\strut  the} \colorbox{Green!0.000}{\strut  Modi} \colorbox{Green!0.000}{\strut  government} \colorbox{Green!0.000}{\strut  defines} \colorbox{Green!0.000}{\strut  its} \colorbox{Green!0.000}{\strut  foreign} \colorbox{Green!0.000}{\strut  policy} \colorbox{Green!0.000}{\strut  priorities} \colorbox{Green!0.000}{\strut ,} \colorbox{Green!0.000}{\strut  one} \colorbox{Green!52.178}{\strut  of} \colorbox{Green!0.000}{\strut  the} \colorbox{Green!0.000}{\strut  issues} \colorbox{Green!0.000}{\strut  that} \\
Output SAE & \num{3.733e+00} & \colorbox{Magenta!0.000}{\strut  As} \colorbox{Magenta!0.000}{\strut  the} \colorbox{Magenta!0.000}{\strut  Modi} \colorbox{Magenta!0.000}{\strut  government} \colorbox{Magenta!0.000}{\strut  defines} \colorbox{Magenta!0.000}{\strut  its} \colorbox{Magenta!0.000}{\strut  foreign} \colorbox{Magenta!0.000}{\strut  policy} \colorbox{Magenta!0.000}{\strut  priorities} \colorbox{Magenta!0.000}{\strut ,} \colorbox{Magenta!0.000}{\strut  one} \colorbox{Magenta!74.451}{\strut  of} \colorbox{Magenta!15.515}{\strut  the} \colorbox{Magenta!0.000}{\strut  issues} \colorbox{Magenta!0.000}{\strut  that} \\
\midrule
Jacobian & \num{2.732e-01} & \colorbox{Cyan!0.000}{\strut  (} \colorbox{Cyan!0.000}{\strut education} \colorbox{Cyan!0.000}{\strut ,} \colorbox{Cyan!0.000}{\strut  family} \colorbox{Cyan!0.000}{\strut  studies} \colorbox{Cyan!0.000}{\strut ,} \colorbox{Cyan!0.000}{\strut  social} \colorbox{Cyan!0.000}{\strut  work} \colorbox{Cyan!0.000}{\strut ).} \colorbox{Cyan!0.000}{\strut  Some} \colorbox{Cyan!99.632}{\strut  of} \colorbox{Cyan!0.000}{\strut  the} \colorbox{Cyan!0.000}{\strut  topics} \colorbox{Cyan!0.000}{\strut  studied} \colorbox{Cyan!0.000}{\strut  include} \\
Input SAE & \num{1.491e+01} & \colorbox{Green!0.000}{\strut  (} \colorbox{Green!0.000}{\strut education} \colorbox{Green!0.000}{\strut ,} \colorbox{Green!0.000}{\strut  family} \colorbox{Green!0.000}{\strut  studies} \colorbox{Green!0.000}{\strut ,} \colorbox{Green!0.000}{\strut  social} \colorbox{Green!0.000}{\strut  work} \colorbox{Green!0.000}{\strut ).} \colorbox{Green!0.000}{\strut  Some} \colorbox{Green!88.050}{\strut  of} \colorbox{Green!0.000}{\strut  the} \colorbox{Green!0.000}{\strut  topics} \colorbox{Green!0.000}{\strut  studied} \colorbox{Green!0.000}{\strut  include} \\
Output SAE & \num{4.679e+00} & \colorbox{Magenta!0.000}{\strut  (} \colorbox{Magenta!0.000}{\strut education} \colorbox{Magenta!0.000}{\strut ,} \colorbox{Magenta!0.000}{\strut  family} \colorbox{Magenta!0.000}{\strut  studies} \colorbox{Magenta!0.000}{\strut ,} \colorbox{Magenta!0.000}{\strut  social} \colorbox{Magenta!0.000}{\strut  work} \colorbox{Magenta!0.000}{\strut ).} \colorbox{Magenta!0.000}{\strut  Some} \colorbox{Magenta!93.341}{\strut  of} \colorbox{Magenta!12.766}{\strut  the} \colorbox{Magenta!0.000}{\strut  topics} \colorbox{Magenta!0.000}{\strut  studied} \colorbox{Magenta!0.000}{\strut  include} \\
\midrule
Jacobian & \num{2.730e-01} & \colorbox{Cyan!0.000}{\strut  geography} \colorbox{Cyan!0.000}{\strut  we} \colorbox{Cyan!0.000}{\strut  will} \colorbox{Cyan!0.000}{\strut  learn} \colorbox{Cyan!0.000}{\strut  about} \colorbox{Cyan!0.000}{\strut  and} \colorbox{Cyan!0.000}{\strut  locate} \colorbox{Cyan!0.000}{\strut  some} \colorbox{Cyan!99.572}{\strut  of} \colorbox{Cyan!0.000}{\strut  the} \colorbox{Cyan!0.000}{\strut  places} \colorbox{Cyan!0.000}{\strut  that} \colorbox{Cyan!0.000}{\strut  Jesus} \colorbox{Cyan!0.000}{\strut  visited} \colorbox{Cyan!0.000}{\strut .} \\
Input SAE & \num{1.347e+01} & \colorbox{Green!0.000}{\strut  geography} \colorbox{Green!0.000}{\strut  we} \colorbox{Green!0.000}{\strut  will} \colorbox{Green!0.000}{\strut  learn} \colorbox{Green!0.000}{\strut  about} \colorbox{Green!0.000}{\strut  and} \colorbox{Green!0.000}{\strut  locate} \colorbox{Green!0.000}{\strut  some} \colorbox{Green!79.540}{\strut  of} \colorbox{Green!0.000}{\strut  the} \colorbox{Green!0.000}{\strut  places} \colorbox{Green!0.000}{\strut  that} \colorbox{Green!0.000}{\strut  Jesus} \colorbox{Green!0.000}{\strut  visited} \colorbox{Green!0.000}{\strut .} \\
Output SAE & \num{4.415e+00} & \colorbox{Magenta!0.000}{\strut  geography} \colorbox{Magenta!0.000}{\strut  we} \colorbox{Magenta!0.000}{\strut  will} \colorbox{Magenta!0.000}{\strut  learn} \colorbox{Magenta!0.000}{\strut  about} \colorbox{Magenta!0.000}{\strut  and} \colorbox{Magenta!0.000}{\strut  locate} \colorbox{Magenta!0.000}{\strut  some} \colorbox{Magenta!88.059}{\strut  of} \colorbox{Magenta!11.910}{\strut  the} \colorbox{Magenta!0.000}{\strut  places} \colorbox{Magenta!0.000}{\strut  that} \colorbox{Magenta!0.000}{\strut  Jesus} \colorbox{Magenta!0.000}{\strut  visited} \colorbox{Magenta!0.000}{\strut .} \\
\midrule
Jacobian & \num{2.729e-01} & \colorbox{Cyan!0.000}{\strut ett} \colorbox{Cyan!0.000}{\strut  to} \colorbox{Cyan!0.000}{\strut  begin} \colorbox{Cyan!0.000}{\strut  your} \colorbox{Cyan!0.000}{\strut  search} \colorbox{Cyan!0.000}{\strut .} \colorbox{Cyan!0.000}{\strut  This} \colorbox{Cyan!0.000}{\strut  list} \colorbox{Cyan!0.000}{\strut  contains} \colorbox{Cyan!0.000}{\strut  some} \colorbox{Cyan!99.523}{\strut  of} \colorbox{Cyan!0.000}{\strut  the} \colorbox{Cyan!0.000}{\strut  largest} \colorbox{Cyan!0.000}{\strut  subdiv} \colorbox{Cyan!0.000}{\strut isions} \\
Input SAE & \num{1.252e+01} & \colorbox{Green!0.000}{\strut ett} \colorbox{Green!0.000}{\strut  to} \colorbox{Green!0.000}{\strut  begin} \colorbox{Green!0.000}{\strut  your} \colorbox{Green!0.000}{\strut  search} \colorbox{Green!0.000}{\strut .} \colorbox{Green!0.000}{\strut  This} \colorbox{Green!0.000}{\strut  list} \colorbox{Green!0.000}{\strut  contains} \colorbox{Green!0.000}{\strut  some} \colorbox{Green!73.925}{\strut  of} \colorbox{Green!0.000}{\strut  the} \colorbox{Green!0.000}{\strut  largest} \colorbox{Green!0.000}{\strut  subdiv} \colorbox{Green!0.000}{\strut isions} \\
Output SAE & \num{4.231e+00} & \colorbox{Magenta!0.000}{\strut ett} \colorbox{Magenta!0.000}{\strut  to} \colorbox{Magenta!0.000}{\strut  begin} \colorbox{Magenta!0.000}{\strut  your} \colorbox{Magenta!0.000}{\strut  search} \colorbox{Magenta!0.000}{\strut .} \colorbox{Magenta!0.000}{\strut  This} \colorbox{Magenta!11.213}{\strut  list} \colorbox{Magenta!0.000}{\strut  contains} \colorbox{Magenta!0.000}{\strut  some} \colorbox{Magenta!84.403}{\strut  of} \colorbox{Magenta!11.190}{\strut  the} \colorbox{Magenta!0.000}{\strut  largest} \colorbox{Magenta!0.000}{\strut  subdiv} \colorbox{Magenta!0.000}{\strut isions} \\
\midrule
Jacobian & \num{2.727e-01} & \colorbox{Cyan!0.000}{\strut  companies} \colorbox{Cyan!0.000}{\strut  are} \colorbox{Cyan!0.000}{\strut  launching} \colorbox{Cyan!0.000}{\strut .} \colorbox{Cyan!0.000}{\strut We} \colorbox{Cyan!0.000}{\strut  then} \colorbox{Cyan!0.000}{\strut  go} \colorbox{Cyan!0.000}{\strut  into} \colorbox{Cyan!0.000}{\strut  some} \colorbox{Cyan!99.446}{\strut  of} \colorbox{Cyan!0.000}{\strut  the} \colorbox{Cyan!0.000}{\strut  mind} \colorbox{Cyan!0.000}{\strut sets} \colorbox{Cyan!0.000}{\strut ,} \\
Input SAE & \num{1.318e+01} & \colorbox{Green!0.000}{\strut  companies} \colorbox{Green!0.000}{\strut  are} \colorbox{Green!0.000}{\strut  launching} \colorbox{Green!0.000}{\strut .} \colorbox{Green!0.000}{\strut We} \colorbox{Green!0.000}{\strut  then} \colorbox{Green!0.000}{\strut  go} \colorbox{Green!0.000}{\strut  into} \colorbox{Green!0.000}{\strut  some} \colorbox{Green!77.831}{\strut  of} \colorbox{Green!0.000}{\strut  the} \colorbox{Green!0.000}{\strut  mind} \colorbox{Green!0.000}{\strut sets} \colorbox{Green!0.000}{\strut ,} \\
Output SAE & \num{4.212e+00} & \colorbox{Magenta!0.000}{\strut  companies} \colorbox{Magenta!0.000}{\strut  are} \colorbox{Magenta!0.000}{\strut  launching} \colorbox{Magenta!0.000}{\strut .} \colorbox{Magenta!0.000}{\strut We} \colorbox{Magenta!0.000}{\strut  then} \colorbox{Magenta!0.000}{\strut  go} \colorbox{Magenta!0.000}{\strut  into} \colorbox{Magenta!0.000}{\strut  some} \colorbox{Magenta!84.014}{\strut  of} \colorbox{Magenta!13.124}{\strut  the} \colorbox{Magenta!0.000}{\strut  mind} \colorbox{Magenta!0.000}{\strut sets} \colorbox{Magenta!0.000}{\strut ,} \\
\midrule
Jacobian & \num{2.720e-01} & \colorbox{Cyan!0.000}{\strut  post} \colorbox{Cyan!0.000}{\strut ,} \colorbox{Cyan!0.000}{\strut  along} \colorbox{Cyan!0.000}{\strut  with} \colorbox{Cyan!0.000}{\strut  showing} \colorbox{Cyan!0.000}{\strut  a} \colorbox{Cyan!0.000}{\strut  1952} \colorbox{Cyan!0.000}{\strut  locom} \colorbox{Cyan!0.000}{\strut otive} \colorbox{Cyan!0.000}{\strut  assignment} \colorbox{Cyan!0.000}{\strut  sheet} \colorbox{Cyan!0.000}{\strut .} \colorbox{Cyan!0.000}{\strut  One} \colorbox{Cyan!99.214}{\strut  of} \colorbox{Cyan!0.000}{\strut  the} \\
Input SAE & \num{1.094e+01} & \colorbox{Green!0.000}{\strut  post} \colorbox{Green!0.000}{\strut ,} \colorbox{Green!0.000}{\strut  along} \colorbox{Green!0.000}{\strut  with} \colorbox{Green!0.000}{\strut  showing} \colorbox{Green!0.000}{\strut  a} \colorbox{Green!0.000}{\strut  1952} \colorbox{Green!0.000}{\strut  locom} \colorbox{Green!0.000}{\strut otive} \colorbox{Green!0.000}{\strut  assignment} \colorbox{Green!0.000}{\strut  sheet} \colorbox{Green!0.000}{\strut .} \colorbox{Green!0.000}{\strut  One} \colorbox{Green!64.600}{\strut  of} \colorbox{Green!0.000}{\strut  the} \\
Output SAE & \num{4.280e+00} & \colorbox{Magenta!0.000}{\strut  post} \colorbox{Magenta!0.000}{\strut ,} \colorbox{Magenta!0.000}{\strut  along} \colorbox{Magenta!0.000}{\strut  with} \colorbox{Magenta!0.000}{\strut  showing} \colorbox{Magenta!0.000}{\strut  a} \colorbox{Magenta!0.000}{\strut  1952} \colorbox{Magenta!0.000}{\strut  locom} \colorbox{Magenta!0.000}{\strut otive} \colorbox{Magenta!0.000}{\strut  assignment} \colorbox{Magenta!0.000}{\strut  sheet} \colorbox{Magenta!0.000}{\strut .} \colorbox{Magenta!0.000}{\strut  One} \colorbox{Magenta!85.376}{\strut  of} \colorbox{Magenta!15.539}{\strut  the} \\
\midrule
Jacobian & \num{2.718e-01} & \colorbox{Cyan!0.000}{\strut  and} \colorbox{Cyan!0.000}{\strut  training} \colorbox{Cyan!0.000}{\strut ,} \colorbox{Cyan!0.000}{\strut  and} \colorbox{Cyan!0.000}{\strut  organizational} \colorbox{Cyan!0.000}{\strut  challenges} \colorbox{Cyan!0.000}{\strut .} \colorbox{Cyan!0.000}{\strut  Some} \colorbox{Cyan!99.125}{\strut  of} \colorbox{Cyan!0.000}{\strut  the} \colorbox{Cyan!0.000}{\strut  report} \colorbox{Cyan!0.000}{\strut \textquotesingle{}s} \colorbox{Cyan!0.000}{\strut  recommendations} \colorbox{Cyan!0.000}{\strut  include} \colorbox{Cyan!0.000}{\strut  more} \\
Input SAE & \num{1.439e+01} & \colorbox{Green!0.000}{\strut  and} \colorbox{Green!0.000}{\strut  training} \colorbox{Green!0.000}{\strut ,} \colorbox{Green!0.000}{\strut  and} \colorbox{Green!0.000}{\strut  organizational} \colorbox{Green!0.000}{\strut  challenges} \colorbox{Green!0.000}{\strut .} \colorbox{Green!0.000}{\strut  Some} \colorbox{Green!84.960}{\strut  of} \colorbox{Green!0.000}{\strut  the} \colorbox{Green!0.000}{\strut  report} \colorbox{Green!0.000}{\strut \textquotesingle{}s} \colorbox{Green!0.000}{\strut  recommendations} \colorbox{Green!0.000}{\strut  include} \colorbox{Green!0.000}{\strut  more} \\
Output SAE & \num{4.555e+00} & \colorbox{Magenta!0.000}{\strut  and} \colorbox{Magenta!0.000}{\strut  training} \colorbox{Magenta!0.000}{\strut ,} \colorbox{Magenta!0.000}{\strut  and} \colorbox{Magenta!0.000}{\strut  organizational} \colorbox{Magenta!0.000}{\strut  challenges} \colorbox{Magenta!0.000}{\strut .} \colorbox{Magenta!0.000}{\strut  Some} \colorbox{Magenta!90.852}{\strut  of} \colorbox{Magenta!13.471}{\strut  the} \colorbox{Magenta!0.000}{\strut  report} \colorbox{Magenta!0.000}{\strut \textquotesingle{}s} \colorbox{Magenta!0.000}{\strut  recommendations} \colorbox{Magenta!0.000}{\strut  include} \colorbox{Magenta!0.000}{\strut  more} \\
\midrule
Jacobian & \num{2.717e-01} & \colorbox{Cyan!0.000}{\strut  world} \colorbox{Cyan!0.000}{\strut  of} \colorbox{Cyan!0.000}{\strut  tennis} \colorbox{Cyan!0.000}{\strut .} \colorbox{Cyan!0.000}{\strut The} \colorbox{Cyan!0.000}{\strut  story} \colorbox{Cyan!0.000}{\strut  of} \colorbox{Cyan!0.000}{\strut  some} \colorbox{Cyan!99.093}{\strut  of} \colorbox{Cyan!0.000}{\strut  the} \colorbox{Cyan!0.000}{\strut  greatest} \colorbox{Cyan!0.000}{\strut  tennis} \colorbox{Cyan!0.000}{\strut  players} \colorbox{Cyan!0.000}{\strut  that} \\
Input SAE & \num{1.480e+01} & \colorbox{Green!0.000}{\strut  world} \colorbox{Green!0.000}{\strut  of} \colorbox{Green!0.000}{\strut  tennis} \colorbox{Green!0.000}{\strut .} \colorbox{Green!0.000}{\strut The} \colorbox{Green!0.000}{\strut  story} \colorbox{Green!0.000}{\strut  of} \colorbox{Green!0.000}{\strut  some} \colorbox{Green!87.421}{\strut  of} \colorbox{Green!0.000}{\strut  the} \colorbox{Green!0.000}{\strut  greatest} \colorbox{Green!0.000}{\strut  tennis} \colorbox{Green!0.000}{\strut  players} \colorbox{Green!0.000}{\strut  that} \\
Output SAE & \num{4.629e+00} & \colorbox{Magenta!0.000}{\strut  world} \colorbox{Magenta!0.000}{\strut  of} \colorbox{Magenta!0.000}{\strut  tennis} \colorbox{Magenta!0.000}{\strut .} \colorbox{Magenta!0.000}{\strut The} \colorbox{Magenta!0.000}{\strut  story} \colorbox{Magenta!0.000}{\strut  of} \colorbox{Magenta!0.000}{\strut  some} \colorbox{Magenta!92.324}{\strut  of} \colorbox{Magenta!13.992}{\strut  the} \colorbox{Magenta!0.000}{\strut  greatest} \colorbox{Magenta!0.000}{\strut  tennis} \colorbox{Magenta!0.000}{\strut  players} \colorbox{Magenta!0.000}{\strut  that} \\
\midrule
Jacobian & \num{2.714e-01} & \colorbox{Cyan!0.000}{\strut  content} \colorbox{Cyan!0.000}{\strut  is} \colorbox{Cyan!0.000}{\strut  mostly} \colorbox{Cyan!0.000}{\strut  news} \colorbox{Cyan!0.000}{\strut  related} \colorbox{Cyan!0.000}{\strut ,} \colorbox{Cyan!0.000}{\strut  but} \colorbox{Cyan!0.000}{\strut  some} \colorbox{Cyan!98.975}{\strut  of} \colorbox{Cyan!0.000}{\strut  the} \colorbox{Cyan!0.000}{\strut  content} \colorbox{Cyan!0.000}{\strut  is} \colorbox{Cyan!0.000}{\strut  also} \colorbox{Cyan!0.000}{\strut  entertaining} \colorbox{Cyan!0.000}{\strut ,} \\
Input SAE & \num{1.382e+01} & \colorbox{Green!0.000}{\strut  content} \colorbox{Green!0.000}{\strut  is} \colorbox{Green!0.000}{\strut  mostly} \colorbox{Green!0.000}{\strut  news} \colorbox{Green!0.000}{\strut  related} \colorbox{Green!0.000}{\strut ,} \colorbox{Green!0.000}{\strut  but} \colorbox{Green!0.000}{\strut  some} \colorbox{Green!81.607}{\strut  of} \colorbox{Green!0.000}{\strut  the} \colorbox{Green!0.000}{\strut  content} \colorbox{Green!0.000}{\strut  is} \colorbox{Green!0.000}{\strut  also} \colorbox{Green!0.000}{\strut  entertaining} \colorbox{Green!0.000}{\strut ,} \\
Output SAE & \num{4.192e+00} & \colorbox{Magenta!0.000}{\strut  content} \colorbox{Magenta!0.000}{\strut  is} \colorbox{Magenta!0.000}{\strut  mostly} \colorbox{Magenta!0.000}{\strut  news} \colorbox{Magenta!0.000}{\strut  related} \colorbox{Magenta!0.000}{\strut ,} \colorbox{Magenta!0.000}{\strut  but} \colorbox{Magenta!0.000}{\strut  some} \colorbox{Magenta!83.609}{\strut  of} \colorbox{Magenta!12.871}{\strut  the} \colorbox{Magenta!0.000}{\strut  content} \colorbox{Magenta!0.000}{\strut  is} \colorbox{Magenta!0.000}{\strut  also} \colorbox{Magenta!0.000}{\strut  entertaining} \colorbox{Magenta!0.000}{\strut ,} \\
\midrule
Jacobian & \num{2.713e-01} & \colorbox{Cyan!0.000}{\strut  photos} \colorbox{Cyan!0.000}{\strut  up} \colorbox{Cyan!0.000}{\strut  on} \colorbox{Cyan!0.000}{\strut  the} \colorbox{Cyan!0.000}{\strut  me} \colorbox{Cyan!0.000}{\strut zz} \colorbox{Cyan!0.000}{\strut anine} \colorbox{Cyan!0.000}{\strut  style} \colorbox{Cyan!0.000}{\strut  balcony} \colorbox{Cyan!0.000}{\strut ,} \colorbox{Cyan!0.000}{\strut  along} \colorbox{Cyan!0.000}{\strut  with} \colorbox{Cyan!0.000}{\strut  some} \colorbox{Cyan!98.962}{\strut  of} \colorbox{Cyan!0.000}{\strut  the} \\
Input SAE & \num{1.347e+01} & \colorbox{Green!0.000}{\strut  photos} \colorbox{Green!0.000}{\strut  up} \colorbox{Green!0.000}{\strut  on} \colorbox{Green!0.000}{\strut  the} \colorbox{Green!0.000}{\strut  me} \colorbox{Green!0.000}{\strut zz} \colorbox{Green!0.000}{\strut anine} \colorbox{Green!0.000}{\strut  style} \colorbox{Green!0.000}{\strut  balcony} \colorbox{Green!0.000}{\strut ,} \colorbox{Green!0.000}{\strut  along} \colorbox{Green!0.000}{\strut  with} \colorbox{Green!0.000}{\strut  some} \colorbox{Green!79.524}{\strut  of} \colorbox{Green!0.000}{\strut  the} \\
Output SAE & \num{4.276e+00} & \colorbox{Magenta!0.000}{\strut  photos} \colorbox{Magenta!0.000}{\strut  up} \colorbox{Magenta!0.000}{\strut  on} \colorbox{Magenta!0.000}{\strut  the} \colorbox{Magenta!0.000}{\strut  me} \colorbox{Magenta!0.000}{\strut zz} \colorbox{Magenta!0.000}{\strut anine} \colorbox{Magenta!0.000}{\strut  style} \colorbox{Magenta!0.000}{\strut  balcony} \colorbox{Magenta!0.000}{\strut ,} \colorbox{Magenta!0.000}{\strut  along} \colorbox{Magenta!0.000}{\strut  with} \colorbox{Magenta!0.000}{\strut  some} \colorbox{Magenta!85.288}{\strut  of} \colorbox{Magenta!11.289}{\strut  the} \\
\midrule
Jacobian & \num{2.713e-01} & \colorbox{Cyan!0.000}{\strut  the} \colorbox{Cyan!0.000}{\strut  difficulties} \colorbox{Cyan!0.000}{\strut  inherent} \colorbox{Cyan!0.000}{\strut  in} \colorbox{Cyan!0.000}{\strut  job} \colorbox{Cyan!0.000}{\strut  hunting} \colorbox{Cyan!0.000}{\strut .} \colorbox{Cyan!0.000}{\strut  One} \colorbox{Cyan!98.945}{\strut  of} \colorbox{Cyan!0.000}{\strut  the} \colorbox{Cyan!0.000}{\strut  issues} \colorbox{Cyan!0.000}{\strut  they} \colorbox{Cyan!0.000}{\strut  raised} \colorbox{Cyan!0.000}{\strut  was} \colorbox{Cyan!0.000}{\strut  how} \\
Input SAE & \num{1.029e+01} & \colorbox{Green!0.000}{\strut  the} \colorbox{Green!0.000}{\strut  difficulties} \colorbox{Green!0.000}{\strut  inherent} \colorbox{Green!0.000}{\strut  in} \colorbox{Green!0.000}{\strut  job} \colorbox{Green!0.000}{\strut  hunting} \colorbox{Green!0.000}{\strut .} \colorbox{Green!0.000}{\strut  One} \colorbox{Green!60.773}{\strut  of} \colorbox{Green!0.000}{\strut  the} \colorbox{Green!0.000}{\strut  issues} \colorbox{Green!0.000}{\strut  they} \colorbox{Green!0.000}{\strut  raised} \colorbox{Green!0.000}{\strut  was} \colorbox{Green!0.000}{\strut  how} \\
Output SAE & \num{3.953e+00} & \colorbox{Magenta!0.000}{\strut  the} \colorbox{Magenta!0.000}{\strut  difficulties} \colorbox{Magenta!0.000}{\strut  inherent} \colorbox{Magenta!0.000}{\strut  in} \colorbox{Magenta!0.000}{\strut  job} \colorbox{Magenta!0.000}{\strut  hunting} \colorbox{Magenta!0.000}{\strut .} \colorbox{Magenta!0.000}{\strut  One} \colorbox{Magenta!78.858}{\strut  of} \colorbox{Magenta!15.399}{\strut  the} \colorbox{Magenta!0.000}{\strut  issues} \colorbox{Magenta!0.000}{\strut  they} \colorbox{Magenta!0.000}{\strut  raised} \colorbox{Magenta!0.000}{\strut  was} \colorbox{Magenta!0.000}{\strut  how} \\
\bottomrule
\end{longtable}
\caption{feature pairs/Layer15-65536-J1-LR5.0e-04-k32-T3.0e+08 abs mean/examples-31040-v-46216 stas c4-en-10k,train,batch size=32,ctx len=16.csv}
\end{table}
% \begin{table}
\centering
\begin{longtable}{lrl}
\toprule
Category & Max. abs. value & Example tokens \\
\midrule
Jacobian & \num{2.744e-01} & \colorbox{Cyan!0.000}{\strut ,} \colorbox{Cyan!0.000}{\strut  and} \colorbox{Cyan!0.000}{\strut  matching} \colorbox{Cyan!0.000}{\strut  socks} \colorbox{Cyan!0.000}{\strut .} \colorbox{Cyan!0.000}{\strut My} \colorbox{Cyan!0.000}{\strut  P} \colorbox{Cyan!0.000}{\strut osh} \colorbox{Cyan!0.000}{\strut  P} \colorbox{Cyan!0.000}{\strut icks} \colorbox{Cyan!0.000}{\strut e } \colorbox{Cyan!0.000}{\strut  Twin} \colorbox{Cyan!0.000}{\strut  boy} \\
Input SAE & \num{1.973e+01} & \colorbox{Green!0.000}{\strut ,} \colorbox{Green!0.000}{\strut  and} \colorbox{Green!0.000}{\strut  matching} \colorbox{Green!0.000}{\strut  socks} \colorbox{Green!0.000}{\strut .} \colorbox{Green!0.000}{\strut My} \colorbox{Green!0.000}{\strut  P} \colorbox{Green!0.000}{\strut osh} \colorbox{Green!0.000}{\strut  P} \colorbox{Green!0.000}{\strut icks} \colorbox{Green!0.000}{\strut e } \colorbox{Green!0.000}{\strut  Twin} \colorbox{Green!0.000}{\strut  boy} \\
Output SAE & \num{5.449e+00} & \colorbox{Magenta!0.000}{\strut ,} \colorbox{Magenta!0.000}{\strut  and} \colorbox{Magenta!0.000}{\strut  matching} \colorbox{Magenta!0.000}{\strut  socks} \colorbox{Magenta!0.000}{\strut .} \colorbox{Magenta!0.000}{\strut My} \colorbox{Magenta!0.000}{\strut  P} \colorbox{Magenta!0.000}{\strut osh} \colorbox{Magenta!0.000}{\strut  P} \colorbox{Magenta!0.000}{\strut icks} \colorbox{Magenta!0.000}{\strut e } \colorbox{Magenta!0.000}{\strut  Twin} \colorbox{Magenta!0.000}{\strut  boy} \\
\midrule
Jacobian & \num{2.697e-01} & \colorbox{Cyan!0.000}{\strut imon} \colorbox{Cyan!0.000}{\strut o} \colorbox{Cyan!0.000}{\strut  robe} \colorbox{Cyan!0.000}{\strut .} \colorbox{Cyan!0.000}{\strut  Never} \colorbox{Cyan!0.000}{\strut  been} \colorbox{Cyan!0.000}{\strut  wore} \colorbox{Cyan!0.000}{\strut .} \colorbox{Cyan!0.000}{\strut My} \colorbox{Cyan!0.000}{\strut  P} \colorbox{Cyan!0.000}{\strut osh} \colorbox{Cyan!0.000}{\strut  P} \colorbox{Cyan!0.000}{\strut icks} \\
Input SAE & \num{2.019e+01} & \colorbox{Green!0.000}{\strut imon} \colorbox{Green!0.000}{\strut o} \colorbox{Green!0.000}{\strut  robe} \colorbox{Green!0.000}{\strut .} \colorbox{Green!0.000}{\strut  Never} \colorbox{Green!0.000}{\strut  been} \colorbox{Green!0.000}{\strut  wore} \colorbox{Green!0.000}{\strut .} \colorbox{Green!0.000}{\strut My} \colorbox{Green!0.000}{\strut  P} \colorbox{Green!0.000}{\strut osh} \colorbox{Green!0.000}{\strut  P} \colorbox{Green!0.000}{\strut icks} \\
Output SAE & \num{5.484e+00} & \colorbox{Magenta!0.000}{\strut imon} \colorbox{Magenta!0.000}{\strut o} \colorbox{Magenta!0.000}{\strut  robe} \colorbox{Magenta!0.000}{\strut .} \colorbox{Magenta!0.000}{\strut  Never} \colorbox{Magenta!0.000}{\strut  been} \colorbox{Magenta!0.000}{\strut  wore} \colorbox{Magenta!0.000}{\strut .} \colorbox{Magenta!0.000}{\strut My} \colorbox{Magenta!0.000}{\strut  P} \colorbox{Magenta!0.000}{\strut osh} \colorbox{Magenta!0.000}{\strut  P} \colorbox{Magenta!0.000}{\strut icks} \\
\midrule
Jacobian & \num{2.689e-01} & \colorbox{Cyan!0.000}{\strut  9} \colorbox{Cyan!0.000}{\strut  mo} \colorbox{Cyan!0.000}{\strut  NEW} \colorbox{Cyan!0.000}{\strut  WITH} \colorbox{Cyan!0.000}{\strut  TAG} \colorbox{Cyan!0.000}{\strut S} \colorbox{Cyan!0.000}{\strut !} \colorbox{Cyan!0.000}{\strut My} \colorbox{Cyan!0.000}{\strut  P} \colorbox{Cyan!0.000}{\strut osh} \colorbox{Cyan!0.000}{\strut  P} \colorbox{Cyan!0.000}{\strut icks} \colorbox{Cyan!0.000}{\strut e } \\
Input SAE & \num{2.038e+01} & \colorbox{Green!0.000}{\strut  9} \colorbox{Green!0.000}{\strut  mo} \colorbox{Green!0.000}{\strut  NEW} \colorbox{Green!0.000}{\strut  WITH} \colorbox{Green!0.000}{\strut  TAG} \colorbox{Green!0.000}{\strut S} \colorbox{Green!0.000}{\strut !} \colorbox{Green!0.000}{\strut My} \colorbox{Green!0.000}{\strut  P} \colorbox{Green!0.000}{\strut osh} \colorbox{Green!0.000}{\strut  P} \colorbox{Green!0.000}{\strut icks} \colorbox{Green!0.000}{\strut e } \\
Output SAE & \num{5.484e+00} & \colorbox{Magenta!0.000}{\strut  9} \colorbox{Magenta!0.000}{\strut  mo} \colorbox{Magenta!0.000}{\strut  NEW} \colorbox{Magenta!0.000}{\strut  WITH} \colorbox{Magenta!0.000}{\strut  TAG} \colorbox{Magenta!0.000}{\strut S} \colorbox{Magenta!0.000}{\strut !} \colorbox{Magenta!0.000}{\strut My} \colorbox{Magenta!0.000}{\strut  P} \colorbox{Magenta!0.000}{\strut osh} \colorbox{Magenta!0.000}{\strut  P} \colorbox{Magenta!0.000}{\strut icks} \colorbox{Magenta!0.000}{\strut e } \\
\midrule
Jacobian & \num{2.680e-01} & \colorbox{Cyan!0.000}{\strut  the} \colorbox{Cyan!0.000}{\strut  full} \colorbox{Cyan!0.000}{\strut  story} \colorbox{Cyan!0.000}{\strut  here} \colorbox{Cyan!0.000}{\strut .} \colorbox{Cyan!0.000}{\strut Take} \colorbox{Cyan!0.000}{\strut  a} \colorbox{Cyan!0.000}{\strut  picture} \colorbox{Cyan!0.000}{\strut of} \colorbox{Cyan!0.000}{\strut  your} \colorbox{Cyan!0.000}{\strut  artwork} \colorbox{Cyan!0.000}{\strut .} \\
Input SAE & \num{2.111e+01} & \colorbox{Green!0.000}{\strut  the} \colorbox{Green!0.000}{\strut  full} \colorbox{Green!0.000}{\strut  story} \colorbox{Green!0.000}{\strut  here} \colorbox{Green!0.000}{\strut .} \colorbox{Green!0.000}{\strut Take} \colorbox{Green!0.000}{\strut  a} \colorbox{Green!0.000}{\strut  picture} \colorbox{Green!0.000}{\strut of} \colorbox{Green!0.000}{\strut  your} \colorbox{Green!0.000}{\strut  artwork} \colorbox{Green!0.000}{\strut .} \\
Output SAE & \num{5.688e+00} & \colorbox{Magenta!0.000}{\strut  the} \colorbox{Magenta!0.000}{\strut  full} \colorbox{Magenta!0.000}{\strut  story} \colorbox{Magenta!0.000}{\strut  here} \colorbox{Magenta!0.000}{\strut .} \colorbox{Magenta!0.000}{\strut Take} \colorbox{Magenta!0.000}{\strut  a} \colorbox{Magenta!0.000}{\strut  picture} \colorbox{Magenta!0.000}{\strut of} \colorbox{Magenta!0.000}{\strut  your} \colorbox{Magenta!0.000}{\strut  artwork} \colorbox{Magenta!0.000}{\strut .} \\
\midrule
Jacobian & \num{2.657e-01} & \colorbox{Cyan!0.000}{\strut .} \colorbox{Cyan!0.000}{\strut 1} \colorbox{Cyan!0.000}{\strut .} \colorbox{Cyan!0.000}{\strut  open} \colorbox{Cyan!0.000}{\strut  the} \colorbox{Cyan!0.000}{\strut  door} \colorbox{Cyan!0.000}{\strut a I} \colorbox{Cyan!0.000}{\strut  answer} \colorbox{Cyan!0.000}{\strut  the} \colorbox{Cyan!0.000}{\strut  door} \colorbox{Cyan!0.000}{\strut :} \colorbox{Cyan!0.000}{\strut aii} \colorbox{Cyan!0.000}{\strut eG} \\
Input SAE & \num{2.054e+01} & \colorbox{Green!0.000}{\strut .} \colorbox{Green!0.000}{\strut 1} \colorbox{Green!0.000}{\strut .} \colorbox{Green!0.000}{\strut  open} \colorbox{Green!0.000}{\strut  the} \colorbox{Green!0.000}{\strut  door} \colorbox{Green!0.000}{\strut a I} \colorbox{Green!0.000}{\strut  answer} \colorbox{Green!0.000}{\strut  the} \colorbox{Green!0.000}{\strut  door} \colorbox{Green!0.000}{\strut :} \colorbox{Green!0.000}{\strut aii} \colorbox{Green!0.000}{\strut eG} \\
Output SAE & \num{5.270e+00} & \colorbox{Magenta!0.000}{\strut .} \colorbox{Magenta!0.000}{\strut 1} \colorbox{Magenta!0.000}{\strut .} \colorbox{Magenta!0.000}{\strut  open} \colorbox{Magenta!0.000}{\strut  the} \colorbox{Magenta!0.000}{\strut  door} \colorbox{Magenta!0.000}{\strut a I} \colorbox{Magenta!0.000}{\strut  answer} \colorbox{Magenta!0.000}{\strut  the} \colorbox{Magenta!0.000}{\strut  door} \colorbox{Magenta!0.000}{\strut :} \colorbox{Magenta!0.000}{\strut aii} \colorbox{Magenta!0.000}{\strut eG} \\
\midrule
Jacobian & \num{2.642e-01} & \colorbox{Cyan!0.000}{\strut  15} \colorbox{Cyan!0.000}{\strut s} \colorbox{Cyan!0.000}{\strut  -} \colorbox{Cyan!0.000}{\strut  not} \colorbox{Cyan!0.000}{\strut  hanging} \colorbox{Cyan!0.000}{\strut  around} \colorbox{Cyan!0.000}{\strut .} \colorbox{Cyan!0.000}{\strut The} \colorbox{Cyan!0.000}{\strut  X} \colorbox{Cyan!0.000}{\strut R} \colorbox{Cyan!0.000}{\strut 8} \colorbox{Cyan!0.000}{\strut ute} \colorbox{Cyan!0.000}{\strut  received} \\
Input SAE & \num{1.848e+01} & \colorbox{Green!0.000}{\strut  15} \colorbox{Green!0.000}{\strut s} \colorbox{Green!0.000}{\strut  -} \colorbox{Green!0.000}{\strut  not} \colorbox{Green!0.000}{\strut  hanging} \colorbox{Green!0.000}{\strut  around} \colorbox{Green!0.000}{\strut .} \colorbox{Green!0.000}{\strut The} \colorbox{Green!0.000}{\strut  X} \colorbox{Green!0.000}{\strut R} \colorbox{Green!0.000}{\strut 8} \colorbox{Green!0.000}{\strut ute} \colorbox{Green!0.000}{\strut  received} \\
Output SAE & \num{5.137e+00} & \colorbox{Magenta!0.000}{\strut  15} \colorbox{Magenta!0.000}{\strut s} \colorbox{Magenta!0.000}{\strut  -} \colorbox{Magenta!0.000}{\strut  not} \colorbox{Magenta!0.000}{\strut  hanging} \colorbox{Magenta!0.000}{\strut  around} \colorbox{Magenta!0.000}{\strut .} \colorbox{Magenta!0.000}{\strut The} \colorbox{Magenta!0.000}{\strut  X} \colorbox{Magenta!0.000}{\strut R} \colorbox{Magenta!0.000}{\strut 8} \colorbox{Magenta!0.000}{\strut ute} \colorbox{Magenta!0.000}{\strut  received} \\
\midrule
Jacobian & \num{2.641e-01} & \colorbox{Cyan!0.000}{\strut  a} \colorbox{Cyan!0.000}{\strut  real} \colorbox{Cyan!0.000}{\strut  professional} \colorbox{Cyan!0.000}{\strut  touch} \colorbox{Cyan!0.000}{\strut .} \colorbox{Cyan!0.000}{\strut My} \colorbox{Cyan!0.000}{\strut  work} \colorbox{Cyan!0.000}{\strut  has} \colorbox{Cyan!0.000}{\strut  started} \colorbox{Cyan!0.000}{\strut  to} \colorbox{Cyan!0.000}{\strut  involve} \colorbox{Cyan!0.000}{\strut  printed} \colorbox{Cyan!0.000}{\strut icing} \\
Input SAE & \num{2.219e+01} & \colorbox{Green!0.000}{\strut  a} \colorbox{Green!0.000}{\strut  real} \colorbox{Green!0.000}{\strut  professional} \colorbox{Green!0.000}{\strut  touch} \colorbox{Green!0.000}{\strut .} \colorbox{Green!0.000}{\strut My} \colorbox{Green!0.000}{\strut  work} \colorbox{Green!0.000}{\strut  has} \colorbox{Green!0.000}{\strut  started} \colorbox{Green!0.000}{\strut  to} \colorbox{Green!0.000}{\strut  involve} \colorbox{Green!0.000}{\strut  printed} \colorbox{Green!0.000}{\strut icing} \\
Output SAE & \num{5.575e+00} & \colorbox{Magenta!0.000}{\strut  a} \colorbox{Magenta!0.000}{\strut  real} \colorbox{Magenta!0.000}{\strut  professional} \colorbox{Magenta!0.000}{\strut  touch} \colorbox{Magenta!0.000}{\strut .} \colorbox{Magenta!0.000}{\strut My} \colorbox{Magenta!0.000}{\strut  work} \colorbox{Magenta!0.000}{\strut  has} \colorbox{Magenta!0.000}{\strut  started} \colorbox{Magenta!0.000}{\strut  to} \colorbox{Magenta!0.000}{\strut  involve} \colorbox{Magenta!0.000}{\strut  printed} \colorbox{Magenta!0.000}{\strut icing} \\
\midrule
Jacobian & \num{2.632e-01} & \colorbox{Cyan!0.000}{\strut  worth} \colorbox{Cyan!0.000}{\strut  it} \colorbox{Cyan!0.000}{\strut !} \colorbox{Cyan!0.000}{\strut \#} \colorbox{Cyan!0.000}{\strut arc} \colorbox{Cyan!0.000}{\strut 1} \colorbox{Cyan!0.000}{\strut o} \colorbox{Cyan!0.000}{\strut 1} \colorbox{Cyan!0.000}{\strut ex} \colorbox{Cyan!0.000}{\strut hibition} \colorbox{Cyan!0.000}{\strut  goes} \colorbox{Cyan!0.000}{\strut  to} \colorbox{Cyan!0.000}{\strut ural} \\
Input SAE & \num{1.877e+01} & \colorbox{Green!0.000}{\strut  worth} \colorbox{Green!0.000}{\strut  it} \colorbox{Green!0.000}{\strut !} \colorbox{Green!0.000}{\strut \#} \colorbox{Green!0.000}{\strut arc} \colorbox{Green!0.000}{\strut 1} \colorbox{Green!0.000}{\strut o} \colorbox{Green!0.000}{\strut 1} \colorbox{Green!0.000}{\strut ex} \colorbox{Green!0.000}{\strut hibition} \colorbox{Green!0.000}{\strut  goes} \colorbox{Green!0.000}{\strut  to} \colorbox{Green!0.000}{\strut ural} \\
Output SAE & \num{5.185e+00} & \colorbox{Magenta!0.000}{\strut  worth} \colorbox{Magenta!0.000}{\strut  it} \colorbox{Magenta!0.000}{\strut !} \colorbox{Magenta!0.000}{\strut \#} \colorbox{Magenta!0.000}{\strut arc} \colorbox{Magenta!0.000}{\strut 1} \colorbox{Magenta!0.000}{\strut o} \colorbox{Magenta!0.000}{\strut 1} \colorbox{Magenta!0.000}{\strut ex} \colorbox{Magenta!0.000}{\strut hibition} \colorbox{Magenta!0.000}{\strut  goes} \colorbox{Magenta!0.000}{\strut  to} \colorbox{Magenta!0.000}{\strut ural} \\
\midrule
Jacobian & \num{2.631e-01} & \colorbox{Cyan!0.000}{\strut ,} \colorbox{Cyan!0.000}{\strut  boiler} \colorbox{Cyan!0.000}{\strut  wiring} \colorbox{Cyan!0.000}{\strut  diagram} \colorbox{Cyan!0.000}{\strut  for} \colorbox{Cyan!0.000}{\strut  therm} \colorbox{Cyan!0.000}{\strut ostat} \colorbox{Cyan!0.000}{\strut  to} \colorbox{Cyan!0.000}{\strut  fl} \colorbox{Cyan!0.000}{\strut air} \colorbox{Cyan!0.000}{\strut 2} \colorbox{Cyan!0.000}{\strut w} \colorbox{Cyan!0.000}{\strut 001} \colorbox{Cyan!0.000}{\strut  d} \\
Input SAE & \num{1.963e+01} & \colorbox{Green!0.000}{\strut ,} \colorbox{Green!0.000}{\strut  boiler} \colorbox{Green!0.000}{\strut  wiring} \colorbox{Green!0.000}{\strut  diagram} \colorbox{Green!0.000}{\strut  for} \colorbox{Green!0.000}{\strut  therm} \colorbox{Green!0.000}{\strut ostat} \colorbox{Green!0.000}{\strut  to} \colorbox{Green!0.000}{\strut  fl} \colorbox{Green!0.000}{\strut air} \colorbox{Green!0.000}{\strut 2} \colorbox{Green!0.000}{\strut w} \colorbox{Green!0.000}{\strut 001} \colorbox{Green!0.000}{\strut  d} \\
Output SAE & \num{5.356e+00} & \colorbox{Magenta!0.000}{\strut ,} \colorbox{Magenta!0.000}{\strut  boiler} \colorbox{Magenta!0.000}{\strut  wiring} \colorbox{Magenta!0.000}{\strut  diagram} \colorbox{Magenta!0.000}{\strut  for} \colorbox{Magenta!0.000}{\strut  therm} \colorbox{Magenta!0.000}{\strut ostat} \colorbox{Magenta!0.000}{\strut  to} \colorbox{Magenta!0.000}{\strut  fl} \colorbox{Magenta!0.000}{\strut air} \colorbox{Magenta!0.000}{\strut 2} \colorbox{Magenta!0.000}{\strut w} \colorbox{Magenta!0.000}{\strut 001} \colorbox{Magenta!0.000}{\strut  d} \\
\midrule
Jacobian & \num{2.629e-01} & \colorbox{Cyan!0.000}{\strut ,} \colorbox{Cyan!0.000}{\strut  brand} \colorbox{Cyan!0.000}{\strut y} \colorbox{Cyan!0.000}{\strut  or} \colorbox{Cyan!0.000}{\strut  li} \colorbox{Cyan!0.000}{\strut que} \colorbox{Cyan!0.000}{\strut ur} \colorbox{Cyan!0.000}{\strut .} \colorbox{Cyan!0.000}{\strut  Add} \colorbox{Cyan!0.000}{\strut  vanilla} \colorbox{Cyan!0.000}{\strut  extract} \colorbox{Cyan!0.000}{\strut .} \colorbox{Cyan!0.000}{\strut If} \\
Input SAE & \num{2.141e+01} & \colorbox{Green!0.000}{\strut ,} \colorbox{Green!0.000}{\strut  brand} \colorbox{Green!0.000}{\strut y} \colorbox{Green!0.000}{\strut  or} \colorbox{Green!0.000}{\strut  li} \colorbox{Green!0.000}{\strut que} \colorbox{Green!0.000}{\strut ur} \colorbox{Green!0.000}{\strut .} \colorbox{Green!0.000}{\strut  Add} \colorbox{Green!0.000}{\strut  vanilla} \colorbox{Green!0.000}{\strut  extract} \colorbox{Green!0.000}{\strut .} \colorbox{Green!0.000}{\strut If} \\
Output SAE & \num{5.276e+00} & \colorbox{Magenta!0.000}{\strut ,} \colorbox{Magenta!0.000}{\strut  brand} \colorbox{Magenta!0.000}{\strut y} \colorbox{Magenta!0.000}{\strut  or} \colorbox{Magenta!0.000}{\strut  li} \colorbox{Magenta!0.000}{\strut que} \colorbox{Magenta!0.000}{\strut ur} \colorbox{Magenta!0.000}{\strut .} \colorbox{Magenta!0.000}{\strut  Add} \colorbox{Magenta!0.000}{\strut  vanilla} \colorbox{Magenta!0.000}{\strut  extract} \colorbox{Magenta!0.000}{\strut .} \colorbox{Magenta!0.000}{\strut If} \\
\midrule
Jacobian & \num{2.624e-01} & \colorbox{Cyan!0.000}{\strut  expects} \colorbox{Cyan!0.000}{\strut  the} \colorbox{Cyan!0.000}{\strut  fund} \colorbox{Cyan!0.000}{\strut  to} \colorbox{Cyan!0.000}{\strut  generate} \colorbox{Cyan!0.000}{\strut  returns} \colorbox{Cyan!0.000}{\strut  sim} \colorbox{Cyan!0.000}{\strut ilar} \colorbox{Cyan!0.000}{\strut  to} \colorbox{Cyan!0.000}{\strut  those} \colorbox{Cyan!0.000}{\strut  of} \colorbox{Cyan!0.000}{\strut  other} \colorbox{Cyan!0.000}{\strut  Ash} \colorbox{Cyan!0.000}{\strut m} \\
Input SAE & \num{1.241e+01} & \colorbox{Green!0.000}{\strut  expects} \colorbox{Green!0.000}{\strut  the} \colorbox{Green!0.000}{\strut  fund} \colorbox{Green!0.000}{\strut  to} \colorbox{Green!0.000}{\strut  generate} \colorbox{Green!0.000}{\strut  returns} \colorbox{Green!0.000}{\strut  sim} \colorbox{Green!0.000}{\strut ilar} \colorbox{Green!0.000}{\strut  to} \colorbox{Green!0.000}{\strut  those} \colorbox{Green!0.000}{\strut  of} \colorbox{Green!0.000}{\strut  other} \colorbox{Green!0.000}{\strut  Ash} \colorbox{Green!0.000}{\strut m} \\
Output SAE & \num{3.429e+00} & \colorbox{Magenta!0.000}{\strut  expects} \colorbox{Magenta!0.000}{\strut  the} \colorbox{Magenta!0.000}{\strut  fund} \colorbox{Magenta!0.000}{\strut  to} \colorbox{Magenta!0.000}{\strut  generate} \colorbox{Magenta!0.000}{\strut  returns} \colorbox{Magenta!0.000}{\strut  sim} \colorbox{Magenta!0.000}{\strut ilar} \colorbox{Magenta!0.000}{\strut  to} \colorbox{Magenta!0.000}{\strut  those} \colorbox{Magenta!0.000}{\strut  of} \colorbox{Magenta!0.000}{\strut  other} \colorbox{Magenta!0.000}{\strut  Ash} \colorbox{Magenta!0.000}{\strut m} \\
\midrule
Jacobian & \num{2.624e-01} & \colorbox{Cyan!0.000}{\strut  Up} \colorbox{Cyan!0.000}{\strut \textquotedbl{}} \colorbox{Cyan!0.000}{\strut  achievement} \colorbox{Cyan!0.000}{\strut  works} \colorbox{Cyan!0.000}{\strut  again} \colorbox{Cyan!0.000}{\strut .} \colorbox{Cyan!0.000}{\strut Good} \colorbox{Cyan!0.000}{\strut  news} \colorbox{Cyan!0.000}{\strut  for} \colorbox{Cyan!0.000}{\strut ely} \colorbox{Cyan!0.000}{\strut tra} \colorbox{Cyan!0.000}{\strut  fl} \colorbox{Cyan!0.000}{\strut iers} \\
Input SAE & \num{1.971e+01} & \colorbox{Green!0.000}{\strut  Up} \colorbox{Green!0.000}{\strut \textquotedbl{}} \colorbox{Green!0.000}{\strut  achievement} \colorbox{Green!0.000}{\strut  works} \colorbox{Green!0.000}{\strut  again} \colorbox{Green!0.000}{\strut .} \colorbox{Green!0.000}{\strut Good} \colorbox{Green!0.000}{\strut  news} \colorbox{Green!0.000}{\strut  for} \colorbox{Green!0.000}{\strut ely} \colorbox{Green!0.000}{\strut tra} \colorbox{Green!0.000}{\strut  fl} \colorbox{Green!0.000}{\strut iers} \\
Output SAE & \num{5.368e+00} & \colorbox{Magenta!0.000}{\strut  Up} \colorbox{Magenta!0.000}{\strut \textquotedbl{}} \colorbox{Magenta!0.000}{\strut  achievement} \colorbox{Magenta!0.000}{\strut  works} \colorbox{Magenta!0.000}{\strut  again} \colorbox{Magenta!0.000}{\strut .} \colorbox{Magenta!0.000}{\strut Good} \colorbox{Magenta!0.000}{\strut  news} \colorbox{Magenta!0.000}{\strut  for} \colorbox{Magenta!0.000}{\strut ely} \colorbox{Magenta!0.000}{\strut tra} \colorbox{Magenta!0.000}{\strut  fl} \colorbox{Magenta!0.000}{\strut iers} \\
\bottomrule
\end{longtable}
\caption{feature pairs/Layer15-65536-J1-LR5.0e-04-k32-T3.0e+08 abs mean/examples-30892-v-64253 stas c4-en-10k,train,batch size=32,ctx len=16.csv}
\end{table}
% \begin{table}
\centering
\begin{longtable}{lrl}
\toprule
Category & Max. abs. value & Example tokens \\
\midrule
Jacobian & \num{2.704e-01} & \colorbox{Cyan!0.000}{\strut  astronaut} \colorbox{Cyan!0.000}{\strut s} \colorbox{Cyan!0.000}{\strut  be} \colorbox{Cyan!0.000}{\strut  able} \colorbox{Cyan!0.000}{\strut  to} \colorbox{Cyan!0.000}{\strut  survive} \colorbox{Cyan!0.000}{\strut  in} \colorbox{Cyan!0.000}{\strut  the} \colorbox{Cyan!0.000}{\strut  spaces} \colorbox{Cyan!0.000}{\strut uits} \colorbox{Cyan!0.000}{\strut ?} \colorbox{Cyan!0.000}{\strut 7} \colorbox{Cyan!0.000}{\strut )} \colorbox{Cyan!100.000}{\strut  What} \\
Input SAE & \num{9.649e+00} & \colorbox{Green!0.000}{\strut  astronaut} \colorbox{Green!0.000}{\strut s} \colorbox{Green!0.000}{\strut  be} \colorbox{Green!0.000}{\strut  able} \colorbox{Green!0.000}{\strut  to} \colorbox{Green!0.000}{\strut  survive} \colorbox{Green!0.000}{\strut  in} \colorbox{Green!0.000}{\strut  the} \colorbox{Green!0.000}{\strut  spaces} \colorbox{Green!0.000}{\strut uits} \colorbox{Green!0.000}{\strut ?} \colorbox{Green!0.000}{\strut 7} \colorbox{Green!0.000}{\strut )} \colorbox{Green!55.145}{\strut  What} \\
Output SAE & \num{3.886e+00} & \colorbox{Magenta!0.000}{\strut  astronaut} \colorbox{Magenta!0.000}{\strut s} \colorbox{Magenta!0.000}{\strut  be} \colorbox{Magenta!0.000}{\strut  able} \colorbox{Magenta!0.000}{\strut  to} \colorbox{Magenta!0.000}{\strut  survive} \colorbox{Magenta!0.000}{\strut  in} \colorbox{Magenta!0.000}{\strut  the} \colorbox{Magenta!0.000}{\strut  spaces} \colorbox{Magenta!0.000}{\strut uits} \colorbox{Magenta!0.000}{\strut ?} \colorbox{Magenta!0.000}{\strut 7} \colorbox{Magenta!0.000}{\strut )} \colorbox{Magenta!69.433}{\strut  What} \\
\midrule
Jacobian & \num{2.675e-01} & \colorbox{Cyan!0.000}{\strut  new} \colorbox{Cyan!0.000}{\strut  spaces} \colorbox{Cyan!0.000}{\strut uit} \colorbox{Cyan!0.000}{\strut ?} \colorbox{Cyan!0.000}{\strut 3} \colorbox{Cyan!0.000}{\strut )} \colorbox{Cyan!98.921}{\strut  What} \colorbox{Cyan!67.696}{\strut  will} \colorbox{Cyan!0.000}{\strut  the} \colorbox{Cyan!0.000}{\strut  spaces} \colorbox{Cyan!0.000}{\strut uit} \colorbox{Cyan!0.000}{\strut  help} \colorbox{Cyan!0.000}{\strut  astronaut} \colorbox{Cyan!0.000}{\strut s} \\
Input SAE & \num{9.854e+00} & \colorbox{Green!0.000}{\strut  new} \colorbox{Green!0.000}{\strut  spaces} \colorbox{Green!0.000}{\strut uit} \colorbox{Green!0.000}{\strut ?} \colorbox{Green!0.000}{\strut 3} \colorbox{Green!0.000}{\strut )} \colorbox{Green!56.317}{\strut  What} \colorbox{Green!10.136}{\strut  will} \colorbox{Green!0.000}{\strut  the} \colorbox{Green!0.000}{\strut  spaces} \colorbox{Green!0.000}{\strut uit} \colorbox{Green!0.000}{\strut  help} \colorbox{Green!0.000}{\strut  astronaut} \colorbox{Green!0.000}{\strut s} \\
Output SAE & \num{3.875e+00} & \colorbox{Magenta!0.000}{\strut  new} \colorbox{Magenta!0.000}{\strut  spaces} \colorbox{Magenta!0.000}{\strut uit} \colorbox{Magenta!0.000}{\strut ?} \colorbox{Magenta!0.000}{\strut 3} \colorbox{Magenta!0.000}{\strut )} \colorbox{Magenta!69.245}{\strut  What} \colorbox{Magenta!28.647}{\strut  will} \colorbox{Magenta!0.000}{\strut  the} \colorbox{Magenta!0.000}{\strut  spaces} \colorbox{Magenta!0.000}{\strut uit} \colorbox{Magenta!0.000}{\strut  help} \colorbox{Magenta!0.000}{\strut  astronaut} \colorbox{Magenta!0.000}{\strut s} \\
\midrule
Jacobian & \num{2.672e-01} & \colorbox{Cyan!0.000}{\strut  I} \colorbox{Cyan!0.000}{\strut  knew} \colorbox{Cyan!0.000}{\strut  about} \colorbox{Cyan!0.000}{\strut  omn} \colorbox{Cyan!0.000}{\strut iv} \colorbox{Cyan!0.000}{\strut ores} \colorbox{Cyan!0.000}{\strut  and} \colorbox{Cyan!0.000}{\strut  herb} \colorbox{Cyan!0.000}{\strut iv} \colorbox{Cyan!0.000}{\strut ores} \colorbox{Cyan!0.000}{\strut ,} \colorbox{Cyan!0.000}{\strut  but} \colorbox{Cyan!98.819}{\strut  what} \colorbox{Cyan!0.000}{\strut  the} \colorbox{Cyan!66.105}{\strut  heck} \\
Input SAE & \num{1.341e+01} & \colorbox{Green!0.000}{\strut  I} \colorbox{Green!0.000}{\strut  knew} \colorbox{Green!0.000}{\strut  about} \colorbox{Green!0.000}{\strut  omn} \colorbox{Green!0.000}{\strut iv} \colorbox{Green!0.000}{\strut ores} \colorbox{Green!0.000}{\strut  and} \colorbox{Green!0.000}{\strut  herb} \colorbox{Green!0.000}{\strut iv} \colorbox{Green!0.000}{\strut ores} \colorbox{Green!0.000}{\strut ,} \colorbox{Green!0.000}{\strut  but} \colorbox{Green!76.633}{\strut  what} \colorbox{Green!0.000}{\strut  the} \colorbox{Green!8.016}{\strut  heck} \\
Output SAE & \num{4.813e+00} & \colorbox{Magenta!0.000}{\strut  I} \colorbox{Magenta!0.000}{\strut  knew} \colorbox{Magenta!0.000}{\strut  about} \colorbox{Magenta!0.000}{\strut  omn} \colorbox{Magenta!0.000}{\strut iv} \colorbox{Magenta!0.000}{\strut ores} \colorbox{Magenta!0.000}{\strut  and} \colorbox{Magenta!0.000}{\strut  herb} \colorbox{Magenta!0.000}{\strut iv} \colorbox{Magenta!0.000}{\strut ores} \colorbox{Magenta!0.000}{\strut ,} \colorbox{Magenta!0.000}{\strut  but} \colorbox{Magenta!86.013}{\strut  what} \colorbox{Magenta!0.000}{\strut  the} \colorbox{Magenta!17.186}{\strut  heck} \\
\midrule
Jacobian & \num{2.648e-01} & \colorbox{Cyan!0.000}{\strut  to} \colorbox{Cyan!0.000}{\strut  do} \colorbox{Cyan!0.000}{\strut .} \colorbox{Cyan!0.000}{\strut Can} \colorbox{Cyan!0.000}{\strut  Ethereum} \colorbox{Cyan!0.000}{\strut  Classic} \colorbox{Cyan!0.000}{\strut  be} \colorbox{Cyan!0.000}{\strut  m} \colorbox{Cyan!0.000}{\strut ined} \colorbox{Cyan!0.000}{\strut ?} \colorbox{Cyan!97.929}{\strut What} \colorbox{Cyan!0.000}{\strut  differs} \colorbox{Cyan!0.000}{\strut  Ethereum} \\
Input SAE & \num{1.023e+01} & \colorbox{Green!0.000}{\strut  to} \colorbox{Green!0.000}{\strut  do} \colorbox{Green!0.000}{\strut .} \colorbox{Green!0.000}{\strut Can} \colorbox{Green!0.000}{\strut  Ethereum} \colorbox{Green!0.000}{\strut  Classic} \colorbox{Green!0.000}{\strut  be} \colorbox{Green!0.000}{\strut  m} \colorbox{Green!0.000}{\strut ined} \colorbox{Green!0.000}{\strut ?} \colorbox{Green!58.473}{\strut What} \colorbox{Green!0.000}{\strut  differs} \colorbox{Green!0.000}{\strut  Ethereum} \\
Output SAE & \num{4.199e+00} & \colorbox{Magenta!0.000}{\strut  to} \colorbox{Magenta!0.000}{\strut  do} \colorbox{Magenta!0.000}{\strut .} \colorbox{Magenta!0.000}{\strut Can} \colorbox{Magenta!0.000}{\strut  Ethereum} \colorbox{Magenta!0.000}{\strut  Classic} \colorbox{Magenta!0.000}{\strut  be} \colorbox{Magenta!0.000}{\strut  m} \colorbox{Magenta!0.000}{\strut ined} \colorbox{Magenta!0.000}{\strut ?} \colorbox{Magenta!75.031}{\strut What} \colorbox{Magenta!0.000}{\strut  differs} \colorbox{Magenta!0.000}{\strut  Ethereum} \\
\midrule
Jacobian & \num{2.646e-01} & \colorbox{Cyan!0.000}{\strut ONS} \colorbox{Cyan!0.000}{\strut OR} \colorbox{Cyan!0.000}{\strut ?} \colorbox{Cyan!0.000}{\strut  AND} \colorbox{Cyan!97.833}{\strut  WHAT} \colorbox{Cyan!0.000}{\strut  ARE} \colorbox{Cyan!0.000}{\strut  THEY} \colorbox{Cyan!0.000}{\strut  L} \colorbox{Cyan!0.000}{\strut OOK} \colorbox{Cyan!0.000}{\strut ING} \colorbox{Cyan!0.000}{\strut  FOR} \colorbox{Cyan!0.000}{\strut  IN} \colorbox{Cyan!0.000}{\strut  SP} \colorbox{Cyan!0.000}{\strut ONS} \colorbox{Cyan!0.000}{\strut OR} \\
Input SAE & \num{1.054e+01} & \colorbox{Green!0.000}{\strut ONS} \colorbox{Green!0.000}{\strut OR} \colorbox{Green!0.000}{\strut ?} \colorbox{Green!0.000}{\strut  AND} \colorbox{Green!60.210}{\strut  WHAT} \colorbox{Green!0.000}{\strut  ARE} \colorbox{Green!0.000}{\strut  THEY} \colorbox{Green!0.000}{\strut  L} \colorbox{Green!0.000}{\strut OOK} \colorbox{Green!0.000}{\strut ING} \colorbox{Green!0.000}{\strut  FOR} \colorbox{Green!0.000}{\strut  IN} \colorbox{Green!0.000}{\strut  SP} \colorbox{Green!0.000}{\strut ONS} \colorbox{Green!0.000}{\strut OR} \\
Output SAE & \num{3.815e+00} & \colorbox{Magenta!0.000}{\strut ONS} \colorbox{Magenta!0.000}{\strut OR} \colorbox{Magenta!0.000}{\strut ?} \colorbox{Magenta!0.000}{\strut  AND} \colorbox{Magenta!68.165}{\strut  WHAT} \colorbox{Magenta!0.000}{\strut  ARE} \colorbox{Magenta!0.000}{\strut  THEY} \colorbox{Magenta!0.000}{\strut  L} \colorbox{Magenta!0.000}{\strut OOK} \colorbox{Magenta!0.000}{\strut ING} \colorbox{Magenta!0.000}{\strut  FOR} \colorbox{Magenta!0.000}{\strut  IN} \colorbox{Magenta!0.000}{\strut  SP} \colorbox{Magenta!0.000}{\strut ONS} \colorbox{Magenta!0.000}{\strut OR} \\
\midrule
Jacobian & \num{2.646e-01} & \colorbox{Cyan!0.000}{\strut the} \colorbox{Cyan!0.000}{\strut  first} \colorbox{Cyan!0.000}{\strut  non} \colorbox{Cyan!0.000}{\strut -} \colorbox{Cyan!0.000}{\strut king} \colorbox{Cyan!0.000}{\strut  ruler} \colorbox{Cyan!0.000}{\strut )} \colorbox{Cyan!0.000}{\strut  to} \colorbox{Cyan!0.000}{\strut  make} \colorbox{Cyan!0.000}{\strut  his} \colorbox{Cyan!0.000}{\strut  move} \colorbox{Cyan!0.000}{\strut ?} \colorbox{Cyan!0.000}{\strut  And} \colorbox{Cyan!97.831}{\strut  what} \colorbox{Cyan!0.000}{\strut  is} \\
Input SAE & \num{1.010e+01} & \colorbox{Green!0.000}{\strut the} \colorbox{Green!0.000}{\strut  first} \colorbox{Green!0.000}{\strut  non} \colorbox{Green!0.000}{\strut -} \colorbox{Green!0.000}{\strut king} \colorbox{Green!0.000}{\strut  ruler} \colorbox{Green!0.000}{\strut )} \colorbox{Green!0.000}{\strut  to} \colorbox{Green!0.000}{\strut  make} \colorbox{Green!0.000}{\strut  his} \colorbox{Green!0.000}{\strut  move} \colorbox{Green!0.000}{\strut ?} \colorbox{Green!0.000}{\strut  And} \colorbox{Green!57.698}{\strut  what} \colorbox{Green!0.000}{\strut  is} \\
Output SAE & \num{4.319e+00} & \colorbox{Magenta!0.000}{\strut the} \colorbox{Magenta!0.000}{\strut  first} \colorbox{Magenta!0.000}{\strut  non} \colorbox{Magenta!0.000}{\strut -} \colorbox{Magenta!0.000}{\strut king} \colorbox{Magenta!0.000}{\strut  ruler} \colorbox{Magenta!0.000}{\strut )} \colorbox{Magenta!0.000}{\strut  to} \colorbox{Magenta!0.000}{\strut  make} \colorbox{Magenta!0.000}{\strut  his} \colorbox{Magenta!0.000}{\strut  move} \colorbox{Magenta!0.000}{\strut ?} \colorbox{Magenta!0.000}{\strut  And} \colorbox{Magenta!77.173}{\strut  what} \colorbox{Magenta!25.091}{\strut  is} \\
\midrule
Jacobian & \num{2.640e-01} & \colorbox{Cyan!0.000}{\strut  people} \colorbox{Cyan!0.000}{\strut  have} \colorbox{Cyan!0.000}{\strut  been} \colorbox{Cyan!0.000}{\strut  to} \colorbox{Cyan!0.000}{\strut  the} \colorbox{Cyan!0.000}{\strut  site} \colorbox{Cyan!0.000}{\strut  before} \colorbox{Cyan!0.000}{\strut ,} \colorbox{Cyan!97.628}{\strut  what} \colorbox{Cyan!0.000}{\strut  time} \colorbox{Cyan!0.000}{\strut  people} \colorbox{Cyan!0.000}{\strut  visited} \colorbox{Cyan!0.000}{\strut  the} \colorbox{Cyan!0.000}{\strut  site} \colorbox{Cyan!0.000}{\strut  and} \\
Input SAE & \num{1.337e+01} & \colorbox{Green!0.000}{\strut  people} \colorbox{Green!0.000}{\strut  have} \colorbox{Green!0.000}{\strut  been} \colorbox{Green!0.000}{\strut  to} \colorbox{Green!0.000}{\strut  the} \colorbox{Green!0.000}{\strut  site} \colorbox{Green!0.000}{\strut  before} \colorbox{Green!0.000}{\strut ,} \colorbox{Green!76.434}{\strut  what} \colorbox{Green!0.000}{\strut  time} \colorbox{Green!0.000}{\strut  people} \colorbox{Green!0.000}{\strut  visited} \colorbox{Green!0.000}{\strut  the} \colorbox{Green!0.000}{\strut  site} \colorbox{Green!0.000}{\strut  and} \\
Output SAE & \num{4.844e+00} & \colorbox{Magenta!0.000}{\strut  people} \colorbox{Magenta!0.000}{\strut  have} \colorbox{Magenta!0.000}{\strut  been} \colorbox{Magenta!0.000}{\strut  to} \colorbox{Magenta!0.000}{\strut  the} \colorbox{Magenta!0.000}{\strut  site} \colorbox{Magenta!0.000}{\strut  before} \colorbox{Magenta!0.000}{\strut ,} \colorbox{Magenta!86.562}{\strut  what} \colorbox{Magenta!0.000}{\strut  time} \colorbox{Magenta!0.000}{\strut  people} \colorbox{Magenta!0.000}{\strut  visited} \colorbox{Magenta!0.000}{\strut  the} \colorbox{Magenta!0.000}{\strut  site} \colorbox{Magenta!0.000}{\strut  and} \\
\midrule
Jacobian & \num{2.638e-01} & \colorbox{Cyan!0.000}{\strut ?} \colorbox{Cyan!0.000}{\strut  W} \colorbox{Cyan!0.000}{\strut iner} \colorbox{Cyan!0.000}{\strut ies} \colorbox{Cyan!0.000}{\strut ,} \colorbox{Cyan!0.000}{\strut  restaurants} \colorbox{Cyan!0.000}{\strut ,} \colorbox{Cyan!0.000}{\strut  are} \colorbox{Cyan!0.000}{\strut  they} \colorbox{Cyan!0.000}{\strut  all} \colorbox{Cyan!0.000}{\strut  open} \colorbox{Cyan!0.000}{\strut ?} \colorbox{Cyan!97.560}{\strut  What} \colorbox{Cyan!0.000}{\strut  are} \colorbox{Cyan!0.000}{\strut  all} \\
Input SAE & \num{1.086e+01} & \colorbox{Green!0.000}{\strut ?} \colorbox{Green!0.000}{\strut  W} \colorbox{Green!0.000}{\strut iner} \colorbox{Green!0.000}{\strut ies} \colorbox{Green!0.000}{\strut ,} \colorbox{Green!0.000}{\strut  restaurants} \colorbox{Green!0.000}{\strut ,} \colorbox{Green!0.000}{\strut  are} \colorbox{Green!0.000}{\strut  they} \colorbox{Green!0.000}{\strut  all} \colorbox{Green!0.000}{\strut  open} \colorbox{Green!0.000}{\strut ?} \colorbox{Green!62.071}{\strut  What} \colorbox{Green!0.000}{\strut  are} \colorbox{Green!0.000}{\strut  all} \\
Output SAE & \num{4.341e+00} & \colorbox{Magenta!0.000}{\strut ?} \colorbox{Magenta!0.000}{\strut  W} \colorbox{Magenta!0.000}{\strut iner} \colorbox{Magenta!0.000}{\strut ies} \colorbox{Magenta!0.000}{\strut ,} \colorbox{Magenta!0.000}{\strut  restaurants} \colorbox{Magenta!0.000}{\strut ,} \colorbox{Magenta!0.000}{\strut  are} \colorbox{Magenta!0.000}{\strut  they} \colorbox{Magenta!0.000}{\strut  all} \colorbox{Magenta!0.000}{\strut  open} \colorbox{Magenta!0.000}{\strut ?} \colorbox{Magenta!77.568}{\strut  What} \colorbox{Magenta!13.639}{\strut  are} \colorbox{Magenta!0.000}{\strut  all} \\
\midrule
Jacobian & \num{2.636e-01} & \colorbox{Cyan!0.000}{\strut  they} \colorbox{Cyan!0.000}{\strut  dealing} \colorbox{Cyan!0.000}{\strut  with} \colorbox{Cyan!0.000}{\strut ?} \colorbox{Cyan!97.491}{\strut  What} \colorbox{Cyan!0.000}{\strut  value} \colorbox{Cyan!0.000}{\strut  am} \colorbox{Cyan!0.000}{\strut  I} \colorbox{Cyan!0.000}{\strut  providing} \colorbox{Cyan!0.000}{\strut  them} \colorbox{Cyan!0.000}{\strut ?} \colorbox{Cyan!0.000}{\strut  Do} \colorbox{Cyan!0.000}{\strut  I} \colorbox{Cyan!0.000}{\strut  even} \colorbox{Cyan!0.000}{\strut  like} \\
Input SAE & \num{1.092e+01} & \colorbox{Green!0.000}{\strut  they} \colorbox{Green!0.000}{\strut  dealing} \colorbox{Green!0.000}{\strut  with} \colorbox{Green!0.000}{\strut ?} \colorbox{Green!62.413}{\strut  What} \colorbox{Green!0.000}{\strut  value} \colorbox{Green!0.000}{\strut  am} \colorbox{Green!0.000}{\strut  I} \colorbox{Green!0.000}{\strut  providing} \colorbox{Green!0.000}{\strut  them} \colorbox{Green!0.000}{\strut ?} \colorbox{Green!0.000}{\strut  Do} \colorbox{Green!0.000}{\strut  I} \colorbox{Green!0.000}{\strut  even} \colorbox{Green!0.000}{\strut  like} \\
Output SAE & \num{4.220e+00} & \colorbox{Magenta!0.000}{\strut  they} \colorbox{Magenta!0.000}{\strut  dealing} \colorbox{Magenta!0.000}{\strut  with} \colorbox{Magenta!0.000}{\strut ?} \colorbox{Magenta!75.412}{\strut  What} \colorbox{Magenta!0.000}{\strut  value} \colorbox{Magenta!0.000}{\strut  am} \colorbox{Magenta!0.000}{\strut  I} \colorbox{Magenta!0.000}{\strut  providing} \colorbox{Magenta!0.000}{\strut  them} \colorbox{Magenta!0.000}{\strut ?} \colorbox{Magenta!0.000}{\strut  Do} \colorbox{Magenta!0.000}{\strut  I} \colorbox{Magenta!0.000}{\strut  even} \colorbox{Magenta!0.000}{\strut  like} \\
\midrule
Jacobian & \num{2.636e-01} & \colorbox{Cyan!0.000}{\strut  who} \colorbox{Cyan!0.000}{\strut  these} \colorbox{Cyan!0.000}{\strut  humans} \colorbox{Cyan!0.000}{\strut  are} \colorbox{Cyan!0.000}{\strut  likely} \colorbox{Cyan!0.000}{\strut  to} \colorbox{Cyan!0.000}{\strut  be} \colorbox{Cyan!0.000}{\strut  and} \colorbox{Cyan!97.486}{\strut  what} \colorbox{Cyan!0.000}{\strut  their} \colorbox{Cyan!0.000}{\strut  personal} \colorbox{Cyan!0.000}{\strut  characteristics} \colorbox{Cyan!0.000}{\strut  suggest} \colorbox{Cyan!0.000}{\strut  about} \colorbox{Cyan!0.000}{\strut  how} \\
Input SAE & \num{1.138e+01} & \colorbox{Green!0.000}{\strut  who} \colorbox{Green!0.000}{\strut  these} \colorbox{Green!0.000}{\strut  humans} \colorbox{Green!0.000}{\strut  are} \colorbox{Green!0.000}{\strut  likely} \colorbox{Green!0.000}{\strut  to} \colorbox{Green!0.000}{\strut  be} \colorbox{Green!0.000}{\strut  and} \colorbox{Green!65.047}{\strut  what} \colorbox{Green!0.000}{\strut  their} \colorbox{Green!0.000}{\strut  personal} \colorbox{Green!0.000}{\strut  characteristics} \colorbox{Green!0.000}{\strut  suggest} \colorbox{Green!0.000}{\strut  about} \colorbox{Green!0.000}{\strut  how} \\
Output SAE & \num{4.053e+00} & \colorbox{Magenta!0.000}{\strut  who} \colorbox{Magenta!0.000}{\strut  these} \colorbox{Magenta!0.000}{\strut  humans} \colorbox{Magenta!0.000}{\strut  are} \colorbox{Magenta!0.000}{\strut  likely} \colorbox{Magenta!0.000}{\strut  to} \colorbox{Magenta!0.000}{\strut  be} \colorbox{Magenta!0.000}{\strut  and} \colorbox{Magenta!72.423}{\strut  what} \colorbox{Magenta!9.903}{\strut  their} \colorbox{Magenta!0.000}{\strut  personal} \colorbox{Magenta!0.000}{\strut  characteristics} \colorbox{Magenta!0.000}{\strut  suggest} \colorbox{Magenta!0.000}{\strut  about} \colorbox{Magenta!0.000}{\strut  how} \\
\midrule
Jacobian & \num{2.631e-01} & \colorbox{Cyan!0.000}{\strut ?} \colorbox{Cyan!0.000}{\strut Can} \colorbox{Cyan!0.000}{\strut  you} \colorbox{Cyan!0.000}{\strut  imagine} \colorbox{Cyan!93.270}{\strut  what} \colorbox{Cyan!0.000}{\strut  a} \colorbox{Cyan!0.000}{\strut  glorious} \colorbox{Cyan!0.000}{\strut  celebration} \colorbox{Cyan!0.000}{\strut  that} \colorbox{Cyan!0.000}{\strut  would} \colorbox{Cyan!0.000}{\strut  be} \colorbox{Cyan!0.000}{\strut ?} \colorbox{Cyan!0.000}{\strut  And} \colorbox{Cyan!97.305}{\strut  what} \\
Input SAE & \num{1.102e+01} & \colorbox{Green!0.000}{\strut ?} \colorbox{Green!0.000}{\strut Can} \colorbox{Green!0.000}{\strut  you} \colorbox{Green!0.000}{\strut  imagine} \colorbox{Green!62.991}{\strut  what} \colorbox{Green!0.000}{\strut  a} \colorbox{Green!0.000}{\strut  glorious} \colorbox{Green!0.000}{\strut  celebration} \colorbox{Green!0.000}{\strut  that} \colorbox{Green!0.000}{\strut  would} \colorbox{Green!0.000}{\strut  be} \colorbox{Green!0.000}{\strut ?} \colorbox{Green!0.000}{\strut  And} \colorbox{Green!57.523}{\strut  what} \\
Output SAE & \num{4.054e+00} & \colorbox{Magenta!0.000}{\strut ?} \colorbox{Magenta!0.000}{\strut Can} \colorbox{Magenta!0.000}{\strut  you} \colorbox{Magenta!0.000}{\strut  imagine} \colorbox{Magenta!62.148}{\strut  what} \colorbox{Magenta!0.000}{\strut  a} \colorbox{Magenta!0.000}{\strut  glorious} \colorbox{Magenta!0.000}{\strut  celebration} \colorbox{Magenta!0.000}{\strut  that} \colorbox{Magenta!0.000}{\strut  would} \colorbox{Magenta!0.000}{\strut  be} \colorbox{Magenta!0.000}{\strut ?} \colorbox{Magenta!0.000}{\strut  And} \colorbox{Magenta!72.447}{\strut  what} \\
\midrule
Jacobian & \num{2.629e-01} & \colorbox{Cyan!0.000}{\strut  suit} \colorbox{Cyan!0.000}{\strut \textquotesingle{}s} \colorbox{Cyan!0.000}{\strut  technology} \colorbox{Cyan!0.000}{\strut  be} \colorbox{Cyan!0.000}{\strut  used} \colorbox{Cyan!0.000}{\strut  in} \colorbox{Cyan!0.000}{\strut  one} \colorbox{Cyan!0.000}{\strut  day} \colorbox{Cyan!0.000}{\strut ?} \colorbox{Cyan!0.000}{\strut 1} \colorbox{Cyan!0.000}{\strut )} \colorbox{Cyan!97.221}{\strut  What} \colorbox{Cyan!0.000}{\strut  kind} \colorbox{Cyan!0.000}{\strut  of} \\
Input SAE & \num{1.003e+01} & \colorbox{Green!0.000}{\strut  suit} \colorbox{Green!0.000}{\strut \textquotesingle{}s} \colorbox{Green!0.000}{\strut  technology} \colorbox{Green!0.000}{\strut  be} \colorbox{Green!0.000}{\strut  used} \colorbox{Green!0.000}{\strut  in} \colorbox{Green!0.000}{\strut  one} \colorbox{Green!0.000}{\strut  day} \colorbox{Green!0.000}{\strut ?} \colorbox{Green!0.000}{\strut 1} \colorbox{Green!0.000}{\strut )} \colorbox{Green!57.316}{\strut  What} \colorbox{Green!0.000}{\strut  kind} \colorbox{Green!0.000}{\strut  of} \\
Output SAE & \num{3.986e+00} & \colorbox{Magenta!0.000}{\strut  suit} \colorbox{Magenta!0.000}{\strut \textquotesingle{}s} \colorbox{Magenta!0.000}{\strut  technology} \colorbox{Magenta!0.000}{\strut  be} \colorbox{Magenta!0.000}{\strut  used} \colorbox{Magenta!0.000}{\strut  in} \colorbox{Magenta!0.000}{\strut  one} \colorbox{Magenta!0.000}{\strut  day} \colorbox{Magenta!0.000}{\strut ?} \colorbox{Magenta!0.000}{\strut 1} \colorbox{Magenta!0.000}{\strut )} \colorbox{Magenta!71.222}{\strut  What} \colorbox{Magenta!0.000}{\strut  kind} \colorbox{Magenta!0.000}{\strut  of} \\
\bottomrule
\end{longtable}
\caption{feature pairs/Layer15-65536-J1-LR5.0e-04-k32-T3.0e+08 abs mean/examples-9820-v-9601 stas c4-en-10k,train,batch size=32,ctx len=16.csv}
\end{table}
% \begin{table}
\centering
\begin{longtable}{lrl}
\toprule
Category & Max. abs. value & Example tokens \\
\midrule
Jacobian & \num{2.660e-01} & \colorbox{Cyan!0.000}{\strut  provide} \colorbox{Cyan!0.000}{\strut  dock} \colorbox{Cyan!0.000}{\strut  service} \colorbox{Cyan!0.000}{\strut  only} \colorbox{Cyan!0.000}{\strut ,} \colorbox{Cyan!0.000}{\strut  and} \colorbox{Cyan!100.000}{\strut  un} \colorbox{Cyan!0.000}{\strut loading} \colorbox{Cyan!0.000}{\strut  is} \colorbox{Cyan!0.000}{\strut  the} \colorbox{Cyan!0.000}{\strut  Customer} \colorbox{Cyan!0.000}{\strut \textquotesingle{}s} \colorbox{Cyan!0.000}{\strut  responsibility} \colorbox{Cyan!0.000}{\strut .} \colorbox{Cyan!0.000}{\strut  Transport} \\
Input SAE & \num{1.125e+01} & \colorbox{Green!0.000}{\strut  provide} \colorbox{Green!0.000}{\strut  dock} \colorbox{Green!0.000}{\strut  service} \colorbox{Green!0.000}{\strut  only} \colorbox{Green!0.000}{\strut ,} \colorbox{Green!0.000}{\strut  and} \colorbox{Green!57.833}{\strut  un} \colorbox{Green!0.000}{\strut loading} \colorbox{Green!0.000}{\strut  is} \colorbox{Green!0.000}{\strut  the} \colorbox{Green!0.000}{\strut  Customer} \colorbox{Green!0.000}{\strut \textquotesingle{}s} \colorbox{Green!0.000}{\strut  responsibility} \colorbox{Green!0.000}{\strut .} \colorbox{Green!0.000}{\strut  Transport} \\
Output SAE & \num{5.100e+00} & \colorbox{Magenta!0.000}{\strut  provide} \colorbox{Magenta!0.000}{\strut  dock} \colorbox{Magenta!0.000}{\strut  service} \colorbox{Magenta!0.000}{\strut  only} \colorbox{Magenta!0.000}{\strut ,} \colorbox{Magenta!0.000}{\strut  and} \colorbox{Magenta!98.471}{\strut  un} \colorbox{Magenta!0.000}{\strut loading} \colorbox{Magenta!0.000}{\strut  is} \colorbox{Magenta!0.000}{\strut  the} \colorbox{Magenta!0.000}{\strut  Customer} \colorbox{Magenta!0.000}{\strut \textquotesingle{}s} \colorbox{Magenta!0.000}{\strut  responsibility} \colorbox{Magenta!0.000}{\strut .} \colorbox{Magenta!0.000}{\strut  Transport} \\
\midrule
Jacobian & \num{2.647e-01} & \colorbox{Cyan!0.000}{\strut  the} \colorbox{Cyan!0.000}{\strut  management} \colorbox{Cyan!0.000}{\strut  system} \colorbox{Cyan!0.000}{\strut  operates} \colorbox{Cyan!0.000}{\strut  and} \colorbox{Cyan!0.000}{\strut  its} \colorbox{Cyan!0.000}{\strut  general} \colorbox{Cyan!0.000}{\strut  approach} \colorbox{Cyan!0.000}{\strut  to} \colorbox{Cyan!0.000}{\strut  both} \colorbox{Cyan!0.000}{\strut  planned} \colorbox{Cyan!0.000}{\strut  and} \colorbox{Cyan!99.510}{\strut  un} \colorbox{Cyan!0.000}{\strut planned} \colorbox{Cyan!0.000}{\strut  risk} \\
Input SAE & \num{1.082e+01} & \colorbox{Green!0.000}{\strut  the} \colorbox{Green!0.000}{\strut  management} \colorbox{Green!0.000}{\strut  system} \colorbox{Green!0.000}{\strut  operates} \colorbox{Green!0.000}{\strut  and} \colorbox{Green!0.000}{\strut  its} \colorbox{Green!0.000}{\strut  general} \colorbox{Green!0.000}{\strut  approach} \colorbox{Green!0.000}{\strut  to} \colorbox{Green!0.000}{\strut  both} \colorbox{Green!0.000}{\strut  planned} \colorbox{Green!0.000}{\strut  and} \colorbox{Green!55.629}{\strut  un} \colorbox{Green!0.000}{\strut planned} \colorbox{Green!0.000}{\strut  risk} \\
Output SAE & \num{4.513e+00} & \colorbox{Magenta!0.000}{\strut  the} \colorbox{Magenta!0.000}{\strut  management} \colorbox{Magenta!0.000}{\strut  system} \colorbox{Magenta!0.000}{\strut  operates} \colorbox{Magenta!0.000}{\strut  and} \colorbox{Magenta!0.000}{\strut  its} \colorbox{Magenta!0.000}{\strut  general} \colorbox{Magenta!0.000}{\strut  approach} \colorbox{Magenta!0.000}{\strut  to} \colorbox{Magenta!0.000}{\strut  both} \colorbox{Magenta!0.000}{\strut  planned} \colorbox{Magenta!0.000}{\strut  and} \colorbox{Magenta!87.124}{\strut  un} \colorbox{Magenta!0.000}{\strut planned} \colorbox{Magenta!0.000}{\strut  risk} \\
\midrule
Jacobian & \num{2.646e-01} & \colorbox{Cyan!0.000}{\strut  the} \colorbox{Cyan!0.000}{\strut  procedure} \colorbox{Cyan!0.000}{\strut  should} \colorbox{Cyan!0.000}{\strut  be} \colorbox{Cyan!0.000}{\strut  integrated} \colorbox{Cyan!0.000}{\strut  with} \colorbox{Cyan!0.000}{\strut  the} \colorbox{Cyan!0.000}{\strut  quality} \colorbox{Cyan!0.000}{\strut  system} \colorbox{Cyan!0.000}{\strut  and} \colorbox{Cyan!0.000}{\strut  apply} \colorbox{Cyan!0.000}{\strut  to} \colorbox{Cyan!0.000}{\strut  planned} \colorbox{Cyan!0.000}{\strut  and} \colorbox{Cyan!99.488}{\strut  un} \\
Input SAE & \num{1.023e+01} & \colorbox{Green!0.000}{\strut  the} \colorbox{Green!0.000}{\strut  procedure} \colorbox{Green!0.000}{\strut  should} \colorbox{Green!0.000}{\strut  be} \colorbox{Green!0.000}{\strut  integrated} \colorbox{Green!0.000}{\strut  with} \colorbox{Green!0.000}{\strut  the} \colorbox{Green!0.000}{\strut  quality} \colorbox{Green!0.000}{\strut  system} \colorbox{Green!0.000}{\strut  and} \colorbox{Green!0.000}{\strut  apply} \colorbox{Green!0.000}{\strut  to} \colorbox{Green!0.000}{\strut  planned} \colorbox{Green!0.000}{\strut  and} \colorbox{Green!52.594}{\strut  un} \\
Output SAE & \num{4.392e+00} & \colorbox{Magenta!0.000}{\strut  the} \colorbox{Magenta!0.000}{\strut  procedure} \colorbox{Magenta!0.000}{\strut  should} \colorbox{Magenta!0.000}{\strut  be} \colorbox{Magenta!0.000}{\strut  integrated} \colorbox{Magenta!0.000}{\strut  with} \colorbox{Magenta!0.000}{\strut  the} \colorbox{Magenta!0.000}{\strut  quality} \colorbox{Magenta!0.000}{\strut  system} \colorbox{Magenta!0.000}{\strut  and} \colorbox{Magenta!0.000}{\strut  apply} \colorbox{Magenta!0.000}{\strut  to} \colorbox{Magenta!0.000}{\strut  planned} \colorbox{Magenta!0.000}{\strut  and} \colorbox{Magenta!84.789}{\strut  un} \\
\midrule
Jacobian & \num{2.628e-01} & \colorbox{Cyan!0.000}{\strut  in} \colorbox{Cyan!0.000}{\strut  1997} \colorbox{Cyan!0.000}{\strut ,} \colorbox{Cyan!0.000}{\strut  provides} \colorbox{Cyan!0.000}{\strut  managed} \colorbox{Cyan!0.000}{\strut  and} \colorbox{Cyan!98.826}{\strut  un} \colorbox{Cyan!0.000}{\strut managed} \colorbox{Cyan!0.000}{\strut  dedicated} \colorbox{Cyan!0.000}{\strut  servers} \colorbox{Cyan!0.000}{\strut  to} \colorbox{Cyan!0.000}{\strut  customers} \colorbox{Cyan!0.000}{\strut  in} \colorbox{Cyan!0.000}{\strut  nearly} \colorbox{Cyan!0.000}{\strut  100} \\
Input SAE & \num{1.095e+01} & \colorbox{Green!0.000}{\strut  in} \colorbox{Green!0.000}{\strut  1997} \colorbox{Green!0.000}{\strut ,} \colorbox{Green!0.000}{\strut  provides} \colorbox{Green!0.000}{\strut  managed} \colorbox{Green!0.000}{\strut  and} \colorbox{Green!56.310}{\strut  un} \colorbox{Green!0.000}{\strut managed} \colorbox{Green!0.000}{\strut  dedicated} \colorbox{Green!0.000}{\strut  servers} \colorbox{Green!0.000}{\strut  to} \colorbox{Green!0.000}{\strut  customers} \colorbox{Green!0.000}{\strut  in} \colorbox{Green!0.000}{\strut  nearly} \colorbox{Green!0.000}{\strut  100} \\
Output SAE & \num{4.522e+00} & \colorbox{Magenta!0.000}{\strut  in} \colorbox{Magenta!0.000}{\strut  1997} \colorbox{Magenta!0.000}{\strut ,} \colorbox{Magenta!0.000}{\strut  provides} \colorbox{Magenta!0.000}{\strut  managed} \colorbox{Magenta!0.000}{\strut  and} \colorbox{Magenta!87.303}{\strut  un} \colorbox{Magenta!0.000}{\strut managed} \colorbox{Magenta!0.000}{\strut  dedicated} \colorbox{Magenta!0.000}{\strut  servers} \colorbox{Magenta!0.000}{\strut  to} \colorbox{Magenta!0.000}{\strut  customers} \colorbox{Magenta!0.000}{\strut  in} \colorbox{Magenta!0.000}{\strut  nearly} \colorbox{Magenta!0.000}{\strut  100} \\
\midrule
Jacobian & \num{2.628e-01} & \colorbox{Cyan!0.000}{\strut  of} \colorbox{Cyan!0.000}{\strut  different} \colorbox{Cyan!0.000}{\strut  statistical} \colorbox{Cyan!0.000}{\strut  categories} \colorbox{Cyan!0.000}{\strut  first} \colorbox{Cyan!0.000}{\strut  serve} \colorbox{Cyan!0.000}{\strut  \%} \colorbox{Cyan!0.000}{\strut ,} \colorbox{Cyan!98.810}{\strut  un} \colorbox{Cyan!0.000}{\strut forced} \colorbox{Cyan!0.000}{\strut  errors} \colorbox{Cyan!0.000}{\strut ,} \colorbox{Cyan!0.000}{\strut  break} \colorbox{Cyan!0.000}{\strut  point} \\
Input SAE & \num{1.238e+01} & \colorbox{Green!0.000}{\strut  of} \colorbox{Green!0.000}{\strut  different} \colorbox{Green!0.000}{\strut  statistical} \colorbox{Green!0.000}{\strut  categories} \colorbox{Green!0.000}{\strut  first} \colorbox{Green!0.000}{\strut  serve} \colorbox{Green!0.000}{\strut  \%} \colorbox{Green!0.000}{\strut ,} \colorbox{Green!63.663}{\strut  un} \colorbox{Green!0.000}{\strut forced} \colorbox{Green!0.000}{\strut  errors} \colorbox{Green!0.000}{\strut ,} \colorbox{Green!0.000}{\strut  break} \colorbox{Green!0.000}{\strut  point} \\
Output SAE & \num{4.900e+00} & \colorbox{Magenta!0.000}{\strut  of} \colorbox{Magenta!0.000}{\strut  different} \colorbox{Magenta!0.000}{\strut  statistical} \colorbox{Magenta!0.000}{\strut  categories} \colorbox{Magenta!0.000}{\strut  first} \colorbox{Magenta!0.000}{\strut  serve} \colorbox{Magenta!0.000}{\strut  \%} \colorbox{Magenta!0.000}{\strut ,} \colorbox{Magenta!94.612}{\strut  un} \colorbox{Magenta!0.000}{\strut forced} \colorbox{Magenta!0.000}{\strut  errors} \colorbox{Magenta!0.000}{\strut ,} \colorbox{Magenta!0.000}{\strut  break} \colorbox{Magenta!0.000}{\strut  point} \\
\midrule
Jacobian & \num{2.622e-01} & \colorbox{Cyan!0.000}{\strut  with} \colorbox{Cyan!0.000}{\strut  the} \colorbox{Cyan!0.000}{\strut  disposal} \colorbox{Cyan!0.000}{\strut  of} \colorbox{Cyan!0.000}{\strut  older} \colorbox{Cyan!0.000}{\strut  models} \colorbox{Cyan!0.000}{\strut  as} \colorbox{Cyan!0.000}{\strut  well} \colorbox{Cyan!0.000}{\strut  as} \colorbox{Cyan!98.605}{\strut  un} \colorbox{Cyan!0.000}{\strut needed} \colorbox{Cyan!0.000}{\strut  kits} \colorbox{Cyan!0.000}{\strut  (} \colorbox{Cyan!0.000}{\strut for} \colorbox{Cyan!0.000}{\strut  example} \\
Input SAE & \num{1.068e+01} & \colorbox{Green!0.000}{\strut  with} \colorbox{Green!0.000}{\strut  the} \colorbox{Green!0.000}{\strut  disposal} \colorbox{Green!0.000}{\strut  of} \colorbox{Green!0.000}{\strut  older} \colorbox{Green!0.000}{\strut  models} \colorbox{Green!0.000}{\strut  as} \colorbox{Green!0.000}{\strut  well} \colorbox{Green!0.000}{\strut  as} \colorbox{Green!54.915}{\strut  un} \colorbox{Green!0.000}{\strut needed} \colorbox{Green!0.000}{\strut  kits} \colorbox{Green!0.000}{\strut  (} \colorbox{Green!0.000}{\strut for} \colorbox{Green!0.000}{\strut  example} \\
Output SAE & \num{4.982e+00} & \colorbox{Magenta!0.000}{\strut  with} \colorbox{Magenta!0.000}{\strut  the} \colorbox{Magenta!0.000}{\strut  disposal} \colorbox{Magenta!0.000}{\strut  of} \colorbox{Magenta!0.000}{\strut  older} \colorbox{Magenta!0.000}{\strut  models} \colorbox{Magenta!0.000}{\strut  as} \colorbox{Magenta!0.000}{\strut  well} \colorbox{Magenta!0.000}{\strut  as} \colorbox{Magenta!96.184}{\strut  un} \colorbox{Magenta!0.000}{\strut needed} \colorbox{Magenta!0.000}{\strut  kits} \colorbox{Magenta!0.000}{\strut  (} \colorbox{Magenta!0.000}{\strut for} \colorbox{Magenta!0.000}{\strut  example} \\
\midrule
Jacobian & \num{2.621e-01} & \colorbox{Cyan!0.000}{\strut  provides} \colorbox{Cyan!0.000}{\strut  managed} \colorbox{Cyan!0.000}{\strut  and} \colorbox{Cyan!98.568}{\strut  un} \colorbox{Cyan!0.000}{\strut managed} \colorbox{Cyan!0.000}{\strut  dedicated} \colorbox{Cyan!0.000}{\strut  web} \colorbox{Cyan!0.000}{\strut  hosting} \colorbox{Cyan!0.000}{\strut  services} \colorbox{Cyan!0.000}{\strut  to} \colorbox{Cyan!0.000}{\strut  thousands} \colorbox{Cyan!0.000}{\strut  of} \colorbox{Cyan!0.000}{\strut  satisfied} \colorbox{Cyan!0.000}{\strut  customers} \colorbox{Cyan!0.000}{\strut .} \\
Input SAE & \num{1.143e+01} & \colorbox{Green!0.000}{\strut  provides} \colorbox{Green!0.000}{\strut  managed} \colorbox{Green!0.000}{\strut  and} \colorbox{Green!58.753}{\strut  un} \colorbox{Green!0.000}{\strut managed} \colorbox{Green!0.000}{\strut  dedicated} \colorbox{Green!0.000}{\strut  web} \colorbox{Green!0.000}{\strut  hosting} \colorbox{Green!0.000}{\strut  services} \colorbox{Green!0.000}{\strut  to} \colorbox{Green!0.000}{\strut  thousands} \colorbox{Green!0.000}{\strut  of} \colorbox{Green!0.000}{\strut  satisfied} \colorbox{Green!0.000}{\strut  customers} \colorbox{Green!0.000}{\strut .} \\
Output SAE & \num{4.772e+00} & \colorbox{Magenta!0.000}{\strut  provides} \colorbox{Magenta!0.000}{\strut  managed} \colorbox{Magenta!0.000}{\strut  and} \colorbox{Magenta!92.135}{\strut  un} \colorbox{Magenta!0.000}{\strut managed} \colorbox{Magenta!0.000}{\strut  dedicated} \colorbox{Magenta!0.000}{\strut  web} \colorbox{Magenta!0.000}{\strut  hosting} \colorbox{Magenta!0.000}{\strut  services} \colorbox{Magenta!0.000}{\strut  to} \colorbox{Magenta!0.000}{\strut  thousands} \colorbox{Magenta!0.000}{\strut  of} \colorbox{Magenta!0.000}{\strut  satisfied} \colorbox{Magenta!0.000}{\strut  customers} \colorbox{Magenta!0.000}{\strut .} \\
\midrule
Jacobian & \num{2.618e-01} & \colorbox{Cyan!0.000}{\strut  cost} \colorbox{Cyan!0.000}{\strut  of} \colorbox{Cyan!0.000}{\strut  amount} \colorbox{Cyan!0.000}{\strut  of} \colorbox{Cyan!98.439}{\strut  un} \colorbox{Cyan!0.000}{\strut labeled} \colorbox{Cyan!0.000}{\strut  samples} \colorbox{Cyan!0.000}{\strut .} \colorbox{Cyan!0.000}{\strut  In} \colorbox{Cyan!0.000}{\strut  the} \colorbox{Cyan!0.000}{\strut  paper} \colorbox{Cyan!0.000}{\strut ,} \colorbox{Cyan!0.000}{\strut  we} \colorbox{Cyan!0.000}{\strut  propose} \colorbox{Cyan!0.000}{\strut  an} \\
Input SAE & \num{1.115e+01} & \colorbox{Green!0.000}{\strut  cost} \colorbox{Green!0.000}{\strut  of} \colorbox{Green!0.000}{\strut  amount} \colorbox{Green!0.000}{\strut  of} \colorbox{Green!57.312}{\strut  un} \colorbox{Green!0.000}{\strut labeled} \colorbox{Green!0.000}{\strut  samples} \colorbox{Green!0.000}{\strut .} \colorbox{Green!0.000}{\strut  In} \colorbox{Green!0.000}{\strut  the} \colorbox{Green!0.000}{\strut  paper} \colorbox{Green!0.000}{\strut ,} \colorbox{Green!0.000}{\strut  we} \colorbox{Green!0.000}{\strut  propose} \colorbox{Green!0.000}{\strut  an} \\
Output SAE & \num{4.546e+00} & \colorbox{Magenta!0.000}{\strut  cost} \colorbox{Magenta!0.000}{\strut  of} \colorbox{Magenta!0.000}{\strut  amount} \colorbox{Magenta!0.000}{\strut  of} \colorbox{Magenta!87.778}{\strut  un} \colorbox{Magenta!0.000}{\strut labeled} \colorbox{Magenta!0.000}{\strut  samples} \colorbox{Magenta!0.000}{\strut .} \colorbox{Magenta!0.000}{\strut  In} \colorbox{Magenta!0.000}{\strut  the} \colorbox{Magenta!0.000}{\strut  paper} \colorbox{Magenta!0.000}{\strut ,} \colorbox{Magenta!0.000}{\strut  we} \colorbox{Magenta!0.000}{\strut  propose} \colorbox{Magenta!0.000}{\strut  an} \\
\midrule
Jacobian & \num{2.612e-01} & \colorbox{Cyan!0.000}{\strut  are} \colorbox{Cyan!0.000}{\strut  comprised} \colorbox{Cyan!0.000}{\strut  of} \colorbox{Cyan!0.000}{\strut  carbon} \colorbox{Cyan!0.000}{\strut  fiber} \colorbox{Cyan!0.000}{\strut  bra} \colorbox{Cyan!0.000}{\strut id} \colorbox{Cyan!0.000}{\strut  and} \colorbox{Cyan!98.211}{\strut  un} \colorbox{Cyan!0.000}{\strut id} \colorbox{Cyan!0.000}{\strut irectional} \colorbox{Cyan!0.000}{\strut  fabrics} \colorbox{Cyan!0.000}{\strut .} \colorbox{Cyan!0.000}{\strut  They} \colorbox{Cyan!0.000}{\strut  offer} \\
Input SAE & \num{1.205e+01} & \colorbox{Green!0.000}{\strut  are} \colorbox{Green!0.000}{\strut  comprised} \colorbox{Green!0.000}{\strut  of} \colorbox{Green!0.000}{\strut  carbon} \colorbox{Green!0.000}{\strut  fiber} \colorbox{Green!0.000}{\strut  bra} \colorbox{Green!0.000}{\strut id} \colorbox{Green!0.000}{\strut  and} \colorbox{Green!61.984}{\strut  un} \colorbox{Green!0.000}{\strut id} \colorbox{Green!0.000}{\strut irectional} \colorbox{Green!0.000}{\strut  fabrics} \colorbox{Green!0.000}{\strut .} \colorbox{Green!0.000}{\strut  They} \colorbox{Green!0.000}{\strut  offer} \\
Output SAE & \num{4.918e+00} & \colorbox{Magenta!0.000}{\strut  are} \colorbox{Magenta!0.000}{\strut  comprised} \colorbox{Magenta!0.000}{\strut  of} \colorbox{Magenta!0.000}{\strut  carbon} \colorbox{Magenta!0.000}{\strut  fiber} \colorbox{Magenta!0.000}{\strut  bra} \colorbox{Magenta!0.000}{\strut id} \colorbox{Magenta!0.000}{\strut  and} \colorbox{Magenta!94.960}{\strut  un} \colorbox{Magenta!0.000}{\strut id} \colorbox{Magenta!0.000}{\strut irectional} \colorbox{Magenta!0.000}{\strut  fabrics} \colorbox{Magenta!0.000}{\strut .} \colorbox{Magenta!0.000}{\strut  They} \colorbox{Magenta!0.000}{\strut  offer} \\
\midrule
Jacobian & \num{2.608e-01} & \colorbox{Cyan!0.000}{\strut ank} \colorbox{Cyan!0.000}{\strut  is} \colorbox{Cyan!0.000}{\strut  of} \colorbox{Cyan!0.000}{\strut  different} \colorbox{Cyan!0.000}{\strut  color} \colorbox{Cyan!0.000}{\strut  and} \colorbox{Cyan!0.000}{\strut  texture} \colorbox{Cyan!0.000}{\strut ,} \colorbox{Cyan!0.000}{\strut  the} \colorbox{Cyan!0.000}{\strut  floors} \colorbox{Cyan!0.000}{\strut  made} \colorbox{Cyan!0.000}{\strut  with} \colorbox{Cyan!98.064}{\strut  un} \colorbox{Cyan!0.000}{\strut processed} \colorbox{Cyan!0.000}{\strut  wood} \\
Input SAE & \num{1.078e+01} & \colorbox{Green!0.000}{\strut ank} \colorbox{Green!0.000}{\strut  is} \colorbox{Green!0.000}{\strut  of} \colorbox{Green!0.000}{\strut  different} \colorbox{Green!0.000}{\strut  color} \colorbox{Green!0.000}{\strut  and} \colorbox{Green!0.000}{\strut  texture} \colorbox{Green!0.000}{\strut ,} \colorbox{Green!0.000}{\strut  the} \colorbox{Green!0.000}{\strut  floors} \colorbox{Green!0.000}{\strut  made} \colorbox{Green!0.000}{\strut  with} \colorbox{Green!55.439}{\strut  un} \colorbox{Green!0.000}{\strut processed} \colorbox{Green!0.000}{\strut  wood} \\
Output SAE & \num{4.713e+00} & \colorbox{Magenta!0.000}{\strut ank} \colorbox{Magenta!0.000}{\strut  is} \colorbox{Magenta!0.000}{\strut  of} \colorbox{Magenta!0.000}{\strut  different} \colorbox{Magenta!0.000}{\strut  color} \colorbox{Magenta!0.000}{\strut  and} \colorbox{Magenta!0.000}{\strut  texture} \colorbox{Magenta!0.000}{\strut ,} \colorbox{Magenta!0.000}{\strut  the} \colorbox{Magenta!0.000}{\strut  floors} \colorbox{Magenta!0.000}{\strut  made} \colorbox{Magenta!0.000}{\strut  with} \colorbox{Magenta!90.991}{\strut  un} \colorbox{Magenta!0.000}{\strut processed} \colorbox{Magenta!0.000}{\strut  wood} \\
\midrule
Jacobian & \num{2.608e-01} & \colorbox{Cyan!0.000}{\strut  and} \colorbox{Cyan!0.000}{\strut  overseas} \colorbox{Cyan!0.000}{\strut ,} \colorbox{Cyan!0.000}{\strut  as} \colorbox{Cyan!0.000}{\strut  well} \colorbox{Cyan!0.000}{\strut  as} \colorbox{Cyan!0.000}{\strut  on} \colorbox{Cyan!0.000}{\strut  a} \colorbox{Cyan!0.000}{\strut  range} \colorbox{Cyan!0.000}{\strut  of} \colorbox{Cyan!98.054}{\strut  un} \colorbox{Cyan!0.000}{\strut structured} \colorbox{Cyan!0.000}{\strut  interviews} \colorbox{Cyan!0.000}{\strut  with} \colorbox{Cyan!0.000}{\strut  LE} \\
Input SAE & \num{1.113e+01} & \colorbox{Green!0.000}{\strut  and} \colorbox{Green!0.000}{\strut  overseas} \colorbox{Green!0.000}{\strut ,} \colorbox{Green!0.000}{\strut  as} \colorbox{Green!0.000}{\strut  well} \colorbox{Green!0.000}{\strut  as} \colorbox{Green!0.000}{\strut  on} \colorbox{Green!0.000}{\strut  a} \colorbox{Green!0.000}{\strut  range} \colorbox{Green!0.000}{\strut  of} \colorbox{Green!57.214}{\strut  un} \colorbox{Green!0.000}{\strut structured} \colorbox{Green!0.000}{\strut  interviews} \colorbox{Green!0.000}{\strut  with} \colorbox{Green!0.000}{\strut  LE} \\
Output SAE & \num{5.179e+00} & \colorbox{Magenta!0.000}{\strut  and} \colorbox{Magenta!0.000}{\strut  overseas} \colorbox{Magenta!0.000}{\strut ,} \colorbox{Magenta!0.000}{\strut  as} \colorbox{Magenta!0.000}{\strut  well} \colorbox{Magenta!0.000}{\strut  as} \colorbox{Magenta!0.000}{\strut  on} \colorbox{Magenta!0.000}{\strut  a} \colorbox{Magenta!0.000}{\strut  range} \colorbox{Magenta!0.000}{\strut  of} \colorbox{Magenta!99.988}{\strut  un} \colorbox{Magenta!0.000}{\strut structured} \colorbox{Magenta!0.000}{\strut  interviews} \colorbox{Magenta!0.000}{\strut  with} \colorbox{Magenta!0.000}{\strut  LE} \\
\midrule
Jacobian & \num{2.608e-01} & \colorbox{Cyan!0.000}{\strut  include} \colorbox{Cyan!0.000}{\strut  space} \colorbox{Cyan!0.000}{\strut  for} \colorbox{Cyan!98.046}{\strut  un} \colorbox{Cyan!0.000}{\strut loading} \colorbox{Cyan!0.000}{\strut .} \colorbox{Cyan!0.000}{\strut Right} \colorbox{Cyan!0.000}{\strut  now} \colorbox{Cyan!0.000}{\strut  the} \colorbox{Cyan!0.000}{\strut  firm} \colorbox{Cyan!0.000}{\strut  is} \colorbox{Cyan!0.000}{\strut  weighing} \colorbox{Cyan!0.000}{\strut  the} \colorbox{Cyan!0.000}{\strut  benefits} \\
Input SAE & \num{1.174e+01} & \colorbox{Green!0.000}{\strut  include} \colorbox{Green!0.000}{\strut  space} \colorbox{Green!0.000}{\strut  for} \colorbox{Green!60.373}{\strut  un} \colorbox{Green!0.000}{\strut loading} \colorbox{Green!0.000}{\strut .} \colorbox{Green!0.000}{\strut Right} \colorbox{Green!0.000}{\strut  now} \colorbox{Green!0.000}{\strut  the} \colorbox{Green!0.000}{\strut  firm} \colorbox{Green!0.000}{\strut  is} \colorbox{Green!0.000}{\strut  weighing} \colorbox{Green!0.000}{\strut  the} \colorbox{Green!0.000}{\strut  benefits} \\
Output SAE & \num{4.795e+00} & \colorbox{Magenta!0.000}{\strut  include} \colorbox{Magenta!0.000}{\strut  space} \colorbox{Magenta!0.000}{\strut  for} \colorbox{Magenta!92.574}{\strut  un} \colorbox{Magenta!0.000}{\strut loading} \colorbox{Magenta!0.000}{\strut .} \colorbox{Magenta!0.000}{\strut Right} \colorbox{Magenta!0.000}{\strut  now} \colorbox{Magenta!0.000}{\strut  the} \colorbox{Magenta!0.000}{\strut  firm} \colorbox{Magenta!0.000}{\strut  is} \colorbox{Magenta!0.000}{\strut  weighing} \colorbox{Magenta!0.000}{\strut  the} \colorbox{Magenta!0.000}{\strut  benefits} \\
\bottomrule
\end{longtable}
\caption{feature pairs/Layer15-65536-J1-LR5.0e-04-k32-T3.0e+08 abs mean/examples-17539-v-49956 stas c4-en-10k,train,batch size=32,ctx len=16.csv}
\end{table}
\begin{table}
\centering
\begin{tabular}{lrl}
\toprule
Category & Max. abs. value & Example tokens \\
\midrule
Jacobian & \num{2.530e-01} & \colorbox{Cyan!0.000}{\strut  they} \colorbox{Cyan!0.000}{\strut  are} \colorbox{Cyan!0.000}{\strut  able} \colorbox{Cyan!0.000}{\strut  to} \colorbox{Cyan!0.000}{\strut  duplicate} \colorbox{Cyan!0.000}{\strut  this} \colorbox{Cyan!0.000}{\strut  to} \colorbox{Cyan!0.000}{\strut  maneuver} \colorbox{Cyan!0.000}{\strut  upon} \colorbox{Cyan!0.000}{\strut .} \colorbox{Cyan!94.996}{\strut Non} \colorbox{Cyan!100.000}{\strut -} \colorbox{Cyan!0.000}{\strut Member} \colorbox{Cyan!0.000}{\strut  \$} \colorbox{Cyan!0.000}{\strut 1} \\
Input SAE & \num{1.659e+01} & \colorbox{Green!0.000}{\strut  they} \colorbox{Green!0.000}{\strut  are} \colorbox{Green!0.000}{\strut  able} \colorbox{Green!0.000}{\strut  to} \colorbox{Green!0.000}{\strut  duplicate} \colorbox{Green!0.000}{\strut  this} \colorbox{Green!0.000}{\strut  to} \colorbox{Green!0.000}{\strut  maneuver} \colorbox{Green!0.000}{\strut  upon} \colorbox{Green!0.000}{\strut .} \colorbox{Green!83.956}{\strut Non} \colorbox{Green!58.519}{\strut -} \colorbox{Green!0.000}{\strut Member} \colorbox{Green!0.000}{\strut  \$} \colorbox{Green!0.000}{\strut 1} \\
Output SAE & \num{4.136e+00} & \colorbox{Magenta!0.000}{\strut  they} \colorbox{Magenta!0.000}{\strut  are} \colorbox{Magenta!0.000}{\strut  able} \colorbox{Magenta!0.000}{\strut  to} \colorbox{Magenta!0.000}{\strut  duplicate} \colorbox{Magenta!0.000}{\strut  this} \colorbox{Magenta!0.000}{\strut  to} \colorbox{Magenta!0.000}{\strut  maneuver} \colorbox{Magenta!0.000}{\strut  upon} \colorbox{Magenta!0.000}{\strut .} \colorbox{Magenta!81.488}{\strut Non} \colorbox{Magenta!86.662}{\strut -} \colorbox{Magenta!0.000}{\strut Member} \colorbox{Magenta!0.000}{\strut  \$} \colorbox{Magenta!0.000}{\strut 1} \\
\midrule
Jacobian & \num{2.530e-01} & \colorbox{Cyan!0.000}{\strut  evaluates} \colorbox{Cyan!0.000}{\strut  financial} \colorbox{Cyan!0.000}{\strut  information} \colorbox{Cyan!0.000}{\strut  for} \colorbox{Cyan!0.000}{\strut  two} \colorbox{Cyan!0.000}{\strut  segments} \colorbox{Cyan!0.000}{\strut :} \colorbox{Cyan!0.000}{\strut  Se} \colorbox{Cyan!0.000}{\strut ismic} \colorbox{Cyan!0.000}{\strut  and} \colorbox{Cyan!95.492}{\strut  Non} \colorbox{Cyan!99.976}{\strut -} \colorbox{Cyan!0.000}{\strut Se} \colorbox{Cyan!0.000}{\strut ismic} \colorbox{Cyan!0.000}{\strut .} \\
Input SAE & \num{1.655e+01} & \colorbox{Green!0.000}{\strut  evaluates} \colorbox{Green!0.000}{\strut  financial} \colorbox{Green!0.000}{\strut  information} \colorbox{Green!0.000}{\strut  for} \colorbox{Green!0.000}{\strut  two} \colorbox{Green!0.000}{\strut  segments} \colorbox{Green!0.000}{\strut :} \colorbox{Green!0.000}{\strut  Se} \colorbox{Green!0.000}{\strut ismic} \colorbox{Green!0.000}{\strut  and} \colorbox{Green!83.728}{\strut  Non} \colorbox{Green!47.285}{\strut -} \colorbox{Green!0.000}{\strut Se} \colorbox{Green!0.000}{\strut ismic} \colorbox{Green!0.000}{\strut .} \\
Output SAE & \num{4.124e+00} & \colorbox{Magenta!0.000}{\strut  evaluates} \colorbox{Magenta!0.000}{\strut  financial} \colorbox{Magenta!0.000}{\strut  information} \colorbox{Magenta!0.000}{\strut  for} \colorbox{Magenta!0.000}{\strut  two} \colorbox{Magenta!0.000}{\strut  segments} \colorbox{Magenta!0.000}{\strut :} \colorbox{Magenta!0.000}{\strut  Se} \colorbox{Magenta!0.000}{\strut ismic} \colorbox{Magenta!0.000}{\strut  and} \colorbox{Magenta!86.421}{\strut  Non} \colorbox{Magenta!82.191}{\strut -} \colorbox{Magenta!11.772}{\strut Se} \colorbox{Magenta!0.000}{\strut ismic} \colorbox{Magenta!0.000}{\strut .} \\
\midrule
Jacobian & \num{2.508e-01} & \colorbox{Cyan!0.000}{\strut .} \colorbox{Cyan!0.000}{\strut  Most} \colorbox{Cyan!0.000}{\strut  patients} \colorbox{Cyan!0.000}{\strut  go} \colorbox{Cyan!0.000}{\strut  back} \colorbox{Cyan!0.000}{\strut  to} \colorbox{Cyan!0.000}{\strut  work} \colorbox{Cyan!0.000}{\strut  the} \colorbox{Cyan!0.000}{\strut  next} \colorbox{Cyan!0.000}{\strut  day} \colorbox{Cyan!0.000}{\strut .} \colorbox{Cyan!95.108}{\strut  Non} \colorbox{Cyan!99.133}{\strut -} \colorbox{Cyan!0.000}{\strut surgical} \colorbox{Cyan!0.000}{\strut  chin} \\
Input SAE & \num{1.835e+01} & \colorbox{Green!0.000}{\strut .} \colorbox{Green!0.000}{\strut  Most} \colorbox{Green!0.000}{\strut  patients} \colorbox{Green!0.000}{\strut  go} \colorbox{Green!0.000}{\strut  back} \colorbox{Green!0.000}{\strut  to} \colorbox{Green!0.000}{\strut  work} \colorbox{Green!0.000}{\strut  the} \colorbox{Green!0.000}{\strut  next} \colorbox{Green!0.000}{\strut  day} \colorbox{Green!0.000}{\strut .} \colorbox{Green!92.851}{\strut  Non} \colorbox{Green!58.206}{\strut -} \colorbox{Green!0.000}{\strut surgical} \colorbox{Green!0.000}{\strut  chin} \\
Output SAE & \num{4.342e+00} & \colorbox{Magenta!0.000}{\strut .} \colorbox{Magenta!0.000}{\strut  Most} \colorbox{Magenta!0.000}{\strut  patients} \colorbox{Magenta!0.000}{\strut  go} \colorbox{Magenta!0.000}{\strut  back} \colorbox{Magenta!0.000}{\strut  to} \colorbox{Magenta!0.000}{\strut  work} \colorbox{Magenta!0.000}{\strut  the} \colorbox{Magenta!0.000}{\strut  next} \colorbox{Magenta!0.000}{\strut  day} \colorbox{Magenta!0.000}{\strut .} \colorbox{Magenta!90.989}{\strut  Non} \colorbox{Magenta!85.596}{\strut -} \colorbox{Magenta!0.000}{\strut surgical} \colorbox{Magenta!0.000}{\strut  chin} \\
\midrule
Jacobian & \num{2.507e-01} & \colorbox{Cyan!0.000}{\strut /} \colorbox{Cyan!94.116}{\strut Non} \colorbox{Cyan!99.096}{\strut -} \colorbox{Cyan!0.000}{\strut standard} \colorbox{Cyan!0.000}{\strut  Fil} \colorbox{Cyan!0.000}{\strut ters} \colorbox{Cyan!0.000}{\strut \textquotedbl{}} \colorbox{Cyan!0.000}{\strut  option} \colorbox{Cyan!0.000}{\strut  so} \colorbox{Cyan!0.000}{\strut  I} \colorbox{Cyan!0.000}{\strut  guess} \colorbox{Cyan!0.000}{\strut  I} \colorbox{Cyan!0.000}{\strut  could} \colorbox{Cyan!0.000}{\strut  use} \colorbox{Cyan!0.000}{\strut  that} \\
Input SAE & \num{1.602e+01} & \colorbox{Green!0.000}{\strut /} \colorbox{Green!81.092}{\strut Non} \colorbox{Green!50.566}{\strut -} \colorbox{Green!0.000}{\strut standard} \colorbox{Green!0.000}{\strut  Fil} \colorbox{Green!0.000}{\strut ters} \colorbox{Green!0.000}{\strut \textquotedbl{}} \colorbox{Green!0.000}{\strut  option} \colorbox{Green!0.000}{\strut  so} \colorbox{Green!0.000}{\strut  I} \colorbox{Green!0.000}{\strut  guess} \colorbox{Green!0.000}{\strut  I} \colorbox{Green!0.000}{\strut  could} \colorbox{Green!0.000}{\strut  use} \colorbox{Green!0.000}{\strut  that} \\
Output SAE & \num{3.794e+00} & \colorbox{Magenta!0.000}{\strut /} \colorbox{Magenta!79.503}{\strut Non} \colorbox{Magenta!77.577}{\strut -} \colorbox{Magenta!0.000}{\strut standard} \colorbox{Magenta!0.000}{\strut  Fil} \colorbox{Magenta!0.000}{\strut ters} \colorbox{Magenta!0.000}{\strut \textquotedbl{}} \colorbox{Magenta!0.000}{\strut  option} \colorbox{Magenta!0.000}{\strut  so} \colorbox{Magenta!0.000}{\strut  I} \colorbox{Magenta!0.000}{\strut  guess} \colorbox{Magenta!0.000}{\strut  I} \colorbox{Magenta!0.000}{\strut  could} \colorbox{Magenta!0.000}{\strut  use} \colorbox{Magenta!0.000}{\strut  that} \\
\midrule
Jacobian & \num{2.507e-01} & \colorbox{Cyan!0.000}{\strut  out} \colorbox{Cyan!0.000}{\strut  facial} \colorbox{Cyan!0.000}{\strut  and} \colorbox{Cyan!0.000}{\strut  neck} \colorbox{Cyan!0.000}{\strut  wr} \colorbox{Cyan!0.000}{\strut inkles} \colorbox{Cyan!0.000}{\strut ,} \colorbox{Cyan!0.000}{\strut  the} \colorbox{Cyan!0.000}{\strut  patient} \colorbox{Cyan!0.000}{\strut  underwent} \colorbox{Cyan!0.000}{\strut  Inf} \colorbox{Cyan!0.000}{\strut ini} \colorbox{Cyan!94.826}{\strut  non} \colorbox{Cyan!99.076}{\strut -} \colorbox{Cyan!0.000}{\strut invasive} \\
Input SAE & \num{1.775e+01} & \colorbox{Green!0.000}{\strut  out} \colorbox{Green!0.000}{\strut  facial} \colorbox{Green!0.000}{\strut  and} \colorbox{Green!0.000}{\strut  neck} \colorbox{Green!0.000}{\strut  wr} \colorbox{Green!0.000}{\strut inkles} \colorbox{Green!0.000}{\strut ,} \colorbox{Green!0.000}{\strut  the} \colorbox{Green!0.000}{\strut  patient} \colorbox{Green!0.000}{\strut  underwent} \colorbox{Green!0.000}{\strut  Inf} \colorbox{Green!0.000}{\strut ini} \colorbox{Green!89.800}{\strut  non} \colorbox{Green!57.285}{\strut -} \colorbox{Green!0.000}{\strut invasive} \\
Output SAE & \num{4.175e+00} & \colorbox{Magenta!0.000}{\strut  out} \colorbox{Magenta!0.000}{\strut  facial} \colorbox{Magenta!0.000}{\strut  and} \colorbox{Magenta!0.000}{\strut  neck} \colorbox{Magenta!0.000}{\strut  wr} \colorbox{Magenta!0.000}{\strut inkles} \colorbox{Magenta!0.000}{\strut ,} \colorbox{Magenta!0.000}{\strut  the} \colorbox{Magenta!0.000}{\strut  patient} \colorbox{Magenta!0.000}{\strut  underwent} \colorbox{Magenta!0.000}{\strut  Inf} \colorbox{Magenta!0.000}{\strut ini} \colorbox{Magenta!86.940}{\strut  non} \colorbox{Magenta!87.490}{\strut -} \colorbox{Magenta!0.000}{\strut invasive} \\
\midrule
Jacobian & \num{2.507e-01} & \colorbox{Cyan!0.000}{\strut  D} \colorbox{Cyan!0.000}{\strut err} \colorbox{Cyan!0.000}{\strut ata} \colorbox{Cyan!0.000}{\strut  2010} \colorbox{Cyan!0.000}{\strut \_} \colorbox{Cyan!0.000}{\strut Q} \colorbox{Cyan!0.000}{\strut 3} \colorbox{Cyan!88.905}{\strut  Non} \colorbox{Cyan!99.071}{\strut -} \colorbox{Cyan!0.000}{\strut conf} \colorbox{Cyan!0.000}{\strut idential} \colorbox{Cyan!0.000}{\strut  Mark} \colorbox{Cyan!0.000}{\strut ed} \colorbox{Cyan!0.000}{\strut -} \colorbox{Cyan!0.000}{\strut up} \\
Input SAE & \num{1.686e+01} & \colorbox{Green!0.000}{\strut  D} \colorbox{Green!0.000}{\strut err} \colorbox{Green!0.000}{\strut ata} \colorbox{Green!0.000}{\strut  2010} \colorbox{Green!0.000}{\strut \_} \colorbox{Green!0.000}{\strut Q} \colorbox{Green!0.000}{\strut 3} \colorbox{Green!85.330}{\strut  Non} \colorbox{Green!62.206}{\strut -} \colorbox{Green!0.000}{\strut conf} \colorbox{Green!0.000}{\strut idential} \colorbox{Green!0.000}{\strut  Mark} \colorbox{Green!0.000}{\strut ed} \colorbox{Green!0.000}{\strut -} \colorbox{Green!0.000}{\strut up} \\
Output SAE & \num{4.192e+00} & \colorbox{Magenta!0.000}{\strut  D} \colorbox{Magenta!0.000}{\strut err} \colorbox{Magenta!0.000}{\strut ata} \colorbox{Magenta!0.000}{\strut  2010} \colorbox{Magenta!0.000}{\strut \_} \colorbox{Magenta!0.000}{\strut Q} \colorbox{Magenta!0.000}{\strut 3} \colorbox{Magenta!76.893}{\strut  Non} \colorbox{Magenta!87.848}{\strut -} \colorbox{Magenta!0.000}{\strut conf} \colorbox{Magenta!0.000}{\strut idential} \colorbox{Magenta!0.000}{\strut  Mark} \colorbox{Magenta!0.000}{\strut ed} \colorbox{Magenta!0.000}{\strut -} \colorbox{Magenta!0.000}{\strut up} \\
\midrule
Jacobian & \num{2.502e-01} & \colorbox{Cyan!0.000}{\strut  weeks} \colorbox{Cyan!0.000}{\strut ,} \colorbox{Cyan!0.000}{\strut  vigorous} \colorbox{Cyan!0.000}{\strut  physical} \colorbox{Cyan!0.000}{\strut  activity} \colorbox{Cyan!0.000}{\strut  can} \colorbox{Cyan!0.000}{\strut  be} \colorbox{Cyan!0.000}{\strut  resumed} \colorbox{Cyan!0.000}{\strut .} \colorbox{Cyan!93.876}{\strut Non} \colorbox{Cyan!98.880}{\strut -} \colorbox{Cyan!0.000}{\strut S} \colorbox{Cyan!0.000}{\strut urgical} \colorbox{Cyan!0.000}{\strut  Chin} \\
Input SAE & \num{1.822e+01} & \colorbox{Green!0.000}{\strut  weeks} \colorbox{Green!0.000}{\strut ,} \colorbox{Green!0.000}{\strut  vigorous} \colorbox{Green!0.000}{\strut  physical} \colorbox{Green!0.000}{\strut  activity} \colorbox{Green!0.000}{\strut  can} \colorbox{Green!0.000}{\strut  be} \colorbox{Green!0.000}{\strut  resumed} \colorbox{Green!0.000}{\strut .} \colorbox{Green!92.185}{\strut Non} \colorbox{Green!61.734}{\strut -} \colorbox{Green!0.000}{\strut S} \colorbox{Green!0.000}{\strut urgical} \colorbox{Green!0.000}{\strut  Chin} \\
Output SAE & \num{4.394e+00} & \colorbox{Magenta!0.000}{\strut  weeks} \colorbox{Magenta!0.000}{\strut ,} \colorbox{Magenta!0.000}{\strut  vigorous} \colorbox{Magenta!0.000}{\strut  physical} \colorbox{Magenta!0.000}{\strut  activity} \colorbox{Magenta!0.000}{\strut  can} \colorbox{Magenta!0.000}{\strut  be} \colorbox{Magenta!0.000}{\strut  resumed} \colorbox{Magenta!0.000}{\strut .} \colorbox{Magenta!92.067}{\strut Non} \colorbox{Magenta!91.220}{\strut -} \colorbox{Magenta!0.000}{\strut S} \colorbox{Magenta!0.000}{\strut urgical} \colorbox{Magenta!0.000}{\strut  Chin} \\
\midrule
Jacobian & \num{2.501e-01} & \colorbox{Cyan!0.000}{\strut  smooth} \colorbox{Cyan!0.000}{\strut  out} \colorbox{Cyan!0.000}{\strut  facial} \colorbox{Cyan!0.000}{\strut  wr} \colorbox{Cyan!0.000}{\strut inkles} \colorbox{Cyan!0.000}{\strut  ,} \colorbox{Cyan!0.000}{\strut  the} \colorbox{Cyan!0.000}{\strut  patient} \colorbox{Cyan!0.000}{\strut  underwent} \colorbox{Cyan!0.000}{\strut  Inf} \colorbox{Cyan!0.000}{\strut ini} \colorbox{Cyan!94.978}{\strut  non} \colorbox{Cyan!98.827}{\strut -} \colorbox{Cyan!0.000}{\strut invasive} \colorbox{Cyan!0.000}{\strut  (} \\
Input SAE & \num{1.748e+01} & \colorbox{Green!0.000}{\strut  smooth} \colorbox{Green!0.000}{\strut  out} \colorbox{Green!0.000}{\strut  facial} \colorbox{Green!0.000}{\strut  wr} \colorbox{Green!0.000}{\strut inkles} \colorbox{Green!0.000}{\strut  ,} \colorbox{Green!0.000}{\strut  the} \colorbox{Green!0.000}{\strut  patient} \colorbox{Green!0.000}{\strut  underwent} \colorbox{Green!0.000}{\strut  Inf} \colorbox{Green!0.000}{\strut ini} \colorbox{Green!88.475}{\strut  non} \colorbox{Green!55.972}{\strut -} \colorbox{Green!0.000}{\strut invasive} \colorbox{Green!0.000}{\strut  (} \\
Output SAE & \num{4.139e+00} & \colorbox{Magenta!0.000}{\strut  smooth} \colorbox{Magenta!0.000}{\strut  out} \colorbox{Magenta!0.000}{\strut  facial} \colorbox{Magenta!0.000}{\strut  wr} \colorbox{Magenta!0.000}{\strut inkles} \colorbox{Magenta!0.000}{\strut  ,} \colorbox{Magenta!0.000}{\strut  the} \colorbox{Magenta!0.000}{\strut  patient} \colorbox{Magenta!0.000}{\strut  underwent} \colorbox{Magenta!0.000}{\strut  Inf} \colorbox{Magenta!0.000}{\strut ini} \colorbox{Magenta!86.726}{\strut  non} \colorbox{Magenta!85.160}{\strut -} \colorbox{Magenta!0.000}{\strut invasive} \colorbox{Magenta!0.000}{\strut  (} \\
\midrule
Jacobian & \num{2.500e-01} & \colorbox{Cyan!0.000}{\strut  those} \colorbox{Cyan!0.000}{\strut  outcomes} \colorbox{Cyan!0.000}{\strut  imply} \colorbox{Cyan!0.000}{\strut  for} \colorbox{Cyan!0.000}{\strut  the} \colorbox{Cyan!0.000}{\strut  future} \colorbox{Cyan!0.000}{\strut .} \colorbox{Cyan!91.278}{\strut Non} \colorbox{Cyan!98.798}{\strut -} \colorbox{Cyan!0.000}{\strut reactive} \colorbox{Cyan!0.000}{\strut --} \colorbox{Cyan!0.000}{\strut  refers} \colorbox{Cyan!0.000}{\strut  to} \colorbox{Cyan!0.000}{\strut  an} \\
Input SAE & \num{1.765e+01} & \colorbox{Green!0.000}{\strut  those} \colorbox{Green!0.000}{\strut  outcomes} \colorbox{Green!0.000}{\strut  imply} \colorbox{Green!0.000}{\strut  for} \colorbox{Green!0.000}{\strut  the} \colorbox{Green!0.000}{\strut  future} \colorbox{Green!0.000}{\strut .} \colorbox{Green!89.308}{\strut Non} \colorbox{Green!59.498}{\strut -} \colorbox{Green!0.000}{\strut reactive} \colorbox{Green!0.000}{\strut --} \colorbox{Green!0.000}{\strut  refers} \colorbox{Green!0.000}{\strut  to} \colorbox{Green!0.000}{\strut  an} \\
Output SAE & \num{4.052e+00} & \colorbox{Magenta!0.000}{\strut  those} \colorbox{Magenta!0.000}{\strut  outcomes} \colorbox{Magenta!0.000}{\strut  imply} \colorbox{Magenta!0.000}{\strut  for} \colorbox{Magenta!0.000}{\strut  the} \colorbox{Magenta!0.000}{\strut  future} \colorbox{Magenta!0.000}{\strut .} \colorbox{Magenta!84.913}{\strut Non} \colorbox{Magenta!84.428}{\strut -} \colorbox{Magenta!0.000}{\strut reactive} \colorbox{Magenta!0.000}{\strut --} \colorbox{Magenta!0.000}{\strut  refers} \colorbox{Magenta!0.000}{\strut  to} \colorbox{Magenta!0.000}{\strut  an} \\
\midrule
Jacobian & \num{2.493e-01} & \colorbox{Cyan!0.000}{\strut \textquotedbl{}} \colorbox{Cyan!0.000}{\strut  fall} \colorbox{Cyan!0.000}{\strut  protection} \colorbox{Cyan!0.000}{\strut  for} \colorbox{Cyan!0.000}{\strut  the} \colorbox{Cyan!0.000}{\strut  most} \colorbox{Cyan!0.000}{\strut  rigorous} \colorbox{Cyan!0.000}{\strut  work} \colorbox{Cyan!0.000}{\strut  environments} \colorbox{Cyan!0.000}{\strut .} \colorbox{Cyan!93.927}{\strut  Non} \colorbox{Cyan!98.516}{\strut -} \colorbox{Cyan!0.000}{\strut sl} \colorbox{Cyan!0.000}{\strut ip} \colorbox{Cyan!0.000}{\strut  chest} \\
Input SAE & \num{1.830e+01} & \colorbox{Green!0.000}{\strut \textquotedbl{}} \colorbox{Green!0.000}{\strut  fall} \colorbox{Green!0.000}{\strut  protection} \colorbox{Green!0.000}{\strut  for} \colorbox{Green!0.000}{\strut  the} \colorbox{Green!0.000}{\strut  most} \colorbox{Green!0.000}{\strut  rigorous} \colorbox{Green!0.000}{\strut  work} \colorbox{Green!0.000}{\strut  environments} \colorbox{Green!0.000}{\strut .} \colorbox{Green!92.616}{\strut  Non} \colorbox{Green!59.796}{\strut -} \colorbox{Green!0.000}{\strut sl} \colorbox{Green!0.000}{\strut ip} \colorbox{Green!0.000}{\strut  chest} \\
Output SAE & \num{4.290e+00} & \colorbox{Magenta!0.000}{\strut \textquotedbl{}} \colorbox{Magenta!0.000}{\strut  fall} \colorbox{Magenta!0.000}{\strut  protection} \colorbox{Magenta!0.000}{\strut  for} \colorbox{Magenta!0.000}{\strut  the} \colorbox{Magenta!0.000}{\strut  most} \colorbox{Magenta!0.000}{\strut  rigorous} \colorbox{Magenta!0.000}{\strut  work} \colorbox{Magenta!0.000}{\strut  environments} \colorbox{Magenta!0.000}{\strut .} \colorbox{Magenta!89.901}{\strut  Non} \colorbox{Magenta!85.585}{\strut -} \colorbox{Magenta!0.000}{\strut sl} \colorbox{Magenta!0.000}{\strut ip} \colorbox{Magenta!0.000}{\strut  chest} \\
\midrule
Jacobian & \num{2.491e-01} & \colorbox{Cyan!0.000}{\strut  basis} \colorbox{Cyan!0.000}{\strut .} \colorbox{Cyan!90.989}{\strut  Non} \colorbox{Cyan!98.463}{\strut -} \colorbox{Cyan!0.000}{\strut profit} \colorbox{Cyan!0.000}{\strut  is} \colorbox{Cyan!0.000}{\strut  the} \colorbox{Cyan!0.000}{\strut  only} \colorbox{Cyan!0.000}{\strut  exception} \colorbox{Cyan!0.000}{\strut  to} \colorbox{Cyan!0.000}{\strut  these} \colorbox{Cyan!0.000}{\strut  terms} \colorbox{Cyan!0.000}{\strut  of} \colorbox{Cyan!0.000}{\strut  agreement} \colorbox{Cyan!0.000}{\strut  you} \\
Input SAE & \num{1.665e+01} & \colorbox{Green!0.000}{\strut  basis} \colorbox{Green!0.000}{\strut .} \colorbox{Green!84.279}{\strut  Non} \colorbox{Green!58.681}{\strut -} \colorbox{Green!0.000}{\strut profit} \colorbox{Green!0.000}{\strut  is} \colorbox{Green!0.000}{\strut  the} \colorbox{Green!0.000}{\strut  only} \colorbox{Green!0.000}{\strut  exception} \colorbox{Green!0.000}{\strut  to} \colorbox{Green!0.000}{\strut  these} \colorbox{Green!0.000}{\strut  terms} \colorbox{Green!0.000}{\strut  of} \colorbox{Green!0.000}{\strut  agreement} \colorbox{Green!0.000}{\strut  you} \\
Output SAE & \num{3.990e+00} & \colorbox{Magenta!0.000}{\strut  basis} \colorbox{Magenta!0.000}{\strut .} \colorbox{Magenta!80.512}{\strut  Non} \colorbox{Magenta!83.597}{\strut -} \colorbox{Magenta!0.000}{\strut profit} \colorbox{Magenta!0.000}{\strut  is} \colorbox{Magenta!0.000}{\strut  the} \colorbox{Magenta!0.000}{\strut  only} \colorbox{Magenta!0.000}{\strut  exception} \colorbox{Magenta!0.000}{\strut  to} \colorbox{Magenta!0.000}{\strut  these} \colorbox{Magenta!0.000}{\strut  terms} \colorbox{Magenta!0.000}{\strut  of} \colorbox{Magenta!0.000}{\strut  agreement} \colorbox{Magenta!0.000}{\strut  you} \\
\midrule
Jacobian & \num{2.489e-01} & \colorbox{Cyan!0.000}{\strut  OR} \colorbox{Cyan!0.000}{\strut GAN} \colorbox{Cyan!0.000}{\strut IC} \colorbox{Cyan!0.000}{\strut .} \colorbox{Cyan!97.294}{\strut  NON} \colorbox{Cyan!98.389}{\strut -} \colorbox{Cyan!0.000}{\strut G} \colorbox{Cyan!0.000}{\strut MO} \colorbox{Cyan!0.000}{\strut ,} \colorbox{Cyan!0.000}{\strut  C} \colorbox{Cyan!0.000}{\strut ERT} \colorbox{Cyan!0.000}{\strut IF} \colorbox{Cyan!0.000}{\strut IED} \colorbox{Cyan!0.000}{\strut  K} \colorbox{Cyan!0.000}{\strut OS} \\
Input SAE & \num{1.474e+01} & \colorbox{Green!0.000}{\strut  OR} \colorbox{Green!0.000}{\strut GAN} \colorbox{Green!0.000}{\strut IC} \colorbox{Green!0.000}{\strut .} \colorbox{Green!74.576}{\strut  NON} \colorbox{Green!51.847}{\strut -} \colorbox{Green!0.000}{\strut G} \colorbox{Green!0.000}{\strut MO} \colorbox{Green!0.000}{\strut ,} \colorbox{Green!0.000}{\strut  C} \colorbox{Green!0.000}{\strut ERT} \colorbox{Green!0.000}{\strut IF} \colorbox{Green!0.000}{\strut IED} \colorbox{Green!0.000}{\strut  K} \colorbox{Green!0.000}{\strut OS} \\
Output SAE & \num{3.587e+00} & \colorbox{Magenta!0.000}{\strut  OR} \colorbox{Magenta!0.000}{\strut GAN} \colorbox{Magenta!0.000}{\strut IC} \colorbox{Magenta!0.000}{\strut .} \colorbox{Magenta!75.160}{\strut  NON} \colorbox{Magenta!73.645}{\strut -} \colorbox{Magenta!0.000}{\strut G} \colorbox{Magenta!0.000}{\strut MO} \colorbox{Magenta!0.000}{\strut ,} \colorbox{Magenta!0.000}{\strut  C} \colorbox{Magenta!0.000}{\strut ERT} \colorbox{Magenta!0.000}{\strut IF} \colorbox{Magenta!0.000}{\strut IED} \colorbox{Magenta!0.000}{\strut  K} \colorbox{Magenta!0.000}{\strut OS} \\
\bottomrule
\end{tabular}
% feature pairs/Layer15-65536-J1-LR5.0e-04-k32-T3.0e+08 abs mean/examples-26438-v-54734 stas c4-en-10k,train,batch size=32,ctx len=16.csv
\caption{
The top $12$ examples that produce the maximum absolute values of the Jacobian element with input SAE latent index $26438$ and output latent index $54734$.
% This pair of latent indices is one of the top $5$ pairs by the mean absolute value of non-zero Jacobian elements.
The Jacobian SAE pair was trained on layer 15 of Pythia-410m with an expansion factor of $R=64$ and sparsity $k=32$.
The examples were collected over the first 10K records of the English subset of the C4 text dataset \citep{raffel_exploring_2020}, with a context length of $16$ tokens.
For each example, the first row shows the values of the Jacobian element, and the second and third show the corresponding activations of the input and output SAE latents.
In this case, both SAE latents appear to strongly activate for the pair of tokens `non-' (in upper- or lowercase).
}
\label{tab:feature_pairs_26438_54734}
\end{table} % N/non-
% \begin{table}
\centering
\begin{longtable}{lrl}
\toprule
Category & Max. abs. value & Example tokens \\
\midrule
Jacobian & \num{2.175e-01} & \colorbox{Cyan!0.000}{\strut  stylish} \colorbox{Cyan!0.000}{\strut  looks} \colorbox{Cyan!0.000}{\strut .} \colorbox{Cyan!0.000}{\strut Im} \colorbox{Cyan!0.000}{\strut pro} \colorbox{Cyan!0.000}{\strut ves} \colorbox{Cyan!0.000}{\strut  core} \colorbox{Cyan!0.000}{\strut  strength} \colorbox{Cyan!0.000}{\strut ,} \colorbox{Cyan!0.000}{\strut  functional} \colorbox{Cyan!0.000}{\strut  movement} \colorbox{Cyan!0.000}{\strut  and} \colorbox{Cyan!0.000}{\strut  flexibility} \colorbox{Cyan!0.000}{\strut .} \\
Input SAE & \num{1.430e+00} & \colorbox{Green!30.888}{\strut  stylish} \colorbox{Green!28.020}{\strut  looks} \colorbox{Green!0.000}{\strut .} \colorbox{Green!0.000}{\strut Im} \colorbox{Green!0.000}{\strut pro} \colorbox{Green!0.000}{\strut ves} \colorbox{Green!0.000}{\strut  core} \colorbox{Green!0.000}{\strut  strength} \colorbox{Green!0.000}{\strut ,} \colorbox{Green!44.316}{\strut  functional} \colorbox{Green!0.000}{\strut  movement} \colorbox{Green!0.000}{\strut  and} \colorbox{Green!41.641}{\strut  flexibility} \colorbox{Green!0.000}{\strut .} \\
Output SAE & \num{1.655e+00} & \colorbox{Magenta!0.000}{\strut  stylish} \colorbox{Magenta!0.000}{\strut  looks} \colorbox{Magenta!0.000}{\strut .} \colorbox{Magenta!0.000}{\strut Im} \colorbox{Magenta!0.000}{\strut pro} \colorbox{Magenta!0.000}{\strut ves} \colorbox{Magenta!0.000}{\strut  core} \colorbox{Magenta!0.000}{\strut  strength} \colorbox{Magenta!0.000}{\strut ,} \colorbox{Magenta!0.000}{\strut  functional} \colorbox{Magenta!0.000}{\strut  movement} \colorbox{Magenta!0.000}{\strut  and} \colorbox{Magenta!0.000}{\strut  flexibility} \colorbox{Magenta!0.000}{\strut .} \\
\midrule
Jacobian & \num{2.175e-01} & \colorbox{Cyan!0.000}{\strut  is} \colorbox{Cyan!0.000}{\strut  OK} \colorbox{Cyan!0.000}{\strut  .} \colorbox{Cyan!0.000}{\strut S} \colorbox{Cyan!0.000}{\strut ony} \colorbox{Cyan!0.000}{\strut  makes} \colorbox{Cyan!0.000}{\strut  a} \colorbox{Cyan!0.000}{\strut  really} \colorbox{Cyan!0.000}{\strut  great} \colorbox{Cyan!0.000}{\strut  product} \colorbox{Cyan!0.000}{\strut  ,} \colorbox{Cyan!0.000}{\strut set} \colorbox{Cyan!0.000}{\strut  up} \colorbox{Cyan!0.000}{\strut  is} \\
Input SAE & \num{2.161e-01} & \colorbox{Green!0.000}{\strut  is} \colorbox{Green!0.000}{\strut  OK} \colorbox{Green!0.000}{\strut  .} \colorbox{Green!0.000}{\strut S} \colorbox{Green!0.000}{\strut ony} \colorbox{Green!0.000}{\strut  makes} \colorbox{Green!0.000}{\strut  a} \colorbox{Green!0.000}{\strut  really} \colorbox{Green!0.000}{\strut  great} \colorbox{Green!0.000}{\strut  product} \colorbox{Green!0.000}{\strut  ,} \colorbox{Green!0.000}{\strut set} \colorbox{Green!0.000}{\strut  up} \colorbox{Green!0.000}{\strut  is} \\
Output SAE & \num{1.658e+00} & \colorbox{Magenta!0.000}{\strut  is} \colorbox{Magenta!0.000}{\strut  OK} \colorbox{Magenta!0.000}{\strut  .} \colorbox{Magenta!0.000}{\strut S} \colorbox{Magenta!0.000}{\strut ony} \colorbox{Magenta!0.000}{\strut  makes} \colorbox{Magenta!0.000}{\strut  a} \colorbox{Magenta!0.000}{\strut  really} \colorbox{Magenta!0.000}{\strut  great} \colorbox{Magenta!0.000}{\strut  product} \colorbox{Magenta!0.000}{\strut  ,} \colorbox{Magenta!0.000}{\strut set} \colorbox{Magenta!0.000}{\strut  up} \colorbox{Magenta!0.000}{\strut  is} \\
\midrule
Jacobian & \num{2.175e-01} & \colorbox{Cyan!0.000}{\strut .} \colorbox{Cyan!0.000}{\strut  Also} \colorbox{Cyan!0.000}{\strut  needs} \colorbox{Cyan!0.000}{\strut  a} \colorbox{Cyan!0.000}{\strut  weight} \colorbox{Cyan!0.000}{\strut  somewhere} \colorbox{Cyan!0.000}{\strut  too} \colorbox{Cyan!0.000}{\strut .} \colorbox{Cyan!0.000}{\strut Ah} \colorbox{Cyan!0.000}{\strut ,} \colorbox{Cyan!0.000}{\strut  no} \colorbox{Cyan!0.000}{\strut .} \colorbox{Cyan!0.000}{\strut  I} \colorbox{Cyan!0.000}{\strut  am} \\
Input SAE & \num{1.541e+00} & \colorbox{Green!0.000}{\strut .} \colorbox{Green!47.763}{\strut  Also} \colorbox{Green!0.000}{\strut  needs} \colorbox{Green!0.000}{\strut  a} \colorbox{Green!0.000}{\strut  weight} \colorbox{Green!29.629}{\strut  somewhere} \colorbox{Green!0.000}{\strut  too} \colorbox{Green!0.000}{\strut .} \colorbox{Green!0.000}{\strut Ah} \colorbox{Green!0.000}{\strut ,} \colorbox{Green!0.000}{\strut  no} \colorbox{Green!0.000}{\strut .} \colorbox{Green!0.000}{\strut  I} \colorbox{Green!30.664}{\strut  am} \\
Output SAE & \num{1.656e+00} & \colorbox{Magenta!0.000}{\strut .} \colorbox{Magenta!0.000}{\strut  Also} \colorbox{Magenta!0.000}{\strut  needs} \colorbox{Magenta!0.000}{\strut  a} \colorbox{Magenta!0.000}{\strut  weight} \colorbox{Magenta!0.000}{\strut  somewhere} \colorbox{Magenta!0.000}{\strut  too} \colorbox{Magenta!0.000}{\strut .} \colorbox{Magenta!0.000}{\strut Ah} \colorbox{Magenta!0.000}{\strut ,} \colorbox{Magenta!0.000}{\strut  no} \colorbox{Magenta!0.000}{\strut .} \colorbox{Magenta!0.000}{\strut  I} \colorbox{Magenta!0.000}{\strut  am} \\
\midrule
Jacobian & \num{2.175e-01} & \colorbox{Cyan!0.000}{\strut  jacket} \colorbox{Cyan!0.000}{\strut  has} \colorbox{Cyan!0.000}{\strut  a} \colorbox{Cyan!0.000}{\strut  removable} \colorbox{Cyan!0.000}{\strut  hood} \colorbox{Cyan!0.000}{\strut  and} \colorbox{Cyan!0.000}{\strut  elastic} \colorbox{Cyan!0.000}{\strut ated} \colorbox{Cyan!0.000}{\strut  toggle} \colorbox{Cyan!0.000}{\strut  at} \colorbox{Cyan!0.000}{\strut  the} \colorbox{Cyan!0.000}{\strut  waist} \colorbox{Cyan!0.000}{\strut .} \colorbox{Cyan!0.000}{\strut Det} \\
Input SAE & \num{1.097e+00} & \colorbox{Green!24.924}{\strut  jacket} \colorbox{Green!21.833}{\strut  has} \colorbox{Green!0.000}{\strut  a} \colorbox{Green!0.000}{\strut  removable} \colorbox{Green!0.000}{\strut  hood} \colorbox{Green!0.000}{\strut  and} \colorbox{Green!33.114}{\strut  elastic} \colorbox{Green!0.000}{\strut ated} \colorbox{Green!0.000}{\strut  toggle} \colorbox{Green!0.000}{\strut  at} \colorbox{Green!0.000}{\strut  the} \colorbox{Green!0.000}{\strut  waist} \colorbox{Green!34.015}{\strut .} \colorbox{Green!21.984}{\strut Det} \\
Output SAE & \num{1.652e+00} & \colorbox{Magenta!0.000}{\strut  jacket} \colorbox{Magenta!0.000}{\strut  has} \colorbox{Magenta!0.000}{\strut  a} \colorbox{Magenta!0.000}{\strut  removable} \colorbox{Magenta!0.000}{\strut  hood} \colorbox{Magenta!0.000}{\strut  and} \colorbox{Magenta!0.000}{\strut  elastic} \colorbox{Magenta!0.000}{\strut ated} \colorbox{Magenta!0.000}{\strut  toggle} \colorbox{Magenta!0.000}{\strut  at} \colorbox{Magenta!0.000}{\strut  the} \colorbox{Magenta!0.000}{\strut  waist} \colorbox{Magenta!0.000}{\strut .} \colorbox{Magenta!0.000}{\strut Det} \\
\midrule
Jacobian & \num{2.175e-01} & \colorbox{Cyan!0.000}{\strut  personnel} \colorbox{Cyan!0.000}{\strut ).} \colorbox{Cyan!0.000}{\strut For} \colorbox{Cyan!0.000}{\strut  more} \colorbox{Cyan!0.000}{\strut  information} \colorbox{Cyan!0.000}{\strut ,} \colorbox{Cyan!0.000}{\strut  ask} \colorbox{Cyan!0.000}{\strut  at} \colorbox{Cyan!0.000}{\strut  The} \colorbox{Cyan!0.000}{\strut  Q} \colorbox{Cyan!0.000}{\strut ant} \colorbox{Cyan!0.000}{\strut as} \colorbox{Cyan!0.000}{\strut  Club} \colorbox{Cyan!0.000}{\strut  Service} \\
Input SAE & \num{1.375e+00} & \colorbox{Green!0.000}{\strut  personnel} \colorbox{Green!0.000}{\strut ).} \colorbox{Green!27.126}{\strut For} \colorbox{Green!33.686}{\strut  more} \colorbox{Green!42.610}{\strut  information} \colorbox{Green!0.000}{\strut ,} \colorbox{Green!0.000}{\strut  ask} \colorbox{Green!0.000}{\strut  at} \colorbox{Green!0.000}{\strut  The} \colorbox{Green!0.000}{\strut  Q} \colorbox{Green!0.000}{\strut ant} \colorbox{Green!0.000}{\strut as} \colorbox{Green!0.000}{\strut  Club} \colorbox{Green!0.000}{\strut  Service} \\
Output SAE & \num{1.658e+00} & \colorbox{Magenta!0.000}{\strut  personnel} \colorbox{Magenta!0.000}{\strut ).} \colorbox{Magenta!0.000}{\strut For} \colorbox{Magenta!0.000}{\strut  more} \colorbox{Magenta!0.000}{\strut  information} \colorbox{Magenta!0.000}{\strut ,} \colorbox{Magenta!0.000}{\strut  ask} \colorbox{Magenta!0.000}{\strut  at} \colorbox{Magenta!0.000}{\strut  The} \colorbox{Magenta!0.000}{\strut  Q} \colorbox{Magenta!0.000}{\strut ant} \colorbox{Magenta!0.000}{\strut as} \colorbox{Magenta!0.000}{\strut  Club} \colorbox{Magenta!0.000}{\strut  Service} \\
\midrule
Jacobian & \num{2.175e-01} & \colorbox{Cyan!0.000}{\strut  kid} \colorbox{Cyan!0.000}{\strut ).} \colorbox{Cyan!0.000}{\strut My} \colorbox{Cyan!0.000}{\strut  friend} \colorbox{Cyan!0.000}{\strut  Jenn} \colorbox{Cyan!0.000}{\strut  is} \colorbox{Cyan!0.000}{\strut  back} \colorbox{Cyan!0.000}{\strut  in} \colorbox{Cyan!0.000}{\strut  town} \colorbox{Cyan!0.000}{\strut ,} \colorbox{Cyan!0.000}{\strut  too} \colorbox{Cyan!0.000}{\strut ,} \colorbox{Cyan!0.000}{\strut  living} \colorbox{Cyan!0.000}{\strut  with} \\
Input SAE & \num{1.334e+00} & \colorbox{Green!0.000}{\strut  kid} \colorbox{Green!0.000}{\strut ).} \colorbox{Green!0.000}{\strut My} \colorbox{Green!0.000}{\strut  friend} \colorbox{Green!20.444}{\strut  Jenn} \colorbox{Green!0.000}{\strut  is} \colorbox{Green!0.000}{\strut  back} \colorbox{Green!0.000}{\strut  in} \colorbox{Green!0.000}{\strut  town} \colorbox{Green!39.582}{\strut ,} \colorbox{Green!0.000}{\strut  too} \colorbox{Green!41.354}{\strut ,} \colorbox{Green!0.000}{\strut  living} \colorbox{Green!0.000}{\strut  with} \\
Output SAE & \num{1.658e+00} & \colorbox{Magenta!0.000}{\strut  kid} \colorbox{Magenta!0.000}{\strut ).} \colorbox{Magenta!0.000}{\strut My} \colorbox{Magenta!0.000}{\strut  friend} \colorbox{Magenta!0.000}{\strut  Jenn} \colorbox{Magenta!0.000}{\strut  is} \colorbox{Magenta!0.000}{\strut  back} \colorbox{Magenta!0.000}{\strut  in} \colorbox{Magenta!0.000}{\strut  town} \colorbox{Magenta!0.000}{\strut ,} \colorbox{Magenta!0.000}{\strut  too} \colorbox{Magenta!0.000}{\strut ,} \colorbox{Magenta!0.000}{\strut  living} \colorbox{Magenta!0.000}{\strut  with} \\
\midrule
Jacobian & \num{2.175e-01} & \colorbox{Cyan!0.000}{\strut  who} \colorbox{Cyan!0.000}{\strut  needs} \colorbox{Cyan!0.000}{\strut  a} \colorbox{Cyan!0.000}{\strut  stand} \colorbox{Cyan!0.000}{\strut  out} \colorbox{Cyan!0.000}{\strut  resume} \colorbox{Cyan!0.000}{\strut !} \colorbox{Cyan!0.000}{\strut I} \colorbox{Cyan!0.000}{\strut  am} \colorbox{Cyan!0.000}{\strut  a} \colorbox{Cyan!0.000}{\strut  current} \colorbox{Cyan!0.000}{\strut  writer} \colorbox{Cyan!0.000}{\strut  for} \colorbox{Cyan!0.000}{\strut  The} \\
Input SAE & \num{1.114e+00} & \colorbox{Green!0.000}{\strut  who} \colorbox{Green!0.000}{\strut  needs} \colorbox{Green!0.000}{\strut  a} \colorbox{Green!0.000}{\strut  stand} \colorbox{Green!0.000}{\strut  out} \colorbox{Green!0.000}{\strut  resume} \colorbox{Green!0.000}{\strut !} \colorbox{Green!34.524}{\strut I} \colorbox{Green!33.866}{\strut  am} \colorbox{Green!0.000}{\strut  a} \colorbox{Green!0.000}{\strut  current} \colorbox{Green!29.547}{\strut  writer} \colorbox{Green!0.000}{\strut  for} \colorbox{Green!0.000}{\strut  The} \\
Output SAE & \num{1.657e+00} & \colorbox{Magenta!0.000}{\strut  who} \colorbox{Magenta!0.000}{\strut  needs} \colorbox{Magenta!0.000}{\strut  a} \colorbox{Magenta!0.000}{\strut  stand} \colorbox{Magenta!0.000}{\strut  out} \colorbox{Magenta!0.000}{\strut  resume} \colorbox{Magenta!0.000}{\strut !} \colorbox{Magenta!0.000}{\strut I} \colorbox{Magenta!0.000}{\strut  am} \colorbox{Magenta!0.000}{\strut  a} \colorbox{Magenta!0.000}{\strut  current} \colorbox{Magenta!0.000}{\strut  writer} \colorbox{Magenta!0.000}{\strut  for} \colorbox{Magenta!0.000}{\strut  The} \\
\midrule
Jacobian & \num{2.175e-01} & \colorbox{Cyan!0.000}{\strut  life} \colorbox{Cyan!0.000}{\strut  a} \colorbox{Cyan!0.000}{\strut  little} \colorbox{Cyan!0.000}{\strut  easier} \colorbox{Cyan!0.000}{\strut ,} \colorbox{Cyan!0.000}{\strut  right} \colorbox{Cyan!0.000}{\strut ?).} \colorbox{Cyan!0.000}{\strut Are} \colorbox{Cyan!0.000}{\strut  account} \colorbox{Cyan!0.000}{\strut  management} \colorbox{Cyan!0.000}{\strut  and} \colorbox{Cyan!0.000}{\strut  technical} \colorbox{Cyan!0.000}{\strut  support} \colorbox{Cyan!0.000}{\strut  available} \\
Input SAE & \num{1.149e+00} & \colorbox{Green!0.000}{\strut  life} \colorbox{Green!31.771}{\strut  a} \colorbox{Green!31.143}{\strut  little} \colorbox{Green!34.530}{\strut  easier} \colorbox{Green!35.612}{\strut ,} \colorbox{Green!29.425}{\strut  right} \colorbox{Green!0.000}{\strut ?).} \colorbox{Green!24.059}{\strut Are} \colorbox{Green!0.000}{\strut  account} \colorbox{Green!0.000}{\strut  management} \colorbox{Green!0.000}{\strut  and} \colorbox{Green!0.000}{\strut  technical} \colorbox{Green!0.000}{\strut  support} \colorbox{Green!0.000}{\strut  available} \\
Output SAE & \num{1.654e+00} & \colorbox{Magenta!0.000}{\strut  life} \colorbox{Magenta!0.000}{\strut  a} \colorbox{Magenta!0.000}{\strut  little} \colorbox{Magenta!0.000}{\strut  easier} \colorbox{Magenta!0.000}{\strut ,} \colorbox{Magenta!0.000}{\strut  right} \colorbox{Magenta!0.000}{\strut ?).} \colorbox{Magenta!0.000}{\strut Are} \colorbox{Magenta!0.000}{\strut  account} \colorbox{Magenta!0.000}{\strut  management} \colorbox{Magenta!0.000}{\strut  and} \colorbox{Magenta!0.000}{\strut  technical} \colorbox{Magenta!0.000}{\strut  support} \colorbox{Magenta!0.000}{\strut  available} \\
\midrule
Jacobian & \num{2.175e-01} & \colorbox{Cyan!0.000}{\strut  with} \colorbox{Cyan!0.000}{\strut  a} \colorbox{Cyan!0.000}{\strut  Tesla} \colorbox{Cyan!0.000}{\strut  police} \colorbox{Cyan!0.000}{\strut  car} \colorbox{Cyan!0.000}{\strut .} \colorbox{Cyan!0.000}{\strut  .} \colorbox{Cyan!0.000}{\strut Windows} \colorbox{Cyan!0.000}{\strut  X} \colorbox{Cyan!0.000}{\strut p} \colorbox{Cyan!0.000}{\strut  Sp} \colorbox{Cyan!0.000}{\strut 2} \colorbox{Cyan!0.000}{\strut  Super} \colorbox{Cyan!0.000}{\strut  L} \\
Input SAE & \num{1.174e+00} & \colorbox{Green!26.825}{\strut  with} \colorbox{Green!32.346}{\strut  a} \colorbox{Green!0.000}{\strut  Tesla} \colorbox{Green!0.000}{\strut  police} \colorbox{Green!0.000}{\strut  car} \colorbox{Green!0.000}{\strut .} \colorbox{Green!36.375}{\strut  .} \colorbox{Green!27.736}{\strut Windows} \colorbox{Green!0.000}{\strut  X} \colorbox{Green!0.000}{\strut p} \colorbox{Green!32.471}{\strut  Sp} \colorbox{Green!0.000}{\strut 2} \colorbox{Green!19.301}{\strut  Super} \colorbox{Green!0.000}{\strut  L} \\
Output SAE & \num{1.656e+00} & \colorbox{Magenta!0.000}{\strut  with} \colorbox{Magenta!0.000}{\strut  a} \colorbox{Magenta!0.000}{\strut  Tesla} \colorbox{Magenta!0.000}{\strut  police} \colorbox{Magenta!0.000}{\strut  car} \colorbox{Magenta!0.000}{\strut .} \colorbox{Magenta!0.000}{\strut  .} \colorbox{Magenta!0.000}{\strut Windows} \colorbox{Magenta!0.000}{\strut  X} \colorbox{Magenta!0.000}{\strut p} \colorbox{Magenta!0.000}{\strut  Sp} \colorbox{Magenta!0.000}{\strut 2} \colorbox{Magenta!0.000}{\strut  Super} \colorbox{Magenta!0.000}{\strut  L} \\
\midrule
Jacobian & \num{2.175e-01} & \colorbox{Cyan!0.000}{\strut \%).} \colorbox{Cyan!0.000}{\strut Is} \colorbox{Cyan!0.000}{\strut  the} \colorbox{Cyan!0.000}{\strut  Dr} \colorbox{Cyan!0.000}{\strut .} \colorbox{Cyan!0.000}{\strut  Pepper} \colorbox{Cyan!0.000}{\strut  good} \colorbox{Cyan!0.000}{\strut  for} \colorbox{Cyan!0.000}{\strut  you} \colorbox{Cyan!0.000}{\strut ?} \colorbox{Cyan!0.000}{\strut Does} \colorbox{Cyan!0.000}{\strut  the} \colorbox{Cyan!0.000}{\strut  Dr} \\
Input SAE & \num{1.254e+00} & \colorbox{Green!0.000}{\strut \%).} \colorbox{Green!0.000}{\strut Is} \colorbox{Green!0.000}{\strut  the} \colorbox{Green!24.312}{\strut  Dr} \colorbox{Green!28.818}{\strut .} \colorbox{Green!0.000}{\strut  Pepper} \colorbox{Green!0.000}{\strut  good} \colorbox{Green!38.252}{\strut  for} \colorbox{Green!0.000}{\strut  you} \colorbox{Green!0.000}{\strut ?} \colorbox{Green!38.869}{\strut Does} \colorbox{Green!0.000}{\strut  the} \colorbox{Green!0.000}{\strut  Dr} \\
Output SAE & \num{1.659e+00} & \colorbox{Magenta!0.000}{\strut \%).} \colorbox{Magenta!0.000}{\strut Is} \colorbox{Magenta!0.000}{\strut  the} \colorbox{Magenta!0.000}{\strut  Dr} \colorbox{Magenta!0.000}{\strut .} \colorbox{Magenta!0.000}{\strut  Pepper} \colorbox{Magenta!0.000}{\strut  good} \colorbox{Magenta!0.000}{\strut  for} \colorbox{Magenta!0.000}{\strut  you} \colorbox{Magenta!0.000}{\strut ?} \colorbox{Magenta!0.000}{\strut Does} \colorbox{Magenta!0.000}{\strut  the} \colorbox{Magenta!0.000}{\strut  Dr} \\
\midrule
Jacobian & \num{2.175e-01} & \colorbox{Cyan!0.000}{\strut  reasonable} \colorbox{Cyan!0.000}{\strut  exercise} \colorbox{Cyan!0.000}{\strut  program} \colorbox{Cyan!0.000}{\strut ).} \colorbox{Cyan!0.000}{\strut aG} \colorbox{Cyan!0.000}{\strut  Avoid} \colorbox{Cyan!0.000}{\strut  foods} \colorbox{Cyan!0.000}{\strut  and} \colorbox{Cyan!0.000}{\strut  drinks} \colorbox{Cyan!0.000}{\strut  with} \colorbox{Cyan!0.000}{\strut  caffeine} \colorbox{Cyan!0.000}{\strut ,} \colorbox{Cyan!0.000}{\strut  as} \colorbox{Cyan!0.000}{\strut  they} \\
Input SAE & \num{2.161e-01} & \colorbox{Green!0.000}{\strut  reasonable} \colorbox{Green!0.000}{\strut  exercise} \colorbox{Green!0.000}{\strut  program} \colorbox{Green!0.000}{\strut ).} \colorbox{Green!0.000}{\strut aG} \colorbox{Green!0.000}{\strut  Avoid} \colorbox{Green!0.000}{\strut  foods} \colorbox{Green!0.000}{\strut  and} \colorbox{Green!0.000}{\strut  drinks} \colorbox{Green!0.000}{\strut  with} \colorbox{Green!0.000}{\strut  caffeine} \colorbox{Green!0.000}{\strut ,} \colorbox{Green!0.000}{\strut  as} \colorbox{Green!0.000}{\strut  they} \\
Output SAE & \num{1.656e+00} & \colorbox{Magenta!0.000}{\strut  reasonable} \colorbox{Magenta!0.000}{\strut  exercise} \colorbox{Magenta!0.000}{\strut  program} \colorbox{Magenta!0.000}{\strut ).} \colorbox{Magenta!0.000}{\strut aG} \colorbox{Magenta!0.000}{\strut  Avoid} \colorbox{Magenta!0.000}{\strut  foods} \colorbox{Magenta!0.000}{\strut  and} \colorbox{Magenta!0.000}{\strut  drinks} \colorbox{Magenta!0.000}{\strut  with} \colorbox{Magenta!0.000}{\strut  caffeine} \colorbox{Magenta!0.000}{\strut ,} \colorbox{Magenta!0.000}{\strut  as} \colorbox{Magenta!0.000}{\strut  they} \\
\midrule
Jacobian & \num{2.175e-01} & \colorbox{Cyan!0.000}{\strut  Airport} \colorbox{Cyan!0.000}{\strut ).} \colorbox{Cyan!0.000}{\strut You} \colorbox{Cyan!0.000}{\strut  may} \colorbox{Cyan!0.000}{\strut  bring} \colorbox{Cyan!0.000}{\strut  along} \colorbox{Cyan!0.000}{\strut  your} \colorbox{Cyan!0.000}{\strut  ordinary} \colorbox{Cyan!0.000}{\strut  bike} \colorbox{Cyan!0.000}{\strut  in} \colorbox{Cyan!0.000}{\strut  Let} \colorbox{Cyan!0.000}{\strut ban} \colorbox{Cyan!0.000}{\strut en} \colorbox{Cyan!0.000}{\strut  except} \\
Input SAE & \num{1.054e+00} & \colorbox{Green!28.527}{\strut  Airport} \colorbox{Green!0.000}{\strut ).} \colorbox{Green!23.113}{\strut You} \colorbox{Green!24.781}{\strut  may} \colorbox{Green!0.000}{\strut  bring} \colorbox{Green!0.000}{\strut  along} \colorbox{Green!0.000}{\strut  your} \colorbox{Green!0.000}{\strut  ordinary} \colorbox{Green!0.000}{\strut  bike} \colorbox{Green!0.000}{\strut  in} \colorbox{Green!0.000}{\strut  Let} \colorbox{Green!0.000}{\strut ban} \colorbox{Green!0.000}{\strut en} \colorbox{Green!32.684}{\strut  except} \\
Output SAE & \num{1.655e+00} & \colorbox{Magenta!0.000}{\strut  Airport} \colorbox{Magenta!0.000}{\strut ).} \colorbox{Magenta!0.000}{\strut You} \colorbox{Magenta!0.000}{\strut  may} \colorbox{Magenta!0.000}{\strut  bring} \colorbox{Magenta!0.000}{\strut  along} \colorbox{Magenta!0.000}{\strut  your} \colorbox{Magenta!0.000}{\strut  ordinary} \colorbox{Magenta!0.000}{\strut  bike} \colorbox{Magenta!0.000}{\strut  in} \colorbox{Magenta!0.000}{\strut  Let} \colorbox{Magenta!0.000}{\strut ban} \colorbox{Magenta!0.000}{\strut en} \colorbox{Magenta!0.000}{\strut  except} \\
\bottomrule
\end{longtable}
\caption{feature pairs/Layer15-65536-J1-LR5.0e-04-k32-T3.0e+08 abs mean/examples-29982-v-33024 stas c4-en-10k,train,batch size=32,ctx len=16.csv}
\end{table}
% \begin{table}
\centering
\begin{longtable}{lrl}
\toprule
Category & Max. abs. value & Example tokens \\
\midrule
Jacobian & \num{2.587e-01} & \colorbox{Cyan!0.000}{\strut  open} \colorbox{Cyan!0.000}{\strut  where} \colorbox{Cyan!0.000}{\strut  to} \colorbox{Cyan!0.000}{\strut  insert} \colorbox{Cyan!0.000}{\strut  my} \colorbox{Cyan!0.000}{\strut  Sim} \colorbox{Cyan!0.000}{\strut  card} \colorbox{Cyan!0.000}{\strut  .} \colorbox{Cyan!0.000}{\strut  I} \colorbox{Cyan!0.000}{\strut  need} \colorbox{Cyan!0.000}{\strut  ur} \colorbox{Cyan!0.000}{\strut  assistance} \colorbox{Cyan!0.000}{\strut  on} \colorbox{Cyan!100.000}{\strut  how} \colorbox{Cyan!0.000}{\strut  to} \\
Input SAE & \num{1.140e+01} & \colorbox{Green!0.000}{\strut  open} \colorbox{Green!0.000}{\strut  where} \colorbox{Green!0.000}{\strut  to} \colorbox{Green!0.000}{\strut  insert} \colorbox{Green!0.000}{\strut  my} \colorbox{Green!0.000}{\strut  Sim} \colorbox{Green!0.000}{\strut  card} \colorbox{Green!0.000}{\strut  .} \colorbox{Green!0.000}{\strut  I} \colorbox{Green!0.000}{\strut  need} \colorbox{Green!0.000}{\strut  ur} \colorbox{Green!0.000}{\strut  assistance} \colorbox{Green!0.000}{\strut  on} \colorbox{Green!67.203}{\strut  how} \colorbox{Green!0.000}{\strut  to} \\
Output SAE & \num{4.092e+00} & \colorbox{Magenta!0.000}{\strut  open} \colorbox{Magenta!12.060}{\strut  where} \colorbox{Magenta!0.000}{\strut  to} \colorbox{Magenta!0.000}{\strut  insert} \colorbox{Magenta!0.000}{\strut  my} \colorbox{Magenta!0.000}{\strut  Sim} \colorbox{Magenta!0.000}{\strut  card} \colorbox{Magenta!0.000}{\strut  .} \colorbox{Magenta!0.000}{\strut  I} \colorbox{Magenta!0.000}{\strut  need} \colorbox{Magenta!0.000}{\strut  ur} \colorbox{Magenta!0.000}{\strut  assistance} \colorbox{Magenta!0.000}{\strut  on} \colorbox{Magenta!76.616}{\strut  how} \colorbox{Magenta!0.000}{\strut  to} \\
\midrule
Jacobian & \num{2.557e-01} & \colorbox{Cyan!0.000}{\strut  for} \colorbox{Cyan!0.000}{\strut  the} \colorbox{Cyan!0.000}{\strut  moment} \colorbox{Cyan!0.000}{\strut ),} \colorbox{Cyan!0.000}{\strut  I} \colorbox{Cyan!0.000}{\strut  need} \colorbox{Cyan!0.000}{\strut  to} \colorbox{Cyan!0.000}{\strut  figure} \colorbox{Cyan!0.000}{\strut  out} \colorbox{Cyan!98.872}{\strut  how} \colorbox{Cyan!0.000}{\strut  to} \colorbox{Cyan!0.000}{\strut  find} \colorbox{Cyan!0.000}{\strut  that} \colorbox{Cyan!0.000}{\strut  balance} \colorbox{Cyan!0.000}{\strut  between} \\
Input SAE & \num{1.295e+01} & \colorbox{Green!0.000}{\strut  for} \colorbox{Green!0.000}{\strut  the} \colorbox{Green!0.000}{\strut  moment} \colorbox{Green!0.000}{\strut ),} \colorbox{Green!0.000}{\strut  I} \colorbox{Green!0.000}{\strut  need} \colorbox{Green!0.000}{\strut  to} \colorbox{Green!0.000}{\strut  figure} \colorbox{Green!0.000}{\strut  out} \colorbox{Green!76.283}{\strut  how} \colorbox{Green!0.000}{\strut  to} \colorbox{Green!0.000}{\strut  find} \colorbox{Green!0.000}{\strut  that} \colorbox{Green!0.000}{\strut  balance} \colorbox{Green!0.000}{\strut  between} \\
Output SAE & \num{4.498e+00} & \colorbox{Magenta!0.000}{\strut  for} \colorbox{Magenta!0.000}{\strut  the} \colorbox{Magenta!0.000}{\strut  moment} \colorbox{Magenta!0.000}{\strut ),} \colorbox{Magenta!0.000}{\strut  I} \colorbox{Magenta!0.000}{\strut  need} \colorbox{Magenta!0.000}{\strut  to} \colorbox{Magenta!0.000}{\strut  figure} \colorbox{Magenta!0.000}{\strut  out} \colorbox{Magenta!84.210}{\strut  how} \colorbox{Magenta!0.000}{\strut  to} \colorbox{Magenta!0.000}{\strut  find} \colorbox{Magenta!0.000}{\strut  that} \colorbox{Magenta!0.000}{\strut  balance} \colorbox{Magenta!0.000}{\strut  between} \\
\midrule
Jacobian & \num{2.551e-01} & \colorbox{Cyan!0.000}{\strut  to} \colorbox{Cyan!0.000}{\strut  post} \colorbox{Cyan!0.000}{\strut  a} \colorbox{Cyan!0.000}{\strut  thread} \colorbox{Cyan!0.000}{\strut  with} \colorbox{Cyan!0.000}{\strut  some} \colorbox{Cyan!0.000}{\strut  images} \colorbox{Cyan!0.000}{\strut  in} \colorbox{Cyan!0.000}{\strut  it} \colorbox{Cyan!0.000}{\strut  but} \colorbox{Cyan!0.000}{\strut  I} \colorbox{Cyan!0.000}{\strut  don} \colorbox{Cyan!0.000}{\strut \textquotesingle{}t} \colorbox{Cyan!0.000}{\strut  know} \colorbox{Cyan!98.632}{\strut  how} \\
Input SAE & \num{1.167e+01} & \colorbox{Green!0.000}{\strut  to} \colorbox{Green!0.000}{\strut  post} \colorbox{Green!0.000}{\strut  a} \colorbox{Green!0.000}{\strut  thread} \colorbox{Green!0.000}{\strut  with} \colorbox{Green!0.000}{\strut  some} \colorbox{Green!0.000}{\strut  images} \colorbox{Green!0.000}{\strut  in} \colorbox{Green!0.000}{\strut  it} \colorbox{Green!0.000}{\strut  but} \colorbox{Green!0.000}{\strut  I} \colorbox{Green!0.000}{\strut  don} \colorbox{Green!0.000}{\strut \textquotesingle{}t} \colorbox{Green!0.000}{\strut  know} \colorbox{Green!68.790}{\strut  how} \\
Output SAE & \num{3.903e+00} & \colorbox{Magenta!0.000}{\strut  to} \colorbox{Magenta!0.000}{\strut  post} \colorbox{Magenta!0.000}{\strut  a} \colorbox{Magenta!0.000}{\strut  thread} \colorbox{Magenta!0.000}{\strut  with} \colorbox{Magenta!0.000}{\strut  some} \colorbox{Magenta!0.000}{\strut  images} \colorbox{Magenta!0.000}{\strut  in} \colorbox{Magenta!0.000}{\strut  it} \colorbox{Magenta!0.000}{\strut  but} \colorbox{Magenta!0.000}{\strut  I} \colorbox{Magenta!0.000}{\strut  don} \colorbox{Magenta!0.000}{\strut \textquotesingle{}t} \colorbox{Magenta!0.000}{\strut  know} \colorbox{Magenta!73.073}{\strut  how} \\
\midrule
Jacobian & \num{2.548e-01} & \colorbox{Cyan!0.000}{\strut \textquotesingle{}s} \colorbox{Cyan!0.000}{\strut  property} \colorbox{Cyan!0.000}{\strut .} \colorbox{Cyan!0.000}{\strut I} \colorbox{Cyan!0.000}{\strut  haven} \colorbox{Cyan!0.000}{\strut \textquotesingle{}} \colorbox{Cyan!0.000}{\strut t} \colorbox{Cyan!0.000}{\strut  been} \colorbox{Cyan!0.000}{\strut  able} \colorbox{Cyan!0.000}{\strut  to} \colorbox{Cyan!0.000}{\strut  learn} \colorbox{Cyan!98.520}{\strut  how} \colorbox{Cyan!0.000}{\strut  the} \colorbox{Cyan!0.000}{\strut  Oregon} \\
Input SAE & \num{1.199e+01} & \colorbox{Green!0.000}{\strut \textquotesingle{}s} \colorbox{Green!0.000}{\strut  property} \colorbox{Green!0.000}{\strut .} \colorbox{Green!0.000}{\strut I} \colorbox{Green!0.000}{\strut  haven} \colorbox{Green!0.000}{\strut \textquotesingle{}} \colorbox{Green!0.000}{\strut t} \colorbox{Green!0.000}{\strut  been} \colorbox{Green!0.000}{\strut  able} \colorbox{Green!0.000}{\strut  to} \colorbox{Green!0.000}{\strut  learn} \colorbox{Green!70.664}{\strut  how} \colorbox{Green!0.000}{\strut  the} \colorbox{Green!0.000}{\strut  Oregon} \\
Output SAE & \num{3.992e+00} & \colorbox{Magenta!0.000}{\strut \textquotesingle{}s} \colorbox{Magenta!0.000}{\strut  property} \colorbox{Magenta!0.000}{\strut .} \colorbox{Magenta!0.000}{\strut I} \colorbox{Magenta!0.000}{\strut  haven} \colorbox{Magenta!0.000}{\strut \textquotesingle{}} \colorbox{Magenta!0.000}{\strut t} \colorbox{Magenta!0.000}{\strut  been} \colorbox{Magenta!0.000}{\strut  able} \colorbox{Magenta!0.000}{\strut  to} \colorbox{Magenta!0.000}{\strut  learn} \colorbox{Magenta!74.737}{\strut  how} \colorbox{Magenta!0.000}{\strut  the} \colorbox{Magenta!0.000}{\strut  Oregon} \\
\midrule
Jacobian & \num{2.540e-01} & \colorbox{Cyan!0.000}{\strut  trying} \colorbox{Cyan!0.000}{\strut  to} \colorbox{Cyan!0.000}{\strut  figure} \colorbox{Cyan!0.000}{\strut  out} \colorbox{Cyan!98.184}{\strut  how} \colorbox{Cyan!0.000}{\strut  to} \colorbox{Cyan!0.000}{\strut  begin} \colorbox{Cyan!0.000}{\strut .} \colorbox{Cyan!0.000}{\strut  Any} \colorbox{Cyan!0.000}{\strut  ideas} \colorbox{Cyan!0.000}{\strut  or} \colorbox{Cyan!0.000}{\strut  hints} \colorbox{Cyan!0.000}{\strut ?} \colorbox{Cyan!0.000}{\strut  Thank} \colorbox{Cyan!0.000}{\strut  you} \\
Input SAE & \num{1.354e+01} & \colorbox{Green!0.000}{\strut  trying} \colorbox{Green!0.000}{\strut  to} \colorbox{Green!0.000}{\strut  figure} \colorbox{Green!0.000}{\strut  out} \colorbox{Green!79.810}{\strut  how} \colorbox{Green!0.000}{\strut  to} \colorbox{Green!0.000}{\strut  begin} \colorbox{Green!0.000}{\strut .} \colorbox{Green!0.000}{\strut  Any} \colorbox{Green!0.000}{\strut  ideas} \colorbox{Green!0.000}{\strut  or} \colorbox{Green!0.000}{\strut  hints} \colorbox{Green!0.000}{\strut ?} \colorbox{Green!0.000}{\strut  Thank} \colorbox{Green!0.000}{\strut  you} \\
Output SAE & \num{4.561e+00} & \colorbox{Magenta!0.000}{\strut  trying} \colorbox{Magenta!0.000}{\strut  to} \colorbox{Magenta!0.000}{\strut  figure} \colorbox{Magenta!0.000}{\strut  out} \colorbox{Magenta!85.394}{\strut  how} \colorbox{Magenta!10.451}{\strut  to} \colorbox{Magenta!0.000}{\strut  begin} \colorbox{Magenta!0.000}{\strut .} \colorbox{Magenta!0.000}{\strut  Any} \colorbox{Magenta!0.000}{\strut  ideas} \colorbox{Magenta!0.000}{\strut  or} \colorbox{Magenta!0.000}{\strut  hints} \colorbox{Magenta!0.000}{\strut ?} \colorbox{Magenta!0.000}{\strut  Thank} \colorbox{Magenta!0.000}{\strut  you} \\
\midrule
Jacobian & \num{2.538e-01} & \colorbox{Cyan!0.000}{\strut  spread} \colorbox{Cyan!0.000}{\strut er} \colorbox{Cyan!0.000}{\strut  and} \colorbox{Cyan!0.000}{\strut  it} \colorbox{Cyan!0.000}{\strut  worked} \colorbox{Cyan!0.000}{\strut  fine} \colorbox{Cyan!0.000}{\strut .} \colorbox{Cyan!0.000}{\strut Don} \colorbox{Cyan!0.000}{\strut \textquotesingle{}t} \colorbox{Cyan!0.000}{\strut  know} \colorbox{Cyan!0.000}{\strut  yet} \colorbox{Cyan!98.126}{\strut  how} \colorbox{Cyan!0.000}{\strut  the} \colorbox{Cyan!0.000}{\strut  service} \\
Input SAE & \num{1.191e+01} & \colorbox{Green!0.000}{\strut  spread} \colorbox{Green!0.000}{\strut er} \colorbox{Green!0.000}{\strut  and} \colorbox{Green!0.000}{\strut  it} \colorbox{Green!0.000}{\strut  worked} \colorbox{Green!0.000}{\strut  fine} \colorbox{Green!0.000}{\strut .} \colorbox{Green!0.000}{\strut Don} \colorbox{Green!0.000}{\strut \textquotesingle{}t} \colorbox{Green!0.000}{\strut  know} \colorbox{Green!0.000}{\strut  yet} \colorbox{Green!70.166}{\strut  how} \colorbox{Green!0.000}{\strut  the} \colorbox{Green!0.000}{\strut  service} \\
Output SAE & \num{3.997e+00} & \colorbox{Magenta!0.000}{\strut  spread} \colorbox{Magenta!0.000}{\strut er} \colorbox{Magenta!0.000}{\strut  and} \colorbox{Magenta!0.000}{\strut  it} \colorbox{Magenta!0.000}{\strut  worked} \colorbox{Magenta!0.000}{\strut  fine} \colorbox{Magenta!0.000}{\strut .} \colorbox{Magenta!0.000}{\strut Don} \colorbox{Magenta!0.000}{\strut \textquotesingle{}t} \colorbox{Magenta!0.000}{\strut  know} \colorbox{Magenta!0.000}{\strut  yet} \colorbox{Magenta!74.835}{\strut  how} \colorbox{Magenta!0.000}{\strut  the} \colorbox{Magenta!0.000}{\strut  service} \\
\midrule
Jacobian & \num{2.534e-01} & \colorbox{Cyan!0.000}{\strut ,} \colorbox{Cyan!0.000}{\strut  the} \colorbox{Cyan!0.000}{\strut  fact} \colorbox{Cyan!0.000}{\strut  that} \colorbox{Cyan!0.000}{\strut  I} \colorbox{Cyan!0.000}{\strut  have} \colorbox{Cyan!0.000}{\strut  no} \colorbox{Cyan!0.000}{\strut  idea} \colorbox{Cyan!0.000}{\strut  on} \colorbox{Cyan!97.988}{\strut  how} \colorbox{Cyan!0.000}{\strut  to} \colorbox{Cyan!0.000}{\strut  program} \colorbox{Cyan!0.000}{\strut  a} \colorbox{Cyan!0.000}{\strut  robot} \colorbox{Cyan!0.000}{\strut  stops} \\
Input SAE & \num{1.337e+01} & \colorbox{Green!0.000}{\strut ,} \colorbox{Green!0.000}{\strut  the} \colorbox{Green!0.000}{\strut  fact} \colorbox{Green!0.000}{\strut  that} \colorbox{Green!0.000}{\strut  I} \colorbox{Green!0.000}{\strut  have} \colorbox{Green!0.000}{\strut  no} \colorbox{Green!0.000}{\strut  idea} \colorbox{Green!0.000}{\strut  on} \colorbox{Green!78.799}{\strut  how} \colorbox{Green!0.000}{\strut  to} \colorbox{Green!0.000}{\strut  program} \colorbox{Green!0.000}{\strut  a} \colorbox{Green!0.000}{\strut  robot} \colorbox{Green!0.000}{\strut  stops} \\
Output SAE & \num{4.261e+00} & \colorbox{Magenta!0.000}{\strut ,} \colorbox{Magenta!0.000}{\strut  the} \colorbox{Magenta!0.000}{\strut  fact} \colorbox{Magenta!0.000}{\strut  that} \colorbox{Magenta!0.000}{\strut  I} \colorbox{Magenta!0.000}{\strut  have} \colorbox{Magenta!0.000}{\strut  no} \colorbox{Magenta!0.000}{\strut  idea} \colorbox{Magenta!0.000}{\strut  on} \colorbox{Magenta!79.776}{\strut  how} \colorbox{Magenta!0.000}{\strut  to} \colorbox{Magenta!0.000}{\strut  program} \colorbox{Magenta!0.000}{\strut  a} \colorbox{Magenta!0.000}{\strut  robot} \colorbox{Magenta!0.000}{\strut  stops} \\
\midrule
Jacobian & \num{2.533e-01} & \colorbox{Cyan!0.000}{\strut  went} \colorbox{Cyan!0.000}{\strut  for} \colorbox{Cyan!0.000}{\strut  a} \colorbox{Cyan!0.000}{\strut  dramatic} \colorbox{Cyan!0.000}{\strut  feel} \colorbox{Cyan!0.000}{\strut .} \colorbox{Cyan!0.000}{\strut Any} \colorbox{Cyan!0.000}{\strut  idea} \colorbox{Cyan!0.000}{\strut \textquotesingle{}s} \colorbox{Cyan!0.000}{\strut  on} \colorbox{Cyan!97.948}{\strut  how} \colorbox{Cyan!0.000}{\strut  I} \colorbox{Cyan!0.000}{\strut  make} \colorbox{Cyan!0.000}{\strut  it} \\
Input SAE & \num{1.196e+01} & \colorbox{Green!0.000}{\strut  went} \colorbox{Green!0.000}{\strut  for} \colorbox{Green!0.000}{\strut  a} \colorbox{Green!0.000}{\strut  dramatic} \colorbox{Green!0.000}{\strut  feel} \colorbox{Green!0.000}{\strut .} \colorbox{Green!0.000}{\strut Any} \colorbox{Green!0.000}{\strut  idea} \colorbox{Green!0.000}{\strut \textquotesingle{}s} \colorbox{Green!0.000}{\strut  on} \colorbox{Green!70.499}{\strut  how} \colorbox{Green!0.000}{\strut  I} \colorbox{Green!0.000}{\strut  make} \colorbox{Green!0.000}{\strut  it} \\
Output SAE & \num{4.171e+00} & \colorbox{Magenta!0.000}{\strut  went} \colorbox{Magenta!0.000}{\strut  for} \colorbox{Magenta!0.000}{\strut  a} \colorbox{Magenta!0.000}{\strut  dramatic} \colorbox{Magenta!11.964}{\strut  feel} \colorbox{Magenta!0.000}{\strut .} \colorbox{Magenta!0.000}{\strut Any} \colorbox{Magenta!0.000}{\strut  idea} \colorbox{Magenta!0.000}{\strut \textquotesingle{}s} \colorbox{Magenta!0.000}{\strut  on} \colorbox{Magenta!78.086}{\strut  how} \colorbox{Magenta!15.406}{\strut  I} \colorbox{Magenta!0.000}{\strut  make} \colorbox{Magenta!0.000}{\strut  it} \\
\midrule
Jacobian & \num{2.533e-01} & \colorbox{Cyan!0.000}{\strut  pl} \colorbox{Cyan!0.000}{\strut z} \colorbox{Cyan!0.000}{\strut  tell} \colorbox{Cyan!0.000}{\strut  me} \colorbox{Cyan!97.921}{\strut  how} \colorbox{Cyan!0.000}{\strut  to} \colorbox{Cyan!0.000}{\strut  make} \colorbox{Cyan!0.000}{\strut  fen} \colorbox{Cyan!0.000}{\strut ug} \colorbox{Cyan!0.000}{\strut reek} \colorbox{Cyan!0.000}{\strut  seeds} \colorbox{Cyan!0.000}{\strut  tea} \colorbox{Cyan!0.000}{\strut  for} \colorbox{Cyan!0.000}{\strut  hair} \colorbox{Cyan!0.000}{\strut  loss} \\
Input SAE & \num{1.258e+01} & \colorbox{Green!0.000}{\strut  pl} \colorbox{Green!0.000}{\strut z} \colorbox{Green!0.000}{\strut  tell} \colorbox{Green!0.000}{\strut  me} \colorbox{Green!74.125}{\strut  how} \colorbox{Green!0.000}{\strut  to} \colorbox{Green!0.000}{\strut  make} \colorbox{Green!0.000}{\strut  fen} \colorbox{Green!0.000}{\strut ug} \colorbox{Green!0.000}{\strut reek} \colorbox{Green!0.000}{\strut  seeds} \colorbox{Green!0.000}{\strut  tea} \colorbox{Green!0.000}{\strut  for} \colorbox{Green!0.000}{\strut  hair} \colorbox{Green!0.000}{\strut  loss} \\
Output SAE & \num{4.398e+00} & \colorbox{Magenta!0.000}{\strut  pl} \colorbox{Magenta!0.000}{\strut z} \colorbox{Magenta!0.000}{\strut  tell} \colorbox{Magenta!0.000}{\strut  me} \colorbox{Magenta!82.330}{\strut  how} \colorbox{Magenta!13.632}{\strut  to} \colorbox{Magenta!10.224}{\strut  make} \colorbox{Magenta!0.000}{\strut  fen} \colorbox{Magenta!0.000}{\strut ug} \colorbox{Magenta!0.000}{\strut reek} \colorbox{Magenta!0.000}{\strut  seeds} \colorbox{Magenta!0.000}{\strut  tea} \colorbox{Magenta!0.000}{\strut  for} \colorbox{Magenta!0.000}{\strut  hair} \colorbox{Magenta!0.000}{\strut  loss} \\
\midrule
Jacobian & \num{2.531e-01} & \colorbox{Cyan!0.000}{\strut  enough} \colorbox{Cyan!0.000}{\strut  to} \colorbox{Cyan!0.000}{\strut  hit} \colorbox{Cyan!0.000}{\strut  25} \colorbox{Cyan!0.000}{\strut 350} \colorbox{Cyan!0.000}{\strut .} \colorbox{Cyan!0.000}{\strut  what} \colorbox{Cyan!0.000}{\strut  i} \colorbox{Cyan!0.000}{\strut  want} \colorbox{Cyan!0.000}{\strut  ot} \colorbox{Cyan!0.000}{\strut  know} \colorbox{Cyan!0.000}{\strut  is} \colorbox{Cyan!97.871}{\strut  how} \colorbox{Cyan!0.000}{\strut  do} \colorbox{Cyan!0.000}{\strut  the} \\
Input SAE & \num{1.330e+01} & \colorbox{Green!0.000}{\strut  enough} \colorbox{Green!0.000}{\strut  to} \colorbox{Green!0.000}{\strut  hit} \colorbox{Green!0.000}{\strut  25} \colorbox{Green!0.000}{\strut 350} \colorbox{Green!0.000}{\strut .} \colorbox{Green!0.000}{\strut  what} \colorbox{Green!0.000}{\strut  i} \colorbox{Green!0.000}{\strut  want} \colorbox{Green!0.000}{\strut  ot} \colorbox{Green!0.000}{\strut  know} \colorbox{Green!0.000}{\strut  is} \colorbox{Green!78.396}{\strut  how} \colorbox{Green!0.000}{\strut  do} \colorbox{Green!0.000}{\strut  the} \\
Output SAE & \num{4.569e+00} & \colorbox{Magenta!0.000}{\strut  enough} \colorbox{Magenta!0.000}{\strut  to} \colorbox{Magenta!0.000}{\strut  hit} \colorbox{Magenta!0.000}{\strut  25} \colorbox{Magenta!0.000}{\strut 350} \colorbox{Magenta!0.000}{\strut .} \colorbox{Magenta!18.221}{\strut  what} \colorbox{Magenta!0.000}{\strut  i} \colorbox{Magenta!0.000}{\strut  want} \colorbox{Magenta!0.000}{\strut  ot} \colorbox{Magenta!0.000}{\strut  know} \colorbox{Magenta!0.000}{\strut  is} \colorbox{Magenta!85.531}{\strut  how} \colorbox{Magenta!32.065}{\strut  do} \colorbox{Magenta!0.000}{\strut  the} \\
\midrule
Jacobian & \num{2.530e-01} & \colorbox{Cyan!0.000}{\strut  you} \colorbox{Cyan!0.000}{\strut  have} \colorbox{Cyan!0.000}{\strut  to} \colorbox{Cyan!0.000}{\strut  figure} \colorbox{Cyan!0.000}{\strut  out} \colorbox{Cyan!97.833}{\strut  how} \colorbox{Cyan!0.000}{\strut  to} \colorbox{Cyan!0.000}{\strut  perfect} \colorbox{Cyan!0.000}{\strut  your} \colorbox{Cyan!0.000}{\strut  form} \colorbox{Cyan!0.000}{\strut  when} \colorbox{Cyan!0.000}{\strut  you} \colorbox{Cyan!0.000}{\strut  can} \colorbox{Cyan!0.000}{\strut \textquotesingle{}t} \colorbox{Cyan!0.000}{\strut  see} \\
Input SAE & \num{1.356e+01} & \colorbox{Green!0.000}{\strut  you} \colorbox{Green!0.000}{\strut  have} \colorbox{Green!0.000}{\strut  to} \colorbox{Green!0.000}{\strut  figure} \colorbox{Green!0.000}{\strut  out} \colorbox{Green!79.905}{\strut  how} \colorbox{Green!0.000}{\strut  to} \colorbox{Green!0.000}{\strut  perfect} \colorbox{Green!0.000}{\strut  your} \colorbox{Green!0.000}{\strut  form} \colorbox{Green!0.000}{\strut  when} \colorbox{Green!0.000}{\strut  you} \colorbox{Green!0.000}{\strut  can} \colorbox{Green!0.000}{\strut \textquotesingle{}t} \colorbox{Green!0.000}{\strut  see} \\
Output SAE & \num{4.391e+00} & \colorbox{Magenta!0.000}{\strut  you} \colorbox{Magenta!0.000}{\strut  have} \colorbox{Magenta!0.000}{\strut  to} \colorbox{Magenta!0.000}{\strut  figure} \colorbox{Magenta!0.000}{\strut  out} \colorbox{Magenta!82.211}{\strut  how} \colorbox{Magenta!0.000}{\strut  to} \colorbox{Magenta!0.000}{\strut  perfect} \colorbox{Magenta!0.000}{\strut  your} \colorbox{Magenta!0.000}{\strut  form} \colorbox{Magenta!0.000}{\strut  when} \colorbox{Magenta!0.000}{\strut  you} \colorbox{Magenta!0.000}{\strut  can} \colorbox{Magenta!0.000}{\strut \textquotesingle{}t} \colorbox{Magenta!0.000}{\strut  see} \\
\midrule
Jacobian & \num{2.529e-01} & \colorbox{Cyan!0.000}{\strut 7} \colorbox{Cyan!0.000}{\strut ?} \colorbox{Cyan!0.000}{\strut [} \colorbox{Cyan!0.000}{\strut SOL} \colorbox{Cyan!0.000}{\strut V} \colorbox{Cyan!0.000}{\strut ED} \colorbox{Cyan!0.000}{\strut ]} \colorbox{Cyan!97.759}{\strut  How} \colorbox{Cyan!0.000}{\strut  to} \colorbox{Cyan!0.000}{\strut  recover} \colorbox{Cyan!0.000}{\strut  deleted} \colorbox{Cyan!0.000}{\strut  contacts} \colorbox{Cyan!0.000}{\strut  from} \colorbox{Cyan!0.000}{\strut  h} \\
Input SAE & \num{1.319e+01} & \colorbox{Green!0.000}{\strut 7} \colorbox{Green!0.000}{\strut ?} \colorbox{Green!0.000}{\strut [} \colorbox{Green!0.000}{\strut SOL} \colorbox{Green!0.000}{\strut V} \colorbox{Green!0.000}{\strut ED} \colorbox{Green!0.000}{\strut ]} \colorbox{Green!77.718}{\strut  How} \colorbox{Green!0.000}{\strut  to} \colorbox{Green!0.000}{\strut  recover} \colorbox{Green!0.000}{\strut  deleted} \colorbox{Green!0.000}{\strut  contacts} \colorbox{Green!0.000}{\strut  from} \colorbox{Green!0.000}{\strut  h} \\
Output SAE & \num{5.148e+00} & \colorbox{Magenta!0.000}{\strut 7} \colorbox{Magenta!0.000}{\strut ?} \colorbox{Magenta!0.000}{\strut [} \colorbox{Magenta!0.000}{\strut SOL} \colorbox{Magenta!0.000}{\strut V} \colorbox{Magenta!0.000}{\strut ED} \colorbox{Magenta!0.000}{\strut ]} \colorbox{Magenta!96.385}{\strut  How} \colorbox{Magenta!11.966}{\strut  to} \colorbox{Magenta!0.000}{\strut  recover} \colorbox{Magenta!0.000}{\strut  deleted} \colorbox{Magenta!0.000}{\strut  contacts} \colorbox{Magenta!0.000}{\strut  from} \colorbox{Magenta!0.000}{\strut  h} \\
\bottomrule
\end{longtable}
\caption{feature pairs/Layer15-65536-J1-LR5.0e-04-k32-T3.0e+08 abs mean/examples-7447-v-21308 stas c4-en-10k,train,batch size=32,ctx len=16.csv}
\end{table} % how
% \begin{table}
\centering
\begin{longtable}{lrl}
\toprule
Category & Max. abs. value & Example tokens \\
\midrule
Jacobian & \num{2.564e-01} & \colorbox{Cyan!0.000}{\strut  The} \colorbox{Cyan!0.000}{\strut  most} \colorbox{Cyan!0.000}{\strut  amazing} \colorbox{Cyan!0.000}{\strut  part} \colorbox{Cyan!0.000}{\strut  is} \colorbox{Cyan!0.000}{\strut ,} \colorbox{Cyan!100.000}{\strut  all} \colorbox{Cyan!78.546}{\strut  of} \colorbox{Cyan!0.000}{\strut  this} \colorbox{Cyan!0.000}{\strut  is} \colorbox{Cyan!0.000}{\strut  done} \colorbox{Cyan!0.000}{\strut  by} \colorbox{Cyan!0.000}{\strut  one} \colorbox{Cyan!0.000}{\strut  worker} \colorbox{Cyan!0.000}{\strut ,} \\
Input SAE & \num{1.733e+01} & \colorbox{Green!0.000}{\strut  The} \colorbox{Green!0.000}{\strut  most} \colorbox{Green!0.000}{\strut  amazing} \colorbox{Green!0.000}{\strut  part} \colorbox{Green!0.000}{\strut  is} \colorbox{Green!0.000}{\strut ,} \colorbox{Green!91.082}{\strut  all} \colorbox{Green!9.753}{\strut  of} \colorbox{Green!0.000}{\strut  this} \colorbox{Green!0.000}{\strut  is} \colorbox{Green!0.000}{\strut  done} \colorbox{Green!0.000}{\strut  by} \colorbox{Green!0.000}{\strut  one} \colorbox{Green!0.000}{\strut  worker} \colorbox{Green!0.000}{\strut ,} \\
Output SAE & \num{4.417e+00} & \colorbox{Magenta!0.000}{\strut  The} \colorbox{Magenta!0.000}{\strut  most} \colorbox{Magenta!0.000}{\strut  amazing} \colorbox{Magenta!0.000}{\strut  part} \colorbox{Magenta!0.000}{\strut  is} \colorbox{Magenta!0.000}{\strut ,} \colorbox{Magenta!96.019}{\strut  all} \colorbox{Magenta!33.676}{\strut  of} \colorbox{Magenta!16.586}{\strut  this} \colorbox{Magenta!15.553}{\strut  is} \colorbox{Magenta!0.000}{\strut  done} \colorbox{Magenta!0.000}{\strut  by} \colorbox{Magenta!0.000}{\strut  one} \colorbox{Magenta!0.000}{\strut  worker} \colorbox{Magenta!0.000}{\strut ,} \\
\midrule
Jacobian & \num{2.542e-01} & \colorbox{Cyan!0.000}{\strut  par} \colorbox{Cyan!0.000}{\strut ades} \colorbox{Cyan!0.000}{\strut  and} \colorbox{Cyan!0.000}{\strut  theatrical} \colorbox{Cyan!0.000}{\strut  performances} \colorbox{Cyan!0.000}{\strut  took} \colorbox{Cyan!0.000}{\strut  place} \colorbox{Cyan!0.000}{\strut .} \colorbox{Cyan!0.000}{\strut  In} \colorbox{Cyan!0.000}{\strut  the} \colorbox{Cyan!0.000}{\strut  17} \colorbox{Cyan!0.000}{\strut th} \colorbox{Cyan!0.000}{\strut  century} \colorbox{Cyan!0.000}{\strut ,} \colorbox{Cyan!99.146}{\strut  all} \\
Input SAE & \num{1.660e+01} & \colorbox{Green!0.000}{\strut  par} \colorbox{Green!0.000}{\strut ades} \colorbox{Green!0.000}{\strut  and} \colorbox{Green!0.000}{\strut  theatrical} \colorbox{Green!0.000}{\strut  performances} \colorbox{Green!0.000}{\strut  took} \colorbox{Green!0.000}{\strut  place} \colorbox{Green!0.000}{\strut .} \colorbox{Green!0.000}{\strut  In} \colorbox{Green!0.000}{\strut  the} \colorbox{Green!0.000}{\strut  17} \colorbox{Green!0.000}{\strut th} \colorbox{Green!0.000}{\strut  century} \colorbox{Green!0.000}{\strut ,} \colorbox{Green!87.260}{\strut  all} \\
Output SAE & \num{4.106e+00} & \colorbox{Magenta!0.000}{\strut  par} \colorbox{Magenta!0.000}{\strut ades} \colorbox{Magenta!0.000}{\strut  and} \colorbox{Magenta!0.000}{\strut  theatrical} \colorbox{Magenta!0.000}{\strut  performances} \colorbox{Magenta!0.000}{\strut  took} \colorbox{Magenta!0.000}{\strut  place} \colorbox{Magenta!0.000}{\strut .} \colorbox{Magenta!0.000}{\strut  In} \colorbox{Magenta!0.000}{\strut  the} \colorbox{Magenta!0.000}{\strut  17} \colorbox{Magenta!0.000}{\strut th} \colorbox{Magenta!0.000}{\strut  century} \colorbox{Magenta!0.000}{\strut ,} \colorbox{Magenta!89.255}{\strut  all} \\
\midrule
Jacobian & \num{2.542e-01} & \colorbox{Cyan!0.000}{\strut  (} \colorbox{Cyan!0.000}{\strut Dec} \colorbox{Cyan!0.000}{\strut  2014} \colorbox{Cyan!0.000}{\strut )} \colorbox{Cyan!0.000}{\strut  we} \colorbox{Cyan!0.000}{\strut  had} \colorbox{Cyan!0.000}{\strut  to} \colorbox{Cyan!0.000}{\strut  drain} \colorbox{Cyan!0.000}{\strut  the} \colorbox{Cyan!0.000}{\strut  tub} \colorbox{Cyan!0.000}{\strut .} \colorbox{Cyan!99.120}{\strut  All} \colorbox{Cyan!0.000}{\strut  four} \colorbox{Cyan!0.000}{\strut  foot} \colorbox{Cyan!0.000}{\strut  jet} \\
Input SAE & \num{1.671e+01} & \colorbox{Green!0.000}{\strut  (} \colorbox{Green!0.000}{\strut Dec} \colorbox{Green!0.000}{\strut  2014} \colorbox{Green!0.000}{\strut )} \colorbox{Green!0.000}{\strut  we} \colorbox{Green!0.000}{\strut  had} \colorbox{Green!0.000}{\strut  to} \colorbox{Green!0.000}{\strut  drain} \colorbox{Green!0.000}{\strut  the} \colorbox{Green!0.000}{\strut  tub} \colorbox{Green!0.000}{\strut .} \colorbox{Green!87.866}{\strut  All} \colorbox{Green!0.000}{\strut  four} \colorbox{Green!0.000}{\strut  foot} \colorbox{Green!0.000}{\strut  jet} \\
Output SAE & \num{4.293e+00} & \colorbox{Magenta!0.000}{\strut  (} \colorbox{Magenta!0.000}{\strut Dec} \colorbox{Magenta!0.000}{\strut  2014} \colorbox{Magenta!0.000}{\strut )} \colorbox{Magenta!0.000}{\strut  we} \colorbox{Magenta!0.000}{\strut  had} \colorbox{Magenta!0.000}{\strut  to} \colorbox{Magenta!0.000}{\strut  drain} \colorbox{Magenta!0.000}{\strut  the} \colorbox{Magenta!0.000}{\strut  tub} \colorbox{Magenta!0.000}{\strut .} \colorbox{Magenta!93.322}{\strut  All} \colorbox{Magenta!0.000}{\strut  four} \colorbox{Magenta!0.000}{\strut  foot} \colorbox{Magenta!0.000}{\strut  jet} \\
\midrule
Jacobian & \num{2.536e-01} & \colorbox{Cyan!0.000}{\strut  approval} \colorbox{Cyan!0.000}{\strut  for} \colorbox{Cyan!0.000}{\strut  a} \colorbox{Cyan!0.000}{\strut  brand} \colorbox{Cyan!0.000}{\strut \textquotesingle{}s} \colorbox{Cyan!0.000}{\strut  product} \colorbox{Cyan!0.000}{\strut .} \colorbox{Cyan!0.000}{\strut  So} \colorbox{Cyan!0.000}{\strut  last} \colorbox{Cyan!0.000}{\strut  night} \colorbox{Cyan!0.000}{\strut ,} \colorbox{Cyan!98.903}{\strut  all} \colorbox{Cyan!0.000}{\strut  the} \colorbox{Cyan!0.000}{\strut  brands} \colorbox{Cyan!0.000}{\strut  who} \\
Input SAE & \num{1.847e+01} & \colorbox{Green!0.000}{\strut  approval} \colorbox{Green!0.000}{\strut  for} \colorbox{Green!0.000}{\strut  a} \colorbox{Green!0.000}{\strut  brand} \colorbox{Green!0.000}{\strut \textquotesingle{}s} \colorbox{Green!0.000}{\strut  product} \colorbox{Green!0.000}{\strut .} \colorbox{Green!0.000}{\strut  So} \colorbox{Green!0.000}{\strut  last} \colorbox{Green!0.000}{\strut  night} \colorbox{Green!0.000}{\strut ,} \colorbox{Green!97.109}{\strut  all} \colorbox{Green!0.000}{\strut  the} \colorbox{Green!0.000}{\strut  brands} \colorbox{Green!0.000}{\strut  who} \\
Output SAE & \num{4.428e+00} & \colorbox{Magenta!0.000}{\strut  approval} \colorbox{Magenta!0.000}{\strut  for} \colorbox{Magenta!0.000}{\strut  a} \colorbox{Magenta!0.000}{\strut  brand} \colorbox{Magenta!0.000}{\strut \textquotesingle{}s} \colorbox{Magenta!0.000}{\strut  product} \colorbox{Magenta!0.000}{\strut .} \colorbox{Magenta!0.000}{\strut  So} \colorbox{Magenta!0.000}{\strut  last} \colorbox{Magenta!0.000}{\strut  night} \colorbox{Magenta!0.000}{\strut ,} \colorbox{Magenta!96.242}{\strut  all} \colorbox{Magenta!31.486}{\strut  the} \colorbox{Magenta!0.000}{\strut  brands} \colorbox{Magenta!0.000}{\strut  who} \\
\midrule
Jacobian & \num{2.531e-01} & \colorbox{Cyan!0.000}{\strut  an} \colorbox{Cyan!0.000}{\strut  earthquake} \colorbox{Cyan!0.000}{\strut  in} \colorbox{Cyan!0.000}{\strut  1933} \colorbox{Cyan!0.000}{\strut ,} \colorbox{Cyan!0.000}{\strut  and} \colorbox{Cyan!0.000}{\strut  now} \colorbox{Cyan!98.713}{\strut  all} \colorbox{Cyan!0.000}{\strut  that} \colorbox{Cyan!0.000}{\strut  remains} \colorbox{Cyan!0.000}{\strut  are} \colorbox{Cyan!0.000}{\strut  maj} \colorbox{Cyan!0.000}{\strut estic} \colorbox{Cyan!0.000}{\strut  palms} \colorbox{Cyan!0.000}{\strut ,} \\
Input SAE & \num{1.741e+01} & \colorbox{Green!0.000}{\strut  an} \colorbox{Green!0.000}{\strut  earthquake} \colorbox{Green!0.000}{\strut  in} \colorbox{Green!0.000}{\strut  1933} \colorbox{Green!0.000}{\strut ,} \colorbox{Green!0.000}{\strut  and} \colorbox{Green!0.000}{\strut  now} \colorbox{Green!91.519}{\strut  all} \colorbox{Green!0.000}{\strut  that} \colorbox{Green!0.000}{\strut  remains} \colorbox{Green!0.000}{\strut  are} \colorbox{Green!0.000}{\strut  maj} \colorbox{Green!0.000}{\strut estic} \colorbox{Green!0.000}{\strut  palms} \colorbox{Green!0.000}{\strut ,} \\
Output SAE & \num{4.345e+00} & \colorbox{Magenta!0.000}{\strut  an} \colorbox{Magenta!0.000}{\strut  earthquake} \colorbox{Magenta!0.000}{\strut  in} \colorbox{Magenta!0.000}{\strut  1933} \colorbox{Magenta!0.000}{\strut ,} \colorbox{Magenta!0.000}{\strut  and} \colorbox{Magenta!0.000}{\strut  now} \colorbox{Magenta!94.452}{\strut  all} \colorbox{Magenta!14.996}{\strut  that} \colorbox{Magenta!0.000}{\strut  remains} \colorbox{Magenta!0.000}{\strut  are} \colorbox{Magenta!0.000}{\strut  maj} \colorbox{Magenta!0.000}{\strut estic} \colorbox{Magenta!0.000}{\strut  palms} \colorbox{Magenta!0.000}{\strut ,} \\
\midrule
Jacobian & \num{2.525e-01} & \colorbox{Cyan!0.000}{\strut  to} \colorbox{Cyan!0.000}{\strut  be} \colorbox{Cyan!0.000}{\strut  seated} \colorbox{Cyan!0.000}{\strut  for} \colorbox{Cyan!0.000}{\strut  more} \colorbox{Cyan!0.000}{\strut  than} \colorbox{Cyan!0.000}{\strut  half} \colorbox{Cyan!0.000}{\strut  an} \colorbox{Cyan!0.000}{\strut  hour} \colorbox{Cyan!0.000}{\strut .} \colorbox{Cyan!98.484}{\strut  All} \colorbox{Cyan!0.000}{\strut  his} \colorbox{Cyan!0.000}{\strut  rhe} \colorbox{Cyan!0.000}{\strut umatic} \colorbox{Cyan!0.000}{\strut  complaints} \\
Input SAE & \num{1.618e+01} & \colorbox{Green!0.000}{\strut  to} \colorbox{Green!0.000}{\strut  be} \colorbox{Green!0.000}{\strut  seated} \colorbox{Green!0.000}{\strut  for} \colorbox{Green!0.000}{\strut  more} \colorbox{Green!0.000}{\strut  than} \colorbox{Green!0.000}{\strut  half} \colorbox{Green!0.000}{\strut  an} \colorbox{Green!0.000}{\strut  hour} \colorbox{Green!0.000}{\strut .} \colorbox{Green!85.066}{\strut  All} \colorbox{Green!0.000}{\strut  his} \colorbox{Green!0.000}{\strut  rhe} \colorbox{Green!0.000}{\strut umatic} \colorbox{Green!0.000}{\strut  complaints} \\
Output SAE & \num{4.073e+00} & \colorbox{Magenta!0.000}{\strut  to} \colorbox{Magenta!0.000}{\strut  be} \colorbox{Magenta!0.000}{\strut  seated} \colorbox{Magenta!0.000}{\strut  for} \colorbox{Magenta!0.000}{\strut  more} \colorbox{Magenta!0.000}{\strut  than} \colorbox{Magenta!16.710}{\strut  half} \colorbox{Magenta!0.000}{\strut  an} \colorbox{Magenta!0.000}{\strut  hour} \colorbox{Magenta!0.000}{\strut .} \colorbox{Magenta!88.521}{\strut  All} \colorbox{Magenta!18.582}{\strut  his} \colorbox{Magenta!0.000}{\strut  rhe} \colorbox{Magenta!0.000}{\strut umatic} \colorbox{Magenta!0.000}{\strut  complaints} \\
\midrule
Jacobian & \num{2.523e-01} & \colorbox{Cyan!0.000}{\strut S} \colorbox{Cyan!0.000}{\strut AL} \colorbox{Cyan!0.000}{\strut EM} \colorbox{Cyan!0.000}{\strut ,} \colorbox{Cyan!0.000}{\strut  O} \colorbox{Cyan!0.000}{\strut RE} \colorbox{Cyan!0.000}{\strut .} \colorbox{Cyan!0.000}{\strut  --} \colorbox{Cyan!98.409}{\strut  All} \colorbox{Cyan!0.000}{\strut  of} \colorbox{Cyan!0.000}{\strut  Will} \colorbox{Cyan!0.000}{\strut am} \colorbox{Cyan!0.000}{\strut ette} \colorbox{Cyan!0.000}{\strut \textquotesingle{}s} \colorbox{Cyan!0.000}{\strut  football} \\
Input SAE & \num{1.692e+01} & \colorbox{Green!0.000}{\strut S} \colorbox{Green!0.000}{\strut AL} \colorbox{Green!0.000}{\strut EM} \colorbox{Green!0.000}{\strut ,} \colorbox{Green!0.000}{\strut  O} \colorbox{Green!0.000}{\strut RE} \colorbox{Green!0.000}{\strut .} \colorbox{Green!0.000}{\strut  --} \colorbox{Green!88.971}{\strut  All} \colorbox{Green!0.000}{\strut  of} \colorbox{Green!0.000}{\strut  Will} \colorbox{Green!0.000}{\strut am} \colorbox{Green!0.000}{\strut ette} \colorbox{Green!0.000}{\strut \textquotesingle{}s} \colorbox{Green!0.000}{\strut  football} \\
Output SAE & \num{4.063e+00} & \colorbox{Magenta!0.000}{\strut S} \colorbox{Magenta!0.000}{\strut AL} \colorbox{Magenta!0.000}{\strut EM} \colorbox{Magenta!0.000}{\strut ,} \colorbox{Magenta!0.000}{\strut  O} \colorbox{Magenta!0.000}{\strut RE} \colorbox{Magenta!0.000}{\strut .} \colorbox{Magenta!0.000}{\strut  --} \colorbox{Magenta!88.317}{\strut  All} \colorbox{Magenta!28.205}{\strut  of} \colorbox{Magenta!0.000}{\strut  Will} \colorbox{Magenta!0.000}{\strut am} \colorbox{Magenta!0.000}{\strut ette} \colorbox{Magenta!12.750}{\strut \textquotesingle{}s} \colorbox{Magenta!0.000}{\strut  football} \\
\midrule
Jacobian & \num{2.522e-01} & \colorbox{Cyan!0.000}{\strut  a} \colorbox{Cyan!0.000}{\strut  left} \colorbox{Cyan!0.000}{\strut  hand} \colorbox{Cyan!0.000}{\strut  turn} \colorbox{Cyan!0.000}{\strut  across} \colorbox{Cyan!0.000}{\strut  an} \colorbox{Cyan!0.000}{\strut  intersection} \colorbox{Cyan!0.000}{\strut .} \colorbox{Cyan!0.000}{\strut  He} \colorbox{Cyan!0.000}{\strut  says} \colorbox{Cyan!98.350}{\strut  all} \colorbox{Cyan!0.000}{\strut  he} \colorbox{Cyan!0.000}{\strut  saw} \colorbox{Cyan!0.000}{\strut  was} \colorbox{Cyan!0.000}{\strut  something} \\
Input SAE & \num{1.697e+01} & \colorbox{Green!0.000}{\strut  a} \colorbox{Green!0.000}{\strut  left} \colorbox{Green!0.000}{\strut  hand} \colorbox{Green!0.000}{\strut  turn} \colorbox{Green!0.000}{\strut  across} \colorbox{Green!0.000}{\strut  an} \colorbox{Green!0.000}{\strut  intersection} \colorbox{Green!0.000}{\strut .} \colorbox{Green!0.000}{\strut  He} \colorbox{Green!0.000}{\strut  says} \colorbox{Green!89.215}{\strut  all} \colorbox{Green!0.000}{\strut  he} \colorbox{Green!0.000}{\strut  saw} \colorbox{Green!0.000}{\strut  was} \colorbox{Green!0.000}{\strut  something} \\
Output SAE & \num{4.212e+00} & \colorbox{Magenta!0.000}{\strut  a} \colorbox{Magenta!0.000}{\strut  left} \colorbox{Magenta!0.000}{\strut  hand} \colorbox{Magenta!0.000}{\strut  turn} \colorbox{Magenta!0.000}{\strut  across} \colorbox{Magenta!0.000}{\strut  an} \colorbox{Magenta!0.000}{\strut  intersection} \colorbox{Magenta!0.000}{\strut .} \colorbox{Magenta!0.000}{\strut  He} \colorbox{Magenta!0.000}{\strut  says} \colorbox{Magenta!91.548}{\strut  all} \colorbox{Magenta!0.000}{\strut  he} \colorbox{Magenta!0.000}{\strut  saw} \colorbox{Magenta!0.000}{\strut  was} \colorbox{Magenta!0.000}{\strut  something} \\
\midrule
Jacobian & \num{2.522e-01} & \colorbox{Cyan!0.000}{\strut  has} \colorbox{Cyan!0.000}{\strut  been} \colorbox{Cyan!0.000}{\strut  declared} \colorbox{Cyan!0.000}{\strut  one} \colorbox{Cyan!0.000}{\strut  of} \colorbox{Cyan!0.000}{\strut  the} \colorbox{Cyan!0.000}{\strut  most} \colorbox{Cyan!0.000}{\strut  beautiful} \colorbox{Cyan!0.000}{\strut  rail} \colorbox{Cyan!0.000}{\strut ways} \colorbox{Cyan!0.000}{\strut  stations} \colorbox{Cyan!0.000}{\strut  in} \colorbox{Cyan!0.000}{\strut  Europe} \colorbox{Cyan!0.000}{\strut .} \colorbox{Cyan!98.345}{\strut  All} \\
Input SAE & \num{1.738e+01} & \colorbox{Green!0.000}{\strut  has} \colorbox{Green!0.000}{\strut  been} \colorbox{Green!0.000}{\strut  declared} \colorbox{Green!0.000}{\strut  one} \colorbox{Green!0.000}{\strut  of} \colorbox{Green!0.000}{\strut  the} \colorbox{Green!0.000}{\strut  most} \colorbox{Green!0.000}{\strut  beautiful} \colorbox{Green!0.000}{\strut  rail} \colorbox{Green!0.000}{\strut ways} \colorbox{Green!0.000}{\strut  stations} \colorbox{Green!0.000}{\strut  in} \colorbox{Green!0.000}{\strut  Europe} \colorbox{Green!0.000}{\strut .} \colorbox{Green!91.387}{\strut  All} \\
Output SAE & \num{4.299e+00} & \colorbox{Magenta!0.000}{\strut  has} \colorbox{Magenta!0.000}{\strut  been} \colorbox{Magenta!0.000}{\strut  declared} \colorbox{Magenta!0.000}{\strut  one} \colorbox{Magenta!0.000}{\strut  of} \colorbox{Magenta!0.000}{\strut  the} \colorbox{Magenta!0.000}{\strut  most} \colorbox{Magenta!0.000}{\strut  beautiful} \colorbox{Magenta!0.000}{\strut  rail} \colorbox{Magenta!0.000}{\strut ways} \colorbox{Magenta!0.000}{\strut  stations} \colorbox{Magenta!0.000}{\strut  in} \colorbox{Magenta!0.000}{\strut  Europe} \colorbox{Magenta!0.000}{\strut .} \colorbox{Magenta!93.438}{\strut  All} \\
\midrule
Jacobian & \num{2.521e-01} & \colorbox{Cyan!0.000}{\strut  group} \colorbox{Cyan!0.000}{\strut  and} \colorbox{Cyan!0.000}{\strut  was} \colorbox{Cyan!0.000}{\strut  very} \colorbox{Cyan!0.000}{\strut  much} \colorbox{Cyan!0.000}{\strut  on} \colorbox{Cyan!0.000}{\strut  my} \colorbox{Cyan!0.000}{\strut  level} \colorbox{Cyan!0.000}{\strut .} \colorbox{Cyan!0.000}{\strut  I} \colorbox{Cyan!0.000}{\strut  realise} \colorbox{Cyan!0.000}{\strut  now} \colorbox{Cyan!0.000}{\strut  that} \colorbox{Cyan!98.320}{\strut  all} \colorbox{Cyan!0.000}{\strut  classes} \\
Input SAE & \num{1.713e+01} & \colorbox{Green!0.000}{\strut  group} \colorbox{Green!0.000}{\strut  and} \colorbox{Green!0.000}{\strut  was} \colorbox{Green!0.000}{\strut  very} \colorbox{Green!0.000}{\strut  much} \colorbox{Green!0.000}{\strut  on} \colorbox{Green!0.000}{\strut  my} \colorbox{Green!0.000}{\strut  level} \colorbox{Green!0.000}{\strut .} \colorbox{Green!0.000}{\strut  I} \colorbox{Green!0.000}{\strut  realise} \colorbox{Green!0.000}{\strut  now} \colorbox{Green!0.000}{\strut  that} \colorbox{Green!90.048}{\strut  all} \colorbox{Green!0.000}{\strut  classes} \\
Output SAE & \num{4.219e+00} & \colorbox{Magenta!0.000}{\strut  group} \colorbox{Magenta!0.000}{\strut  and} \colorbox{Magenta!0.000}{\strut  was} \colorbox{Magenta!0.000}{\strut  very} \colorbox{Magenta!0.000}{\strut  much} \colorbox{Magenta!0.000}{\strut  on} \colorbox{Magenta!0.000}{\strut  my} \colorbox{Magenta!0.000}{\strut  level} \colorbox{Magenta!0.000}{\strut .} \colorbox{Magenta!0.000}{\strut  I} \colorbox{Magenta!0.000}{\strut  realise} \colorbox{Magenta!0.000}{\strut  now} \colorbox{Magenta!0.000}{\strut  that} \colorbox{Magenta!91.702}{\strut  all} \colorbox{Magenta!0.000}{\strut  classes} \\
\midrule
Jacobian & \num{2.521e-01} & \colorbox{Cyan!0.000}{\strut  had} \colorbox{Cyan!0.000}{\strut  to} \colorbox{Cyan!0.000}{\strut  learn} \colorbox{Cyan!0.000}{\strut  English} \colorbox{Cyan!0.000}{\strut .} \colorbox{Cyan!0.000}{\strut  I} \colorbox{Cyan!0.000}{\strut  got} \colorbox{Cyan!0.000}{\strut  lucky} \colorbox{Cyan!0.000}{\strut  because} \colorbox{Cyan!98.316}{\strut  all} \colorbox{Cyan!0.000}{\strut  my} \colorbox{Cyan!0.000}{\strut  room} \colorbox{Cyan!0.000}{\strut mates} \colorbox{Cyan!0.000}{\strut  knew} \colorbox{Cyan!0.000}{\strut  Spanish} \\
Input SAE & \num{1.565e+01} & \colorbox{Green!0.000}{\strut  had} \colorbox{Green!0.000}{\strut  to} \colorbox{Green!0.000}{\strut  learn} \colorbox{Green!0.000}{\strut  English} \colorbox{Green!0.000}{\strut .} \colorbox{Green!0.000}{\strut  I} \colorbox{Green!0.000}{\strut  got} \colorbox{Green!0.000}{\strut  lucky} \colorbox{Green!0.000}{\strut  because} \colorbox{Green!82.278}{\strut  all} \colorbox{Green!0.000}{\strut  my} \colorbox{Green!0.000}{\strut  room} \colorbox{Green!0.000}{\strut mates} \colorbox{Green!0.000}{\strut  knew} \colorbox{Green!0.000}{\strut  Spanish} \\
Output SAE & \num{3.950e+00} & \colorbox{Magenta!0.000}{\strut  had} \colorbox{Magenta!0.000}{\strut  to} \colorbox{Magenta!0.000}{\strut  learn} \colorbox{Magenta!0.000}{\strut  English} \colorbox{Magenta!0.000}{\strut .} \colorbox{Magenta!0.000}{\strut  I} \colorbox{Magenta!0.000}{\strut  got} \colorbox{Magenta!0.000}{\strut  lucky} \colorbox{Magenta!0.000}{\strut  because} \colorbox{Magenta!85.860}{\strut  all} \colorbox{Magenta!16.692}{\strut  my} \colorbox{Magenta!0.000}{\strut  room} \colorbox{Magenta!0.000}{\strut mates} \colorbox{Magenta!0.000}{\strut  knew} \colorbox{Magenta!0.000}{\strut  Spanish} \\
\midrule
Jacobian & \num{2.520e-01} & \colorbox{Cyan!0.000}{\strut  happened} \colorbox{Cyan!0.000}{\strut  again} \colorbox{Cyan!0.000}{\strut  but} \colorbox{Cyan!0.000}{\strut  on} \colorbox{Cyan!0.000}{\strut  Halloween} \colorbox{Cyan!0.000}{\strut .} \colorbox{Cyan!98.270}{\strut  All} \colorbox{Cyan!0.000}{\strut  Wal} \colorbox{Cyan!0.000}{\strut m} \colorbox{Cyan!0.000}{\strut arts} \colorbox{Cyan!0.000}{\strut ,} \colorbox{Cyan!0.000}{\strut  the} \colorbox{Cyan!0.000}{\strut  specialty} \colorbox{Cyan!0.000}{\strut  Halloween} \colorbox{Cyan!0.000}{\strut  shops} \\
Input SAE & \num{1.789e+01} & \colorbox{Green!0.000}{\strut  happened} \colorbox{Green!0.000}{\strut  again} \colorbox{Green!0.000}{\strut  but} \colorbox{Green!0.000}{\strut  on} \colorbox{Green!0.000}{\strut  Halloween} \colorbox{Green!0.000}{\strut .} \colorbox{Green!94.072}{\strut  All} \colorbox{Green!0.000}{\strut  Wal} \colorbox{Green!0.000}{\strut m} \colorbox{Green!0.000}{\strut arts} \colorbox{Green!0.000}{\strut ,} \colorbox{Green!0.000}{\strut  the} \colorbox{Green!0.000}{\strut  specialty} \colorbox{Green!0.000}{\strut  Halloween} \colorbox{Green!0.000}{\strut  shops} \\
Output SAE & \num{4.494e+00} & \colorbox{Magenta!0.000}{\strut  happened} \colorbox{Magenta!0.000}{\strut  again} \colorbox{Magenta!0.000}{\strut  but} \colorbox{Magenta!0.000}{\strut  on} \colorbox{Magenta!0.000}{\strut  Halloween} \colorbox{Magenta!0.000}{\strut .} \colorbox{Magenta!97.681}{\strut  All} \colorbox{Magenta!0.000}{\strut  Wal} \colorbox{Magenta!0.000}{\strut m} \colorbox{Magenta!0.000}{\strut arts} \colorbox{Magenta!0.000}{\strut ,} \colorbox{Magenta!0.000}{\strut  the} \colorbox{Magenta!0.000}{\strut  specialty} \colorbox{Magenta!0.000}{\strut  Halloween} \colorbox{Magenta!0.000}{\strut  shops} \\
\bottomrule
\end{longtable}
\caption{feature pairs/Layer15-65536-J1-LR5.0e-04-k32-T3.0e+08 abs mean/examples-38236-v-43039 stas c4-en-10k,train,batch size=32,ctx len=16.csv}
\end{table} % A/all
% \begin{table}
\centering
\begin{longtable}{lrl}
\toprule
Category & Max. abs. value & Example tokens \\
\midrule
Jacobian & \num{2.652e-01} & \colorbox{Cyan!0.000}{\strut  or} \colorbox{Cyan!0.000}{\strut  Asp} \colorbox{Cyan!0.000}{\strut ire} \colorbox{Cyan!74.969}{\strut \textquotesingle{}} \colorbox{Cyan!0.000}{\strut s} \colorbox{Cyan!0.000}{\strut  failure} \colorbox{Cyan!0.000}{\strut  to} \colorbox{Cyan!0.000}{\strut  provide} \colorbox{Cyan!0.000}{\strut  the} \colorbox{Cyan!0.000}{\strut  Site} \colorbox{Cyan!0.000}{\strut .} \colorbox{Cyan!0.000}{\strut  Asp} \colorbox{Cyan!0.000}{\strut ire} \colorbox{Cyan!100.000}{\strut \textquotesingle{}} \colorbox{Cyan!0.000}{\strut s} \\
Input SAE & \num{2.067e+01} & \colorbox{Green!0.000}{\strut  or} \colorbox{Green!0.000}{\strut  Asp} \colorbox{Green!0.000}{\strut ire} \colorbox{Green!51.885}{\strut \textquotesingle{}} \colorbox{Green!0.000}{\strut s} \colorbox{Green!0.000}{\strut  failure} \colorbox{Green!0.000}{\strut  to} \colorbox{Green!0.000}{\strut  provide} \colorbox{Green!0.000}{\strut  the} \colorbox{Green!0.000}{\strut  Site} \colorbox{Green!0.000}{\strut .} \colorbox{Green!0.000}{\strut  Asp} \colorbox{Green!0.000}{\strut ire} \colorbox{Green!90.184}{\strut \textquotesingle{}} \colorbox{Green!0.000}{\strut s} \\
Output SAE & \num{5.279e+00} & \colorbox{Magenta!0.000}{\strut  or} \colorbox{Magenta!0.000}{\strut  Asp} \colorbox{Magenta!0.000}{\strut ire} \colorbox{Magenta!41.540}{\strut \textquotesingle{}} \colorbox{Magenta!0.000}{\strut s} \colorbox{Magenta!0.000}{\strut  failure} \colorbox{Magenta!0.000}{\strut  to} \colorbox{Magenta!0.000}{\strut  provide} \colorbox{Magenta!0.000}{\strut  the} \colorbox{Magenta!0.000}{\strut  Site} \colorbox{Magenta!0.000}{\strut .} \colorbox{Magenta!0.000}{\strut  Asp} \colorbox{Magenta!0.000}{\strut ire} \colorbox{Magenta!97.696}{\strut \textquotesingle{}} \colorbox{Magenta!0.000}{\strut s} \\
\midrule
Jacobian & \num{2.631e-01} & \colorbox{Cyan!0.000}{\strut  influence} \colorbox{Cyan!0.000}{\strut  on} \colorbox{Cyan!0.000}{\strut  Jamie} \colorbox{Cyan!85.285}{\strut \textquotesingle{}} \colorbox{Cyan!0.000}{\strut s} \colorbox{Cyan!0.000}{\strut  food} \colorbox{Cyan!0.000}{\strut  and} \colorbox{Cyan!0.000}{\strut  cooking} \colorbox{Cyan!0.000}{\strut .} \colorbox{Cyan!0.000}{\strut  In} \colorbox{Cyan!0.000}{\strut  Jamie} \colorbox{Cyan!99.184}{\strut \textquotesingle{}} \colorbox{Cyan!0.000}{\strut s} \colorbox{Cyan!0.000}{\strut  Italy} \colorbox{Cyan!0.000}{\strut ,} \\
Input SAE & \num{2.070e+01} & \colorbox{Green!0.000}{\strut  influence} \colorbox{Green!0.000}{\strut  on} \colorbox{Green!0.000}{\strut  Jamie} \colorbox{Green!59.649}{\strut \textquotesingle{}} \colorbox{Green!0.000}{\strut s} \colorbox{Green!0.000}{\strut  food} \colorbox{Green!0.000}{\strut  and} \colorbox{Green!0.000}{\strut  cooking} \colorbox{Green!0.000}{\strut .} \colorbox{Green!0.000}{\strut  In} \colorbox{Green!0.000}{\strut  Jamie} \colorbox{Green!90.338}{\strut \textquotesingle{}} \colorbox{Green!0.000}{\strut s} \colorbox{Green!0.000}{\strut  Italy} \colorbox{Green!0.000}{\strut ,} \\
Output SAE & \num{5.305e+00} & \colorbox{Magenta!0.000}{\strut  influence} \colorbox{Magenta!0.000}{\strut  on} \colorbox{Magenta!0.000}{\strut  Jamie} \colorbox{Magenta!59.356}{\strut \textquotesingle{}} \colorbox{Magenta!0.000}{\strut s} \colorbox{Magenta!0.000}{\strut  food} \colorbox{Magenta!0.000}{\strut  and} \colorbox{Magenta!0.000}{\strut  cooking} \colorbox{Magenta!0.000}{\strut .} \colorbox{Magenta!0.000}{\strut  In} \colorbox{Magenta!0.000}{\strut  Jamie} \colorbox{Magenta!98.184}{\strut \textquotesingle{}} \colorbox{Magenta!0.000}{\strut s} \colorbox{Magenta!0.000}{\strut  Italy} \colorbox{Magenta!0.000}{\strut ,} \\
\midrule
Jacobian & \num{2.625e-01} & \colorbox{Cyan!0.000}{\strut .} \colorbox{Cyan!0.000}{\strut  Second} \colorbox{Cyan!0.000}{\strut  Self} \colorbox{Cyan!81.742}{\strut \textquotesingle{}} \colorbox{Cyan!0.000}{\strut s} \colorbox{Cyan!0.000}{\strut  \textquotedbl{}} \colorbox{Cyan!0.000}{\strut Th} \colorbox{Cyan!0.000}{\strut ai} \colorbox{Cyan!0.000}{\strut  Wheat} \colorbox{Cyan!0.000}{\strut \textquotedbl{}} \colorbox{Cyan!0.000}{\strut  was} \colorbox{Cyan!0.000}{\strut  inspired} \colorbox{Cyan!0.000}{\strut  by} \colorbox{Cyan!0.000}{\strut  Jason} \colorbox{Cyan!98.961}{\strut \textquotesingle{}} \\
Input SAE & \num{2.133e+01} & \colorbox{Green!0.000}{\strut .} \colorbox{Green!0.000}{\strut  Second} \colorbox{Green!0.000}{\strut  Self} \colorbox{Green!63.872}{\strut \textquotesingle{}} \colorbox{Green!0.000}{\strut s} \colorbox{Green!0.000}{\strut  \textquotedbl{}} \colorbox{Green!0.000}{\strut Th} \colorbox{Green!0.000}{\strut ai} \colorbox{Green!0.000}{\strut  Wheat} \colorbox{Green!0.000}{\strut \textquotedbl{}} \colorbox{Green!0.000}{\strut  was} \colorbox{Green!0.000}{\strut  inspired} \colorbox{Green!0.000}{\strut  by} \colorbox{Green!0.000}{\strut  Jason} \colorbox{Green!93.077}{\strut \textquotesingle{}} \\
Output SAE & \num{5.403e+00} & \colorbox{Magenta!0.000}{\strut .} \colorbox{Magenta!0.000}{\strut  Second} \colorbox{Magenta!0.000}{\strut  Self} \colorbox{Magenta!59.811}{\strut \textquotesingle{}} \colorbox{Magenta!0.000}{\strut s} \colorbox{Magenta!0.000}{\strut  \textquotedbl{}} \colorbox{Magenta!0.000}{\strut Th} \colorbox{Magenta!0.000}{\strut ai} \colorbox{Magenta!0.000}{\strut  Wheat} \colorbox{Magenta!0.000}{\strut \textquotedbl{}} \colorbox{Magenta!0.000}{\strut  was} \colorbox{Magenta!0.000}{\strut  inspired} \colorbox{Magenta!0.000}{\strut  by} \colorbox{Magenta!0.000}{\strut  Jason} \colorbox{Magenta!100.000}{\strut \textquotesingle{}} \\
\midrule
Jacobian & \num{2.624e-01} & \colorbox{Cyan!0.000}{\strut  actually} \colorbox{Cyan!0.000}{\strut  not} \colorbox{Cyan!0.000}{\strut  their} \colorbox{Cyan!0.000}{\strut  own} \colorbox{Cyan!0.000}{\strut ;} \colorbox{Cyan!0.000}{\strut  it} \colorbox{Cyan!0.000}{\strut  is} \colorbox{Cyan!0.000}{\strut  God} \colorbox{Cyan!91.638}{\strut \textquotesingle{}} \colorbox{Cyan!0.000}{\strut s} \colorbox{Cyan!0.000}{\strut  initiative} \colorbox{Cyan!0.000}{\strut  and} \colorbox{Cyan!0.000}{\strut  God} \colorbox{Cyan!98.932}{\strut \textquotesingle{}} \colorbox{Cyan!0.000}{\strut s} \\
Input SAE & \num{2.172e+01} & \colorbox{Green!0.000}{\strut  actually} \colorbox{Green!0.000}{\strut  not} \colorbox{Green!0.000}{\strut  their} \colorbox{Green!0.000}{\strut  own} \colorbox{Green!0.000}{\strut ;} \colorbox{Green!0.000}{\strut  it} \colorbox{Green!0.000}{\strut  is} \colorbox{Green!0.000}{\strut  God} \colorbox{Green!72.331}{\strut \textquotesingle{}} \colorbox{Green!0.000}{\strut s} \colorbox{Green!0.000}{\strut  initiative} \colorbox{Green!0.000}{\strut  and} \colorbox{Green!0.000}{\strut  God} \colorbox{Green!94.774}{\strut \textquotesingle{}} \colorbox{Green!0.000}{\strut s} \\
Output SAE & \num{5.066e+00} & \colorbox{Magenta!0.000}{\strut  actually} \colorbox{Magenta!0.000}{\strut  not} \colorbox{Magenta!0.000}{\strut  their} \colorbox{Magenta!0.000}{\strut  own} \colorbox{Magenta!0.000}{\strut ;} \colorbox{Magenta!0.000}{\strut  it} \colorbox{Magenta!0.000}{\strut  is} \colorbox{Magenta!0.000}{\strut  God} \colorbox{Magenta!74.373}{\strut \textquotesingle{}} \colorbox{Magenta!0.000}{\strut s} \colorbox{Magenta!0.000}{\strut  initiative} \colorbox{Magenta!0.000}{\strut  and} \colorbox{Magenta!0.000}{\strut  God} \colorbox{Magenta!93.761}{\strut \textquotesingle{}} \colorbox{Magenta!0.000}{\strut s} \\
\midrule
Jacobian & \num{2.623e-01} & \colorbox{Cyan!0.000}{\strut Pl} \colorbox{Cyan!0.000}{\strut atte} \colorbox{Cyan!87.704}{\strut \textquotesingle{}} \colorbox{Cyan!0.000}{\strut s} \colorbox{Cyan!0.000}{\strut  father} \colorbox{Cyan!0.000}{\strut  encouraged} \colorbox{Cyan!0.000}{\strut  his} \colorbox{Cyan!0.000}{\strut  son} \colorbox{Cyan!98.890}{\strut \textquotesingle{}} \colorbox{Cyan!0.000}{\strut s} \colorbox{Cyan!0.000}{\strut  passion} \colorbox{Cyan!0.000}{\strut  for} \colorbox{Cyan!0.000}{\strut  aviation} \colorbox{Cyan!0.000}{\strut  by} \\
Input SAE & \num{2.032e+01} & \colorbox{Green!0.000}{\strut Pl} \colorbox{Green!0.000}{\strut atte} \colorbox{Green!70.390}{\strut \textquotesingle{}} \colorbox{Green!0.000}{\strut s} \colorbox{Green!0.000}{\strut  father} \colorbox{Green!0.000}{\strut  encouraged} \colorbox{Green!0.000}{\strut  his} \colorbox{Green!0.000}{\strut  son} \colorbox{Green!88.663}{\strut \textquotesingle{}} \colorbox{Green!0.000}{\strut s} \colorbox{Green!0.000}{\strut  passion} \colorbox{Green!0.000}{\strut  for} \colorbox{Green!0.000}{\strut  aviation} \colorbox{Green!0.000}{\strut  by} \\
Output SAE & \num{5.391e+00} & \colorbox{Magenta!0.000}{\strut Pl} \colorbox{Magenta!0.000}{\strut atte} \colorbox{Magenta!70.834}{\strut \textquotesingle{}} \colorbox{Magenta!0.000}{\strut s} \colorbox{Magenta!0.000}{\strut  father} \colorbox{Magenta!0.000}{\strut  encouraged} \colorbox{Magenta!0.000}{\strut  his} \colorbox{Magenta!0.000}{\strut  son} \colorbox{Magenta!99.783}{\strut \textquotesingle{}} \colorbox{Magenta!0.000}{\strut s} \colorbox{Magenta!0.000}{\strut  passion} \colorbox{Magenta!0.000}{\strut  for} \colorbox{Magenta!0.000}{\strut  aviation} \colorbox{Magenta!0.000}{\strut  by} \\
\midrule
Jacobian & \num{2.611e-01} & \colorbox{Cyan!0.000}{\strut  The} \colorbox{Cyan!0.000}{\strut  upper} \colorbox{Cyan!0.000}{\strut  section} \colorbox{Cyan!0.000}{\strut  of} \colorbox{Cyan!0.000}{\strut  the} \colorbox{Cyan!0.000}{\strut  Henry} \colorbox{Cyan!86.988}{\strut \textquotesingle{}} \colorbox{Cyan!0.000}{\strut s} \colorbox{Cyan!0.000}{\strut  F} \colorbox{Cyan!0.000}{\strut ork} \colorbox{Cyan!0.000}{\strut  is} \colorbox{Cyan!0.000}{\strut  formed} \colorbox{Cyan!0.000}{\strut  where} \colorbox{Cyan!0.000}{\strut  Henry} \colorbox{Cyan!98.436}{\strut \textquotesingle{}} \\
Input SAE & \num{2.189e+01} & \colorbox{Green!0.000}{\strut  The} \colorbox{Green!0.000}{\strut  upper} \colorbox{Green!0.000}{\strut  section} \colorbox{Green!0.000}{\strut  of} \colorbox{Green!0.000}{\strut  the} \colorbox{Green!0.000}{\strut  Henry} \colorbox{Green!64.371}{\strut \textquotesingle{}} \colorbox{Green!0.000}{\strut s} \colorbox{Green!0.000}{\strut  F} \colorbox{Green!0.000}{\strut ork} \colorbox{Green!0.000}{\strut  is} \colorbox{Green!0.000}{\strut  formed} \colorbox{Green!0.000}{\strut  where} \colorbox{Green!0.000}{\strut  Henry} \colorbox{Green!95.526}{\strut \textquotesingle{}} \\
Output SAE & \num{5.370e+00} & \colorbox{Magenta!0.000}{\strut  The} \colorbox{Magenta!0.000}{\strut  upper} \colorbox{Magenta!0.000}{\strut  section} \colorbox{Magenta!0.000}{\strut  of} \colorbox{Magenta!0.000}{\strut  the} \colorbox{Magenta!0.000}{\strut  Henry} \colorbox{Magenta!65.593}{\strut \textquotesingle{}} \colorbox{Magenta!0.000}{\strut s} \colorbox{Magenta!0.000}{\strut  F} \colorbox{Magenta!0.000}{\strut ork} \colorbox{Magenta!0.000}{\strut  is} \colorbox{Magenta!0.000}{\strut  formed} \colorbox{Magenta!0.000}{\strut  where} \colorbox{Magenta!0.000}{\strut  Henry} \colorbox{Magenta!99.383}{\strut \textquotesingle{}} \\
\midrule
Jacobian & \num{2.594e-01} & \colorbox{Cyan!0.000}{\strut  on} \colorbox{Cyan!0.000}{\strut  a} \colorbox{Cyan!0.000}{\strut  particular} \colorbox{Cyan!0.000}{\strut  case} \colorbox{Cyan!0.000}{\strut ;} \colorbox{Cyan!0.000}{\strut  it} \colorbox{Cyan!79.630}{\strut \textquotesingle{}} \colorbox{Cyan!0.000}{\strut s} \colorbox{Cyan!0.000}{\strut  not} \colorbox{Cyan!0.000}{\strut  a} \colorbox{Cyan!0.000}{\strut  statement} \colorbox{Cyan!0.000}{\strut  on} \colorbox{Cyan!0.000}{\strut  the} \colorbox{Cyan!0.000}{\strut  commissioner} \colorbox{Cyan!97.789}{\strut \textquotesingle{}} \\
Input SAE & \num{1.900e+01} & \colorbox{Green!0.000}{\strut  on} \colorbox{Green!0.000}{\strut  a} \colorbox{Green!0.000}{\strut  particular} \colorbox{Green!0.000}{\strut  case} \colorbox{Green!0.000}{\strut ;} \colorbox{Green!0.000}{\strut  it} \colorbox{Green!49.473}{\strut \textquotesingle{}} \colorbox{Green!0.000}{\strut s} \colorbox{Green!0.000}{\strut  not} \colorbox{Green!0.000}{\strut  a} \colorbox{Green!0.000}{\strut  statement} \colorbox{Green!0.000}{\strut  on} \colorbox{Green!0.000}{\strut  the} \colorbox{Green!0.000}{\strut  commissioner} \colorbox{Green!82.900}{\strut \textquotesingle{}} \\
Output SAE & \num{5.000e+00} & \colorbox{Magenta!0.000}{\strut  on} \colorbox{Magenta!0.000}{\strut  a} \colorbox{Magenta!0.000}{\strut  particular} \colorbox{Magenta!0.000}{\strut  case} \colorbox{Magenta!0.000}{\strut ;} \colorbox{Magenta!0.000}{\strut  it} \colorbox{Magenta!52.547}{\strut \textquotesingle{}} \colorbox{Magenta!0.000}{\strut s} \colorbox{Magenta!0.000}{\strut  not} \colorbox{Magenta!0.000}{\strut  a} \colorbox{Magenta!0.000}{\strut  statement} \colorbox{Magenta!0.000}{\strut  on} \colorbox{Magenta!0.000}{\strut  the} \colorbox{Magenta!0.000}{\strut  commissioner} \colorbox{Magenta!92.537}{\strut \textquotesingle{}} \\
\midrule
Jacobian & \num{2.592e-01} & \colorbox{Cyan!0.000}{\strut The} \colorbox{Cyan!0.000}{\strut  hotel} \colorbox{Cyan!87.944}{\strut \textquotesingle{}} \colorbox{Cyan!0.000}{\strut s} \colorbox{Cyan!0.000}{\strut  sk} \colorbox{Cyan!0.000}{\strut inc} \colorbox{Cyan!0.000}{\strut  spa} \colorbox{Cyan!0.000}{\strut  is} \colorbox{Cyan!0.000}{\strut  home} \colorbox{Cyan!0.000}{\strut  to} \colorbox{Cyan!0.000}{\strut  the} \colorbox{Cyan!0.000}{\strut  world} \colorbox{Cyan!97.711}{\strut \textquotesingle{}} \colorbox{Cyan!0.000}{\strut s} \\
Input SAE & \num{1.911e+01} & \colorbox{Green!0.000}{\strut The} \colorbox{Green!0.000}{\strut  hotel} \colorbox{Green!67.229}{\strut \textquotesingle{}} \colorbox{Green!0.000}{\strut s} \colorbox{Green!0.000}{\strut  sk} \colorbox{Green!0.000}{\strut inc} \colorbox{Green!0.000}{\strut  spa} \colorbox{Green!0.000}{\strut  is} \colorbox{Green!0.000}{\strut  home} \colorbox{Green!0.000}{\strut  to} \colorbox{Green!0.000}{\strut  the} \colorbox{Green!0.000}{\strut  world} \colorbox{Green!83.394}{\strut \textquotesingle{}} \colorbox{Green!0.000}{\strut s} \\
Output SAE & \num{4.852e+00} & \colorbox{Magenta!0.000}{\strut The} \colorbox{Magenta!0.000}{\strut  hotel} \colorbox{Magenta!68.786}{\strut \textquotesingle{}} \colorbox{Magenta!0.000}{\strut s} \colorbox{Magenta!0.000}{\strut  sk} \colorbox{Magenta!0.000}{\strut inc} \colorbox{Magenta!0.000}{\strut  spa} \colorbox{Magenta!0.000}{\strut  is} \colorbox{Magenta!0.000}{\strut  home} \colorbox{Magenta!0.000}{\strut  to} \colorbox{Magenta!0.000}{\strut  the} \colorbox{Magenta!0.000}{\strut  world} \colorbox{Magenta!89.793}{\strut \textquotesingle{}} \colorbox{Magenta!0.000}{\strut s} \\
\midrule
Jacobian & \num{2.590e-01} & \colorbox{Cyan!0.000}{\strut .} \colorbox{Cyan!0.000}{\strut  But} \colorbox{Cyan!0.000}{\strut  it} \colorbox{Cyan!80.422}{\strut \textquotesingle{}} \colorbox{Cyan!0.000}{\strut s} \colorbox{Cyan!0.000}{\strut  caused} \colorbox{Cyan!0.000}{\strut  by} \colorbox{Cyan!0.000}{\strut  two} \colorbox{Cyan!0.000}{\strut  things} \colorbox{Cyan!0.000}{\strut :} \colorbox{Cyan!0.000}{\strut  The} \colorbox{Cyan!0.000}{\strut  Earth} \colorbox{Cyan!97.663}{\strut \textquotesingle{}} \colorbox{Cyan!0.000}{\strut s} \colorbox{Cyan!0.000}{\strut  axis} \\
Input SAE & \num{2.156e+01} & \colorbox{Green!0.000}{\strut .} \colorbox{Green!0.000}{\strut  But} \colorbox{Green!0.000}{\strut  it} \colorbox{Green!52.985}{\strut \textquotesingle{}} \colorbox{Green!0.000}{\strut s} \colorbox{Green!0.000}{\strut  caused} \colorbox{Green!0.000}{\strut  by} \colorbox{Green!0.000}{\strut  two} \colorbox{Green!0.000}{\strut  things} \colorbox{Green!0.000}{\strut :} \colorbox{Green!0.000}{\strut  The} \colorbox{Green!0.000}{\strut  Earth} \colorbox{Green!94.087}{\strut \textquotesingle{}} \colorbox{Green!0.000}{\strut s} \colorbox{Green!0.000}{\strut  axis} \\
Output SAE & \num{5.328e+00} & \colorbox{Magenta!0.000}{\strut .} \colorbox{Magenta!0.000}{\strut  But} \colorbox{Magenta!0.000}{\strut  it} \colorbox{Magenta!54.366}{\strut \textquotesingle{}} \colorbox{Magenta!10.491}{\strut s} \colorbox{Magenta!0.000}{\strut  caused} \colorbox{Magenta!0.000}{\strut  by} \colorbox{Magenta!0.000}{\strut  two} \colorbox{Magenta!0.000}{\strut  things} \colorbox{Magenta!0.000}{\strut :} \colorbox{Magenta!0.000}{\strut  The} \colorbox{Magenta!0.000}{\strut  Earth} \colorbox{Magenta!98.604}{\strut \textquotesingle{}} \colorbox{Magenta!0.000}{\strut s} \colorbox{Magenta!0.000}{\strut  axis} \\
\midrule
Jacobian & \num{2.588e-01} & \colorbox{Cyan!0.000}{\strut Every} \colorbox{Cyan!0.000}{\strut  book} \colorbox{Cyan!0.000}{\strut  worm} \colorbox{Cyan!86.301}{\strut \textquotesingle{}} \colorbox{Cyan!0.000}{\strut s} \colorbox{Cyan!0.000}{\strut  dreams} \colorbox{Cyan!0.000}{\strut  exist} \colorbox{Cyan!0.000}{\strut  at} \colorbox{Cyan!0.000}{\strut  Powell} \colorbox{Cyan!97.589}{\strut \textquotesingle{}} \colorbox{Cyan!0.000}{\strut s} \colorbox{Cyan!0.000}{\strut  Books} \colorbox{Cyan!0.000}{\strut ,} \colorbox{Cyan!0.000}{\strut  the} \\
Input SAE & \num{1.943e+01} & \colorbox{Green!0.000}{\strut Every} \colorbox{Green!0.000}{\strut  book} \colorbox{Green!0.000}{\strut  worm} \colorbox{Green!68.848}{\strut \textquotesingle{}} \colorbox{Green!0.000}{\strut s} \colorbox{Green!0.000}{\strut  dreams} \colorbox{Green!0.000}{\strut  exist} \colorbox{Green!0.000}{\strut  at} \colorbox{Green!0.000}{\strut  Powell} \colorbox{Green!84.768}{\strut \textquotesingle{}} \colorbox{Green!0.000}{\strut s} \colorbox{Green!0.000}{\strut  Books} \colorbox{Green!0.000}{\strut ,} \colorbox{Green!0.000}{\strut  the} \\
Output SAE & \num{4.929e+00} & \colorbox{Magenta!0.000}{\strut Every} \colorbox{Magenta!0.000}{\strut  book} \colorbox{Magenta!0.000}{\strut  worm} \colorbox{Magenta!67.477}{\strut \textquotesingle{}} \colorbox{Magenta!0.000}{\strut s} \colorbox{Magenta!0.000}{\strut  dreams} \colorbox{Magenta!0.000}{\strut  exist} \colorbox{Magenta!0.000}{\strut  at} \colorbox{Magenta!0.000}{\strut  Powell} \colorbox{Magenta!91.225}{\strut \textquotesingle{}} \colorbox{Magenta!0.000}{\strut s} \colorbox{Magenta!0.000}{\strut  Books} \colorbox{Magenta!12.560}{\strut ,} \colorbox{Magenta!0.000}{\strut  the} \\
\midrule
Jacobian & \num{2.588e-01} & \colorbox{Cyan!0.000}{\strut  the} \colorbox{Cyan!0.000}{\strut  hotel} \colorbox{Cyan!83.741}{\strut \textquotesingle{}} \colorbox{Cyan!0.000}{\strut s} \colorbox{Cyan!0.000}{\strut  99} \colorbox{Cyan!0.000}{\strut  bedrooms} \colorbox{Cyan!0.000}{\strut .} \colorbox{Cyan!0.000}{\strut  The} \colorbox{Cyan!0.000}{\strut  hotel} \colorbox{Cyan!97.565}{\strut \textquotesingle{}} \colorbox{Cyan!0.000}{\strut s} \colorbox{Cyan!0.000}{\strut  Classic} \colorbox{Cyan!0.000}{\strut  rooms} \colorbox{Cyan!0.000}{\strut  offer} \colorbox{Cyan!0.000}{\strut  everything} \\
Input SAE & \num{2.020e+01} & \colorbox{Green!0.000}{\strut  the} \colorbox{Green!0.000}{\strut  hotel} \colorbox{Green!63.603}{\strut \textquotesingle{}} \colorbox{Green!0.000}{\strut s} \colorbox{Green!0.000}{\strut  99} \colorbox{Green!0.000}{\strut  bedrooms} \colorbox{Green!0.000}{\strut .} \colorbox{Green!0.000}{\strut  The} \colorbox{Green!0.000}{\strut  hotel} \colorbox{Green!88.170}{\strut \textquotesingle{}} \colorbox{Green!0.000}{\strut s} \colorbox{Green!0.000}{\strut  Classic} \colorbox{Green!0.000}{\strut  rooms} \colorbox{Green!0.000}{\strut  offer} \colorbox{Green!0.000}{\strut  everything} \\
Output SAE & \num{5.086e+00} & \colorbox{Magenta!0.000}{\strut  the} \colorbox{Magenta!0.000}{\strut  hotel} \colorbox{Magenta!60.813}{\strut \textquotesingle{}} \colorbox{Magenta!0.000}{\strut s} \colorbox{Magenta!0.000}{\strut  99} \colorbox{Magenta!0.000}{\strut  bedrooms} \colorbox{Magenta!0.000}{\strut .} \colorbox{Magenta!0.000}{\strut  The} \colorbox{Magenta!0.000}{\strut  hotel} \colorbox{Magenta!94.126}{\strut \textquotesingle{}} \colorbox{Magenta!0.000}{\strut s} \colorbox{Magenta!0.000}{\strut  Classic} \colorbox{Magenta!0.000}{\strut  rooms} \colorbox{Magenta!0.000}{\strut  offer} \colorbox{Magenta!0.000}{\strut  everything} \\
\midrule
Jacobian & \num{2.587e-01} & \colorbox{Cyan!0.000}{\strut  a} \colorbox{Cyan!0.000}{\strut  success} \colorbox{Cyan!0.000}{\strut .} \colorbox{Cyan!0.000}{\strut \textquotedbl{}} \colorbox{Cyan!0.000}{\strut People} \colorbox{Cyan!0.000}{\strut  will} \colorbox{Cyan!0.000}{\strut  look} \colorbox{Cyan!0.000}{\strut  back} \colorbox{Cyan!0.000}{\strut  upon} \colorbox{Cyan!0.000}{\strut  Jamie} \colorbox{Cyan!0.000}{\strut  D} \colorbox{Cyan!0.000}{\strut imon} \colorbox{Cyan!97.529}{\strut \textquotesingle{}} \colorbox{Cyan!0.000}{\strut s} \\
Input SAE & \num{2.018e+01} & \colorbox{Green!0.000}{\strut  a} \colorbox{Green!0.000}{\strut  success} \colorbox{Green!0.000}{\strut .} \colorbox{Green!0.000}{\strut \textquotedbl{}} \colorbox{Green!0.000}{\strut People} \colorbox{Green!0.000}{\strut  will} \colorbox{Green!0.000}{\strut  look} \colorbox{Green!0.000}{\strut  back} \colorbox{Green!0.000}{\strut  upon} \colorbox{Green!0.000}{\strut  Jamie} \colorbox{Green!0.000}{\strut  D} \colorbox{Green!0.000}{\strut imon} \colorbox{Green!88.063}{\strut \textquotesingle{}} \colorbox{Green!0.000}{\strut s} \\
Output SAE & \num{5.019e+00} & \colorbox{Magenta!0.000}{\strut  a} \colorbox{Magenta!0.000}{\strut  success} \colorbox{Magenta!0.000}{\strut .} \colorbox{Magenta!11.158}{\strut \textquotedbl{}} \colorbox{Magenta!0.000}{\strut People} \colorbox{Magenta!0.000}{\strut  will} \colorbox{Magenta!0.000}{\strut  look} \colorbox{Magenta!0.000}{\strut  back} \colorbox{Magenta!0.000}{\strut  upon} \colorbox{Magenta!0.000}{\strut  Jamie} \colorbox{Magenta!0.000}{\strut  D} \colorbox{Magenta!0.000}{\strut imon} \colorbox{Magenta!92.894}{\strut \textquotesingle{}} \colorbox{Magenta!0.000}{\strut s} \\
\bottomrule
\end{longtable}
\caption{feature pairs/Layer15-65536-J1-LR5.0e-04-k32-T3.0e+08 abs mean/examples-40314-v-43623 stas c4-en-10k,train,batch size=32,ctx len=16.csv}
\end{table} % single quotes
% \begin{table}
\centering
\begin{longtable}{lrl}
\toprule
Category & Max. abs. value & Example tokens \\
\midrule
Jacobian & \num{2.467e-01} & \colorbox{Cyan!0.000}{\strut  E} \colorbox{Cyan!0.000}{\strut  PR} \colorbox{Cyan!0.000}{\strut OD} \colorbox{Cyan!0.000}{\strut UN} \colorbox{Cyan!0.000}{\strut CTIONS} \colorbox{Cyan!0.000}{\strut .} \colorbox{Cyan!0.000}{\strut damn} \colorbox{Cyan!0.000}{\strut .} \colorbox{Cyan!0.000}{\strut  great} \colorbox{Cyan!0.000}{\strut  talent} \colorbox{Cyan!0.000}{\strut  gone} \colorbox{Cyan!100.000}{\strut  just} \colorbox{Cyan!0.000}{\strut ly} \colorbox{Cyan!0.000}{\strut k} \\
Input SAE & \num{1.814e+01} & \colorbox{Green!0.000}{\strut  E} \colorbox{Green!0.000}{\strut  PR} \colorbox{Green!0.000}{\strut OD} \colorbox{Green!0.000}{\strut UN} \colorbox{Green!0.000}{\strut CTIONS} \colorbox{Green!0.000}{\strut .} \colorbox{Green!0.000}{\strut damn} \colorbox{Green!0.000}{\strut .} \colorbox{Green!0.000}{\strut  great} \colorbox{Green!0.000}{\strut  talent} \colorbox{Green!0.000}{\strut  gone} \colorbox{Green!86.562}{\strut  just} \colorbox{Green!0.000}{\strut ly} \colorbox{Green!0.000}{\strut k} \\
Output SAE & \num{4.366e+00} & \colorbox{Magenta!0.000}{\strut  E} \colorbox{Magenta!0.000}{\strut  PR} \colorbox{Magenta!0.000}{\strut OD} \colorbox{Magenta!0.000}{\strut UN} \colorbox{Magenta!0.000}{\strut CTIONS} \colorbox{Magenta!0.000}{\strut .} \colorbox{Magenta!0.000}{\strut damn} \colorbox{Magenta!0.000}{\strut .} \colorbox{Magenta!0.000}{\strut  great} \colorbox{Magenta!0.000}{\strut  talent} \colorbox{Magenta!0.000}{\strut  gone} \colorbox{Magenta!86.511}{\strut  just} \colorbox{Magenta!0.000}{\strut ly} \colorbox{Magenta!0.000}{\strut k} \\
\midrule
Jacobian & \num{2.466e-01} & \colorbox{Cyan!0.000}{\strut et} \colorbox{Cyan!0.000}{\strut \textquotedbl{}} \colorbox{Cyan!0.000}{\strut  came} \colorbox{Cyan!0.000}{\strut  out} \colorbox{Cyan!0.000}{\strut  from} \colorbox{Cyan!0.000}{\strut  my} \colorbox{Cyan!0.000}{\strut  lips} \colorbox{Cyan!0.000}{\strut  it} \colorbox{Cyan!0.000}{\strut  came} \colorbox{Cyan!0.000}{\strut  out} \colorbox{Cyan!99.954}{\strut  just} \colorbox{Cyan!0.000}{\strut  as} \colorbox{Cyan!0.000}{\strut  a} \colorbox{Cyan!0.000}{\strut  whisper} \colorbox{Cyan!0.000}{\strut  and} \\
Input SAE & \num{1.700e+01} & \colorbox{Green!0.000}{\strut et} \colorbox{Green!0.000}{\strut \textquotedbl{}} \colorbox{Green!0.000}{\strut  came} \colorbox{Green!0.000}{\strut  out} \colorbox{Green!0.000}{\strut  from} \colorbox{Green!0.000}{\strut  my} \colorbox{Green!0.000}{\strut  lips} \colorbox{Green!0.000}{\strut  it} \colorbox{Green!0.000}{\strut  came} \colorbox{Green!0.000}{\strut  out} \colorbox{Green!81.110}{\strut  just} \colorbox{Green!0.000}{\strut  as} \colorbox{Green!0.000}{\strut  a} \colorbox{Green!0.000}{\strut  whisper} \colorbox{Green!0.000}{\strut  and} \\
Output SAE & \num{4.294e+00} & \colorbox{Magenta!0.000}{\strut et} \colorbox{Magenta!0.000}{\strut \textquotedbl{}} \colorbox{Magenta!0.000}{\strut  came} \colorbox{Magenta!0.000}{\strut  out} \colorbox{Magenta!0.000}{\strut  from} \colorbox{Magenta!0.000}{\strut  my} \colorbox{Magenta!0.000}{\strut  lips} \colorbox{Magenta!0.000}{\strut  it} \colorbox{Magenta!0.000}{\strut  came} \colorbox{Magenta!0.000}{\strut  out} \colorbox{Magenta!85.083}{\strut  just} \colorbox{Magenta!0.000}{\strut  as} \colorbox{Magenta!12.873}{\strut  a} \colorbox{Magenta!0.000}{\strut  whisper} \colorbox{Magenta!0.000}{\strut  and} \\
\midrule
Jacobian & \num{2.464e-01} & \colorbox{Cyan!0.000}{\strut  is} \colorbox{Cyan!0.000}{\strut  still} \colorbox{Cyan!0.000}{\strut  very} \colorbox{Cyan!0.000}{\strut  young} \colorbox{Cyan!0.000}{\strut ,} \colorbox{Cyan!0.000}{\strut  exc} \colorbox{Cyan!0.000}{\strut itable} \colorbox{Cyan!0.000}{\strut ,} \colorbox{Cyan!0.000}{\strut  and} \colorbox{Cyan!0.000}{\strut  interested} \colorbox{Cyan!0.000}{\strut  in} \colorbox{Cyan!99.894}{\strut  just} \colorbox{Cyan!72.743}{\strut  about} \colorbox{Cyan!0.000}{\strut  everything} \colorbox{Cyan!0.000}{\strut .} \\
Input SAE & \num{1.886e+01} & \colorbox{Green!0.000}{\strut  is} \colorbox{Green!0.000}{\strut  still} \colorbox{Green!0.000}{\strut  very} \colorbox{Green!0.000}{\strut  young} \colorbox{Green!0.000}{\strut ,} \colorbox{Green!0.000}{\strut  exc} \colorbox{Green!0.000}{\strut itable} \colorbox{Green!0.000}{\strut ,} \colorbox{Green!0.000}{\strut  and} \colorbox{Green!0.000}{\strut  interested} \colorbox{Green!0.000}{\strut  in} \colorbox{Green!89.985}{\strut  just} \colorbox{Green!7.364}{\strut  about} \colorbox{Green!0.000}{\strut  everything} \colorbox{Green!0.000}{\strut .} \\
Output SAE & \num{4.400e+00} & \colorbox{Magenta!0.000}{\strut  is} \colorbox{Magenta!0.000}{\strut  still} \colorbox{Magenta!0.000}{\strut  very} \colorbox{Magenta!0.000}{\strut  young} \colorbox{Magenta!0.000}{\strut ,} \colorbox{Magenta!0.000}{\strut  exc} \colorbox{Magenta!0.000}{\strut itable} \colorbox{Magenta!0.000}{\strut ,} \colorbox{Magenta!0.000}{\strut  and} \colorbox{Magenta!0.000}{\strut  interested} \colorbox{Magenta!0.000}{\strut  in} \colorbox{Magenta!87.199}{\strut  just} \colorbox{Magenta!23.252}{\strut  about} \colorbox{Magenta!0.000}{\strut  everything} \colorbox{Magenta!0.000}{\strut .} \\
\midrule
Jacobian & \num{2.463e-01} & \colorbox{Cyan!0.000}{\strut  so} \colorbox{Cyan!0.000}{\strut  much} \colorbox{Cyan!0.000}{\strut  Miss} \colorbox{Cyan!0.000}{\strut  G} \colorbox{Cyan!0.000}{\strut !} \colorbox{Cyan!0.000}{\strut This} \colorbox{Cyan!0.000}{\strut  happened} \colorbox{Cyan!0.000}{\strut  to} \colorbox{Cyan!0.000}{\strut  me} \colorbox{Cyan!99.821}{\strut  just} \colorbox{Cyan!0.000}{\strut  today} \colorbox{Cyan!0.000}{\strut .} \colorbox{Cyan!0.000}{\strut  I} \colorbox{Cyan!0.000}{\strut  had} \\
Input SAE & \num{1.703e+01} & \colorbox{Green!0.000}{\strut  so} \colorbox{Green!0.000}{\strut  much} \colorbox{Green!0.000}{\strut  Miss} \colorbox{Green!0.000}{\strut  G} \colorbox{Green!0.000}{\strut !} \colorbox{Green!0.000}{\strut This} \colorbox{Green!0.000}{\strut  happened} \colorbox{Green!0.000}{\strut  to} \colorbox{Green!0.000}{\strut  me} \colorbox{Green!81.241}{\strut  just} \colorbox{Green!0.000}{\strut  today} \colorbox{Green!0.000}{\strut .} \colorbox{Green!0.000}{\strut  I} \colorbox{Green!0.000}{\strut  had} \\
Output SAE & \num{4.430e+00} & \colorbox{Magenta!0.000}{\strut  so} \colorbox{Magenta!0.000}{\strut  much} \colorbox{Magenta!0.000}{\strut  Miss} \colorbox{Magenta!0.000}{\strut  G} \colorbox{Magenta!0.000}{\strut !} \colorbox{Magenta!0.000}{\strut This} \colorbox{Magenta!0.000}{\strut  happened} \colorbox{Magenta!0.000}{\strut  to} \colorbox{Magenta!0.000}{\strut  me} \colorbox{Magenta!87.787}{\strut  just} \colorbox{Magenta!0.000}{\strut  today} \colorbox{Magenta!0.000}{\strut .} \colorbox{Magenta!0.000}{\strut  I} \colorbox{Magenta!0.000}{\strut  had} \\
\midrule
Jacobian & \num{2.460e-01} & \colorbox{Cyan!0.000}{\strut an} \colorbox{Cyan!0.000}{\strut ufact} \colorbox{Cyan!0.000}{\strut ured} \colorbox{Cyan!0.000}{\strut  tread} \colorbox{Cyan!0.000}{\strut m} \colorbox{Cyan!0.000}{\strut ills} \colorbox{Cyan!0.000}{\strut  or} \colorbox{Cyan!0.000}{\strut  ref} \colorbox{Cyan!0.000}{\strut urb} \colorbox{Cyan!0.000}{\strut ished} \colorbox{Cyan!0.000}{\strut  tread} \colorbox{Cyan!0.000}{\strut m} \colorbox{Cyan!0.000}{\strut ills} \colorbox{Cyan!0.000}{\strut  means} \colorbox{Cyan!99.735}{\strut  just} \\
Input SAE & \num{1.774e+01} & \colorbox{Green!0.000}{\strut an} \colorbox{Green!0.000}{\strut ufact} \colorbox{Green!0.000}{\strut ured} \colorbox{Green!0.000}{\strut  tread} \colorbox{Green!0.000}{\strut m} \colorbox{Green!0.000}{\strut ills} \colorbox{Green!0.000}{\strut  or} \colorbox{Green!0.000}{\strut  ref} \colorbox{Green!0.000}{\strut urb} \colorbox{Green!0.000}{\strut ished} \colorbox{Green!0.000}{\strut  tread} \colorbox{Green!0.000}{\strut m} \colorbox{Green!0.000}{\strut ills} \colorbox{Green!0.000}{\strut  means} \colorbox{Green!84.654}{\strut  just} \\
Output SAE & \num{4.468e+00} & \colorbox{Magenta!0.000}{\strut an} \colorbox{Magenta!0.000}{\strut ufact} \colorbox{Magenta!0.000}{\strut ured} \colorbox{Magenta!0.000}{\strut  tread} \colorbox{Magenta!0.000}{\strut m} \colorbox{Magenta!0.000}{\strut ills} \colorbox{Magenta!0.000}{\strut  or} \colorbox{Magenta!0.000}{\strut  ref} \colorbox{Magenta!0.000}{\strut urb} \colorbox{Magenta!0.000}{\strut ished} \colorbox{Magenta!0.000}{\strut  tread} \colorbox{Magenta!0.000}{\strut m} \colorbox{Magenta!0.000}{\strut ills} \colorbox{Magenta!0.000}{\strut  means} \colorbox{Magenta!88.541}{\strut  just} \\
\midrule
Jacobian & \num{2.455e-01} & \colorbox{Cyan!0.000}{\strut  look} \colorbox{Cyan!0.000}{\strut  thinking} \colorbox{Cyan!0.000}{\strut  who} \colorbox{Cyan!0.000}{\strut  b} \colorbox{Cyan!0.000}{\strut anged} \colorbox{Cyan!0.000}{\strut  on} \colorbox{Cyan!0.000}{\strut  the} \colorbox{Cyan!0.000}{\strut  door} \colorbox{Cyan!0.000}{\strut .} \colorbox{Cyan!99.516}{\strut  Just} \colorbox{Cyan!0.000}{\strut  when} \colorbox{Cyan!0.000}{\strut  she} \colorbox{Cyan!0.000}{\strut  thought} \colorbox{Cyan!0.000}{\strut  to} \colorbox{Cyan!0.000}{\strut  close} \\
Input SAE & \num{1.720e+01} & \colorbox{Green!0.000}{\strut  look} \colorbox{Green!0.000}{\strut  thinking} \colorbox{Green!0.000}{\strut  who} \colorbox{Green!0.000}{\strut  b} \colorbox{Green!0.000}{\strut anged} \colorbox{Green!0.000}{\strut  on} \colorbox{Green!0.000}{\strut  the} \colorbox{Green!0.000}{\strut  door} \colorbox{Green!0.000}{\strut .} \colorbox{Green!82.077}{\strut  Just} \colorbox{Green!0.000}{\strut  when} \colorbox{Green!0.000}{\strut  she} \colorbox{Green!0.000}{\strut  thought} \colorbox{Green!0.000}{\strut  to} \colorbox{Green!0.000}{\strut  close} \\
Output SAE & \num{4.718e+00} & \colorbox{Magenta!0.000}{\strut  look} \colorbox{Magenta!0.000}{\strut  thinking} \colorbox{Magenta!0.000}{\strut  who} \colorbox{Magenta!0.000}{\strut  b} \colorbox{Magenta!0.000}{\strut anged} \colorbox{Magenta!0.000}{\strut  on} \colorbox{Magenta!0.000}{\strut  the} \colorbox{Magenta!0.000}{\strut  door} \colorbox{Magenta!0.000}{\strut .} \colorbox{Magenta!93.491}{\strut  Just} \colorbox{Magenta!0.000}{\strut  when} \colorbox{Magenta!0.000}{\strut  she} \colorbox{Magenta!0.000}{\strut  thought} \colorbox{Magenta!0.000}{\strut  to} \colorbox{Magenta!0.000}{\strut  close} \\
\midrule
Jacobian & \num{2.451e-01} & \colorbox{Cyan!0.000}{\strut  one} \colorbox{Cyan!0.000}{\strut  of} \colorbox{Cyan!0.000}{\strut  these} \colorbox{Cyan!0.000}{\strut  books} \colorbox{Cyan!0.000}{\strut  and} \colorbox{Cyan!0.000}{\strut  finished} \colorbox{Cyan!0.000}{\strut  reading} \colorbox{Cyan!0.000}{\strut  it} \colorbox{Cyan!99.354}{\strut  just} \colorbox{Cyan!0.000}{\strut  this} \colorbox{Cyan!0.000}{\strut  morning} \colorbox{Cyan!0.000}{\strut ,} \colorbox{Cyan!0.000}{\strut  and} \colorbox{Cyan!0.000}{\strut  one} \colorbox{Cyan!0.000}{\strut  of} \\
Input SAE & \num{1.810e+01} & \colorbox{Green!0.000}{\strut  one} \colorbox{Green!0.000}{\strut  of} \colorbox{Green!0.000}{\strut  these} \colorbox{Green!0.000}{\strut  books} \colorbox{Green!0.000}{\strut  and} \colorbox{Green!0.000}{\strut  finished} \colorbox{Green!0.000}{\strut  reading} \colorbox{Green!0.000}{\strut  it} \colorbox{Green!86.347}{\strut  just} \colorbox{Green!0.000}{\strut  this} \colorbox{Green!0.000}{\strut  morning} \colorbox{Green!0.000}{\strut ,} \colorbox{Green!0.000}{\strut  and} \colorbox{Green!0.000}{\strut  one} \colorbox{Green!0.000}{\strut  of} \\
Output SAE & \num{4.388e+00} & \colorbox{Magenta!0.000}{\strut  one} \colorbox{Magenta!0.000}{\strut  of} \colorbox{Magenta!0.000}{\strut  these} \colorbox{Magenta!0.000}{\strut  books} \colorbox{Magenta!0.000}{\strut  and} \colorbox{Magenta!0.000}{\strut  finished} \colorbox{Magenta!0.000}{\strut  reading} \colorbox{Magenta!0.000}{\strut  it} \colorbox{Magenta!86.948}{\strut  just} \colorbox{Magenta!12.492}{\strut  this} \colorbox{Magenta!0.000}{\strut  morning} \colorbox{Magenta!0.000}{\strut ,} \colorbox{Magenta!0.000}{\strut  and} \colorbox{Magenta!0.000}{\strut  one} \colorbox{Magenta!0.000}{\strut  of} \\
\midrule
Jacobian & \num{2.445e-01} & \colorbox{Cyan!0.000}{\strut  lovers} \colorbox{Cyan!0.000}{\strut ,} \colorbox{Cyan!0.000}{\strut  and} \colorbox{Cyan!0.000}{\strut  those} \colorbox{Cyan!0.000}{\strut  that} \colorbox{Cyan!0.000}{\strut  travel} \colorbox{Cyan!0.000}{\strut  there} \colorbox{Cyan!99.121}{\strut  just} \colorbox{Cyan!0.000}{\strut  to} \colorbox{Cyan!0.000}{\strut  search} \colorbox{Cyan!0.000}{\strut  out} \colorbox{Cyan!0.000}{\strut  new} \colorbox{Cyan!0.000}{\strut  romantic} \colorbox{Cyan!0.000}{\strut  relationships} \colorbox{Cyan!0.000}{\strut .} \\
Input SAE & \num{1.687e+01} & \colorbox{Green!0.000}{\strut  lovers} \colorbox{Green!0.000}{\strut ,} \colorbox{Green!0.000}{\strut  and} \colorbox{Green!0.000}{\strut  those} \colorbox{Green!0.000}{\strut  that} \colorbox{Green!0.000}{\strut  travel} \colorbox{Green!0.000}{\strut  there} \colorbox{Green!80.484}{\strut  just} \colorbox{Green!0.000}{\strut  to} \colorbox{Green!0.000}{\strut  search} \colorbox{Green!0.000}{\strut  out} \colorbox{Green!0.000}{\strut  new} \colorbox{Green!0.000}{\strut  romantic} \colorbox{Green!0.000}{\strut  relationships} \colorbox{Green!0.000}{\strut .} \\
Output SAE & \num{4.316e+00} & \colorbox{Magenta!0.000}{\strut  lovers} \colorbox{Magenta!0.000}{\strut ,} \colorbox{Magenta!0.000}{\strut  and} \colorbox{Magenta!0.000}{\strut  those} \colorbox{Magenta!0.000}{\strut  that} \colorbox{Magenta!0.000}{\strut  travel} \colorbox{Magenta!0.000}{\strut  there} \colorbox{Magenta!85.520}{\strut  just} \colorbox{Magenta!17.694}{\strut  to} \colorbox{Magenta!0.000}{\strut  search} \colorbox{Magenta!0.000}{\strut  out} \colorbox{Magenta!0.000}{\strut  new} \colorbox{Magenta!0.000}{\strut  romantic} \colorbox{Magenta!0.000}{\strut  relationships} \colorbox{Magenta!0.000}{\strut .} \\
\midrule
Jacobian & \num{2.443e-01} & \colorbox{Cyan!0.000}{\strut -} \colorbox{Cyan!0.000}{\strut cr} \colorbox{Cyan!0.000}{\strut ushing} \colorbox{Cyan!0.000}{\strut  hit} \colorbox{Cyan!0.000}{\strut ,} \colorbox{Cyan!0.000}{\strut  La} \colorbox{Cyan!0.000}{\strut  Gr} \colorbox{Cyan!0.000}{\strut ange} \colorbox{Cyan!0.000}{\strut ,} \colorbox{Cyan!0.000}{\strut  that} \colorbox{Cyan!0.000}{\strut  the} \colorbox{Cyan!0.000}{\strut  music} \colorbox{Cyan!0.000}{\strut  world} \colorbox{Cyan!0.000}{\strut  saw} \colorbox{Cyan!99.010}{\strut  just} \\
Input SAE & \num{2.073e+01} & \colorbox{Green!0.000}{\strut -} \colorbox{Green!0.000}{\strut cr} \colorbox{Green!0.000}{\strut ushing} \colorbox{Green!0.000}{\strut  hit} \colorbox{Green!0.000}{\strut ,} \colorbox{Green!0.000}{\strut  La} \colorbox{Green!0.000}{\strut  Gr} \colorbox{Green!0.000}{\strut ange} \colorbox{Green!0.000}{\strut ,} \colorbox{Green!0.000}{\strut  that} \colorbox{Green!0.000}{\strut  the} \colorbox{Green!0.000}{\strut  music} \colorbox{Green!0.000}{\strut  world} \colorbox{Green!0.000}{\strut  saw} \colorbox{Green!98.928}{\strut  just} \\
Output SAE & \num{4.773e+00} & \colorbox{Magenta!0.000}{\strut -} \colorbox{Magenta!0.000}{\strut cr} \colorbox{Magenta!0.000}{\strut ushing} \colorbox{Magenta!0.000}{\strut  hit} \colorbox{Magenta!0.000}{\strut ,} \colorbox{Magenta!0.000}{\strut  La} \colorbox{Magenta!0.000}{\strut  Gr} \colorbox{Magenta!0.000}{\strut ange} \colorbox{Magenta!0.000}{\strut ,} \colorbox{Magenta!0.000}{\strut  that} \colorbox{Magenta!0.000}{\strut  the} \colorbox{Magenta!0.000}{\strut  music} \colorbox{Magenta!0.000}{\strut  world} \colorbox{Magenta!0.000}{\strut  saw} \colorbox{Magenta!94.574}{\strut  just} \\
\midrule
Jacobian & \num{2.442e-01} & \colorbox{Cyan!0.000}{\strut  extremely} \colorbox{Cyan!0.000}{\strut  valuable} \colorbox{Cyan!0.000}{\strut  and} \colorbox{Cyan!99.002}{\strut  just} \colorbox{Cyan!68.272}{\strut  a} \colorbox{Cyan!0.000}{\strut  single} \colorbox{Cyan!0.000}{\strut  piece} \colorbox{Cyan!0.000}{\strut  of} \colorbox{Cyan!0.000}{\strut  a} \colorbox{Cyan!0.000}{\strut  much} \colorbox{Cyan!0.000}{\strut  larger} \colorbox{Cyan!0.000}{\strut  business} \colorbox{Cyan!0.000}{\strut  but} \colorbox{Cyan!0.000}{\strut  also} \\
Input SAE & \num{1.737e+01} & \colorbox{Green!0.000}{\strut  extremely} \colorbox{Green!0.000}{\strut  valuable} \colorbox{Green!0.000}{\strut  and} \colorbox{Green!82.863}{\strut  just} \colorbox{Green!12.452}{\strut  a} \colorbox{Green!0.000}{\strut  single} \colorbox{Green!0.000}{\strut  piece} \colorbox{Green!0.000}{\strut  of} \colorbox{Green!0.000}{\strut  a} \colorbox{Green!0.000}{\strut  much} \colorbox{Green!0.000}{\strut  larger} \colorbox{Green!0.000}{\strut  business} \colorbox{Green!0.000}{\strut  but} \colorbox{Green!0.000}{\strut  also} \\
Output SAE & \num{4.595e+00} & \colorbox{Magenta!0.000}{\strut  extremely} \colorbox{Magenta!0.000}{\strut  valuable} \colorbox{Magenta!0.000}{\strut  and} \colorbox{Magenta!91.061}{\strut  just} \colorbox{Magenta!26.628}{\strut  a} \colorbox{Magenta!11.262}{\strut  single} \colorbox{Magenta!0.000}{\strut  piece} \colorbox{Magenta!0.000}{\strut  of} \colorbox{Magenta!0.000}{\strut  a} \colorbox{Magenta!0.000}{\strut  much} \colorbox{Magenta!0.000}{\strut  larger} \colorbox{Magenta!0.000}{\strut  business} \colorbox{Magenta!0.000}{\strut  but} \colorbox{Magenta!0.000}{\strut  also} \\
\midrule
Jacobian & \num{2.433e-01} & \colorbox{Cyan!0.000}{\strut  get} \colorbox{Cyan!0.000}{\strut  out} \colorbox{Cyan!0.000}{\strut  of} \colorbox{Cyan!0.000}{\strut  it} \colorbox{Cyan!98.612}{\strut  just} \colorbox{Cyan!0.000}{\strut  means} \colorbox{Cyan!0.000}{\strut  more} \colorbox{Cyan!0.000}{\strut  drama} \colorbox{Cyan!0.000}{\strut .} \colorbox{Cyan!0.000}{\strut  For} \colorbox{Cyan!0.000}{\strut  example} \colorbox{Cyan!0.000}{\strut ,} \colorbox{Cyan!0.000}{\strut  if} \colorbox{Cyan!0.000}{\strut  your} \colorbox{Cyan!0.000}{\strut  friend} \\
Input SAE & \num{1.684e+01} & \colorbox{Green!0.000}{\strut  get} \colorbox{Green!0.000}{\strut  out} \colorbox{Green!0.000}{\strut  of} \colorbox{Green!0.000}{\strut  it} \colorbox{Green!80.328}{\strut  just} \colorbox{Green!0.000}{\strut  means} \colorbox{Green!0.000}{\strut  more} \colorbox{Green!0.000}{\strut  drama} \colorbox{Green!0.000}{\strut .} \colorbox{Green!0.000}{\strut  For} \colorbox{Green!0.000}{\strut  example} \colorbox{Green!0.000}{\strut ,} \colorbox{Green!0.000}{\strut  if} \colorbox{Green!0.000}{\strut  your} \colorbox{Green!0.000}{\strut  friend} \\
Output SAE & \num{4.324e+00} & \colorbox{Magenta!0.000}{\strut  get} \colorbox{Magenta!0.000}{\strut  out} \colorbox{Magenta!0.000}{\strut  of} \colorbox{Magenta!0.000}{\strut  it} \colorbox{Magenta!85.681}{\strut  just} \colorbox{Magenta!12.062}{\strut  means} \colorbox{Magenta!0.000}{\strut  more} \colorbox{Magenta!0.000}{\strut  drama} \colorbox{Magenta!0.000}{\strut .} \colorbox{Magenta!0.000}{\strut  For} \colorbox{Magenta!0.000}{\strut  example} \colorbox{Magenta!0.000}{\strut ,} \colorbox{Magenta!0.000}{\strut  if} \colorbox{Magenta!0.000}{\strut  your} \colorbox{Magenta!0.000}{\strut  friend} \\
\midrule
Jacobian & \num{2.432e-01} & \colorbox{Cyan!0.000}{\strut  coral} \colorbox{Cyan!0.000}{\strut  gen} \colorbox{Cyan!0.000}{\strut omics} \colorbox{Cyan!0.000}{\strut  at} \colorbox{Cyan!0.000}{\strut  the} \colorbox{Cyan!0.000}{\strut  Red} \colorbox{Cyan!0.000}{\strut  Sea} \colorbox{Cyan!0.000}{\strut  Research} \colorbox{Cyan!0.000}{\strut  Center} \colorbox{Cyan!0.000}{\strut .} \colorbox{Cyan!0.000}{\strut I} \colorbox{Cyan!0.000}{\strut  arrived} \colorbox{Cyan!98.584}{\strut  just} \colorbox{Cyan!72.047}{\strut  in} \colorbox{Cyan!0.000}{\strut  time} \\
Input SAE & \num{1.707e+01} & \colorbox{Green!0.000}{\strut  coral} \colorbox{Green!0.000}{\strut  gen} \colorbox{Green!0.000}{\strut omics} \colorbox{Green!0.000}{\strut  at} \colorbox{Green!0.000}{\strut  the} \colorbox{Green!0.000}{\strut  Red} \colorbox{Green!0.000}{\strut  Sea} \colorbox{Green!0.000}{\strut  Research} \colorbox{Green!0.000}{\strut  Center} \colorbox{Green!0.000}{\strut .} \colorbox{Green!0.000}{\strut I} \colorbox{Green!0.000}{\strut  arrived} \colorbox{Green!81.458}{\strut  just} \colorbox{Green!10.152}{\strut  in} \colorbox{Green!0.000}{\strut  time} \\
Output SAE & \num{4.206e+00} & \colorbox{Magenta!0.000}{\strut  coral} \colorbox{Magenta!0.000}{\strut  gen} \colorbox{Magenta!0.000}{\strut omics} \colorbox{Magenta!0.000}{\strut  at} \colorbox{Magenta!0.000}{\strut  the} \colorbox{Magenta!0.000}{\strut  Red} \colorbox{Magenta!0.000}{\strut  Sea} \colorbox{Magenta!0.000}{\strut  Research} \colorbox{Magenta!0.000}{\strut  Center} \colorbox{Magenta!0.000}{\strut .} \colorbox{Magenta!0.000}{\strut I} \colorbox{Magenta!0.000}{\strut  arrived} \colorbox{Magenta!83.352}{\strut  just} \colorbox{Magenta!30.580}{\strut  in} \colorbox{Magenta!0.000}{\strut  time} \\
\bottomrule
\end{longtable}
\caption{feature pairs/Layer15-65536-J1-LR5.0e-04-k32-T3.0e+08 abs mean/examples-34296-v-43107 stas c4-en-10k,train,batch size=32,ctx len=16.csv}
\end{table} % just...
% \begin{table}
\centering
\begin{longtable}{lrl}
\toprule
Category & Max. abs. value & Example tokens \\
\midrule
Jacobian & \num{2.078e-01} & \colorbox{Cyan!0.000}{\strut  Tips} \colorbox{Cyan!0.000}{\strut  And} \colorbox{Cyan!0.000}{\strut  T} \colorbox{Cyan!0.000}{\strut ricks} \colorbox{Cyan!0.000}{\strut  With} \colorbox{Cyan!0.000}{\strut  .} \colorbox{Cyan!0.000}{\strut 5} \colorbox{Cyan!0.000}{\strut  Res} \colorbox{Cyan!0.000}{\strut ume} \colorbox{Cyan!0.000}{\strut  Writing} \colorbox{Cyan!0.000}{\strut  Tips} \colorbox{Cyan!0.000}{\strut  5} \colorbox{Cyan!0.000}{\strut  Res} \colorbox{Cyan!0.000}{\strut ume} \\
Input SAE & \num{9.586e-01} & \colorbox{Green!0.000}{\strut  Tips} \colorbox{Green!0.000}{\strut  And} \colorbox{Green!0.000}{\strut  T} \colorbox{Green!0.000}{\strut ricks} \colorbox{Green!0.000}{\strut  With} \colorbox{Green!0.000}{\strut  .} \colorbox{Green!25.398}{\strut 5} \colorbox{Green!24.275}{\strut  Res} \colorbox{Green!0.000}{\strut ume} \colorbox{Green!0.000}{\strut  Writing} \colorbox{Green!29.713}{\strut  Tips} \colorbox{Green!0.000}{\strut  5} \colorbox{Green!0.000}{\strut  Res} \colorbox{Green!0.000}{\strut ume} \\
Output SAE & \num{1.543e+00} & \colorbox{Magenta!0.000}{\strut  Tips} \colorbox{Magenta!0.000}{\strut  And} \colorbox{Magenta!0.000}{\strut  T} \colorbox{Magenta!0.000}{\strut ricks} \colorbox{Magenta!0.000}{\strut  With} \colorbox{Magenta!0.000}{\strut  .} \colorbox{Magenta!0.000}{\strut 5} \colorbox{Magenta!0.000}{\strut  Res} \colorbox{Magenta!0.000}{\strut ume} \colorbox{Magenta!0.000}{\strut  Writing} \colorbox{Magenta!0.000}{\strut  Tips} \colorbox{Magenta!0.000}{\strut  5} \colorbox{Magenta!0.000}{\strut  Res} \colorbox{Magenta!0.000}{\strut ume} \\
\midrule
Jacobian & \num{2.078e-01} & \colorbox{Cyan!0.000}{\strut  TA} \colorbox{Cyan!0.000}{\strut KE} \colorbox{Cyan!0.000}{\strut  HIM} \colorbox{Cyan!0.000}{\strut  BACK} \colorbox{Cyan!0.000}{\strut !} \colorbox{Cyan!0.000}{\strut I} \colorbox{Cyan!0.000}{\strut  get} \colorbox{Cyan!0.000}{\strut  the} \colorbox{Cyan!0.000}{\strut  foot} \colorbox{Cyan!0.000}{\strut  cr} \colorbox{Cyan!0.000}{\strut amps} \colorbox{Cyan!0.000}{\strut  like} \colorbox{Cyan!0.000}{\strut  crazy} \colorbox{Cyan!0.000}{\strut .} \\
Input SAE & \num{7.619e-01} & \colorbox{Green!0.000}{\strut  TA} \colorbox{Green!0.000}{\strut KE} \colorbox{Green!0.000}{\strut  HIM} \colorbox{Green!0.000}{\strut  BACK} \colorbox{Green!0.000}{\strut !} \colorbox{Green!18.404}{\strut I} \colorbox{Green!23.616}{\strut  get} \colorbox{Green!0.000}{\strut  the} \colorbox{Green!21.220}{\strut  foot} \colorbox{Green!0.000}{\strut  cr} \colorbox{Green!0.000}{\strut amps} \colorbox{Green!0.000}{\strut  like} \colorbox{Green!0.000}{\strut  crazy} \colorbox{Green!0.000}{\strut .} \\
Output SAE & \num{1.542e+00} & \colorbox{Magenta!0.000}{\strut  TA} \colorbox{Magenta!0.000}{\strut KE} \colorbox{Magenta!0.000}{\strut  HIM} \colorbox{Magenta!0.000}{\strut  BACK} \colorbox{Magenta!0.000}{\strut !} \colorbox{Magenta!0.000}{\strut I} \colorbox{Magenta!0.000}{\strut  get} \colorbox{Magenta!0.000}{\strut  the} \colorbox{Magenta!0.000}{\strut  foot} \colorbox{Magenta!0.000}{\strut  cr} \colorbox{Magenta!0.000}{\strut amps} \colorbox{Magenta!0.000}{\strut  like} \colorbox{Magenta!0.000}{\strut  crazy} \colorbox{Magenta!0.000}{\strut .} \\
\midrule
Jacobian & \num{2.078e-01} & \colorbox{Cyan!0.000}{\strut  BRE} \colorbox{Cyan!0.000}{\strut ATH} \colorbox{Cyan!0.000}{\strut !} \colorbox{Cyan!0.000}{\strut Such} \colorbox{Cyan!0.000}{\strut  a} \colorbox{Cyan!0.000}{\strut  simple} \colorbox{Cyan!0.000}{\strut  solution} \colorbox{Cyan!0.000}{\strut  to} \colorbox{Cyan!0.000}{\strut  help} \colorbox{Cyan!0.000}{\strut  ease} \colorbox{Cyan!0.000}{\strut  the} \colorbox{Cyan!0.000}{\strut  journey} \colorbox{Cyan!0.000}{\strut .} \colorbox{Cyan!0.000}{\strut  Safe} \\
Input SAE & \num{1.095e+00} & \colorbox{Green!0.000}{\strut  BRE} \colorbox{Green!0.000}{\strut ATH} \colorbox{Green!0.000}{\strut !} \colorbox{Green!0.000}{\strut Such} \colorbox{Green!0.000}{\strut  a} \colorbox{Green!0.000}{\strut  simple} \colorbox{Green!0.000}{\strut  solution} \colorbox{Green!33.955}{\strut  to} \colorbox{Green!0.000}{\strut  help} \colorbox{Green!0.000}{\strut  ease} \colorbox{Green!0.000}{\strut  the} \colorbox{Green!0.000}{\strut  journey} \colorbox{Green!0.000}{\strut .} \colorbox{Green!0.000}{\strut  Safe} \\
Output SAE & \num{1.539e+00} & \colorbox{Magenta!0.000}{\strut  BRE} \colorbox{Magenta!0.000}{\strut ATH} \colorbox{Magenta!0.000}{\strut !} \colorbox{Magenta!0.000}{\strut Such} \colorbox{Magenta!0.000}{\strut  a} \colorbox{Magenta!0.000}{\strut  simple} \colorbox{Magenta!0.000}{\strut  solution} \colorbox{Magenta!0.000}{\strut  to} \colorbox{Magenta!0.000}{\strut  help} \colorbox{Magenta!0.000}{\strut  ease} \colorbox{Magenta!0.000}{\strut  the} \colorbox{Magenta!0.000}{\strut  journey} \colorbox{Magenta!0.000}{\strut .} \colorbox{Magenta!0.000}{\strut  Safe} \\
\midrule
Jacobian & \num{2.078e-01} & \colorbox{Cyan!0.000}{\strut  Tips} \colorbox{Cyan!0.000}{\strut  Choose} \colorbox{Cyan!0.000}{\strut  .} \colorbox{Cyan!0.000}{\strut Best} \colorbox{Cyan!0.000}{\strut  Res} \colorbox{Cyan!0.000}{\strut ume} \colorbox{Cyan!0.000}{\strut  Format} \colorbox{Cyan!0.000}{\strut  Forbes} \colorbox{Cyan!0.000}{\strut  Tips} \colorbox{Cyan!0.000}{\strut  On} \colorbox{Cyan!0.000}{\strut  Res} \colorbox{Cyan!0.000}{\strut ume} \colorbox{Cyan!0.000}{\strut  Writing} \colorbox{Cyan!0.000}{\strut  Res} \\
Input SAE & \num{2.161e-01} & \colorbox{Green!0.000}{\strut  Tips} \colorbox{Green!0.000}{\strut  Choose} \colorbox{Green!0.000}{\strut  .} \colorbox{Green!0.000}{\strut Best} \colorbox{Green!0.000}{\strut  Res} \colorbox{Green!0.000}{\strut ume} \colorbox{Green!0.000}{\strut  Format} \colorbox{Green!0.000}{\strut  Forbes} \colorbox{Green!0.000}{\strut  Tips} \colorbox{Green!0.000}{\strut  On} \colorbox{Green!0.000}{\strut  Res} \colorbox{Green!0.000}{\strut ume} \colorbox{Green!0.000}{\strut  Writing} \colorbox{Green!0.000}{\strut  Res} \\
Output SAE & \num{1.540e+00} & \colorbox{Magenta!0.000}{\strut  Tips} \colorbox{Magenta!0.000}{\strut  Choose} \colorbox{Magenta!0.000}{\strut  .} \colorbox{Magenta!0.000}{\strut Best} \colorbox{Magenta!0.000}{\strut  Res} \colorbox{Magenta!0.000}{\strut ume} \colorbox{Magenta!0.000}{\strut  Format} \colorbox{Magenta!0.000}{\strut  Forbes} \colorbox{Magenta!0.000}{\strut  Tips} \colorbox{Magenta!0.000}{\strut  On} \colorbox{Magenta!0.000}{\strut  Res} \colorbox{Magenta!0.000}{\strut ume} \colorbox{Magenta!0.000}{\strut  Writing} \colorbox{Magenta!0.000}{\strut  Res} \\
\midrule
Jacobian & \num{2.078e-01} & \colorbox{Cyan!0.000}{\strut  Tips} \colorbox{Cyan!0.000}{\strut  For} \colorbox{Cyan!0.000}{\strut  .} \colorbox{Cyan!0.000}{\strut Res} \colorbox{Cyan!0.000}{\strut ume} \colorbox{Cyan!0.000}{\strut  Template} \colorbox{Cyan!0.000}{\strut  Res} \colorbox{Cyan!0.000}{\strut ume} \colorbox{Cyan!0.000}{\strut  Format} \colorbox{Cyan!0.000}{\strut  Tips} \colorbox{Cyan!0.000}{\strut  Sample} \colorbox{Cyan!0.000}{\strut  Res} \colorbox{Cyan!0.000}{\strut ume} \colorbox{Cyan!0.000}{\strut  Template} \\
Input SAE & \num{8.074e-01} & \colorbox{Green!0.000}{\strut  Tips} \colorbox{Green!0.000}{\strut  For} \colorbox{Green!0.000}{\strut  .} \colorbox{Green!25.026}{\strut Res} \colorbox{Green!0.000}{\strut ume} \colorbox{Green!0.000}{\strut  Template} \colorbox{Green!0.000}{\strut  Res} \colorbox{Green!0.000}{\strut ume} \colorbox{Green!0.000}{\strut  Format} \colorbox{Green!0.000}{\strut  Tips} \colorbox{Green!0.000}{\strut  Sample} \colorbox{Green!0.000}{\strut  Res} \colorbox{Green!0.000}{\strut ume} \colorbox{Green!0.000}{\strut  Template} \\
Output SAE & \num{1.544e+00} & \colorbox{Magenta!0.000}{\strut  Tips} \colorbox{Magenta!0.000}{\strut  For} \colorbox{Magenta!0.000}{\strut  .} \colorbox{Magenta!0.000}{\strut Res} \colorbox{Magenta!0.000}{\strut ume} \colorbox{Magenta!0.000}{\strut  Template} \colorbox{Magenta!0.000}{\strut  Res} \colorbox{Magenta!0.000}{\strut ume} \colorbox{Magenta!0.000}{\strut  Format} \colorbox{Magenta!0.000}{\strut  Tips} \colorbox{Magenta!0.000}{\strut  Sample} \colorbox{Magenta!0.000}{\strut  Res} \colorbox{Magenta!0.000}{\strut ume} \colorbox{Magenta!0.000}{\strut  Template} \\
\midrule
Jacobian & \num{2.078e-01} & \colorbox{Cyan!0.000}{\strut ,} \colorbox{Cyan!0.000}{\strut  the} \colorbox{Cyan!0.000}{\strut  .} \colorbox{Cyan!0.000}{\strut Watch} \colorbox{Cyan!0.000}{\strut  Online} \colorbox{Cyan!0.000}{\strut  Hans} \colorbox{Cyan!0.000}{\strut el} \colorbox{Cyan!0.000}{\strut  \&} \colorbox{Cyan!0.000}{\strut  G} \colorbox{Cyan!0.000}{\strut ret} \colorbox{Cyan!0.000}{\strut el} \colorbox{Cyan!0.000}{\strut  Movie} \colorbox{Cyan!0.000}{\strut  for} \colorbox{Cyan!0.000}{\strut  Free} \\
Input SAE & \num{1.238e+00} & \colorbox{Green!0.000}{\strut ,} \colorbox{Green!23.125}{\strut  the} \colorbox{Green!22.051}{\strut  .} \colorbox{Green!29.862}{\strut Watch} \colorbox{Green!0.000}{\strut  Online} \colorbox{Green!0.000}{\strut  Hans} \colorbox{Green!0.000}{\strut el} \colorbox{Green!0.000}{\strut  \&} \colorbox{Green!0.000}{\strut  G} \colorbox{Green!0.000}{\strut ret} \colorbox{Green!0.000}{\strut el} \colorbox{Green!29.274}{\strut  Movie} \colorbox{Green!38.385}{\strut  for} \colorbox{Green!0.000}{\strut  Free} \\
Output SAE & \num{1.539e+00} & \colorbox{Magenta!0.000}{\strut ,} \colorbox{Magenta!0.000}{\strut  the} \colorbox{Magenta!0.000}{\strut  .} \colorbox{Magenta!0.000}{\strut Watch} \colorbox{Magenta!0.000}{\strut  Online} \colorbox{Magenta!0.000}{\strut  Hans} \colorbox{Magenta!0.000}{\strut el} \colorbox{Magenta!0.000}{\strut  \&} \colorbox{Magenta!0.000}{\strut  G} \colorbox{Magenta!0.000}{\strut ret} \colorbox{Magenta!0.000}{\strut el} \colorbox{Magenta!0.000}{\strut  Movie} \colorbox{Magenta!0.000}{\strut  for} \colorbox{Magenta!0.000}{\strut  Free} \\
\midrule
Jacobian & \num{2.078e-01} & \colorbox{Cyan!0.000}{\strut  well} \colorbox{Cyan!0.000}{\strut  .} \colorbox{Cyan!0.000}{\strut It} \colorbox{Cyan!0.000}{\strut  becomes} \colorbox{Cyan!0.000}{\strut  easy} \colorbox{Cyan!0.000}{\strut  for} \colorbox{Cyan!0.000}{\strut  them} \colorbox{Cyan!0.000}{\strut  to} \colorbox{Cyan!0.000}{\strut  understand} \colorbox{Cyan!0.000}{\strut  what} \colorbox{Cyan!0.000}{\strut  I} \colorbox{Cyan!0.000}{\strut  teach} \colorbox{Cyan!0.000}{\strut .} \\
Input SAE & \num{1.623e+00} & \colorbox{Green!29.154}{\strut  well} \colorbox{Green!0.000}{\strut  .} \colorbox{Green!0.000}{\strut It} \colorbox{Green!50.303}{\strut  becomes} \colorbox{Green!0.000}{\strut  easy} \colorbox{Green!35.001}{\strut  for} \colorbox{Green!0.000}{\strut  them} \colorbox{Green!0.000}{\strut  to} \colorbox{Green!0.000}{\strut  understand} \colorbox{Green!0.000}{\strut  what} \colorbox{Green!0.000}{\strut  I} \colorbox{Green!0.000}{\strut  teach} \colorbox{Green!0.000}{\strut .} \\
Output SAE & \num{1.538e+00} & \colorbox{Magenta!0.000}{\strut  well} \colorbox{Magenta!0.000}{\strut  .} \colorbox{Magenta!0.000}{\strut It} \colorbox{Magenta!0.000}{\strut  becomes} \colorbox{Magenta!0.000}{\strut  easy} \colorbox{Magenta!0.000}{\strut  for} \colorbox{Magenta!0.000}{\strut  them} \colorbox{Magenta!0.000}{\strut  to} \colorbox{Magenta!0.000}{\strut  understand} \colorbox{Magenta!0.000}{\strut  what} \colorbox{Magenta!0.000}{\strut  I} \colorbox{Magenta!0.000}{\strut  teach} \colorbox{Magenta!0.000}{\strut .} \\
\midrule
Jacobian & \num{2.078e-01} & \colorbox{Cyan!0.000}{\strut  \&} \colorbox{Cyan!0.000}{\strut  .} \colorbox{Cyan!0.000}{\strut Sand} \colorbox{Cyan!0.000}{\strut  and} \colorbox{Cyan!0.000}{\strut  Gra} \colorbox{Cyan!0.000}{\strut vel} \colorbox{Cyan!0.000}{\strut  Direct} \colorbox{Cyan!0.000}{\strut  supply} \colorbox{Cyan!0.000}{\strut  Cr} \colorbox{Cyan!0.000}{\strut ushed} \colorbox{Cyan!0.000}{\strut  As} \colorbox{Cyan!0.000}{\strut phalt} \colorbox{Cyan!0.000}{\strut  available} \colorbox{Cyan!0.000}{\strut  for} \\
Input SAE & \num{1.672e+00} & \colorbox{Green!51.833}{\strut  \&} \colorbox{Green!29.182}{\strut  .} \colorbox{Green!24.003}{\strut Sand} \colorbox{Green!0.000}{\strut  and} \colorbox{Green!19.562}{\strut  Gra} \colorbox{Green!0.000}{\strut vel} \colorbox{Green!0.000}{\strut  Direct} \colorbox{Green!0.000}{\strut  supply} \colorbox{Green!0.000}{\strut  Cr} \colorbox{Green!0.000}{\strut ushed} \colorbox{Green!0.000}{\strut  As} \colorbox{Green!0.000}{\strut phalt} \colorbox{Green!0.000}{\strut  available} \colorbox{Green!44.459}{\strut  for} \\
Output SAE & \num{1.540e+00} & \colorbox{Magenta!0.000}{\strut  \&} \colorbox{Magenta!0.000}{\strut  .} \colorbox{Magenta!0.000}{\strut Sand} \colorbox{Magenta!0.000}{\strut  and} \colorbox{Magenta!0.000}{\strut  Gra} \colorbox{Magenta!0.000}{\strut vel} \colorbox{Magenta!0.000}{\strut  Direct} \colorbox{Magenta!0.000}{\strut  supply} \colorbox{Magenta!0.000}{\strut  Cr} \colorbox{Magenta!0.000}{\strut ushed} \colorbox{Magenta!0.000}{\strut  As} \colorbox{Magenta!0.000}{\strut phalt} \colorbox{Magenta!0.000}{\strut  available} \colorbox{Magenta!0.000}{\strut  for} \\
\midrule
Jacobian & \num{2.077e-01} & \colorbox{Cyan!0.000}{\strut  So} \colorbox{Cyan!0.000}{\strut  let} \colorbox{Cyan!0.000}{\strut \textquotesingle{}} \colorbox{Cyan!0.000}{\strut s} \colorbox{Cyan!0.000}{\strut  get} \colorbox{Cyan!0.000}{\strut  right} \colorbox{Cyan!0.000}{\strut  to} \colorbox{Cyan!0.000}{\strut  it} \colorbox{Cyan!0.000}{\strut  .} \colorbox{Cyan!0.000}{\strut  .} \colorbox{Cyan!0.000}{\strut  .} \colorbox{Cyan!0.000}{\strut Build} \colorbox{Cyan!0.000}{\strut ium} \colorbox{Cyan!0.000}{\strut  recently} \\
Input SAE & \num{1.417e+00} & \colorbox{Green!0.000}{\strut  So} \colorbox{Green!41.371}{\strut  let} \colorbox{Green!35.953}{\strut \textquotesingle{}} \colorbox{Green!0.000}{\strut s} \colorbox{Green!35.468}{\strut  get} \colorbox{Green!0.000}{\strut  right} \colorbox{Green!43.926}{\strut  to} \colorbox{Green!0.000}{\strut  it} \colorbox{Green!0.000}{\strut  .} \colorbox{Green!33.815}{\strut  .} \colorbox{Green!0.000}{\strut  .} \colorbox{Green!0.000}{\strut Build} \colorbox{Green!0.000}{\strut ium} \colorbox{Green!28.994}{\strut  recently} \\
Output SAE & \num{1.539e+00} & \colorbox{Magenta!0.000}{\strut  So} \colorbox{Magenta!0.000}{\strut  let} \colorbox{Magenta!0.000}{\strut \textquotesingle{}} \colorbox{Magenta!0.000}{\strut s} \colorbox{Magenta!0.000}{\strut  get} \colorbox{Magenta!0.000}{\strut  right} \colorbox{Magenta!0.000}{\strut  to} \colorbox{Magenta!0.000}{\strut  it} \colorbox{Magenta!0.000}{\strut  .} \colorbox{Magenta!0.000}{\strut  .} \colorbox{Magenta!0.000}{\strut  .} \colorbox{Magenta!0.000}{\strut Build} \colorbox{Magenta!0.000}{\strut ium} \colorbox{Magenta!0.000}{\strut  recently} \\
\midrule
Jacobian & \num{2.077e-01} & \colorbox{Cyan!0.000}{\strut  is} \colorbox{Cyan!0.000}{\strut  OK} \colorbox{Cyan!0.000}{\strut  .} \colorbox{Cyan!0.000}{\strut S} \colorbox{Cyan!0.000}{\strut ony} \colorbox{Cyan!0.000}{\strut  makes} \colorbox{Cyan!0.000}{\strut  a} \colorbox{Cyan!0.000}{\strut  really} \colorbox{Cyan!0.000}{\strut  great} \colorbox{Cyan!0.000}{\strut  product} \colorbox{Cyan!0.000}{\strut  ,} \colorbox{Cyan!0.000}{\strut set} \colorbox{Cyan!0.000}{\strut  up} \colorbox{Cyan!0.000}{\strut  is} \\
Input SAE & \num{2.161e-01} & \colorbox{Green!0.000}{\strut  is} \colorbox{Green!0.000}{\strut  OK} \colorbox{Green!0.000}{\strut  .} \colorbox{Green!0.000}{\strut S} \colorbox{Green!0.000}{\strut ony} \colorbox{Green!0.000}{\strut  makes} \colorbox{Green!0.000}{\strut  a} \colorbox{Green!0.000}{\strut  really} \colorbox{Green!0.000}{\strut  great} \colorbox{Green!0.000}{\strut  product} \colorbox{Green!0.000}{\strut  ,} \colorbox{Green!0.000}{\strut set} \colorbox{Green!0.000}{\strut  up} \colorbox{Green!0.000}{\strut  is} \\
Output SAE & \num{1.540e+00} & \colorbox{Magenta!0.000}{\strut  is} \colorbox{Magenta!0.000}{\strut  OK} \colorbox{Magenta!0.000}{\strut  .} \colorbox{Magenta!0.000}{\strut S} \colorbox{Magenta!0.000}{\strut ony} \colorbox{Magenta!0.000}{\strut  makes} \colorbox{Magenta!0.000}{\strut  a} \colorbox{Magenta!0.000}{\strut  really} \colorbox{Magenta!0.000}{\strut  great} \colorbox{Magenta!0.000}{\strut  product} \colorbox{Magenta!0.000}{\strut  ,} \colorbox{Magenta!0.000}{\strut set} \colorbox{Magenta!0.000}{\strut  up} \colorbox{Magenta!0.000}{\strut  is} \\
\midrule
Jacobian & \num{2.077e-01} & \colorbox{Cyan!0.000}{\strut  YOUR} \colorbox{Cyan!0.000}{\strut  HE} \colorbox{Cyan!0.000}{\strut ALTH} \colorbox{Cyan!0.000}{\strut  AND} \colorbox{Cyan!0.000}{\strut  V} \colorbox{Cyan!0.000}{\strut ITAL} \colorbox{Cyan!0.000}{\strut ITY} \colorbox{Cyan!0.000}{\strut !} \colorbox{Cyan!0.000}{\strut \textquotedbl{}} \colorbox{Cyan!0.000}{\strut My} \colorbox{Cyan!0.000}{\strut  depression} \colorbox{Cyan!0.000}{\strut  is} \colorbox{Cyan!0.000}{\strut  gone} \colorbox{Cyan!0.000}{\strut ,} \\
Input SAE & \num{1.547e+00} & \colorbox{Green!0.000}{\strut  YOUR} \colorbox{Green!29.909}{\strut  HE} \colorbox{Green!0.000}{\strut ALTH} \colorbox{Green!0.000}{\strut  AND} \colorbox{Green!40.170}{\strut  V} \colorbox{Green!0.000}{\strut ITAL} \colorbox{Green!0.000}{\strut ITY} \colorbox{Green!30.965}{\strut !} \colorbox{Green!37.062}{\strut \textquotedbl{}} \colorbox{Green!0.000}{\strut My} \colorbox{Green!0.000}{\strut  depression} \colorbox{Green!33.325}{\strut  is} \colorbox{Green!0.000}{\strut  gone} \colorbox{Green!47.946}{\strut ,} \\
Output SAE & \num{1.541e+00} & \colorbox{Magenta!0.000}{\strut  YOUR} \colorbox{Magenta!0.000}{\strut  HE} \colorbox{Magenta!0.000}{\strut ALTH} \colorbox{Magenta!0.000}{\strut  AND} \colorbox{Magenta!0.000}{\strut  V} \colorbox{Magenta!0.000}{\strut ITAL} \colorbox{Magenta!0.000}{\strut ITY} \colorbox{Magenta!0.000}{\strut !} \colorbox{Magenta!0.000}{\strut \textquotedbl{}} \colorbox{Magenta!0.000}{\strut My} \colorbox{Magenta!0.000}{\strut  depression} \colorbox{Magenta!0.000}{\strut  is} \colorbox{Magenta!0.000}{\strut  gone} \colorbox{Magenta!0.000}{\strut ,} \\
\midrule
Jacobian & \num{2.077e-01} & \colorbox{Cyan!0.000}{\strut  your} \colorbox{Cyan!0.000}{\strut  organization} \colorbox{Cyan!0.000}{\strut .} \colorbox{Cyan!0.000}{\strut  Read} \colorbox{Cyan!0.000}{\strut  Now} \colorbox{Cyan!0.000}{\strut .} \colorbox{Cyan!0.000}{\strut What} \colorbox{Cyan!0.000}{\strut  will} \colorbox{Cyan!0.000}{\strut  differentiate} \colorbox{Cyan!0.000}{\strut  the} \colorbox{Cyan!0.000}{\strut  best} \colorbox{Cyan!0.000}{\strut  HR} \colorbox{Cyan!0.000}{\strut  leaders} \colorbox{Cyan!0.000}{\strut ?} \\
Input SAE & \num{2.161e-01} & \colorbox{Green!0.000}{\strut  your} \colorbox{Green!0.000}{\strut  organization} \colorbox{Green!0.000}{\strut .} \colorbox{Green!0.000}{\strut  Read} \colorbox{Green!0.000}{\strut  Now} \colorbox{Green!0.000}{\strut .} \colorbox{Green!0.000}{\strut What} \colorbox{Green!0.000}{\strut  will} \colorbox{Green!0.000}{\strut  differentiate} \colorbox{Green!0.000}{\strut  the} \colorbox{Green!0.000}{\strut  best} \colorbox{Green!0.000}{\strut  HR} \colorbox{Green!0.000}{\strut  leaders} \colorbox{Green!0.000}{\strut ?} \\
Output SAE & \num{1.545e+00} & \colorbox{Magenta!0.000}{\strut  your} \colorbox{Magenta!0.000}{\strut  organization} \colorbox{Magenta!0.000}{\strut .} \colorbox{Magenta!0.000}{\strut  Read} \colorbox{Magenta!0.000}{\strut  Now} \colorbox{Magenta!0.000}{\strut .} \colorbox{Magenta!0.000}{\strut What} \colorbox{Magenta!0.000}{\strut  will} \colorbox{Magenta!0.000}{\strut  differentiate} \colorbox{Magenta!0.000}{\strut  the} \colorbox{Magenta!0.000}{\strut  best} \colorbox{Magenta!0.000}{\strut  HR} \colorbox{Magenta!0.000}{\strut  leaders} \colorbox{Magenta!0.000}{\strut ?} \\
\bottomrule
\end{longtable}
\caption{feature pairs/Layer15-65536-J1-LR5.0e-04-k32-T3.0e+08 abs mean/examples-29982-v-53492 stas c4-en-10k,train,batch size=32,ctx len=16.csv}
\end{table}
% \begin{table}
\centering
\begin{longtable}{lrl}
\toprule
Category & Max. abs. value & Example tokens \\
\midrule
Jacobian & \num{2.059e-01} & \colorbox{Cyan!0.000}{\strut  BRE} \colorbox{Cyan!0.000}{\strut ATH} \colorbox{Cyan!0.000}{\strut !} \colorbox{Cyan!0.000}{\strut Such} \colorbox{Cyan!0.000}{\strut  a} \colorbox{Cyan!0.000}{\strut  simple} \colorbox{Cyan!0.000}{\strut  solution} \colorbox{Cyan!0.000}{\strut  to} \colorbox{Cyan!0.000}{\strut  help} \colorbox{Cyan!0.000}{\strut  ease} \colorbox{Cyan!0.000}{\strut  the} \colorbox{Cyan!0.000}{\strut  journey} \colorbox{Cyan!0.000}{\strut .} \colorbox{Cyan!0.000}{\strut  Safe} \\
Input SAE & \num{1.095e+00} & \colorbox{Green!0.000}{\strut  BRE} \colorbox{Green!0.000}{\strut ATH} \colorbox{Green!0.000}{\strut !} \colorbox{Green!0.000}{\strut Such} \colorbox{Green!0.000}{\strut  a} \colorbox{Green!0.000}{\strut  simple} \colorbox{Green!0.000}{\strut  solution} \colorbox{Green!33.955}{\strut  to} \colorbox{Green!0.000}{\strut  help} \colorbox{Green!0.000}{\strut  ease} \colorbox{Green!0.000}{\strut  the} \colorbox{Green!0.000}{\strut  journey} \colorbox{Green!0.000}{\strut .} \colorbox{Green!0.000}{\strut  Safe} \\
Output SAE & \num{1.504e+00} & \colorbox{Magenta!0.000}{\strut  BRE} \colorbox{Magenta!0.000}{\strut ATH} \colorbox{Magenta!0.000}{\strut !} \colorbox{Magenta!0.000}{\strut Such} \colorbox{Magenta!0.000}{\strut  a} \colorbox{Magenta!0.000}{\strut  simple} \colorbox{Magenta!0.000}{\strut  solution} \colorbox{Magenta!0.000}{\strut  to} \colorbox{Magenta!0.000}{\strut  help} \colorbox{Magenta!0.000}{\strut  ease} \colorbox{Magenta!0.000}{\strut  the} \colorbox{Magenta!0.000}{\strut  journey} \colorbox{Magenta!0.000}{\strut .} \colorbox{Magenta!0.000}{\strut  Safe} \\
\midrule
Jacobian & \num{2.059e-01} & \colorbox{Cyan!0.000}{\strut L} \colorbox{Cyan!0.000}{\strut IFE} \colorbox{Cyan!0.000}{\strut !} \colorbox{Cyan!0.000}{\strut It} \colorbox{Cyan!0.000}{\strut \textquotesingle{}} \colorbox{Cyan!0.000}{\strut s} \colorbox{Cyan!0.000}{\strut  simple} \colorbox{Cyan!0.000}{\strut ,} \colorbox{Cyan!0.000}{\strut  I} \colorbox{Cyan!0.000}{\strut \textquotesingle{}} \colorbox{Cyan!0.000}{\strut ll} \colorbox{Cyan!0.000}{\strut  take} \colorbox{Cyan!0.000}{\strut  out} \colorbox{Cyan!0.000}{\strut  a} \\
Input SAE & \num{1.334e+00} & \colorbox{Green!0.000}{\strut L} \colorbox{Green!0.000}{\strut IFE} \colorbox{Green!0.000}{\strut !} \colorbox{Green!0.000}{\strut It} \colorbox{Green!41.346}{\strut \textquotesingle{}} \colorbox{Green!25.268}{\strut s} \colorbox{Green!0.000}{\strut  simple} \colorbox{Green!0.000}{\strut ,} \colorbox{Green!0.000}{\strut  I} \colorbox{Green!35.067}{\strut \textquotesingle{}} \colorbox{Green!0.000}{\strut ll} \colorbox{Green!0.000}{\strut  take} \colorbox{Green!0.000}{\strut  out} \colorbox{Green!0.000}{\strut  a} \\
Output SAE & \num{1.507e+00} & \colorbox{Magenta!0.000}{\strut L} \colorbox{Magenta!0.000}{\strut IFE} \colorbox{Magenta!0.000}{\strut !} \colorbox{Magenta!0.000}{\strut It} \colorbox{Magenta!0.000}{\strut \textquotesingle{}} \colorbox{Magenta!0.000}{\strut s} \colorbox{Magenta!0.000}{\strut  simple} \colorbox{Magenta!0.000}{\strut ,} \colorbox{Magenta!0.000}{\strut  I} \colorbox{Magenta!0.000}{\strut \textquotesingle{}} \colorbox{Magenta!0.000}{\strut ll} \colorbox{Magenta!0.000}{\strut  take} \colorbox{Magenta!0.000}{\strut  out} \colorbox{Magenta!0.000}{\strut  a} \\
\midrule
Jacobian & \num{2.059e-01} & \colorbox{Cyan!0.000}{\strut  TA} \colorbox{Cyan!0.000}{\strut KE} \colorbox{Cyan!0.000}{\strut  HIM} \colorbox{Cyan!0.000}{\strut  BACK} \colorbox{Cyan!0.000}{\strut !} \colorbox{Cyan!0.000}{\strut I} \colorbox{Cyan!0.000}{\strut  get} \colorbox{Cyan!0.000}{\strut  the} \colorbox{Cyan!0.000}{\strut  foot} \colorbox{Cyan!0.000}{\strut  cr} \colorbox{Cyan!0.000}{\strut amps} \colorbox{Cyan!0.000}{\strut  like} \colorbox{Cyan!0.000}{\strut  crazy} \colorbox{Cyan!0.000}{\strut .} \\
Input SAE & \num{7.619e-01} & \colorbox{Green!0.000}{\strut  TA} \colorbox{Green!0.000}{\strut KE} \colorbox{Green!0.000}{\strut  HIM} \colorbox{Green!0.000}{\strut  BACK} \colorbox{Green!0.000}{\strut !} \colorbox{Green!18.404}{\strut I} \colorbox{Green!23.616}{\strut  get} \colorbox{Green!0.000}{\strut  the} \colorbox{Green!21.220}{\strut  foot} \colorbox{Green!0.000}{\strut  cr} \colorbox{Green!0.000}{\strut amps} \colorbox{Green!0.000}{\strut  like} \colorbox{Green!0.000}{\strut  crazy} \colorbox{Green!0.000}{\strut .} \\
Output SAE & \num{1.505e+00} & \colorbox{Magenta!0.000}{\strut  TA} \colorbox{Magenta!0.000}{\strut KE} \colorbox{Magenta!0.000}{\strut  HIM} \colorbox{Magenta!0.000}{\strut  BACK} \colorbox{Magenta!0.000}{\strut !} \colorbox{Magenta!0.000}{\strut I} \colorbox{Magenta!0.000}{\strut  get} \colorbox{Magenta!0.000}{\strut  the} \colorbox{Magenta!0.000}{\strut  foot} \colorbox{Magenta!0.000}{\strut  cr} \colorbox{Magenta!0.000}{\strut amps} \colorbox{Magenta!0.000}{\strut  like} \colorbox{Magenta!0.000}{\strut  crazy} \colorbox{Magenta!0.000}{\strut .} \\
\midrule
Jacobian & \num{2.059e-01} & \colorbox{Cyan!0.000}{\strut  Sh} \colorbox{Cyan!0.000}{\strut own} \colorbox{Cyan!0.000}{\strut  .} \colorbox{Cyan!0.000}{\strut the} \colorbox{Cyan!0.000}{\strut  bucket} \colorbox{Cyan!0.000}{\strut  wheel} \colorbox{Cyan!0.000}{\strut  excav} \colorbox{Cyan!0.000}{\strut ator} \colorbox{Cyan!0.000}{\strut .} \colorbox{Cyan!0.000}{\strut  Key} \colorbox{Cyan!0.000}{\strut  Words} \colorbox{Cyan!0.000}{\strut :} \colorbox{Cyan!0.000}{\strut  Mod} \colorbox{Cyan!0.000}{\strut eling} \\
Input SAE & \num{9.919e-01} & \colorbox{Green!22.277}{\strut  Sh} \colorbox{Green!0.000}{\strut own} \colorbox{Green!0.000}{\strut  .} \colorbox{Green!0.000}{\strut the} \colorbox{Green!0.000}{\strut  bucket} \colorbox{Green!29.921}{\strut  wheel} \colorbox{Green!0.000}{\strut  excav} \colorbox{Green!0.000}{\strut ator} \colorbox{Green!25.333}{\strut .} \colorbox{Green!30.743}{\strut  Key} \colorbox{Green!0.000}{\strut  Words} \colorbox{Green!0.000}{\strut :} \colorbox{Green!0.000}{\strut  Mod} \colorbox{Green!0.000}{\strut eling} \\
Output SAE & \num{1.505e+00} & \colorbox{Magenta!0.000}{\strut  Sh} \colorbox{Magenta!0.000}{\strut own} \colorbox{Magenta!0.000}{\strut  .} \colorbox{Magenta!0.000}{\strut the} \colorbox{Magenta!0.000}{\strut  bucket} \colorbox{Magenta!0.000}{\strut  wheel} \colorbox{Magenta!0.000}{\strut  excav} \colorbox{Magenta!0.000}{\strut ator} \colorbox{Magenta!0.000}{\strut .} \colorbox{Magenta!0.000}{\strut  Key} \colorbox{Magenta!0.000}{\strut  Words} \colorbox{Magenta!0.000}{\strut :} \colorbox{Magenta!0.000}{\strut  Mod} \colorbox{Magenta!0.000}{\strut eling} \\
\midrule
Jacobian & \num{2.059e-01} & \colorbox{Cyan!0.000}{\strut  \textquotedbl{}} \colorbox{Cyan!0.000}{\strut St} \colorbox{Cyan!0.000}{\strut amps} \colorbox{Cyan!0.000}{\strut  are} \colorbox{Cyan!0.000}{\strut  St} \colorbox{Cyan!0.000}{\strut amp} \colorbox{Cyan!0.000}{\strut in} \colorbox{Cyan!0.000}{\strut  up} \colorbox{Cyan!0.000}{\strut ,} \colorbox{Cyan!0.000}{\strut  products} \colorbox{Cyan!0.000}{\strut ,} \colorbox{Cyan!0.000}{\strut  couldn} \colorbox{Cyan!0.000}{\strut \textquotesingle{}t} \colorbox{Cyan!0.000}{\strut  do} \\
Input SAE & \num{1.444e+00} & \colorbox{Green!0.000}{\strut  \textquotedbl{}} \colorbox{Green!25.874}{\strut St} \colorbox{Green!0.000}{\strut amps} \colorbox{Green!24.173}{\strut  are} \colorbox{Green!0.000}{\strut  St} \colorbox{Green!0.000}{\strut amp} \colorbox{Green!0.000}{\strut in} \colorbox{Green!44.750}{\strut  up} \colorbox{Green!0.000}{\strut ,} \colorbox{Green!42.703}{\strut  products} \colorbox{Green!0.000}{\strut ,} \colorbox{Green!31.495}{\strut  couldn} \colorbox{Green!40.088}{\strut \textquotesingle{}t} \colorbox{Green!0.000}{\strut  do} \\
Output SAE & \num{1.508e+00} & \colorbox{Magenta!0.000}{\strut  \textquotedbl{}} \colorbox{Magenta!0.000}{\strut St} \colorbox{Magenta!0.000}{\strut amps} \colorbox{Magenta!0.000}{\strut  are} \colorbox{Magenta!0.000}{\strut  St} \colorbox{Magenta!0.000}{\strut amp} \colorbox{Magenta!0.000}{\strut in} \colorbox{Magenta!0.000}{\strut  up} \colorbox{Magenta!0.000}{\strut ,} \colorbox{Magenta!0.000}{\strut  products} \colorbox{Magenta!0.000}{\strut ,} \colorbox{Magenta!0.000}{\strut  couldn} \colorbox{Magenta!0.000}{\strut \textquotesingle{}t} \colorbox{Magenta!0.000}{\strut  do} \\
\midrule
Jacobian & \num{2.059e-01} & \colorbox{Cyan!0.000}{\strut  R} \colorbox{Cyan!0.000}{\strut ear} \colorbox{Cyan!0.000}{\strut  .} \colorbox{Cyan!0.000}{\strut Buy} \colorbox{Cyan!0.000}{\strut  Saf} \colorbox{Cyan!0.000}{\strut star} \colorbox{Cyan!0.000}{\strut  Motor} \colorbox{Cyan!0.000}{\strut cycle} \colorbox{Cyan!0.000}{\strut  Stand} \colorbox{Cyan!0.000}{\strut  Sport} \colorbox{Cyan!0.000}{\strut  B} \colorbox{Cyan!0.000}{\strut ike} \colorbox{Cyan!0.000}{\strut  R} \colorbox{Cyan!0.000}{\strut ear} \\
Input SAE & \num{1.315e+00} & \colorbox{Green!31.102}{\strut  R} \colorbox{Green!40.763}{\strut ear} \colorbox{Green!0.000}{\strut  .} \colorbox{Green!0.000}{\strut Buy} \colorbox{Green!0.000}{\strut  Saf} \colorbox{Green!0.000}{\strut star} \colorbox{Green!0.000}{\strut  Motor} \colorbox{Green!0.000}{\strut cycle} \colorbox{Green!0.000}{\strut  Stand} \colorbox{Green!0.000}{\strut  Sport} \colorbox{Green!0.000}{\strut  B} \colorbox{Green!0.000}{\strut ike} \colorbox{Green!0.000}{\strut  R} \colorbox{Green!0.000}{\strut ear} \\
Output SAE & \num{1.505e+00} & \colorbox{Magenta!0.000}{\strut  R} \colorbox{Magenta!0.000}{\strut ear} \colorbox{Magenta!0.000}{\strut  .} \colorbox{Magenta!0.000}{\strut Buy} \colorbox{Magenta!0.000}{\strut  Saf} \colorbox{Magenta!0.000}{\strut star} \colorbox{Magenta!0.000}{\strut  Motor} \colorbox{Magenta!0.000}{\strut cycle} \colorbox{Magenta!0.000}{\strut  Stand} \colorbox{Magenta!0.000}{\strut  Sport} \colorbox{Magenta!0.000}{\strut  B} \colorbox{Magenta!0.000}{\strut ike} \colorbox{Magenta!0.000}{\strut  R} \colorbox{Magenta!0.000}{\strut ear} \\
\midrule
Jacobian & \num{2.059e-01} & \colorbox{Cyan!0.000}{\strut  Tips} \colorbox{Cyan!0.000}{\strut  And} \colorbox{Cyan!0.000}{\strut  T} \colorbox{Cyan!0.000}{\strut ricks} \colorbox{Cyan!0.000}{\strut  With} \colorbox{Cyan!0.000}{\strut  .} \colorbox{Cyan!0.000}{\strut 5} \colorbox{Cyan!0.000}{\strut  Res} \colorbox{Cyan!0.000}{\strut ume} \colorbox{Cyan!0.000}{\strut  Writing} \colorbox{Cyan!0.000}{\strut  Tips} \colorbox{Cyan!0.000}{\strut  5} \colorbox{Cyan!0.000}{\strut  Res} \colorbox{Cyan!0.000}{\strut ume} \\
Input SAE & \num{9.586e-01} & \colorbox{Green!0.000}{\strut  Tips} \colorbox{Green!0.000}{\strut  And} \colorbox{Green!0.000}{\strut  T} \colorbox{Green!0.000}{\strut ricks} \colorbox{Green!0.000}{\strut  With} \colorbox{Green!0.000}{\strut  .} \colorbox{Green!25.398}{\strut 5} \colorbox{Green!24.275}{\strut  Res} \colorbox{Green!0.000}{\strut ume} \colorbox{Green!0.000}{\strut  Writing} \colorbox{Green!29.713}{\strut  Tips} \colorbox{Green!0.000}{\strut  5} \colorbox{Green!0.000}{\strut  Res} \colorbox{Green!0.000}{\strut ume} \\
Output SAE & \num{1.506e+00} & \colorbox{Magenta!0.000}{\strut  Tips} \colorbox{Magenta!0.000}{\strut  And} \colorbox{Magenta!0.000}{\strut  T} \colorbox{Magenta!0.000}{\strut ricks} \colorbox{Magenta!0.000}{\strut  With} \colorbox{Magenta!0.000}{\strut  .} \colorbox{Magenta!0.000}{\strut 5} \colorbox{Magenta!0.000}{\strut  Res} \colorbox{Magenta!0.000}{\strut ume} \colorbox{Magenta!0.000}{\strut  Writing} \colorbox{Magenta!0.000}{\strut  Tips} \colorbox{Magenta!0.000}{\strut  5} \colorbox{Magenta!0.000}{\strut  Res} \colorbox{Magenta!0.000}{\strut ume} \\
\midrule
Jacobian & \num{2.058e-01} & \colorbox{Cyan!0.000}{\strut  Wow} \colorbox{Cyan!0.000}{\strut !} \colorbox{Cyan!0.000}{\strut So} \colorbox{Cyan!0.000}{\strut  in} \colorbox{Cyan!0.000}{\strut  light} \colorbox{Cyan!0.000}{\strut  of} \colorbox{Cyan!0.000}{\strut  that} \colorbox{Cyan!0.000}{\strut  you} \colorbox{Cyan!0.000}{\strut  can} \colorbox{Cyan!0.000}{\strut  read} \colorbox{Cyan!0.000}{\strut  my} \colorbox{Cyan!0.000}{\strut  oh} \colorbox{Cyan!0.000}{\strut  so} \colorbox{Cyan!0.000}{\strut  regret} \\
Input SAE & \num{1.413e+00} & \colorbox{Green!0.000}{\strut  Wow} \colorbox{Green!0.000}{\strut !} \colorbox{Green!19.469}{\strut So} \colorbox{Green!32.965}{\strut  in} \colorbox{Green!26.496}{\strut  light} \colorbox{Green!0.000}{\strut  of} \colorbox{Green!0.000}{\strut  that} \colorbox{Green!0.000}{\strut  you} \colorbox{Green!36.252}{\strut  can} \colorbox{Green!29.545}{\strut  read} \colorbox{Green!0.000}{\strut  my} \colorbox{Green!0.000}{\strut  oh} \colorbox{Green!0.000}{\strut  so} \colorbox{Green!43.809}{\strut  regret} \\
Output SAE & \num{1.505e+00} & \colorbox{Magenta!0.000}{\strut  Wow} \colorbox{Magenta!0.000}{\strut !} \colorbox{Magenta!0.000}{\strut So} \colorbox{Magenta!0.000}{\strut  in} \colorbox{Magenta!0.000}{\strut  light} \colorbox{Magenta!0.000}{\strut  of} \colorbox{Magenta!0.000}{\strut  that} \colorbox{Magenta!0.000}{\strut  you} \colorbox{Magenta!0.000}{\strut  can} \colorbox{Magenta!0.000}{\strut  read} \colorbox{Magenta!0.000}{\strut  my} \colorbox{Magenta!0.000}{\strut  oh} \colorbox{Magenta!0.000}{\strut  so} \colorbox{Magenta!0.000}{\strut  regret} \\
\midrule
Jacobian & \num{2.058e-01} & \colorbox{Cyan!0.000}{\strut OOK} \colorbox{Cyan!0.000}{\strut  LIKE} \colorbox{Cyan!0.000}{\strut !} \colorbox{Cyan!0.000}{\strut IS} \colorbox{Cyan!0.000}{\strut  THE} \colorbox{Cyan!0.000}{\strut  B} \colorbox{Cyan!0.000}{\strut ALL} \colorbox{Cyan!0.000}{\strut OT} \colorbox{Cyan!0.000}{\strut  QUE} \colorbox{Cyan!0.000}{\strut ST} \colorbox{Cyan!0.000}{\strut ION} \colorbox{Cyan!0.000}{\strut  F} \colorbox{Cyan!0.000}{\strut LA} \colorbox{Cyan!0.000}{\strut W} \\
Input SAE & \num{1.699e+00} & \colorbox{Green!0.000}{\strut OOK} \colorbox{Green!0.000}{\strut  LIKE} \colorbox{Green!0.000}{\strut !} \colorbox{Green!0.000}{\strut IS} \colorbox{Green!0.000}{\strut  THE} \colorbox{Green!46.473}{\strut  B} \colorbox{Green!0.000}{\strut ALL} \colorbox{Green!0.000}{\strut OT} \colorbox{Green!38.678}{\strut  QUE} \colorbox{Green!34.962}{\strut ST} \colorbox{Green!0.000}{\strut ION} \colorbox{Green!52.677}{\strut  F} \colorbox{Green!52.077}{\strut LA} \colorbox{Green!0.000}{\strut W} \\
Output SAE & \num{1.505e+00} & \colorbox{Magenta!0.000}{\strut OOK} \colorbox{Magenta!0.000}{\strut  LIKE} \colorbox{Magenta!0.000}{\strut !} \colorbox{Magenta!0.000}{\strut IS} \colorbox{Magenta!0.000}{\strut  THE} \colorbox{Magenta!0.000}{\strut  B} \colorbox{Magenta!0.000}{\strut ALL} \colorbox{Magenta!0.000}{\strut OT} \colorbox{Magenta!0.000}{\strut  QUE} \colorbox{Magenta!0.000}{\strut ST} \colorbox{Magenta!0.000}{\strut ION} \colorbox{Magenta!0.000}{\strut  F} \colorbox{Magenta!0.000}{\strut LA} \colorbox{Magenta!0.000}{\strut W} \\
\midrule
Jacobian & \num{2.058e-01} & \colorbox{Cyan!0.000}{\strut  Tips} \colorbox{Cyan!0.000}{\strut  Choose} \colorbox{Cyan!0.000}{\strut  .} \colorbox{Cyan!0.000}{\strut Best} \colorbox{Cyan!0.000}{\strut  Res} \colorbox{Cyan!0.000}{\strut ume} \colorbox{Cyan!0.000}{\strut  Format} \colorbox{Cyan!0.000}{\strut  Forbes} \colorbox{Cyan!0.000}{\strut  Tips} \colorbox{Cyan!0.000}{\strut  On} \colorbox{Cyan!0.000}{\strut  Res} \colorbox{Cyan!0.000}{\strut ume} \colorbox{Cyan!0.000}{\strut  Writing} \colorbox{Cyan!0.000}{\strut  Res} \\
Input SAE & \num{2.161e-01} & \colorbox{Green!0.000}{\strut  Tips} \colorbox{Green!0.000}{\strut  Choose} \colorbox{Green!0.000}{\strut  .} \colorbox{Green!0.000}{\strut Best} \colorbox{Green!0.000}{\strut  Res} \colorbox{Green!0.000}{\strut ume} \colorbox{Green!0.000}{\strut  Format} \colorbox{Green!0.000}{\strut  Forbes} \colorbox{Green!0.000}{\strut  Tips} \colorbox{Green!0.000}{\strut  On} \colorbox{Green!0.000}{\strut  Res} \colorbox{Green!0.000}{\strut ume} \colorbox{Green!0.000}{\strut  Writing} \colorbox{Green!0.000}{\strut  Res} \\
Output SAE & \num{1.505e+00} & \colorbox{Magenta!0.000}{\strut  Tips} \colorbox{Magenta!0.000}{\strut  Choose} \colorbox{Magenta!0.000}{\strut  .} \colorbox{Magenta!0.000}{\strut Best} \colorbox{Magenta!0.000}{\strut  Res} \colorbox{Magenta!0.000}{\strut ume} \colorbox{Magenta!0.000}{\strut  Format} \colorbox{Magenta!0.000}{\strut  Forbes} \colorbox{Magenta!0.000}{\strut  Tips} \colorbox{Magenta!0.000}{\strut  On} \colorbox{Magenta!0.000}{\strut  Res} \colorbox{Magenta!0.000}{\strut ume} \colorbox{Magenta!0.000}{\strut  Writing} \colorbox{Magenta!0.000}{\strut  Res} \\
\midrule
Jacobian & \num{2.058e-01} & \colorbox{Cyan!0.000}{\strut  100} \colorbox{Cyan!0.000}{\strut \%} \colorbox{Cyan!0.000}{\strut  .} \colorbox{Cyan!0.000}{\strut G} \colorbox{Cyan!0.000}{\strut arant} \colorbox{Cyan!0.000}{\strut is} \colorbox{Cyan!0.000}{\strut  2} \colorbox{Cyan!0.000}{\strut .} \colorbox{Cyan!0.000}{\strut 5} \colorbox{Cyan!0.000}{\strut  3} \colorbox{Cyan!0.000}{\strut .} \colorbox{Cyan!0.000}{\strut 5} \colorbox{Cyan!0.000}{\strut  S} \colorbox{Cyan!0.000}{\strut ATA} \\
Input SAE & \num{9.834e-01} & \colorbox{Green!0.000}{\strut  100} \colorbox{Green!0.000}{\strut \%} \colorbox{Green!0.000}{\strut  .} \colorbox{Green!30.483}{\strut G} \colorbox{Green!0.000}{\strut arant} \colorbox{Green!0.000}{\strut is} \colorbox{Green!0.000}{\strut  2} \colorbox{Green!0.000}{\strut .} \colorbox{Green!0.000}{\strut 5} \colorbox{Green!0.000}{\strut  3} \colorbox{Green!0.000}{\strut .} \colorbox{Green!0.000}{\strut 5} \colorbox{Green!0.000}{\strut  S} \colorbox{Green!0.000}{\strut ATA} \\
Output SAE & \num{1.505e+00} & \colorbox{Magenta!0.000}{\strut  100} \colorbox{Magenta!0.000}{\strut \%} \colorbox{Magenta!0.000}{\strut  .} \colorbox{Magenta!0.000}{\strut G} \colorbox{Magenta!0.000}{\strut arant} \colorbox{Magenta!0.000}{\strut is} \colorbox{Magenta!0.000}{\strut  2} \colorbox{Magenta!0.000}{\strut .} \colorbox{Magenta!0.000}{\strut 5} \colorbox{Magenta!0.000}{\strut  3} \colorbox{Magenta!0.000}{\strut .} \colorbox{Magenta!0.000}{\strut 5} \colorbox{Magenta!0.000}{\strut  S} \colorbox{Magenta!0.000}{\strut ATA} \\
\midrule
Jacobian & \num{2.058e-01} & \colorbox{Cyan!0.000}{\strut  Like} \colorbox{Cyan!0.000}{\strut ?} \colorbox{Cyan!0.000}{\strut In} \colorbox{Cyan!0.000}{\strut  Your} \colorbox{Cyan!0.000}{\strut  Light} \colorbox{Cyan!0.000}{\strut ,} \colorbox{Cyan!0.000}{\strut  We} \colorbox{Cyan!0.000}{\strut  Shall} \colorbox{Cyan!0.000}{\strut  See} \colorbox{Cyan!0.000}{\strut  Light} \colorbox{Cyan!0.000}{\strut !} \colorbox{Cyan!0.000}{\strut The} \colorbox{Cyan!0.000}{\strut  second} \\
Input SAE & \num{1.064e+00} & \colorbox{Green!0.000}{\strut  Like} \colorbox{Green!0.000}{\strut ?} \colorbox{Green!32.991}{\strut In} \colorbox{Green!0.000}{\strut  Your} \colorbox{Green!0.000}{\strut  Light} \colorbox{Green!30.179}{\strut ,} \colorbox{Green!0.000}{\strut  We} \colorbox{Green!0.000}{\strut  Shall} \colorbox{Green!0.000}{\strut  See} \colorbox{Green!0.000}{\strut  Light} \colorbox{Green!0.000}{\strut !} \colorbox{Green!0.000}{\strut The} \colorbox{Green!0.000}{\strut  second} \\
Output SAE & \num{1.505e+00} & \colorbox{Magenta!0.000}{\strut  Like} \colorbox{Magenta!0.000}{\strut ?} \colorbox{Magenta!0.000}{\strut In} \colorbox{Magenta!0.000}{\strut  Your} \colorbox{Magenta!0.000}{\strut  Light} \colorbox{Magenta!0.000}{\strut ,} \colorbox{Magenta!0.000}{\strut  We} \colorbox{Magenta!0.000}{\strut  Shall} \colorbox{Magenta!0.000}{\strut  See} \colorbox{Magenta!0.000}{\strut  Light} \colorbox{Magenta!0.000}{\strut !} \colorbox{Magenta!0.000}{\strut The} \colorbox{Magenta!0.000}{\strut  second} \\
\bottomrule
\end{longtable}
\caption{feature pairs/Layer15-65536-J1-LR5.0e-04-k32-T3.0e+08 abs mean/examples-29982-v-18929 stas c4-en-10k,train,batch size=32,ctx len=16.csv}
\end{table}
% \begin{table}
\centering
\begin{longtable}{lrl}
\toprule
Category & Max. abs. value & Example tokens \\
\midrule
Jacobian & \num{2.049e-01} & \colorbox{Cyan!0.000}{\strut L} \colorbox{Cyan!0.000}{\strut IFE} \colorbox{Cyan!0.000}{\strut !} \colorbox{Cyan!0.000}{\strut It} \colorbox{Cyan!0.000}{\strut \textquotesingle{}} \colorbox{Cyan!0.000}{\strut s} \colorbox{Cyan!0.000}{\strut  simple} \colorbox{Cyan!0.000}{\strut ,} \colorbox{Cyan!0.000}{\strut  I} \colorbox{Cyan!0.000}{\strut \textquotesingle{}} \colorbox{Cyan!0.000}{\strut ll} \colorbox{Cyan!0.000}{\strut  take} \colorbox{Cyan!0.000}{\strut  out} \colorbox{Cyan!0.000}{\strut  a} \\
Input SAE & \num{1.334e+00} & \colorbox{Green!0.000}{\strut L} \colorbox{Green!0.000}{\strut IFE} \colorbox{Green!0.000}{\strut !} \colorbox{Green!0.000}{\strut It} \colorbox{Green!41.346}{\strut \textquotesingle{}} \colorbox{Green!25.268}{\strut s} \colorbox{Green!0.000}{\strut  simple} \colorbox{Green!0.000}{\strut ,} \colorbox{Green!0.000}{\strut  I} \colorbox{Green!35.067}{\strut \textquotesingle{}} \colorbox{Green!0.000}{\strut ll} \colorbox{Green!0.000}{\strut  take} \colorbox{Green!0.000}{\strut  out} \colorbox{Green!0.000}{\strut  a} \\
Output SAE & \num{1.534e+00} & \colorbox{Magenta!0.000}{\strut L} \colorbox{Magenta!0.000}{\strut IFE} \colorbox{Magenta!0.000}{\strut !} \colorbox{Magenta!0.000}{\strut It} \colorbox{Magenta!0.000}{\strut \textquotesingle{}} \colorbox{Magenta!0.000}{\strut s} \colorbox{Magenta!0.000}{\strut  simple} \colorbox{Magenta!0.000}{\strut ,} \colorbox{Magenta!0.000}{\strut  I} \colorbox{Magenta!0.000}{\strut \textquotesingle{}} \colorbox{Magenta!0.000}{\strut ll} \colorbox{Magenta!0.000}{\strut  take} \colorbox{Magenta!0.000}{\strut  out} \colorbox{Magenta!0.000}{\strut  a} \\
\midrule
Jacobian & \num{2.049e-01} & \colorbox{Cyan!0.000}{\strut  TA} \colorbox{Cyan!0.000}{\strut KE} \colorbox{Cyan!0.000}{\strut  HIM} \colorbox{Cyan!0.000}{\strut  BACK} \colorbox{Cyan!0.000}{\strut !} \colorbox{Cyan!0.000}{\strut I} \colorbox{Cyan!0.000}{\strut  get} \colorbox{Cyan!0.000}{\strut  the} \colorbox{Cyan!0.000}{\strut  foot} \colorbox{Cyan!0.000}{\strut  cr} \colorbox{Cyan!0.000}{\strut amps} \colorbox{Cyan!0.000}{\strut  like} \colorbox{Cyan!0.000}{\strut  crazy} \colorbox{Cyan!0.000}{\strut .} \\
Input SAE & \num{7.619e-01} & \colorbox{Green!0.000}{\strut  TA} \colorbox{Green!0.000}{\strut KE} \colorbox{Green!0.000}{\strut  HIM} \colorbox{Green!0.000}{\strut  BACK} \colorbox{Green!0.000}{\strut !} \colorbox{Green!18.404}{\strut I} \colorbox{Green!23.616}{\strut  get} \colorbox{Green!0.000}{\strut  the} \colorbox{Green!21.220}{\strut  foot} \colorbox{Green!0.000}{\strut  cr} \colorbox{Green!0.000}{\strut amps} \colorbox{Green!0.000}{\strut  like} \colorbox{Green!0.000}{\strut  crazy} \colorbox{Green!0.000}{\strut .} \\
Output SAE & \num{1.533e+00} & \colorbox{Magenta!0.000}{\strut  TA} \colorbox{Magenta!0.000}{\strut KE} \colorbox{Magenta!0.000}{\strut  HIM} \colorbox{Magenta!0.000}{\strut  BACK} \colorbox{Magenta!0.000}{\strut !} \colorbox{Magenta!0.000}{\strut I} \colorbox{Magenta!0.000}{\strut  get} \colorbox{Magenta!0.000}{\strut  the} \colorbox{Magenta!0.000}{\strut  foot} \colorbox{Magenta!0.000}{\strut  cr} \colorbox{Magenta!0.000}{\strut amps} \colorbox{Magenta!0.000}{\strut  like} \colorbox{Magenta!0.000}{\strut  crazy} \colorbox{Magenta!0.000}{\strut .} \\
\midrule
Jacobian & \num{2.049e-01} & \colorbox{Cyan!0.000}{\strut  BRE} \colorbox{Cyan!0.000}{\strut ATH} \colorbox{Cyan!0.000}{\strut !} \colorbox{Cyan!0.000}{\strut Such} \colorbox{Cyan!0.000}{\strut  a} \colorbox{Cyan!0.000}{\strut  simple} \colorbox{Cyan!0.000}{\strut  solution} \colorbox{Cyan!0.000}{\strut  to} \colorbox{Cyan!0.000}{\strut  help} \colorbox{Cyan!0.000}{\strut  ease} \colorbox{Cyan!0.000}{\strut  the} \colorbox{Cyan!0.000}{\strut  journey} \colorbox{Cyan!0.000}{\strut .} \colorbox{Cyan!0.000}{\strut  Safe} \\
Input SAE & \num{1.095e+00} & \colorbox{Green!0.000}{\strut  BRE} \colorbox{Green!0.000}{\strut ATH} \colorbox{Green!0.000}{\strut !} \colorbox{Green!0.000}{\strut Such} \colorbox{Green!0.000}{\strut  a} \colorbox{Green!0.000}{\strut  simple} \colorbox{Green!0.000}{\strut  solution} \colorbox{Green!33.955}{\strut  to} \colorbox{Green!0.000}{\strut  help} \colorbox{Green!0.000}{\strut  ease} \colorbox{Green!0.000}{\strut  the} \colorbox{Green!0.000}{\strut  journey} \colorbox{Green!0.000}{\strut .} \colorbox{Green!0.000}{\strut  Safe} \\
Output SAE & \num{1.532e+00} & \colorbox{Magenta!0.000}{\strut  BRE} \colorbox{Magenta!0.000}{\strut ATH} \colorbox{Magenta!0.000}{\strut !} \colorbox{Magenta!0.000}{\strut Such} \colorbox{Magenta!0.000}{\strut  a} \colorbox{Magenta!0.000}{\strut  simple} \colorbox{Magenta!0.000}{\strut  solution} \colorbox{Magenta!0.000}{\strut  to} \colorbox{Magenta!0.000}{\strut  help} \colorbox{Magenta!0.000}{\strut  ease} \colorbox{Magenta!0.000}{\strut  the} \colorbox{Magenta!0.000}{\strut  journey} \colorbox{Magenta!0.000}{\strut .} \colorbox{Magenta!0.000}{\strut  Safe} \\
\midrule
Jacobian & \num{2.049e-01} & \colorbox{Cyan!0.000}{\strut  Tips} \colorbox{Cyan!0.000}{\strut  And} \colorbox{Cyan!0.000}{\strut  T} \colorbox{Cyan!0.000}{\strut ricks} \colorbox{Cyan!0.000}{\strut  With} \colorbox{Cyan!0.000}{\strut  .} \colorbox{Cyan!0.000}{\strut 5} \colorbox{Cyan!0.000}{\strut  Res} \colorbox{Cyan!0.000}{\strut ume} \colorbox{Cyan!0.000}{\strut  Writing} \colorbox{Cyan!0.000}{\strut  Tips} \colorbox{Cyan!0.000}{\strut  5} \colorbox{Cyan!0.000}{\strut  Res} \colorbox{Cyan!0.000}{\strut ume} \\
Input SAE & \num{9.586e-01} & \colorbox{Green!0.000}{\strut  Tips} \colorbox{Green!0.000}{\strut  And} \colorbox{Green!0.000}{\strut  T} \colorbox{Green!0.000}{\strut ricks} \colorbox{Green!0.000}{\strut  With} \colorbox{Green!0.000}{\strut  .} \colorbox{Green!25.398}{\strut 5} \colorbox{Green!24.275}{\strut  Res} \colorbox{Green!0.000}{\strut ume} \colorbox{Green!0.000}{\strut  Writing} \colorbox{Green!29.713}{\strut  Tips} \colorbox{Green!0.000}{\strut  5} \colorbox{Green!0.000}{\strut  Res} \colorbox{Green!0.000}{\strut ume} \\
Output SAE & \num{1.534e+00} & \colorbox{Magenta!0.000}{\strut  Tips} \colorbox{Magenta!0.000}{\strut  And} \colorbox{Magenta!0.000}{\strut  T} \colorbox{Magenta!0.000}{\strut ricks} \colorbox{Magenta!0.000}{\strut  With} \colorbox{Magenta!0.000}{\strut  .} \colorbox{Magenta!0.000}{\strut 5} \colorbox{Magenta!0.000}{\strut  Res} \colorbox{Magenta!0.000}{\strut ume} \colorbox{Magenta!0.000}{\strut  Writing} \colorbox{Magenta!0.000}{\strut  Tips} \colorbox{Magenta!0.000}{\strut  5} \colorbox{Magenta!0.000}{\strut  Res} \colorbox{Magenta!0.000}{\strut ume} \\
\midrule
Jacobian & \num{2.049e-01} & \colorbox{Cyan!0.000}{\strut  Tips} \colorbox{Cyan!0.000}{\strut  For} \colorbox{Cyan!0.000}{\strut  .} \colorbox{Cyan!0.000}{\strut Res} \colorbox{Cyan!0.000}{\strut ume} \colorbox{Cyan!0.000}{\strut  Template} \colorbox{Cyan!0.000}{\strut  Res} \colorbox{Cyan!0.000}{\strut ume} \colorbox{Cyan!0.000}{\strut  Format} \colorbox{Cyan!0.000}{\strut  Tips} \colorbox{Cyan!0.000}{\strut  Sample} \colorbox{Cyan!0.000}{\strut  Res} \colorbox{Cyan!0.000}{\strut ume} \colorbox{Cyan!0.000}{\strut  Template} \\
Input SAE & \num{8.074e-01} & \colorbox{Green!0.000}{\strut  Tips} \colorbox{Green!0.000}{\strut  For} \colorbox{Green!0.000}{\strut  .} \colorbox{Green!25.026}{\strut Res} \colorbox{Green!0.000}{\strut ume} \colorbox{Green!0.000}{\strut  Template} \colorbox{Green!0.000}{\strut  Res} \colorbox{Green!0.000}{\strut ume} \colorbox{Green!0.000}{\strut  Format} \colorbox{Green!0.000}{\strut  Tips} \colorbox{Green!0.000}{\strut  Sample} \colorbox{Green!0.000}{\strut  Res} \colorbox{Green!0.000}{\strut ume} \colorbox{Green!0.000}{\strut  Template} \\
Output SAE & \num{1.536e+00} & \colorbox{Magenta!0.000}{\strut  Tips} \colorbox{Magenta!0.000}{\strut  For} \colorbox{Magenta!0.000}{\strut  .} \colorbox{Magenta!0.000}{\strut Res} \colorbox{Magenta!0.000}{\strut ume} \colorbox{Magenta!0.000}{\strut  Template} \colorbox{Magenta!0.000}{\strut  Res} \colorbox{Magenta!0.000}{\strut ume} \colorbox{Magenta!0.000}{\strut  Format} \colorbox{Magenta!0.000}{\strut  Tips} \colorbox{Magenta!0.000}{\strut  Sample} \colorbox{Magenta!0.000}{\strut  Res} \colorbox{Magenta!0.000}{\strut ume} \colorbox{Magenta!0.000}{\strut  Template} \\
\midrule
Jacobian & \num{2.049e-01} & \colorbox{Cyan!0.000}{\strut  Tips} \colorbox{Cyan!0.000}{\strut  Choose} \colorbox{Cyan!0.000}{\strut  .} \colorbox{Cyan!0.000}{\strut Best} \colorbox{Cyan!0.000}{\strut  Res} \colorbox{Cyan!0.000}{\strut ume} \colorbox{Cyan!0.000}{\strut  Format} \colorbox{Cyan!0.000}{\strut  Forbes} \colorbox{Cyan!0.000}{\strut  Tips} \colorbox{Cyan!0.000}{\strut  On} \colorbox{Cyan!0.000}{\strut  Res} \colorbox{Cyan!0.000}{\strut ume} \colorbox{Cyan!0.000}{\strut  Writing} \colorbox{Cyan!0.000}{\strut  Res} \\
Input SAE & \num{2.161e-01} & \colorbox{Green!0.000}{\strut  Tips} \colorbox{Green!0.000}{\strut  Choose} \colorbox{Green!0.000}{\strut  .} \colorbox{Green!0.000}{\strut Best} \colorbox{Green!0.000}{\strut  Res} \colorbox{Green!0.000}{\strut ume} \colorbox{Green!0.000}{\strut  Format} \colorbox{Green!0.000}{\strut  Forbes} \colorbox{Green!0.000}{\strut  Tips} \colorbox{Green!0.000}{\strut  On} \colorbox{Green!0.000}{\strut  Res} \colorbox{Green!0.000}{\strut ume} \colorbox{Green!0.000}{\strut  Writing} \colorbox{Green!0.000}{\strut  Res} \\
Output SAE & \num{1.532e+00} & \colorbox{Magenta!0.000}{\strut  Tips} \colorbox{Magenta!0.000}{\strut  Choose} \colorbox{Magenta!0.000}{\strut  .} \colorbox{Magenta!0.000}{\strut Best} \colorbox{Magenta!0.000}{\strut  Res} \colorbox{Magenta!0.000}{\strut ume} \colorbox{Magenta!0.000}{\strut  Format} \colorbox{Magenta!0.000}{\strut  Forbes} \colorbox{Magenta!0.000}{\strut  Tips} \colorbox{Magenta!0.000}{\strut  On} \colorbox{Magenta!0.000}{\strut  Res} \colorbox{Magenta!0.000}{\strut ume} \colorbox{Magenta!0.000}{\strut  Writing} \colorbox{Magenta!0.000}{\strut  Res} \\
\midrule
Jacobian & \num{2.049e-01} & \colorbox{Cyan!0.000}{\strut  Tips} \colorbox{Cyan!0.000}{\strut  Photo} \colorbox{Cyan!0.000}{\strut  Album} \colorbox{Cyan!0.000}{\strut  For} \colorbox{Cyan!0.000}{\strut  .} \colorbox{Cyan!0.000}{\strut 17} \colorbox{Cyan!0.000}{\strut  Lux} \colorbox{Cyan!0.000}{\strut ury} \colorbox{Cyan!0.000}{\strut  Res} \colorbox{Cyan!0.000}{\strut ume} \colorbox{Cyan!0.000}{\strut  Format} \colorbox{Cyan!0.000}{\strut ting} \colorbox{Cyan!0.000}{\strut  Tips} \colorbox{Cyan!0.000}{\strut  Tony} \\
Input SAE & \num{2.161e-01} & \colorbox{Green!0.000}{\strut  Tips} \colorbox{Green!0.000}{\strut  Photo} \colorbox{Green!0.000}{\strut  Album} \colorbox{Green!0.000}{\strut  For} \colorbox{Green!0.000}{\strut  .} \colorbox{Green!0.000}{\strut 17} \colorbox{Green!0.000}{\strut  Lux} \colorbox{Green!0.000}{\strut ury} \colorbox{Green!0.000}{\strut  Res} \colorbox{Green!0.000}{\strut ume} \colorbox{Green!0.000}{\strut  Format} \colorbox{Green!0.000}{\strut ting} \colorbox{Green!0.000}{\strut  Tips} \colorbox{Green!0.000}{\strut  Tony} \\
Output SAE & \num{1.533e+00} & \colorbox{Magenta!0.000}{\strut  Tips} \colorbox{Magenta!0.000}{\strut  Photo} \colorbox{Magenta!0.000}{\strut  Album} \colorbox{Magenta!0.000}{\strut  For} \colorbox{Magenta!0.000}{\strut  .} \colorbox{Magenta!0.000}{\strut 17} \colorbox{Magenta!0.000}{\strut  Lux} \colorbox{Magenta!0.000}{\strut ury} \colorbox{Magenta!0.000}{\strut  Res} \colorbox{Magenta!0.000}{\strut ume} \colorbox{Magenta!0.000}{\strut  Format} \colorbox{Magenta!0.000}{\strut ting} \colorbox{Magenta!0.000}{\strut  Tips} \colorbox{Magenta!0.000}{\strut  Tony} \\
\midrule
Jacobian & \num{2.048e-01} & \colorbox{Cyan!0.000}{\strut OOK} \colorbox{Cyan!0.000}{\strut  LIKE} \colorbox{Cyan!0.000}{\strut !} \colorbox{Cyan!0.000}{\strut IS} \colorbox{Cyan!0.000}{\strut  THE} \colorbox{Cyan!0.000}{\strut  B} \colorbox{Cyan!0.000}{\strut ALL} \colorbox{Cyan!0.000}{\strut OT} \colorbox{Cyan!0.000}{\strut  QUE} \colorbox{Cyan!0.000}{\strut ST} \colorbox{Cyan!0.000}{\strut ION} \colorbox{Cyan!0.000}{\strut  F} \colorbox{Cyan!0.000}{\strut LA} \colorbox{Cyan!0.000}{\strut W} \\
Input SAE & \num{1.699e+00} & \colorbox{Green!0.000}{\strut OOK} \colorbox{Green!0.000}{\strut  LIKE} \colorbox{Green!0.000}{\strut !} \colorbox{Green!0.000}{\strut IS} \colorbox{Green!0.000}{\strut  THE} \colorbox{Green!46.473}{\strut  B} \colorbox{Green!0.000}{\strut ALL} \colorbox{Green!0.000}{\strut OT} \colorbox{Green!38.678}{\strut  QUE} \colorbox{Green!34.962}{\strut ST} \colorbox{Green!0.000}{\strut ION} \colorbox{Green!52.677}{\strut  F} \colorbox{Green!52.077}{\strut LA} \colorbox{Green!0.000}{\strut W} \\
Output SAE & \num{1.533e+00} & \colorbox{Magenta!0.000}{\strut OOK} \colorbox{Magenta!0.000}{\strut  LIKE} \colorbox{Magenta!0.000}{\strut !} \colorbox{Magenta!0.000}{\strut IS} \colorbox{Magenta!0.000}{\strut  THE} \colorbox{Magenta!0.000}{\strut  B} \colorbox{Magenta!0.000}{\strut ALL} \colorbox{Magenta!0.000}{\strut OT} \colorbox{Magenta!0.000}{\strut  QUE} \colorbox{Magenta!0.000}{\strut ST} \colorbox{Magenta!0.000}{\strut ION} \colorbox{Magenta!0.000}{\strut  F} \colorbox{Magenta!0.000}{\strut LA} \colorbox{Magenta!0.000}{\strut W} \\
\midrule
Jacobian & \num{2.048e-01} & \colorbox{Cyan!0.000}{\strut rength} \colorbox{Cyan!0.000}{\strut -} \colorbox{Cyan!0.000}{\strut  Benefits} \colorbox{Cyan!0.000}{\strut !} \colorbox{Cyan!0.000}{\strut Th} \colorbox{Cyan!0.000}{\strut ousands} \colorbox{Cyan!0.000}{\strut  of} \colorbox{Cyan!0.000}{\strut  years} \colorbox{Cyan!0.000}{\strut  ago} \colorbox{Cyan!0.000}{\strut  yoga} \colorbox{Cyan!0.000}{\strut  originated} \colorbox{Cyan!0.000}{\strut  in} \colorbox{Cyan!0.000}{\strut  India} \colorbox{Cyan!0.000}{\strut ,} \\
Input SAE & \num{1.620e+00} & \colorbox{Green!0.000}{\strut rength} \colorbox{Green!0.000}{\strut -} \colorbox{Green!0.000}{\strut  Benefits} \colorbox{Green!0.000}{\strut !} \colorbox{Green!35.631}{\strut Th} \colorbox{Green!32.724}{\strut ousands} \colorbox{Green!28.010}{\strut  of} \colorbox{Green!50.225}{\strut  years} \colorbox{Green!0.000}{\strut  ago} \colorbox{Green!0.000}{\strut  yoga} \colorbox{Green!0.000}{\strut  originated} \colorbox{Green!0.000}{\strut  in} \colorbox{Green!0.000}{\strut  India} \colorbox{Green!0.000}{\strut ,} \\
Output SAE & \num{1.532e+00} & \colorbox{Magenta!0.000}{\strut rength} \colorbox{Magenta!0.000}{\strut -} \colorbox{Magenta!0.000}{\strut  Benefits} \colorbox{Magenta!0.000}{\strut !} \colorbox{Magenta!0.000}{\strut Th} \colorbox{Magenta!0.000}{\strut ousands} \colorbox{Magenta!0.000}{\strut  of} \colorbox{Magenta!0.000}{\strut  years} \colorbox{Magenta!0.000}{\strut  ago} \colorbox{Magenta!0.000}{\strut  yoga} \colorbox{Magenta!0.000}{\strut  originated} \colorbox{Magenta!0.000}{\strut  in} \colorbox{Magenta!0.000}{\strut  India} \colorbox{Magenta!0.000}{\strut ,} \\
\midrule
Jacobian & \num{2.048e-01} & \colorbox{Cyan!0.000}{\strut  who} \colorbox{Cyan!0.000}{\strut  needs} \colorbox{Cyan!0.000}{\strut  a} \colorbox{Cyan!0.000}{\strut  stand} \colorbox{Cyan!0.000}{\strut  out} \colorbox{Cyan!0.000}{\strut  resume} \colorbox{Cyan!0.000}{\strut !} \colorbox{Cyan!0.000}{\strut I} \colorbox{Cyan!0.000}{\strut  am} \colorbox{Cyan!0.000}{\strut  a} \colorbox{Cyan!0.000}{\strut  current} \colorbox{Cyan!0.000}{\strut  writer} \colorbox{Cyan!0.000}{\strut  for} \colorbox{Cyan!0.000}{\strut  The} \\
Input SAE & \num{1.114e+00} & \colorbox{Green!0.000}{\strut  who} \colorbox{Green!0.000}{\strut  needs} \colorbox{Green!0.000}{\strut  a} \colorbox{Green!0.000}{\strut  stand} \colorbox{Green!0.000}{\strut  out} \colorbox{Green!0.000}{\strut  resume} \colorbox{Green!0.000}{\strut !} \colorbox{Green!34.524}{\strut I} \colorbox{Green!33.866}{\strut  am} \colorbox{Green!0.000}{\strut  a} \colorbox{Green!0.000}{\strut  current} \colorbox{Green!29.547}{\strut  writer} \colorbox{Green!0.000}{\strut  for} \colorbox{Green!0.000}{\strut  The} \\
Output SAE & \num{1.533e+00} & \colorbox{Magenta!0.000}{\strut  who} \colorbox{Magenta!0.000}{\strut  needs} \colorbox{Magenta!0.000}{\strut  a} \colorbox{Magenta!0.000}{\strut  stand} \colorbox{Magenta!0.000}{\strut  out} \colorbox{Magenta!0.000}{\strut  resume} \colorbox{Magenta!0.000}{\strut !} \colorbox{Magenta!0.000}{\strut I} \colorbox{Magenta!0.000}{\strut  am} \colorbox{Magenta!0.000}{\strut  a} \colorbox{Magenta!0.000}{\strut  current} \colorbox{Magenta!0.000}{\strut  writer} \colorbox{Magenta!0.000}{\strut  for} \colorbox{Magenta!0.000}{\strut  The} \\
\midrule
Jacobian & \num{2.048e-01} & \colorbox{Cyan!0.000}{\strut  Wow} \colorbox{Cyan!0.000}{\strut !} \colorbox{Cyan!0.000}{\strut So} \colorbox{Cyan!0.000}{\strut  in} \colorbox{Cyan!0.000}{\strut  light} \colorbox{Cyan!0.000}{\strut  of} \colorbox{Cyan!0.000}{\strut  that} \colorbox{Cyan!0.000}{\strut  you} \colorbox{Cyan!0.000}{\strut  can} \colorbox{Cyan!0.000}{\strut  read} \colorbox{Cyan!0.000}{\strut  my} \colorbox{Cyan!0.000}{\strut  oh} \colorbox{Cyan!0.000}{\strut  so} \colorbox{Cyan!0.000}{\strut  regret} \\
Input SAE & \num{1.413e+00} & \colorbox{Green!0.000}{\strut  Wow} \colorbox{Green!0.000}{\strut !} \colorbox{Green!19.469}{\strut So} \colorbox{Green!32.965}{\strut  in} \colorbox{Green!26.496}{\strut  light} \colorbox{Green!0.000}{\strut  of} \colorbox{Green!0.000}{\strut  that} \colorbox{Green!0.000}{\strut  you} \colorbox{Green!36.252}{\strut  can} \colorbox{Green!29.545}{\strut  read} \colorbox{Green!0.000}{\strut  my} \colorbox{Green!0.000}{\strut  oh} \colorbox{Green!0.000}{\strut  so} \colorbox{Green!43.809}{\strut  regret} \\
Output SAE & \num{1.533e+00} & \colorbox{Magenta!0.000}{\strut  Wow} \colorbox{Magenta!0.000}{\strut !} \colorbox{Magenta!0.000}{\strut So} \colorbox{Magenta!0.000}{\strut  in} \colorbox{Magenta!0.000}{\strut  light} \colorbox{Magenta!0.000}{\strut  of} \colorbox{Magenta!0.000}{\strut  that} \colorbox{Magenta!0.000}{\strut  you} \colorbox{Magenta!0.000}{\strut  can} \colorbox{Magenta!0.000}{\strut  read} \colorbox{Magenta!0.000}{\strut  my} \colorbox{Magenta!0.000}{\strut  oh} \colorbox{Magenta!0.000}{\strut  so} \colorbox{Magenta!0.000}{\strut  regret} \\
\midrule
Jacobian & \num{2.048e-01} & \colorbox{Cyan!0.000}{\strut  poorly} \colorbox{Cyan!0.000}{\strut ?} \colorbox{Cyan!0.000}{\strut leave} \colorbox{Cyan!0.000}{\strut  the} \colorbox{Cyan!0.000}{\strut  red} \colorbox{Cyan!0.000}{\strut box} \colorbox{Cyan!0.000}{\strut  alone} \colorbox{Cyan!0.000}{\strut .} \colorbox{Cyan!0.000}{\strut  block} \colorbox{Cyan!0.000}{\strut b} \colorbox{Cyan!0.000}{\strut uster} \colorbox{Cyan!0.000}{\strut  will} \colorbox{Cyan!0.000}{\strut  raise} \colorbox{Cyan!0.000}{\strut  the} \\
Input SAE & \num{8.551e-01} & \colorbox{Green!21.532}{\strut  poorly} \colorbox{Green!0.000}{\strut ?} \colorbox{Green!24.232}{\strut leave} \colorbox{Green!26.504}{\strut  the} \colorbox{Green!20.306}{\strut  red} \colorbox{Green!0.000}{\strut box} \colorbox{Green!0.000}{\strut  alone} \colorbox{Green!0.000}{\strut .} \colorbox{Green!0.000}{\strut  block} \colorbox{Green!0.000}{\strut b} \colorbox{Green!0.000}{\strut uster} \colorbox{Green!0.000}{\strut  will} \colorbox{Green!0.000}{\strut  raise} \colorbox{Green!0.000}{\strut  the} \\
Output SAE & \num{1.534e+00} & \colorbox{Magenta!0.000}{\strut  poorly} \colorbox{Magenta!0.000}{\strut ?} \colorbox{Magenta!0.000}{\strut leave} \colorbox{Magenta!0.000}{\strut  the} \colorbox{Magenta!0.000}{\strut  red} \colorbox{Magenta!0.000}{\strut box} \colorbox{Magenta!0.000}{\strut  alone} \colorbox{Magenta!0.000}{\strut .} \colorbox{Magenta!0.000}{\strut  block} \colorbox{Magenta!0.000}{\strut b} \colorbox{Magenta!0.000}{\strut uster} \colorbox{Magenta!0.000}{\strut  will} \colorbox{Magenta!0.000}{\strut  raise} \colorbox{Magenta!0.000}{\strut  the} \\
\bottomrule
\end{longtable}
\caption{feature pairs/Layer15-65536-J1-LR5.0e-04-k32-T3.0e+08 abs mean/examples-29982-v-28845 stas c4-en-10k,train,batch size=32,ctx len=16.csv}
\end{table}
% \begin{table}
\centering
\begin{longtable}{lrl}
\toprule
Category & Max. abs. value & Example tokens \\
\midrule
Jacobian & \num{2.375e-01} & \colorbox{Cyan!0.000}{\strut  many} \colorbox{Cyan!0.000}{\strut  new} \colorbox{Cyan!0.000}{\strut  exciting} \colorbox{Cyan!0.000}{\strut  functions} \colorbox{Cyan!0.000}{\strut  (} \colorbox{Cyan!100.000}{\strut R} \colorbox{Cyan!0.000}{\strut BL} \colorbox{Cyan!0.000}{\strut ,} \colorbox{Cyan!95.528}{\strut  R} \colorbox{Cyan!0.000}{\strut WL} \colorbox{Cyan!0.000}{\strut ,} \colorbox{Cyan!0.000}{\strut  Automatic} \colorbox{Cyan!0.000}{\strut  white} \colorbox{Cyan!0.000}{\strut  listing} \colorbox{Cyan!0.000}{\strut )} \\
Input SAE & \num{1.339e+01} & \colorbox{Green!0.000}{\strut  many} \colorbox{Green!0.000}{\strut  new} \colorbox{Green!0.000}{\strut  exciting} \colorbox{Green!0.000}{\strut  functions} \colorbox{Green!0.000}{\strut  (} \colorbox{Green!85.940}{\strut R} \colorbox{Green!0.000}{\strut BL} \colorbox{Green!0.000}{\strut ,} \colorbox{Green!63.357}{\strut  R} \colorbox{Green!0.000}{\strut WL} \colorbox{Green!0.000}{\strut ,} \colorbox{Green!0.000}{\strut  Automatic} \colorbox{Green!0.000}{\strut  white} \colorbox{Green!0.000}{\strut  listing} \colorbox{Green!0.000}{\strut )} \\
Output SAE & \num{3.797e+00} & \colorbox{Magenta!0.000}{\strut  many} \colorbox{Magenta!0.000}{\strut  new} \colorbox{Magenta!0.000}{\strut  exciting} \colorbox{Magenta!0.000}{\strut  functions} \colorbox{Magenta!0.000}{\strut  (} \colorbox{Magenta!90.001}{\strut R} \colorbox{Magenta!0.000}{\strut BL} \colorbox{Magenta!0.000}{\strut ,} \colorbox{Magenta!67.103}{\strut  R} \colorbox{Magenta!0.000}{\strut WL} \colorbox{Magenta!0.000}{\strut ,} \colorbox{Magenta!0.000}{\strut  Automatic} \colorbox{Magenta!0.000}{\strut  white} \colorbox{Magenta!0.000}{\strut  listing} \colorbox{Magenta!0.000}{\strut )} \\
\midrule
Jacobian & \num{2.359e-01} & \colorbox{Cyan!0.000}{\strut  activity} \colorbox{Cyan!0.000}{\strut  in} \colorbox{Cyan!0.000}{\strut  the} \colorbox{Cyan!0.000}{\strut  areas} \colorbox{Cyan!0.000}{\strut  where} \colorbox{Cyan!0.000}{\strut  asbestos} \colorbox{Cyan!0.000}{\strut  was} \colorbox{Cyan!0.000}{\strut  present} \colorbox{Cyan!0.000}{\strut  [} \colorbox{Cyan!99.329}{\strut R} \colorbox{Cyan!0.000}{\strut .} \colorbox{Cyan!0.000}{\strut  v} \colorbox{Cyan!0.000}{\strut .} \colorbox{Cyan!0.000}{\strut  D} \colorbox{Cyan!0.000}{\strut ella} \\
Input SAE & \num{1.230e+01} & \colorbox{Green!0.000}{\strut  activity} \colorbox{Green!0.000}{\strut  in} \colorbox{Green!0.000}{\strut  the} \colorbox{Green!0.000}{\strut  areas} \colorbox{Green!0.000}{\strut  where} \colorbox{Green!0.000}{\strut  asbestos} \colorbox{Green!0.000}{\strut  was} \colorbox{Green!0.000}{\strut  present} \colorbox{Green!0.000}{\strut  [} \colorbox{Green!78.957}{\strut R} \colorbox{Green!0.000}{\strut .} \colorbox{Green!0.000}{\strut  v} \colorbox{Green!0.000}{\strut .} \colorbox{Green!0.000}{\strut  D} \colorbox{Green!0.000}{\strut ella} \\
Output SAE & \num{3.538e+00} & \colorbox{Magenta!0.000}{\strut  activity} \colorbox{Magenta!0.000}{\strut  in} \colorbox{Magenta!0.000}{\strut  the} \colorbox{Magenta!0.000}{\strut  areas} \colorbox{Magenta!0.000}{\strut  where} \colorbox{Magenta!0.000}{\strut  asbestos} \colorbox{Magenta!0.000}{\strut  was} \colorbox{Magenta!0.000}{\strut  present} \colorbox{Magenta!0.000}{\strut  [} \colorbox{Magenta!83.866}{\strut R} \colorbox{Magenta!16.195}{\strut .} \colorbox{Magenta!0.000}{\strut  v} \colorbox{Magenta!0.000}{\strut .} \colorbox{Magenta!0.000}{\strut  D} \colorbox{Magenta!0.000}{\strut ella} \\
\midrule
Jacobian & \num{2.357e-01} & \colorbox{Cyan!0.000}{\strut  available} \colorbox{Cyan!0.000}{\strut  to} \colorbox{Cyan!93.621}{\strut  R} \colorbox{Cyan!0.000}{\strut ICS} \colorbox{Cyan!0.000}{\strut  members} \colorbox{Cyan!0.000}{\strut  to} \colorbox{Cyan!0.000}{\strut  download} \colorbox{Cyan!0.000}{\strut  from} \colorbox{Cyan!0.000}{\strut  the} \colorbox{Cyan!99.230}{\strut  R} \colorbox{Cyan!0.000}{\strut ICS} \colorbox{Cyan!0.000}{\strut  website} \colorbox{Cyan!0.000}{\strut  and} \colorbox{Cyan!0.000}{\strut  to} \colorbox{Cyan!0.000}{\strut  non} \\
Input SAE & \num{1.207e+01} & \colorbox{Green!0.000}{\strut  available} \colorbox{Green!0.000}{\strut  to} \colorbox{Green!77.499}{\strut  R} \colorbox{Green!0.000}{\strut ICS} \colorbox{Green!0.000}{\strut  members} \colorbox{Green!0.000}{\strut  to} \colorbox{Green!0.000}{\strut  download} \colorbox{Green!0.000}{\strut  from} \colorbox{Green!0.000}{\strut  the} \colorbox{Green!57.290}{\strut  R} \colorbox{Green!0.000}{\strut ICS} \colorbox{Green!0.000}{\strut  website} \colorbox{Green!0.000}{\strut  and} \colorbox{Green!0.000}{\strut  to} \colorbox{Green!0.000}{\strut  non} \\
Output SAE & \num{3.262e+00} & \colorbox{Magenta!0.000}{\strut  available} \colorbox{Magenta!0.000}{\strut  to} \colorbox{Magenta!77.325}{\strut  R} \colorbox{Magenta!0.000}{\strut ICS} \colorbox{Magenta!0.000}{\strut  members} \colorbox{Magenta!0.000}{\strut  to} \colorbox{Magenta!0.000}{\strut  download} \colorbox{Magenta!0.000}{\strut  from} \colorbox{Magenta!0.000}{\strut  the} \colorbox{Magenta!62.994}{\strut  R} \colorbox{Magenta!0.000}{\strut ICS} \colorbox{Magenta!0.000}{\strut  website} \colorbox{Magenta!0.000}{\strut  and} \colorbox{Magenta!0.000}{\strut  to} \colorbox{Magenta!0.000}{\strut  non} \\
\midrule
Jacobian & \num{2.349e-01} & \colorbox{Cyan!0.000}{\strut  or} \colorbox{Cyan!0.000}{\strut  damage} \colorbox{Cyan!0.000}{\strut \textquotedbl{}} \colorbox{Cyan!0.000}{\strut  [} \colorbox{Cyan!98.915}{\strut R} \colorbox{Cyan!0.000}{\strut .} \colorbox{Cyan!0.000}{\strut  v} \colorbox{Cyan!0.000}{\strut .} \colorbox{Cyan!0.000}{\strut  Government} \colorbox{Cyan!0.000}{\strut  of} \colorbox{Cyan!0.000}{\strut  Yuk} \colorbox{Cyan!0.000}{\strut on} \colorbox{Cyan!0.000}{\strut ].} \colorbox{Cyan!0.000}{\strut Example} \\
Input SAE & \num{1.341e+01} & \colorbox{Green!0.000}{\strut  or} \colorbox{Green!0.000}{\strut  damage} \colorbox{Green!0.000}{\strut \textquotedbl{}} \colorbox{Green!0.000}{\strut  [} \colorbox{Green!86.052}{\strut R} \colorbox{Green!0.000}{\strut .} \colorbox{Green!0.000}{\strut  v} \colorbox{Green!0.000}{\strut .} \colorbox{Green!0.000}{\strut  Government} \colorbox{Green!0.000}{\strut  of} \colorbox{Green!0.000}{\strut  Yuk} \colorbox{Green!0.000}{\strut on} \colorbox{Green!0.000}{\strut ].} \colorbox{Green!0.000}{\strut Example} \\
Output SAE & \num{3.801e+00} & \colorbox{Magenta!0.000}{\strut  or} \colorbox{Magenta!0.000}{\strut  damage} \colorbox{Magenta!0.000}{\strut \textquotedbl{}} \colorbox{Magenta!0.000}{\strut  [} \colorbox{Magenta!90.111}{\strut R} \colorbox{Magenta!19.216}{\strut .} \colorbox{Magenta!0.000}{\strut  v} \colorbox{Magenta!0.000}{\strut .} \colorbox{Magenta!0.000}{\strut  Government} \colorbox{Magenta!0.000}{\strut  of} \colorbox{Magenta!0.000}{\strut  Yuk} \colorbox{Magenta!0.000}{\strut on} \colorbox{Magenta!0.000}{\strut ].} \colorbox{Magenta!0.000}{\strut Example} \\
\midrule
Jacobian & \num{2.349e-01} & \colorbox{Cyan!0.000}{\strut  apply} \colorbox{Cyan!0.000}{\strut :} \colorbox{Cyan!0.000}{\strut  see} \colorbox{Cyan!98.874}{\strut  R} \colorbox{Cyan!0.000}{\strut FP} \colorbox{Cyan!0.000}{\strut  or} \colorbox{Cyan!0.000}{\strut  program} \colorbox{Cyan!0.000}{\strut  page} \colorbox{Cyan!0.000}{\strut  for} \colorbox{Cyan!0.000}{\strut  full} \colorbox{Cyan!0.000}{\strut  details} \colorbox{Cyan!0.000}{\strut .} \colorbox{Cyan!0.000}{\strut  Award} \colorbox{Cyan!0.000}{\strut  of} \colorbox{Cyan!0.000}{\strut  \$} \\
Input SAE & \num{1.318e+01} & \colorbox{Green!0.000}{\strut  apply} \colorbox{Green!0.000}{\strut :} \colorbox{Green!0.000}{\strut  see} \colorbox{Green!84.574}{\strut  R} \colorbox{Green!0.000}{\strut FP} \colorbox{Green!0.000}{\strut  or} \colorbox{Green!0.000}{\strut  program} \colorbox{Green!0.000}{\strut  page} \colorbox{Green!0.000}{\strut  for} \colorbox{Green!0.000}{\strut  full} \colorbox{Green!0.000}{\strut  details} \colorbox{Green!0.000}{\strut .} \colorbox{Green!0.000}{\strut  Award} \colorbox{Green!0.000}{\strut  of} \colorbox{Green!0.000}{\strut  \$} \\
Output SAE & \num{3.760e+00} & \colorbox{Magenta!0.000}{\strut  apply} \colorbox{Magenta!0.000}{\strut :} \colorbox{Magenta!0.000}{\strut  see} \colorbox{Magenta!89.132}{\strut  R} \colorbox{Magenta!0.000}{\strut FP} \colorbox{Magenta!0.000}{\strut  or} \colorbox{Magenta!0.000}{\strut  program} \colorbox{Magenta!0.000}{\strut  page} \colorbox{Magenta!0.000}{\strut  for} \colorbox{Magenta!0.000}{\strut  full} \colorbox{Magenta!0.000}{\strut  details} \colorbox{Magenta!0.000}{\strut .} \colorbox{Magenta!0.000}{\strut  Award} \colorbox{Magenta!0.000}{\strut  of} \colorbox{Magenta!0.000}{\strut  \$} \\
\midrule
Jacobian & \num{2.347e-01} & \colorbox{Cyan!0.000}{\strut  lower} \colorbox{Cyan!0.000}{\strut  in} \colorbox{Cyan!0.000}{\strut  England} \colorbox{Cyan!0.000}{\strut  compared} \colorbox{Cyan!0.000}{\strut  with} \colorbox{Cyan!0.000}{\strut  Belgium} \colorbox{Cyan!0.000}{\strut  (} \colorbox{Cyan!98.818}{\strut R} \colorbox{Cyan!0.000}{\strut ER} \colorbox{Cyan!0.000}{\strut  2} \colorbox{Cyan!0.000}{\strut .} \colorbox{Cyan!0.000}{\strut 96} \colorbox{Cyan!0.000}{\strut ,} \colorbox{Cyan!0.000}{\strut  95} \colorbox{Cyan!0.000}{\strut \%} \\
Input SAE & \num{1.238e+01} & \colorbox{Green!0.000}{\strut  lower} \colorbox{Green!0.000}{\strut  in} \colorbox{Green!0.000}{\strut  England} \colorbox{Green!0.000}{\strut  compared} \colorbox{Green!0.000}{\strut  with} \colorbox{Green!0.000}{\strut  Belgium} \colorbox{Green!0.000}{\strut  (} \colorbox{Green!79.465}{\strut R} \colorbox{Green!0.000}{\strut ER} \colorbox{Green!0.000}{\strut  2} \colorbox{Green!0.000}{\strut .} \colorbox{Green!0.000}{\strut 96} \colorbox{Green!0.000}{\strut ,} \colorbox{Green!0.000}{\strut  95} \colorbox{Green!0.000}{\strut \%} \\
Output SAE & \num{3.600e+00} & \colorbox{Magenta!0.000}{\strut  lower} \colorbox{Magenta!0.000}{\strut  in} \colorbox{Magenta!0.000}{\strut  England} \colorbox{Magenta!0.000}{\strut  compared} \colorbox{Magenta!0.000}{\strut  with} \colorbox{Magenta!0.000}{\strut  Belgium} \colorbox{Magenta!0.000}{\strut  (} \colorbox{Magenta!85.346}{\strut R} \colorbox{Magenta!13.709}{\strut ER} \colorbox{Magenta!0.000}{\strut  2} \colorbox{Magenta!0.000}{\strut .} \colorbox{Magenta!0.000}{\strut 96} \colorbox{Magenta!0.000}{\strut ,} \colorbox{Magenta!0.000}{\strut  95} \colorbox{Magenta!0.000}{\strut \%} \\
\midrule
Jacobian & \num{2.347e-01} & \colorbox{Cyan!0.000}{\strut :} \colorbox{Cyan!0.000}{\strut  D} \colorbox{Cyan!0.000}{\strut AX} \colorbox{Cyan!0.000}{\strut  Expression} \colorbox{Cyan!0.000}{\strut  to} \colorbox{Cyan!0.000}{\strut  calculate} \colorbox{Cyan!0.000}{\strut  a} \colorbox{Cyan!98.804}{\strut  R} \colorbox{Cyan!0.000}{\strut ANK} \colorbox{Cyan!0.000}{\strut  number} \colorbox{Cyan!0.000}{\strut  List} \colorbox{Cyan!0.000}{\strut ing} \colorbox{Cyan!0.000}{\strut  3} \colorbox{Cyan!0.000}{\strut :} \colorbox{Cyan!0.000}{\strut  D} \\
Input SAE & \num{1.331e+01} & \colorbox{Green!0.000}{\strut :} \colorbox{Green!0.000}{\strut  D} \colorbox{Green!0.000}{\strut AX} \colorbox{Green!0.000}{\strut  Expression} \colorbox{Green!0.000}{\strut  to} \colorbox{Green!0.000}{\strut  calculate} \colorbox{Green!0.000}{\strut  a} \colorbox{Green!85.462}{\strut  R} \colorbox{Green!0.000}{\strut ANK} \colorbox{Green!0.000}{\strut  number} \colorbox{Green!0.000}{\strut  List} \colorbox{Green!0.000}{\strut ing} \colorbox{Green!0.000}{\strut  3} \colorbox{Green!0.000}{\strut :} \colorbox{Green!3.856}{\strut  D} \\
Output SAE & \num{3.694e+00} & \colorbox{Magenta!0.000}{\strut :} \colorbox{Magenta!0.000}{\strut  D} \colorbox{Magenta!0.000}{\strut AX} \colorbox{Magenta!0.000}{\strut  Expression} \colorbox{Magenta!0.000}{\strut  to} \colorbox{Magenta!0.000}{\strut  calculate} \colorbox{Magenta!0.000}{\strut  a} \colorbox{Magenta!87.563}{\strut  R} \colorbox{Magenta!0.000}{\strut ANK} \colorbox{Magenta!0.000}{\strut  number} \colorbox{Magenta!0.000}{\strut  List} \colorbox{Magenta!0.000}{\strut ing} \colorbox{Magenta!0.000}{\strut  3} \colorbox{Magenta!0.000}{\strut :} \colorbox{Magenta!0.000}{\strut  D} \\
\midrule
Jacobian & \num{2.346e-01} & \colorbox{Cyan!0.000}{\strut  Cisco} \colorbox{Cyan!0.000}{\strut  CC} \colorbox{Cyan!0.000}{\strut NP} \colorbox{Cyan!98.787}{\strut  R} \colorbox{Cyan!0.000}{\strut outing} \colorbox{Cyan!0.000}{\strut  and} \colorbox{Cyan!0.000}{\strut  Switch} \colorbox{Cyan!0.000}{\strut ing} \colorbox{Cyan!0.000}{\strut  certification} \colorbox{Cyan!0.000}{\strut  passed} \colorbox{Cyan!0.000}{\strut  with} \colorbox{Cyan!0.000}{\strut  guaranteed} \colorbox{Cyan!0.000}{\strut  success} \colorbox{Cyan!0.000}{\strut .} \colorbox{Cyan!0.000}{\strut  Currently} \\
Input SAE & \num{1.283e+01} & \colorbox{Green!0.000}{\strut  Cisco} \colorbox{Green!0.000}{\strut  CC} \colorbox{Green!0.000}{\strut NP} \colorbox{Green!82.383}{\strut  R} \colorbox{Green!0.000}{\strut outing} \colorbox{Green!0.000}{\strut  and} \colorbox{Green!0.000}{\strut  Switch} \colorbox{Green!0.000}{\strut ing} \colorbox{Green!0.000}{\strut  certification} \colorbox{Green!0.000}{\strut  passed} \colorbox{Green!0.000}{\strut  with} \colorbox{Green!0.000}{\strut  guaranteed} \colorbox{Green!0.000}{\strut  success} \colorbox{Green!0.000}{\strut .} \colorbox{Green!0.000}{\strut  Currently} \\
Output SAE & \num{3.771e+00} & \colorbox{Magenta!0.000}{\strut  Cisco} \colorbox{Magenta!0.000}{\strut  CC} \colorbox{Magenta!0.000}{\strut NP} \colorbox{Magenta!89.382}{\strut  R} \colorbox{Magenta!0.000}{\strut outing} \colorbox{Magenta!0.000}{\strut  and} \colorbox{Magenta!0.000}{\strut  Switch} \colorbox{Magenta!0.000}{\strut ing} \colorbox{Magenta!0.000}{\strut  certification} \colorbox{Magenta!0.000}{\strut  passed} \colorbox{Magenta!0.000}{\strut  with} \colorbox{Magenta!0.000}{\strut  guaranteed} \colorbox{Magenta!0.000}{\strut  success} \colorbox{Magenta!0.000}{\strut .} \colorbox{Magenta!0.000}{\strut  Currently} \\
\midrule
Jacobian & \num{2.346e-01} & \colorbox{Cyan!0.000}{\strut f} \colorbox{Cyan!0.000}{\strut ps} \colorbox{Cyan!0.000}{\strut .} \colorbox{Cyan!0.000}{\strut  It} \colorbox{Cyan!0.000}{\strut  was} \colorbox{Cyan!0.000}{\strut  also} \colorbox{Cyan!0.000}{\strut  31} \colorbox{Cyan!0.000}{\strut \%} \colorbox{Cyan!0.000}{\strut  faster} \colorbox{Cyan!0.000}{\strut  than} \colorbox{Cyan!0.000}{\strut  the} \colorbox{Cyan!98.750}{\strut  R} \colorbox{Cyan!0.000}{\strut adeon} \colorbox{Cyan!0.000}{\strut  HD} \colorbox{Cyan!0.000}{\strut  48} \\
Input SAE & \num{1.235e+01} & \colorbox{Green!0.000}{\strut f} \colorbox{Green!0.000}{\strut ps} \colorbox{Green!0.000}{\strut .} \colorbox{Green!0.000}{\strut  It} \colorbox{Green!0.000}{\strut  was} \colorbox{Green!0.000}{\strut  also} \colorbox{Green!0.000}{\strut  31} \colorbox{Green!0.000}{\strut \%} \colorbox{Green!0.000}{\strut  faster} \colorbox{Green!0.000}{\strut  than} \colorbox{Green!0.000}{\strut  the} \colorbox{Green!79.244}{\strut  R} \colorbox{Green!0.000}{\strut adeon} \colorbox{Green!0.000}{\strut  HD} \colorbox{Green!0.000}{\strut  48} \\
Output SAE & \num{3.449e+00} & \colorbox{Magenta!0.000}{\strut f} \colorbox{Magenta!0.000}{\strut ps} \colorbox{Magenta!0.000}{\strut .} \colorbox{Magenta!0.000}{\strut  It} \colorbox{Magenta!0.000}{\strut  was} \colorbox{Magenta!0.000}{\strut  also} \colorbox{Magenta!0.000}{\strut  31} \colorbox{Magenta!0.000}{\strut \%} \colorbox{Magenta!0.000}{\strut  faster} \colorbox{Magenta!0.000}{\strut  than} \colorbox{Magenta!0.000}{\strut  the} \colorbox{Magenta!81.745}{\strut  R} \colorbox{Magenta!0.000}{\strut adeon} \colorbox{Magenta!0.000}{\strut  HD} \colorbox{Magenta!0.000}{\strut  48} \\
\midrule
Jacobian & \num{2.344e-01} & \colorbox{Cyan!0.000}{\strut  no} \colorbox{Cyan!0.000}{\strut  vest} \colorbox{Cyan!0.000}{\strut  beneath} \colorbox{Cyan!0.000}{\strut  their} \colorbox{Cyan!0.000}{\strut  jacket} \colorbox{Cyan!0.000}{\strut \textquotedbl{}} \colorbox{Cyan!0.000}{\strut  (} \colorbox{Cyan!98.680}{\strut R} \colorbox{Cyan!0.000}{\strut ang} \colorbox{Cyan!0.000}{\strut oon} \colorbox{Cyan!0.000}{\strut  Gaz} \colorbox{Cyan!0.000}{\strut ette} \colorbox{Cyan!0.000}{\strut  Weekly} \colorbox{Cyan!0.000}{\strut  Budget} \colorbox{Cyan!0.000}{\strut  1899} \\
Input SAE & \num{1.276e+01} & \colorbox{Green!0.000}{\strut  no} \colorbox{Green!0.000}{\strut  vest} \colorbox{Green!0.000}{\strut  beneath} \colorbox{Green!0.000}{\strut  their} \colorbox{Green!0.000}{\strut  jacket} \colorbox{Green!0.000}{\strut \textquotedbl{}} \colorbox{Green!0.000}{\strut  (} \colorbox{Green!81.882}{\strut R} \colorbox{Green!0.000}{\strut ang} \colorbox{Green!0.000}{\strut oon} \colorbox{Green!0.000}{\strut  Gaz} \colorbox{Green!0.000}{\strut ette} \colorbox{Green!0.000}{\strut  Weekly} \colorbox{Green!0.000}{\strut  Budget} \colorbox{Green!0.000}{\strut  1899} \\
Output SAE & \num{3.616e+00} & \colorbox{Magenta!0.000}{\strut  no} \colorbox{Magenta!0.000}{\strut  vest} \colorbox{Magenta!0.000}{\strut  beneath} \colorbox{Magenta!0.000}{\strut  their} \colorbox{Magenta!0.000}{\strut  jacket} \colorbox{Magenta!0.000}{\strut \textquotedbl{}} \colorbox{Magenta!0.000}{\strut  (} \colorbox{Magenta!85.722}{\strut R} \colorbox{Magenta!0.000}{\strut ang} \colorbox{Magenta!0.000}{\strut oon} \colorbox{Magenta!0.000}{\strut  Gaz} \colorbox{Magenta!0.000}{\strut ette} \colorbox{Magenta!0.000}{\strut  Weekly} \colorbox{Magenta!0.000}{\strut  Budget} \colorbox{Magenta!0.000}{\strut  1899} \\
\midrule
Jacobian & \num{2.344e-01} & \colorbox{Cyan!0.000}{\strut  and} \colorbox{Cyan!0.000}{\strut  total} \colorbox{Cyan!0.000}{\strut  2} \colorbox{Cyan!0.000}{\strut .} \colorbox{Cyan!0.000}{\strut 8} \colorbox{Cyan!0.000}{\strut A} \colorbox{Cyan!0.000}{\strut  max} \colorbox{Cyan!0.000}{\strut  output} \colorbox{Cyan!0.000}{\strut  under} \colorbox{Cyan!0.000}{\strut  direct} \colorbox{Cyan!0.000}{\strut  sunlight} \colorbox{Cyan!0.000}{\strut .} \colorbox{Cyan!98.671}{\strut  R} \colorbox{Cyan!0.000}{\strut ated} \colorbox{Cyan!0.000}{\strut  4} \\
Input SAE & \num{1.411e+01} & \colorbox{Green!0.000}{\strut  and} \colorbox{Green!0.000}{\strut  total} \colorbox{Green!0.000}{\strut  2} \colorbox{Green!0.000}{\strut .} \colorbox{Green!0.000}{\strut 8} \colorbox{Green!0.000}{\strut A} \colorbox{Green!0.000}{\strut  max} \colorbox{Green!0.000}{\strut  output} \colorbox{Green!0.000}{\strut  under} \colorbox{Green!0.000}{\strut  direct} \colorbox{Green!0.000}{\strut  sunlight} \colorbox{Green!0.000}{\strut .} \colorbox{Green!90.561}{\strut  R} \colorbox{Green!0.000}{\strut ated} \colorbox{Green!0.000}{\strut  4} \\
Output SAE & \num{4.042e+00} & \colorbox{Magenta!0.000}{\strut  and} \colorbox{Magenta!0.000}{\strut  total} \colorbox{Magenta!0.000}{\strut  2} \colorbox{Magenta!0.000}{\strut .} \colorbox{Magenta!0.000}{\strut 8} \colorbox{Magenta!0.000}{\strut A} \colorbox{Magenta!0.000}{\strut  max} \colorbox{Magenta!0.000}{\strut  output} \colorbox{Magenta!0.000}{\strut  under} \colorbox{Magenta!0.000}{\strut  direct} \colorbox{Magenta!0.000}{\strut  sunlight} \colorbox{Magenta!0.000}{\strut .} \colorbox{Magenta!95.807}{\strut  R} \colorbox{Magenta!0.000}{\strut ated} \colorbox{Magenta!0.000}{\strut  4} \\
\midrule
Jacobian & \num{2.341e-01} & \colorbox{Cyan!0.000}{\strut  appointed} \colorbox{Cyan!0.000}{\strut  by} \colorbox{Cyan!0.000}{\strut  P} \colorbox{Cyan!0.000}{\strut \&} \colorbox{Cyan!0.000}{\strut I} \colorbox{Cyan!0.000}{\strut  Under} \colorbox{Cyan!0.000}{\strut writers} \colorbox{Cyan!0.000}{\strut  as} \colorbox{Cyan!0.000}{\strut  consultant} \colorbox{Cyan!0.000}{\strut  to} \colorbox{Cyan!0.000}{\strut  investigate} \colorbox{Cyan!0.000}{\strut  a} \colorbox{Cyan!0.000}{\strut  collision} \colorbox{Cyan!0.000}{\strut  (} \colorbox{Cyan!98.539}{\strut R} \\
Input SAE & \num{1.296e+01} & \colorbox{Green!0.000}{\strut  appointed} \colorbox{Green!0.000}{\strut  by} \colorbox{Green!0.000}{\strut  P} \colorbox{Green!0.000}{\strut \&} \colorbox{Green!0.000}{\strut I} \colorbox{Green!0.000}{\strut  Under} \colorbox{Green!0.000}{\strut writers} \colorbox{Green!0.000}{\strut  as} \colorbox{Green!0.000}{\strut  consultant} \colorbox{Green!0.000}{\strut  to} \colorbox{Green!0.000}{\strut  investigate} \colorbox{Green!0.000}{\strut  a} \colorbox{Green!0.000}{\strut  collision} \colorbox{Green!0.000}{\strut  (} \colorbox{Green!83.195}{\strut R} \\
Output SAE & \num{3.513e+00} & \colorbox{Magenta!0.000}{\strut  appointed} \colorbox{Magenta!0.000}{\strut  by} \colorbox{Magenta!0.000}{\strut  P} \colorbox{Magenta!0.000}{\strut \&} \colorbox{Magenta!0.000}{\strut I} \colorbox{Magenta!0.000}{\strut  Under} \colorbox{Magenta!0.000}{\strut writers} \colorbox{Magenta!0.000}{\strut  as} \colorbox{Magenta!0.000}{\strut  consultant} \colorbox{Magenta!0.000}{\strut  to} \colorbox{Magenta!0.000}{\strut  investigate} \colorbox{Magenta!0.000}{\strut  a} \colorbox{Magenta!0.000}{\strut  collision} \colorbox{Magenta!0.000}{\strut  (} \colorbox{Magenta!83.279}{\strut R} \\
\bottomrule
\end{longtable}
\caption{feature pairs/Layer15-65536-J1-LR5.0e-04-k32-T3.0e+08 abs mean/examples-30688-v-11160 stas c4-en-10k,train,batch size=32,ctx len=16.csv}
\end{table}
\begin{table}
\centering
\begin{tabular}{lrl}
\toprule
Category & Max. abs. value & Example tokens \\
\midrule
Jacobian & \num{2.331e-01} & \colorbox{Cyan!0.000}{\strut s} \colorbox{Cyan!100.000}{\strut  favourite} \colorbox{Cyan!86.116}{\strut  chefs} \colorbox{Cyan!0.000}{\strut .} \colorbox{Cyan!0.000}{\strut The} \colorbox{Cyan!0.000}{\strut  exciting} \colorbox{Cyan!0.000}{\strut  event} \colorbox{Cyan!0.000}{\strut  has} \colorbox{Cyan!0.000}{\strut  not} \colorbox{Cyan!0.000}{\strut  only} \colorbox{Cyan!0.000}{\strut  been} \colorbox{Cyan!0.000}{\strut  about} \colorbox{Cyan!0.000}{\strut  fun} \colorbox{Cyan!0.000}{\strut  and} \\
Input SAE & \num{5.784e+00} & \colorbox{Green!0.000}{\strut s} \colorbox{Green!8.663}{\strut  favourite} \colorbox{Green!46.727}{\strut  chefs} \colorbox{Green!0.000}{\strut .} \colorbox{Green!0.000}{\strut The} \colorbox{Green!0.000}{\strut  exciting} \colorbox{Green!0.000}{\strut  event} \colorbox{Green!0.000}{\strut  has} \colorbox{Green!0.000}{\strut  not} \colorbox{Green!0.000}{\strut  only} \colorbox{Green!0.000}{\strut  been} \colorbox{Green!0.000}{\strut  about} \colorbox{Green!0.000}{\strut  fun} \colorbox{Green!0.000}{\strut  and} \\
Output SAE & \num{2.349e+00} & \colorbox{Magenta!0.000}{\strut s} \colorbox{Magenta!23.597}{\strut  favourite} \colorbox{Magenta!57.990}{\strut  chefs} \colorbox{Magenta!0.000}{\strut .} \colorbox{Magenta!0.000}{\strut The} \colorbox{Magenta!0.000}{\strut  exciting} \colorbox{Magenta!0.000}{\strut  event} \colorbox{Magenta!0.000}{\strut  has} \colorbox{Magenta!0.000}{\strut  not} \colorbox{Magenta!0.000}{\strut  only} \colorbox{Magenta!0.000}{\strut  been} \colorbox{Magenta!0.000}{\strut  about} \colorbox{Magenta!0.000}{\strut  fun} \colorbox{Magenta!0.000}{\strut  and} \\
\midrule
Jacobian & \num{2.313e-01} & \colorbox{Cyan!0.000}{\strut cup} \colorbox{Cyan!0.000}{\strut  dec} \colorbox{Cyan!0.000}{\strut an} \colorbox{Cyan!0.000}{\strut ter} \colorbox{Cyan!0.000}{\strut  brew} \colorbox{Cyan!99.235}{\strut ers} \colorbox{Cyan!0.000}{\strut .} \colorbox{Cyan!0.000}{\strut  Special} \colorbox{Cyan!0.000}{\strut  paper} \colorbox{Cyan!0.000}{\strut  grade} \colorbox{Cyan!0.000}{\strut  ass} \colorbox{Cyan!0.000}{\strut ures} \colorbox{Cyan!0.000}{\strut  optimum} \colorbox{Cyan!0.000}{\strut  extraction} \colorbox{Cyan!0.000}{\strut  of} \\
Input SAE & \num{1.220e+00} & \colorbox{Green!0.000}{\strut cup} \colorbox{Green!0.000}{\strut  dec} \colorbox{Green!0.000}{\strut an} \colorbox{Green!0.000}{\strut ter} \colorbox{Green!0.000}{\strut  brew} \colorbox{Green!9.853}{\strut ers} \colorbox{Green!0.000}{\strut .} \colorbox{Green!0.000}{\strut  Special} \colorbox{Green!0.000}{\strut  paper} \colorbox{Green!0.000}{\strut  grade} \colorbox{Green!0.000}{\strut  ass} \colorbox{Green!0.000}{\strut ures} \colorbox{Green!0.000}{\strut  optimum} \colorbox{Green!0.000}{\strut  extraction} \colorbox{Green!0.000}{\strut  of} \\
Output SAE & \num{1.417e+00} & \colorbox{Magenta!13.990}{\strut cup} \colorbox{Magenta!0.000}{\strut  dec} \colorbox{Magenta!16.567}{\strut an} \colorbox{Magenta!28.637}{\strut ter} \colorbox{Magenta!34.988}{\strut  brew} \colorbox{Magenta!22.382}{\strut ers} \colorbox{Magenta!0.000}{\strut .} \colorbox{Magenta!0.000}{\strut  Special} \colorbox{Magenta!0.000}{\strut  paper} \colorbox{Magenta!0.000}{\strut  grade} \colorbox{Magenta!0.000}{\strut  ass} \colorbox{Magenta!0.000}{\strut ures} \colorbox{Magenta!0.000}{\strut  optimum} \colorbox{Magenta!0.000}{\strut  extraction} \colorbox{Magenta!0.000}{\strut  of} \\
\midrule
Jacobian & \num{2.302e-01} & \colorbox{Cyan!0.000}{\strut  working} \colorbox{Cyan!0.000}{\strut  one} \colorbox{Cyan!0.000}{\strut -} \colorbox{Cyan!0.000}{\strut on} \colorbox{Cyan!0.000}{\strut -} \colorbox{Cyan!0.000}{\strut one} \colorbox{Cyan!0.000}{\strut  with} \colorbox{Cyan!0.000}{\strut  the} \colorbox{Cyan!0.000}{\strut  pur} \colorbox{Cyan!0.000}{\strut vey} \colorbox{Cyan!98.776}{\strut ors} \colorbox{Cyan!0.000}{\strut  of} \colorbox{Cyan!87.659}{\strut  food} \colorbox{Cyan!0.000}{\strut .} \\
Input SAE & \num{1.184e+00} & \colorbox{Green!0.000}{\strut  working} \colorbox{Green!0.000}{\strut  one} \colorbox{Green!0.000}{\strut -} \colorbox{Green!0.000}{\strut on} \colorbox{Green!0.000}{\strut -} \colorbox{Green!0.000}{\strut one} \colorbox{Green!0.000}{\strut  with} \colorbox{Green!0.000}{\strut  the} \colorbox{Green!0.000}{\strut  pur} \colorbox{Green!0.000}{\strut vey} \colorbox{Green!8.928}{\strut ors} \colorbox{Green!0.000}{\strut  of} \colorbox{Green!9.563}{\strut  food} \colorbox{Green!0.000}{\strut .} \\
Output SAE & \num{3.366e+00} & \colorbox{Magenta!0.000}{\strut  working} \colorbox{Magenta!0.000}{\strut  one} \colorbox{Magenta!0.000}{\strut -} \colorbox{Magenta!0.000}{\strut on} \colorbox{Magenta!0.000}{\strut -} \colorbox{Magenta!0.000}{\strut one} \colorbox{Magenta!0.000}{\strut  with} \colorbox{Magenta!0.000}{\strut  the} \colorbox{Magenta!0.000}{\strut  pur} \colorbox{Magenta!0.000}{\strut vey} \colorbox{Magenta!21.373}{\strut ors} \colorbox{Magenta!17.737}{\strut  of} \colorbox{Magenta!83.099}{\strut  food} \colorbox{Magenta!0.000}{\strut .} \\
\midrule
Jacobian & \num{2.292e-01} & \colorbox{Cyan!0.000}{\strut ns} \colorbox{Cyan!0.000}{\strut  cater} \colorbox{Cyan!98.341}{\strut ing} \colorbox{Cyan!0.000}{\strut  specifically} \colorbox{Cyan!0.000}{\strut  to} \colorbox{Cyan!0.000}{\strut  the} \colorbox{Cyan!0.000}{\strut  Japanese} \colorbox{Cyan!0.000}{\strut  tourist} \colorbox{Cyan!0.000}{\strut .} \colorbox{Cyan!0.000}{\strut  For} \colorbox{Cyan!0.000}{\strut  many} \colorbox{Cyan!0.000}{\strut  Japanese} \colorbox{Cyan!0.000}{\strut  visitors} \colorbox{Cyan!0.000}{\strut  Bro} \colorbox{Cyan!0.000}{\strut ome} \\
Input SAE & \num{1.161e+00} & \colorbox{Green!0.000}{\strut ns} \colorbox{Green!0.000}{\strut  cater} \colorbox{Green!9.381}{\strut ing} \colorbox{Green!0.000}{\strut  specifically} \colorbox{Green!0.000}{\strut  to} \colorbox{Green!0.000}{\strut  the} \colorbox{Green!0.000}{\strut  Japanese} \colorbox{Green!0.000}{\strut  tourist} \colorbox{Green!0.000}{\strut .} \colorbox{Green!0.000}{\strut  For} \colorbox{Green!0.000}{\strut  many} \colorbox{Green!0.000}{\strut  Japanese} \colorbox{Green!0.000}{\strut  visitors} \colorbox{Green!0.000}{\strut  Bro} \colorbox{Green!0.000}{\strut ome} \\
Output SAE & \num{2.027e+00} & \colorbox{Magenta!0.000}{\strut ns} \colorbox{Magenta!50.038}{\strut  cater} \colorbox{Magenta!28.665}{\strut ing} \colorbox{Magenta!0.000}{\strut  specifically} \colorbox{Magenta!0.000}{\strut  to} \colorbox{Magenta!0.000}{\strut  the} \colorbox{Magenta!0.000}{\strut  Japanese} \colorbox{Magenta!0.000}{\strut  tourist} \colorbox{Magenta!0.000}{\strut .} \colorbox{Magenta!0.000}{\strut  For} \colorbox{Magenta!0.000}{\strut  many} \colorbox{Magenta!0.000}{\strut  Japanese} \colorbox{Magenta!0.000}{\strut  visitors} \colorbox{Magenta!0.000}{\strut  Bro} \colorbox{Magenta!0.000}{\strut ome} \\
\midrule
Jacobian & \num{2.275e-01} & \colorbox{Cyan!0.000}{\strut \textquotesingle{}} \colorbox{Cyan!0.000}{\strut s} \colorbox{Cyan!0.000}{\strut  re} \colorbox{Cyan!0.000}{\strut pert} \colorbox{Cyan!97.615}{\strut ory} \colorbox{Cyan!0.000}{\strut  production} \colorbox{Cyan!0.000}{\strut  Creative} \colorbox{Cyan!0.000}{\strut  Con} \colorbox{Cyan!0.000}{\strut vergence} \colorbox{Cyan!0.000}{\strut .} \colorbox{Cyan!0.000}{\strut  In} \colorbox{Cyan!0.000}{\strut  March} \colorbox{Cyan!0.000}{\strut ,} \colorbox{Cyan!0.000}{\strut  Peter} \colorbox{Cyan!0.000}{\strut  Pan} \\
Input SAE & \num{1.527e+00} & \colorbox{Green!0.000}{\strut \textquotesingle{}} \colorbox{Green!0.000}{\strut s} \colorbox{Green!0.000}{\strut  re} \colorbox{Green!0.000}{\strut pert} \colorbox{Green!12.339}{\strut ory} \colorbox{Green!0.000}{\strut  production} \colorbox{Green!0.000}{\strut  Creative} \colorbox{Green!0.000}{\strut  Con} \colorbox{Green!0.000}{\strut vergence} \colorbox{Green!0.000}{\strut .} \colorbox{Green!0.000}{\strut  In} \colorbox{Green!0.000}{\strut  March} \colorbox{Green!0.000}{\strut ,} \colorbox{Green!0.000}{\strut  Peter} \colorbox{Green!0.000}{\strut  Pan} \\
Output SAE & \num{8.890e-01} & \colorbox{Magenta!0.000}{\strut \textquotesingle{}} \colorbox{Magenta!0.000}{\strut s} \colorbox{Magenta!0.000}{\strut  re} \colorbox{Magenta!0.000}{\strut pert} \colorbox{Magenta!21.946}{\strut ory} \colorbox{Magenta!0.000}{\strut  production} \colorbox{Magenta!0.000}{\strut  Creative} \colorbox{Magenta!0.000}{\strut  Con} \colorbox{Magenta!0.000}{\strut vergence} \colorbox{Magenta!0.000}{\strut .} \colorbox{Magenta!0.000}{\strut  In} \colorbox{Magenta!0.000}{\strut  March} \colorbox{Magenta!0.000}{\strut ,} \colorbox{Magenta!0.000}{\strut  Peter} \colorbox{Magenta!0.000}{\strut  Pan} \\
\midrule
Jacobian & \num{2.275e-01} & \colorbox{Cyan!0.000}{\strut  Pe} \colorbox{Cyan!0.000}{\strut aks} \colorbox{Cyan!87.752}{\strut  Brewing} \colorbox{Cyan!94.407}{\strut  Company} \colorbox{Cyan!0.000}{\strut ,} \colorbox{Cyan!0.000}{\strut  beer} \colorbox{Cyan!97.589}{\strut  offerings} \colorbox{Cyan!0.000}{\strut  and} \colorbox{Cyan!0.000}{\strut  specialty} \colorbox{Cyan!84.221}{\strut  menu} \colorbox{Cyan!89.294}{\strut  items} \colorbox{Cyan!0.000}{\strut .} \colorbox{Cyan!0.000}{\strut  PU} \colorbox{Cyan!0.000}{\strut B} \colorbox{Cyan!0.000}{\strut  365} \\
Input SAE & \num{5.791e+00} & \colorbox{Green!0.000}{\strut  Pe} \colorbox{Green!0.000}{\strut aks} \colorbox{Green!17.207}{\strut  Brewing} \colorbox{Green!12.653}{\strut  Company} \colorbox{Green!0.000}{\strut ,} \colorbox{Green!0.000}{\strut  beer} \colorbox{Green!10.557}{\strut  offerings} \colorbox{Green!0.000}{\strut  and} \colorbox{Green!0.000}{\strut  specialty} \colorbox{Green!46.783}{\strut  menu} \colorbox{Green!14.816}{\strut  items} \colorbox{Green!0.000}{\strut .} \colorbox{Green!0.000}{\strut  PU} \colorbox{Green!0.000}{\strut B} \colorbox{Green!0.000}{\strut  365} \\
Output SAE & \num{2.536e+00} & \colorbox{Magenta!0.000}{\strut  Pe} \colorbox{Magenta!0.000}{\strut aks} \colorbox{Magenta!39.978}{\strut  Brewing} \colorbox{Magenta!21.935}{\strut  Company} \colorbox{Magenta!0.000}{\strut ,} \colorbox{Magenta!48.339}{\strut  beer} \colorbox{Magenta!27.711}{\strut  offerings} \colorbox{Magenta!0.000}{\strut  and} \colorbox{Magenta!28.268}{\strut  specialty} \colorbox{Magenta!62.599}{\strut  menu} \colorbox{Magenta!34.464}{\strut  items} \colorbox{Magenta!0.000}{\strut .} \colorbox{Magenta!0.000}{\strut  PU} \colorbox{Magenta!18.889}{\strut B} \colorbox{Magenta!0.000}{\strut  365} \\
\midrule
Jacobian & \num{2.273e-01} & \colorbox{Cyan!0.000}{\strut um} \colorbox{Cyan!0.000}{\strut min} \colorbox{Cyan!0.000}{\strut ess} \colorbox{Cyan!0.000}{\strut  from} \colorbox{Cyan!0.000}{\strut  local} \colorbox{Cyan!0.000}{\strut  ch} \colorbox{Cyan!0.000}{\strut ocol} \colorbox{Cyan!0.000}{\strut at} \colorbox{Cyan!97.510}{\strut ier} \colorbox{Cyan!0.000}{\strut  Eclipse} \colorbox{Cyan!0.000}{\strut  Chocolate} \colorbox{Cyan!0.000}{\strut !} \colorbox{Cyan!0.000}{\strut  Chocolate} \colorbox{Cyan!0.000}{\strut  Un} \colorbox{Cyan!0.000}{\strut w} \\
Input SAE & \num{1.811e+00} & \colorbox{Green!0.000}{\strut um} \colorbox{Green!0.000}{\strut min} \colorbox{Green!0.000}{\strut ess} \colorbox{Green!0.000}{\strut  from} \colorbox{Green!0.000}{\strut  local} \colorbox{Green!0.000}{\strut  ch} \colorbox{Green!0.000}{\strut ocol} \colorbox{Green!0.000}{\strut at} \colorbox{Green!14.633}{\strut ier} \colorbox{Green!0.000}{\strut  Eclipse} \colorbox{Green!0.000}{\strut  Chocolate} \colorbox{Green!0.000}{\strut !} \colorbox{Green!0.000}{\strut  Chocolate} \colorbox{Green!0.000}{\strut  Un} \colorbox{Green!0.000}{\strut w} \\
Output SAE & \num{1.769e+00} & \colorbox{Magenta!15.071}{\strut um} \colorbox{Magenta!0.000}{\strut min} \colorbox{Magenta!0.000}{\strut ess} \colorbox{Magenta!13.749}{\strut  from} \colorbox{Magenta!0.000}{\strut  local} \colorbox{Magenta!14.900}{\strut  ch} \colorbox{Magenta!39.259}{\strut ocol} \colorbox{Magenta!36.753}{\strut at} \colorbox{Magenta!27.682}{\strut ier} \colorbox{Magenta!0.000}{\strut  Eclipse} \colorbox{Magenta!41.774}{\strut  Chocolate} \colorbox{Magenta!0.000}{\strut !} \colorbox{Magenta!43.665}{\strut  Chocolate} \colorbox{Magenta!0.000}{\strut  Un} \colorbox{Magenta!0.000}{\strut w} \\
\midrule
Jacobian & \num{2.272e-01} & \colorbox{Cyan!88.010}{\strut  food} \colorbox{Cyan!97.468}{\strut  establishment} \colorbox{Cyan!0.000}{\strut  to} \colorbox{Cyan!0.000}{\strut  have} \colorbox{Cyan!0.000}{\strut  a} \colorbox{Cyan!0.000}{\strut  Cert} \colorbox{Cyan!0.000}{\strut ified} \colorbox{Cyan!87.705}{\strut  Food} \colorbox{Cyan!0.000}{\strut  Protection} \colorbox{Cyan!0.000}{\strut  Manager} \colorbox{Cyan!0.000}{\strut  (} \colorbox{Cyan!0.000}{\strut C} \colorbox{Cyan!0.000}{\strut FP} \colorbox{Cyan!0.000}{\strut M} \colorbox{Cyan!0.000}{\strut )} \\
Input SAE & \num{2.957e+00} & \colorbox{Green!16.923}{\strut  food} \colorbox{Green!23.893}{\strut  establishment} \colorbox{Green!0.000}{\strut  to} \colorbox{Green!0.000}{\strut  have} \colorbox{Green!0.000}{\strut  a} \colorbox{Green!0.000}{\strut  Cert} \colorbox{Green!0.000}{\strut ified} \colorbox{Green!11.725}{\strut  Food} \colorbox{Green!0.000}{\strut  Protection} \colorbox{Green!0.000}{\strut  Manager} \colorbox{Green!0.000}{\strut  (} \colorbox{Green!0.000}{\strut C} \colorbox{Green!0.000}{\strut FP} \colorbox{Green!0.000}{\strut M} \colorbox{Green!0.000}{\strut )} \\
Output SAE & \num{3.219e+00} & \colorbox{Magenta!79.463}{\strut  food} \colorbox{Magenta!25.914}{\strut  establishment} \colorbox{Magenta!0.000}{\strut  to} \colorbox{Magenta!0.000}{\strut  have} \colorbox{Magenta!0.000}{\strut  a} \colorbox{Magenta!0.000}{\strut  Cert} \colorbox{Magenta!0.000}{\strut ified} \colorbox{Magenta!70.435}{\strut  Food} \colorbox{Magenta!0.000}{\strut  Protection} \colorbox{Magenta!0.000}{\strut  Manager} \colorbox{Magenta!0.000}{\strut  (} \colorbox{Magenta!0.000}{\strut C} \colorbox{Magenta!0.000}{\strut FP} \colorbox{Magenta!0.000}{\strut M} \colorbox{Magenta!0.000}{\strut )} \\
\midrule
Jacobian & \num{2.271e-01} & \colorbox{Cyan!0.000}{\strut  and} \colorbox{Cyan!0.000}{\strut  Steve} \colorbox{Cyan!0.000}{\strut  from} \colorbox{Cyan!0.000}{\strut  Si} \colorbox{Cyan!0.000}{\strut ren} \colorbox{Cyan!0.000}{\strut  Craft} \colorbox{Cyan!0.000}{\strut  Brew} \colorbox{Cyan!97.419}{\strut s} \colorbox{Cyan!0.000}{\strut  Bar} \colorbox{Cyan!0.000}{\strut rel} \colorbox{Cyan!0.000}{\strut  Project} \colorbox{Cyan!0.000}{\strut  have} \colorbox{Cyan!0.000}{\strut  team} \colorbox{Cyan!0.000}{\strut ed} \colorbox{Cyan!0.000}{\strut  up} \\
Input SAE & \num{1.034e+00} & \colorbox{Green!0.000}{\strut  and} \colorbox{Green!0.000}{\strut  Steve} \colorbox{Green!0.000}{\strut  from} \colorbox{Green!0.000}{\strut  Si} \colorbox{Green!0.000}{\strut ren} \colorbox{Green!0.000}{\strut  Craft} \colorbox{Green!0.000}{\strut  Brew} \colorbox{Green!8.354}{\strut s} \colorbox{Green!0.000}{\strut  Bar} \colorbox{Green!0.000}{\strut rel} \colorbox{Green!0.000}{\strut  Project} \colorbox{Green!0.000}{\strut  have} \colorbox{Green!0.000}{\strut  team} \colorbox{Green!0.000}{\strut ed} \colorbox{Green!0.000}{\strut  up} \\
Output SAE & \num{1.831e+00} & \colorbox{Magenta!0.000}{\strut  and} \colorbox{Magenta!0.000}{\strut  Steve} \colorbox{Magenta!0.000}{\strut  from} \colorbox{Magenta!0.000}{\strut  Si} \colorbox{Magenta!0.000}{\strut ren} \colorbox{Magenta!0.000}{\strut  Craft} \colorbox{Magenta!45.197}{\strut  Brew} \colorbox{Magenta!23.482}{\strut s} \colorbox{Magenta!36.071}{\strut  Bar} \colorbox{Magenta!0.000}{\strut rel} \colorbox{Magenta!0.000}{\strut  Project} \colorbox{Magenta!0.000}{\strut  have} \colorbox{Magenta!0.000}{\strut  team} \colorbox{Magenta!0.000}{\strut ed} \colorbox{Magenta!0.000}{\strut  up} \\
\midrule
Jacobian & \num{2.270e-01} & \colorbox{Cyan!0.000}{\strut  first} \colorbox{Cyan!0.000}{\strut  event} \colorbox{Cyan!0.000}{\strut  came} \colorbox{Cyan!0.000}{\strut  up} \colorbox{Cyan!0.000}{\strut  unexpectedly} \colorbox{Cyan!0.000}{\strut ,} \colorbox{Cyan!0.000}{\strut  the} \colorbox{Cyan!0.000}{\strut  Bridge} \colorbox{Cyan!0.000}{\strut  and} \colorbox{Cyan!0.000}{\strut  T} \colorbox{Cyan!0.000}{\strut unnel} \colorbox{Cyan!0.000}{\strut  Brew} \colorbox{Cyan!97.383}{\strut ery} \colorbox{Cyan!0.000}{\strut  was} \colorbox{Cyan!0.000}{\strut  just} \\
Input SAE & \num{1.922e+00} & \colorbox{Green!0.000}{\strut  first} \colorbox{Green!0.000}{\strut  event} \colorbox{Green!0.000}{\strut  came} \colorbox{Green!0.000}{\strut  up} \colorbox{Green!0.000}{\strut  unexpectedly} \colorbox{Green!0.000}{\strut ,} \colorbox{Green!0.000}{\strut  the} \colorbox{Green!0.000}{\strut  Bridge} \colorbox{Green!0.000}{\strut  and} \colorbox{Green!0.000}{\strut  T} \colorbox{Green!0.000}{\strut unnel} \colorbox{Green!0.000}{\strut  Brew} \colorbox{Green!15.526}{\strut ery} \colorbox{Green!0.000}{\strut  was} \colorbox{Green!0.000}{\strut  just} \\
Output SAE & \num{1.807e+00} & \colorbox{Magenta!0.000}{\strut  first} \colorbox{Magenta!0.000}{\strut  event} \colorbox{Magenta!0.000}{\strut  came} \colorbox{Magenta!0.000}{\strut  up} \colorbox{Magenta!0.000}{\strut  unexpectedly} \colorbox{Magenta!0.000}{\strut ,} \colorbox{Magenta!0.000}{\strut  the} \colorbox{Magenta!0.000}{\strut  Bridge} \colorbox{Magenta!0.000}{\strut  and} \colorbox{Magenta!0.000}{\strut  T} \colorbox{Magenta!0.000}{\strut unnel} \colorbox{Magenta!44.601}{\strut  Brew} \colorbox{Magenta!26.002}{\strut ery} \colorbox{Magenta!0.000}{\strut  was} \colorbox{Magenta!0.000}{\strut  just} \\
\midrule
Jacobian & \num{2.269e-01} & \colorbox{Cyan!0.000}{\strut  in} \colorbox{Cyan!0.000}{\strut  other} \colorbox{Cyan!0.000}{\strut  colors} \colorbox{Cyan!0.000}{\strut ,} \colorbox{Cyan!97.355}{\strut  ordering} \colorbox{Cyan!0.000}{\strut  in} \colorbox{Cyan!0.000}{\strut  colors} \colorbox{Cyan!0.000}{\strut  other} \colorbox{Cyan!0.000}{\strut  than} \colorbox{Cyan!0.000}{\strut  fl} \colorbox{Cyan!0.000}{\strut orescent} \colorbox{Cyan!0.000}{\strut  green} \colorbox{Cyan!0.000}{\strut  or} \colorbox{Cyan!0.000}{\strut  orange} \colorbox{Cyan!0.000}{\strut  WILL} \\
Input SAE & \num{1.327e+00} & \colorbox{Green!0.000}{\strut  in} \colorbox{Green!0.000}{\strut  other} \colorbox{Green!0.000}{\strut  colors} \colorbox{Green!0.000}{\strut ,} \colorbox{Green!10.717}{\strut  ordering} \colorbox{Green!0.000}{\strut  in} \colorbox{Green!0.000}{\strut  colors} \colorbox{Green!0.000}{\strut  other} \colorbox{Green!0.000}{\strut  than} \colorbox{Green!0.000}{\strut  fl} \colorbox{Green!0.000}{\strut orescent} \colorbox{Green!0.000}{\strut  green} \colorbox{Green!0.000}{\strut  or} \colorbox{Green!0.000}{\strut  orange} \colorbox{Green!0.000}{\strut  WILL} \\
Output SAE & \num{1.055e+00} & \colorbox{Magenta!0.000}{\strut  in} \colorbox{Magenta!0.000}{\strut  other} \colorbox{Magenta!0.000}{\strut  colors} \colorbox{Magenta!0.000}{\strut ,} \colorbox{Magenta!26.038}{\strut  ordering} \colorbox{Magenta!0.000}{\strut  in} \colorbox{Magenta!0.000}{\strut  colors} \colorbox{Magenta!0.000}{\strut  other} \colorbox{Magenta!0.000}{\strut  than} \colorbox{Magenta!0.000}{\strut  fl} \colorbox{Magenta!0.000}{\strut orescent} \colorbox{Magenta!0.000}{\strut  green} \colorbox{Magenta!0.000}{\strut  or} \colorbox{Magenta!14.569}{\strut  orange} \colorbox{Magenta!0.000}{\strut  WILL} \\
\midrule
Jacobian & \num{2.269e-01} & \colorbox{Cyan!88.387}{\strut  eating} \colorbox{Cyan!97.337}{\strut  out} \colorbox{Cyan!0.000}{\strut  or} \colorbox{Cyan!0.000}{\strut  not} \colorbox{Cyan!0.000}{\strut  buying} \colorbox{Cyan!0.000}{\strut  a} \colorbox{Cyan!0.000}{\strut  new} \colorbox{Cyan!0.000}{\strut  BMW} \colorbox{Cyan!0.000}{\strut  creates} \colorbox{Cyan!0.000}{\strut  job} \colorbox{Cyan!0.000}{\strut  cuts} \colorbox{Cyan!0.000}{\strut  but} \colorbox{Cyan!0.000}{\strut  not} \colorbox{Cyan!0.000}{\strut  def} \colorbox{Cyan!0.000}{\strut lation} \\
Input SAE & \num{2.020e+00} & \colorbox{Green!10.715}{\strut  eating} \colorbox{Green!16.324}{\strut  out} \colorbox{Green!0.000}{\strut  or} \colorbox{Green!0.000}{\strut  not} \colorbox{Green!0.000}{\strut  buying} \colorbox{Green!0.000}{\strut  a} \colorbox{Green!0.000}{\strut  new} \colorbox{Green!0.000}{\strut  BMW} \colorbox{Green!0.000}{\strut  creates} \colorbox{Green!0.000}{\strut  job} \colorbox{Green!0.000}{\strut  cuts} \colorbox{Green!0.000}{\strut  but} \colorbox{Green!0.000}{\strut  not} \colorbox{Green!0.000}{\strut  def} \colorbox{Green!0.000}{\strut lation} \\
Output SAE & \num{2.951e+00} & \colorbox{Magenta!72.843}{\strut  eating} \colorbox{Magenta!24.310}{\strut  out} \colorbox{Magenta!0.000}{\strut  or} \colorbox{Magenta!0.000}{\strut  not} \colorbox{Magenta!13.457}{\strut  buying} \colorbox{Magenta!0.000}{\strut  a} \colorbox{Magenta!0.000}{\strut  new} \colorbox{Magenta!0.000}{\strut  BMW} \colorbox{Magenta!0.000}{\strut  creates} \colorbox{Magenta!0.000}{\strut  job} \colorbox{Magenta!0.000}{\strut  cuts} \colorbox{Magenta!0.000}{\strut  but} \colorbox{Magenta!0.000}{\strut  not} \colorbox{Magenta!0.000}{\strut  def} \colorbox{Magenta!0.000}{\strut lation} \\
\bottomrule
\end{tabular}
% feature pairs/Layer15-65536-J1-LR5.0e-04-k32-T3.0e+08 abs mean/examples-54846-v-30912 stas c4-en-10k,train,batch size=32,ctx len=16.csv
\caption{
The top $12$ examples that produce the maximum absolute values of the Jacobian element with input SAE latent index $54846$ and output latent index $30912$.
% This pair of latent indices is one of the top $5$ pairs by the mean absolute value of non-zero Jacobian elements.
The Jacobian SAE pair was trained on layer 15 of Pythia-410m with an expansion factor of $R=64$ and sparsity $k=32$.
The examples were collected over the first 10K records of the English subset of the C4 text dataset \citep{raffel_exploring_2020}, with a context length of $16$ tokens.
For each example, the first row shows the values of the Jacobian element, and the second and third show the corresponding activations of the input and output SAE latents.
In this case, both SAE latents appear to activate for tokens or contexts relating to food service.
}
\label{tab:feature_pairs_54846_30912}
\end{table} % food-related
% \begin{table}
\centering
\begin{longtable}{lrl}
\toprule
Category & Max. abs. value & Example tokens \\
\midrule
Jacobian & \num{2.063e-01} & \colorbox{Cyan!0.000}{\strut  by} \colorbox{Cyan!0.000}{\strut  opt} \colorbox{Cyan!0.000}{\strut ing} \colorbox{Cyan!0.000}{\strut  for} \colorbox{Cyan!0.000}{\strut  more} \colorbox{Cyan!0.000}{\strut  fruits} \colorbox{Cyan!0.000}{\strut ,} \colorbox{Cyan!0.000}{\strut  vegetables} \colorbox{Cyan!100.000}{\strut ,} \colorbox{Cyan!0.000}{\strut  lean} \colorbox{Cyan!0.000}{\strut  meats} \colorbox{Cyan!0.000}{\strut  and} \colorbox{Cyan!0.000}{\strut  whole} \colorbox{Cyan!0.000}{\strut  grains} \colorbox{Cyan!0.000}{\strut  and} \\
Input SAE & \num{1.230e+00} & \colorbox{Green!0.000}{\strut  by} \colorbox{Green!0.000}{\strut  opt} \colorbox{Green!0.000}{\strut ing} \colorbox{Green!0.000}{\strut  for} \colorbox{Green!0.000}{\strut  more} \colorbox{Green!0.000}{\strut  fruits} \colorbox{Green!0.000}{\strut ,} \colorbox{Green!0.000}{\strut  vegetables} \colorbox{Green!100.000}{\strut ,} \colorbox{Green!0.000}{\strut  lean} \colorbox{Green!0.000}{\strut  meats} \colorbox{Green!0.000}{\strut  and} \colorbox{Green!0.000}{\strut  whole} \colorbox{Green!0.000}{\strut  grains} \colorbox{Green!0.000}{\strut  and} \\
Output SAE & \num{6.758e-01} & \colorbox{Magenta!0.000}{\strut  by} \colorbox{Magenta!0.000}{\strut  opt} \colorbox{Magenta!0.000}{\strut ing} \colorbox{Magenta!0.000}{\strut  for} \colorbox{Magenta!0.000}{\strut  more} \colorbox{Magenta!0.000}{\strut  fruits} \colorbox{Magenta!0.000}{\strut ,} \colorbox{Magenta!0.000}{\strut  vegetables} \colorbox{Magenta!87.116}{\strut ,} \colorbox{Magenta!0.000}{\strut  lean} \colorbox{Magenta!0.000}{\strut  meats} \colorbox{Magenta!100.000}{\strut  and} \colorbox{Magenta!0.000}{\strut  whole} \colorbox{Magenta!0.000}{\strut  grains} \colorbox{Magenta!0.000}{\strut  and} \\
\midrule
Jacobian & \num{1.944e-01} & \colorbox{Cyan!0.000}{\strut  tablets} \colorbox{Cyan!0.000}{\strut ,} \colorbox{Cyan!0.000}{\strut  laptops} \colorbox{Cyan!94.200}{\strut ,} \colorbox{Cyan!0.000}{\strut  and} \colorbox{Cyan!0.000}{\strut  PCs} \colorbox{Cyan!0.000}{\strut  via} \colorbox{Cyan!0.000}{\strut  free} \colorbox{Cyan!0.000}{\strut  apps} \colorbox{Cyan!0.000}{\strut .} \colorbox{Cyan!0.000}{\strut Wire} \colorbox{Cyan!0.000}{\strut less} \colorbox{Cyan!0.000}{\strut  printing} \colorbox{Cyan!0.000}{\strut  is} \\
Input SAE & \num{8.849e-01} & \colorbox{Green!0.000}{\strut  tablets} \colorbox{Green!0.000}{\strut ,} \colorbox{Green!0.000}{\strut  laptops} \colorbox{Green!71.930}{\strut ,} \colorbox{Green!0.000}{\strut  and} \colorbox{Green!0.000}{\strut  PCs} \colorbox{Green!0.000}{\strut  via} \colorbox{Green!0.000}{\strut  free} \colorbox{Green!0.000}{\strut  apps} \colorbox{Green!0.000}{\strut .} \colorbox{Green!0.000}{\strut Wire} \colorbox{Green!0.000}{\strut less} \colorbox{Green!0.000}{\strut  printing} \colorbox{Green!0.000}{\strut  is} \\
Output SAE & \num{5.771e-01} & \colorbox{Magenta!0.000}{\strut  tablets} \colorbox{Magenta!0.000}{\strut ,} \colorbox{Magenta!0.000}{\strut  laptops} \colorbox{Magenta!85.390}{\strut ,} \colorbox{Magenta!0.000}{\strut  and} \colorbox{Magenta!0.000}{\strut  PCs} \colorbox{Magenta!0.000}{\strut  via} \colorbox{Magenta!0.000}{\strut  free} \colorbox{Magenta!0.000}{\strut  apps} \colorbox{Magenta!0.000}{\strut .} \colorbox{Magenta!0.000}{\strut Wire} \colorbox{Magenta!0.000}{\strut less} \colorbox{Magenta!0.000}{\strut  printing} \colorbox{Magenta!0.000}{\strut  is} \\
\bottomrule
\end{longtable}
\caption{feature pairs/Layer15-65536-J1-LR5.0e-04-k32-T3.0e+08 abs mean/examples-64039-v-38992 stas c4-en-10k,train,batch size=32,ctx len=16.csv}
\end{table}

\end{document}
