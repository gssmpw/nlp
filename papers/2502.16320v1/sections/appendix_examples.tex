\section{PEW Surveys Experiments: Details and Additional Examples} \label{sec:pew_additional_examples} 
In this section, we expand on our PEW surveys experiment where we used polling data on key political and social issues to show: (i) how $\nbc$ rankings can differ from those maximizing the average reward; (ii) How sensitive $\nbc$ is to the sampling distribution of the pairwise preference data. 

\niparagraph{Data.} We use several Pew Research Center surveys, specifically the American Trends Panel surveys number 35, 52, 79, 83, 99,  109, 
111, 112,  114, 119, 120, 121, 126, 127, 128, 129, 130, 131, and 132. The choices are a mix of recent surveys and  those relevant to science, technology, data and AI. 
%
Each survey include questions asked to thousands of participants. We categorize participants by political party leanings to define types.
%
When processing the questions, we discard responses that are empty, as well as discarding the option "Refused". We note that discarding the option "Refused" had no effect on the results as it is not frequently chosen. 

\niparagraph{Reward Estimation.}
Although we observe how often each group selects a particular option, we don't directly observe respondents' internal rewards. To estimate this, we apply the Luce-Shepherd model~\citep{shepard_stimulus_1957, luce1959individual}:
 \begin{equation}
 \label{eq:luce-shep}
        \Pr\big(\text{option } i \text{ is chosen from } \mathcal{S}\big) =  \frac{\exp{(r(i;u))}}{\sum_{{j\in \mathcal{S}}} \exp{(r(j;u))}},
\end{equation}
where $\mathcal{S}$ is the set of options, and $r(\cdot; u)$ is the reward for type $u$. 
This allows us to estimate each option's reward (up to a constant additive term) for each type. From these estimates and observed probabilities, we compute both the expected reward and the $\nbc$ metric, where in the latter we assume the uniform probability for alternatives unless specified otherwise.

\niparagraph{Sensitivity Experiments.} In \cref{sec:sensitivity}, we estimate the sensitivity of $\nbc$ rankings to the sampling distribution of pairwise preference data by determining the minimum Total Variation (TV) distance required, if possible, to alter $\nbc$ rankings from that attained under the uniform distribution.

Recall that $\nbc$ is defined as:
\[
\nbc(\vy; \gD) \coloneqq \E_{\vy' \sim \gD(\cdot)} \big[\Pr(\vy \succ \vy' \mid \vx; r)\big].
\]
To compute this, we first estimate the reward function \(r\), then evaluate \(\Pr(\vy \succ \vy'; r)\) for all \((\vy, \vy') \in \gY \times \gY\), where \(\gY\) is the set of alternatives. 
Next, consider the feasibility of swapping the ranking induced by the uniform distribution \(\gD_U\) for alternatives \(\vy_i\) and \(\vy_j\) with a new distribution \(\gD_a\), assuming \(\nbc(\vy_i; \gD_U) > \nbc(\vy_j; \gD_U)\). This is equivalent to solving the following linear program:
\[
\begin{aligned}
     \text{minimize}& \quad \frac{1}{2} \mathbf{1}^\top \mathbf{s}, \\
    \text{subject to:} 
    & \quad \mathbf{q} > \epsilon \mathbf{1}, \\
    & \quad \mathbf{s} \geq \frac{1}{N} \mathbf{1} - \mathbf{q}, \\
    & \quad \mathbf{s} \geq \mathbf{q} - \frac{1}{N} \mathbf{1}, \\
    & \quad \mathbf{P}_{i} \mathbf{q} < \mathbf{P}_{j} \mathbf{q} + \delta, \\
    & \quad \mathbf{1}^\top \mathbf{q} = 1, \\
    & \quad \mathbf{q} > \mathbf{0}.
\end{aligned}
\]
Here, \(N = |\gY|\), and labeling the alternatives as \(\vy_1, \dots, \vy_N\), we define \(\mathbf{P}_{ij} = \Pr(\vy_i \succ \vy_j; r)\), \(q_i = \gD_a(\vy_i)\), and \(\mathbf{P}_k\) as the \(k\)-th row of \(\mathbf{P}\). The parameter \(\epsilon > 0\) ensures support for all alternatives, and \(\delta > 0\) controls the required magnitude of change in \(\nbc\) beyond what is required for the swap. We set \(\epsilon = \delta = 10^{-5}\). 

To compute the minimum TV distance, we solve the program for all pairs \((i, j)\) where \(\nbc(\vy_i; \gD_U) > \nbc(\vy_j; \gD_U)\) and record the smallest objective value. In this analysis, we group respondents by political leaning (specifically, the column \texttt{F\_PARTYSUM\_FINAL}). We also note that in this analysis we exclude survey questions with fewer than three options. 


% \niparagraph{Sensitivity Experiments.} In \cref{sec:sensitiviy}, to estimate the sensitivity of $\nbc$ to the sampling distribution of the pairwise preference data, we find the minimum TV distance required to alter $\nbc$ ranking under uniform. 
% First recall that $\nbc$ is defined as,
% \begin{equation}
%     \nbc(\vy; \gD) \coloneqq \E_{\vy' \sim \gD(\cdot)} \Big[\Pr(\vy \succ \vy' \mid \vx; r)\Big].
% \end{equation}
% Hence, to estimate it, we first estimate the reward $r$ and then compute the quantity $\Pr(\vy \succ \vy'; r) \forall (\vy, \vy') \in \gY \times \gY$, where $\gY$ is the set of alternatives. Note that the feasibility of swapping the ranking induced by the uniform distribution $\gD_U$ of options $\vy_i$ and $\vy_j$ with a new distribution $\gD_a$, and assuming $\nbc(\vy_i; \gD_u) > \nbc(\vy_j; \gD_U)$, is equivalent to the feasibility of the following linear program,
% \[
% \begin{aligned}
%     & \text{minimize} \quad \frac{1}{2} \mathbf{1}^\top \mathbf{s} \\
%     \text{subject to:} 
%     & \quad \mathbf{q} > \epsilon \mathbf{1}, \\
%     & \quad \mathbf{s} \geq \frac{1}{N} \mathbf{1} - \mathbf{q}, \\
%     & \quad \mathbf{s} \geq \mathbf{q} - \frac{1}{N} \mathbf{1}, \\
%     & \quad \mathbf{P}_{i} \mathbf{q} < \mathbf{P}_{j} \mathbf{q} + \delta, \\
%     & \quad \mathbf{1}^\top \mathbf{q} = 1, \\
%     & \quad \mathbf{q} > \mathbf{0}.
% \end{aligned}
% \]
% Where in the above,  $N = |\gY|$, and with labeling of the alternatives as $\vy_1, \dots, \vy_N$, let   $\mathbf{P}_{ij} = \Pr(\vy_i \succ \vy_j; r)$, $q_i = \gD_a(y_i)$, and $\mathbf{P}_k$ be the $k$-th row of $\mathbf{P}$. Note that $\epsilon > 0$ is used to retain the support for all alternatives, and $\delta > 0$ controls the magnitude of the change  required in $\nbc$. We set both $\delta$ and $\epsilon$ to $10^{-5}$. 
% %
% Now to find the minimum TV distance required, we find the minimum value attained by this program across all pairs $(i,j)$ such that $\nbc(\vy_i; \gD_u) > \nbc(\vy_j); \gD_u$.
% Note that in the analysis above, we use the political leaning (specifically the column \texttt{F\_PARTYSUM\_FINAL}) to group respondents.

\niparagraph{Additional Examples.} We highlight some of the examples of discrepancies that we find in some of the surveys listed above in Figures \ref{fig:example5}, \ref{fig:example6}, \ref{fig:example7}, \ref{fig:example8},  and \ref{fig:example9}. 
%
 We note though that while we indeed find a few examples of the discrepancy, $\nbc$ rankings is actually aligned with average reward rankings in most cases.  One possible explanation for this is that in many questions, the distributions of responses across types were very similar, suggesting that a homogeneous reward model would have been appropriate, causing $\nbc$ and average reward to align.


\begin{figure}[!htbp]
    \centering
    \begin{minipage}{0.6\textwidth}
        \centering
    % First Row
    \subfigure[Rewards per Type]{%
        \includegraphics[width=0.98\linewidth]{figs/PEW/rew_BIDENADM_W83.pdf}
        \label{fig:prop_rew5}
    }%
    \hfill
    \subfigure[Avg Reward vs. NBC]{%
        \includegraphics[width=0.98\linewidth]{figs/PEW/reward_vs_NBC_BIDENADM_W83.pdf}
        \label{fig:reward_vs_nbc5}
    }
    \end{minipage} 
    \begin{minipage}{0.2\textwidth}
        \subfigure[Ranking]{%
        \includegraphics[width=0.95\linewidth]{figs/PEW/ranking_comparison_BIDENADM_W83.pdf}
        \label{fig:ranking5}
    }
    \end{minipage}
    \caption{Do you think the plans and policies of the Biden administration will make the country’s response to the coronavirus outbreak:
A: A lot better; B: A little better; C: Not much different; D: A little worse; E: A lot worse}
\label{fig:example5}
\end{figure}

\begin{figure}[!htbp]
    \centering
    \begin{minipage}{0.6\textwidth}
        \centering
    % First Row
    \subfigure[Rewards per Type]{%
        \includegraphics[width=0.98\linewidth]{figs/PEW/rew_COVIDEGFP_a_W114.pdf}
        \label{fig:prop_rew6}
    }%
    \hfill
    \subfigure[Avg Reward vs. NBC]{%
        \includegraphics[width=0.98\linewidth]{figs/PEW/reward_vs_NBC_COVIDEGFP_a_W114.pdf}
        \label{fig:reward_vs_nbc6}
    }
    \end{minipage} 
    \begin{minipage}{0.2\textwidth}
        \subfigure[Ranking]{%
        \includegraphics[width=0.95\linewidth]{figs/PEW/ranking_comparison_COVIDEGFP_a_W114.pdf}
        \label{fig:ranking6}
    }
    \end{minipage}
    
    \caption{ How would you rate the job Joe Biden is doing responding to the coronavirus outbreak?    
A: Excellent; B: Good; C: Only fair; D: Poor}
\label{fig:example6}
\end{figure}

\begin{figure}[!htbp]
    \centering
    \begin{minipage}{0.6\textwidth}
        \centering
    % First Row
    \subfigure[Rewards per Type]{%
        \includegraphics[width=0.98\linewidth]{figs/PEW/rew_EVCAR2_W128.pdf}
        \label{fig:prop_rew7}
    }%
    \hfill
    \subfigure[Avg Reward vs. NBC]{%
        \includegraphics[width=0.98\linewidth]{figs/PEW/reward_vs_NBC_EVCAR2_W128.pdf}
        \label{fig:reward_vs_nbc7}
    }
    \end{minipage} 
    \begin{minipage}{0.2\textwidth}
        \subfigure[Ranking]{%
        \includegraphics[width=0.95\linewidth]{figs/PEW/ranking_comparison_EVCAR2_W128.pdf}
        \label{fig:ranking7}
    }
    \end{minipage}
    
    \caption{The next time you purchase a vehicle, how likely are you to seriously consider purchasing an electric vehicle?
A: Very likely; B: Somewhat likely; C: Not too likely; D: Not at all likely; E: I do not expect to purchase a vehicle
}
\label{fig:example7}
\end{figure}

\begin{figure}[!htbp]
    \centering
    \begin{minipage}{0.6\textwidth}
        \centering
    % First Row
    \subfigure[Rewards per Type]{%
        \includegraphics[width=0.98\linewidth]{figs/PEW/rew_FAVPOL_HARRIS_W130.pdf}
        \label{fig:prop_rew8}
    }%
    \hfill
    \subfigure[Avg Reward vs. NBC]{%
        \includegraphics[width=0.98\linewidth]{figs/PEW/reward_vs_NBC_FAVPOL_HARRIS_W130.pdf}
        \label{fig:reward_vs_nbc8}
    }
    \end{minipage} 
    \begin{minipage}{0.2\textwidth}
        \subfigure[Ranking]{%
        \includegraphics[width=0.95\linewidth]{figs/PEW/ranking_comparison_FAVPOL_HARRIS_W130.pdf}
        \label{fig:ranking8}
    }
    \end{minipage}
    
    \caption{ What is your overall opinion of    Kamala Harris?
A: Very favorable; B: Mostly favorable; C: Mostly unfavorable; D: Very unfavorable; E: Never heard of this person.
}
\label{fig:example8}
\end{figure}

\begin{figure}[!htbp]
    \centering
    \begin{minipage}{0.6\textwidth}
        \centering
    % First Row
    \subfigure[Rewards per Type]{%
        \includegraphics[width=0.98\linewidth]{figs/PEW/rew_FAVPOL_BIDEN_W130.pdf}
        \label{fig:prop_rew9}
    }%
    \hfill
    \subfigure[Avg Reward vs. NBC]{%
        \includegraphics[width=0.98\linewidth]{figs/PEW/reward_vs_NBC_FAVPOL_BIDEN_W130.pdf}
        \label{fig:reward_vs_nbc9}
    }
    \end{minipage} 
    \begin{minipage}{0.2\textwidth}
        \subfigure[Ranking]{%
        \includegraphics[width=0.95\linewidth]{figs/PEW/ranking_comparison_FAVPOL_BIDEN_W130.pdf}
        \label{fig:ranking9}
    }
    \end{minipage}
    
    \caption{ What is your overall opinion of Joe Biden?
A: Very favorable; B: Mostly favorable; C: Mostly unfavorable; D: Very unfavorable; E: Never heard of this person.
}
\label{fig:example9}
\end{figure}


% \begin{figure}[!htbp]
%     \centering
%     \begin{minipage}{0.6\textwidth}
%         \centering
%     % First Row
%     \subfigure[Rewards per Type]{%
%         \includegraphics[width=0.98\linewidth]{figs/PEW/rew_POLPROB_W127.pdf}
%         \label{fig:prop_rew10}
%     }%
%     \hfill
%     \subfigure[Avg Reward vs. NBC]{%
%         \includegraphics[width=0.98\linewidth]{figs/PEW/reward_vs_NBC_POLPROB_W127.pdf}
%         \label{fig:reward_vs_nbc10}
%     }
%     \end{minipage} 
%     \begin{minipage}{0.2\textwidth}
%         \subfigure[Ranking]{%
%         \includegraphics[width=0.95\linewidth]{figs/PEW/ranking_comparison_POLPROB_W127.pdf}
%         \label{fig:ranking10}
%     }
%     \end{minipage}
    
%     \caption{ How much of a problem do you think police violence against Black people is in the United States today? A: Major problem; B: Minor problem; C: Not a problem.
% }
% \label{fig:example10}
% \end{figure}
