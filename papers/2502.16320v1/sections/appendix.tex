%%%%%%%%%%
\section{Additional Drawbacks of DPO under Heterogeneity}
\label{sec:drawbacks_appendix}

%%%
\paragraph{Violating Independence of Irrelevant Alternatives (IIA).}
Suppose $\gU = \{A, B\}$ and types are equally represented. Given two possible responses~$\vy_1$ and~$\vy_2$, type~$A$ prefers~$\vy_1$ but type~$B$ prefers~$\vy_2$:
%
\begin{align*}
    &r^*(\vy_1; A) = 6,\, r^*(\vy_2; A) = 1 \,, \\
    &r^*(\vy_1; B) = 3,\, r^*(\vy_2; B) = 9 \,.
\end{align*}
%
A direct calculation shows $\E_u[r^*(\vy_2)] > \E_u[r^*(\vy_1)]$ and $\nbc(\vy_2) > \nbc(\vy_1)$. So, both $\pi^*$ and~$\pi_\dpo$ prefer~$\vy_2$. Let's consider another possible response~$\vy_3$ which is not the preferred response for any user type:
%
\begin{equation*}
    r^*(\vy_3; A) = r^*(\vy_3; B) = 2 
    \,.
\end{equation*}
%
While $\pi^*$~still prefers~$\vy_2$ to~$\vy_1$, now $\nbc(\vy_1) \approx 0.62 > \nbc(\vy_2) \approx 0.55$, so, introducing an irrelevant alternative can alter DPO's ranking over existing alternatives.
%%%


%%%
\paragraph{Tyranny of Majority.}
Suppose $\gU = \{A, B\}$ with type~$A$ shaping~$90\%$ of the population. Given two responses~$\vy_1, \vy_2$, type~$A$ slightly favors~$\vy_2$ but type~$B$ finds~$\vy_2$ offensive:
%
\begin{align*}
    &r^*(\vy_1; A) = 0.5,\, r^*(\vy_2; A) = 1 \,, \\
    &r^*(\vy_1; B) = 0.5,\, r^*(\vy_2; B) = -10 \,.
\end{align*}
%
In this case, $\pi^*$ prefers~$\vy_1$ even though type~$B$ is a minority. In contrast, we have $\nbc(\vy_1) \approx 0.47, \nbc(\vy_2) \approx 0.53$, which implies that the majority dominates in DPO.
%%%

%%%%%%%%%%


%%%%%%%%%%
%\ifnotarxiv
\section{Additional Related Work}
\label{sec:related_appendix}
The challenge of handling heterogeneous preferences in alignment has been recognized as a significant problem in alignment research~\citep{anwar_foundational_2024, casper_open_2023,ge2024axioms,sorensen_roadmap_2024}. This problem has attracted considerable attention from researchers in the field. Here, we highlight a few representative works that address key directions in tackling this challenge.

\paragraph{Analysis of DPO.}
Our study of how standard preference learning methods, such as DPO, behave in the presence of heterogeneous preferences was inspired by \citet{siththaranjan2023distributional}'s result, which shows that RLHF aggregates preferences according to a well-known voting rule called Borda count. \citet{chakraborty2024maxmin} highlights the impossibility of aligning with a singular reward model in RLHF by providing a lower bound on the gap between the optimal policy and a subpopulation's optimal policy. \citet{dumoulin2023density} adopts a density estimation perspective on learning from human feedback to illustrate the challenges of preference learning from a population of annotators with diverse viewpoints. \citet{rosset_direct_2024} and~\citet{gao_rebel_2024} point out the limitations of point-wise reward models in expressing complex, intransitive preferences that may arise due to the aggregation of diverse preferences. Additionally, frameworks that generalize DPO and unify different alignment methods have been proposed to analyze current approaches and explore possible alternatives~\citep{chen_mallowspo_2024, tang_generalized_2024, azar2024general, meng2024simpo}.


\paragraph{Policy Personalization.} 
Many works in the literature have proposed personalization as a solution to the problem of pluralistic alignment. \citet{poddar_personalizing_2024} propose a latent variable formulation of the problem and learn rewards and policies conditioned on it. \citet{chen_pal_2024} use an ideal point model for preferences and learn latent spaces representing different preferences. Mapping user information to user representations, \citet{li_personalized_2024} perform personalized DPO to jointly learn a user model and a personalized language model. \citet{balepur2025boatdoesfloatimproving} use abductive reasoning to infer user personas and train models to tailor responses accordingly. \citet{lee_aligning_2024} explore the possibility of steering a language model to align with a user's intentions through system messages. \citet{dang_personalized_2025} extend personalized alignment to text-to-image diffusion models. \citet{jang_personalized_2023} perform personalized alignment by decomposing preferences into multiple dimensions. \citet{lau_personalized_2024} dynamically adapt the model to individual preferences using in-context learning.


\paragraph{Preference Aggregation.} 
Closely aligned with our goal of serving the entire population with a single policy, several works have explored ways to aggregate diverse preferences. The rich literature on social choice theory has proven to be a valuable source of inspiration for studying existing preference learning approaches and proposing new ones~\citep{conitzer2024social, qiu_representative_2024,alamdari_policy_2024,ge2024axioms,dai2024mapping}. Drawing insights from social choice theory, robustness to approximate clones has been proposed as a desirable property of RLHF algorithms, which current methods lack~\citep{procaccia2025clone}. The Minimax Winner, a concept in preference aggregation, has inspired the use of the proportion of wins as the reward for a particular trajectory to align a model through self-play~\citep{swamy_minimaximalist_2024}. The impact of heterogeneity on strategic behavior in feedback and its effects on aggregation are also explored in~\citet{park2024rlhf}, which further examines the use of different social welfare functions for preference aggregation.


\paragraph{Methods.} 
Solutions proposed to address different formulations of the problem span a wide range of methods. \citet{siththaranjan2023distributional} estimate a distribution of scores for alternatives to account for heterogeneity as hidden context. \citet{chidambaram2024direct} propose an Expectation-Maximization (EM) version of DPO to minimize a notion of worst-case regret. Multi-objective reinforcement learning~\citep{harland_adaptive_2024, jang_personalized_2023} and its direct optimization variant~\citep{zhou_beyond_2024} have also been proposed to align with diverse preferences. \citet{wang_arithmetic_2024} train a multi-objective reward model to capture diverse preferences. \citet{zhong_provable_2024} use meta-learning to learn diverse preferences and aggregate them using different social welfare functions. \citet{li_aligning_2024} design an optimal-transport-based loss to calibrate their model with the categorical distribution of preferences.
 Producing a Pareto front of models has also been explored as a solution. \citet{boldi_pareto-optimal_2024} employ an iterative process to select solutions, while~\citet{rame_rewarded_2023} interpolate the weights of independent networks linearly to achieve a Pareto-optimal generalization across preferences. 

\paragraph{Empirical Observations.} 
Empirical studies of alignment methods have had a significant impact on the study of preference learning. \citet{zhang_diverging_2024} demonstrate the Bradley-Terry model's failure to distinguish between unanimous agreement among annotators and the majority opinion in cases of diverging user preferences. \citet{chen_preference_2024} show that RLHF and DPO struggle to improve ranking accuracy. \citet{zeng_diversified_2024} study the role of model size and data size in the impact of diversified human preferences. \citet{bansal_peering_2024} demonstrate the significant influence of feedback protocol choice on alignment evaluation. \citet{santurkar_whose_2023} explore the opinions reflected by a language model, while \citet{bakker_fine-tuning_2022} investigate a language model's ability to generate consensus statements by training it to predict individual preferences. \citet{jiang_can_2024} propose individualistic alignment to predict an individual's values, and \citet{zollo_personalllm_2024} introduce the PersonalLLM benchmark to measure a model's adaptation to a particular user's preferences.


%\fi
%%%%%%%%%%


%%%%%%%%%%
\section{Additional Statements}

%%%
\begin{theoremEnd}[restate]{proposition}
\label{prop:mixture_bt}
There exists a mixture of BTs that a single BT cannot represent.
\end{theoremEnd}
%%%
\begin{proofEnd}
    Suppose the pairwise comparison distribution over a set of alternatives \((\vy_1, \vy_2, \vy_3, \dots)\)  satisfies the Bradley-Terry (BT) model; i.e. \(\Pr(\vy_i \succ \vy_j) = \sigma\big(r^*(\vy_2) - r^*(\vy_1)\big)\). Then:
    \begin{align*}
    \Pr(\vy_1 \succ \vy_2)\Pr(\vy_2 \succ \vy_3)\Pr(\vy_3 \succ \vy_1) &= \frac{\prod_{i=1}^3 \exp(r^*(\vy_i))}{\prod_{i=1}^3 \left(\exp(r^*(\vy_i)) + \exp( r^*(\vy_{(i+1)\bmod 3 + 1}))\right)}\\
    &= \Pr(\vy_1 \succ \vy_3)\Pr(\vy_3 \succ \vy_2)\Pr(\vy_2 \succ \vy_1)
    \,.
    \end{align*}
    Now, consider two BT models corresponding to \(u_1\) and \(u_2\), with a uniform mixture over them. For the mixture:
    \[
    \Pr(\vy_i \succ \vy_j) = \frac{\Pr(\vy_i \succ \vy_j \mid u_1) + \Pr(\vy_i \succ \vy_j \mid u_2)}{2}
    \,.
    \]
    The probability of cyclic preferences in one direction is given by
    \[
    \Pr(\vy_1 \succ \vy_2)\Pr(\vy_2 \succ \vy_3)\Pr(\vy_3 \succ \vy_1) = \frac{\sum_{s \in \{1, 2\}^3} \prod_{i=1}^3 \Pr(\vy_i \succ \vy_{(i+1)\bmod 3 + 1} \mid u_{s_i})}{8}
    \,,
    \]
    which is not necessarily equal to the probability of the cyclic preferences in the reverse direction:
    \[
    \Pr(\vy_1 \succ \vy_3)\Pr(\vy_3 \succ \vy_2)\Pr(\vy_2 \succ \vy_1) = \frac{\sum_{s \in \{1, 2\}^3} \prod_{i=1}^3 \Pr(\vy_{(i+1)\bmod 3 + 1} \succ \vy_i \mid u_{s_i})}{8}
    \,.
    \]
    To verify this, consider specific examples such as \(\Pr(\vy_i \succ \vy_j \mid u_k) = \frac{\exp(r^i_k)}{\exp(r^i_k) + \exp(r^j_k)}\) with \(r_1 = (1, 2, 3)\) and \(r_2 = (1, 2, 4)\).
    %, or run the provided code to test arbitrary mixtures. 
    More generally, the BT assumption implies that, for a fixed reward \(r^*\), the likelihood of a set of pairwise comparisons \(\{(\vy_{p, 1} > \vy_{p, 2})\}_{p \in [P]}\) is proportional to \(\prod_i \exp(r^*(\vy_i))^{|\{p \in [P] \,\mid\, \vy_{p, 1} = i\}|}\) and depends only on the number of times each option is preferred in the comparisons. However, as demonstrated above, this property does not hold for a mixture of BT models.
\end{proofEnd}
%%% 

%%%
\begin{definition}[Learnability]
\label{def:learnability}
Denote by~$\gD_{r, \sigma}$ an i.i.d. sampled pairwise preference dataset labeled by random users with reward~$r$ and preference model~$\sigma$. Let $\barr(\vy) \coloneqq \E_u[r(\vy; u)]$. We say that the ranking based on~$\barr$ is (weakly) learnable if, for some~$\epsilon > 0$, there exists an algorithm with a bounded sample complexity~$m$, such that for every reward~$r$, when given a dataset~$\gD_{r,\sigma}$ of size $|\gD_{r,\sigma}| \ge m(\epsilon, \barr)$, it outputs a ranking consistent with~$\barr$ with a probability at least $\epsilon$ above the chance level.
\end{definition}
%%%

%%%
\begin{theoremEnd}[restate]{proposition}
\label{prop:consistent_loss_2}
Defining~$l$ in \cref{eq:decompose_L} as follows results in a consistent estimation of the optimal policy when preferences follow the BT model:
%
\begin{equation*}
    l(\vy_1, \vy_2, \vo; \pi) = \begin{cases}
        -\sigma\big(h(\vy_1, \vy_2; \pi)\big) - I\big(\sigma\big(h(\vy_1, \vy_2; \pi)\big)\big) \,, & \vo = \vone \,, \\
        -\sigma\big(h(\vy_2, \vy_1; \pi)\big) - I\big(\sigma\big(h(\vy_2, \vy_1; \pi)\big)\big) \,, & \vo = \vzero \,, \\
        0 & \text{o.w.}
    \end{cases}
\end{equation*}
%
Here, we define $I(\theta) \coloneqq \int_1^\theta \big(\frac{1}{\theta'} - 1\big)^{|\gU|} \dif \theta'$, and $h$ is the difference of $\pi$'s induced rewards (\cref{eq:def_h}).
\end{theoremEnd}
%%%
\begin{proofEnd}
    Recall $\Pr(o_u = 1 \mid \vx, \vy_1, \vy_2) = \sigma\big(\Delta r^*(\vx, \vy_1, \vy_2; u)\big)$. We use $z_u(\vx, \vy_1, \vy_2)$ as a shorthand for this quantity and will drop the dependence on~$(\vx, \vy_1, \vy_2)$ whenever it is clear from the context. We also use~$s$ as a shorthand for~$\sigma(h)$.
    In the limit of a very large dataset, the proposed loss approaches
    %
    \begin{equation*}
        \gL(s) = -\E_{\vx, \vy_1, \vy_2} \Big[
        \big(\prod_{u \in \gU} z_u \big) \big(s + I(s)\big)
        + \big(\prod_{u \in \gU} (1 - z_u) \big) \big(1-s + I(1-s)\big)
        \Big] 
        \,.
    \end{equation*}
    %
    Note that we wrote~$\gL$ as a function of~$s$ instead of~$\pi$ since $s$~is the only place that~$\pi$ appears. We first show that $\gL(s)$ has a unique global minimizer. To show an~$s$ is a global minimizer of~$\gL$, it suffices to show that $s$~minimizes the term inside expectation for every~$(\vx, \vy_1, \vy_2)$. Such a minimizer meets the first-order condition: 
    %
    \begin{equation*}
        \big(\prod_{u \in \gU} z_u \big)\Big(1 + (\frac{1}{s} - 1)^{|\gU|} \Big) 
        + \big(\prod_{u \in \gU} (1 - z_u) \big)\Big(-1 - (\frac{1}{1 - s} - 1)^{|\gU|} \Big) 
        = 0
        \,.
    \end{equation*}
    %
    Here, we used $\od{I}{\theta} = (\frac{1}{\theta} - 1)^{|\gU|}$. Define $w \coloneqq (\frac{1 - s}{s})^{|\gU|}$. Then, the above condition reduces to a quadratic equation in terms of~$w$:
    %
    \begin{equation*}
        1 + w - \big(\prod_{u \in \gU} (\frac{1}{z_u} - 1) \big) (1 + w^{-1}) = (1 + w^{-1})\Big[w - \prod_{u \in \gU} (\frac{1}{z_u} - 1) \Big] = 0
        \,.
    \end{equation*}
    %
    Solving for~$w$, we obtain
    %
    \begin{equation*}
        s^* = \frac{1}{1 + \big(\prod_{u \in \gU} (\frac{1}{z_u} - 1) \big)^{\frac{1}{|\gU|}}}
        \,.
    \end{equation*}
    %
    For the BT model, a direct calculation then shows
    %
    \begin{equation}
    \label{eq:_proof_s_star}
        s^*(\vx, \vy_1, \vy_2) = \sigma\Big(\frac{1}{|\gU|}\sum_{u \in \gU}\Delta r(\vx, \vy_1, \vy_2; u)\Big)
        \,.
    \end{equation}
    %
    In fact, $s^*$ is the only global minimizer of~$\gL(s)$. This is because~$\gL(s)$ is convex in~$s$:
    %
    \begin{equation*}
        \od[2]{\gL}{s} = \E_{\vx, \vy_1, \vy_2}\Big[
        \big(\prod_{u \in \gU} z_u \big) \cdot \frac{|\gU|}{s^2} (\frac{1}{s} - 1)^{|\gU| - 1} 
        + \big(\prod_{u \in \gU} (1 - z_u) \big) \cdot \frac{|\gU|}{(1 - s)^2}(\frac{1}{1 - s} - 1)^{|\gU| - 1} 
        \Big] \ge 0
        \,.
    \end{equation*}
    %
    Finally, one can verify that the policy that results in~$s^*$ (\cref{eq:_proof_s_star}) is the optimal policy~$\pi^*$. This completes the proof that the proposed loss is a consistent loss for~$\pi^*$.  
\end{proofEnd}
%%%

%%%%%%%%%%


%%%%%%%%%%
\section{Missing Proofs}

\printProofs
%%%%%%%%%%


%%%%%%%%%%
\section{More Examples for \Cref{sec-general}}\label{sec-examples}

In this section, we provide more examples for the general problem framework in \Cref{sec-general}. In these examples, the survey responses can be real-valued or multi-dimensional.


\begin{example}[Market research]
Suppose a company is interested in learning its customers' willingness-to-pay (WTP) for a new product, which is the highest price a customer is willing to pay for the product. Then, each $\profile \in \profilespace$ can represent a customer profile (e.g., age, gender, occupation), each survey question $\testfunction$ is about a certain product, and a customer's response $\response$ is a noisy observation of the customer's WTP. Then $\responsedistalt(~\cdot\mid \testfunction )$ is the distribution of the customer population's WTP. We may take $\statistic (\testfunction)$ as the $\tau$-quantile of the WTP distribution $\responsedistalt(~\cdot\mid \testfunction )$, for some $\tau\in(0,1)$:
\[
\statistic (\testfunction) = \inf \left\{ q\in[0,\infty) : \PP_{\response \sim \responsedistalt(\cdot\mid \testfunction )} (\response\le q) \ge \tau \right\}.
\] 
An LLM can be used to simulate customers' WTP for the product.
\end{example}

\begin{example}[Public survey, multi-dimensional]\label{example-public-survey}
Suppose an organization is interested in performing a public survey in a city. Each survey question $\testfunction$ is a multiple-choice question with $5$ options. An example is ``How often do you talk to your neighbors?'', with $5$ choices ``Basically every day'', ``A few times a week'', ``A few times a month'', ``Once a month'', and ``Less than once a month''. Every $\profile \in \profilespace$ is a person's profile (e.g., age, gender, occupation), the response space $\responsespace$ is the standard orthonormal basis $\{e_i\}_{i=1}^5$ in $\RR^5$, where $\response = e_i$ indicates that a person chooses the $i$-th option. We can take $\statistic ( \testfunction ) = \EE_{\response \sim \responsedist(\cdot\mid\testfunction)}[y]\in\RR^5$, which summarizes the proportion of people that choose the $i$-th option. An LLM can be used to simulate people's answers to the survey question.
\end{example}

\begin{example}[Public survey, one-dimensional]\label{example-public-survey-1D}
Consider the setup in \Cref{example-public-survey}. When the $5$ choices in a survey question correspond to ordered sentiments, we can map them to numeric scores, say, $v = (-1,-\frac{1}{3},0,\frac{1}{3},1)^\top$. Then the statistic $\widetilde{\statistic} (\testfunction) = \langle v,\statistic (\testfunction)  \rangle$ reflects the population's average sentiment in the survey question $\testfunction$.
\end{example}


\newpage
\section{Semi-Synthetic Experiment: Fine-Tuning Llama-3-8B on HH-RLHF}
\label{app:llama}

\begin{figure}
    \centering
    \includegraphics[width=0.95\columnwidth]{figs/rews.pdf}
    \caption{Reward definition for three user types in semi-synthetic experiments (\cref{subsec:exp:semi-synth}) based on the length of prompt response combination. The first user type prefers long prompt response combinations, the second user type prefers short prompt response combinations, and the third user type prefers mid-length prompt response combinations. The dashed cyan line shows the average reward across the three user types.}
    \label{fig:app:rewards}
\end{figure}

\paragraph{Reward Models.}
\cref{fig:app:rewards} shows the three distinct rewards we use for the three user types along with their average.
In order to have a reliable ground-truth reward which we can rely on in evaluation, we define these rewards as functions of the number of tokens in prompt-response combinations.

\paragraph{Anonymous Dataset.}
We use prompts and response pairs from both helpfulness and harmlessness subsets of Anthropic's HH-RLHF dataset~\citep{hh-rlhf} and relabel the \textit{chosen} and \textit{rejected} responses manually.
We filter for data points in which the sum of the number of tokens in the prompt and the number of tokens in the longer response do not exceed $512$.
This leaves us with $160,800$ training and $17,104$ test data points. 
For every data point (a prompt with a pair of responses), we sample one of the three user types uniformly at random.
Given the type of user, we sample a preference based on BT~\citep{bt} to label the two alternatives.

\paragraph{Dataset with Maximum Annotator Information.}
We use prompts and response pairs from both helpfulness and harmlessness subsets of Anthropic's HH-RLHF dataset~\citep{hh-rlhf} and relabel the \textit{chosen} and \textit{rejected} responses manually.
We filter for data points in which the sum of the number of tokens in the prompt and the number of tokens in the longer response do not exceed $512$.
This leaves us with $160,800$ training and $17,104$ test data points. 
For every data point (a prompt with a pair of responses), we keep sampling BT~\citep{bt} preferences from all user types until they agree with each other.
Once the consensus is achieved, we stop sampling and use the agreed-upon preference as the label for this data point.

\paragraph{Fine-Tuning Details.}
We fine-tune Llama-3-8B~\citep{llama3modelcard} base model with LoRA~\citep{lora}.
We fine-tune for one epoch with a batch size of $2$, and use a linear learning rate schedule that starts with $3\times10^{-5}$ and decreases to zero.
We use the Adam optimizer with a weight decay of $0.001$~\citep{adamw}.
Regarding LoRA's hyper-parameters, we use the matrix rank of $r=8$, $\alpha=32$, and the dropout probability of $0.1$.

For direct alignment experiments, we use a uniform reference policy.
When ignoring heterogeneity, we do vanilla DPO over the anonymous dataset.
When modeling heterogeneity, we use the loss function we propose in \cref{prop:consistent_loss_1} over the dataset with maximum annotator information.
We use the ordinal agreement between the ground-truth average reward and the reward induced by the aligned policy as the measure of accuracy.

For the reward learning experiments, we fine-tune the Llama-3-8B as a reward model.
When ignoring heterogeneity, we assume BT and maximize the probability of the anonymous preference dataset under the learned reward model.
When modeling heterogeneity, we use the loss function in \cref{prop:consistent_loss_1} over the dataset with maximum annotator information, but replace $h(\vy_1, \vy_2; \pi)$ with the difference in rewards, i.e., $r(\vy_2) - r(\vy_1)$.
We use the ordinal agreement between the ground-truth average reward and the learned reward as the measure of accuracy.

\begin{table}
\caption{Raw Accuracy $(\%)$ in Alignment Experiments}
\label{tab:raw-aligns}
% \vskip 0.15in
\begin{center}
\begin{small}
\begin{sc}
\begin{tabular}{ccc}
\toprule
Seed & Ignoring Homogeneity & Modeling Heterogeneity \\
\midrule
0    &  65.61 & 66.63\\
1    &  67.55 & 69.55\\
2    &  68.77 & 75.21\\
3    &  68.70 & 72.04\\
4    &  66.28 & 74.95\\
\bottomrule
\end{tabular}
\end{sc}
\end{small}
\end{center}
% \vskip -0.1in
\end{table}

\begin{table}
\caption{Raw Accuracy $(\%)$ in Reward Learning Experiments}
\label{tab:raw-rews}
% \vskip 0.15in
\begin{center}
\begin{small}
\begin{sc}
\begin{tabular}{ccc}
\toprule
Seed & Ignoring Homogeneity & Modeling Heterogeneity \\
\midrule
0    &  92.33 & 95.26\\
1    &  85.38 & 92.01\\
2    &  88.46 & 94.6\\
3    &  89.69 & 93.57\\
4    &  91.94 & 93.87\\
\bottomrule
\end{tabular}
\end{sc}
\end{small}
\end{center}
% \vskip -0.1in
\end{table}

\paragraph{Detailed Results.}
We conduct every experiment with five different random seeds.
\cref{fig:semi-synthetic} shows the average, $25^\text{th}$ percentile, and $75^\text{th}$ percentile of accuracy across the five random seeds.
\cref{tab:raw-aligns,tab:raw-rews} show the raw accuracy numbers across the five random seeds.
%%%%%%%%%%