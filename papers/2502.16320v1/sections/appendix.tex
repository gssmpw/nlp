%%%%%%%%%%
\section{Additional Drawbacks of DPO under Heterogeneity}
\label{sec:drawbacks_appendix}

%%%
\paragraph{Violating Independence of Irrelevant Alternatives (IIA).}
Suppose $\gU = \{A, B\}$ and types are equally represented. Given two possible responses~$\vy_1$ and~$\vy_2$, type~$A$ prefers~$\vy_1$ but type~$B$ prefers~$\vy_2$:
%
\begin{align*}
    &r^*(\vy_1; A) = 6,\, r^*(\vy_2; A) = 1 \,, \\
    &r^*(\vy_1; B) = 3,\, r^*(\vy_2; B) = 9 \,.
\end{align*}
%
A direct calculation shows $\E_u[r^*(\vy_2)] > \E_u[r^*(\vy_1)]$ and $\nbc(\vy_2) > \nbc(\vy_1)$. So, both $\pi^*$ and~$\pi_\dpo$ prefer~$\vy_2$. Let's consider another possible response~$\vy_3$ which is not the preferred response for any user type:
%
\begin{equation*}
    r^*(\vy_3; A) = r^*(\vy_3; B) = 2 
    \,.
\end{equation*}
%
While $\pi^*$~still prefers~$\vy_2$ to~$\vy_1$, now $\nbc(\vy_1) \approx 0.62 > \nbc(\vy_2) \approx 0.55$, so, introducing an irrelevant alternative can alter DPO's ranking over existing alternatives.
%%%


%%%
\paragraph{Tyranny of Majority.}
Suppose $\gU = \{A, B\}$ with type~$A$ shaping~$90\%$ of the population. Given two responses~$\vy_1, \vy_2$, type~$A$ slightly favors~$\vy_2$ but type~$B$ finds~$\vy_2$ offensive:
%
\begin{align*}
    &r^*(\vy_1; A) = 0.5,\, r^*(\vy_2; A) = 1 \,, \\
    &r^*(\vy_1; B) = 0.5,\, r^*(\vy_2; B) = -10 \,.
\end{align*}
%
In this case, $\pi^*$ prefers~$\vy_1$ even though type~$B$ is a minority. In contrast, we have $\nbc(\vy_1) \approx 0.47, \nbc(\vy_2) \approx 0.53$, which implies that the majority dominates in DPO.
%%%

%%%%%%%%%%


%%%%%%%%%%
%\ifnotarxiv
\section{Additional Related Works}
\label{app:related}


\textbf{Structured compositional generative models.} Structured generative models leverage architectural inductive biases in an encoder-decoder framework, such as recurrent attention mechanisms \cite{gregor2015drawrecurrentneuralnetwork} or slot-attention \cite{Wang2023SlotVAEOS}. These models decompose scenes into background and parts-based representations in an unsupervised manner guided by modeling priors. While these approaches can flexibly generate scenes with single or multiple objects, they are not explicitly controllable, and require specific model pre-training on datasets containing compositions of interest.

\textbf{Controllable generation.} Composition at inference-time is one potential mechanism for exerting control over the generation process. Another way to modify compositions of style and/or content attributes is through spatial conditioning a pre-trained diffusion model on a structural attribute (e.g., pose or depth) as in  \citet{zhang2023adding}, or on multiple attributes of style and/or content as in \citet{conditional-loradapter}. Another option is control through resampling, as in \citet{liu2024correcting}. These methods are complementary to single or multiple model conditioning mechanisms based on score composition that we study in the current work.

\textbf{Single model conditioning.} We distinguish the kind of composition we study in this paper from approaches that rely on a single model but use OOD conditioners to achieve novel combinations of concepts never seen together during training; for example, passing OOD text prompts to text-to-image models \citep{nichol2021glide, podell2023sdxl}, or works like \citet{okawa2024compositional, park2024emergence} where a single model conditions simultaneously on multiple attributes like shape and color, with some combinations held out during training.
In contrast, the compositions we study recombine the outputs of multiple separate models at inference time.
Though less powerful, this can still be surprisingly effective, and is more amenable to theoretical study since it disentangles the potential role of conditional embeddings.

\textbf{Multiple model composition.} Among compositions involving multiple separate models, many different variants have been explored with different goals and applications.
Some definitions of composition are inspired by logical operators like AND and OR, usually taken to mean that the composed distribution should have high probability under all of the conditional distributions to be composed, or at least one of them, respectively.
Given two conditional probabilities $p_0(x), p_1(x)$, AND is typically implemented as the product $p_0(x)p_1(x)$ and OR as sum $p_0(x) + p_1(x)$
(though these only loosely correspond to the logical operators and other implementations are also possible).
Some composition methods are based on diffusion models and use the learned scores (mainly for product compositions), others use energy-based models (which allows for OR-inspired sum compositions, as well as more sophisticated samplers, in particular sampling at $t=0$ \citep{du2020visualenergy, du2023reduce, liu2021learning}, and still others work directly with the densities \cite{skreta2024superposition} (enabling an even greater variety of compositions, including a different style of AND, taken to mean $p_0(x) = p_1(x)$). \citet{mcallister2025decentralized} explore another type of OR composition. \cite{wiedemer2024compositional} take a different approach of taking the final rendered images generated by separate diffusion models and ``adding them up'' in pixel-space, as part of a study on generalization of data-generating processes. Task-arithmetic \cite{zhang2023composing, ilharco2022editing}, often using LoRAs \cite{hu2021lora}, is a kind of composition in weight-space that has had significant practical impact.

\textbf{Product compositions.} In this work, we focus specifically on product compositions (broadly defined to allow for a ``background'' distribution, i.e. compositions of the form $\hat{p}(x) = p_b(x) \prod_i \frac{p_i(x)}{p_b(x)}$) implemented with diffusion models, which allows the composition to be implemented via a linear combinations of scores as in \citet{du2023reduce, liu2022compositional}. Our goal is not to propose
a wholly new method of composition but rather to improve theoretical understanding of existing methods.

\textbf{Learning and Generalization.}
Recently, \citet{kamb2024analytic}
demonstrated how a type of compositional generalization
arises from inductive bias in the learning procedure (equivariance
and locality).
Their findings are relevant to our broader motivation,
but complementary to the focus of this work.
Specifically, we focus only on mathematical aspects
of defining and sampling from compositional distributions,
and we do not consider any learning-theoretic aspects
such as inductive bias or sample complexity.
This allows us to study the behavior of
compositional sampling methods
even assuming perfect knowledge of the underlying distributions.

%\fi
%%%%%%%%%%


%%%%%%%%%%
\section{Additional Statements}

%%%
\begin{theoremEnd}[restate]{proposition}
\label{prop:mixture_bt}
There exists a mixture of BTs that a single BT cannot represent.
\end{theoremEnd}
%%%
\begin{proofEnd}
    Suppose the pairwise comparison distribution over a set of alternatives \((\vy_1, \vy_2, \vy_3, \dots)\)  satisfies the Bradley-Terry (BT) model; i.e. \(\Pr(\vy_i \succ \vy_j) = \sigma\big(r^*(\vy_2) - r^*(\vy_1)\big)\). Then:
    \begin{align*}
    \Pr(\vy_1 \succ \vy_2)\Pr(\vy_2 \succ \vy_3)\Pr(\vy_3 \succ \vy_1) &= \frac{\prod_{i=1}^3 \exp(r^*(\vy_i))}{\prod_{i=1}^3 \left(\exp(r^*(\vy_i)) + \exp( r^*(\vy_{(i+1)\bmod 3 + 1}))\right)}\\
    &= \Pr(\vy_1 \succ \vy_3)\Pr(\vy_3 \succ \vy_2)\Pr(\vy_2 \succ \vy_1)
    \,.
    \end{align*}
    Now, consider two BT models corresponding to \(u_1\) and \(u_2\), with a uniform mixture over them. For the mixture:
    \[
    \Pr(\vy_i \succ \vy_j) = \frac{\Pr(\vy_i \succ \vy_j \mid u_1) + \Pr(\vy_i \succ \vy_j \mid u_2)}{2}
    \,.
    \]
    The probability of cyclic preferences in one direction is given by
    \[
    \Pr(\vy_1 \succ \vy_2)\Pr(\vy_2 \succ \vy_3)\Pr(\vy_3 \succ \vy_1) = \frac{\sum_{s \in \{1, 2\}^3} \prod_{i=1}^3 \Pr(\vy_i \succ \vy_{(i+1)\bmod 3 + 1} \mid u_{s_i})}{8}
    \,,
    \]
    which is not necessarily equal to the probability of the cyclic preferences in the reverse direction:
    \[
    \Pr(\vy_1 \succ \vy_3)\Pr(\vy_3 \succ \vy_2)\Pr(\vy_2 \succ \vy_1) = \frac{\sum_{s \in \{1, 2\}^3} \prod_{i=1}^3 \Pr(\vy_{(i+1)\bmod 3 + 1} \succ \vy_i \mid u_{s_i})}{8}
    \,.
    \]
    To verify this, consider specific examples such as \(\Pr(\vy_i \succ \vy_j \mid u_k) = \frac{\exp(r^i_k)}{\exp(r^i_k) + \exp(r^j_k)}\) with \(r_1 = (1, 2, 3)\) and \(r_2 = (1, 2, 4)\).
    %, or run the provided code to test arbitrary mixtures. 
    More generally, the BT assumption implies that, for a fixed reward \(r^*\), the likelihood of a set of pairwise comparisons \(\{(\vy_{p, 1} > \vy_{p, 2})\}_{p \in [P]}\) is proportional to \(\prod_i \exp(r^*(\vy_i))^{|\{p \in [P] \,\mid\, \vy_{p, 1} = i\}|}\) and depends only on the number of times each option is preferred in the comparisons. However, as demonstrated above, this property does not hold for a mixture of BT models.
\end{proofEnd}
%%% 

%%%
\begin{definition}[Learnability]
\label{def:learnability}
Denote by~$\gD_{r, \sigma}$ an i.i.d. sampled pairwise preference dataset labeled by random users with reward~$r$ and preference model~$\sigma$. Let $\barr(\vy) \coloneqq \E_u[r(\vy; u)]$. We say that the ranking based on~$\barr$ is (weakly) learnable if, for some~$\epsilon > 0$, there exists an algorithm with a bounded sample complexity~$m$, such that for every reward~$r$, when given a dataset~$\gD_{r,\sigma}$ of size $|\gD_{r,\sigma}| \ge m(\epsilon, \barr)$, it outputs a ranking consistent with~$\barr$ with a probability at least $\epsilon$ above the chance level.
\end{definition}
%%%

%%%
\begin{theoremEnd}[restate]{proposition}
\label{prop:consistent_loss_2}
Defining~$l$ in \cref{eq:decompose_L} as follows results in a consistent estimation of the optimal policy when preferences follow the BT model:
%
\begin{equation*}
    l(\vy_1, \vy_2, \vo; \pi) = \begin{cases}
        -\sigma\big(h(\vy_1, \vy_2; \pi)\big) - I\big(\sigma\big(h(\vy_1, \vy_2; \pi)\big)\big) \,, & \vo = \vone \,, \\
        -\sigma\big(h(\vy_2, \vy_1; \pi)\big) - I\big(\sigma\big(h(\vy_2, \vy_1; \pi)\big)\big) \,, & \vo = \vzero \,, \\
        0 & \text{o.w.}
    \end{cases}
\end{equation*}
%
Here, we define $I(\theta) \coloneqq \int_1^\theta \big(\frac{1}{\theta'} - 1\big)^{|\gU|} \dif \theta'$, and $h$ is the difference of $\pi$'s induced rewards (\cref{eq:def_h}).
\end{theoremEnd}
%%%
\begin{proofEnd}
    Recall $\Pr(o_u = 1 \mid \vx, \vy_1, \vy_2) = \sigma\big(\Delta r^*(\vx, \vy_1, \vy_2; u)\big)$. We use $z_u(\vx, \vy_1, \vy_2)$ as a shorthand for this quantity and will drop the dependence on~$(\vx, \vy_1, \vy_2)$ whenever it is clear from the context. We also use~$s$ as a shorthand for~$\sigma(h)$.
    In the limit of a very large dataset, the proposed loss approaches
    %
    \begin{equation*}
        \gL(s) = -\E_{\vx, \vy_1, \vy_2} \Big[
        \big(\prod_{u \in \gU} z_u \big) \big(s + I(s)\big)
        + \big(\prod_{u \in \gU} (1 - z_u) \big) \big(1-s + I(1-s)\big)
        \Big] 
        \,.
    \end{equation*}
    %
    Note that we wrote~$\gL$ as a function of~$s$ instead of~$\pi$ since $s$~is the only place that~$\pi$ appears. We first show that $\gL(s)$ has a unique global minimizer. To show an~$s$ is a global minimizer of~$\gL$, it suffices to show that $s$~minimizes the term inside expectation for every~$(\vx, \vy_1, \vy_2)$. Such a minimizer meets the first-order condition: 
    %
    \begin{equation*}
        \big(\prod_{u \in \gU} z_u \big)\Big(1 + (\frac{1}{s} - 1)^{|\gU|} \Big) 
        + \big(\prod_{u \in \gU} (1 - z_u) \big)\Big(-1 - (\frac{1}{1 - s} - 1)^{|\gU|} \Big) 
        = 0
        \,.
    \end{equation*}
    %
    Here, we used $\od{I}{\theta} = (\frac{1}{\theta} - 1)^{|\gU|}$. Define $w \coloneqq (\frac{1 - s}{s})^{|\gU|}$. Then, the above condition reduces to a quadratic equation in terms of~$w$:
    %
    \begin{equation*}
        1 + w - \big(\prod_{u \in \gU} (\frac{1}{z_u} - 1) \big) (1 + w^{-1}) = (1 + w^{-1})\Big[w - \prod_{u \in \gU} (\frac{1}{z_u} - 1) \Big] = 0
        \,.
    \end{equation*}
    %
    Solving for~$w$, we obtain
    %
    \begin{equation*}
        s^* = \frac{1}{1 + \big(\prod_{u \in \gU} (\frac{1}{z_u} - 1) \big)^{\frac{1}{|\gU|}}}
        \,.
    \end{equation*}
    %
    For the BT model, a direct calculation then shows
    %
    \begin{equation}
    \label{eq:_proof_s_star}
        s^*(\vx, \vy_1, \vy_2) = \sigma\Big(\frac{1}{|\gU|}\sum_{u \in \gU}\Delta r(\vx, \vy_1, \vy_2; u)\Big)
        \,.
    \end{equation}
    %
    In fact, $s^*$ is the only global minimizer of~$\gL(s)$. This is because~$\gL(s)$ is convex in~$s$:
    %
    \begin{equation*}
        \od[2]{\gL}{s} = \E_{\vx, \vy_1, \vy_2}\Big[
        \big(\prod_{u \in \gU} z_u \big) \cdot \frac{|\gU|}{s^2} (\frac{1}{s} - 1)^{|\gU| - 1} 
        + \big(\prod_{u \in \gU} (1 - z_u) \big) \cdot \frac{|\gU|}{(1 - s)^2}(\frac{1}{1 - s} - 1)^{|\gU| - 1} 
        \Big] \ge 0
        \,.
    \end{equation*}
    %
    Finally, one can verify that the policy that results in~$s^*$ (\cref{eq:_proof_s_star}) is the optimal policy~$\pi^*$. This completes the proof that the proposed loss is a consistent loss for~$\pi^*$.  
\end{proofEnd}
%%%

%%%%%%%%%%


%%%%%%%%%%
\section{Missing Proofs}

\printProofs
%%%%%%%%%%


%%%%%%%%%%
\section{PEW Surveys Experiments: Details and Additional Examples} \label{sec:pew_additional_examples} 
In this section, we expand on our PEW surveys experiment where we used polling data on key political and social issues to show: (i) how $\nbc$ rankings can differ from those maximizing the average reward; (ii) How sensitive $\nbc$ is to the sampling distribution of the pairwise preference data. 

\niparagraph{Data.} We use several Pew Research Center surveys, specifically the American Trends Panel surveys number 35, 52, 79, 83, 99,  109, 
111, 112,  114, 119, 120, 121, 126, 127, 128, 129, 130, 131, and 132. The choices are a mix of recent surveys and  those relevant to science, technology, data and AI. 
%
Each survey include questions asked to thousands of participants. We categorize participants by political party leanings to define types.
%
When processing the questions, we discard responses that are empty, as well as discarding the option "Refused". We note that discarding the option "Refused" had no effect on the results as it is not frequently chosen. 

\niparagraph{Reward Estimation.}
Although we observe how often each group selects a particular option, we don't directly observe respondents' internal rewards. To estimate this, we apply the Luce-Shepherd model~\citep{shepard_stimulus_1957, luce1959individual}:
 \begin{equation}
 \label{eq:luce-shep}
        \Pr\big(\text{option } i \text{ is chosen from } \mathcal{S}\big) =  \frac{\exp{(r(i;u))}}{\sum_{{j\in \mathcal{S}}} \exp{(r(j;u))}},
\end{equation}
where $\mathcal{S}$ is the set of options, and $r(\cdot; u)$ is the reward for type $u$. 
This allows us to estimate each option's reward (up to a constant additive term) for each type. From these estimates and observed probabilities, we compute both the expected reward and the $\nbc$ metric, where in the latter we assume the uniform probability for alternatives unless specified otherwise.

\niparagraph{Sensitivity Experiments.} In \cref{sec:sensitivity}, we estimate the sensitivity of $\nbc$ rankings to the sampling distribution of pairwise preference data by determining the minimum Total Variation (TV) distance required, if possible, to alter $\nbc$ rankings from that attained under the uniform distribution.

Recall that $\nbc$ is defined as:
\[
\nbc(\vy; \gD) \coloneqq \E_{\vy' \sim \gD(\cdot)} \big[\Pr(\vy \succ \vy' \mid \vx; r)\big].
\]
To compute this, we first estimate the reward function \(r\), then evaluate \(\Pr(\vy \succ \vy'; r)\) for all \((\vy, \vy') \in \gY \times \gY\), where \(\gY\) is the set of alternatives. 
Next, consider the feasibility of swapping the ranking induced by the uniform distribution \(\gD_U\) for alternatives \(\vy_i\) and \(\vy_j\) with a new distribution \(\gD_a\), assuming \(\nbc(\vy_i; \gD_U) > \nbc(\vy_j; \gD_U)\). This is equivalent to solving the following linear program:
\[
\begin{aligned}
     \text{minimize}& \quad \frac{1}{2} \mathbf{1}^\top \mathbf{s}, \\
    \text{subject to:} 
    & \quad \mathbf{q} > \epsilon \mathbf{1}, \\
    & \quad \mathbf{s} \geq \frac{1}{N} \mathbf{1} - \mathbf{q}, \\
    & \quad \mathbf{s} \geq \mathbf{q} - \frac{1}{N} \mathbf{1}, \\
    & \quad \mathbf{P}_{i} \mathbf{q} < \mathbf{P}_{j} \mathbf{q} + \delta, \\
    & \quad \mathbf{1}^\top \mathbf{q} = 1, \\
    & \quad \mathbf{q} > \mathbf{0}.
\end{aligned}
\]
Here, \(N = |\gY|\), and labeling the alternatives as \(\vy_1, \dots, \vy_N\), we define \(\mathbf{P}_{ij} = \Pr(\vy_i \succ \vy_j; r)\), \(q_i = \gD_a(\vy_i)\), and \(\mathbf{P}_k\) as the \(k\)-th row of \(\mathbf{P}\). The parameter \(\epsilon > 0\) ensures support for all alternatives, and \(\delta > 0\) controls the required magnitude of change in \(\nbc\) beyond what is required for the swap. We set \(\epsilon = \delta = 10^{-5}\). 

To compute the minimum TV distance, we solve the program for all pairs \((i, j)\) where \(\nbc(\vy_i; \gD_U) > \nbc(\vy_j; \gD_U)\) and record the smallest objective value. In this analysis, we group respondents by political leaning (specifically, the column \texttt{F\_PARTYSUM\_FINAL}). We also note that in this analysis we exclude survey questions with fewer than three options. 


% \niparagraph{Sensitivity Experiments.} In \cref{sec:sensitiviy}, to estimate the sensitivity of $\nbc$ to the sampling distribution of the pairwise preference data, we find the minimum TV distance required to alter $\nbc$ ranking under uniform. 
% First recall that $\nbc$ is defined as,
% \begin{equation}
%     \nbc(\vy; \gD) \coloneqq \E_{\vy' \sim \gD(\cdot)} \Big[\Pr(\vy \succ \vy' \mid \vx; r)\Big].
% \end{equation}
% Hence, to estimate it, we first estimate the reward $r$ and then compute the quantity $\Pr(\vy \succ \vy'; r) \forall (\vy, \vy') \in \gY \times \gY$, where $\gY$ is the set of alternatives. Note that the feasibility of swapping the ranking induced by the uniform distribution $\gD_U$ of options $\vy_i$ and $\vy_j$ with a new distribution $\gD_a$, and assuming $\nbc(\vy_i; \gD_u) > \nbc(\vy_j; \gD_U)$, is equivalent to the feasibility of the following linear program,
% \[
% \begin{aligned}
%     & \text{minimize} \quad \frac{1}{2} \mathbf{1}^\top \mathbf{s} \\
%     \text{subject to:} 
%     & \quad \mathbf{q} > \epsilon \mathbf{1}, \\
%     & \quad \mathbf{s} \geq \frac{1}{N} \mathbf{1} - \mathbf{q}, \\
%     & \quad \mathbf{s} \geq \mathbf{q} - \frac{1}{N} \mathbf{1}, \\
%     & \quad \mathbf{P}_{i} \mathbf{q} < \mathbf{P}_{j} \mathbf{q} + \delta, \\
%     & \quad \mathbf{1}^\top \mathbf{q} = 1, \\
%     & \quad \mathbf{q} > \mathbf{0}.
% \end{aligned}
% \]
% Where in the above,  $N = |\gY|$, and with labeling of the alternatives as $\vy_1, \dots, \vy_N$, let   $\mathbf{P}_{ij} = \Pr(\vy_i \succ \vy_j; r)$, $q_i = \gD_a(y_i)$, and $\mathbf{P}_k$ be the $k$-th row of $\mathbf{P}$. Note that $\epsilon > 0$ is used to retain the support for all alternatives, and $\delta > 0$ controls the magnitude of the change  required in $\nbc$. We set both $\delta$ and $\epsilon$ to $10^{-5}$. 
% %
% Now to find the minimum TV distance required, we find the minimum value attained by this program across all pairs $(i,j)$ such that $\nbc(\vy_i; \gD_u) > \nbc(\vy_j); \gD_u$.
% Note that in the analysis above, we use the political leaning (specifically the column \texttt{F\_PARTYSUM\_FINAL}) to group respondents.

\niparagraph{Additional Examples.} We highlight some of the examples of discrepancies that we find in some of the surveys listed above in Figures \ref{fig:example5}, \ref{fig:example6}, \ref{fig:example7}, \ref{fig:example8},  and \ref{fig:example9}. 
%
 We note though that while we indeed find a few examples of the discrepancy, $\nbc$ rankings is actually aligned with average reward rankings in most cases.  One possible explanation for this is that in many questions, the distributions of responses across types were very similar, suggesting that a homogeneous reward model would have been appropriate, causing $\nbc$ and average reward to align.


\begin{figure}[!htbp]
    \centering
    \begin{minipage}{0.6\textwidth}
        \centering
    % First Row
    \subfigure[Rewards per Type]{%
        \includegraphics[width=0.98\linewidth]{figs/PEW/rew_BIDENADM_W83.pdf}
        \label{fig:prop_rew5}
    }%
    \hfill
    \subfigure[Avg Reward vs. NBC]{%
        \includegraphics[width=0.98\linewidth]{figs/PEW/reward_vs_NBC_BIDENADM_W83.pdf}
        \label{fig:reward_vs_nbc5}
    }
    \end{minipage} 
    \begin{minipage}{0.2\textwidth}
        \subfigure[Ranking]{%
        \includegraphics[width=0.95\linewidth]{figs/PEW/ranking_comparison_BIDENADM_W83.pdf}
        \label{fig:ranking5}
    }
    \end{minipage}
    \caption{Do you think the plans and policies of the Biden administration will make the country’s response to the coronavirus outbreak:
A: A lot better; B: A little better; C: Not much different; D: A little worse; E: A lot worse}
\label{fig:example5}
\end{figure}

\begin{figure}[!htbp]
    \centering
    \begin{minipage}{0.6\textwidth}
        \centering
    % First Row
    \subfigure[Rewards per Type]{%
        \includegraphics[width=0.98\linewidth]{figs/PEW/rew_COVIDEGFP_a_W114.pdf}
        \label{fig:prop_rew6}
    }%
    \hfill
    \subfigure[Avg Reward vs. NBC]{%
        \includegraphics[width=0.98\linewidth]{figs/PEW/reward_vs_NBC_COVIDEGFP_a_W114.pdf}
        \label{fig:reward_vs_nbc6}
    }
    \end{minipage} 
    \begin{minipage}{0.2\textwidth}
        \subfigure[Ranking]{%
        \includegraphics[width=0.95\linewidth]{figs/PEW/ranking_comparison_COVIDEGFP_a_W114.pdf}
        \label{fig:ranking6}
    }
    \end{minipage}
    
    \caption{ How would you rate the job Joe Biden is doing responding to the coronavirus outbreak?    
A: Excellent; B: Good; C: Only fair; D: Poor}
\label{fig:example6}
\end{figure}

\begin{figure}[!htbp]
    \centering
    \begin{minipage}{0.6\textwidth}
        \centering
    % First Row
    \subfigure[Rewards per Type]{%
        \includegraphics[width=0.98\linewidth]{figs/PEW/rew_EVCAR2_W128.pdf}
        \label{fig:prop_rew7}
    }%
    \hfill
    \subfigure[Avg Reward vs. NBC]{%
        \includegraphics[width=0.98\linewidth]{figs/PEW/reward_vs_NBC_EVCAR2_W128.pdf}
        \label{fig:reward_vs_nbc7}
    }
    \end{minipage} 
    \begin{minipage}{0.2\textwidth}
        \subfigure[Ranking]{%
        \includegraphics[width=0.95\linewidth]{figs/PEW/ranking_comparison_EVCAR2_W128.pdf}
        \label{fig:ranking7}
    }
    \end{minipage}
    
    \caption{The next time you purchase a vehicle, how likely are you to seriously consider purchasing an electric vehicle?
A: Very likely; B: Somewhat likely; C: Not too likely; D: Not at all likely; E: I do not expect to purchase a vehicle
}
\label{fig:example7}
\end{figure}

\begin{figure}[!htbp]
    \centering
    \begin{minipage}{0.6\textwidth}
        \centering
    % First Row
    \subfigure[Rewards per Type]{%
        \includegraphics[width=0.98\linewidth]{figs/PEW/rew_FAVPOL_HARRIS_W130.pdf}
        \label{fig:prop_rew8}
    }%
    \hfill
    \subfigure[Avg Reward vs. NBC]{%
        \includegraphics[width=0.98\linewidth]{figs/PEW/reward_vs_NBC_FAVPOL_HARRIS_W130.pdf}
        \label{fig:reward_vs_nbc8}
    }
    \end{minipage} 
    \begin{minipage}{0.2\textwidth}
        \subfigure[Ranking]{%
        \includegraphics[width=0.95\linewidth]{figs/PEW/ranking_comparison_FAVPOL_HARRIS_W130.pdf}
        \label{fig:ranking8}
    }
    \end{minipage}
    
    \caption{ What is your overall opinion of    Kamala Harris?
A: Very favorable; B: Mostly favorable; C: Mostly unfavorable; D: Very unfavorable; E: Never heard of this person.
}
\label{fig:example8}
\end{figure}

\begin{figure}[!htbp]
    \centering
    \begin{minipage}{0.6\textwidth}
        \centering
    % First Row
    \subfigure[Rewards per Type]{%
        \includegraphics[width=0.98\linewidth]{figs/PEW/rew_FAVPOL_BIDEN_W130.pdf}
        \label{fig:prop_rew9}
    }%
    \hfill
    \subfigure[Avg Reward vs. NBC]{%
        \includegraphics[width=0.98\linewidth]{figs/PEW/reward_vs_NBC_FAVPOL_BIDEN_W130.pdf}
        \label{fig:reward_vs_nbc9}
    }
    \end{minipage} 
    \begin{minipage}{0.2\textwidth}
        \subfigure[Ranking]{%
        \includegraphics[width=0.95\linewidth]{figs/PEW/ranking_comparison_FAVPOL_BIDEN_W130.pdf}
        \label{fig:ranking9}
    }
    \end{minipage}
    
    \caption{ What is your overall opinion of Joe Biden?
A: Very favorable; B: Mostly favorable; C: Mostly unfavorable; D: Very unfavorable; E: Never heard of this person.
}
\label{fig:example9}
\end{figure}


% \begin{figure}[!htbp]
%     \centering
%     \begin{minipage}{0.6\textwidth}
%         \centering
%     % First Row
%     \subfigure[Rewards per Type]{%
%         \includegraphics[width=0.98\linewidth]{figs/PEW/rew_POLPROB_W127.pdf}
%         \label{fig:prop_rew10}
%     }%
%     \hfill
%     \subfigure[Avg Reward vs. NBC]{%
%         \includegraphics[width=0.98\linewidth]{figs/PEW/reward_vs_NBC_POLPROB_W127.pdf}
%         \label{fig:reward_vs_nbc10}
%     }
%     \end{minipage} 
%     \begin{minipage}{0.2\textwidth}
%         \subfigure[Ranking]{%
%         \includegraphics[width=0.95\linewidth]{figs/PEW/ranking_comparison_POLPROB_W127.pdf}
%         \label{fig:ranking10}
%     }
%     \end{minipage}
    
%     \caption{ How much of a problem do you think police violence against Black people is in the United States today? A: Major problem; B: Minor problem; C: Not a problem.
% }
% \label{fig:example10}
% \end{figure}

\newpage
\section{Semi-Synthetic Experiment: Fine-Tuning Llama-3-8B on HH-RLHF}
\label{app:llama}

\begin{figure}
    \centering
    \includegraphics[width=0.95\columnwidth]{figs/rews.pdf}
    \caption{Reward definition for three user types in semi-synthetic experiments (\cref{subsec:exp:semi-synth}) based on the length of prompt response combination. The first user type prefers long prompt response combinations, the second user type prefers short prompt response combinations, and the third user type prefers mid-length prompt response combinations. The dashed cyan line shows the average reward across the three user types.}
    \label{fig:app:rewards}
\end{figure}

\paragraph{Reward Models.}
\cref{fig:app:rewards} shows the three distinct rewards we use for the three user types along with their average.
In order to have a reliable ground-truth reward which we can rely on in evaluation, we define these rewards as functions of the number of tokens in prompt-response combinations.

\paragraph{Anonymous Dataset.}
We use prompts and response pairs from both helpfulness and harmlessness subsets of Anthropic's HH-RLHF dataset~\citep{hh-rlhf} and relabel the \textit{chosen} and \textit{rejected} responses manually.
We filter for data points in which the sum of the number of tokens in the prompt and the number of tokens in the longer response do not exceed $512$.
This leaves us with $160,800$ training and $17,104$ test data points. 
For every data point (a prompt with a pair of responses), we sample one of the three user types uniformly at random.
Given the type of user, we sample a preference based on BT~\citep{bt} to label the two alternatives.

\paragraph{Dataset with Maximum Annotator Information.}
We use prompts and response pairs from both helpfulness and harmlessness subsets of Anthropic's HH-RLHF dataset~\citep{hh-rlhf} and relabel the \textit{chosen} and \textit{rejected} responses manually.
We filter for data points in which the sum of the number of tokens in the prompt and the number of tokens in the longer response do not exceed $512$.
This leaves us with $160,800$ training and $17,104$ test data points. 
For every data point (a prompt with a pair of responses), we keep sampling BT~\citep{bt} preferences from all user types until they agree with each other.
Once the consensus is achieved, we stop sampling and use the agreed-upon preference as the label for this data point.

\paragraph{Fine-Tuning Details.}
We fine-tune Llama-3-8B~\citep{llama3modelcard} base model with LoRA~\citep{lora}.
We fine-tune for one epoch with a batch size of $2$, and use a linear learning rate schedule that starts with $3\times10^{-5}$ and decreases to zero.
We use the Adam optimizer with a weight decay of $0.001$~\citep{adamw}.
Regarding LoRA's hyper-parameters, we use the matrix rank of $r=8$, $\alpha=32$, and the dropout probability of $0.1$.

For direct alignment experiments, we use a uniform reference policy.
When ignoring heterogeneity, we do vanilla DPO over the anonymous dataset.
When modeling heterogeneity, we use the loss function we propose in \cref{prop:consistent_loss_1} over the dataset with maximum annotator information.
We use the ordinal agreement between the ground-truth average reward and the reward induced by the aligned policy as the measure of accuracy.

For the reward learning experiments, we fine-tune the Llama-3-8B as a reward model.
When ignoring heterogeneity, we assume BT and maximize the probability of the anonymous preference dataset under the learned reward model.
When modeling heterogeneity, we use the loss function in \cref{prop:consistent_loss_1} over the dataset with maximum annotator information, but replace $h(\vy_1, \vy_2; \pi)$ with the difference in rewards, i.e., $r(\vy_2) - r(\vy_1)$.
We use the ordinal agreement between the ground-truth average reward and the learned reward as the measure of accuracy.

\begin{table}
\caption{Raw Accuracy $(\%)$ in Alignment Experiments}
\label{tab:raw-aligns}
% \vskip 0.15in
\begin{center}
\begin{small}
\begin{sc}
\begin{tabular}{ccc}
\toprule
Seed & Ignoring Homogeneity & Modeling Heterogeneity \\
\midrule
0    &  65.61 & 66.63\\
1    &  67.55 & 69.55\\
2    &  68.77 & 75.21\\
3    &  68.70 & 72.04\\
4    &  66.28 & 74.95\\
\bottomrule
\end{tabular}
\end{sc}
\end{small}
\end{center}
% \vskip -0.1in
\end{table}

\begin{table}
\caption{Raw Accuracy $(\%)$ in Reward Learning Experiments}
\label{tab:raw-rews}
% \vskip 0.15in
\begin{center}
\begin{small}
\begin{sc}
\begin{tabular}{ccc}
\toprule
Seed & Ignoring Homogeneity & Modeling Heterogeneity \\
\midrule
0    &  92.33 & 95.26\\
1    &  85.38 & 92.01\\
2    &  88.46 & 94.6\\
3    &  89.69 & 93.57\\
4    &  91.94 & 93.87\\
\bottomrule
\end{tabular}
\end{sc}
\end{small}
\end{center}
% \vskip -0.1in
\end{table}

\paragraph{Detailed Results.}
We conduct every experiment with five different random seeds.
\cref{fig:semi-synthetic} shows the average, $25^\text{th}$ percentile, and $75^\text{th}$ percentile of accuracy across the five random seeds.
\cref{tab:raw-aligns,tab:raw-rews} show the raw accuracy numbers across the five random seeds.
%%%%%%%%%%