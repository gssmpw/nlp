\section{Introduction}
The complexity of drug discovery lies in understanding the intricate relationships between drugs and diseases, making the identification of potential therapeutic uses a challenging and resource-intensive endeavor. In recent years, the availability of large-scale biomedical knowledge graphs~\cite{pubmed_kg_2024, kg_ctg_2024, citation_sum_2023}, such as the Drug Repurposing Knowledge Graph (DRKG)~\cite{drkg}, has enabled significant advances in this field by linking vast amounts of biomedical entities and relationships. These structured databases capture a wealth of information across drug interactions, disease associations, and biological pathways. Yet, fully exploiting this information for drug discovery, and more specifically drug repurposing, requires efficient methods to extract meaningful insights that can guide therapeutic reasoning.


With the rapid growth of biomedical literature, particularly in repositories like PubMed, there is an unprecedented opportunity to tap into this body of knowledge to inform drug-disease relationships. However, manually curating, understanding, and drawing conclusions from this literature is impractical due to its sheer volume and the specificity of each study. Traditional knowledge graphs provide a static representation of these connections, but their utility is limited when it comes to reasoning about complex therapeutic mechanisms or the nuanced interplay between drug efficacy and disease pathology. This gap highlights the need for automated approaches that can contextualize and synthesize existing literature to support and enhance the process of drug design and discovery.


The challenges in leveraging biomedical literature for drug discovery are multifaceted. Firstly, the process of identifying and validating drug-disease pairs from a vast knowledge graph is inherently complex, given the heterogeneity of relationships and the potential noise in the data~\cite{Peng2017, Wang2020, Zhu2019}. Secondly, even when relevant pairs are identified, understanding their context and implications requires sifting through large volumes of literature, which can vary greatly in quality, relevance, and specificity~\cite{haddaway2020eight,unfoldai2023rag}. Standard retrieval-based systems often fall short in providing insightful reasoning over the retrieved content. Moreover, transforming this wealth of information into a format that can be easily used for computational models in drug design remains an unmet challenge. In essence, there is a critical need for a systematic, automated approach to extract, understand, and reason over biomedical literature in a way that is aligned with the complexities of drug-disease associations~\cite{huang2024computational,bblgat2022, gendrin2023investigating}.


To address these challenges, we propose a novel framework that integrates knowledge graph extraction, literature mining, and language model-based reasoning. Our approach begins by extracting a subset of drug-disease pairs from the DRKG, thereby forming a focused sub-knowledge graph tailored for drug discovery. For each pair in this graph, we employ a Retrieval-Augmented Generation (RAG)~\cite{lewis2020retrieval} technique which searches for relevant literature content in PubMed and Clinical Trials, leveraging the wealth of biomedical literature to contextualize each drug-disease relationship. To facilitate reasoning over these pairs, we design a set of domain-specific questions that are crucial for understanding the therapeutic potential of each relationship. An instructional template is then used to fit the pairs as well as retrieved background information for a teacher model generating responses to these questions. These responses provide rich, contextually informed insights into each pair, transforming unstructured text into structured, reasoned answers.

Building on this, we use these generated responses to train a LLaMA model~\cite{touvron2023llama2} as the local model for learning those distilled knowledge. This model is designed to understand and reason over drug-disease relationships, providing a tool that not only synthesizes biomedical knowledge but also supports hypothesis generation and decision-making in drug design. By leveraging both structured knowledge graphs and unstructured literature, our approach enables a deeper understanding of drug-disease associations, facilitating the discovery of novel therapeutic opportunities.


This paper makes the following contributions:
\begin{itemize}[left=0pt .. \parindent]
\item \textbf{RAG-Based Literature Reasoning} We design domain-specific questions and utilize a GPT-based Retrieval-Augmented Generation (RAG) pipeline to extract and reason over biomedical literature from PubMed for each drug-disease pair.

\item \textbf{Domain-Specific local LLM} We use the structured responses generated by RAG and a powerful teacher model to train a specialized LLaMA model, tailored for drug design reasoning. This model captures nuanced drug-disease relationships and supports inference-making in therapeutic research.

\item \textbf{Experimental Resources} We demonstrate the effectiveness of our approach in explainable drug recommendation, open-sourcing a new dataset and a local model to the community. 
\end{itemize}
Our work represents a step forward in the integration of knowledge graphs, literature mining, and language model-based reasoning for explainable drug recommendation.