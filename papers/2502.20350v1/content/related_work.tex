\section{Related Work}
The integration of large language models (LLMs) and knowledge graphs has significantly advanced drug discovery and repurposing. LLMs, with their capacity to process and reason over unstructured biomedical data, have shown promise in uncovering target-disease linkages and facilitating clinical trial optimization. ~\citet{zheng2024large} provide an overview of LLM applications in drug development, while \citet{gangwal2024generative} demonstrate the ability of generative transformers to design novel drug molecules using extensive biomedical datasets.

Knowledge graphs have also played a crucial role in drug repurposing. The Drug Repurposing Knowledge Graph (DRKG)~\citep{drkg}, which integrates data from multiple sources, including DrugBank~\citep{drugbank} and Hetionet~\citep{himmelstein2017systematic}, enables computational methods to identify novel drug-disease associations. Techniques such as graph neural networks (GNNs) have further improved prediction accuracy~\citep{doshi2022computational}. Additionally, hybrid approaches that combine LLMs with knowledge graphs, such as those proposed by ~\citet{fei2021enriching}, enhance reasoning over structured and unstructured biomedical data.

While these approaches emphasize either LLMs or knowledge graphs independently, our work uniquely combines these technologies with retrieval-augmented generation (RAG)~\citep{lewis2020retrieval} to incorporate structured and unstructured background information. Unlike prior works, we focus on building a framework that enables explainable drug recommendation by integrating knowledge distillation techniques to improve model interpretability and decision-making. By leveraging background retrieval from PubMed and Clinical Trials, our approach bridges the gap between LLM reasoning and biomedical knowledge, offering a robust solution for distinguishing drug candidates with supporting evidence.
