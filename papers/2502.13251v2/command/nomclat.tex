% Everything supposed to be used in equation mode?
% use in preamble: \input{macros_DAC.tex}

% Necessary packages for definitions
\usepackage{amsmath}
\usepackage{mathtools}  % for :=
\usepackage{amsfonts} 

\newcommand{\code}[1]{\small{\texttt{#1}}} 

%beginmacros

%--------- General Math Notation
\DeclareMathOperator*{\E}{\mathbb{E}}           % Expectation as a math operator
\DeclareMathOperator*{\expectation}{\mathbb{E}} % Expectation as a math operator
\renewcommand{\vec}[1]{\mathbf{#1}}             % Bold emphasis for vectors
\DeclareMathOperator*{\argmin}{arg\,min}        % Argmin
\DeclareMathOperator*{\argmax}{arg\,max}        % Argmax
\newcommand{\natnums}{{\mathbb{N}}}              % Notation for set of natural numbers
\newcommand{\realnums}{{\mathbb R}}             % Notation for set of real numbers
\newcommand{\extset}[2]{\{#1 \; | \; #2\}}      % Set of a given b. Renders {{a | b}}.    
\newcommand{\ddfrac}[2]{\frac{\displaystyle #1}{\displaystyle #2}}    % Double Display Fraction, forces large displays for everything in numerator and denominator

\newcommand{\diff}{\mathop{}\!\mathrm{d}}       % ???????
\newcommand{\transpose}[0]{{\textrm{\tiny{\sf{T}}}}} % Transpose T. Usage: $A^\transpose$
\newcommand{\norm}{{\mathcal{N}}}               % Normal distribution
\newcommand{\normaldist}{{\mathcal{N}}}         % Normal distribution
\newcommand{\iter}[2][\bocount]{{#2}^{(#1)}}    % Iteration specific instance of variable/function/anything

%-- Stochastic
\newcommand{\pdf}{\phi}                         % Standard Normal PDF
\newcommand{\cdf}{\Phi}                         % Standard Normal CDF
\newcommand{\mean}{\mu}                         % Mean
\newcommand{\stddev}{\sigma}                    % Standard Deviation
\newcommand{\variance}{\sigma^2}                % Variance
\newcommand{\noise}{\nu}                        % Noise
\newcommand{\given}[1][]{\:#1\vert\:}           % Conditional Probability "Given That" Relation, source: https://tex.stackexchange.com/a/141685/205886
\newcommand{\prob}[0]{p}                        % Probability p
\newcommand{\Prob}[0]{P}                        % Probability distribution P


%------- Notation for Configuration(s)

\newcommand{\confspace}[0]{\pmb{\Lambda}}       % Configuration space of parameters
\newcommand{\conf}[0]{\pmb{\lambda}}            % Configuration of parameters
%\newcommand{\bx}[0]{\conf}                     % Configuration of parameters
\newcommand{\hyperparam}{\lambda}               % Single hyperparameter of configuration
\newcommand{\hyperparami}[1][i]{{\hyperparam}_{#1}}   % Single hyperparameter within a hyperparameter configuration
\newcommand{\confinc}[0]{\pmb{\hat{\lambda}}}   % Incumbent configuration
%\newcommand{\confI}[1]{{\conf}^{(#1)}}      % Configuration corresponding to a given iteration -- better use \iter!
\newcommand{\confdef}[0]{{\conf}_{\text{def}}}  % Default configuration
\newcommand{\incumbent}[1][\bocount]{\iter[#1]{\confinc}}   % Incumbent configuration
\newcommand{\confincfin}{\incumbent[\bobudget]} % Final incumbent configuration (at end of run)
\newcommand{\confopt}[0]{{\conf}^*}             % Optimal configuration


%------- Notation for Cost, Risk, Loss, Performance Metric or Objective Functions

\newcommand{\loss}[0]{\mathcal{L}}              % Loss
\newcommand{\risk}{\mathcal{R}}                 % Risk
\newcommand{\riskemp}{\mathcal{R}_{\text{emp}}} % Empirical risk
\newcommand{\cost}[0]{c}                        % Cost
\newcommand{\costi}[1]{c^{(#1)}}                % Cost of instance/identifier
\newcommand{\objF}{F}                           % Family of objective functions
\newcommand{\func}[0]{f}                        % Function
\newcommand{\perfdomain}[0]{\mathbb{R}}         % Performance domain


%------- Notation for Algorithms
\newcommand{\algo}[0]{A}                        % One algorithm
\newcommand{\algos}[0]{\mathbf{A}}              % Set of algorithms    


\newcommand{\feat}[0]{\x_{\text{meta}}}                 % Meta features
\newcommand{\feats}[0]{\mathcal{X}_{\text{meta}}}       % Set of meta features


%------- Notation for Machine Learning
\newcommand{\dset}[0]{\mathcal{D}}                          % Dataset (instance)
\newcommand{\dsetmeta}[0]{\dset_{\text{meta}}}              % Dataset: meta
\newcommand{\dsettrain}[0]{\dset_{\text{train}}}            % Dataset: train
\newcommand{\dsetval}[0]{\dset_{\text{val}}}                % Dataset: val
\newcommand{\dsettest}[0]{\dset_{\text{test}}}              % Dataset: test
\newcommand{\x}[0]{\mathbf{x}}                              % Input vector x
\newcommand{\y}[0]{y}                                       % Output y
\newcommand{\xI}[1]{\mathbf{x}^{(#1)}}                      % i-th component of input
\newcommand{\yI}[1]{y^{(#1)}}                               % i-th component of output
\newcommand{\fx}{f(\mathbf{x})}                             % f(x), continuous prediction function
\newcommand{\Hspace}{\mathcal{H}}                           % hypothesis space where f is from
\newcommand{\fh}{\hat{f}}                                   % f hat, estimated prediction function


%------- Notation for Deep Learning
\newcommand{\weights}[0]{\mathbf{\theta}}                   % Weights of neural network
\newcommand{\metaweights}[0]{\phi}                          % Weights of a meta-network


%------- Notation for AutoML
\newcommand{\portfolio}[0]{\mathbf{P}}                      % Portfolio
\newcommand{\schedule}[0]{\mathcal{S}}                      % Schedule
\newcommand{\hist}[0]{\dset_{\text{hist}}}                  % History


%------- Notation for Bayesian Optimization
%-- Components
\newcommand{\GP}{\mathcal{G}}                   % Gaussian Process
\newcommand{\kernel}{\kappa}                    % Kernel
\newcommand{\acq}{u}                            % Acquisition Function
\newcommand{\constraintg}{g}                    % Constraint function
\newcommand{\surro}[0]{\hat{\cost}}             % Surrogate model

%-- Loop
\newcommand{\bocount}{t}                        % BO loop counter
\newcommand{\bobudget}{T}                       % BO loop counter max, the counter runs from 1 to this value
\newcommand{\obs}[1][\conf]{\cost({#1})}        % BO loop observation
\newcommand{\obsspace}{\mathcal{Y}}             % BO loop observation space
\newcommand{\bonextobs}{\obs[\iter{\conf}]}     % BO loop next observation
\newcommand{\bonextsample}{\iter{\conf}}        % BO loop next selected sample

%-- Dataset
\newcommand{\dsetHPOdef}{{\langle \bonextsample,\,\bonextobs \rangle}_{\bocount=1}^{\bobudget}}     % Observed cost for selected configurations during BO run


%------- Notation for Reinforcement Learning
%-- Markov Decision Process with Components
\newcommand{\MDP}[0]{M}                       % MDP (Markov Decision Process)
\newcommand{\statespace}[0]{\mathcal{S}}                % State space
\newcommand{\state}[0]{s}                               % State
\newcommand{\statet}[1][t]{\state_{#1}}                 % State at time t
\newcommand{\actionspace}[0]{\mathcal{A}}               % Action space
\newcommand{\actionrl}[0]{a}                            % Action
\newcommand{\transdomain}[0]{\mathcal{T}}               % Transition domain (probability distribution of algorithm state transitions)
\newcommand{\trans}[0]{t}                               % Transition
\newcommand{\rewards}[0]{\mathcal{R}}                   % Reward function
\newcommand{\reward}[0]{r}                              % Reward

\newcommand{\policies}[0]{\mathbf{\Pi}}                 % Policies
\newcommand{\policy}[0]{\pi}                            % Policy
\newcommand{\policyopt}[0]{{\policy}^*}                 % Optimal policy

\newcommand{\discount}[0]{\gamma}                       % Discount rate
\newcommand{\valueS}[0]{\mathcal{V}}                    % State value function
\newcommand{\valueSip}[1][\inst]{\valueS^{\policy}_{#1}} % State value function under policy pi and instance i
\newcommand{\Qfunc}[0]{\mathcal{Q}}                     % State action value function (Q)
\newcommand{\Qfuncip}[1][\inst]{\Qfunc^{\policy}_{#1}}  % State action value function (Q) under policy pi and instance i

\newcommand{\egreedy}[0]{$\epsilon$-greedy}             % $\egreedy$ 

\newcommand{\trajectory}{\tau}                          % Trajectory
\newcommand{\episode}[0]{\mathcal{E}}                  % Episode
\newcommand{\expreturn}[0]{G}                           % Expected return
\newcommand{\cutoff}[0]{\kappa}                         % Cutoff kappa


%-- Dynamic Algorithm Configuration (DAC)
\newcommand{\inst}[0]{i}                    % One instance
\newcommand{\insts}[0]{\mathcal{I}}         % Set of instances
\newcommand{\insti}[1]{i^{(#1)}}            % Numbered instance. Usage: $\insti{3}$ for 3rd instance.

\newcommand{\cMDP}[0]{\mathcal{M}}                      % cMDP (contextual MDP)
\newcommand{\cMDPdef}[0]{\cMDP \coloneqq \{\MDPi\}_{\inst \sim \insts}} % cMDP: Definition as set of MDPs per instance
\newcommand{\MDPi}[0]{\MDP_\inst}                       % MDP of one instance
\newcommand{\MDPidef}[0]{\MDPi \coloneqq (\statespace, \actionspace, \transdomaini, \rewardsi)}  % MDP: Definition as tuple for one instance

\newcommand{\transdomaini}[0]{\transdomain_\inst}       % Transitions (instance-specific)
\newcommand{\rewardsi}[0]{\rewards_\inst}               % Rewards (instance-specific)    
\newcommand{\curric}[0]{\mathbf{d}}                     % Curriculum
\newcommand{\contexti}[1][i]{c_{#1}}                    % Context of instance i

\newcommand{\configspaceDAC}[0]{\Theta}                 % Configuration space
\newcommand{\configDAC}[0]{\theta}                      % Configuration
\newcommand{\hyperparamset}{H}                          % Hyperparameter set
\newcommand{\hyperparamname}[0]{h}                      % Name of hyper-/configuration parameter

%endmacros


% Regret
% Return