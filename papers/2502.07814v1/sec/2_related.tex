\section{Related Workds}
\label{sec:related}

\textbf{Downscaling.}
Downscaling aims to transform original low-resolution maps into their smaller-scale counterparts to accurately obtain surface meteorological states at minor locations~\cite{chajaei2024machine, doll2024streamflow, wang2024spatial}. 
Based on this objective, various deep learning methods~\cite{menon2020pulse,maraun2016bias} have been employed for the downscaling of meteorological maps. 
For instance, generative adversarial network (GAN) based methods incorporate a generative adversarial loss which enables the generator to produce results indistinguishable from real maps by the discriminator~\cite{chen2020adversarial}. 
CliGAN~\cite{chaudhuri2020cligan} is a GAN-based downscaling approach for precipitation data derived from global climate models to regional-level gridded data by employing a convolutional encoder-dense decoder network.
% , thus achieving commendable results. 
% Although GAN-based methods possess the capability to accomplish downscaling, there remains room for improvement in aspects such as map variability and accuracy. 

In recent years, diffusion models have been widely utilized due to their diversity and their capability to produce high-quality downscaling maps~\cite{fei2023generative}. 
Recent investigation~\cite{bischoff2024unpaired} has centered on the downscaling of fluid flows utilizing generative models founded on diffusion maps and Latent Diffusion Models (LDM). 
% These endeavors aim to facilitate dynamical downscaling and replicate high-resolution simulations through diffusion-based methodologies. 
Moreover,~\cite{lopez2024dynamical} integrates dynamical downscaling with generative methods to augment the uncertainty estimates of downscaled climate projections, thereby demonstrating the potential of diffusion models in refining downscaling methodologies.
However, these methods overlook the coupling relationship that exists between ERA5 reanalysis data and satellite observations.
Therefore we aim to employ GridSat maps as conditions and downscale the resolution of ERA5 maps from $25km$ to $6.25km$ based on a conditional diffusion model. 
% Due to the high resolution of ERA5 maps, we have incorporated optimizable convolutional kernels to simulate the inverse process of downscaling in real-time, which leverages low-resolution maps as guidance to generate high-resolution ERA5 maps with faithful details.

\textbf{Diffusion-based SR.}
Super-resolution is an image processing technique that aims to increase the resolution of the image and enhance the clarity and intricate details within the image~\cite{tu2024taming}. 
Recent advancements in super resolution~\cite{li2022srdiff,liu2022diffusion,gao2023implicit} have seen the emergence of diffusion models as a promising approach to address various challenges in this task, such as over-smoothing, mode collapse, and computational inefficiency. 
~\cite{ho2022cascaded} illustrates the efficacy of cascaded diffusion models in producing high-fidelity images, with a particular emphasis on class-conditional ImageNet generation.
% This methodology employs a series of interconnected diffusion models, among which super-resolution diffusion models play a key role by incrementally upsampling images and incorporating finer details. 
Furthermore,~\cite{liu2024patchscaler} presented PatchScaler, a patch-independent diffusion-based method for single image super-resolution. 
% This technique optimizes the efficiency of the inference process by adaptively assigning sampling configurations based on the complexity of patch-level reconstruction. 
% Various diffusion-based models have exhibited significant promise in propelling super-resolution techniques forward across a wide range of applications, effectively mitigating issues such as over-smoothing, mode collapse, and computational inefficiency. 
% Nonetheless, in the realm of meteorology, meteorological data often exhibits pronounced specificity, such as uneven spatial distributions and intricate spatial correlations. 
% Generalized diffusion models may fail to adequately capture these particular characteristics of the data, resulting in the decline of precision and accuracy of generated outcomes.
% Consequently, we employ ERA5 data for training, integrating information from various variables within it, and utilize optimizable convolutional kernels to construct scale variations process in real-time, thereby enabling SGD to incorporate prior knowledge from the meteorological field and enhance the model's generative quality for meteorological data. 
Nonetheless, meteorological data owns its characteristics such as uneven spatial distributions and intricate spatial correlations compared with natural images.
Therefore, we aim to train a meteorology-specific diffusion model that is conditioned on satellite observations, which is able to generate high-resolution meteorological data under the guidance of current low-resolution reanalysis data and station-level observations.