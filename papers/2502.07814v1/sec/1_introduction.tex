\section{Introduction}
\label{sec:intro}
% Weather forecasting endeavors to predict future meteorological conditions by mapping current weather phenomena onto subsequent atmospheric states~\cite{mukkavilli2023ai,chen2023foundation,li2024deepphysinet}. 
%Precision in meteorological forecasts significantly assists citizens, businesses, and nations in making informed decisions regarding future societal activities. 

Precision in acquiring meteorological variable states at a small scale is pivotal to weather forecasting, which endeavors to predict forthcoming meteorological conditions by correlating current weather phenomena with subsequent atmospheric states, further aiding citizens, businesses, and nations in making informed decisions regarding future societal activities~\cite{mukkavilli2023ai,chen2023foundation,li2024deepphysinet}. 
Therefore, enhancing the accuracy of weather forecasts holds considerable significance for enterprise production and daily life, which necessitates extracting more accurate meteorological data on a finer scale~\cite{liu2024deriving,liu2024mambads}.

\begin{figure}[t]
    \centering
\includegraphics[width=\linewidth]{Figures/teaser_7.pdf}
    \vspace{-1.6cm}
    \caption{
    The difference between \textbf{(a)} the previous super-resolution (SR)-based, interpolation-based downscaling methods and \textbf{(b)} the proposed  SGD. 
    The SR-based methods attempt to model the downscaling process directly from low-resolution (LR) maps, while the interpolation-based methods solely rely on interpolation.
    % , which utilizes unlearnable parameters. 
    However, these approaches introduce systematic biases and the loss of detail when dealing with maps at a small scale of $6.25km$. 
    In contrast, SGD endeavors to commence with high-resolution (HR) maps, employing satellite observations to conditionally sample via a diffusion model. 
    Simultaneously, it simulates and constructs the inverse process of downscaling by utilizing the original LR ERA5 maps and observations from weather stations, thereby guiding the sampling results to ensure the fidelity of detailed information. }
    \label{fig:teaser}
\vspace{-0.5cm}
\end{figure}

Downscaling is an effective method for capturing details meteorological data at a finer scale, which aims to transform low-resolution maps from meteorological data such as ERA5 reanalysis dataset into corresponding high-resolution maps at a small scale to obtain more accurate and detailed meteorological data~\cite{fang2013spatial, de2018estimating, aich2024conditional}. 
As shown in \cref{fig:teaser} \textbf{(a)}, in the task of downscaling meteorological data, spatial interpolation-based methods such as linear and bilinear interpolations are common and feasible approaches. 
These methods do not use learnable parameters to model the downscaling process but instead obtain meteorological states at the detail level of small-scale maps through interpolation from grid meteorological field data. 
Consequently, achieving a highly satisfactory level of precision in the downscaling process is challenging when dealing with complex and high-resolution grid information. 
In recent years, artificial intelligence technologies have demonstrated remarkable performance in this task~\cite{sun2024deep,wang2021method}. 
For instance, SwinRDM~\cite{chen2023swinrdm} utilizes diffusion models to recover high spatial resolution and finer-scale atmospheric details. 
\cite{pozo2021dynamically} utilized a high-resolution regional ocean circulation model to dynamically downscale Earth System Models (ESMs) and produce climate projections for the California Current System. 

Existing methods rely only on the original meteorological field data at relatively coarse resolutions and directly construct and model the downscaling process~\cite{hess2024fast, zhu2024downscaling}. 
However, for the downscaling task of ERA5 maps, a coupling relationship exists between ERA5 reanalysis data and satellite observations~\cite{vaughan2024aardvark,vandal2024global}. 
% The brightness temperature from satellite observations primarily is a major factor influencing the atmospheric states. 
Satellite observations, such as brightness temperature, and humidity, are the major factors influencing the atmospheric states~\cite{mcnally2024data}. 
Consequently, compared with directly constructing a downscaling model of ERA5 maps, incorporating satellite observations as conditional inputs via cross-attention module during the downscaling process is essential to ensure that the generated high-resolution ERA5 meteorological data aligns more accurately with actual conditions~\cite{huang2019ccnet, feng2022training, qi2023fatezero}. 

% This approach is inconsistent with the inherent multiscale characteristics of meteorological variables when dealing with higher-resolution maps, such as ERA5 reanalysis data at a scale of $25\times25km$, leading to an inability to accurately capture of dispersed meteorological state~\cite{liu2024deriving}. 
% \textcolor{red}{we might need to re-identify the key challenge. Moreover, we need to claim the help of satellite observation.}

To this end, we propose the \textbf{S}atellite-observations \textbf{G}uided \textbf{D}iffusion Model (\textbf{SGD}) based on the conditional diffusion model. 
As shown in \cref{fig:teaser} \textbf{(b)}, the conditional diffusion model is utilized for the conditional generation of high-resolution ERA5 atmospheric data through the integration of brightness temperature information from GridSat satellite observations, thereby aligning more closely with real-world conditions. 
During the training process, a satellite encoder is pre-trained to extract features from the GridSat maps, which are then fused with ERA5 maps through a cross-attention module. 
The conditional denoising function is trained by the UNet module of the diffusion model. 
During the sampling process, instead of directly modeling the downscaling process, we leveraged its inverse to guide the generation of high-resolution ERA5 maps by incorporating guidance from low-resolution maps and observations from weather stations, which enables SGD to yield high-quality small-scale maps with faithful details.
A convolutional kernel $\mathcal{D}^t_{\varphi}$ with optimizable parameters ${\varphi}$ is utilized to simulate the upscaling process, which is updated in real-time across reverse steps $t$, aiming to guide the details of generated small-scale maps closer to both the original ERA5 maps and station-level observations. 
A distance function is proposed to measure the discrepancy between the two, and its gradient was used to update the mean of the samples, guiding the generation of small-scale maps with fine details based on both the low-resolution maps and station-level observations. 
The convolutional kernel parameters are updated through the gradient with respect to the parameters themselves, ensuring that the model dynamically refines its ability to construct the resolution conversion process. 
Extensive experiments have demonstrated that SGD can generate more accurate high-resolution ERA5 maps than off-the-shelf interpolation-based and diffusion-based methods. 
Ablation studies further validate the effectiveness of utilizing GridSat maps as conditioned inputs. 
Simultaneously, we have also analyzed further optimization solutions for the performance and generation accuracy of SGD through comprehensive experiments. 
Our contributions are three folds:
\begin{itemize}
\item 
We propose SGD for ERA5 meteorological states downscaling. 
By integrating GridSat satellite observations into the conditional diffusion model to capture the coupling between satellite observations and ERA5 maps, SGD is capable of generating atmospheric states that are more aligned with real-world conditions. 
\item
We employ an optimizable convolutional kernel to simulate the upscaling during the sampling process. 
By drawing upon a distance function, we incorporate guidance from both low-resolution ERA5 maps and station-level observations at a scale of $25km$ into the generation of small-scale ERA5 maps, thereby enabling the model to produce high-quality ERA5 maps at $6.25km$ or even smaller scale that exhibit faithful details. 
\end{itemize}

\begin{figure*}[t]
    \centering   
    \includegraphics[width=\linewidth]{Figures/main_fig_13.pdf}
    \vspace{-1.1cm}
    \caption{
\textbf{(a)} Overview of the conditional diffusion-based downscaling model. 
\textbf{(b)} The satellite observations, GridSat, undergo a feature extraction encoder to serve as the conditional input.
% , aiming to obtain more realistic atmospheric variables. 
GridSat is then fused with ERA5 via cross-attention to train the conditional diffusion model.
\textbf{(c)} During the sampling process, low-resolution ERA5 maps are utilized to guide the generation of high-resolution maps. 
This is achieved through convolutional kernels $\mathcal{D}$ with optimizable parameters $\varphi$ that facilitate resolution transformation, while a distance function $\mathcal{L}$ is introduced to quantify the disparity between the upscaled convolution-generated map $\mathcal{D}(\tilde{x}_0)$ and the original ERA5 map $z_t$, where $\tilde{x}_0$ refers to the real-time estimation of the generated maps. 
% \textcolor{red}{maybe we can add some symbols to make the illustration correspond to the figure.}
The gradient of the distance function with respect to $\tilde{x}_0$ is utilized to update the mean value used in sampling. 
% \textcolor{red}{x0 should be mentioned before} 
Simultaneously, the gradient of the distance function concerning the convolutional kernel parameters is employed to update these parameters, thereby enabling a more accurate simulation of the inverse process of downscaling.
}
    \label{fig:main_fig}
\vspace{-0.5cm}
\end{figure*}
