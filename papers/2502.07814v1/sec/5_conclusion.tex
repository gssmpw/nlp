\section{Conclusion}
% In this paper, we propose SGD, a conditional diffusion model based robust downscaling method, which enables the downscaling of ERA5 maps to arbitrary resolutions for the extraction of intricate meteorological states. 
We propose SGD, a conditional diffusion model for robust downscaling, which enables the downscaling of ERA5 maps to arbitrary resolutions for the extraction of intricate meteorological states. 
Specifically, considering that ERA5 data is derived from satellite observation data and the brightness temperature data from satellite observations significantly influences the meteorological states within ERA5, SGD employs GridSat satellite observation maps as conditions to generate downscaled ERA5 maps that more accurately align with actual meteorological states. 
During the sampling process, SGD utilizes the generative prior within the conditional diffusion model, thereby incorporating guidance both from low-resolution ERA5 maps and station-level observations through the optimizable kernel and distance functions. 
Experiments demonstrate that SGD is capable of generating atmospheric states in high-resolution maps with more ideal accuracy and faithful details than various off-the-shelf methods
SGD also showcases its capability of producing high-quality ERA5 maps at a small scale of $6.25km$. 

\noindent \textbf{Limitations and Future Work.}
The training data in SGD consists of ERA5 and GridSat.
However, SGD serves as a versatile framework that could incorporate more modalities such as other reanalysis data, observations from polar-orbiting satellites, sounding and radar data.
Once these systematic data are all integrated into SGD, more accurate weather conditions near the surface can be achieved.
% a fast and robust data assimilation system can be achieved.
% The limitation of SGD is that the training patterns are mainly from ERA5 with $25km$ and GridSat with $9km$