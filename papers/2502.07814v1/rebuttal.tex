\documentclass[10pt,twocolumn,letterpaper]{article}
\usepackage[rebuttal]{cvpr}

% Include other packages here, before hyperref.
\usepackage{graphicx}
\usepackage{amsmath}
\usepackage{amssymb}
\usepackage{booktabs}
\usepackage{xcolor}

\usepackage{booktabs}
\usepackage{multirow}
\usepackage{floatrow}
\usepackage{bbding}
\usepackage{makecell}
\usepackage{caption}
\floatsetup[table]{capposition=top}


% Import additional packages in the preamble file, before hyperref
%
% --- inline annotations
%
\newcommand{\red}[1]{{\color{red}#1}}
\newcommand{\todo}[1]{{\color{red}#1}}
\newcommand{\TODO}[1]{\textbf{\color{red}[TODO: #1]}}
% --- disable by uncommenting  
% \renewcommand{\TODO}[1]{}
% \renewcommand{\todo}[1]{#1}



\newcommand{\VLM}{LVLM\xspace} 
\newcommand{\ours}{PeKit\xspace}
\newcommand{\yollava}{Yo’LLaVA\xspace}

\newcommand{\thisismy}{This-Is-My-Img\xspace}
\newcommand{\myparagraph}[1]{\noindent\textbf{#1}}
\newcommand{\vdoro}[1]{{\color[rgb]{0.4, 0.18, 0.78} {[V] #1}}}
% --- disable by uncommenting  
% \renewcommand{\TODO}[1]{}
% \renewcommand{\todo}[1]{#1}
\usepackage{slashbox}
% Vectors
\newcommand{\bB}{\mathcal{B}}
\newcommand{\bw}{\mathbf{w}}
\newcommand{\bs}{\mathbf{s}}
\newcommand{\bo}{\mathbf{o}}
\newcommand{\bn}{\mathbf{n}}
\newcommand{\bc}{\mathbf{c}}
\newcommand{\bp}{\mathbf{p}}
\newcommand{\bS}{\mathbf{S}}
\newcommand{\bk}{\mathbf{k}}
\newcommand{\bmu}{\boldsymbol{\mu}}
\newcommand{\bx}{\mathbf{x}}
\newcommand{\bg}{\mathbf{g}}
\newcommand{\be}{\mathbf{e}}
\newcommand{\bX}{\mathbf{X}}
\newcommand{\by}{\mathbf{y}}
\newcommand{\bv}{\mathbf{v}}
\newcommand{\bz}{\mathbf{z}}
\newcommand{\bq}{\mathbf{q}}
\newcommand{\bff}{\mathbf{f}}
\newcommand{\bu}{\mathbf{u}}
\newcommand{\bh}{\mathbf{h}}
\newcommand{\bb}{\mathbf{b}}

\newcommand{\rone}{\textcolor{green}{R1}}
\newcommand{\rtwo}{\textcolor{orange}{R2}}
\newcommand{\rthree}{\textcolor{red}{R3}}
\usepackage{amsmath}
%\usepackage{arydshln}
\DeclareMathOperator{\similarity}{sim}
\DeclareMathOperator{\AvgPool}{AvgPool}

\newcommand{\argmax}{\mathop{\mathrm{argmax}}}     



% If you comment hyperref and then uncomment it, you should delete
% egpaper.aux before re-running latex.  (Or just hit 'q' on the first latex
% run, let it finish, and you should be clear).
\definecolor{cvprblue}{rgb}{0.21,0.49,0.74}
\usepackage[pagebackref,breaklinks,colorlinks,allcolors=cvprblue]{hyperref}

% If you wish to avoid re-using figure, table, and equation numbers from
% the main paper, please uncomment the following and change the numbers
% appropriately.
%\setcounter{figure}{2}
%\setcounter{table}{1}
%\setcounter{equation}{2}

% If you wish to avoid re-using reference numbers from the main paper,
% please uncomment the following and change the counter value to the
% number of references you have in the main paper (here, 100).
%\makeatletter
%\apptocmd{\thebibliography}{\global\c@NAT@ctr 100\relax}{}{}
%\makeatother

%%%%%%%%% PAPER ID  - PLEASE UPDATE
\def\paperID{3251} % *** Enter the Paper ID here
\def\confName{CVPR}
\def\confYear{2025}


\begin{document}

%%%%%%%%% TITLE - PLEASE UPDATE
% \title{Satellite Observations Guided Diffusion Model for Accurate Meteorological States at Arbitrary Resolution}  % **** Enter the paper title here

% \maketitle
\thispagestyle{empty}
\appendix

%%%%%%%%% BODY TEXT - ENTER YOUR RESPONSE BELOW
We thank the reviewers for approvals of our contributions:
1) \emph{\textcolor{blue}{An innovative methodological}} (\textcolor[rgb]{1,0,1}{R-yyPm}, \textcolor[rgb]{0.847,0.565,0.055}{R-PSDu}).
2) \emph{\textcolor{blue}{Outstanding research significance}} (\textcolor[rgb]{1,0,1}{R-yyPm}). 
3) \emph{\textcolor{blue}{An effective method}} (\textcolor[rgb]{1,0,1}{R-yyPm}, \textcolor[rgb]{0,0.690,0.706}{R-gS3c}).
4) \emph{\textcolor{blue}{A rigorous experiment}} (\textcolor[rgb]{1,0,1}{R-yyPm}) with sufficiently detailed comparisons (\textcolor[rgb]{0.847,0.565,0.055}{R-PSDu}).
5) \emph{\textcolor{blue}{Effective feature fusion and feature extraction modules}} (\textcolor[rgb]{0,0.690,0.706}{R-gS3c}).
6) \emph{\textcolor{blue}{Better results}} (\textcolor[rgb]{1,0,1}{R-yyPm}) with detailed weather states and accurate real-world condition (\textcolor[rgb]{0.847,0.565,0.055}{R-PSDu}).
Below please find the point-to-point response. 

\noindent \textbf{[\textcolor[rgb]{1,0,1}{MajorQ1-yyPm}][\textcolor[rgb]{0.847,0.565,0.055}{MajorQ3-PSDu}] Running time and resource consumption:} 
\cref{time} shows the running time and resource consumption of SGD during the training and the sampling process. 
To enhance the inference efficiency, we have also tested our SGD on 50-step DDIM sampling to generate HR ERA5 maps within one minute, making it a feasible approach for practical use.
The efficiency and performance of DDIM sampling will be added to the revision.
% All downscaling results of SGD mentioned in the main text are derived from DDPM sampling, yet SGD has the capability to generate downscaling ERA5 maps more rapidly through the utilization of DDIM. 

\noindent \textbf{[\textcolor[rgb]{1,0,1}{MajorQ2-yyPm}][\textcolor[rgb]{0,0.690,0.706}{Sugg.2-gS3c}][\textcolor[rgb]{0.847,0.565,0.055}{Sugg.-PSDu}] Additional comparisons:}
We conducted a comparative analysis of SGD against SwinRDM, HyperDS (MambaDS is not publicly available), and Reference-Based Super-Resolution methods.
% MambaDS is not publicly available, thus we choose HyperDS for comparison.
% in downscaling ERA5 maps. 
\cref{tab:compare} shows that SGD, when GridSat is employed as the condition, exhibits advantages in variables $T_{2m}$ and $MSL$ which is associated with brightness temperature data. 
Furthermore, it outperforms the majority of these methods across other variables as well. 

\noindent \textbf{[\textcolor[rgb]{1,0,1}{MajorQ3-yyPm}] ``Sampling while training'':}
% The training and sampling processes of SGD are distinct. 
% In SGD, we also follow the training and then sampling paradigm.
SGD is a training and then sampling paradigm.
$\mathcal{D}$ refers to the upscaling kernel employed in the sampling process, rather than the diffusion model. 
$\mathcal{D}$ is utilized to simulate the resolution transformation process, aiming to ensure that the model's output, after upscaling $\mathcal{D}$, closely approximates the low-resolution ERA5 maps
% During the training process, we incorporate the GridSat map as a condition to facilitate the learning of the coupling relationship between the GridSat and ERA5 maps. 
In the sampling phase, the well-trained conditional diffusion model $\epsilon_\theta$ is utilized to generate high-resolution ERA5 images, with the model parameters remaining unchanged.
To enhance the discriminability of these symbols, we will make modifications in the revision.
% During this process, low-resolution ERA5 data is integrated as guidance, and an upscaling kernel is utilized to mimic the resolution transformation process, aiming to ensure that the model's output, after upscaling, closely approximates the low-resolution ERA5 maps, thereby yielding high-resolution maps with enhanced fidelity and finer details. 
% To ensure more accurate simulations in the subsequent reverse steps, it is essential to update the kernel parameters based on gradient information, enabling an adaptive simulation and learning of the upscaling process for each ERA5 map generation.


% Having scrutinized the sentence structures and formatting typo carefully, we have refined them to ensure greater clarity. The revised version will incorporate these enhancements. 

\noindent \textbf{[\textcolor[rgb]{0,0.690,0.706}{Sugg.3,Sugg.4-gS3c}] Ablation studies on the relationship of variables:}
% Ablation study is carried out to investigate the impact of each GridSat variable on the variables of ERA5. 
% As shown in \cref{table:gridsat}, when only the brightness temperature variables from GridSat (IrWin\_Cdr, IrWin\_VZA\_Adj) are employed as conditions, the performance indicators of the variables in SGD are relatively superior. 
When only the brightness temperature variables from GridSat (IrWin\_Cdr or IrWin\_VZA\_Adj) are employed as the condition, a satisfactory performance can be obtained, demonstrating that brightness temperature is an important condition for ERA5 maps downscaling (\cref{table:gridsat}).
All variables from GridSat could guide the SGD to yield higher-quality HR ERA5 maps.
% This shows that the brightness temperature data from GridSat is capable of guiding SGD to yield higher-quality downscaled maps.
% \cref{table:era5} indicates that incorporating GridSat as conditions and utilizing only a single ERA5 variable, the introduction of GridSat yields the most significant enhancement for the ERA5 temperature variable ($T_{2m}$).
When incorporating GridSat as conditions and utilizing only a single LR ERA5 variable as guidance to generate the single HR ERA5 variable, the introduction of GridSat yields the most significant enhancement for the ERA5 temperature variable ($T_{2m}$) (\cref{table:era5}).
% , compared to the original ERA5 maps
Considering the correlation between sea-level pressure and brightness temperature, the incorporation of GridSat as conditions for generating the single $MSL$ variable also contributes to enhancing the $MSL$ downscaling.

\noindent \textbf{[\textcolor[rgb]{0,0.690,0.706}{MinorQ1-gS3c}][\textcolor[rgb]{0.847,0.565,0.055}{MajorQ2-PSDu}] 
Coupling relationship:}
% Lack the explanation of coupling relationship and the rationale for employing cross attention:}  
ERA5, as a reanalysis dataset, is derived from satellite observations and other data.
Among them, the brightness temperature from satellite observations provides temperature variations being the primary driver of atmospheric changes. 
Therefore, the high-quality brightness temperature data from GridSat play a crucial role in the ERA5 reanalysis process.
Moreover, atmospheric state variables influence observations through radiative processes, while observational data, in turn, feed back into ERA5 via data assimilation systems. 
These analyses, corresponding experiments, and references will be added to the revision.
% We will incorporate and refine this in the revised version.

\noindent \textbf{[\textcolor[rgb]{0.847,0.565,0.055}{MajorQ1-PSDu}] Details about data:} 
The dimensions of ERA5 maps are (N, 4, 720, 1440), whereas those of the GridSat maps are (N, 3, 2000, 5143). 
Our SGD could generate accurate meteorological states at arbitrary resolution.
% Our SGD possesses a theoretically limitless downscaling capability.
For the $4\times$ downscaling scenario, each channel of the generated ERA5 maps has a dimension of (2880, 5760). 
Detailed descriptions of the input and output data, especially on the pre-trained encoder and Satellite Conditioning Mechanisms, will be carefully added to the revised paper.
% Detailed elaborations on this can be found in Section 4.2 and Figure 2 of the main text. 
% The comprehensive description of the pre-trained encoder module, including the data dimensions at each step, is provided in Section C and Figure 1 in the Appendix. 
% The GridSat maps undergo three successive 3×3 convolutional operations with channel progression of 3, 32, and 64, respectively. 
% Ultimately, the processed GridSat data is integrated into the cross attention module. 

\noindent \textbf{[\textcolor[rgb]{0.847,0.565,0.055}{MajorQ2-PSDu}] Cross-attention:}
It is revealed in Stable Diffusion(CVPR'22) and FateZero(ICLR'23) that cross-attention facilitates feature fusion to ensure that target predictions better align with the characteristics and correlations of the conditions. 
Considering the contribution of satellite observations in ERA5 reanalysis, SGD employs cross-attention to learn the coupling relationships between target ERA5 maps and conditional GridSat maps, enabling the integration of satellite observations to generate meteorological data that more accurately reflect real-world atmosphere.
% The main observation of satellites is brightness temperature, with temperature variations being the primary driver of atmospheric changes. 
% Meanwhile, GridSat data furnishes comprehensive insights into the radiative properties of Earth's surface and atmosphere. 
% These datasets, though overlapping in certain aspects, are complementary. 
% Consequently, GridSat observations can validate and enhance the accuracy of ERA5 simulation data. 
% The utilization of cross-attention facilitates the model's ability to learn specified conditional features more effectively and precisely through feature fusion. 
% In traditional attention mechanisms, the QKV components are derived from the same sequence. 
% However, in cross-attention, the KV components originate from a distinct sequence. 
% By computing the relevance between each position in one sequence and another, it weights and sums the feature representations of the conditional sequence, thereby generating a contextual vector for each position in the target sequence. 
% SGD employs cross-attention to learn the coupling relationships between target ERA5 maps and conditional GridSat maps, enabling the integration of satellite observations to generate meteorological data that more accurately reflect real-world conditions.

\noindent \textbf{[\textcolor[rgb]{1,0,1}{MinorQ1-yyPm}][\textcolor[rgb]{0.847,0.565,0.055}{MinorQ1-PSDu}] Complex sentence \& typos:} 
We will simplify the sentences and fix formatting typos carefully to ensure greater clarity in revision.
% \textbf{All minor weaknesses will be meticulously addressed in the revision.} 
% \textbf{Thanks again for your insightful feedback.}


\begin{table}[t]\small
\centering
\caption{The running time and resource consumption of SGD.}
% \setlength{\abovecaptionskip}{0cm}
\vspace{-0.4cm}
\resizebox{\textwidth}{!}{
\begin{tabular}{c|c c c}
    \toprule[1pt]
     Mode&SGD Training&SGD Sampling&SGD Sampling with DDIM\\
    \midrule
    \multirow{1}{*}{Running Time}
    &48h&6min&1min\\
    \midrule
    \multirow{1}{*}{Resource Consumption}
    &$5\times10^4$MiB&$1.8\times10^4$MiB&$1.6\times10^4$MiB\\
    \bottomrule[1pt]

  \end{tabular}
  }
 \label{time}
\vspace{-0.4cm}
\end{table}
% \vspace{-0.3cm}


\begin{table}[t]\small
  \centering
  % \setlength{\abovecaptionskip}{0cm}
  % \caption{Extra comparison with SwinRDM, Reference-based Super-Resolution methods and latest methods.}
  \caption{Additional comparisons.}
  \vspace{-0.4cm}
  \resizebox{\textwidth}{!}
  {
  \begin{tabular}{c|c c |c c |c c |c c }
    \toprule[1pt]
     \multirow{2}{*}{Methods}&\multicolumn{2}{c|}{$U_{10}$}&\multicolumn{2}{c|}{$V_{10}$}&\multicolumn{2}{c|}{$T_{2m}$}&\multicolumn{2}{c}{$MSL$}\\
     \cmidrule(lr){2-9}
     
    &MSE&MAE&MSE&MAE&MSE&MAE&MSE&MAE\\
    \midrule
SwinRDM (AAAI'23)& 53.18 & 5.95 & \textbf{38.51} & \textbf{4.95} & 216.27 & 11.39 & 470.06 & 15.78 \\
Ref-SR (ECCV'22)& 62.72 & 6.15 & 43.12 & 5.17 & 195.42 & 11.02 & 395.42 & 15.10 \\
$C^2$-Matching(CVPR'21)& 65.12 & 6.02 & 44.57 & 5.41 & 200.17 & 11.36 & 410.72 & 15.32 \\
HyperDS (TGRS'24)& 53.72 & 6.01 & 41.37 & 5.26 & 191.83 & 10.87 & 384.72 & 14.72 \\
\midrule
SGD& \textbf{51.65} & \textbf{5.84} & 39.82 & 5.05 & \textbf{187.69} & \textbf{10.63} & \textbf{374.39} & \textbf{14.49} \\
    \bottomrule[1pt]

  \end{tabular}
  }
\label{tab:compare}
\vspace{-0.4cm}
\end{table}

\begin{table}[t]\small
  \centering
  \caption{Ablation study employing a single GridSat variable.}
  % \setlength{\abovecaptionskip}{0cm}
  \vspace{-0.4cm}
  \label{table:gridsat}
  \resizebox{\textwidth}{!}{
  \begin{tabular}{c|c c |c c |c c |c c }
    \toprule[1pt]
     \multirow{2}{*}{Methods}&\multicolumn{2}{c|}{$U_{10}$}&\multicolumn{2}{c|}{$V_{10}$}&\multicolumn{2}{c|}{$T_{2m}$}&\multicolumn{2}{c}{$MSL$}\\
     \cmidrule(lr){2-9}
     
    &MSE&MAE&MSE&MAE&MSE&MAE&MSE&MAE\\
    \midrule
Only IrWin\_Cdr  & 55.74 & 6.07 & 46.11 & 5.76 & 191.42 & 11.04 & 412.11 & 16.70 \\
Only IrWin\_VZA\_Adj  & 56.08 & 5.99 & 47.53 & 5.84 & 194.05 & 10.87 & 405.74 & 16.55 \\
Only IrWVP  & 60.42 & 6.74 & 55.08 & 6.10 & 214.07 & 12.07 & 424.17 & 17.52 \\
\midrule
All Variables in GridSat& \textbf{51.65} & \textbf{5.84} & \textbf{39.82} & \textbf{5.05} & \textbf{187.69} & \textbf{10.63} & \textbf{374.39} & \textbf{14.49} \\
    \bottomrule[1pt]

  \end{tabular}
  }
\vspace{-0.4cm}
\end{table}


\begin{table}[t]\small
\centering
\caption{Ablation study employing a single ERA5 variable.}
% \setlength{\abovecaptionskip}{0cm}
\vspace{-0.4cm}
\resizebox{\textwidth}{!}{
\begin{tabular}{c|c c | c | c c |c|c c | c | c c}
    \toprule[1pt]
     \multirow{2}{*}{Methods
     } &\multicolumn{2}{c|}{$U_{10}$} & \multirow{2}{*}{Methods}&\multicolumn{2}{c|}{$V_{10}$}&\multirow{2}{*}{Methods
     } &\multicolumn{2}{c|}{$T_{2m}$} & \multirow{2}{*}{Methods}&\multicolumn{2}{c}{$MSL$}\\
     \cmidrule(lr){2-3}
     \cmidrule(lr){5-6}
     \cmidrule(lr){8-9}
     \cmidrule(lr){11-12}

    & MSE & MAE    &&   MSE & MAE && MSE & MAE    &&   MSE & MAE  \\
    \midrule
    ERA5 $1^{\circ} $  & 53.18 & 5.95 & Era5 $1^{\circ} $ & 38.51 & 4.95 &  ERA5 $1^{\circ} $   & 216.27 & 11.39 & Era5 $1^{\circ} $ & 470.06 & 15.78 \\
    Only $U_{10}$   & 56.72 & 6.45 & Only $V_{10}$  & 47.28 & 5.84 & Only $T_{2m}$   & 194.15 & 10.71 & Only $MSL$  & 398.05 & 14.90   \\


    \bottomrule[1pt]


  \end{tabular}
  }
 % \label{Tab3}
\label{table:era5}
\vspace{-0.7cm}
\end{table}

\end{document}
