\section{Station observation-guided downscaling bias with stations in Weather2K}
In this section, we endeavor to integrate the MSE loss from ERA5 LR maps and MAE loss from the observation stations in Weather5K within the distance function utilized in the sampling process. 
Subsequently, we evaluated the high-resolution ERA5 maps derived from SGD with this setting across all stations within the Weather2K dataset, thereby further assessing the efficacy of the guided sampling and the accuracy of the downscaling results. 

Weather2K dataset~\cite{zhu2023weather2k} is a benchmark dataset that aims to address the shortcomings of existing weather forecasting datasets in terms of real-time relevance, reliability, and diversity, as well as the critical impediment posed by data quality. 
The data is available from January 2017 to August 2021. 
It encompasses the meteorological data from 2130 ground weather stations across 40896 time steps, with each data incorporates 3 position variables and 20 meteorological variables. 

Specifically, we incorporate the MAE loss between the generated HR ERA5 maps and station observations from the Weather5K dataset~\cite{han2024weather} with equal weights into our distance function to measure bias. 
Subsequently, we calculate the biases between the downscaling results obtained under this setting with the meteorological data at the stations from the Weather2K dataset. 
The evaluation metrics we employed are the MSE and MAE loss of the variable $T_{2M}$. 

We compared our results with those of interpolation-based and diffusion-based methods using the same metrics. 
As shown in ~\cref{tab:app_ablation}, the discrepancy between ERA5+station guided SGD and Weather2K stations is smaller, indicating that using ERA5 and Weather5K with equal weights as the distance function yields more ideal downscaling results for stations beyond Weather5K. 

~\cref{fig:app_ablation} illustrates the differences in downscaling results among various methods at parts of the stations within Weather2K, with darker colors indicating smaller discrepancies at the stations. 
In terms of the bias between the downscaled results at the station locations in the image and the actual observations,  SGD with mixed guidance downscaling results has less extreme bias stations, which is symbolized as yellow-labeled stations. 
Moreover, the overall station coloration appears deeper. 
This suggests that utilizing weather5k as guidance can enhance the model's performance in downscaling at the local scale, aligning more closely with the real conditions. 


\begin{figure*}[t]
    \centering
\includegraphics[width=\linewidth]{Figures/app_ablation_2.pdf}
    % \vspace{-1.6cm}
    \caption{Visualization comparison of SGD downscaling to station-scale employing various distance function, where the coloration of each Weather2k observation station signifies the MAE loss between the downscaled results and their corresponding observed values. }
    \label{fig:app_ablation}
% \vspace{-0.5cm}
\end{figure*}

\begin{table*}[t]
\centering
\caption{Station-level downscaling results for $T_{2M}$, which utilize the stations from Weather2k to assess the bias between the downscaling maps and Weather2k station observation values. 
ERA5 guided and ERA5 + station guided SGD respectively denote the SGD models that employ the MSE loss between the generated maps and ERA5 maps as the sole distance function, and the SGD model that integrates the Weather5k station observations into its distance function. }
% \vspace{-0.3cm}
\begin{tabular}{c| c| c c c c c}
    \toprule[1pt]
     Variable&Metrics&ERA5 1$^\circ$&ERA5 0.25$^\circ$&GDP &ERA5 Guided SGD&ERA5+Station Guided SGD\\
    \midrule
    \multirow{2}{*}{$T_{2M}$}
    &MSE& 17.51 & 17.80 & 18.87 & 18.08 & \textbf{15.61}\\
    &MAE& 407.81 & 420.38 & 466.00 & 431.33 & \textbf{355.31}\\
    \bottomrule[1pt]
  \end{tabular}

 \label{tab:app_ablation}
% \vspace{-0.3cm}
\end{table*}