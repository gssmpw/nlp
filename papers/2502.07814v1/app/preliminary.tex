\section{Preliminary}
\label{app:preliminary}
Unconditional diffusion model, proposed by~\cite{ho2020denoising}, is a powerful generative model composed of a forward process and a reverse process. 
The former aims to gradually introduce random Gaussian noise into the original images over $T$ diffusion steps, ultimately resulting in pure Gaussian noise $x_T\sim \mathcal{N}(0,I)$. 
The latter, being the reverse of the forward process, intends to denoise and sample the generated images from the pure Gaussian noise through a pre-trained noise estimation network. 

The forward process is a Markov chain without learnable parameters. 
The denoising method for each step is defined by the following equation, where $\beta_t$ refers to the variance of the forward process, which is experimentally set as a hyperparameter solely dependent on the diffusion steps $t$. 
\begin{align}
q (x_t|x_{t-1})=\mathcal{N}(x_t;\sqrt{1-\beta_t}x_{t-1},\beta_tI).
\end{align}

For each steps in the reverse process $p(x_{t-1}|x_t)=\mathcal{N}(x_{t-1};\mu_\theta(x_t,t),\Sigma_\theta I)$, the mean of the distribution is hard to compute directly as the forward process. 
Consequently, we necessitate the utilization of a neural network with parameter $\theta$ to estimate the noise inherent within the image $x_t$. 
By employing Bayes theorem, we can express the mean and variance of the reverse process as follows:
\begin{align}
\mu_\theta(x_t,t)&=\frac{1}{\sqrt{\alpha_t}}(x_t-\frac{\beta_t}{\sqrt{1-\bar{\alpha}_t}}\epsilon_\theta(x_t,t))\\
\Sigma_\theta(x_t)&=\frac{1-\bar{\alpha}_{t-1} }{1-\bar{\alpha}_{t}}\beta_t,
\end{align}

Among them, $\epsilon_\theta(x_t,t)$ represents the noise estimation function, which is pre-trained by utilizing the low-resolution ERA5 maps. 
It performs real-time estimation and simulation of the noise contained within the maps, thereby enabling denoising to sample $x_{t-1}$. 
The unconditional diffusion model is trained utilizing maximum likelihood estimation, with the objective for each training iteration defined as follows: 
\begin{align}
E_{\epsilon \sim \mathcal{N}(0,I),t\sim [0,T]}[\left \|\epsilon - \epsilon_\theta(x_t,t) \right \|^2 ].
\end{align}