%%%%%%%%%%%%%%%%%%%%%%%%%%%%%%%%%%%%%%%%%%%%%%%%%%%%%%%%%%%%%%%%%%%%%%%%

%%% LaTeX Template for AAMAS-2025 (based on sample-sigconf.tex)
%%% Prepared by the AAMAS-2025 Program Chairs based on the version from AAMAS-2025. 

%%%%%%%%%%%%%%%%%%%%%%%%%%%%%%%%%%%%%%%%%%%%%%%%%%%%%%%%%%%%%%%%%%%%%%%%

%%% Start your document with the \documentclass command.


%%% == IMPORTANT ==
%%% Use the first variant below for the final paper (including auithor information).
%%% Use the second variant below to anonymize your submission (no authoir information shown).
%%% For further information on anonymity and double-blind reviewing, 
%%% please consult the call for paper information
%%% https://aamas2025.org/index.php/conference/calls/submission-instructions-main-technical-track/

%%%% For anonymized submission, use this
%\documentclass[sigconf,anonymous]{aamas} 

%%%% For camera-ready, use this
\documentclass[sigconf]{aamas} 


%%% Load required packages here (note that many are included already).

\usepackage{balance} % for balancing columns on the final page


% newly loaded packages
\usepackage{amstext}
\usepackage{amsmath}
\usepackage{amsthm}
\usepackage{amsfonts}
\usepackage{bm}
%\usepackage{bbm}
\usepackage{multirow}
\usepackage{multicol}
\usepackage{array}
\usepackage{enumitem}
\usepackage{algorithm}
\usepackage{algpseudocode}
\renewcommand{\algorithmicrequire}{\textbf{Input:}}
\renewcommand{\algorithmicensure}{\textbf{Output:}}


%%%%%%%%%%%%%%%%%%%%%%%%%%%%%%%%%%%%%%%%%%%%%%%%%%%%%%%%%%%%%%%%%%%%%%%%

%%% AAMAS-2025 copyright block (do not change!)

\makeatletter
\gdef\@copyrightpermission{
  \begin{minipage}{0.2\columnwidth}
   \href{https://creativecommons.org/licenses/by/4.0/}{\includegraphics[width=0.90\textwidth]{by}}
  \end{minipage}\hfill
  \begin{minipage}{0.8\columnwidth}
   \href{https://creativecommons.org/licenses/by/4.0/}{This work is licensed under a Creative Commons Attribution International 4.0 License.}
  \end{minipage}
  \vspace{5pt}
}
\makeatother

\setcopyright{ifaamas}
\acmConference[AAMAS '25]{Proc.\@ of the 24th International Conference
on Autonomous Agents and Multiagent Systems (AAMAS 2025)}{May 19 -- 23, 2025}
{Detroit, Michigan, USA}{Y.~Vorobeychik, S.~Das, A.~Nowé  (eds.)}
\copyrightyear{2025}
\acmYear{2025}
\acmDOI{}
\acmPrice{}
\acmISBN{}

%%%%%%%%%%%%%%%%%%%%%%%%%%%%%%%%%%%%%%%%%%%%%%%%%%%%%%%%%%%%%%%%%%%%%%%%

%%% == IMPORTANT ==
%%% Use this command to specify your OpenReview submission number.
%%% In anonymous mode, it will be printed on the first page.

\acmSubmissionID{714}

%%% Use this command to specify the title of your paper.

\title[AAMAS-2025 Formatting Instructions]{Single-Agent Planning in a Multi-Agent System: \\A Unified Framework for Type-Based Planners}

% Add the subtitle below for an extended abstract
%\subtitle{Extended Abstract}

%%% Provide names, affiliations, and email addresses for all authors.

\author{Fengming Zhu}
\affiliation{
  \institution{Hong Kong University of Science and Technology}
  \city{Hong Kong SAR}
  \country{China}}
\email{fzhuae@connect.ust.hk}

\author{Fangzhen Lin}
\affiliation{
  \institution{Hong Kong University of Science and Technology}
  \city{Hong Kong SAR}
  \country{China}}
\email{flin@cs.ust.hk}

%%% Use this environment to specify a short abstract for your paper.

\begin{abstract}
We consider a general problem where an agent is in a multi-agent environment and must plan for herself without any prior information about her opponents.
At each moment, this pivotal agent is faced with a trade-off between exploiting her currently accumulated information about the other agents and exploring further to improve future (re-)planning.
We propose a theoretic framework that unifies a spectrum of planners for the pivotal agent to address this trade-off.
The planner at one end of this spectrum aims to find exact solutions, while those towards the other end yield approximate solutions as the problem scales up.
Beyond theoretical analysis, we also implement \textbf{13} planners and conduct experiments in a specific domain called \textit{multi-agent route planning} with the number of agents \textbf{up to~50}, to compare their performaces in various scenarios.
One interesting observation comes from a class of planners that we call \textit{safe-agents} and their enhanced variants by incorporating domain-specific knowledge, which is a simple special case under the proposed general framework, but performs sufficiently well in most cases.
%Our unified framework, as well as those induced planners, provides new insights on multi-agent decision-making, with potential applications to related areas such as e-market design, negotiation, teaching, etc.
Our unified framework, as well as those induced planners, provides new insights on multi-agent decision-making, with potential applications to related areas such as mechanism design.
\end{abstract}

%%% The code below was generated by the tool at http://dl.acm.org/ccs.cfm.
%%% Please replace this example with code appropriate for your own paper.


%%% Use this command to specify a few keywords describing your work.
%%% Keywords should be separated by commas.

\keywords{Multi-agent planning; Opponent modelling; Tree search}

%%%%%%%%%%%%%%%%%%%%%%%%%%%%%%%%%%%%%%%%%%%%%%%%%%%%%%%%%%%%%%%%%%%%%%%%

%%% Include any author-defined commands here.
         
\newcommand{\BibTeX}{\rm B\kern-.05em{\sc i\kern-.025em b}\kern-.08em\TeX}

%%%%%%%%%%%%%%%%%%%%%%%%%%%%%%%%%%%%%%%%%%%%%%%%%%%%%%%%%%%%%%%%%%%%%%%%

\begin{document}

%%% The following commands remove the headers in your paper. For final 
%%% papers, these will be inserted during the pagination process.

\pagestyle{fancy}
\fancyhead{}

%%% The next command prints the information defined in the preamble.

\maketitle 

%%%%%%%%%%%%%%%%%%%%%%%%%%%%%%%%%%%%%%%%%%%%%%%%%%%%%%%%%%%%%%%%%%%%%%%%

\section{Introduction}
\label{sec:intro}

%NE online repair,
%
%HBA
%
%max$^n$ \cite{samothrakis2011fast} vs  minimax vs expecti-max?
%
%$smart^n$-players \cite{stahl1993evolution}
%
%rational play leads to nash equilibrium \cite{kalai1993rational}
%
%An ex-post sample NE is easier than BNE 
%
%tree search mc simu?
%
%existing work of exploring opponent optimally
%
%evaluation, in-domain, out-of-domain

%% v0 intro starts =================
%% background/example
%While AI-assisted driving is becoming prevalent, fully autonomous driving is still faced with plenties of challenges.
%Apart from sensing capabilities, one key aspect lies in the decision-making side. It is (informally) believed that autonomous driving is not reliable and therefore cannot be trust until when most of cars become autonomous vehicles.
%This belief has revealed the nature of multi-agent decision-making underneath autonomous driving.
%That is, an AI driver is not only supposed to plan for her own trajectories but should also take the possible plans of the surrounding cars into consideration, as there might be conflicting interests.
%
%% why no NE
%One can argue that the above circumstance can be formulated as a game, where a Nash equilibrium (NE) serves as a solution protocol that every autonomous vehicle should comply.
%%To some extent, this is true if we turn to look at a warehouse logistic settings where a fleet of robots are programed to pick up and delivery packages.
%Such a claim is debatable in two aspects.
%For one, there is no central controller planning for all the agents simultaneously, let alone that part of the agents are human beings.
%For another, even if each individual agent can compute an NE, chances are that there are multiple NEs or the other agents are simply not playing the equilibrium strategies.
%
%% difficulty
%The hardness that distinguishes such problems from those like playing Go/Poker \cite{doi:10.1126/science.aar6404, schrittwieser2020mastering, danihelka2022policy, antonoglou2022planning, zinkevich2007regret}, or programming robot fleets in warehouses \cite{stern2019multi-overview, stern2019multi, sharon2015conflict, li2021eecbs}, is brought by the lack of knowledge about the opponents and the general-sum nature of interests. In two-player board games like Go, both players can assume the other one is doing the most harm, and therefore, compute a minimax strategy, which coincides with the NE of this zero-sum game and serves as a never-lose strategy regardless of what strategy the opponent eventually follows. In cooperative settings like warehouse robots, the system can compute a throughput-maximizing plan and forces the robots to jointly execute the plan. However, all the aforementioned approaches are infeasible when the game is general-sum and there is no central authorities.
%
%% problem
%We address this general class of problems by conducting planning and opponent modelling at the same time. 
%Given no knowledge about the upcoming opponents, the controlled agent will start with a uniform belief. With the game going on, actions done by the other players can be observed, which can be utilized to update the belief. Meanwhile, the revised belief can be used for replanning. In fact, to control the agent is to find a trade-off between exploiting the current knowledge and acquiring more information about opponents for more accurate future planning.
%%Note that, one can never completely predict in advance what kind of opponents will join the game until the game begins.
%Computing a plan over all possible future beliefs completely in advance is nearly impossible. 
%However, it is feasible to interleave belief update and replanning in an online fashion. As we mentioned above, this replanning operator should take into account both the value of the current state and the potential information gain of future belief revision.
%
%%Back to the example of autonomous driving, it is reasonable that a driver initially assumes any other is either a normal one or a drunk one.
%%A normal one will go straight towards her destination and avoid collisions with others, while a drunk one may seem to drive wonkily and even risk bumping into cars or trees.
%%After observing that the other one is actually taking irrational moves occasionally, the driver can probably conclude that she is against a drunk one and should therefore drive herself away from that.
%
%% solution
%We present a general framework that unifies across a spectrum of useful formulations.
%The theoretic framework is basically to do lookahead tree search and backup from values of leaf nodes, whereas different formulations implement the lookahead search differently.
%The formulation at one extreme of this spectrum aims at solving exact solutions, while the formulations towards the other extreme lead to approximated solutions as the problem scales up.
%More specifically, to compute an optimal plan, one has to solve for an exact solution for the underlying infinite-horizon \textit{partially observable Markov decision process} (POMDP) \cite{papadimitriou1987complexity, madani1999undecidability}.
%This only works out for tiny problem instances, as optimally solving infinite-horizon POMDPs is computationally intractable.
%As the environment scales up in terms of the number of agents and the length of feasible plans, we have to appeal to approximated formulations.
%Under our proposed framework, the approximation can be done by controlling the depth of tree search, altering node evaluation functions, adopting sampling-based backup, toggling heuristic action selection, etc. The framework eventually bridges \textit{POMDP}, \textit{Contextual-MDP}, \textit{Contextual-RL}, and even \textit{Constraint Satisfaction} together.
%
%% implementation
%It is highly non-trivial to compare those formulations purely in theory. Therefore, we implement a group of planners resulted from those formulations in a concrete domain, named \textit{multi-agent route planning} (MARP).
%Most of the contemporary work eventually designs a specific planner for a specific problem (mostly repeated games), which on the one hand makes the planner itself highly problem-dependent, and on the other hand adds extra difficulty to compare those planners.
%To this end, this evaluating domain is deliberately selected because it can provide us with problem instances of various sizes, parameterized by the map size and the number of agents, and the agents are supposed to do long-horizon planning to reach goals.
%%so that we can put all the implemented planners under a fair comparison.
%Another possibly hidden aspect of evaluation lies in the choices of opponents.
%One can always construct ``counter-opponents'' on purpose to break down the current way of belief modelling of the controlled agent. It is impossible to have a one-fits-all way to form beliefs about the opponents as doing so requires initial assumptions in the first place. We alleviate this concern by testing our planners against three representative types of opponents, including rational ones, malicious ones, and self-playing ones.
%Among all the planners, we draw readers attention to a particular class, what we call \textit{safe-agents} and their enhanced variants, which turns out to be a very coarse-grained special case of our general framework but performs sufficiently well in most cases. It is simple enough to save a great amount of computation.
%
%% generalization
%Although the empirical evaluation is done in a specific domain, the formulations themselves are general enough to apply to other related domains. 
%% maybe in conclusion?
%For example, in e-markets, the owner designs a mechanism to maximize the revenue from potential payments of consumers. The revenue maximization is done by assuming a probabilistic profile of consumers. By the insight of our work, we advocate it should be revised to additionally take future profile change into account \cite{tang2017reinforcement}. Another possible application will be for negotiation \cite{kraus1997negotiation, beer1999negotiation, jennings2001automated}, where multiple stakeholders try to strategically persuade each other to agree on a common goal. For each one, it is not only needed to postulate how the others would act, but also necessary to notice their actions reflect their persuasion strategies. It is therefore believed that the development of negotiation theory will also benefit that of persuasion theory \cite{4ea43cce77d34e00bbf627204843ce91} in psychology by offering a multi-agent perspective and computational techniques.
%
%% v0 intro ends ===================

%\section{new intro}

% v1 intro starts =================

We consider a general problem where an agent is in a multi-agent system and is supposed to plan for herself without any prior information about her opponents.
A motivating example of such a scenario is in autonomous driving~\cite{chen2024end}, where a fully autonomous vehicle needs to navigate herself to the destination, but inevitably shares the roads with other cars that she may not know much about, ones either driven by human beings or possibly controlled by AI softwares from another company.
Therefore, she must learn the behavioral patterns of the other cars in real-time and act accordingly.

There are three unique features that distinguish this setting from other seemingly related ones.
\textbf{Firstly}, it is clearly different from a system where all the agents are under the control of a central planner, such as programming robot fleets to deliver packages in warehouses~\cite{stern2019multi-overview, stern2019multi, sharon2015conflict, li2021eecbs}, which is also known as \textit{multi-agent path finding} (MAPF) problems.
%
\textbf{Secondly}, the setting that we are investigating does not ensure pure competitiveness among participants, and therefore, cannot be modelled as zero-sum games (win or lose).
In typical zero-sum games, e.g., Go/chess~\cite{doi:10.1126/science.aar6404, schrittwieser2020mastering, danihelka2022policy, antonoglou2022planning} and poker~\cite{zinkevich2007regret},
a player can assume the other one will do the most harm, and compute a minimax strategy, which coincides with the Nash equilibrium (NE) and later serves as a never-lose strategy regardless of what strategy the opponent eventually follows.
%
\textbf{Thirdly}, an oracle that computes a sample NE dose not \textit{directly} offer much help, even if all agents are aligned with common interests~\cite{boutilier1996planning, claus1998dynamics}.
Chances are that there may be multiple NEs or the other agents are simply not playing the equilibrium strategies at all.

Conceptually, 
this general class of problems is usually addressed by conducting opponent modelling and opponent-aware replanning at the same time. 
Given no knowledge about the upcoming opponents, the controlled agent will start with an initial belief over possible opponent models (or types). With time going on, actions done by the other players are observed, which can be utilized to update the belief. Meanwhile, the revised belief can be employed for replanning. Essentially, to control the agent is to find a trade-off between exploiting current knowledge and acquiring more information about  opponents to enhance future replanning~\cite{carmel1998explore, carmel1999exploration}.

% previously `Mathematically, `
In principle, it can be formalized as a stochastic game~\cite{shapley1953stochastic, solan2015stochastic} with incomplete information~\cite{harsanyi1967games}, and the incompleteness is due to the lack of knowledge about opponents.
There is a vast literature on how to devise an effective strategy for the controlled agent, mostly based on best responses~\cite{kalai1993rational, boutilier1996planning, claus1998dynamics, carmel1998explore, carmel1999exploration, albrecht2015game, eck2020scalable, schwartz2023bayes, schwartz2023combining, rahman2023general}.
%rational learning, designed or learn conventions, IL, JAL, lookahead-explore, HBA,
However, one can clearly observe two extremes along this line of work.
At one extreme, researchers apply deliberate formulations on toy artificial problems such as repeated games with one state transiting to itself~\cite{kalai1993rational, boutilier1996planning, claus1998dynamics, carmel1998explore, carmel1999exploration, albrecht2015game}, serving primarily as proof-of-concept examples.
In fact, repeated games cannot fully reflect the complexity underneath as they are just special cases of stochastic games.  
At the other extreme, people jump to highly approximated ones, namely Monte Carlo methods~\cite{albrecht2015game, eck2020scalable, schwartz2023bayes, schwartz2023combining} or complex Neural Network (NN) pipelines~\cite{rahman2023general}, yet the largest known domain remains only $10^{14}$ states with four agents~\cite{schwartz2023bayes, schwartz2023combining, rahman2023general}.
We here ask a significant research question: \textbf{Is there any viable formulation in between, and even beyond?}
More precisely, given such a multi-agent planning problem with $10^6$ states ($\ll 10^{14}$),
can we identify a planner that is more accurate than Monte Carlo ones, 
and is there any planner that can scale to even larger state spaces ($\gg 10^{14}$)?
%, e.g., in sizes of $10^{40}$ and $10^{140}$?

%More precisely, is there a suitable planner that can solve such a multi-agent planning problem with $10^4$ states, and furthermore, does there exist any planner that can solve problems with even larger state spaces, e.g., in sizes of $10^{40}$ and $10^{140}$?

%But most of them are repeated games, or reactive states, or very coarse implementation of their claimed formulation.
%
%small number of agents, or small number of states
%
%some are unnecessary appealing the complex learning algorithms
%
%map size bxb $b\in [8,12]$
%
%largest of which has four agents and on the order of 1014 states and 108 observations.
%
%from simple repeated games jump to mcts

To this end,
we first present a set of formulations that each of them \textit{rediscovers and generalizes} an existing one from the literature.
More specifically, to find an optimal plan, one has to compute an exact solution for the underlying infinite-horizon \textit{partially observable Markov decision process} (POMDP)~\cite{smallwood1973optimal, sondik1978optimal, kaelbling1998planning}.
This is only feasible for tiny problem instances, as optimally solving infinite-horizon POMDPs is computationally expensive~\cite{papadimitriou1987complexity, madani1999undecidability}.
As the environment scales up in terms of the number of agents, we have to appeal to approximated formulations.
We then propose a general framework that unifies this entire set of formulations.
This unified framework is designed to encompass a spectrum that interpolates the aforementioned formulations, enabling users to devise new ones for those problems of intermediate scales as well as much larger scales.
\textbf{The theoretic framework is basically to perform layered lookahead search and backup from values of leaf nodes in each layer, wherein different formulations implement different lookahead search.}
The induced formulation at one end of this spectrum aims at solving exact solutions, while the formulations towards the other extreme lead to approximated ones as the problem scales up.
%More specifically, to compute an optimal plan, one has to compute an exact solution for the underlying infinite-horizon \textit{partially observable Markov decision process} (POMDP)~\cite{smallwood1973optimal, sondik1978optimal, kaelbling1998planning}.
%This only works out for tiny problem instances, as optimally solving infinite-horizon POMDPs is computationally intractable~\cite{papadimitriou1987complexity, madani1999undecidability}.
%As the environment scales up in terms of the number of agents, we have to gradually approximate the formulation.
Within our proposed framework, approximations can be made by controlling the depth of tree search, altering node evaluation functions, adopting sampling-based backup, toggling heuristic action selection, etc. The framework effectively connects \textit{POMDP}~\cite{kaelbling1998planning}, \textit{contextual-MDP/RL}~\cite{hallak2015contextual, benjamins2021carl}, \textit{tree search}~\cite{browne2012survey} and even \textit{constraint satisfaction}~\cite{brailsford1999constraint}.

%More specifically, to compute an optimal plan, one has to solve for an exact solution for the underlying infinite-horizon \textit{partially observable Markov decision process} (POMDP) \cite{papadimitriou1987complexity, madani1999undecidability}.
%This only works out for tiny problem instances, as optimally solving infinite-horizon POMDPs is computationally intractable.
%As the environment scales up in terms of the number of agents and the length of feasible plans, we have to appeal to approximated formulations.
%Under our proposed framework, the approximation can be done by controlling the depth of tree search, altering node evaluation functions, adopting sampling-based backup, toggling heuristic action selection, etc. The framework eventually bridges \textit{POMDP}, \textit{Contextual-MDP}, \textit{Contextual-RL}, and even \textit{Constraint Satisfaction} together.


%upper triangular experimental results
%Answer the research question:
%We fill this gap by 1) a spectrum of formulations and 2) a unified framework.
%Our framework \textsc{Layered ExpectiMax} can rediscover all the aforementioned formulations and moreover shed lights on new ones
%highly extendable, 1) can incorporates new plug-ins 2) shifts from \textsc{ExpectiMax} to \textsc{MiniMax} and \textsc{Max}$^n$ by simplying altering the belief types of opponents.

It is highly non-trivial to compare those formulations purely in theory. Therefore, we implement a group of planners\footnote{Code available at
\url{https://github.com/Fernadoo/BestResponsePlanning}.}
resulted from those formulations in a concrete domain, called \textit{multi-agent route planning} (MARP)
%Most of the contemporary work eventually designs a specific planner for a specific problem (mostly repeated games), which on the one hand makes the planner itself highly problem-dependent, and on the other hand adds extra difficulty to compare those planners.
This benchmark is challenging in two aspects: 1)~it is no longer repeated games, instead, involves complex state transitions; 2)~it necessitates long-horizon planning, as the rewards are goal-conditioned and hence sparse.
We select this domain on purpose to take the advantage that it can 
generate problem instances (parameterized by the map size and the number of agents) along a fine-grained axis of scales, from around 70 states with two agents to around $4.62 \times 10^{145}$ states with 50 agents, matching our need of studying the capacities of the planners in the spectrum.
As it is impossible to run each proposed planner against all possible hypothetical opponents and their combinations,
we instead test our planners against three representative groups of opponents, including rational ones, malicious ones, and self-playing ones.
Among all the planners, we also draw one's attention to a particular class, what we call \textit{safe-agents} and their enhanced variants, which turns out to be a special case of our general framework but performs sufficiently well in most cases. It is also simple enough to save a great amount of computation.

%The full paper is available at
%\url{https://arxiv.org/abs/xxxx.xxxx}.



%Benchmark:
%challenging in 1) no longer repeated games, 2) long-horizon planning needed (rewards sparse)
%advantages in xxx
%helps to illustrate ours
%side products by observing xxx
%
%

%\textbf{Paper organization.} We list a few related areas in Section~\ref{sec:related}.
%Then the model and potential solution formulations are presented in Section~\ref{sec:model}.
%Later in Section~\ref{sec:unification}, we propose the unified framework.
%The multi-agent routing domain is described in Section~\ref{sec:benchmark}, followed by some experimental results.
%The paper is concluded with a few valuable future directions in Section~\ref{sec:conclusion}.




% v1 intro ends ===================



%\section{Related Work}
%
%% v0 related starts ==========
%%The problem that we investigate is related to quite a few research areas.
%
%% Ad hoc coordination
%\textit{Ad-Hoc Coordination.}
%Coordination in multi-agent systems has been of great interest for both game theorists and computer scientists for a long history.
%\citet{boutilier1996planning} studied the problem of multi-agent planning where individual agents have to coordinate as there are multiple NEs. The author's focus was merely on coordinating agents by either social conventions or a learned protocol as the agents' interests are aligned with each other. This further leads to a flourishing area named \textit{zero-shot coordination} (or, more generally as \textit{ad-hoc teamwork}) \cite{mirsky2022survey}. A direct application is to the card game \textit{Hanabi}, where players who have never met each other before are supposed to collaborate for higher scores \cite{pmlr-v119-hu20a, pmlr-v139-hu21c}.
%In contrast, an orthogonal line of work of multi-agent reinforcement learning (MARL) \cite{lowe2017multi, foerster2018counterfactual, yu2022the, pmlr-v162-fu22d, JMLR:v24:22-0169, samvelyan19smac} aims at solving a sample NE, which does \textbf{not} exactly fall into the scope of our discussion as they force all agents to execute the NE policy that is found during the training phase.
%
%% opponent modelling
%\textit{Opponent Modelling.}
%\citet{carmel1998explore, carmel1999exploration} investigated a similar problem as ours, which also partly inspired this work. Nevertheless, the authors only discussed applications in toy examples like the \textit{Iterated Prisoner's Dilemma} (IPD). Our work can be viewed as an extension where we put everything in a unified framework and drew a spectrum of formulations as well as implementations in a more complex domain.
%\citet{albrecht2018autonomous} surveyed a number of methodologies on opponent modeling,
%where our work falls into the category of type-based reasoning. The most recently related work goes to \cite{albrecht2019convergence, albrecht2016belief}, where the authors proposed a belief-dependent dynamic programming operator, mainly contributed to the convergence and optimality analysis from the theoretical side. Our work can be seen as a generalized implementation along with empirical studies in a highly non-trivial domain.
%%supplementing the understanding of such problems from the practical side via comprehensive experiments.
%
%% NE; recursive reasoning; bounded rationality
%\textit{Bounded Rationality.}
%%To some extent, we are actually arguing that the solution concept of Nash equilibrium does not offer much help in terms of how to construct an appropriate strategy given that the agent knows almost nothing about her opponents.
%%To best respond to her current hypothetical opponent model is all the agent can do. However, whether it will eventually leads to a whatever equilibrium if all agents follow such best response dynamics should not be our concern.
%%The most significant part is to define how best responses should be. A fully rational agent would presumably do infinitely recursive reasoning, going like ``I believe that you believe that I believe ..'', and do backward induction all the way up to the current state and figure out her best action. In some special cases like two-agent zero-sum games, one can always do minimax search to the terminal state.
%%Nevertheless, in general, it is obviously impossible for computer programs to implement such behavior.
%The most closely related formalization is called \textit{Interactive POMDP} (I-POMDP) \cite{gmytrasiewicz2005framework} modelling bounded rationality via recursive reasoning, where in theory the depth of recursion can be arbitrarily large but in practice it is usually a very small integer like 1 or 2 with toy examples.
%It is also shown in experimental psychology that in most cases human beings merely demonstrate a shallow capability of recursive reasoning. The empirical results  \cite{camerer2004cognitive}  suggest participants on average only do around 1.5 steps of reasoning, that is, assuming their opponents are a fair mixture of random ones and level-1 rational ones. % slightly inaccurate
%
%% POMDP; CMDP
%\textit{POMDP.}
%As we adopt POMDP \cite{kaelbling1998planning} as the underlying model, one can leverage any existing techniques to alleviate the computation burden \cite{littman1995learning, zhang2001speeding}. It is arguably correct to alternatively formalizing it as \textit{Contextual MDP} (CMDP), where transitions (and rewards) can be factorized into ones over only subsets of states \cite{hallak2015contextual}. However, CMDPs are degrade to POMDPs when those contexts are not revealed. In this sense, some meta/contextual learning techniques developed by the RL community might also be related \cite{beck2023survey, benjamins2021carl}.
%
%% MAPF
%\textit{MAPF.}
%Last but not least, we would like to claim that our work might \textbf{not} align with the interest of researchers who are investigating the problem of \textit{Multi-Agent Path Finding} (MAPF) \cite{stern2019multi-overview, stern2019multi, sharon2015conflict, zhang2022multi, li2021eecbs}. The line of work in MAPF aims at inventing algorithms controlling multiple moving robots that run fast in practice. It is either the case where the central controller computes the joint-plan in advance and forces all robots to rigorously execute it, or the case where each robot follows a certain protocol that enables them to communicate and iteratively adjust their routes. None of these cases falls into our scope. We borrow this domain only for evaluating and illustrating the capability of those aforementioned planners.
%
%%as it offers a testbed where we can highly customize the map layout and the number of agents, and therefore, alter the scale of the planning problem for a comprehensive study.
%
%% v0 related ends ==========

\section{Related Work}
\label{sec:related}

% v1 related starts ==========
The problem that we investigate pertains to a board range of theoretic areas, primarily \textit{opponent modelling} and the control theory of \textit{POMDP}, alongside  practical tools, such as efficient \textit{NE solvers}.

%A primary area that is related is termed as \textit{opponent modelling}.
\textbf{Opponent Modelling.}
\citet{albrecht2018autonomous} surveyed a number of methodologies under this topic, where our work falls into the category of type-based reasoning.
Types usually refer to predefined opponent models.
For instance, \citet{carmel1998explore, carmel1999exploration} investigate a similar problem by modelling opponent strategies as deterministic finite automata,
while other researchers have approached this by using reactive policy oracles, either given by human experts~\cite{albrecht2015game} or trained by separated pipelines~\cite{schwartz2023bayes, schwartz2023combining}.
An critical follow-up issue is how to integrate opponent modelling with opponent-aware planning. 
For repeated games, much of the literature emphasizes planners that best respond to the current belief over hypothetical opponent models, e.g., $\epsilon$-BR~\cite{kalai1993rational}, IL/JAL~\cite{boutilier1996planning, claus1998dynamics}, CJAL~\cite{banerjee2007reaching}, etc.
For a more general setting, HBA~\cite{albrecht2015game} is proposed as a belief-dependent planning operator that takes long-term returns into account. The authors have also examined some convergence results~\cite{albrecht2019convergence}, and the conditions under which the belief correctly reveals the truth~\cite{albrecht2016belief}.
Recent studies have also explored MCTS methods~\cite{schwartz2023bayes, schwartz2023combining}, and neural network approximations~\cite{rahman2023general}.


\textbf{POMDP.}
We later formalize the underlying problem as a \textit{Contextual MDP} (CMDP)~\cite{hallak2015contextual}, where the set of contexts corresponds to the set of all possible opponent models.
However, CMDPs reduce to POMDPs~\cite{smallwood1973optimal, sondik1978optimal, kaelbling1998planning} when those contexts are not revealed. 
One can therefore resort to existing techniques in the literature of the POMDP theory to alleviate the computational burden~\cite{littman1995learning, zhang2001speeding}.
Moreover, some meta/contextual learning techniques developed by the RL community may also be related \cite{beck2023survey, benjamins2021carl}.

% I-POMDP?

\textbf{NE Solvers.}
Lastly, we want to clarify that our work may \textit{not directly} align with the interests of researchers who are investigating the problem of \textit{Multi-Agent Reinforcement Learning} (MARL)~\cite{lowe2017multi, foerster2018counterfactual, yu2022the, pmlr-v162-fu22d, JMLR:v24:22-0169, samvelyan19smac} or \textit{Multi-Agent Path Finding} (MAPF)~\cite{stern2019multi-overview, stern2019multi, sharon2015conflict, zhang2022multi, li2021eecbs}.
Both lines of work focus on solving a sample NE in a centralized manner, which does not strictly fall into the scope of our discussion as they force all agents to execute the NE plan that is found upfront.
Nevertheless, as we will show that with the help of an oracle that computes an NE very fast in real-time, our proposed framework can effectively repair it while replanning.
We emphasize that this routing domain resembling MAPF is selected as our benchmark solely for the purpose of evaluating and illustrating the capabilities of our proposed planners.


%
%The line of work in MAPF aims at inventing algorithms controlling multiple moving robots that run fast in practice. It is either the case where the central controller computes the joint-plan in advance and forces all robots to rigorously execute it, or the case where each robot follows a certain protocol that enables them to communicate and iteratively adjust their routes. None of these cases falls into our scope. 
%In contrast, one should be cautious about an orthogonal line of work of multi-agent reinforcement learning (MARL) \cite{lowe2017multi, foerster2018counterfactual, yu2022the, pmlr-v162-fu22d, JMLR:v24:22-0169, samvelyan19smac}, which aims at learning to solve a sample NE, which does \textbf{not} exactly fall into the scope of our discussion as they force all agents to execute the NE policy that is found during the training phase.

%Despite of the list related areas above, to the best of our knowledge, none of them clearly answers the research question that we have asked in the previous section,
% which therefore makes this work of great significance.

%recursive reasoning
%
%1. Problem background: opponent modelling, mostly ad-hoc coordination, including bounded rationality
%
%2. Problem formulation: POMDP, Contextual MDP, various tree search and bandit
%
%3. Not aligned but used: Centralized cooperative MAS: MAPF, MARL

% v1 related ends ==========


%\vspace{-10mm}
\section{Model and Solution Concepts}
\label{sec:model}

\subsection{Stochastic Game}

The whole system where the agents interact is modelled as a \textit{stochastic game} (SG, also known as Markov games)~\cite{shapley1953stochastic, solan2015stochastic}, which 
can be seen as an extension of both \textit{normal-form games} (to dynamic situations with stochastic transitions) and \textit{Markov decision processes} (to strategic situations with multiple agents).
A stochastic game
is a 5-tuple $\langle\mathcal{N},\mathcal{S}, \mathcal{A}, T, R\rangle$ given as follows,
\begin{enumerate}
	\item $\mathcal{N}$ is a finite set of agents.
	\item $\mathcal{S}$ is a finite set of normal form games. We call each possible $S \in \mathcal{S}$ a stage game. In the proceeding discussion, each $S$ is also called an environment state, or just a state.
	\item $\mathcal{A} = \mathcal{A}_1 \times \cdots \times \mathcal{A}_n$ is a set of joint actions, where $\mathcal{A}_i$ is the action set of agent $i$. We write $a_i$ as the action of agent $i$ and $a = (a_i, a_{-i})$ (without any subscript) as the joint action.
	\item $T: \mathcal{S} \times \mathcal{A}_1 \times \cdots \mathcal{A}_n \mapsto \Delta(\mathcal{S})$ defines a stochastic transition among stage games.
	\item $R_i: \mathcal{S} \times \mathcal{A}_1 \times \cdots \mathcal{A}_n \mapsto \mathbb{R}$ denote the utility function of agent $i$ in each stage game.
\end{enumerate}
Each agent $i$ is only aware of her own utilities $R_i$. In fact, knowing other agents' utility functions does not help, as she has no knowledge about the opponents until the game begins and will be possibly exposed to different opponents in each match.
This adds extra difficulty of evaluation as we will cover that later in Section \ref{sec:benchmark}.
Following the terminology in \cite{albrecht2018autonomous}, we also call agent $i$ the modelling agent under our control, and agent $-i$ the modelled agents.


\subsection{The Modelling Agent's Problem}

% problem assumptions
The modelling agent under such an environment acts in a manner of maximizing her own accumulated discounted utility, i.e.,
%\begin{equation}
$
	\mathbb{E}_{\tau \sim SG}[\sum_{t=0}^{T(\tau)} \gamma^tR_{i,t}],
$
%\end{equation}
where $\tau$ denotes the trajectory of one single match of the SG, $\gamma < 1$ denotes the discount factor, and $T(\tau)$ denotes the length of the trajectory.
%With slight abuses of notations, by $\tau \sim SG$, we mean that on the one hand the stochastic game itself involves randomness mainly due to its stochastic transition, on the other hand the opponents are also unknown but there is no probabilistic distribution prescribing them\footnote{It is usually noted as aleatoric uncertainty (due to randomness) versus epistemic uncertainty (due to a lack of knowledge).}.
%Given that agents act simultaneously and each of them only knows their own utility, we may term the whole game an imperfect and incomplete one.
For the sake of simplicity, we assume perfect recall, i.e., the modelling agent can remember the entire history from the beginning of the match up to the current moment, including every past states and joint actions.
Formally, a history segment is given as $h \in \mathcal{H} \triangleq (\mathcal{S} \times \mathcal{A})^*$, i.e. a sequence consisting of alternating states and joint-actions.

% solutions concept; 
Therefore, a candidate solution concept for the modelling agent is a history-depenent stochastic policy $\pi_i: \mathcal{H} \times \mathcal{S} \mapsto \Delta(\mathcal{A}_i)$ that maps from all possible history segments and current states to (randomized) available actions.
%The stationary structure of the stage transition also leads one to consider stationary Markov strategies, i.e. history-independent policies.
%It is proved that NE formed by stationary Markov strategy profiles do exist when there are finitely many states \cite{fink1964equilibrium, takahashi1964equilibrium}.
%Nevertheless, we have argued that NE merely serves as a characterization of stable outcomes when agents perform mutual best response. 
%As there is no golden standard on how to model opponents without any prior knowledge about them,
%despite of our previous argument that NE merely serves as a characterization of stable outcomes when agents perform mutual best response, the existence of such NEs at least offers some justification that one can simply look at stationary Markov strategies.
% opponent assumptions
%There is no golden standard on how to model opponents without any prior knowledge about them.
% To some extent, the existence of NE in stationary Markov strategies, however, offers justification that one can simply look at strategies of such simple forms.
%A direct way is to do it in a ``monolithic'' way, if one can somehow obtained an oracle, e.g., by learning from an artifact dataset that takes as input any history and outputs the predicted most likely action.
%As an alternative, we would rather do it in a ``compositional'' way as follows.
From agent $i$'s perspective, she also assumes every other opponent $j$'s potential strategy is drawn from a set of basis policies $\Pi_j$, where each $\pi_j^k \in \Pi_j$ is a stationary Markov policy that maps states to actions,
i.e. $\pi_j^k: \mathcal{S} \mapsto \Delta(\mathcal{A}_j)$.
This is indeed a strong assumption that suggests those opponents do not care about histories, and they are not learning and evolving during the game play.

% problem formulation
We then show, from the modelling agent's perspective, conditioned on the strategies of opponents, one can model the rest of the problem as a \textit{Contextual  Markov Decision Process} (CMDP)~\cite{hallak2015contextual}.
The resulted formulation is given as a 3-tuple $\langle\mathcal{C}, \mathcal{A}, \mathcal{M}\rangle$,
\begin{enumerate}
	\item $\mathcal{C}$ is a set of contexts. In our setting, $\mathcal{C} \triangleq \Pi_{-i} = \prod_{j\neq i} \Pi_j $, i.e., the set of contexts are exactly the set of all possible combinations of opponent models.
	\item $\mathcal{A}$ is a set of actions. In our setting, $\mathcal{A} \triangleq \mathcal{A}_i$, as we are approaching the problem from agent $i$'s perspective.
	\item $\mathcal{M}$ is a mapping from contexts to corresponding Markov decision processes.
	An interpretation is, if the modelling agent knows for sure how her opponents would act, then she is basically faced with a single-agent MDP with opponent behaviors subsumed into transition/utility functions as noises.
	The resulted MDP from the perspective of agent $i$, denoted as $\mathcal{M}(\pi_{-i})$,
	is induced as a 5-tuple
	$\langle\mathcal{S}, \mathcal{A}_i, T^{\pi_{-i}}, R^{\pi_{-i}}, \gamma\rangle$:
	\begin{itemize}
		\item $\mathcal{S}, \mathcal{A}_i$ and $\gamma$ inherit from the previous setup,
		\item $T^{\pi_{-i}}(S'| S, a_i) \triangleq \sum_{a_{-i} \in \mathcal{A}_{-i}}T(S'|S,a)\pi_{-i}(a_{-i}|S)$,
		\item $R^{\pi_{-i}}(S, a_i) \triangleq \sum_{a_{-i} \in \mathcal{A}_{-i}}R_i(S,a)\pi_{-i}(a_{-i}|S)$,
	\end{itemize}
%	where $\pi_{-i}(a_{-i}|s) \triangleq \prod_{j\neq i}\pi_j(a_j|s)$ by the independent modeling assumption.
\end{enumerate}
If the context is revealed, it is equivalent to augmenting the MDP state by an additional dimension representing the context, and such a CMDP can therefore be factorized into $|\mathcal{C}|$ MDPs and solved one by one via any off-the-shelf solver accordingly.
However, the issue is that in our situation the context is unobservable as it represents the unknown strategies of opponents.
It then reduces to solving the corresponding POMDP~\cite{smallwood1973optimal, sondik1978optimal, kaelbling1998planning}, where a state in the POMDP consists of the environment state and the underlying context (opponent model), and the observation function takes as input a POMDP state, and returns only the environment state exposed to everyone yet keeps the context (opponent model) hidden.

We also assume that the pivotal agent model every other opponent \textit{independently}, and each independent belief is a distribution over potential basis policies. Mathematically, for any belief $b \in \mathcal{B} \triangleq \prod_{j\neq i}\Delta(\Pi_j)$, we have $\pi_{-i}(a_{-i}|s) = \prod_{j\neq i}\pi_j(a_j|s)$ and $b(\pi_{-i}) = \prod_{j\neq i}b(\pi_j)$.
Depending on the set of presumed opponent policies, any belief $b_t$ at time step $t$ can be updated to a successor belief $b_{t+1}$ inductively by a revision operator, denoted as $\xi: \mathcal{B}\times \mathcal{S}\times \mathcal{A} \mapsto \mathcal{B}$. Equivalently speaking, given a prior belief and a sequence consisting of past states, and the actions taken under each corresponding state, a posterior belief can be inferred. Here we formulate it in a Bayesian way, $b_{t+1} = \xi(b_t, S, a)$ is given as,
\begin{equation}
\begin{split}
	b_{t+1}(\pi_j^k) \triangleq \mathbb{P}[\pi_j^k | S, a_j]
	& = \frac{ (\mathbb{P}[a_j|S, \pi_j^k]\cdot \mathbb{P}[\pi_j^k|S])^{1/\beta} }
	{\sum_{\pi_j^l \in \Pi_j} (\mathbb{P}[a_j|S, \pi_j^l]\cdot \mathbb{P}[\pi_j^l|S])^{1/\beta} } \\
	& = \frac{ (\pi_j^k(a_j|S)\cdot b_t(\pi_j^k))^{1/\beta} }
	{\sum_{\pi_j^l \in \Pi_j} (\pi_j^l(a_j|S)\cdot b_t(\pi_j^l))^{1/\beta} } \\
%	& \propto (\pi_j^k(a_j|S)\cdot b_t(\pi_j^k))^{1/\beta} 
\end{split}
\label{eq:bayes}
\end{equation}
where $\beta$ is a tunable temperature parameter.
If $\beta = 1$ is always the case, it will be exactly the same as how beliefs are computed in the corresponding POMDP. However, altering the value of $\beta$ will open up a broader set of choices for implementations, as we will illustrate in Table \ref{tab:all_planners}.
%Also, we denote $b(\pi_{-i}) \triangleq \prod_{j\neq i}b(\pi_j)$ by the independent modeling assumption.

% playing against a population of agents == against a mixed one?




\subsection{Potential Solution Formulations}
\label{sec:solutions}

\subsubsection{Exact Dynamic Programming}


%approach from  Bayesian game~\cite{harsanyi1967games} and seek Bayesian-Nash equilibria~\cite{harsanyi1968games}?

As we mentioned, the modelling agent is faced with a CMDP with unobservable contexts which subsequently reduces to a POMDP.
Following from the control theory of POMDP~\cite{smallwood1973optimal}, one can actually find out that at each moment a compound \textit{information state} consisting of the environment state and the belief serves as a sufficient statistic, summarizing all the past history.
Hence, the transitions among those information states from the modelling agent's perspective, denoted as $\mathcal{T}: \mathcal{S}\times \mathcal{B} \times \mathcal{A}_i \mapsto  \Delta(\mathcal{S})\times \mathcal{B}$, are resulted from a joint effect of the environment transition and a sampling process of opponents' actions according to the belief. Mathematically,
{\small
\[
\mathcal{T} \Big ( (S', b') \Big| (S, b), a_i  \Big ) \triangleq 
\sum_{\pi_{-i}\in b} b(\pi_{-i})\sum_{a_{-i} \in \mathcal{A}_{-i}} T(S'|S, a) \pi_{-i}(a_{-i}|S)
\]
}
when $T(S'|S, a) > 0 \text{ and } b' = \xi(b, S, a)$, and all the other transitions are invalid and assigned zero probability.
The expected reward from the modelling agent's perspective is induced analogously,
{\small
\[
\mathcal{R} \Big((S, b), a_i \Big ) \triangleq 
\sum_{\pi_{-i}\in b} b(\pi_{-i})\sum_{a_{-i} \in \mathcal{A}_{-i}} R_i(S, a) \pi_{-i}(a_{-i}|S)
\]
}
Then, the Bellman optimality equation can be established over a continuous space,
\begin{equation}
%\hspace{-3mm}
{\small
\begin{split}
	& V_i(S,b) \\
	& = \max_{a_i \in \mathcal{A}_i}
	\bigg\{  \mathcal{R} \Big( (S, b), a_i \Big)
	+ \gamma \sum_{S', b'} \mathcal{T} \Big( (S', b') \Big| (S, b), a_i  \Big)\cdot V_i(S',b') \bigg\} \\
	& = \max_{a_i\in \mathcal{A}_i}\{\mathbb{E}_{ \substack{\pi_{-i} \sim b, \\ a_{-i}\sim \pi_{-i}} }
	[R_i(S,a) +  \gamma \sum_{S'}T(S'|S,a) V_i(S', b')]\}
\end{split}
}
\label{eq:pomdp}	
\end{equation}
when $b' = \xi(b, S, a)$, and $V_i(\cdot,\cdot)$ is the desired optimal value function for the modelling agent.
%
Again, if $\xi$ is implemented with $\beta = 1$, then solving Equation~(\ref{eq:pomdp}) is equivalent to optimally solving the corresponding infinite-horizon POMDP.
Despite that it is computationally costly in theory as revealed by~\cite{papadimitriou1987complexity, madani1999undecidability},
we can resort to off-the-shelf POMDP solvers like \textit{pomdp-solve}
\footnote{https://pomdp.org/code/} to solve small problem instances in practice, as we will show in Appendix~\ref{apd:case_study}.


%While we can appeal to off-the-shelf POMDP solvers like \textit{pomdp-solve}
%\footnote{https://pomdp.org/code/}
%, it is computationally costly both in theory as revealed by~\cite{papadimitriou1987complexity, madani1999undecidability}, and in practice as we will illustrate with experiments in Appendix \ref{apd:case_study}.

As one may notice, this formulation resembles the HBA operator proposed by~\cite{albrecht2015game}, where the authors thereof call it a best-response rule based on Bellman control. We here point out that it is not just any random rule, but also the exact characterization of what the modelling agent is faced with, and the optimal control strategy can be therefore \textit{derived} and computed upfront.

% HBA = one step belief update + QMDP

%\subsubsection{Contextual-RL}
In fact, Equation~(\ref{eq:pomdp}) also leads to potential (contextual-)RL solutions~\cite{benjamins2021carl}.
%To this end, one needs to wrap the given multi-agent environment as a single-agent learning environment, by (i) first initializing each episode by sampling opponent strategies according to the initial belief, and (ii) then proceeding the environment by the given action and the sampled opponent actions, updating the belief, and returning it to the learning agent.
Subsequently, the most challenging part lies in how to effectively train such a policy that converges to the desired optimum.
%As we will show in Appendix~C, there is usually an intermediate plateau phase before convergence, which we conjecture is caused due to the gap between feasible strategies and optimal strategies. 
As we will show in Appendix~\ref{apd:exp_setting}, there is usually an intermediate plateau phase before convergence, which we conjecture is caused due to the gap between feasible strategies and optimal strategies. 

%actor-critic?


\subsubsection{Belief-Induced MDP}
\label{sec:belief_induced_mdp}

%TODO: IL~\cite{claus1998dynamics}

As an alternative, one can amortize the burden of computing an optimal strategy completely in advance over repeated online replanning during the game play.
We first extend the aforementioned context-to-MDP mapping $\mathcal{M}(\pi_{-i})$ to $\mathcal{M}(b)$, allowing inputs of distributions of contexts,
\begin{itemize}
\item $T^{b}(S'| S, a_i) \triangleq \sum_{\pi_{-i}\in b} b(\pi_{-i})\sum_{a_{-i} \in \mathcal{A}_{-i}}T(S'|S,a)\pi_{-i}(a_{-i}|S)$
\item $R^{b}(S, a_i) \triangleq \sum_{\pi_{-i}\in b} b(\pi_{-i})\sum_{a_{-i} \in \mathcal{A}_{-i}}R_i(S,a)\pi_{-i}(a_{-i}|S)$
\end{itemize}
Therefore, optimally solving the belief-induced MDP $\mathcal{M}(b_t)$ at each time $t$ is equivalent to solving a surrogate target,
\begin{equation}
\begin{split}
& V_i(S,b_t) \\
&  = \max_{a_i\in A_i}\{\mathbb{E}_{ \substack{\pi_{-i} \sim b_t, \\ a_{-i}\sim \pi_{-i}} }
[R_i(S,a) +  \gamma \sum_{S'}T(S'|S,a) V_i(S', b_t)]\}	
\end{split}
\label{eq:belief_mdp}
\end{equation}
where $V_i(\cdot,b_t)$ is the optimal value function characterizing the optimal control of the belief-induced MDP $\mathcal{M}(b_t)$.
Note that the belief on the LHS is the same as that on the RHS, i.e., $b_t$ is fixed as a hyperparameter.
That is to say, the modelling agent is assuming the others will play the stationary mixed strategies throughout the rest of the game according to the current belief.

One may notice that solving such an online surrogate target involves inherent inconsistency issues. That is, the effect on revising the belief by observing future plays of the opponents is ignored while inducing such MDPs, which is exactly the cause that leads to sub-optimal strategies for the modelling agent.

Understandably, one can then come up with two alternatives. If solving a newly induced MDP at each move is computationally feasible, then repeatedly updating the belief and solving the MDP induced by the updated belief will be a better choice.
In contrast, if compiling and solving an MDP is costly or the belief itself is not worthy of iterative update, one might as well only compute the optimal policy of the MDP induced by a cautiously selected initial belief and commit to it until the end of the game. 

%Usually, computation done before the first action is viewed as initialization cost and computation done per action onwards will be viewed as replanning cost. It is acceptable that the initialization cost can be reasonably larger than the replanning cost. 

%Not seen in the POMDP community, actually observation-MDP. The  agent pretends she is faced with a MDP instead of a POMDP, and the MDP is built over the set of observations. Therefore, it has some inconsistency issue while planning ahead.



By Equation (\ref{eq:belief_mdp}) we generalize what is called \textit{independent learners} (ILs)~\cite{claus1998dynamics} in repeated games, which evaluate the accumulated return of the modelling agents with the opponents subsumed in the environment as stationary noises and best respond accordingly, to the ones that act under the same principle in stochastic games. 

\subsubsection{Belief-mixed MDP (QMDP)}

One may want to figure out the best response against each opponent model in the first place and then mix them afterwards according to the real-time belief, i.e.,
the action choice at each time step $t$ is hence given as
\begin{equation}
\begin{split}
a_t^* 
& \in \arg\max_{a_i\in \mathcal{A}_i}  \sum_{\pi_{-i} \in \Pi_{-i}} b_t(\pi_{-i})\cdot
	Q_{\mathcal{M}(\pi_{-i})}(S, a_i) \\
& = \arg\max_{a_i\in \mathcal{A}_i} \sum_{(S, \pi_{-i}) \in \mathcal{S}\times \Pi_{-i}} \mathbb{P}[(S, \pi_{-i})] \cdot Q((S, \pi_{-i}), a_i)  \\
\end{split}
\label{eq:qmdp}
\end{equation}
where $Q_{\mathcal{M}(\pi_{-i})}$ is the optimal Q-function of $\mathcal{M}(\pi_{-i})$ that can be solved upfront, but there are in total $|\Pi_{-i}|$ MDPs to solve, which grows exponentially in terms of the number of agents.
%Therefore, it is equivalent to evaluating the value of the current information state as
%\[
%\begin{split}
%& V_i(S, b_t) \\
%& \gets \max_{a_i\in A_i}\{\mathbb{E}_{\pi_{-i} \sim b_t, a_{-i}\sim \pi_{-i}}[R_i(S,a) +  \gamma \sum_{S'}T(S'|S,a) V_i(S', \pi_{-i})] \\
%%& V_i(S, \pi_{-i}) =
%%\max_{a_i\in A_i}
%%\{
%%\mathbb{E}_{a_{-i}\sim \pi_{-i}}[R_i(S,a) +  \gamma \sum_{S'}T(S'|S,a) V_i(S', \pi_{-i})]
%%\}
%\end{split}	
%\]
%where $V_i(\cdot, \pi_{-i})$ is the optimal value function of $\mathcal{M}(\pi_{-i})$.
In fact,  the second line is exactly the QMDP formulation from the POMDP community~\cite{littman1995learning},
which pretends that the modelling agent can observe the underlying opponent model, and the Q-function $Q((S, \pi_{-i}), a_i)$ is obtained by optimally solving the underlying hypothetical MDP.
%is consequently obtained by solving the hypothetical MDP.
The equality holds because the opponents are assumed not evolving their strategies, and therefore, those factorized MDPs, namely all such $\mathcal{M}(\pi_{-i}), \forall \pi_{-i} \in \Pi_{-i}$, are independent of each other.
%$Q_{\mathcal{M}(\pi_{-i})}$ is therefore the Q-function obtained by optimally solving the MDP $\mathcal{M}(\pi_{-i})$.

%QMDP is a formulation exactly borrowed from the POMDP community \cite{littman1995learning} but indeed offers extra semantics under the multi-agent context. In our case, the underlying CMDP can actually be factorized in to sub-MDPs conditioning on opponent models. 
%The formulation of QMDP suggests that one can first solve all sub-MDPs, and then mix the respective Q-values during the game play according to beliefs. 



%It assumes the agent is aware of the underlying MDP and solves it in the first place. At each move during execution, the agent just mix the Q-value according the belief at that time.


%The good news is, all the sub-MDPs can be solved in advance, the agent only has to mix the Q-values on demand during execution. The bad new is, there are in total $|\Pi_{-i}|$ sub-MDPs to solve, which grows exponentially in terms of the number of agents.

Moreover, Equation (\ref{eq:qmdp}) extends 
\textit{joint action learners} (JALs)~\cite{claus1998dynamics} in repeated games to the ones that act under the same principle in stochastic games. A JAL learns Q-values with respect to all possible opponent actions, and then assesses the best expected return by mixing the Q-values over the observed action distribution of the opponent.
Clearly, ours generalizes opponent actions (in single-state scenarios) to presumed opponent policies (in multi-state scenarios), on which a corresponding Q-function is derived.

%\textit{We postpone the pseudocode for these three planners to Appendix~B, and some case studies to Appendix~D.}
\textit{We postpone the pseudocode for these three planners to Appendix~\ref{apd:planner_detail}, and some case studies to Appendix~\ref{apd:case_study}.}

%\subsubsection{NE as Constraint Satisfaction}
%Despite that we have argued NE as a solution concept offers little help,
%an agent can still compute one, if possible, and execute the NE strategy expecting that the others will do the same.
%Computing NE is in general computationally intractable.
%Nevertheless, there is a special case that greatly alleviates the heavy computation when
%\begin{enumerate}
%	\item the transition is deterministic, i.e. $T: \mathcal{S} \times \mathcal{A}_1 \times \cdots \mathcal{A}_n \mapsto \mathcal{S}$,
%	\item there is a particular goal state $G \in \mathcal{S}$ and set of conflicting states $\mathcal{S}_c \subset \mathcal{S}$, such that the utilities are shaped as $\forall i \in N$,
%	\[u_i(S, a) 
%	\begin{cases}
%		> 0, & \text{if } T(S, a) = G,\\
%		< - c_{p}, & \text{if } T(S, a) \in \mathcal{S}_c,\\	
%		\in [- c_0 , 0], & \text{otherwise},\\	
%	\end{cases}
%	\]
%\end{enumerate}
%where $c_p \gg c_0 \geq 0$.
%Intuitively, such special SGs are resulted from those goal-conditioned planning problems where all agents try to reach the goals while avoiding certain situations with severe conflicting interests, e.g., classic MAPF problems \cite{stern2019multi-overview, stern2019multi}.
%Every NE under this setting will drive all agents away from states in $\mathcal{S}_c$, if possible, and leads, to some extent, to shortest trajectories from every possible initial state $S_0$ to the goal state $G$. 
%
%Instead of numerical utilities, constraints are therefore probably the right language to describe the interaction among agents. More specifically, one can use a predicate $Nogood(S, a_1, \cdots, a_n)$ to rule out solutions that allow agents to simultaneously perform $(a_1, \cdots, a_n)$ in state $S$, if $T(S, a) \in \mathcal{S}_c$.
%Consequently, to compute NE, a highly applicable but usually neglected solution is to appeal to constraint satisfaction and optimization techniques, to make great use of constraint propagation.
%The resulted plan can be seen as an approximate ex-post NE (w.r.t initial states) for the original stochastic game, as those state-action pairs with large penalties have been converted to hard constraints.
%The solving process will be drastically sped up in the reason that constraints usually propagate efficiently in practice.
%Exemplary solvers can be found mostly in the MAPF literature, e.g.,
%optimal planners\cite{sharon2015conflict, zhang2022multi},
%bounded sub-optimal planners\cite{li2021eecbs},
%universal planners\cite{zhu2024computinguniversalplanspartially}, etc.
%We postpone the elaboration on how to cast conflict-free plans to state-action values to Appendix \ref{apd:exp_setting}. For now, one can assume that an oracle for computing an $\epsilon$-NE is given.
%
%%An ex-post sample NE is easier than BNE 
%
%\subsubsection{Contextual-RL}
%In fact, equation (\ref{eq:pomdp}) also leads to potential RL solutions.
%To this end, one needs to wrap the given multi-agent environment as a single-agent learning environment, by (i) first initializing each episode by sampling opponent strategies according to the initial belief, and (ii) then proceeding the environment by the given action and the sampled opponent actions, updating the belief, and returning it to the learning agent.
%
%Consequently, the most challenging part lies in how to efficiently train such a policy that converges to the desired optimum.
%As we will show in Appendix \ref{apd:exp_setting}, there is usually a clear intermediate plateau phase before convergence, which we conjecture is caused due to the gap between feasible strategies and optimal strategies. 
%
%%actor-critic?



\section{Unification}

\label{sec:unification}

%\subsection{The Framework}
In this section, we propose a unified planning framework conceptually based on tree search.
The general framework has all the aforementioned formulations embedded and even sheds lights on new ones.
%It is characterized recursively as a three-level planner.
Given the state $S_t$ and the belief $b_t$ at each step $t$, we do \textit{three layers of lookahead search} rooted at the node $(S_t, b_t)$, denoted as $(S^0_t, b^0_t)$ inside the tree (with $t$ omitted below),

%Belief-updated lookahead for $n$ levels, i.e., for $l\in[0, n)$:
%	\[
%		V_i(S^{l},b^l)
%		\gets \max_{a_i\in A_i}
%%		\{\mathbb{E}_{\pi_{-i} \sim b^{l}}
%		\sum_{\pi_{-i} \in b^{l}}
%		[R_i(S^{l},a)
%		+  \gamma \sum_{S^{l+1}}T(S^{l+1}|S^l,a) V_i(S^{l+1}, b^{l+1})],
%	\label{eq:belief_bellman}
%	\]
%	where $b^{l+1} = \xi(b^l, S^l, a)$,
%%	we omit $b^{l+1}$ under the summation as the outcome $\xi$ is deterministic.
%	
%Belief-fixed lookahead for $m$ levels, i.e., for $l\in [n, n+m)$:
%	\[
%		V_i(S^{l},b^{n})
%		\gets \max_{a_i\in A_i}
%%		\{\mathbb{E}_{\pi_{-i} \sim b^{n}}
%		\sum_{\pi_{-i} \in b^{n}}
%		[R_i(S^{l},a)
%		+  \gamma \sum_{S^{l+1}}T(S^{l+1}|S^l,a)V_i(S^{l+1}, b^{n})],
%	\label{eq:van_bellman}
%	\]
%
%Heuristic evaluation for leaf nodes in level $(n+m)$:
%	\[
%		V_i(S^{n+m},b^{n}) \gets \textsc{Eval}_i(S^{n+m},b^{n}),
%	\]
%	% eval is actually best response?
%	where $\textsc{Eval}_i(\cdot)$ can be any heuristic function that estimates a reasonable value for the modelling agent $i$.
\begin{enumerate}[left=0pt, itemsep=0pt]
	\item Belief-updated lookahead for the first $n$ levels, i.e., for $l\in[0, n)$:
	\[
		V_i(S^{l},b^l)
		\gets \max_{a_i\in \mathcal{A}_i}
%		\{\mathbb{E}_{\pi_{-i} \sim b^{l}}
		\sum_{ \substack{\pi_{-i} \in b^{l}, \\ a_{-i} \in \pi_{-i}} }
		[R_i(S^{l},a)
		+  \gamma \sum_{S^{l+1}}T(S^{l+1}|S^l,a) V_i(S^{l+1}, b^{l+1})],
	\label{eq:belief_bellman}
	\]
	where $b^{l+1} = \xi(b^l, S^l, a)$,
%	we omit $b^{l+1}$ under the summation as the outcome $\xi$ is deterministic.
	
	\item Belief-fixed lookahead for the next $m$ levels, for $l\in [n,n+m)$:
	\[
		V_i(S^{l},b^{n})
		\gets \max_{a_i\in \mathcal{A}_i}
%		\{\mathbb{E}_{\pi_{-i} \sim b^{n}}
		\sum_{ \substack{\pi_{-i} \in b^{n}, \\ a_{-i} \in \pi_{-i}} }
		[R_i(S^{l},a)
		+  \gamma \sum_{S^{l+1}}T(S^{l+1}|S^l,a)V_i(S^{l+1}, b^{n})],
	\label{eq:van_bellman}
	\]
	\item Heuristic evaluation for the leaf nodes in level $(n+m)$:
	\[
		V_i(S^{n+m},b^{n}) \gets \textsc{Eval}_i(S^{n+m},b^{n}),
	\]
	% eval is actually best response?
	where $\textsc{Eval}_i(\cdot)$ can be any heuristic value function that estimates a reasonable future return for the modelling agent $i$.
\end{enumerate}
Note that, for long-horizon planning problems, especially goal-conditioned ones, the planner is supposed to lookahead in depth to ensure non-trivial backup from leaf nodes, which implies that either $(n+m)$ should be large enough or $\textsc{Eval}_i$ should be sufficiently informative.
In fact, the possession of such a heuristic enables one to evaluate any given situation by directly enquiring $\textsc{Eval}_i(S_t,b_t)$, without doing any lookahead search.
\textit{Nevertheless, the lookahead search by the first two layers is highly necessary as it serves as an online policy improvement operator.}
%We postpone the proof of this rather intuitive statement to Appendix~A.
We postpone the proof of this rather intuitive statement to Appendix~\ref{apd:converge}.

%\subsection{Rediscovery of Previous Formulations}
By various options of $(n, m, \textsc{Eval}_i)$,
we here reproduce the aforementioned three formulations, and introduce a new forth one, 
%some of which can be instantiated to our aforementioned formulations,
\begin{enumerate}[label=\texttt{F\arabic*.}]
	\item If $n = \infty$, then it is equivalent to exactly solving the infinite-horizon POMDP \textbf{as Equation~(\ref{eq:pomdp})}, and there is no need to go to layer~(2) and (3), i.e., $m=0$ and $\textsc{Eval}_i(\cdot) = any$. No replanning will be needed.
	\item If $n=0, m=\infty, \textsc{Eval}_i(\cdot) = any$, then it is equivalent to solving the belief-induced MDP \textbf{as Equation~(\ref{eq:belief_mdp})}. One should note that $\mathcal{S}$ is a finite set. Therefore, layer (2) search can instead be implemented as dynamic programming to handle value backup on repeated states. 
	\item If $n=0$, $m=0$, and
	\[
%	\textsc{Eval}_i(S,b) =  \sum_{\pi_{-i}\in b} b(\pi_{-i})V_{\mathcal{M}(\pi_{-i})}(S),
	\textsc{Eval}_i(S,b) =
	\max_{a_i\in \mathcal{A}_i} \sum_{\pi_{-i}} b(\pi_{-i})\cdot Q_{\mathcal{M}(\pi_{-i})}(S, a_i),
	\]
%	where $V_{\mathcal{M}(\pi_{-i})}(S)$ is the value function for the MDP $\mathcal{M}(\pi_{-i})$, 
	then it is equivalent to the QMDP approach \textbf{as Equation~(\ref{eq:qmdp})}.
%	We postpone the equivalence proof to Appendix \ref{apd:planner_detail}. TODO??
	\item If $0<n<\infty, m=\infty$ and $\textsc{Eval}_i(\cdot) = any$, then it is equivalent to solving a (finite) $n$-horizon POMDP with terminal states evaluated by respective MDPs. Considerably, this will perform better then \texttt{F2}, but replanning will be needed. 
\end{enumerate}
Note that once $n =\infty$ or $m = \infty$, it does not matter which $\textsc{Eval}_i$ is adopted, as the backup operators in the first two layers are both $\gamma$-contractions which eventually lead to unique fixed points.
%The proof is postponed to Appendix~A.
The proof is postponed to Appendix~\ref{apd:converge}.
Although one can resort to off-the-shelf MDP solvers, formulations involving optimally solving MDPs will soon become computationally infeasible as the number of agents grows.
The framework then guides one to additional scalable implementations,
%when solving an MDP until convergence at each move might be computationally expensive,
\begin{enumerate}[label=\texttt{F\arabic*.}, start=5]
	\item $0<n < \infty$ and $0<m < \infty$, 
	then it is equivalent to solving a finite-horizon POMDP with terminal states evaluated as finite-horizon MDPs with terminal states evaluated by a heuristic.
	Chances are that belief update itself is not that costly. Then one may reallocate all the computational budget of $m$ to $n$, i.e., set $(n,m) \gets (n+m, 0)$, since the complexity of doing lookahead search is the same for the first two layers.	
\end{enumerate}
For both cases in \texttt{F5}, the choice of $\textsc{Eval}_i$ is no longer trivial.
It has to serve as an oracle that returns a heuristic value that is informative enough in a flash,
\textit{which is why we draw one's attention to contextual-RL policies in Section~\ref{sec:solutions} as well as efficient NE solvers later in Section~\ref{sec:benchmark}.}
Once such an oracle is ready, \texttt{F5} is all about improving (or sometimes, repairing) the oracle in real-time by online lookahead search.
%In fact, \texttt{F5} leads to a full-width \textsc{ExpectiMax} tree search paradigm as Algorithm~\ref{alg:uts} in Appendix~B,
In fact, \texttt{F5} leads to a full-width \textsc{ExpectiMax} tree search paradigm as Algorithm~\ref{alg:uts} in Appendix~\ref{apd:planner_detail},
while in the literature there is indeed some alternative framework like \textsc{Max}$^n$ tree search~\cite{stahl1993evolution, samothrakis2011fast}. However, the latter one aims at converging to NEs, which imposes much stronger assumptions on the opponents.

%As illustrated in Figure \ref{fig:ts}, the general framework is implemented as an \textsc{ExpectiMax} tree.
%The red triangle nodes are called ``MAX'' nodes, representing the states of the modelling agent $i$, while the orange diamond nodes are called ``EXP'' nodes, representing the hypothetical states\footnote{Also usually known as after-states.} that follow from states given agent $i$'s committed action. The green diamond nodes are not explicitly implemented as they are just conceptual ones to show that each $a_{-i}$ is drawn probabilistically from the potential policies based on the belief. Algorithm \ref{alg:ts} shows the skeleton of how to use an online tree search planner, and Algorithm \ref{alg:uts} shows the detailed procedure of how \textsc{ExpectiMax} backup works.


%\begin{figure}[!ht]
%	\centering
%	\includegraphics[width=80mm]{fig/tree_search.pdf}
%	\caption{A tree search diagram that unifies the spectrum of implementations}
%	\label{fig:ts}
%\end{figure}




One should notice that in order to perform exact backup and compute $\max\{\cdot\}$, the agent has to exhaustively enumerate all joint actions over belief distributions, which is of complexity $\bigr(|\mathcal{A}_i|\times \prod_{j\neq i}|\Pi_{j}|\times \prod_{j\neq i}|\mathcal{A}_j|\bigl)^{n+m}$ and hence expensive when the problem further scales up. 
The framework is then extended to alleviate the heavy computation, but unavoidably compromises the accuracy, by the following two implementations.

%compute $\max\{\cdot\}$ and do exact backup, the agent has to exhaustively enumerate all joint actions and perform exact backup over belief distributions, which is of complexity $\bigr(|\mathcal{A}_i|\times \prod_{j\neq i}|\Pi_{j}|\times \prod_{j\neq i}|\mathcal{A}_j|\bigl)^{n+m}$ and hence not negligible when the problem further scales up. The framework is then extended to alleviate the computation burden, but unavoidably compromises the accuracy, by the following two implementations.

\begin{enumerate}[label=\texttt{F\arabic*.}, start=6]
	\item Sample actions over belief distributions and use sample mean to approximate the exact backup.
	\item Use bandit-based lookahead search to approximate $\max\{\cdot\}$.
\end{enumerate} 
For bandit-based lookahead search to work, the planner has to utilize a node selection function that well balances exploration and exploitation, to eventually minimize accumulated regret~\cite{browne2012survey}.
Besides the most widely used UCT formula~\cite{kocsis2006bandit}, it is also proposed to use a certain prior policy to guide the selection, resulting the pUCT formula~\cite{rosin2011multi, silver2017mastering, schrittwieser2020mastering, schwartz2023bayes, schwartz2023combining}.
%
%Let $t$ denote any tree node during search, then $t.Q$ represents the accumulated values, $t.N$ represents the number of visits to this node, and $t.\pi$ represents the probability that leads to this node given by a prior policy.
%The node selection by pUCT is formally given as
%\begin{equation}
%	t_c^* \in \arg\max_{t_c \in t.children} \frac{t_c.Q}{t_c.N} + t_c.\pi \cdot \frac{\sqrt{\ln(t.N)}}{1 + t_c.N}\cdot C
%\end{equation}
%where $C= c_1 + \ln (\frac{t.N + c_2 + 1}{c_2})$, and $c_1$ and $c_2$ are two constants controlling the influence of the prior policy.
%
Note a the prior policy is usually not easy to acquire, which leads to the following discussion
on how to make use of NE strategies, to extract such priors as well as value estimates.
%
%Luckily, as we have mentioned in Section \ref{sec:solutions} and will illustrate later with experiments in Section \ref{sec:benchmark}, one can use two ways to obtain policy priors: 1) the approximate NE strategies computed by the constraint solver, and 2) policy predictions made by the actor trained by any actor-critic RL algorithm.
%An ultimate version integrating \texttt{F6} and \texttt{F7} leads to an opponent-aware MCTS  planner as Algorithm~\ref{alg:mcts} in Appendix~B.
An ultimate version integrating \texttt{F6} and \texttt{F7} leads to an opponent-aware MCTS  planner as Algorithm~\ref{alg:mcts} in Appendix~\ref{apd:planner_detail}.
It resembles the POMCP implementation~\cite{silver2010monte},
%and falls into the category of ``Nondeterministic MCTS'' in the survey article~\cite{browne2012survey}.
while we avoid using random Monte-Carlo rollouts as value estimates, since they usually cannot backpropagate useful information when the problem is goal-conditioned or the rewards are sparse.
\textit{With the help of efficient NE solvers, we scale the implementation up to suit 50-agent long-horizon planning problems, which to the best of our knowledge is the first opponent-modelling planner of this capability.}


To conclude, this unified framework draws a spectrum of planners.
On top of it, one can predict the performance of a planner devised within this framework, as the deeper (and more deliberately) it searches, the better it performs.
For example,
with the same choice of $\textsc{Eval}_i$, one can easily predict a performance ranking among them:
$\texttt{F1} \succcurlyeq \texttt{F4} \succcurlyeq \texttt{F2} \sim \texttt{F5} \succcurlyeq \texttt{F6} \sim \texttt{F7}$.
%by just looking at the fine-grained framework.
One can later see such a trend as expected in the experimental results in Table~\ref{tab:performance}.

%tree search mc simu?

%DP policy improvement?
%Performance ranking?

%The most widely used formula is the UCT formula \cite{kocsis2006bandit},
%\begin{equation}
%	t_c \in \arg\max_{t_c \in t.children} \frac{t_c.v}{t_c.N} + c\cdot \sqrt{\frac{\ln(t.N)}{t_c.N}}
%\end{equation}
%where $c$ is a constant controlling the weight of exploration and exploitation, with its best empirical value $\sqrt{2}$.
%It is also proposed to use certain prior policies to guide the choices, resulting the pUCT formula \cite{rosin2011multi, silver2017mastering, schrittwieser2020mastering},
%\begin{equation}
%	t_c \in \arg\max_{t_c \in t.children} \frac{t_c.v}{t_c.N} + t_c.policy\_prior \cdot \sqrt{\frac{\ln(t.N)}{t_c.N}}\Bigl(c_1 + \ln \bigl(\frac{t.N + c_2}{c_2}  \bigr)  \Bigr)
%\end{equation}
%where $c_1$ and $c_2$ are two constants controlling the influence of the prior policy, with their empirical values $c_1=1.25$ and $c_2=19625$.
%Usually the prior policy is not easy to acquire. Luckily in our situation, as we will show later in Section \ref{sec:benchmark}, the approximate NE strategies computed by the constraint solver can, to some extent, serve as prior policies.

%NE and actor-critic RL can both work.



\begin{table*}[!ht]
%\vspace{-3mm}
\footnotesize
\caption{Detailed description of the implemented planners under the unified framework.}
\vspace{-2mm}
\centering
\begin{tabular}{@{}c|cccccccc@{}}
\toprule
\textbf{Planners}    & \textbf{n}   & \textbf{m} & $\mathbf\beta$ & \textbf{Collision penalty} & \textbf{\textsc{Eval}$_i$} & \textbf{Backup}     & \textbf{Lookahead} & \textbf{Replanning} \\ \midrule
$A^*$-Agent          &              &            &                &                            &                                             &                     &                    &                          \\
Safe-Agent           &              & 1          &                & $\infty$                   & single-agent $A^*$                          & exact               & full-width         & \checkmark                \\
EnhancedSafe-Agent   &              & 1          & 1 $\to$ 0      & $\infty$                   & single-agent $A^*$                          & exact               & full-width         & \checkmark                \\ \midrule
MDPAgentFixedBelief  & 0            & $\infty$   &                & $< \infty$                 &                                          &    exact           &                    &                          \\
MDPAgentUpdateBelief & 0            & $\infty$   &                & $< \infty$                 &                                          &    exact            &                    & \checkmark                \\ \midrule
RLAgentFixedBelief   & $\infty$     & 0          &                & $< \infty$                 &                                &                     &                    &                          \\
RLAgentUpdateBelief  & $\infty$     & 0          &                & $< \infty$                 &                                &                     &                    &                          \\
UniformTSAgentRL     & $(0,\infty)$ &            &                & $< \infty$                 & contextual-RL                               & exact               & full-width         & \checkmark                \\ \midrule
CBSAgentFixedBelief  & 0            & 0          &                & $< \infty$                 & NE by EECBS                                 &                     &                    &                          \\
CBSAgentUpdateBelief & 0            & 0          &                & $< \infty$                 & NE by EECBS                                 &                     &                    & \checkmark                \\
UniformTSAgentCBS    & $(0,\infty)$ &            &                & $< \infty$                 & NE by EECBS                                 & \{exact, sampling\} & full-width         & \checkmark                \\
MCTSAgentCBSuct      &              & 0          &                & $< \infty$                 & NE by EECBS                                 & sampling            & bandit-based       & \checkmark                \\
MCTSAgentCBSpuct     &              & 0          &                & $< \infty$                 & NE by EECBS                                 & sampling            & bandit-based       & \checkmark                \\ \bottomrule
\end{tabular}
\label{tab:all_planners}
%\vspace{-5mm}
\end{table*}


%
%\subsubsection*{NE as Constraint Satisfaction}
%Despite that we have argued NE as a solution concept offers little help,
%an agent can still compute one, if possible, and execute the NE strategy expecting that the others will do the same.
%Computing NE is in general computationally intractable.
%Nevertheless, there is a special case that greatly alleviates the heavy computation when
%\begin{enumerate}
%	\item the transition is deterministic, i.e. $T: \mathcal{S} \times \mathcal{A}_1 \times \cdots \mathcal{A}_n \mapsto \mathcal{S}$,
%	\item there is a particular goal state $G \in \mathcal{S}$ and set of conflicting states $\mathcal{S}_c \subset \mathcal{S}$, such that the utilities are shaped as $\forall i \in N$,
%	\[u_i(S, a) 
%	\begin{cases}
%		> 0, & \text{if } T(S, a) = G,\\
%		< - c_{p}, & \text{if } T(S, a) \in \mathcal{S}_c,\\	
%		\in [- c_0 , 0], & \text{otherwise},\\	
%	\end{cases}
%	\]
%\end{enumerate}
%where $c_p \gg c_0 \geq 0$.
%Intuitively, such special SGs are resulted from those goal-conditioned planning problems where all agents try to reach the goals while avoiding certain situations with severe conflicting interests, e.g., classic MAPF problems \cite{stern2019multi-overview, stern2019multi}.
%Every NE under this setting will drive all agents away from states in $\mathcal{S}_c$, if possible, and leads, to some extent, to shortest trajectories from every possible initial state $S_0$ to the goal state $G$. 
%
%Instead of numerical utilities, constraints are therefore probably the right language to describe the interaction among agents. More specifically, one can use a predicate $Nogood(S, a_1, \cdots, a_n)$ to rule out solutions that allow agents to simultaneously perform $(a_1, \cdots, a_n)$ in state $S$, if $T(S, a) \in \mathcal{S}_c$.
%Consequently, to compute NE, a highly applicable but usually neglected solution is to appeal to constraint satisfaction and optimization techniques, to make great use of constraint propagation.
%The resulted plan can be seen as an approximate ex-post NE (w.r.t initial states) for the original stochastic game, as those state-action pairs with large penalties have been converted to hard constraints.
%The solving process will be drastically sped up in the reason that constraints usually propagate efficiently in practice.
%Exemplary solvers can be found mostly in the MAPF literature, e.g.,
%optimal planners\cite{sharon2015conflict, zhang2022multi},
%bounded sub-optimal planners\cite{li2021eecbs},
%universal planners\cite{zhu2024computinguniversalplanspartially}, etc.
%We postpone the elaboration on how to cast conflict-free plans to state-action values to Appendix \ref{apd:exp_setting}. For now, one can assume that an oracle for computing an $\epsilon$-NE is given.
%
%%An ex-post sample NE is easier than BNE 
%
%\subsubsection*{Contextual-RL}
%In fact, equation (\ref{eq:pomdp}) also leads to potential RL solutions.
%To this end, one needs to wrap the given multi-agent environment as a single-agent learning environment, by (i) first initializing each episode by sampling opponent strategies according to the initial belief, and (ii) then proceeding the environment by the given action and the sampled opponent actions, updating the belief, and returning it to the learning agent.
%
%Consequently, the most challenging part lies in how to efficiently train such a policy that converges to the desired optimum.
%As we will show in Appendix \ref{apd:exp_setting}, there is usually a clear intermediate plateau phase before convergence, which we conjecture is caused due to the gap between feasible strategies and optimal strategies. 



\section{Experiments}

\label{sec:benchmark}
\subsection{Benchmark}


\begin{table*}[t]
%\vspace{-10mm}
\footnotesize
\caption{Statistics in each scenario. The numbers in the ``lower bounds'' row are means (standard deviations).}
\vspace{-2mm}
\centering
\begin{tabular}{@{}r|cc|cc|cc|cc|cc@{}}
\toprule
                  & \multicolumn{2}{c|}{\textbf{Small2a (8x8 map)}}                          & \multicolumn{2}{c|}{\textbf{Square2a (12x12 map)}}                       & \multicolumn{2}{c|}{\textbf{Square4a (12x12 map)}}                       & \multicolumn{2}{c|}{\textbf{Medium20a (18x18 map)}}                         & \multicolumn{2}{c}{\textbf{Large50a (32x32 map)}}                          \\ \midrule
                  & \begin{tabular}[c]{@{}c@{}}Rational\\ Malicious\end{tabular} & Self       & \begin{tabular}[c]{@{}c@{}}Rational\\ Malicious\end{tabular} & Self       & \begin{tabular}[c]{@{}c@{}}Rational\\ Malicious\end{tabular} & Self       & \begin{tabular}[c]{@{}c@{}}Rational\\ Malicious\end{tabular}  & Self         & \begin{tabular}[c]{@{}c@{}}Rational\\ Malicious\end{tabular}  & Self        \\ \midrule
\#(states)        & \multicolumn{2}{c|}{930}                                                  & \multicolumn{2}{c|}{7310}                                                 & \multicolumn{2}{c|}{$5.47\times 10^7$}                           & \multicolumn{2}{c|}{$6.44\times 10^{46}$}                             & \multicolumn{2}{c}{$4.62\times 10^{145}$}                            \\
\#(joint-actions) & \multicolumn{2}{c|}{25}                                                   & \multicolumn{2}{c|}{25}                                                   & \multicolumn{2}{c|}{625}                                                  & \multicolumn{2}{c|}{$5^{20} \sim 9.54\times 10^{13}$} & \multicolumn{2}{c}{$5^{50} \sim 8.88\times 10^{34}$} \\
\#(op-types each) & \multicolumn{2}{c|}{30}                                                   & \multicolumn{2}{c|}{85}                                                   & \multicolumn{2}{c|}{85}                                                   & \multicolumn{2}{c|}{218}                                                     & \multicolumn{2}{c}{818}                                                     \\ \midrule
Lower bounds      & 4.16(2.14)                                                   & 4.26(1.54) & 6.78(3.34)                                                   & 6.79(2.54) & 6.61(3.35)                                                   & 6.85(1.77) & 11.12(5.56)                                                   & 11.32(1.19)  & 23.2(11.25)                                                   & 23.33(1.54) \\
Upper bounds      & \multicolumn{2}{c|}{32}                                                   & \multicolumn{2}{c|}{48}                                                   & \multicolumn{2}{c|}{48}                                                   & \multicolumn{2}{c|}{144}                                                      & \multicolumn{2}{c}{256}                                                     \\ \bottomrule
\end{tabular}
\label{tab:testbed}
\vspace{-1mm}
\end{table*}




\subsubsection{Multi-Agent Route Planning}
In the domain of \textit{multi-agent route planning} (MARP), a group of agents are placed in a grid world with possible obstacles. Each of them is designated a goal. 
We here name the benchmark as MAPP in order to drag one's attention away from the particular research area called \textit{multi-agent path finding} (MAPF)~\cite{stern2019multi-overview, stern2019multi}. 
In MAPF, this fleet of robots are supposed to find a set of paths to reach their goals from the initial positions without any collision among them.
%In other words, the robots must cooperate.
Usually, the set of paths is computed upfront by an efficient centralized planner~\cite{sharon2015conflict, li2021eecbs} and then robots are forced to strictly execute them.
In other words, all robots are under the control of a central planner.
By contrast, in our MARP setting, only one agent is under our control,
without knowing others' goals and strategies.
%who has no prior knowledge about other participating agents.
The agent will be positively rewarded when she arrives at her goal, and will be penalized if she hits others or gets hit.
The consequences of actions are deterministic and commonly known. The environment state consists of all agents' locations and is completely observable to everyone.
We borrow this domain for the reason that it provides us with problem instances of customizable scales.
Our experiments are conducted on configurations
%varying from 3x3 maps with 2 agents
up to 32x32 maps with 50 agents. 
%As in general the complexity grows exponentially (even doubly exponential) in terms of the size of the map and the number of agents, it might be a bit odd if we categorize the sizes by those two numbers.
%Instead, we here separate the categories by the practical computational limits of different planners, in Table \ref{tab:category}.
Compared with contemporary work~\cite{carmel1998explore, albrecht2015game, eck2020scalable},
%this benchmark is challenging for the reason that it is no longer repeated matrix games and deliberate long-horizon planning is definitely needed.
This benchmark is challenging for two reasons: (1)~it is no longer repeated games, instead, involves a large number of states and joint-actions, as shown in Table~\ref{tab:testbed}; (2)~deliberate long-horizon planning is definitely needed, as the rewards are goal-conditioned and hence sparse.


%\subsubsection{NE as Constraint Optimization}
\subsubsection{Nash Equilibrium}
An NE typically reveals a predictable future outcome. 
In this routing domain, given a set of initial positions and goals,
if a set of collision-free paths is found optimal up to certain metrics, e.g., minimizing the sum of lengths, it can serve as an NE as no agent will unitarily deviate in the sense of finding a shorter path without colliding to any others.
In general, it is an NP-hard problem~\cite{yu2013structure}.
However, efficient planners can solve it very fast in practice, by making good use of constraint propagation, e.g., conflict-based search~\cite{sharon2015conflict}, its bounded sub-optimal variant~\cite{li2021eecbs}, and even some universal planner~\cite{zhu2024computinguniversalplanspartially}.

%% collision-free path CSP
%In this routing domain, given a set of initial positions and goals,
%a set of collision-free paths can be computed upfront by a central planner.
%In general, it is an NP-hard problem~\cite{yu2013structure}.
%However, efficient planners can solve it very fast in practice, by making good use of constraint propagation, e.g., conflict-based search~\cite{sharon2015conflict}, and its bounded sub-optimal variant~\cite{li2021eecbs}.

% as NE and online repair
%One can easily see that if the set of collision-free paths is optimal up to certain metrics,
%it also serves as an (approximate-)NE of the underlying stochastic game.
Despite our argument that NE as a solution concept does not \textit{directly} offer much help,
an agent can still compute one by sampling hypothetical opponents from her belief, and make a smart use of it: (1) either directly execute the NE strategy expecting that the others will do the same, or (2) transform it into the $\textsc{Eval}_i$ heuristic as we mentioned, and therefore, can plug it into our tree search framework to repair it in an online manner.
\textit{A heuristic extracted from NEs is considerably more informative than random Monte Carlo rollouts.}
%Detailed procedures will be revealed in Appendix~C.
Detailed procedures will be revealed in Appendix~\ref{apd:exp_setting}.


% MARL
For other domains, one can always appeal to MARL methods for NEs, e.g., MADDPG~\cite{lowe2017multi}, QMIX~\cite{rashid2020monotonic}, IPPO/MAPPO~\cite{yu2022the}, etc. 

%Despite that we have argued NE as a solution concept offers little help,
%an agent can still compute one, if possible, and execute the NE strategy expecting that the others will do the same.


%
%\subsubsection{NE as Constraint Satisfaction}
%Despite that we have argued NE as a solution concept offers little help,
%an agent can still compute one, if possible, and execute the NE strategy expecting that the others will do the same.
%Computing NE is in general computationally intractable.
%Nevertheless, there is a special case that greatly alleviates the heavy computation when
%\begin{enumerate}
%	\item the transition is deterministic, i.e. $T: \mathcal{S} \times \mathcal{A}_1 \times \cdots \mathcal{A}_n \mapsto \mathcal{S}$,
%	\item there is a particular goal state $G \in \mathcal{S}$ and set of conflicting states $\mathcal{S}_c \subset \mathcal{S}$, such that the utilities are shaped as $\forall i \in N$,
%	\[u_i(S, a) 
%	\begin{cases}
%		> 0, & \text{if } T(S, a) = G,\\
%		< - c_{p}, & \text{if } T(S, a) \in \mathcal{S}_c,\\	
%		\in [- c_0 , 0], & \text{otherwise},\\	
%	\end{cases}
%	\]
%\end{enumerate}
%where $c_p \gg c_0 \geq 0$.
%Intuitively, such special SGs are resulted from those goal-conditioned planning problems where all agents try to reach the goals while avoiding certain situations with severe conflicting interests, e.g., classic MAPF problems \cite{stern2019multi-overview, stern2019multi}.
%Every NE under this setting will drive all agents away from states in $\mathcal{S}_c$, if possible, and leads, to some extent, to shortest trajectories from every possible initial state $S_0$ to the goal state $G$. 
%
%Instead of numerical utilities, constraints are therefore probably the right language to describe the interaction among agents. More specifically, one can use a predicate $Nogood(S, a_1, \cdots, a_n)$ to rule out solutions that allow agents to simultaneously perform $(a_1, \cdots, a_n)$ in state $S$, if $T(S, a) \in \mathcal{S}_c$.
%Consequently, to compute NE, a highly applicable but usually neglected solution is to appeal to constraint satisfaction and optimization techniques, to make great use of constraint propagation.
%The resulted plan can be seen as an approximate ex-post NE (w.r.t initial states) for the original stochastic game, as those state-action pairs with large penalties have been converted to hard constraints.
%The solving process will be drastically sped up in the reason that constraints usually propagate efficiently in practice.
%Exemplary solvers can be found mostly in the MAPF literature, e.g.,
%optimal planners\cite{sharon2015conflict, zhang2022multi},
%bounded sub-optimal planners\cite{li2021eecbs},
%universal planners\cite{zhu2024computinguniversalplanspartially}, etc.
%We postpone the elaboration on how to cast conflict-free plans to state-action values to Appendix \ref{apd:exp_setting}. For now, one can assume that an oracle for computing an $\epsilon$-NE is given.
%
%%An ex-post sample NE is easier than BNE 

\subsubsection{A Special Case: Safe Agents}
Given this domain, we now present one simple-yet-powerful agent which is a myopic special case of the unified tree search framework, called \textit{safe-agents}. 
An action $a_i \in \mathcal{A}_i$ in state $S$ is unsafe if there exists another agent $j$ and an action $a_j \in \mathcal{A}_j$ that will drive agent $i$ and $j$ into collision in the successor state.
For each step, a \textit{safe-agent} first rules out all unsafe actions and takes the best one among the rest. By ``the best action'' here, we mean that the agent simply finds the action through which the shortest distance (ignoring other agents) to her goal can be minimized. 
% relation with tree search
In fact, one can easily construct a counter-example to get a \textit{safe-agent} stuck at an intermediate position forever.
For example, if one of the opponents stops right in the way of a \textit{safe-agent}'s only shortest path to her goal, then this \textit{safe-agent} would rather stop forever than move to a nearby safe position since that leads to a longer path.
We can therefore make \textit{safe-agents} slightly cleverer by incorporating domain-specific knowledge: no matter how random an opponent is, she will no longer move once she reaches her goal.
This leads to the \textit{enhanced safe-agents}, who will simply treat a nearby agent that has stayed still for a sufficiently long time as an obstacle, and replan a shortest path to get around.

In the language of our unified framework, \textit{safe-agents} are induced by ignoring belief update, doing one-depth full-width lookahead search, using constraints to rule out unsafe actions, and evaluating the leaf nodes by values of the onward shortest paths.
The \textit{enhanced} version is further obtained by descending the belief temperature $\beta$ from 1 to 0 when the aforementioned situations are detected, making the belief distribution a hard-max one. In this sense, these two \textit{safe-agent} variants are both online improved versions of $A^*$ agents (ignoring opponents) by one depth of tree search. 

%\begin{table}[t]
%\centering
%\begin{tabular}{cccccc}
%        & \begin{tabular}[c]{@{}c@{}}POMDP\\ (offline)\end{tabular} & \begin{tabular}[c]{@{}c@{}}MDP\\ (online)\end{tabular} & \begin{tabular}[c]{@{}c@{}}MDP\\ (offline)\end{tabular} & \begin{tabular}[c]{@{}c@{}}Lookahead\\ (full-width)\end{tabular} & \begin{tabular}[c]{@{}c@{}}Lookahead\\ (bandit)\end{tabular} \\ \hline
%Small   & \checkmark                                 & \checkmark                              & \checkmark                               & \checkmark                                        & \checkmark                                    \\
%Medium  &                                                           & \checkmark                              & \checkmark                               & \checkmark                                        & \checkmark                                    \\
%Large   &                                                           &                                                        & \checkmark                               & \checkmark                                        & \checkmark                                    \\
%Huge    &                                                           &                                                        &                                                         & \checkmark                                        & \checkmark                                    \\
%Massive &                                                           &                                                        &                                                         &                                                                  & \checkmark                                   
%\end{tabular}
%\caption{Categories of problem sizes in the MARP domain.}
%\label{tab:category}
%\end{table}

%\vspace{-1mm}


\subsubsection{Setup}
We now instantiate each theoretic module to the corresponding domain-specific implementation.

\textit{Belief Initialization and Update.}
Since the specific goal of each opponent is not revealed to the modelling agent, the modelling agent therefore associates each of the others a \textit{uniform} initial belief over all empty cells.
She also assumes everyone else is a goal-directed agent up to some randomness. More precisely, for each opponent $j$, for each empty cell $g_k$ in the map layout, a plan will be hypothesized as a shortest path tree from all possible positions to $g_k$, denoted as $\pi_j^k$ and $\mathbb{P}[\pi_j^k] = 1 / \#(empty\ cells)$.
 The modelling agent $i$ assumes agent $j$ will play $\pi_j^k$ with probability $(1-\epsilon)\cdot~\mathbb{P}[\pi_j^k]$, and go to a random adjacent position with probability $\epsilon$. 
 As for belief update, we simply use Equation (\ref{eq:bayes}) with $\beta=1$ except for the \textit{enhanced safe-agents}. Intuitively, when $\epsilon \to 0$, the term $\pi_j^k(a_j|s)\cdot~b_t(\pi_j^k)$ will be zero if agent $j$ does not follow the course of shortest-path actions towards $g_k$.
 Therefore, the belief update,
% with non-zero $\epsilon$,
 to some extent, actually makes soft inference about opponents' true goals.

\textit{Solvers.}
We use 
% the open-source software \textit{pomdp-solve}\footnote{http://www.pomdp.org/code/index.html} as the POMDP solver, and
\textit{mdptoolbox}\footnote{https://pymdptoolbox.readthedocs.io/en/latest/api/mdptoolbox.html} as the MDP solver, 
EECBS
\footnote{https://github.com/Jiaoyang-Li/EECBS}~\cite{li2021eecbs} as the constraint-based approximate NE solver, which returns joint-plans whose lengths are no more than a user-specified factor away from the optimum, and PPO~\cite{schulman2017proximal} implemented by \textit{stable-baseline3}\footnote{https://stable-baselines3.readthedocs.io/en/master/}~\cite{stable-baselines3} as the contextual-RL algorithm.

%\textit{RL setup.} We wrap our environment as a gym-like one, and then use PPO \cite{schulman2017proximal} implemented by \textit{stable-baseline3}\footnote{https://stable-baselines3.readthedocs.io/en/master/} \cite{stable-baselines3} to train our modelling agent. We also tried other alternatives like DQN and A2C, but they do not end up with acceptably good returns. Figure \ref{fig:rl} shows a sample experiment of the training phase, there is clearly an intermediate plateau before convergence. It is conjectured that in a certain phase the modelling agent does find a feasible plan to reach the goal without any collision, and in the later phase she eventually manages to find a much shorter plan.

\textit{Hardware.}
The RL experiments are conducted on a Linux machine with one NVIDIA 2080Ti GPU. The rest are done on a Mac mini (M1 CPU, 16GB memory) with multiprocessing.


%
%\begin{table*}[!ht]
%%\vspace{-5mm}
%\small
%\caption{Detailed implementation of the provided planners.}
%\centering
%\begin{tabular}{@{}c|cccccccc@{}}
%\toprule
%\textbf{Planners}    & \textbf{n}   & \textbf{m} & $\mathbf\beta$ & \textbf{Collision penalty} & \textbf{\textsc{Eval}$_i$} & \textbf{Backup}     & \textbf{Lookahead} & \textbf{Need replanning} \\ \midrule
%$A^*$-Agent          &              &            &                &                            &                                             &                     &                    &                          \\
%Safe-Agent           &              & 1          &                & $\infty$                   & single-agent $A^*$                          & exact               & full-width         & \checkmark                \\
%EnhancedSafe-Agent   &              & 1          & 1 $\to$ 0      & $\infty$                   & single-agent $A^*$                          & exact               & full-width         & \checkmark                \\ \midrule
%MDPAgentFixedBelief  & 0            & $\infty$   &                & $< \infty$                 & MDP                                         &                     &                    &                          \\
%MDPAgentUpdateBelief & 0            & $\infty$   &                & $< \infty$                 & MDP                                         &                     &                    & \checkmark                \\ \midrule
%RLAgentFixedBelief   & $\infty$     & 0          &                & $< \infty$                 & contextual-RL                               &                     &                    &                          \\
%RLAgentUpdateBelief  & $\infty$     & 0          &                & $< \infty$                 & contextual-RL                               &                     &                    &                          \\
%UniformTSAgentRL     & $(0,\infty)$ &            &                & $< \infty$                 & contextual-RL                               & exact               & full-width         & \checkmark                \\ \midrule
%CBSAgentFixedBelief  & 0            & 0          &                & $< \infty$                 & NE by EECBS                                 &                     &                    &                          \\
%CBSAgentUpdateBelief & 0            & 0          &                & $< \infty$                 & NE by EECBS                                 &                     &                    & \checkmark                \\
%UniformTSAgentCBS    & $(0,\infty)$ &            &                & $< \infty$                 & NE by EECBS                                 & \{exact, sampling\} & full-width         & \checkmark                \\
%MCTSAgentCBSuct      &              & 0          &                & $< \infty$                 & NE by EECBS                                 & sampling            & bandit-based       & \checkmark                \\
%MCTSAgentCBSpuct     &              & 0          &                & $< \infty$                 & NE by EECBS                                 & sampling            & bandit-based       & \checkmark                \\ \bottomrule
%\end{tabular}
%\label{tab:all_planners}
%%\vspace{-5mm}
%\end{table*}


\begin{table*}[t]
%\vspace{-3mm}
\footnotesize
\caption{Detailed comparison across all planners. The numbers are means (standard deviations). Those numbers in bold fonts are the best ones in each scenario. Slots filled with ``\textbf{/}'' means the corresponding planners are computationally infeasible at that problem scale. We also copy the leftmost three columns of ``\textbf{Small2a}'' and ``\textbf{Square2a}'' to the lower table for better  comparison.} 
\centering
\begin{tabular}{@{}>{\raggedright\arraybackslash}p{1.2cm}>{\raggedright\arraybackslash}p{1cm}|>{\raggedright\arraybackslash}p{1.4cm}>{\raggedright\arraybackslash}p{1.2cm}>{\raggedright\arraybackslash}p{1.8cm}|>{\raggedright\arraybackslash}p{1.5cm}>{\raggedright\arraybackslash}p{1.5cm}>{\raggedright\arraybackslash}p{1.5cm}>{\raggedright\arraybackslash}p{1.5cm}>{\raggedright\arraybackslash}p{2cm}@{}}
%\begin{tabular}{@{}>{\raggedright\arraybackslash}p{1.2cm}>{\raggedleft\arraybackslash}p{1cm}|>{\raggedleft\arraybackslash}p{1.4cm}>{\raggedleft\arraybackslash}p{1.2cm}>{\raggedleft\arraybackslash}p{1.8cm}|>{\raggedleft\arraybackslash}p{1.5cm}>{\raggedleft\arraybackslash}p{1.5cm}>{\raggedleft\arraybackslash}p{1.5cm}>{\raggedleft\arraybackslash}p{1.5cm}>{\raggedleft\arraybackslash}p{2cm}@{}}
\toprule
                   &           & \textbf{Astar} & \textbf{Safe}         & \textbf{EnhancedSafe} & \textbf{MDPFixed}   & \textbf{MDPUpdate}  & \textbf{RLFixed} & \textbf{RLUpdate} & \textbf{UnifTSRL} \\ \midrule
\textbf{Small2a}     & Rational  & 7.25(9.18)     & 7.33(8.90)             & 4.95(3.67)            & 6.62(7.85)          & 4.76(2.91)          & 5.92(6.91)         & 5.74(6.56)          & 5.06(4.51)             \\
\textbf{}          & Malicious & 12.33(13.11)   & 5.18(3.04)            & 5.18(3.04)            & 5.00(2.84)           & \textbf{4.96(2.79)} & 9.59(11.14)        & 9.47(11.05)         & 6.45(6.91)             \\
                   & Self      & 9.19(10.92)    & 9.96(10.89)           & 5.98(5.95)            & 9.35(10.62)         & \textbf{4.96(2.94)} & 6.44(7.54)         & 6.18(7.11)          & 5.56(5.38)             \\ \midrule
\textbf{Square2a}  & Rational  & 9.42(10.82)    & 9.60(10.85)            & 7.14(3.87)            & 8.90(9.49)           & \textbf{/}          & 9.44(10.82)        & 9.31(10.55)         & 8.19(7.84)             \\
\textbf{}          & Malicious & 17.40(18.64)    & 7.75(4.41)            & 7.75(4.41)            & \textbf{7.28(3.58)} & \textbf{/}          & 17.11(18.45)       & 17.10(18.45)         & 10.10(11.14)            \\
                   & Self      & 11.18(13.02)   & 11.60(13.12)           & 7.70(5.83)             & 10.91(12.38)        & \textbf{/}          & 12.08(14.13)       & 11.78(13.76)        & 8.92(9.08)             \\
%\midrule
%\textbf{Square4a}  & Rational  & 13.4(15.92)    & 13.53(15.41)          & 8.44(6.6)             & \textbf{/}          & \textbf{/}          & \textbf{/}         & \textbf{/}          & \textbf{/}             \\
%\textbf{}          & Malicious & 20.89(20.43)   & \textbf{11.43(10.68)} & 11.76(11.33)          & \textbf{/}          & \textbf{/}          & \textbf{/}         & \textbf{/}          & \textbf{/}             \\
%                   & Self      & 27.42(20.78)   & 27.45(19.3)           & 11.59(10.5)           & \textbf{/}          & \textbf{/}          & \textbf{/}         & \textbf{/}          & \textbf{/}             \\ \midrule
%\textbf{Medium20a} & Rational  & 45.97(31.4)    & 42.88(29.44)          & \textbf{28.1(22.86)}  & \textbf{/}          & \textbf{/}          & \textbf{/}         & \textbf{/}          & \textbf{/}             \\
%\textbf{}          & Malicious & 49.62(30.53)   & \textbf{28.02(23.59)} & 28.09(23.64)          & \textbf{/}          & \textbf{/}          & \textbf{/}         & \textbf{/}          & \textbf{/}             \\
%                   & Self      & 72.0(0.0)      & 43.37(9.85)           & 44.92(19.12)          & \textbf{/}          & \textbf{/}          & \textbf{/}         & \textbf{/}          & \textbf{/}             \\ \midrule
%\textbf{Random}    & Rational  & 93.43(52.09)   & 76.22(46.5)           & 57.36(42.21)          & \textbf{/}          & \textbf{/}          & \textbf{/}         & \textbf{/}          & \textbf{/}             \\
%\textbf{}          & Malicious & 99.09(49.02)   & 52.71(42.74)          & 53.05(42.72)          & \textbf{/}          & \textbf{/}          & \textbf{/}         & \textbf{/}          & \textbf{/}             \\
%                   & Self      & 128.0(0.0)     & 76.75(5.3)            & 112.47(27.51)         & \textbf{/}          & \textbf{/}          & \textbf{/}         & \textbf{/}          & \textbf{/}             \\
\bottomrule
\end{tabular}

\footnotesize
\centering
\begin{tabular}{@{}>{\raggedright\arraybackslash}p{1.2cm}>{\raggedright\arraybackslash}p{1cm}|>{\raggedright\arraybackslash}p{1.4cm}>{\raggedright\arraybackslash}p{1.2cm}>{\raggedright\arraybackslash}p{1.8cm}|>{\raggedright\arraybackslash}p{1.5cm}>{\raggedright\arraybackslash}p{1.5cm}>{\raggedright\arraybackslash}p{1.5cm}>{\raggedright\arraybackslash}p{1.5cm}>{\raggedright\arraybackslash}p{2cm}@{}}
%\begin{tabular}{@{}>{\raggedright\arraybackslash}p{1.2cm}>{\raggedleft\arraybackslash}p{1cm}|>{\raggedleft\arraybackslash}p{1.4cm}>{\raggedleft\arraybackslash}p{1.2cm}>{\raggedleft\arraybackslash}p{1.8cm}|>{\raggedleft\arraybackslash}p{1.5cm}>{\raggedleft\arraybackslash}p{1.5cm}>{\raggedleft\arraybackslash}p{1.5cm}>{\raggedleft\arraybackslash}p{1.5cm}>{\raggedleft\arraybackslash}p{2cm}@{}}
\toprule
                   &           & \textbf{Astar} & \textbf{Safe}         & \textbf{EnhancedSafe} & \textbf{CBSFixed} & \textbf{CBSUpdate} & \textbf{UnifTSCBS} & \textbf{MCTSCBSuct} & \textbf{MCTSCBSpuct} \\ \midrule
\textbf{Small2a}     & Rational  & 7.25(9.18)     & 7.33(8.90)             & 4.95(3.67)            & 6.85(8.62)        & 6.09(7.33)         & \textbf{4.74(3.19)}   & 5.48(5.07)          & 5.47(5.24)           \\
\textbf{}          & Malicious & 12.33(13.11)   & 5.18(3.04)            & 5.18(3.04)            & 12.30(13.08)       & 11.81(12.79)       & 5.32(4.08)            & 7.15(7.8)           & 6.93(7.53)           \\
                   & Self      & 9.19(10.92)    & 9.96(10.89)           & 5.98(5.95)            & 8.46(10.22)       & 6.77(8.09)         & 5.20(4.30)              & 5.48(4.21)          & 5.32(3.86)           \\ \midrule
\textbf{Square2a}  & Rational  & 9.42(10.82)    & 9.60(10.85)            & 7.14(3.87)            & 9.48(10.89)       & 8.74(9.41)         & \textbf{7.13(4.14)}   & 8.41(7.00)           & 8.27(6.95)           \\
\textbf{}          & Malicious & 17.40(18.64)    & 7.75(4.41)            & 7.75(4.41)            & 17.28(18.57)      & 16.77(18.22)       & 8.41(7.63)            & 12.77(13.89)        & 11.92(13.00)          \\
                   & Self      & 11.18(13.02)   & 11.60(13.12)           & 7.70(5.83)             & 10.94(12.71)      & 9.10(9.77)          & \textbf{7.05(3.16)}   & 7.98(5.17)          & 7.61(3.96)           \\ \midrule
\textbf{Square4a}  & Rational  & 13.40(15.92)    & 13.53(15.41)          & 8.44(6.60)             & 13.04(15.56)      & 11.50(13.72)        & \textbf{8.26(7.42)}   & 12.51(12.00)         & 11.71(11.47)          \\
\textbf{}          & Malicious & 20.89(20.43)   & $\textbf{11.43(10.68)}$ & 11.76(11.33)          & 20.06(20.03)      & 19.71(19.87)       & 13.65(15.29)          & 19.58(17.88)        & 18.54(17.71)          \\
                   & Self      & 27.42(20.78)   & 27.45(19.30)           & 11.59(10.50)           & 25.57(20.69)      & 19.14(18.92)       & \textbf{9.56(9.55)}   & 15.93(13.99)        & 14.04(12.80)           \\ \midrule
\textbf{Medium20a} & Rational  & 88.45(66.72)   & 73.92(62.80)           & \textbf{35.52(40.65)} & 86.14(67.10)       & 71.56(66.62)       & \textbf{/}            & 59.15(52.10)         & 56.04(52.59)          \\
\textbf{}          & Malicious & 96.27(64.83)   & \textbf{40.26(48.63)} & 40.62(49.03)          & 91.89(65.88)      & 90.82(65.98)       & \textbf{/}            & 64.97(56.53)        & 64.02(57.07)          \\
                   & Self      & 144.00(0.00)     & \textbf{49.27(28.73)} & 67.38(51.92)          & 144.00(0.00)        & 144.00(0.00)         & \textbf{/}            & /                   & /                     \\ \midrule
\textbf{Large50a}  & Rational  & 182.01(111.03) & 111.80(95.58)          & \textbf{74.60(78.27)}  & 187.32(108.28)    & 163.06(114.24)     & \textbf{/}            & 120.48(91.51)      & 119.26(96.18) \\
\textbf{}          & Malicious & 193.73(105.03) & \textbf{79.42(92.44)} & 79.84(92.48)          & 182.22(109.31)    & 188.22(106.83)      & \textbf{/}            & 145.88(106.18)      & 132.87(104.97)        \\
                   & Self      & 256.00(0.00)     & \textbf{76.65(5.30)}   & 209.15(82.39)         & /                 & /                  & \textbf{/}            & /                   & /                     \\ \bottomrule
\end{tabular}
\label{tab:performance}
%\vspace{-3mm}
\end{table*}




\subsection{Evaluation}

%\subsubsection{Empirical Results}
\label{sec:empirical}





%Although the framework provides us with such a spectrum of formulations that one can customize her own planner according to the given problem on hand, it is highly non-trivial to analyze their performances entirely in theory.
In this section, we conduct a comprehensive study over \textbf{13} planners resulted from our unified framework. 
Table~\ref{tab:all_planners} shows how each implemented planner is induced by the unified framework.
For each class of ``FixedBelief'' and its opposite ``UpdateBelief'', we mean whether the planner will update the belief after observing opponents' actions.
For example, ``MDPAgentFixedBelief'' means the planner only solves the MDP induced by the initial belief and does no replanning, while ``MDPAgentFixedBelief'' means the planner will update the belief at each step and solve a new MDP. ``RL'' agents  solve the underlying POMDP and thus no replanning is needed.

As we are investigating the setting where no prior information about opponents' goals or strategies are revealed to the modelling agents, the golden standard is to run each proposed planner against all possible hypothetical opponents and their combinations, which is clearly impossible.
To get around, we prepare three representative sets of opponents against which every planner will be tested for an average performance.

%\textit{Rational types}: a mixture of opponents that are, to some extent, goal-directed, mainly including three types:
%	\textit{ShortestPathAgent}, \textit{RandomAgent(p)} and \textit{SafeAgent}. A \textit{ShortestPathAgent} is an agent that goes towards her own goal ignoring other agents. A \textit{RandomAgent(p)} is a modified version who does random actions with probability $p$, while replans a shortest path to her goal with probability $(1-p)$. 
%
%\textit{Malicious types}: a mixture of opponents that sometimes intend to do harm to the modelling agent, what we call \textit{ChasingAgent(p)}. A \textit{ChasingAgent(p)} is an agent that plans a shortest path to the modelling agent with probability $p$, while plans a shortest path to her own goal with probability $(1-p)$. This type of malicious agents is dangerous since they significantly attacks the belief modelling process.
%
%\textit{Self-play}: opponents that run the same planners as everyone herself is an agent modelling others simultaneously. Empirical results under this scenario try to answer a potential question whether we can simply distribute the computation towards the ex-ante Bayesian NE over each individuals.

\begin{enumerate}
	\item ``Rational types'': a mixture of opponents that are, to some extent, goal-directed, mainly including three types:
	\textit{ShortestPathAgent}, \textit{RandomAgent(p)} and \textit{SafeAgent}. A \textit{ShortestPathAgent} is an agent that goes towards her own goal ignoring other agents. A \textit{RandomAgent(p)} is a modified version who does random actions with probability $p$, while replans a shortest path to her goal with probability $(1-p)$. 
	
	\item ``Malicious types'': a mixture of opponents that sometimes intend to do harm to the modelling agent, what we call \textit{ChasingAgent(p)}. A \textit{ChasingAgent(p)} is an agent that plans a shortest path to the modelling agent with probability $p$, while replans a shortest path to her own goal otherwise. 
	
	\item ``Self-play'': all agents run the same planner as every one is an autonomous agent modelling others simultaneously. 
\end{enumerate}
Experiments against ``rational'' opponents aim at simulating the situations where belief modelling more or less aligns with the underlying truth, while experiments against ``malicious'' ones examine whether belief-dependent planning will be broken down as  belief modelling is significantly attacked by the chasing behavior.
Empirical results under ``self-play'' scenarios try to figure out the gap between the outcome and the potential (but unreachable) ex-post NE (of the SG realized by specific goals).

Table~\ref{tab:testbed} presents some statistics for each testing scenario. For example, the one named \textbf{Medium20a} refers to a configuration with 20 agents initially randomly spawned on maps of size 18x18 with a few obstacles.
Each opponent will be associated with 218 hypothetical policies by the modelling agent.
The scenario contains around $6.44\times 10^{46}$ states and $9.54\times 10^{13}$ joint-actions under each state.
The lower bounds for ``rational/malicious'' cases are computed as the average lengths of single-agent shortest paths (ignoring all collisions) for the modelling agent, while the lower bounds for ``self-play'' cases are computed as the average lengths per agent of the multi-agent optimal joint paths (avoiding any collision). By definition, these lower bounds are impossible to reach and hence only for reference.
The upper bounds are the maximum number of steps allowed for route planning. Also, once any collision happens, the length of the route will be set to this upper bound.

%An alternative angle is to see \textit{safe-agent} variants as \textsc{Minimax} tree search that guarantees a not-that-bad worst case performance.



% trend
We attach the overall comparison among those 13 planners in Table \ref{tab:performance}.
The numbers are the average path lengths penalized by collisions.
%Additional details about raw path lengths and collision ratios are attached in Appendix~C.
Additional details about raw path lengths and collision ratios are attached in Appendix~\ref{apd:exp_setting}.
We intend to present Table~\ref{tab:performance} as a practical handbook for one to select a suitable planner, or devise their own ones in a similar way, given any problem in a certain scale, and therefore, answer the research question in Section~\ref{sec:intro}.

One can imagine Table~\ref{tab:performance} as an upper-triangular matrix, where each column corresponds to a particular planner, and each row shows the 
performances of all planners in a particular testing scenario with the best numbers in bold fonts. 
\textit{One should compare the numbers in each row, while their absolute values may not be meaningful as once a collision happens the penalty will be huge.}

\textbf{As an overview}, with the testing scenario scaling up, one can notice that the most capable planners gradually become computationally infeasible as they do deliberate computation, starting from ``{MDPUpdate}'', then ``{MDPFixed}'', finally ``{UnifTSCBS}''.
Understandably, when exhaustive enumeration of joint-actions becomes infeasible, the only options left are {MCTS} planners and \textit{safe-agent} variants. 
Despite that MCTS is in principle an anytime algorithm, if the increase of computational budget is not able to catch up with the inherent exponential complexity, it will definitely lead to compromised performance. 

\textbf{Look closer}, Table~\ref{tab:performance} indeed echoes the potential of tree search that serves as an online policy improvement operator, even while modelling a large number of opponents, e.g., by comparing vanilla RL/CBS agents against their improved tree search versions.
Additionally, MCTS planners with pUCT formula guided by NE policy priors beat those with vanilla UCT formula, but the gap is not significantly large, mainly because policy priors out of NE strategies presume others will do the same, which indeed causes some inconsistency.
One should also notice that CBS agents are never the best ones, which supports our argument that NE offers little help if one aims at computing effective strategies against priorly unknown opponents.
Contextual-RL here rather serves as a proof-of-concept example. Although online tree search further improves the learned contextual-policy, it is still challenging to invent a learning diagram that can be immediately applied to any configuration once trained.




\textbf{Unexpectedly,}  \textit{safe-agents} and its \textit{enhanced} versions perform surprisingly well, especially in cases of larger scales and against malicious opponents.
For large-scale instances, due to the limited computational budget, tree search agents are not capable of doing deliberate lookahead search to significantly outperform \textit{safe-agents}, while the latter ones easily beat all the rest in terms of fast replanning due to their simple algorithmic structure.
For situations against malicious opponents where belief modelling does not match the underlying truth, \textit{safe-agents} conservatively rule out all unsafe action and therefore guarantee a lower chance of collision. 
Alternatively, one highly interpretable angle is to see \textit{safe-agents} as \textsc{Minimax} tree search that offers a fairly good worst-case performance, especially when your opponents turn out to be harmful ones.

\textbf{For case-by-case recommendations}, Table~\ref{tab:performance} roughly shows the computational limit of each planner. One should always identify the scale of the problem at first and find the most suitable planner accordingly. For example, given a problem of $10^6$ states with three agents, we would definitely vote for two-depth full-width tree search with an NE oracle against MCTS. However, in a huge scale where value estimate oracles like RL/NE become out of reach, the only choice left is to devise rule-based planners such as \textit{safe-agents}, which at least offer certain conservative guarantees.

%\textit{For other planners that may not be suitable for a multi-run evaluation, we attach a few case studies about them in Appendix~D.}
\textit{For other planners that may not be suitable for a multi-run evaluation, we attach a few case studies about them in Appendix~\ref{apd:case_study}.}

%xxx
%
%xxx
%
%xxx
%
%xxx
%
%xxx

%It has clearly witnessed the trend that in scenarios with smaller maps and less agents, deliberate planners like replanning via MDP with updated belief or full-width tree search significantly outperform the others.
% Nevertheless, as the problem scales up where exhaustive enumeration of joint-actions becomes infeasible, the only options left are MCTS planners and \textit{safe-agent} variants. 
% Despite that MCTS is in principle an anytime algorithm, if the increase of computational budget is not able to catch up with the inherent exponential complexity, it will definitely lead to compromised performance. 
%
%% tree search policy improvement
%The good new is, Table \ref{tab:performance} indeed shows the potential of tree search that serves as an online policy improvement operator, even while modelling a large number of opponents, e.g., by comparing \textbf{UnifTSRL} against \textbf{RLxxx} or \textbf{UnifTSCBS}/\textbf{MCTSTSCBSxxx} against \textbf{CBSxxx}.
%Additionally, MCTS planners with pUCT formula guided by NE policy priors beat those with vanilla UCT formula, but the gap is not significantly large mainly because policy priors out of NE strategies presume others will do the same, which indeed causes some inconsistency.
%One should also notice that \textbf{CBSxxx}'s are never the best ones, which supports our argument that NE offers little help for the purpose of computing effective strategies.
%Contextual-RL here rather serves as a proof-of-concept example. Although online tree search further improves the learned contextual-policy, it is still challenging to invent a learning diagram that can be immediately applied to any instance once trained.

%% safe agents
%An unexpected observation is that \textit{safe-agents} and its \textit{enhanced} version performs surprisingly well, especially in cases of larger scales and against malicious opponents.
%For large-scale instances, tree search agents is not capable of doing deep lookahead search to significantly outperform \textit{safe-agents}, while the latter ones beat the former ones and the rest in terms of fast replanning due to their simple algorithmic structure.
%For situations against malicious opponents where belief modelling does not match the underlying truth, \textit{safe-agents} conservatively rule out all unsafe action and therefore guarantee a low chance of collision. 
%Alternatively, one highly interpretable angle is to see \textit{safe-agent} variants resembling \textsc{Minimax} tree search that offers a not-that-bad worst case performance. 

%\begin{table*}[!ht]
%\centering
%\small
%\begin{tabular}{@{}>{\raggedright\arraybackslash}p{1.5cm}>{\raggedright\arraybackslash}p{1.2cm}|>{\raggedright\arraybackslash}p{1cm}>{\raggedright\arraybackslash}p{2cm}|>{\raggedright\arraybackslash}p{1.8cm}>{\raggedright\arraybackslash}p{1.8cm}>{\raggedright\arraybackslash}p{1.8cm}>{\raggedright\arraybackslash}p{1.8cm}>{\raggedright\arraybackslash}p{2cm}|c@{}}
%\toprule
%                                    &           & \textbf{Safe} & \textbf{EnhancedSafe} & \textbf{MDPFixed} & \textbf{MDPUpdate} & \textbf{MetaFixed} & \textbf{MetaUpdate} & \textbf{UnifTSMeta} & \textbf{ref} \\ \midrule
%\multirow{3}{*}{\textbf{Small}}     & Rational  &               &                       &                   &                    &                    &                     &                        &              \\
%                                    & Malicious &               &                       &                   &                    &                    &                     &                        &              \\
%                                    & Self      &               &                       &                   &                    &                    &                     &                        &              \\ \midrule
%\multirow{3}{*}{\textbf{Square2a}}  & Rational  &               &                       &                   &                    &                    &                     &                        &              \\
%                                    & Malicious &               &                       &                   &                    &                    &                     &                        &              \\
%                                    & Self      &               &                       &                   &                    &                    &                     &                        &              \\ \midrule
%\multirow{3}{*}{\textbf{Square4a}}  & Rational  &               &                       &                   &                    &                    &                     &                        &              \\
%                                    & Malicious &               &                       &                   &                    &                    &                     &                        &              \\
%                                    & Self      &               &                       &                   &                    &                    &                     &                        &              \\ \midrule
%\multirow{3}{*}{\textbf{Medium20a}} & Rational  &               &                       &                   &                    &                    &                     &                        &              \\
%                                    & Malicious &               &                       &                   &                    &                    &                     &                        &              \\
%                                    & Self      &               &                       &                   &                    &                    &                     &                        &              \\ \midrule
%\multirow{3}{*}{\textbf{Random50a}} & Rational  &               &                       &                   &                    &                    &                     &                        &              \\
%                                    & Malicious &               &                       &                   &                    &                    &                     &                        &              \\
%                                    & Self      &               &                       &                   &                    &                    &                     &                        &              \\ \bottomrule
%\end{tabular}
%\vspace{1em}
%
%\centering
%\begin{tabular}{@{}>{\raggedright\arraybackslash}p{1.5cm}>{\raggedright\arraybackslash}p{1.2cm}|>{\raggedright\arraybackslash}p{1cm}>{\raggedright\arraybackslash}p{2cm}|>{\raggedright\arraybackslash}p{1.8cm}>{\raggedright\arraybackslash}p{1.8cm}>{\raggedright\arraybackslash}p{1.8cm}>{\raggedright\arraybackslash}p{1.8cm}>{\raggedright\arraybackslash}p{2cm}|c@{}}
%\toprule
%                                    &           & \textbf{Safe} & \textbf{EnhancedSafe} & \textbf{CBSFixed} & \textbf{CBSUpdate} & \textbf{UnifTSCBS} & \textbf{MCTSCBSucb} & \textbf{MCTSCBSpuct} & \textbf{ref} \\ \midrule
%\multirow{3}{*}{\textbf{Small}}     & Rational  &               &                       &                   &                    &                       &                     &                      &              \\
%                                    & Malicious &               &                       &                   &                    &                       &                     &                      &              \\
%                                    & Self      &               &                       &                   &                    &                       &                     &                      &              \\ \midrule
%\multirow{3}{*}{\textbf{Square2a}}  & Rational  &               &                       &                   &                    &                       &                     &                      &              \\
%                                    & Malicious &               &                       &                   &                    &                       &                     &                      &              \\
%                                    & Self      &               &                       &                   &                    &                       &                     &                      &              \\ \midrule
%\multirow{3}{*}{\textbf{Square4a}}  & Rational  &               &                       &                   &                    &                       &                     &                      &              \\
%                                    & Malicious &               &                       &                   &                    &                       &                     &                      &              \\
%                                    & Self      &               &                       &                   &                    &                       &                     &                      &              \\ \midrule
%\multirow{3}{*}{\textbf{Medium20a}} & Rational  &               &                       &                   &                    &                       &                     &                      &              \\
%                                    & Malicious &               &                       &                   &                    &                       &                     &                      &              \\
%                                    & Self      &               &                       &                   &                    &                       &                     &                      &              \\ \midrule
%\multirow{3}{*}{\textbf{Random50a}} & Rational  &               &                       &                   &                    &                       &                     &                      &              \\
%                                    & Malicious &               &                       &                   &                    &                       &                     &                      &              \\
%                                    & Self      &               &                       &                   &                    &                       &                     &                      &              \\ \bottomrule
%\end{tabular}
%\end{table*}







%\vspace{-1mm}
\section{Conclusion}
\label{sec:conclusion}

To conclude, we formalize the problem of controlling one single agent against multiple opponents that are priorly unknown. A spectrum of formulations are drawn, all of which are further unified under a tree search perspective. We underpin the investigation by offering a challenging benchmark, namely multi-agent route planning. We also show by this general framework how to customize domain-specific planners such as \textit{safe-agents}. To offer a practical handbook of proper selection of the induced planners, we have empirically tested each of them against three representative groups of opponents. One interesting observation from our experiments is that those conservative and myopic \textit{safe-agents} perform sufficiently well in most cases, especially when belief modelling does not match the underlying truth or deliberate replanning is computationally infeasible.
%As for \textbf{future work}, we point out a few valuable directions: 
%(1)~How to design a formal language to describe domain knowledge that can be embedded into those general planners \cite{gao2020embedding}?
%(2)~How to model dependencies among agents which may decompose the complexity of modelling joint-transitions and joint-utilities \cite{ma2024efficient}, and what if such dependencies are dynamic, i.e., agents may come and go?
%(3)~What if there are infinitely many opponents, is it justifiable to model the problem as a mean-field game \cite{lasry2007mean}?
%%	\item , by assuming all the others are simply homogeneous ones resembling the modelling agent herself?
%%Another possible application will be for negotiation \cite{kraus1997negotiation, beer1999negotiation, jennings2001automated}, where multiple stakeholders try to strategically persuade each other to agree on a common goal.
%(4)~Other applicable domains like mechanism design~\cite{tang2017reinforcement}, and even negotiation \cite{kraus1997negotiation, beer1999negotiation, jennings2001automated},
%%	and teaching \cite{camerer2001strategic, camerer2004behavioural}, 
%	 which turn out also involving strategic behaviors that explore and exploit opponents during repeated interaction. 
%%%%%%%%%%%%%%%%%%%%%%
As for \textbf{future work}, we point out a few valuable directions: 
\begin{enumerate}
	\item How to design a formal language to describe domain knowledge that can be embedded into those general planners \cite{gao2020embedding}?
	\item How to model dependencies among agents which may potentially decompose the complexity of modelling joint-transitions and joint-utilities \cite{ma2024efficient}, and what if such dependencies are dynamic, i.e., agents may come and go?
	\item What if there are infinitely many opponents, is it justifiable to model the problem as a mean-field game \cite{lasry2007mean}?
%	\item , by assuming all the others are simply homogeneous ones resembling the modelling agent herself?
	\item
%Another possible application will be for negotiation \cite{kraus1997negotiation, beer1999negotiation, jennings2001automated}, where multiple stakeholders try to strategically persuade each other to agree on a common goal.
	Other applicable domains like mechanism design~\cite{tang2017reinforcement} and even negotiation \cite{kraus1997negotiation, beer1999negotiation, jennings2001automated},
%	and teaching \cite{camerer2001strategic, camerer2004behavioural}, 
	 which also turn out to involve strategic behaviors that explore and exploit opponents during repeated interaction. 
\end{enumerate}
%%%%%%%%%%%%%%%%%%%%%%

%(1) How to design a formal language to describe domain knowledge that can be embedded into those general planners~\cite{gao2020embedding}?
%(2) How to model dependencies among agents which can potentially decompose the complexity of modelling joint-transitions and joint-utilities~\cite{ma2024efficient}?
%(3) When the number of agents is sufficiently large, is it justifiable to model the problem as a mean-field game~\cite{lasry2007mean}, by assuming all the others are simply homogeneous ones resembling the modelling agent herself?
%(4) Other applicable domains like learning and teaching~\cite{camerer2001strategic, camerer2004behavioural}, which turns out also involving strategic behaviors that probe and exploit opponents during repeated interaction. 

%%%%%%%%%%%%%%%%%%%%%%%%%%%%%%%%%%%%%%%%%%%%%%%%%%%%%%%%%%%%%%%%%%%%%%%%

%%% The acknowledgments section is defined using the "acks" environment
%%% (rather than an unnumbered section). The use of this environment 
%%% ensures the proper identification of the section in the article 
%%% metadata as well as the consistent spelling of the heading.

\begin{acks}
This work was supported in part by a generous research grant from Xiaoi Robot Technology Limited.
%We thank Yuxin Pan, Chenglin Wang, Haozhe Wang, Yangfan Wu, and the anonymous reviewers for their insightful comments.
We thank Yuxin Pan, Chenglin Wang, Haozhe Wang, and Yangfan Wu for their valuable feedback during the revision of this paper. We also appreciate the anonymous reviewers for their insightful comments.
\end{acks}

%%%%%%%%%%%%%%%%%%%%%%%%%%%%%%%%%%%%%%%%%%%%%%%%%%%%%%%%%%%%%%%%%%%%%%%%

%%% The next two lines define, first, the bibliography style to be 
%%% applied, and, second, the bibliography file to be used.

\bibliographystyle{ACM-Reference-Format} 
\bibliography{sample}


%\end{document}

\balance

\clearpage
\onecolumn
\appendix

%\section*{Appendix}

\section{Theoretic Analysis of The Unified Framework}

\label{apd:converge}

For the belief-fixed lookahead search presented in Section~\ref{sec:unification}.(\ref{eq:van_bellman}),
since the belief is fixed, we can make the notations simpler.
Let $v(S) \triangleq V_i(S,b)$, and let $\mathcal{V}$ be the space of all possible such value functions,
and $\Gamma: \mathcal{V} \mapsto \mathcal{V}$ be the operator that does the job of this backup equation.
\begin{theorem}
	The backup operator $\Gamma$ in Section~\ref{sec:unification}.(\ref{eq:van_bellman}) is a $\gamma$-contraction. Mathematically, for $u,v \in \mathcal{V}$, we have
	\[
	\|\Gamma(u) - \Gamma(v) \|_\infty \leq \gamma \|u - v\|_\infty
	\]
%	where $\|\cdot\|_\infty$ is the infinity norm.	
\label{thm:van_bellman_contraction}
\end{theorem}

\begin{proof}
First, we fist convert it from the expectation notation back to the summation notation,
\[
\begin{split}
v(S)
& = \max_{a_i\in \mathcal{A}_i}\{\mathbb{E}_{\pi_{-i} \sim b, a_{-i}\in \mathcal{A}_{-i}}[R_i(s,a) +  \gamma \sum_{S'}T(S'|S,a) v(S')]\} \\
& = \max_{a_i\in \mathcal{A}_i}\{\sum_{\pi_{-i}\in b} b(\pi_{-i})\sum_{a_{-i} \in \mathcal{A}_{-i}}R_i(S,a)\pi_{-i}(a_{-i}|S) + \gamma \sum_{\pi_{-i}\in b} b(\pi_{-i})\sum_{a_{-i} \in \mathcal{A}_{-i}}T(S'|S,a)\pi_{-i}(a_{-i}|S) v(S')\} \\
%\item $T^{b}(S'| S, a_i) \triangleq \sum_{\pi_{-i}\in b} b(\pi_{-i})\sum_{a_{-i} \in \mathcal{A}_{-i}}T(S'|S,a)\pi_{-i}(a_{-i}|S)$,
%\item $R^{b}(S, a_i) \triangleq \sum_{\pi_{-i}\in b} b(\pi_{-i})\sum_{a_{-i} \in \mathcal{A}_{-i}}R_i(S,a)\pi_{-i}(a_{-i}|S)$.
& = \max_{a_i\in \mathcal{A}_i} \{R^{b}(S, a_i) + \gamma T^{b}(S'| S, a_i) v'(S) \} \\
%& = \max_{a_i\in A_i}
%\{\mathbb{E}_{\pi_{-i} \sim b}[R_i(s,a)]
%+ \gamma \sum_{S'} \sum_{\bm{a}_{-i} \in \mathcal{A}_{-i}} T^b(S'|S,a)
%\sum_{\pi_{-i} \in b}\pi_{-i}(\bm{a}_{-i}|s)b(\pi_{-i})v(S')\} \\
%& = \max_{a_i\in A_i}
%\{r(s,a)
%+ \gamma \sum_{S'} T^b(S'|S,a_i)v(S')\} \\
\end{split}
\]
The last line is to simplify the equation, by our belief-induced reward and transition functions defined in Section \ref{sec:belief_induced_mdp}. With slight abuses of notations, we write $\Gamma_v(S)$ for $\Gamma(v)(S)$. In the following, we prove our target in two sub-cases,
\begin{enumerate}
\item For $S \in \mathcal{S}$, such that $\Gamma_u(S) > \Gamma_v(S)$.
We choose $a_i^* \in \arg\max_{a_i\in\mathcal{A}_i}\{R^b(S,a_i) + \gamma\sum_{S'} T^b(S'|S,a_i)u(S')\}$, then
\[
\begin{split}
|\Gamma_u(S) - \Gamma_v(S)|
& = \Gamma_u(S) - \Gamma_v(S) \\
& = R^b(S,a_i^*) + \gamma\sum_{S'} T^b(S'|S,a_i^*)u(S')
- \max_{a_i\in A_i}\{R^b(S,a_i) + \gamma\sum_{S'} T^b(S'|S,a_i)v(S')\} \\
& \leq R^b(S,a_i^*) + \gamma\sum_{S'} T^b(S'|S,a_i^*)u(S')
- [R^b(S,a_i^*) + \gamma\sum_{S'} T^b(S'|S,a_i^*)v(S')] \\
& \leq \gamma\sum_{S'} T^b(S'|S,a_i^*) [u(S') - v(S')] \\
& \leq \gamma\sum_{S'} T^b(S'|S,a_i^*) |u(S') - v(S')| \\
& \leq \gamma\sum_{S'} T^b(S'|S,a_i^*) \|u - v\|_\infty
\end{split}
\]

We then show that $\sum_{S'} T^b(S'|S,a_i) = 1$ given any $a_i \in \mathcal{A}_i$,
i.e., $T^b$ is indeed a valid (stochastic) transition function,
\[
\begin{split}
\sum_{S'\in \mathcal{S}} T^b(S'|S,a_i^*)
& = \sum_{S'\in \mathcal{S}}\sum_{\pi_{-i}\in b} b(\pi_{-i})\sum_{a_{-i} \in \mathcal{A}_{-i}}T(S'|S,a)\pi_{-i}(a_{-i}|S) \\
& = \sum_{S'\in \mathcal{S}} \sum_{a_{-i} \in \mathcal{A}_{-i}} \sum_{\pi_{-i} \in b}
T(S'|S,a)\pi_{-i}(a_{-i}|s)b_{-i}(\pi_{-i}) \\
& = \sum_{\pi_{-i} \in b}  b(\pi_{-i})
\sum_{a_{-i} \in \mathcal{A}_{-i}} \pi_{-i}(a_{-i}|s)
\sum_{S'\in \mathcal{S}} T(S'|S,a) \\
& = \sum_{\pi_{-i} \in b}  b(\pi_{-i})
\sum_{a_{-i} \in \mathcal{A}_{-i}} \pi_{-i}(a_{-i}|s) \\
& = \sum_{\pi_{-i} \in b}  b(\pi_{-i}) \\
& = 1
\end{split}
\]

\item For $S \in \mathcal{S}$, such that $\Gamma_u(S) < \Gamma_v(S)$. Similarly, we can also have
\[
|\Gamma_u(S) - \Gamma_v(S)|
= \Gamma_v(S) - \Gamma_u(S)
\leq \gamma |v(S') - u(S')|
\leq \gamma \|u - v\|_\infty
\]
\end{enumerate}
As a result, for all $S\in \mathcal{S}$, we have $|\Gamma_u(S) - \Gamma_v(S)| \leq \gamma \|u - v\|_\infty$. Hence,
$
\|\Gamma(u) - \Gamma(v)\|_\infty \leq \gamma \|u - v\|_\infty
$.
\end{proof}

By the above theorem, we can conclude that by the second-level belief-fixed lookahead search $V_i(S^{n},b^{n})$ converges to the optimal value function $v^*(S^n)$ of the induced MDP $\mathcal{M}(b^n)$, as $m$ approaches infinity.
For the first-level belief-updated lookahead search presented in Section \ref{sec:unification}.(1), we can also have a similar property.
\begin{theorem}
The backup operator in Section~\ref{sec:unification}.(\ref{eq:belief_bellman}) is a $\gamma$-contraction.
\end{theorem}
	
\begin{proof}
\[
V_i(S,b)
= \max_{a_i \in \mathcal{A}_i}\bigg\{ \mathcal{R}\Big( (S, b), a_i \Big)
+ \gamma \sum_{S',b'} \mathcal{T} \Big( (S', b') \Big| (S, b), a_i  \Big)\cdot V_i(S',b') \bigg\}
\]
Now that $V_i: \mathcal{S} \times \mathcal{B} \mapsto \mathbb{R}$ is a function over continuous variables, then $\|V_i\|_\infty \triangleq  \sup_{s,b} |V_i(S,b)|$. 
In fact, $V_i$ is piece-wise linear and convex, as it reveals the value function of the underlying POMDP \cite{smallwood1973optimal}, then ``$\sup$'' simply becomes ``$\max$'' and the rest of the proof will naturally proceed as that for Theorem \ref{thm:van_bellman_contraction}, as showing $\sum_{S',b'} \mathcal{T} ( (S', b') | (S, b), a_i  ) = 1$ for any given $a_i \in \mathcal{A}_i$ is also straightforward,
\[
\begin{split}
\sum_{S',b'} \mathcal{T} \Big( (S', b') \Big| (S, b), a_i  \Big)
& = \sum_{b' \in \mathcal{B}}\sum_{S' \in \mathcal{S}}\sum_{\pi_{-i}\in b} b(\pi_{-i})\sum_{a_{-i} \in \mathcal{A}_{-i}} T(S'|S, a) \pi_{-i}(a_{-i}|S) \\
& = \sum_{S' \in \mathcal{S}}\sum_{\pi_{-i}\in b} b(\pi_{-i})\sum_{a_{-i} \in \mathcal{A}_{-i}} T(S'|S, a) \pi_{-i}(a_{-i}|S) \\
& = 1
\end{split}
\]
The second last equality holds because the belief update process is deterministic, hence the unique successor belief, while the last equality is proving the same target as what we did in the proof of Theorem \ref{thm:van_bellman_contraction}.

\end{proof}

Therefore, the above theorem shows $V_i(S^0,b^0)$ will converge to the solution of Equation~(\ref{eq:pomdp}), as $n$ approaches infinity, i.e. the optimal value function $V_i^*(S^0,b^0)$ of the underlying POMDP.
More illustratively, it can be informally shown by Figure \ref{fig:conv},
where $\|EB\| = \gamma^m \|CB\|$, $\|AG\| = \gamma^m \|AC\|$, and $\|AD\| =\gamma^n \|AB\|$, $\|AF\| =\gamma^n \|AE\|$, $\|AH\| =\gamma^n \|AG\|$, therefore, by simple geometry,
$FD \parallel EB$ and $HF \parallel GE \parallel AB$.
Projecting our planning procedures to the diagram,
we start from C, go to E, and end up with F.
For several other alternatives,
\begin{enumerate}
	\item $C\rightarrow B \rightarrow D$ means one optimally solves the belief induced MDP first and then backup the value $n$ levels with updated beliefs. 
	\item $C\rightarrow G \rightarrow H$ means one optimally solves the finite-$(n+m)$-horizon POMDP with terminating states evaluated by $\textsc{Eval}_i$.
\end{enumerate}
Consequently, $\|FD\|$ and $\|HF\|$ are the respective distances of these two alternative solutions to that given by what we have proposed.

\begin{figure}[!ht]
	\centering
%	\includegraphics[height=60mm]{fig/convergence}
	\includegraphics[height=60mm]{fig/convergence_pdf}
	\caption{The convergence dynamics.}
	\label{fig:conv}
\end{figure}

%\begin{theorem}
%	The backup operator $\Gamma$ in (\ref{eq:van_bellman}) is a policy improvement operator.
%\end{theorem}
%
%\begin{proof}
%	TBC
%\end{proof}
%
%\begin{corollary}
%	The backup operator $\Gamma$ in (\ref{eq:belief_bellman}) is a policy improvement operator.
%\end{corollary}
%
%\begin{proof}
%	TBC
%\end{proof}




\section{Planners in Pseudocode}
\label{apd:planner_detail}

By convention, we use $SG$ to denote an instantiated multi-agent environment, $SG.reset()$ launches a new episode and returns the initial state, and $SG.step(a_i, a_{-i})$ proceeds the environment by the given joint actions and returns the successor state. An unbounded \textbf{while} loop is used to represent a running episode, and will terminate automatically if $SG$ reaches an end state.

\begin{algorithm}[!ht]
\caption{Planning via belief-fixed MDPs}
%\small
\begin{algorithmic}[1]
%%\Require $n \geq 0$
%%\Ensure $y = x^n$
%\Function{InduceMDP}{$b$}
%	\State $<\mathcal{S}^{mdp}, \mathcal{A}^{mdp}, \gamma^{mdp}>\ \gets \ <\mathcal{S}, \mathcal{A}_i, \gamma>$
%	\State Induce the transition function
%	\[
%	T^{mdp}(s'| s, a_i) \gets \sum_{\bm{a}_{-i} \in \mathcal{A}_{-i}} T^b(S'|S,a)
%	\sum_{\pi_{-i} \in b}\pi_{-i}(\bm{a}_{-i}|s)b(\pi_{-i})
%	\]
%	\State Induce the reward function
%	\[
%	R^{mdp}(s, a_i) \gets \sum_{\bm{a}_{-i} \in \mathcal{A}_{-i}} R_i(s, \bm{a}) 
%	\sum_{\pi_{-i} \in b}\pi_{-i}(\bm{a}_{-i}|s)\bm{b}_{-1}(\bm{\pi}_{-1})
%	\]
%
%	\State \Return $<\mathcal{S}^{mdp}, \mathcal{A}^{mdp}, T^{mdp}, R^{mdp}, \gamma^{mdp}>$
%\EndFunction
%
%\State
\Function{Plan-Belief-Fixed}{$b^0$}
	\State $M \gets \mathcal{M}(b^0)$
	\Comment Induce an MDP
	\State $\pi_i \gets \textsc{solve}(M)$
	\State $S \gets SG.reset()$
	\While{True} \Comment Game loop
		\State $a_i \gets \pi_i(\cdot|S)$
		\Comment No replanning
		\State $S\gets SG.step(a_i, a_{-i})$
	\EndWhile
\EndFunction
\end{algorithmic}
\label{alg:mdp_fixed}
\end{algorithm}



\begin{algorithm}[!ht]
\caption{Planning via belief-updated MDPs}
%\small
\begin{algorithmic}[1]
%\Require $n \geq 0$
%\Ensure $y = x^n$
\Function{Plan-Belief-updated}{$b^0$}
	\State $b \gets b^0$
	\State $M \gets \mathcal{M}(b)$
	\Comment Induce an MDP
	\State $\pi_i \gets \textsc{solve}(M)$
	\State $S \gets SG.reset()$
	\While{True} \Comment Game loop
		\State $a_i \gets \pi_i(\cdot|S)$
		\State $S \gets SG.step(a_i, a_{-i})$
		\State $b \gets \xi(b, S, a)$
		\Comment $a$ is a shorthand for $(a_i, a_{-i})$
		\State $M \gets \mathcal{M}(b)$
		\Comment Replanning on the revised MDP
		\State $\pi_i \gets \textsc{solve}(M)$
	\EndWhile
\EndFunction
\end{algorithmic}
\label{alg:mdp_updated}
\end{algorithm}




\begin{algorithm}[!ht]
\caption{Planning via QMDPs}
%\small
\label{alg:qmdp}
\begin{algorithmic}[1]
%\Require $n \geq 0$
%\Ensure $y = x^n$
\Function{Plan-QMDP}{$b^0$}
	\For{each $\pi_{-i} \in b^0$}
		\State $Q_{\pi_{-i}} \gets \textsc{Solve}(\mathcal{M}(\pi_{-i}))$
		\Comment Get the Q values instead of the policy
	\EndFor
	\State $b\gets b^0$
	\State $S \gets SG.reset()$
	\While{True} \Comment Game loop
		\State $a_i \in \arg\max_{a\in \mathcal{A}_i} \sum_{\pi_{-i} \in b}Q_{\pi_{-i}}(S, a) \cdot b(\pi_{-i})$
		\Comment No need to replan
		\State $S\gets SG.step(a_i, a_{-i})$
		\State $b \gets \xi(b, S, a)$
	\EndWhile
\EndFunction
\end{algorithmic}
\end{algorithm}




\begin{algorithm}[!ht]
\caption{Planning via ContextualRL}
%\small
\label{alg:crl}
\begin{algorithmic}[1]
%\Require $n \geq 0$
%\Ensure $y = x^n$

\Function{Plan-ContextualRL}{$b0$}
	\State $SG' \gets \textsc{WrapAsSamplingEnv}(SG)$
	\State $\pi^* \gets \textsc{AnyLearner}(SG')$
	\State $b \gets b^0$
	\State $S \gets SG.reset()$
	\While{True} \Comment Game loop
		\State $a_i \sim \pi^*(\cdot|S,b)$
		\Comment $\pi$ might be a stochastic policy
		\State $S\gets SG.step(a_i, a_{-i})$
		\State $b \gets \xi(b, S, a)$
	\EndWhile
\EndFunction
\end{algorithmic}
\end{algorithm}

We first present the pseudocode of the planners we mentioned in Section \ref{sec:solutions}, as Algorithm \ref{alg:mdp_fixed}, \ref{alg:mdp_updated}, \ref{alg:qmdp}, \ref{alg:crl}, respectively.
%In Section \ref{sec:unification}, we have elaborated how those algorithms can be instantiated by the unified framework, where we did not give the rigorous proof for the claim in \texttt{F4}.
%
%\begin{theorem}
%	If $n=0$, $m=1$, and $$\textsc{Eval}_i(S,b) = \sum_{\pi_{-i} \in \Pi_{-i}}
%	V_{\mathcal{M}(\pi_{-i})}(S) \cdot b(\pi_{-i}),$$ where $V_{\mathcal{M}(\pi_{-i})}(S)$ is the value function for the MDP $\mathcal{M}(\pi_{-i})$, then it is equivalent to the QMDP solution, characterized by equation (\ref{eq:qmdp}).
%\end{theorem}
%
%\begin{proof}
%By the QMPD formulation given by equation (\ref{eq:qmdp}),
%\begin{equation}
%\begin{split}
%a_t^* 
%& \in \arg\max_{a_i\in \mathcal{A}_i} \sum_{x\in \mathcal{S}\times \Pi_{-i}} Q(x, a_i) \mathbb{P}[x] \\
%& = \arg\max_{a_i\in \mathcal{A}_i} \sum_{\pi_{-i} \in \Pi_{-i}}
%	Q_{\mathcal{M}(\pi_{-i})}(S, a_i) \cdot b(\pi_{-i}) \\
%& = \arg\max_{a_i\in \mathcal{A}_i}
%\sum_{\pi_{-i} \in \Pi_{-i}} b(\pi_{-i})\cdot [ R^{\pi_{-i}}(S, a_i) + \gamma \sum_{S'} T^{\pi_{-i}}(S'|S, a_i)V_{\mathcal{M}(\pi_{-i})}(S')] \\
%\end{split}
%\label{eq:qmdp_expand}
%\end{equation}
%By the definitions of the belief induced transition/reward functions,
%\[
%\begin{split}
%\sum_{\pi_{-i} \in \Pi_{-i}} b(\pi_{-i}) R^{\pi_{-i}}(S, a_i)
%& = \sum_{\pi_{-i} \in \Pi_{-i}} b(\pi_{-i}) \sum_{a_{-i} \in \mathcal{A}_{-i}}R_i(S,a)\pi_{-i}(a_{-i}|S)
%= R^{b}(S, a_i) \\
%%\triangleq \sum_{\pi_{-i}\in b} b(\pi_{-i})\sum_{a_{-i} \in \mathcal{A}_{-i}}R_i(S,a)\pi_{-i}(a_{-i}|S).
%\sum_{\pi_{-i} \in \Pi_{-i}} b(\pi_{-i}) T^{\pi_{-i}}(S'| S, a_i)
%& = \sum_{\pi_{-i} \in \Pi_{-i}} b(\pi_{-i}) \sum_{a_{-i} \in \mathcal{A}_{-i}}T(S'|S,a)\pi_{-i}(a_{-i}|S)
%= T^{b}(S'| S, a_i)
%%\triangleq \sum_{\pi_{-i}\in b} b(\pi_{-i})\sum_{a_{-i} \in \mathcal{A}_{-i}}T(S'|S,a)\pi_{-i}(a_{-i}|S),
%\end{split}
%\]
%Then equation (\ref{eq:qmdp_expand}) can be re-written as
%\[
%a_t^* \in 
%\arg\max_{a_i\in \mathcal{A}_i}
%\big\{ R^{b}(S, a_i) + \gamma \sum_{S'} T^{b}(S'| S, a_i)V_{\mathcal{M}(\pi_{-i})}(S') \big\}
%\]
%Which is exactly one-depth lookahead search with leaf node evaluated by  
%\end{proof}
%
We did not attach the pseudocode for the POMDP planner, as it simply operates this way: one compiles the problem into a POMDP instance, gives it to a POMDP solver, and enquires the returned policy at each step without any replanning.

As illustrated in Figure \ref{fig:ts}, the general framework is implemented as an \textsc{ExpectiMax} tree.
The red triangle nodes are called ``MAX'' nodes, representing the states of the modelling agent $i$, while the orange diamond nodes are called ``EXP'' nodes, representing the hypothetical states\footnote{Also usually known as after-states.} that follow from the states given agent $i$'s committed action. The green diamond nodes are not explicitly implemented as they are just conceptual ones to show that each $a_{-i}$ is drawn probabilistically from the potential policies (or types) based on the belief. Algorithm \ref{alg:ts} shows the skeleton of how to use tree search as an online (re-)planner, and Algorithm \ref{alg:uts} shows the detailed procedure of how this \textsc{ExpectiMax} backup works.

\begin{figure}[!ht]
	\centering
%	\includegraphics[height=45mm]{fig/approx_plan}
%	\includegraphics[height=45mm]{fig/tree_search.pdf}
%	\caption{The exact optimal plan (left), and a possible approximated online plan (middle). A closer look at the tree diagram for one depth of lookahead search (right).}
	\includegraphics[height=55mm]{fig/approx_tree_pdf}
	\caption{The exact optimal plan~(a), a potential approximated online plan with repeated replanning by layered tree search~(b), and a closer look at the tree diagram for one depth of the lookahead search~(c).}
	\label{fig:ts}
\end{figure}

\begin{algorithm}[!ht]
\caption{Planning via look-ahead tree search}
\label{alg:ts}
\begin{algorithmic}[1]
%\Require $n \geq 0$
%\Ensure $y = x^n$

\Function{Plan-TS}{$b^0, compute\_budget$}
	\State $b \gets b^0$
	\State $S \gets SG.reset()$
	\While{True} \Comment Game loop
		\State $a_i \gets \textsc{TreeSearch.BestResponse}(S, b, compute\_budget)$
		\State $S \gets SG.step(a_i, a_{-i})$
		\State $b \gets \xi(b, S, a)$
	\EndWhile
\EndFunction
\end{algorithmic}
\label{alg:ts}
\end{algorithm}


The ultimate version of MCTS-like planner is described in Algorithm \ref{alg:mcts}.
Note that, in line 19, it has to utilize a exploration-exploitation-balanced choice function for node selection.
Given a tree node $t$, by $t.v$ we denote the accumulated return from this node state onwards, and $t.N$ the number of visits to this node.
The most widely used formula is the UCT formula \cite{kocsis2006bandit},
\begin{equation}
	t_c \in \arg\max_{t_c \in t.children} \frac{t_c.v}{t_c.N} + c\cdot \sqrt{\frac{\ln(t.N)}{t_c.N}}
\end{equation}
where $c$ is a constant controlling the weight of exploration and exploitation, with its best empirical value $\sqrt{2}$.
It is also proposed to use certain prior policies to guide the choices, resulting the pUCT formula \cite{rosin2011multi, silver2017mastering, schrittwieser2020mastering},
\begin{equation}
	t_c \in \arg\max_{t_c \in t.children} \frac{t_c.v}{t_c.N} + t_c.policy\_prior \cdot \sqrt{\frac{\ln(t.N)}{t_c.N}}\Bigl(c_1 + \ln \bigl(\frac{t.N + c_2}{c_2}  \bigr)  \Bigr)
\end{equation}
where $c_1$ and $c_2$ are two constants controlling the influence of the prior policy, with their commonly adopted empirical values $c_1=1.25$ and $c_2=19625$.
In principle, the prior policies are harder to acquire than the value estimations of the leaf nodes.
However, we here mention two ways to obtain both in practice,
\begin{enumerate}
	\item Via the approximate NE strategies computed by \textit{constraint satisfaction} solvers. One can sample the opponent types from the belief distribution for multiple runs, compute an approximate (ex-post) NE w.r.t. the sampled types at each run, and eventually obtain an average policy. The value estimate can be computed by the average utility of those approximate NE strategies. We will elaborate, in our MARP domain, how to convert CBS plans to these two components in Appendix \ref{apd:exp_setting}.
	\item Via policy predictions by the actor in any actor-critic RL algorithm.
%	Any actor-critic RL algorithm like PPO \cite{schulman2017proximal} has an actor network to output logits that are further used to parameterize an action sampling distribution. Such distributions can also serve as prior policies for pUCT. Additionally, one can also utilize the critic network to output a value estimate for the given state, which can serve as the evaluation heuristic for leaf nodes during tree search.
	Any actor-critic RL algorithm like PPO \cite{schulman2017proximal} has an actor network to output certain logits, and a value network to predict a rough value of a given state.
	The action sampling distribution parameterized by the logits can serve as the policy prior, while the value prediction can directly be the desired value estimate.
%	Such distributions can also serve as prior policies for pUCT. Additionally, one can also utilize the critic network to output a value estimate for the given state, which can serve as the evaluation heuristic for leaf nodes during tree search.
\end{enumerate}






\begin{algorithm}[htpb]
\caption{Opponent-Modelling Uniform Tree Search (Exact Backup)}
\label{alg:uts}
\begin{algorithmic}[1]
%\Require $n \geq 0$
%\Ensure $y = x^n$
\Function{BestResponse}{$S, b, total\_depth$}
	\State Initialize the root node as 
	\[
	t_{root} \gets
	(type=\texttt{`max'}, state=S, height=0, belief=b, a_{prev}=null, children=[], reward = 0, v=null)
	\]
	\For{$a_i \in \mathcal{A}_i$}
		\State $t_c \gets \textsc{NewChildNode}(\texttt{`exp'}, t_{root}, a_i)$
		\State $t_{root}.children.append(t_c)$
	\EndFor
	\State \textsc{MaxVal}$(t_{root}, 0, total\_depth)$
	\State $best\_child \gets \arg\max_{t_c \in t_{root}.chilren} t_c.v$
	\State \Return $best\_child.a_{prev}$
	
\EndFunction

%\State
%\Function{ExpandTree}{$t, height, total\_depth$}
%	\If{$height \leq total\_depth$}
%		\For{$a_i \in \mathcal{A}_{t.i}$}
%			\State $t_c \gets \textsc{NewChildNode}(t, a_i)$
%			\State $t.children.append(t_c)$
%			\State $\textsc{ExpandTree}(t_c, height+1, total\_depth)$
%		\EndFor
%	\EndIf
%
%\EndFunction

\State
\Function{NewChildNode}{$type, parent, a$}
	\If{$type == \texttt{`exp'}$}
		\State $S' \gets parent.state$
		\State $h' \gets parent.height$
		\State $b' \gets parent.belief$
		\State
		$t_{new} \gets (type=\texttt{`exp'},
			   state=S', height=h', belief=b', a_{prev}=a,
			   children=[], reward=0, v=null)$
		
	\ElsIf{$type == \texttt{`max'}$}
		\State $a' \gets concatenate(parent.a_{prev}, a)$
		\Comment To compose a joint action
		\State $S', r' \gets transit\_and\_reward(parent.state, a')$
		\State $h' \gets parent.height + 1$
		\State $b' \gets belief\_update(parent.belief, parent.state, a')$
		\State
		$t_{new} \gets (type=\texttt{`max'},
			   state=S', height=h', belief=b', a_{prev}=a',
			   children=[], reward=r', v=null)$
	
	\EndIf
	\State $parent.children.append(t_{new})$
	\State \Return $t_{new}$
\EndFunction

\State
\Function{ExpVal}{$exp\_node, height, total\_depth$}
  	\For{$a_{-i} \in \mathcal{A}_{-i}$}
  		\State $child_{-i} \gets \textsc{NewChildNode}(\texttt{`max'}, exp\_node, a_{-i})$
  		\State $exp\_node.children.append(child_{-i})$
  	\EndFor
  	\State $exp\_node.v \gets$
%  	\Statex
%  	\qquad 
  	$\sum_{a_{-i} \in \mathcal{A}_{-i}}
  			\sum_{\pi_{-i} \in b}b(\pi_{-i}) \pi_{-i}(a_{-i}|exp\_node.state)\cdot
  			[child_{-i}.reward + \gamma \cdot \textsc{MaxVal}(child_{-i}, height + 1, total\_depth)]$
  	\State \Return $t.v$
\EndFunction

\State
\Function{MaxVal}{$max\_node, height, total\_depth$}
  	\If{$height == total\_depth$}
  		\State $max\_node.v \gets \textsc{Eval}_i(max\_node.state, max\_node.belief)$
  	\Else
  		\For{$a_{i} \in \mathcal{A}_{i}$}
  			\State $child_{i} \gets \textsc{NewChildNode}(\texttt{`max'}, max\_node, a_i)$
  			\State $max\_node.children.append(child_{i})$
  		\EndFor
  		\State $max\_node.v \gets \max_{t_c\in t.children} \textsc{ExpVal}(max\_node, height, total\_depth)$
  	\EndIf
  	\State \Return $max\_node.v$
\EndFunction
\end{algorithmic}
\label{alg:uts}
\end{algorithm}


\begin{algorithm}[htbp]
\caption{Opponent-Modelling Monte Carlo Tree Search }
\label{alg:mcts}
\begin{algorithmic}[1]
%\Require $n \geq 0$
%\Ensure $y = x^n$
\Function{BestResponse}{$S, b, time\_limit$}
	\State Initialize the root node as 
	\[
	t_{root} \gets
	(type=\texttt{`max'}, state=S, height=0, belief=b, a_{prev}=null, children=[], reward = 0, v=null, N=0)
	\]
	\While{not exceeding $time\_limit$}
		\State $t_{candidate} \gets \textsc{Select}(t_{root})$
		\State $t_{new} \gets \textsc{Expand}(t_{candidate})$
		\State $\textsc{Evaluate}(t_{new})$
		\Comment The node to evaluate must be a MAX node
		\State $\textsc{Backup}(t_{new})$
		\Comment Backup the value given by $\textsc{Eval}_i(t_{new}.state, t_{new}.belief)$
	\EndWhile
	\State $best\_child \gets \arg\max_{t_c \in t_{root}.children} t_c.N$
	\Comment select the action according to the Categorial distribution parameterized by $N$'s
	\State \Return $best\_child.a_{prev}$
	
\EndFunction

%\State
%\Function{ExpandTree}{$t, height, total\_depth$}
%	\If{$height \leq total\_depth$}
%		\For{$a_i \in \mathcal{A}_{t.i}$}
%			\State $t_c \gets \textsc{NewChildNode}(t, a_i)$
%			\State $t.children.append(t_c)$
%			\State $\textsc{ExpandTree}(t_c, height+1, total\_depth)$
%		\EndFor
%	\EndIf
%
%\EndFunction

%\State
%\Function{NewChildNode}{$type, parent, a$}
%	\If{$type == \texttt{`exp'}$}
%		\State $S' \gets parent.state$
%		\State $h' \gets parent.height$
%		\State $b' \gets parent.belief$
%		\State
%		$t_{new} \gets (type=\texttt{`exp'},
%			   state=S', height=h', belief=b', a_{prev}=a,
%			   children=[], reward=0, v=0)$
%		
%	\ElsIf{$type == \texttt{`max'}$}
%		\State $a' \gets concatenate(parent.a_{prev}, a)$
%		\Comment To compose a joint action
%		\State $S', r' \gets transit\_and\_reward(parent.state, a_{new})$
%		\State $h' \gets parent.height + 1$
%		\State $b' \gets belief\_update(parent.belief, parent.state, a_{new})$
%		\State
%		$t_{new} \gets (type=\texttt{`max'},
%			   state=S', height=h', belief=b', a_{prev}=a',
%			   children=[], reward=r', v=0)$
%	
%	\EndIf
%	\State \Return $t_{new}$
%\EndFunction

\State
\Function{Select}{$node$}
	\While{ True}
		\If{$node.type == \texttt{`max'}$}
			\If{$node$ is not fully expanded}
				\State \Return $node$ 
			\Else
				\State $node \gets \textsc{EEBalancedChoice}(node)$
				\Comment Balance exploration and exploitation
			\EndIf
		\ElsIf{$node.type == \texttt{`exp'}$}
			\State Sample $\pi_{-i} \sim node.belief$
				   for enough times to obtain a mean policy $\tilde\pi_{-i}$
			\State Sample $a_{-i} \sim \tilde\pi_{-i}(\cdot|node.state)$
			\If{$a_{-i}$ is not tried yet}
			\Comment Encounter a new MAX node that is not evaluated
				\State $child \gets \textsc{NewChildNode}(\texttt{`max'}, node, a_{-i})$
				\Comment Additionally set $child.N \gets 0$
				\State \Return $child$
			\Else
				\State $node \gets node.children[-i]$
				\Comment ``[-i]'' means the index corresponding to $a_{-i}$
			\EndIf
		\EndIf
	\EndWhile
	\State \Return node
\EndFunction


\State
\Function{Expand}{$node$}
	\If{$node$ is not yet evaluated}
	\Comment This can be done by checking whether $node.v$ is still $null$
		\State \Return $node$ 
	\EndIf
	\State $a_i \gets$ an untried but available action from $\mathcal{A}_i(node.State)$
	\State $child \gets \textsc{NewChildNode}(\texttt{`exp'}, node, a_i)$
	\State Sample $\pi_{-i} \sim child.belief$
				   for enough times to obtain a mean policy $\tilde\pi_{-i}$
	\State Sample $a_{-i} \sim \tilde\pi_{-i}(\cdot|child.state)$
	\State \Return $\textsc{NewChildNode}(\texttt{`max'}, child, a_{-i})$ 
\EndFunction

%\State
%\Function{Evaluate}{$max\_node$}
%	\State $max\_node.v \gets \textsc{EvalForAgent}_i(max\_node.state, max\_node.belief)$
%\EndFunction

\State
\Function{Backup}{$node$}
	\State $G \gets node.v$
	\While{$node$ is not $null$}
		\If{$node.type == \texttt{`max'}$}
		\Comment Only compute discounted rewards at MAX nodes
			\State $G \gets node.reward + \gamma \cdot G$
		\EndIf
		\State $node \gets node.parent$
		\State $node.v \gets node.v + G$
		\State $node.N \gets node.N + 1$
		
	\EndWhile
\EndFunction
\end{algorithmic}
\label{alg:mcts}
\end{algorithm}


\clearpage

\section{More Experimental Details}
\label{apd:exp_setting}



\subsection{Convert CBS plans to NE Strategies}

Given a grid map and a set of initial positions and goals,
MAPF solvers, such as CBS~\cite{sharon2015conflict} and EECBS~\cite{li2021eecbs},
compute a set of collision-free paths. As we mentioned, if this set of collision-free paths is optimal up to certain metrics, e.g., minimizing the sum of lengths, it serves as an NE as no agent will deviate in the sense of finding a shorter path without colliding to any others. We here use EECBS as it supports users to specify an error bound $\epsilon$ and returns a bounded sub-optimal solution of total length no more than $(1+\epsilon)$ times the optimum. By merely sacrificing a bounded amount of solution quality, EECBS can speed up the solving process drastically, e.g., only needs an amount of time of the order of 10ms to solve instances of 32x32 maps with 50 agents. In our experiments, $\epsilon = 0.2$.

By such a solver as an oracle that can compute a sample NE very fast in real-time, we are then able to extract value estimates and policy priors for the usage in tree search algorithms.
Given a tree node $(S,b)$, where $S$ represent the current locations of all agents and $b$ is a distribution over all possible goals of the opponents that is inferred by the modelling agent, the modelling agent will go through the following procedure for multiple rounds,
\begin{enumerate}
	\item Samples a set of opponents' goals from $b$ and calls EECBS to compute a set of collision-free path from the current locations to the sampled goals.
	\item Extract her own path, which is a sequence of actions leading to her goal without colliding to others.
	\item Suppose the path is of length $l$ and the first action is $a^0$, the value  will be estimated as $\gamma^l \times R_i(goal)$ and $e_{a^0}$ will be the policy prior, where $e_i$ is the unit vector with the $i$-th element being 1.
\end{enumerate}
Finally, the value estimate and policy prior for this node $(S,b)$ will be computed as the mean of these values and policy priors collected above.



%Despite that we have argued NE as a solution concept offers little help,
%an agent can still compute one, if possible, and execute the NE strategy expecting that the others will do the same.
%Computing NE is in general computationally intractable.
%Nevertheless, there is a special case that greatly alleviates the heavy computation when
%\begin{enumerate}
%	\item the transition is deterministic, i.e. $T: \mathcal{S} \times \mathcal{A}_1 \times \cdots \mathcal{A}_n \mapsto \mathcal{S}$,
%	\item there is a particular goal state $G \in \mathcal{S}$ and set of conflicting states $\mathcal{S}_c \subset \mathcal{S}$, such that the utilities are shaped as $\forall i \in N$,
%	\[u_i(S, a) 
%	\begin{cases}
%		> 0, & \text{if } T(S, a) = G,\\
%		< - c_{p}, & \text{if } T(S, a) \in \mathcal{S}_c,\\	
%		\in [- c_0 , 0], & \text{otherwise},\\	
%	\end{cases}
%	\]
%\end{enumerate}
%where $c_p \gg c_0 \geq 0$.
%Intuitively, such special SGs are resulted from those goal-conditioned planning problems where all agents try to reach the goals while avoiding certain situations with severe conflicting interests, e.g., classic MAPF problems \cite{stern2019multi-overview, stern2019multi}.
%Every NE under this setting will drive all agents away from states in $\mathcal{S}_c$, if possible, and leads, to some extent, to shortest trajectories from every possible initial state $S_0$ to the goal state $G$. 
%
%Instead of numerical utilities, constraints are therefore probably the right language to describe the interaction among agents. More specifically, one can use a predicate $Nogood(S, a_1, \cdots, a_n)$ to rule out solutions that allow agents to simultaneously perform $(a_1, \cdots, a_n)$ in state $S$, if $T(S, a) \in \mathcal{S}_c$.
%Consequently, to compute NE, a highly applicable but usually neglected solution is to appeal to constraint satisfaction and optimization techniques, to make great use of constraint propagation.
%The resulted plan can be seen as an approximate ex-post NE (w.r.t initial states) for the original stochastic game, as those state-action pairs with large penalties have been converted to hard constraints.
%The solving process will be drastically sped up in the reason that constraints usually propagate efficiently in practice.
%Exemplary solvers can be found mostly in the MAPF literature, e.g.,
%optimal planners\cite{sharon2015conflict, zhang2022multi},
%bounded sub-optimal planners\cite{li2021eecbs},
%universal planners\cite{zhu2024computinguniversalplanspartially}, etc.
%We postpone the elaboration on how to cast conflict-free plans to state-action values to Appendix \ref{apd:exp_setting}. For now, one can assume that an oracle for computing an $\epsilon$-NE is given.
%
%An ex-post sample NE is easier than BNE 
%

\subsection{Contextual-RL}
As mentioned in Section~\ref{sec:model}, Equation~(\ref{eq:pomdp}) also leads to potential (contextual-)RL solutions~\cite{benjamins2021carl}.
To this end, one needs to cast the given multi-agent stochastic game as a single-agent learning environment from the perspective of the modelling agent, by (i) first initializing each episode by sampling opponent strategies according to the initial belief, and (ii) then proceeding the environment by the given action of the modelling agent and the sampled actions of the opponents, updating the belief, and returning it to the modelling agent.

Consequently, the most challenging part lies in how to efficiently train such a policy that converges to the desired optimum.
%As we will show in Figure~\ref{fig:crl}, there is usually a clear intermediate plateau phase before convergence, which we conjecture is caused due to the gap between feasible strategies and optimal strategies. 
We wrap our environment as a gym-like one, and then use PPO \cite{schulman2017proximal} implemented by \textit{stable-baseline3}\footnote{https://stable-baselines3.readthedocs.io/en/master/} \cite{stable-baselines3} to train our modelling agent. We also tried other alternatives like DQN and A2C, but they do not end up with acceptably good returns.
Figure \ref{fig:crl} shows sample experiments of the training phase, for the two configurations ``Small\_2a'' and ``Square\_2a'', respectively.
In both figures, there is clearly an intermediate plateau before convergence. It is usually the case that in a certain early phase the modelling agent does find a feasible plan to reach the goal without any collision, and in the later phase she eventually manages to find a much shorter plan, hence a much improved return.

\begin{figure}[!ht]
	\centering
	\includegraphics[height=80mm]{fig/exp/small_2a}
	\includegraphics[height=80mm]{fig/exp/square_2a}
	\caption{Statistics of the RL training samples.}
	\label{fig:crl}
\end{figure}

\subsection{Detailed Results Additional to Table~\ref{tab:performance}}

Table~\ref{tab:params} shows the detailed parameters for the planners tested in the experiments shown in Table~\ref{tab:performance}.
\begin{enumerate}
	\item ``\#(runs)'' means how many rounds we test each planner to calculate an average.
	\item ``$\epsilon$'' means the ration of randomness associated to the opponents that is assumed by the modelling agent.
	\item ``depth'' means the depth of lookahead search in the corresponding full-width tree search planner.
	\item ``eval\_samples'' means the number of calls to EECBS while evaluating a tree node.
	\item ``backup\_samples'' means the number of samples to perform sampling-based backup.
	\item ``max\_iter'' means the number of simulations performed by the corresponding MCTS planner.
	\item ``select\_samples'' means the number of samples performed at each ``EXP'' node.
\end{enumerate}


\begin{table}[!htpb]
\begin{tabular}{@{}l|lllllll@{}}
\toprule
maps      & \#(runs) & $\epsilon$ & depth & eval\_samples & backup\_samples & max\_iter & select\_samples \\ \midrule
Small2a     & 500      & 7E-04                   & 2     & 10           & exact              & 30      & 50             \\
Square2a  & 1000     & 2E-04                   & 2     & 10           & exact              & 50      & 50             \\
Square4a  & 1500     & 2E-04                   & 1     & 5            & 10             & 60      & 50             \\
Medium20a & 1000     & 8E-05                   & /     & 5            & /              & 80      & 80             \\
Large50a    & 500      & 2E-05                   & /     & 2            & /              & 100     & 125   \\ \bottomrule       
\end{tabular}
\caption{Detailed Parameters for Table \ref{tab:performance}.}
\label{tab:params}
\end{table}


In table~\ref{tab:performance}, we have shown the path lengths penalized by collisions.
Here by Figure~\ref{fig:detail_samll}~-~\ref{fig:detail_large}, we show the raw path lengths as well as collision ratios.
As one can see,
\begin{enumerate}
	\item \textit{Safe-agents} and their \textit{Enhanced} versions lead to possibly longer raw paths but lower chance of collisions, as they may easily get stuck but can avoid most of the collisions.
	\item Compared to each planner by a vanilla oracle, the improved version by tree search usually leads to slightly longer paths but significantly lower collision ratios.
	\item For \textbf{malicious} opponents, it is sometimes better if the modelling agent sticks to the initial uniform belief and does not update it, as in this case belief modelling is ``severely attacked'' by their chasing behavior. 
\end{enumerate}


%\clearpage
\begin{figure}[!ht]
	\flushleft
	\includegraphics[height=35mm]{fig/exp/small_rand_rational/map.png}
	\hspace{-3mm}
	\includegraphics[height=35mm]{fig/exp/small_rand_rational/mean_std_rand}
	\hspace{-8mm}
	\includegraphics[height=35mm]{fig/exp/small_rand_malicious/mean_std_rand}
	\hspace{-8mm}
	\includegraphics[height=35mm]{fig/exp/small_rand_self/mean_std_rand}
%	\vspace{-2mm}
	\caption{Detailed experiments for ``Small2a'' configurations.}
	\label{fig:detail_samll}
\end{figure}

\begin{figure}[!ht]
	\flushleft
	\includegraphics[height=35mm]{fig/exp/square_rand_rational/map.png}
	\hspace{-3mm}
	\includegraphics[height=35mm]{fig/exp/square_rand_rational/mean_std_rand}
	\hspace{-8mm}
	\includegraphics[height=35mm]{fig/exp/square_rand_malicious/mean_std_rand}
	\hspace{-8mm}
	\includegraphics[height=35mm]{fig/exp/square_rand_self/mean_std_rand}
%	\vspace{-2mm}
	\caption{Detailed experiments for ``Square2a'' configurations.}
\end{figure}

\begin{figure}[!ht]
%\vspace{-8mm}
	\flushleft
	\includegraphics[height=35mm]{fig/exp/square4a_rand_rational/map.png}
	\hspace{-3mm}
	\includegraphics[height=35mm]{fig/exp/square4a_rand_rational/mean_std_rand}
	\hspace{-8mm}
	\includegraphics[height=35mm]{fig/exp/square4a_rand_malicious/mean_std_rand}
	\hspace{-8mm}
	\includegraphics[height=35mm]{fig/exp/square4a_rand_self/mean_std_rand}
%	\vspace{-2mm}
	\caption{Detailed experiments for ``Sqaure4a'' configurations.}
\end{figure}

\begin{figure}[!ht]
%\vspace{-8mm}
	\flushleft
	\includegraphics[height=35mm]{fig/exp/medium20a_rand_rational/map.png}
	\hspace{-3mm}
	\includegraphics[height=35mm]{fig/exp/medium20a_rand_rational/mean_std_rand}
	\hspace{-8mm}
	\includegraphics[height=35mm]{fig/exp/medium20a_rand_malicious/mean_std_rand}
	\hspace{-8mm}
	\includegraphics[height=35mm]{fig/exp/medium20a_rand_self/mean_std_rand}
%	\vspace{-2mm}
	\caption{Detailed experiments for ``Medium20a'' configurations.}
\end{figure}

\begin{figure}[!ht]
%\vspace{-6mm}
	\flushleft
	\includegraphics[height=35mm]{fig/exp/random50a_rand_rational/map.png}
	\hspace{-3mm}
	\includegraphics[height=35mm]{fig/exp/random50a_rand_rational/mean_std_rand}
	\hspace{-8mm}
	\includegraphics[height=35mm]{fig/exp/random50a_rand_malicious/mean_std_rand}
	\hspace{-8mm}
	\includegraphics[height=35mm]{fig/exp/random50a_rand_self/mean_std_rand}
%	\vspace{-2mm}
	\caption{Detailed experiments for ``Large50a'' configurations.}
	\label{fig:detail_large}
\end{figure}


%\newpage
\section{Case Studies}
\label{apd:case_study}

We here present a few additional planners in Table~\ref{tab:additional_planner}, as they are computationally expensive and therefore not suitable for a multi-round testing.
Instead, in this section we show a few case studies to help one understand the capability of each planner.

% Please add the following required packages to your document preamble:
% \usepackage{booktabs}
\begin{table}[!htpb]
\begin{tabular}{@{}c|cccccccc@{}}
\toprule
\textbf{Planners}  & \textbf{n}   & \textbf{m} & $\mathbf\beta$ & \textbf{Collision penalty} & \textbf{\textsc{Eval}$_i$} & \textbf{Backup} & \textbf{Lookahead} & \textbf{Need replanning}  \\ \midrule
POMDP        & $\infty$     &            & 1              & $< \infty$                 &                                             &                 &                    &                           \\
QMDP         & 0            & 0          & 1              & $< \infty$                 & QMDP                                        &                 &                    &                           \\
UniformTS-MDP      & $(0,\infty)$ & $\infty$   & 1              & $< \infty$                 &                                             & exact           & full-width         & \checkmark \\
UniformTS-reactive & $(0,\infty)$ & 0   & 1              & $< \infty$                 & Euclidean distance                          & exact           & full-width         & \checkmark \\ \bottomrule
\end{tabular}
\caption{Additional planners.}
\label{tab:additional_planner}
\end{table}

\subsubsection*{Case 1}
It is predictable that by Bayesian-like belief update, the belief held by the modelling agent will converge to the underlying ground truth, if given many enough rounds.
The challenging case is when the agent is under a rather small grid world and may not have too many steps to go. In this case, the modelling agent has to plan over hypothetical beliefs upfront, instead of updating her belief per move.
As shown in Figure~\ref{fig:small_compare}, the POMDP agent only needs 5 steps, while the QMDP agent needs 6 steps and both MDP agents (with and without belief update) need the longest 7 steps.

\begin{figure}[!ht]
	\flushleft
%	\includegraphics[height=100mm]{fig/small_compare}
	\includegraphics[height=100mm]{fig/small_compare_pdf}
	\caption{Detailed paths planned in a tiny situation (3x3 map two agents): agent 1 is a shorest-path agent while agent 2 is the modelling agent. Agent 1 is using a different shortest path algorithm from the way how agent 2 does in belief modelling. Squares with the corresponding colors are the respective goals.}
	\label{fig:small_compare}
\end{figure}

\subsubsection*{Case 2}
Here we show the necessity of belief update in Figure~\ref{fig:noupdate_stuck}.
Even if the modelling agent is capably of planning over states for infinite horizons, she will still easily get stuck under such a crowded environment if she does not update her belief.
With belief update, as long as she has observed the opponent has stayed in a position for a sufficiently long time, she can therefore infer that the probability of getting hit becomes negligible.

%When belief update matters: no update $\rightarrow$ stuck


\begin{figure}[!ht]
	\flushleft
%	\includegraphics[height=55mm]{fig/noupdate_stuck}
	\includegraphics[height=55mm]{fig/noupdate_stuck_pdf}
	\caption{Different results by MDP agents with and without belief update. Agent 2 is the modelling agent using MDP planners, while agent 1 and 3 are two shortest-path agents using different algorithms. Squares with the corresponding colors are the respective goals.}
	\label{fig:noupdate_stuck}
\end{figure}

\subsubsection*{Case 3}
Figure~\ref{fig:probe} illustrate a special type of actions, what we term as \textit{probing actions}.
A probing action is such an action by which the modelling agent gains more information about the opponents without sacrificing her own future return.
In both cases, agent 2 is at C5 when agent 1 is at D1.
For agent 2, going down is the only action that leads to shortest paths, but at the same time of her taking this action, the action made by agent 1 will reveal more information about her underlying goal. More specifically, in the right figure, when agent 2 proceeds to D5 by this probing action, she realizes that agent 1's goal must be below row E and hence decides to go left for the next step.
In contrast, the planners used in the left figure are not able to utilize the probed information. 

\begin{figure}[!ht]
	\centering
%	\includegraphics[width=105mm]{fig/probe_action}
	\includegraphics[width=105mm]{fig/probe_action_pdf}
	\caption{Agent 2 is the modelling agent using any of the planners noted at the lower left corner of each sub-figure, while agent 1 is a shortest-path agent. Squares with the corresponding colors are the respective goals.}
	\label{fig:probe}
\end{figure}



\subsubsection*{Case 4} Figure~\ref{fig:probe} presents a situation where the modelling agent will sooner or later get stuck if she only does myopic planning like \textit{safe-agents}. 
The enhanced version can later resolve it but has to wait until the others stop moving.
The full-width tree search agent here does only one-depth lookahead search, and therefore, also takes a late turn,
while the MCTS agent gets around the opponents in a much earlier time primarily because she manages to lookahead for more steps with the help of heuristic node selection within the given budget.

\begin{figure}[!ht]
	\centering
%	\includegraphics[width=170mm]{fig/plan_ahead}
	\includegraphics[width=170mm]{fig/plan_ahead_pdf}
	\caption{Agent 1 is the modelling agent using the respective planners noted at the lower left corner of each sub-figure, while agent 2, 3, and 4 are shortest-path agents. Squares with the corresponding colors are the respective goals.}
	\label{fig:probe}
\end{figure}

\subsubsection*{Case 5}
For a large map with a dense population of agents, we attach two videos to show the capability of our planners, especially the potential of tree search.
In both videos, we control agent 1 as the modelling agent, while all the others are naive A$^*$ agents ignoring others, and therefore, may collide into each other. Initially agents are randomly spawn on the map. In \textbf{large50a\_cbs.mp4} agent 1 is born at the upper left corner, while in \textbf{large50a\_mctscbs.mp4} agent 1 is born at the middle right part. In the former one  agent 1 directly enquires CBS plans at each move, while in the latter one she instead uses CBS plans as node heuristics and applies MCTS with pUCT to improve them in real-time. As one can see the vanilla version encounters many intermediate collisions, while the MCTS improved one perfectly avoids all collisions and finds a fairly short path.


%\paragraph{Case 5} Lookahead deep or wide?
%
%TBC

\section{Computing Time}
Besides a comparison over the performances of our proposed planners, we here attach two additional tables for computing times, in order to help readers make more informed decisions on selecting suitable planners. 
Table~\ref{tab:comp_time_main} shows computing times for the planners in Table~\ref{tab:performance},
while Table~\ref{tab:comp_time_additional} are for the extra planners mentioned in Table~\ref{tab:additional_planner} in Appendix~\ref{apd:case_study}, with the following elaboration:
\begin{enumerate}
	\item Each cell records the average computing time in seconds, except for the POMDP planner where we explicitly write roughly 2 hours. We run 10 times, as the computing time is quite stable, for each one  to calculate the average (just two runs for the POMDP planner), with the standard deviation omitted to make the table more concise.
	\item The additional Tiny2a configuration means the one (3-by-3 maps with 2 agents) we present in Figure~\ref{fig:small_compare} in Appendix~\ref{apd:case_study}.
	\item Some planners need no replanning, therefore, we report the time they take for the initial plans.
	\item ``/'' means the planner is not feasible in that scenario.
\end{enumerate}

\begin{table}[!ht]
\small
\centering
\begin{tabular}{@{}lrrrrrrrrr@{}}
\toprule
                  & \textbf{Astar} & \textbf{Safe} & \textbf{MDP} & \textbf{RL} & \textbf{UniformTSRL} & \textbf{CBS} & \textbf{UniformTSCBS} & \textbf{MCTSCBSucb} & \textbf{MCTSCBSpuct} \\ \midrule
\textbf{Small2a}   & 6.40E-04                           & 3.96E-04                                    & 3.38E+00                                       & 1.67E-06                                        & 1.85E-01                                   & 1.24E-01                                       & 7.65E+01                                  & 3.87E+00                                & 3.93E+00                                 \\
\textbf{Square2a}  & 1.54E-03                           & 1.64E-03                                    & 1.11E+02                                       & 1.57E-06                                        & 2.57E-01                                   & 1.27E-01                                       & 7.80E+01                                  & 6.53E+00                                & 6.12E+00                                 \\
\textbf{Square4a}  & 1.55E-03                           & 1.62E-03                                    & /                                              & /                                               & /                                          & 6.47E-02                                       & 3.67E+01                                  & 4.12E+00                                & 4.34E+00                                 \\
\textbf{Medium20a} & 2.44E-03                           & 4.96E-03                                    & /                                              & /                                               & /                                          & 7.17E-02                                       & /                                         & 1.73E+01                                & 3.16E+01                                 \\
\textbf{Random}    & 1.29E-02                           & 1.52E-02                                    & /                                              & /                                               & /                                          & 6.04E-02                                       & /                                         & 1.81E+02                                & 2.19E+02                                 \\ \bottomrule
\end{tabular}
\caption{For the planners in Table 3}
\label{tab:comp_time_main}

\small
\begin{tabular}{@{}lrrrr@{}}
\toprule
                 & \textbf{POMDP*} & \textbf{QMDP*} & \textbf{UnifTSMDP} & \textbf{UnifTS-reactive} \\ \midrule
\textbf{Tiny2a}  & $\sim$2hr       & 4.16E-01       & 5.52E+01           & 4.48E-02                 \\
\textbf{Small2a} & /               & 5.10E+01       & 8.67E+01           & 7.70E-02                 \\ \bottomrule
\end{tabular}
\caption{For the planners in Table 5 in Appendix D}
\label{tab:comp_time_additional}
\end{table}






%%%%%%%%%%%%%%%%%%%%%%%%%%%%%%%%%%%%%%%%%%%%%%%%%%%%%%%%%%%%%%%%%%%%%%%%

\end{document}

%%%%%%%%%%%%%%%%%%%%%%%%%%%%%%%%%%%%%%%%%%%%%%%%%%%%%%%%%%%%%%%%%%%%%%%%

