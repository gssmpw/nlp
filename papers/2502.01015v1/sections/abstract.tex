Task vectors, which are derived from the difference between pre-trained and fine-tuned model weights, enable flexible task adaptation and model merging through arithmetic operations such as addition and negation. However, existing approaches often rely on heuristics with limited theoretical support, often leading to performance gaps comparing to direct task fine tuning. Meanwhile, although it is easy to manipulate saved task vectors with arithmetic for different purposes, such compositional flexibility demands high memory usage, especially when dealing with a huge number of tasks, limiting scalability. This work addresses these issues with a theoretically grounded framework that explains task vector arithmetic and introduces the task vector bases framework. Building upon existing task arithmetic literature, our method significantly reduces the memory cost for downstream arithmetic with little effort, while achieving competitive performance and maintaining compositional advantage, providing a practical solution for large-scale task arithmetic.