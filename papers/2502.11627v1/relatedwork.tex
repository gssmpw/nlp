\section{Related Work}
To clarify our research motivation and research gap, in this section, we reviewed relevant studies on social connectedness among older adults, VR to promote social connection, and background of Chinese chess and its benefits for seniors.

\subsection{Social Connectedness in Later Life}

Social connectedness, i.e. the experience of belonging and relatedness between people \cite{van2009social}, is an essential human need \cite{easton2013investigation}. As a key \rv{factor in} healthy aging, \rv{social connectedness} \rv{brings} numerous benefits to the health and well-being of older adults \cite{ashida2008differential}. Research has shown that maintaining social connections can improve life satisfaction of older adults \cite{o2017definition}, prolong their independence and integration into social life \cite{michael2001living}, and also contribute to active aging \cite{annele2019definitions}. Additionally, a strong social support network helps to strengthen the immune system and increases the chances of living a longer life \cite{holt2010social}.

Community, as an important social structure, plays a vital role in maintaining social connections and providing emotional support for older adults \cite{brossoie2003community,markle2018community,salman2021community}. Participation in community-based social programs such as craft groups, board games, square dance, and fitness classes, serves as an important way for many older people to sustain their social ties \cite{agedcareguide_social_support}. These programs offer older adults ample opportunities for social engagement, including building new friendships, obtaining social support, and fostering a sense of belonging through participation in group activities \cite{lindsay2018mixed,xing2023keeping}. By engaging in these activities, older adults can establish friendships and social support networks within their communities, which contributes to both their physical and mental well-being \cite{agedcareguide_social_support,lindsay2018mixed}.

However, not every older adult has the opportunity to participate in community-based social activities in person. Due to life course changes such as deteriorating health, retirement, and loss of intimate relationships \cite{coyle2012social,pugh2009social}, many older adults face a decline in social connectedness. This decline could further lead to increased social isolation and loneliness in later life, which are significant risk factors for health and well-being \cite{house1988social,nicholson2012review}. Studies have also shown that older adults who lack social connectedness are more likely to suffer from chronic diseases, depression, cognitive decline, and even risk premature death \cite{eng2002social,holt2021loneliness}. Therefore, there is an urgent need to investigate how to help older adults engage in these community-based activities and strengthen their social ties remotely through a digital connection. 

\subsection{VR to Support Social Connection}

Recently, the rapid development of VR has provided a new interactive platform for people to engage in meaningful social activities without geographical constraints. Social VR, as an emerging immersive remote communication tool \cite{li2021social}, allows multiple users to join a collaborative virtual environment and communicate with each other through avatars, thereby simulating face-to-face social experiences. As social VR platforms (e.g., VRChat, Rec Room, and AltspaceVR) become increasingly prevalent in the market, many people turn to these platforms to experience social connectedness. 
To date, researchers have studied various aspects of social VR experiences, including \sout{user preferences for avatar use \cite{gonzalez2018avatar},} how social VR supports mental health \cite{deighan2023social}, facilitates everyday collaborative activities \cite{freeman2022working}, understands and measures social behaviors and interactions \cite{mcveigh2018s,li2019measuring}, as well as communicating effectively in embodied environments and addressing platform governance challenges \cite{smith2018communication,blackwell2019harassment}. Researchers have also explored how social VR affects meaningful relationships \cite{zamanifard2019togetherness}, and that social VR users begin to value social VR communities and make friends easily \cite{piitulainen2022vibing}.

%It is also important to note that the goal of social VR is not to replicate reality completely, but to facilitate and extend existing communication channels in the physical world \cite{li2021social}.

\sout{VR has been widely demonstrated to have benefits in promoting older adults' cognition, health, and physical functioning \cite{appel2020older,mirelman2016addition,roberts2019older,eisapour2018game,dickens2011interventions,du2024lightsword}.}
\rv{VR offers numerous potential benefits for older adults, including promoting health and physical functions \cite{appel2020older, mirelman2016addition}, positively impacting cognitive abilities \cite{appel2020older, eisapour2018game,du2024lightsword}, and improving their social and emotional well-being \cite{dickens2011interventions,roberts2019older}. Previous studies have explored VR’s usability among older adults, revealing both challenges and opportunities \cite{brown2019exploration,baker2020evaluating,seifert2021use}. Key issues include usability concerns, particularly for those with functional limitations or dementia \cite{baker2020evaluating}. Despite these challenges, VR shows promise in engaging older adults, potentially improving quality of life and social interactions \cite{brown2019exploration,baker2020evaluating,seifert2021use}.}
Recently, researchers began to explore how VR can be used by older adults to sustain their social connections.
One of the advantages of VR interventions is that they allow older adults, who may have physical limitations or live far away, to safely and comfortably participate in meaningful social activities from the comfort of their own homes. Baker et al. reported a series of studies investigating how older adults communicate with each other in VR \cite{baker2019interrogating}, how avatars in social VR meet the communication needs of older adults \cite{baker2021avatar}, and how social VR can support group reminiscence among seniors \cite{baker2021school,baker2019exploring}. Their work shows that VR can not only provide immersive social experiences, but also help older adults build and maintain social connections, effectively reducing feelings of loneliness and social isolation. Similarly, research by Afifi et al. indicated that VR can serve as a bridge, bringing older adults closer to family members \cite{afifi2023using} and fostering inter-generational communication and emotional bonding \cite{wei2023bridging}. These studies highlight the importance of using VR to foster reciprocal and enjoyable social connections among older adults, suggesting that social VR has great potential for application in future aging society.

Although VR shows great potential in enhancing social connectedness, most existing studies focus on general user groups. There is still a lack of research on specifically designing social VR experiences to enhance social connections among older adults. Furthermore, the few existing social VR platforms designed for older adults tend to prioritize verbal communication, such as reminiscence conversations \cite{baker2021school} and casual chatting \cite{baker2019interrogating}. While verbal interaction plays a significant role in fostering emotional connection, social bonding extends beyond just conversation. Participation in shared activities is another key factor in strengthening relationships. Research indicates that participating in shared activities allows individuals to naturally share ideas and emotions while having fun together \cite{evjemo2004supporting,richter2018relations,viguer2010grandparent}. A study by Wei et al. has highlighted the need for social VR platforms to move beyond solely communication-based interactions and instead offer a variety of shared experiences to promote deeper social connections \cite{wei2023bridging}.
\sout{In this work} \rv{As VR becomes more accessible}, we sought to find a shared activity familiar to older adults and construct a \rv{VR social} community \sout{in VR}. By leveraging the social benefits of both VR and shared activities, we hope to provide a unique platform for older adults to maintain and strengthen their social ties, thereby contributing to their overall well-being.

\subsection{Chinese Chess \sout{(Chinese chess)} Among Older Adults}

Chinese chess is a traditional strategic board game that represents a battle between two players \cite{leventhal1978}. The game is played on a flat board with pieces divided into two sets, typically red and black, each set consisting of 7 different Chinese characters and 16 pieces in total \cite{lau2011chinese}. Participants need to capture the enemy's general (king). In China, Chinese chess is a highly popular form of social entertainment among older adults, as it offers a familiar and intellectually stimulating pastime. In order to win or play well in the game, players need to coordinate and work with various abilities such as attention, memory, logical thinking, and decision-making. Repeated use of these abilities during gameplay can help maintain and improve cognitive functioning in older adults \cite{hu2012effects,shi2023effect}. Nuria et al. \cite{cibeira2021effectiveness} suggest that there is a significant correlation between regular board gameplay and a better quality of older life. Moreover, the social aspect of the game is equally significant; Chinese chess facilitates social interaction, providing a platform for emotional support and community bonding among seniors \cite{lin2023social,lee2018effect}.

Despite the benefits, not every older person has the opportunity to participate in such activities in an outdoor environment due to poor health and disabilities, geographic distance, or limited income. This prevents them from maintaining social networks and developing cognitive exercises. Various digital platforms and technologies have been developed to enable remote chess gameplay in previous research and market, such as computer or mobile chess games \cite{qqchess_webpage,bontchev2008mobile}, chess robots \cite{larregay2018design,kolosowski2020collaborative}, augmented reality (AR) applications \cite{chen2008remote,cerron2023multiplayer,yusof2019collaborative}, and VR platforms \cite{very_real_chess_steam,chess_vr_multiverse_journey_steam,vrchat_chess_world}. However, these existing systems tend to offer generalized social experiences that may not resonate with older adults. They often fail to replicate the rich social dynamics of face-to-face Chinese chess gatherings, which include not only the game itself but also social chit-chat, story sharing, and cultural exchange.