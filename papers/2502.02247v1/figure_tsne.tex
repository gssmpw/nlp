\begin{figure*}[t]
    % \flushleft
    \centering
    \subfloat[Baseline]{%[b]{0.45\textwidth}
        \label{fig:tsne_randrot}
        % \centering
        \includegraphics[width=0.24\textwidth]{./resources/visualization/randrot.pdf}
        % \vspace{10mm}
    }
    \subfloat[Ours]{%[b]{0.45\textwidth}
        \label{fig:tsne_ours}
        % \centering
        \includegraphics[width=0.24\textwidth]{./resources/visualization/ours.pdf}
        % \vspace{10mm}
    }
    \subfloat[Ablation of $\lambda_{oc}$]{%[b]{0.45\textwidth}
        \label{fig:plot_lambda_oc}
        % \centering
        \includegraphics[width=0.24\textwidth]{./resources/visualization/ablation_cons_weights.pdf}
        % \vspace{10mm}
    }
    \subfloat[Ablation of $\lambda_{ms}$]{%[b]{0.45\textwidth}
        \label{fig:plot_lambda_ms}
        % \centering
        \includegraphics[width=0.24\textwidth]{./resources/visualization/ablation_reg_weights.pdf}
        % \vspace{10mm}
    }
    \vspace{-1mm}
    \caption{T-SNE visualizations of the features for class \textcolor[RGB]{0,47,167}{"plant"}, \textcolor[RGB]{153,102,204}{"sofa"} and \textcolor[RGB]{240,0,86}{"chair"} on $M\to S$ are presented in (a) and (b), where the $\cdot$ and $\times$ denote source domain and target domain, respectively. The curves of performance under varying $\lambda_{oc}$ and $\lambda_{ms}$ are post in (c) and (d).}
    \label{fig:visualization}
    \vspace{-6mm}
\end{figure*} 