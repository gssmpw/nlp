\documentclass[journal,compsoc]{IEEEtran}
% \documentclass[lettersize,journal]{IEEEtran}
\usepackage{epsfig}
\usepackage{graphicx}
\usepackage{amsmath}
\usepackage{amssymb}
\usepackage{algorithm, algorithmic}

\usepackage{diagbox}
\usepackage{float}
\usepackage{afterpage}
\usepackage{bm}
\usepackage{subfig}

\usepackage{multirow}
\usepackage{color}
\usepackage{tablefootnote}
\usepackage{adjustbox}
\usepackage{wrapfig}

\usepackage{hyperref}       % hyperlinks
\usepackage{url}            % simple URL typesetting
\usepackage{booktabs}       % professional-quality tables
\usepackage{amsfonts}       % blackboard math symbols
\usepackage{nicefrac}       % compact symbols for 1/2, etc.
\usepackage{microtype}      % microtypography
\usepackage{times}
\usepackage{epsfig}

\usepackage{bbding}
\usepackage{etoolbox}
\usepackage{paralist}
\usepackage{ulem}
\usepackage{tikz}
\usepackage{color}

\usepackage{makecell}

\usepackage{xcolor,colortbl}



\newcolumntype{Y}{p{0.5cm}<{\centering}}
\newcommand{\mc}[2]{\multicolumn{#1}{c}{#2}}
\definecolor{Gray}{gray}{0.5}
\definecolor{LightCyan}{rgb}{0.88,1,1}

\newcolumntype{a}{>{\columncolor{Gray}}c}
\newcolumntype{b}{>{\columncolor{white}}c}



\DeclareMathOperator*{\cat}{Cat}


\def\etal{\textit{et al}.}
\def\ie{\textit{i.e.}}
\def\eg{\textit{e.g.}}
\def\etc{\textit{etc}}
\def\wrt{\textit{w.r.t. }}

\def\bz{\textcolor{red}}
\def\xc{\textcolor{blue}}

\newcommand{\tb}[1]{\textbf{#1}}
\newcommand{\bc}[1]{\textcolor[RGB]{192,0,0}{\text{#1}}}
\newcommand{\rc}[1]{\textcolor{blue}{\text{#1}}}
\newcommand{\bb}[1]{\textcolor[RGB]{192,0,0}{\textbf{#1}}}
\newcommand{\rb}[1]{\textcolor{blue}{\textbf{#1}}}
\newcommand{\todo}[1]{{\color{blue}{[TODO: #1]}}}

\renewcommand{\thefootnote}{\fnsymbol{footnote}}

%For paragraph heading that is NOT immediately after a (sub)section heading
% \newcommand{\midparaheading}[1]{\vspace*{-0.8em}\paragraph{#1}}

\normalem

\begin{document}

\title{Rotation-Adaptive Point Cloud Domain Generalization via Intricate Orientation Learning}

\author{{Bangzhen~Liu,~Chenxi~Zheng,~Xuemiao~Xu,~Cheng Xu,~Huaidong~Zhang, \\ and~Shengfeng~He,~\IEEEmembership{Senior Member,~IEEE}}

\thanks{This work is supported by the China National Key R\&D Program (No. 2023YFE0202700, 2024YFB4709200), the Key-Area Research and Development Program of Guangzhou City (No. 2023B01J0022), the Guangdong Provincial Natural Science Foundation for Outstanding Youth Team Project (No. 2024B1515040010), the NSFC Key Project (No. U23A20391), the Guangdong Natural Science Funds for Distinguished Young Scholars (No. 2023B1515020097), the AI Singapore Programme under the National Research Foundation Singapore (No. AISG3-GV-2023-011), and the Lee Kong Chian Fellowships. (Bangzhen~Liu and Chenxi~Zheng contributed equally to this work.) (Corresponding authors: Xuemiao~Xu; Cheng~Xu.)
}
\thanks{Bangzhen Liu,~Chenxi~Zheng, Xuemiao Xu, Cheng Xu, and Huaidong Zhang are with the South China University of Technology, Guangzhou, China.~E-mail: liubz.scut@gmail.com,~cszcx@mail.scut.edu.cn, xuemx@scut.edu.cn, cschengxu@gmail.com, and huaidongz@scut.edu.cn. Xuemiao Xu is also with the Guangdong Engineering Center for Large Model and GenAI Technology, the State Key Laboratory of Subtropical Building and Urban Science, the Ministry of Education Key Laboratory of Big Data and Intelligent Robot and the Guangdong Provincial Key Lab of Computational Intelligence and Cyberspace Information.}
\thanks{ Shengfeng He is with the Singapore Management University, Singapore. E-mail: shengfenghe@smu.edu.sg.} 
% the School of Computing and Information Systems, 
}

\markboth{IEEE Transactions on Pattern Analysis and Machine Intelligence}%
{Shell \MakeLowercase{\textit{Liu et al.}}: Rotation-Adaptive Point Cloud Domain Generalization via Intricate Orientation Learning}


\IEEEtitleabstractindextext{
\begin{abstract}
  The vulnerability of 3D point cloud analysis to unpredictable rotations poses an open yet challenging problem: \emph{orientation-aware 3D domain generalization}. Cross-domain robustness and adaptability of 3D representations are crucial but not easily achieved through rotation augmentation. Motivated by the inherent advantages of intricate orientations in enhancing generalizability, we propose an innovative rotation-adaptive domain generalization framework for 3D point cloud analysis. Our approach aims to alleviate orientational shifts by leveraging intricate samples in an iterative learning process. Specifically, we identify the most challenging rotation for each point cloud and construct an intricate orientation set by optimizing intricate orientations. Subsequently, we employ an orientation-aware contrastive learning framework that incorporates an orientation consistency loss and a margin separation loss, enabling effective learning of categorically discriminative and generalizable features with rotation consistency. Extensive experiments and ablations conducted on 3D cross-domain benchmarks firmly establish the state-of-the-art performance of our proposed approach in the context of orientation-aware 3D domain generalization. 
\end{abstract}

\begin{IEEEkeywords}
  Point cloud domain generalization, contrastive learning, rotation robustness, intricate orientation mining
\end{IEEEkeywords}
}

\maketitle

\IEEEdisplaynontitleabstractindextext

\IEEEpeerreviewmaketitle

\section{Introduction}
\label{sec::intro}

Embodied Question Answering (EQA) \cite{das2018embodied} represents a challenging task at the intersection of natural language processing, computer vision, and robotics, where an embodied agent (e.g., a UAV) must actively explore its environment to answer questions posed in natural language. While most existing research has concentrated on indoor EQA tasks \cite{gao2023room, pena2023visual}, such as exploring and answering questions within confined spaces like homes or offices \cite{liu2024aligning}, relatively little attention has been dedicated to EQA tasks in  open-ended city space. Nevertheless, extending EQA to city space is crucial for numerous real-world applications, including autonomous systems \cite{kalinowska2023embodied}, urban region profiling \cite{yan2024urbanclip}, and city planning \cite{gao2024embodiedcity}. 
% 1. 环境复杂性   
%    - 地标重复性问题(如区分相似建筑)  
%    - 动态干扰因素(交通流、行人)  
% 2. 行动复杂性  
%    - 长程导航路径规划  
%    - 移动控制、角度等  
% 3. 感知复杂性  
%    - 复合空间关系推理("A楼东侧商铺西边的车辆")  
%    - 时序依赖的观察结果整合

EQA tasks in city space (referred to as CityEQA) introduce a unique set of challenges that fundamentally differ from those encountered in indoor environments. Compared to indoor EQA, CityEQA faces three main challenges: 

1) \textbf{Environmental complexity with ambiguous objects}: 
Urban environments are inherently more complex,  featuring a diverse range of objects and structures, many of which are visually similar and difficult to distinguish without detailed semantic information (e.g., buildings, roads, and vehicles). This complexity makes it challenging to construct task instructions and specify the desired information accurately, as shown in Figure \ref{fig:example}. 

2) \textbf{Action complexity in cross-scale space}: 
The vast geographical scale of city space compels agents to adopt larger movement amplitudes to enhance exploration efficiency. However, it might risk overlooking detailed information within the scene. Therefore, agents require cross-scale action adjustment capabilities to effectively balance long-distance path planning with fine-grained movement and angular control.

3) \textbf{Perception complexity with observation dynamics}: 
% Rich semantic information in urban settings leads to varying observations depending on distance and orientation, which can impact the accuracy of answer generation. 
Observations can vary greatly depending on distance, orientation, and perspective. For example, an object may look completely different up close than it does from afar or from different angles. These differences pose challenges for consistency and can affect the accuracy of answer generation, as embodied agents must adapt to the dynamic and complex nature of urban environments.


\begin{table}
\centering
\caption{CityEQA-EC vs existing benchmarks.}
\label{table:dataset}
\renewcommand\arraystretch{1.2}
\resizebox{\linewidth}{!}{
\begin{tabular}{cccccc}
             & Place  & Open Vocab & Active & Platform  & Reference \\ \hline
EQA-v1      & Indoor & \textcolor{red}{\ding{55}}          & \textcolor{green}{\ding{51}}      & House3D      & \cite{das2018embodied}  \\
IQUAD        & Indoor & \textcolor{red}{\ding{55}}          & \textcolor{green}{\ding{51}}      & AI2-THOR     & \cite{gordon2018iqa} \\
MP3D-EQA     & Indoor & \textcolor{red}{\ding{55}}          & \textcolor{green}{\ding{51}}      & Matterport3D & \cite{wijmans2019embodied} \\
MT-EQA       & Indoor & \textcolor{red}{\ding{55}}          & \textcolor{green}{\ding{51}}      & House3D      & \cite{yu2019multi} \\
ScanQA       & Indoor & \textcolor{red}{\ding{55}}          & \textcolor{red}{\ding{55}}      & -            & \cite{azuma2022scanqa} \\
SQA3D        & Indoor & \textcolor{red}{\ding{55}}          & \textcolor{red}{\ding{55}}      & -            & \cite{masqa3d} \\
K-EQA        & Indoor & \textcolor{green}{\ding{51}}          & \textcolor{green}{\ding{51}}      & AI2-THOR     & \cite{tan2023knowledge} \\
OpenEQA      & Indoor & \textcolor{green}{\ding{51}}          & \textcolor{green}{\ding{51}}      & ScanNet/HM3D & \cite{majumdar2024openeqa} \\
 \hline
CityEQA-EC   & City (Outdoor)  & \textcolor{green}{\ding{51}}          & \textcolor{green}{\ding{51}}      & EmbodiedCity & - \\ \hline
\end{tabular}}
\end{table}

\begin{figure*}[!htb]
\centering
    \includegraphics[width=0.78\linewidth]{figures/example.pdf}
% \vspace{-0.2cm}
\caption{The typical workflow of the PMA to address City EQA tasks. There are two cars in this area, thus a valid question must contain landmarks and spatial relationships to specify a car. Given the task, PMA will sequentially complete multiple sub-tasks to find the answer.}
% \vspace{-0.2cm}
\label{fig:example}
\end{figure*}

As an initial step toward CityEQA, we developed \textbf{CityEQA-EC}, a benchmark dataset to evaluate embodied agents' performance on CityEQA tasks. The distinctions between this dataset and other EQA benchmarks are summarized in Table \ref{table:dataset}. CityEQA-EC comprises six task types characterized by open-vocabulary questions. These tasks utilize urban landmarks and spatial relationships to delineate the expected answer, adhering to human conventions while addressing object ambiguity. This design introduces significant complexity, turning CityEQA into long-horizon tasks that require embodied agents to identify and use landmarks, explore urban environments effectively, and refine observation to generate high-quality answers.

To address CityEQA tasks, we introduce the \textbf{Planner-Manager-Actor (PMA)}, a novel baseline agent powered by large models, designed to emulate human-like rationale for solving long-horizon tasks in urban environments, as illustrated in Figure \ref{fig:example}. PMA employs a hierarchical framework to generate actions and derive answers. The Planner module parses tasks and creates plans consisting of three sub-task types: navigation, exploration, and collection. The Manager oversees the execution of these plans while maintaining a global object-centric cognitive map \cite{deng2024opengraph}. This 2D grid-based representation enables precise object identification (retrieval) and efficient management of long-term landmark information. The Actor generates specific actions based on the Manager's instructions through its components: Navigator, Explorer, and Collector. Notably, the Collector integrates a Multi-Modal Large Language Model (MM-LLM) as its Vision-Language-Action (VLA) module to refine observations and generate high-quality answers.
PMA's performance is assessed against four baselines, including humans. 
Results show that humans perform best in CityEQA, while PMA achieves 60.73\% of human accuracy in answering questions, highlighting both the challenge and validity of the proposed benchmarks. 

% The Frontier-Based Exploration (FBE) Agent, widely used in indoor EQA tasks, performs worse than even a blind LLM. This underscores the importance of PMA's hierarchical framework and its use of landmarks and spatial relationships for tackling CityEQA tasks.

In summary, this paper makes the following significant contributions:
\vspace{-8pt}
\begin{itemize}[leftmargin=*]
    \item To the best of our knowledge, we present the first open-ended embodied question answering benchmark for city space, namely CityEQA-EC.
    \vspace{-7pt}
    \item We propose a novel baseline model, PMA, which is capable of solving long-horizon tasks for CityEQA tasks with a human-like rationale.
     \vspace{-7pt}
    \item Experimental results demonstrate that our approach outperforms existing baselines in tackling the CityEQA task. However, the gap with human performance highlights opportunities for future research to improve visual thinking and reasoning in embodied intelligence for city spaces.
\end{itemize}




\section{Related Works}


\noindent\textbf{3D Point Cloud Domain Adaptation and Generalization.}
Early endeavors within 3D domain adaptation (3DDA) focused on extending 2D adversarial methodologies~\cite{qin2019pointdan} to align point cloud features. Alternative methods have delved into geometry-aware self-supervised pre-tasks. Achituve \etal~\cite{achituve2021self} introduced DefRec, a technique employing self-complement tasks by reconstructing point clouds from a non-rigid distorted version, while Zou \etal~\cite{zou2021geometry} incorporating norm curves prediction as an auxiliary task. Liang \etal~\cite{liang2022point} put forth MLSP, focusing on point estimation tasks like cardinality, position, and normal. SDDA~\cite{cardace2023self} employs self-distillation to learn the point-based features. Additionally, post-hoc self-paced training~\cite{zou2021geometry,fan2022self,park2023pcadapter} has been embraced to refine adaptation to target distributions by accessing target data and conducting further finetuning based on prior knowledge from the source domain.
In contrast, the landscape of 3D domain generalization (3DDG) research remains nascent. Metasets~\cite{huang2021metasets} leverage meta-learning to address geometric variations, while PDG~\cite{wei2022learning} decomposes 3D shapes into part-based features to enhance generalization capabilities.
Despite the remarkable progress, existing studies assume that objects in both the source and target domains share the same orientation, limiting their practical application. This limitation propels our exploration into orientation-aware 3D domain generalization through intricate orientation learning.


\noindent\textbf{Rotation-generalizable Point Cloud Analysis.}
Previous works in point cloud analysis~\cite{qi2017pointnet, wang2019dynamic} enhance rotation robustness by introducing random rotations to augment point clouds. {However, generating a comprehensive set of rotated data is impractical, resulting in variable model performance across different scenes. To robustify the networks \wrt randomly rotated point clouds,} rotation-equivariance methods explore equivalent model architectures by incorporating equivalent operations~\cite{su2022svnet, Deng_2021_ICCV, luo2022equivariant} or steerable convolutions~\cite{chen2021equivariant, poulenard2021functional}.
Alternatively, rotation-invariance approaches aim to identify geometric descriptors invariant to rotations, such as distances and angles between local points~\cite{chen2019clusternet, zhang2020global} or point norms~\cite{zhao2019rotation, li2021rotation}. Besides, {Li \etal~\cite{li2021closer} have explored disambiguating the number of PCA-based canonical poses, while Kim \etal~\cite{kim2020rotation} and Chen \etal~\cite{chen2022devil} have transformed local point coordinates according to local reference frames to maintain rotation invariance. However, these methods focus on improving in-domain rotation robustness, neglecting domain shift and consequently exhibiting limited performance when applied to diverse domains. This study addresses the challenge of cross-domain generalizability together with rotation robustness and proposes novel solutions.} 

\noindent\textbf{Intricate Sample Mining}, aimed at identifying or synthesizing challenging samples that are difficult to classify correctly, seeks to rectify the imbalance between positive and negative samples for enhancing a model's discriminability. While traditional works have explored this concept in SVM optimization~\cite{felzenszwalb2009object}, shallow neural networks~\cite{dollar2009integral}, and boosted decision trees~\cite{yu2019unsupervised}, recent advances in deep learning have catalyzed a proliferation of researches in this area across various computer vision tasks. For instance, 
Lin \etal~\cite{lin2017focal} proposed a focal loss to concentrate training efforts on a selected group of hard examples in object detection, while Yu \etal~\cite{yu2019unsupervised} devised a soft multilabel-guided hard negative mining method to learn discriminative embeddings for person Re-ID. Schroff \etal~\cite{schroff2015facenet} introduced an online negative exemplar mining process to encourage spherical clusters in face embeddings for individual recognition, and Wang \etal~\cite{wang2021instance} designed an adversarially trained negative generator to yield instance-wise negative samples, bolstering the learning of unpaired image-to-image translation. In contrast to existing studies, our work presents the first attempt to mitigate the orientational shift in 3D point cloud domain generalization, by developing an effective intricate orientation mining strategy to achieve orientation-aware learning.



\section{Orientational Shift Analysis} \label{sec:analysis}


\textbf{Problem Definition.}
In practical scenarios, objects often originate from various domains, and concurrently, present diverse orientations. To achieve an accurate understanding of these targets, our goal is to grant 3D recognition systems with cross-domain generalizability and robustness to rotational transformations. To this end, we introduce a novel task: orientation-aware 3D domain generalization, and explore the applicability of existing 3D recognition systems under orientational shifts.
Let $\mathbb{X}$ and $\mathbb{Y}$ be the input and label spaces, respectively. We consider a labeled source domain $D_s = {\{(P_i^s, y_i^s)}\}^{n_s}_{i=1}$ with $n_s$ samples and an unlabeled target domain $D_t = \{{P_i^t}\}^{n_t}_{i=1}$. Here, $y^s\in \mathbb{Y}$ is the source labels, and $P^s, P^t \in \mathbb{X}$ represent point clouds with a specified orientation relative to the world coordinate system, where each $P_i \in \mathbb{R}^{N_c \times 3}$ consists of $N_c$ points. The orientation is defined by a $3\times3$ rotation matrix $M_i \in \mathcal{O} \subseteq$ SO(3), where $\mathcal{O}$ is the set of orientations in the given domain and SO(3) represents all rotations in $\mathbb{R}^3$. The objective of orientation-aware 3D domain generalization is to learn a projection function $g: \mathbb{X} \to \mathbb{Y}$ that can be applied to any given target domain $D_t$ with arbitrary orientations in $\mathcal{O}$, by solely training on the labeled source domain $D_s$.


\begin{figure}[t]
    % \flushleft
    \centering
    \subfloat[Random rotation augmentation]{%[b]{0.45\textwidth}
        \label{fig:statement_random}
        % \centering
        \includegraphics[width=0.23\textwidth]{./resources/teaser/analysis/random.png}
    }
    % \hspace{5mm}
    \subfloat[Intricate orientation training]{%[b]{0.45\textwidth}
        \label{fig:statement_ours}
        % \centering
        \includegraphics[width=0.23\textwidth]{./resources/teaser/analysis/full.png}
    }
    \vspace{-2mm}
    \caption{t-SNE visualization of the feature spaces trained with (a) random rotation augmentation and (b) our proposed intricate orientation training. For the sake of simplicity, we only visualize samples from three categories (table, chair, and sofa), which share part of distinguishing features (\eg, slim legs, plane seats, \etc). Samples from ModelNet (the source domain) and ShapeNet (the target domain) are denoted by star (`$\star$') and plus (`$+$'), respectively. The gray dash line denotes the approximated decision boundaries.}
    \label{fig:feature_statement}
    \vspace{-5mm}
\end{figure}


\noindent\textbf{Generalization Analysis of Orientations.}
To accurately understand 3D shapes, information from multiple perspectives is often necessary. Some of the perspectives are easy-to-understand by deep models but less informative, incurring a phenomenon known as ``taking the whole from a part''. Specifically, in the context of rotation-robust point cloud analysis, these learning perspectives refer to those rotated variants with small gradients during training. Random rotation attempts to capture comprehensive information through a Monte Carlo approach, but the vast number of possible rotated angles introduces difficulties and leads to imbalanced learning.
This biased learning problem could give rise to inaccurate selection of point features, making false decisions. As shown in Fig.~\ref{fig:feature_statement}(a), the learned model cannot well distinguish the rotated sample from ``table" and ``chair", as they may contain similar features such as slim legs, which are learned from easy-to-understand samples. 
On the other hand, intricate samples lying beyond current cognitive boundaries can offer complementary knowledge for learning a more accurate discriminative boundary.

We empirically validate these insights by employing Maximum Mean Discrepancy (MMD)~\cite{gretton2012kernel} as a measure of distributional shifts. The MMD calculates the distance between source and target distributions in the Reproducing Kernel Hilbert Space (RKHS). We train a DGCNN~\cite{wang2019dynamic} on the ModelNet (source domain) and test it on ShapeNet (target domain), augmenting the data with random rotations and intricate orientations, respectively. The intricate orientations of the training dataset are obtained by optimizing the rotational angles of each point cloud to maximize the cross-entropy loss with a baseline model trained on aligned data. Fig.~\ref{fig:analysis}(a) depicts the per-class MMD values, demonstrating that training with intricate orientations reduces orientational shifts more effectively. To evaluate discriminability and consistency under varying rotations, we further train a linear SVM on the source domain and test it on the target domain. We augment each point cloud 64 times as illustrated in Section~\ref{sec:experiment} and measure consistency by calculating the mean KL-divergence between their output probabilities. As shown in Fig.~\ref{fig:analysis}(a), the discriminability of the model augmented with intricate orientations matches that of the randomly rotated version, while better preserving consistency across different rotations.
Based on these findings, we aim to leverage intricate samples to reduce orientational shift, enhance output consistency, improve model discriminability to arbitrary rotations, and finally obtain a better classifier that is generalizable towards various domains (Fig.~\ref{fig:analysis}(b)).



% \vspace{-3mm}
\section{Intricate Orientation Learning} \label{sec:method}

\begin{figure}[t]
    \captionsetup[subfloat]{justification=centering,singlelinecheck=false}
    % \flushleft
    \centering
    \subfloat[Per-class maximum mean discrepancy]{%[b]{0.45\textwidth}
        \label{fig:analysis_a}
        % \centering
        \includegraphics[width=0.32\textwidth]{./resources/teaser/analysis/divergence.pdf}
    }
    % \hspace{.4in}
    \subfloat[Consistency \\\& discrimination]{%[b]{0.45\textwidth}
        \label{fig:analysis_b}
        % \centering
        \includegraphics[width=0.16\textwidth]{./resources/teaser/analysis/consistency.pdf}
    }
    \vspace{-3mm}
    \caption{(a) Maximum Mean Discrepancy~\cite{borgwardt2006integrating} between the features of the ModelNet and ShapeNet subdomains in PointDA~\cite{qin2019pointdan}, learned with different orientation augmentations. (b) The mean consistency rate was computed using KL-divergence between multiple augmented variants and the mean accuracy on ShapeNet.}
    \vspace{-5mm}
    \label{fig:analysis}
\end{figure}


The pipeline of our framework is depicted in Fig.~\ref{fig:pipeline}. It consists of an iterative optimization process that alternates between intricate orientation mining and orientation-aware contrastive training. We first generate an alternative set of intricate rotation angles through intricate orientation mining. Then we augment the training point clouds by applying the selected angles from the alternative set, creating a series of intricate sample pairs. To enhance the generalizability of the learned representations, we employ orientation-aware contrastive training using these intricate pairs. Our framework incorporates an orientation consistency loss, which encourages the learning of more representative features while promoting their consistency \wrt random rotations. Additionally, we introduce a margin separation loss to further improve the categorical discriminability of the learned representation space.
In the following sections, we elaborate on each component of our framework in detail.


\begin{figure*}[t]
    % \flushleft
    \centering
    \includegraphics[width=1\textwidth]{./resources/pipeline/pipeline.pdf}
    \vspace{-6mm}
    \caption{Pipeline of our intricate orientation learning framework for point cloud classification, which alternatively optimizes between the intricate orientation mining (Left) and orientation-aware contrastive training (Right). }
    \label{fig:pipeline}
    \vspace{-6mm}
\end{figure*}


% \vspace{-3mm}
\subsection{Intricate Orientation Mining}
Let us first consider a standard classification problem over the given data distribution $D = {\{(P_i, y_i)}\}^{n}_{i=1}$. Suppose we have a point cloud classification model, the goal is to search the model parameters $\omega^*$ that optimize the empirical risk $E_{(P, y)\sim D}\left[L(w, P, y)\right]$, where $L$ is a proper loss function. To improve the robustness, the most adopted technique is data augmentation, such that the model can resist possible perturbations on the input data, resulting in the following objective:
\begin{equation}
    \omega^*=\mathop{\arg\min_{\omega}} \mathbb{E}_{(P, y)\sim D} \left[L(\omega, f(P), y)\right],
\end{equation}
where $f$ is the perturbation function. Considering the orientational shift only, the perturbation function of a given point cloud $P_i$ can be regarded as augmenting the aligned pose with a rotation matrix $M_i$. According to Euler's rotation theorem, the rotation matrix is defined in $\mathbb{R}^3$ as $M_i = R_{\theta_{x_i}}\cdot R_{\theta_{y_i}}\cdot R_{\theta_{z_i}}$, 
where $R_{\theta_j}$ is the rotation matrix about the axis-j by angle $\theta_{j} \in \left[-\pi, \pi\right)$ over the cartesian coordinates. In this case, the rotation matrix is parameterized by $\Theta_i=\left[\theta_{x_i}, \theta_{y_i}, \theta_{z_i}\right]$. Since the trigonometric function is differential in the scope $\left[-\pi, \pi\right)$, we can optimize $\Theta_i$ through gradient descent.



The essence of intricate orientation mining lies in the deliberate search for intricate orientations, instead of directly optimizing the model with random rotations. This endeavor aims to identify a rotation matrix $\hat{\Theta}_i$ that introduces perturbations to the point cloud, such that the 3D model is confounded:
\begin{equation}
    \hat{\Theta}_i = \mathop{\arg\max_{\Theta_i}}L(\omega^*, f(\Theta_i, P_i), y_i), \label{formula:iom}
\end{equation}
where $f(\Theta_i, P_i)=M_i\cdot P_i$, and $L$ is a task-specific optimization function~(\eg, cross-entropy loss for classification task, hereafter referred to as $L_{cls}$). By iteratively maximizing this objective, we can calculate the most intricate orientations for all the point clouds within $D$. The details of gradient calculation are provided in the supplement. We adopt Projected Gradient Descent~\cite{madrytowards} to ensure that the optimized $\Theta_i$ can generate a pure rotation matrix. By applying intricate orientation mining to the whole training dataset, we obtain the intricate orientation set $I=\{\hat{\Theta}_{i}\}^{n}_{i=1}$, which is essential for the construction of intricate sample pairs. Note that each orientation is specified for the according sample. To increase the diversity of $I$, we further initialize $\Theta$ with random values and repeat the optimization multiple times. Finally, the intricate set is defined as $I=\{\{\hat{\Theta}^j_{i}\}^{AT}_{j=1}\}^{n}_{i=1}$, where $AT$ is the number of augmentation times. During the training stage, we periodically update the intricate set $I$ for every $T$ epochs. 




% \vspace{-3mm}
\subsection{Orientation-aware Contrastive Training}

Based on the intricate orientation set $I$, we construct the intricate sample pairs and build a contrastive learning framework upon them to obtain categorically discriminative and generalizable features with rotational consistency. As Fig.~\ref{fig:pipeline} shows, our model consists of an optimizable student network $F_s\!=\!C_s\circ E_s$ and a frozen teacher network $F_t\!=\!C_t\circ E_t$, where $E_{s/t}$ is a feature extractor which encodes point cloud to the representation space and derives the final classification result through a linear classifier $C_{s/t}$. The teacher network is optimized by Exponential Moving Average (EMA)~\cite{tarvainen2017mean}.

\noindent\textbf{Inner-sample Rotational Consistency.} The main idea is to maximize the agreement between the original shape and its intricate augmented variants via self-contrastive learning. Each time we sample a mini-batch of $N$ point clouds $\left\{P_{m}\right\}^N_{m=1}$ and randomly select one intricate rotation $\hat{\Theta}_m$ from $I$ for each point cloud. Then we construct the intricate sample pairs $B_{it} = \{(P_{m}, \hat{P}_m )\}^N_{m=1}$, where $\hat{P}_m = f(\hat{\Theta}_{m}, P_m)$ is the augmented version of the original point cloud with the intricate rotation. $B_{it}$ serve as the input of (${F_t}$, $F_s$) and the output logits of $(P_{m}, \hat{P}_m )$ are denoted as $p_m$ and $\hat{p}_m$, respectively. To ensure the consistency of the intricate pairs, we perform knowledge distillation~\cite{hinton2015distilling} on the output probabilities, the orientation consistency loss is therefore formulated as:
\begin{equation}\label{eq5-0}
    \sigma(p/\tau) = \frac{exp(p/\tau)}{\sum_{j=1}^{K}{exp(p^{(j)}/\tau)}},
\end{equation}
\begin{equation}
    L_{oc} = -\frac{1}{N}\sum_{m=1}^{N} \sigma(\hat{p}_m/\tau_t)\log(\sigma(p_m /\tau_s)), \label{eq5}
\end{equation}
where $K$ is the number of categories and $\sigma(\cdot)$ is the softmax function. $\tau \in \mathbb{R^+}$ is the temperature parameter controlling the magnitude of the output logits, where the subscripts in $\left[\tau_s, \tau_t\right]$ denote the source and target domains, respectively. By iteratively refining the distance between the aligned object and its intricate variant, the model can gradually adapt to the harder rotational variations and learn common knowledge concerning different rotations.


\noindent\textbf{Intra-category Discriminability.} To further enhance discriminability, we aim for the learned representation space to maintain rotational consistency while improving the classification ability. To achieve this, we introduce a margin separation loss that yields a classifier with a concise and accurate boundary, capable of handling the unforeseen rotational perturbations. Specifically, we first construct the intra-category intricate pairs by considering the relationships between different categories in the same mini-batch. The intricate pairs can be divided into two groups, including the positive pairs and the negative pairs.
Simply, the positive pairs can be found by traversing the training batch and sorting out samples that belong to the same class. But the embedding space under multiple orientations may not be well simulated by the originally aligned point clouds. Instead of learning from only one intricate orientation, we further augment each point cloud $P_{i}$ with $V$ number of intricate angles sampled from $\{\hat{\Theta}^l_{i}\}^{AT}_{l=1}$ to better simulate the feature space. The loss is designed to minimize the cosine distance between the positive samples as follows:
\begin{equation}
    L_{pos}\!=\!- \frac{1}{K}\sum_{k=1}^{K} \!\frac{1}{V{N^k_p}^2}\!\!\!\sum_{i=1,j=1 \atop j\neq i}^{N^k_{p}, N^k_{p}}\!\sum_{v=1}^{V} \cos(\frac{E_s(\hat{P}_{i,v})}{\tau'},\! \frac{E_s(\hat{P}_j)}{\tau'}),
    \label{eq6}
\end{equation}
where $\cos(\cdot, \cdot)$ denotes the cosine similarity between the two inputs, $N^k_{p}$ is the number of samples of class $k$ within the mini-batch and $\tau'$ is the temperature parameter.

The negative pairs are formed by samples with different labels.
To enlarge the discrepancy of different categories in the representation space, we plan to maximize the distance between each negative pair with:
\begin{equation}
    L_{neg}\!=\!\frac{1}{K}\sum_{k=1}^{K}  \frac{1}{{N^k_p}{N^k_n}}\sum_{i=1}^{N^k_p} \sum_{j=1}^{N^k_{n}} \cos( \frac{E_s(\hat{P}_i)}{\tau'}, \frac{E_s(\hat{P}_j)}{\tau'}),
    \label{eq7}
\end{equation}
where $N^k_{n}$ is the number of negative samples.
The overall margin separation loss is summarized as follows:
\begin{equation}
    L_{ms} = L_{pos} + L_{neg}.
\end{equation}Through contrastive tasks, the point representations under varying rotations are further clustered within the same category, while the other categories are pushed away. A more compact and consistent representation space is formed, thereby enhancing the discriminability for rotated shape classification.

\begin{table*}[t] % table for OSDA setting on Office31
    \caption{Comparison of the macro-average precision score \textit{Avg.}~($\%$) under the orientation-aware 3D domain generalization setting. The value after $\pm$ denotes the standard deviation on the 64 evaluated series. RE, RI, DA and DG represent Rotation-Equivalent, Rotation-Invariant, 3D Domain Adaptation, and 3D Domain Generalization. The top 2 records are marked in \bc{red} and \rc{blue}, respectively.}
    \label{tab:pointda10_avg} 
    \vspace{-4mm}
    \small
    \begin{center}
    \setlength{\tabcolsep}{0.4cm}{
    \resizebox{1\textwidth}{!}{
    \begin{tabular}{l|c|l|l|l|l|l|l|c}
    \hline
    \multicolumn{9}{c}{DGCNN~\cite{wang2019dynamic}}\\
    \hline
    \makecell[c]{Methods}
    &Type & \makecell[c]{M$\to$S} & \makecell[c]{M$\to$S*} & \makecell[c]{S$\to$M} & \makecell[c]{S$\to$S*} & \makecell[c]{S*$\to$M} & \makecell[c]{S*$\to$S} & {AVG} \\


    \hline
    Supervised                                &\multirow{2}{*}{-}                  &{74.4 $\pm$ 5.5}   &{56.7 $\pm$ 3.9}   &{87.3 $\pm$ 14.5}   &{56.7 $\pm$ 3.9}  &{87.3 $\pm$ 14.5}  &{74.4 $\pm$ 5.5}    &{72.8}\\
    w/o Adapt                                 &                  &{54.7 $\pm$ 9.2}   &{25.3 $\pm$ 2.3}   &{57.6 $\pm$ 8.6}   &{27.9 $\pm$ 2.8}  &{45.3 $\pm$ 4.4}  &{37.1 $\pm$ 2.8}    &{41.3}    \\
    \hline
    VN~\cite{Deng_2021_ICCV}                  &\multirow{3}{*}{RE}                  &{61.2 $\pm$ 0.0}    &{27.3 $\pm$ 0.0} &{68.9 $\pm$ 0.0} &{23.5 $\pm$ 0.0} &{35.1 $\pm$ 0.0} &{31.5 $\pm$ 0.0} &{41.3}\\
    SVN~\cite{su2022svnet}                  &                  &{58.4 $\pm$ 0.7}    &{26.3 $\pm$ 0.6} &{64.9 $\pm$ 0.8}  &{23.6 $\pm$ 0.4} &{33.1 $\pm$ 0.8}    &{31.1 $\pm$ 0.8} &{39.6}\\
    EOMP~\cite{luo2022equivariant}            &                  &{56.7 $\pm$ 0.6}    &{27.0 $\pm$ 0.5} &{64.2 $\pm$ 0.9}  &{26.6 $\pm$ 0.5} &{25.6 $\pm$ 0.6}    &{26.9 $\pm$ 0.4} &{37.8}\\
    \hline
    SPRIN~\cite{you2021prin}              &\multirow{5}{*}{RI}             &{58.6 $\pm$ 0.7} &{27.3 $\pm$ 0.7}   &{73.9 $\pm$ 0.6} &{20.7 $\pm$ 0.9}    &{45.7 $\pm$ 0.7} &{37.3 $\pm$ 0.8}  &{43.9}\\
    RIPCA~\cite{li2021closer}              &            &{62.5 $\pm$ 1.2} &\rc{30.5 $\pm$ 1.2}   &{72.9 $\pm$ 0.8} &{27.2 $\pm$ 1.1}    &{47.5 $\pm$ 1.8} &{39.0 $\pm$ 1.2}  &\rc{46.6}\\
    RIConv++~\cite{zhang2022riconv}        &             &{38.9 $\pm$ 0.4} &{19.0 $\pm$ 0.6}   &{57.2 $\pm$ 0.8} &\rc{30.2 $\pm$ 1.0}    &{32.5 $\pm$ 0.7} &{38.4 $\pm$ 0.7}  &{36.0}\\ % special design for Pointnet++
    PaRI~\cite{chen2022devil}              &             &{40.9 $\pm$ 0.0} &{25.0 $\pm$ 0.5}   &{55.8 $\pm$ 0.7} &\bc{32.5 $\pm$ 0.7}    &{44.0 $\pm$ 1.0} &{39.4 $\pm$ 0.1}  &{39.6}\\
    LocoTrans~\cite{chen2024local}              &                  &\rc{65.2 $\pm$ 0.2}   &{30.4 $\pm$ 0.5}    &\rc{76.5 $\pm$ 0.4}  &{25.7 $\pm$ 0.4}  &{37.7 $\pm$ 0.7} &{26.9 $\pm$ 0.0}    &{43.7}\\ 
    \hline
    PointDAN~\cite{qin2019pointdan}           &\multirow{6}{*}{DA}                   &{50.5 $\pm$ 9.4}   &{27.5 $\pm$ 2.5}   &{58.9 $\pm$ 7.0}   &{26.1 $\pm$ 2.4}   &{36.6 $\pm$ 5.1}   &{34.7 $\pm$ 3.1}    &{39.1}    \\
    DefRec~\cite{achituve2021self}            &                   &{53.3 $\pm$ 9.2}   &{24.0 $\pm$ 3.6}   &{58.2 $\pm$ 8.0}   &{21.5 $\pm$ 2.9}   &{37.4 $\pm$ 6.3}   &{30.0 $\pm$ 3.0}    &{37.4}     \\
    GAST~\cite{zou2021geometry}               &                   &{37.9 $\pm$ 6.3}   &{20.7 $\pm$ 1.9}   &{47.4 $\pm$ 2.5}   &{14.7 $\pm$ 2.0}   &{28.9 $\pm$ 1.1}   &{26.0 $\pm$ 1.0}    &{29.3}     \\
    MLSP~\cite{liang2022point}                &                   &{60.9 $\pm$ 9.2}   &{30.0 $\pm$ 3.9}   &{62.6 $\pm$ 5.3}   &{23.4 $\pm$ 3.7}   &{44.9 $\pm$ 4.7}   &{37.3 $\pm$ 3.1}    &{43.2}     \\
    SDDA~\cite{cardace2023self}               &                   &{58.4 $\pm$ 9.8}   &\rc{30.5 $\pm$ 3.8}   &{64.4 $\pm$ 5.3}   &{25.9 $\pm$ 3.8}   &{40.1 $\pm$ 6.1}   &{39.5 $\pm$ 3.4}    &{43.1}    \\
    PCFEA~\cite{wang2024progressive}     &                   &{62.2 $\pm$ 10.3}   &{10.5 $\pm$ 0.2}   &{59.2 $\pm$ 8.7}   &{21.6 $\pm$ 2.4}   &\rc{48.0 $\pm$ 3.9}   &\rc{39.7 $\pm$ 3.9}    &{40.2}    \\
    \hline
    {Metasets~\cite{huang2021metasets}}       &\multirow{3}{*}{DG}                   &{47.4 $\pm$ 9.3}    &{14.0 $\pm$ 0.5} &{38.4 $\pm$ 12.2}  &{10.1 $\pm$ 0.8} &{21.8 $\pm$ 3.9}    &{19.6 $\pm$ 3.8} &{25.2}\\
    {PDG~\cite{wei2022learning}}              &                  &{31.9 $\pm$ 22.4}   &{24.4 $\pm$ 14.3}    &{43.8 $\pm$ 17.2}   &{14.0 $\pm$ 2.7}  &{30.6 $\pm$ 5.2}  &{30.5 $\pm$ 6.8}    &{29.2}\\
    {Ours}                                    &                  &\bb{66.2 $\pm$ 1.5}   &\bb{30.9 $\pm$ 2.2}   &\bb{81.7 $\pm$ 0.7}   &{30.1 $\pm$ 1.3}   &\bb{48.6 $\pm$ 2.3}  &\bb{39.8 $\pm$ 2.4}    &\bb{49.6}    \\
    \hline
    \hline
    \multicolumn{9}{c}{PointNet~\cite{qi2017pointnet}}\\
    \hline
    Supervised                    &\multirow{2}{*}{-}     &{70.1 $\pm$ 7.9}   &{40.9 $\pm$ 4.0}   &{86.4 $\pm$ 18.8}   &{40.9 $\pm$ 4.0}   &{86.4 $\pm$ 18.8}   &{70.1 $\pm$ 7.9}    &{65.8} \\
    w/o Adapt                     &                  &{51.8 $\pm$ 11.7}   &{21.9 $\pm$ 3.4}   &{67.5 $\pm$ 6.6}   &{24.5 $\pm$ 2.2}   &{33.8 $\pm$ 4.2}   &{32.9 $\pm$ 2.6}       &{38.7}   \\
    \hline
    VN~\cite{Deng_2021_ICCV}      &\multirow{2}{*}{RE}    &{57.3 $\pm$ 0.0}   &{22.0 $\pm$ 0.0}   &\rc{68.9 $\pm$ 0.0}   &{20.1 $\pm$ 0.0}   &\rc{34.5 $\pm$ 0.0}   &{26.3 $\pm$ 0.0}  &\rc{38.2} \\
    SVN~\cite{su2022svnet}      &    &{52.7 $\pm$ 0.2}   &{22.3 $\pm$ 0.3}   &{56.2 $\pm$ 0.2}   &{18.1 $\pm$ 0.0}   &{17.0 $\pm$ 0.3}   &{16.7 $\pm$ 0.2}  &{30.5} \\
    \hline
    
    % \hline
    PointDAN~\cite{qin2019pointdan} &\multirow{6}{*}{DA}   &{40.4 $\pm$ 10.4}   &{17.5 $\pm$ 2.5}   &{37.6 $\pm$ 7.4}   &{14.8 $\pm$ 3.5}   &{10.8 $\pm$ 0.5}   &{11.8 $\pm$ 0.8}    &{22.2}    \\
    DefRec~\cite{achituve2021self}  &                   &{34.8 $\pm$ 8.6}   &{15.0 $\pm$ 3.1}   &{35.3 $\pm$ 9.9}   &{14.3 $\pm$ 5.4}   &{13.9 $\pm$ 2.8}   &{11.8 $\pm$ 2.2}    &{20.9}     \\
    GAST~\cite{zou2021geometry}     &                   &{45.3 $\pm$ 4.1}   &\rc{24.6 $\pm$ 1.5}   &{53.6 $\pm$ 3.0}   &{20.3 $\pm$ 1.9}   &{21.4 $\pm$ 2.8}   &{24.4 $\pm$ 3.1}    &{31.6}     \\
    MLSP~\cite{liang2022point}      &                   &{44.2 $\pm$ 10.5}   &{21.8 $\pm$ 4.4}   &{46.6 $\pm$ 8.4}   &{18.3 $\pm$ 3.5}   &{20.2 $\pm$ 5.7}   &{19.0 $\pm$ 3.8}    &{28.4}     \\
    SDDA~\cite{cardace2023self}     &                   &\rc{58.4 $\pm$ 12.3}   &{19.0 $\pm$ 2.5}   &{55.0 $\pm$ 7.1}   &\rc{21.9 $\pm$ 3.7}   &{32.3 $\pm$ 6.9}   &{28.2 $\pm$ 4.5}    &{35.8}    \\
    PCFEA~\cite{wang2024progressive}     &                   &{50.2 $\pm$ 11.3}   &{10.4 $\pm$ 0.2}   &{44.8 $\pm$ 13.2}   &{10.4 $\pm$ 0.5}   &{17.1 $\pm$ 2.3}   &\rc{33.1 $\pm$ 3,2}    &{27.7}    \\
    \hline
    {Metasets~\cite{huang2021metasets}}&\multirow{3}{*}{DG}                   &{31.0 $\pm$ 7.3}    &{11.9 $\pm$ 0.9} &{32.5 $\pm$ 12.7}  &{10.0 $\pm$ 1.0} &{14.3 $\pm$ 4.5}    &{12.3 $\pm$ 2.4} &{18.7}\\
    {PDG~\cite{wei2022learning}}   &                  &{34.0 $\pm$ 11.5}   &{21.1 $\pm$ 3.0}    &{37.3 $\pm$ 17.8}   &{20.0 $\pm$ 4.2}  &{31.0 $\pm$ 8.9}  &{27.0 $\pm$ 5.3}    &{28.4}\\
    {Ours}                         &                    &\bb{63.3 $\pm$ 1.4}   &\bb{27.3 $\pm$ 1.3}   &\bb{81.4 $\pm$ 1.9}   &\bb{29.0 $\pm$ 1.2}   &\bb{42.7 $\pm$ 2.6}   &\bb{34.7 $\pm$ 1.5}       &\bb{46.4} \\
    \hline

    \end{tabular}

    }
    }
    \end{center}
    \vspace{-7mm}
\end{table*}



The entire framework is optimized in an alternative scheme between the intricate orientation mining and the orientation-aware contrastive training.
For the intricate orientation mining, the final objective is the same as Eq.~\ref{formula:iom}, where $L$ is the standard cross-entropy function over the originally aligned point clouds.
The intricate set is periodically updated after $T$ epochs of training to avoid overfitting to the current intricate rotations.
For orientation-aware contrastive training, the final loss function is
\begin{equation}
    L_{final} = L_{cls} + \lambda_{oc} L_{oc} + \lambda_{ms} L_{ms},
    \label{eq9}
\end{equation}
where $L_{cls}$ is the cross-entropy loss on the intricate point clouds. 


\section{Experiments}
\begin{table}[t]
\centering
{\resizebox{\columnwidth}{!}{
\begin{tabular}{lccccccc}
\toprule
\multicolumn{1}{c}{\multirow{2}{*}{\textbf{Model}}} & \multirow{2}{*}{\begin{tabular}[c]{@{}c@{}}\textbf{Parameter}\\ \textbf{Scale}\end{tabular}}  & \multicolumn{5}{c}{\textbf{Multifacet}} & \multirow{2}{*}{\textbf{Average}} \\ \cmidrule{3-7}  
\multicolumn{1}{c}{} &  & \multicolumn{1}{l}{\textbf{AE}} & \multicolumn{1}{l}{\textbf{FL}} & \multicolumn{1}{l}{\textbf{Ko}} & \multicolumn{1}{l}{\textbf{MT}} & \multicolumn{1}{l}{\textbf{SI}} & \\ \midrule
\textit{Open-Source Models} \\ \midrule
Solar-10.7B-instruct & 10.7B & 3.30 & 3.31 & 3.09 & 3.19 & 3.08 & 3.19  \\  
Gemma-2-9b-it & 9B & 4.10 & 3.80 & 4.26 & 4.15 & 3.92 & 4.05  \\ 
\midrule
\multicolumn{8}{l}{\textit{Open-source Models} $+$ \textit{KD (Fine-tuning on \textbf{\textsc{SysGen}} dataset)}} \\
\midrule 
Solar-10.7B-instruct & 10.7B & 3.97 & 3.73 & 3.64 & 3.98 & 3.52 & 3.76 (+0.57) \\ 
Gemma-2-9b-it & 9B & 4.40 & 4.04 & 4.30 & 4.23 & 4.18 & 4.23 (+0.18) \\ 
\bottomrule
\end{tabular}}}
\caption{
We conduct a knowledge distillation (KD) experiments leveraging data generated by \textsc{SysGen} pipeline using Phi-4.}
\label{tab:knowledge_distillation}
\vspace{-0.3cm}
\end{table}
\begin{table*}[t]
\centering
{\resizebox{\textwidth}{!}{
\begin{tabular}{lcccccccccc}
\toprule
\multicolumn{1}{c}{\multirow{2}{*}{\textbf{Model}}} & \multirow{2}{*}{\begin{tabular}[c]{@{}c@{}}\textbf{Parameter}\\ \textbf{Scale}\end{tabular}}  & \multicolumn{8}{c}{\textbf{Unseen Benchmarks}} & \multirow{2}{*}{\textbf{Average}} \\  \cmidrule{3-10} 
\multicolumn{1}{c}{} &  & \multicolumn{1}{l}{\textbf{MMLU}} & \multicolumn{1}{l}{\textbf{MMLU-Pro}} & \multicolumn{1}{l}{\textbf{ARC-c}} & \multicolumn{1}{l}{\textbf{GPQA}} & \multicolumn{1}{l}{\textbf{HellaSwag}} & \multicolumn{1}{l}{\textbf{IFEVAL}} & \textbf{MATHQA} & \textbf{BBH} & \\ \midrule
\multicolumn{11}{l}{\textit{Open-Source Models}} \\ \midrule 
Solar-10.7B-instruct & 10.7B  &  63.28 & 30.20 & 63.99 & 30.36 & 86.35 & 38.59 &  36.38 & 37.28 & 48.31 \\
Gemma-2-9b-it & 9B & 73.27 & 32.78 & 67.89 & 31.05 & 81.92 & 74.78 & 38.87 & 41.98 & 55.31 \\ 
LLaMA-3.1-8B-instruct & 8B & 67.95 & 40.87 & 54.95 & 34.60 & 79.18 & 50.71 & 39.53 & 70.85 & 54.83 \\ 
% Mixtral-8x22B-instruct & 8x22B & 75.62 &  52.63 & 67.83 & 36.83 & 87.68 & 60.43 & 50.08 & 83.03 & \\   
Qwen2.5-14B-instruct & 14B &  79.73  & 51.22 & 67.39 & 45.51 & 82.31 & 79.83 & 42.12 & 78.25 & 65.79 \\ 
Phi-4 & 14B & 84.56 & 70.12  & 68.26 & 55.93 & 84.42 & 62.98 & 48.87 & 79.87 & 69.37 \\  
\midrule
\multicolumn{11}{l}{\textit{Open-Source Models (Fine-tuning on original SFT Dataset)}} \\ \midrule
Solar-10.7B-instruct & 10.7B & 62.38 & 29.12 & 58.87 & 29.17 & 81.58 & 31.27 & 37.21 & 32.85 & 45.30 (-3.01) \\ 
Gemma-2-9b-it & 9B & 71.85 & 31.67  & 62.57 & 30.51 & 77.54 & 69.25 & 39.12 & 37.25 & 52.47 (-2.84) \\ 
LLaMA-3.1-8B-instruct & 8B & 65.34 &  36.85 &  
54.18 & 33.93 & 77.98 & 35.64 & 40.03 & 62.83 & 50.85 (-3.98)  \\ 
% Mixtral-8x22B-instruct & 8x22B&  &   & & & & & &  & \\ 
Qwen2.5-14B-instruct & 14B & 75.87  & 49.85  & 66.89 & 43.98 & 80.99 & 62.57 & 43.28 & 71.17 & 61.82 (-3.97) \\ 
Phi-4 & 14B & 80.27 & 66.58  & 66.27 & 52.89 & 83.39 & 55.83 & 49.98 & 75.49 & 66.33 (-6.04) \\ 
\midrule
\multicolumn{11}{l}{\textit{Open-Source Models (Fine-tuning on \textbf{\textsc{SysGen}} dataset)}} \\ \midrule 
LLaMA-3.1-8B-instruct & 8B & 66.89 & 39.77 & 54.55 & 34.21 & 78.89 & 46.75 & 42.11 & 68.98 & 54.02 (-0.81) \\ 
% Gemma-2-9B-instruct & 9B &  &   & & & & & &  & \\ 
% Mixtral-8x22B-instruct & 8x22B&  &   & & & & & &  & \\ 
% Solar-10.7B-instruct & 10.7B & 63.28 & 30.20 & 63.99 & 30.36 & 86.35 & 38.59 &  36.38 & 37.28 & \\ 
Qwen2.5-14B-instruct & 14B & 78.92 & 43.38 & 66.82 & 44.46 & 80.98 & 74.59 & 43.23 & 76.28 & 63.58 (-2.20) \\ 
Phi-4 & 14B & 83.27 & 68.77  & 67.89 & 55.18 & 84.31 & 57.87 & 50.23 & 77.12 & 68.08 (-1.29) \\ 
\midrule
\multicolumn{11}{l}{\textit{Open-source Models} $+$ \textit{Knowledge Distillation (Fine-tuning on \textbf{\textsc{SysGen}} dataset))}} \\
\midrule 
Solar-10.7B-instruct & 10.7B & 59.98  & 29.26  & 62.81 & 30.25 & 85.91 & 34.58 & 38.25 & 35.97 & 47.12 (-1.19) \\ 
Gemma-2-9b-it & 9B & 72.19 & 31.56 & 66.75 & 30.89 & 81.53 & 71.37 & 40.27 & 40.38 & 54.37 (-0.94) \\ 
\bottomrule
\end{tabular}}}
\caption{We utilize the Open LLM Leaderboard 2 score as the unseen benchmark. This reveals the key finding that adding system messages to existing SFT datasets does not lead to significant performance degradation.}
\label{tab:unseen_experiments}
\end{table*}
The primary goal of \textsc{SysGen} pipeline is to enhance the utilization of the \emph{system role} while minimizing performance degradation on unseen benchmarks, thereby improving the effectiveness of supervised fine-tuning (SFT).
To validate this, we evaluate how well the models trained on \textsc{SysGen} data generate appropriate assistant responses given both the system messages and user instructions, using the Multifacet~\citep{lee2024aligning} dataset.
For models that cannot generate data independently, we apply knowledge distillation to assess their effectiveness.
Additionally, we leverage the widely used Open LLM Leaderboard 2~\citep{myrzakhan2024open} as an unseen benchmark to determine whether our approach can be effectively integrated into existing SFT workflows.


\paragraph{\textsc{SysGen} provides better system message and assistant response to align with user instructions.}
Given the system messages and user instructions, the assistant's response is evaluated across four dimensions: style, background knowledge, harmlessness, and informativeness.
Each of these four aspects is scored on a scale of 1 to 5 using a rubric, and the average score is presented as the final score for the given instruction.
As shown in Table~\ref{tab:main_experiments}, recent open-source models achieve comparable scores to the proprietary models, indicating that open-source models have already undergone training related to system roles~\citep{meta2024introducing, yang2024qwen2, abdin2024phi}.

When trained on \textsc{SysGen} data, both LLaMA (4.12 → 4.21) and Phi (4.41 → 4.54) show score improvements.
Among the four dimensions, LLaMA exhibits score increases in style (4.15 → 4.32) and harmlessness (4.23 → 4.29).
Similarly, Phi shows the improvements in style (4.42 → 4.61) and informativeness (4.37 → 4.49).
As a result, even open-source models that have already been trained on system roles demonstrate their positive effects on style, informativeness, and harmlessness.



\paragraph{Knowledge distillation through \textsc{SysGen} data.}
If an open-source model does not support the system roles, it may not generate the system messages properly using \textsc{SysGen} pipeline. 
However, the effectiveness of knowledge distillation, using data generated by another open-source model without the limitation, remains uncertain.
To explore this, we train Gemma~\citep{team2024gemma} and Solar~\citep{kim-etal-2024-solar} using data generated by Phi-4~\citep{abdin2024phi}.
We use the Phi-4 data because it preserves most of the data and provides high quality  assistant responses as shown in Table~\ref{tab:statistics_generated_answer} and \ref{tab:data_statistics}.

As shown in Table~\ref{tab:knowledge_distillation}, even for models that do not inherently support system roles, modifying the chat template to incorporate system role and training on knowledge distilled dataset leads to an improvement in Multifacet performance, as observed in Gemma (4.05 → 4.23).
We describe the details in the Appendix~\ref{app:system_role_support}.
Additionally, for the Solar model, which had not been trained on system roles, we observe a dramatic performance improvement (3.19 → 3.76).\footnote{We speculate that Solar model did not properly learn the system role because its initial Multifacet score was low.}
This demonstrates that the data generated by the \textsc{SysGen} pipeline effectively supports the system roles.


\paragraph{\textsc{SysGen} data minimizes the performance degradation in unseen benchmarks.}
When incorporating system messages that were not present in the original SFT datasets and modifying the corresponding assistant responses, it is crucial to ensure that the model’s existing performance should not degrade.
For example, one key consideration in post-training is maintaining the model's original performance.
To assess this, we observed performance difference in unseen benchmark after applying supervised fine-tuning.
As shown in Table~\ref{tab:unseen_experiments}, we use the Open LLM Leaderboard 2 dataset as an unseen benchmark, with performance categorized into four groups:
\begin{itemize}
    \item Performance of existing open-source models (row 1-6)
    \item Performance of fine-tuning with open-source models using SFT datasets (row 7-12)
    \item Performance of fine-tuning with \textsc{SysGen} data (row 13-16)
    \item Performance after applying knowledge distillation using Phi-4 \textsc{SysGen} data (row 17-19)
\end{itemize}
The average performance degradation reflects the scores missing from each open-source model's original performance (row 1-6).

When fine-tuning with independently generated data using \textsc{SysGen}, the performance degradation is significantly lower than fine-tuning with the original SFT datasets selected under the same conditions.
Additionally, even for models that cannot generate data independently (e.g., those that do not support system roles), knowledge distillation helps mitigate performance drops considerably.









\vspace{-5pt}
\section{Discussion}
\textbf{Conclusion.}
In this work, we propose the \textit{\methodname{}} metric, $M_{AP}$, to evaluate preference data quality in alignment.
By measuring the gap from the model's current implicit reward margin to the target explicit reward margin, $M_{AP}$ quantifies the discrepancy between the current model and the aligned optimum, thereby indicating the potential for alignment enhancement.
Extensive experiments validate the efficacy of $M_{AP}$ across various training settings under offline and self-play preference learning scenarios.

\textbf{Limitations and future work}. 
Despite the performance improvements, $M_{AP}$ requires tuning a parameter $\beta$ to combine the explicit and implicit margins; future work could explore how to set this ratio automatically.
Additionally, while our experiments focus on the widely applied DPO and SimPO objectives, a broader investigation with alternative preference learning methods is crucial in future works.

% \section{Conclusion}
% In this paper, we introduce the \methodname{} metric to evaluate preference data quality in LLM alignment.
% By measuring the discrepancy between the model's current implicit reward margin to the target explicit reward margin, this metric quantifies the gap between the current model and the aligned optimum, thereby indicating the potential for alignment enhancement.
% Empirical results demonstrate that training on data selected by our metric consistently improves alignment performance, outperforming existing metrics across different base models and training objectives.
% Moreover, this metric extends to data generation scenarios (\ie self-play alignment): by identifying high-quality data from the intrinsic self-generated context, our metric yields superior results across various training settings, providing a comprehensive solution for enhancing LLM alignment through optimized
% preference data generation, selection, and utilization.


\section*{Impact Statement}
This paper presents work whose goal is to advance the field of Machine Learning. There are many potential societal consequences of our work, none which we feel must be specifically highlighted here.



{\small
\bibliographystyle{ieee_fullname}
\bibliography{egbib}
}

\end{document} 